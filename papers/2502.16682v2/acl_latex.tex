 \pdfoutput=1

\documentclass[11pt]{article}
\usepackage{float}

\usepackage{acl}

\usepackage[utf8]{inputenc}
\usepackage{pgfplots}
\usepackage{dsfont}

\DeclareUnicodeCharacter{2212}{−}
\usepgfplotslibrary{groupplots,dateplot}
\usetikzlibrary{patterns,shapes.arrows}
\pgfplotsset{compat=newest}

\usepackage{tikzscale}
\usepackage{amsmath}

\usepackage{multirow, colortbl}
\usepackage{color}
\usepackage{array}
\usepackage{multirow}
\usepackage{CJKutf8}
% \usepackage[usenames,dvipsnames,svgnames,table]{xcolor}
% Define your own color
\usepackage{tabularx,booktabs}
\usepgfplotslibrary{groupplots}
\usepackage{makecell}
\usepackage{enumitem}
\usepackage{placeins}

\usepackage[normalem]{ulem}
\usepackage{graphicx}
\usepackage{booktabs}

\usepackage{pifont}
\newcommand{\cmark}{\ding{51}} % Check mark
\newcommand{\xmark}{\ding{55}} % X mark

\usepackage{pgfplots}
\pgfplotsset{width=10cm,compat=1.9}

\definecolor{ablation6}{HTML}{fcefed}
\definecolor{ablation_tie}{HTML}{fce3e1}

\definecolor{ablation5}{HTML}{fcd8d4}
\definecolor{ablation4}{HTML}{FBC3BC}
\definecolor{ablation3}{HTML}{F7A399}
\definecolor{ablation2}{HTML}{F38375}
\definecolor{ablation1}{HTML}{EF6351}

% \definecolor{ablation1}{HTML}{e3faf8}
% \definecolor{ablation2}{HTML}{C4FFF9}
% \definecolor{ablation3}{HTML}{9CEAEF}
% \definecolor{ablation4}{HTML}{68D8D6}
% \definecolor{ablation5}{HTML}{3DCCC7}
% \definecolor{ablation6}{HTML}{07BEB8}



\usepackage[]{algpseudocode}
\usepackage[]{algorithm}
\usepackage{float}
\algtext*{EndFor}%
\algtext*{EndProcedure}%

\usepackage{booktabs}
% Standard package includes
\usepackage{times}
\usepackage{latexsym}
\usepackage{adjustbox}


% For proper rendering and hyphenation of words containing Latin characters (including in bib files)
\usepackage[T1]{fontenc}
\usepackage[utf8]{inputenc}
\usepackage[russian,english]{babel}
\newcommand{\russian}[1]{{\fontencoding{T2A}\selectfont\foreignlanguage{russian}{#1}}}


\usepackage{amsmath}
\usepackage{amssymb}
\usepackage{booktabs}
\usepackage{multirow}
% For Vietnamese characters
% \usepackage[T5]{fontenc}
% See https://www.latex-project.org/help/documentation/encguide.pdf for other character sets
\usepackage{scalerel,xparse}
% This assumes your files are encoded as UTF8
\usepackage[utf8]{inputenc}
%\usepackage[dvipsnames]{xcolor}

\renewcommand{\floatpagefraction}{.8}%
\renewcommand{\textfraction}{.1}%
\setcounter{totalnumber}{5}

% This is not strictly necessary, and may be commented out,
% but it will improve the layout of the manuscript,
% and will typically save some space.
\usepackage{cleveref}
\usepackage{microtype}
\usepackage{enumitem}\usepackage[utf8]{inputenc}
\usepackage{pgfplots}
\usepackage{dsfont}

\DeclareUnicodeCharacter{2212}{−}
\usepgfplotslibrary{groupplots,dateplot}
\usetikzlibrary{patterns,shapes.arrows}
\pgfplotsset{compat=newest}

\usepackage{tikzscale}
\usepackage{amsmath}
\newcommand{\probP}{\text{I\kern-0.15em P}}

\usepackage{multirow, colortbl}
\usepackage{color}
\usepackage{array}
\usepackage{multirow}
\usepackage{CJKutf8}
% \usepackage[usenames,dvipsnames,svgnames,table]{xcolor}
% Define your own color
\usepackage{tabularx,booktabs}
\usepgfplotslibrary{groupplots}
\usepackage{makecell}
\usepackage{enumitem}
\usepackage{placeins}

\usepackage{soul}
\definecolor{light blue}{RGB}{215, 242, 252}
\definecolor{light purple}{RGB}{247, 215, 252}
\definecolor{light orange}{rgb}{0.9961, 0.875, 0.7188}
\sethlcolor{light blue}
\newcommand{\hlpurple}[1]{\sethlcolor{light purple}\hl{#1}\sethlcolor{light blue}}
\newcommand{\hlorange}[1]{\sethlcolor{light orange}\hl{#1}\sethlcolor{light blue}}

\usepackage{graphicx}
\usepackage{booktabs}

\usepackage{pgfplots}
\pgfplotsset{width=10cm,compat=1.9}

\definecolor{ablation6}{HTML}{fcefed}
\definecolor{ablation_tie}{HTML}{fce3e1}

\definecolor{ablation5}{HTML}{fcd8d4}
\definecolor{ablation4}{HTML}{FBC3BC}
\definecolor{ablation3}{HTML}{F7A399}
\definecolor{ablation2}{HTML}{F38375}
\definecolor{ablation1}{HTML}{EF6351}

% \definecolor{ablation1}{HTML}{e3faf8}
% \definecolor{ablation2}{HTML}{C4FFF9}
% \definecolor{ablation3}{HTML}{9CEAEF}
% \definecolor{ablation4}{HTML}{68D8D6}
% \definecolor{ablation5}{HTML}{3DCCC7}
% \definecolor{ablation6}{HTML}{07BEB8}



\usepackage[]{algpseudocode}
\usepackage[]{algorithm}
\usepackage{float}
\algtext*{EndFor}%
\algtext*{EndProcedure}%


% Standard package includes
\usepackage{times}
\usepackage{latexsym}
\usepackage{adjustbox}


% For proper rendering and hyphenation of words containing Latin characters (including in bib files)
\usepackage[T1]{fontenc}
\usepackage[utf8]{inputenc}
\usepackage[russian,english]{babel}

\usepackage{amsmath}
\usepackage{amssymb}
\usepackage{booktabs}
\usepackage{multirow}
% For Vietnamese characters
% \usepackage[T5]{fontenc}
% See https://www.latex-project.org/help/documentation/encguide.pdf for other character sets
\usepackage{scalerel,xparse}
% This assumes your files are encoded as UTF8
\usepackage[utf8]{inputenc}
%\usepackage[dvipsnames]{xcolor}

\renewcommand{\floatpagefraction}{.8}%
\renewcommand{\textfraction}{.1}%
\setcounter{totalnumber}{5}

% This is not strictly necessary, and may be commented out,
% but it will improve the layout of the manuscript,
% and will typically save some space.
\usepackage{cleveref}
\usepackage{microtype}
\usepackage{enumitem}
\usepackage{placeins}

\crefformat{section}{\S#2#1#3}
\crefformat{subsection}{\S#2#1#3}
\crefformat{subsubsection}{\S#2#1#3}

\newcommand\mymathop[1]{\mathop{\operatorname{#1}}}


% -------------- prompt box setup ---------------- %
% \definecolor{bggray}{rgb}{0.95, 0.95, 0.95}
% \usepackage[%
%     framemethod=tikz,
%     skipbelow=\topskip,
%     skipabove=\topskip
% ]{mdframed}
% \DeclareUnicodeCharacter{2212}{−}
% \usepgfplotslibrary{groupplots,dateplot}
% \usetikzlibrary{patterns,shapes.arrows}
% \pgfplotsset{compat=newest}


% comments
\definecolor{zoey green}{rgb}{0.684,0.836,0.227}
%\usepackage{xcolor}
\newcommand{\ensuretext}[1]{#1}
%Marine
\newcommand{\mcmarker}{\ensuretext{\textcolor{magenta}{\ensuremath{^{\textsc{M}}_{\textsc{C}}}}}}
%Aquia
\newcommand{\armarker}{\ensuretext{\textcolor{blue}{\ensuremath{^{\textsc{A}}_{\textsc{R}}}}}}
%Marianna
\newcommand{\mjmarker}{\ensuretext{\textcolor{cyan}{\ensuremath{^{\textsc{M}}_{\textsc{M}}}}}}
%Eleftheria
\newcommand{\ebmarker}{\ensuretext{\textcolor{green}{\ensuremath{^{\textsc{E}}_{\textsc{B}}}}}}
%Sweta
\newcommand{\samarker}{\ensuretext{\textcolor{orange}{\ensuremath{^{\textsc{S}}_{\textsc{A}}}}}}
%Calvin
\newcommand{\cbmarker}{\ensuretext{\textcolor{purple}{\ensuremath{^{\textsc{C}}_{\textsc{B}}}}}}
%Zoey
\newcommand{\zkmarker}{\ensuretext{\textcolor{zoey green}{\ensuremath{^{\textsc{Z}}_{\textsc{K}}}}}}

%Review
\newcommand{\rxmarker}{\ensuretext{\textcolor{cyan}{\ensuremath{^{\textsc{R}}_{\textsc{X}}}}}}

% enable comments here
\newcommand{\mycomment}[3]{\ensuretext{\textcolor{#3}{[#1 #2]}}}
%disable comments here
%\newcommand{\zk}[1]{\ignore{#1}}
%\newcommand{\mycomment}[3]{}
\newcommand{\mc}[1]{\mycomment{\mcmarker}{#1}{magenta}}
\newcommand{\ar}[1]{\mycomment{\armarker}{#1}{blue}}
\newcommand{\mjm}[1]{\mycomment{\mjmarker}{#1}{cyan}}
\newcommand{\eb}[1]{\mycomment{\ebmarker}{#1}{green}}
\newcommand{\wx}[1]{\mycomment{\wxmarker}{#1}{purple}}
\newcommand{\sa}[1]{\mycomment{\samarker}{#1}{orange}}
\newcommand{\rx}[1]{\mycomment{\rxmarker}{#1}{cyan}}
\newcommand{\zk}[1]{\mycomment{\zkmarker}{#1}{zoey green}}
\newcommand{\ignore}[1]{}


% \title{Rewritten Inputs, Refined Outputs: Enhancing Machine Translation through Source Text Rewrites}
% \title{Does Rewriting Inputs Improve Translations from Large Language Models?}
% \title{Rewriting Inputs for Translation with Large Language Models}
% \title{\textit{Rewritten} Inputs, \textit{Refined} Outputs: \\
% Does Rewriting with Language Models Improve Translation?}
\title{Automatic Input Rewriting Improves Translation \\ with Large Language Models}

% \mc{or should we mention simplification?}
% \zk{I think current title is okay without mentioning simplification}

\author{Dayeon Ki \\
  University of Maryland \\
  \texttt{dayeonki@umd.edu} \\\And
  Marine Carpuat \\
  University of Maryland \\
  \texttt{marine@cs.umd.edu} \\}


\begin{document}
\maketitle


% \mdfsetup{%
%     leftmargin=0pt,
%     rightmargin=0pt,
%     backgroundcolor=bggray,
%     middlelinecolor=black,
%     roundcorner=3
% }
% \newtcolorbox[list inside=prompt,auto counter,number within=section]{prompt}[1][]{
%     colbacktitle=black!60,
%     fonttitle=\small,
%     coltitle=white,
%     fontupper=\footnotesize,
%     boxsep=4pt,
%     left=0pt,
%     % right=0pt,
%     top=0pt,
%     bottom=0pt,
%     boxrule=1pt,
%     #1,
% }


\begin{abstract}
Can we improve machine translation (MT) with LLMs by rewriting their inputs automatically? Users commonly rely on the intuition that well-written text is easier to translate when using off-the-shelf MT systems. LLMs can rewrite text in many ways but in the context of MT, these capabilities have been primarily exploited to rewrite outputs via post-editing. We present an empirical study of 21 input rewriting methods with 3 open-weight LLMs for translating from English into 6 target languages. We show that text simplification is the most effective MT-agnostic rewrite strategy and that it can be improved further when using quality estimation to assess translatability. Human evaluation further confirms that simplified rewrites and their MT outputs both largely preserve the original meaning of the source and MT. These results suggest LLM-assisted input rewriting as a promising direction for improving translations.\footnote{We release our code and dataset at \url{https://github.com/dayeonki/rewrite_mt}.}


% using the \textsc{Tower} LLM \citep{alves2024tower},\mc{Need to update to reflect experiments actually included in the paper}

% \mc{The single most important part of the human evaluation is meaning preservation since it is harder to assess automatically, so the take-away needs to be mentioned here, and in all the other relevant spots in the paper (intro, conclusion, etc.).} 


% \mc{X tasks}

% \mc{Is the 21 number still correct?}
% \zk{Yes, I counted and it totals 21 methods.}

% We first consider stylistic rewrites that are agnostic to MT, and tailor them to the MT task by assessing input translatability with quality estimation.

%sign  according to MT-agnostic style dimensions, an stylisti
%Rewriting inputs is a common strategy to enhance translation quality exploited by end users and in dedicated machine translation (MT) architectures.

%how effective is this approach when translating with Large Language Models (LLMs)? 

%We conduct an empirical study using the Tower multilingual LLM, primarily trained for translation-related tasks, across two distinct rewriting methods: \textbf{MT-Agnostic} (without translation-related knowledge) and \textbf{MT-Aware} (with translation signal), yielding three key findings. First, MT-agnostic style rewrites do not uniformly improve translations but re-ranking and fine-tuning to guide rewrites with reference-free quality estimation improves translation quality according to both automatic and human evaluations. Second, when a rewrite is easier to translate, it often fails to preserve the original meaning, which poses a challenge of Pareto optimization. Third, rewriting inputs complements post-editing outputs, demonstrating the effectiveness of combining both strategies to further improve translation quality\footnote{We will release all models, datasets, and code.}.

% \footnote{We release our code and dataset at \url{https://github.com/dayeonki/rewrite_mt}.}.


% \mc{TODOs in preparation for author response:
% (1) do a read through post deadline to catch any remaining issues
% (2) manual evaluation
% (3) possibly try other LLMs with the winning strategies (Aya? larger Tower?)
% (4) possibly try other held-out test sets that are not out-of-En lang pairs
% (5) possibly optimize simplification+MT prompts to maximize translatability and/or reference-based metrics on a devset through DSPy
% }

\end{abstract}

\section{Introduction}

\begin{figure*}[ht]
    \centering
    \includegraphics[width=0.98\textwidth]{figs/MainFigure.pdf}
    \caption{Overview of our interactive pipeline for coding evaluation. 
    (A) We obfuscate the input of existing fully specified datasets to reflect how programmers tend to underspecify requests to LLMs (e.g., via docstrings or comments) in practice. (B) As developers may interact with models in a variety of ways, we explore $4$ different feedback types and introduce a pipeline that mimics the iterative refinement loop that programmers often use with chat models, (C) where the \cm{} generates a solution using feedback on its previous solution, (D) and the \user{} provides updated feedback to the \cm.
    }
    
    \label{fig:main_figure}
\end{figure*}


Machine translation (MT) users and developers have long exploited the idea that some texts are easier to translate than others. For instance, guiding people to edit their inputs so that they are well formed is a cornerstone of MT literacy courses \citep{bowker-2021-promoting,steigerwald-etal-2022-overcoming}, and adopting plain language has been shown to improve the readability of translated health content \citep{Rossetti2019}. In MT research, a wealth of studies have considered pre-processing strategies to rewrite inputs, particularly for statistical MT \citep{XiaMcCord2004,callison-burch-etal-2006-improved,stajner-popovic-2016-text}.%TODO: expand cites at camera-ready or in related work section

%In MT research, a wealth of studies have considered pre-processing strategies to rewrite inputs, particularly to improve statistical MT via input paraphrasing \citep{callison-burch-etal-2006-improved, mirkin-etal-2009-source, marton-etal-2009-improved, aziz-etal-2010-learning}, reordering \citep{XiaMcCord2004,WangCollinsKoehn2007}, or simplification \citep{stajner-popovic-2016-text,stajner-popovic-2019-automated} or to design systems that help users pre-edit their inputs \citep{mirkin-etal-2013-sort,Miyata2017DissectingHP}. 

The growing use of Large Language Models (LLMs) for translation leads us to revisit the impact of rewriting inputs on MT. On the one hand, rewriting inputs for LLM translation aligns with the re-framing of MT as a multi-step process \citep{Briakou}. LLMs have shown promise in rewriting MT outputs \citep{ki2024guiding, zeng2024improving, xu2024llmrefine}, and can rewrite text according to various style specifications \citep{raheja-etal-2023-coedit, hallinan2023steer, shu2023rewritelm, dipper}. On the other hand, current models might already be robust to input variability, since they are trained on vast amounts of heterogeneous data \citep{touvron2023llama}, fine-tuned on diverse tasks \citep{raffel-etal-2020-exploring,alves2024tower} and operate at a much higher quality level compared to the statistical MT systems used in previous pre-processing studies.%todo later: add cite for this.

How should inputs be rewritten for MT? The assumption that well-written texts are easier to translate drives recommendations for MT literacy, as well as the use of paraphrasing \citep{callison-burch-etal-2006-improved, mirkin-etal-2009-source, marton-etal-2009-improved, aziz-etal-2010-learning} and simplification  \citep{stajner-popovic-2016-text,stajner-popovic-2019-automated}. However, can we more directly rewrite inputs so that they are easier to translate? Generic translatability has been defined as “a measurement of the time and effort
it takes to translate a text” \citep{kumhyr-etal-1994-internationalization}. \citet{uchimoto-etal-2005-automatic} introduced a metric to quantify MT translatability based on back-translation of MT hypotheses in the source language. Given recent progress in quality estimation \citep{fernandes-etal-2023-devil, naskar-etal-2023-quality, tomani2024qualityaware}, we propose instead to use reference-free quality estimation scores as a measure of translatability.

We thus ask the following research questions:
\begin{enumerate}[label=(\arabic*),topsep=0pt,itemsep=-1ex,partopsep=-1ex,parsep=1ex]
    \item Can we improve MT quality from LLMs by rewriting inputs for style?
    \item Do quality estimation metrics provide useful translatability signals for input rewriting?
\end{enumerate}

% We conduct an empirical study of the \textsc{Tower-Instruct} LLM \citep{alves2024tower} for a total of 21 input rewriting methods with varying levels of MT-awareness on translation from English into German, Russian and Chinese, and we further evaluate the generalizability of our best performing approach on translation from English into Czech, Hebrew and Japanese. 
We conduct an empirical study with 3 open-weight LLMs for a total of 21 input rewriting methods with varying levels of MT-awareness on translation from English into German, Russian and Chinese, and we further evaluate the generalizability of our best performing approach on translation from English into Czech, Hebrew and Japanese (\S \ref{sec:newlanguages}). 
Our results show that simple \textbf{MT-Agnostic rewrites} obtained by prompting LLMs to simplify, paraphrase, or change the style of the input, improve translatability, and that simplification most reliably improves translation quality. Interestingly, these MT-agnostic rewrites are more effective than \textbf{Task-Aware rewrites}, where LLMs are prompted to rewrite inputs for the purpose of MT (\S \ref{simplification best}). Finally, using quality estimation signals to assess \textbf{translatability} at the segment level and select when to use rewrites further improves MT quality, outperforming more expensive fine-tuning strategies (\S \ref{input selection}). Human evaluation further confirms that simplified rewrites and their MT largely preserve the original meaning of the source and MT (\S \ref{human evaluation}).

% \mc{X New Tasks}

%We conduct an empirical study of the Tower LLM \citep{alves2024tower} for a total of 21 input rewriting methods with varying levels of MT-awareness on translation from English into German, Russian and Chinese. We further show that the benefits of our best performing approach generalize to new test sets from the WMT23 general translation task for X language pairs.

%To address these questions, we first generate \textbf{MT-Agnostic rewrites} by prompting LLMs to simplify, paraphrase or change the style of the original input (\S \ref{3.1 mt-agnostic}). 

%Next, we design three strategies to obtain \textbf{MT-Aware rewrites}: using Chain-of-Thought \citep{wei2023chainofthought} prompting, selecting generic rewrites with reference-free quality estimation metrics \citep{guerreiro2023xcomet}, and fine-tuning LLMs to rewrite for translation (\S \ref{3.2 task-aware}).
%
%We conduct an empirical study of the Tower LLM \citep{alves2024tower} for a total of 21 input rewriting methods with varying levels of MT-awareness on translation from English into German, Russian and Chinese. Our findings suggest that rewriting inputs can improve translation quality (\S \ref{res:mt agnostic}) and that quality estimation feedback helps generate inputs that are better translated (\S \ref{res:mt aware}), even outperforming post-editing translation outputs (\S \ref{res:post-editing}).

\definecolor{light blue}{RGB}{215, 242, 252}
\definecolor{light purple}{RGB}{247, 215, 252}
\definecolor{light orange}{rgb}{0.9961, 0.875, 0.7188}

\section{Input Rewriting Methods}

\label{3 method}
Within the process of source rewriting, the goal of a rewrite model is to rewrite the original source sentence $s$ into another form that is easier to translate while preserving its intended meaning. For \textbf{MT-Agnostic} rewriting methods (\S \ref{3.1 mt-agnostic}), which lacks translation-related knowledge, the rewrite model $\mathcal{M}_{\theta}$ can rewrite $s$ into $s'$:
\begin{equation}
    s' = \mathcal{M}_{\theta}(s)
\end{equation}

On the contrary, both \textbf{Task-Aware} (\S \ref{3.2 task-aware}) and \textbf{Translatability-Aware} (\S \ref{3.3 translatability-aware}) rewriting methods incorporate some translation signal. For Task-Aware, $\mathcal{M}_{\theta}$ rewrites $s$ with the information of the end-task (MT):

\begin{equation}
    s' = \mathcal{M}_{\theta}(s, \text{MT task})
\end{equation}

For Translatability-Aware method, it rewrites with the knowledge of segment level quality estimation scores between source and the output of a specific MT system MT($t$):

\begin{equation}
    s' = \mathcal{M}_{\theta}(s, \text{\textsc{xCOMET}}(s,\text{MT}(t)))
\end{equation}
Figure~\ref{fig:main_figure} shows the overview of our proposed rewriting pipeline. To find the most effective $\mathcal{M}_{\theta}$, we test a total of 21 input rewriting methods.


\subsection{\fcolorbox{white}{light blue}{\raisebox{-0.2em}{\includegraphics[height=1em]{figures/logos/agnostic.png}} MT-Agnostic} Rewriting}
\label{3.1 mt-agnostic}
MT-agnostic rewriting methods reflect various a priori assumptions on what makes text easier to translate. They do not take as input any signal of translatability or knowledge about the end-task. We consider three prompting variants here, all inspired by prior works on source rewriting \citep{mirkin-etal-2009-source, mirkin-etal-2013-sort, stajner-popovic-2016-text}.

\paragraph{Simplification.}
Simplification includes replacing complex words with simpler ones, rephrasing complex syntactic structures, and shortening sentences \citep{article, Feng2008}. Prior works show that simplified inputs are more conducive to MT, and particularly improve the fluency of MT outputs \citep{stajner-popovic-2019-automated}.

\paragraph{Paraphrase.}
Paraphrases are alternative ways of expressing the same information within one language, which can help resolve unknown or complex words \citep{callison-burch-etal-2006-improved}. Paraphrasing with LLMs might benefit MT by normalizing inputs using language patterns that are more frequent in LLM training data. Further, some LLMs, such as \textsc{Tower} \citep{alves2024tower}, are fine-tuned on both paraphrasing and MT tasks, and might thus produce paraphrases that are useful for MT.

\paragraph{Stylistic.}
We employ an off-the-shelf text editing tool \textsc{CoEdIT-XL} \citep{raheja-etal-2023-coedit} to rewrite inputs according to diverse style specifications:
\begin{itemize}[leftmargin=*, itemsep=2pt, parsep=-1pt]
 \item \textbf{Grammar}: Fix the grammar.
 \item \textbf{Coherent}: Make the text more coherent.
 \item \textbf{Understandable}: Make it easier to understand.
 \item \textbf{Formal}: Rewrite the text more formally.
\end{itemize}
These operationalize the assumption that well-formed text is easier to translate.
%In addition to translation-related tasks, Tower-Instruct is trained on Grammatical Error Correction (GEC) task, which motivates us to consider the Grammar prompt. The Coherent and Understandable prompts function similarly to the Simplification rewrites, making the text easier to translate by using a dedicated text editing tool. Further, we explore the impact of increasing formality on MT quality \citep{lee-etal-2023-improving-formality} with Formal prompt.
All prompt templates are shown in Appendix Table~\ref{tab:prompting_template}.

\subsection{\fcolorbox{white}{light purple}{\raisebox{-0.2em}{\includegraphics[height=1em]{figures/logos/task.png}} Task-Aware} Rewriting}
\label{3.2 task-aware}

For task-aware rewriting methods, we design prompts that account for the fact that rewrites are aimed at MT. 
Prior work has shown that LLMs can post-edit errors in MT outputs \citep{ki2024guiding, zeng2024improving, treviso-etal-2024-xtower, xu2024llmrefine, briakou-etal-2024-translating}, raising the question of whether this ability can be extended to rewriting inputs to enhance translatability. Additionally, \textsc{Tower-Instruct} has been jointly trained on paraphrasing, grammatical error correction (GEC), and translation tasks, suggesting it may be well-suited for performing translatability rewrites in a zero-shot fashion. We consider two prompting strategies (Refer to Appendix Table~\ref{tab:prompting_template} for exact templates):

\paragraph{Easy Translation.} We prompt LLMs to rewrite inputs in a way that specifically facilitates translation into the target language.

\paragraph{Chain of Thought Rewrite+Translate.} We use a Chain of Thought (\citet{wei2023chainofthought}, CoT) style prompt where LLMs are prompted to handle the entire rewriting and translation process in one sequence of CoT instructions within a single model.



\subsection{\fcolorbox{white}{light orange}{\raisebox{-0.2em}{\includegraphics[height=1em]{figures/logos/translatability.png}} Translatability-Aware} Rewriting}
\label{3.3 translatability-aware}
We propose to use quality estimation scores for a given input and output pair to assess the translatability of inputs at the segment level. This makes it possible to inject translatability signals at inference or training time. We introduce a lightweight inference-time selection strategy, and contrast it against a more expensive fine-tuning approach.

\paragraph{Inference-Time Selection.}
Input segments might not benefit from rewriting uniformly, since the quality of the original inputs and of their rewrites might vary. We thus propose to use translatability scores to decide whether or not to replace the original input with a rewrite  at inference time. We use the state-of-the-art \textsc{xCOMET} quality estimation tool \citep{guerreiro2023xcomet} to assess how good the translation $t'$ of a rewrite $s'$ is: \textsc{xCOMET}$(s',t')$. We compare this score with the estimated quality of the translation $t$ of the original source $s$, choosing to use the rewrite if \textsc{xCOMET}$(s',t')$ > \textsc{xCOMET}$(s,t)$, and keeping the original source otherwise. This straightforward approach allows us incorporate translatability signals at inference time, with little additional cost.

% we consider the best-performing rewrite among MT-Agnostic and Task-Aware methods. 

\paragraph{Supervised Fine-tuning.}
The translatability-based selection process described above for inference could also be used to gather examples of good rewrites and enable instruction fine-tuning of models to rewrite text for improved translation.
%
%
%Quantifying whether a given rewrite improves translatability at the segment level as above makes it possible to training data, and thus fine-tuning
%We explore fine-tuning LLMs to enhance their capability in rewriting the source for improved translation.
While designing an optimal approach for this task is out of scope for this work,  we wish to compare our inference-time selection strategy with a straightforward training strategy. We construct a fine-tuning dataset of positive rewrite examples $\mathcal{D}_{pos}$, as follows: for a given input $s$, we generate rewrites using all MT-agnostic methods. We add to our training set the rewrites that improve translatability as measured by \textsc{xCOMET}$(s', t')$ > \textsc{xCOMET}$(s,t)$. The base LLM is then instruction fine-tuned based to rewrite input $s$ so that it is better translated, using $s'$ as supervision. Detailed prompt templates are shown in Appendix \ref{appendix:prompt_templates}.

%The original data is drawn from the English-German and English-Russian subset from WMT-20, 21, and 22 General MT task datasets \citep{freitag2021experts}\footnote{We do not consider English-Chinese pair here since this language pair is not supported in the dataset.}. We first prompt LLMs using MT-agnostic rewriting methods to generate rewrites $s'$ for each $s$. Then, we translate the original source and rewrites using Tower-Instruct 7B to get MT outputs $t$ and $t'$. We use \textsc{xCOMET} to compute scores between $s'$ and $t'$, which estimates translation quality for each rewrite. Our goal of fine-tuning is to generate better rewrites than the original source. Thus, we create $\mathcal{D}_{pos}$, a subset of $\mathcal{D}_{ft}$, which only contains the rewrites where \textsc{xCOMET}$(s', t')$ > \textsc{xCOMET}$(s,t)$. 

%For prompt template, we compare three variants:
%\begin{itemize}[leftmargin=*, itemsep=2pt, parsep=-1pt]
% \item \textbf{Basic}: Given $s$, generate $s'$.
% \item \textbf{MT}: Given $s$ and $t$, generate $s'$.
% \item \textbf{Reference}: Given $s$ and $r$ (reference translation), generate $s'$.
%\end{itemize}
%Specific prompt templates are outlined in Appendix \ref{appendix:prompt_templates}. All parameters are listed in Appendix \ref{appendix:parameters}.

\section{Experimental Setup}


    
    
\begin{table*}[t!]
    \centering
    \small
    
    \scalebox{0.90}{
    \setlength{\tabcolsep}{1.0pt}
    \begin{tabular}{l c c c r | c c c c c c |c  c c }
    \toprule
    \multirow{1}{*}{Method} & \multirow{1}{*}{Recipe} & \multirow{1}{*}{Complexity} & \multirow{1}{*}{\# P.} & \multirow{1}{*}{\# T.P.}& MME & MMB &POPE & \multicolumn{1}{c} {SEED} & MMMU & MM-Vet& TQA & SQA-I  & \multicolumn{1}{c}{GQA} \\
    \midrule
    \rowcolor{gray!14}
    \multicolumn{14}{l}{\textbf{\textit{Encoder-based VLMs}}} \\ 
    OpenFlamingo~\cite{openflamingo} & \underline{PT, SFT}& Quadratic & 9B& 96.6\%  & - & 4.6 & - & - & - & - & 33.6 & - & - \\
    MiniGPT-4~\cite{minigpt} & \underline{PT, SFT}& Quadratic & 13B& 94.8\%  & 581.7 & 23.0 & - & - & -& 22.1 & - & - & 32.2  \\
    Qwen-VL~\cite{qwenvl} & \underline{PT, SFT}& Quadratic & 7B& 100.0\%  & - & 38.2 & - & 56.3 & - & - & 63.8 & 67.1 & 59.3\\ 
    LLaVA-Phi~\cite{llavaphi}  & \underline{PT, SFT}& Quadratic & 3B& 90.0\%  & 1335.1 & 59.8 & 85.0 & - & - & 28.9& 48.6 & 68.4 & - \\
    MobileVLM-3B~\cite{mobilevlm} & \underline{PT, SFT}& Quadratic & 3B& 90.0\%  & 1288.9 & 59.6 & 84.9 & - & - & - & 47.5 & 61.0 & 59.0  \\
    VisualRWKV~\cite{visualrwkv} & \underline{PT, SFT}&  \textbf{Linear} & 3B& 90.0\%  & 1369.2 & 59.5 & 83.1 & - & - & - & 48.7 & 65.3 & 59.6 \\
    VL-Mamba~\cite{vlmamba} & \underline{PT, SFT}&  \textbf{Linear} & 3B& 90.0\%  & 1369.6 & 57.0 & 84.4 & - & -& 32.6 & 48.9 & 65.4 & 56.2 \\
    Cobra~\cite{cobra} & \underline{PT, SFT}&  \textbf{Linear} & 3.5B& 82.6\%  & - & - & \textbf{88.4} & - & - & - & 58.2 & - & \textbf{62.3}\\
    \midrule
    \rowcolor{gray!14}
    \multicolumn{14}{l}{\textbf{\textit{Decoder-only VLMs}}} \\
    Fuyu-8B (HD)~\cite{fuyu} & \underline{PT, SFT}& Quadratic & 8B& 100.0\%  & 728.6 & 10.7 & 74.1 & - & - & 21.4 & - & - & -\\
    SOLO~\cite{solo} & \underline{PT, SFT}& Quadratic &  7B& 100.0\%   & 1001.3 & - & - & 64.4 & - & - & - & 73.3 & -   \\    
    Chameleon-7B~\cite{chameleon}  & \underline{PT, SFT}& Quadratic &  7B& 100.0\%   & 170 & 31.1 & - & 30.6 & 25.4 & 8.3 & 4.8 & 47.2 & -\\  
    EVE-7B~\cite{eve}  & \underline{PT, SFT}& Quadratic &  7B& 100.0\%  & 1217.3 & 49.5 & 83.6 & 61.3 & \underline{32.3} & 25.6& 51.9 & 63.0 & 60.8 \\
    Emu3~\cite{emu3} & \underline{PT, SFT}& Quadratic & 8B& 100.0\%  & - & 58.5 & 85.2 & \underline{68.2} & 31.6 & \underline{37.2} & \underline{64.7} & \underline{89.2} & 60.3\\
    HoVLE~\cite{hovle} & DT, PT, SFT & Quadratic & \textbf{2.6B}& 100.0\%  & \textbf{1433.5} & \textbf{71.9} & \underline{87.6} & \textbf{70.7} & \textbf{33.7} & \textbf{44.3} & \textbf{66.0} & \textbf{94.8} & \underline{60.9} \\
    \rowcolor{green!15}
    \name{} & \textbf{DT} & \textbf{Linear} & \underline{2.7B}& \underline{14.7\%}  &1303.5 & 57.2 & 85.2 & 62.9& 30.7  & 31.1 &47.7 & 79.2 & 57.4 \\
    \rowcolor{yellow!15}
    \name{} & \textbf{DT} & \underline{Hybrid} & \underline{2.7B}& \textbf{11.2\%}  & \underline{1371.1} & \underline{63.7} & 86.7 & 66.3 & \underline{32.3} & 36.9 & 55.1 & 86.9 & 59.3  \\
    
    \bottomrule
    \end{tabular}
    }
    \vspace{-1em}
    \caption{\textbf{Comparison with existing VLMs on general VLM benchmarks.} ``Recipe'' denotes the adopted training recipe. ``PT'', ``SFT'', and ``DT'' denote the pre-training, supervised fine-tuning, and distillation training, respectively. ``Complexity'' denotes the model computation complexity with respect to the number of tokens. ``\# P.'' denotes the number of total parameters. ``\# T.P.'' denotes the percentage of trainable parameters ($\frac{\text{trainable paramters}}{\text{total parameters}}$). The best performance is highlighted in \textbf{bold} and the second-best result is \underline{underlined}.}
    \label{tab:results_general}
    \end{table*}



\subsection{Model \& Data}

\paragraph{MT System.} We use \textsc{Tower-Instruct} 7B as our MT system for all our experiments since it is specifically trained for translation-related tasks and has demonstrated superior MT performance compared to other LLMs \citep{alves2024tower}.

\paragraph{Rewriting Models.} For prompting experiments, we use 7B variant of three open-weight LLMs in zero-shot setting: \textsc{LLaMA-2} \citep{touvron2023llama} \---\ the base model for \textsc{Tower-Instruct}, \textsc{LLaMA-3} \citep{grattafiori2024llama3herdmodels} \---\ more recent multilingual model compared to \textsc{LLaMA-2}, and \textsc{Tower-Instruct} \citep{alves2024tower} \---\ the same LLM as used for our MT system.\footnote{The HuggingFace model names are detailed in Appendix Table \ref{tab:huggingface_api}.} For supervised fine-tuning, we draw training samples from the English-German and English-Russian subset from WMT-20, 21, and 22 General MT task datasets \citep{freitag2021experts}\footnote{We do not consider English-Chinese pair here since this language pair is not supported in the dataset.}, and provide detailed parameter settings in Appendix~\ref{appendix:parameters}.


%We first prompt LLMs using MT-agnostic rewriting methods to generate rewrites $s'$ for each $s$. Then, we translate the original source and rewrites using Tower-Instruct 7B to get MT outputs $t$ and $t'$. We use \textsc{xCOMET} to compute scores between $s'$ and $t'$, which estimates translation quality for each rewrite. Our goal of fine-tuning is to generate better rewrites than the original source. Thus, we create $\mathcal{D}_{pos}$, a subset of $\mathcal{D}_{ft}$, which only contains the rewrites where \textsc{xCOMET}$(s', t')$ > \textsc{xCOMET}$(s,t)$. 
%For prompt template, we compare three variants:
%\begin{itemize}[leftmargin=*, itemsep=2pt, parsep=-1pt]
% \item \textbf{Basic}: Given $s$, generate $s'$.
% \item \textbf{MT}: Given $s$ and $t$, generate $s'$.
% \item \textbf{Reference}: Given $s$ and $r$ (reference translation), generate $s'$.
%\end{itemize}
%Specific prompt templates are outlined in Appendix \ref{appendix:prompt_templates}.


\paragraph{Test Data.}
We use the WMT-23 General MT task\footnote{\url{https://www2.statmt.org/wmt23/translation-task.html}} from the \textsc{TowerEval} dataset\footnote{\url{https://huggingface.co/datasets/Unbabel/TowerEval-Data-v0.1}} to guarantee that it was held out from the various training stages. We focus on translation from English into German (\textsc{En-De}), Russian (\textsc{En-Ru}) and Chinese (\textsc{En-Zh}) for an extensive empirical comparison, and then test whether the most promising approaches generalize to translation from English into Czech (\textsc{En-Cs}), Hebrew (\textsc{En-He}) and Japanese (\textsc{En-Ja}). % We use combination of dev and test split from the original dataset\footnote{We translate each source sentence using our MT system and do not use the original NLLB 3B \citep{nllbteam2022language} translations.}. Reference translations are human-produced directly based on the source sentences.
See Appendix Table~\ref{tab:dataset_details} for data statistics.

% \mc{the key quesiton is how? directly based on the source? or are the reference translations generated by asking people to post-edit NLLB output?}
% \zk{I checked again the Tower paper, TowerEval huggingface page and the dataset I used -- the datasets that TowerEval name as "WMT23 Automatic Post-Edition" does not corresponds to actual WMT23 Automatic Post-edition (only has En-Marathi pair). The dataset here are source and reference from WMT23 general mt task with translations from nllb model. So I'll reframe it as "we combine development and test sets of WMT23 General MT task" and also get rid of en-ru, en-zh results as held-out test sets.}
% \mc{okay}

% \mc{How were the references created? Are they translatinos of the source from scratch or are they created by post-editing the NLLB output? if it is the latter it might create a weird bias.}

\subsection{Evaluation Metrics}
We use \textsc{xCOMET} \citep{guerreiro2023xcomet} and \textsc{MetricX} \citep{juraska-etal-2023-metricx} to evaluate different aspects of rewrite quality. Specifically, we use \textsc{xCOMET-XL}\footnote{\url{https://huggingface.co/Unbabel/XCOMET-XL}} and \textsc{MetricX-23-XL}.\footnote{\url{https://huggingface.co/google/metricx-23-xl-v2p0}} Higher scores indicate better performance for \textsc{xCOMET}, while lower scores are better with \textsc{MetricX}.


\paragraph{Translatability.}
%Generic \textit{translatability} has been defined as ``a measurement of the time and effort it takes to translate a text'' \citep{kumhyr-etal-1994-internationalization}. Here, we define translatability as a measure of how well a given source sentence can be translated by a particular MT system. More translatable inputs yield better MT outputs \citep{uchimoto-etal-2005-automatic}, so 
We quantify translatability with the quality estimation score for a specific input--output pair (\textsc{xCOMET}$(s', t')$ or \textsc{MetricX-QE}$(s',t')$). A rewrite $s'$ of the original input $s$ is considered easier to translate if \textsc{xCOMET}$(s', t')$ is higher than \textsc{xCOMET}$(s, t)$.

\paragraph{Meaning Preservation.} We do not want rewrites that are easier to translate at the expense of changing the original meaning. Our meaning preservation metric evaluates how well the rewrite maintains the intended meaning of the translation as represented by the reference \citep{Graham2015CanMT}. We use a reference-based metric as opposed to using the semantic similarity between $s$ and $s'$ because it abstracts the meaning away from the specific formulation of $s$, reducing overfitting. We compute \textsc{xCOMET} scores between the rewrites and reference translations (\textsc{xCOMET}$(s',r)$). The desired behavior is to minimize the deterioration in \textsc{xCOMET}$(s',r)$ compared to \textsc{xCOMET}$(s,r)$.


\paragraph{Translation Quality.} We additionally report the combined evaluation metric, \textsc{xCOMET}$(s',t',r)$ to take into account of the trade-off between the two above metrics, and \textsc{MetricX}$(t',r)$ which also assesses translation quality of the rewrite but is not informed by the updated source $s'$.

%\paragraph{\textsc{MetricX}.} In addition to \textsc{xCOMET}, we consider \textsc{MetricX-23-XL} \citep{juraska-etal-2023-metricx} as a secondary evaluation metric. We use two score variants: \textsc{MetricX}$(s',t')$ for quality estimation and \textsc{MetricX}$(t',r)$ as a reference-based score.


\subsection{Use case example}

The \gls{hpo} experiment run by using the scripts the code boxes above refer to results, at the time of writing, in the state-of-the-art classification of Braille characters from the reduced dataset used in~\cite{pedersen_neuromorphic_2024}. Figure~\ref{fig:par_coord} summarizes the exploration carried out across the search space defined in Code~\ref{code:HPO_conf}, reporting both the best training accuracy and the best validation accuracy achieved during the learning phase of each trial.
In the rightmost part, the test accuracy is reported.
As it is reported in Code~\ref{code:HPO_train}, the value for the \texttt{default} key in the dictionary given to \texttt{nni.report\_final\_result()}, namely the objective metrics for optimization with this experiment, is set to be the validation accuracy; particularly, the best validation accuracy achieved throughout the training epochs.\\
From Code~\ref{code:HPO_train}, it is also possible to track down the selection criterion for the optimal model.
At the end of the training stage, the weights from the highest validation accuracy are loaded, and test is performed.
The resulting accuracy is passed to \texttt{nni.report\_final\_result()} as value for the key \texttt{test}, and the best test accuracy at the end of the \gls{hpo} experiment will identify what combination of hyperparameters is the optimal one for the model under optimization in the selected task.\\
In Figure~\ref{fig:cm}, the confusion matrix produced on the test set by the optimized \gls{snn} is shown, with partial misclassification in two classes only and an overall accuracy of 97.14\%.

\begin{figure}[b]
    \centering
    \includegraphics[width=\textwidth]{Figures/parallel_coordinates.pdf}
    \caption{Exploration of the search space with the resulting test accuracy for each combination of hyperparameters}
    \label{fig:par_coord}
\end{figure}

\begin{figure}[t]
    \centering
    \includegraphics[width=0.5\textwidth]{Figures/cm.pdf}
    \caption{Confusion matrix produced on the test set by the optimal model. The overall accuracy is 97.14\%.}
    \label{fig:cm}
\end{figure}


\subsection{Published works}

The application-oriented automatic \gls{hpo} procedure described in this document is the result of ongoing efforts that lead to continuous refinement and customization of the pipeline initially proposed in~\cite{fra_human_2022}.
Its adaptability, rooted in the wide range of possibilities offered by \gls{nni}, is at the same time the key feature for its employment and the driving force for its never-ending development. In Table~\ref{table:works}, a summary of the published works that use it for spiking models is reported.

\begin{table*}[h]
    \renewcommand{\arraystretch}{1.15}
    \centering
    \caption{Summary of published works that performed application-oriented automatic \gls{hpo} through \gls{nni} based on the procedure presented here}
    \label{table:works}
    % \begin{tabular}{{|>{\centering}m{1.5cm}|>{\centering}m{2.7cm}|>{\centering}m{1.4cm}|>{\centering}m{2.1cm}|>{\centering}m{1.9cm}|>{\centering\arraybackslash}m{1.8cm}|}}
    %     \hline
    %     { Neuron model} & { Device } & { Used RAM } & { Mean inference time } & { Mean energy per inference } & { Accuracy } \\
    %     \hline
    %     \multirow{3.1}{*}{ \texttt{Leaky} } & { STM32MP1 } & { 65.7 MB } & { 0.13 s } & { 215.1 mJ } & \multirow{3.3}{*}{ 93.91\% } \\
    %     \cline{2-5}
    %     {  } & { Raspberry Pi 3B+ } & { 77.8 MB } & { 0.06 s } & { 268.8 mJ } & {  } \\
    %     \cline{2-5}
    %     {  } & { Raspberry Pi 4B } & { 77.4 MB } & { 0.03 s } & { 153.9 mJ } & {  } \\
    %     \hline
    %     \multirow{3.1}{*}{ \texttt{Synaptic} } & { STM32MP1 } & { 167.9 MB } & { 0.22 s } & { 383.4 mJ } & \multirow{3.3}{*}{ 93.84\% } \\
    %     \cline{2-5}
    %     {  } & { Raspberry Pi 3B+ } & { 187.5 MB } & { 0.15 s } & { 727.5 mJ } & {  } \\
    %     \cline{2-5}
    %     {  } & { Raspberry Pi 4B } & { 187.4 MB } & { 0.07 s } & { 348.9 mJ } & {  } \\
    %     \hline
    % \end{tabular}
    \begin{tabular}{{|>{\centering}m{2cm}|>{\centering}m{2.7cm}|>{\centering}m{2.5cm}|>{\centering}m{1.9cm}|>{\centering}m{2cm}|>{\centering\arraybackslash}m{1.8cm}|}}
        \hline
        { Reference } & { Task } & { Architecture } & { Event/Frame data } & { Dataset } & {Framework} \\
        \hline
        { \cite{fra_human_2022} } & { Human activity recognition } & { LMU } & { Frame } & { \cite{Weiss2019a,Weiss2019} } & { \texttt{TensorFlow} } \\
        \hline
        { \cite{muller-cleve_braille_2022} } & { Braille letter reading } & { Fully connected } & { Both } & { \cite{muller-cleve_tactile_2022} } & { \texttt{PyTorch} } \\
        \hline
        { \cite{pedersen_neuromorphic_2024} } & { Braille letter reading } & { Fully connected } & { Event } & { \href{https://github.com/neuromorphs/NIR/tree/main/paper/03_rnn/data}{Braille subset for \cite{pedersen_neuromorphic_2024}} } & { \texttt{snnTorch} } \\
        \hline
        { \cite{wand_natively_2024} } & { Human activity recognition } & { L$^2$MU } & { Frame } & { \cite{Weiss2019a,Weiss2019} } & { \texttt{snnTorch} } \\
        \hline
        { \cite{meo_neu-brauer_2025} } & { Braille letter reading } & { Fully connected } & { Frame } & { \cite{muller-cleve_tactile_2022} } & { \texttt{snnTorch} } \\
        \hline
        { \cite{fra_win-gui_2025} } & { Spike pattern classification } & { Fully connected } & { Event } & { Spike patterns from \cite{Mihalas2009} } & { \texttt{snnTorch} } \\
        \hline
        { [NICE2025] } & { Braille letter reading } & { L$^2$MU } & { Event } & { \cite{muller-cleve_tactile_2022} and \href{https://github.com/neuromorphs/NIR/tree/main/paper/03_rnn/data}{Braille subset for \cite{pedersen_neuromorphic_2024}} } & { \texttt{snnTorch} } \\
        \hline
    \end{tabular}
\end{table*}

\section{Analysis}
\label{5 analysis}

\subsection{Simplifying Inputs Improves MT Readability}
\label{readability}

Simplification as an input rewriting strategy can balance translatability and meaning preservation, leading to overall improvements in translation quality. We also examine whether this enhances the readability of both inputs and, subsequently, translation outputs. In Table~\ref{tab:readability}, we present the Flesch Reading Ease score\footnote{\url{https://en.wikipedia.org/wiki/Flesch-Kincaid_readability_tests}} and Gunning Fog index\footnote{\url{https://en.wikipedia.org/wiki/Gunning_fog_index}} to measure input readability, and the Vienna formula (WSTF) \citep{zowalla2023readability} and the Russian version of Flesch Readability test \citep{inbook} to assess output readability for \textsc{En-De} and \textsc{En-Ru}, respectively.

As expected, input readability improves across all simplification methods, whether used in MT-Agnostic (\textsc{LLaMA-2}, \textsc{LLaMA-3}, and \textsc{Tower-Instruct} in Table~\ref{tab:readability}) or Translatability-Aware (Selection in Table~\ref{tab:readability}) manner. Interestingly, simplification not only leads to more readable input but also more readable outputs, with gains of up to 0.22 WSTF scores for \textsc{En-De} and 0.95 Flesch scores for \textsc{En-Ru}. We provide several qualitative examples in Appendix Tables \ref{tab:readability_ende} to \ref{tab:readability_enzh} that illustrate how simplification rewrites can lead to varying degrees of readability improvements in both inputs and translation outputs.


\begin{table}[!htp]
\centering
\resizebox{\linewidth}{!}{%
    \begin{tabular}{l l l l l l}
    \specialrule{1.3pt}{0pt}{0pt}
    \textbf{Language} & \textbf{Prompt/Model} & \textbf{Flesch} & \textbf{GFI} & \textbf{WSTF} & \textbf{Flesch-Ru} \\
    \toprule

    \multirow{5}{*}{\large \textbf{\Large{\textsc{en-de}}}} & Original & 60.79 & 10.56 & 1.35 & - \\
    & \textsc{LLaMA-2} & 66.69 & 9.25 & 1.15 & - \\
    & \textsc{LLaMA-3} & 64.00 & 9.98 & 1.24 & - \\
    & \textsc{Tower-Instruct} & \textbf{68.17} & \textbf{8.99} & \textbf{1.13} & - \\
    & Selection & 63.27 & 10.09 & 1.26 & - \\ \midrule

    \multirow{5}{*}{\large \textbf{\Large{\textsc{en-ru}}}} & Original & 69.93 & 9.91 & - & 65.67 \\
    & \textsc{LLaMA-2} & \textbf{74.73} & 8.37 & - & \textbf{66.62} \\
    & \textsc{LLaMA-3} & 72.88 & 9.20 & - & 66.36 \\
    & \textsc{Tower-Instruct} & 74.14 & \textbf{8.19} & - & 65.40 \\
    & Selection & 72.24 & 9.37 & - & 65.89 \\ \midrule

    \multirow{5}{*}{\large \textbf{\Large{\textsc{en-zh}}}} & Original & 66.51 & 10.08 & - & - \\
    & \textsc{LLaMA-2} & 71.64 & 8.74 & - & - \\
    & \textsc{LLaMA-3} & 69.32 & 9.48 & - & - \\
    & \textsc{Tower-Instruct} & \textbf{72.22} & \textbf{8.42} & - & - \\
    & Selection & 68.41 & 9.68 & - & - \\

    \specialrule{1.3pt}{0pt}{0pt}
    \end{tabular}
}
\caption{Input and output readability scores for simplification rewriting method. \textbf{Flesch}: Flesch Reading Ease score (↑); \textbf{GFI}: Gunning Fog Index (↓); \textbf{WSTF}: Vienna formula (↓); \textbf{Flesch-Ru}: Russian version of Flesch (↑).
}
\label{tab:readability}
\end{table}
\begin{table}[!htp]
\centering
\resizebox{\linewidth}{!}{%
    \begin{tabular}{l l l l l l l}
    \specialrule{1.3pt}{0pt}{0pt}
    \textbf{Language} & \textbf{Type} & \textbf{\textsc{x}$(s,t)$} & \textbf{\textsc{x}$(s,t,r)$} & \textbf{\textsc{M}$(s,t)$} & \textbf{\textsc{M}$(t,r)$}\\
    \toprule
        
    \multirow{5}{*}{\textbf{\large{\textsc{en-de}}}} & Original & 0.893 & 0.898 & 2.038 & 1.534 \\
    & \textbf{I} & \textbf{0.922} & \textbf{0.907} & \textbf{1.504} & 1.519 \\
    & \textbf{Owo} & 0.863 & 0.879 & 2.941 & 2.200 \\ 
    & \textbf{Ow} & 0.879 & 0.894 & 2.515 & 1.858 \\
    & \textbf{I+O} & 0.915 & \textbf{0.907} & 1.751 & \textbf{1.502} \\ \midrule

    \multirow{5}{*}{\textbf{\large{\textsc{en-ru}}}} & Original & 0.861 & 0.854 & 2.535 & 2.028 \\
    & \textbf{I} & \textbf{0.921} & 0.891 & \textbf{1.135} & \textbf{1.921} \\
    & \textbf{Owo} & 0.868 & 0.864 & 2.815 & 2.384 \\ 
    & \textbf{Ow} & 0.872 & 0.869 & 2.674 & 2.259 \\
    & \textbf{I+O} & 0.917 & \textbf{0.892} & 1.632 & 2.045 \\ \midrule

    \multirow{5}{*}{\textbf{\large{\textsc{en-zh}}}} & Original & 0.786 & 0.794 & 3.445 & \textbf{2.282} \\
    & \textbf{I} & \textbf{0.821} & 0.802 & \textbf{1.521} & 2.327 \\
    & \textbf{Owo} & 0.713 & 0.751 & 5.585 & 4.262 \\ 
    & \textbf{Ow} & 0.746 & 0.780 & 4.363 & 2.676 \\
    & \textbf{I+O} & 0.818 & \textbf{0.804} & 3.335 & 2.323 \\

    \specialrule{1.3pt}{0pt}{0pt}
    \end{tabular}
}
\caption{Results for input rewriting (\textbf{I}), post-editing output without source signal (\textbf{Owo}), with source signal (\textbf{Ow}), and the combination of both strategies (\textbf{I+O}). Best scores for each metric is \textbf{bold}. We use the same abbreviations for metrics as in Table \ref{tab:heldout}.} 
\label{tab:complementary}
\end{table}



\begin{figure*}
    \centering
    \includegraphics[width=\linewidth]{figures/human_dist.pdf}
    \caption{Win rates for human evaluation comparing Original MT vs. Rewrite MT across three language pairs (\textsc{En-De}, \textsc{En-Ru}, \textsc{En-Zh}) and four evaluation criteria: Fluency, Understandability, Level of detail (Detailed), and Meaning preservation relative to the reference translation.}
    \label{fig:human_results}
\end{figure*}


\subsection{Input Rewriting outperforms Post-Editing}
\label{res:post-editing}

The symmetric task to input rewriting is post-editing, which focuses on improving and correcting errors in translation outputs. Can post-editing alone achieve the same improvements, or are both strategies \textit{complementary}? To explore this, we compare input rewriting to post-editing by prompting \textsc{Tower-Instruct}\footnote{We focus on \textsc{Tower-Instruct} as it is a multilingual LLM capable of rewriting in non-English target languages.} to simplify either inputs or outputs. As shown in Table~\ref{tab:complementary}, rewriting inputs (\textbf{I}) offers a notable advantage over post-editing outputs (\textbf{Owo}), even when post-editing is guided by the input sentence (\textbf{Ow}). Combining input rewriting and post-editing (\textbf{I+O}) yields the highest translation quality, though the difference compared to input rewriting alone is not statistically significant. This confirms that rewriting text for better translatability before translation plays a more decisive role than post-editing the output.\footnote{We compare time and computational efficiency for input rewriting and output post-editing in Appendix \ref{appendix:inference_cost}.}




\subsection{Human Evaluation}
\label{human evaluation}


\paragraph{Original MT vs. Rewrite MT.}
We conduct a manual evaluation to determine whether bilingual human annotators rate translations generated using our winning rewrite method (simplification with \textsc{Tower-Instruct}) as superior to the original translations. For each language pairs (\textsc{En-De}, \textsc{En-Ru}, \textsc{En-Zh}), we randomly select 20 pairs of instances, resulting in a total of 180 annotations from three annotators per pair. Inter-annotator agreement, measured by Fleiss' Kappa\footnote{\url{https://en.wikipedia.org/wiki/Fleiss_kappa}}, is moderate, with values of 0.43, 0.39, and 0.51 for \textsc{En-De}, \textsc{En-Ru}, and \textsc{En-Zh}, respectively. For each instance, annotators are first provided with two translations and asked to evaluate on three axes: \textbf{1)} Fluency, \textbf{2)} Understandability, and \textbf{3)} Level of detail. Subsequently, we provide the reference translation, and annotators are asked to assess \textbf{4)} Meaning preservation. Annotators are also given the option to provide free form comments. Further details on the annotation set-up are available in Appendix \ref{appendix:mt_details}.

As illustrated in Figure~\ref{fig:human_results}, the human evaluation results confirm that translations from simplified inputs are rated as more fluent, understandable, and better at preserving the meaning of the reference translation. While this improvement is clear for the \textsc{En-De} and \textsc{En-Zh} pair, for \textsc{En-Ru} pair, annotators rate original MT as more fluent and more faithful to the original meaning.\footnote{Note that the Fleiss's Kappa scores indicate that there is more disagreement between annotators for \textsc{En-Ru} pair.} Some \textsc{En-Ru} annotators who preferred the original MT noted that it often retained a more accurate sense of the words in the reference. In contrast, those who favored the simplified rewrite MT highlighted that translations are more contextually appropriate, easier to read, and more comprehensible than the original MT.


\paragraph{Original vs. Rewrite.}
Our automatic meaning preservation metric evaluates the extent to which the original meaning is retained in the rewrite by comparing the rewritten source to the reference translation, rather than to the original source \citep{Graham2015CanMT}. Comparing to the original source is in the same language, but introduces a bias toward the original wording. On the other hand, comparing to the reference involves a cross-lingual comparison and is affected by unstable quality of references \citep{kocmi-etal-2022-findings}, but is less biased toward the original wording of the source.

To complement our automatic metric, we conduct a manual evaluation to assess how well the rewrites from simplification with \textsc{Tower-Instruct} preserve the meaning of the original source. We randomly sample 30 pairs of instances and collect three annotations per pair, totaling 90 annotations. Annotators are presented with both the original and rewritten sources and asked to evaluate how well the rewrite captures the meaning of the original source using a 4-point Likert scale (1: Does not capture meaning, 2: Partially, 3: Mostly, 4: Fully). Inter-annotator agreement by Fleiss' Kappa is 0.45. Of the 90 annotations, 55 were rated as 4, 27 as 3, 7 as 2, and 1 as 1, resulting an average score of 3.51. These results indicate that simplified rewrites generated by \textsc{Tower-Instruct}, although compared against the original source, still largely preserve the original meaning. Further details are provided in Appendix \ref{appendix:details}.
\section{Related Work}

\paragraph{Rewriting with LLMs.} 
Recent advances in LLMs have demonstrated impressive zero-shot capabilities in rewriting textual input based on user requirements \citep{shu2023rewritelm}. Most LLM-assisted rewriting tasks focus on query rewriting \citep{efthimiadis1996query}, which aims to reformulate text-based queries to enhance their representativeness and improve recall with retrieval-augmented LLMs \citep{mao-etal-2023-search, Zhu_2024}. Rewriting methods include prompting LLMs both as rewriters and rewrite editors \citep{ye-etal-2023-enhancing, kunilovskaya-etal-2024-mitigating}, and training LLMs as rewriters using feedback alignment learning \citep{ma-etal-2023-query, mao2024rafe}. Another line of work focuses on style transfer, where the goal is to rewrite textual input into a specified style \citep{wordcraft, hallinan2023steer}. Our research aligns with efforts to rewrite texts with LLM assistance; however, unlike these works, we focus on rewriting source inputs to enhance MT quality.

\paragraph{Quality Estimation Metrics.}
% These metrics use complex neural networks to estimate the quality of MT outputs more effectively. 
The discrepancy between lexical-based metrics (e.g., \textsc{BLEU} \citep{papineni-etal-2002-bleu}, \textsc{chrF} \citep{popovic-2015-chrf}) and human judgments \citep{ma-etal-2019-results} has led to research in \textit{neural} metrics. Particularly, quality estimation (QE) metrics, which compute a quality score for the translation conditioned only on the source sentence, have demonstrated benefits in improving MT quality. QE metrics are used for various purposes, including filtering out low-quality translations during training \citep{tomani2024qualityaware}, applying to post-editing workflows \citep{bechara2021role}, and providing feedback to users of MT systems \citep{mehandru2023physician}. In our experiments, we use \textsc{xCOMET} as our main evaluation metric, as it shows the best correlation with human judgments \citep{agrawal2024automatic}. We primarily use \textsc{xCOMET} as a QE metric to compute translatability, further providing this information as knowledge to LLMs to improve MT quality.

\paragraph{Rewriting MT Outputs.} 
The symmetric task of post-editing MT outputs has received significantly more attention than rewriting MT inputs. Most recent work relies on LLMs to automatically detect and correct errors in MT outputs using their internal knowledge \citep{raunak-etal-2023-leveraging, zeng2024improving, chen2024iterative}, with the help of external feedback \citep{ki2024guiding, xu2024llmrefine} or through fine-tuning \citep{treviso2024xtowermultilingualllmexplaining}. In contrast, the task of rewriting MT inputs to make them more suitable for translation has been relatively underexplored with LLMs. While there have been some efforts in query rewriting and style transfer to improve retrieval \citep{mao-etal-2023-search, Zhu_2024} and stylistic coherence \citep{ye-etal-2023-enhancing, hallinan2023steer}, the specific application of LLMs to rewrite inputs for the purpose of enhancing MT quality is still emerging. Our research addresses this gap by focusing on the potential of LLM-assisted input rewriting to improve the translatability and quality of the resulting translations.

% Traditional automatic metrics for MT evaluation rely on lexical-based approaches, calculating the evaluation score based on lexical overlap between a candidate translation and a reference translation (ex. \textsc{BLEU} \citep{papineni-etal-2002-bleu}, \textsc{METEOR} \citep{banerjee-lavie-2005-meteor}, and \textsc{chrF} \citep{popovic-2015-chrf}). However, evidence indicate that these lexical metrics do not consistently correlate with human judgments \citep{ma-etal-2019-results}. This discrepancy led to

% \mc{This paragraph is a little weak --- generic overview of QE methods. It would be much stronger to focus instead on discussing how QE has been sued to improve MT quality in the past.}

% \paragraph{Preference Alignment Learning.}
% An increasing body of work seeks to align LLMs with preference datasets, either collected from humans (RLHF, \citet{ouyang2022training}) or AI (RLAIF, \citet{bai2022constitutional}). However, RLHF procedure is complex, where we need to first fit a reward model over human preferences, and then use reinforcement learning (RL) algorithms such as Proximal Policy Optimization (\citet{schulman2017proximal}, PPO) to find a policy that maximizes the learned reward \citep{ziegler2020finetuning, ouyang2022training}. In contrast, reward-free methods offer simpler training procedure by directly training LLMs on preference without the need of reward modeling or RL \citep{yuan2023rrhf}. Among these, Direct Preference Optimization (\citet{rafailov2023direct}, DPO) has shown strong performance while being simple, thus we adopt DPO for our preference alignment learning.
\section{Conclusion and future work}
In this study, we examined the ability of LLMs to produce self-generated counterfactual explanations (SCEs).
We design a prompt-based setup for evaluating the efficacy of \SCEs.
Our results show that LLMs consistently struggle with generating valid \SCEs. In many cases model prediction on a \SCE does not yield the same target prediction for which the model crafted the \SCE.
Surprisingly, we find that LLMs put significant emphasis on the context---the prediction on \SCE is significantly impacted by the presence of original prediction and instructions for generating the \SCE.
Based on this empirical evidence, we argue that LLMs are still far from being able to explain their own predictions counterfactually.
Our findings add to similar insights from recent studies on other forms of self-explanations~\cite{lanham2023measuring,tanneru2024quantifying}.



Our work opens several avenues for future work. Inspired by counterfactual data augmentation~\cite{sachdeva2023catfood}, one could include the counterfactual explanation capabilities a part of the LLM training process. This inclusion may enhance the counterfactual reasoning capabilities of the LLM. Follow ups should also explore the effect of prompt tuning, specifically, model-tailored prompts for generating \SCEs. These approaches might lead to better quality \SCEs.


We limited our investigation to open source models of upto 70B parameters. Extending our analysis to larger and more recent models, \eg, DeepSeek R1 671B, and closed source models like OpenAI o3 would be an interesting avenue for future work.

Finally, our experiments were limited to relatively simple tasks: classification and mathematics problems where the solution is an integer. This limitation was mainly due to the fact that it is difficult to automatically judge validity of answers for more open-ended language generation tasks like search and information retrieval. Scaling our analysis to such tasks would require significant human-annotation resources, and is an important direction for future investigations.

\section{Limitations}


\paragraph{Probing Tasks} While the probing tasks we have proposed provide valuable insights into the visual arithmetic capabilities of VLMs, it is important to acknowledge that they may not encompass all possible dimensions of visual reasoning. Our choice to limit the scope of these tasks was intentional, as they serve as initial, simple tests to determine whether VLMs exhibit failure in fundamental aspects of visual arithmetic. These tasks allow us to iterate different experiments in a controlled and efficient manner, providing clear, actionable insights without the complexity that more comprehensive tasks might introduce. However, there is potential to explore additional tasks that involve more complex interactions of basic geometric properties. For instance, tasks requiring the model to simultaneously assess both length and angle, or combinations of length and area, could be valuable for understanding the compositionality of these atomic tasks. \looseness=-1

\paragraph{Training Data Synthesis}The training data synthesis method of \method~ is not only scalable but also effectively enhances the visual arithmetic capabilities of VLMs. Our approach serves as a proof-of-concept, demonstrating the potential of automated data generation for improving models' understanding of basic geometric properties. To further enrich the training data, we could consider utilizing additional configurations for each task. For instance, in generating positive and negative responses, we could leverage LLMs to produce rationales based on the specific configuration of each figure. By including explanations or justifications for why a particular geometric property holds or does not hold, we could foster deeper understanding within the VLMs. \looseness=-1


% \input{page/08 societal_concern}
\section*{Acknowledgments}

We thank the anonymous reviewers and the members of the \textsc{clip} lab at University of Maryland for their constructive feedback. This work was supported in part by NSF Fairness in AI Grant 2147292, by the Institute for Trustworthy AI in Law and Society (TRAILS), which is supported by the National Science Foundation under Award No. 2229885, and by the Office of the Director of National Intelligence (ODNI), Intelligence Advanced Research Projects Activity (IARPA), via the HIATUS Program contract \#2022-22072200006, by NSF grant 2147292. The views and conclusions contained herein are those of the authors and should not be interpreted as necessarily representing the official policies, either expressed or implied, of ODNI, IARPA, NSF or the U.S. Government. The U.S. Government is authorized to reproduce and distribute reprints for governmental purposes notwithstanding any copyright annotation therein.


\bibliography{custom,inputrewrite}

\appendix


\section{Model and Experiment Details}
\subsection{Prompt Templates}
\label{appendix:prompt_templates}
In Tables \ref{tab:prompting_template} and \ref{tab:training_template}, we describe the prompt templates used for prompting and fine-tuning experiments, respectively. For stylistic rewriting, we use the same prompts as those used to train the \textsc{CoEdIT-XL} model. During prompting, we provide the original source as the input, while for fine-tuning, we provide the positive rewrite along with the source.


\subsection{Training Setup}
\label{appendix:parameters}
All models are trained using one NVIDIA RTX A5000 GPU. In practice, we find that fine-tuning converges in around 3 hours. We use a 90/10 train/validation data split and adopt QLoRA \citep{dettmers2023qlora}, a quantized version of LoRA \citep{hu2021lora}, for parameter-efficient training. We train \textsc{Tower-Instruct 7B} with 8-bit quantization, a LoRA rank of 16, a scaling parameter ($\alpha$) of 32, and a dropout probability of 0.05 for layers. We train for 10 epochs. All unspecified hyperparameters are set to default values.


\subsection{Decoding Strategy}
We use greedy decoding (no sampling) when generating rewrites for prompting experiments. We fix the temperature value to 0 throughout the experiments in order to eliminate sampling variations.


\begin{table}
\centering
\resizebox{\linewidth}{!}{%
    \begin{tabular}{ll}
    \specialrule{1.3pt}{0pt}{0pt}
    \textbf{Model} & \textbf{HuggingFace Model Name} \\
    \toprule

    \textsc{LLaMA-2} & \texttt{meta-llama/Llama-2-7b-chat-hf} \\
    \textsc{LLaMA-3} & \texttt{meta-llama/Meta-Llama-3-8B-Instruct} \\
    \textsc{Tower-Instruct} & \texttt{Unbabel/TowerInstruct-7B-v0.1} \\
    \specialrule{1.3pt}{0pt}{0pt}
    \end{tabular}
}
\caption{HuggingFace model names for all tested LLMs.} 
\label{tab:huggingface_api}
\end{table}


\subsection{Dataset Details}
\label{appendix:dataset_details}
We provide detailed statistics of our training ($\mathcal{D}_{pos}$) and test dataset in Table \ref{tab:dataset_details}. For $\mathcal{D}_{pos}$, we only use rewrites where the \textsc{xCOMET}$(s', t')$ score is higher than the original \textsc{xCOMET}$(s, t)$ score. We further conduct a two-step pre-processing procedure: \textbf{1)} Remove duplicate instances and \textbf{2)} Remove lengthy instances where the upper threshold is set as Q$3 + 1.5 \times \text{IQR}$.


\section{Detailed Results}
\subsection{Full Results}
\label{appendix:detailed results}
In Tables \ref{tab:detailed_results_ende} to \ref{tab:detailed_results_enzh}, we present the detailed numerical results for all tested variations. Most rewrites yield higher \textsc{xCOMET}$(s,t)$ scores, indicating better translatability compared to the original baseline. For stylistic rewrites with \textsc{CoEdIT}, prompting to make the text easier to understand (Understandable) achieves the highest translatability score, while prompting to rewrite the text more formally (Formal) results in the highest translation quality. The Coherent prompt achieves the highest meaning preservation score but this is because most rewrites are merely copies of the original source (Appendix \ref{appendix:direct_copy}). Overall, we demonstrate that translatability-based selection method remains the most effective method, even outperforming scores from our fine-tuned LLMs.


\subsection{Impact of LLM}
\label{appendix:impact of llm}
Among the three LLMs used for prompting, \textsc{Tower-Instruct} performs the best in terms of the combined metric \textsc{xCOMET}$(s,t,r)$. Although it lags behind \textsc{LLaMA-2} and \textsc{LLaMA-3} in translatability, its meaning preservation score deteriorates the least, resulting in the highest overall score. \textsc{LLaMA-3} performs the best in terms of translatability, likely due to its more multilingual training data, with over 5\% of its pre-training dataset consisting of high-quality non-English data.\footnote{\url{https://ai.meta.com/blog/meta-LLaMA-3/}} This suggests that the amount of multilingual data in the pre-training phase may enhance the model's ability to generate more translatable rewrites. However, this advantage does not extend when comparing the \textsc{LLaMA} models to \textsc{Tower-Instruct}. Despite being inherently multilingual primarily trained on translation-related tasks, \textsc{Tower-Instruct} performs lower than the \textsc{LLaMA} models in translatability. This discrepancy can be attributed to \textsc{Tower-Instruct} not being specifically trained on rewriting tasks to improve MT quality, highlighting the importance of introducing translation-related knowledge for effective rewriting.


We further compare the results with off-the-shelf paraphrasing (\textsc{DIPPER}) and text-editing (\textsc{CoEdIT-XL}) tools. Despite being specifically trained for rewriting tasks, their rewrites are not as translatable as those generated by the prompted LLMs. For \textsc{DIPPER}, this may be due to its primary focus on paraphrasing, which has been shown to be less effective (\S \ref{simplification best}). In the case of \textsc{CoEdIT}, we attribute the lower performance to the model's smaller size (3B) compared to the 7B LLMs used for prompting.


\subsection{Same LLM vs. Different LLM}
\label{appendix:same llm}
We distinguish whether the LLM being prompted is the same as the one used as the MT system. Initially, we expected the highest improvements when prompting \textsc{Tower-Instruct}, which may incur self-preference bias, where the LLM favors its own outputs due to recognition \citep{panickssery2024llm}. However, our results indicate that prompting \textsc{Tower-Instruct} does not yield the most translatable rewrites. Instead, the LLaMA series models consistently outperform in this aspect. Interestingly, \textsc{Tower-Instruct} consistently produces rewrites that are more meaning-preserving compared to \textsc{LLaMA-2} or \textsc{LLaMA-3}, resulting in higher \textsc{xCOMET}$(s,t,r)$ scores overall. We conclude that prompting the same LLM used for the MT system is not helpful in generating more translatable rewrites, but these rewrites are better at preserving the intended meaning.


% \section{Re-ranking Details}
% \subsection{Top-1 Rewrites}
% A natural question that arises from using a re-ranker is whether the higher-ranked rewrites are consistent across language pairs. To find this, we track the source of the Top-1 rewrites. We increment a count for each rewriting method if the Top-1 rewrite originates from that method. If the Top-1 rewrite is coming from multiple rewriting methods, we count them towards both. As illustrated in Table \ref{tab:top1_rewrites}, Top-3 rewrites are coming from the same rewriting methods across tested language pairs are: simplification \small(\textsc{LLaMA-3})\normalsize, paraphrase \small(\textsc{LLaMA-3})\normalsize, and stylistic \small(\textsc{CoEdIT} Understandable)\normalsize. Thus, the most effective prompting rewriting methods are generalizable across different languages.

% \begin{table}[!htp]
\centering
\resizebox{\linewidth}{!}{%
    \begin{tabular}{l l l l l}
    \toprule
    \multirow{2}{*}{\textbf{Rewrite}} & \multirow{2}{*}{\textbf{Prompt/Model}} & \multicolumn{3}{c}{\textbf{Count (Rank)}} \\
    
    \cmidrule(lr){3-5}
    
    & & \textbf{\textsc{EN-DE}} & \textbf{\textsc{EN-RU}} & \textbf{\textsc{EN-ZH}}  \\ \midrule

    \multirow{3}{*}{\textbf{Simplification}} & LLaMA-2 & 303 (6) & 568 (8) & 397 (8) \\
    & LLaMA-3 & \underline{380 (2)} & \textbf{679 (1)} & \underline{529 (2)} \\
    & Tower-Instruct & 191 (10) & 465 (10) & 277 (10) \\ \midrule

    \multirow{6}{*}{\textbf{Paraphrase}} & LLaMA-2 & 314 (5) & 629 (4) & 413 (6) \\
    & LLaMA-3 & \textit{357 (3)} & \underline{675 (2)} & \textit{483 (3)} \\
    & Tower-Instruct & 174 (11) & 395 (12) & 227 (11) \\
    & \textsc{DIPPER} (L80/O60) & 319 (4) & 588 (5) & 469 (4) \\
    & (L80/O40) & 289 (8) & 583 (6) & 443 (5) \\
    & (L60/O40) & 248 (9) & 588 (5) & 483 (3) \\

    \midrule

    \multirow{4}{*}{\textbf{Stylistic}} & \textsc{CoEdIT} (GEC) & 158 (13) & 403 (11) & 202 (12) \\
    & (Coherent) & 160 (12) & 358 (13) & 196 (13) \\
    & (Understandable) & \textbf{492 (1)} & \textit{639 (3)} & \textbf{591 (1)} \\
    % & (Paraphrase) & 253 (9) & 525 (9) & 280 (10) \\
    & (Formal) & 302 (7) & 574 (7) & 402 (7) \\
    
    
    \bottomrule
    \end{tabular}
}
\caption{Count of Top-1 rewrites per prompting variation. Highest rank is marked in \textbf{bold}, second in \underline{underline}, and third as \textit{italic}.}
\label{tab:top1_rewrites}
\end{table}


% \subsection{Top-$k$ Rewrites}
% \label{appendix:top k rewrites}
% Our default re-ranker considers all 13 MT-agnostic rewriting methods. However, in practical scenarios, generating a large number of rewrites incurs significant costs. To simulate this, we limit our re-ranker to consider only the Top-$k$ rewrites, as outlined in Table \ref{tab:top-k rerank}. Figure \ref{fig:topk_rerank} illustrates the impact of varying $k$ on the \textsc{xCOMET}$(s,t)$ score. We observe that increasing the number of rewriting methods (higher $k$) improves translatability, as evidenced by a consistent increase in the \textsc{xCOMET}$(s,t)$ score. However, this improvement diminishes gradually, particularly for the \textsc{En-De} pair (0.027 → 0.013 → 0.006), where the score converges around Top-3. Overall, this underscores that re-ranking using a small subset of the best rewriting methods is sufficient.

% \begin{table*}[!htp]
\centering
\resizebox{\textwidth}{!}{%
    \begin{tabular}{c l}
    \toprule
    \textbf{Top-k} & \textbf{Rewrite Methods} \\ \midrule
    
    2 & \textsc{CoEdIT} \small{(Understand)}\normalsize{, Simplification} \small{(LLaMA-3)} \\
    3 & \textsc{CoEdIT} \small{(Understand)}\normalsize{, Simplification} \small{(LLaMA-3)}\normalsize{, Paraphrase} \small{(LLaMA-3)} \\
    4 & \textsc{CoEdIT} \small{(Understand)}\normalsize{, Simplification} \small{(LLaMA-3)}\normalsize{, Paraphrase} \small{(LLaMA-3)}\normalsize{, \textsc{DIPPER}} \small{(L80/O60)} \\
    5 & \textsc{CoEdIT} \small{(Understand)}\normalsize{, Simplification} \small{(LLaMA-3)}\normalsize{, Paraphrase} \small{(LLaMA-3)}\normalsize{, \textsc{DIPPER}} \small{(L80/O60)}\normalsize{, Paraphrase} \small{(LLaMA-2)} \\
    
    \bottomrule
    \end{tabular}
}
\caption{Top-$k$ rewrite methods tested.}
\label{tab:top-k rerank}
\end{table*}


\section{Qualitative Evaluation}
\label{appendix: qualitative eval details}
\subsection{Copying Behavior}
\label{appendix:direct_copy}
To prevent LLMs from directly copying the original source, we explicitly state in the prompt to ``\textit{avoid directly copying the source}'' (Appendix \ref{appendix:prompt_templates}). However, we still observe some rewrites that are identical to the source sentence. We count the occurrences and compute the percentage per language pair in Table \ref{tab:direct_copy}. Note that we do not consider Translatability-Aware Selection rewrite method here since this involves selecting whether to keep the original source or use the rewrite based on translatability scores. The highest occurrence appears for stylistic rewrites using the \textsc{CoEdIT-XL} Coherent prompt, where the source is copied most of the time (82.2\%, 91.9\%, 93.2\% for \textsc{En-De}, \textsc{En-Ru}, and \textsc{En-Zh}, respectively).

\begin{figure}
    \centering
    \includegraphics[width=\linewidth]{figures/success_dist.pdf}
    \caption{Distribution of properties of good rewrites.}
    \label{fig:success}
\end{figure}

\subsection{What makes a Good Rewrite for MT?}
Qualitatively examining translation outputs reveals several common patterns, which motivate us to conduct a detailed qualitative analysis. Here, we aim to identify the properties that lead to meaning-equivalent rewrites that are easier to translate. We examine 200 data instances where each rewrite is the highest performing rewrite based on the \textsc{xCOMET}$(s,t)$ score. To focus on successful rewrites, we filter instances where \textsc{xCOMET}$(s',t')$ $>$ \textsc{xCOMET}$(s,t)$. Each rewrite is annotated with the following labels: (1) \textbf{Simplified}: Replaces complex words with simpler ones or reduces structural complexity; (2) \textbf{Detailed}: Adds information for better context; (3) \textbf{Fluency}: Restructures the sentence for better flow and readability. 
Examples of rewrites for each annotation label are in  Table \ref{tab:success_types}. 

As shown in Figure \ref{fig:success}, most successful rewrites are labeled as \textbf{Simplified}. This highlights the effectiveness of simplification, which has been consistently effective even in the context of LLMs. Notably, many simplified rewrites involve changing complex words to simpler, more conventional alternatives (e.g., ``Derry City \textit{emerged victorious} in the President's Cup as they \textit{ran out} 2-0 \textit{winners} over Shamrock Rovers.'' → ``Derry City \textit{won} the President's Cup title by \textit{defeating} Shamrock Rovers 2-0.''). This finding aligns with our conclusions from MT-Agnostic rewriting methods (\S \ref{3.1 mt-agnostic}), where simplification emerged as the best rewrite method among the prompting variations.


\section{Additional Results}
\subsection{Additional LLM Baselines}
\label{appendix:more_llms}
\paragraph{LLMs for Rewriting.}
Our initial experiments consist of 21 input rewriting methods across 3 LLMs (\textsc{LLaMA-2 7B}, \textsc{LLaMA-3 8B}, and \textsc{Tower-Instruct 7B}). In Table \ref{tab:more_rewrite_llms}, we present extended experiment results by applying simplification rewriting with two additional LLMs: \textsc{Aya-23 8B} \citep{aryabumi2024aya23openweight} and \textsc{Tower-Instruct 13B} \citep{alves2024tower}. The results confirms that simplification rewriting improves translation quality measured by \textsc{xCOMET}$(s,t,r)$ compared to the original baseline.

\paragraph{LLMs for MT.}
Furthermore, we initially relied on \textsc{Tower-Instruct 7B} as our MT system for all our experiments since it is specifically trained for translation-related tasks and has demonstrated superior MT performance (\S \ref{3 method}). However, we extend our analysis by comparing the original baseline and our winning strategy (simplification with \textsc{Tower-Instruct 7B}) using two additional LLMs as the MT system. As shown in Table \ref{tab:more_mt_llms}, our method outperforms the original baseline in terms of both the translation quality (\textsc{xCOMET}$(s,t,r)$) and \textsc{MetricX}$(s,t)$, regardless of the LLM used as the MT system.


\subsection{Additional Language Pairs}
\label{appendix:more_lang_pairs}
To assess the generalizability to other source languages, we test two of our winning strategies (simplification with \textsc{Tower-Instruct 7B} and inference-time selection) on seven additional into-English and non-English language pairs from the WMT-23 General MT task test set.\footnote{\url{https://www2.statmt.org/wmt23/translation-task.html}} As shown in Table \ref{tab:more_lang_pairs}, while translatability scores (\textsc{xCOMET}$(s,t)$) improve across all language pairs, translation quality (\textsc{xCOMET}$(s,t,r)$) improvements are less pronounced compared to out-of-English pairs. Notably, gains in translation quality are observed only for German-English (\textsc{De-En}) and Chinese-English (\textsc{Zh-En}) pairs. These results highlight the importance of input rewrites' quality, which is currently higher for high-resource source languages. This motivates further work to strengthen input rewriting for broader range of source languages.



% \section{Direct Preference Optimization}
% \label{appendix:dpo}
% \subsection{Preference Dataset}
% Further motivated by recent efforts to utilize preference learning strategies \citep{rafailov2023direct, xu2024contrastive, xu2024advancing, he2024improving}, we collect triplets of dispreferred (negative) and preferred (positive) rewrites for each source sentence via pairwise comparisons. We then use this signal with Direct Preference Optimization (\citet{rafailov2023direct}, DPO) to further align the initial fine-tuned model with an MT-based objective. Direct Preference Optimization (\citet{rafailov2023direct}, DPO) needs input $x$ and preferred/dispreferred outputs $y_w$/$y_l$, where $y_w > y_l$ amongst preference pair. We set $x$ as the above prompt templates with corresponding inputs from $\mathcal{D}_{ft}$. For each source sentence $s$, we collect a triplet of source, positive (preferred), and negative (dispreferred) rewrite $(s, s'_p, s'_n)$ where we define positive rewrites as rewrites having higher \textsc{xCOMET}$(s,t)$ score and negative rewrites as those having lower score than the original. We consider all combinations of positive and negative rewrites, thus there can be multiple triplets for the same source sentence. We filter to only use unique triplet combination in our final preference dataset $\mathcal{D}_{dpo}$. We conduct the same 2-step pre-processing procedure taken for the fine-tuning dataset $\mathcal{D}_{pos}$: 1) Remove duplicated instances; and 2) Remove lengthy instances where the upper threshold is set as Q$3 + 1.5 \times \text{IQR}$. We show the number of instances in Table \ref{tab:dpo_dataset_details}.

% \begin{table}[!htp]
\centering
\resizebox{200}{!}{%
    \begin{tabular}{l l l}
    \toprule
    \textbf{Dataset} & \textbf{\# Sentences}  \\ \midrule
    
    $\mathcal{D}_{dpo}$ (English-German) & 38,716 \\ 
    $\mathcal{D}_{dpo}$ (English-Russian) & 34,178 \\
    
    \bottomrule
    \end{tabular}
}
\caption{Summary statistics of DPO preference dataset.}
\label{tab:dpo_dataset_details}
\end{table}

% With this data, we update our initial model $\pi_0$  to align better rewrite model $\pi_{dpo}$ with DPO. We explore the same prompt template variants as in the fine-tuning setting (Appendix \ref{appendix:prompt_templates}).



% \subsection{Training Setup}
% We find that DPO converges in approximately 10 hours. We use a 90/10 train/validation data split and start with a supervised fine-tuned model as our initial model, $\pi_0$. We perform DPO with a training batch size of 4, a beta value of 0.1, a learning rate of 2e-4, a warm-up ratio of 0.05, a cosine learning rate scheduler, a maximum gradient norm clipping value of 0.3, a maximum prompt length of 1024 tokens, and 10 training epochs. All unspecified hyperparameters are set to their default values.

% \subsection{Results}
% In Tables \ref{tab:detailed_results_ende} to \ref{tab:detailed_results_enzh}, we compare DPO results to other rewriting methods. While previous works \citep{rafailov2023direct, wang2023making, tunstall2023zephyr} have demonstrated the effectiveness of DPO in aligning LLMs with preference datasets, we find that alignment-based learning is less effective for generating better rewrites for MT. Although \textsc{xCOMET}$(s,t)$ scores improve by +0.59 and +0.71 for \textsc{En-De} and \textsc{En-Ru}, respectively, \textsc{xCOMET}$(s,r)$ scores decrease by even wider margins. Among the three prompt template variations (Basic, MT, Ref), providing both the source and MT (MT) is most effective. We attribute this to the use of both preferred and dispreferred rewrites in DPO, suggesting that providing the MT as context for comparison may be helpful. Overall, DPO outperforms the baseline, but translatability-based selection remains the best rewriting method.


\section{Human Annotation Details}
\label{appendix:annotation details}
We use Qualtrics\footnote{\url{https://www.qualtrics.com}} to design our survey and Prolific\footnote{\url{https://www.prolific.com}} to recruit human annotators fluent in the tested target language.

\subsection{Original MT vs. Rewrite MT Details}
\label{appendix:mt_details}
We randomize the order of the two sentences (original MT and rewrite MT) to mitigate position bias. Annotators evaluate which sentence is better across four dimensions: fluency, understandability, level of detail, and meaning preservation. The entire survey is estimated to take approximately 20 minutes to complete. We recruit a total of 9 annotators and provide a compensation of 5 US dollars per survey (15 US dollars/hr), totaling 45 US dollars.


\subsection{Original vs. Rewrite Details}
\label{appendix:details}
Each annotator is tasked to judge how well the rewritten sentence preserves the meaning of the original source sentence. The survey is estimated to take approximately 30 minutes to complete. We recruit a total of 3 annotators. We offer a compensation of 7.5 US dollars per survey (15 US dollars/hr), totaling 22.5 US dollars.


\subsection{Annotator Instructions}
In Figures \ref{fig:human_intro} to \ref{fig:human_example_3}, we present the instructions and survey content provided to annotators. For the Original MT vs. Rewrite MT evaluation, each annotator reviews 20 sets of examples. Each question consists of two parts: \textbf{1)} comparing the two sentences based on fluency, understandability, and level of detail, and \textbf{2)} selecting which sentence better preserves the meaning of the reference translation. For the Original vs. Rewrite evaluation, each annotator reviews 30 sets of examples. Additionally, a free-form text box is provided alongside each example for annotators to offer feedback or suggestions.


\section{Time \& Computational Efficiency}
\label{appendix:inference_cost}
We show that on average, rewriting with our winning strategy is not a resource-intensive option for downstream applications in terms of both time and computation. For approximately 1.5K sentences, the rewrite and MT pipeline using our winning strategy (simplification with \textsc{Tower-Instruct 7b} takes 1 hour, compared to 30 minutes for the MT process alone. All variants of our prompting experiments are conducted using a single NVIDIA RTX958 A5000 GPU. In terms of efficiency compared to automatic post-editing (\S \ref{res:post-editing}), both approaches remains equivalent in time and computational requirements since the rewriting or post-editing process only differs in its position within the pipeline. Input rewriting modifies the source before the MT system, while output post-editing adjusts the translation after the MT system.


% \subsection{Annotator Feedback Details}
% \label{appendix:further comments}
% In our survey, we include an optional question for each example to collect additional feedback or comments from the annotators. We present some of this feedback in Table \ref{tab:annotator_feedback}. Annotators commented that they prefer the translations of rewrites over the original translations because 1) they are easier to understand; 2) they contain words that better suit the target language context; and 3) they are more precise. These comments align with the higher winning rates in the fluency, understandability, and detail dimensions.


\begin{table*}[!htp]
\centering
\resizebox{\textwidth}{!}{%
    \begin{tabular}{l l}
    \specialrule{1.3pt}{0pt}{0pt}
    \textbf{Rewrite} & \textbf{Prompt} \\ \midrule
    
    \multirow{4}{*}{\textbf{Simplification}} & Simplify the English sentence. Simplification may include identifying complex words and replacing with simpler \\
    & or shorter words or using active voice instead of passive voice. Try to keep the meaning of the Original sentence. \\
    & Original: \textit{This is a very nice skirt. The lacy pattern is classy and soft.} \\
    & Simplified: \\ \midrule

    \multirow{3}{*}{\textbf{Paraphrase}} & Paraphrase the English sentence. Try to not directly copy but keep the meaning of the Original sentence. \\
    & Original: \textit{This is a very nice skirt. The lacy pattern is classy and soft.} \\
    & Paraphrase: \\ \midrule

    \multirow{4}{*}{\textbf{Stylistic} (\textsc{CoEdIT})} & \textbf{(GEC)} Fix the grammar: \\
    & \textbf{(Coherent)} Make this text coherent: \\
    & \textbf{(Understandable)} Rewrite to make this easier to understand: \\
    % & \textbf{(Paraphrase)} Paraphrase this: \\
    & (\textbf{Formal)} Write this more formally: \\ \midrule

    \multirow{3}{*}{\textbf{Easy Translation}} & Rewrite the Original sentence to be easier for translation in {target language}. New sentence should be in English. \\
    & Original: \textit{This is a very nice skirt. The lacy pattern is classy and soft.} \\
    & New: \\ \midrule

    \multirow{8}{*}{\textbf{CoT}} & \textbf{(Step 1)} Rewrite the Original English sentence to New English sentence that translates better into German. \\
    & Avoid directly copying the Original sentence while keeping its meaning. New sentence should be in English. \\
    & Original: \textit{This is a very nice skirt. The lacy pattern is classy and soft.} \\
    & New: \\
    
    % \cmidrule(lr){2}
    \\
    & \textbf{(Step 2)} Now, translate the English sentence to German. \\
    & English: \\
    & German: \\
    
    \specialrule{1.3pt}{0pt}{0pt}
    \end{tabular}
}
\caption{Exemplar prompt templates for English-German language pair used for prompting experiments. \textit{Italic} represents the source sentence used in this example.}
\label{tab:prompting_template}
\end{table*}
\definecolor{light gray}{rgb}{0.898, 0.902, 0.906}
\definecolor{light pink}{rgb}{0.996, 0.9102, 0.9219}


\begin{table*}[htbp]
\centering
\resizebox{\textwidth}{!}
{
    \begin{tabular}{l}
    \specialrule{1.3pt}{0pt}{0pt}
    \textbf{Basic} \\ \midrule
    
    {\textsc{\#\#\#} \textbf{Instruction:}} Rewrite this English sentence to give a better translation. {\texttt{\textbackslash n\textbackslash n}} \\
        
    {\textsc{\#\#\#} \textbf{English}:} This is a very nice skirt. The lacy pattern is classy and soft.{\texttt{\textbackslash n}} \\
    
    {\textsc{\#\#\#} \textbf{English rewrite}:} The lacy pattern on this skirt is elegant and soft. \\
    \midrule

    \textbf{MT} \\ \midrule
    
    {\textsc{\#\#\#} \textbf{Instruction:}} Rewrite this English sentence to give a better translation in German. German sentence is the hypothesis translation that \\
    we are trying to improve.{\texttt{\textbackslash n\textbackslash n}} \\
    
    {\textsc{\#\#\#} \textbf{English}:} This is a very nice skirt. The lacy pattern is classy and soft.{\texttt{\textbackslash n}} \\

    {\textsc{\#\#\#} \textbf{German}:} Das ist eine sehr schöne Röhre. Das schicke Spitzenmuster ist weich und elegant.{\texttt{\textbackslash n}} \\
    
    {\textsc{\#\#\#} \textbf{English rewrite}:} The lacy pattern on this skirt is elegant and soft. \\
    \midrule

    \textbf{Reference} \\ \midrule
    
    {\textsc{\#\#\#} \textbf{Instruction:}} Rewrite this English sentence to give a better translation in German. German sentence is the human-annotated translation \\
    that we are trying to pursue.{\texttt{\textbackslash n\textbackslash n}} \\
    
    {\textsc{\#\#\#} \textbf{English}:} This is a very nice skirt. The lacy pattern is classy and soft.{\texttt{\textbackslash n}} \\

    {\textsc{\#\#\#} \textbf{German}:} Das ist ein sehr schöner Rock. Das Spitzenmuster ist stilvoll und weich.{\texttt{\textbackslash n}} \\
    
    {\textsc{\#\#\#} \textbf{English rewrite}:} The lacy pattern on this skirt is elegant and soft. \\
    \specialrule{1.3pt}{0pt}{0pt}
    \end{tabular}
}
\caption{Exemplar prompt templates for supervised fine-tuning experiments (English-German pair). We additionally give machine translation for the \textbf{MT} prompt and reference translation for the \textbf{Reference} prompt after \textsc{\#\#\#} \textbf{German:}.}
\label{tab:training_template}
\end{table*}



% \begin{prompt}[title={Prompt A.1.1. Basic SFT}]
% \textbf{\#\#\# Instruction:} Rewrite the English sentence to give a better translation.\\ \\
% \textbf{\#\#\# English:} \texttt{\{source\}} \\
% \textbf{\#\#\# English rewrite:} \texttt{\{rewrite\}} \\
% \end{prompt}
\clearpage

% \begin{table*}[!htp]
\centering
\resizebox{\textwidth}{!}{%
    \begin{tabular}{c l}
    \toprule
    \textbf{Top-k} & \textbf{Rewrite Methods} \\ \midrule
    
    2 & \textsc{CoEdIT} \small{(Understand)}\normalsize{, Simplification} \small{(LLaMA-3)} \\
    3 & \textsc{CoEdIT} \small{(Understand)}\normalsize{, Simplification} \small{(LLaMA-3)}\normalsize{, Paraphrase} \small{(LLaMA-3)} \\
    4 & \textsc{CoEdIT} \small{(Understand)}\normalsize{, Simplification} \small{(LLaMA-3)}\normalsize{, Paraphrase} \small{(LLaMA-3)}\normalsize{, \textsc{DIPPER}} \small{(L80/O60)} \\
    5 & \textsc{CoEdIT} \small{(Understand)}\normalsize{, Simplification} \small{(LLaMA-3)}\normalsize{, Paraphrase} \small{(LLaMA-3)}\normalsize{, \textsc{DIPPER}} \small{(L80/O60)}\normalsize{, Paraphrase} \small{(LLaMA-2)} \\
    
    \bottomrule
    \end{tabular}
}
\caption{Top-$k$ rewrite methods tested.}
\label{tab:top-k rerank}
\end{table*}
\section{Dataset Details}
Below we report the data statistics of our \dataname dataset, detail the data preprocessing steps, pipeline configurations used for each dataset in our experiments, and additional data pipeline experiments that utilize 3D instance mask predictions in caption generation process.

\subsection{Data Statistics}
In Tab.~\ref{tab:datasets}, we report the statistics of our generated dataset, including the number of scenes, RGB-D frames, generated captions, and total tokens in captions for each source dataset. Our dataset contains over 30K scenes, and 5.6M captions with a total of 30M tokens across both real and synthetic indoor environments.
\section{Dataset Generation}
\label{sec:dataset}
\revise{
To train the proposed GNN, we constructed a dataset of building structures and a subset of these structures were subjected to fire simulations using FEA. The dataset generation process is illustrated in \figref{fig:dataset_generation_procedure}. Initially, a total of 33,000 building structures with geometrical details, material properties, and gravity loads were created. Due to randomness in generating these structures, a filter is applied to remove unreasonable data after gravity load simulation, which included 15,377 structures. A trade-off between computational feasibility and model performance is made among the remaining 17,623 structures. As further labeling structures with MIDR requires resource-intensive fire simulations via OpenSeesRT, a large proportion of 16,050 structures is selected as unlabeled dataset. On the other hand, each of the other 1,573 structures was further subjected to 30 different fire simulations, forming the labeled dataset containing $1,573\times 30 = 47,190$ fire cases.} This section details the step-by-step process for generating the dataset, including geometry creation, material property assignment, and simulations due to gravity loads and fire scenarios. 
% To train the proposed neural network, we constructed a dataset comprising building structure data and a subset of fire scenario data. The dataset generation process is illustrated in \figref{fig:dataset_generation_procedure}. 
% A total of 33,000 building structures with geometric details, material properties, and gravity loads were initially created. Out of these, 3,000 structures were selected as labeled data, and the remaining 30,000 were designated as unlabeled data. Further, about half of them filtered out due to instability under gravity loads only. 
\begin{figure*}[h!]
    \centering
    \includegraphics[width=0.8\linewidth]{figures/dataset_filter_procedure.pdf}
    \caption{Workflow for dataset generation (geometry, material property, gravity loads, and fire scenarios).}
    \label{fig:dataset_generation_procedure}
\end{figure*}

\subsection{Geometry Generation}
\label{subsec:geometry_generation}
The geometry of the building structures forms the foundation of the dataset. Regular 
\revise{3D structures} resembling multi-story parking structures or shopping malls were generated, with parameters such as building floor dimensions and story heights selected randomly. Each building structure is composed of multiple rooms, which serve as the basic unit in this study. A room herein is a cuboid space defined by specific length, width, and height. Within a structure, rooms of the same dimensions are uniformly arranged along the length, width, and height, corresponding to the $x$-, $y$-, and $z$-axes, respectively. Structures vary in room size and number of rooms along each axis. Specifically, the room length, width, and height are independently sampled from a uniform distribution within the interval $[2, 5]$ meters along the three directions of the structure. Similarly, the room number along each axis is uniformly sampled independently as an integer within the interval $[2, 7]$, i.e., the maximum number of stories of the buildings simulated in this study is 7.

To introduce variability and simulate real-world scenarios, approximately $8\%$ of structural elements (beams or columns) are randomly removed after initial geometry creation. 
\revise{Such removal is not fire-induced damage, but reflects functional diversity often observed in real buildings, such as open spaces designed for activities in shopping malls, e.g., ice skating rinks. Examples of the generated geometries are illustrated in \figref{fig:example_generated_geometry}, showcasing the diversity and realism of the dataset. This element removal does not affect the definition of room's geometry in the structure and nor does it affect the number of considered fire scenarios.} 

\revise{A range of coefficient of variation values ($3.3\%$ to $17.5\%$) was derived from prior studies that investigated the statistics of geometrical and material properties of structural components of buildings (e.g., \cite{mirza1979variations, lee2004probabilistic}). These studies provide empirical data on the natural variability in parameters such as Young's modulus, yield strength, and dimensions of structural elements due to manufacturing tolerances and material inconsistencies. By selecting $8\%$ for the removal of structural elements in our database, we aimed to maintain a level of variability that is representative of real-world uncertainties while ensuring computational feasibility. This choice ensures that the database captures realistic deviations without introducing extreme cases that may not be commonly encountered in practice.}

\begin{figure*}[h!]
    \centering
    \includegraphics[width=\linewidth]{figures/example_generated_geometry.pdf}
    \caption{Examples of generated structural geometry of different sizes (all dimensions in meters).}
    \label{fig:example_generated_geometry} 
\end{figure*}

{\blockRevise

In this study, we opted for a deterministic square, dimension of $0.1$ m, solid cross-sectional steel elements due to their simplicity in modeling and analysis. Square sections exhibit uniform geometrical properties in all directions, simplifying the computation of structural responses and avoiding complications associated with more complex shapes, such as wide-flange sections, facilitating the computational efficiency and scalability to generate a large dataset. This choice also helps to mitigate issues related to stress concentrations and facilitates a more straightforward representation of structural behavior under thermal loads. 

\textit{Remark:} The selected cross-section provides a comparable flexural rigidity to a $W 130 \times 130 \times 28.1$ wide-flange section (metric units), albeit with significantly higher axial rigidity. This cross-section is acceptable for gravity-load-designed frames under service loading conditions where the models assume fully rigid, moment-resisting beam-column connections for the evaluation of the IDR under thermal loading. This assumption is reasonable in this computational study where the primary interest is to understand the global deformation response of frames under fire conditions. The selection of uniform square cross-sections for both beams and columns, rather than adherence to standard capacity design principles, was made here primarily for computational efficiency and to reduce design parameters in the database generation process. This choice allows for simplified and scalable approach to analyze the fire-induced response of generic steel frames without the need for large section variations, where this study mainly focuses on the fire vulnerability assessment using ML-based predictions. However, if additional loading conditions, e.g., seismic or wind loads, were to be considered, larger sections, strong-column/weak-beam principle, and ductile detailing would be required in the generated buildings for realistic structural behavior under combined loading conditions. Future studies may also consider investigating the influence of variable cross-sectional dimensions and semi-rigid connections on the structural performance under fire conditions. 
} % blockRevise

\subsection{Material Properties}
Steel is chosen as the material for the structures. To reflect real-world variations, we randomly assign one of five slightly different steel material types to each structural element. \revise{
The ranges of material properties are provided in \tabref{tab:material_property_ranges} and the properties are sampled from uniform distributions of the corresponding ranges. These variations simulate differences arising from manufacturing batches or regional material properties. That these properties are at ambient temperature and change when the temperature rises due to a fire. The selection of materials with varying properties is aimed at increasing the diversity of the data. Our goal is to represent as wide a range of data as possible with a limited amount of building structure data, thereby enhancing the generalization ability of the GNN. Our assumed material property ranges are expected to be wider than the real-world conditions based on findings in \cite{mirza1979variations, lee2004probabilistic}. Therefore, we are essentially tackling a more challenging and general task. If we can solve this problem, we are confident that our method will perform equally well or even better in real-world scenarios.
}
\begin{table}[h!]
    \centering
    \caption{Material properties ranges for considered steel structures.}
    \begin{tabular}{lc}
        \toprule
        Property & Range \\
        \midrule
        Young's modulus & [168, 252] GPa \\
        Yield strength & [220, 330] MPa \\
        Strain-hardening ratio & [0.8, 1.2] \% \\
        \bottomrule
    \end{tabular}
    \label{tab:material_property_ranges}
\end{table}

\subsection{Gravity Loads}
Gravity loads are applied to columns and beams based on their \revise{influence (tributary) areas as typically conducted in structural analysis. The considered ``service'' load conditions include the column self-weight and the additional loads directly supported on the beams from their self-weight and weights of the reinforced concrete slabs, people as live load, and building content. An edge beam typically carries approximately half the gravity load supported by a parallel interior beam}. The ranges of gravity loads are listed in \tabref{tab:gravity_load_ranges}. \revise{The loads are sampled from uniform distributions of the corresponding ranges.} Structures that failed to meet an MIDR threshold of $1\%$ under gravity loads were deemed unacceptable designs and filtered out, as such configurations of randomly chosen geometry, material, and gravity load combinations were considered unrealistic from a regulatory and practicality points of view.
\begin{table}[h!]
    \centering
    \caption{Gravity load ranges for considered beams and columns.}
    \begin{tabular}{lc}
        \toprule
        Element & Range (kN/m)  \\
        \midrule
        Column & [0.5, 1.0]  \\
        Edge beam & [1.5, 4.5]  \\
        Interior beam & [3.0, 7.5]  \\
        \bottomrule
    \end{tabular}
    \label{tab:gravity_load_ranges}
\end{table} 

\subsection{Rule-based Thermal Load Generation}
\label{subsec:thermal_load_generation}
To evaluate a building's structural response during a fire event, we employed a simplified rule-based approach for thermal load generation. 
% Previous studies \cite{nan_structuralfire_2023} have demonstrated that steel structures rapidly equilibrate with surrounding gases temperatures due to efficient heat exchange. Consequently, gas temperatures can be directly used as inputs for FEA tools, e.g., OpenSees, simplifying the process of modeling thermal loads. 
% Accurately simulating temperature fields in fire scenarios poses significant challenges. Advanced thermodynamic simulations, such as those performed using Fire Dynamics Simulator (FDS) \cite{mcgrattan_fire_2000}, provide precise temperature predictions. However, these methods are hindered by high computational costs, prolonging execution times, and limited scalability, making them impractical for generating large datasets. Additionally, real-world fire loads often display substantial spatial variability across different rooms \cite{dundar_fire_2023}, resulting in scenario-specific temperature fields with limited generalizability. For example, studies on bridge fires \cite{he_study_2024} have demonstrated that environmental factors, such as wind speeds, can significantly influence temperature distributions. Furthermore, even within identical scenarios, variations in fire modeling methodologies can produce distinctly different temperature fields \cite{zhang_temperature_2020, du_new_2012}. These challenges emphasize the need for efficient and adaptable methods to generate fire temperature data.
% To address these issues, we adopted a rule-based approach to model temperature variations. 
According to \cite{spearpoint_fire_2008}, a typical fire development follows a predictable pattern. During the {\em{growth stage}}, the temperature rises slowly and approximately linearly after ignition. This is followed by the {\em{flashover stage}}, where temperatures increase rapidly to peak values. After reaching the peak, the temperature either stabilizes or continues to rise slowly until the {\em{decay stage}} begins. Inspired by this fire development pattern, we describe the temperature evolution in time, $t$, prior to the decay stage in two distinct stages:
\begin{enumerate}
    \item {\bf{Initial linear increase stage}}: For $t \in [0, t_1)$, temperature increases gradually and linearly as the fire spreads through the building. This stage represents the time before the fire directly affects a structural element.  
    \item {\bf{ISO 834 fire curve stage}}: For $t \in [t_1, t_{\thre}]$, temperature rises rapidly following the ISO 834 curve \cite{ISO834}, modeling the direct impact of the fire on the structural element. 
\end{enumerate}
The slope of the linear temperature increase, $c$, and the transition time, $t_1$, are influenced by the spatial relationship between the fire source and the structural element. For the second stage of temperature evolution, we utilize the ISO 834 curve, a widely accepted standard for fire resistance testing. This standardized fire curve describes the temperature rise over time, enabling rapid and consistent thermal fields across various scenarios. The duration of fire simulation in this study is set to $t_{\thre}=60$ minutes. This value represents the upper limit for the temperature evolution of each structural element, providing a consistent basis for analyzing the structural response to fire.

Let $(x, y, z)$ represents the midpoint of a structural element and $(x_{\subfire}, y_{\subfire}, z_{\subfire})$ the fire source point. \revise{Integer parameters $h$ and $h_{\subfire}$ correspond to the respective floor levels of the element and the fire source}. The temperature evolution for each element is expressed as follows:
\begin{enumerate}
    \item Linear increase stage ($0 < t < t_1$):
    \begin{equation}
    T(t) = c \cdot t,
    \end{equation}
    where $c$, the rate of temperature increase ($^\circ\mathrm{C}/\mathrm{min}$), depends on the height difference between the element, $h$, and the fire source, $h_{\subfire}$:
    \begin{equation}
        c = 
        \begin{cases} 
        5\left/\left(h - h_{\subfire} + 1\right)\right., & h \geq h_{\subfire}, \\
        2\left/\left(h_{\subfire} - h\right)\right., & h < h_{\subfire}.
        \end{cases}
    \end{equation}
     \item ISO 834 stage ($t \geq t_1$):
\begin{equation}
    T(t) = c \cdot t_1 + 345 \log_{10} \left(8 \left(t - t_1\right) + 1\right).
\end{equation}
\end{enumerate}

The transition (arrival) time $t_1$, marking the end of the linear stage, depends on the spatial distance between the fire source and the element. We define the following two Euclidean distances $L_p$ in the $xy$ plane and $L_s$ in the $xyz$ space:
\begin{eqnarray}
L_p & \triangleq & \sqrt{(x - x_{\subfire})^2 + (y - y_{\subfire})^2}, \\
\label{eq:Lp}
L_s & \triangleq & \sqrt{(x - x_{\subfire})^2 + (y - y_{\subfire})^2 + (z - z_{\subfire})^2}.
\label{eq:Ls}
\end{eqnarray}
Accordingly, the transition time, $t_1$, is expressed as follows:
\begin{equation}
    t_1 = 
    \begin{cases}
    \beta_{1} \cdot \left(1 - \exp\left\{- L_s\left/\alpha_{1}\right.\right\}\right), & h > h_{\subfire}, \\
    \beta_{2} \cdot \left(1 - \exp\left\{- L_p\left/\alpha_{2}\right.\right\}\right), & h = h_{\subfire}, \\
    \beta_{3} \cdot \left(1 - \exp\left\{- L_s\left/\alpha_{3}\right.\right\}\right), & h < h_{\subfire} .
    \end{cases}
    \label{eq:t1}
\end{equation}
The parameters $\beta_i$ and $\alpha_i$ for determining $t_1$ are summarized in Table~\ref{tab:fire_spread_parameters}. In this study, we take $r_{\mathrm{up}}=0.95$ and $r_{\mathrm{down}}=0.97$.
\begin{table}[ht]
    \centering
    \caption{Fire spread parameters for $t_1$ calculations.}
    \begin{tabular}{lcc}
        \toprule
        Case  & $\beta_i$ & $\alpha_i$  \\
        \midrule
        $i=1$, Upward spread & $16 \left.\left(1-r_{\mathrm{up}}^{\left|h-h_{\subfire}\right|}\right)\right/\left(1-r_{\mathrm{up}}\right)$ & $10$  \\
        $i=2$, Horizontal spread & $18$ & $18$  \\
        $i=3$, Downward spread & $30 \left.\left(1-r_{\mathrm{down}}^{\left|h-h_{\subfire}\right|}\right)\right/\left(1-r_{\mathrm{down}}\right)$ & $5$  \\
        \bottomrule
    \end{tabular}
    \label{tab:fire_spread_parameters}
\end{table}

\figref{fig:t1_curve} illustrates the $t_1$ curves for various fire scenarios: (1) fire originating on the lower floor, $h-h_{\subfire}=1$ with rapid upward spread, (2) fire on the same floor, $h=h_{\subfire}$ with the fastest spread, and (3) fire on the upper floor, $h_{\subfire}-h=1$ with slow downward spread. The exponential decay in $t_1$ reflects the accelerating fire propagation speed as the distance increases. \figref{fig:t1_curve} also indicates that the employed simplified model is consistent with the Markov chain-based dynamic model given by \cite{cheng_dynamic_2011}, where the rooms at the same floor of the fire point start flashover slightly before the corresponding upper floors. Additionally, $\beta_{1}$ and $\beta_{3}$ are the summation of a geometric sequence, where story level $h$ is the index. The common ratios $r_{\mathrm{up}}<1$ in $\beta_{1}$ and $r_{\mathrm{down}}<1$ in $\beta_{3}$ indicate that the fire speeds up to spread through the next story, which is consistent with the real-world fire spread mechanism given in \cite{hokugo_mechanism_2000}. The temperature profile within the range $t \in [0, t_{\thre}]$ is subsequently used as the thermal load in OpenSeesRT simulations to compute displacements at each structural node at time $t_{\thre}$.
\begin{figure}[h!]
    \centering
    \includegraphics[width=0.8\linewidth]{figures/m204_t1_curve.pdf}
    \caption{Three examples for the $t_1$ curve.}
    \label{fig:t1_curve}
\end{figure}

\revise{
\textit{Remark:} The effects of structural elements, such as concrete floor slabs and partitions, are not explicitly modeled in our approach. Instead, their influence is implicitly captured through the careful selection of the parameters $ \alpha, \beta, r_\mathrm{up} $, and $ r_\mathrm{down} $. This parameterization provides a unified framework for generating temperature fields. Indeed, fire propagation is governed by a multitude of factors and remains an open research question. For instance, if the fire resistance of a floor slab is enhanced by fire protective coating, the corresponding model can account for this by decreasing $\alpha_1$ \& $\alpha_3$, increasing $\beta_1$ \& $\beta_3$, and adopting larger values for $r_\mathrm{up}$ \& $r_\mathrm{down}$, which collectively slow down the vertical spread of fire. Conversely, scenarios involving higher amounts of combustible materials would warrant the opposite adjustments. This flexible and integrated approach avoids the need to design separate models for different fire propagation scenarios while still capturing the essential effects.
}

\revise{
In conclusion, our rule-based approach is a computationally efficient method for approximating fire temperature fields, enabling large-scale dataset generation to train predictive models. By combining ISO 834 fire curves with spatial considerations and embedding structural effects through parameter calibration, the method achieves a balanced trade-off between accuracy and scalability, making it a practical solution for thermal load modeling in fire scenarios. After generating the temperature of each beam or column according to the middle point, the temperature is applied as uniform thermal load to the elements of the structure in question using OpenSeesRT. 
}

% In conclusion, this rule-based approach is a computationally efficient method to approximate fire temperature fields, enabling large-scale dataset generation to train predictive models. By combining ISO 834 fire curves with spatial considerations, the method balances accuracy and scalability, making it a practical solution for thermal load modeling in fire scenarios.

% \subsection{Interstory Drift Ratio}
\subsection{OpenSeesRT Simulation}
\label{subsec:opensees_simulation}

The thermal and mechanical responses of 3D frame structures under combined fire and gravity loads are simulated using OpenSeesRT \cite{perez2024openseesrt}. \revise{In the simulation, the IDR of each node at $t_{\thre}$ is computed using the computed nodal displacements. Each structural model features six degrees of freedom per node (3 translational  and 3 rotational), with linear geometrical transformations (\texttt{geomTransf: Linear}) defining how the element local coordinate systems are mapped to the global coordinate system and assuming small displacements and rotations. Although OpenSeesRT allows a variety of options for modeling finite deformations, in the present simulations and mainly for simplicity, we did not consider large deformations. All bottom nodes (nodes on the ground) are fully constrained in all six degrees of freedom, while degrees of freedom os all other nodes are free.} Material behavior is temperature-dependent and modeled with \texttt{Steel01Thermal}, while fiber-based sections (\texttt{FiberThermal}) capture nonlinear interactions between thermal and mechanical responses at the cross-section level. \revise{Structural elements are represented as displacement-based Euler-Bernoulli beam-columns (\texttt{dispBeamColumnThermal}). This element  formulation accounts for thermal strains (temperature gradients) in the section, which is discretized into fibers. Numerical integration is used along the length of each element using three integration (Gauss) points, one at each end and the third in the middle of the element.}

{\revise{Thermal expansion of steel members plays a crucial role in IDR development. In reality, reinforced concrete floor slabs heat at a different rate than steel members due to their higher thermal mass and lower thermal conductivity. This differential heating can lead to restrained thermal expansion, introducing axial compression in beams and affecting the overall structural response. In this study, explicit {\em{composite action}} between steel members and concrete slabs is not modeled. Instead, our approach focuses on isolating the response of the steel structural frame, which is often the critical load-bearing component in fire scenarios. This assumption aligns with prior studies \cite{Possidente_2024} demonstrating that steel structures reach thermal equilibrium with surrounding gases quickly, allowing the use of uniform thermal loading in fire analysis. Future work could enhance this framework by incorporating slab-beam interaction effects, through a refined FEA for an extended dataset where constraints imposed by floor slabs are explicitly considered.}

The analysis begins with the application of gravity loads, followed by incremental thermal loads simulating the fire exposure. A static nonlinear solver using  \texttt{ExpressNewton} algorithm ensures convergence, while the \texttt{NormDispIncr} test maintains accuracy. An incremental \texttt{LoadControl} scheme with small step sizes is employed to guarantee numerical stability, using 10\% for gravity loads and 1\% for thermal loads. 

\revise{
In the thermal load analysis, uniform thermal load is applied to each beam or column, i.e., the temperature of each element is set to be that at the middle point, according to \secref{subsec:thermal_load_generation}. The \texttt{Steel01Thermal} material allows the properties (e.g., Young's modulus and yield strength) to be adjusted at increasing temperatures according to \cite{EN1993} using its Table 3.1: Reduction factors for the stress-strain relationship of carbon steel at elevated temperatures. For example, if the Young’s modulus at ambient temperature is $E_0$, then as the temperature ($T$) increases, the modulus changes as $E(T) = \eta (T) \times E_0$. \cite{EN1993} directly provides the values of $\eta(T) \in \left[0,1\right] $ at every $100 ^\circ\mathrm{C}$ interval and recommends using linear interpolation to obtain $\eta(T)$ for intermediate values of $T$.
} OpenSeesRT documentation \cite{OpenSeesThermalExamples} provides several examples of thermal analyses.

This modeling framework accommodates variations in material properties, cross-sectional geometries, and temperature profiles, providing robust simulations of structural behavior under fire conditions. The primary settings and configurations for the OpenSeesRT simulations are summarized in \tabref{tab:ops_detail}.
\begin{table}[h!]
    \centering
        \caption{Key settings of OpenSeesRT simulations.}
    \begin{tabular}{l|>{\raggedright\arraybackslash}p{0.6\linewidth}} %
    \toprule
    Modeling Aspect     & Details \\
    \midrule
    Geometry            & 3D models; 6 degrees of freedom per node \\
    Transformation      & geomTransf: Linear \\ 
    Material            & Steel01Thermal \\
    Section             & FiberThermal; Cross-section: $0.1$ m $\times$ $0.1$ m \\ 
    Element type        & {dispBeamColumnThermal} \\ 
    Loading             & Gravity loads: {beamUniform}; Thermal loads: {beamThermal} \\
    Integration scheme  & Incremental {LoadControl}; Step size: $10\%$ (gravity analysis), $1\%$ (thermal analysis) \\
    Nonlinear solver    & {ExpressNewton} algorithm; {UmfPack} solver; Convergence test: {NormDispIncr} tolerance: $10^{-8}$; Maximum \# iterations per step: $1000$. \\ 
    \bottomrule
    \end{tabular}
    \label{tab:ops_detail}
\end{table}

For each structure in the labeled dataset, 30 fire points are selected using a dual-granularity approach, \revise{i.e., two-stage sampling strategy,} to ensure they are well-distributed. Specifically, rooms are sequentially selected, with one fire point randomly chosen within each selected room. If a building is large and contains more than 30 rooms, we randomly select 30 rooms without replacement, i.e., ensuring that no more than one fire point is located in the same room. Conversely, if the building is small and has fewer than 30 rooms, all rooms are initially selected, with one fire point randomly assigned to each room. Additionally, rooms are then selected with replacement until a total of 30 fire points are assigned. \revise{The room-level sampling prioritizes selecting distinct rooms to avoid spatial clustering of fire points, while the point-level sampling ensures intra-room variability. This approach aligns with stratified sampling principles commonly used for efficient spatial representation, where multi-stage sampling strategies optimize coverage and variability, e.g., \cite{arunachalam_generalized_2023}, and enables a more comprehensive characterizing of how the structures respond under fire conditions.}
% This selection method prevents fire points from clustering too closely while maintaining an element of randomness. By distributing fire points in this manner, the 30 fire scenarios are effectively utilized, enabling a more comprehensive characterizing of how the structures respond under fire conditions.

\subsection{Summary of the Dataset Generation}
As discussed in this section and related to  \figref{fig:dataset_generation_procedure}, three key steps were considered in the development of the dataset: 
\begin{enumerate}
    \item {\bf{Filtering process}}: Structures with MIDR exceeding $1\%$ under gravity loads were excluded,  resulting in $1,573$ labeled structures retained for fire simulation and $16,050$ unlabeled structures for training the MFSP predictor.
    \item {\bf{Fire simulations}}: For each retained labeled structure, 30 fire scenarios were simulated using OpenSeesRT, yielding $47,190$ fire cases.
    \item {\bf{Data distribution check}}: MIDR distributions for labeled and unlabeled data under gravity loads were highly similar, because both datasets were generated using the same method. Under fire conditions, the MIDR distribution shifted, reflecting significant structural deformation with values reaching a maximum of about 6\%, an average of 1.70\%, and a standard deviation of 1.12\%. This step ensured a diverse and comprehensive dataset for the proposed predictive framework.
\end{enumerate}
The statistical distribution histograms for MIDR (after applying the $1\%$ filtering threshold \revise{for gravity load responses}) under different loading conditions are plotted in \figref{fig:histogram_mdr}. Figures \ref{fig:histogram_mdr}(a) and \ref{fig:histogram_mdr}(b) show the MIDR distributions of the labeled and unlabeled data, respectively, under gravity loads only. \figref{fig:histogram_mdr}(c) shows the MIDR distribution of the labeled data under the combined effects of gravity and fire loads. Fire load causes the structures to significantly deform, leading to a noticeably \revise{right-skewed} MIDR distribution.

\begin{figure*}[h!]
    \centering
    \includegraphics[width=\linewidth]{figures/histogram_mdr.pdf}
    \caption{Histograms of MIDR for labeled and unlabeled structures with gravity loads and fire cases.}
    \label{fig:histogram_mdr}
\end{figure*}

\revise{
This dataset provides the basis for training and testing the performance of the GNN-based framework. Although we employed a simplified rule-based thermal load generation method compared with conventional CFD-based simulations, the temperature field, the changes of the material properties, and the response of the structures, are all still highly nonlinear and complex. Therefore, it is still a challenging task for the NN to predict the MIDRs based on this dataset.
}

In Tab.~\ref{tab:existing_dataset}, we evaluate caption and 3D mask quality across datasets using three metrics. 
The \textit{unique normalized nouns count} measures the total number of unique normalized nouns in captions, with higher count indicating richer and more diverse captions.
\textit{Mask coverage} (\%) calculates the mean percentage of 3D points with associated captions per scene, where higher coverage enables more effective training.
\textit{Mask entropy} (bits) measures mask quality for datasets with partial masks generated from multi-view images (\ie OV3D, RegionPLC, and Mosaic3D-5.6M) without using GT.
It calculates Shannon entropy of GT instance ID distributions within each mask--higher entropy indicates that a mask contains multiple GT instances, suggesting less accurate mask boundaries.
Mosaic3D-5.6M demonstrates superior caption diversity and mask quality compared to both existing large-scale 3D-text datasets and previous open-vocabulary 3D segmentation datasets, validating its value as a new dataset.
\begin{table*}[t]
    \centering
    \renewcommand{\arraystretch}{1.3} % Adjust row height
    \setlength{\tabcolsep}{4pt} % Reduce space between columns
    \resizebox{1\textwidth}{!}{ \begin{tabular}{lrlcccccc}
            \toprule
            \textbf{Dataset} & \textbf{Size} & \textbf{Data Construction Method} & \textbf{Cultural?} & \textbf{Location?} & \textbf{\#Topic} & \textbf{\#Country} & \textbf{Reasoning?} \\
            \midrule
            \datasetname{} (\textbf{Ours}) & 3,482 & Manually built,  validated by native & \checkmark & \checkmark &  54 & 13 & \checkmark \\
            AraDiCE-Culture \cite{mousi-etal-2025-aradice} & 180 & Manually built, validated by native & \checkmark & \checkmark & 9 &  1 & -- \\
            AraDiCE-WinoGrande \cite{mousi-etal-2025-aradice} & 1,267 & Machine-translated, post-edited & -- & -- & -- & --& \checkmark \\
            AraDiCE-PIQA \cite{mousi-etal-2025-aradice} & 1,838 & Machine-translated, post-edited & -- & -- & -- & --& \checkmark \\
            AraDiCE-OpenBookQA \cite{mousi-etal-2025-aradice} & 500 & Machine-translated, post-edited &  -- & -- & -- & -- & \checkmark \\
            AlGhafa (COPA Ar) \cite{almazrouei-etal-2023-alghafa} & 89 & Machine-translated, verified by humans & -- & -- & -- & -- & \checkmark \\
            ACVA \cite{huang2023acegpt} & 2,486 & ChatGPT generated, verified by humans & \checkmark &  -- & 50 & -- & -- \\
            \bottomrule
        \end{tabular}
    }
    \caption{Comparison of our dataset with other Arabic cultural commonsense reasoning datasets. The metadata includes\textbf{ Size} (number of Arabic instances), \textbf{Cultural?} (whether the data considers cultural nuances), \textbf{Location?} ( whether the data includes fine-grained location information, such as regions and countries per region), \textbf{\#Topic} (number of fine-grained topics covered), \textbf{\#Country} (total number of countries across all regions) and \textbf{Reasoning?} (whether the data emphasizes commonsense reasoning or not). }
    \label{tab:dataset_comparison} 
\end{table*}

\subsection{Data Preprocessing}
\begin{itemize}[leftmargin=*,itemsep=1pt]
    \item \textbf{ScanNet}~\cite{dai2017scannet} To optimize computational efficiency while maintaining adequate spatial coverage, we process every 20th RGB-D frame from each scene. Prior to processing, we resize all RGB-D frames to 640$\times$480 resolution.
    \item \textbf{ScanNet++}~\cite{yeshwanth2023scannet++} From the official dataset, we utilize the \textit{``DSLR''} image collection. Following repository guidelines, we generate synthetic depth images using the reconstructed mesh and camera parameters. After correcting for distortion in both RGB and depth images and adjusting camera intrinsics, we process every 10th frame through our annotation pipeline. Point clouds are generated via surface sampling on the reconstructed meshes.
    \item \textbf{ARKitScenes}~\cite{baruch2021arkitscenes} We leverage the \textit{``3D Object Detection (3DOD)''} subset, utilizing its RGB-D frames and reconstructed meshes. We use every 10th frame at low resolution (256$\times$192), and apply surface point sampling on mesh for point clouds.
    \item \textbf{Matterport3D}~\cite{chang2017matterport3d} We use preprocesed RGB-D frames and point clouds provided by the author of OpenScene~\cite{Peng2023OpenScene}.
    \item \textbf{Structured3D}~\cite{zheng2020structured3d} We utilize RGB-D frames from both perspective and panoramic camera. We utilize preprocessed point clouds from the \textit{Pointcept}~\cite{pointcept2023} library, which fuses multi-view depth unprojection with voxel downsampling to get point clouds.
\end{itemize}

\subsection{Pipeline Configurations}
Our data generation pipeline leverages multiple Visual Foundation Models to automate the data annotation process. Below we detail the configuration of each model in our pipeline.
\begin{itemize}[leftmargin=*,itemsep=1pt]
    \item \textbf{RAM++~\cite{ram_pp}}: we utilize the official pretrained checkpoint \texttt{ram\_plus\_swin\_large\_14m} available at \url{https://huggingface.co/xinyu1205/recognize-anything-plus-model}.
    \item \textbf{Grounded-SAM~\cite{ren2024grounded}}: We employ the official checkpoint of Grounding-DINO~\cite{liu2023grounding} \texttt{IDEA-Research/grounding-dino-tiny} accessed through HuggingFace at \url{https://huggingface.co/IDEA-Research/grounding-dino-tiny}, together with SAM2~\cite{ravi2024sam} with checkpoint \texttt{sam2\_hiera\_l}, available at \url{https://huggingface.co/facebook/sam2-hiera-large}. For the postprocessing, we process the output bounding boxes from Grounding-DINO using a box score threshold of 0.25 and a text score threshold of 0.2. We then apply non-maximum suppression (NMS) with an IoU threshold of 0.5 to remove redundancy. To ensure meaningful region proposals, we filter out excessively large boxes that occupy more than 95\% of the image area. These refined bounding boxes are then passed to SAM2 for mask prediction.
    \item \textbf{Osprey~\cite{yuan2024osprey}}: We utilize the official pretrained \texttt{sunshine-lwt/Osprey-Chat-7b} checkpoint, available at \url{https://huggingface.co/sunshine-lwt/Osprey-Chat-7b}. The generation parameters are set with a temperature of 1.0, top\_p of 1.0, beam search size of 1, and the maximum number of new tokens to 512.
\end{itemize}

\begin{table}[ht]
    \centering
    \begin{minipage}{0.99\columnwidth}\vspace{0mm}    \centering
    \begin{tcolorbox} 
        \raggedright
        \small
        $\texttt{\textbf{System}}$: \texttt{A chat between a curious human and an artificial intelligence assistant. The assistant gives helpful, detailed, and polite answers to the human's questions.} \\
        $\texttt{\textbf{User}}$: \PredSty{\texttt{<image>}} \texttt{This provides an overview of the picture. Please give me a short description of} \PredSty{\texttt{<mask><pos>}} \texttt{, using a short phrase.}
    \end{tcolorbox}
        \label{tab:osprey_prompt}
    \end{minipage}
    \caption{\textbf{Osprey region caption prompt}. Osprey~\cite{yuan2024osprey} utilizes this prompt along with segmentation masks generated by Grounded-SAM to produce descriptive captions for each region.}
    \vspace{-4mm}
\end{table}



\subsection{Additional Pipeline Experiments}
We explore two additional data pipeline configurations that use Segment3D~\cite{huang2024segment3d} masks for segmentation while maintaining Osprey~\cite{yuan2024osprey} for captioning:
\begin{itemize}[leftmargin=*,itemsep=1pt]
    \item \textbf{Segment3D}: We utilize complete Segment3D masks and obtain captions by aggregating descriptions from multiple projected views of each mask. This approach maintains mask completeness but may result in multiple captions being assigned to a single mask from different viewpoints.
    \item \textbf{Segment3D - Mosaic}: We use partial Segment3D masks as seen from individual views and generate captions based on these view-specific projections. While masks are partial, each mask-caption pair is aligned since it represents the exact visible region from a specific viewpoint.
\end{itemize}
The results in Tab.~\ref{tab:segment3d_pipeline} demonstrate that Segment3D - Mosaic outperforms the baseline Segment3D approach, highlighting the importance of precise mask-text pair alignment.
However, both Segment3D variants are outperformed by our \nickname pipeline, which suggests that our combination of RAM++~\cite{ram_pp}, Grounded-SAM~\cite{ren2024grounded}, and SEEM~\cite{zou2024segment} provides superior segmentation quality.

\begin{table}[h]
\centering
\resizebox{\linewidth}{!}{
\begin{tabular}{l|rr|rr}
\midrule
\multirow{2}{*}{Pipeline} & \multicolumn{2}{c|}{ScanNet20~\cite{dai2017scannet}} & \multicolumn{2}{c}{ScanNet200~\cite{scannet200}} \\
 & f-mIoU & f-mAcc & f-mIoU & f-mAcc \\
\midrule
Segment3D~\cite{huang2024segment3d} & 50.6 & 76.6 & 8.3 & 19.1 \\
Segment3D~\cite{huang2024segment3d} - Mosaic & 57.3 & 79.6 & 10.6 & 22.8 \\
\rowcolor{gray!15} \nickname & \textbf{65.0} & \textbf{82.5} & \textbf{13.0} & \textbf{24.5} \\
\midrule
\end{tabular}
}
\caption{\textbf{Segment3D data pipeline evaluation results.}}
\label{tab:segment3d_pipeline}
\end{table}


\definecolor{light blue}{RGB}{215, 242, 252}
\definecolor{light purple}{RGB}{247, 215, 252}
\definecolor{light orange}{rgb}{0.9961, 0.875, 0.7188}
    
    
\begin{table*}[!htp]
\centering
\resizebox{\textwidth}{!}{%
\renewcommand{\arraystretch}{1.1}
\begin{tabular}{c l l l l l l l}
    \specialrule{1.3pt}{0pt}{0pt}
    \textbf{Language} & \textbf{Type} & \textbf{Prompt/Model} & \textbf{\textsc{xCOMET}$(s,t)$} & \textbf{\textsc{xCOMET}$(s,r)$} & \textbf{\textsc{xCOMET}$(s,t,r)$} & \textbf{\textsc{MetricX}$(s,t)$} & \textbf{\textsc{MetricX}$(t,r)$} \\ \midrule
    
    \multirow{22}{*}{\large \textbf{\textsc{en-de}}} & \textbf{Original} & - & 0.893 & 0.904 & 0.898 & 2.038 & 1.534 \\
    \cmidrule(lr){2-8}
    
    & \fcolorbox{white}{light blue}{\raisebox{-0.2em}{\includegraphics[height=1em]{figures/logos/agnostic.png}} \textbf{MT-Agnostic}} & \textbf{Simplification} (\small \textsc{LLaMA-2}) & \textbf{0.931} & 0.846 & 0.900 & \textbf{1.185} & 1.727 \\
    & &(\small \textsc{LLaMA-3}) & \textbf{0.944} & 0.820 & 0.903 & \textbf{0.925*} & 1.600 \\
    & &(\small \textsc{Tower-Instruct}) & \textbf{0.922} & 0.885 & \textbf{0.907} & 1.504 & 1.519 \\
    
    & & \textbf{Paraphrase} (\small \textsc{LLaMA-2}) & \textbf{0.926} & 0.823 & 0.889 & \textbf{1.126} & \textbf{1.480} \\
    & & (\small \textsc{LLaMA-3}) & \textbf{0.938} & 0.796 & 0.892 & \textbf{0.955} & \textbf{1.469} \\
    & & (\small \textsc{Tower-Instruct}) & 0.902 & 0.887 & 0.901 & \textbf{1.310} & 1.534 \\

    & &(\small \textsc{DIPPER} (L80/O60)) & \textbf{0.904} & 0.745 & 0.838 & \textbf{1.674} & 2.757 \\
    & &(\small \textsc{DIPPER} (L80/O40)) & \textbf{0.913} & 0.797 & 0.863 & \textbf{1.461} & 2.266 \\
    & &(\small \textsc{DIPPER} (L60/O40)) & \textbf{0.917} & 0.847 & 0.892 & \textbf{1.555} & 1.958 \\

    & &\textbf{Stylistic} (\small \textsc{CoEdIT} GEC) & \textbf{0.901} & 0.899 & 0.900 & \textbf{1.709} & 1.555 \\
    & & (\small \textsc{CoEdIT} Coherent) & 0.898 & 0.900 & 0.898 & \textbf{1.728} & 1.595 \\
    & & (\small \textsc{CoEdIT} Understandable) & \textbf{0.949} & \textbf{0.758} & 0.862 & \textbf{0.989} & 2.610 \\
    & & (\small \textsc{CoEdIT} Formal) & \textbf{0.937} & 0.830 & 0.900 & \textbf{1.063} & 1.879 \\

    \cmidrule(lr){2-8}
    & \fcolorbox{white}{light purple}{\raisebox{-0.2em}{\includegraphics[height=1em]{figures/logos/task.png}} \textbf{Task-Aware}} & \textbf{Easy Translation} (\small \textsc{LLaMA-2}) & \textbf{0.916} & 0.857 & 0.893 & \textbf{1.654} & 2.482 \\
    & & (\small \textsc{LLaMA-3}) & \textbf{0.932} & 0.827 & 0.899 & \textbf{1.151} & 2.241 \\
    & & (\small \textsc{Tower-Instruct}) & \textbf{0.901} & 0.900 & 0.903 & \textbf{1.759} & 2.427 \\
    & & \textbf{CoT} (\small \textsc{Tower-Instruct}) & \textbf{0.907} & 0.816 & 0.897 & \textbf{1.892} & 1.578 \\
    
    \cmidrule(lr){2-8}
    & \fcolorbox{white}{light orange}{\raisebox{-0.2em}{\includegraphics[height=1em]{figures/logos/translatability.png}} \textbf{Translatability-Aware}} & \textbf{Selection} & \textbf{0.921} & 0.907 & \textbf{0.915*} & \textbf{1.734} & \textbf{1.461*} \\
    & & \textbf{Fine-tune} (\small Basic) & \textbf{0.934} & 0.851 & \textbf{0.909} & \textbf{1.878} & \textbf{1.499} \\
    & & (\small MT) & \textbf{0.919} & 0.856 & 0.903 & \textbf{1.947} & 1.593 \\
    & & (\small Reference) & 0.896 & 0.836 & 0.876 & 2.023 & 2.028 \\
    
    % & &\textbf{DPO} (\small Basic) & \textbf{0.937} & 0.654 & 0.752 & \textbf{1.829} & 1.512 \\
    % & &(\small MT) & \textbf{0.952*} & 0.728 & 0.847 & \textbf{0.944} & \textbf{1.488} \\
    % & &(\small Reference) & \textbf{0.933} & 0.742 & 0.836 & \textbf{1.846} & 1.562 \\

    \specialrule{1.3pt}{0pt}{0pt}
    \end{tabular}
}
\caption{Detailed results of English-German pair using different rewrite methods. Statistically significant average improvements ($p$-value $< 0.05$) are \textbf{bold}. Best scores for each metric is \textbf{bold} with \textbf{*}. \textsc{xCOMET}$(s,t)$: translatability (↑); \textsc{xCOMET}$(s,r)$: meaning preservation (↑); \textsc{xCOMET}$(s,t,r)$: overall translation quality (↑); \textsc{MetricX}$(s,t)$: quality estimation (↓); \textsc{MetricX}$(t,r)$: reference-based metric (↓). For \textsc{DIPPER} \cite{dipper} variations, L and O denote lexical and order diversity, respectively.}
\label{tab:detailed_results_ende}
\end{table*}
\clearpage

\definecolor{light blue}{RGB}{215, 242, 252}
\definecolor{light purple}{RGB}{247, 215, 252}
\definecolor{light orange}{rgb}{0.9961, 0.875, 0.7188}


\begin{table*}[!htp]
\centering
\resizebox{\textwidth}{!}{%
\renewcommand{\arraystretch}{1.1}
\begin{tabular}{c l l l l l l l}
    \specialrule{1.3pt}{0pt}{0pt}
    \textbf{Language} & \textbf{Type} & \textbf{Prompt/Model} & \textbf{\textsc{xCOMET}$(s,t)$} & \textbf{\textsc{xCOMET}$(s,r)$} & \textbf{\textsc{xCOMET}$(s,t,r)$} & \textbf{\textsc{MetricX}$(s,t)$} & \textbf{\textsc{MetricX}$(t,r)$} \\ \midrule

    \multirow{22}{*}{\large \textbf{\textsc{en-ru}}} & \textbf{Original} & - & 0.872 & 0.884 & 0.868 & 2.535 & 2.028 \\
    \cmidrule(lr){2-8}

    & \fcolorbox{white}{light blue}{\raisebox{-0.2em}{\includegraphics[height=1em]{figures/logos/agnostic.png}} \textbf{MT-Agnostic}} & \textbf{Simplification} (\small \textsc{LLaMA-2}) & \textbf{0.916} & 0.839 & \textbf{0.882} & \textbf{0.951} & 2.160 \\
    & &(\small \textsc{LLaMA-3}) & \textbf{0.919} & 0.812 & \textbf{0.885} & \textbf{0.804} & 2.039 \\
    & &(\small \textsc{Tower-Instruct}) & \textbf{0.921} & 0.870 & \textbf{0.891} & \textbf{1.135} & \textbf{1.921} \\

    & & \textbf{Paraphrase} (\small \textsc{LLaMA-2}) & \textbf{0.923} & 0.804 & \textbf{0.881} & \textbf{0.882} & \textbf{1.853} \\
    & &  (\small \textsc{LLaMA-3}) & \textbf{0.930} & 0.788 & \textbf{0.882} & \textbf{0.855} & \textbf{1.863} \\
    & & (\small \textsc{Tower-Instruct}) & \textbf{0.887} & 0.878 & \textbf{0.878} & \textbf{1.095} & \textbf{1.976} \\

    & & (\small \textsc{DIPPER} (L80/O60)) & \textbf{0.904} & 0.735 & 0.821 & \textbf{1.249} & 3.476 \\
    & & (\small \textsc{DIPPER} (L80/O40)) & \textbf{0.909} & 0.790 & 0.853 & \textbf{1.105} & 2.773 \\
    & & (\small \textsc{DIPPER} (L60/O40)) & \textbf{0.905} & 0.834 & 0.873 & \textbf{1.119} & 2.418 \\

    & & \textbf{Stylistic} (\small \textsc{CoEdIT} GEC) & 0.873 & 0.884 & 0.869 & \textbf{1.327} & \textbf{1.969} \\
    & & (\small \textsc{CoEdIT} Coherent) & 0.873 & 0.884 & 0.869 & \textbf{1.368} & \textbf{1.989} \\
    & & (\small \textsc{CoEdIT} Understandable) & \textbf{0.918} & 0.801 & 0.873 & \textbf{0.991} & 2.726 \\
    & & (\small \textsc{CoEdIT} Formal) & \textbf{0.916} & 0.841 & \textbf{0.887} & \textbf{0.922} & 2.020 \\
    \cmidrule(lr){2-8}


    & \fcolorbox{white}{light purple}{\raisebox{-0.2em}{\includegraphics[height=1em]{figures/logos/task.png}} \textbf{Task-Aware}} & \textbf{Easy Translation} (\small \textsc{LLaMA-2}) & \textbf{0.914} & 0.839 & \textbf{0.884} & \textbf{1.037} & 10.849 \\
    & & (\small \textsc{LLaMA-3}) & \textbf{0.917} & 0.808 & \textbf{0.881} & \textbf{0.801*} & 10.401 \\
    & & (\small \textsc{Tower-Instruct}) & \textbf{0.885} & 0.883 & \textbf{0.878} & \textbf{1.277} & 11.137 \\
    
    & & \textbf{CoT} (\small \textsc{Tower-Instruct}) & \textbf{0.903} & 0.871 & 0.875 & \textbf{2.432} & 2.024 \\
    \cmidrule(lr){2-8}

    & \fcolorbox{white}{light orange}{\raisebox{-0.2em}{\includegraphics[height=1em]{figures/logos/translatability.png}} \textbf{Translatability-Aware}} & \textbf{Selection} & \textbf{0.914} & \textbf{0.891*} & \textbf{0.899*} & \textbf{2.096} & \textbf{1.830*} \\
    
    & & \textbf{Fine-tune} (\small Basic) & \textbf{0.912} & 0.848 & \textbf{0.886} & \textbf{2.123} & \textbf{1.932} \\
    & & (\small MT) & \textbf{0.904} & 0.851 & 0.871 & \textbf{2.119} & \textbf{1.997} \\
    & & (\small Reference) & \textbf{0.881} & 0.812 & 0.859 & \textbf{2.284} & 2.012 \\
    
    % & & \textbf{DPO} (\small Basic) & \textbf{0.916} & 0.728 & 0.791 & \textbf{2.153} & \textbf{1.924} \\
    % & & (\small MT) & \textbf{0.943*} & 0.769 & 0.859 & \textbf{2.099} & \textbf{1.903} \\
    % & & (\small Reference) & \textbf{0.925} & 0.754 & 0.852 & \textbf{2.211} & 2.014 \\

    \specialrule{1.3pt}{0pt}{0pt}
    \end{tabular}
}
\caption{Detailed results of English-Russian pair using different rewrite methods.}
\label{tab:detailed_results_enru}
\end{table*}
\definecolor{light blue}{RGB}{215, 242, 252}
\definecolor{light purple}{RGB}{247, 215, 252}
\definecolor{light orange}{rgb}{0.9961, 0.875, 0.7188}



\begin{table*}[!htp]
\centering
\resizebox{\textwidth}{!}{%
\renewcommand{\arraystretch}{1.1}
\begin{tabular}{c l l l l l l l}
    \specialrule{1.3pt}{0pt}{0pt}
    \textbf{Language} & \textbf{Type} & \textbf{Prompt/Model} & \textbf{\textsc{xCOMET}$(s,t)$} & \textbf{\textsc{xCOMET}$(s,r)$} & \textbf{\textsc{xCOMET}$(s,t,r)$} & \textbf{\textsc{MetricX}$(s,t)$} & \textbf{\textsc{MetricX}$(t,r)$} \\ \midrule

    \multirow{19}{*}{\large \textbf{\textsc{en-zh}}} & \textbf{Original} & - & 0.786 & 0.775 & 0.794 & 3.445 & 2.282 \\

    \cmidrule(lr){2-8}

    & \fcolorbox{white}{light blue}{\raisebox{-0.2em}{\includegraphics[height=1em]{figures/logos/agnostic.png}} \textbf{MT-Agnostic}} & \textbf{Simplification} (\small \textsc{LLaMA-2}) & \textbf{0.828} & 0.728 & 0.796 & \textbf{1.321} & 2.537 \\
    &  & (\small \textsc{LLaMA-3}) & \textbf{0.823} & 0.701 & 0.795 & \textbf{1.252*} & 2.572 \\
    & & (\small \textsc{Tower-Instruct}) & \textbf{0.821} & 0.759 & \textbf{0.802} & \textbf{1.521} & \textbf{2.227} \\

    & & \textbf{Paraphrase} (\small \textsc{LLaMA-2}) & \textbf{0.818} & 0.683 & 0.771 & \textbf{1.330} & 2.478 \\
    & & (\small \textsc{LLaMA-3}) & \textbf{0.826} & 0.662 & 0.766 & \textbf{1.341} & 2.534 \\
    & & (\small \textsc{Tower-Instruct}) & \textbf{0.797} & 0.765 & 0.798 & \textbf{1.580} & 2.283 \\

    & & (\small \textsc{DIPPER} (L80/O60)) & \textbf{0.813} & 0.622 & 0.722 & \textbf{1.583} & 4.009 \\
    & &  (\small \textsc{DIPPER} (L80/O40)) & \textbf{0.816} & 0.670 & 0.750 & \textbf{1.499} & 3.196 \\
    & & (\small \textsc{DIPPER} (L60/O40)) & \textbf{0.809} & 0.711 & 0.775 & \textbf{1.503} & 2.725\\
    
    & & \textbf{Stylistic} (\small \textsc{CoEdIT} GEC) & 0.789 & 0.772 & 0.795 & \textbf{1.632} & 2.251 \\
    & &  (\small \textsc{CoEdIT} Coherent) & 0.786 & 0.774 & 0.794 & \textbf{1.658} & 2.267 \\
    & &  (\small \textsc{CoEdIT} Understandable) & \textbf{0.839*} & 0.677 & 0.774 & \textbf{1.358} & 3.174 \\
    & &  (\small \textsc{CoEdIT} Formal) & \textbf{0.823} & 0.730 & 0.798 & \textbf{1.336} & 2.443 \\
    \cmidrule(lr){2-8}

    & \fcolorbox{white}{light purple}{\raisebox{-0.2em}{\includegraphics[height=1em]{figures/logos/task.png}} \textbf{Task-Aware}} & \textbf{Easy Translation} (\small \textsc{LLaMA-2}) & \textbf{0.821} & 0.721 & 0.784 & \textbf{1.900} & 7.732 \\
    & & (\small \textsc{LLaMA-3}) & \textbf{0.830} & 0.687 & 0.783 & \textbf{1.360} & 7.608 \\
    & & (\small \textsc{Tower-Instruct}) & \textbf{0.793} & 0.762 & 0.791 & \textbf{1.618} & 7.650 \\
    
    & & \textbf{CoT} (\small \textsc{Tower-Instruct}) & \textbf{0.821} & 0.769 & 0.771 & \textbf{3.321} & 2.432 \\

    \cmidrule(lr){2-8}
    & \fcolorbox{white}{light orange}{\raisebox{-0.2em}{\includegraphics[height=1em]{figures/logos/translatability.png}} \textbf{Translatability-Aware}} & \textbf{Selection} & \textbf{0.823} & \textbf{0.783*} & \textbf{0.819*} & \textbf{3.149} & \textbf{2.206*} \\

    \specialrule{1.3pt}{0pt}{0pt}
    \end{tabular}
}
\caption{Detailed results of English-Chinese pair using different rewrite methods.}
\label{tab:detailed_results_enzh}
\end{table*}
\clearpage

\definecolor{light blue}{RGB}{215, 242, 252}
\definecolor{light purple}{RGB}{247, 215, 252}
\definecolor{light orange}{rgb}{0.9961, 0.875, 0.7188}



\begin{table*}[!htp]
\centering
\resizebox{330pt}{!}{%
\begin{tabular}{l l l l l l}
    \specialrule{1.3pt}{0pt}{0pt}
    \textbf{Type} & \textbf{Prompt/Model} & \textbf{\textsc{en-de}} & \textbf{\textsc{en-ru}} & \textbf{\textsc{en-zh}} \\ \midrule
    
    \fcolorbox{white}{light blue}{\textbf{MT-Agnostic}} & \textbf{Simplification} (\small \textsc{LLaMA-2}) & 2.06	& 2.37	& 2.37 \\
    & (\small \textsc{LLaMA-3}) & 0.39	& 0.33	& 0.29 \\
    & (\small \textsc{Tower-Instruct}) & 28 &	29.3	& 30.2 \\

    & \textbf{Paraphrase} (\small \textsc{LLaMA-2}) & 0 & 0 & 0 \\
    &  (\small \textsc{LLaMA-3}) & 0.06	&0.07	&0.03 \\
    & (\small \textsc{Tower-Instruct}) & 37.3	& 38.2	& 38 \\

    & (\small \textsc{DIPPER} (L80/O60)) & 0.19	& 0.94	& 1.04 \\
    & (\small \textsc{DIPPER} (L80/O40)) & 0.51& 1.5 &	1.53 \\
    & (\small \textsc{DIPPER} (L60/O40)) & 1.48	& 2.5	& 2.44 \\

    & \textbf{Stylistic} (\small \textsc{CoEdIT} GEC) & 42.6& 44 & 48.3 \\
    & (\small \textsc{CoEdIT} Coherent) & \color{red}{82.2}	& \color{red}{91.9}	& \color{red}{93.2} \\
    & (\small \textsc{CoEdIT} Understandable) & 1.61& 1.88	& 1.53 \\
    & (\small \textsc{CoEdIT} Formal) & 5.33& 3.76	& 5.5 \\
    \midrule


    \fcolorbox{white}{light purple}{\textbf{Task-Aware}} & \textbf{Easy Translation} (\small \textsc{LLaMA-2}) & 3.04 & 3.55 & 3.63 \\
    & (\small \textsc{LLaMA-3}) & 0.24 & 0.66 & 0.27 \\
    & (\small \textsc{Tower-Instruct}) & 12.3 & 18.6 & 15.4 \\
    & \textbf{CoT} (\small \textsc{Tower-Instruct}) & 0.71& 1.45& 1.53 \\
    \midrule

    \fcolorbox{white}{light orange}{\textbf{Translatability-Aware}} & \textbf{Fine-tune} (\small Basic) & 4.5	& 3.91 & - \\
    & (\small MT) &3.73	& 3.42 & - \\
    & (\small Reference) & 6.17	& 7.85 & - \\

    % & \textbf{DPO} (\small Basic) & 0.71 & 0.98 & - \\
    % & (\small MT) &0.77	& 1.37 & - \\
    % & (\small Reference) & 1.48	& 1.95 & - \\

    \specialrule{1.3pt}{0pt}{0pt}
    \end{tabular}
}
\caption{Percentage of occurrence (\%) where the rewrite is a direct copy of the original source sentence.}
\label{tab:direct_copy}
\end{table*}

% \begin{table*}[!htp]
\centering
\resizebox{\textwidth}{!}{%
    \begin{tabular}{c | p{3.5cm} p{3.5cm} p{3.5cm} p{4.5cm}}
    \toprule
    \textbf{Rewrite Method} & \textbf{Original MT} & \textbf{Rewrite MT} & \textbf{Reference} & \textbf{Comment}  \\ \midrule

    \multirow{5}{*}{\textbf{Stylistic Prompting}} & Er wurde oft von Mobbern geärgert, aber von seinem Bruder tapfer verteidigt. & Er wurde oft von Mobbern gehänselt, aber immer von seinem Bruder unterstützt. & Oft von Mobbern angegriffen, aber hartnäckig von seinem Bruder verteidigt. & ``geärgert'' and ``gehänselt'' are both good choices but ``gehänselt'' definitely \textbf{suits better} when talking about mobbing. \\
    \midrule
    
    \multirow{16}{*}{\textbf{Re-ranking}} & Trage einen warmen Anzug, einen Barett, einen Regenmantel und gute Stiefel mit Nägeln. & Tragen Sie warme Kleidung, einen Hut und wasserdichte Stiefel. & Tragen Sie einen warmen Anzug, eine Baskenmütze, einen Regenmantel und ein gutes Paar genagelte Stiefel. & 1) Original MT has more terminology from the reference. \par 2) Rewrite MT is \textbf{easier to understand} than original MT in this context. \\
    \cmidrule(lr){2-5}
    
    & Wann immer er ins Gefängnis kommt, bezahlt Eddie immer wieder seine Kaution. & Eddie holt ihn immer aus dem Gefängnis. & Wann immer er im Gefängnis landet, holt Eddie ihn ständig raus. & 1) Rewrite MT seems \textbf{more precise}. \par 2) Original MT does not explicitly tells you that Eddie safes someone from prison. \\
    \cmidrule(lr){2-5}

    & ``Es ist nicht so, dass man oft hört: `Ich fahre ihn einfach nur nach Glasgow''', sagte der Ticketverkäufer. & ``Das ist nicht etwas, was man oft hört'', sagte der Kassierer. ``Ich bringe ihn einfach nach Glasgow.'' & ``Das hört man nicht oft, `Ich bringe ihn nur nach Glasgow''', sagte der Schaffner. & Original MT has a weird structure, rewrite MT is a \textbf{lot easier to understand.} \\
    \midrule

    \multirow{9}{*}{\textbf{SFT}} & Das ist eine sehr schöne Röhre. Das schicke Spitzenmuster ist weich und elegant. & Diese Röcke sind schön und elegant. Das Spitzenmuster ist stilvoll und weich. & Das ist ein sehr schöner Rock. Das Spitzenmuster ist stilvoll und weich. & Rewrite MT is an \textbf{easy sentence}, but original MT is better capturing the meaning with the word ``Nominierung''. \\
    \cmidrule(lr){2-5}

    & Wir sind alle so unordentliche Wesen. & Wir sind alle ein bisschen unordentlich. & Wir sind alle so chaotische Typen. & Original MT better captures the reference, but it's weired to use ``Wesen'' in this context. \\

    % \multirow{4}{*}{\textbf{DPO}} & Die Schützen feuerten von der Kante der kleineren Gruben aus. & Die Schützen positionierten sich auf den Rändern der kleinen Gräben. Sie feuerten. & Die Schützen feuerten vom Rand der kleineren Gruben. & Writing ``Kante'' in the original MT is misleading, rewrite MT \textbf{better captures the reference.} \\
    
    \bottomrule
    \end{tabular}
}
\caption{Comments from the human annotators. We show the original MT, rewrite MT, and reference translation for each example. Comments in \textbf{bold} represent reasons for preferring rewrite translations over those of original.}
\label{tab:annotator_feedback}
\end{table*}
% \clearpage

% \input{table/good_rewrites_ende}
% \input{table/good_rewrites_enru}
% \input{table/good_rewrites_enzh}
% \clearpage

\begin{CJK*}{UTF8}{gbsn}

\begin{table*}[!htp]
\centering
\resizebox{\textwidth}{!}{%
\renewcommand{\arraystretch}{1.1}
\begin{tabular}{l p{4cm} p{4cm} p{4cm} p{4cm} l l}
    \specialrule{1.3pt}{0pt}{0pt}
    \textbf{Label} & \textbf{Original} & \textbf{Rewrite} & \textbf{Original MT}  & \textbf{Rewrite MT} & \textbf{\textsc{xCOMET}$(s,t)$} & \textbf{\textsc{xCOMET}$(s',t')$} \\ \midrule

    \textbf{Simplified} & When Michael ``Hopper'' McGrath \textbf{lobbed} a ball in, Molloy \textbf{leapt} highest before rifling a sublime goal to the roof of the net. & When Michael McGrath \textbf{threw} the ball in, Molloy \textbf{jumped} highest and scored a beautiful goal to the top of the net. & Als Michael ``Hopper'' McGrath einen Ball hereinwarf, sprang Molloy am höchsten und schoss einen herrlichen Treffer auf das Dach des Netzes. & Als Michael McGrath den Ball in die Luft warf, sprang Molloy am höchsten und erzielte einen wunderschönen Treffer in die obere Netzhöhe. & 0.906 & 0.945 \\
        
    \cmidrule(lr){2-7}

    & Derry City \textbf{emerged victorious} in the President's Cup as they ran out 2-0 winners over Shamrock Rovers. & Derry City \textbf{won} the President's Cup title by defeating Shamrock Rovers 2-0. & Derry City 在总统杯赛中获胜,以 2-0 的比分击败尚洛克罗弗斯。& Derry City 以 2-0 的 比分击败 Shamrock Rovers,获得了总统杯冠军。& 0.648 & 0.952 \\
    \midrule

    \textbf{Detailed} & The great majority of rankers never advanced beyond principalis. & The vast majority of soldiers remained in the lowest rank throughout their careers. & Die große Mehrheit der Reiter schaffte es nie über den Rang eines principalis. & Die überwiegende Mehrheit der Soldaten blieb während ihrer gesamten Karriere in der niedrigsten Ränge. & 0.938 & 0.982 \\
    \cmidrule(lr){2-7}

    & I've noticed you almost need line of sight for it to work. & It appears that visibility plays a crucial role in the effectiveness of the process. & \russian{Я заметил, что для работы вам почти все время нужен прямой свет.} & \russian{Похоже, что видимость играет решающую роль в эффективности процесса.} & 0.98 & 1.0 \\
    \midrule

    \textbf{Fluency} & It's a thing I've never said before either. & I've never said that before either. & \russian{Это то, что я никогда не говорил раньше.} & \russian{Я никогда этого не говорил и раньше.} & 0.989 & 1.0 \\
    \cmidrule(lr){2-7}

    & When I started in summer with those multi-source experiments. & I began a series of experiments in the summer. & 我在夏天开始进行多来源实验时。& 我在夏天开始了一系列的实验。& 0.858 & 1.0 \\

    \specialrule{1.3pt}{0pt}{0pt}
    \end{tabular}
}
\caption{Examples of rewrites for each annotation label (\textbf{Simplified}, \textbf{Detailed} and \textbf{Fluency}).}
\label{tab:success_types}
\end{table*}

\end{CJK*}

\clearpage
% \begin{table*}[!htp]
\centering
\resizebox{400}{!}{%
    \begin{tabular}{l l l l l}
    \toprule
    \textbf{Language Pair} & \textbf{Prompt} & \textbf{Type} & \textbf{\textsc{xCOMET}$(s,t)$} & \textbf{\textsc{xCOMET}$(s,t,r)$}  \\
    \toprule
        
    \multirow{10}{*}{\textbf{\large{\textsc{en-de}}}} & \textbf{Original} & - & 0.893 & 0.898 \\
    \cmidrule{2-5}
    & \textbf{Simplification} & \textbf{I} & \textbf{0.915} & 0.907 \\
    & & \textbf{Owo} & 0.863 & 0.879 \\
    & & \textbf{Ow} & 0.879 & 0.894 \\
    & & \textbf{I+O} & \textbf{0.915} & \textbf{0.907} \\
    \cmidrule{2-5}
    & \textbf{Paraphrase} & \textbf{I} & \textbf{0.902} & \textbf{0.901} \\
    & & \textbf{Owo} & 0.864 & 0.881 \\
    & & \textbf{Ow} & 0.892 & 0.896 \\
    & & \textbf{I+O} & \textbf{0.902} & \textbf{0.901} \\

    \midrule

    \multirow{10}{*}{\textbf{\large{\textsc{en-ru}}}} & \textbf{Original} & - & 0.861 & 0.854 \\
    \cmidrule{2-5}
    & \textbf{Simplification} & \textbf{I} & \textbf{0.901} & \textbf{0.885} \\
    & & \textbf{Owo} & 0.868 & 0.864 \\
    & & \textbf{Ow} & 0.872 & 0.869 \\
    & & \textbf{I+O} & 0.888 & 0.869 \\
    \cmidrule{2-5}
    & \textbf{Paraphrase} & \textbf{I} & \textbf{0.887} & \textbf{0.878} \\
    & & \textbf{Owo} & 0.867 & 0.865 \\
    & & \textbf{Ow} & 0.883 & 0.871 \\
    & & \textbf{I+O} & 0.876 & 0.862 \\

    \midrule

    \multirow{10}{*}{\textbf{\large{\textsc{en-zh}}}} & \textbf{Original} & - & 0.786 & 0.794 \\
    \cmidrule{2-5}
    & \textbf{Simplification} & \textbf{I} & \textbf{0.805} & \textbf{0.798} \\
    & & \textbf{Owo} & 0.713 & 0.751 \\
    & & \textbf{Ow} & 0.746 & 0.780 \\
    & & \textbf{I+O} & \textbf{0.805} & \textbf{0.798} \\
    \cmidrule{2-5}
    & \textbf{Paraphrase} & \textbf{I} & \textbf{0.797} & \textbf{0.798} \\
    & & \textbf{Owo} & 0.708 & 0.751 \\
    & & \textbf{Ow} & 0.790 & 0.796 \\
    & & \textbf{I+O} & 0.795 & 0.797 \\

    \bottomrule
    \end{tabular}
}
\caption{Results for simplification and paraphrase prompting: input rewriting (\textbf{I}), post-editing output without source signal (\textbf{Owo}), with source signal (\textbf{Ow}), and the combination of both strategies (\textbf{I+O}). Best scores for each metric is \textbf{bold}.} 
\label{tab:inputvsoutput}
\end{table*}

\begin{table*}[!htp]
\centering
\resizebox{\textwidth}{!}{%
\renewcommand{\arraystretch}{1.1}
\begin{tabular}{p{3.5cm} p{4cm} p{4cm} p{4cm} p{4cm} l l l l}
    \specialrule{1.3pt}{0pt}{0pt}
    \textbf{Prompt/Model} & \textbf{Original} & \textbf{Rewrite} & \textbf{Original MT}  & \textbf{Rewrite MT} & \textbf{Flesch$(s)$} & \textbf{Flesch$(s')$} & \textbf{WSTF$(t)$} & \textbf{WSTF$(t')$} \\ \midrule

    \textbf{Simplification} (\textsc{LLaMA-3}) & She \textbf{steamed via} Hawaii, Midway, Guam, and Subic Bay for Vietnam and anchored in the Saigon River on 13 September. & She \textbf{sailed from} Hawaii to Vietnam, stopping at Midway, Guam, and Subic Bay, and \textbf{arrived at} the Saigon River on September 13. & Sie fuhr über Hawaii, Midway, Guam und Subic Bay nach Vietnam und ankerte am 13. September in der Saigon-Schifffahrt. & Sie segelte von Hawaii nach Vietnam, machte Halt in Midway, Guam und Subic Bay und erreichte am 13. September in der Saigon River. & 74.53 & \textbf{76.56} & 1.032 & \textbf{0.838} \\ \midrule

    \textbf{Simplification} (\textsc{Tower-Instruct}) & The remnants of Felix continued northeastward across the Atlantic until dissipating near Shetland on August 25. & Felix's remnants continued northeastward across the Atlantic until dissipating near Shetland on August 25. & Die Überreste von Felix zogen sich über den Atlantik in nordöstlicher Richtung bis zum 25. August, als sie sich in der Nähe von Shetland auflösten. & Felix's Reste zogen sich über den Atlantik in nordöstlicher Richtung bis zum 25. August, als sie sich in der Nähe von Shetland auflösten. & 31.89 & \textbf{38.32} & 1.193 & \textbf{1.109} \\ 
    \cmidrule{2-9}

     & Cambrai thus \textbf{reverted}, but only briefly, to the Western Frankish Realm. & Cambrai \textbf{returned} to the Western Frankish Realm, but only briefly. & Cambrai fiel daher, aber nur kurzzeitig, wieder an das Westfrankenreich zurück. & Cambrai kehrte zum Westfrankenreich zurück, aber nur kurz. & \textbf{68.77} & 54.22 & 0.728 & \textbf{0.429} \\
    
    \specialrule{1.3pt}{0pt}{0pt}
    \end{tabular}
}
\caption{Examples of simplification rewrites for English-German (\textsc{En-De}) pair and their corresponding input and output readability scores. \textbf{Flesch}: Flesch Reading Ease score (↑); \textbf{WSTF}: Vienna formula (↓).}
\label{tab:readability_ende}
\end{table*}

\begin{table*}[!htp]
\centering
\resizebox{\textwidth}{!}{%
\renewcommand{\arraystretch}{1.1}
\begin{tabular}{p{3.5cm} p{4cm} p{4cm} p{4cm} p{4cm} l l l l}
    \specialrule{1.3pt}{0pt}{0pt}
    \textbf{Prompt/Model} & \textbf{Original} & \textbf{Rewrite} & \textbf{Original MT}  & \textbf{Rewrite MT} & \textbf{Flesch$(s)$} & \textbf{Flesch$(s')$} & \textbf{Flesch-Ru$(t)$} & \textbf{Flesch-Ru$(t')$} \\ \midrule

    \textbf{Simplification} (\textsc{LLaMA-3}) & Later, Wallachia's Vornic Radu Socol traveled to Suceava, bringing Despot two steeds, a kuka hat with precious stones, and 24,000 ducats. & Radu Socol, the Vornic of Wallachia, visited Suceava and brought two horses, a hat with precious stones, and 24,000 ducats to Despot. & \russian{Позже, Ворник Раду Соколь из Валахии отправился в Сучаву, привезнув деспоту двух лошадей, кукушку с драгоценными камнями и 24 000 дукатов.} & \russian{Раду Сокол, вонник Валахии, посетил Сучаву и принес деспоту два коня, шляпу с драгоценными камнями и 24 000 дукатов.} & \textbf{67.08} & 66.07 & 55.81 & \textbf{64.80} \\ \midrule

    \textbf{Simplification} (\textsc{Tower-Instruct}) & \textbf{Appalled} at the thought of Emily \textbf{cavorting} with Casey, Margo \textbf{vindictively revealed} Emily's \textbf{hooker past} to Tom and Casey. & Margo was shocked that Emily was hanging out with Casey and so she told Tom and Casey about Emily's past as a prostitute. & \russian{Потрясенная мыслью о том, что Эмили развлекается с Кейси, Марго мстительно рассказала Тому и Кейси о прошлом Эмили проституткой.} & \russian{Марго была потрясена тем, что Эмили общалась с Кейси, и поэтому она рассказала Тому и Кейси о прошлом Эмили как проститутке.} & 60.65 & \textbf{81.97} & 58.47 & \textbf{64.40} \\
    
    \specialrule{1.3pt}{0pt}{0pt}
    \end{tabular}
}
\caption{Examples of simplification rewrites for English-Russian (\textsc{En-Ru}) pair and their corresponding input and output readability scores. \textbf{Flesch}: Flesch Reading Ease score (↑); \textbf{Flesch-Ru}: Russian version of Flesch (↑).}
\label{tab:readability_enru}
\end{table*}

\begin{CJK*}{UTF8}{gbsn}

\begin{table*}[!htp]
\centering
\resizebox{\textwidth}{!}{%
\renewcommand{\arraystretch}{1.1}
\begin{tabular}{p{3.5cm} p{4cm} p{4cm} p{4cm} p{4cm} l l l l}
    \specialrule{1.3pt}{0pt}{0pt}
    \textbf{Prompt/Model} & \textbf{Original} & \textbf{Rewrite} & \textbf{Original MT}  & \textbf{Rewrite MT} & \textbf{Flesch$(s)$} & \textbf{Flesch$(s')$} \\ \midrule

    \textbf{Simplification} (\textsc{LLaMA-3}) & During the delay, the tire carcass wrapped itself around the axle, costing him several laps. & The tire wrapped around the axle, causing him to lose several laps. & 延迟期间,轮胎壳破损,裹住了轮毂,让他失去了几圈的速度。& 轮胎缠在轴上,让他失去了几圈。& 64.71 & \textbf{84.68} \\ \midrule

    \textbf{Simplification} (\textsc{Tower-Instruct}) & Japanese artillery attempted to engage them but South Dakota and the other battleships easily outranged them. & Japanese artillery tried to attack them but South Dakota and the other battleships were too far away. & 日本炮兵试图与他们交战,但南达科他和其他战舰的射程远远超过他们。& 日本炮兵试图袭击他们,但南达科他和其他战舰太远了。 & 38.32 & \textbf{62.68} \\

    
    \specialrule{1.3pt}{0pt}{0pt}
    \end{tabular}
}
\caption{Examples of simplification rewrites for English-Chinese (\textsc{En-Zh}) pair and their corresponding input readability scores. \textbf{Flesch}: Flesch Reading Ease score (↑).}
\label{tab:readability_enzh}
\end{table*}

\end{CJK*}
\clearpage

\begin{table*}[!htp]
\centering
\resizebox{\textwidth}{!}{%
\renewcommand{\arraystretch}{1.1}
\begin{tabular}{c l l l l l l}
    \specialrule{1.3pt}{0pt}{0pt}
    \textbf{Language} & \textbf{Prompt/Model} & \textbf{\textsc{xCOMET}$(s,t)$} & \textbf{\textsc{xCOMET}$(s,t,r)$} & \textbf{\textsc{MetricX}$(s,t)$} & \textbf{\textsc{MetricX}$(t,r)$} \\ \midrule
    
    \multirow{3}{*}{\large \textbf{\textsc{en-de}}} & Original & 0.893 & 0.898 & 2.038 & 1.534 \\
    & Simplification (\small \textsc{Aya-23 8B}) & 0.901 & 0.900 & 1.956 & 1.624 \\
    & Simplification (\small \textsc{Tower-Instruct 13B}) & \textbf{0.924} & \textbf{0.912} & \textbf{1.562} & \textbf{1.445} \\
    \midrule

    \multirow{3}{*}{\large \textbf{\textsc{en-ru}}} & Original & 0.872 & 0.868 & 2.535 & 2.028 \\
    & Simplification (\small \textsc{Aya-23 8B}) & 0.880 & \textbf{0.875} & 2.428 & 1.938 \\
    & Simplification (\small \textsc{Tower-Instruct 13B}) & \textbf{0.901} & \textbf{0.889} & \textbf{2.137} & \textbf{1.861} \\

    \bottomrule
    \end{tabular}
}
\caption{Results with two additional LLMs for rewriting: \textsc{Aya-23 8B} and \textsc{Tower-Instruct 13B}. Statistically significant average improvements ($p$-value $< 0.05$) are \textbf{bold}. \textsc{xCOMET}$(s,t)$: translatability (↑); \textsc{xCOMET}$(s,t,r)$: overall translation quality (↑); \textsc{MetricX}$(s,t)$: quality estimation (↓); \textsc{MetricX}$(t,r)$: reference-based metric (↓).}
\label{tab:more_rewrite_llms}
\end{table*}
\begin{table*}[!htp]
\centering
\resizebox{\textwidth}{!}{%
\renewcommand{\arraystretch}{1.1}
\begin{tabular}{c l l l l l l l}
    \specialrule{1.3pt}{0pt}{0pt}
    \textbf{Language} & \textbf{MT System} & \textbf{Prompt/Model} & \textbf{\textsc{xCOMET}$(s,t)$} & \textbf{\textsc{xCOMET}$(s,t,r)$} & \textbf{\textsc{MetricX}$(s,t)$} & \textbf{\textsc{MetricX}$(t,r)$} \\ \midrule
    
    \multirow{6}{*}{\large \textbf{\textsc{en-de}}} & \multirow{2}{*}{\textsc{Tower-Instruct 7B}} & Original & 0.893 & 0.898 & 2.038 & 1.534 \\
    & & Simplification & \textbf{0.915} & \textbf{0.907} & 1.504 & 1.519 \\
    \cmidrule{2-7}
    
    & \multirow{2}{*}{\textsc{Aya-23 8B}} & Original & 0.887 & 0.891 & 1.926 & 1.554 \\
    & & Simplification & \textbf{0.911} & \textbf{0.902} & 1.660 & 1.571 \\
    \cmidrule{2-7}
    
    & \multirow{2}{*}{\textsc{Tower-Instruct 13B}} & Original & 0.880 & 0.887 & 2.043 & 1.522 \\
    & & Simplification & \textbf{0.900} & \textbf{0.893} & \textbf{1.778} & 1.556 \\
    \midrule


    \multirow{6}{*}{\large \textbf{\textsc{en-ru}}} & \multirow{2}{*}{\textsc{Tower-Instruct 7B}} & Original & 0.872 & 0.868 & 2.535 & 2.028 \\
    & & Simplification & \textbf{0.921} & \textbf{0.891} & \textbf{1.135} & \textbf{1.921} \\
    \cmidrule{2-7}
    
    & \multirow{2}{*}{\textsc{Aya-23 8B}} & Original & 0.863 & 0.852 & 2.711 & 2.323 \\
    & & Simplification & \textbf{0.892} & \textbf{0.872} & \textbf{2.300} & \textbf{2.173} \\
    \cmidrule{2-7}
    
    & \multirow{2}{*}{\textsc{Tower-Instruct 13B}} & Original & 0.887 & 0.882 & 2.290 & 1.915 \\
    & & Simplification & \textbf{0.894} & 0.875 & 2.296 & 1.915 \\
    \midrule


    \multirow{6}{*}{\large \textbf{\textsc{en-zh}}} & \multirow{2}{*}{\textsc{Tower-Instruct 7B}} & Original & 0.786 & 0.794 & 3.445 & 2.282 \\
    & & Simplification & \textbf{0.821} & \textbf{0.802} & \textbf{1.521} & \textbf{2.227} \\
    \cmidrule{2-7}
    
    & \multirow{2}{*}{\textsc{Aya-23 8B}} & Original & 0.769 & 0.779 & 3.758 & 2.572 \\
    & & Simplification & \textbf{0.793} & \textbf{0.788} & \textbf{3.433} & 2.530 \\
    \cmidrule{2-7}
    
    & \multirow{2}{*}{\textsc{Tower-Instruct 13B}} & Original & 0.755 & 0.764 & 3.421 & 2.341 \\
    & & Simplification & \textbf{0.772} & 0.767 & 3.236 & 2.413 \\

    \specialrule{1.3pt}{0pt}{0pt}
    \end{tabular}
}
\caption{Results with two additional LLMs as MT system: \textsc{Aya-23 8B} and \textsc{Tower-Instruct 13B}. Simplification is done with \textsc{Tower-Instruct 7b}. Statistically significant average improvements ($p$-value $< 0.05$) over their respective original baselines are \textbf{bold}. \textsc{xCOMET}$(s,t)$: translatability (↑); \textsc{xCOMET}$(s,t,r)$: overall translation quality (↑); \textsc{MetricX}$(s,t)$: quality estimation (↓); \textsc{MetricX}$(t,r)$: reference-based metric (↓).}
\label{tab:more_mt_llms}
\end{table*}
\begin{table*}[!htp]
\centering
\resizebox{\textwidth}{!}{%
\renewcommand{\arraystretch}{1.1}
\begin{tabular}{c l l l l l l l}
    \specialrule{1.3pt}{0pt}{0pt}
    \textbf{Language} & \textbf{Prompt/Model} & \textbf{\textsc{xCOMET}$(s,t)$} & \textbf{\textsc{xCOMET}$(s,t,r)$} & \textbf{\textsc{MetricX}$(s,t)$} & \textbf{\textsc{MetricX}$(t,r)$} \\ \midrule
    
    \multirow{3}{*}{\large \textbf{\textsc{cs-uk}}} & Original & 0.866 & 0.755 & 2.437 & 4.033 \\
    & Simplification & \textbf{0.885} & 0.749 & 2.355 & 4.053 \\
    & Selection & \textbf{0.930} & 0.748 & 3.050 & 4.053 \\
    \midrule

    \multirow{3}{*}{\large \textbf{\textsc{de-en}}} & Original & 0.969 & 0.622 & 1.869 & 4.760 \\
    & Simplification & \textbf{0.975} & \textbf{0.632} & 1.856 & 4.600 \\
    & Selection & \textbf{0.979} & \textbf{0.631} & 1.856 & 4.599 \\
    \midrule

    \multirow{3}{*}{\large \textbf{\textsc{he-en}}} & Original & 0.582 & 0.556 & 8.057 & 5.758 \\
    & Simplification & 0.562 & 0.514 & 8.671 & 6.374 \\
    & Selection & \textbf{0.639} & 0.514 & 9.192 & 6.541 \\
    \midrule

    \multirow{3}{*}{\large \textbf{\textsc{ja-en}}} & Original & 0.884 & 0.841 & 3.473 & 2.688 \\
    & Simplification & 0.896 & 0.828 & 3.303 & 2.929 \\
    & Selection & \textbf{0.918} & 0.827 & 3.659 & 2.964 \\
    \midrule

    \multirow{3}{*}{\large \textbf{\textsc{ru-en}}} & Original & 0.938 & 0.921 & 3.024 & 1.823 \\
    & Simplification & 0.945 & 0.922 & 2.909 & 1.879 \\
    & Selection & \textbf{0.954} & 0.923 & 3.079 & 1.912 \\
    \midrule

    \multirow{3}{*}{\large \textbf{\textsc{uk-en}}} & Original & 0.934 & 0.929 & 2.959 & 1.507 \\
    & Simplification & \textbf{0.951} & 0.929 & 2.684 & 1.595 \\
    & Selection & \textbf{0.962} & 0.929 & 3.055 & 1.656 \\
    \midrule

    \multirow{3}{*}{\large \textbf{\textsc{zh-en}}} & Original & 0.797 & 0.524 & 5.099 & 5.666 \\
    & Simplification & \textbf{0.809} & \textbf{0.530} & 4.849 & 5.582 \\
    & Selection & \textbf{0.827} & 0.528 & 5.202 & 5.800 \\

    \bottomrule
    \end{tabular}
}
\caption{Results with into-English and non-English language pairs. Simplification is done with \textsc{Tower-Instruct 7b}. Statistically significant average improvements ($p$-value $< 0.05$) over their respective original baselines are \textbf{bold}. \textsc{xCOMET}$(s,t)$: translatability (↑); \textsc{xCOMET}$(s,t,r)$: overall translation quality (↑); \textsc{MetricX}$(s,t)$: quality estimation (↓); \textsc{MetricX}$(t,r)$: reference-based metric (↓).}
\label{tab:more_lang_pairs}
\end{table*}

\begin{figure*}
    \centering
        \fbox
        \includegraphics[width=350pt]{figures/human_example_1_fig.png}
    }
    \caption{Survey content of the first part to compare Original MT vs. Rewrite MT. To avoid position bias, we randomly shuffle the order of original translations ($t$) and translations of rewrites ($t'$) for \textbf{Sentence 1} and \textbf{2}.}
    \label{fig:human_example_1}
\end{figure*}
\definecolor{maryred}{rgb}{0.758, 0.109, 0.0234}


\begin{figure*}
    \centering
        \fbox{%}
        \includegraphics[width=350pt]{figures/human_example_2_fig.png}
    }
    \caption{Survey content of the second part to compare to the \color{maryred}{\textbf{Reference translation}}\color{black}{. An optional text box is given for each example for further comments.}}
    \label{fig:human_example_2}
\end{figure*}
\begin{figure*}
    \centering
        \fbox{
        \includegraphics[width=350pt]{figures/human_example_3_fig.png}
    }
    \caption{Survey content to compare Original (\textbf{Sentence 1}) vs. Rewrite (\textbf{Sentence 2}).}
    \label{fig:human_example_3}
\end{figure*}

\end{document}
