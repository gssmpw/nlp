\section{Societal Considerations}

%Current evaluation methods for machine translation (MT) are inherently limited, as they primarily focus on traditional quality assessment metrics without considering the broader context in which translations are used. These methods often fail to account for the actual needs of users who rely on MT systems in real-world scenarios. We should better understand how improvements in translation quality impact the actual experiences of diverse user groups, particularly in multilingual and multicultural settings.

One critical area of future research lies in developing rewriting tools that support a wider range of languages beyond English. Existing tools are often biased toward high-resource languages, making it difficult to generalize translation improvements to other language pairs, especially those involving low-resource languages. By expanding the availability of rewriting tools and methodologies that cater to broader languages, researchers can ensure that the benefits of MT advancements are distributed more equitably.

Even within current tools, there are inherent biases that tend to favor expressions and conventions common in the United States, particularly for English-centric language models. For example, when rewriting the sentence ``\textit{Resuming her patrols, Constitution managed to recapture the American sloop Neutrality on 27 March and, a few days later, the French ship Carteret.}'' using the easy translate rewriting method, \textsc{LLaMA-2} rewrites ``\textit{on 27 March}'' to ``\textit{on March 27th},'' following the American date format. Further, \textsc{LLaMA-3} adds unnecessary explication by changing ``\textit{Constitution}'' to ``\textit{USS Constitution},'' which is a name of an American ship. While these rewrites may improve the overall MT quality, they also introduce a subtle standardization toward American expressions and cultural references. Such biases not only risk distorting the linguistic characteristics of the target language but also potentially erase its cultural nuances, making the translated content less relatable to non-American audiences. This emphasizes the importance of developing translation and rewriting systems that can preserve the linguistic and cultural identity of the source language.