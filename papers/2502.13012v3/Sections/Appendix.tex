\clearpage
\appendix

\section{Inclusion and Exclusion Criteria}
\label{tab: inclusion and exclusion criteria}


We summarize the inclusion and exclusion criteria in Table~\ref{tab:criteria}. Briefly, the \textbf{Inclusion Criteria (IC)} ensure that the reviewed studies focus on LLM agents exhibiting human-like behavior—either implicitly (e.g., preference or behavioral patterns) or explicitly (e.g., emotions or personality)—along with key cognitive processes such as reasoning and decision-making. Moreover, an open-ended action space and the capacity to tackle multifaceted tasks are essential attributes for inclusion.

By contrast, the \textbf{Exclusion Criteria (EC)} eliminate studies employing LLMs purely as chatbots, single-purpose systems, or evaluation tools, rather than as agents mimicking human cognition. Likewise, if the LLM agents are restricted to fixed, close-ended tasks or limited to algorithmic optimization without simulating cognitive processes, they fall outside the scope of this work.
\begin{table}[t]
\small
\caption{Inclusion and exclusion criteria.}
\label{tab:criteria}
\begin{tabularx}{\columnwidth}{lX}
    \toprule
    \multicolumn{2}{l}{\textbf{Inclusion Criteria (IC)}} \\
    \midrule
        IC-1 & The LLM agents in the paper simulate humanoid behavior with implicit personality (e.g., preference and behavior pattern) or explicit personality (e.g., emotion or characteristics).  \\
        IC-2 & The LLM agents in the paper have cognitive activities such as decision-making, reasoning, and planning. \\
        IC-3 & The LLM agents in the paper are capable of completing complicated and general tasks. \\
        IC-4 & The LLM agents' action set in the paper is neither predefined nor finite. \\
    \midrule
    \multicolumn{2}{l}{\textbf{Exclusion Criteria (EC)}} \\
    \midrule
        EC-1 & The study does not employ LLM agents for simulation purposes but rather uses them as chatbots, task-specific agents, or evaluators. \\
        EC-2 & The paper's research objectives, methodologies, and evaluations are not focused on simulating human-like behavior with LLM agents, but rather on optimizing LLM algorithms. \\
        EC-3 & The study primarily investigates the perception or action capabilities of LLM agents without simulating the cognitive process. \\
        EC-4 & The LLM agents are restricted to handling specific, close-ended tasks. \\
        EC-5 & The LLM agents' actions are either predefined or limited. \\
    \bottomrule
\end{tabularx}
\end{table}

\section{Query String}
\label{query string}

We employed the following query to guide our literature retrieval process:

\begin{quote}
\texttt{(“large language model” OR LLM) AND (agent OR persona OR "human digital twin" OR simulacra) AND (simulat* OR generat* OR eval*) AND “human behavior” AND cognit*}
\end{quote}

This query was designed to capture a broad spectrum of studies on large language models that simulate or replicate human-like behavior. It combines keywords related to LLM agents (\emph{LLM}, \emph{persona}, \emph{simulacra}), their capabilities (\emph{simulat*}, \emph{generat*}, \emph{eval*}), and the focus on cognitively grounded human behavior (\emph{cognit*}). This ensures that the resulting literature is relevant to our exploration of how LLM-based systems can mimic or exhibit human-like cognition and behavior patterns.


\section{Evaluation Approach Usage for Agent- and Task-Oriented Metrics}
We present a breakdown of evaluation approach usage by agent-oriented metrics (Fig.~\ref{fig:stacked-bar-agent-oriented}) and task-oriented metrics (Fig.~\ref{fig:stacked-bar-task-oriented}).

\begin{figure}[t]
    \centering
    \includegraphics[width=\linewidth]{Figures/agent_metrics_approach.png}
    \caption{Usage ratio of evaluation approaches for each category of agent-oriented metrics.}
    \label{fig:stacked-bar-agent-oriented}
\end{figure}


\begin{figure}[t]
    \centering
    \includegraphics[width=\linewidth]{Figures/task_metrics_approach.png}
    \caption{Usage ratio of evaluation approaches for each category of task-oriented metrics.}
    \label{fig:stacked-bar-task-oriented}
\end{figure}

% \section{Top Three Metrics for Agent and Task Attributes}
% \label{sec: top3}
% We present two tables for referencing the top three frequently used metrics for agent attributes (Tab.~\ref{tab: top3-agent}) and task attributes (Tab.~\ref{tab: top3-task}).

% % Please add the following required packages to your document preamble:
% \usepackage{booktabs}
\begin{table}[]
\small
\resizebox{\linewidth}{!}{%
\begin{tabular}{@{}p{0.32\linewidth}p{0.7\linewidth}@{}}
\toprule
\textbf{Agent Attributes}        & \textbf{Top 3 Agent-Oriented Metrics}                                                           \\ \midrule
Activity History        & External alignment metrics, internal consistency metrics, content and textual metrics  \\
Belief and Value        & Psychological metrics, bias, fairness, and ethics metrics                              \\
Demographic Info. & Psychological metrics, internal consistency metrics, external alignment metrics        \\
Psychological Traits    & Psychological metrics, internal consistency metrics, content and textual metrics       \\
Skill and Expertise     & External alignment metrics, internal consistency metrics,  content and textual metrics \\
Social Relationship     & Psychological metrics, external alignment metrics, social and decision-making metrics  \\ \bottomrule
\end{tabular}
}
\caption{Top 3 frequently used agent-oriented metrics for each agent attribute}
\label{tab: top3-agent}
\vspace{-1em}
\end{table}
% \begin{table}[t]
\small
\resizebox{\linewidth}{!}{%
\begin{tabular}{@{}p{0.37\linewidth}p{0.63\linewidth}@{}}
\toprule
\textbf{Task Attributes}          & \textbf{Top 3 Task-Oriented Metrics}                                                             \\ \midrule
Simulated Individuals    & Psychological, performance, and internal consistency metrics                            \\
Simulated Society        & Social and decision-making metrics, performance metrics, and psychological metrics      \\
Opinion Dynamics         & Performance metrics, external alignment metrics, and bias, fairness, and ethics metrics \\
Decision Making          & Social and decision-making, performance, and psychological metrics                      \\
Psychological Experiment & Psychological, content and textual, and performance metrics                             \\
Educational Training     & Psychological, performance, and content and textual metrics                             \\
Writing                  & Content and textual, psychological, and performance metrics                             \\ \bottomrule
\end{tabular}
}
\caption{Top 3 frequently used task-oriented metrics for each task attribute}
\label{tab: top3-task}
\vspace{-1em}
\end{table}

\section{Case Study: Flawed Example}
\label{appendix:bad_example}

Fig.~\ref{fig:bad example} visualized how the authors in the flawed example selected their evaluation metrics how further evaluation metrics could be uncovered through our proposed guideline.

\begin{figure}[t]
    \centering
    \includegraphics[width=\linewidth]{Figures/bad_example.pdf}
    \caption{Case study of a flawed example in Section \ref{sec_bad_example}. Given agent attributes (yellow) and task attributes (pink). The original authors' selection of evaluation metrics (purple and blue). The missing metrics that are recommended by our proposed guideline (orange) align with the reviewer's criticism in red text.}
    \label{fig:bad example}
\end{figure}


\section{Metrics Glossary}
\label{sec: metrics glossary}

We present two glossary tables for referencing the source of agent-oriented metrics (Tab.~\ref{tab: long_table_agent_metrics_src}) and task-oriented metrics (Tab.~\ref{tab: long_table_task_metrics_src}).

\onecolumn

\begin{small}
\begin{center}
\begin{longtable}{@{}p{0.12\textwidth}p{0.22\textwidth}p{0.3\textwidth}p{0.06\textwidth}p{0.2\textwidth}@{}}
\caption{Agent-oriented evaluation metrics glossary.} \label{tab: long_table_agent_metrics_src} \\

\toprule 
\textbf{Attribute} & \textbf{Category} & \textbf{Agent-oriented Metrics} & \textbf{Approach}  & \textbf{Source} \\ \midrule 
\endfirsthead

\\
\midrule 
\textbf{Attribute} & \textbf{Category} & \textbf{Agent-oriented Metrics} & \textbf{Approach}  & \textbf{Source} \\ \midrule
\endhead

\hline \multicolumn{5}{c}{{Continued on next page}} \\ \hline
\endfoot

\hline \bottomrule
\endlastfoot

Belief \& Value        & Bias, fairness, ethics metrics      & Exaggeration (normalized average cosine similarity)                                                                                         & Automatic         & \cite{cheng-etal-2023-compost}                                                                                                                                \\
Belief \& Value        & Bias, fairness, ethics metrics      & Individuation (classification accuracy)                                                                                                     & Automatic         & \cite{cheng-etal-2023-compost}                                                                                                                                \\
Belief \& Value        & Bias, fairness, ethics metrics      & Bias (performance disparity, prevalence, magnitude, variation, attitude shift)                                                              & Automatic         & \cite{gupta2024bias}\\

Belief \& Value        & Bias, fairness, ethics metrics      & Bias (performance disparity, prevalence, magnitude, variation, attitude shift)                                                              & Automatic         & \cite{taubenfeld-etal-2024-systematic}                                                                                                                              \\
Demographic Information & Bias, fairness, ethics metrics      & Exaggeration (normalized average cosine similarity)                                                                                         & Automatic         & \cite{cheng-etal-2023-compost}                                                                                                                               \\
Demographic Information & Bias, fairness, ethics metrics      & Individuation (classification accuracy)                                                                                                     & Automatic         & \cite{cheng-etal-2023-compost}                                                                                                                               \\
Demographic Information & Bias, fairness, ethics metrics      & Bias (performance disparity, prevalence, magnitude, variation, attitude shift)                                                              & Automatic         & \cite{gupta2024bias}                                                                                                                                \\
Demographic Information & Bias, fairness, ethics metrics      & Bias (performance disparity, prevalence, magnitude, variation, attitude shift)                                                              & Automatic         & \cite{neuberger2024sauce}                                                                                                                               \\
Demographic Information & Bias, fairness, ethics metrics      & Bias (performance disparity, prevalence, magnitude, variation, attitude shift)                                                              & Automatic         & \cite{taubenfeld-etal-2024-systematic}                                                                                                                               \\
Demographic Information & Bias, fairness, ethics metrics      & Message toxicity                                                                                                                            & Automatic         & \cite{fang2024llm}                                                                                                                              \\
Activity History        & Content and textual metrics         & Coherence                                                                                                                                   & LLM               & \cite{Li2024SchemaGuidedCC}                                                                                                                \\
Activity History        & Content and textual metrics         & Clarity                                                                                                                                     & Human             & \cite{10.1145/3613904.3642363}                                                                                                                \\
Activity History        & Content and textual metrics         & Diversity of dialog (Shannon entropy, intra-remote-clique, inter-remote-clique, semantic similarity, longest common subsequence similarity) & Automatic         & \cite{Ha2024CloChatUH}                                                                                                             \\
Belief \& Value        & Content and textual metrics         & Diversity of dialog (Shannon entropy, intra-remote-clique, inter-remote-clique, semantic similarity, longest common subsequence similarity) & Automatic         & \cite{gu2024agentgroupchatinteractivegroupchat}                                                                                                                              \\
Demographic Information & Content and textual metrics         & Coherence                                                                                                                                   & LLM               & \cite{Li2024SchemaGuidedCC}                                                                                                                 \\
Demographic Information & Content and textual metrics         & Attitudes (topic term frequency)                                                                                                            & Automatic         & \cite{fang2024llm}                                                                                                                               \\
Demographic Information & Content and textual metrics         & Diversity of dialog (Shannon entropy, intra-remote-clique, inter-remote-clique, semantic similarity, longest common subsequence similarity) & Automatic         & \cite{fang2024llm}                                                                                                                                 \\
Demographic Information & Content and textual metrics         & Clarity                                                                                                                                     & Human             & \cite{10.1145/3613904.3642363}                                                                                                                \\
Demographic Information & Content and textual metrics         & Diversity of dialog (Shannon entropy, intra-remote-clique, inter-remote-clique, semantic similarity, longest common subsequence similarity) & Automatic         & \cite{Ha2024CloChatUH}                                                                                                             \\
Demographic Information & Content and textual metrics         & Linguistic complexity (utterance length, Kolmogorov complexity)                                                                             & Automatic         & \cite{milivcka2024large}                                                                                      \\
Psychological Traits    & Content and textual metrics         & Text similarity (BLEU, ROUGE)                                                                                                               & Automatic         & \cite{zeng2024persllm}                                                                                                                               \\
Psychological Traits    & Content and textual metrics         & Tone Alignment                                                                                                                              & LLM               & \cite{zeng2024persllm}                                                                                                                            \\
Skills and Expertise    & Content and textual metrics         & Coherence                                                                                                                                   & LLM               & \cite{Li2024SchemaGuidedCC}                                                                                                                \\
Activity History        & External alignment metrics          & Hallucination                                                                                                                               & LLM               & \cite{shao2023character}                                                                                                           \\
Activity History        & External alignment metrics          & Entailment                                                                                                                                  & LLM               & \cite{Li2024SchemaGuidedCC}                                                                                                                \\
Activity History        & External alignment metrics          & Believability/Credibility(self-knowledge, memory, plans, reactions, reflections)                                                            & Human             & \cite{park2023generative}                                                                                                            \\
Demographic Information & External alignment metrics          & Entailment                                                                                                                                  & LLM               & \cite{Li2024SchemaGuidedCC}                                                                                                           \\
Demographic Information & External alignment metrics          & Believability/Credibility(self-knowledge, memory, plans, reactions, reflections)                                                            & Human             & \cite{park2023generative}                                                                                                           \\
Psychological Traits    & External alignment metrics          & Fact Accuracy                                                                                                                               & LLM               & \cite{zeng2024persllm}                                                                                                                              \\
Skills and Expertise    & External alignment metrics          & Hallucination                                                                                                                               & LLM               & \cite{shao2023character}                                                                                                            \\
Skills and Expertise    & External alignment metrics          & Entailment                                                                                                                                  & LLM               & \cite{Li2024SchemaGuidedCC}                                                                                                               \\
Activity History        & Internal consistency metrics        & Stability                                                                                                                                   & LLM               & \cite{shao2023character}                                                                                                                  \\
Activity History        & Internal consistency metrics        & Consistency of information                                                                                                                  & Human             & \cite{10.1145/3613904.3642363}                                                                                                                  \\
Belief \& Value        & Internal consistency metrics        & Attitude shift                                                                                                                              & LLM               & \cite{wang2024connecting}                                                                                                                              \\
Demographic Information & Internal consistency metrics        & Stability                                                                                                                                   & LLM               & \cite{shao2023character}                                                                                                                  \\
Demographic Information & Internal consistency metrics        & Attitude shift                                                                                                                              & LLM               & \cite{neuberger2024sauce}                                                                                                                              \\
Demographic Information & Internal consistency metrics        & Attitude shift                                                                                                                              & LLM               & \cite{taubenfeld-etal-2024-systematic}                                                                                                                               \\
Demographic Information & Internal consistency metrics        & Behavior stability (mean, standard deviation)                                                                                               & Automatic         & \cite{Wang2024IntelligentCS}                                                                                                                              \\
Demographic Information & Internal consistency metrics        & Consistency of information                                                                                                                  & Human             & \cite{10.1145/3613904.3642363}                                                                                                                  \\
Demographic Information & Internal consistency metrics        & Consistency of psychological state / personalities                                                                                          & Human             & \cite{10.1145/3613904.3642363}                                                                                                                  \\
Demographic Information & Internal consistency metrics        & Consistency of information                                                                                                                  & Human             & \cite{zeng2024persllm}                                                                                                                          \\
Psychological Traits    & Internal consistency metrics        & Stability                                                                                                                                   & LLM               & \cite{shao2023character}                                                                                                                  \\
Psychological Traits    & Internal consistency metrics        & Consistency of information                                                                                                                  & Human             & \cite{zeng2024persllm}                                                                                                                              \\
Psychological Traits    & Internal consistency metrics        & Consistency of psychological state / personalities                                                                                          & Human             & \cite{zeng2024persllm}                                                                                                                              \\
Psychological Traits    & Internal consistency metrics        & Consistency of information                                                                                                                  & Human             & \cite{cai2024digital} \\
Psychological Traits    & Internal consistency metrics        & Consistency of psychological state / personalities                                                                                          & Human             & \cite{cai2024digital} \\
Skills and Expertise    & Internal consistency metrics        & Stability                                                                                                                                   & LLM               & \cite{shao2023character}                                                                                                                  \\
Activity History        & Performance metrics                 & Memorization                                                                                                                                & LLM               & \cite{shao2023character}                                                                                                               \\
Demographic Information & Performance metrics                 & Memorization                                                                                                                                & LLM               & \cite{10.1145/3613904.3642363}                                                                                                                     \\
Demographic Information & Performance metrics                 & Communication ability (win rates)                                                                                                           & Automatic         & \cite{liu2024roleagent}                                                                                                                   \\
Demographic Information & Performance metrics                 & Reaction (accuracy)                                                                                                                         & Automatic         & \cite{liu2024roleagent}                                                                                                               \\
Demographic Information & Performance metrics                 & Self-knowledge (accuracy)                                                                                                                   & Automatic         & \cite{liu2024roleagent}                                                                                                                   \\
Activity History        & Psychological metrics & Empathy                                                                                                                                     & Human             & \cite{10.1145/3613904.3642363}                                                                                                                \\
Belief \& Value        & Psychological metrics & Value                                                                                                                                       & LLM               & \cite{shao2023character}                                                                                                              \\
Demographic Information & Psychological metrics & Personality consistency                                                                                                                     & Automatic         & \cite{wang2024incharacter}                                                                                                                        \\
Demographic Information & Psychological metrics & Measured alignment for personality                                                                                                          & Human             & \cite{wang2024incharacter}                                                                                                                   \\
Demographic Information & Psychological metrics & Sentiment                                                                                                                                   & Automatic         & \cite{fang2024llm}                                                                                                                               \\
Demographic Information & Psychological metrics & Empathy                                                                                                                                     & Human             & \cite{10.1145/3613904.3642363}                                                                                                                \\
Demographic Information & Psychological metrics & Belief (stability, evolution, correlation with behavior)                                                                                    & Automatic         & \cite{lei2024fairmindsim}                                                                                                                             \\
Psychological Traits    & Psychological metrics & Personality                                                                                                                                 & Automatic         & \cite{shao2023character}                                                                                                             \\
Psychological Traits    & Psychological metrics & Belief (stability, evolution, correlation with behavior)                                                                                    & Automatic         & \cite{shao2023character}                                                                                                                                 \\
Psychological Traits    & Psychological metrics & Emotion responses (entropy of valence and arousal)                                                                                          & Automatic         & \cite{shao2023character}                                                                                                                                \\
Psychological Traits    & Psychological metrics & Personality (Machine Personality Inventory, PsychoBench)                                                                                    & Automatic         & \cite{NEURIPS2023_21f7b745}                                               \\
Psychological Traits    & Psychological metrics & Personality (vignette tests)                                                                                                                & Human             & \cite{NEURIPS2023_21f7b745}                                               \\
Belief \& Value        & Social and decision-making metrics  & Social value orientation (SVO-based Value Rationality Measurement)                                                                          & Automatic         & \cite{zhang2023heterogeneous}                                                                                                                                \\
\end{longtable}
\end{center}
\end{small}
\twocolumn
% \label{tab: long_table_agent_metrics_src}

\onecolumn

\begin{small}
\begin{center}
\begin{longtable}{@{}p{0.1\textwidth}p{0.24\textwidth}p{0.3\textwidth}p{0.06\textwidth}p{0.2\textwidth}@{}}
\caption[Task-oriented evaluation metrics glossary]{Task-oriented evaluation metrics glossary.} \label{tab: long_table_task_metrics_src} \\

\toprule
\textbf{Task} & \textbf{Category} & \textbf{Task-oriented Metrics} & \textbf{Approach}  & \textbf{Source} \\ \midrule 
\endfirsthead

\\
\midrule 
\textbf{Task} & \textbf{Category} & \textbf{Task-oriented Metrics} & \textbf{Approach}  & \textbf{Source} \\ \midrule
\endhead

\hline \multicolumn{5}{c}{{Continued on next page}} \\ \hline
\endfoot

\hline \bottomrule
\endlastfoot

Decision Making          & Social and economic metrics         & Negotiation (Concession Rate, Negotiation Success Rate, Average Negotiation Round)                                                                                                                          & Automatic & \cite{huang2024personality}                                                                                                                                                                                                                                                                                                                                                                                                 \\
Decision Making          & Social and economic metrics         & Societal Satisfaction (average per-capita living area size, average waiting time, social welfare)                                                                                                           & Automatic & \cite{ji2024srap}                                                                                                                                                                                                                                                                                                                                                                                   \\
Decision Making          & Social and economic metrics         & Societal Fairness (variance in per capita living area size, number of inverse order pairs in house allocation, Gini coefficient)                                                                            & Automatic & \cite{ji2024srap}                                                                                                                                                                                                                                                                                                                                                                                   \\
Decision Making          & Social and economic metrics         & Macroeconomic (Inflation rate, Unemployment rate, Nominal GDP, Nominal GDP growth, Wage inflation, Real GDP growth, Expected monthly income, Consumption)                                                   & Automatic & \cite{li2024econagent}                                                                                                                                                                                                                                                                                                                                                                                                 \\
Decision Making          & Social and economic metrics         & Market and Consumer (Purchase probability, Expected competing product price, Customer counts, Price consistency between competitors)                                                                        & Automatic & \cite{Gui2023TheCO}                                                                                                                                                                                                                                                                                                                                                                                                      \\
Decision Making          & Social and economic metrics         & Market and Consumer (Purchase probability, Expected competing product price, Customer counts, Price consistency between competitors)                                                                        & Automatic & \cite{Zhao2023CompeteAIUT}                                                                                                                                                                                                                                                                                                                                                                                                      \\
Decision Making          & Social and economic metrics         & Probability weighting                                                                                                                                                                                       & Automatic & \cite{Jia2024DecisionMakingBE}                                                                                                                                                                                                                                                                                                                                                                                                      \\
Decision Making          & Social and economic metrics         & Utility (Intrinsic Utility, Joint Utility)                                                                                                                                                                  & Automatic & \cite{huang2024personality}                                                                                                                                                                                                                                                                                                                                                                                                 \\
Decision Making          & Psychological metrics & Level of trust (distribution of amounts sent, trust rate)                                                                                                                                                   & Automatic & \cite{Xie2024CanLL}                                                                                                                                                                                                                                                                                                                                                                                                      \\
Decision Making          & Psychological metrics & Risk preference                                                                                                                                                                                             & Automatic & \cite{Jia2024DecisionMakingBE}                                                                                                                                                                                                                                                                                                                                                                                                      \\
Decision Making          & Psychological metrics & Loss aversion                                                                                                                                                                                               & Automatic & \cite{Jia2024DecisionMakingBE}                                                                                                                                                                                                                                                                                                                                                                                                      \\
Decision Making          & Psychological metrics & Selfishness (Selfishness Index, Difference Index)                                                                                                                                                           & Automatic & \cite{Kim2024WillLS}                                                                                                                                                                                                                                                                                                                                                                               \\
Decision Making          & Performance metrics                 & Frequency (distribution of expert type)                                                                                                                                                                     & Automatic & \cite{Wang2024DEEMDE}                                                                                                                                                                                                                                                                                                                                                                                     \\
Decision Making          & Performance metrics                 & Valid response rate                                                                                                                                                                                         & Automatic & \cite{Xie2024CanLL}                                                                                                                                                                                                                                                                                                                                                                                                      \\
Decision Making          & Performance metrics                 & Web search quality (Mean reciprocal rank, Mean reciprocal rank)                                                                                                                                             & Automatic & \cite{Ren2024BASESLW}                                                                                                                                                                                                                                                                                                                                                                                 \\
Decision Making          & Performance metrics                 & Performance deviations/alignment from the baseline (accuracy, Jaccard Index, Cohen’s Kappa Coefficient, Percentage Agreement, overlapping ratio between prediction and targets)                             & Automatic & \cite{Kim2024WillLS}                                                                                                                                                                                                                                                                                                                                                                               \\
Decision Making          & Performance metrics                 & Performance deviations/alignment from the baseline (accuracy, Jaccard Index, Cohen’s Kappa Coefficient, Percentage Agreement, overlapping ratio between prediction and targets)                             & Automatic & \cite{Jin2024AgentReviewEP}                                                                                                                                                                                                                                                                                                                                                                                     \\
Decision Making          & Performance metrics                 & Performance deviations/alignment from the baseline (accuracy, Jaccard Index, Cohen’s Kappa Coefficient, Percentage Agreement, overlapping ratio between prediction and targets)                             & Automatic & \cite{Wang2024DEEMDE}                                                                                                                                                                                                                                                                                                                                                                                     \\
Decision Making          & Performance metrics                 & Performance deviations/alignment from the baseline (accuracy, Jaccard Index, Cohen’s Kappa Coefficient, Percentage Agreement, overlapping ratio between prediction and targets)                             & Automatic & \cite{wang-etal-2024-unleashing}                                                                                                                                                                                                                                                                                                                                                                                             \\
Decision Making          & Internal consistency metrics        & Behavioral alignment (lottery rate, behavior dynamic, Imitation and differentiation behavior, Proportion of similar and different dishes)                                                                   & Automatic & \cite{Xie2024CanLL}                                                                                                                                                                                                                                                                                                                                                                                               \\
Decision Making          & Internal consistency metrics        & Behavioral alignment (lottery rate, behavior dynamic, Imitation and differentiation behavior, Proportion of similar and different dishes)                                                                   & Automatic & \cite{Zhao2023CompeteAIUT}                                                                                                                                                                                                                                                                                                                                                                                                      \\
Decision Making          & Internal consistency metrics        & Cultural appropriateness (Alignment between persona information and its assigned nationality)                                                                                                               & LLM       & \cite{Li2024SchemaGuidedCC}                                                                                                                                                                                                                                                                                                                                                                                       \\
Decision Making          & External alignment metrics          & Factual hallucinations (String matching overlap ratio)                                                                                                                                                      & Automatic & \cite{wang-etal-2024-unleashing}                                                                                                                                                                                                                                                                                                                                                                                             \\
Decision Making          & External alignment metrics          & Simulation capability (Turing test)                                                                                                                                                                         & Human     & \cite{ji2024srap}                                                                                                                                                                                                                                                                                                                                                                                   \\
Decision Making          & External alignment metrics          & Entailment                                                                                                                                                                                                  & LLM       & \cite{Li2024SchemaGuidedCC}                                                                                                                                                                                                                                                                                                                                                                                       \\
Decision Making          & External alignment metrics          & Realism                                                                                                                                                                                                     & LLM       & \cite{Li2024SchemaGuidedCC}                                                                                                                                                                                                                                                                                                                                                                                       \\
Educational Training     & Psychological metrics & Perceived reflection on the development of essential non-cognitive skills                                                                                                                                   & Human     & \cite{Yan2024SocialLS}                                                                                                                                                                                                                                                                                                                                                                                                     \\
Educational Training     & Psychological metrics & Non-cognitive skill scale                                                                                                                                                                                   & Automatic & \cite{Yan2024SocialLS}                                                                                                                                                                                                                                                                                                                                                                                                     \\
Educational Training     & Psychological metrics & Sense of immersion / Perceived immersion                                                                                                                                                                    & Human     & \cite{lee2023generative} \\
Educational Training     & Psychological metrics & Perceived intelligence                                                                                                                                                                                      & Human     & \cite{Cheng2024LLMPoweredAT}                                                                                                                                                                                                                                                                                                                                                                                               \\
Educational Training     & Psychological metrics & Perceived enjoyment                                                                                                                                                                                         & Human     & \cite{Cheng2024LLMPoweredAT}                                                                                                                                                                                                                                                                                                                                                                                               \\
Educational Training     & Psychological metrics & Perceived trust                                                                                                                                                                                             & Human     & \cite{Cheng2024LLMPoweredAT}                                                                                                                                                                                                                                                                                                                                                                                               \\
Educational Training     & Psychological metrics & Perceived sense of connection                                                                                                                                                                               & Human     & \cite{Cheng2024LLMPoweredAT}                                                                                                                                                                                                                                                                                                                                                                                               \\
Educational Training     & Psychological metrics & Personality (Big Five Invertory, MBTI score, SD3 score, Linguistic Inquiry and Word Count framework, HEXACO)                                                                                                & Automatic & \cite{Sonlu2024TheEO}                                                                                                                                                                                                                                                                                                                                                                                                   \\
Educational Training     & Psychological metrics & Personality (Big Five Invertory, MBTI score, SD3 score, Linguistic Inquiry and Word Count framework, HEXACO)                                                                                                & Automatic & \cite{Liu2024PersonalityawareSS}                                                                                                                                                                                                                                                                                                                                                                                                 \\
Educational Training     & Psychological metrics & Perceived usefulness                                                                                                                                                                                        & Human     & \cite{Cheng2024LLMPoweredAT}                                                                                                                                                                                                                                                                                                                                                                                               \\
Educational Training     & Performance metrics                 & Density of knowledge-building                                                                                                                                                                               & Automatic & \cite{Jin2023TeachAH}                                                                                                                                                                                                                                                                                                                                                                                                  \\
Educational Training     & Performance metrics                 & Effectiveness of questioning                                                                                                                                                                                & Human     & \cite{Shi2023CGMICG}                                                                                                                                                                                                                                                                                                                                                                                                   \\
Educational Training     & Performance metrics                 & Success criterion function outputs before operation and after operation                                                                                                                                     & Human     & \cite{Li2023MetaAgentsSI}                                                                                                                                                                                                                                                                                                                                                                                                  \\
Educational Training     & External alignment metrics          & Knowledge level (reconfigurability, persistence, and adaptability)                                                                                                                                          & Automatic & \cite{Jin2023TeachAH}                                                                                                                                                                                                                                                                                                                                                                                                  \\
Educational Training     & External alignment metrics          & Perceived human-likeness                                                                                                                                                                                    & Human     & \cite{Cheng2024LLMPoweredAT}                                                                                                                                                                                                                                                                                                                                                                                               \\
Educational Training     & Content and textual metrics         & Story Content Generation (narratives staging score)                                                                                                                                                         & Automatic & \cite{Yan2024SocialLS}                                                                                                                                                                                                                                                                                                                                                                                                     \\
Educational Training     & Content and textual metrics         & Willingness to speak                                                                                                                                                                                        & Human     & \cite{Shi2023CGMICG}                                                                                                                                                                                                                                                                                                                                                                                                   \\
Educational Training     & Content and textual metrics         & Authenticity                                                                                                                                                                                                & Human     & \cite{lee2023generative} \\
Opinion Dynamics         & Psychological metrics & Opinion change                                                                                                                                                                                              & Human     & \cite{triem2024tipping}                                                                                                                                                                                                                                                                                                                                                         \\
Opinion Dynamics         & Psychological metrics & Emotional density                                                                                                                                                                                           & Automatic & \cite{Gao2023S3SS}                                                                                                                                                                                                                                                                                                                                                                                                  \\
Opinion Dynamics         & Performance metrics                 & Prediction accuracy (F1 score, AUC, MSE, MAE, depression risk prediction accuracy, suicide risk prediction accuracy)                                                                                        & Automatic & \cite{Gao2023S3SS}                                                                                                                                                                                                                                                                                                                                                                                                  \\
Opinion Dynamics         & Performance metrics                 & Prediction accuracy (F1 score, AUC, MSE, MAE, depression risk prediction accuracy, suicide risk prediction accuracy)                                                                                        & Automatic & \cite{Mou2024UnveilingTT}                                                                                                                                                                                                                                                                                                                                                                                                   \\
Opinion Dynamics         & Performance metrics                 & Prediction accuracy (F1 score, AUC, MSE, MAE, depression risk prediction accuracy, suicide risk prediction accuracy)                                                                                        & Automatic & \cite{Yu2024TowardsMA}                                                                                                                                                                                                                                                                                                                                                                                                   \\
Opinion Dynamics         & Performance metrics                 & Classification accuracy                                                                                                                                                                                     & Human     & \cite{Chan2023ChatEvalTB}                                                                                                                                                                                                                                                                                                                                                                                                  \\
Opinion Dynamics         & Performance metrics                 & Rephrase accuracy                                                                                                                                                                                           & Automatic & \cite{Ju2024FloodingSO}                                                                                                                                                                                                                                                                                                                                                                                                \\
Opinion Dynamics         & Performance metrics                 & Legal articles evaluation (precision, recall, F1)                                                                                                                                                           & Automatic & \cite{He2024AgentsCourtBJ}                                                                                                                                                                                                                                                                                                                                                                                \\
Opinion Dynamics         & Performance metrics                 & Judgment evaluation for civil and administrative cases (precision, recall, F1)                                                                                                                              & Automatic & \cite{He2024AgentsCourtBJ}                                                                                                                                                                                                                                                                                                                                                                                \\
Opinion Dynamics         & Performance metrics                 & Judgment evaluation for criminal cases (accuracy)                                                                                                                                                           & Automatic & \cite{He2024AgentsCourtBJ}                                                                                                                                                                                                                                                                                                                                                                                \\
Opinion Dynamics         & Performance metrics                 & Prediction error rate                                                                                                                                                                                       & Automatic & \cite{Gao2023S3SS}                                                                                                                                                                                                                                                                                                                                                                                                  \\
Opinion Dynamics         & Performance metrics                 & Locality accuracy                                                                                                                                                                                           & Automatic & \cite{Ju2024FloodingSO}                                                                                                                                                                                                                                                                                                                                                                                                \\
Opinion Dynamics         & Performance metrics                 & Decision probability                                                                                                                                                                                        & Human     & \cite{triem2024tipping}                                                                                                                                                                                                                                                                                                                                                         \\
Opinion Dynamics         & Performance metrics                 & Decision volatility                                                                                                                                                                                         & Human     & \cite{triem2024tipping}                                                                                                                                                                                                                                                                                                                                                         \\
Opinion Dynamics         & Performance metrics                 & Case complexity                                                                                                                                                                                             & Human     & \cite{triem2024tipping}                                                                                                                                                                                                                                                                                                                                                         \\
Opinion Dynamics         & Performance metrics                 & Alignment (compare simulation results with actual social outcomes)                                                                                                                                          & Automatic & \cite{Wang2024IntelligentCS}                                                                                                                                                                                                                                                                                                                                                                                                  \\
Opinion Dynamics         & Internal consistency metrics        & Alignment (stance, content, behavior, static attitude distribution, time series of the average attitude)                                                                                                    & Automatic & \cite{Mou2024UnveilingTT}                                                                                                                                                                                                                                                                                                                                                                                               \\
Opinion Dynamics         & Internal consistency metrics        & Personality-behavior alignment                                                                                                                                                                              & Human     & \cite{Navarro2024DesigningRE}                                                                                                                                                                                                                                                                                                                                                                                                \\
Opinion Dynamics         & Internal consistency metrics        & Similarity between initial and post preference (KL-divergence, RMSE)                                                                                                                                        & Automatic & \cite{Namikoshi2024UsingLT}                                                                                                                                                                                                                                                                                                                                                                                                   \\
Opinion Dynamics         & Internal consistency metrics        & Role playing                                                                                                                                                                                                & Human     & \cite{lv2024coggpt}                                                                                                                                                                                                                                                                                                                                                                                              \\
Opinion Dynamics         & External alignment metrics          & Correctness                                                                                                                                                                                                 & Human     & \cite{He2024AgentsCourtBJ}                                                                                                                                                                                                                                                                                                                                                                                \\
Opinion Dynamics         & External alignment metrics          & Accuracy (correctness)                                                                                                                                                                                      & Automatic & \cite{Ju2024FloodingSO}                                                                                                                                                                                                                                                                                                                                                                                                   \\
Opinion Dynamics         & External alignment metrics          & Logicality                                                                                                                                                                                                  & Human     & \cite{He2024AgentsCourtBJ}                                                                                                                                                                                                                                                                                                                                                                                \\
Opinion Dynamics         & External alignment metrics          & Concision                                                                                                                                                                                                   & Human     & \cite{He2024AgentsCourtBJ}                                                                                                                                                                                                                                                                                                                                                                                \\
Opinion Dynamics         & External alignment metrics          & Human likeness index                                                                                                                                                                                        & Automatic & \cite{Chuang2023TheWO}                                                                                                                                                                                                                                                                                                                                                                                        \\
Opinion Dynamics         & External alignment metrics          & Alignment between model and human (Kappa correlation coefficient, MAE), Authenticity (alignment of ratings between the agent and human annotators)                                                          & Human     & \cite{Chan2023ChatEvalTB}                                                                                                                                                                                                                                                                                                                                                                                                  \\
Opinion Dynamics         & External alignment metrics          & Alignment between model and human (Kappa correlation coefficient, MAE), Authenticity (alignment of ratings between the agent and human annotators)                                                          & Human     & \cite{triem2024tipping}                                                                                                                                                                                                                                                                                                                                                                  \\
Opinion Dynamics         & External alignment metrics          & Alignment between model and human (Kappa correlation coefficient, MAE), Authenticity (alignment of ratings between the agent and human annotators)                                                          & Human     & \cite{lv2024coggpt}                                                                                                                                                                                                                                                                                                                                                                                              \\
Opinion Dynamics         & Content and textual metrics         & Turn-level Kendall-Tau correlation (naturalness, coherence, engagingness and groundedness)                                                                                                                  & Automatic & \cite{Chan2023ChatEvalTB}                                                                                                                                                                                                                                                                                                                                                                                                  \\
Opinion Dynamics         & Content and textual metrics         & Turn-level Spearman correlation (naturalness, coherence, engagingness and groundedness)                                                                                                                     & Automatic & \cite{Chan2023ChatEvalTB}                                                                                                                                                                                                                                                                                                                                                                                                  \\
Opinion Dynamics         & Bias, fairness, and ethic metrics   & Partisan bias                                                                                                                                                                                               & Automatic & \cite{Chuang2023TheWO}                                                                                                                                                                                                                                                                                                                                                                                        \\
Opinion Dynamics         & Bias, fairness, and ethic metrics   & Bias (cultural, linguistic, economic, demographic, ideological)                                                                                                                                             & Automatic & \cite{qu2024performance}                                                                                                                                                                                                                                                                                                                                                                                  \\
Opinion Dynamics         & Bias, fairness, and ethic metrics   & Bias (mean)                                                                                                                                                                                                 & Automatic & \cite{chuang2023simulating}                                                                                                                                                                                                                                                                                                                                                                                           \\
Opinion Dynamics         & Bias, fairness, and ethic metrics   & Extreme values                                                                                                                                                                                              & Automatic & \cite{Chuang2023TheWO}                                                                                                                                                                                                                                                                                                                                                                                        \\
Opinion Dynamics         & Bias, fairness, and ethic metrics   & Wisdom of Partisan Crowds effect                                                                                                                                                                            & Automatic & \cite{Chuang2023TheWO}                                                                                                                                                                                                                                                                                                                                                                                        \\
Opinion Dynamics         & Bias, fairness, and ethic metrics   & Opinion diversity                                                                                                                                                                                           & Automatic & \cite{chuang2023simulating}                                                                                                                                                                                                                                                                                                                                                                                           \\
Psychological Experiment & Social and economic metrics         & Money allocation                                                                                                                                                                                            & Automatic & \cite{lei2024fairmindsim}                                                                                                                                                                                                                                                                                                                                                                                                   \\
Psychological Experiment & Psychological metrics & Attitude change                                                                                                                                                                                             & Automatic & \cite{Wang2023UserBS}                                                                                                                                                                                                                                                                                                                                                                                            \\
Psychological Experiment & Psychological metrics & Average happiness value per time step                                                                                                                                                                       & Automatic & \cite{He2024AFSPPAF}                                                                                                                                                                                                                                                                                                                                                                                                \\
Psychological Experiment & Psychological metrics & Belief value                                                                                                                                                                                                & Automatic & \cite{lei2024fairmindsim}                                                                                                                                                                                                                                                                                                                                                                                                   \\
Psychological Experiment & Psychological metrics & Personality (Big Five Invertory, MBTI score, SD3 score, Linguistic Inquiry and Word Count framework, HEXACO)                                                                                                & Automatic & \cite{He2024AFSPPAF}                                                                                                                                                                                                                                                                                                                                                                                                \\
Psychological Experiment & Psychological metrics & Personality (Big Five Invertory, MBTI score, SD3 score, Linguistic Inquiry and Word Count framework, HEXACO)                                                                                                & Automatic & \cite{deWinter2024TheUO}                                                                                                        \\
Psychological Experiment & Psychological metrics & Personality (Big Five Invertory, MBTI score, SD3 score, Linguistic Inquiry and Word Count framework, HEXACO)                                                                                                & Automatic & \cite{bose2024assessing}                                                                                                                                                                                                                                                                                                                                                                                      \\
Psychological Experiment & Psychological metrics & Personality (Big Five Invertory, MBTI score, SD3 score, Linguistic Inquiry and Word Count framework, HEXACO)                                                                                                & Automatic & \cite{Jiang2023PersonaLLMIT}                                                                                                                                                                                                                                                                                                                                                                              \\
Psychological Experiment & Psychological metrics & Longitudinal trajectories of emotions                                                                                                                                                                       & Automatic & \cite{de2025introducing}                                                                                                                                                                                                                                                                                                                \\
Psychological Experiment & Psychological metrics & Valence entropy                                                                                                                                                                                             & Automatic & \cite{lei2024fairmindsim}                                                                                                                                                                                                                                                                                                                                                                                                   \\
Psychological Experiment & Psychological metrics & Arousal entropy                                                                                                                                                                                             & Automatic & \cite{lei2024fairmindsim}                                                                                                                                                                                                                                                                                                                                                                                                   \\
Psychological Experiment & Performance metrics                 & Precision of item recommendation                                                                                                                                                                            & Automatic & \cite{Wang2023UserBS}                                                                                                                                                                                                                                                                                                                                                                                            \\
Psychological Experiment & Performance metrics                 & Missing rate                                                                                                                                                                                                & Automatic & \cite{lei2024fairmindsim}                                                                                                                                                                                                                                                                                                                                                                                                   \\
Psychological Experiment & Performance metrics                 & Rejection rate                                                                                                                                                                                              & Automatic & \cite{lei2024fairmindsim}                                                                                                                                                                                                                                                                                                                                                                                                   \\
Psychological Experiment & Internal consistency metrics        & Correlation between social dilemma game outcome and agent personality                                                                                                                                       & Automatic & \cite{bose2024assessing}                                                                                                                                                                                                                                                                                                                                                                                      \\
Psychological Experiment & Internal consistency metrics        & Behavioral similarity                                                                                                                                                                                       & Automatic & \cite{Li2024EvolvingAI}                                                                                                                                                                                                                                                                                                                                                                                                    \\
Psychological Experiment & Internal consistency metrics        & Perception consistency (agent perceived safety, agent perceived liveliness)                                                                                                                                 & LLM       & \cite{Verma2023GenerativeAI}                                                                                                                                                                                                                                                                                                                                                                                                    \\
Psychological Experiment & External alignment metrics          & Rationality of the agent memory                                                                                                                                                                             & Automatic & \cite{Wang2023UserBS}                                                                                                                                                                                                                                                                                                                                                                                            \\
Psychological Experiment & External alignment metrics          & Believability of behavior                                                                                                                                                                                   & Automatic & \cite{Wang2023UserBS}                                                                                                                                                                                                                                                                                                                                                                                                 \\
Psychological Experiment & Content and textual metrics         & Salience of individual words                                                                                                                                                                                & Automatic & \cite{de2025introducing}                                                                                                                                                                                                                                                                                                           \\
Psychological Experiment & Content and textual metrics         & Absolutist words                                                                                                                                                                                            & Automatic & \cite{de2025introducing}                                                                                                                                                                                                                                                                                                           \\
Psychological Experiment & Content and textual metrics         & Personal pronouns or emotions                                                                                                                                                                               & Automatic & \cite{de2025introducing}                                                                                                                                                                                                                                                                                                           \\
Psychological Experiment & Content and textual metrics         & Information entropy                                                                                                                                                                                         & Automatic & \cite{Wang2023UserBS}                                                                                                                                                                                                                                                                                                                                                                                            \\
Psychological Experiment & Content and textual metrics         & Story (readability, personalness, redundancy, cohesiveness, likeability, believability)                                                                                                                     & Human     & \cite{Jiang2023PersonaLLMIT}                                                                                                                                                                                                                                                                                                                                                                              \\
Psychological Experiment & Content and textual metrics         & Story (readability, personalness, redundancy, cohesiveness, likeability, believability)                                                                                                                     & LLM       & \cite{Jiang2023PersonaLLMIT}                                                                                                                                                                                                                                                                                                                                                                              \\
Simulated Individual     & Social and economic metrics         & Numbers of generated peer support strategies                                                                                                                                                                & Automatic & \cite{Liu2024ComPeerAG}                                                                                                                                                                                                                                                                                                                                                                              \\
Simulated Individual     & Social and economic metrics         & Perceived social support questionnaire                                                                                                                                                                      & Human     & \cite{Liu2024ComPeerAG}                                                                                                                                                                                                                                                                                                                                                                              \\
Simulated Individual     & Psychological metrics & Emotions                                                                                                                                                                                                    & Human     & \cite{Pataranutaporn2024FutureYA}                                                                                                                                                                                                                                                                                                                                                                                                  \\
Simulated Individual     & Psychological metrics & Agency                                                                                                                                                                                                      & Human     & \cite{Pataranutaporn2024FutureYA}                                                                                                                                                                                                                                                                                                                                                                                                  \\
Simulated Individual     & Psychological metrics & Future consideration                                                                                                                                                                                        & Human     & \cite{Pataranutaporn2024FutureYA}                                                                                                                                                                                                                                                                                                                                                                                                  \\
Simulated Individual     & Psychological metrics & Self-reflection                                                                                                                                                                                             & Human     & \cite{Pataranutaporn2024FutureYA}                                                                                                                                                                                                                                                                                                                                                                                                  \\
Simulated Individual     & Psychological metrics & Insight                                                                                                                                                                                                     & Human     & \cite{Pataranutaporn2024FutureYA}                                                                                                                                                                                                                                                                                                                                                                                                  \\
Simulated Individual     & Psychological metrics & Persona Perception Scale                                                                                                                                                                                    & Human     & \cite{Salminen2024PicturingTF}                                                                                                                                                                                                                                                                                                                                                                   \\
Simulated Individual     & Psychological metrics & Persona Perception Scale                                                                                                                                                                                    & Human     & \cite{10.1145/3643834.3660729}                                                                                                                                                                                                                                                                                                                                                                                \\
Simulated Individual     & Psychological metrics & Persona Perception Scale                                                                                                                                                                                    & Human     & \cite{Ha2024CloChatUH}                                                                                                                                                                                                                                                                                                                                                                                \\
Simulated Individual     & Psychological metrics & Persona Perception Scale                                                                                                                                                                                    & Human     & \cite{10.1145/3613904.3642363}                                                                                                                                                                                                                                                                                                                                                                                    \\
Simulated Individual     & Psychological metrics & Engagement                                                                                                                                                                                                  & Human     & \cite{Zhang2024SpeechAgentsHS}                                                                                                                                                                                                                                                                                                                                                                                                   \\
Simulated Individual     & Psychological metrics & Safety                                                                                                                                                                                                      & Human     & \cite{Zhang2024SpeechAgentsHS}                                                                                                                                                                                                                                                                                                                                                                                                   \\
Simulated Individual     & Psychological metrics & Sensitivity to personalization                                                                                                                                                                              & Automatic & \cite{giorgi2024humanllmbiaseshate}                                                                                                                                                                                                                                                                                                                                                                                                    \\
Simulated Individual     & Psychological metrics & Agent self-awareness                                                                                                                                                                                        & LLM       & \cite{xie2024humansimulacrabenchmarkingpersonification}                                                                                                                                                                                                                                                                                                                                                                                                  \\
Simulated Individual     & Psychological metrics & Personality (Big Five Invertory rated by LLM)                                                                                                                                                               & LLM       & \cite{NEURIPS2023_21f7b745}                                                                                                                                                                                                                                                                                                                    \\
Simulated Individual     & Psychological metrics & Positively mention rate                                                                                                                                                                                     & Automatic & \cite{kamruzzaman2024exploringchangesnationperception}                                                                                                                                                                                                                                                                                                                                                                                            \\
Simulated Individual     & Psychological metrics & Optimism                                                                                                                                                                                                    & Human     & \cite{Pataranutaporn2024FutureYA}                                                                                                                                                                                                                                                                                                                                                                                                  \\
Simulated Individual     & Psychological metrics & Self-esteem                                                                                                                                                                                                 & Human     & \cite{Pataranutaporn2024FutureYA}                                                                                                                                                                                                                                                                                                                                                                                                  \\
Simulated Individual     & Psychological metrics & Pressure perceived scale                                                                                                                                                                                    & Human     & \cite{Liu2024ComPeerAG}                                                                                                                                                                                                                                                                                                                                                                              \\
Simulated Individual     & Performance metrics                 & Error rates (error of average, error of dispersion)                                                                                                                                                         & Automatic & \cite{lin2024diversedialoguemethodologydesigningchatbots}                                                                                                                                                                                                                                                                                                                                                                                            \\
Simulated Individual     & Performance metrics                 & Model fit indices (Chi-square to degrees of freedom ratio, Comparative Fit Index, Tucker-Lewis Index, Root Mean Square Error of Approximation)                                                              & Automatic & \cite{Ke2024HumanAISI}                                                                                                                                                                                                                                                                                                                                                                                           \\
Simulated Individual     & Performance metrics                 & Knowledge accuracy (WikiRoleEval with human evaluators)                                                                                                                                                     & Human     & \cite{tang2024erabalenhancingroleplayingagents}                                                                                                                                                                                                                                                                                                                                                                                              \\
Simulated Individual     & Performance metrics                 & Knowledge accuracy (WikiRoleEval)                                                                                                                                                                           & LLM       & \cite{tang2024erabalenhancingroleplayingagents}                                                                                                                                                                                                                                                                                                                                                                                              \\
Simulated Individual     & Performance metrics                 & Win rates                                                                                                                                                                                                   & Automatic & \cite{chi2024amongagentsevaluatinglargelanguage}                                                                                                                                                                                                                                                                                                                                                                                             \\
Simulated Individual     & Performance metrics                 & Comprehension                                                                                                                                                                                               & Automatic & \cite{10.1145/3643834.3660729}                                                                                                                                                                                                                                                                                                                                                                                \\
Simulated Individual     & Performance metrics                 & Completeness                                                                                                                                                                                                & Automatic & \cite{10.1145/3643834.3660729}                                                                                                                                                                                                                                                                                                                                                                                \\
Simulated Individual     & Performance metrics                 & Validity (average variance extracted, inter-construct correlations)                                                                                                                                         & Automatic & \cite{Ke2024HumanAISI}                                                                                                                                                                                                                                                                                                                                                                                           \\
Simulated Individual     & Performance metrics                 & Composite reliability                                                                                                                                                                                       & Automatic & \cite{Ke2024HumanAISI}                                                                                                                                                                                                                                                                                                                                                                                           \\
Simulated Individual     & Performance metrics                 & Rated statement quality                                                                                                                                                                                     & Human     & \cite{liu2023improvinginterpersonalcommunicationsimulating}                                                                                                                                                                                                                                                                                                                                                                                              \\
Simulated Individual     & Performance metrics                 & Rated statement quality                                                                                                                                                                                     & LLM       & \cite{liu2023improvinginterpersonalcommunicationsimulating}                                                                                                                                                                                                                                                                                                                                                                                              \\
Simulated Individual     & Performance metrics                 & Conversational ability (CharacterEval)                                                                                                                                                                      & LLM       & \cite{tang2024erabalenhancingroleplayingagents}                                                                                                                                                                                                                                                                                                                                                                                              \\
Simulated Individual     & Performance metrics                 & Roleplay subset of MT-Bench                                                                                                                                                                                 & LLM       & \cite{tang2024erabalenhancingroleplayingagents}                                                                                                                                                                                                                                                                                                                                                                                              \\
Simulated Individual     & Performance metrics                 & Professional scale (accuracy in replicating profession-specific knowledge)                                                                                                                                  & LLM       & \cite{sun2024identitydrivenhierarchicalroleplayingagents}                                                                                                                                                                                                                                                                                                                                                                                           \\
Simulated Individual     & Performance metrics                 & Language quality                                                                                                                                                                                            & LLM       & \cite{Zhang2024SpeechAgentsHS}                                                                                                                                                                                                                                                                                                                                                                                                   \\
Simulated Individual     & Performance metrics                 & Prediction accuracy between real data and generated data (Replication success rate, Kullback-Leibler divergence)                                                                                            & Automatic & \cite{assaf2024human}                                                                                                                                                                                                                                                                                                                                                                     \\
Simulated Individual     & Performance metrics                 & Prediction accuracy between real data and generated data (Replication success rate, Kullback-Leibler divergence)                                                                                            & Automatic & \cite{tamaki_2024_chronon}                                                                                                                                                                                                                                                                                                                                                                                                  \\
Simulated Individual     & Performance metrics                 & Prediction accuracy between real data and generated data (Replication success rate, Kullback-Leibler divergence)                                                                                            & Automatic & \cite{park2024generativeagentsimulations1000}                                                                                                                                                                                                                                                                                                                                                                                                \\
Simulated Individual     & Performance metrics                 & Prediction accuracy between real data and generated data (Replication success rate, Kullback-Leibler divergence)                                                                                            & Automatic & \cite{yeykelis2024usinglargelanguagemodels}                                                                                                                                                                                                                                                                                                                                                                                             \\
Simulated Individual     & Performance metrics                 & Accuracy of distinguishing between AI-generated and human-built solutions                                                                                                                                   & Automatic & \cite{10.1145/3613905.3650860}                                                                                                                                                                                                                                                                                                                                                                               \\
Simulated Individual     & Internal consistency metrics        & Accuracy of reaction based on social relationship                                                                                                                                                           & Automatic & \cite{liu2024roleagent}                                                                                                                                                                                                                                                                                                                                                                                           \\
Simulated Individual     & Internal consistency metrics        & Perceived connection between personas and system outcomes                                                                                                                                                   & Human     & \cite{10.1145/3613904.3642363}                                                                                                                                                                                                                                                                                                                                                                                    \\
Simulated Individual     & Internal consistency metrics        & Representativeness (Wasserstein distance, respond with similar answers to individual survey questions), Consistency (Frobenius norm, the correlation across responses to a set of questions in each survey) & Automatic & \cite{moon2024virtualpersonaslanguagemodels}                                                                                                                                                                                                                                                                                                                                                                                                 \\
Simulated Individual     & Internal consistency metrics        & Role consistency (WikiRoleEval with human evaluators)                                                                                                                                                       & Human     & \cite{tang2024erabalenhancingroleplayingagents}                                                                                                                                                                                                                                                                                                                                                                                              \\
Simulated Individual     & Internal consistency metrics        & Role consistency/attractiveness (WikiRoleEval, CharacterEval)                                                                                                                                               & LLM       & \cite{tang2024erabalenhancingroleplayingagents}                                                                                                                                                                                                                                                                                                                                                                                              \\
Simulated Individual     & Internal consistency metrics        & Consistency                                                                                                                                                                                                 & Human     & \cite{Zhang2024SpeechAgentsHS}                                                                                                                                                                                                                                                                                                                                                                                                   \\
Simulated Individual     & Internal consistency metrics        & Consistency                                                                                                                                                                                                 & Human     & \cite{mishra-etal-2023-e}                                                                                                                                                                                                                                                                                                                                                                                       \\
Simulated Individual     & Internal consistency metrics        & Future self-continuity                                                                                                                                                                                      & Human     & \cite{Pataranutaporn2024FutureYA}                                                                                                                                                                                                                                                                                                                                                                                                  \\
Simulated Individual     & Internal consistency metrics        & Agreement between a synthetic annotator both with and without a leave-one-out attribute (Cohen's Kappa)                                                                                                     & Automatic & \cite{castricato2024personareproducibletestbedpluralistic}                                                                                                                                                                                                                                                                                                                                                                                         \\
Simulated Individual     & Internal consistency metrics        & Consistency with the scenario and characters                                                                                                                                                                & Automatic & \cite{Zhang2024SpeechAgentsHS}                                                                                                                                                                                                                                                                                                                                                                                                   \\
Simulated Individual     & Internal consistency metrics        & Quality and logical coherence of the script content                                                                                                                                                         & Automatic & \cite{Zhang2024SpeechAgentsHS}                                                                                                                                                                                                                                                                                                                                                                                                   \\
Simulated Individual     & Internal consistency metrics        & Nation-related response percentage                                                                                                                                                                          & Automatic & \cite{kamruzzaman2024exploringchangesnationperception}                                                                                                                                                                                                                                                                                                                                                                                            \\
Simulated Individual     & External alignment metrics          & Unknown question rejection (WikiRoleEval with human evaluators)                                                                                                                                             & Human     & \cite{tang2024erabalenhancingroleplayingagents}                                                                                                                                                                                                                                                                                                                                                                                              \\
Simulated Individual     & External alignment metrics          & Unknown question rejection (WikiRoleEval)                                                                                                                                                                   & LLM       & \cite{tang2024erabalenhancingroleplayingagents}                                                                                                                                                                                                                                                                                                                                                                                              \\
Simulated Individual     & External alignment metrics          & Accuracy of self-knowledge                                                                                                                                                                                  & Automatic & \cite{liu2024roleagent}                                                                                                                                                                                                                                                                                                                                                                                           \\
Simulated Individual     & External alignment metrics          & Correctness                                                                                                                                                                                                 & Human     & \cite{Zhang2024SpeechAgentsHS}                                                                                                                                                                                                                                                                                                                                                                                                   \\
Simulated Individual     & External alignment metrics          & Correctness                                                                                                                                                                                                 & Human     & \cite{milivcka2024large}                                                                                                                                                                                                                                                                                                                                                            \\
Simulated Individual     & External alignment metrics          & Agreement score between human raters and LLM,                                                                                                                                                               & Automatic & \cite{liu2023improvinginterpersonalcommunicationsimulating}                                                                                                                                                                                                                                                                                                                                                                                              \\
Simulated Individual     & External alignment metrics          & Agreement score between human raters and LLM,                                                                                                                                                               & Automatic & \cite{NEURIPS2023_21f7b745}                                                                                                                                                                                                                                                                                                                    \\
Simulated Individual     & External alignment metrics          & Agreement score between human raters and LLM,                                                                                                                                                               & Automatic & \cite{liu2024roleagent}                                                                                                                                                                                                                                                                                                                                                                                           \\
Simulated Individual     & External alignment metrics          & Human-likeness                                                                                                                                                                                              & Human     & \cite{Zhang2024SpeechAgentsHS}                                                                                                                                                                                                                                                                                                                                                                                                   \\
Simulated Individual     & Content and textual metrics         & Content similarity (ROUGE-L, BERTScore, GPT-based-similarity, G-eval)                                                                                                                                       & Automatic & \cite{10.1145/3643834.3660729}                                                                                                                                                                                                                                                                                                                                                                                \\
Simulated Individual     & Content and textual metrics         & Entity density of summarization                                                                                                                                                                             & Automatic & \cite{liu2024roleagent}                                                                                                                                                                                                                                                                                                                                                                                           \\
Simulated Individual     & Content and textual metrics         & Entity recall of summarization                                                                                                                                                                              & Automatic & \cite{liu2024roleagent}                                                                                                                                                                                                                                                                                                                                                                                           \\
Simulated Individual     & Content and textual metrics         & Dialog diversity                                                                                                                                                                                            & Automatic & \cite{lin2024diversedialoguemethodologydesigningchatbots}                                                                                                                                                                                                                                                                                                                                                                                            \\
Simulated Individual     & Bias, fairness, and ethic metrics   & Hate speech detection accuracy                                                                                                                                                                              & Automatic & \cite{giorgi2024humanllmbiaseshate}                                                                                                                                                                                                                                                                                                                                                                                                    \\
Simulated Individual     & Bias, fairness, and ethic metrics   & Population heterogeneity                                                                                                                                                                                    & Automatic & \cite{murthy2024fishfishseaalignment}                                                                                                                                                                                                                                                                                                                                                                                                \\
Simulated Society        & Social and economic metrics         & Social Conflict Count                                                                                                                                                                                       & Automatic & \cite{ren2024emergencesocialnormsgenerative}                                                                                                                                                                                                                                                                                                                                                                                                     \\
Simulated Society        & Social and economic metrics         & Social Rules                                                                                                                                                                                                & Human     & \cite{zhou2024sotopiainteractiveevaluationsocial}                                                                                                                                                                                                                                                                                                                                                                                              \\
Simulated Society        & Social and economic metrics         & Social Rules                                                                                                                                                                                                & LLM       & \cite{zhou2024sotopiainteractiveevaluationsocial}                                                                                                                                                                                                                                                                                                                                                                                              \\
Simulated Society        & Social and economic metrics         & Financial and Material Benefits                                                                                                                                                                             & Human     & \cite{zhou2024sotopiainteractiveevaluationsocial}                                                                                                                                                                                                                                                                                                                                                                                              \\
Simulated Society        & Social and economic metrics         & Financial and Material Benefits                                                                                                                                                                             & LLM       & \cite{zhou2024sotopiainteractiveevaluationsocial}                                                                                                                                                                                                                                                                                                                                                                                              \\
Simulated Society        & Social and economic metrics         & Converged price                                                                                                                                                                                             & Automatic & \cite{toledozucco2024scatteringpassivestructurepreservingfiniteelement}                                                                                                                                                                                                                                                                                                                                                                                              \\
Simulated Society        & Social and economic metrics         & Information diffusion                                                                                                                                                                                       & Automatic & \cite{park2023generative}                                                                                                                                                                                                                                                                                                                                                                                \\
Simulated Society        & Social and economic metrics         & Relationship formation                                                                                                                                                                                      & Automatic & \cite{park2023generative}                                                                                                                                                                                                                                                                                                                                                                                \\
Simulated Society        & Social and economic metrics         & Relationship                                                                                                                                                                                                & LLM       & \cite{zhou2024sotopiainteractiveevaluationsocial}                                                                                                                                                                                                                                                                                                                                                                                              \\
Simulated Society        & Social and economic metrics         & Coordination within other agents                                                                                                                                                                            & Automatic & \cite{park2023generative}                                                                                                                                                                                                                                                                                                                                                                                \\
Simulated Society        & Social and economic metrics         & Probability of social connection formation                                                                                                                                                                  & Automatic & \cite{leng2024llmagentsexhibitsocial}                                                                                                                                                                                                                                                                                                                                                                                              \\
Simulated Society        & Social and economic metrics         & Percent of social welfare maximization choices                                                                                                                                                              & Automatic & \cite{leng2024llmagentsexhibitsocial}                                                                                                                                                                                                                                                                                                                                                                                              \\
Simulated Society        & Social and economic metrics         & Persuasion (distribution of persuasion outcomes, odds ratios)                                                                                                                                               & Automatic & \cite{campedelli2024iwantbreakfree}                                                                                                                                                                                                                                                                                                                                                                                                \\
Simulated Society        & Social and economic metrics         & Anti-social behavior (effect on toxic messages)                                                                                                                                                             & Automatic & \cite{campedelli2024iwantbreakfree}                                                                                                                                                                                                                                                                                                                                                                                                \\
Simulated Society        & Social and economic metrics         & Norm Internalization Rate                                                                                                                                                                                   & Automatic & \cite{ren2024emergencesocialnormsgenerative}                                                                                                                                                                                                                                                                                                                                                                                                     \\
Simulated Society        & Social and economic metrics         & Norm Compliance Rate                                                                                                                                                                                        & Automatic & \cite{ren2024emergencesocialnormsgenerative}                                                                                                                                                                                                                                                                                                                                                                                                     \\
Simulated Society        & Psychological metrics & NASA-TLX Scores                                                                                                                                                                                             & Human     & \cite{10.1145/3613904.3642545}                                                                                                                                                                                                                                                                                                                                                                          \\
Simulated Society        & Psychological metrics & Helpfulness rating                                                                                                                                                                                          & Human     & \cite{10.1145/3613904.3642545}                                                                                                                                                                                                                                                                                                                                                                          \\
Simulated Society        & Psychological metrics & Personality (Big Five Invertory, MBTI score, SD3 score, Linguistic Inquiry and Word Count framework, HEXACO)                                                                                                & Automatic & \cite{frisch2024llmagentsinteractionmeasuring}                                                                                                                                                                                                                                                                                                                                                                                      \\
Simulated Society        & Psychological metrics & Personality (Big Five Invertory, MBTI score, SD3 score, Linguistic Inquiry and Word Count framework, HEXACO)                                                                                                & Automatic & \cite{Li2024EvolvingAI}                                                                                                                                                                                                                                                                                                                                                                                                    \\
Simulated Society        & Psychological metrics & Degree of reciprocity                                                                                                                                                                                       & Automatic & \cite{leng2024llmagentsexhibitsocial}                                                                                                                                                                                                                                                                                                                                                                                              \\
Simulated Society        & Psychological metrics & Pleasure rating                                                                                                                                                                                             & Human     & \cite{10.1145/3613904.3642545}                                                                                                                                                                                                                                                                                                                                                                          \\
Simulated Society        & Psychological metrics & Trend of Favorability Decline                                                                                                                                                                               & Automatic & \cite{gu2024agentgroupchatinteractivegroupchat}                                                                                                                                                                                                                                                                                                                                                                                        \\
Simulated Society        & Psychological metrics & Negative Favorability Achievement                                                                                                                                                                           & Automatic & \cite{gu2024agentgroupchatinteractivegroupchat}                                                                                                                                                                                                                                                                                                                                                                                        \\
Simulated Society        & Psychological metrics & Trend of Favorability Decline                                                                                                                                                                               & Automatic & \cite{gu2024agentgroupchatinteractivegroupchat}                                                                                                                                                                                                                                                                                                                                                                                        \\
Simulated Society        & Psychological metrics & Negative Favorability Achievement                                                                                                                                                                           & Automatic & \cite{gu2024agentgroupchatinteractivegroupchat}                                                                                                                                                                                                                                                                                                                                                                                        \\
Simulated Society        & Performance metrics                 & Abstention accuracy                                                                                                                                                                                         & Automatic & \cite{ashkinaze2024pluralsguidingllmssimulated}                                                                                                                                                                                                                                                                                                                                                                                              \\
Simulated Society        & Performance metrics                 & Accuracy of information gathering                                                                                                                                                                           & Automatic & \cite{kaiya2023lyfeagentsgenerativeagents}                                                                                                                                                                                                                                                                                                                                                                                              \\
Simulated Society        & Performance metrics                 & Implicit reasoning accuracy                                                                                                                                                                                 & Automatic & \cite{mou2024agentsensebenchmarkingsocialintelligence}                                                                                                                                                                                                                                                                                                                                                                                               \\
Simulated Society        & Performance metrics                 & Prediction accuracy (F1 score, AUC, MSE, MAE, depression risk prediction accuracy, suicide risk prediction accuracy)                                                                                        & Automatic & \cite{lan2024depressiondiagnosisdialoguesimulation}                                                                                                                                                                                                                                                                                                                                                                                          \\
Simulated Society        & Performance metrics                 & Guess accuracy                                                                                                                                                                                              & Automatic & \cite{leng2024llmagentsexhibitsocial}                                                                                                                                                                                                                                                                                                                                                                                              \\
Simulated Society        & Performance metrics                 & Classification accuracy                                                                                                                                                                                     & Automatic & \cite{li2024hello}                                                                                                                                                                                                                                                                                                                                                                                                        \\
Simulated Society        & Performance metrics                 & Success rate                                                                                                                                                                                                & Automatic & \cite{kaiya2023lyfeagentsgenerativeagents}                                                                                                                                                                                                                                                                                                                                                                                              \\
Simulated Society        & Performance metrics                 & Success rate                                                                                                                                                                                                & Automatic & \cite{li2023metaagentssimulatinginteractionshuman}                                                                                                                                                                                                                                                                                                                                                                                          \\
Simulated Society        & Performance metrics                 & Success rate                                                                                                                                                                                                & Automatic & \cite{li2023metaagentssimulatinginteractionshuman}                                                                                                                                                                                                                                                                                                                                                                                          \\
Simulated Society        & Performance metrics                 & Success rate for coordination (identification accuracy, workflow correctness, alignment between job and agent's skill)                                                                                      & Automatic & \cite{Li2023MetaAgentsSI}                                                                                                                                                                                                                                                                                                                                                                                                  \\
Simulated Society        & Performance metrics                 & Success rate for coordination (identification accuracy, workflow correctness, alignment between job and agent's skill)                                                                                      & Automatic & \cite{Li2023MetaAgentsSI}                                                                                                                                                                                                                                                                                                                                                                                                  \\
Simulated Society        & Performance metrics                 & Task Accuracy                                                                                                                                                                                               & Automatic & \cite{Zhang2023ExploringCM}                                                                                                                                                                                                                                                                                                                                                                               \\
Simulated Society        & Performance metrics                 & Task Accuracy                                                                                                                                                                                               & Automatic & \cite{lan2024depressiondiagnosisdialoguesimulation}                                                                                                                                                                                                                                                                                                                                                                                          \\
Simulated Society        & Performance metrics                 & Errors in the prompting sequence                                                                                                                                                                            & Human     & \cite{10.1145/3570945.3607303}                                                                                                                                                                                                                                                                                                                                                                              \\
Simulated Society        & Performance metrics                 & Error-free execution                                                                                                                                                                                        & Automatic & \cite{wang2024megaagentpracticalframeworkautonomous}                                                                                                                                                                                                                                                                                                                                                                                              \\
Simulated Society        & Performance metrics                 & Goal completion                                                                                                                                                                                             & Human     & \cite{mou2024agentsensebenchmarkingsocialintelligence}                                                                                                                                                                                                                                                                                                                                                                                               \\
Simulated Society        & Performance metrics                 & Goal completion                                                                                                                                                                                             & LLM       & \cite{Zhou2024IsTT}                                                                                                                                                                                                                                                                                                                                                                            \\
Simulated Society        & Performance metrics                 & Goal completion                                                                                                                                                                                             & LLM       & \cite{mou2024agentsensebenchmarkingsocialintelligence}                                                                                                                                                                                                                                                                                                                                                                                               \\
Simulated Society        & Performance metrics                 & Goal completion                                                                                                                                                                                             & LLM       & \cite{zhou2024sotopiainteractiveevaluationsocial}                                                                                                                                                                                                                                                                                                                                                                                              \\
Simulated Society        & Performance metrics                 & Efficacy                                                                                                                                                                                                    & Human     & \cite{ashkinaze2024pluralsguidingllmssimulated}                                                                                                                                                                                                                                                                                                                                                                                              \\
Simulated Society        & Performance metrics                 & Knowledge                                                                                                                                                                                                   & Human     & \cite{zhou2024sotopiainteractiveevaluationsocial}                                                                                                                                                                                                                                                                                                                                                                                              \\
Simulated Society        & Performance metrics                 & Knowledge                                                                                                                                                                                                   & LLM       & \cite{zhou2024sotopiainteractiveevaluationsocial}                                                                                                                                                                                                                                                                                                                                                                                              \\
Simulated Society        & Performance metrics                 & Reasoning abilities                                                                                                                                                                                         & Automatic & \cite{chen2023agentversefacilitatingmultiagentcollaboration}                                                                                                                                                                                                                                                                                                                                                                                                 \\
Simulated Society        & Performance metrics                 & Reasoning abilities                                                                                                                                                                                         & Human     & \cite{chen2023agentversefacilitatingmultiagentcollaboration}                                                                                                                                                                                                                                                                                                                                                                                                 \\
Simulated Society        & Performance metrics                 & Efficiency                                                                                                                                                                                                  & Automatic & \cite{piatti2024cooperatecollapseemergencesustainable}                                                                                                                                                                                                                                                                                                                                                                                                  \\
Simulated Society        & Performance metrics                 & Text understanding and creative writing abilities (Dialogue response dataset, Commongen Challenge)                                                                                                          & LLM       & \cite{chen2023agentversefacilitatingmultiagentcollaboration}                                                                                                                                                                                                                                                                                                                                                                                                 \\
Simulated Society        & Performance metrics                 & Probabilities of receiving, storing, and retrieving the key information across the population                                                                                                               & Automatic & \cite{kaiya2023lyfeagentsgenerativeagents}                                                                                                                                                                                                                                                                                                                                                                                              \\
Simulated Society        & Performance metrics                 & Correlation between predicted and real results                                                                                                                                                              & Automatic & \cite{Mitsopoulos2024PsychologicallyValidGA}                                                                                                                                                                                                                                                                                                                                                                            \\
Simulated Society        & Internal consistency metrics        & Behavioral similarity                                                                                                                                                                                       & Automatic & \cite{Li2024EvolvingAI}                                                                                                                                                                                                                                                                                                                                                                                                    \\
Simulated Society        & Internal consistency metrics        & Semantic consistency (cosine similarity)                                                                                                                                                                    & Automatic & \cite{qiu2024interactiveagentssimulatingcounselorclient}                                                                                                                                                                                                                                                                                                                                                                                               \\
Simulated Society        & External alignment metrics          & Alignment (Environmental understanding and response accuracy, adherence to predefined settings)                                                                                                             & Automatic & \cite{gu2024agentgroupchatinteractivegroupchat}                                                                                                                                                                                                                                                                                                                                                                                        \\
Simulated Society        & External alignment metrics          & Strategy accuracy (strategies provided by the models vs. by human experts and evaluate the accuracy)                                                                                                        & Automatic & \cite{zhang2024selfemotionblendeddialoguegeneration}                                                                                                                                                                                                                                                                                                                                                                                       \\
Simulated Society        & External alignment metrics          & Believability of behavior                                                                                                                                                                                   & Human     & \cite{zhou2024sotopiainteractiveevaluationsocial}                                                                                                                                                                                                                                                                                                                                                                                              \\
Simulated Society        & External alignment metrics          & Believability of behavior                                                                                                                                                                                   & Human     & \cite{park2023generative}                                                                                                                                                                                                                                                                                                                                                                                \\
Simulated Society        & Content and textual metrics         & Content similarity (ROUGE-L, BERTScore, GPT-based-similarity, G-eval, BLEU-4)                                                                                                                               & Automatic & \cite{li2024hello}                                                                                                                                                                                                                                                                                                                                                                                                        \\
Simulated Society        & Content and textual metrics         & Content similarity (ROUGE-L, BERTScore, GPT-based-similarity, G-eval)                                                                                                                                       & Automatic & \cite{chen2024towards}                                                                                                                                                                                                                                                                                                                                                                                       \\
Simulated Society        & Content and textual metrics         & Content similarity (ROUGE-L, BERTScore, GPT-based-similarity, G-eval)                                                                                                                                       & Automatic & \cite{mishra-etal-2023-e}                                                                                                                                                                                                                                                                                                                                                                                             \\
Simulated Society        & Content and textual metrics         & Semantic understanding                                                                                                                                                                                      & Automatic & \cite{gu2024agentgroupchatinteractivegroupchat}                                                                                                                                                                                                                                                                                                                                                                                        \\
Simulated Society        & Content and textual metrics         & Complexity of generated content                                                                                                                                                                             & Automatic & \cite{10.1145/3570945.3607303}                                                                                                                                                                                                                                                                                                                                                                              \\
Simulated Society        & Content and textual metrics         & Dialogue generation quality                                                                                                                                                                                 & Automatic & \cite{10.1145/3570945.3607303}                                                                                                                                                                                                                                                                                                                                                                              \\
Simulated Society        & Content and textual metrics         & Number of conversation rounds                                                                                                                                                                               & Automatic & \cite{10.1145/3613904.3642545}                                                                                                                                                                                                                                                                                                                                                                          \\
Simulated Society        & Bias, fairness, and ethic metrics   & Bias rate (herd effect, authority effect, ban franklin effect, rumor chain effect, gambler's fallacy, confirmation bias, halo effect)                                                                       & Human     & \cite{liu2025exploringprosocialirrationalityllm}                                                                                                                                                                                                                                                                                                                                                                                            \\
Simulated Society        & Bias, fairness, and ethic metrics   & Bias rate (herd effect, authority effect, ban franklin effect, rumor chain effect, gambler's fallacy, confirmation bias, halo effect)                                                                       & LLM       & \cite{liu2025exploringprosocialirrationalityllm}                                                                                                                                                                                                                                                                                                                                                                                            \\
Simulated Society        & Bias, fairness, and ethic metrics   & Bias rate (herd effect, authority effect, ban franklin effect, rumor chain effect, gambler's fallacy, confirmation bias, halo effect)                                                                       & Automatic & \cite{liu2025exploringprosocialirrationalityllm}                                                                                                                                                                                                                                                                                                                                                                                            \\
Simulated Society        & Bias, fairness, and ethic metrics   & Equality                                                                                                                                                                                                    & Automatic & \cite{piatti2024cooperatecollapseemergencesustainable}                                                                                                                                                                                                                                                                                                                                                                                                  \\
Writing                  & Psychological metrics & Qualitative feedback (expertise, social relation, valence, level of involvement)                                                                                                                            & Human     & \cite{10.1145/3613904.3642406}                                                                                                                                                                                                                                                                                                                                                                                \\
Writing                  & Performance metrics                 & Prediction accuracy (F1 score, AUC, MSE, MAE, depression risk prediction accuracy, suicide risk prediction accuracy)                                                                                        & Automatic & \cite{wang-etal-2024-unleashing}                                                                         \\
Writing                  & Performance metrics                 & Success rate                                                                                                                                                                                                & Automatic & \cite{10.1145/3649921.3656987}                                                                                                                                                                                                                                                                                                                                                                                       \\
Writing                  & Performance metrics                 & Behavioral patterns                                                                                                                                                                                         & Human     & \cite{10.1145/3613904.3642545}                                                                                                                                                                                                                                                                                                                                                                          \\
Writing                  & Internal consistency metrics        & Consistency (user profile, psychotherapeutic approach)                                                                                                                                                      & Automatic & \cite{mishra-etal-2023-e}                                                                                                                                                                                                                                                                                                                                                                                        \\
Writing                  & Internal consistency metrics        & Motivational consistency                                                                                                                                                                                    & LLM       & \cite{10.1145/3649921.3656987}                                                                                                                                                                                                                                                                                                                                                                                       \\
Writing                  & Internal consistency metrics        & Audience similarity                                                                                                                                                                                         & Human     & \cite{choi2024proxona}                                                                                                                                                                                                                                                                                                                                                                                                          \\
Writing                  & Internal consistency metrics        & Quality of generated dimension \& values (relevance, mutual exclusiveness)                                                                                                                                  & Human     & \cite{choi2024proxona}                                                                                                                                                                                                                                                                                                                                                                                                          \\
Writing                  & External alignment metrics          & Factual error rate                                                                                                                                                                                          & Automatic & \cite{wang-etal-2024-unleashing}                                                                         \\
Writing                  & External alignment metrics          & Correctness (politeness, interpersonal behaviour)                                                                                                                                                           & Automatic & \cite{mishra-etal-2023-e}                                                                                                                                                                                                                                                                                                                                                                                      \\
Writing                  & External alignment metrics          & Hallucination (groundedness of the chat responses)                                                                                                                                                          & Human     & \cite{choi2024proxona}                                                                                                                                                                                                                                                                                                                                                                                                          \\
Writing                  & Content and textual metrics         & Linguistic similarity                                                                                                                                                                                       & Human     & \cite{choi2024proxona}                                                                                                                                                                                                                                                                                                                                                                                                       \\
Writing                  & Content and textual metrics         & Fluency                                                                                                                                                                                                     & Human     & \cite{mishra-etal-2023-e}                                                                                                                                                                                                                                                                                                                                                                                            \\
Writing                  & Content and textual metrics         & Perplexity                                                                                                                                                                                                  & Automatic & \cite{mishra-etal-2023-e}                                                                                                                                                                                                                                                                                                                                                                                           \\
Writing                  & Content and textual metrics         & Non-Repetitiveness                                                                                                                                                                                          & Human     & \cite{mishra-etal-2023-e}                                                                                                                                                                                                                                                                                                                                                                                        \\
Writing                  & Content and textual metrics         & response generation quality                                                                                                                                                                                 & Automatic & \cite{li2024hello}                                                                                                                                                                                                                                                                                                                                                                                                         \\
Writing                  & Content and textual metrics         & Coherency                                                                                                                                                                                                   & LLM       &  \cite{10.1145/3649921.3656987}    



\end{longtable}
\end{center}
\end{small}

\clearpage
\twocolumn
% \label{tab: long_table_task_metrics_src}

