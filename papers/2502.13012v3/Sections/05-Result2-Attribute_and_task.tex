
\begin{figure}[t]
    \centering
    \includegraphics[width=\linewidth]{Figures/heatmap.png}
    \caption{Relationships between agent attributes and downstream tasks. The numbers in the heatmap represent the paper counts.}
    \label{fig:heatmap}
    % \vspace{-1em}
\end{figure}

\section{Relationships Between Agent Attributes and Downstream Tasks}




Both agent attributes and downstream task attributes play a crucial role in selecting appropriate RPA evaluation metrics. Researchers predefine these factors when designing and evaluating RPAs, yet their interrelation remains an open question.  In this section, we analyze how agent attributes correspond to different downstream tasks, uncovering several recurring patterns (Fig.~\ref{fig:heatmap}).

Demographic information and psychological traits are fundamental across all downstream tasks. Whether in decision-making, opinion dynamics, or simulated environments, these attributes consistently shape RPA design. As shown in Fig.~\ref{fig:heatmap}, they are the most frequently incorporated factors, underscoring their central role in modeling agent behavior across diverse applications.

For tasks where simulation itself is the primary objective, such as Simulated Individuals and Simulated Society, the selection of agent attributes becomes broader. In addition to demographic and psychological factors, these tasks frequently incorporate skills, expertise, and social relationships, reflecting the need for richer agent representations to capture complex social and individual interactions. By contrast, tasks that use simulation as a means to study specific research fields tend to prioritize certain agent attributes. For instance, in Opinion Dynamics, beliefs and values play a distinctive role, as they directly influence how agents interact and form opinions. Similarly, tasks related to Educational Training and Writing exhibit a different pattern, emphasizing skills and expertise over broad demographic or psychological considerations.

In contrast, attributes such as activity history and social relationships receive significantly less emphasis across tasks. This raises a question: is their impact inherently limited, or are they simply underexplored in current RPA applications?

Overall, these findings highlight the nuanced interplay between agent attributes and downstream tasks. While demographic information and psychological traits are universally relevant, attributes like beliefs and values gain importance in specific contexts. At the same time, the relative absence of activity history and social relationships in current evaluations presents an open research question, particularly in scenarios requiring long-term modeling and complex social interactions.
