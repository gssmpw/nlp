\section{Experiments}
In this section, we conduct comprehensive experiments to compare SAISA with existing MLLMs.
Furthermore, we perform a series of ablation experiments to further validate the effectiveness of the SAISA architecture.

\subsection{Setups}
\paragraph{Model Configuration.}
To make an apple-to-apple comparison between SAISA and LLaVA-1.5~\cite{liu2024improvedbaselinesvisualinstruction}, we use the same settings as LLaVA-1.5.
Specifically, we employ Vicuna-7B-v1.5~\cite{vicuna} as the default LLM and CLIP-ViT-L/14-336~\cite{radford2021learningtransferablevisualmodels} as the default visual encoder.

\vspace{-0.35cm}
\paragraph{Training Details.}
We utilize the same training data as LLaVA-1.5.
During pre-training, we adopt the pre-train dataset with 558k samples from LLaVA~\cite{liu2023visualinstructiontuning}. This stage takes around 1.5 hours on 8 A800 (80G) GPUs. 
During fine-tuning, we use the mixture instruction-following dataset from LLaVA-1.5. The dataset contains 665k samples from LLaVA-Instruct~\cite{liu2023visualinstructiontuning}, ShareGPT~\cite{sharegpt2023}, VQAv2, GQA~\cite{hudson2019gqanewdatasetrealworld}, OK-VQA~\cite{marino2019okvqavisualquestionanswering}, OCR-VQA~\cite{mishra2019ocrvqa}, A-OKVQA~\cite{schwenk2022aokvqabenchmarkvisualquestion}, TextCaps~\cite{sidorov2020textcapsdatasetimagecaptioning}, RefCOCO~\cite{kazemzadeh2014referitgame, mao2016generationcomprehensionunambiguousobject}, and Visual Genome~\cite{krishna2016visualgenomeconnectinglanguage}.
This stage takes around 8.5 hours on 8 A800 (80G) GPUs.
The total training budget  of SAISA is approximately 80 GPU hours.

\begin{table*}
  [t]
  \centering
  \resizebox{\textwidth}{!}{%
  \begin{tabular}{cccccccccccc}
    \toprule \multicolumn{2}{c}{Components}                                                             & \multicolumn{5}{c}{Re-executability Rate (\%)} & \multicolumn{5}{c}{Readability (\#)} \\
    \cmidrule(lr){1-2} \cmidrule(lr){3-7} \cmidrule(lr){8-12}        \hspace{8pt}\labelemoji\hspace{8pt}                                                                & \hspace{8pt}\toolemoji\hspace{8pt}                                      & O0                                 & O1             & O2             & O3             & AVG            & O0             & O1             & O2             & O3             & AVG            \\
    \hline
    \rowcolor[rgb]{0.93,0.93,0.93}\multicolumn{12}{c}{\textbf{Initialize with LLM4Decompile-End-6.7B~\citep{llm4decompile}}}   \\
    \xmark                                                                                              & \xmark                                    & 69.51                              & 46.95          & 50.61          & 46.34          & 53.35          & 3.98 & 3.41 & 3.44 & 3.38 & 3.55 \\
    \cmark                                                                                              & \xmark                                    & 75.61                              & 50.61          & 50.00          & 50.00          & 56.55          & 4.01 & 3.44 & 3.39 & \textbf{3.49} & 3.58 \\
    \xmark                                                                                              & \cmark                                    & 83.54                     & \textbf{56.10}          & 51.22          & 50.61 & 60.37 & 4.05 & 3.51 & 3.51 & 3.42 & 3.62 \\
    \cmark                                                                                              & \cmark                                    & \textbf{85.37}                            & \textbf{56.10}                     & \textbf{51.83} & \textbf{52.43}          & \textbf{61.43} & \textbf{4.13} & \textbf{3.60} & \textbf{3.54} & \textbf{3.49} & \textbf{3.69} \\

    \rowcolor[rgb]{0.93,0.93,0.93}\multicolumn{12}{c}{\textbf{Initialize with Deepseek-Coder-6.7B-base~\citep{deepseekcoder}}} \\
    \xmark                                                                                              & \xmark                                    & 59.15                              & 35.98          & 39.02          & 37.80          & 42.99          & 3.71 & 3.05 & 3.16 & 3.05 & 3.24 \\
    \cmark                                                                                              & \xmark                                    & 66.46                              & 41.46          & 38.41          & 36.59          & 45.73          & 3.76 & 3.17 & \textbf{3.21} & 3.08 & 3.31 \\
    \xmark                                                                                              & \cmark                                    & 70.73                              & 39.63          & 39.02          & 40.24          & 47.41          & 3.90 & 3.17 & 3.08 & 3.11 & 3.31 \\
    \cmark                                                                                              & \cmark                                    & \textbf{79.88}                     & \textbf{45.73} & \textbf{43.90} & \textbf{42.68} & \textbf{53.05} & \textbf{3.96} & \textbf{3.21} & 3.18 & \textbf{3.19} & \textbf{3.38} \\
    \bottomrule
  \end{tabular}%
  }
  \caption{The ablation study of different methods across four optimization levels
  (O0, O1, O2, O3), as well as their average scores (AVG). The results in bold represent the optimal performance. The ~\labelemoji~ and ~\toolemoji~ means Relabedling and Function Call. \textbf{Bold} denotes the best performance.}
  \label{tab:ablation}
\end{table*}
% Table generated by Excel2LaTeX from sheet 'Sheet1'
\begin{table}[t]
  \centering
    % \resizebox{\linewidth}{!}
    \scalebox{0.9}
    {
    \begin{tabular}{l|c|ccc}
    \toprule
    \multirow{2}[2]{*}{Method} & Inference &  MMVP  & \multicolumn{2}{c}{CV-Bench~\cite{tong2024cambrian1fullyopenvisioncentric}}     \\
         & TFLOPs$\downarrow$      &   \cite{tong2024eyeswideshutexploring}          & 2D & 3D  \\
    \midrule
    LLaVA-1.5 & 8.53 & 24.7    & \textbf{56.6}  & 59.5   \\
    \rowcolor{cyan!20} SAISA (Ours) & 2.86 & \textbf{26.0}  & 56.2  & \textbf{59.8}   \\
    \bottomrule
    \end{tabular}
    }
    \caption{\textbf{Performance on vision-centric MLLM benchmarks,} based on Vicuna and CLIP.}
  \label{tab:vision-centric}
\end{table}

\section{Network Join Time}
\label{sec:latency}
One of the desirable qualities of wireless sensor networks is self-organisation: Nodes can independently establish a multi-hop network; new nodes can join the network; and nodes, which leave the network for various reasons, can rejoin. The MAC protocol plays a key role during self-organisation. As stated above, in TSCH, the coordinator node broadcasts EB packets regularly. A node wishing to join the network listens to these packets, and upon receiving one, contends for the medium and sends a request-to-join packet using a unicast channel. Because initially no time synchronisation does take place between the coordinator and the nodes joining the network, almost certainly there is a time difference between the coordinator and these nodes. Expecting this condition, TSCH defines two types of time offsets. The first offset is intended to prevent a receiver from early sleeping in case a packet (preamble, SFD, Headers, payload, FCS) does not arrive according to the receiver's local time. The second time offset is intended for a transmitter to receive a delayed acknowledgement packet.

    \begin{figure}[h!]
        \centering
        \includegraphics[width=0.45\textwidth]{latency_no_interference.pdf}
        \caption{The histogram of network join latency during a moderate CTI.}
        \label{fig: latency_with_interference}
    \end{figure}  
            
    
    \begin{figure}[h!]
        \centering
        \includegraphics[width=0.45\textwidth]{latency_with_interference.pdf}
        \caption{The histogram of network join latency during an high CTI.}
        \label{fig: latency_no_interference}
    \end{figure}  

When the network is under the influence of a CTI, as we already discussed in Section~\ref{sec:ts}, the time drifts between the coordinator and the new nodes increase and the two time offsets are not sufficient to establish a reliable communication. This creates a join delay. We measured the join delay with and without a CTI. The distributions of these delay are given in Figs. \ref{fig: latency_with_interference} and \ref{fig: latency_no_interference}.   Without CTI, $83.3\%$  of the case, the join delay  is between $40$ and $70$ ms, whereas in the presence of CTI, $96.82\%$ of the time, the join delay is between $100$ and $200$ ms. In other words, the join delay in the presence of CTI is about five times higher than without CTI. Additionally, the maximum join delay we observed during CTI was $800$ ms, whereas it was $200$ ms when there was no CTI.     

%-------------------------------------------------------------------------

\subsection{Main Results}
The performance on benchmarks is shown in Table~\ref{tab:mllms}, Table~\ref{tab:vqa} and Table~\ref{tab:vision-centric}, and the result of the inference latency test is shown in Table~\ref{tab:latency}.

We evaluate SAISA on a range of benchmarks, including: (1) comprehensive benchmarks for instruction-following MLLMs such as MMMU~\cite{yue2024mmmumassivemultidisciplinemultimodal}, MME~\cite{fu2024mmecomprehensiveevaluationbenchmark}, MMBench~\cite{liu2024mmbenchmultimodalmodelallaround}, MMBench-CN~\cite{liu2024mmbenchmultimodalmodelallaround}, and SEED-bench~\cite{li2023seedbenchbenchmarkingmultimodalllms}; (2) hallucination benchmark such as POPE~\cite{li2023evaluatingobjecthallucinationlarge}, which evaluates MLLMs' degree of hallucination on three subsets: random, popular, and adversarial; (3) general visual question answering benchmarks such as GQA~\cite{hudson2019gqanewdatasetrealworld} and ScienceQA IMG~\cite{lu2022learnexplainmultimodalreasoning}; (4) fine-grained visual question answering benchmarks such as OK-VQA~\cite{marino2019okvqavisualquestionanswering} and TextVQA~\cite{singh2019vqamodelsread}, OK-VQA requires fine-grained image understanding and spatial understanding, and TextVQA is an OCR-related benchmark; (5) vision-centric MLLM benchmarks such as MMVP~\cite{tong2024eyeswideshutexploring} and CV-Bench~\cite{tong2024cambrian1fullyopenvisioncentric}.
In the inference latency test, the latency is reported as the time of LLM prefilling during inference with varying numbers of text tokens.
Table~\ref{tab:mllms} shows the comparison on the comprehensive benchmarks for instruct-following MLLMs.
SAISA outperforms BLIP-2~\cite{li2023blip2bootstrappinglanguageimagepretraining}, InstructBLIP~\cite{dai2023instructblipgeneralpurposevisionlanguagemodels}, MiniGPT-4~\cite{zhu2023minigpt4enhancingvisionlanguageunderstanding}, MiniGPT-v2\cite{chen2023minigptv2largelanguagemodel}, Otter~\cite{li2023ottermultimodalmodelincontext}, Shikra~\cite{chen2023shikraunleashingmultimodalllms}, and IDEFICS~\cite{idefics} utilizing LLama-7B and LLama-65B~\cite{touvron2023llamaopenefficientfoundation} across all these benchmarks.
Compared to Qwen-VL-Chat~\cite{Qwen-VL} trained on data with 1.4B samples, SAISA performs better on 4 out of 5 benchmarks.
Compared to LLaVA-1.5~\cite{liu2024improvedbaselinesvisualinstruction}, SAISA performs better on 3 out of 5 benchmarks.
Table~\ref{tab:vqa} shows the comparison on the hallucination and visual question answering benchmarks, and Table~\ref{tab:vision-centric} shows the comparison on vision-centric MLLM benchmarks.
Since most previous models do not evaluate performance on vision-centric MLLM benchmarks, we compare SAISA with LLaVA-1.5.
SAISA achieves the best overall performance compared to other baseline MLLMs, and strikes the optimal balance between effectiveness and efficiency.

\subsection{Ablation Study}
\paragraph{Ablation on LLMs and Visual Encoders.}
As presented in Table~\ref{tab:ablation}, we perform multiple ablation experiments on both LLMs and visual encoders to evaluate the robustness of SAISA.
We tune a set of SAISA models using a variety of LLM backbones and visual encoders.
The ablated LLMs include Vicuna-7B~\cite{vicuna} and two LLMs using grouped query attention (GQA)~\cite{ainslie2023gqatraininggeneralizedmultiquery}, such as Mistral-7B~\cite{jiang2023mistral7b} and Llama3-8B~\cite{llama3v}.
The ablated visual encoders include two ViT-based~\cite{dosovitskiy2021imageworth16x16words} visual backbones such as CLIP-ViT-L/14-336~\cite{radford2021learningtransferablevisualmodels} and  SigLIP-ViT-SO400M/14-384~\cite{zhai2023sigmoidlosslanguageimage}, and a ConvNeXt-based~\cite{liu2022convnet2020s} visual encoder such as ConvNeXt-XXL-1024 from OpenCLIP~\cite{ilharco_gabriel_2021_5143773, schuhmann2022laion5bopenlargescaledataset}.
The experimental results demonstrate that SAISA consistently achieves superior performance to LLaVA-1.5 across different LLM backbones and visual encoders, while dramatically reducing computational costs.

\vspace{-0.35cm}
\paragraph{Ablation on Pre-training Strategies.}
As shown in Table~\ref{tab:ablation_train}, we conduct an ablation study to investigate the effects of SAISA's pre-training strategies.
We tune a SAISA model where the full projector (32 MLPs) is tunable during pre-training, and the other settings keep the same as the original SAISA.
With more randomly initialized parameters, we observe a performance drop when pre-training the full projector.
We attribute this drop to the small amount of pre-training data with only 558k samples.
The ablation study demonstrates the effectiveness of our pre-training strategy, which provides a robust initialization for the subsequent fine-tuning stage.

\vspace{-0.35cm}
\paragraph{Ablation on Projector Designs.}
Previous works find that replacing linear projection with MLP projection improves performance in MLLM~\cite{liu2024improvedbaselinesvisualinstruction} and self-supervised learning~\cite{chen2020simpleframeworkcontrastivelearning, chen2020improvedbaselinesmomentumcontrastive}.
We conduct an experiment to investigate the impact of projector designs in SAISA.
We tune a model under the same configuration as the original SAISA model but replace the MLPs in the projector with linear layers.
Table ~\ref{tab:linear} shows that the model with MLPs in the projector performs better than the model with linear layers, which is consistent with the finding of the previous study~\cite{liu2024improvedbaselinesvisualinstruction}.
Notably, we note that even the SAISA model with linear layers achieves comparable performance to LLaVA-1.5 with MLP projection.
This observation provides additional evidence for the effectiveness of SAISA.

% Table generated by Excel2LaTeX from sheet 'Sheet1'
\begin{table}[t]
  \centering
    \resizebox{\linewidth}{!}
    % \scalebox{0.8}
    {
    \begin{tabular}{l|cccccc}
    \toprule
    Pre-trained &  MMMU & \multicolumn{2}{c}{MMBench} & \multirow{2}[2]{*}{POPE}  & SQA & OK-  \\
      Parameters  & VAL      &  EN & CN &           & IMG & VQA \\
    \midrule
    Full Projector & 34.8 & 59.2 & 51.1   & 85.6  & 67.8 & 53.1 \\
    \rowcolor{cyan!20} Shared MLP & 36.9 & 65.7 & 59.0 & 87.2  & 70.1 & 56.8  \\
    \bottomrule
    \end{tabular}
    }
    \caption{\textbf{Ablation on Pre-training Strategies.} 
    ``Full Projector" denotes we pre-train 32 MLPs.
    ``Shared MLP" denotes our strategy, which involves tuning a shared MLP for efficiency.
    Our strategy provides an effective initialization for visul fine-tuning when using the small pre-training dataset with 558k samples.
    }
  \label{tab:ablation_train}
\end{table}

\begin{table*}[t]
    \centering
    \small
    % \footnotesize	    
    \def\arraystretch{1.25}
    \setlength{\tabcolsep}{3pt}
    \begin{tabular}{lcr cc cccc}
        \toprule
         &  &  & \multicolumn{2}{c}{\textit{$L_2$ Loss} ($\downarrow$)} & \multicolumn{2}{c}{\textit{Accuracy} ($\uparrow$)} \\
         \cmidrule(lr){4-5}\cmidrule(lr){6-9}
         \textbf{Objective} & $q_\varphi$ & \textbf{Model} & \multicolumn{2}{c}{\textbf{Linear Regression}} & \multicolumn{4}{c}{\textbf{Linear Classification}} \\
         \cmidrule(lr){4-5}\cmidrule(lr){6-9}
         & & & \textit{2D} & \textit{100D} & \textit{2D-2cl} & \textit{2D-5cl} & \textit{100D-2cl} & \textit{100D-5cl} \\
         \midrule

\multirow{4}{*}{Baseline} & - & Random & $4.178$\sstd{$0.018$} & $202.601$\sstd{$0.321$} & $50.498$\sstd{$0.357$} & $19.891$\sstd{$0.028$} & $50.046$\sstd{$0.047$} & $20.054$\sstd{$0.053$} \\
& - & Optimization & $0.257$\sstd{$0.000$} & $25.083$\sstd{$0.006$} & $96.982$\sstd{$0.000$} & $93.449$\sstd{$0.002$} & $70.258$\sstd{$0.012$} & $41.338$\sstd{$0.012$} \\
& - & Langevin & $0.263$\sstd{$0.002$} & $23.340$\sstd{$0.689$} & $95.034$\sstd{$0.412$} & $88.277$\sstd{$0.290$} & $65.123$\sstd{$0.370$} & $32.544$\sstd{$0.422$} \\
& - & HMC & $0.263$\sstd{$0.001$} & $18.659$\sstd{$0.189$} & $92.659$\sstd{$0.344$} & $82.169$\sstd{$0.518$} & $62.145$\sstd{$0.245$} & $29.582$\sstd{$0.371$} \\
\cmidrule{2-9}

\multirow{3}{*}{Fwd-KL} & \multirow{6}{*}{\rotatebox[origin=c]{90}{Gaussian}} & GRU & \highlight{$0.264$\sstd{$0.001$}} & $124.823$\sstd{$0.135$} & $81.170$\sstd{$0.389$} & $71.170$\sstd{$0.275$} & $59.740$\sstd{$0.102$} & $23.042$\sstd{$0.246$} \\
& & DeepSets &$0.264$\sstd{$0.000$} & $123.133$\sstd{$1.080$} & $81.281$\sstd{$0.278$} & $70.993$\sstd{$0.191$} & $50.047$\sstd{$0.051$} & $20.053$\sstd{$0.045$} \\
& & Transformer &$0.264$\sstd{$0.000$} & $45.856$\sstd{$1.331$} & $80.960$\sstd{$0.285$} & $71.484$\sstd{$0.437$} & $62.954$\sstd{$0.062$} & $26.789$\sstd{$0.110$} \\
\cmidrule{3-9}

\multirow{3}{*}{Rev-KL} & & GRU & \highlight{$0.263$\sstd{$0.000$}} & $60.215$\sstd{$0.866$} & $94.258$\sstd{$0.034$} & $87.339$\sstd{$0.023$} & $63.465$\sstd{$0.307$} & $28.270$\sstd{$0.462$} \\
& & DeepSets & \highlight{$0.263$\sstd{$0.000$}} & $62.837$\sstd{$0.617$} & $94.285$\sstd{$0.116$} & $87.342$\sstd{$0.021$} & $60.867$\sstd{$0.265$} & $21.339$\sstd{$0.085$} \\
& & Transformer & \highlight{$0.264$\sstd{$0.001$}} & \highlight{$28.735$\sstd{$0.252$}} & $94.302$\sstd{$0.054$} & $87.540$\sstd{$0.117$} & $68.185$\sstd{$0.007$} & \highlight{$32.950$\sstd{$0.284$}} \\
\cmidrule{2-9}

\multirow{3}{*}{Fwd-KL} & \multirow{6}{*}{\rotatebox[origin=c]{90}{Flow}} & GRU & \highlight{$0.264$\sstd{$0.001$}} & $119.119$\sstd{$0.233$} & $96.305$\sstd{$0.008$} & $88.927$\sstd{$0.200$} & $59.920$\sstd{$0.221$} & $23.025$\sstd{$0.077$} \\
& & DeepSets & \highlight{$0.264$\sstd{$0.001$}} & $125.677$\sstd{$3.731$} & $96.191$\sstd{$0.021$} & $88.643$\sstd{$0.102$} & $50.061$\sstd{$0.021$} & $20.021$\sstd{$0.094$} \\
& & Transformer &$0.264$\sstd{$0.000$} & $43.272$\sstd{$2.700$} & \highlight{$96.344$\sstd{$0.059$}} & \highlight{$89.624$\sstd{$0.215$}} & $64.349$\sstd{$0.147$} & $26.952$\sstd{$0.203$} \\
\cmidrule{3-9}

\multirow{3}{*}{Rev-KL} & & GRU & \highlight{$0.263$\sstd{$0.000$}} & $61.295$\sstd{$1.008$} & $95.241$\sstd{$0.012$} & $88.429$\sstd{$0.024$} & $64.669$\sstd{$0.207$} & $28.409$\sstd{$1.167$} \\
& & DeepSets & \highlight{$0.263$\sstd{$0.001$}} & $76.412$\sstd{$2.038$} & $95.296$\sstd{$0.021$} & $88.464$\sstd{$0.061$} & $58.384$\sstd{$0.812$} & $21.569$\sstd{$0.117$} \\
& & Transformer & \highlight{$0.263$\sstd{$0.000$}} & \highlight{$29.358$\sstd{$1.569$}} & $95.339$\sstd{$0.063$} & $88.644$\sstd{$0.047$} & \highlight{$68.721$\sstd{$0.121$}} & \highlight{$33.107$\sstd{$0.333$}} \\
\bottomrule
    \end{tabular}
    \caption{\textbf{Fixed-Dimensional}. Results for estimating the parameters of linear regression (LR) and classification (LC) models with the expected $L_2$ loss and accuracy according to the posterior predictive as metrics.}
    \vspace{-4mm}
    \label{tab:}
\end{table*}

% \begin{table*}[t]
%     \centering
%     \small
%     % \footnotesize	    
%     \def\arraystretch{1.25}
%     \setlength{\tabcolsep}{5pt}
%     \begin{tabular}{lcr cc cccc}
%         \toprule
%          &  &  & \multicolumn{6 }{c}{\textit{Conditional Negative Log Likelihood} ($\downarrow$)} \\
%          \cmidrule(lr){4-9}
%          \textbf{Objective} & $q_\varphi$ & \textbf{Model} & \textit{2D} & \textit{100D} & \textit{2D-2cl} & \textit{2D-5cl} & \textit{100D-2cl} & \textit{100D-5cl} \\
%          \midrule

% \multirow{4}{*}{Baseline} & - & Random & $792.2$\sstd{$3.6$} & $37777.0$\sstd{$65.5$} & $109.7$\sstd{$0.7$} & $234.4$\sstd{$1.5$} & $532.4$\sstd{$0.5$} & $1097.9$\sstd{$2.8$} \\
% & - & Optimization & $69.2$\sstd{$0.0$} & $3980.9$\sstd{$0.9$} & $9.2$\sstd{$0.0$} & $26.2$\sstd{$0.0$} & $261.4$\sstd{$0.1$} & $442.3$\sstd{$0.0$} \\
% & - & Langevin & $70.3$\sstd{$0.3$} & $3671.3$\sstd{$110.7$} & $14.1$\sstd{$0.2$} & $36.2$\sstd{$0.2$} & $228.3$\sstd{$6.3$} & $585.8$\sstd{$5.3$} \\
% & - & HMC & $70.2$\sstd{$0.2$} & $3093.6$\sstd{$27.5$} & $27.7$\sstd{$0.5$} & $71.9$\sstd{$0.7$} & $105.3$\sstd{$0.2$} & $255.9$\sstd{$2.9$} \\
% \cmidrule{2-9}

% \multirow{3}{*}{Fwd-KL} & \multirow{6}{*}{\rotatebox[origin=c]{90}{Gaussian}} & GRU &$70.4$\sstd{$0.1$} & $22964.7$\sstd{$43.5$} & $38.5$\sstd{$0.5$} & $72.9$\sstd{$0.5$} & $397.1$\sstd{$1.0$} & $1013.3$\sstd{$5.4$} \\
%  & & DeepSets &$70.4$\sstd{$0.1$} & $22802.5$\sstd{$191.9$} & $38.4$\sstd{$0.5$} & $73.3$\sstd{$0.4$} & $532.4$\sstd{$0.5$} & $1098.0$\sstd{$2.8$} \\
%  & & Transformer &$70.4$\sstd{$0.0$} & $8002.2$\sstd{$250.1$} & $38.8$\sstd{$0.5$} & $72.0$\sstd{$0.8$} & $349.9$\sstd{$0.6$} & $912.3$\sstd{$2.7$} \\
% \cmidrule{3-9}

% \multirow{3}{*}{Rev-KL} & & GRU &$70.2$\sstd{$0.0$} & $11089.1$\sstd{$153.0$} & $14.1$\sstd{$0.0$} & $34.9$\sstd{$0.2$} & $265.1$\sstd{$1.9$} & $603.6$\sstd{$2.8$} \\
%  & & DeepSets &$70.2$\sstd{$0.1$} & $11604.3$\sstd{$110.4$} & $14.1$\sstd{$0.2$} & $35.1$\sstd{$0.2$} & $306.9$\sstd{$2.1$} & $497.8$\sstd{$22.8$} \\
%  & & Transformer &$70.3$\sstd{$0.1$} & $5111.2$\sstd{$49.1$} & $14.0$\sstd{$0.1$} & $34.3$\sstd{$0.3$} & $263.5$\sstd{$0.8$} & $664.5$\sstd{$8.1$} \\
% \cmidrule{2-9}

% \multirow{3}{*}{Fwd-KL} & \multirow{6}{*}{\rotatebox[origin=c]{90}{Flow}} & GRU &$70.3$\sstd{$0.1$} & $21890.2$\sstd{$58.5$} & $23.8$\sstd{$0.2$} & $57.6$\sstd{$0.5$} & $366.8$\sstd{$2.6$} & $1007.0$\sstd{$2.5$} \\
%  & & DeepSets &$70.4$\sstd{$0.2$} & $23320.5$\sstd{$691.7$} & $23.5$\sstd{$0.6$} & $58.0$\sstd{$0.7$} & $532.1$\sstd{$0.3$} & $1099.1$\sstd{$2.7$} \\
%  & & Transformer &$70.4$\sstd{$0.1$} & $7519.8$\sstd{$507.3$} & $23.5$\sstd{$0.1$} & $56.0$\sstd{$0.5$} & $297.1$\sstd{$1.0$} & $889.7$\sstd{$5.2$} \\
% \cmidrule{3-9}

% \multirow{3}{*}{Rev-KL} & & GRU &$70.3$\sstd{$0.1$} & $11306.4$\sstd{$209.1$} & $13.2$\sstd{$0.0$} & $33.5$\sstd{$0.2$} & $196.7$\sstd{$2.4$} & $556.5$\sstd{$10.3$} \\
%  & & DeepSets &$70.2$\sstd{$0.1$} & $14107.0$\sstd{$354.7$} & $13.1$\sstd{$0.1$} & $33.4$\sstd{$0.3$} & $181.8$\sstd{$10.9$} & $442.8$\sstd{$9.9$} \\
%  & & Transformer &$70.3$\sstd{$0.1$} & $5225.3$\sstd{$292.6$} & $13.0$\sstd{$0.1$} & $33.1$\sstd{$0.1$} & $209.2$\sstd{$1.5$} & $619.8$\sstd{$4.4$} \\
% \bottomrule
%     \end{tabular}
%     \caption{}
%     \vspace{-4mm}
%     \label{tab:}
% \end{table*}