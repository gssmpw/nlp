\section{Experiments}
In this section, we conduct comprehensive experiments to compare SAISA with existing MLLMs.
Furthermore, we perform a series of ablation experiments to further validate the effectiveness of the SAISA architecture.

\subsection{Setups}
\paragraph{Model Configuration.}
To make an apple-to-apple comparison between SAISA and LLaVA-1.5~\cite{liu2024improvedbaselinesvisualinstruction}, we use the same settings as LLaVA-1.5.
Specifically, we employ Vicuna-7B-v1.5~\cite{vicuna} as the default LLM and CLIP-ViT-L/14-336~\cite{radford2021learningtransferablevisualmodels} as the default visual encoder.

\vspace{-0.35cm}
\paragraph{Training Details.}
We utilize the same training data as LLaVA-1.5.
During pre-training, we adopt the pre-train dataset with 558k samples from LLaVA~\cite{liu2023visualinstructiontuning}. This stage takes around 1.5 hours on 8 A800 (80G) GPUs. 
During fine-tuning, we use the mixture instruction-following dataset from LLaVA-1.5. The dataset contains 665k samples from LLaVA-Instruct~\cite{liu2023visualinstructiontuning}, ShareGPT~\cite{sharegpt2023}, VQAv2, GQA~\cite{hudson2019gqanewdatasetrealworld}, OK-VQA~\cite{marino2019okvqavisualquestionanswering}, OCR-VQA~\cite{mishra2019ocrvqa}, A-OKVQA~\cite{schwenk2022aokvqabenchmarkvisualquestion}, TextCaps~\cite{sidorov2020textcapsdatasetimagecaptioning}, RefCOCO~\cite{kazemzadeh2014referitgame, mao2016generationcomprehensionunambiguousobject}, and Visual Genome~\cite{krishna2016visualgenomeconnectinglanguage}.
This stage takes around 8.5 hours on 8 A800 (80G) GPUs.
The total training budget  of SAISA is approximately 80 GPU hours.

% \begin{table}[!t]
% \centering
% \scalebox{0.68}{
%     \begin{tabular}{ll cccc}
%       \toprule
%       & \multicolumn{4}{c}{\textbf{Intellipro Dataset}}\\
%       & \multicolumn{2}{c}{Rank Resume} & \multicolumn{2}{c}{Rank Job} \\
%       \cmidrule(lr){2-3} \cmidrule(lr){4-5} 
%       \textbf{Method}
%       &  Recall@100 & nDCG@100 & Recall@10 & nDCG@10 \\
%       \midrule
%       \confitold{}
%       & 71.28 &34.79 &76.50 &52.57 
%       \\
%       \cmidrule{2-5}
%       \confitsimple{}
%     & 82.53 &48.17
%        & 85.58 &64.91
     
%        \\
%        +\RunnerUpMiningShort{}
%     &85.43 &50.99 &91.38 &71.34 
%       \\
%       +\HyReShort
%         &- & -
%        &-&-\\
       
%       \bottomrule

%     \end{tabular}
%   }
% \caption{Ablation studies using Jina-v2-base as the encoder. ``\confitsimple{}'' refers using a simplified encoder architecture. \framework{} trains \confitsimple{} with \RunnerUpMiningShort{} and \HyReShort{}.}
% \label{tbl:ablation}
% \end{table}
\begin{table*}[!t]
\centering
\scalebox{0.75}{
    \begin{tabular}{l cccc cccc}
      \toprule
      & \multicolumn{4}{c}{\textbf{Recruiting Dataset}}
      & \multicolumn{4}{c}{\textbf{AliYun Dataset}}\\
      & \multicolumn{2}{c}{Rank Resume} & \multicolumn{2}{c}{Rank Job} 
      & \multicolumn{2}{c}{Rank Resume} & \multicolumn{2}{c}{Rank Job}\\
      \cmidrule(lr){2-3} \cmidrule(lr){4-5} 
      \cmidrule(lr){6-7} \cmidrule(lr){8-9} 
      \textbf{Method}
      & Recall@100 & nDCG@100 & Recall@10 & nDCG@10
      & Recall@100 & nDCG@100 & Recall@10 & nDCG@10\\
      \midrule
      \confitold{}
      & 71.28 & 34.79 & 76.50 & 52.57 
      & 87.81 & 65.06 & 72.39 & 56.12
      \\
      \cmidrule{2-9}
      \confitsimple{}
      & 82.53 & 48.17 & 85.58 & 64.91
      & 94.90&78.40 & 78.70& 65.45
       \\
      +\HyReShort{}
       &85.28 & 49.50
       &90.25 & 70.22
       & 96.62&81.99 & \textbf{81.16}& 67.63
       \\
      +\RunnerUpMiningShort{}
       % & 85.14& 49.82
       % &90.75&72.51
       & \textbf{86.13}&\textbf{51.90} & \textbf{94.25}&\textbf{73.32}
       & \textbf{97.07}&\textbf{83.11} & 80.49& \textbf{68.02}
       \\
   %     +\RunnerUpMiningShort{}
   %    & 85.43 & 50.99 & 91.38 & 71.34 
   %    & 96.24 & 82.95 & 80.12 & 66.96
   %    \\
   %    +\HyReShort{} old
   %     &85.28 & 49.50
   %     &90.25 & 70.22
   %     & 96.62&81.99 & 81.16& 67.63
   %     \\
   % +\HyReShort{} 
   %     % & 85.14& 49.82
   %     % &90.75&72.51
   %     & 86.83&51.77 &92.00 &72.04
   %     & 97.07&83.11 & 80.49& 68.02
   %     \\
      \bottomrule

    \end{tabular}
  }
\caption{\framework{} ablation studies. ``\confitsimple{}'' refers using a simplified encoder architecture. \framework{} trains \confitsimple{} with \RunnerUpMiningShort{} and \HyReShort{}. We use Jina-v2-base as the encoder due to its better performance.
}
\label{tbl:ablation}
\end{table*}
% Table generated by Excel2LaTeX from sheet 'Sheet1'
\begin{table}[t]
  \centering
    % \resizebox{\linewidth}{!}
    \scalebox{0.9}
    {
    \begin{tabular}{l|c|ccc}
    \toprule
    \multirow{2}[2]{*}{Method} & Inference &  MMVP  & \multicolumn{2}{c}{CV-Bench~\cite{tong2024cambrian1fullyopenvisioncentric}}     \\
         & TFLOPs$\downarrow$      &   \cite{tong2024eyeswideshutexploring}          & 2D & 3D  \\
    \midrule
    LLaVA-1.5 & 8.53 & 24.7    & \textbf{56.6}  & 59.5   \\
    \rowcolor{cyan!20} SAISA (Ours) & 2.86 & \textbf{26.0}  & 56.2  & \textbf{59.8}   \\
    \bottomrule
    \end{tabular}
    }
    \caption{\textbf{Performance on vision-centric MLLM benchmarks,} based on Vicuna and CLIP.}
  \label{tab:vision-centric}
\end{table}

\begin{figure}[ht]
    \centering
    \begin{subfigure}[b]{0.245\textwidth}
        \centering
        \includegraphics[width=\textwidth]{figures/latency_1_prefill.pdf}
        \caption{Prefill Time (\color{gray}{bsz=1})}
    \end{subfigure}
    \hfill
    \begin{subfigure}[b]{0.23\textwidth}
        \centering
        \includegraphics[width=\textwidth]{figures/latency_4_prefill.pdf}
        \caption{Prefill Time (\color{gray}{bsz=4})}
    \end{subfigure}
    \hfill
    \begin{subfigure}[b]{0.23\textwidth}
        \centering
        \includegraphics[width=\textwidth]{figures/latency_1_generate.pdf}
        \caption{Total Time (\color{gray}{bsz=1})}
    \end{subfigure}
    \hfill
    \begin{subfigure}[b]{0.23\textwidth}
        \centering
        \includegraphics[width=\textwidth]{figures/latency_4_generate.pdf}
        \caption{Total Time (\color{gray}{bsz=4})}
    \end{subfigure}
    \caption{\textbf{Latency on H100 GPU: prefill and total inference time (s).} The gray text in brackets is batch size.} %See Appendix~\ref{} for other architectures.}
    \label{fig:latency}
\end{figure}

%-------------------------------------------------------------------------

\subsection{Main Results}
The performance on benchmarks is shown in Table~\ref{tab:mllms}, Table~\ref{tab:vqa} and Table~\ref{tab:vision-centric}, and the result of the inference latency test is shown in Table~\ref{tab:latency}.

We evaluate SAISA on a range of benchmarks, including: (1) comprehensive benchmarks for instruction-following MLLMs such as MMMU~\cite{yue2024mmmumassivemultidisciplinemultimodal}, MME~\cite{fu2024mmecomprehensiveevaluationbenchmark}, MMBench~\cite{liu2024mmbenchmultimodalmodelallaround}, MMBench-CN~\cite{liu2024mmbenchmultimodalmodelallaround}, and SEED-bench~\cite{li2023seedbenchbenchmarkingmultimodalllms}; (2) hallucination benchmark such as POPE~\cite{li2023evaluatingobjecthallucinationlarge}, which evaluates MLLMs' degree of hallucination on three subsets: random, popular, and adversarial; (3) general visual question answering benchmarks such as GQA~\cite{hudson2019gqanewdatasetrealworld} and ScienceQA IMG~\cite{lu2022learnexplainmultimodalreasoning}; (4) fine-grained visual question answering benchmarks such as OK-VQA~\cite{marino2019okvqavisualquestionanswering} and TextVQA~\cite{singh2019vqamodelsread}, OK-VQA requires fine-grained image understanding and spatial understanding, and TextVQA is an OCR-related benchmark; (5) vision-centric MLLM benchmarks such as MMVP~\cite{tong2024eyeswideshutexploring} and CV-Bench~\cite{tong2024cambrian1fullyopenvisioncentric}.
In the inference latency test, the latency is reported as the time of LLM prefilling during inference with varying numbers of text tokens.
Table~\ref{tab:mllms} shows the comparison on the comprehensive benchmarks for instruct-following MLLMs.
SAISA outperforms BLIP-2~\cite{li2023blip2bootstrappinglanguageimagepretraining}, InstructBLIP~\cite{dai2023instructblipgeneralpurposevisionlanguagemodels}, MiniGPT-4~\cite{zhu2023minigpt4enhancingvisionlanguageunderstanding}, MiniGPT-v2\cite{chen2023minigptv2largelanguagemodel}, Otter~\cite{li2023ottermultimodalmodelincontext}, Shikra~\cite{chen2023shikraunleashingmultimodalllms}, and IDEFICS~\cite{idefics} utilizing LLama-7B and LLama-65B~\cite{touvron2023llamaopenefficientfoundation} across all these benchmarks.
Compared to Qwen-VL-Chat~\cite{Qwen-VL} trained on data with 1.4B samples, SAISA performs better on 4 out of 5 benchmarks.
Compared to LLaVA-1.5~\cite{liu2024improvedbaselinesvisualinstruction}, SAISA performs better on 3 out of 5 benchmarks.
Table~\ref{tab:vqa} shows the comparison on the hallucination and visual question answering benchmarks, and Table~\ref{tab:vision-centric} shows the comparison on vision-centric MLLM benchmarks.
Since most previous models do not evaluate performance on vision-centric MLLM benchmarks, we compare SAISA with LLaVA-1.5.
SAISA achieves the best overall performance compared to other baseline MLLMs, and strikes the optimal balance between effectiveness and efficiency.

\subsection{Ablation Study}
\paragraph{Ablation on LLMs and Visual Encoders.}
As presented in Table~\ref{tab:ablation}, we perform multiple ablation experiments on both LLMs and visual encoders to evaluate the robustness of SAISA.
We tune a set of SAISA models using a variety of LLM backbones and visual encoders.
The ablated LLMs include Vicuna-7B~\cite{vicuna} and two LLMs using grouped query attention (GQA)~\cite{ainslie2023gqatraininggeneralizedmultiquery}, such as Mistral-7B~\cite{jiang2023mistral7b} and Llama3-8B~\cite{llama3v}.
The ablated visual encoders include two ViT-based~\cite{dosovitskiy2021imageworth16x16words} visual backbones such as CLIP-ViT-L/14-336~\cite{radford2021learningtransferablevisualmodels} and  SigLIP-ViT-SO400M/14-384~\cite{zhai2023sigmoidlosslanguageimage}, and a ConvNeXt-based~\cite{liu2022convnet2020s} visual encoder such as ConvNeXt-XXL-1024 from OpenCLIP~\cite{ilharco_gabriel_2021_5143773, schuhmann2022laion5bopenlargescaledataset}.
The experimental results demonstrate that SAISA consistently achieves superior performance to LLaVA-1.5 across different LLM backbones and visual encoders, while dramatically reducing computational costs.

\vspace{-0.35cm}
\paragraph{Ablation on Pre-training Strategies.}
As shown in Table~\ref{tab:ablation_train}, we conduct an ablation study to investigate the effects of SAISA's pre-training strategies.
We tune a SAISA model where the full projector (32 MLPs) is tunable during pre-training, and the other settings keep the same as the original SAISA.
With more randomly initialized parameters, we observe a performance drop when pre-training the full projector.
We attribute this drop to the small amount of pre-training data with only 558k samples.
The ablation study demonstrates the effectiveness of our pre-training strategy, which provides a robust initialization for the subsequent fine-tuning stage.

\vspace{-0.35cm}
\paragraph{Ablation on Projector Designs.}
Previous works find that replacing linear projection with MLP projection improves performance in MLLM~\cite{liu2024improvedbaselinesvisualinstruction} and self-supervised learning~\cite{chen2020simpleframeworkcontrastivelearning, chen2020improvedbaselinesmomentumcontrastive}.
We conduct an experiment to investigate the impact of projector designs in SAISA.
We tune a model under the same configuration as the original SAISA model but replace the MLPs in the projector with linear layers.
Table ~\ref{tab:linear} shows that the model with MLPs in the projector performs better than the model with linear layers, which is consistent with the finding of the previous study~\cite{liu2024improvedbaselinesvisualinstruction}.
Notably, we note that even the SAISA model with linear layers achieves comparable performance to LLaVA-1.5 with MLP projection.
This observation provides additional evidence for the effectiveness of SAISA.

% Table generated by Excel2LaTeX from sheet 'Sheet1'
\begin{table}[t]
  \centering
    \resizebox{\linewidth}{!}
    % \scalebox{0.8}
    {
    \begin{tabular}{l|cccccc}
    \toprule
    Pre-trained &  MMMU & \multicolumn{2}{c}{MMBench} & \multirow{2}[2]{*}{POPE}  & SQA & OK-  \\
      Parameters  & VAL      &  EN & CN &           & IMG & VQA \\
    \midrule
    Full Projector & 34.8 & 59.2 & 51.1   & 85.6  & 67.8 & 53.1 \\
    \rowcolor{cyan!20} Shared MLP & 36.9 & 65.7 & 59.0 & 87.2  & 70.1 & 56.8  \\
    \bottomrule
    \end{tabular}
    }
    \caption{\textbf{Ablation on Pre-training Strategies.} 
    ``Full Projector" denotes we pre-train 32 MLPs.
    ``Shared MLP" denotes our strategy, which involves tuning a shared MLP for efficiency.
    Our strategy provides an effective initialization for visul fine-tuning when using the small pre-training dataset with 558k samples.
    }
  \label{tab:ablation_train}
\end{table}

\begin{table}[t]
  \centering
    \resizebox{\linewidth}{!}
    % \scalebox{0.8}
    {
    \begin{tabular}{l|c|cccccc}
    \toprule
    \multirow{2}[2]{*}{Method} & \multirow{2}[2]{*}{Proj.} & MMMU & \multicolumn{2}{c}{MMBench} & \multirow{2}[2]{*}{POPE}  & SQA & OK-  \\
        &  & VAL  & EN & CN     &             & IMG & VQA \\
    \midrule
    LLaVA-1.5 & MLP & 35.7  & 64.3 & 58.3  & 86.8  & 66.8 & 53.4  \\
    SAISA & Linear & 35.7 & 65.3 & 56.6   & 85.8  & 69.2 & 53.6  \\
    \rowcolor{cyan!20} SAISA & MLP & 36.9 & 65.7 & 59.0 & 87.2  & 70.1 & 56.8  \\
    \bottomrule
    \end{tabular}
    }
    \caption{\textbf{Ablation on Projector Designs.} ``Proj." denotes projector type.
    The SAISA model that uses MLPs outperforms the model that uses linear layers.
    Notably, SAISA with linear layers achieves comparable performance to LLaVA-1.5 with MLP.
    }
  \label{tab:linear}
\end{table}
