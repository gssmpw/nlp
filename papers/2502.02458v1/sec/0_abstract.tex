\begin{abstract}

Multimodal Large Language Models (MLLMs) mainly fall into two architectures, each involving a trade-off between training and inference efficiency: embedding space alignment (e.g., LLaVA-1.5) is inefficient during inference, while cross-attention space alignment (e.g., Flamingo) is inefficient in training.
 this paper, we compare these two architectures and identify the key factors for building efficient MLLMs.
A primary difference between them lies in how attention is applied to visual tokens, particularly in their interactions with each other.
To investigate whether attention among visual tokens is necessary, we propose a new self-attention mechanism, NAAViT (\textbf{N}o \textbf{A}ttention \textbf{A}mong \textbf{Vi}sual \textbf{T}okens), h eliminates this type of attention.
Our pilot experiment on LLaVA-1.5 shows that attention among visual tokens is highly redundant.
, we introduce SAISA (\textbf{S}elf-\textbf{A}ttention \textbf{I}nput \textbf{S}pace \textbf{A}lignment), a novel architecture that enhance both training and inference efficiency.
SAISA directly aligns visual features with the input spaces of NAAViT self-attention blocks, reducing computational overhead in both self-attention blocks and feed-forward networks (FFNs).
Using the same configuration as LLaVA-1.5, SAISA reduces inference FLOPs by 66\% and training budget by 26\%, while achieving superior performance in terms of accuracy.
Comprehensive ablation studies further validate the effectiveness of SAISA across various LLMs and visual encoders.
The code and model will be publicly available at \href{https://github.com/icip-cas/SAISA}{https://github.com/icip-cas/SAISA}.
\end{abstract}