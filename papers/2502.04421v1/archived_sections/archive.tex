% Archived 2024-02-13
\subsection{Refining the Dataset}
In order to get the best results with the chosen machine learning parameters, the entire dataset was ingested and then filtered by analysts. The 9574 victim records on the site were filtered down to those whose descriptions were more than 255 characters long and whose adversaries were previously identified by our analyst through the CTI extraction pipeline. This narrowed the total amount of records down to 1,556 from 9,754. Using OpenAI's ChatGPT-4, we generated profiles for the remaining organizations, encompassing attributes such as annual revenue, strategic importance, sector, and reputational risk (refer to Table \ref{tab:company_profile_attributes}).After the initial filtering process, the profiled companies underwent a final review by analysts. Those with undesirable values or missing attributes were subsequently excluded from the final dataset. The refined dataset comprised 490 unique companies and 33 distinct ransomware groups. 

The machine learning dataset was built upon the refined set of 490 companies and 33 threat actors. Employing the OSINT CTI on adversaries, extracted and stored through our application, we executed a join operation on the adversary name. We utilized adversary information from our local database, comparing it with data from the refined set. For each instance where an adversary targeted a company's country of origin or industry sector, a new machine learning record was generated. The process of generating the machine learning dataset is further illustrated in Figure \ref{fig:ml_dataset_generation}. This thorough integration process yielded a total of 6,892 records, establishing the foundational dataset for our machine learning endeavors.

\begin{figure}[H]
  \captionsetup{
    belowskip=-8pt,
  }
  \centering
  \fbox{\includegraphics[width=0.985\linewidth]{figures/ml_dataset_generation.png}}%
  \caption[Machine Learning Dataset Generation]{
    This class diagram depicts the process of extracting, integrating, and transforming data from ransomware.live into our machine learning dataset, aimed at identifying potential ransomware victims. The process employs a web scraper to extract information about ransomware victims and their attackers. It utilizes the CTI Extraction, as illustrated in Figure 2, to establish a connection between our repository of adversarial data and the information obtained from ransomware.live. The resultant dataset is stored in a CSV file.
  }
  \label{fig:ml_dataset_generation}
\end{figure}


\begin{figure}
  \captionsetup{
    belowskip=-8pt,
  }
  \centering
  \includegraphics[width=0.985\linewidth]{figures/oneclass_svm_attribute_spread.png}%
\caption[Dataset Records Attribute Count]{
Total count of values (0-9) for organizational attributes as defined in \ref{tab:company_profile_attributes}. The graph illustrates the distribution, indicating a focus on medium values with a range from somewhat-low to somewhat-high for our organizational victims.}
  \label{fig:attribute_spread}
\end{figure}