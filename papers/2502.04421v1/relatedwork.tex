\section{Related Works}
\label{sec:relatedworks}
\subsection{Automating Cyber Threat Intelligence Extraction}
Recent studies have showcased advancements in Cyber Threat Intelligence (CTI) extraction, particularly through the use of Large Language Models (LLMs). Various methods have been developed to refine the automation of extracting different types of CTI, leveraging LLM capabilities for efficient analysis.

The "Time for aCTIon" framework employs GPT-3.5 to streamline CTI extraction using prompt engineering, zero-shot learning, and in-context learning. This approach obviates the need for in-house model training by utilizing LLMs as a Service (LLMaaS), effectively handling diverse CTI scenarios. The framework has been rigorously evaluated against other tools, demonstrating its superiority in quickly and accurately extracting critical CTI elements while optimizing resource utilization \cite{aCTIon}.

Other initiatives, such as TTPDrill and ThreatKG, utilize NLP and information retrieval models to transform adversarial TTPs into structured formats like the Structured Threat Information eXpression  v2.1 (STIX) framework and develop comprehensive threat actor profiles through knowledge graphs \cite{TTPDrill, ThreatKG}. The Ladder framework further extends this by employing models like XLM-RoBERTa and Roberta to associate extracted attack patterns with Mitre Att\&ck IDs, providing a systematic strategy to preempt threats \cite{LADDER, mitre2023attack}.

Further, models like those showcased in CTI View and research by Irshad and Siddiqui extract a combination of strategic CTI and TTPs to pinpoint specific threat actors behind cyber-attacks, aiding the analysis of APT threat trends and the development of proactive defense strategies \cite{CtiView, Irshad2023}.

Building on these advancements, our unique approach leverages LLMs for CTI extraction and employs techniques such as zero-shot learning, chain-of-thought, and polymorphic prompting. This strategy allows for a one-time querying process, significantly enhancing resource utilization and speeding up the extraction of critical CTI elements. Unlike traditional methods, which often rely heavily on manual efforts and provide incomplete threat landscapes, our method offers clear, rationale-driven outputs that improve extraction accuracy and provide deeper insights for analysts. By incorporating these innovations, we deliver a more sophisticated and effective solution for navigating and addressing the complexities of the threat landscape.

\subsection{Ransomware Prevention and Mitigation}
Numerous studies have focused on identifying and mitigating ransomware threats \cite{McIntosh:2023,Oujezsky:2023,Thomas:2018,Chayal:2022, Kim2022}. Recognizing its effectiveness in safeguarding against this pervasive threat, the primary goal of these measures is to prevent and mitigate potential risks and security threats, effectively protecting systems, data, and networks from unauthorized access, breaches, or malicious activities \cite{Aldaraani:2018}.

Levesque et al. \cite{Levesque2014} developed a user-risk prediction model to identify potential malware targets by analyzing social, demographic, and behavioral factors, identifying twelve key features as crucial predictors. Similarly, Yilmaz et al. \cite{Yilmaz2023} explored the human element by investigating the relationship between personality types and ransomware victimization, though they found no significant correlation.

In contrast to these individual-focused studies, our solution emphasizes the broader context of company profiles and the operational characteristics of ransomware groups using the SKRAM (Skills, Knowledge, Resources, Authorities, and Motivation) model. This shift enables a more strategic approach to cybersecurity, allowing organizations to proactively adjust their defenses by understanding both the potential targets and the attackers. Additionally, our model generates a risk score for each identified threat, providing a quantifiable measure that organizations can use to prioritize and filter threats based on their potential impact. This holistic view significantly improves organizational resilience to ransomware attacks by reducing the noise in threat detection and focusing on the most pertinent threats.