% !TeX root = main.tex
% \documentclass{uai2024}
\documentclass[accepted]{uai2024} % after acceptance, for a revised version;
                        
%% There is a class option to choose the math font
% \documentclass[mathfont=newtx]{uai2024} 

% \documentclass[mathfont=newtx]{uai2024} NOTE: Only keep *one* line above as
% appropriate, as it will be replaced automatically for papers to be published. Do not
% make any other change above this note for an accepted version.

\usepackage[american]{babel}
% \usepackage[british]{babel}

\usepackage{natbib} % has a nice set of citation styles and commands
    \bibliographystyle{authordate3}
\renewcommand{\bibsection}{\subsubsection*{References}} \usepackage{mathtools}

\usepackage{booktabs} % commands to create good-looking tables
\usepackage{tikz} % nice language for creating drawings and diagrams
\usepackage{cleveref}
\usepackage{pifont}
\usepackage{subcaption}
\usepackage{amssymb}
\usepackage{amsthm}
\newtheorem{theorem}{Theorem}
\newtheorem{definition}{Definition}
\newtheorem{remark}{Remark}
\Crefname{lemma}{Lemma}{Lemmas}
\Crefname{opup}{Operator Update}{Operator Updates}

\crefname{equation}{Eq.}{Eqs.}
\crefname{figure}{Fig.}{Figs.}
\crefname{theorem}{Th.}{Ths.}
% \crefname{table}{Tab.}{Tabs.}
\crefname{algocf}{Algorithm}{Algorithms}

%% Provided macros \smaller: Because the class footnote size is essentially LaTeX's
% \small, redefining \footnotesize, we provide the original \footnotesize using this
% macro. (Use only sparingly, e.g., in drawings, as it is quite small.)

% \usepackage{mathabx}
\usepackage{array}
\newcolumntype{L}[1]{>{\raggedright\let\newline\\\arraybackslash\hspace{0pt}}m{#1}}
\newcolumntype{C}[1]{>{\centering\let\newline\\\arraybackslash\hspace{0pt}}m{#1}}
\newcolumntype{R}[1]{>{\raggedleft\let\newline\\\arraybackslash\hspace{0pt}}m{#1}}

% \usepackage{algorithm}
% \usepackage{algpseudocode}
% \algnewcommand{\IfThenElse}[3]{% \IfThenElse{<if>}{<then>}{<else>}
%   \State \algorithmicif\ #1\ \algorithmicthen\ #2\ \algorithmicelse\ #3}
% \algnewcommand{\IfThen}[2]{% \IfThen{<if>}{<then>}
%   \State \algorithmicif\ #1\ \algorithmicthen\ #2} 

\usepackage[ruled,vlined]{algorithm2e}
\newcommand\mycommfont[1]{\smaller\textcolor{black}{#1}}
\SetCommentSty{mycommfont}
\SetKwInput{KwDefine}{Define}
\SetKwInput{KwOptional}{Optional data}
\SetKwInput{KwRequire}{Require}
\SetKwInput{KwEnsure}{Ensure}
% \setlength{\algomargin}{1.2em}

\usepackage{etoolbox}
\usepackage{thm-restate}

\newcommand*\squeezespaces[1]{% %% <- #1 is a number between 0 and 1
  \thickmuskip=\scalemuskip{\thickmuskip}{#1}%
  \medmuskip=\scalemuskip{\medmuskip}{#1}%
  \thinmuskip=\scalemuskip{\thinmuskip}{#1}%
  \nulldelimiterspace=#1\nulldelimiterspace
  \scriptspace=#1\scriptspace
}
\newcommand*\scalemuskip[2]{%
  \muexpr #1*\numexpr\dimexpr#2pt\relax\relax/65536\relax
} %% <- based on  https://tex.stackexchange.com/a/198966/156366


\makeatletter
% Remove the right-hand margin in algorithms
\patchcmd{\@algocf@start}% <cmd>
  {-1.5em}% <search>
  {0pt}% <replace>
  {}{}% <success><failure>
\makeatother

\newcommand{\Pa}{\operatorname{Pa}}
\newcommand{\Ch}{\operatorname{Ch}} \newcommand{\Ne}{\operatorname{Ne}}
\newcommand{\Ad}{\operatorname{Ad}} \DeclareMathOperator*{\argmax}{arg\,max}
\newcommand{\iid}{\stackrel{\text{iid}}{\sim}}
\newcommand{\indep}{\perp \!\!\! \perp}

\newcommand{\R}{\mathbb{R}} \newcommand{\N}{\mathbb{N}} \newcommand{\C}{\mathbb{C}}
\newcommand{\curlystack}[1]{ \left\{\begin{matrix}#1\end{matrix}\right.}
\newcommand{\stack}[1]{\begin{matrix}#1\end{matrix}}

\DeclareRobustCommand{\parhead}[1]{\noindent \textbf{#1} }

\newcommand{\data}{\bm{D}}
\usepackage[textsize=smaller]{todonotes}

\title{Extremely Greedy Equivalence Search}

\author[1]{\href{mailto:<aon2108@columbia.edu>?Subject=Your UAI 2024 paper}{Achille
Nazaret}{}}
\author[1,2]{David Blei}
\affil[1]{%
    Department of Computer Science\\
    Columbia University\\
    New York\\
    USA } 
\affil[2]{%
    Department of Statistics \\
    Columbia University\\
    New York\\
    USA } 
\begin{document}
\maketitle

\begin{abstract}
% \begin{abstract}
% Adversarial attacks pose significant threats to deploying Graph Neural Networks (GNNs) in real-world applications. Lines of studies have made progress in minimizing the influence of adversarial perturbations. However, existing methods often rely on fixed priors about the dataset or attacker, limiting their ability to generalize across diverse scenarios. These approaches cannot adaptively learn the intrinsic properties of the dataset.
% In this paper, we propose a novel framework, \ModelName (Graph \textbf{P}urification through t\textbf{R}ansfer \textbf{EN}tropy-guided \textbf{N}on-i\textbf{S}otropic Diffu\textbf{S}ion), which leverages a graph diffusion generative model to learn intrinsic properties and recover the clean structure of adversarial graphs. However, two key challenges arise: (1) The graph diffusion model’s uniform noise injection to all nodes during the forward process can over-perturb the graph, erasing valuable information and making recovery difficult, and (2) the diversity of the diffusion model in the reverse denoising process may cause the generated graph to deviate from the target clean structure.
% To address these challenges, we introduce a LID-Driven Non-Isotropic Diffusion process, which injects noise selectively, focusing on adversarial nodes while preserving the clean structure. Additionally, we propose a Graph Transfer Entropy-Guided Reverse Denoising process that maximizes transfer entropy to reduce uncertainty in the reverse process, ensuring that the generated graph remains aligned with the clean structure.
% Extensive experiments on both graph and node classification demonstrate our proposed \ModelName framework's robustness and superior generalization.
% Our code is available at \textcolor{mytablecolor}{\url{https:///}}
% \end{abstract}


% \begin{abstract}
% % priors and no priors-free 继续总结凝练
% % 鲁棒和各向同性有关
% Adversarial attacks pose significant threats to deploying Graph Neural Networks (GNNs) in real-world applications. Lines of studies have made progress in minimizing the influence of adversarial perturbations. 
% They often rely on priors such as neighbor similarity in clean graphs to restore the correct structure. However, this approach is less effective on datasets where these priors do not hold.
% % Their robustness methods often rely on priors of clean graphs or attacks.
% To achieve more generalized robustness, we need methods that can learn clean graph properties and recover the correct structure based on those learned properties, rather than depending on prior assumptions.
% Driven by this goal, in this work, we approach adversarial attacks from a distribution perspective: these attacks cause the graph distribution to deviate from the original clean distribution.
% % From this perspective, we propose using a graph generative model to learn the clean graph distribution without relying on priors and to purify adversarial graphs through distribution mapping.
% % 前面不要,直接提diffusion
% % Among various graph generative models, the diffusion model’s reverse denoising process naturally aligns with the removal of adversarial perturbations,
% % making it an ideal choice for mapping between adversarial and clean distributions. 
% From this perspective, we propose using the graph diffusion model to learn the clean graph distribution and purify adversarial graphs through distribution mapping.
% % The diffusion model’s reverse denoising process naturally aligns with the removal of adversarial perturbations, making it an ideal choice for mapping between adversarial and clean distributions.
% However, in graph diffusion models, 1) the indiscriminate noise injection across all nodes during the forward process can remove useful information still present in adversarial samples, making it difficult to recover the clean structure during reverse purification. 2) the diversity of the reverse denoising process may cause the generated graph to deviate from the target clean structure.
% To address these challenges, we propose a novel framework, \ModelName, to enhance gra\textbf{P}h rob\textbf{U}stness through t\textbf{R}ansfer \textbf{EN}tropy guid\textbf{E}d non-i\textbf{S}otropic diffu\textbf{S}ion purification.
% Our method introduces a LID-based Non-Isotropic Diffusion process, where we use local intrinsic dimensionality (LID) to estimate the adversarial degree of each node, enabling selective noise injection to focus on adversarial nodes while preserving the clean structure. Additionally, we propose a Graph Transfer Entropy-Guided Denoising process, which maximizes transfer entropy at each step to reduce uncertainty during the reverse process, 
% % ensuring the generated graph stays aligned with the clean structure without deviation.
% ensuring the generated graph matches the target clean graph without deviation.
% Extensive experiments on both graph and node classification tasks demonstrate the robustness of our \ModelName framework. Our code is available at \textcolor{mytablecolor}{\url{https:///}}.
% \end{abstract}

\begin{abstract}
Adversarial evasion attacks pose significant threats to graph learning, with lines of studies that have improved the robustness of Graph Neural Networks (GNNs).
However, existing works rely on priors about clean graphs or attacking strategies, which are often heuristic and inconsistent.
To achieve robust graph learning over different types of evasion attacks and diverse datasets, we investigate this problem from a prior-free structure purification perspective.
Specifically, we propose a novel \underline{\textbf{Diff}}usion-based \underline{\textbf{S}}tructure \underline{\textbf{P}}urification framework named \textbf{\ModelName}, which creatively incorporates the graph diffusion model to learn intrinsic distributions of clean graphs and purify the perturbed structures by removing adversaries under the direction of the captured predictive patterns without relying on priors.
\ModelName~is divided into the forward diffusion process and the reverse denoising process, during which structure purification is achieved.
To avoid valuable information loss during the forward process, we propose an LID-driven non-isotropic diffusion mechanism to selectively inject noise anisotropically.
To promote semantic alignment between the clean graph and the purified graph generated during the reverse process, we reduce the generation uncertainty by the proposed graph transfer entropy guided denoising mechanism.
Extensive experiments demonstrate the superior robustness of \ModelName~against evasion attacks.
% The reverse denoising process of diffusion models naturally aligns with removing graph adversarial perturbations, making them suitable for learning clean graph distribution and removing adversarial perturbations based on the learned distributional patterns without relying on priors.
% purifying adversarial graphs through distribution mapping.
% However, the indiscriminate noise injection in graph diffusion models can erase useful information, while the diversity of the reverse process may cause generated graphs to deviate from the target clean graph, making it difficult to directly apply them for purifying adversarial graph data.
% To address these challenges, 
% In this work, we propose a novel framework \ModelName, which introduces a LID-driven non-isotropic forward diffusion process and a transfer entropy-guided reverse denoising process to precisely remove adversarial perturbations and guide the generation toward the target clean graph.
% Our code is available at \textcolor{mytablecolor}{\url{https:///}}.
\end{abstract}


\keywords{robust graph learning, adversarial evasion attack, graph structure purification, graph diffuison}
\end{abstract}

\newcommand{\bm}[1]{\mathbf{#1}}

\newcommand{\cone}{\raisebox{.5pt}{\textcircled{\raisebox{-.9pt} {1}}}}
\newcommand{\ctwo}{\raisebox{.5pt}{\textcircled{\raisebox{-.9pt} {2}}}}



\section{Introduction}
\IEEEPARstart{I}{n} recent years, flourishing of Artificial Intelligence Generated Content (AIGC) has sparked significant advancements in modalities such as text, image, audio, and even video. 
Among these, AI-Generated Image (AGI) has garnered considerable interest from both researchers and the public.
Plenty of remarkable AGI models and online services, such as StableDiffusion\footnote{\url{https://stability.ai/}}, Midjourney\footnote{\url{https://www.midjourney.com/}}, and FLUX\footnote{\url{https://blackforestlabs.ai/}}, offer users an excellent creative experience.
However, users often remain critical of the quality of the AGI due to image distortions or mismatches with user intentions.
Consequently, methods for assessing the quality of AGI are becoming increasingly crucial to help improve the generative capabilities of these models.

Unlike Natural Scene Image (NSI) quality assessment, which focuses primarily on perception aspects such as sharpness, color, and brightness, AI-Generated Image Quality Assessment (AGIQA) encompasses additional aspects like correspondence and authenticity. 
Since AGI is generated on the basis of user text prompts, it may fail to capture key user intentions, resulting in misalignment with the prompt.
Furthermore, authenticity refers to how closely the generated image resembles real-world artworks, as AGI can sometimes exhibit logical inconsistencies.
While traditional IQA models may effectively evaluate perceptual quality, they are often less capable of adequately assessing aspects such as correspondence and authenticity.

\begin{figure}\label{fig:radar}
    \centering
    \includegraphics[width=1.0\linewidth]{figures/radar_plot.pdf}
    \caption{A comparison on quality, correspondence, and authenticity aspects of AIGCIQA2023~\cite{wang2023aigciqa2023} dataset illustrates the superior performance of our method.}
\end{figure}

Several methods have been proposed specifically for the AGIQA task, including metrics designed to evaluate the authenticity and diversity of generated images~\cite{gulrajani2017improved,heusel2017gans}. 
Nevertheless, these methods tend to compare and evaluate grouped images rather than single instances, which limits their utility for single image assessment.
Beginning with AGIQA-1k~\cite{zhang2023perceptual}, a series of AGIQA databases have been introduced, including AGIQA-3k~\cite{li2023agiqa}, AIGCIQA-20k~\cite{li2024aigiqa}, etc.
Concurrently, there has been a surge in research utilizing deep learning methods~\cite{zhou2024adaptive,peng2024aigc,yu2024sf}, which have significantly benefited from pre-trained models such as CLIP~\cite{radford2021learning}. 
These approaches enhance the analysis by leveraging the correlations between images and their descriptive texts.
While these models are effective in capturing general text-image alignments, they may not effectively detect subtle inconsistencies or mismatches between the generated image content and the detailed nuances of the textual description.
Moreover, as these models are pre-trained on large-scale datasets for broad tasks, they might not fully exploit the textual information pertinent to the specific context of AGIQA without task-specific fine-tuning.
To overcome these limitations, methods that leverage Multimodal Large Language Models (MLLMs)~\cite{wang2024large,wang2024understanding} have been proposed.
These methods aim to fully exploit the synergies of image captioning and textual analysis for AGIQA.
Although they benefit from advanced prompt understanding, instruction following, and generation capabilities, they often do not utilize MLLMs as encoders capable of producing a sequence of logits that integrate both image and text context.

In conclusion, the field of AI-Generated Image Quality Assessment (AGIQA) continues to face significant challenges: 
(1) Developing comprehensive methods to assess AGIs from multiple dimensions, including quality, correspondence, and authenticity; 
(2) Enhancing assessment techniques to more accurately reflect human perception and the nuanced intentions embedded within prompts; 
(3) Optimizing the use of Multimodal Large Language Models (MLLMs) to fully exploit their multimodal encoding capabilities.

To address these challenges, we propose a novel method M3-AGIQA (\textbf{M}ultimodal, \textbf{M}ulti-Round, \textbf{M}ulti-Aspect AI-Generated Image Quality Assessment) which leverages MLLMs as both image and text encoders. 
This approach incorporates an additional network to align human perception and intentions, aiming to enhance assessment accuracy. 
Specially, we distill the rich image captioning capability from online MLLMs into a local MLLM through Low-Rank Adaption (LoRA) fine-tuning, and train this model with human-labeled data. The key contributions of this paper are as follows:
\begin{itemize}
    \item We propose a novel AGIQA method that distills multi-aspect image captioning capabilities to enable comprehensive evaluation. Specifically, we use an online MLLM service to generate aspect-specific image descriptions and fine-tune a local MLLM with these descriptions in a structured two-round conversational format.
    \item We investigate the encoding potential of MLLMs to better align with human perceptual judgments and intentions, uncovering previously underestimated capabilities of MLLMs in the AGIQA domain. To leverage sequential information, we append an xLSTM feature extractor and a regression head to the encoding output.
    \item Extensive experiments across multiple datasets demonstrate that our method achieves superior performance, setting a new state-of-the-art (SOTA) benchmark in AGIQA.
\end{itemize}

In this work, we present related works in Sec.~\ref{sec:related}, followed by the details of our M3-AGIQA method in Sec.~\ref{sec:method}. Sec.~\ref{sec:exp} outlines our experimental design and presents the results. Sec.~\ref{sec:limit},~\ref{sec:ethics} and~\ref{sec:conclusion} discuss the limitations, ethical concerns, future directions and conclusions of our study. 
\section{Related Work}
\label{sec:related}


\noindentbold{2D visual foundation models}
In recent years, we have witnessed the emergence of large pretrained models—so-called foundation models that are trained on large-scale datasets and serve as a \textit{foundation} for many downstream tasks.
These models demonstrate remarkable versatility across multiple modalities, including language~\cite{team2023gemini,touvron2023llama,touvron2023llama2,dubey2024llama3,vicuna2023,radford2019language,brown2020language,chung2024scaling,achiam2023gpt,bai2023qwen,yang2024qwen2,jiang2023mistral,jiang2024mixtral}, vision~\cite{sam,ravi2024sam,dino_v1,oquab2023dinov2,zou2024segment,rombach2022high,ho2020denoising,nichol2021improved,songdenoising,songscore}, audio~\cite{deshmukh2023pengi,zhang2023speechgpt,rubenstein2023audiopalm,borsos2023audiolm}. 
Furthermore, they enable multi-modal reasoning capabilities that bridge across different modalities~\cite{girdharImageBindOneEmbedding2023,Qwen-VL,llava,radfordLearningTransferableVisual2021,jia2021scaling,team2024gemini}.
Among these models, those that operate on visual modalities are known as visual foundation models (VFM).
VFMs excel in various computer vision tasks such as image segmentation~\cite{sam,ravi2024sam,zou2024segment,zou2023generalized,cheng2021per,cheng2022masked,jain2023oneformer,li2024semantic}, object detection~\cite{liu2023grounding,carion2020end}, representation learning~\cite{dino_v1,oquab2023dinov2}, and open-vocabulary understanding~\cite{radfordLearningTransferableVisual2021,li2022language,ghiasi2022scaling,ram,ram_pp,yu2023convolutions,kang2024defense,naeem2024silc,cho2024cat}.
When integrated with large language models, they enable sophisticated visual reasoning and natural language interactions~\cite{llava,Qwen-VL,girdharImageBindOneEmbedding2023,team2024gemini,guo2024regiongpt,yuan2024osprey,you2023ferret}.
We use such vision language models to construct open vocabulary segmentation and captions for point clouds based on multiview images.







\noindentbold{Open-vocabulary 3D segmentation}
Building on the success of 2D VFMs, recent work have extended open-vocabulary capabilities to 3D scene understanding.
OpenScene~\cite{Peng2023OpenScene} first introduced zero-shot 3D semantic segmentation by distilling knowledge from language-aligned image encoders~\cite{li2022language,ghiasi2022scaling}.
Subsequent methods~\cite{ding2022pla,yang2024regionplc,jiang2024open} leverage multiview images to generate textual captions, which then serve as training supervision.
However, these methods face challenges in generating high-quality 3D mask-text pairs at scale.
For open-vocabulary 3D instance segmentation, existing methods~\cite{takmaz2023openmask3d,nguyen2024open3dis,huang2024openins3d} typically rely on closed-vocabulary proposal networks such as Mask3D~\cite{schult2023mask3d}, which inherently constrains their ability to detect novel object categories. 
Moreover, these methods leverage 2D VFMs like CLIP~\cite{radfordLearningTransferableVisual2021} for region classification by projecting 3D regions onto multiple 2D views.
This approach requires both 2D images and 3D point clouds during inference. Additionally, it necessitates multiple inferences of large 2D models on projected masks, resulting in high computational costs. 
We address these limitations by developing the first single-stage open-vocabulary 3D instance segmentation model that operates directly in 3D without ground truth labels, using our \dataname dataset and Segment3D~\cite{huang2024segment3d} proposals.

\noindentbold{3D vision-language datasets}
Several datasets align 3D scenes with textual annotations to facilitate language-driven 3D understanding. 
ScanRefer~\cite{chen2020scanrefer}, ReferIt3D~\cite{achlioptas2020referit_3d} and EmbodiedScan~\cite{wangEmbodiedScanHolisticMultiModal2023} provide fine-grained object-level localization through detailed referential phrases, while ScanQA~\cite{azuma2022scanqa} targets spatially grounded question-answering. 
In contrast, SceneVerse~\cite{jiaSceneVerseScaling3D2024} and MMScan~\cite{lyu2024mmscan} employ large-language models or vision-language models to partially automate annotation.
Despite leveraging advanced models, these datasets depend significantly on costly human annotations derived from closed-vocabulary sources, limiting their support for open-vocabulary and scalability for large-scale 3D segmentation tasks.

% !TEX root = ../main.tex

\section{Causal Discovery and GES}\label{sec:background}  

We first review the causal discovery problem and the necessary details of the greedy
equivalence search (GES) method.

\subsection{Causal Graphical Models }
Causal discovery aims to identify cause-and-effect relationships between random
variables $\{X_1, ..., X_d\}$. We reason about causal relationships using causal
graphical models (CGM). A CGM has two components:
\begin{enumerate}
\item a directed acyclic graph (DAG), $G^* = (V,E)$, where a node $j \in V$ represents
variable $X_j$ and an edge $(j, k) \in E$ denotes a direct causal link from $X_j$ to
$X_k$,
\item conditional distributions $p(X_j  \mid X_{\Pa_j^{G^*}})$, defining the
distribution of $X_j$ given its causal parents $X_{\Pa^{G^*}_j}$.
\end{enumerate}
The joint distribution of the variables $X_1,..., X_d$ writes:
\begin{equation}\label{eq:factor_joint}
p^*(X) = \prod\limits_{j \in V} p^*(X_j \mid X_{\Pa^{G^*}_j}).
\end{equation}
The goal of causal discovery is to recover the graph $G^*$ from the joint distribution
$p^*$ or from samples drawn from $p^*$. 

However, multiple CGMs with different graphs can generate the same $p^*\!.$ Two
important concepts address this difficulty: faithfulness and Markov equivalence
\citep{spirtes2000causation}.

\parhead{Faithfulness.} In \Cref{eq:factor_joint}, $G^*$ induces a factorization of
$p^*$, which, in turn, induces independencies between variables: each $X_j$ is
independent of its non-descendants given its parents $X_{\Pa^{G^*}_j}$
\citep{pearl1988probabilistic}. Reciprocally, we say that a distribution $p^*$ and a
graph $G^*$ are \textit{faithful} if all the independencies in $p^*$ are exactly those
implied by $G^*$ and no more. \looseness=-1

%(Notice it is the lack of edge in $G^*$ that imposes independencies in $p^*$.)
For example, $H=(\{1,2\}, \{1\rightarrow 2\})$ and $q=q(X_1)q(X_2)$ form a
valid CGM. But the independence $X_1 \indep X_2$ present in $q$ is not suggested by $H$.
Rather, $q$ is faithful to $H'=(\{1,2\}, \varnothing)$, which has no superfluous edges
like $1 \rightarrow 2$.

Limiting the search to faithful graphs reduces the possible CGMs that could have
generated $p^*$. But this is not enough.



\parhead{Markov Equivalence.} 
Two distinct graphs can both be faithful to $p^*$ if they induce the same set of
independencies on $p^*$. For example, $A\rightarrow B \rightarrow C$ and $A \leftarrow B
\leftarrow C$ impose the same set of independencies,  $\{A \indep C \mid B\}$. 

Graphs inducing the same independencies are called \textit{Markov equivalent}, they form
\textit{Markov equivalence classes} (MEC).
Since $G^*$ is identifiable only up to Markov equivalence, the task of causal discovery
becomes finding the MEC of $G^*$. 

\begin{remark}
    With more assumptions (e.g. about the form of $p^*$) or special data (e.g.,
    interventions), other causal discovery methods focus on identifying the possible
    $G^*$ beyond Markov equivalence. See \citet{glymour2019review} for an excellent
    review. We focus on Markov equivalence classes.
\end{remark}

\subsection{Score-based Causal Discovery}
In this work, we assume to have $n$ iid samples, denoted $\data = \{(x_1^{i}, ...,
x_d^{i})\}_{i=1}^n$, from a distribution $p^*$ that is faithful to some $G^*$. We aim to
recover the MEC of $G^*$ from $\data$.

Greedy Equivalence Search (GES) searches for the MEC of $G^*$ among all the possible
MECs. It does so by searching for the MEC whose DAGs maximize a specific score function.
It is a particular case of score-based causal discovery.


{Score-based methods} assign a score $S(G; \data)$ to every possible DAG $G$ given the
data $\data$. The score function $S$ is designed to be maximized by the true graph
$G^*$. This turns causal discovery into an optimization problem. 
\begin{equation}
    G^* = \argmax_{G} S(G; \data)
\end{equation}
Some scores have properties that are important for GES.

\begin{definition}
    A score $S$ is \emph{score equivalent} if it assigns the same score to all the
    graphs in the same MEC.
\end{definition}

A score equivalent $S$ enables defining the score of a MEC as the score of any of its
constituent graphs. 

\begin{definition}[Local Consistency, \citep{chickering2002optimal}]
Let $\data$ contain $n$ iid samples from some $p^*$. Let $G$ be any DAG and $G'$ be a
different DAG obtained by adding the edge $i \rightarrow j$ to $G$. A score $S$ is
\emph{locally consistent} if both hold:
\begin{enumerate}
    \item $X_i \not\hspace*{-0.5mm}\indep_{p^*} X_j \mid X_{\Pa_j^G} \Rightarrow S(G';
    \data)>S(G; \data)$,
\item $X_i \indep_{p^*} X_j \mid X_{\Pa_j^G} \Rightarrow S(G'; \data)<S(G; \data)$.
\end{enumerate}
The independence statements are with respect to $p^*$.

\end{definition}

A locally consistent score increases when we add an edge that captures a dependency not
yet represented in the graph. It decreases when we add an edge that does not capture any
new dependency.

\subsection{Greedy Equivalence Search}
\citet{chickering2002optimal} introduced Greedy Equivalence Search (GES). It is a
score-based method that is guaranteed to return the MEC of $G^*$ whenever the score is
locally consistent. The main characteristic of GES is to navigate the space of MECs.

GES begins with the MEC of the empty DAG and iteratively modifies it to improve the
score. At each step, only a few modifications are allowed. These
modifications are of three types: insertions, deletions, and reversals. 
\begin{figure}
    \centering
    \includegraphics[width=\linewidth]{fig/mecv2.pdf}
    \caption{Illustration of insertions from a MEC A to MECs B or C: (i) choose a DAG in
    A, (ii) insert the edge $2\leftarrow3$ to obtain another DAG (iii) consider its MEC.
    Each MEC has all its DAGs on a white plate, and its canonical PDAG on a gray
    plate (see \Cref{sec:implementation} for a definition).}
    \label{fig:mec}
\end{figure}

\parhead{MEC modifications.}An \textit{insertion} on MEC $M$ selects a DAG $G$ in $M$,
adds an edge $x \rightarrow y$ to $G$ to obtain a different DAG $G'$ and replaces $M$
with the MEC of $G'$ (see \Cref{fig:mec}).

A \textit{deletion} on MEC $M$ selects a DAG $G$ in $M$, removes an edge
$x \rightarrow y$ from $G$ to obtain a different DAG $G'$ and replaces $M$ with the MEC
of $G'$.

A \textit{reversal} on MEC $M$ selects a DAG $G$ in $M$, reverses an edge
$x \rightarrow y$ to $x \leftarrow y$ in $G$ to obtain a different DAG $G'$ and replaces
$M$ with the MEC of $G'$.

GES applies these modifications in three separate phases.

\parhead{Phase 1: Insert.} First, GES finds all possible insertions, 
applies the one leading to the largest score increase, and repeats until no insertion
increases the score.

\parhead{Phase 2: Delete.} Then, GES finds all possible deletions, applies the one
leading to the largest score increase, and repeats until no deletion increases the
score.

The MEC obtained at the end of phase 2 is exactly the MEC of $G^*$ if the score is
locally consistent \citep{chickering2002optimal}

\parhead{Phase 3: Reverse.} In theory, phases 1 and 2 are sufficient to recover the
MEC of $G^*$. Yet, \citet{hauser2012characterization} showed that adding a third
phase with reversals can improve the search in practice with finite data (where the score might not be
locally consistent). In this phase, GES finds all possible reversals for the MEC,
applies the one that increases the score most, and repeats until none do.

The pseudocode of GES is given in \Cref{alg:ges-vanilla}.

\begin{algorithm}[t]
    \caption{Greedy Equivalence Search (GES)}\label{alg:ges-vanilla}  
    \DontPrintSemicolon
    
    \KwIn{Data $\data \in \R^{n\times d}$, score function $S$}  
    \KwDefine{$\delta_{\data,M}(O) = S(\text{Apply}(O,M) ; \data) - S(M; \data)$}  
    \KwOut{MEC of $G^*$}  
    $M \leftarrow \{([d], \varnothing)\}$ \tcp*{Empty graph's MEC}  
    $\mathcal{I} \leftarrow$ get all insertions valid for $M$ \;  
    \While{$|\mathcal{I}| > 0$}{  
        $O^* \leftarrow \argmax_{I \in \mathcal{I}}\{\delta_{\data,M}(I)\}$ \tcp*{Get
        best insertion}  
        \lIf{$\delta_{\data,M}(O^*) \leq 0$}{\textbf{break}}  
        $M \leftarrow$ Apply($O^*, M$) \tcp*{Apply best insertion}  
        $\mathcal{I} \leftarrow$ get all insertions valid for $M$ \;  
    }  
    $\mathcal{D} \leftarrow$ get all deletions valid for $M$ \;  
    \While{$|\mathcal{D}| > 0$}{  
        $O^* \leftarrow \argmax_{D \in \mathcal{D}} \{\delta_{\data,M}(D)\}$ \tcp*{Get
        best deletion}  
        \lIf{$\delta_{\data,M}(O^*) \leq 0$}{\textbf{break}}  
        $M \leftarrow$ Apply($O^*, M$) \tcp*{Apply best deletion}  
        $\mathcal{D} \leftarrow$ get all deletions valid for $M$ \;  
    }  
    \tcc{(Optional) 3rd phase like above but with reversals}  
    \Return $M$ \;  
\end{algorithm}

\parhead{Correctness.} GES's correctness relies on two properties:
    (i) the greedy scheme will reach the global maximum of any locally consistent
    score \citep[Lemma 10]{chickering2002optimal}, and 
    (ii) the true graph $G^*$ and its MEC are the unique global maximizers of any
    locally consistent score \citep[Proposition 8]{chickering2002optimal}.

With these two properties, GES is guaranteed to recover the MEC of $G^*$ when the score
is locally consistent.
\label{sec:greedy_search}

\parhead{The BIC score.} A score commonly used by GES is the Bayesian Information Criterion
(BIC) \citep{schwarz1978estimating}. Given a model class
$\mathcal{M}_G$ for each $G$, and our data $\data$ with its $n$ samples, the BIC defines a score:
\begin{equation}
    S(G; \data) = \log p_{\hat\theta}(\data ; G) - \frac{\alpha}{2} \log n \cdot |\hat\theta|,
\end{equation}
where
$\alpha$ is a hyperparameter, $\hat \theta$ is the likelihood maximizer over
$\mathcal{M}_G$, and the number of parameters $|\hat\theta|$ is usually the number of edges in $G$. The BIC is a model selection criterion trading off log-likelihood and model complexity. Models with higher BIC are preferred. The original BIC has $\alpha=1$. 


\parhead{Gaussian Linear Models. } A common model class used for continuous data with the BIC is the class of Gaussian linear models, where given a graph $G$, each variable is Gaussian with a linear
conditional mean and a specific variance:
\begin{equation}\textstyle
    p_{\theta}(x_j \mid x_{\Pa_j^G}) \sim \mathcal{N}\Bigl(\sum_{k \in \Pa_j^G} 
    \theta_{jk} x_k + \theta_{j0}, \theta_{j(d+1)}^2\Bigr).
\end{equation}
For Gaussian linear models, the BIC is score equivalent. It can also be locally
consistent under some conditions.

\begin{theorem}[Local Consistency of BIC \citep{haughton1988choice,chickering2002optimal}]
    For $\alpha>0$, the BIC for Gaussian linear models is locally consistent once $n$
    is large enough.
    \label{th:local_consistency}
\end{theorem}

% More generally, the local consistency theorem also applies to most model classes of the exponential family (e.g. the multinomial distribution for discrete data).
Hence, for Gaussian linear models, GES is guaranteed to recover the MEC of $G^*$ with
infinite data. 
In practice, however, data is finite and GES can return incorrect MECs. 

\parhead{Example of Failure.} In \Cref{fig:simulation_main}, we report the performance
of GES (orange) on simulated data, along with its variants and the proposed XGES. We
simulate CGMs $(G^*, p^*)$ for $d \in\{25,50\}$ variables, $\rho d \in \{2d,3d,4d\}$
edges ($\rho$ is an edge density parameter) and draw $n=10,000$ samples from $p^*$. The
simulation is detailed in \Cref{sec:experiments:evaluation_setup}. We compare the
methods' results to the true graph $G^*$ using the structural Hamming distance for MECs
(SHD), which counts the number of different edges between graphs of two MECs (see
\Cref{sec:experiments:evaluation_setup}). \Cref{fig:simulation_main} shows that GES can fail (SHD $> 0$),
especially in denser graphs.

In addition, we confirm that including the third phase of reversals in GES improves its
performance (GES-r, in green).


% figure: simulation_main
\begin{figure}[t]
    \centering
    \includegraphics[width=1\linewidth]{fig/simulation_main_shd.pdf}
    \caption{Performance comparison of GES and XGES variants, measured with SHD for
    different edge densities $\rho$. XGES heuristics outperform GES and its variants in
    all scenarios. The dashed lines indicate the number of edges of the true graph. Each boxplot
    is computed over 30 seeds.\looseness=-1}
    \label{fig:simulation_main}
\end{figure}


% !TEX root = ../main.tex

\section{Extremely Greedy Equivalence Search}\label{sec:xges} 


In this section, we first investigate why GES can fail and then
propose simple solutions to mitigate failure.

\subsection{Scenarios of GES failure}
\label{sec:ges_failure}
As reviewed in \Cref{sec:greedy_search}, GES's correctness relies on two conditions: (i)
the ability of its greedy search to find the score's global maximizer, and (ii) the true
graph's MEC being the score's global maximizer (and the only one).

These two properties might not hold if the data cannot render $S$
locally consistent. We investigate GES failure.
Is the issue that GES cannot reach the global maximizer with greedy search? If so,
changing the search heuristic could help bypass local maxima. Or is it that GES
effectively finds the global maximizer, but this maximizer is not the true graph?
In this case, designing a new score might help. To check these hypotheses, we conduct
empirical experiments.
\begin{figure}
    \centering
    \includegraphics[width=\linewidth]{fig/ges_fail.pdf}
    \caption{
    Empirical study of GES failure, on 90 simulated datasets with varying variables $d$ and graph densities $\rho$. (left) Differences in BIC
    between GES and ground-truth are negative. GES does not find the score's
    global maximum. (right) Ratios of GES-edges to true edges exceed 1. GES returns many more edges than the true graph.
}
    \label{fig:ges_fail}
\end{figure}



\parhead{Greedy Optimization Fails.}
We apply GES on the data simulated for \Cref{fig:simulation_main}, this time including
$d \in \{10,50,100\}$ variables. We obtain a MEC $\hat M$ that we compare to $G^*$ with:
\begin{equation}
    \Delta S = \frac{S(\hat M; \data) - S(G^*; \data)}{d}.
\end{equation}
A negative $\Delta S$ indicates that GES failed to identify a global maximizer and that
$G^*$ still scores higher than $\hat M$ (note that it does not imply $G^*$ is the global maximizer). A positive $\Delta S$ demonstrates
that $G^*$ is no longer the global maximizer and that GES legitimately identified a
MEC with high score.

In \Cref{fig:ges_fail} (left), we consistently find $\Delta < 0$ across different
numbers of variables $d$ and edge densities $\rho$. This is evidence that GES fails to
reach the global maximum and that the true graph $G^*$ still has a better score than
$\hat M$.

We measure next the ratio $\zeta = |\hat M| /|G^*|$ between the number of edges found by
GES and the number of true edges. \Cref{fig:ges_fail} (right) shows that GES
over-inserts edges i.e. $\zeta > 1$, especially with dense graphs. The idea behind GES
is to over-insert edges in the first phase and then delete them in the second phase.
However, we hypothesize that over-inserting may lead GES into local maxima before the
second phase can correct it.

Following these two observations, we focus on ensuring that GES reaches the global
maximizer. To do so, we design novel search heuristics aimed at preventing
over-insertion.



\subsection{The Heuristic XGES-0}\label{sec:xges-0}

GES considers insertions, deletions, and optionally reversals in three separate phases.
Rather, we consider all operations simultaneously, where deletions, insertions, and
reversals can interleave in any order. And when both insertions and deletions can 
increase the score, we prioritize deletions. 

\parhead{Heuristic XGES-0.} At each step, identify all the valid insertions,
deletions, and reversals. If some deletes would increase the score, apply the best one.
Otherwise, if some reversals would increase the score, apply the best one. Otherwise,
apply the best insert. Repeat until no deletions, reversals, or insertions can increase
the score. 



\begin{algorithm}[t]
    \DontPrintSemicolon  
     % align line on the right
    \KwIn{Data $\data \in \mathbb{R}^{n\times d}$, score $S$.}
    \KwDefine{$\delta_{\data,M}(O) = S(\text{Apply}(O,M) ; \data) - S(M; \data)$}  
    \KwOut{MEC of $G^*$}  
    $M \leftarrow \{([d], \varnothing)\}$\;  
    $\mathcal{I},\mathcal{D},\mathcal{R} \leftarrow$ all insertions, deletions, reversals
    valid for $M$\;  
    \While{$|\mathcal{I}| + |\mathcal{D}| + |\mathcal{R}| > 0$}{  
        \uIf{$|\mathcal{D}| > 0$ \textnormal{\textbf{and}} $\max_{D \in \mathcal{D}}
        \{\delta_{\data,M}(D)\} \geq 0 $}{  
            $O^* \leftarrow \argmax_{D \in \mathcal{D}} \{\delta_{\data,M}(D)\}$ \;  
        }  
        \uElseIf{$|\mathcal{R}| > 0$ \textnormal{\textbf{and}} $\max_{R \in \mathcal{R}}
        \{\delta_{\data,M}(R)\} > 0 $}{  
            $O^* \leftarrow \argmax_{R \in \mathcal{R}} \{\delta_{\data,M}(R)\}$\;  
        }  
        \uElseIf{$|\mathcal{I}| > 0$ \textnormal{\textbf{and}} $\max_{I \in \mathcal{I}}
        \{\delta_{\data,M}(I)\} > 0 $}{  
            $O^* \leftarrow \argmax_{I \in \mathcal{I}} \{\delta_{\data,M}(I)\}$\;  
        }  
        \uElse{  
            \textbf{break} \tcp*{No more operations available}  
        }  
        $M \leftarrow$ Apply($O^*, M$) \;
        $\mathcal{I},\!\mathcal{D},\!\mathcal{R} \leftarrow\!$ all insertions, deletions, reversals
        valid for \!$M$\;  
    }  
    \Return $M$\;  
    \caption{XGES-0.   
    }  
    \label{alg:xges-0}  
\end{algorithm}

We call this heuristic XGES-0 for eXtremely Greedy Equivalence Search and we detail it in
\Cref{alg:xges-0}. XGES-0 retains the same theoretical correctness as GES.

\begin{restatable}[]{theorem}{thXgesZero}
    For any locally consistent score $S$, the MEC $\hat M$ returned by XGES-0 contains
    the true graph $G^*$.
    \label{thm:xges0}
\end{restatable}


The proof leverages the same theorems as those used to prove GES's correctness. It is
provided in \Cref{appendix:sec:theoretical_guarantees_xges0}.

In \Cref{fig:simulation_main}, we find empirically that XGES-0 (red) obtains MECs
with better scores and closer to $G^*$ than GES. Early deletions effectively reduce
encounters with local maxima.

\begin{remark}
    Alongside GES, \citet{chickering2002optimal} proposed a variant called OPS that
    also considered insertions and deletions simultaneously. But OPS did not prioritize
    deletions over insertions, resulting in no improvements to GES in practice. 
    We provide more details in \Cref{appendix:sec:empirical:ops}.
    %Mathematically, deleting an edge can only increase the BIC score by at most $\frac{\alpha}{2} \log n$ which is usually smaller than the increase from inserting an edge. Hence even if OPS considered deletions, it would still mostly insert edges first and encounter the same local maxima as GES.
\end{remark}

\subsection{The Heuristic XGES}

Building upon XGES-0, we introduce the heuristic XGES. XGES complements XGES-0. It
repeatedly uses XGES-0, each time deleting an edge that causes a local maximum.

\parhead{Heuristic XGES.} XGES begins by applying XGES-0 until no operations can
increase the score. Then, it enumerates all valid deletions, all of which will
decrease the score. For each deletion, XGES copies the MEC, applies the deletion, and resumes XGES-0 on the
copy, but without ever reinserting the edge removed by the deletion.
If the final score is worse than the original MEC, the copy is discarded, and the search
continues with the next deletion. If the final score is better, the copy becomes the new
MEC. XGES then restarts with the new MEC and all its new deletions. XGES stops once all
deletions of a MEC have been tried. We provide the pseudocode of XGES in \Cref{alg:xges}
in \Cref{appendix:sec:theoretical_guarantees_xges}.


\parhead{Intuition.} The XGES heuristic aims to remove incorrect edges that were
inserted early in the search and might be causing local maxima, preventing their
deletion. By forcefully deleting these edges, XGES can get around the local maximum and
discover better graphs.


\begin{restatable}[]{theorem}{thXges}
    For any score $S$, XGES returns a MEC $\hat M$ with a higher or equal score than
    XGES-0. If $S$ is locally consistent, then $\hat M$ contains the true graph $G^*$. 
    \label{thm:xges}
\end{restatable}

The proof follows from the design of XGES and by \Cref{thm:xges0}. 

\Cref{fig:simulation_main} illustrate the performance of XGES (purple), showing that it
significantly improves over all other GES variants.  XGES enables
non-trivial causal discovery in denser graphs ($\rho\geq2$) with many variables
($d\geq50$). However, XGES is computationally more expensive than
XGES-0. To alleviate this, we develop an efficient implementation for it.

% !TEX root = ../main.tex

\section{Efficient Algorithm}\label{sec:implementation} GES, if naively implemented, is
slow. In this section, we develop new ways of implementing its details to significantly speed it up. As we will see in the empirical studies, these details are crucial to scaling up XGES to large, dense graphs. We now review how GES manipulates MECs in
practice and then show how to make it more efficient.

\subsection{Manipulating MECs with CPDAGs}\label{sec:manipulating_mecs}
MECs are sets of DAGs whose size can grow exponentially with the number of nodes $d$
\citep{he2015counting}. To manipulate them practically, GES builds on the
following theorem.

\begin{theorem}[\citep{verma1991equivalence}]
    Two DAGs are Markov equivalent if and only if they have the same skeletons and the
    same v-structures. 
    \label{thm:mec}
 \end{theorem} 

 The \textit{skeleton} of a graph is the undirected graph obtained
 by removing the direction of all edges; a \textit{v-structure} is a triple of nodes
such that $x \rightarrow y \leftarrow z$ with no edge between $x$ and
 $z$.\looseness=-1

\Cref{thm:mec} shows that all the graphs of a MEC share the same skeleton and differ
only on edges that can be reversed without changing the set of v-structures. So within a
MEC, some edges are consistently oriented in one direction, while others may have
different orientations between graphs. They are respectively called \textit{compelled}
and \textit{reversible} edges.


\parhead{CPDAGs.}
Each MEC can be represented by a \textit{partially directed acyclic
graph} (PDAG). A PDAG is a graph with both directed and undirected edges and no cycles of
directed edges. The \textit{canonical PDAG of a MEC} contains all the compelled edges as directed
edges and all the reversible edges as undirected edges (see \Cref{fig:mec}). A PDAG that is the canonical
representation of a MEC is called a \textit{completed PDAG} (CPDAG).
A PDAG $P$ that is not a CPDAG but has the same skeleton and v-structures as another CPDAG $P'$
can be transformed into $P'$ with a method called 
\textit{completing} the PDAG \citep{meek1995causal,chickering2002learning}.

We will use the following terminology when discussing a PDAG. For node $x$: its \textit{neighbors}
$\Ne(x)$ are its neighbors from undirected edges, its \textit{children} $\Ch(x)$ are its
children from directed edges, its \textit{parents} $\Pa(x)$ are its parents from
directed edges, and its \textit{adjacent} nodes $\Ad(x)$ are any of all three. 
A \textit{semi-directed path} from $x$ to $y$ is a path from $x$ to $y$ with edges that
are either undirected or directed toward the direction of $y$. A \textit{clique} is a
set of all adjacent nodes.


\parhead{Operators on CPDAGs.}
GES associates each operation on a MEC $M$  with an operator acting on its CPDAG $P$,
such that an operation changing $M$ into $M'$ is associated with an operator changing
$P$ into $P'$, the CPDAG of $M'$. 

For insertions, the operators used by GES are of the form \text{Insert}$(x,y,T)$ where
$x,y \in V$ and $T \subset V$. The action of Insert$(x,y,T)$ on $P$ is to insert the
edge $x \rightarrow y$, orient any undirected edges $t - y$ as $t \rightarrow y$ for $t
\in T$ and finally complete the resulting PDAG into a CPDAG. 

Given a MEC $M$ and its CPDAG $P$, \citet{chickering2002optimal} shows that there is a
bijection between (a) the set of possible insertions on $M$ and (b) the set of operators
Insert$(x,y,T)$ satisfying the following validity conditions relative to $P$:
\begin{align}
    \bm{I1.}\quad&x \not\in \mathrm{Ad}(y). \label{eq:valid1} \\
    \bm{I2.}\quad&T \subset \mathrm{Ne}(y) \setminus \mathrm{Ad}(x). \label{eq:valid2} \\
    \bm{I3.}\quad&(\mathrm{Ne}(y) \cap \mathrm{Ad}(x)) \cup T \text{ is a clique.} \label{eq:valid3} \\
    \bm{I4.}\quad&\text{All semi-directed paths from } y \text{ to } x \text{ have a node in }\nonumber \\
    &(\mathrm{Ne}(y) \cap \mathrm{Ad}(x)) \cup T. \label{eq:valid4} 
\end{align}
Insert operators satisfying these conditions, with $\Ad, \Ne,$ $\Pa,$ clique and paths
computed in $P$, are called \textit{valid} for $P$. To navigate from one MEC to another
with an insertion, GES applies the corresponding valid Insert operator from one CPDAG to
another.

\parhead{Score of Operators.}
The increase in score after an Insert operation can be efficiently computed when the
score is BIC. Indeed, the BIC for a graph $G$ equivalently rewrites as: 
\begin{equation}
    \textstyle
    S(G; \data) = \sum_{j=1}^d s(j, \Pa_j^G ; \data), \label{eq:bic-decomp}
\end{equation}
where $s(j, \Pa_j^G ; \data)$ is called the \textit{local score} of $j$ and equals:
\begin{equation}
    \sum\limits_{i=1}^n \log p_{\hat
\theta}(x^i_j | x^i_{\Pa_j^G}) - \frac{\alpha}{2}\log n \cdot |\Pa_j^G|.
\end{equation}
A score decomposing as \Cref{eq:bic-decomp} is called \textit{decomposable}.


With a decomposable score, the increase in score for an operator Insert($x,y,T$) applied
to $P$ is:
\begin{multline}
    \delta = s(y, (\Ne(y) \cap \Ad(x)) \cup T \cup \Pa(y) \cup \{x\}) \\
    - s(y, (\Ne(y) \cap \Ad(x)) \cup \Pa(y )),
    \label{eq:insert-score}
\end{multline}
where each term $\Ad, \Ne, \Pa$ is computed relative to $P$. For convenience, we say
that $\delta$ is the \textit{score} of the operator.

Similar derivations are made for Delete and Reversal in \citet{chickering2002optimal}
 and \citet{hauser2012characterization} (reversal is called turning). We review them in
 \Cref{appendix:sec:ges_parametrization}.

\parhead{GES with CPDAGs.}
In sum, GES implements \Cref{alg:ges-vanilla} using CPDAGs and operators. It begins with
the empty CPDAG, identifies all the Insert (or Delete, Reversal) that are valid for the
current CPDAG, computes their scores, applies the best one if it has a positive score,
and repeats.

XGES could proceed similarly. However, whether for GES or XGES, constructing the list of
valid operators and scoring them at each step is computationally expensive. We now turn
to new ways to more efficiently implement these operations.

\subsection{Efficient Algorithmic Formulation}
\label{sec:efficient_algorithmic_formulation}
When applying an operator on $P$ to form $P'$, the validity conditions of the other
operators (\Cref{eq:valid1,eq:valid2,eq:valid3,eq:valid4}) can become valid or invalid.
Similarly, the score of the other operators in \Cref{eq:insert-score} can
change. Yet, as noticed in \citet{ramsey2017million}, only a few edges changed from $P$ to $P'$. As a result, most
other operators that were computed for $P$ but not applied remain valid operators for
$P'$. Similarly, the scores of most operators remain identical.

Each step of XGES involves the following sub-steps:
\begin{enumerate}
    \item Start with a CPDAG $P$ and a list of candidate operators $\mathcal{C}$,
    where $\mathcal{C}$ is guaranteed to include all the valid operators for $P$, and
    their scores.
    \item Choose the best operator $O^*$ from
    $\mathcal{C}$ using XGES's heuristic (deletion before reversal, before insertion).
    \item Verify that $O^*$ is valid for $P$, otherwise re-run the heuristic on
    $\mathcal{C} \setminus \{O^*\}$ until a valid operator is found.
    \item Apply $O^*$ to $P$ to form $P'$ and add to $\mathcal{C}$ all the operators that
    became valid for $P'$, with their scores. Return to 1, with $P \leftarrow P'$, as we
    have just guaranteed that $\mathcal{C}$ includes all the valid operators for $P$.
\end{enumerate}

The operators that became invalid for $P'$ are not removed from $\mathcal{C}$. It is more efficient to leave them in the list and only check the validity of an operator in step 3 just before applying it (and discarding it if invalid). Indeed, if we recheck the validity of all operators after each operation, a single operator will be rechecked at each step until it is applied, instead of being checked only once before being applied.

No steps were included to recompute the scores of any operators in
$\mathcal{C}$. We explain how we can avoid it next.

\subsubsection{Updating the Score of Operators.}
\label{sec:updating_score}
To avoid recomputing the scores of operators at each step, we
change the parametrization of the
operators to make their scores independent of the CPDAG they are applied to. 

We parametrize each Insert by an additional set $E \subset V$ and an extra validity
condition that completes \Cref{eq:valid1,eq:valid2,eq:valid3,eq:valid4}: 
\begin{flalign}
    ~~~\bm{I5.}\quad & E = (\Ne(y) \cap \Ad(x)) \cup T \cup \Pa(y).&&
\end{flalign} 
The score of Insert($x,y,T,E$) from \Cref{eq:insert-score} becomes $s(y, E \cup \{x\}) -
s(y, E )$, which only depends on the Insert parameters. We reparametrize Delete and
Reversal operators similarly in \Cref{appendix:sec:xges_parametrization}.

With \citet{chickering2002optimal}'s parametrization, the score of an Insert would
change if $(\Ne(y) \cap \Ad(x)) \cup T \cup \Pa(y)$ changes. Now, with $E$ as a fixed
parameter of the operator, it is the status of condition \textbf{I5} that would change.
We turned a change in score into a change in validity.

We now turn to efficiently update the valid operators.

\begin{table}[t]
    \centering
    \begin{tabular}{ L{1.7cm}C{3.4cm}C{1.8cm} }
    \toprule
    Pre-update & $a \quad b$ & $a
    \rightarrow b$ \\
    Post-update& $a - b$ &  $a- b$  \\
    \midrule
    Necessary conditions
    &\begin{itemize}[leftmargin=*, itemsep=-1pt]
        \vspace*{-4mm}
        \item[] $y \in \{a,b\}$ 
        \item[or] $y \in \Ne(a)\cap \Ne(b)$
        \item[or] $(x = a) \wedge (y \in \Ne(b))$
        \item[or] $(x = b) \wedge (y \in \Ne(a))$
    \end{itemize}
    & \begin{itemize}[leftmargin=*]
        \item[] $y \in \{a,b\}$
    \end{itemize}
    \\[-3mm]
    \bottomrule
    \end{tabular}
    \caption{Necessary conditions for an Insert($x,y,T,E$) to become valid after the
    $(a,b)$ update. Excerpt of \Cref{tab:operator_updates} from
    \Cref{appendix:sec:efficient_algorithmic_formulation} with only two types of
    updates. }
    \label{tab:operator_updates_example}
\end{table}

\subsubsection{Updating the Validity of Operators.}
After updating a CPDAG $P$ into $P'$, our goal is to efficiently add to $\mathcal{C}$
the operators that became valid for $P'$.

To do so, we decompose the update from $P$ to $P'$ into a succession of single edge
updates $P_1, \hdots P_k$, with $P_1=P$, $P_k = P'$ and where $P_i$ and $P_{i+1}$ only
differ on the orientation or presence of a single edge, e.g. $a \rightarrow b$ vs $a -
b$. We then have the following theorem.

\begin{theorem}
    Write $P_1, \hdots P_k $ a sequence of single edge updates that transforms $P$ into
    $P'$. Take an operator $O$ that is invalid for $P$ and becomes valid for $P'$ and
    write $\{c_1, \ldots, c_m \}$ its validity conditions, e.g. $\bm{I1}$ to $\bm{I5}$
    for an Insert. Then there exists $i^* \in \{1,k-1\}$ and one validity condition
    $c_{j^*}$ such that $c_{j^*}$ is false for $P_{i^*}$, true for $P_{i^*+1}$, and all
    other conditions $c_{j} \neq c_{j^*}$ are true for $P_{i^*+1}$.
    \label{thm:update_validity}
\end{theorem}
\begin{proof}
    All $c_j$ are true for $P'$ i.e. $P_k$. So let us step back from $P'$ to $P$ until
    one of the conditions $c_{j^*}$ becomes false for some $P_{i^*}$. Such an $i^{*}$
    must exist since some condition is false for $P$ i.e. $P_1$. $P_{i^*}$ and $c_{j^*}$
    satisfy the theorem.
\end{proof}

With \Cref{thm:update_validity}, we can efficiently update $\mathcal{C}$ if we can
identify which operators are susceptible to having one of their conditions
become true after single-edge updates.

In \Cref{appendix:sec:efficient_algorithmic_formulation} we study the necessary
conditions on the parameters of an operator to have one of its validity conditions
become true after a single-edge update. We report the necessary conditions for all
validity conditions of all operators against all types of edge updates in
\Cref{tab:operator_updates} in \Cref{appendix:sec:efficient_algorithmic_formulation}. We
provide an excerpt in \Cref{tab:operator_updates_example} with only two types of edge
updates, for the Insert operator only, and where we grouped the necessary conditions for
each validity condition into a single set of necessary conditions (with or).

For example, if edge $a \rightarrow b$ is changed into $a - b$,
\Cref{tab:operator_updates_example} shows that the only Insert($x,y,T,E$) that 
can become
valid are those with $y \in \{a,b\}$. 
If the edge $a - b$ is changed into $a \rightarrow
b$, then the necessary conditions for an Insert operator to become valid are more
involved but still efficient.\looseness=-1

In sum, we can efficiently update $\mathcal{C}$ after each CPDAG update using
\Cref{tab:operator_updates} in \Cref{appendix:sec:efficient_algorithmic_formulation}.

\subsubsection{XGES Implementation.}

We implement the efficient algorithmic formulation of XGES-0 and XGES  at \href{https://github.com/ANazaret/XGES}{https://github.com/ANazaret/XGES}
 We provide
code in C++ and Python.


% !TEX root = ../main.tex

\section{Empirical Studies}\label{sec:experiments} We compare the XGES heuristics to
different variants of GES. We find that
XGES recovers causal graphs with significantly better accuracy and up to 10 times faster.


\subsection{Evaluation Setup}
\label{sec:experiments:evaluation_setup}
\parhead{Data Simulation.}
We simulate CGMs and data for different numbers of variables $d$, edge density $\rho$
(average number of parents) and number of samples $n$. 
We first draw a random DAG $G^*$
from an Erdos-Renyi distribution. We then obtain $p^*$ by choosing each conditional
distribution $p^*(x_i \mid x_{\Pa^i_{G^*}})$ as a Gaussian $x_i \sim
\mathcal{N}(W_i^\top x_{\Pa^i_{G^*}}, \varepsilon_i)$ where $W_i, \varepsilon_i$ are
random selected. To ensure faithfulness, we sample $W_i$ away from $0$. More details are
in \Cref{appendix:sec:simulated_data}.

\parhead{Baseline Algorithms.}
We compare our algorithms against GES without reversals (GES), and with reversals
(GES-r, a.k.a GIES), using the C++ implementation in the R package \texttt{pcalg}
\citep{kalisch2012causal}. We also include fast-GES (fGES) from the Java software Tetrad
\citep{ramsey2017million}.
An additional baseline, OPS, is provided in \Cref{appendix:sec:empirical:ops}.

\parhead{Evaluation Metrics}
We evaluate the algorithms with the structural Hamming distance on MECs (SHD) between
the method's results $\hat M$ and the ground-truth MEC $M^*$ \citep{peters2014causal}. The SHD is the number
of different edges between the CPDAGs of $\hat M$ and $M^*$. 
We also consider causal discovery as a binary classification task, where $\hat M$ predicts the presence of edges in $M^*$. We report the F1 score, precision, and recall for this task. Error bars are computed over multiple random datasets (seeds) and reported as bootstrapped 95\% confidence intervals \citep{waskom2021seaborn}.


\begin{figure}[t]
    \centering
    \includegraphics[width=\linewidth]{fig/simulation_correlation_speed.pdf}
    \caption{
    (left) The BIC scores of the graphs returned by each method are strongly correlated with the SHD to ground truth (shown for $d=50$, $\rho=3$, 30 seeds). XGES finds the highest scores and lowest SHDs. (right) Runtime of GES and XGES for a wide
    range of $d$. XGES-0 is up to 30 times faster than GES, and XGES up to 10 times
    faster. fGES may have overhead due to Java while other methods are in C++.}
    \label{fig:simulation_correlation_speed}
\end{figure}

\subsection{Results}
\parhead{General Performance.} % when we vary d, rho and fix n=10000, alpha=2
In \Cref{fig:simulation_main}, we find that XGES-0 and XGES outperform all baselines.
The improvement is more significant for larger density $\rho$ and larger $d$. The
conclusions are identical with precision and recall in \Cref{fig:simulation_main_precision_recall} of \Cref{appendix:sec:empirical:f1_score},
which are both improved by XGES. We emphasize that even though XGES favors deleting
edges, the proportion of true edges recovered by XGES is higher than GES (the recall).
We also report the F1 metric in \Cref{appendix:sec:empirical:f1_score}. We note that the
performance of fGES is slightly worse than GES. We explain in
\Cref{appendix:subsec:fast_ges} that one of the optimizations of fGES removes
some valid insertions.

\parhead{Choice of Metrics.}
\Cref{fig:simulation_correlation_speed} (left) shows that the BIC scores of the graphs returned by each method are strongly correlated with the SHD to ground truth. This is a comforting observation: maximizing the BIC score on finite data is indeed a good proxy for minimizing SHD to ground-truth.


\parhead{Impact of the Edge Density $\rho$.}
In \Cref{fig:simulation_d_and_rho}, we see that when $\rho=1$ (a very sparse graph where
nodes have $\rho=1$ parent on average), then all methods perform similarly well. The
advantage of XGES over GES is visible as soon as $\rho=2$ and widens as $\rho$ increases, see also
\Cref{fig:simulation_main}.

\parhead{Robustness to the Sample Size $n$.}
In \Cref{fig:simulation_n_and_alpha}~(left), we vary the number of samples $n$ and fix
$d=50$ and $\rho=3$. XGES's performance improves with $n$, coherent with
\Cref{th:local_consistency}. In contrast, GES and its variants are hurt when $n$ increases
beyond $10^4$. But this is not incoherent with GES's correctness in the limit of
infinite data. Instead, this reveals that the finite sample behavior of GES is nontrivial and that GES may require very large $n$ -- beyond what is practical -- to perform well.\looseness=-1

In \Cref{fig:simulation_double_descent}, we study sample sizes up to $n=10^8$ on a small graph with $d=15$ and $\rho=2$. We find again that GES worsens around $10^4$ samples, but this time, it improves again after $10^5$ samples, thereby exhibiting a double descent behavior. We discuss it in more detail in \Cref{appendix:sec:empirical:double_descent}.


% figure: simulation_n and simulation_alpha
\begin{figure}
    \centering
    \includegraphics[width=\linewidth]{fig/simulation_n_and_alpha.pdf}
    \caption{Performance of GES and XGES when varying (left) the number of samples $n$, and
    (right) the regularization strength $\alpha$. Increasing $n$ improves XGES while it hurts GES and its
    variants. Increasing
    $\alpha$ initially improves GES but eventually hurts all methods. The dashed lines indicate the number of edges of the true graph. Error bars over 30 seeds.\looseness=-1}
    \label{fig:simulation_n_and_alpha}
\end{figure}

\parhead{Robustness to the Regularization Strength $\alpha$.}
In \Cref{fig:simulation_n_and_alpha} (right), we vary
$\alpha$ and fix $d=50$, $\rho=3$ and $n=10000$. We find that increasing $\alpha$ helps
GES from $\alpha=1$ to $\alpha=10$ but then hurts it. No value of $\alpha$ enables GES to
catch up to XGES. Echoing \Cref{sec:ges_failure}, we conclude that the solution to GES's
over-inserting is not to change the score function, but to change the search strategy,
as XGES does.


\parhead{Robustness to the Data Simulation.}
We vary the procedure to sample the weights $W_i$ in two ways: changing the scale
and changing the shape of their distribution. We report the results in
\Cref{appendix:sec:empirical:impact_of_simulation} with \Cref{fig:simulation_epsilon,fig:simulation_main_negative}.
We find similar conclusions as in \Cref{fig:simulation_main}.


\parhead{Implementation Speed.}
We measure the runtime of the different methods for a wide range of $d$ and 
$\rho \in \{2,4\}$ in \Cref{fig:simulation_correlation_speed} (right). We find that XGES-0 is up to 30 times
faster than GES, and XGES up to 10 times faster. While fGES's slower runtime may be 
attributed to Java overhead, the other methods are implemented in C++.
Higher densities slow down all methods, with a stronger impact on GES, which is coherent
with GES over-inserting in denser graphs. 

We find the same conclusions by reporting the number of calls to the scoring function as another measure
of efficiency in \Cref{appendix:sec:empirical:number_of_calls}. Interestingly, even though XGES repeatedly applies XGES-0, it only makes around one order of magnitude more BIC score evaluations than XGES-0.

\section*{Conclusion and Future Work}
We introduced XGES, an algorithm that significantly improves on GES. With XGES, we can learn larger and denser graphs from data. XGES offers several avenues for future work. One direction is to study its finite sample guarantees. A second is to study its applicability to more complex model classes beyond linear models. Another is to relax its assumptions: e.g. unfaithful graphs, or scores that are not asymptotically locally consistent \citep{schultheiss2023pitfalls}. Finally, its efficient implementation could be used to analyze large real-world datasets.\looseness=-1



\begin{acknowledgements}
We thank the anonymous reviewers for their helpful comments.
A.N. was supported by funding from the Eric and Wendy Schmidt Center at the Broad Institute of MIT and Harvard, and the Africk Family Fund. 
D.B. was funded by NSF IIS-2127869, NSF DMS-2311108, NSF/DoD PHY-2229929, ONR N00014-17-1-2131, ONR N00014-15-1-2209, the Simons Foundation, and Open Philanthropy.
\end{acknowledgements}

% References
\bibliography{main}

\newpage

\onecolumn

\title{Extremely Greedy Equivalence Search\\(Supplementary Material)}
\maketitle
\appendix

\subsection{Correctness of the encoding rules}
\label{app:correctness}

\ThemSoundAndComplete*
\begin{proof} By a case analysis of the encoding rules and semantic definitions: 

\begin{enumerate}[itemsep=1.5em,leftmargin=!]
\vspace{1em}
\item Atomic Proposition: 
{
\small 
\begin{align*}
\frac{
\begin{matrix}
\widetilde{\drule} = [\nmNEW(\interval) \hornarrow \nm\_{\m{TS}}(\interval).]
\end{matrix}
}{\encoding {\nm}{\nmNEW}{\widetilde{\drule}}}\ [\trans\text{-}\m{AP}]
\end{align*}
\vspace{-1mm}
\begin{align*}
(\history, \timepoint) &\models 
\nm &\m{iff}&~ 
\m{\exists\,\interval}.~ 
\llbracket \nm\_{\m{TS}}(\interval) \text{$\rrbracket_{\history}$}{=}\m{true}
~\m{and}~
\timepoint\,{\in}\,\interval
\\[0.1em]
%, \Subj, \Obj
\end{align*}
\vspace{-8mm}
}

In   
$[\trans\text{-}\m{AP}]$, with $\nm$, 
$\forall \interval.~\llbracket \nmNEW(I) \rrbracket_{\Prolog} {=} \m{true}$, 
its premise indicates that 
$\llbracket  \nm_{\m{TS}}(\interval)\rrbracket_{\history} {=} \m{true}$. 
Next, from the semantic definition, we have 
$\forall  \timepoint\,{\in}\,\interval, (\history, \timepoint) {\models} \nm$; thus, the rule is sound. 
\\
From the semantic definition, $\forall   (\history, \timepoint) {\models} \nm$, it indicates that $\exists\interval.~\llbracket  \nm_{\m{TS}}(\interval)\rrbracket_{\history} {=} \m{true} ~\m{and}~
\timepoint\,{\in}\,\interval$.  
Next, from $[\trans\text{-}\m{AP}]$, we obtain 
$\llbracket \nmNEW(I) \rrbracket_{\Prolog} {=} \m{true}$; thus, the rule is complete. 



\item  Finally:
{
\small 
\begin{align*}
\frac{
\begin{matrix}
\widetilde{\drule} {=} 
[\nmNEW([\interval^\prime_\m{start}\text{-}\interval_{\m{end}}, \interval^\prime_\m{end}\text{-}\interval_{\m{start}}]) \hornarrow \nm(\interval^\prime).]
\end{matrix}
}{\encoding {\mathcal{F}_\interval\,\mtl}{\nmNEW}{\widetilde{\drule} }}\ [\trans\text{-}\m{Finally}]
\end{align*}
\vspace{-1mm}
\begin{align*}
(\history, \timepoint) &\models \mathcal{F}_\interval \,\mtl & 
\m{iff}&~ 
\m{\exists\,\distance}.~\distance\,{\in}\,I  ~ \m{and}
~ (\history, \timepoint\plus\distance)\models\mtl
\\[0.1em]
\end{align*}
\vspace{-8mm}
}



In $[\trans\text{-}\m{Finally}]$  with 
$\mathcal{F}_{[\Istart, \Iend]}\,\mtl$, \\
$\forall \interval^{\prime\prime}.~\llbracket \nmNEW(\interval^{\prime\prime}) \rrbracket_{\Prolog} {=} \m{true}$, 
its premise indicates that \\
$\llbracket  \nm([\interval^{\prime\prime}_\m{start}\plus\Iend, \interval^{\prime\prime}_\m{end}\plus\Istart])\rrbracket_{\history} {=} \m{true}$, 
which means that $\forall \timepoint \,{\in}\,\interval^{\prime\prime}$, there exists $\distance\,{\in}\,[\Istart, \Iend]$ such that $(\history, \timepoint\plus\distance) {\models} \mtl$
\\
Next, from the semantic definition, we have \\
$\forall  \timepoint\,{\in}\,\interval^{\prime\prime}, 
(\history, \timepoint) {\models} \mathcal{F}_{[\Istart, \Iend]}\,\mtl$; thus the rule is sound. \\
From the semantic definition, $\forall   
(\history, \timepoint) {\models} \mathcal{F}_{[\Istart, \Iend]}\,\mtl$; it indicates that 
$\exists \distance\,{\in}\,[\Istart, \Iend] ~\m{and}~ (\history, \timepoint\plus\distance)\models\mtl$, 
which means that 
$\exists \interval^\prime.~ \timepoint\plus\distance\,{\in}\, \interval^\prime ~\m{and} ~\llbracket \nm(\interval^\prime) \rrbracket_{\Prolog} {=} \m{true}$
\\
Next, from $[\trans\text{-}\m{Finally}]$, 
we obtain \\
$\llbracket \nmNEW([\interval^\prime_\m{start}\text{-}\Iend, \interval^\prime_\m{end}\text{-}\Istart]) \rrbracket_{\Prolog} {=} \m{true}$,  and thus \\
$\timepoint\,{\in}\,[\interval^\prime_\m{start}\text{-}\Iend, \interval^\prime_\m{end}\text{-}\Istart]$; 
thus, the rule is complete. 

\item  Globally:
{
\small 
\begin{align*}
\frac{
\begin{matrix}
\widetilde{\drule} {=} 
[\nmNEW([\interval^\prime_\m{start}\text{-}\interval_{\m{start}}, \interval^\prime_\m{end}\text{-}\interval_{\m{end}}]) \hornarrow \nm(\interval^\prime).]
\end{matrix}
}{\encoding {\mathcal{G}_\interval\,\mtl}{\nmNEW}{\widetilde{\drule} }}\ [\trans\text{-}\m{Globally}]
\end{align*}
\vspace{-1mm}
\begin{align*}
(\history, \timepoint) &\models \mathcal{G}_\interval\,\mtl & 
\m{iff}&~ 
\m{\forall\,\distance}.~\distance\,{\in}\,I  ~ \m{and}
~ (\history, \timepoint\plus\distance)\models\mtl
\\[0.1em]
\end{align*}
\vspace{-8mm}
}

In $[\trans\text{-}\m{Globally}]$  with 
$\mathcal{G}_{[\Istart, \Iend]}\,\mtl$, \\
$\forall \interval^{\prime\prime}.~\llbracket \nmNEW(\interval^{\prime\prime}) \rrbracket_{\Prolog} {=} \m{true}$, 
its premise indicates that \\
$\llbracket  \nm([\interval^{\prime\prime}_\m{start}\plus\Istart, \interval^{\prime\prime}_\m{end}\plus\Iend])\rrbracket_{\history} {=} \m{true}$, which means that $\forall \timepoint \,{\in}\,\interval^{\prime\prime}$, for all $\distance\,{\in}\,[\Istart, \Iend]$ such that $(\history, \timepoint\plus\distance) {\models} \mtl$. \\
Next, from the semantic definition, we have \\
$\forall  \timepoint\,{\in}\,\interval^{\prime\prime}, 
(\history, \timepoint) {\models} \mathcal{G}_{[\Istart, \Iend]}\,\mtl$; thus the rule is sound. \\
From the semantic definition, $\forall   
(\history, \timepoint) {\models} \mathcal{G}_{[\Istart, \Iend]}\,\mtl$; it indicates that $\forall\distance\,{\in}\,[\Istart, \Iend] ~\m{and}~ (\history, \timepoint\plus\distance)\models\mtl$, 
which means that 
$\exists \interval^\prime.~ \timepoint\plus\distance\,{\in}\, \interval^\prime ~\m{and} ~\llbracket \nm(\interval^\prime) \rrbracket_{\Prolog} {=} \m{true}$
\\
Next, from $[\trans\text{-}\m{Globally}]$, 
we obtain \\ 
$\llbracket \nmNEW([\interval^\prime_\m{start}\text{-}\Istart, \interval^\prime_\m{end}\text{-}\Iend]) \rrbracket_{\Prolog} {=} \m{true}$,  and thus \\  $\timepoint\,{\in}\,[\interval^\prime_\m{start}\text{-}\Istart, \interval^\prime_\m{end}\text{-}\Iend]$; 
thus, the rule is complete. 



\item  Next:
{
\small 
\begin{align*}
\frac{
\begin{matrix}
\widetilde{\drule} {=} 
[\nmNEW([\interval^\prime_\m{start}\text{-}1, \interval^\prime_\m{end}\text{-}1]) \hornarrow \nm(\interval^\prime).]
\end{matrix}
}{\encoding {\mathcal{N}\,\mtl}{\nmNEW}{\widetilde{\drule} }}\ [\trans\text{-}\m{Next}]
\end{align*}
\vspace{-1mm}
\begin{align*}
(\history, \timepoint) &\models \mathcal{N}\,\mtl & 
\m{iff}&~ 
(\history, \timepoint\plus 1)\models\mtl
\\[0.1em]
\end{align*}
\vspace{-8mm}
}


In $[\trans\text{-}\m{Next}]$  with 
$\mathcal{N}\,\mtl$, \\
$\forall \interval^{\prime\prime}.~\llbracket \nmNEW(\interval^{\prime\prime}) \rrbracket_{\Prolog} {=} \m{true}$, 
its premise indicates that \\
$\llbracket  \nm([\interval^{\prime\prime}_\m{start}\plus 1, \interval^{\prime\prime}_\m{end}\plus 1])\rrbracket_{\history} {=} \m{true}$. \\
Next, from the semantic definition, we have \\
$\forall  \timepoint\,{\in}\,\interval^{\prime\prime}, 
(\history, \timepoint) {\models} \mathcal{N}\,\mtl$; thus the rule is sound. \\
From the semantic definition, $\forall   
(\history, \timepoint) {\models} \mathcal{N}\,\mtl$, \\ 
it indicates that $ (\history, \timepoint\plus 1)\models\mtl$, which means that \\ 
$\exists \interval^\prime.~ \timepoint\plus1\,{\in}\, \interval^\prime ~\m{and} ~\llbracket \nm(\interval^\prime) \rrbracket_{\Prolog} {=} \m{true}$.\\
Next, from $[\trans\text{-}\m{Next}]$, 
we obtain  \\
$\llbracket \nmNEW([\interval^\prime_\m{start}\text{-}1, \interval^\prime_\m{end}\text{-}1]) \rrbracket_{\Prolog} {=} \m{true}$, and thus \\  $\timepoint\,{\in}\,[\interval^\prime_\m{start}\text{-}1, \interval^\prime_\m{end}\text{-}1]$; 
thus, the rule is complete. 


\item  Until: 
{
\small 
\begin{align*}
\frac{
\begin{matrix}
\encoding{\mtl_1}{\nm_1}{\widetilde{\drule}_1}
\qquad 
\encoding{\mtl_2}{\nm_2}{\widetilde{\drule}_2}
\\[0.2em]
\widetilde{\drule}_3{=} [\m{helper1}([\interval^\prime_{\m{start}}\plus\interval_{\m{start}}, \interval^\prime_{\m{end}}\plus1]) \hornarrow 
\nm_1(\interval^\prime).]
\\[0.2em]
\widetilde{\drule}_4{=} [\m{helper2}(\interval_1\,{\cap}\,\interval_2) \hornarrow 
\m{helper1}(\interval_1), 
\nm_2(\interval_2).] 
\\[0.2em]
\encoding {\mathcal{F}_\interval\,(\m{helper2})}{\nm_f}{\widetilde{\drule}_5 }
\\[0.2em]
\widetilde{\drule}_6{=} [\nmNEW(\interval_1\cap \interval_2) \hornarrow 
\nm_1(\interval_1), 
\nm_
f(\interval_2). ] 
\end{matrix}
}{\encoding{\mtl_1\,\mathcal{U}_\interval\,\mtl_2}{\nmNEW}{\widetilde{\drule}_1\cup \widetilde{\drule}_2\cup
\widetilde{\drule}_3\cup
\widetilde{\drule}_4\cup
\widetilde{\drule}_5\cup
\widetilde{\drule}_6}}\ [\trans\text{-}\m{Until}]
%\shil{
%~2. ~what's ~the~ definition ~\interval ~in~ \mathcal{F}? }
%\\ \shil{~3. ~what's ~the ~meaning ~of ~;?}\text{\syh{to~construct~list~from~single~rules}}
\end{align*}
%\vspace{-1mm}
\begin{align*}
(\history, \timepoint) &\models \mtl_1 \, \mathcal{U}_\interval \,\mtl_2  & \m{iff}&~  \m{\exists\,\distance}.~ \distance\,{\in}\,\interval  ~ \m{and}~ (\history, \timepoint\plus\distance)\models\mtl_2 ~ \m{and}
\\[0.1em] 
&&& ~ 
\m{\forall}\, 
k~\m{with} ~\timepoint{<}k{<}(\timepoint\plus\distance), 
(\history, k)\models \mtl_1
\\[0.1em]
\end{align*}
\vspace{-8mm}
}



In $[\trans\text{-}\m{Until}]$  with 
$\mtl_1\,\mathcal{U}_{[\Istart, \Iend]}\,\mtl_2$, \\
$\forall \interval^{\prime}.~\llbracket \nmNEW(\interval^{\prime}) \rrbracket_{\Prolog} {=} \m{true}$, 
its premise indicates that \\
$\forall \timepoint\,{\in}\,\interval^{\prime}, 
\m{exists~\distance}.~ \distance\,{\in}\, {[\Istart, \Iend]}  ~ \m{and}~ $\\ 
$
(\history, k)\models \mtl_1 ~\m{forall}~ 
k~\m{with} ~\timepoint{<}k{<}(\timepoint\plus\distance)
~ \m{and}~(\history, \timepoint\plus\distance)\models\mtl_2 $, \\
guaranteed by the helper functions $\m{helper1}$, $\m{helper2}$, and $\nm_f$.  Next, from the semantic definition, we have \\
$(\history, \timepoint) {\models} \mtl_1\,\mathcal{U}_{[\Istart, \Iend]}\,\mtl_2$; thus the rule is sound. \\
From the semantic definition, $\forall   
(\history, \timepoint) {\models} \mtl_1\,\mathcal{U}_{[\Istart, \Iend]}\,\mtl_2$, \\ it indicates that $ \m{exists~\distance}.~ \distance\,{\in}\,{[\Istart, \Iend]}  ~ \m{and}~ (\history, \timepoint\plus\distance)\models\mtl_2 ~ \m{and}~(\history, k)\models \mtl_1 ~\m{forall}~ 
k~\m{with} ~\timepoint{<}k{<}(\timepoint\plus\distance)$. \\ 
Next, from $[\trans\text{-}\m{Until}]$, 
$\m{helper1}$ produces the superset of the possible values of $\timepoint{\plus}\distance$ which satisfy the first constrain, then $\m{helper2}$ produces the exact set of the possible values of $\timepoint{\plus}\distance$ which also satisfy the second constrain. 
Lastly, $\nmNEW$ produces the exact set of the possible values of $\timepoint$; thus, the rule is complete. 

\item $\mtl_1  
\,\mathcal{U}_{[0, 0]} \,  \mtl_2$ $\equiv$ $\mtl_2$: 
{
\small 
\begin{align*}
(\history, \timepoint) &\models \mtl_1 \, \mathcal{U}_\interval \,\mtl_2  & \m{iff}&~  \m{\exists\,\distance}.~ \distance\,{\in}\,\interval  ~ \m{and}~ (\history, \timepoint\plus\distance)\models\mtl_2 ~ \m{and}
\\[0.1em] 
&&& ~ 
\m{\forall}\, 
k~\m{with} ~\timepoint{<}k{<}(\timepoint\plus\distance), 
(\history, k)\models \mtl_1
\\[0.1em]
\end{align*}
\vspace{-8mm}
}

By instantiating the above semantic rule for Until operators with $\interval{=}[0, 0]$, we obtain the following semantic rule: 

{
\small 
\begin{align*}
(\history, \timepoint) &\models \mtl_1 \, \mathcal{U}_{[0, 0]} \,\mtl_2  & \m{iff}&~  \m{exists~\distance}.~ \distance{=}0  ~ \m{and}~ (\history, \timepoint\plus 0)\models\mtl_2 ~ \m{and}
\\[0.1em] 
&&& ~ 
(\history, k)\models \mtl_1 ~\m{forall}~ 
k~\m{with} ~\timepoint{<}k{<}(\timepoint\plus 0)
\end{align*}}

Which is essentially: 

{
\small 
\begin{align*}
(\history, \timepoint) &\models \mtl_1 \, \mathcal{U}_{[0, 0]} \,\mtl_2  & \m{iff}&~   (\history, \timepoint)\models\mtl_2 
\end{align*}}

Thus, the conclusion $\mtl_1  
\,\mathcal{U}_{[0, 0]} \,  \mtl_2$ $\equiv$ $\mtl_2$ is sound and complete. 

\item  Negation:
{
\small 
\begin{align*}
\frac{
\begin{matrix}
\encoding{\mtl}{\nm}{\widetilde{\drule}_1}
\\ 
\widetilde{\drule}{=}[\nmNEW(\interval) \hornarrow
\m{findall}(\interval_1, \nm), \m{compl}(\interval_1, \interval).]
\end{matrix}
}{
\encoding{\neg\mtl}{\nmNEW}{
\widetilde{\drule}_1\,{\cup}\,\widetilde{\drule}}
}\ [\trans\text{-}\m{Neg}]
\end{align*}
\vspace{-1mm}
\begin{align*}
(\history, \timepoint) &\models\neg \mtl & \m{iff}&~
(\history, \timepoint)\not\models\mtl
\\[0.1em]
\end{align*}
\vspace{-8mm}
}

In $[\trans\text{-}\m{Neg}]$  with 
$\neg\,\mtl$, \\
$\forall \interval.~\llbracket \nmNEW(\interval) \rrbracket_{\Prolog} {=} \m{true}$, 
its premise indicates that \\ 
$\forall \interval_1.~ \nm(\interval_1), \m{and} ~ \interval \cap \interval_1 \,{=}\, \emptyset$. 
\\
Next, from the semantic definition, we have \\
$\forall  \timepoint\,{\in}\,\interval, 
(\history, \timepoint) {\models} \neg\,\mtl$; thus the rule is sound. \\
From the semantic definition, $\forall   
(\history, \timepoint) {\models} \neg\,\mtl$, it indicates that $ (\history, \timepoint){\not\models}\mtl$, 
which means that 
$\forall \interval'. ~ \timepoint\,{\not\in}\,\interval'$ and $\llbracket \nm(\interval^\prime) \rrbracket_{\Prolog} {=} \m{true}$. 
\\
Next, from $[\trans\text{-}\m{Neg}]$, 
we obtain  \\
$\exists\interval.~ \timepoint\,{\in}\,\interval, \interval \cap (\m{findall}(\interval_1, \nm)) {=} \emptyset,$ and $ \llbracket \nmNEW(\interval) \rrbracket_{\Prolog} {=} \m{true}$; 
thus, the rule is complete. 

\item Disjunction:
{
\small 
\begin{align*}
\frac{
\begin{matrix}
[\trans\text{-}\m{Disj}]\\
\encoding{\mtl_1}{\nm_1}{\widetilde{\drule}_1}
\qquad 
\encoding{\mtl_2}{\nm_1}{\widetilde{\drule}_2}
\\
\widetilde{\drule}{=}[\nmNEW(\interval_1\,{\cup}\,\interval_2) \hornarrow
\m{findall}(\interval_1, \nm_1), \m{findall}(\interval_2, \nm_2)]
\end{matrix}
}{
\encoding{\mtl_1{\vee}\mtl_2}{\nmNEW}{ \widetilde{\drule}_1\,{\cup}\,\widetilde{\drule}_2\,{\cup}\,\widetilde{\drule}}
}
\end{align*}
%\vspace{-1mm}
\begin{align*}
(\history, \timepoint) &\models\mtl_1 \, {\vee} \,\mtl_2 & \m{iff}&~ (\history, \timepoint)\models\mtl_1 ~\m{or}~ (\history, \timepoint)\models\mtl_2 
\end{align*}
\vspace{2mm}
}

In $[\trans\text{-}\m{Disj}]$  with 
$\mtl_1{\vee}\mtl_2$, \\
$\forall \interval.~\llbracket \nmNEW(\interval) \rrbracket_{\Prolog} {=} \m{true}$, 
its premise indicates that \\ 
$\forall \interval_1. \forall \interval_2.~ \nm_1(\interval_1), \nm_2(\interval_2)~ \m{and} ~  \interval {=} \interval_1 \cup \interval_2$. 
Next, from the semantic definition, we have \\
$\forall  \timepoint\,{\in}\,\interval, 
(\history, \timepoint) {\models} \mtl_1 \vee \mtl_2$; thus the rule is sound. \\
From the semantic definition, $\forall   
(\history, \timepoint) {\models} \mtl_1 \vee \mtl_2$, it indicates that $(\history, \timepoint)\models\mtl_1$ or $(\history, \timepoint)\models\mtl_2$.\\
Next, from $[\trans\text{-}\m{Disj}]$, 
$\exists \interval.~\llbracket \nm_1(\interval) \rrbracket_{\Prolog} {=} \m{true}$ or $\llbracket \nm_2(\interval) \rrbracket_{\Prolog} {=} \m{true}$
we obtain  \\
$\llbracket \nmNEW(\interval) \rrbracket_{\Prolog} {=} \m{true}$,  and $\timepoint\,{\in}\,\interval$; 
thus, the rule is complete. 

\item Conjunction:
{
\small 
\begin{align*}
\frac{
\begin{matrix}
[\trans\text{-}\m{Conj}]\\
\encoding{\mtl_1}{\nm_1}{\widetilde{\drule}_1}
\qquad 
\encoding{\mtl_2}{\nm_1}{\widetilde{\drule}_2}
\\
\widetilde{\drule}{=}[\nmNEW(\interval_1\,{\cap}\,\interval_2) \hornarrow
\m{findall}(\interval_1, \nm_1), \m{findall}(\interval_2, \nm_2)]
\end{matrix}
}{
\encoding{\mtl_1{\wedge}\mtl_2}{\nmNEW}{ \widetilde{\drule}_1\,{\cup}\,\widetilde{\drule}_2\,{\cup}\,\widetilde{\drule}}
}
\end{align*}
%\vspace{-1mm}
\begin{align*}
(\history, \timepoint) &\models\mtl_1 \, {\wedge} \,\mtl_2 & \m{iff}&~ (\history, \timepoint)\models\mtl_1 ~\m{and}~ (\history, \timepoint)\models\mtl_2
\\[0.1em]
\end{align*}
\vspace{-2mm}
}

In $[\trans\text{-}\m{Conj}]$  with 
$\mtl_1{\wedge}\mtl_2$, \\
$\forall \interval.~\llbracket \nmNEW(\interval) \rrbracket_{\Prolog} {=} \m{true}$, 
its premise indicates that \\ 
$\forall \interval_1. \forall \interval_2.~ \nm_1(\interval_1), \nm_2(\interval_2)~ \m{and} ~  \interval {=} \interval_1 \cap \interval_2$. 
Next, from the semantic definition, we have \\
$\forall  \timepoint\,{\in}\,\interval, 
(\history, \timepoint) {\models} \mtl_1 \wedge \mtl_2$; thus the rule is sound. \\
From the semantic definition, $\forall   
(\history, \timepoint) {\models} \mtl_1 \wedge \mtl_2$, it indicates that $(\history, \timepoint)\models\mtl_1$ and $(\history, \timepoint)\models\mtl_2$.\\
Next, from $[\trans\text{-}\m{Conj}]$, 
$\exists \interval.~\llbracket \nm_1(\interval) \rrbracket_{\Prolog} {=} \m{true}$ and $\llbracket \nm_2(\interval) \rrbracket_{\Prolog} {=} \m{true}$
we obtain  \\
$\llbracket \nmNEW(\interval) \rrbracket_{\Prolog} {=} \m{true}$,  and $\timepoint\,{\in}\,\interval$; 
thus, the rule is complete. 
\end{enumerate}
\vspace{3mm}

All the encoding rules are sound and complete. 

\end{proof}

\end{document}


