% !TEX root = ../main.tex

\parhead{Related Work.}
Causal discovery encompasses a wide range of methods \citep{glymour2019review}.
Here, we focus on score-based methods, which posit a proper scoring rule, and proceed
to find the sets of graphs that maximize it. The Greedy Equivalence Search (GES) algorithm maximizes it using a
greedy search strategy \citep{chickering2002optimal}. Extensions and variants of GES include OPS \citep{chickering2002optimal}, GIES
\citep{hauser2012characterization}, GDS-EEV \citep{peters2014identifiability},  and ARGES \citep{nandy2018high}.

Fast-GES (fGES) is a more efficient implementation of GES\citep{ramsey2017million}. But
we find that it does not reproduce exactly GES's search strategy and hurts performance
(see \Cref{sec:experiments}). Selective GES (SGES) guarantees a polynomial
worst-case complexity but has limited speed improvement in practice
\citep{chickering2015selective}.

Other works improve GES by randomly perturbing the search
\citep{alonso2018use,liu2023improving}. However, they are computationally
expensive. 

More recently, \textit{differentiable causal discovery} methods have been proposed to
maximize the score using gradient-based methods
\citep{zheng2018dags,brouillard2020differentiable,nazaret2023stable}. These methods can model causal
relations using neural networks, but unlike GES and XGES (proposed here), their optimization procedures have no
theoretical guarantees of converging to the true graph. 
