\section{Results and Discussion}

In this section, we present the result obtained from the CoM estimation with the U-GRAPH pipeline. We evaluated the performance of our model on the customized data collection setup with unseen weight distributions. In addition, we experiment on real-world objects with known CoM to validate the effectiveness of our proposed framework. 

\subsection{Baseline Methods}%\wenzhen{Will it be better to put this in IV?}
%\vspace{-1mm}
In contrast to our proposed method, we implement the following 3 different baseline methods:

\textbf{Analytical Solution}: The analytical solution assumes a perfect-world scenario with no noise in the F/T measurement and no in-hand slip. We can easily obtain the CoM of any object using the following formula: $\Vec{r}_{CoM} = \frac{\Vec{\tau} \times \Vec{F}}{|\Vec{F}|^2}$,
where $\Vec{F}$ denotes the force reading and $\Vec{\tau}$ denotes the torque readings. Since we only use the $[0,0]$ orientation, no torque should be caused by offsetting the $Z$-axis. However, real-world measurement noise introduces randomness into this calculation. Additionally, the F/T sensor produces inefficiently accurate torque measurement when the sensor is not placed vertically. Therefore, in this paper, we use analytical solutions only in the default pose as the baseline method.

% \yuchen{Explain why this gives random value on z-axis: in raw data corresponding torque should be 0 but in practice some small noise}

\textbf{One Grasp}: The One Grasp method only uses the first part of our proposed pipeline which is the model that takes in one F/T measurement and tries to infer the CoM. Unlike the analytical solution, this method also uses a neural network for estimation. The MLP used is the same one we used for active perception inference. %This method serves as an ablation study of using a second action

The previous two baselines have a fundamental flaw that based on one grasp it is impossible to evaluate the offset on the Z-axis.
% \yuchen{So is the z value from these methods simply random?}
% \samuel{sort of}

\textbf{Random Rotate}: The random rotation method uses two measurements similar to our proposed method. Instead of an informed action, this method uses a random action selected from the continuous action space to perform the measurement again. After getting the new reading, we will use the same BNN and joint distribution methods as our proposed method to estimate the CoM. %This method serves as an ablation study to use an informed action.


% \yuchen{Table 1 axis should be in world frame}
% \yuchen{Shouldn't OOD stand for Out-of-Distribution?}

\subsection{CoM Estimation on Customized Training Objects}


Our first experiment used the same plate and box setup but varied weight configurations. We tested five different weight configurations across both objects: no weight, a single 100-gram weight, two 100-gram weights placed together, two 100-gram weights placed separately, and two separate weights that weigh 300 grams in total. For each configuration, we performed five randomly selected grasps.  The data for this experiment are captured using the same overhead camera and AprilTags setup as training. The results of these experiments are detailed in Figure \ref{result1}, where we present a comprehensive analysis including the mean error and the mean standard deviation for each method applied.

\begin{figure}[hbtp]
\vspace{2mm}
\begin{center}
\includegraphics[scale=0.23]{fig/error_bar.png}
\end{center}
\caption{Mean error and mean standard deviation (shown with the error bar) of the estimated center of mass for customized objects obtained from different methods.}
\label{result1}
\vspace{-5mm}
\end{figure}

\subsection{CoM Estimation on Unseen Real-World Objects}
We also performed experiments on a set of 12 real-world objects that are commonly seen in daily life. We predefined the grasping point and found the ground truth CoM by balancing the object on each axis with a gripper.
% \yuchen{Sounds not trivial to me}
The objects have weights ranging from 43.4 grams to 613.2 grams with maximum dimensions from 56 mm to 285 mm. We try to create variations on the $X$, $Y$, and $Z$ axes of the measurement to assess the robustness of the methods. We present the result on the error of each axis for each object in Tab. \ref{table2} along with the dimension and weight of each object. We will give a more comprehensive discussion and analysis in Sec. \ref{discuss}.
%\setlength{\tabcolsep}{2pt}
\renewcommand{\arraystretch}{1.1}
\begin{table*}[btp]
\centering
\footnotesize
\vspace{2mm}
\begin{tabular}{m{14mm}|m{15mm}*{6}{|c@{\hspace{2mm}}c@{\hspace{2mm}}c@{\hspace{2mm}}}} % Adjusted spacing after Z columns

\multirow{3}{=}{\raisebox{5mm}{\centering Objects}}
& \multicolumn{1}{c|}{\raisebox{4mm}{\centering Image}} & \multicolumn{3}{c|}{\centering \includegraphics[width=11mm]{fig/1.png}}& \multicolumn{3}{c|}{\centering \includegraphics[width=11mm]{fig/2.png}} & \multicolumn{3}{c|}{\centering \includegraphics[width=11mm]{fig/3.png}}& \multicolumn{3}{c|}{\centering \includegraphics[width=11mm]{fig/4.png}}& \multicolumn{3}{c|}{\centering \includegraphics[width=11mm]{fig/5.png}}& \multicolumn{3}{c}{\centering \includegraphics[width=11mm]{fig/6.png}} \\

&\centering Dimensions (mm) & \multicolumn{3}{c|}{285 $\times$ 50 $\times$ 50}& \multicolumn{3}{c|}{185 $\times$ 45 $\times$ 40}& \multicolumn{3}{c|}{244 $\times$ 42 $\times$ 39}& \multicolumn{3}{c|}{51 $\times$ 152 $\times$ 99}& \multicolumn{3}{c|}{140 $\times$ 65 $\times$ 25}& \multicolumn{3}{c}{105 $\times$ 105 $\times$ 25}\\
& \centering Weight (g) & \multicolumn{3}{c|}{260.7} & \multicolumn{3}{c|}{325.1}& \multicolumn{3}{c|}{136.9}& \multicolumn{3}{c|}{307.7}& \multicolumn{3}{c|}{144.8}& \multicolumn{3}{c}{122.7}\\
 \hline
 \multirow{5}{10mm}{\newline \newline \newline \newline Prediction Error (mm)} &
\centering Axis & X & Y & Z & X & Y & Z & X & Y & Z & X & Y & Z & X & Y & Z  & X & Y & Z \\ 
\cline{2-20}
& \centering One Grasp Only & 11.4 & 14.2 & 14.0 & 17.8 & 16.1 & 6.2 & 12.0 & 5.9 & 8.4 &16.8&16.9&19.8& 8.1 &16.7&11.5&2.6&9.1&18.4  \\
& \centering Analytical Solution&6.9&5.5&8.4&6.3&\textbf{4.7}&17.9&14.0&3.1&9.3&6.1&3.8&10.7&9.9&5.5&19.4&2.3&\textbf{3.2}&21.2 \\
& \centering Random Rotate&10.1&22.2&13.7&13.6&18.8&\textbf{3.3}&12.2&4.9&13.4&\textbf{5.2}&14.8&17.1&13.3&11.2&9.2&2.4&9.4&18.5\\
 & \centering \textbf{U-GRAPH (Ours)}&\textbf{2.7}&\textbf{4.0}&\textbf{5.6}&\textbf{5.0}&5.6&4.3&\textbf{3.9}&\textbf{2.2}&\textbf{6.3}&\textbf{5.2}&\textbf{3.0}&\textbf{9.3}&\textbf{3.3}&\textbf{3.0}&\textbf{5.0}&\textbf{2.2}&6.8&\textbf{7.9}\\
\multicolumn{1}{c}{\newline }\\
\multirow{3}{=}{\raisebox{5mm}{Objects}}
& \multicolumn{1}{c|}{\raisebox{5mm}{Image}} & \multicolumn{3}{c|}{\centering \includegraphics[width=12mm]{fig/7.png}}& \multicolumn{3}{c|}{\centering \includegraphics[width=12mm]{fig/8.png}} & \multicolumn{3}{c|}{\centering \includegraphics[width=12mm]{fig/9.png}}& \multicolumn{3}{c|}{\centering \includegraphics[width=12mm]{fig/10.png}}& \multicolumn{3}{c|}{\centering \includegraphics[width=12mm]{fig/11.png}}& \multicolumn{3}{c}{\centering \includegraphics[width=12mm]{fig/12.png}} \\

& \centering Dimensions (mm) & \multicolumn{3}{c|}{218 $\times$ 3 $\times$ 3} & \multicolumn{3}{c|}{156 $\times$ 96 $\times$ 64}& \multicolumn{3}{c|}{135 $\times$ 109 $\times$ 94}& \multicolumn{3}{c|}{175 $\times$ 51 $\times$ 188}& \multicolumn{3}{c|}{92 $\times$ 56 $\times$ 192}& \multicolumn{3}{c}{56 $\times$ 56 $\times$ 56}\\
&\centering Weight (g) & \multicolumn{3}{c|}{197.4} & \multicolumn{3}{c|}{172.6}& \multicolumn{3}{c|}{400.0}& \multicolumn{3}{c|}{613.2 (OOD)}& \multicolumn{3}{c|}{43.4 (OOD)}&\multicolumn{3}{c}{76.8 (OOD)}\\
 \hline
\multirow{5}{10mm}{\newline \newline \newline \newline Prediction Error (mm)}  &\centering
Axis & X & Y & Z & X & Y & Z & X & Y & Z & X & Y & Z & X & Y & Z  & X & Y & Z \\ 
\cline{2-20}
&\centering One Grasp Only& 14.4&17.5&9.4&31.0&9.0&16.2&13.7&\textbf{6.1}&21.1&18.4 &25.3 & 26.4 &25.7&15.3&\textbf{8.1}&17.2&8.4&8.0\\
&\centering Analytical Solution&4.6&\textbf{3.0}&13.7&27.9&4.8&9.3&6.8&6.5&17.3&13.5&\textbf{9.2}&17.9&\textbf{7.3}&8.2&11.4&\textbf{1.8}&4.9&7.5\\
&\centering Random Rotate&15.4&23.2&18.0&26.2&3.8&20.3&11.9&20.9&14.4&\textbf{11.0}&20.1&25.3&25.9&8.5&12.1&15.9&\textbf{4.3}&4.2\\
 &\centering \textbf{U-GRAPH (Ours)}&\textbf{4.5}&7.7&\textbf{8.7}&\textbf{21.6}&\textbf{3.6}&\textbf{8.5}&\textbf{6.7}&6.5&\textbf{10.9}&11.8&15.8&\textbf{15.6}&12.1&\textbf{7.4}&10.1&13.5&9.3&\textbf{3.2}\\

\end{tabular}
\caption{The table shows the mean error of each axis of all 12 real-world objects. We performed 5 different grasp configurations on each object and tried to maximize the variations of $\mathrm{d}x$, $\mathrm{d}y$, and $\mathrm{d}z$ for each grasp. We also show the results of the baseline methods and bold the best-performing estimation for each axis of each object. The $X$, $Y$, and $Z$ axes are defined by the world frame. The OOD label in the last three objects' weight means that their weight is out of our collected data distribution.}
\label{table2}
\vspace{-5mm}
\end{table*}
\subsection{Additional Study on the Effect of Weights of the Object}
\label{add}
The performance of our algorithm is observed to decline when the object weight falls outside the range of our initial data collection, as highlighted in Tab. \ref{table2}. We have set up a focused experiment using the Mustard Bottle (Object 11 in Tab. \ref{table2}) as our primary test subject to further investigate. For this experiment, we supplemented the Mustard Bottle with three different sets of weights, bringing the total weights to 244.6 grams, 446.2 grams, and 648.1 grams, respectively. We tape sets of standard laboratory weights on the side of the mustard bottle around the measured CoM location.
%\yuchen{how}
The first two weights fall within the weight range of our collected dataset, while the last weight surpasses the upper limit of our previous data collection. We maintain a consistent grasping position across all weight variations to isolate the effect of weight on our CoM estimation accuracy. The specific grasping locations and the errors in the CoM predictions made using our method for each weight configuration are documented in Tab. \ref{table3}.

\begin{table}[htbp]
    \normalsize
    \centering
    \begin{tabular}{cm{20mm}|ccc}
     \multirow{6}{*}{\centering \includegraphics[width=25mm]{fig/grasp2.png}} &  \multirow{2}{=}{\centering Weights (g)} & \multicolumn{3}{c}{Mean Error (mm)}\\
         & & X &Y&Z\\
         \cline{2-5}
         &\centering 43.4 (OOD) & 12.1 & 7.4 & 10.1\\
         &\centering 244.6 & 5.0 & 6.1& 6.4 \\
         &\centering 446.2 & 6.1 &4.4 &5.6 \\
         &\centering 648.1 (OOD) & 15.6 &12.7 & 13.2\\
    \end{tabular}
    \caption{The left image shows the grasping locations on the mustard bottle. The right table shows the mean prediction error on each axis for different weights. The $X$, $Y$, and $Z$ axes are defined by the world coordinate. The OOD label means the weight is out of our training data distribution.}
    \label{table3}
    \vspace{-5mm}
\end{table}
\subsection{Limitation and Discussion}
\label{discuss}

Our approach aims to minimize the total error in CoM estimation, which occasionally compensates for the different axes. As evidenced in Tab. \ref{table2}, while our method might yield slightly poorer results on one axis, it significantly enhances performance on others. However, our method demonstrates superior accuracy in 10 of the 12 test objects compared to other techniques. This demonstrates the effectiveness of taking a second measurement, as highlighted by our comparison with the baseline method, \textbf{One Grasp}, and the importance of informed active perception noted against the \textbf{Random Rotate} method, which lacks the informed approach of our second measurements. We also show that we can always improve the estimation along the z-axis. This aligns with the intuition that a new orientation will introduce new information about the offset of the $z$-axis, with a second action. 

To further expand this work, predictions could benefit from multiple actions and continuous updates to the estimated center of mass. In this paper, our objective is to demonstrate that this predictive framework can enhance CoM estimation, rather than to claim a complete solution to the problem. Future research should investigate the optimal number of rotations and explore whether a series of actions and predictions can converge to the true CoM.
%\yuchen{Update the z-axis discussion}

Sec. \ref{add} illustrates how the objects’ weight range, spanning 127.36 to 585.36 grams, affects our algorithm’s performance. However, in practice, heavier objects are often inherently unstable for grasping. Extremely light objects fail to generate sufficient F/T signals to overcome noise, making it difficult to expand the dataset in these regions. Despite these challenges, our method demonstrates robust performance across diverse real-world objects that differ significantly in contact geometry, surface friction, and density from the training set, confirming its strong generalizability.

Moreover, our system encounters difficulties with large slips, especially with heavier objects or when the CoM is significantly offset from the grasp points. This is a common challenge in achieving stable grasps and accurate CoM estimations. To address these difficulties, we plan to integrate fingertip GelSight sensors \cite{gelsight} into our system in the future. These sensors will enable precise measurement of slips during manipulation, allowing us to gather critical data to refine our algorithm further. By enhancing our ability to detect and adjust for slips, we aim to improve both the stability of grasps and the robustness of CoM estimations. 




