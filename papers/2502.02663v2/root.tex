%%%%%%%%%%%%%%%%%%%%%%%%%%%%%%%%%%%%%%%%%%%%%%%%%%%%%%%%%%%%%%%%%%%%%%%%%%%%%%%%
%2345678901234567890123456789012345678901234567890123456789012345678901234567890
%        1         2         3         4         5         6         7         8

\documentclass[letterpaper, 10 pt, conference]{ieeeconf}  % Comment this line out if you need a4paper

%\documentclass[a4paper, 10pt, conference]{ieeeconf}      % Use this line for a4 paper

\IEEEoverridecommandlockouts% This command is only needed if 
% you want to use the \thanks command

\overrideIEEEmargins                                      % Needed to meet printer requirements.

%In case you encounter the following error:
%Error 1010 The PDF file may be corrupt (unable to open PDF file) OR
%Error 1000 An error occurred while parsing a content stream. Unable to analyze the PDF file.
%This is a known problem with the pdfLaTeX conversion filter. The file cannot be opened with Acrobat Reader
%Please use one of the alternatives below to circumvent this error by uncommenting one or the other
%\pdfobjcompresslevel=0
%\pdfminorversion=4

% See the \addtolength command later in the file to balance the column lengths
% on the last page of the document

% The following packages can be found at http:\\www.ctan.org
% \usepackage{graphics} % for pdf, bitmapped graphics files
\usepackage{subfig}
\usepackage{graphicx}
\usepackage{comment}
\usepackage{booktabs}
\usepackage{diagbox}
\usepackage{multirow}
\usepackage{colortbl}
\usepackage{makecell}
\usepackage{float}
\usepackage{placeins}
\usepackage{stfloats}
%\usepackage{epsfig} % for postscript graphics files
%\usepackage{mathptmx} % assumes new font selection scheme installed
%\usepackage{times} % assumes new font selection scheme installed
\usepackage{amsmath} % assumes amsmath package installed
\usepackage[bb=px]{mathalfa} 
%\usepackage{amssymb}  % assumes amsmath package installed
%\usepackage[disable]{todonotes}
\usepackage[colorinlistoftodos]{todonotes}
\newcommand{\wenzhen}[1]{\todo[inline,color=red!40]{Wenzhen: #1}}
\newcommand{\yuchen}[1]{\todo[inline,color=green!20]{Yuchen: #1}}
\newcommand{\samuel}[1]{\todo[inline,color=yellow!40]{Samuel: #1}}
\renewcommand{\arraystretch}{1.2}

\title{\LARGE \bf
Learning to Double Guess: An Active Perception Approach for Estimating the Center of Mass of Arbitrary Objects
}


\author{Shengmiao Jin, Yuchen Mo, Wenzhen Yuan$^{1}$% <-this % stops a space
\thanks{$^{1}$ University of Illinois Urbana-Champaign}
\thanks{\{\tt\small jin45, yuchenm7, yuanwz\}@illinois.edu}%
}


\begin{document}



\maketitle
\thispagestyle{empty}
\pagestyle{empty}


%%%%%%%%%%%%%%%%%%%%%%%%%%%%%%%%%%%%%%%%%%%%%%%%%%%%%%%%%%%%%%%%%%%%%%%%%%%%%%%%
\begin{abstract}

Manipulating arbitrary objects in unstructured environments is a significant challenge in robotics, primarily due to difficulties in determining an object's center of mass. This paper introduces U-GRAPH: Uncertainty-Guided Rotational Active Perception with Haptics, a novel framework to enhance the center of mass estimation using active perception. Traditional methods often rely on single interaction and are limited by the inherent inaccuracies of Force-Torque (F/T) sensors. Our approach circumvents these limitations by integrating a Bayesian Neural Network (BNN) to quantify uncertainty and guide the robotic system through multiple, information-rich interactions via grid search and a neural network that scores each action. We demonstrate the remarkable generalizability and transferability of our method with training on a small dataset with limited variation yet still perform well on unseen complex real-world objects. 


\end{abstract}


%%%%%%%%%%%%%%%%%%%%%%%%%%%%%%%%%%%%%%%%%%%%%%%%%%%%%%%%%%%%%%%%%%%%%%%%%%%%%%%%

\section{Introduction}

Lossy scientific data compression has emerged as a vitally important area in the past decade. The volume and velocity of scientific data heighten the urgency of the requirement of good data compression algorithms, specifically methods that can provide performance guarantees in terms of error bounds on the primary data (PD) of interest. The concomitant rise of machine learning has seen the flowering of different learning-based compression paradigms. The primary ones are super-resolution, transform-based, and more recently methods based on generative AI. We first examine these paradigms before turning to the relatively new approaches based on generative models---the paradigm adopted in the present work.

Lossy compression based on \textbf{super-resolution} \cite{Khani2021,Conde2022swin2sr} is based on the premise that the data of interest can be faithfully reconstructed from a small set of ``true'' samples. Machine learning methods based on this paradigm attempt data reconstruction (of the original tensor) from this sample set. \textbf{Transform-based} methods have traditionally been the most popular paradigm with discrete cosine transforms (DCT), wavelets, principal component analysis (PCA) and dictionary-based methods leading the way. More recently, autoencoders (AE) which transform the data into a compact and quantized latent space from which learned decoders reconstruct the original tensor have been the paradigm of choice among ML practitioners. However, these paradigms do not leverage recent advances in generative AI. In this newer approach---termed \textbf{conditional diffusion (CD)} \cite{Yang2023cd}---the original tensor is first gradually converted into zero mean, Gaussian noise. Then, a decoder is learned which gradually denoises the tensor through stages to finally produce a tensor approximately drawn from the probability distribution of the original images. A latent space embedding is used to guide the diffusion process. We propose to work within this paradigm but in the context of scientific data compression.

%\begin{figure}
    %\centering
    %\includegraphics[width=\textwidth]{Figures/Overview.pdf}
    %\caption{Overview of our conditional diffusion model for compression. The $\boldsymbol{x}_i$ denotes the $i^\mathrm{th}$ slice of a 3D block. We compress 3D blocks to capture spatiotemporal correlations in scientific datasets. The compressed (latent) codecs guide a 2D denoising diffusion process. Our diffusion model reconstructs each of the 2D slices in 3D blocks based on its corresponding latent data $\boldsymbol{z}_i$. This enables us to keep a relatively simple U-Net architecture while getting effective latent variables via 3D block compression.}\label{fig:overview}
%\end{figure}

\begin{wrapfigure}{r}{0.5\textwidth} 
\vspace{-0.4cm}
    \centering
    \includegraphics[width=\linewidth]{Figures/Overview.pdf}
    %\hspace{0.25cm}
  \caption{Overview of our conditional diffusion model for compression. We compress 3D blocks to capture spatiotemporal correlations in scientific datasets. The latent variables guide a 2D denoising diffusion process. Our denoising decoder reconstructs each of the 2D slices in 3D blocks based on its corresponding latent data $\boldsymbol{z}_i$. This enables us to keep a relatively simple U-Net architecture while getting effective conditioning via 3D block compression.}
    \label{fig:overview}
    \vspace{-0.4cm}
\end{wrapfigure}
 
Our approach to scientific data compression, situated within the conditional diffusion paradigm is now described. Figure~\ref{fig:overview} illustrates the overview of our proposed conditional diffusion models. Our model is a mixture of 3D block conditioning and 2D denoising diffusion. We first divide the original data into blocks of 3D tensors. 3D tensors are encoded into latent variables, which results in compressed codecs. We construct 3D latent embeddings using these codecs and they act as the conditioning information in CD. Unlike latent embeddings, we learn the denoising decoder in 2D space. Each 2D slice $\boldsymbol{x}_i$ of the 3D tensors is gradually converted into white noise in a stovepiped manner as described above. The denoising decoder estimates the noise of the 2D slice at the diffusion stage $t$. Using an embedding for the diffusion stage index, we learn the denoising decoder and the latent space embeddings in an end-to-end fashion. After training the machine, the original 3D tensor blocks are reconstructed with zero noise at the input (so that the decoder is entirely deterministic). The reconstructed primary data are examined to see if error bounds are violated and if so, we correct for the PD to be within pre-specified error bounds (using PCA or via a separate error bounding neural network). The main contributions are:
\begin{itemize}
    \item We propose a CD model for lossy scientific data compression. We divide the entire data into 3D tensors and map them into compressed codecs to capture spatiotemporal correlations in scientific datasets.
    \item The proposed CD model is a mixture of 3D conditioning and 2D diffusion. We prevent a complexity increase of our denoising decoder by avoiding a 3D diffusion process.
    \item To the best of our knowledge, this application of CD to scientific data compression with error guarantees---termed guaranteed conditional diffusion (GCD)---is a new contribution within a relatively new data compression paradigm.
\end{itemize}





\subsection{Retrieval-Augmented Generation}
Retrieval-Augmented Generation (RAG) has been widely applied to enhance the performance of Large Language Models (LLMs) by retrieving relevant information from external sources, addressing the limitation of LLMs' restricted context windows, improving factual accuracy, and mitigating hallucinations~\cite{fan2024survey, gao2023retrieval}. Most RAG systems primarily process text data by first splitting it into chunks~\cite{finardi2024chronicles}. When a query is received, RAG retrieves relevant chunks either through lexical search~\cite{ram2023context} or by computing semantic similarity~\cite{karpukhin2020dense}, embeddings both the query and text chunks into a shared vector space. Advanced techniques, such as pre-retrieval processing~\cite{ma2023query, zheng2023take} and post-retrieval processing~\cite{dong2024don, xu2023recomp}, as well as fine-tuning strategies~\cite{li2023structure}, have further enhanced RAG’s effectiveness across various domains, including QA)~\cite{yan2024corrective}, dialogue generation~\cite{izacard2023atlas}, and text summarization~\cite{jiang2023active}.

Several studies have evaluated the effectiveness of RAG systems across various tasks~\cite{yu2024evaluation, chen2024benchmarking, es2023ragas}, such as multi-hop question answering~\cite{tang2024multihop}, biomedical question answering~\cite{xiong2024benchmarking}, and text generation~\cite{liu2023recall}. However, no existing study has simultaneously and systematically evaluated and compared RAG and GraphRAG on these general text-based tasks.
% \yu{However, no existing study has simultaneously and systematically evaluated and compared RAG and GraphRAG on these general tasks. Add simultaneously to highlight our novelty is not evaluate any of them but at the same time and so we can compare.}

\begin{figure*}[!htb]
    \centering
   \includegraphics[width=\linewidth]{figures/RAGGraphRAG1.pdf}
    \caption{The illustration of RAG, KG-based GraphRAGs and Community-based GraphRAGs.}
    \label{fig:framework}
    \vspace{-0.2in}
\end{figure*}

\subsection{Graph Retrieval-Augmented Generation}
While RAG primarily processes text data, many real-world scenarios involve graph-structured data, such as knowledge graphs (KGs), social graphs, and molecular graphs~\cite{xia2021graph, ma2021deep}. GraphRAG~\cite{han2024retrieval, peng2024graph} aims to retrieve information from various types of graph-structured data. The inherent structure of graphs enhances retrieval by capturing relationships between connected nodes. For example, hyperlinks between documents can improve retrieval effectiveness in question answering tasks\cite{li2022dynamic}.  Currently, most GraphRAG studies focus on retrieving information from existing KGs for downstream tasks such as KG-based QA~\cite{tian2024graph, yasunaga2021qa} and Fact-Checking~\cite{kim2023factkg}. 
% The retrieved information can be subgraphs~\cite{he2024g} relevant to the query or reasoning paths~\cite{luo2023reasoning} that facilitate inference. 

Despite leveraging the existing graphs, recent studies have explored incorporating graph construction into GraphRAG to enhance text-based tasks. For example, \citet{dong2024don} construct document graphs using Abstract Meaning Representation (AMR) to improve document ranking. \citet{edge2024local} construct graphs from documents using LLMs, where nodes represent entities and edges capture relationships between them. Based on these graphs, they generate hierarchical communities and corresponding community summaries or reports. Their approach focuses on the global query summarization task, retrieving information from both the constructed graphs and their hierarchical communities. Additionally, \citet{han2025reasoning} propose an iterative graph construction approach using LLMs to improve reasoning tasks.
% \yu{Here I feel we spend so much context on introducing graph-based GraphRAG. However, the motivation of this work (also in the introduction part) is more around the issue of GraphRAG applied on text data. So it might be better to  emphasize more on that aspect}

These studies highlight the potential of GraphRAG in processing text-based tasks by constructing graphs from textual data.
However, their focus is limited to specific tasks and evaluation settings. It remains unclear how GraphRAG performs on general text-based tasks compared to RAG. More importantly, when and how should GraphRAG be applied to such tasks for optimal effectiveness? Our work aims to bridge this gap by systematically evaluating GraphRAG and comparing it with RAG on general text-based tasks.


% However, most existing studies focus on specific tasks and datasets, and there is no systematic evaluation—like those conducted for RAG—to assess the effectiveness of GraphRAG on text-based tasks using widely adopted datasets and evaluation metrics.\yu{I think here the thing is if we spend so much context on graph-based GraphRAG, then our evaluation should also be in that regard. However, our systematically evaluation is on GraphRAG for traditional text data. It might be better to focus our narratives more on that perspective.} Our work aims to fill this gap by systematically evaluating GraphRAG and comparing it with RAG.


\section{Method}
%\wenzhen{In general I think this section is unclear. You need to explain specifically your pipeline of estimating CoM. The definition of the poses to grasp objects is also part of the method}

%\wenzhen{start with a high-level overview of your method -- assuming people forgot most of things in introduction}
Targeting a generalized and robust CoM estimation framework, we propose U-GRAPH: Uncertainty-Guided Rotational Active Perception with Haptics. This system incorporates a BNN that processes 6-dimensional force-torque readings and 2-dimensional orientation data to yield a 3-dimensional CoM estimation. U-GRAPH also features ActiveNet, which utilizes the output from the BNN to determine the next best action. Assuming that the robot has already grasped the object, we perform two measurements at different orientations to accurately estimate its CoM. The BNN supplies both prior predictions and quantifies uncertainty through the standard deviation. The ActiveNet takes in prior estimation distribution and uses grid search to calculate a score for each action to determine the best one. Specifically, the action space is the 2-dimensional orientation of the grasping pose. We define the action executed as changing the pose. In the subsequent subsections, we discuss an intuitive physics model, introduce the individual modules of this framework, and present the implementation of online inference.


%To obtain a robust and generalizable estimation of the center of mass, we propose U-GRAPH: Uncertainty-Guided Rotational Active Perception with Haptic.
%U-GRAPH features a Bayesian Neural Network with mean estimation and uncertainty quantification to provide both the prior and secondary estimation of the CoM of the grasping object from the Force Torque reading. This network is trained with a 6-dimensional force-torque reading input, and 2-dimensional orientation input, with the supervised label as the 3-dimensional center of mass. We also believe that this can be generalized to other physical property problems with various input and output modalities. Our system also includes a second part that focuses on determining the next best action. We refer to this part as the ActiveNet, which trains from the result of the first networks, with data of the same label but went through different actions. In our setup, the action is the 2-dimensional rotation orientation of the grasping pose. The second part can also be substituted for any continuous action space. A visualization of the system can be found in Fig. \ref{active}, and we will explain the two parts in detail in the following subsections.
\subsection{Intuitive Model of Arbitrary Object's Center Of Mass}
\label{model}
\begin{figure}[htbp]
\begin{center}
\includegraphics[scale=0.4]{fig/modeling3.png}
\end{center}
\caption{Illustration of the simplified model of CoM on a real-world object. In our setup, we try to estimate the $\mathrm{d}x$, $\mathrm{d}y$, and $\mathrm{d}z$ which are the displacement of CoM away from the grasping point.}
\label{model_figure}
\vspace{-3mm}
\end{figure}

%\wenzhen{Shoudl this be in the method section?}
After grasping the object, we define its CoM by some displacement $dx$, $dy$, and $dz$ away from the grasping point. These axes are defined in the world coordinates, as illustrated in Fig. \ref{model}.
Ideally, we could directly employ an analytical solution using the 6-dimensional F/T reading from an F/T sensor to determine the CoM. However, real-world complications, such as measurement noise and potential in-hand slipping of the object, complicate this process. To counteract these issues, a second measurement is necessary. %This follow-up allows for recalibration and adjustment based on any discrepancies noted from the initial data, providing a more accurate and reliable estimation of the CoM. 
Our method aims to reduce the effect of real-world challenges towards a more robust and accurate prediction.


\begin{figure*}[htbp]
\vspace{2mm}
\begin{center}
\includegraphics[scale=0.26]{fig/active.png}
\end{center}
\caption{a) Flowchart for training Bayesian Neural Network. We train BNN with Markov Chain Monte Carlo and No U-Turn Sampler iteratively. b) Flowchart for training an active perception module. We calculate the score from two orientations as the supervised label of the ActiveNet. We use the first prediction's mean and standard deviation along with the second angle as the input to the network. c) Flowchart for inferencing with U-GRAPH. The robot first grasps with a fixed orientation, then passes the F/T reading with (0, 0) as orientations into the BNN. We use ActiveNet and grid search to find the second action. We pass the second F/T reading with the orientation through BNN to get a secondary prediction and join that with the first prediction to form the posterior prediction.}
%\wenzhen{This caption is too long}}
\label{active}
\vspace{-3mm}
\end{figure*} 

\subsection{Bayesian Neural Network for Uncertainty Qualification}
%\wenzhen{It's unclear how this section is relevant to your method. At least somewhere you need to explain you are using this network to estimate CoM}
%\wenzhen{Start your subsection with directly what you are doing here, or what problem you are trying to solve. Don't start with the background or motivation. Check with other method sections too}

%\wenzhen{It's not clear what do you want to show in the paragraph. It's always nice to start with a sentence for explaining what you want to show, and then goes to details. }
%A common approach to data-driven solutions is to use a Neural Network such as a Multi-Layer Perceptron (MLP) to regress from a large dataset. However, in our scenario, the challenge extends beyond mere predictions; we also aim to quantify the uncertainty associated with these predictions. For this purpose, we utilize a Bayesian Neural Network (BNN), which maintains a similar structure to a traditional MLP but operates under non-deterministic principles. The variability in these outputs allows us to determine the uncertainty of predictions, quantifying it as the standard deviation of the estimated center of mass.


The purpose of using BNN is to get a standard deviation for its output value. Instead of training to specify the exact weight of each network node, in the BNN framework, 
we want to learn a posterior distribution $p(w|D)$ given the input dataset $D$. Each node in our BNN will have a distribution instead of a deterministic value. Given this distribution, we can obtain the estimated distribution of unseen data     $P(\hat{\mathbf{y}}|\hat{\mathbf{x}})$ by getting the expectation of the predictive distribution: $P(\hat{\mathbf{y}}|\hat{\mathbf{x}}) = \mathbb{E}_{P(\mathbf{w}|\mathcal{D})}[P(\hat{\mathbf{y}}|\hat{\mathbf{x}},\mathbf{w})]$, $\mathbf{w}$ denotes the posterior distributions of the nodes in the BNN, $\hat{\mathbf{x}}$ denotes the input testing data and $\hat{\mathbf{y}}$ denotes the output prediction.

However, to evaluate this expectation value, we will need an infinite ensemble of networks as mentioned in \cite{weightuncertainty}. To practically approximate this, Monte Carlo sampling methods, particularly Markov Chain Monte Carlo (MCMC), are employed to reduce training and inference costs. MCMC provides unbiased samples from the posterior, facilitating effective posterior inference and backpropagation.  
Further improving this approach, the Hamiltonian Monte Carlo (HMC) and No U-Turn Sampler (NUTS) are incorporated to avoid the inefficient random walk behavior and dynamically determine the optimal number of steps in the HMC. This automatically adjusts the BNN parameters after each sample to enhance convergence and accuracy \cite{hoffman2011nouturnsampleradaptivelysetting} \cite{Brooks_2011}. 

To implement the BNN and MCMC with NUTS, we used Pyro \cite{pyro} to construct the network, train on our dataset, and evaluate its predictive function. This method gives us reliable uncertainty of the regression prediction of our MLP for active perception. The illustration of the BNN training process can be found in Fig. \ref{active} a).

\subsection{ActiveNet: Action Selection Network}
As mentioned before, our actions have 2 degrees of freedom, the last two joints on the robot are free to move, while all other joints are fixed during perception.
We always keep the orientation [0,0] as the first orientation. This is the orientation where the gripper points straight down, as shown in Fig. \ref{active} c).
To find the best second orientation that improves the prediction result, we consequently design ActiveNet and use grid search to find such orientation. The most intuitive way to generate a new action is to directly estimate from the prediction of the BNN and train the network to predict the best subsequent orientation. In our case, there are usually multiple orientations that the robot can take to minimize the error of CoM prediction. A simple regression model explicitly predicts a single ``best" action, but it can overlook other ``good" actions, especially if these are localized away from the highest peak. 

We therefore try to perform a grid search through the action space and estimate a score to determine how good each action is. For simplicity, we define this score as the error of estimation obtained by the BNN after we perform a specific rotation that results in the orientation $a$. 
%During data collection, we did not collect data explicitly according to the grid we used for data collection, and our grid size (2500) is much larger than our dataset size (100), proving that our network can do robust interpolation of data points for score estimation. 
As a result, the input of our ActiveNet as illustrated in Fig. \ref{active} b) has three parts, the estimation from BNN, the standard deviation from BNN, and a new proposed action that be scored on. 
The output of the ActiveNet is a score of this new proposed action. %For simplicity, we defined this as the error of the mean of the joint prior and posterior distribution after the second orientation. Both estimations are obtained from the pre-trained BNN.
%During training, the proposed action and ground truth action score is from a randomly selected action, and its error when passed through a pre-trained BNN. 

\subsection{Inference}
Our inference pipeline is illustrated in Fig. \ref{active} c). We first use the fixed orientation to generate a prior estimation of the CoM location. Then ActiveNet performs a grid search over the entire action space and calculates the score for each action with prior estimation as input. It uses the action with the minimum action score to proceed. %\wenzhen{The following part is not about action selection. You should use a new sub-section, or merge it with the overview, or find some other ways to accommodate it.}
After we obtain the new F/T reading from the second orientation, we then predict the CoM again with the same BNN. Finally, we treat each orientation as an independently observed measurement of CoM. Since our network can provide a quantified uncertainty, we assume the two measurements are Gaussian. Therefore we can obtain the posterior estimation with: $$
%with the following formula:$$
\mu_{final} =\frac{ (\frac{\mu_1}{\sigma_1^2} + \frac{\mu_2}{\sigma_2^2})}{(\frac{1}{\sigma_1^2} + \frac{1}{\sigma_2^2})}$$
%\yuchen{$\sigma_{final}$?}

\section{Experiment Setup}

In this section, we discuss the hardware setup of the CoM estimation problem. We also explain how to set up the hardware, collect training data, and implement models.

\subsection{Hardware Setup}
\label{hardware}
The hardware system features a 6-DoF UR5e robot arm. Attached to the robot's wrist is a 6-axis NRS-6050-D80 F/T sensor from Nordbo Robotics with a sampling rate of 1000 Hz. The arm is also equipped with a WSG-50 2-fingered gripper from Weiss Robotics with customized 3D-printed PLA fingers. The system is shown in Fig. \ref{setup} a).

For data collection, we designed and 3D-printed two objects with dimensions of 15cm $\times$ 15cm $\times$ 8cm, each including two holders sized 4cm $\times$ 4cm for placing AprilTags \cite{olson2011tags}. The plate object weighs 127.36 grams and allows grasping onto the center. The box object weighs 185.36 grams and is designated to be grasped on the side. We utilize standard laboratory weights for the experiments, specifically two 100-gram weights and one 200-gram weight. Fig. \ref{setup} b) shows the printed version of these objects, as well as weights that are randomly placed on them. 

\begin{figure}[htbp]
\vspace{2mm}
\begin{center}
\includegraphics[scale=0.36]{fig/setup.png}
\end{center}
\caption{a) Example of a data collection robot grasping with the location of F/T Sensor and gripper. b) Printed data collection objects in the real world, and standard lab weights for training data collection. AprilTags are placed on all of the objects. We refer to the object on the left as Plate and the object on the right as Box. }
\label{setup}
\vspace{-5mm}
\end{figure}

\subsection{Data Collection}
%\wenzhen{I think you should highlight the experiment design part-- the creation of those standard objects and your design thoughts are very interesting. You can mention this in the introduction. Also in this section you should emphasize it. E.g. separate the data collection part; add it in the subsection title or the opening sentence}

As mentioned in Sec. \ref{hardware}, we only collect data from the two customized objects for CoM estimation. Our model is based on the premise that the CoM of any grasped object can be fundamentally described by the offsets $\mathrm{d}x$, $\mathrm{d}y$, $\mathrm{d}z$, and the gravitational force $G$ acting on the object. During each trial, we randomly select 0 to 2 weights to be fixed onto one of the objects. On the software side, our data collection algorithm first uses the overhead camera to detect AprilTag and calculate the CoM while figuring out the graspable zone on the object. It randomly generates a valid grasping point from the graspable zone and calculates the $dx$, $dy$, and $dz$ from the CoM. Finally, the robot moves to the location and grasps the object to start a trial of data collection. This data collection algorithm saves the trouble of intensive human labor and allows us to do a larger scale of data collection. 
%\yuchen{``This'' refers to?}

After securing a grasp, the robot rotates the object to 100 different orientations (excluding the default [0,0]), recording the F/T readings at each position. We use the difference between the F/T reading with the object gripped and the F/T reading with nothing between fingers. Then, we loop through the entire 100 different orientations to calculate the action score for each of them. In total, we spent about 150 hours on data collection to generate a dataset from 204 different grasps, comprising 18,893 F/T readings. 

%For the active perception part, we assume the first action is fixed. We use the rotational orientation of $[0,0]$ as the fixed first orientation, which is the orientation that the gripper points straight down, as shown in Fig. \ref{active}. ActiveNet utilizes the variety of rotational data collected from identical grasps but differing orientations to train for better predictive performance. During the evaluation and testing phases, we employ a grid search across a 50 by 50 grid to comprehensively explore a rotational space spanning $2\pi$ by $1\pi$ radians. This method allows us to systematically assess the potential reorientation angles, ensuring optimal grasp and manipulation strategies are developed based on the CoM predictions from the BNN.
\subsection{Model Implementation}
The BNN has a backbone with a hidden size of [256, 128, 64]. To speed up the training process, we first train with PyTorch for a deterministic MLP with RAdam \cite{liu2019radam} optimizer and a learning rate of 0.001 for 500 epochs. Then we use the pre-trained weights as mean and standard deviation of 0.5 as the initialization for our BNN. We train the BNN using Pyro with 1000 samples and 200 warmup steps. 

The ActiveNet is a 5-layer MLP with a hidden size of [1024, 1024, 512, 64]. We train the ActiveNet with RAdam optimizer with a learning rate of 0.0001 for 500 epochs.
\section{Results and Discussion}

In this section, we present the result obtained from the CoM estimation with the U-GRAPH pipeline. We evaluated the performance of our model on the customized data collection setup with unseen weight distributions. In addition, we experiment on real-world objects with known CoM to validate the effectiveness of our proposed framework. 

\subsection{Baseline Methods}%\wenzhen{Will it be better to put this in IV?}
%\vspace{-1mm}
In contrast to our proposed method, we implement the following 3 different baseline methods:

\textbf{Analytical Solution}: The analytical solution assumes a perfect-world scenario with no noise in the F/T measurement and no in-hand slip. We can easily obtain the CoM of any object using the following formula: $\Vec{r}_{CoM} = \frac{\Vec{\tau} \times \Vec{F}}{|\Vec{F}|^2}$,
where $\Vec{F}$ denotes the force reading and $\Vec{\tau}$ denotes the torque readings. Since we only use the $[0,0]$ orientation, no torque should be caused by offsetting the $Z$-axis. However, real-world measurement noise introduces randomness into this calculation. Additionally, the F/T sensor produces inefficiently accurate torque measurement when the sensor is not placed vertically. Therefore, in this paper, we use analytical solutions only in the default pose as the baseline method.

% \yuchen{Explain why this gives random value on z-axis: in raw data corresponding torque should be 0 but in practice some small noise}

\textbf{One Grasp}: The One Grasp method only uses the first part of our proposed pipeline which is the model that takes in one F/T measurement and tries to infer the CoM. Unlike the analytical solution, this method also uses a neural network for estimation. The MLP used is the same one we used for active perception inference. %This method serves as an ablation study of using a second action

The previous two baselines have a fundamental flaw that based on one grasp it is impossible to evaluate the offset on the Z-axis.
% \yuchen{So is the z value from these methods simply random?}
% \samuel{sort of}

\textbf{Random Rotate}: The random rotation method uses two measurements similar to our proposed method. Instead of an informed action, this method uses a random action selected from the continuous action space to perform the measurement again. After getting the new reading, we will use the same BNN and joint distribution methods as our proposed method to estimate the CoM. %This method serves as an ablation study to use an informed action.


% \yuchen{Table 1 axis should be in world frame}
% \yuchen{Shouldn't OOD stand for Out-of-Distribution?}

\subsection{CoM Estimation on Customized Training Objects}


Our first experiment used the same plate and box setup but varied weight configurations. We tested five different weight configurations across both objects: no weight, a single 100-gram weight, two 100-gram weights placed together, two 100-gram weights placed separately, and two separate weights that weigh 300 grams in total. For each configuration, we performed five randomly selected grasps.  The data for this experiment are captured using the same overhead camera and AprilTags setup as training. The results of these experiments are detailed in Figure \ref{result1}, where we present a comprehensive analysis including the mean error and the mean standard deviation for each method applied.

\begin{figure}[hbtp]
\vspace{2mm}
\begin{center}
\includegraphics[scale=0.23]{fig/error_bar.png}
\end{center}
\caption{Mean error and mean standard deviation (shown with the error bar) of the estimated center of mass for customized objects obtained from different methods.}
\label{result1}
\vspace{-5mm}
\end{figure}

\subsection{CoM Estimation on Unseen Real-World Objects}
We also performed experiments on a set of 12 real-world objects that are commonly seen in daily life. We predefined the grasping point and found the ground truth CoM by balancing the object on each axis with a gripper.
% \yuchen{Sounds not trivial to me}
The objects have weights ranging from 43.4 grams to 613.2 grams with maximum dimensions from 56 mm to 285 mm. We try to create variations on the $X$, $Y$, and $Z$ axes of the measurement to assess the robustness of the methods. We present the result on the error of each axis for each object in Tab. \ref{table2} along with the dimension and weight of each object. We will give a more comprehensive discussion and analysis in Sec. \ref{discuss}.
%\setlength{\tabcolsep}{2pt}
\renewcommand{\arraystretch}{1.1}
\begin{table*}[btp]
\centering
\footnotesize
\vspace{2mm}
\begin{tabular}{m{14mm}|m{15mm}*{6}{|c@{\hspace{2mm}}c@{\hspace{2mm}}c@{\hspace{2mm}}}} % Adjusted spacing after Z columns

\multirow{3}{=}{\raisebox{5mm}{\centering Objects}}
& \multicolumn{1}{c|}{\raisebox{4mm}{\centering Image}} & \multicolumn{3}{c|}{\centering \includegraphics[width=11mm]{fig/1.png}}& \multicolumn{3}{c|}{\centering \includegraphics[width=11mm]{fig/2.png}} & \multicolumn{3}{c|}{\centering \includegraphics[width=11mm]{fig/3.png}}& \multicolumn{3}{c|}{\centering \includegraphics[width=11mm]{fig/4.png}}& \multicolumn{3}{c|}{\centering \includegraphics[width=11mm]{fig/5.png}}& \multicolumn{3}{c}{\centering \includegraphics[width=11mm]{fig/6.png}} \\

&\centering Dimensions (mm) & \multicolumn{3}{c|}{285 $\times$ 50 $\times$ 50}& \multicolumn{3}{c|}{185 $\times$ 45 $\times$ 40}& \multicolumn{3}{c|}{244 $\times$ 42 $\times$ 39}& \multicolumn{3}{c|}{51 $\times$ 152 $\times$ 99}& \multicolumn{3}{c|}{140 $\times$ 65 $\times$ 25}& \multicolumn{3}{c}{105 $\times$ 105 $\times$ 25}\\
& \centering Weight (g) & \multicolumn{3}{c|}{260.7} & \multicolumn{3}{c|}{325.1}& \multicolumn{3}{c|}{136.9}& \multicolumn{3}{c|}{307.7}& \multicolumn{3}{c|}{144.8}& \multicolumn{3}{c}{122.7}\\
 \hline
 \multirow{5}{10mm}{\newline \newline \newline \newline Prediction Error (mm)} &
\centering Axis & X & Y & Z & X & Y & Z & X & Y & Z & X & Y & Z & X & Y & Z  & X & Y & Z \\ 
\cline{2-20}
& \centering One Grasp Only & 11.4 & 14.2 & 14.0 & 17.8 & 16.1 & 6.2 & 12.0 & 5.9 & 8.4 &16.8&16.9&19.8& 8.1 &16.7&11.5&2.6&9.1&18.4  \\
& \centering Analytical Solution&6.9&5.5&8.4&6.3&\textbf{4.7}&17.9&14.0&3.1&9.3&6.1&3.8&10.7&9.9&5.5&19.4&2.3&\textbf{3.2}&21.2 \\
& \centering Random Rotate&10.1&22.2&13.7&13.6&18.8&\textbf{3.3}&12.2&4.9&13.4&\textbf{5.2}&14.8&17.1&13.3&11.2&9.2&2.4&9.4&18.5\\
 & \centering \textbf{U-GRAPH (Ours)}&\textbf{2.7}&\textbf{4.0}&\textbf{5.6}&\textbf{5.0}&5.6&4.3&\textbf{3.9}&\textbf{2.2}&\textbf{6.3}&\textbf{5.2}&\textbf{3.0}&\textbf{9.3}&\textbf{3.3}&\textbf{3.0}&\textbf{5.0}&\textbf{2.2}&6.8&\textbf{7.9}\\
\multicolumn{1}{c}{\newline }\\
\multirow{3}{=}{\raisebox{5mm}{Objects}}
& \multicolumn{1}{c|}{\raisebox{5mm}{Image}} & \multicolumn{3}{c|}{\centering \includegraphics[width=12mm]{fig/7.png}}& \multicolumn{3}{c|}{\centering \includegraphics[width=12mm]{fig/8.png}} & \multicolumn{3}{c|}{\centering \includegraphics[width=12mm]{fig/9.png}}& \multicolumn{3}{c|}{\centering \includegraphics[width=12mm]{fig/10.png}}& \multicolumn{3}{c|}{\centering \includegraphics[width=12mm]{fig/11.png}}& \multicolumn{3}{c}{\centering \includegraphics[width=12mm]{fig/12.png}} \\

& \centering Dimensions (mm) & \multicolumn{3}{c|}{218 $\times$ 3 $\times$ 3} & \multicolumn{3}{c|}{156 $\times$ 96 $\times$ 64}& \multicolumn{3}{c|}{135 $\times$ 109 $\times$ 94}& \multicolumn{3}{c|}{175 $\times$ 51 $\times$ 188}& \multicolumn{3}{c|}{92 $\times$ 56 $\times$ 192}& \multicolumn{3}{c}{56 $\times$ 56 $\times$ 56}\\
&\centering Weight (g) & \multicolumn{3}{c|}{197.4} & \multicolumn{3}{c|}{172.6}& \multicolumn{3}{c|}{400.0}& \multicolumn{3}{c|}{613.2 (OOD)}& \multicolumn{3}{c|}{43.4 (OOD)}&\multicolumn{3}{c}{76.8 (OOD)}\\
 \hline
\multirow{5}{10mm}{\newline \newline \newline \newline Prediction Error (mm)}  &\centering
Axis & X & Y & Z & X & Y & Z & X & Y & Z & X & Y & Z & X & Y & Z  & X & Y & Z \\ 
\cline{2-20}
&\centering One Grasp Only& 14.4&17.5&9.4&31.0&9.0&16.2&13.7&\textbf{6.1}&21.1&18.4 &25.3 & 26.4 &25.7&15.3&\textbf{8.1}&17.2&8.4&8.0\\
&\centering Analytical Solution&4.6&\textbf{3.0}&13.7&27.9&4.8&9.3&6.8&6.5&17.3&13.5&\textbf{9.2}&17.9&\textbf{7.3}&8.2&11.4&\textbf{1.8}&4.9&7.5\\
&\centering Random Rotate&15.4&23.2&18.0&26.2&3.8&20.3&11.9&20.9&14.4&\textbf{11.0}&20.1&25.3&25.9&8.5&12.1&15.9&\textbf{4.3}&4.2\\
 &\centering \textbf{U-GRAPH (Ours)}&\textbf{4.5}&7.7&\textbf{8.7}&\textbf{21.6}&\textbf{3.6}&\textbf{8.5}&\textbf{6.7}&6.5&\textbf{10.9}&11.8&15.8&\textbf{15.6}&12.1&\textbf{7.4}&10.1&13.5&9.3&\textbf{3.2}\\

\end{tabular}
\caption{The table shows the mean error of each axis of all 12 real-world objects. We performed 5 different grasp configurations on each object and tried to maximize the variations of $\mathrm{d}x$, $\mathrm{d}y$, and $\mathrm{d}z$ for each grasp. We also show the results of the baseline methods and bold the best-performing estimation for each axis of each object. The $X$, $Y$, and $Z$ axes are defined by the world frame. The OOD label in the last three objects' weight means that their weight is out of our collected data distribution.}
\label{table2}
\vspace{-5mm}
\end{table*}
\subsection{Additional Study on the Effect of Weights of the Object}
\label{add}
The performance of our algorithm is observed to decline when the object weight falls outside the range of our initial data collection, as highlighted in Tab. \ref{table2}. We have set up a focused experiment using the Mustard Bottle (Object 11 in Tab. \ref{table2}) as our primary test subject to further investigate. For this experiment, we supplemented the Mustard Bottle with three different sets of weights, bringing the total weights to 244.6 grams, 446.2 grams, and 648.1 grams, respectively. We tape sets of standard laboratory weights on the side of the mustard bottle around the measured CoM location.
%\yuchen{how}
The first two weights fall within the weight range of our collected dataset, while the last weight surpasses the upper limit of our previous data collection. We maintain a consistent grasping position across all weight variations to isolate the effect of weight on our CoM estimation accuracy. The specific grasping locations and the errors in the CoM predictions made using our method for each weight configuration are documented in Tab. \ref{table3}.

\begin{table}[htbp]
    \normalsize
    \centering
    \begin{tabular}{cm{20mm}|ccc}
     \multirow{6}{*}{\centering \includegraphics[width=25mm]{fig/grasp2.png}} &  \multirow{2}{=}{\centering Weights (g)} & \multicolumn{3}{c}{Mean Error (mm)}\\
         & & X &Y&Z\\
         \cline{2-5}
         &\centering 43.4 (OOD) & 12.1 & 7.4 & 10.1\\
         &\centering 244.6 & 5.0 & 6.1& 6.4 \\
         &\centering 446.2 & 6.1 &4.4 &5.6 \\
         &\centering 648.1 (OOD) & 15.6 &12.7 & 13.2\\
    \end{tabular}
    \caption{The left image shows the grasping locations on the mustard bottle. The right table shows the mean prediction error on each axis for different weights. The $X$, $Y$, and $Z$ axes are defined by the world coordinate. The OOD label means the weight is out of our training data distribution.}
    \label{table3}
    \vspace{-5mm}
\end{table}
\subsection{Limitation and Discussion}
\label{discuss}

Our approach aims to minimize the total error in CoM estimation, which occasionally compensates for the different axes. As evidenced in Tab. \ref{table2}, while our method might yield slightly poorer results on one axis, it significantly enhances performance on others. However, our method demonstrates superior accuracy in 10 of the 12 test objects compared to other techniques. This demonstrates the effectiveness of taking a second measurement, as highlighted by our comparison with the baseline method, \textbf{One Grasp}, and the importance of informed active perception noted against the \textbf{Random Rotate} method, which lacks the informed approach of our second measurements. We also show that we can always improve the estimation along the z-axis. This aligns with the intuition that a new orientation will introduce new information about the offset of the $z$-axis, with a second action. 

To further expand this work, predictions could benefit from multiple actions and continuous updates to the estimated center of mass. In this paper, our objective is to demonstrate that this predictive framework can enhance CoM estimation, rather than to claim a complete solution to the problem. Future research should investigate the optimal number of rotations and explore whether a series of actions and predictions can converge to the true CoM.
%\yuchen{Update the z-axis discussion}

Sec. \ref{add} illustrates how the objects’ weight range, spanning 127.36 to 585.36 grams, affects our algorithm’s performance. However, in practice, heavier objects are often inherently unstable for grasping. Extremely light objects fail to generate sufficient F/T signals to overcome noise, making it difficult to expand the dataset in these regions. Despite these challenges, our method demonstrates robust performance across diverse real-world objects that differ significantly in contact geometry, surface friction, and density from the training set, confirming its strong generalizability.

Moreover, our system encounters difficulties with large slips, especially with heavier objects or when the CoM is significantly offset from the grasp points. This is a common challenge in achieving stable grasps and accurate CoM estimations. To address these difficulties, we plan to integrate fingertip GelSight sensors \cite{gelsight} into our system in the future. These sensors will enable precise measurement of slips during manipulation, allowing us to gather critical data to refine our algorithm further. By enhancing our ability to detect and adjust for slips, we aim to improve both the stability of grasps and the robustness of CoM estimations. 





\section{Conclusions}

This paper presents U-GRAPH, a novel approach to the center of mass estimation with active perception. We design a pipeline that contains two main components: a Bayesian Neural Network that can provide prediction and its associated uncertainty, and an ActiveNet that produces an informed rotation based on the prior estimation. This approach reduces the need for repetitive grasping by replacing it with an efficient and effective rotation. Our experiments validate the effectiveness of U-GRAPH, which consistently outperforms traditional methods and adapts well to real-world scenarios. 



\addtolength{\textheight}{0cm}   

\section*{ACKNOWLEDGMENT}

% The authors would like to thank the entire RoboTouch Lab, especially Amin for his help with photos and figures, Harsh for revising the paper, and Xiping for helping make the video.

The authors would like to thank Mohammad Amin Mirzaee for his help with photos and figures, Hung-Jui (Joe) Huang for thoughtful discussions, and Harsh Gupta for revising the paper. We also thank Xiping Sun and Yilong Niu for helping make the video.
% The authors would like to thank Ruohan, Arpit, Jingyi, Dakarai, and Yuchen for their help towards this paper. Special Thanks to Amin for helping with picture making and hardware design.


\bibliographystyle{IEEEtran} 
\bibliography{bibliography.bib}


\end{document}
