\section{Introduction}

%\wenzhen{Opening is not very relevant and not well connected with the content}
%\wenzhen{Somewhere in the introduction you need to briefly review the state of the current research on the topic}
 %\wenzhen{Lots of things in this paragraph is not very relevant}
With the growing interest in robotics manipulation in the wild, researchers have been investigating ways for robots to interact with different objects. A key factor in achieving a secure grasp is the proximity of the grasp point to the object's center of mass (CoM). Yet, not much research has been trying to address a generalizable approach to estimate the CoM of arbitrary objects. Much recent research on grasping has been focused solely on grasping onto the object \cite{kosmose24} \cite{yuchen23}, often overlooking how the choice of grasping points and the physical properties of objects affect successful manipulation. It has been noted in works like \cite{Poseit} and \cite{rotate_stability} that improper grasping points or poses on objects with certain properties can lead to failures, typically due to rotational and incidental slips. Knowing the CoM of the object, on the other hand, can help decide a more stable and robust grasping strategy, and ensure safety during manipulation.


\begin{figure}[htbp]
\vspace{-5mm}
\begin{center}
\includegraphics[scale=0.31]{fig/teaser3.png}
\end{center}
\caption{We design an active perception algorithm to estimate the center of mass of arbitrary objects. Our algorithm uses the first estimation from the F/T reading to infer a new rotational orientation that improves the estimation, then executes the action and estimates again with a second F/T reading. }
\label{teaser}
\vspace{-5mm}
\end{figure}


This work provides a framework for perceiving the CoM of an arbitrary object. object. The most intuitive solution is using the Force-Torque (F/T) sensor's reading and solving CoM analytically. However, the physical limitations of F/T sensors provide inaccurate readings caused by the measurement noise and in-hand slips, therefore analytical solutions often fail. Additionally, analytical solutions fails to provide uncertainty measurements, making it impossible to estimate an informed action to improve measurements. Instead, we turn to a data-driven solution to learn the correlation between the F/T Reading and the CoM locations with neural networks. Moreover, a vertical grasping pose prevents obtaining the location of CoM on the z-axis. Hence, we cannot predict precisely enough with a single F/T reading. Our strategy incorporates active perception techniques to enhance accuracy, which identifies the orientation that maximizes information gained from the initial grasp. Our methodology involves actively guiding the robot to adjust its orientation which will provide a more accurate prediction. 

Therefore, we propose \textbf{U-GRAPH}: Uncertainty-Guided Rotational Active Perception with Haptics to address the difficulty found in the CoM Estimation problems as illustrated in Fig. \ref{teaser}. We construct a Bayesian Neural Network (BNN) \cite{1992bnn} for uncertainty quantification, obtaining a standard measurement deviation aligned with physical intuition. First, the robot picks up the object with a fixed orientation, providing a prior estimation of the CoM. With the estimation's mean and standard deviation, we employ ActiveNet to infer the orientation that can provide the greatest information gain. From the subsequent orientation, the BNN can provide a new estimation of CoM from the latest F/T reading. 

Our system is trained on 204 grasp trials each with 100 rotations with 20 weight distributions. Despite being trained only on two customized simple data collection objects, we demonstrate its ability to generalize to any arbitrary rigid object in real life. It shows an average of 1.47-centimeter error with a 7.6\% error on unseen complex real-world objects in a zero-shot transfer manner. Our system shows generalizability to all kinds of objects with different contact geometry, surface friction, and overall shapes. %With the result from CoM estimation, the robot can interact with any object more safely, preventing failure caused by slips. Our method shows the potential to apply to any non-visual properties that are needed for object manipulation. %Our results with CoM estimation indicate a step towards more intelligent robotic manipulation with complex objects.% In summary, the key contribution of this work is the active perception algorithm that can perform robust and accurate center of mass estimation on arbitrary objects with a small and limited dataset.
%\wenzhen{A few sentence to talk about the future: like how the methodology can be developed or used in other tasks? How the result of CoM measurement can help robots?}



