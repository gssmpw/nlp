\section{Related works}
%\wenzhen{You need to have a subsection to talk about CoM estimation}
\subsection{Physical Property Estimation}
%\wenzhen{I don't think this topic is very relevant. It's not wrong to keep it, but if you need to save space you can remove this}
A critical premise for the successful manipulation of different objects is an understanding of their physical properties. For example, liquid properties were well-studied by previous research \cite{joe-liquid} \cite{xiaofeng-solid} \cite{matl}. Moreover, many papers also showed that with perception, the precision and accuracy of manipulation can be increased \cite{stir-to-pour} \cite{joe-kitchen} \cite{pancake24}. 
Rigid body properties are also important for manipulation. Zeng et al. used residual physics to help decide a better-tossing policy for a wide range of objects \cite{zeng2019tossingbot}. Wang et al. demonstrated their algorithm can learn implicit properties and improves their swing policy with tactile explorations \cite{swingbot}. Murooka et al. embedded physical reasoning into manipulation skills \cite{murooka_physics}. Most previous works only consider physical property as a vague or distilled representation, while we focus on estimating the explicit measurement of the center of mass.  
\subsection{Center of Mass Estimation}
A more related set of works is directly aimed at estimating the center of mass. Hyland et al. utilized iterative pushing to find a 2-dimensional CoM \cite{com23}. McGovern et al. pointed out that with reinforcement learning of stacking random shape objects in a simulator, they can estimate the CoM \cite{com19}. McGovern and Xiao also proposed a Reinforcement Learning pipeline in the real world to estimate the CoM of utensils with torque-sensing \cite{com22}. However, to our knowledge, none of the methods are generalizable to find the 3-dimensional CoM for arbitrary objects in real life, which is the problem we are trying to solve. 

\subsection{Active Perception}
%\wenzhen{The best way to write the related work is to emphasize the connection of those works with your work. Active perception is very wide. In your work, you want to highlight the algorithms for active perception, and explain how your method is new, so in the review you should talk more about their contribution in the algorithm}
Humans naturally possess the ability to explore an object actively with touch \cite{activetouch}. Inspired by this, many works have studied active perception in robotics with haptic or tactile sensing. 
Xu et al. employed active tactile perception to classify objects and showed improvement in both accuracy and efficiency \cite{xu23tandem3d}. Uttayopas et al. utilized active haptic sensing to classify objects with different properties \cite{Utta23haptic}. Kuzliak et al. designed a framework to interactively learn the physical properties of an object with informed action selection \cite{kruzliak2024interactivelearningphysicalobject}. The works mentioned above only concern a discrete action space, but in our problem setting, we have a continuous action space. Yuan et al. used active perception to decide the next best grasping location to help increase accuracy for clothes material classification \cite{clothes}. Ketchum used active exploration on a scene to understand its property and shows that with active planning the exploration can performed better \cite{biotacHaptic24}. In our work, on the other hand, instead of classification, we are interested in regression tasks with continuous action space and aim to solve the CoM estimation with only two actions.