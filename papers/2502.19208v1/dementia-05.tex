\balance
\section{Conclusion AND FUTURE WORK}
\label{sec:concl}


This paper presents a geographically diverse, multilingual conversational dataset for Alzheimer's detection, comprising 16 publicly available datasets, featuring a range of cognitive assessment tasks, in English, Spanish, Chinese, and Greek. Additionally, an English-translated version of the multilingual dataset has been created to evaluate the impact of translation on Alzheimer's detection across the included languages. The main contributions include unifying datasets with varying formats into a consistent text format. The work also investigates how model framing (binary vs. multiclass) and dataset composition (monolingual vs. combined-multilingual, vs. combined-translated) affect Alzheimer’s detection performance and model differentiation between cognitive decline stages.



The results of this study highlight the fundamental challenges in Alzheimer’s detection when transitioning from a binary to a multiclass classification approach. While distinguishing between
cases of Alzheimer’s Disease (AD) and Healthy Controls (HC) is often achievable with reasonable accuracy, achieving the same level of accuracy in a multiclass setting that also includes patients with Mild Cognitive Impairment (MCI) proves significantly more difficult. This is largely due to the clinical and cognitive overlap between MCI and the other two categories, leading to increased misclassification rates. Our findings also shed light on the potential benefits and limitations of multilingual and translated datasets. While some languages, such as Chinese, show notable improvements with combined-multilingual or translated training, others, such as Greek, experience a decline in performance. This suggests that the effectiveness of multilingual learning is highly language-dependent, likely influenced by underlying cognitive tasks conducted in the dataset collection step, dataset size, linguistic features, and model adaptation. Furthermore, while translation can enhance performance in certain cases, it does not provide a universal advantage across languages, sometimes introducing additional classification errors, e.g., in Greek. These findings emphasize the need for tailored approaches in multilingual Alzheimer’s detection, considering language-specific challenges and dataset characteristics to optimize diagnostic accuracy.





This multilingual conversational dataset opens up significant avenues for future research in Alzheimer's detection. While our results demonstrate the feasibility of automatic detection, the substantial headroom for improvement suggests several promising directions.
First, future research may focus on optimizing language-specific approaches, tailoring methods to the unique characteristics of each language. A deeper investigation into syntactic, semantic, and lexical features could reveal subtle, language-specific indicators of cognitive decline. Second, advanced preprocessing techniques to explicitly identify participant roles (i.e., patient vs. interviewer) within the conversations could help create more nuanced models that specifically target the patient's language use.
Third, investigating cross-lingual patterns and leveraging transfer learning techniques could enhance model generalization and robustness, particularly for languages with limited data. Finally, exploring advanced data augmentation methods to address the class imbalance across diagnostic groups (HC, MCI, and AD) could help improve classification performance.
