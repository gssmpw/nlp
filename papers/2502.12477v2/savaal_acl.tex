% This must be in the first 5 lines to tell arXiv to use pdfLaTeX, which is strongly recommended.
% This must be in the first 5 lines to tell arXiv to use pdfLaTeX, which is strongly recommended.
\pdfoutput=1
% In particular, the hyperref package requires pdfLaTeX in order to break URLs across lines.

\documentclass[11pt]{article}

% Change "review" to "final" to generate the final (sometimes called camera-ready) version.
% Change to "preprint" to generate a non-anonymous version with page numbers.
\usepackage[final]{acl}
% \usepackage{acl}

% \usepackage[subtle, title=normal]{savetrees}


% Standard package includes
\usepackage{times}
\usepackage{latexsym}
\usepackage{xspace}
\usepackage{enumitem}
\usepackage{tabto}
\usepackage{subcaption}
\usepackage{tabularx}
\usepackage{array}
\usepackage{placeins}

% For proper rendering and hyphenation of words containing Latin characters (including in bib files)
\usepackage[T1]{fontenc}
% For Vietnamese characters
% \usepackage[T5]{fontenc}
% See https://www.latex-project.org/help/documentation/encguide.pdf for other character sets

% This assumes your files are encoded as UTF8
\usepackage[utf8]{inputenc}
\usepackage{csquotes}

% This is not strictly necessary, and may be commented out,
% but it will improve the layout of the manuscript,
% and will typically save some space.
\usepackage{microtype}

% This is also not strictly necessary, and may be commented out.
% However, it will improve the aesthetics of text in
% the typewriter font.
\usepackage{inconsolata}

%Including images in your LaTeX document requires adding
%additional package(s)
\usepackage{graphicx}
\usepackage{multirow}
\usepackage{amsmath}
\usepackage{amssymb}
\usepackage{mathtools}
\usepackage{amsthm}
\usepackage{caption} 
\captionsetup{aboveskip=8pt, belowskip=8pt}
\usepackage{tcolorbox}
\usepackage{aliascnt}
\usepackage{xcolor}
\usepackage{tikz}
\usepackage{colortbl}
\usepackage{pgf} % Ensure this package is included in your preamble
\usepackage{pgfmath}
\usepackage{ragged2e}


\makeatletter
\renewcommand{\sectionautorefname}{\S\@gobble}
\makeatother % Changes 'subs
\makeatletter
\renewcommand{\subsectionautorefname}{\S\@gobble}
\makeatother % Changes 'subs
\makeatletter
\renewcommand{\subsubsectionautorefname}{\S\@gobble} % Changes 'subs
\makeatother
\renewcommand{\equationautorefname}{Eqn} % Changes 'subs
\renewcommand{\figureautorefname}{Fig.} % Changes 'Figure' to 'Fig.'
\renewcommand{\tableautorefname}{Table} % Keeps 'Table' as is or change as needed

\def\compactify{\itemsep=0pt \topsep=0pt \partopsep=0pt \parsep=0pt}
\let\latexusecounter=\usecounter
\newenvironment{CompactEnumerate}
  {\def\usecounter{\compactify\latexusecounter}
   \begin{enumerate}}
  {\end{enumerate}\let\usecounter=\latexusecounter}



\tcbset{
    colframe=gray!40, 
    colback=gray!5,   
    coltitle=black,   
    fonttitle=\bfseries,
    sharp corners,
    boxrule=0.5mm,  
    width=\columnwidth,
    left=0.1mm,        
    right=0.1mm,      
    toptitle=0.1mm,
    bottomtitle=0.1mm,
    title=Question Generation Prompt
}
% If the title and author information does not fit in the area allocated, uncomment the following
%
%\setlength\titlebox{<dim>}
%
% and set <dim> to something 5cm or larger.

\title{\name: Scalable Concept-Driven Question Generation\\to Enhance Human Learning}

% Author information can be set in various styles:
% For several authors from the same institution:
% \author{Author 1 \and ... \and Author n \\
%         Address line \\ ... \\ Address line}
% if the names do not fit well on one line use
%         Author 1 \\ {\bf Author 2} \\ ... \\ {\bf Author n} \\
% For authors from different institutions:
% \author{Author 1 \\ Address line \\  ... \\ Address line
%         \And  ... \And
%         Author n \\ Address line \\ ... \\ Address line}
% To start a separate ``row'' of authors use \AND, as in
% \author{Author 1 \\ Address line \\  ... \\ Address line
%         \AND
%         Author 2 \\ Address line \\ ... \\ Address line \And
%         Author 3 \\ Address line \\ ... \\ Address line}

\author{
 \textbf{Kimia Noorbakhsh\textsuperscript{*}}, 
 \textbf{Joseph Chandler\textsuperscript{*}}, 
 \textbf{Pantea Karimi\textsuperscript{*}}, 
 \\
 \textbf{Mohammad Alizadeh}, 
 \textbf{Hari Balakrishnan}
\\
\\
 M.I.T. Computer Science and Artificial Intelligence Lab (CSAIL)
}


% \renewcommand{\thefootnote}{\textsuperscript{*}}
% \footnotetext{Equal contribution.}

%##########################Custom commands ###############
\definecolor{customblue}{HTML}{DAE8FC}
\definecolor{customred}{HTML}{F8CECC}
\definecolor{customgreen}{HTML}{D5E8D4}
\definecolor{custompurple}{HTML}{E1D5E7}
\definecolor{customorange}{HTML}{FFE6CC}


\newcommand*\circled[2]{\tikz[baseline=(char.base)]{
            \node[shape=circle, line width=0.75pt, draw=black, fill=#2, inner sep=1pt] 
            (char) {\textcolor{black}{\small\textsf{#1}}};}}


\newcommand{\bluecircle}{\raisebox{0pt}{\protect\tikz[baseline=(char.base)]{\protect\node[shape=circle,draw, fill=customblue, minimum size=1.8mm, inner sep=1pt] (char) {\footnotesize 1};}}}
\newcommand{\redcircle}{\raisebox{0pt}{\protect\tikz[baseline=(char.base)]{\protect\node[shape=circle,draw, fill=customred, minimum size=1.8mm, inner sep=1pt] (char) {\footnotesize 2};}}}
\newcommand{\greencircle}{\raisebox{0pt}{\protect\tikz[baseline=(char.base)]{\protect\node[shape=circle,draw, fill=customgreen, minimum size=1.8mm, inner sep=1pt] (char) {\footnotesize 3};}}}
\newcommand{\purplecircle}{\raisebox{0pt}{\protect\tikz[baseline=(char.base)]{\protect\node[shape=circle,draw, fill=custompurple, minimum size=1.8mm, inner sep=1pt] (char) {\footnotesize 4};}}}
\newcommand{\orangecircle}{\raisebox{0pt}{\protect\tikz[baseline=(char.base)]{\protect\node[shape=circle,draw, fill=customorange, minimum size=1.8mm, inner sep=1pt] (char) {\footnotesize 5};}}}



\newcommand*\sfboxed[1]{\tikz[baseline=(char.base)]{
            \node[shape=rectangle,line width=0.75pt, draw=black,inner sep=2pt, rounded corners=2pt] (char) {\textcolor{black}{\small\textsf{#1}}};}}
\newcommand{\NewPara}[1]{\noindent{\bf #1}}

\newcommand{\TheSystem}{Savaal\xspace}
\newcommand{\Baseline}{Direct\xspace}
\newcommand{\BaselineMT}{Direct\xspace}
\newcommand{\name}{Savaal\xspace}
\newcommand{\arxiv}{arXiv\xspace}
\newcommand{\gpt}{GPT-4o}

\begin{document}
\maketitle
\def\thefootnote{*}\footnotetext{These authors contributed equally to this work.}\def\thefootnote{\arabic{footnote}}
% Scatter plot PhD

\newcommand{\hitresponse}{38\%\xspace}

\newcommand{\understandingpreferration}{6.5$\times$\xspace}
\newcommand{\choicespreferration}{3$\times$\xspace}
\newcommand{\usabilitypreferration}{2$\times$\xspace}

\newcommand{\savaalunderstandingpreferrationpercent}{61.9\%\xspace}
\newcommand{\baselineunderstandingpreferrationpercent}{9.5\%\xspace}
\newcommand{\sameunderstandingpreferrationpercent}{28.6\%\xspace}

\newcommand{\savaalchoicespreferrationpercent}{57.1\%\xspace}
\newcommand{\baselinechoicespreferrationpercent}{19.0\%\xspace}
\newcommand{\samechoicespreferrationpercent}{23.9\%\xspace}

\newcommand{\savaalusabilitypreferrationpercent}{47.6\%\xspace}
\newcommand{\baselineusabilitypreferrationpercent}{23.8\%\xspace}
\newcommand{\sameusabilitypreferrationpercent}{28.6\%\xspace}

% Weighted average difference
\newcommand{\phdunderstandingWAD}{17\%\xspace}
\newcommand{\phdchoicesWAD}{10\%\xspace}
\newcommand{\phdusabilityWAD}{11.4\%\xspace}

% PhD


\newcommand{\baselinephdunderstandingvalue}{29.0}
\newcommand{\savaalphdunderstandingvalue}{11.9}

\newcommand{\baselinephdchoicesvalue}{39.0}
\newcommand{\savaalphdchoicesvalue}{29.0}


\newcommand{\baselinephdusabilityvalue}{32.9}
\newcommand{\savaalphdusabilityvalue}{21.4}


% Paper 
\newcommand{\baselinepaperunderstandingvalue}{16.7}
\newcommand{\savaalpaperunderstandingvalue}{10.9}

\newcommand{\baselinepaperchoicesvalue}{22.1}
\newcommand{\savaalpaperchoicesvalue}{21.8}


\newcommand{\baselinepaperusabilityvalue}{15.3}
\newcommand{\savaalpaperusabilityvalue}{13.8}



%%%%%%%%%%% CHNAGEABLES UP %%%%%%%%%%%

% Percent disagree macros
\newcommand{\savaalphdusability}{\savaalphdusabilityvalue\%\xspace}
\newcommand{\baselinephdusability}{\baselinephdusabilityvalue\%\xspace}
\newcommand{\phdusabilityreductionvalue}{%
  \pgfmathparse{round(\baselinephdusabilityvalue/\savaalphdusabilityvalue*100 )/100}%
  \pgfmathprintnumber[fixed, precision=1]{\pgfmathresult}%
}

\newcommand{\phdusabilityreduction}{\phdusabilityreductionvalue$\times$\xspace}

\newcommand{\savaalphdchoice}{\savaalphdchoicevalue\%\xspace}
\newcommand{\baselinephdchoice}{\baselinephdchoicevalue\%\xspace}
\newcommand{\phdchoicereductionvalue}{%
  \pgfmathparse{round(\baselinephdchoicevalue/\savaalphdchoicevalue*100 )/100}%
  \pgfmathprintnumber[fixed, precision=1]{\pgfmathresult}%
}

\newcommand{\phdchoicereduction}{\phdchoicereductionvalue$\times$\xspace}

\newcommand{\savaalphdunderstanding}{\savaalphdunderstandingvalue\%\xspace}
\newcommand{\baselinephdunderstanding}{\baselinephdunderstandingvalue\%\xspace}
\newcommand{\phdunderstandingreductionvalue}{%
  \pgfmathparse{round(\baselinephdunderstandingvalue/\savaalphdunderstandingvalue*100 )/100}%
  \pgfmathprintnumber[fixed, precision=1]{\pgfmathresult}%
}
\newcommand{\phdunderstandingreduction}{\phdunderstandingreductionvalue$\times$\xspace}


\newcommand{\savaalphdchoices}{\savaalphdchoicesvalue\%\xspace}
\newcommand{\baselinephdchoices}{\baselinephdchoicesvalue\%\xspace}
\newcommand{\phdchoicesreductionvalue}{%
  \pgfmathparse{round(\baselinephdchoicesvalue/\savaalphdchoicesvalue*100 )/100}%
  \pgfmathprintnumber[fixed, precision=1]{\pgfmathresult}%
}
\newcommand{\phdchoicesreduction}{\phdchoicesreductionvalue$\times$\xspace}

\newcommand{\savaalpaperusability}{\savaalpaperusabilityvalue\%\xspace}
\newcommand{\baselinepaperusability}{\baselinepaperusabilityvalue\%\xspace}
\newcommand{\paperusabilityreductionvalue}{%
  \pgfmathparse{round(\baselinepaperusabilityvalue/\savaalpaperusabilityvalue*100 )/100}%
  \pgfmathprintnumber[fixed, precision=1]{\pgfmathresult}%
}
\newcommand{\paperusabilityreduction}{\paperusabilityreductionvalue$\times$\xspace}

\newcommand{\savaalpaperunderstanding}{\savaalpaperunderstandingvalue\%\xspace}
\newcommand{\baselinepaperunderstanding}{\baselinepaperunderstandingvalue\%\xspace}
\newcommand{\paperunderstandingreductionvalue}{%
  \pgfmathparse{round(\baselinepaperunderstandingvalue/\savaalpaperunderstandingvalue*100 )/100}%
  \pgfmathprintnumber[fixed, precision=1]{\pgfmathresult}%
}

\newcommand{\paperunderstandingreduction}{\paperunderstandingreductionvalue$\times$\xspace}

\newcommand{\savaalpaperchoices}{\savaalpaperchoicesvalue\%\xspace}
\newcommand{\baselinepaperchoices}{\baselinepaperchoicesvalue\%\xspace}
\newcommand{\paperchoicesreductionvalue}{%
  \pgfmathparse{round(\baselinepaperchoicesvalue/\savaalpaperchoicesvalue*100 )/100}%
  \pgfmathprintnumber[fixed, precision=1]{\pgfmathresult}%
}

\newcommand{\paperchoicesreduction}{\paperchoicesreductionvalue$\times$\xspace}

% Num evaluators macros
\newcommand{\numemails}{200\xspace}

\newcommand{\numpaperevaluatorsvalue}{55}
\newcommand{\numphdevaluatorsvalue}{21}

\newcommand{\numpapersvalue}{50}
\newcommand{\numphdvalue}{21}

\newcommand{\numloogle}{48\xspace}
\newcommand{\numfields}{5\xspace}

% Scale of dataset macros
\newcommand{\avgphdpagesvalue}{142}
\newcommand{\avgphdpages}{\avgphdpagesvalue\xspace}
\newcommand{\avgpaperpagesvalue}{19}
\newcommand{\avgpaperpages}{\avgpaperpagesvalue\xspace}


% Math relations
\newcommand{\numphd}{\numphdvalue\xspace}
\newcommand{\numpapers}{\numpapersvalue\xspace}

\newcommand{\totaldocuments}{\pgfmathparse{int(\numphdvalue+\numpapersvalue)}\pgfmathresult\xspace}%

\newcommand{\numpaperevaluators}{\numpaperevaluatorsvalue\xspace}
\newcommand{\numphdevaluators}{\numphdevaluatorsvalue\xspace}

\newcommand{\totalevaluatorsvalue}{\pgfmathparse{int(\numpaperevaluatorsvalue+\numphdevaluatorsvalue)}\pgfmathresult}

\newcommand{\numpaperquestionsvalue}{\pgfmathparse{int(\numpaperevaluatorsvalue*20)}\pgfmathresult}


\newcommand{\numphdquestionsvalue}{\pgfmathparse{int(\numphdevaluatorsvalue*20)}\pgfmathresult}

\newcommand{\totalevaluators}{\totalevaluatorsvalue\xspace}
\newcommand{\numpaperquestions}{\numpaperquestionsvalue\xspace}
\newcommand{\numphdquestions}{\numphdquestionsvalue\xspace}


\newcommand{\numtotalhumanquestionvalue}{\pgfmathparse{int(\numphdevaluatorsvalue*20 + \numpaperevaluatorsvalue*20)}\pgfmathresult}

\newcommand{\numtotalhumanquestion}
{\numtotalhumanquestionvalue\xspace}


\newcommand{\savaalcostreduction}{1.39$\times$\xspace}
\newcommand{\directcostinflation}{1.64$\times$\xspace}

\begin{abstract}

Hypotheses are central to information acquisition, decision-making, and discovery. However, many real-world hypotheses are abstract, high-level statements that are difficult to validate directly. 
This challenge is further intensified by the rise of hypothesis generation from Large Language Models (LLMs), which are prone to hallucination and produce hypotheses in volumes that make manual validation impractical. Here we propose \mname, an agentic framework for rigorous automated validation of free-form hypotheses. 
Guided by Karl Popper's principle of falsification, \mname validates a hypothesis using LLM agents that design and execute falsification experiments targeting its measurable implications. A novel sequential testing framework ensures strict Type-I error control while actively gathering evidence from diverse observations, whether drawn from existing data or newly conducted procedures.
We demonstrate \mname on six domains including biology, economics, and sociology. \mname delivers robust error control, high power, and scalability. Furthermore, compared to human scientists, \mname achieved comparable performance in validating complex biological hypotheses while reducing time by 10 folds, providing a scalable, rigorous solution for hypothesis validation. \mname is freely available at \url{https://github.com/snap-stanford/POPPER}.




\end{abstract}

\section{Introduction}
\label{sec:intro}

Many people learn new material effectively by taking quizzes. Answering questions not only assesses knowledge, but also  reinforces learning by strengthening correct responses and revealing gaps in understanding. A major challenge in the 21st century is the rapid expansion of knowledge across fields like science, technology, medicine, law, finance, and more. AI tools are accelerating this growth, making it increasingly difficult for students, researchers, and professionals---from engineers to salespeople---to stay current. The need to learn efficiently and at scale has never been greater.


One response is to rely on AI for answers, effectively outsourcing expertise. While sometimes necessary, this does little to improve human understanding. Instead, we advocate using AI to enhance {\em our} ability to learn and master new material. %, and our work aims to advance this goal.


Programs like ChatGPT, Gemini, Claude, NotebookLM, Perplexity, and DeepSeek built atop large language models (LLMs) have made remarkable strides in summarization and question-answering. However, less attention has been given to {\em question generation}, specifically, creating high-quality questions that test human understanding and mastery of knowledge. That is the focus of this paper.

Anyone who has made an exam knows how difficult and time-consuming it is to make a good set of questions. Our goal is to produce questions automatically with three objectives:
\begin{CompactEnumerate}
\item {\em Scalability}: Generating questions across vast document corpora, such as rapidly evolving research fields or enterprise knowledge bases.
\item {\em Depth of understanding}: Producing questions beyond memorization and the superficial, requiring conceptual reasoning, synthesis, and analysis.
\item {\em Domain-independence}: Creating high-quality questions across diverse fields, including new material absent in an LLM’s pre-training data.
\end{CompactEnumerate}


Prior work on question generation has produced a small number of questions from short passages, but has not demonstrated scalability~\citep{du-etal-2017-learning, Neural_QG, chan-fan-2019-bert, li-etal-2021-addressing-semantic, knowledge_base_prompting, reading_comprehension_language_llm, code_QG, mcq_mult_sentence}. Our results (\autoref{sec:eval}) show that even well-engineered prompts to an LLM produce poor, repetitive questions on large text contexts (tens to hundreds of pages), highlighting the scalability challenge.
 

We present \textbf{\name}, a scalable question generation system for large documents. Savaal uses a three-stage pipeline. The first stage extracts and ranks the key concepts in a corpus of documents\footnote{We use ``document'' to also refer to the corpus of documents used to generate a quiz.} using a map-reduce computation. The second stage retrieves relevant passages corresponding to each concept with an efficient vector embedding retrieval model such as ColBERT~\cite{colbert}. Finally, the third stage prompts an LLM to generate questions for each ranked concept using the retrieved passages as context.

This approach scales well because each LLM computation is confined to a distinct, self-contained task while operating within a manageable context size. By first identifying core concepts and later synthesizing questions from all relevant passages, \name ensures that the generated questions are both targeted and conceptually rich, requiring deeper understanding by linking a given concept across different sections of a document.

We compare \name to a direct-prompting baseline (\Baseline) using \totalevaluators human expert evaluators (the primary authors of \numpapers recent conference papers and \numphd PhD dissertations in subfields of computer science and aeronautics) on \numtotalhumanquestion questions. We also evaluate each paper, as well as 48 arXiv papers, using an LLM as an AI judge.
%and also using an LLM judge on \numloogle papers from the Loogle \cite{loogle} dataset and \arxiv. 
%These evaluations span \numfields distinct fields. 
We find that: 
\begin{CompactEnumerate}
    \item On \numphdquestions questions from \numphdevaluators large documents (dissertations with average \avgphdpages pages), experts reported that \baselinephdunderstanding of \Baseline's questions {\em did not} test understanding, compared to \savaalphdunderstanding of \name, a \phdunderstandingreduction improvement. 
    They reported that \baselinephdchoices of \Baseline's questions lacked good choice quality, compared to \name's \savaalphdchoices, improving by \phdchoicesreduction. They found \baselinephdusability of \Baseline's questions {\em unusable} in a quiz, compared to \savaalphdusability of \name's questions, a \phdusabilityreduction reduction. Moreover, among experts with a preference, \understandingpreferration more favored \name over baseline in understanding, \choicespreferration in choice quality, and \usabilitypreferration in usability.


    \item Even on shorter documents, experts rated \name better in terms of depth of understanding and usability. On \numpaperquestions questions from \numpapers conference papers, \numpaperevaluators experts reported that \baselinepaperunderstanding of baseline's questions {\em did not} test understanding, compared to \savaalpaperunderstanding of \name, a \paperunderstandingreduction improvement. 
    

    
    \item \name is less expensive than \Baseline as the number of questions grows: \Baseline's cost for 100 questions generated from the dissertations is \directcostinflation higher than \name (\$0.47 vs. \$0.77 on average per document).

    
    \item There is a large gap between AI judgments and human evaluations. Despite several attempts to align the AI judge to human responses, scores remained misaligned.% with human expert evaluations.

\end{CompactEnumerate}



\section{Why is Generating Good Questions Hard?}
\label{sec:insights}

Our goal is to enhance human learning from large documents spanning dozens to hundreds of pages by generating multiple-choice questions. Multiple-choice questions are widely used in assessments, are easy to use by learners, and are easy to grade. The task involves generating a set of clear questions, each with four  choices and a correct answer.

High-quality questions assess {\em depth of understanding}, requiring conceptual reasoning and plausible choices (distractors) that challenge the learner. Beyond clarity and precision, our notion of a good question is one that could appear in an advanced quiz on the material as judged by a human expert. While this paper focuses on generating individual high-quality questions, effective quiz sessions should ensure {\em concept coverage} and {\em adapting the difficulty} to prior answers in the session, both avenues for future work.


The main challenge in scalable question generation using LLMs is selecting an appropriate context to use with LLM prompts. We examine four potential strategies: (i) using the full document corpus, (ii) dividing the corpus into sections, (iii) summarizing the corpus, and (iv) using content selection classifiers~\citep{context_Steuer, Context_diverse_hadifar}. Although each strategy has merits, we show that each strategy fails on at least one of our key objectives: {\em scalability}, {\em depth of understanding}, or {\em domain-independence}.

\newcommand{\questionbox}[1]{%
    \colorbox{customblue}{\parbox{0.97\linewidth}{\vspace{1pt}\textbf{#1}\vspace{1pt}}}
}

\begin{table*}[t]
\centering
\renewcommand{\arraystretch}{1.0} % Adjust row height for better readability
\begin{tabular}{|p{0.2\textwidth}|m{0.5\textwidth}|m{0.2\textwidth}|} 
\hline
\rowcolor{gray!20} % Light gray background for header row
\textbf{Context} & \textbf{Generated Question} & \textbf{Issue} \\ \hline
\footnotesize
\sfboxed{1} \textbf{Entire Document} & 
\footnotesize
\questionbox{What is the primary benefit of using the Adam optimizer in training the Transformer model?}
A. It reduces the need for dropout regularization. \newline
\textbf{B.} It adapts the learning rate based on the training step, improving convergence. \newline
C. It eliminates the need for positional encodings. \newline
D. It simplifies the model architecture by reducing the number of layers. 
&
\footnotesize
\textbf{Too general}: The question is about a basic property of the Adam optimizer rather than the key ideas of the paper.

$\Rightarrow$ Does not test depth of understanding
\\ \hline
\footnotesize
\sfboxed{2} \textbf{Document Section}& 
\footnotesize
\questionbox{In evaluating the performance and efficiency of the Transformer (big) model on the WMT 2014 English-to-French translation task, which of the following factors most significantly contributes to its ability to outperform previous models at a reduced training cost?}
A. The use of a dropout rate of 0.1 instead of 0.3, which enhances model regularization and reduces overfitting.\newline
B. The implementation of beam search with a beam size of 4 and a length penalty $\alpha$ = 0.6, which optimizes the translation output quality.\newline
\textbf{C.} The averaging of the last 20 checkpoints, which stabilizes the model's performance and improves translation accuracy.\newline
D. The reduction in training time to less than 1/4 of the previous state-of-the-art model, which directly correlates with improved BLEU scores.&
\footnotesize
\textbf{Irrelevant detail:} Because the method looks at one section at a time, it fixates on minutiae and irrelevant details (e.g., “averaging the last 20 checkpoints”) that may seem important in isolation, but are not.

$\Rightarrow$ Does not test depth of understanding
\\ \hline
\footnotesize
\sfboxed{3} \textbf{Document Summary} &
\footnotesize
\questionbox{How does the Transformer model address the challenge of learning dependencies between distant positions in sequences compared to models like ConvS2S and ByteNet?}
A. By using convolutional layers to capture long-range dependencies\newline
B. By increasing the number of layers in the encoder and decoder stacks\newline
C. By employing a recurrent neural network to process sequences\newline
\textbf{D.} By reducing the number of operations to a constant using self-attention mechanisms"
& 
\footnotesize
\textbf{Missing context:} The summary mentions ``...The Transformer model addresses this by reducing the number of operations to a constant, using self-attention mechanisms.'' which led the LLM design this incomplete question.


$\Rightarrow$ Leads to inaccurate questions
\\ \hline
\end{tabular}
\caption{Examples from the ``Attention Is All You Need'' paper \citep{attention_is_all_you_need} using three different context selection methods. The questions are drawn from three separate 20-question quizzes, each generated using a different method via OpenAI's API \citep{openai_api} with the \texttt{gpt-4o} model.}
\label{tab:bad-examples-attention}
\vspace{-10 pt}
\end{table*}


\subsection{Using the Entire Document Corpus}
\label{sec:insights-whole-context}

One approach is to provide the entire document as context to an LLM for quiz generation. However, this method has two major drawbacks.
First, as prior research shows~\citep{lost-in-the-middle}, LLMs allocate attention unevenly across long documents, focusing more on the beginning and end while largely neglecting the middle. 

Second, LLMs struggle to capture dependencies between different sections of a long document~\citep{loogle}, leading to superficial questions and missing key concepts. When we prompted OpenAI's \texttt{gpt-4o} model with the full text of the ``Attention Is All You Need'' paper~\citep{attention_is_all_you_need}, many of the 20 generated questions overlooked key ideas. See Example \sfboxed{1} in \autoref{tab:bad-examples-attention} for a question, which is not relevant to the paper's key ideas.

We found that LLMs struggle to follow instructions when the context length is large~\cite{gao2024insights}. For example, we instruct the LLM not to repeat questions. While it avoids repetition when generating a few questions, larger batches (e.g., 20 questions) often contain duplicates. 


\subsection{Using Document Sections}
\label{sec:insights-section-context}

An alternative is to split the document into sections, generate a limited number of questions per section, and later combine them into a quiz. While this method mitigates long-context issues, it introduces {\em context fragmentation}: the LLM cannot connect concepts spanning multiple sections. It often misses deeper connections that can assess stronger conceptual understanding. For example, key insights in a paper’s Algorithm or Methods section may be essential for understanding its Results, but treating these sections independently leads to disjointed, narrow questions.

Another issue is {\em uneven importance weighting}. Not all sections contribute equally to the document’s  ideas, yet a naïve section-based approach may overemphasize minor details while missing key concepts. Example \sfboxed{2} in \autoref{tab:bad-examples-attention} shows how this can generate irrelevant memorization questions.


\subsection{Summarization}
\label{sec:insights-summary}

Providing a {\em document summary} as context offers another way to streamline question generation. While LLMs are effective at summarization, summaries often lack critical details, leading to vague or incomplete questions. More concerning, summaries can introduce hallucinations~\citep{llm_hallucination}, distorting or misrepresenting causal relationships and fabricating details, further degrading question quality.

Example \sfboxed{3} in \autoref{tab:bad-examples-attention} illustrates how summarization can result in misleading or imprecise questions. Here, the summary includes a statement about using self-attention to ``reduce the number of operations to a constant'', but omits that this refers to {\em sequential} operations and maximum path length (Sec. 4 of \citep{attention_is_all_you_need}), leading to an inaccurate question. 



\subsection{Content Selection Classifiers}

Some methods attempt to select relevant content for question generation, often using trained models to identify key passages~\citep{context_Steuer, Context_diverse_hadifar}. However, these approaches typically require domain-specific training data (e.g., pre-existing question-answer pairs), making them {\em domain-dependent}. Such approaches are frequently limited in scope, making them neither reliable nor generalizable to diverse domains. 

\begin{figure*}[!t]
\centering
    \includegraphics[width=1\linewidth]{FIG/savaal.drawio.pdf}
    \caption{\name's Pipeline. \bluecircle\ \name extracts main ideas from sections of the document in parallel, \redcircle\ combines them into a succinct list, and \greencircle\ ranks them in order of importance. Next, \purplecircle\ \name fetches relevant passages from the document using a vector-based retrieval model. Finally, \orangecircle\ given a main idea and fetched passages, \name generates questions.}
    \label{fig:savaal-workflow}
\vspace{-10 pt}
\end{figure*}

\section{\name's Question-Generation Pipeline}
\label{sec:pipeline}


To address challenges of \autoref{sec:insights}, we propose a novel three-stage pipeline: \emph{main idea extraction}, \emph{relevant passage retrieval}, and \emph{question generation}. \autoref{fig:savaal-workflow} shows Savaal's workflow. The idea is to generate questions targeted at key explicitly determined concepts and to retrieve passages relevant to the concept from the source document.
% to generate questions.


\subsection{Extracting Main Ideas}
\label{sec:pipeline-main-idea}
This stage extracts succinct main ideas from different document chapters. This is done in a map-combine-reduce fashion~\cite{langchain_mapreduce}. First, we use GROBID~\citep{GROBID} to parse and segment documents into distinct sections.


In the map stage, \circled{1}{customblue}, we use an LLM to extract the main ideas for each section individually. These extracted main ideas are aggregated and deduplicated in the combine stage, \circled{2}{customred}, into a single, cohesive list of the paper’s main ideas. If the combined output exceeds a predefined length threshold (set to the maximum token window of the LLM), the reduce stage collapses the list further for brevity and clarity. The result is a curated list of main ideas, including main idea titles and their short descriptions (see \autoref{subsubsec:example_main_idea} for examples). The same (or a different) LLM then ranks the main ideas based on their importance in the ranking stage in \circled{3}{customgreen} (see \autoref{subsec:appendix_prompts} for the prompts).

Initially, we attempted to extract the main ideas for the entire document in one shot. However, as noted in \autoref{sec:insights-whole-context}, as the context length grew, this became less effective. We found that using map-reduce extracted main ideas that were more detailed and useful for question generation, particularly on large documents.



\subsection{Retrieving Relevant Passages}
\label{sec:pipeline-retrieval}

Because the main ideas in \autoref{sec:pipeline-main-idea} are concise, they lack sufficient content to generate a question. As discussed in \autoref{sec:insights-summary}, asking an LLM to generate questions based on a concept alone (a main idea or even a summary) has shortcomings. To overcome this problem, \name retrieves relevant text segments directly from the original document to provide granular content for generating a question and to ensure that the questions are grounded in truth.


\name's retriever uses ColBERT, a late-interaction retrieval method~\citep{colbert, colbertv2}, to find the most relevant passages for each main idea (stage \circled{4}{custompurple}).
% integrated in the RAGatouille \footnote{\url{https://github.com/AnswerDotAI/RAGatouille}} library.
For each ranked main idea in \circled{3}{customgreen}, we retrieve the top $k$ passages as added context for the next stage ($k=3$ in our experiments).

We chose ColBERT for its state-of-the-art performance and wide adoption, but any high-performing retrieval method could be used. We also tried using the LLM to identify passages related to a main idea, but as in \autoref{sec:insights-whole-context} and \autoref{sec:pipeline-main-idea}, it struggled with large context sizes.


\subsection{Generating Questions and Choices}
\label{sec:pipeline-QG}


After retrieving the passages for each main idea, stage \circled{5}{customorange} instructs an LLM to generate questions. To create $N$ questions from $M$ ideas, we generate $N/M$ questions per idea.\footnote{We use only the top $N$ ranked main ideas if $N < M$.}  The prompt (\autoref{fig:question_generation}) includes the main idea and its retrieved passages.


Although LLMs often produce good questions, generating good {\em choices} is more challenging. Poorly designed choices can make the correct answer too obvious or, worse, introduce ambiguity or multiple correct options. We experimented with many prompt variations to improve choice quality, yielding mixed results. We also tested a separate ``choice refinement'' stage, where an LLM was specifically instructed to improve the answer choices for a given question. This prompt included detailed constraints, such as ensuring alignment with the question's intent (e.g., a question about benefits should not include limitations as choices; see \autoref{appendix:choice-refine}).
Although this additional step produced more challenging choices, we found that it caused excessive ambiguity and was less preferred by human expert evaluators. Therefore, \name does not include a choice refinement stage. Instead, its question-generation prompt explicitly emphasizes that the choices should be ``plausible distractors''.

Finally, we observed {\em positional biases} in the placement of the correct choice, corroborating prior findings~\cite{pezeshkpour2023large}. For example, in a set of 1000 questions from 50 papers (20 per paper) generated by \texttt{GPT-4o}, choice B was correct 73.3\% of the time! Thus, we randomize the choices to eliminate this bias.

\section{Evaluation}
\label{sec:eval}

We evaluated \name on \totaldocuments documents using both human experts and an AI judge. We used \texttt{GPT-4o} via the OpenAI API as our primary LLM. We also evaluated \texttt{Meta-Llama-3.3-70B-Instruct} (\autoref{subsec:ablation-model}). All models are set to temperature 0.0 for all experiments, with default settings for all other parameters. \name is model-agnostic and is compatible with current LLMs. We implemented our pipeline using LangChain~\cite{langchain} and traced our experiments in Weights \& Biases~\cite{wandb}.
%any commercial or open-source LLM.

\subsection{Datasets}
\label{sec:evaluation-data}
%We gathered three datasets for evaluation:
\begin{itemize}[topsep=0pt, itemsep=0pt, leftmargin=*]
    \item \textbf{PhD dissertations}: \numphd long documents in Aerospace, Machine Learning, Networks, Systems, and Databases (\autoref{tab:human-eval-dataset-stats}).
    \item \textbf{Conference papers}: \numpapers papers from conferences in CS and Aeronautics in 2023 and 2024.
    \item \textbf{Diverse \arxiv papers}: \numloogle papers from CS, Physics, Mathematics, Economics, and Biology (\autoref{tab:benchmark-stats}). 
\end{itemize}

\begin{table}[h]
\centering
\renewcommand{\arraystretch}{1} % Increase row height for better readability
\setlength{\tabcolsep}{1pt} % Adjust column spacing for better fit
% \begin{small}
\begin{tabular}{|l|c|c|}
\hline
\small  \textbf{Statistic} & \small  \textbf{Conference Papers} & \small \textbf{Dissertations} \\
\hline
\small \textbf{No. Documents} & \small \numpapers & \small \numphd \\ 
\hline
\small \textbf{Avg. Words} & \small 10,354 & \small 26,511 \\ 
\hline
\small \textbf{Avg. Pages} & \small \avgpaperpages & \small \avgphdpages \\
\hline
\end{tabular}
\caption{Statistics for the number of words in the conference papers and PhD dissertations.}
\label{tab:human-eval-dataset-stats}
% \end{small}
\end{table}



\subsection{Methods Compared}
\label{sec:evaluation-baselines}
We compare \name to \Baseline, a direct-prompting baseline (\autoref{sec:insights-whole-context}) that provides the entire document to the LLM with a detailed prompt to generate $N$ multiple-choice questions (\autoref{fig:baseline_question_generation_prompt}). We found that when $N$ exceeds $\approx$ 20, \Baseline fails to produce $N$ distinct questions. Since broad concept coverage requires generating a large pool of questions and sampling for shorter quizzes, we generate $N > 20$ questions in batches of $b=20$ using an additional prompt (\autoref{fig:baseline_large_question_generation_prompt}). We use this {\em multi-turn method} for \Baseline on longer documents. 


We evaluate other methods using the AI judge: Summary (\autoref{sec:insights-summary}) and Single-Prompt Savaal, which condenses Savaal's idea extraction, retrieval, and question generation into a single prompt (\autoref{subsec:ablation-methods}).


\subsection{Evaluation Criteria}
\label{sec:evaluation-metrics}

Evaluating the quality of questions is challenging because it involves subjective human judgment~\cite{fu2024qgeval}. We primarily rely on human evaluations but also use \texttt{GPT-4o} as an AI judge~\cite{naismith2023automated} to expand the scope of our evaluation to more approaches, documents, and criteria. 

% Exempt ID: E-6417

\paragraph{Human experts:} We generated 10 multiple-choice questions from Savaal and 10 from \Baseline for each of the \numphd PhD dissertations and \numpapers conference papers. We contacted the primary authors to evaluate the quality of questions via a secure web-based feedback form.\footnote{\emph{MIT} Institutional Review Board exempted this study (Exemption Number: E-6417). All the personnel were certified, and participants were over 18 years of age and provided informed consent before participating.} We asked each expert to rate their questions on clarity, depth of understanding\footnote{Used interchangeably with understanding.}, and quality of choices using a four-point scale: \emph{Disagree}, \emph{Somewhat Disagree}, \emph{Somewhat Agree}, and \emph{Agree}. They also assessed usability by answering: ``Would I use this question on a graduate-level quiz?'' with options: {\em Yes}, {\em Yes with small changes}, and {\em No}. The questions were randomly mixed and the evaluators were blind to their source. In all, \totalevaluators experts participated (\autoref{subsec:appendix_human_eval_conduct}).
 

\label{sec:metrics-auto}
\paragraph{AI judge:} We prompted \texttt{GPT-4o} at temperature 0.0 to score each question on a 1–4 scale (\autoref{subsubsec:eval-prompts}) on Depth of Understanding, Quality of Choices, Clarity, Usability, Difficulty, Cognitive Level, and Engagement (\autoref{subsec:ablation-metrics}). Our evaluation prompts provide detailed guidelines than those given to humans, including explicit criteria for each rating (\autoref{subsubsec:eval-prompts}).



\subsection{Results with Human Experts}
\label{sec:evaluation-results}

\begin{figure}[!t]
    \centering
    \begin{subfigure}[b]{0.9\linewidth}
        \centering
        \includegraphics[width=1\linewidth]{FIG/thesis_only_disagree_all_metrics_bar_charts.pdf}
        \caption{PhD dissertations: \numphdquestions questions, \numphdevaluators experts.}
        \label{fig:human-eval-disagree-phd}
    \end{subfigure}
    \hfill
    \begin{subfigure}[b]{0.9\linewidth}
        \centering
        \includegraphics[width=1\linewidth]{FIG/no_dedup_no_refine_final_only_disagree_all_metrics_bar_charts.pdf}
        \caption{Conference papers: \numpaperquestions questions, \numpaperevaluators experts.}
        \label{fig:human-eval-disagree-paper}
    \end{subfigure}
    \vspace{-10 pt}
    \caption{ Summary of human evaluation: The charts show the percentage and standard error of respondents who {\em Disagree} or {\em Somewhat Disagree} with questions on understanding, choice quality, and usability. {\bf Lower values indicate better performance.}}
    \label{fig:human-eval-disagree}
\vspace{-20 pt}
\end{figure}

\begin{figure*}[!t]
    \centering
    \begin{subfigure}[b]{0.32\linewidth}
        \centering
        \includegraphics[width=\linewidth]{FIG/thesis_understanding_num_AGREE_scatter.pdf}
        \caption{Depth of understanding: 61.9\% prefer \name, 9.5\% \Baseline.}
        \label{fig:thesis-scatter-understanding}
    \end{subfigure}
    \hfill
    \begin{subfigure}[b]{0.32\linewidth}
        \centering
        \includegraphics[width=\linewidth]{FIG/thesis_quality_of_choices_num_AGREE_scatter.pdf}
        \caption{Quality of choices: 57.1\% prefer \name, 19\% \Baseline.}
        \label{fig:thesis-scatter-choices}
    \end{subfigure}
    \hfill
    \begin{subfigure}[b]{0.32\linewidth}
        \centering
        \includegraphics[width=\linewidth]{FIG/thesis_overall_quality_num_AGREE_scatter.pdf}
        \caption{Usability: 47.6\% prefer \name, 23.8\% \Baseline.}
        \label{fig:thesis-scatter-overall}
    \end{subfigure}
    
    \caption{Expert preferences for \numphdevaluators PhD dissertations. Each point shows the number of \emph{Agree}s or \emph{Somewhat Agree}s in a 10-question quiz for each of \name and \Baseline. The majority of experts prefer \name to \Baseline on depth of understanding, quality of choices, and usability on long documents (experts above $y=x$ prefer \name).}
    \label{fig:human-eval-scatter}
    \vspace{-10 pt}
\end{figure*}


\label{sec:evaluation-human}

\autoref{fig:human-eval-disagree} summarizes the results of our expert human evaluation on PhD dissertations and papers. We show here the negative sentiment of the experts, i.e., the percentage of questions that experts responded with \emph{Disagree} or \emph{Somewhat Disagree} for each criterion (see \autoref{fig:human-phd-breakdown} and \autoref{fig:human_paper_breakdown} for the full breakdown). 

For the \numphdquestions questions from \numphd PhD dissertations (\autoref{fig:human-eval-disagree-phd}), the experts responded that \baselinephdunderstanding of \Baseline's questions {\em did not test understanding}; by contrast, only \savaalphdunderstanding of \name's questions did not, a  \phdunderstandingreduction reduction in negative sentiment. They also rated \baselinephdusability of \Baseline's questions as {\em unusable in a quiz}, versus \savaalphdusability for \name, a \phdusabilityreduction reduction.

For  conference papers (\autoref{fig:human-eval-disagree-paper}), on \numpaperquestions questions, \numpaperevaluators experts\footnote{Some papers had multiple expert respondents.} found that \savaalpaperunderstanding of \name's questions {\em did not} test understanding, versus \baselinepaperunderstanding for \Baseline, a \paperunderstandingreduction improvement. They also rated \baselinepaperusability of \Baseline's questions as {\em unusable}, versus \savaalpaperusability for \name.

The experts agreed or somewhat agreed that over 90\% of the questions in both \Baseline and \name had clarity (not shown in the figure). This result is unsurprising because LLMs can be prompted to generate coherent and unambiguous text. 

For PhD dissertations, \autoref{fig:human-eval-scatter} shows how each of the \numphdevaluators experts scored \name vs. \Baseline on the metrics for the PhD dissertations. The $x$ and $y$ axes show number of \emph{Agree} or \emph{Somewhat Agree} for \Baseline and \name, respectively. Each point represents one expert evaluator. 

We observe that \savaalunderstandingpreferrationpercent favor \name over \Baseline for understanding (\autoref{fig:thesis-scatter-understanding}), whereas only \baselineunderstandingpreferrationpercent (\understandingpreferration fewer) prefer \Baseline over \name (\sameunderstandingpreferrationpercent rate the two systems the same). For choice quality, \savaalchoicespreferrationpercent prefer \name compared to \baselinechoicespreferrationpercent for \Baseline (\choicespreferration more, see \autoref{fig:thesis-scatter-choices}), while for usability \savaalusabilitypreferrationpercent prefer \name compared to \baselineusabilitypreferrationpercent for \Baseline (\usabilitypreferration more, see \autoref{fig:thesis-scatter-overall}). 

The data in \autoref{fig:human-eval-scatter} also shows that, on average, expert evaluators rated \emph{Agree} or \emph{Somewhat Agree} for more questions in \name quizzes than \Baseline: \phdunderstandingWAD more for understanding, \phdchoicesWAD more for quality of choices, and \phdusabilityWAD more for usability.

\autoref{fig:human_paper_breakdown} shows the breakdown of expert responses for \numpaperquestions questions from the conference papers. On these shorter documents, experts slightly prefer \name over \Baseline in terms of depth of understanding. They reported that 16.7\% of \TheSystem's questions {\em did not} test understanding, compared to 10.9\% for \Baseline. Experts rated the two methods similarly for choice quality and usability. As in the results for Ph.D. dissertations (\autoref{fig:human-auto-correlation}), the \texttt{GPT-4o} scores (\autoref{fig:AI_paper_breakdown}) correlated poorly with expert evaluations.


\autoref{fig:paper-human-eval-scatter} shows how each of the \numpaperevaluators experts scored \name vs. \Baseline. The $x$-axis shows the number of \emph{Agree} or \emph{Somewhat Agree} for \Baseline, and the $y$-axis shows the same for \name. Each point represents one expert evaluator. Among evaluators with a preference, 1.5$\times$ more experts favor \TheSystem over \Baseline in understanding (34.5\% for \name vs 21.8\% for \Baseline, \autoref{fig:paper-scatter-understanding}). Experts do not exhibit a strong preference between \name and \Baseline for choice quality (\autoref{fig:paper-scatter-choices}) or usability (\autoref{fig:paper-scatter-overall}). The average relative increase in the Agree score for \TheSystem compared to \Baseline is 5.8\% for understanding, 4\% for quality of choices, and 1.5\% for usability.
% , meaning that on average, experts like at least one more question in \name's quizzes compared to \Baseline.


\begin{figure*}[h]
    \centering
    \begin{subfigure}[b]{0.32\linewidth}
        \centering
        \includegraphics[width=\linewidth]{FIG/no_dedup_no_refine_final_understanding_num_AGREE_scatter.pdf}
        \caption{Depth of understanding: 34.5\% prefer \name, 21.8\% prefer \Baseline.}
        \label{fig:paper-scatter-understanding}
    \end{subfigure}
    \hfill
    \begin{subfigure}[b]{0.32\linewidth}
        \centering
        \includegraphics[width=\linewidth]{FIG/no_dedup_no_refine_final_quality_of_choices_num_AGREE_scatter.pdf}
        \caption{Quality of choices: no specific preference exhibited.}
        \label{fig:paper-scatter-choices}
    \end{subfigure}
    \hfill
    \begin{subfigure}[b]{0.32\linewidth}
        \centering
        \includegraphics[width=\linewidth]{FIG/no_dedup_no_refine_final_overall_quality_num_AGREE_scatter.pdf}
        \caption{Usability: no specific preference exhibited.}
        \label{fig:paper-scatter-overall}
    \end{subfigure}
    
    \caption{Human expert preferences for \numpaperevaluators experts on short conference papers. Each point shows the number of \emph{Agree}s in a 10-question quiz for \name and \Baseline respectively. More experts prefer \name to \Baseline on the depth of understanding. Experts don't exhibit any preference between the quality of choices and usability on short documents (experts above $y=x$ prefer \name).}
    \label{fig:paper-human-eval-scatter}
\end{figure*}


\subsection{Results with an AI Judge}
\label{sec:evaluation-auto}


We used an AI judge to scale evaluations across more documents and criteria. We first examined its alignment with human experts by having \texttt{GPT-4o} evaluate the same \numphdquestions questions from the expert-reviewed dissertations dataset. 

\autoref{fig:human-auto-correlation} compares the AI judge with human experts. The AI judge rarely assigns \emph{Disagree} or \emph{Somewhat Disagree} for understanding and usability and slightly favors \name, giving it 28.6\% Agree rating in comparison to 14.3\% Agree ratings for \Baseline for understanding. However, for quality of choices, it rates both schemes poorly, with only 9.6\% \emph{Agree} or \emph{Somewhat Agree} for \name and 19\% for \Baseline.

We observed similar trends in the \numpaperquestions questions from the conference-paper dataset (\autoref{fig:paper_breakdown}), where the AI judge again slightly preferred \name but remained misaligned with human expert evaluations. For completeness, we also present AI judge results on the Diverse \arxiv dataset in \autoref{subsec:ablation}.

%\paragraph{Limitations of the AI judge.} 
Our takeaway is that our \texttt{GPT-4o} AI judge was unaligned with human expert judgments (see \autoref{fig:auto-correlation-ai} vs. \autoref{fig:human-phd-breakdown}). Despite our extensive efforts in prompt engineering to maximize alignment---including using the prompt optimizer program in DSPy~\citep{khattab2024dspy}---AI-human correlation did not improve. Our experience calls into question the wisdom of using only AI judges in research studies. 




\begin{figure}[t]
    \centering
    \begin{subfigure}[b]{1\linewidth}
        \centering
        \includegraphics[width=1\linewidth]{FIG/thesis_no_combine_all_metrics_bar_charts.pdf}
        \caption{Breakdown of human expert scores on PhD dissertations.}
        \label{fig:human-phd-breakdown}
    \end{subfigure}  
    \hfill
    \begin{subfigure}[b]{1\linewidth}
        \centering
        \includegraphics[width=1\linewidth]{FIG/FINAL_PLOTS_AUTO/thesis_human_auto_all_metrics_bar_charts.pdf}
        \caption{Breakdown of GPT-4o AI judge scores on PhD dissertations.}
        \label{fig:auto-correlation-ai}
    \end{subfigure}
    \caption{Score distribution for  \protect\numphdquestions questions from dissertations: GPT-4o as a judge does not align with humans for assessing the metrics.}
    \label{fig:human-auto-correlation}
    \vspace{-20 pt}
\end{figure}


\begin{figure}[h]
    \centering
    \begin{subfigure}{\linewidth}
        \centering
        \includegraphics[width=\linewidth]{FIG/no_dedup_no_refine_final_no_combine_all_metrics_bar_charts.pdf}
        \caption{Breakdown of human expert scores on conference papers.}
        \label{fig:human_paper_breakdown}
    \end{subfigure}
    \hfill
    \begin{subfigure}{\linewidth}
        \centering
        \includegraphics[width=\linewidth]{FIG/FINAL_PLOTS_AUTO/papers_human_auto_all_metrics_bar_charts.pdf}
        \caption{Breakdown of GPT-4o scores on conference papers.}
        \label{fig:AI_paper_breakdown}
    \end{subfigure}
    \caption{Score distribution for  \protect \numpaperquestions questions from conference papers.}
    \label{fig:paper_breakdown}
    \vspace{-10 pt}
\end{figure}





\subsection{Cost Scalability}
\label{sec:scalability-case}

\autoref{fig:cost-scalability} compares the costs of \name and \BaselineMT on the dissertations. While \name incurs a higher one-time cost to generate the concepts, it becomes less expensive when generating more questions. At $N = 60$ questions, \name has the same cost as \BaselineMT; when $N$ grows to 100 questions, \BaselineMT is \directcostinflation more expensive. 

% Details are in \autoref{appendix:costs}.

\begin{figure}[h]
    \centering
    \includegraphics[width=0.8\linewidth]{FIG/phd_cost_comparison.pdf}
    \caption{Average cost comparison of \BaselineMT and \name when generating questions from \numphd PhD dissertations. \name becomes less expensive as $N$ grows. We calculated costs by tracing prompt and completion tokens with OpenAI's February 2025 API pricing.}
    \label{fig:cost-scalability}
    \vspace{-20 pt}
\end{figure}

% \subsection{Discussion of Cost Scalability}
% \label{appendix:costs}

\name is also more cost-effective as the size of the document, $D$, grows. \BaselineMT costs $\approx \frac{N}{b} \cdot (A \cdot D + 100b \cdot B)$, where $A$ is cost per input token, $B$ is cost per output token, $N$ is the number of questions, $b$ is the batch size of \BaselineMT, and $100b$ is the approximate number of output tokens when generating $b$ questions. By contrast, \name costs $\approx f(D) + 100NB$ where $f(D)$ is the cost of main idea extraction, and $N$ is the number of questions. Thus, \name incurs a fixed cost that depends on the size of the document, but the marginal cost of generating additional questions is then independent of document size. By contrast, \Baseline incurs additional input token cost of $AD$ for each  batch of generated questions. 

In our experiments, for a PhD dissertation, $f(D) \approx 1.48A \cdot D$ on average.  Therefore, \name has lower cost when $\frac{N}{b} > 1.48$. For $N = 100$, \Baseline requires $b \approx 67$ to incur the same cost as \name, which is impractical with current LLMs. Both \texttt{GPT-4o} and \texttt{Meta-Llama-3.3-70B-Instruct} do not reliably generate more than $\approx$ 20 questions in a batch. 



In \autoref{fig:cost-scalability}, we also notate \BaselineMT with caching. Prompt caching is a feature made available from various LLM providers. It works by matching a prompt prefix, like a long system prompt or other long context from previous multi-turn conversations, to reduce computation time and API costs. As of writing in February 2025, the OpenAI API charged 50\% less for cached prompt tokens, resulting in up-to 80\% latency improvements. The \BaselineMT method benefits from this caching scheme, as it repeatedly sends the entire document as a cache prefix to the API. As shown in \autoref{fig:cost-scalability}, \BaselineMT is more cost-effective than \name up until $N \approx 80$ with prompt caching, as opposed to $N \approx 60$ without prompt caching.

However, prompt caching has several limitations. First, many providers evict cache entries after a short period of time, around 5-10 minutes. Thus, all $N$ questions must be generated within a set time frame to benefit. Moreover, many open-source model providers do not include prompt caching as a feature (as of the time of writing). Therefore, while we present the benefits that prompt caching may provide \BaselineMT, we still demonstrate that \name is more cost effective at large scale.

\section{Related Work}
\label{sec:related-work}

\textbf{Automated question-generation} has evolved from early Seq2Seq models \cite{du-etal-2017-learning, Neural_QG} to transformer-based approaches \cite{attention_is_all_you_need}. Models like BERT \cite{devlin-etal-2019-bert}, T5 \cite{t5}, BART \cite{lewis-etal-2020-bart}, and GPT-3 \cite{gpt_3} have significantly improved question generation \cite{chan-fan-2019-bert, li-etal-2021-addressing-semantic}. However, reliance on labeled datasets such as SQuAD \cite{rajpurkar-etal-2016-squad} and HotpotQA \cite{yang-etal-2018-hotpotqa} limits generalizability to other domains.

Researchers have explored LLMs for question generation~\cite{knowledge_base_prompting, reading_comprehension_language_llm, code_QG, mcq_computing, MIT_law}. However, these efforts have focused on generating questions from short, domain-specific context. Our work mitigates this limitation and generates high-quality questions from long documents.


Prior methods for \textbf{automated evaluation using LLMs} use metrics like ROUGE \cite{lin-2004-rouge} and BLEU \cite{papineni-etal-2002-bleu}, but these often misalign with humans~\cite{guo2024survey-neural-question-gen}. Some papers fine-tune small models for specific metrics \cite{zhu2023judgelm, wang2024pandalm}, but they face scalability issues, annotation reliance, or poor generalizability \cite{zhu2023judgelm}. Recent work uses  LLMs like GPT-4o as evaluators \cite{zheng2023judging, lin2023llmevalunifiedmultidimensionalautomatic}. While they achieve good human alignment, they focus on multi-turn conversations, a different context from ours.

For multiple-choice question generation, small models like BART and T5 assess relevance and usability \cite{moon-etal-2024-generative, raina2022multiplechoicequestiongenerationautomated} but require ground-truth data, limiting scalability. Others use LLM judges to rate relevance, coverage, and fluency on a 1-5 Likert scale \cite{microsoft_agriculture}. 
We adopt a similar approach with GPT-4o on a 1-4 scale. 
%for better human alignment. 
LLM judges can introduce positional \cite{zheng2024large, wang-etal-2024-large-language-models-fair}, egocentric \cite{koo-etal-2024-benchmarking}, and misinformation biases \cite{chen-etal-2024-humans, koo-etal-2024-benchmarking}.
%, which require mitigation.


\textbf{Retrieval-Augmented Generation (RAG)} enhances language model accuracy by retrieving relevant information to ground responses and reduce hallucinations~\cite{lewis2020retrieval, shuster2021retrieval, colbertv2, gottumukkala2022investigating}. Advances like dense passage retrieval~\cite{karpukhin2020dense} and late interaction models~\cite{colbert} improve efficiency. \name's pipeline uses recent advances in information retrieval models to fetch the most relevant context for question generation.

\section{Conclusion and Future Work}
\label{sec:concl}

\name uses LLMs and RAG in a concept-driven, three-stage framework to generate multiple-choice quizzes that assess deep understanding of large documents. Evaluations with \totalevaluators experts on \totaldocuments papers and dissertations show that, among those with a preference, \name outperforms a direct-prompting LLM baseline by 6.5$\times$ for dissertations and 1.5$\times$ for papers. Additionally, as document length increases, \name's advantages in question quality and cost efficiency become more pronounced.


We now discuss several avenues for future work.
While \name generates conceptual questions that test depth of understanding, few of them require mathematical analysis, logical reasoning, or creative thinking. \name produces quiz sessions, but we have not yet evaluated session quality. Currently, \name has not utilized human feedback to improve, which could be done using direct-preference optimization (DPO)~\cite{dpo}, Kahneman-Twersky Optimization (KTO) \cite{kto}, or reinforcement learning with human feedback (RLHF) \cite{rlhf}. To help learners, \name should adapt the difficulty of questions to the learner's answering accuracy and the time to answer questions. 


Our attempts to align AI-generated evaluations with human expert judgments have been unsuccessful. Further research is necessary to improve AI judges in educational contexts.
Finally, validating \name's domain-independence requires testing across a broader spectrum of fields. 
\section{Limitation}
The use of 3D-printed PLA for structural components improves improving ease of assembly and reduces weight and cost, yet it causes deformation under heavy load, which can diminish end-effector precision. Using metal, such as aluminum, would remedy this problem. Additionally, \robot relies on integrated joint relative encoders, requiring manual initialization in a fixed joint configuration each time the system is powered on. Using absolute joint encoders could significantly improve accuracy and ease of use, although it would increase the overall cost. 

%Reliance on commercially available actuators simplifies integration but imposes constraints on control frequency and customization, further limiting the potential for tailored performance improvements.

% The 6 DoF configuration provides sufficient mobility for most tasks; however, certain bimanual operations could benefit from an additional degree of freedom to handle complex joint constraints more effectively. Furthermore, the limited torque density of commercially available proprioceptive actuators restricts the payload and torque output, making the system less suitability for handling heavier loads or high-torque applications. 

The 6 DoF configuration of the arm provides sufficient mobility for single-arm manipulation tasks, yet it shows a limitation in certain bimanual manipulation problems. Specifically, when \robot holds onto a rigid object with both hands, each arm loses 1 DoF because the hands are fixed to the object during grasping. This leads to an underactuated kinematic chain which has a limited mobility in 3D space. We can achieve more mobility by letting the object slip inside the grippers, yet this renders the grasp less robust and simulation difficult. Therefore, we anticipate that designing a lightweight 3 DoF wrist in place of the current 2 DoF wrist allows a more diverse repertoire of manipulation in bimanual tasks.

Finally, the limited torque density of commercially available proprioceptive actuators restricts the performance. Currently, all of our actuators feature a 1:10 gear ratio, so \robot can handle up to 2.5 kg of payload. To handle a heavier object and manipulate it with higher torque, we expect the actuator to have 1:20$\sim$30 gear ratio, but it is difficult to find an off-the-shelf product that meets our requirements. Customizing the actuator to increase the torque density while minimizing the weight will enable \robot to move faster and handle more diverse objects.

%These constraints highlight opportunities for improvement in future iterations, including alternative materials for enhanced rigidity, custom actuator designs for higher control precision and torque density, the adoption of absolute joint encoders, and optimized configurations to balance dexterity and weight.



\section*{Acknowledgments}
We thank all the expert evaluators for their time, insights, and feedback. This work was funded in part by Quanta Computer, Inc. under the AIR Project.

\newpage
% \bibliography{refs}
% This must be in the first 5 lines to tell arXiv to use pdfLaTeX, which is strongly recommended.
% This must be in the first 5 lines to tell arXiv to use pdfLaTeX, which is strongly recommended.
\pdfoutput=1
% In particular, the hyperref package requires pdfLaTeX in order to break URLs across lines.

\documentclass[11pt]{article}

% Change "review" to "final" to generate the final (sometimes called camera-ready) version.
% Change to "preprint" to generate a non-anonymous version with page numbers.
\usepackage[final]{acl}
% \usepackage{acl}

% \usepackage[subtle, title=normal]{savetrees}


% Standard package includes
\usepackage{times}
\usepackage{latexsym}
\usepackage{xspace}
\usepackage{enumitem}
\usepackage{tabto}
\usepackage{subcaption}
\usepackage{tabularx}
\usepackage{array}
\usepackage{placeins}

% For proper rendering and hyphenation of words containing Latin characters (including in bib files)
\usepackage[T1]{fontenc}
% For Vietnamese characters
% \usepackage[T5]{fontenc}
% See https://www.latex-project.org/help/documentation/encguide.pdf for other character sets

% This assumes your files are encoded as UTF8
\usepackage[utf8]{inputenc}
\usepackage{csquotes}

% This is not strictly necessary, and may be commented out,
% but it will improve the layout of the manuscript,
% and will typically save some space.
\usepackage{microtype}

% This is also not strictly necessary, and may be commented out.
% However, it will improve the aesthetics of text in
% the typewriter font.
\usepackage{inconsolata}

%Including images in your LaTeX document requires adding
%additional package(s)
\usepackage{graphicx}
\usepackage{multirow}
\usepackage{amsmath}
\usepackage{amssymb}
\usepackage{mathtools}
\usepackage{amsthm}
\usepackage{caption} 
\captionsetup{aboveskip=8pt, belowskip=8pt}
\usepackage{tcolorbox}
\usepackage{aliascnt}
\usepackage{xcolor}
\usepackage{tikz}
\usepackage{colortbl}
\usepackage{pgf} % Ensure this package is included in your preamble
\usepackage{pgfmath}
\usepackage{ragged2e}


\makeatletter
\renewcommand{\sectionautorefname}{\S\@gobble}
\makeatother % Changes 'subs
\makeatletter
\renewcommand{\subsectionautorefname}{\S\@gobble}
\makeatother % Changes 'subs
\makeatletter
\renewcommand{\subsubsectionautorefname}{\S\@gobble} % Changes 'subs
\makeatother
\renewcommand{\equationautorefname}{Eqn} % Changes 'subs
\renewcommand{\figureautorefname}{Fig.} % Changes 'Figure' to 'Fig.'
\renewcommand{\tableautorefname}{Table} % Keeps 'Table' as is or change as needed

\def\compactify{\itemsep=0pt \topsep=0pt \partopsep=0pt \parsep=0pt}
\let\latexusecounter=\usecounter
\newenvironment{CompactEnumerate}
  {\def\usecounter{\compactify\latexusecounter}
   \begin{enumerate}}
  {\end{enumerate}\let\usecounter=\latexusecounter}



\tcbset{
    colframe=gray!40, 
    colback=gray!5,   
    coltitle=black,   
    fonttitle=\bfseries,
    sharp corners,
    boxrule=0.5mm,  
    width=\columnwidth,
    left=0.1mm,        
    right=0.1mm,      
    toptitle=0.1mm,
    bottomtitle=0.1mm,
    title=Question Generation Prompt
}
% If the title and author information does not fit in the area allocated, uncomment the following
%
%\setlength\titlebox{<dim>}
%
% and set <dim> to something 5cm or larger.

\title{\name: Scalable Concept-Driven Question Generation\\to Enhance Human Learning}

% Author information can be set in various styles:
% For several authors from the same institution:
% \author{Author 1 \and ... \and Author n \\
%         Address line \\ ... \\ Address line}
% if the names do not fit well on one line use
%         Author 1 \\ {\bf Author 2} \\ ... \\ {\bf Author n} \\
% For authors from different institutions:
% \author{Author 1 \\ Address line \\  ... \\ Address line
%         \And  ... \And
%         Author n \\ Address line \\ ... \\ Address line}
% To start a separate ``row'' of authors use \AND, as in
% \author{Author 1 \\ Address line \\  ... \\ Address line
%         \AND
%         Author 2 \\ Address line \\ ... \\ Address line \And
%         Author 3 \\ Address line \\ ... \\ Address line}

\author{
 \textbf{Kimia Noorbakhsh\textsuperscript{*}}, 
 \textbf{Joseph Chandler\textsuperscript{*}}, 
 \textbf{Pantea Karimi\textsuperscript{*}}, 
 \\
 \textbf{Mohammad Alizadeh}, 
 \textbf{Hari Balakrishnan}
\\
\\
 M.I.T. Computer Science and Artificial Intelligence Lab (CSAIL)
}


% \renewcommand{\thefootnote}{\textsuperscript{*}}
% \footnotetext{Equal contribution.}

%##########################Custom commands ###############
\definecolor{customblue}{HTML}{DAE8FC}
\definecolor{customred}{HTML}{F8CECC}
\definecolor{customgreen}{HTML}{D5E8D4}
\definecolor{custompurple}{HTML}{E1D5E7}
\definecolor{customorange}{HTML}{FFE6CC}


\newcommand*\circled[2]{\tikz[baseline=(char.base)]{
            \node[shape=circle, line width=0.75pt, draw=black, fill=#2, inner sep=1pt] 
            (char) {\textcolor{black}{\small\textsf{#1}}};}}


\newcommand{\bluecircle}{\raisebox{0pt}{\protect\tikz[baseline=(char.base)]{\protect\node[shape=circle,draw, fill=customblue, minimum size=1.8mm, inner sep=1pt] (char) {\footnotesize 1};}}}
\newcommand{\redcircle}{\raisebox{0pt}{\protect\tikz[baseline=(char.base)]{\protect\node[shape=circle,draw, fill=customred, minimum size=1.8mm, inner sep=1pt] (char) {\footnotesize 2};}}}
\newcommand{\greencircle}{\raisebox{0pt}{\protect\tikz[baseline=(char.base)]{\protect\node[shape=circle,draw, fill=customgreen, minimum size=1.8mm, inner sep=1pt] (char) {\footnotesize 3};}}}
\newcommand{\purplecircle}{\raisebox{0pt}{\protect\tikz[baseline=(char.base)]{\protect\node[shape=circle,draw, fill=custompurple, minimum size=1.8mm, inner sep=1pt] (char) {\footnotesize 4};}}}
\newcommand{\orangecircle}{\raisebox{0pt}{\protect\tikz[baseline=(char.base)]{\protect\node[shape=circle,draw, fill=customorange, minimum size=1.8mm, inner sep=1pt] (char) {\footnotesize 5};}}}



\newcommand*\sfboxed[1]{\tikz[baseline=(char.base)]{
            \node[shape=rectangle,line width=0.75pt, draw=black,inner sep=2pt, rounded corners=2pt] (char) {\textcolor{black}{\small\textsf{#1}}};}}
\newcommand{\NewPara}[1]{\noindent{\bf #1}}

\newcommand{\TheSystem}{Savaal\xspace}
\newcommand{\Baseline}{Direct\xspace}
\newcommand{\BaselineMT}{Direct\xspace}
\newcommand{\name}{Savaal\xspace}
\newcommand{\arxiv}{arXiv\xspace}
\newcommand{\gpt}{GPT-4o}

\begin{document}
\maketitle
\def\thefootnote{*}\footnotetext{These authors contributed equally to this work.}\def\thefootnote{\arabic{footnote}}
% Scatter plot PhD

\newcommand{\hitresponse}{38\%\xspace}

\newcommand{\understandingpreferration}{6.5$\times$\xspace}
\newcommand{\choicespreferration}{3$\times$\xspace}
\newcommand{\usabilitypreferration}{2$\times$\xspace}

\newcommand{\savaalunderstandingpreferrationpercent}{61.9\%\xspace}
\newcommand{\baselineunderstandingpreferrationpercent}{9.5\%\xspace}
\newcommand{\sameunderstandingpreferrationpercent}{28.6\%\xspace}

\newcommand{\savaalchoicespreferrationpercent}{57.1\%\xspace}
\newcommand{\baselinechoicespreferrationpercent}{19.0\%\xspace}
\newcommand{\samechoicespreferrationpercent}{23.9\%\xspace}

\newcommand{\savaalusabilitypreferrationpercent}{47.6\%\xspace}
\newcommand{\baselineusabilitypreferrationpercent}{23.8\%\xspace}
\newcommand{\sameusabilitypreferrationpercent}{28.6\%\xspace}

% Weighted average difference
\newcommand{\phdunderstandingWAD}{17\%\xspace}
\newcommand{\phdchoicesWAD}{10\%\xspace}
\newcommand{\phdusabilityWAD}{11.4\%\xspace}

% PhD


\newcommand{\baselinephdunderstandingvalue}{29.0}
\newcommand{\savaalphdunderstandingvalue}{11.9}

\newcommand{\baselinephdchoicesvalue}{39.0}
\newcommand{\savaalphdchoicesvalue}{29.0}


\newcommand{\baselinephdusabilityvalue}{32.9}
\newcommand{\savaalphdusabilityvalue}{21.4}


% Paper 
\newcommand{\baselinepaperunderstandingvalue}{16.7}
\newcommand{\savaalpaperunderstandingvalue}{10.9}

\newcommand{\baselinepaperchoicesvalue}{22.1}
\newcommand{\savaalpaperchoicesvalue}{21.8}


\newcommand{\baselinepaperusabilityvalue}{15.3}
\newcommand{\savaalpaperusabilityvalue}{13.8}



%%%%%%%%%%% CHNAGEABLES UP %%%%%%%%%%%

% Percent disagree macros
\newcommand{\savaalphdusability}{\savaalphdusabilityvalue\%\xspace}
\newcommand{\baselinephdusability}{\baselinephdusabilityvalue\%\xspace}
\newcommand{\phdusabilityreductionvalue}{%
  \pgfmathparse{round(\baselinephdusabilityvalue/\savaalphdusabilityvalue*100 )/100}%
  \pgfmathprintnumber[fixed, precision=1]{\pgfmathresult}%
}

\newcommand{\phdusabilityreduction}{\phdusabilityreductionvalue$\times$\xspace}

\newcommand{\savaalphdchoice}{\savaalphdchoicevalue\%\xspace}
\newcommand{\baselinephdchoice}{\baselinephdchoicevalue\%\xspace}
\newcommand{\phdchoicereductionvalue}{%
  \pgfmathparse{round(\baselinephdchoicevalue/\savaalphdchoicevalue*100 )/100}%
  \pgfmathprintnumber[fixed, precision=1]{\pgfmathresult}%
}

\newcommand{\phdchoicereduction}{\phdchoicereductionvalue$\times$\xspace}

\newcommand{\savaalphdunderstanding}{\savaalphdunderstandingvalue\%\xspace}
\newcommand{\baselinephdunderstanding}{\baselinephdunderstandingvalue\%\xspace}
\newcommand{\phdunderstandingreductionvalue}{%
  \pgfmathparse{round(\baselinephdunderstandingvalue/\savaalphdunderstandingvalue*100 )/100}%
  \pgfmathprintnumber[fixed, precision=1]{\pgfmathresult}%
}
\newcommand{\phdunderstandingreduction}{\phdunderstandingreductionvalue$\times$\xspace}


\newcommand{\savaalphdchoices}{\savaalphdchoicesvalue\%\xspace}
\newcommand{\baselinephdchoices}{\baselinephdchoicesvalue\%\xspace}
\newcommand{\phdchoicesreductionvalue}{%
  \pgfmathparse{round(\baselinephdchoicesvalue/\savaalphdchoicesvalue*100 )/100}%
  \pgfmathprintnumber[fixed, precision=1]{\pgfmathresult}%
}
\newcommand{\phdchoicesreduction}{\phdchoicesreductionvalue$\times$\xspace}

\newcommand{\savaalpaperusability}{\savaalpaperusabilityvalue\%\xspace}
\newcommand{\baselinepaperusability}{\baselinepaperusabilityvalue\%\xspace}
\newcommand{\paperusabilityreductionvalue}{%
  \pgfmathparse{round(\baselinepaperusabilityvalue/\savaalpaperusabilityvalue*100 )/100}%
  \pgfmathprintnumber[fixed, precision=1]{\pgfmathresult}%
}
\newcommand{\paperusabilityreduction}{\paperusabilityreductionvalue$\times$\xspace}

\newcommand{\savaalpaperunderstanding}{\savaalpaperunderstandingvalue\%\xspace}
\newcommand{\baselinepaperunderstanding}{\baselinepaperunderstandingvalue\%\xspace}
\newcommand{\paperunderstandingreductionvalue}{%
  \pgfmathparse{round(\baselinepaperunderstandingvalue/\savaalpaperunderstandingvalue*100 )/100}%
  \pgfmathprintnumber[fixed, precision=1]{\pgfmathresult}%
}

\newcommand{\paperunderstandingreduction}{\paperunderstandingreductionvalue$\times$\xspace}

\newcommand{\savaalpaperchoices}{\savaalpaperchoicesvalue\%\xspace}
\newcommand{\baselinepaperchoices}{\baselinepaperchoicesvalue\%\xspace}
\newcommand{\paperchoicesreductionvalue}{%
  \pgfmathparse{round(\baselinepaperchoicesvalue/\savaalpaperchoicesvalue*100 )/100}%
  \pgfmathprintnumber[fixed, precision=1]{\pgfmathresult}%
}

\newcommand{\paperchoicesreduction}{\paperchoicesreductionvalue$\times$\xspace}

% Num evaluators macros
\newcommand{\numemails}{200\xspace}

\newcommand{\numpaperevaluatorsvalue}{55}
\newcommand{\numphdevaluatorsvalue}{21}

\newcommand{\numpapersvalue}{50}
\newcommand{\numphdvalue}{21}

\newcommand{\numloogle}{48\xspace}
\newcommand{\numfields}{5\xspace}

% Scale of dataset macros
\newcommand{\avgphdpagesvalue}{142}
\newcommand{\avgphdpages}{\avgphdpagesvalue\xspace}
\newcommand{\avgpaperpagesvalue}{19}
\newcommand{\avgpaperpages}{\avgpaperpagesvalue\xspace}


% Math relations
\newcommand{\numphd}{\numphdvalue\xspace}
\newcommand{\numpapers}{\numpapersvalue\xspace}

\newcommand{\totaldocuments}{\pgfmathparse{int(\numphdvalue+\numpapersvalue)}\pgfmathresult\xspace}%

\newcommand{\numpaperevaluators}{\numpaperevaluatorsvalue\xspace}
\newcommand{\numphdevaluators}{\numphdevaluatorsvalue\xspace}

\newcommand{\totalevaluatorsvalue}{\pgfmathparse{int(\numpaperevaluatorsvalue+\numphdevaluatorsvalue)}\pgfmathresult}

\newcommand{\numpaperquestionsvalue}{\pgfmathparse{int(\numpaperevaluatorsvalue*20)}\pgfmathresult}


\newcommand{\numphdquestionsvalue}{\pgfmathparse{int(\numphdevaluatorsvalue*20)}\pgfmathresult}

\newcommand{\totalevaluators}{\totalevaluatorsvalue\xspace}
\newcommand{\numpaperquestions}{\numpaperquestionsvalue\xspace}
\newcommand{\numphdquestions}{\numphdquestionsvalue\xspace}


\newcommand{\numtotalhumanquestionvalue}{\pgfmathparse{int(\numphdevaluatorsvalue*20 + \numpaperevaluatorsvalue*20)}\pgfmathresult}

\newcommand{\numtotalhumanquestion}
{\numtotalhumanquestionvalue\xspace}


\newcommand{\savaalcostreduction}{1.39$\times$\xspace}
\newcommand{\directcostinflation}{1.64$\times$\xspace}

\begin{abstract}

Hypotheses are central to information acquisition, decision-making, and discovery. However, many real-world hypotheses are abstract, high-level statements that are difficult to validate directly. 
This challenge is further intensified by the rise of hypothesis generation from Large Language Models (LLMs), which are prone to hallucination and produce hypotheses in volumes that make manual validation impractical. Here we propose \mname, an agentic framework for rigorous automated validation of free-form hypotheses. 
Guided by Karl Popper's principle of falsification, \mname validates a hypothesis using LLM agents that design and execute falsification experiments targeting its measurable implications. A novel sequential testing framework ensures strict Type-I error control while actively gathering evidence from diverse observations, whether drawn from existing data or newly conducted procedures.
We demonstrate \mname on six domains including biology, economics, and sociology. \mname delivers robust error control, high power, and scalability. Furthermore, compared to human scientists, \mname achieved comparable performance in validating complex biological hypotheses while reducing time by 10 folds, providing a scalable, rigorous solution for hypothesis validation. \mname is freely available at \url{https://github.com/snap-stanford/POPPER}.




\end{abstract}

\section{Introduction}
\label{sec:intro}

Many people learn new material effectively by taking quizzes. Answering questions not only assesses knowledge, but also  reinforces learning by strengthening correct responses and revealing gaps in understanding. A major challenge in the 21st century is the rapid expansion of knowledge across fields like science, technology, medicine, law, finance, and more. AI tools are accelerating this growth, making it increasingly difficult for students, researchers, and professionals---from engineers to salespeople---to stay current. The need to learn efficiently and at scale has never been greater.


One response is to rely on AI for answers, effectively outsourcing expertise. While sometimes necessary, this does little to improve human understanding. Instead, we advocate using AI to enhance {\em our} ability to learn and master new material. %, and our work aims to advance this goal.


Programs like ChatGPT, Gemini, Claude, NotebookLM, Perplexity, and DeepSeek built atop large language models (LLMs) have made remarkable strides in summarization and question-answering. However, less attention has been given to {\em question generation}, specifically, creating high-quality questions that test human understanding and mastery of knowledge. That is the focus of this paper.

Anyone who has made an exam knows how difficult and time-consuming it is to make a good set of questions. Our goal is to produce questions automatically with three objectives:
\begin{CompactEnumerate}
\item {\em Scalability}: Generating questions across vast document corpora, such as rapidly evolving research fields or enterprise knowledge bases.
\item {\em Depth of understanding}: Producing questions beyond memorization and the superficial, requiring conceptual reasoning, synthesis, and analysis.
\item {\em Domain-independence}: Creating high-quality questions across diverse fields, including new material absent in an LLM’s pre-training data.
\end{CompactEnumerate}


Prior work on question generation has produced a small number of questions from short passages, but has not demonstrated scalability~\citep{du-etal-2017-learning, Neural_QG, chan-fan-2019-bert, li-etal-2021-addressing-semantic, knowledge_base_prompting, reading_comprehension_language_llm, code_QG, mcq_mult_sentence}. Our results (\autoref{sec:eval}) show that even well-engineered prompts to an LLM produce poor, repetitive questions on large text contexts (tens to hundreds of pages), highlighting the scalability challenge.
 

We present \textbf{\name}, a scalable question generation system for large documents. Savaal uses a three-stage pipeline. The first stage extracts and ranks the key concepts in a corpus of documents\footnote{We use ``document'' to also refer to the corpus of documents used to generate a quiz.} using a map-reduce computation. The second stage retrieves relevant passages corresponding to each concept with an efficient vector embedding retrieval model such as ColBERT~\cite{colbert}. Finally, the third stage prompts an LLM to generate questions for each ranked concept using the retrieved passages as context.

This approach scales well because each LLM computation is confined to a distinct, self-contained task while operating within a manageable context size. By first identifying core concepts and later synthesizing questions from all relevant passages, \name ensures that the generated questions are both targeted and conceptually rich, requiring deeper understanding by linking a given concept across different sections of a document.

We compare \name to a direct-prompting baseline (\Baseline) using \totalevaluators human expert evaluators (the primary authors of \numpapers recent conference papers and \numphd PhD dissertations in subfields of computer science and aeronautics) on \numtotalhumanquestion questions. We also evaluate each paper, as well as 48 arXiv papers, using an LLM as an AI judge.
%and also using an LLM judge on \numloogle papers from the Loogle \cite{loogle} dataset and \arxiv. 
%These evaluations span \numfields distinct fields. 
We find that: 
\begin{CompactEnumerate}
    \item On \numphdquestions questions from \numphdevaluators large documents (dissertations with average \avgphdpages pages), experts reported that \baselinephdunderstanding of \Baseline's questions {\em did not} test understanding, compared to \savaalphdunderstanding of \name, a \phdunderstandingreduction improvement. 
    They reported that \baselinephdchoices of \Baseline's questions lacked good choice quality, compared to \name's \savaalphdchoices, improving by \phdchoicesreduction. They found \baselinephdusability of \Baseline's questions {\em unusable} in a quiz, compared to \savaalphdusability of \name's questions, a \phdusabilityreduction reduction. Moreover, among experts with a preference, \understandingpreferration more favored \name over baseline in understanding, \choicespreferration in choice quality, and \usabilitypreferration in usability.


    \item Even on shorter documents, experts rated \name better in terms of depth of understanding and usability. On \numpaperquestions questions from \numpapers conference papers, \numpaperevaluators experts reported that \baselinepaperunderstanding of baseline's questions {\em did not} test understanding, compared to \savaalpaperunderstanding of \name, a \paperunderstandingreduction improvement. 
    

    
    \item \name is less expensive than \Baseline as the number of questions grows: \Baseline's cost for 100 questions generated from the dissertations is \directcostinflation higher than \name (\$0.47 vs. \$0.77 on average per document).

    
    \item There is a large gap between AI judgments and human evaluations. Despite several attempts to align the AI judge to human responses, scores remained misaligned.% with human expert evaluations.

\end{CompactEnumerate}



\section{Why is Generating Good Questions Hard?}
\label{sec:insights}

Our goal is to enhance human learning from large documents spanning dozens to hundreds of pages by generating multiple-choice questions. Multiple-choice questions are widely used in assessments, are easy to use by learners, and are easy to grade. The task involves generating a set of clear questions, each with four  choices and a correct answer.

High-quality questions assess {\em depth of understanding}, requiring conceptual reasoning and plausible choices (distractors) that challenge the learner. Beyond clarity and precision, our notion of a good question is one that could appear in an advanced quiz on the material as judged by a human expert. While this paper focuses on generating individual high-quality questions, effective quiz sessions should ensure {\em concept coverage} and {\em adapting the difficulty} to prior answers in the session, both avenues for future work.


The main challenge in scalable question generation using LLMs is selecting an appropriate context to use with LLM prompts. We examine four potential strategies: (i) using the full document corpus, (ii) dividing the corpus into sections, (iii) summarizing the corpus, and (iv) using content selection classifiers~\citep{context_Steuer, Context_diverse_hadifar}. Although each strategy has merits, we show that each strategy fails on at least one of our key objectives: {\em scalability}, {\em depth of understanding}, or {\em domain-independence}.

\newcommand{\questionbox}[1]{%
    \colorbox{customblue}{\parbox{0.97\linewidth}{\vspace{1pt}\textbf{#1}\vspace{1pt}}}
}

\begin{table*}[t]
\centering
\renewcommand{\arraystretch}{1.0} % Adjust row height for better readability
\begin{tabular}{|p{0.2\textwidth}|m{0.5\textwidth}|m{0.2\textwidth}|} 
\hline
\rowcolor{gray!20} % Light gray background for header row
\textbf{Context} & \textbf{Generated Question} & \textbf{Issue} \\ \hline
\footnotesize
\sfboxed{1} \textbf{Entire Document} & 
\footnotesize
\questionbox{What is the primary benefit of using the Adam optimizer in training the Transformer model?}
A. It reduces the need for dropout regularization. \newline
\textbf{B.} It adapts the learning rate based on the training step, improving convergence. \newline
C. It eliminates the need for positional encodings. \newline
D. It simplifies the model architecture by reducing the number of layers. 
&
\footnotesize
\textbf{Too general}: The question is about a basic property of the Adam optimizer rather than the key ideas of the paper.

$\Rightarrow$ Does not test depth of understanding
\\ \hline
\footnotesize
\sfboxed{2} \textbf{Document Section}& 
\footnotesize
\questionbox{In evaluating the performance and efficiency of the Transformer (big) model on the WMT 2014 English-to-French translation task, which of the following factors most significantly contributes to its ability to outperform previous models at a reduced training cost?}
A. The use of a dropout rate of 0.1 instead of 0.3, which enhances model regularization and reduces overfitting.\newline
B. The implementation of beam search with a beam size of 4 and a length penalty $\alpha$ = 0.6, which optimizes the translation output quality.\newline
\textbf{C.} The averaging of the last 20 checkpoints, which stabilizes the model's performance and improves translation accuracy.\newline
D. The reduction in training time to less than 1/4 of the previous state-of-the-art model, which directly correlates with improved BLEU scores.&
\footnotesize
\textbf{Irrelevant detail:} Because the method looks at one section at a time, it fixates on minutiae and irrelevant details (e.g., “averaging the last 20 checkpoints”) that may seem important in isolation, but are not.

$\Rightarrow$ Does not test depth of understanding
\\ \hline
\footnotesize
\sfboxed{3} \textbf{Document Summary} &
\footnotesize
\questionbox{How does the Transformer model address the challenge of learning dependencies between distant positions in sequences compared to models like ConvS2S and ByteNet?}
A. By using convolutional layers to capture long-range dependencies\newline
B. By increasing the number of layers in the encoder and decoder stacks\newline
C. By employing a recurrent neural network to process sequences\newline
\textbf{D.} By reducing the number of operations to a constant using self-attention mechanisms"
& 
\footnotesize
\textbf{Missing context:} The summary mentions ``...The Transformer model addresses this by reducing the number of operations to a constant, using self-attention mechanisms.'' which led the LLM design this incomplete question.


$\Rightarrow$ Leads to inaccurate questions
\\ \hline
\end{tabular}
\caption{Examples from the ``Attention Is All You Need'' paper \citep{attention_is_all_you_need} using three different context selection methods. The questions are drawn from three separate 20-question quizzes, each generated using a different method via OpenAI's API \citep{openai_api} with the \texttt{gpt-4o} model.}
\label{tab:bad-examples-attention}
\vspace{-10 pt}
\end{table*}


\subsection{Using the Entire Document Corpus}
\label{sec:insights-whole-context}

One approach is to provide the entire document as context to an LLM for quiz generation. However, this method has two major drawbacks.
First, as prior research shows~\citep{lost-in-the-middle}, LLMs allocate attention unevenly across long documents, focusing more on the beginning and end while largely neglecting the middle. 

Second, LLMs struggle to capture dependencies between different sections of a long document~\citep{loogle}, leading to superficial questions and missing key concepts. When we prompted OpenAI's \texttt{gpt-4o} model with the full text of the ``Attention Is All You Need'' paper~\citep{attention_is_all_you_need}, many of the 20 generated questions overlooked key ideas. See Example \sfboxed{1} in \autoref{tab:bad-examples-attention} for a question, which is not relevant to the paper's key ideas.

We found that LLMs struggle to follow instructions when the context length is large~\cite{gao2024insights}. For example, we instruct the LLM not to repeat questions. While it avoids repetition when generating a few questions, larger batches (e.g., 20 questions) often contain duplicates. 


\subsection{Using Document Sections}
\label{sec:insights-section-context}

An alternative is to split the document into sections, generate a limited number of questions per section, and later combine them into a quiz. While this method mitigates long-context issues, it introduces {\em context fragmentation}: the LLM cannot connect concepts spanning multiple sections. It often misses deeper connections that can assess stronger conceptual understanding. For example, key insights in a paper’s Algorithm or Methods section may be essential for understanding its Results, but treating these sections independently leads to disjointed, narrow questions.

Another issue is {\em uneven importance weighting}. Not all sections contribute equally to the document’s  ideas, yet a naïve section-based approach may overemphasize minor details while missing key concepts. Example \sfboxed{2} in \autoref{tab:bad-examples-attention} shows how this can generate irrelevant memorization questions.


\subsection{Summarization}
\label{sec:insights-summary}

Providing a {\em document summary} as context offers another way to streamline question generation. While LLMs are effective at summarization, summaries often lack critical details, leading to vague or incomplete questions. More concerning, summaries can introduce hallucinations~\citep{llm_hallucination}, distorting or misrepresenting causal relationships and fabricating details, further degrading question quality.

Example \sfboxed{3} in \autoref{tab:bad-examples-attention} illustrates how summarization can result in misleading or imprecise questions. Here, the summary includes a statement about using self-attention to ``reduce the number of operations to a constant'', but omits that this refers to {\em sequential} operations and maximum path length (Sec. 4 of \citep{attention_is_all_you_need}), leading to an inaccurate question. 



\subsection{Content Selection Classifiers}

Some methods attempt to select relevant content for question generation, often using trained models to identify key passages~\citep{context_Steuer, Context_diverse_hadifar}. However, these approaches typically require domain-specific training data (e.g., pre-existing question-answer pairs), making them {\em domain-dependent}. Such approaches are frequently limited in scope, making them neither reliable nor generalizable to diverse domains. 

\begin{figure*}[!t]
\centering
    \includegraphics[width=1\linewidth]{FIG/savaal.drawio.pdf}
    \caption{\name's Pipeline. \bluecircle\ \name extracts main ideas from sections of the document in parallel, \redcircle\ combines them into a succinct list, and \greencircle\ ranks them in order of importance. Next, \purplecircle\ \name fetches relevant passages from the document using a vector-based retrieval model. Finally, \orangecircle\ given a main idea and fetched passages, \name generates questions.}
    \label{fig:savaal-workflow}
\vspace{-10 pt}
\end{figure*}

\section{\name's Question-Generation Pipeline}
\label{sec:pipeline}


To address challenges of \autoref{sec:insights}, we propose a novel three-stage pipeline: \emph{main idea extraction}, \emph{relevant passage retrieval}, and \emph{question generation}. \autoref{fig:savaal-workflow} shows Savaal's workflow. The idea is to generate questions targeted at key explicitly determined concepts and to retrieve passages relevant to the concept from the source document.
% to generate questions.


\subsection{Extracting Main Ideas}
\label{sec:pipeline-main-idea}
This stage extracts succinct main ideas from different document chapters. This is done in a map-combine-reduce fashion~\cite{langchain_mapreduce}. First, we use GROBID~\citep{GROBID} to parse and segment documents into distinct sections.


In the map stage, \circled{1}{customblue}, we use an LLM to extract the main ideas for each section individually. These extracted main ideas are aggregated and deduplicated in the combine stage, \circled{2}{customred}, into a single, cohesive list of the paper’s main ideas. If the combined output exceeds a predefined length threshold (set to the maximum token window of the LLM), the reduce stage collapses the list further for brevity and clarity. The result is a curated list of main ideas, including main idea titles and their short descriptions (see \autoref{subsubsec:example_main_idea} for examples). The same (or a different) LLM then ranks the main ideas based on their importance in the ranking stage in \circled{3}{customgreen} (see \autoref{subsec:appendix_prompts} for the prompts).

Initially, we attempted to extract the main ideas for the entire document in one shot. However, as noted in \autoref{sec:insights-whole-context}, as the context length grew, this became less effective. We found that using map-reduce extracted main ideas that were more detailed and useful for question generation, particularly on large documents.



\subsection{Retrieving Relevant Passages}
\label{sec:pipeline-retrieval}

Because the main ideas in \autoref{sec:pipeline-main-idea} are concise, they lack sufficient content to generate a question. As discussed in \autoref{sec:insights-summary}, asking an LLM to generate questions based on a concept alone (a main idea or even a summary) has shortcomings. To overcome this problem, \name retrieves relevant text segments directly from the original document to provide granular content for generating a question and to ensure that the questions are grounded in truth.


\name's retriever uses ColBERT, a late-interaction retrieval method~\citep{colbert, colbertv2}, to find the most relevant passages for each main idea (stage \circled{4}{custompurple}).
% integrated in the RAGatouille \footnote{\url{https://github.com/AnswerDotAI/RAGatouille}} library.
For each ranked main idea in \circled{3}{customgreen}, we retrieve the top $k$ passages as added context for the next stage ($k=3$ in our experiments).

We chose ColBERT for its state-of-the-art performance and wide adoption, but any high-performing retrieval method could be used. We also tried using the LLM to identify passages related to a main idea, but as in \autoref{sec:insights-whole-context} and \autoref{sec:pipeline-main-idea}, it struggled with large context sizes.


\subsection{Generating Questions and Choices}
\label{sec:pipeline-QG}


After retrieving the passages for each main idea, stage \circled{5}{customorange} instructs an LLM to generate questions. To create $N$ questions from $M$ ideas, we generate $N/M$ questions per idea.\footnote{We use only the top $N$ ranked main ideas if $N < M$.}  The prompt (\autoref{fig:question_generation}) includes the main idea and its retrieved passages.


Although LLMs often produce good questions, generating good {\em choices} is more challenging. Poorly designed choices can make the correct answer too obvious or, worse, introduce ambiguity or multiple correct options. We experimented with many prompt variations to improve choice quality, yielding mixed results. We also tested a separate ``choice refinement'' stage, where an LLM was specifically instructed to improve the answer choices for a given question. This prompt included detailed constraints, such as ensuring alignment with the question's intent (e.g., a question about benefits should not include limitations as choices; see \autoref{appendix:choice-refine}).
Although this additional step produced more challenging choices, we found that it caused excessive ambiguity and was less preferred by human expert evaluators. Therefore, \name does not include a choice refinement stage. Instead, its question-generation prompt explicitly emphasizes that the choices should be ``plausible distractors''.

Finally, we observed {\em positional biases} in the placement of the correct choice, corroborating prior findings~\cite{pezeshkpour2023large}. For example, in a set of 1000 questions from 50 papers (20 per paper) generated by \texttt{GPT-4o}, choice B was correct 73.3\% of the time! Thus, we randomize the choices to eliminate this bias.

\section{Evaluation}
\label{sec:eval}

We evaluated \name on \totaldocuments documents using both human experts and an AI judge. We used \texttt{GPT-4o} via the OpenAI API as our primary LLM. We also evaluated \texttt{Meta-Llama-3.3-70B-Instruct} (\autoref{subsec:ablation-model}). All models are set to temperature 0.0 for all experiments, with default settings for all other parameters. \name is model-agnostic and is compatible with current LLMs. We implemented our pipeline using LangChain~\cite{langchain} and traced our experiments in Weights \& Biases~\cite{wandb}.
%any commercial or open-source LLM.

\subsection{Datasets}
\label{sec:evaluation-data}
%We gathered three datasets for evaluation:
\begin{itemize}[topsep=0pt, itemsep=0pt, leftmargin=*]
    \item \textbf{PhD dissertations}: \numphd long documents in Aerospace, Machine Learning, Networks, Systems, and Databases (\autoref{tab:human-eval-dataset-stats}).
    \item \textbf{Conference papers}: \numpapers papers from conferences in CS and Aeronautics in 2023 and 2024.
    \item \textbf{Diverse \arxiv papers}: \numloogle papers from CS, Physics, Mathematics, Economics, and Biology (\autoref{tab:benchmark-stats}). 
\end{itemize}

\begin{table}[h]
\centering
\renewcommand{\arraystretch}{1} % Increase row height for better readability
\setlength{\tabcolsep}{1pt} % Adjust column spacing for better fit
% \begin{small}
\begin{tabular}{|l|c|c|}
\hline
\small  \textbf{Statistic} & \small  \textbf{Conference Papers} & \small \textbf{Dissertations} \\
\hline
\small \textbf{No. Documents} & \small \numpapers & \small \numphd \\ 
\hline
\small \textbf{Avg. Words} & \small 10,354 & \small 26,511 \\ 
\hline
\small \textbf{Avg. Pages} & \small \avgpaperpages & \small \avgphdpages \\
\hline
\end{tabular}
\caption{Statistics for the number of words in the conference papers and PhD dissertations.}
\label{tab:human-eval-dataset-stats}
% \end{small}
\end{table}



\subsection{Methods Compared}
\label{sec:evaluation-baselines}
We compare \name to \Baseline, a direct-prompting baseline (\autoref{sec:insights-whole-context}) that provides the entire document to the LLM with a detailed prompt to generate $N$ multiple-choice questions (\autoref{fig:baseline_question_generation_prompt}). We found that when $N$ exceeds $\approx$ 20, \Baseline fails to produce $N$ distinct questions. Since broad concept coverage requires generating a large pool of questions and sampling for shorter quizzes, we generate $N > 20$ questions in batches of $b=20$ using an additional prompt (\autoref{fig:baseline_large_question_generation_prompt}). We use this {\em multi-turn method} for \Baseline on longer documents. 


We evaluate other methods using the AI judge: Summary (\autoref{sec:insights-summary}) and Single-Prompt Savaal, which condenses Savaal's idea extraction, retrieval, and question generation into a single prompt (\autoref{subsec:ablation-methods}).


\subsection{Evaluation Criteria}
\label{sec:evaluation-metrics}

Evaluating the quality of questions is challenging because it involves subjective human judgment~\cite{fu2024qgeval}. We primarily rely on human evaluations but also use \texttt{GPT-4o} as an AI judge~\cite{naismith2023automated} to expand the scope of our evaluation to more approaches, documents, and criteria. 

% Exempt ID: E-6417

\paragraph{Human experts:} We generated 10 multiple-choice questions from Savaal and 10 from \Baseline for each of the \numphd PhD dissertations and \numpapers conference papers. We contacted the primary authors to evaluate the quality of questions via a secure web-based feedback form.\footnote{\emph{MIT} Institutional Review Board exempted this study (Exemption Number: E-6417). All the personnel were certified, and participants were over 18 years of age and provided informed consent before participating.} We asked each expert to rate their questions on clarity, depth of understanding\footnote{Used interchangeably with understanding.}, and quality of choices using a four-point scale: \emph{Disagree}, \emph{Somewhat Disagree}, \emph{Somewhat Agree}, and \emph{Agree}. They also assessed usability by answering: ``Would I use this question on a graduate-level quiz?'' with options: {\em Yes}, {\em Yes with small changes}, and {\em No}. The questions were randomly mixed and the evaluators were blind to their source. In all, \totalevaluators experts participated (\autoref{subsec:appendix_human_eval_conduct}).
 

\label{sec:metrics-auto}
\paragraph{AI judge:} We prompted \texttt{GPT-4o} at temperature 0.0 to score each question on a 1–4 scale (\autoref{subsubsec:eval-prompts}) on Depth of Understanding, Quality of Choices, Clarity, Usability, Difficulty, Cognitive Level, and Engagement (\autoref{subsec:ablation-metrics}). Our evaluation prompts provide detailed guidelines than those given to humans, including explicit criteria for each rating (\autoref{subsubsec:eval-prompts}).



\subsection{Results with Human Experts}
\label{sec:evaluation-results}

\begin{figure}[!t]
    \centering
    \begin{subfigure}[b]{0.9\linewidth}
        \centering
        \includegraphics[width=1\linewidth]{FIG/thesis_only_disagree_all_metrics_bar_charts.pdf}
        \caption{PhD dissertations: \numphdquestions questions, \numphdevaluators experts.}
        \label{fig:human-eval-disagree-phd}
    \end{subfigure}
    \hfill
    \begin{subfigure}[b]{0.9\linewidth}
        \centering
        \includegraphics[width=1\linewidth]{FIG/no_dedup_no_refine_final_only_disagree_all_metrics_bar_charts.pdf}
        \caption{Conference papers: \numpaperquestions questions, \numpaperevaluators experts.}
        \label{fig:human-eval-disagree-paper}
    \end{subfigure}
    \vspace{-10 pt}
    \caption{ Summary of human evaluation: The charts show the percentage and standard error of respondents who {\em Disagree} or {\em Somewhat Disagree} with questions on understanding, choice quality, and usability. {\bf Lower values indicate better performance.}}
    \label{fig:human-eval-disagree}
\vspace{-20 pt}
\end{figure}

\begin{figure*}[!t]
    \centering
    \begin{subfigure}[b]{0.32\linewidth}
        \centering
        \includegraphics[width=\linewidth]{FIG/thesis_understanding_num_AGREE_scatter.pdf}
        \caption{Depth of understanding: 61.9\% prefer \name, 9.5\% \Baseline.}
        \label{fig:thesis-scatter-understanding}
    \end{subfigure}
    \hfill
    \begin{subfigure}[b]{0.32\linewidth}
        \centering
        \includegraphics[width=\linewidth]{FIG/thesis_quality_of_choices_num_AGREE_scatter.pdf}
        \caption{Quality of choices: 57.1\% prefer \name, 19\% \Baseline.}
        \label{fig:thesis-scatter-choices}
    \end{subfigure}
    \hfill
    \begin{subfigure}[b]{0.32\linewidth}
        \centering
        \includegraphics[width=\linewidth]{FIG/thesis_overall_quality_num_AGREE_scatter.pdf}
        \caption{Usability: 47.6\% prefer \name, 23.8\% \Baseline.}
        \label{fig:thesis-scatter-overall}
    \end{subfigure}
    
    \caption{Expert preferences for \numphdevaluators PhD dissertations. Each point shows the number of \emph{Agree}s or \emph{Somewhat Agree}s in a 10-question quiz for each of \name and \Baseline. The majority of experts prefer \name to \Baseline on depth of understanding, quality of choices, and usability on long documents (experts above $y=x$ prefer \name).}
    \label{fig:human-eval-scatter}
    \vspace{-10 pt}
\end{figure*}


\label{sec:evaluation-human}

\autoref{fig:human-eval-disagree} summarizes the results of our expert human evaluation on PhD dissertations and papers. We show here the negative sentiment of the experts, i.e., the percentage of questions that experts responded with \emph{Disagree} or \emph{Somewhat Disagree} for each criterion (see \autoref{fig:human-phd-breakdown} and \autoref{fig:human_paper_breakdown} for the full breakdown). 

For the \numphdquestions questions from \numphd PhD dissertations (\autoref{fig:human-eval-disagree-phd}), the experts responded that \baselinephdunderstanding of \Baseline's questions {\em did not test understanding}; by contrast, only \savaalphdunderstanding of \name's questions did not, a  \phdunderstandingreduction reduction in negative sentiment. They also rated \baselinephdusability of \Baseline's questions as {\em unusable in a quiz}, versus \savaalphdusability for \name, a \phdusabilityreduction reduction.

For  conference papers (\autoref{fig:human-eval-disagree-paper}), on \numpaperquestions questions, \numpaperevaluators experts\footnote{Some papers had multiple expert respondents.} found that \savaalpaperunderstanding of \name's questions {\em did not} test understanding, versus \baselinepaperunderstanding for \Baseline, a \paperunderstandingreduction improvement. They also rated \baselinepaperusability of \Baseline's questions as {\em unusable}, versus \savaalpaperusability for \name.

The experts agreed or somewhat agreed that over 90\% of the questions in both \Baseline and \name had clarity (not shown in the figure). This result is unsurprising because LLMs can be prompted to generate coherent and unambiguous text. 

For PhD dissertations, \autoref{fig:human-eval-scatter} shows how each of the \numphdevaluators experts scored \name vs. \Baseline on the metrics for the PhD dissertations. The $x$ and $y$ axes show number of \emph{Agree} or \emph{Somewhat Agree} for \Baseline and \name, respectively. Each point represents one expert evaluator. 

We observe that \savaalunderstandingpreferrationpercent favor \name over \Baseline for understanding (\autoref{fig:thesis-scatter-understanding}), whereas only \baselineunderstandingpreferrationpercent (\understandingpreferration fewer) prefer \Baseline over \name (\sameunderstandingpreferrationpercent rate the two systems the same). For choice quality, \savaalchoicespreferrationpercent prefer \name compared to \baselinechoicespreferrationpercent for \Baseline (\choicespreferration more, see \autoref{fig:thesis-scatter-choices}), while for usability \savaalusabilitypreferrationpercent prefer \name compared to \baselineusabilitypreferrationpercent for \Baseline (\usabilitypreferration more, see \autoref{fig:thesis-scatter-overall}). 

The data in \autoref{fig:human-eval-scatter} also shows that, on average, expert evaluators rated \emph{Agree} or \emph{Somewhat Agree} for more questions in \name quizzes than \Baseline: \phdunderstandingWAD more for understanding, \phdchoicesWAD more for quality of choices, and \phdusabilityWAD more for usability.

\autoref{fig:human_paper_breakdown} shows the breakdown of expert responses for \numpaperquestions questions from the conference papers. On these shorter documents, experts slightly prefer \name over \Baseline in terms of depth of understanding. They reported that 16.7\% of \TheSystem's questions {\em did not} test understanding, compared to 10.9\% for \Baseline. Experts rated the two methods similarly for choice quality and usability. As in the results for Ph.D. dissertations (\autoref{fig:human-auto-correlation}), the \texttt{GPT-4o} scores (\autoref{fig:AI_paper_breakdown}) correlated poorly with expert evaluations.


\autoref{fig:paper-human-eval-scatter} shows how each of the \numpaperevaluators experts scored \name vs. \Baseline. The $x$-axis shows the number of \emph{Agree} or \emph{Somewhat Agree} for \Baseline, and the $y$-axis shows the same for \name. Each point represents one expert evaluator. Among evaluators with a preference, 1.5$\times$ more experts favor \TheSystem over \Baseline in understanding (34.5\% for \name vs 21.8\% for \Baseline, \autoref{fig:paper-scatter-understanding}). Experts do not exhibit a strong preference between \name and \Baseline for choice quality (\autoref{fig:paper-scatter-choices}) or usability (\autoref{fig:paper-scatter-overall}). The average relative increase in the Agree score for \TheSystem compared to \Baseline is 5.8\% for understanding, 4\% for quality of choices, and 1.5\% for usability.
% , meaning that on average, experts like at least one more question in \name's quizzes compared to \Baseline.


\begin{figure*}[h]
    \centering
    \begin{subfigure}[b]{0.32\linewidth}
        \centering
        \includegraphics[width=\linewidth]{FIG/no_dedup_no_refine_final_understanding_num_AGREE_scatter.pdf}
        \caption{Depth of understanding: 34.5\% prefer \name, 21.8\% prefer \Baseline.}
        \label{fig:paper-scatter-understanding}
    \end{subfigure}
    \hfill
    \begin{subfigure}[b]{0.32\linewidth}
        \centering
        \includegraphics[width=\linewidth]{FIG/no_dedup_no_refine_final_quality_of_choices_num_AGREE_scatter.pdf}
        \caption{Quality of choices: no specific preference exhibited.}
        \label{fig:paper-scatter-choices}
    \end{subfigure}
    \hfill
    \begin{subfigure}[b]{0.32\linewidth}
        \centering
        \includegraphics[width=\linewidth]{FIG/no_dedup_no_refine_final_overall_quality_num_AGREE_scatter.pdf}
        \caption{Usability: no specific preference exhibited.}
        \label{fig:paper-scatter-overall}
    \end{subfigure}
    
    \caption{Human expert preferences for \numpaperevaluators experts on short conference papers. Each point shows the number of \emph{Agree}s in a 10-question quiz for \name and \Baseline respectively. More experts prefer \name to \Baseline on the depth of understanding. Experts don't exhibit any preference between the quality of choices and usability on short documents (experts above $y=x$ prefer \name).}
    \label{fig:paper-human-eval-scatter}
\end{figure*}


\subsection{Results with an AI Judge}
\label{sec:evaluation-auto}


We used an AI judge to scale evaluations across more documents and criteria. We first examined its alignment with human experts by having \texttt{GPT-4o} evaluate the same \numphdquestions questions from the expert-reviewed dissertations dataset. 

\autoref{fig:human-auto-correlation} compares the AI judge with human experts. The AI judge rarely assigns \emph{Disagree} or \emph{Somewhat Disagree} for understanding and usability and slightly favors \name, giving it 28.6\% Agree rating in comparison to 14.3\% Agree ratings for \Baseline for understanding. However, for quality of choices, it rates both schemes poorly, with only 9.6\% \emph{Agree} or \emph{Somewhat Agree} for \name and 19\% for \Baseline.

We observed similar trends in the \numpaperquestions questions from the conference-paper dataset (\autoref{fig:paper_breakdown}), where the AI judge again slightly preferred \name but remained misaligned with human expert evaluations. For completeness, we also present AI judge results on the Diverse \arxiv dataset in \autoref{subsec:ablation}.

%\paragraph{Limitations of the AI judge.} 
Our takeaway is that our \texttt{GPT-4o} AI judge was unaligned with human expert judgments (see \autoref{fig:auto-correlation-ai} vs. \autoref{fig:human-phd-breakdown}). Despite our extensive efforts in prompt engineering to maximize alignment---including using the prompt optimizer program in DSPy~\citep{khattab2024dspy}---AI-human correlation did not improve. Our experience calls into question the wisdom of using only AI judges in research studies. 




\begin{figure}[t]
    \centering
    \begin{subfigure}[b]{1\linewidth}
        \centering
        \includegraphics[width=1\linewidth]{FIG/thesis_no_combine_all_metrics_bar_charts.pdf}
        \caption{Breakdown of human expert scores on PhD dissertations.}
        \label{fig:human-phd-breakdown}
    \end{subfigure}  
    \hfill
    \begin{subfigure}[b]{1\linewidth}
        \centering
        \includegraphics[width=1\linewidth]{FIG/FINAL_PLOTS_AUTO/thesis_human_auto_all_metrics_bar_charts.pdf}
        \caption{Breakdown of GPT-4o AI judge scores on PhD dissertations.}
        \label{fig:auto-correlation-ai}
    \end{subfigure}
    \caption{Score distribution for  \protect\numphdquestions questions from dissertations: GPT-4o as a judge does not align with humans for assessing the metrics.}
    \label{fig:human-auto-correlation}
    \vspace{-20 pt}
\end{figure}


\begin{figure}[h]
    \centering
    \begin{subfigure}{\linewidth}
        \centering
        \includegraphics[width=\linewidth]{FIG/no_dedup_no_refine_final_no_combine_all_metrics_bar_charts.pdf}
        \caption{Breakdown of human expert scores on conference papers.}
        \label{fig:human_paper_breakdown}
    \end{subfigure}
    \hfill
    \begin{subfigure}{\linewidth}
        \centering
        \includegraphics[width=\linewidth]{FIG/FINAL_PLOTS_AUTO/papers_human_auto_all_metrics_bar_charts.pdf}
        \caption{Breakdown of GPT-4o scores on conference papers.}
        \label{fig:AI_paper_breakdown}
    \end{subfigure}
    \caption{Score distribution for  \protect \numpaperquestions questions from conference papers.}
    \label{fig:paper_breakdown}
    \vspace{-10 pt}
\end{figure}





\subsection{Cost Scalability}
\label{sec:scalability-case}

\autoref{fig:cost-scalability} compares the costs of \name and \BaselineMT on the dissertations. While \name incurs a higher one-time cost to generate the concepts, it becomes less expensive when generating more questions. At $N = 60$ questions, \name has the same cost as \BaselineMT; when $N$ grows to 100 questions, \BaselineMT is \directcostinflation more expensive. 

% Details are in \autoref{appendix:costs}.

\begin{figure}[h]
    \centering
    \includegraphics[width=0.8\linewidth]{FIG/phd_cost_comparison.pdf}
    \caption{Average cost comparison of \BaselineMT and \name when generating questions from \numphd PhD dissertations. \name becomes less expensive as $N$ grows. We calculated costs by tracing prompt and completion tokens with OpenAI's February 2025 API pricing.}
    \label{fig:cost-scalability}
    \vspace{-20 pt}
\end{figure}

% \subsection{Discussion of Cost Scalability}
% \label{appendix:costs}

\name is also more cost-effective as the size of the document, $D$, grows. \BaselineMT costs $\approx \frac{N}{b} \cdot (A \cdot D + 100b \cdot B)$, where $A$ is cost per input token, $B$ is cost per output token, $N$ is the number of questions, $b$ is the batch size of \BaselineMT, and $100b$ is the approximate number of output tokens when generating $b$ questions. By contrast, \name costs $\approx f(D) + 100NB$ where $f(D)$ is the cost of main idea extraction, and $N$ is the number of questions. Thus, \name incurs a fixed cost that depends on the size of the document, but the marginal cost of generating additional questions is then independent of document size. By contrast, \Baseline incurs additional input token cost of $AD$ for each  batch of generated questions. 

In our experiments, for a PhD dissertation, $f(D) \approx 1.48A \cdot D$ on average.  Therefore, \name has lower cost when $\frac{N}{b} > 1.48$. For $N = 100$, \Baseline requires $b \approx 67$ to incur the same cost as \name, which is impractical with current LLMs. Both \texttt{GPT-4o} and \texttt{Meta-Llama-3.3-70B-Instruct} do not reliably generate more than $\approx$ 20 questions in a batch. 



In \autoref{fig:cost-scalability}, we also notate \BaselineMT with caching. Prompt caching is a feature made available from various LLM providers. It works by matching a prompt prefix, like a long system prompt or other long context from previous multi-turn conversations, to reduce computation time and API costs. As of writing in February 2025, the OpenAI API charged 50\% less for cached prompt tokens, resulting in up-to 80\% latency improvements. The \BaselineMT method benefits from this caching scheme, as it repeatedly sends the entire document as a cache prefix to the API. As shown in \autoref{fig:cost-scalability}, \BaselineMT is more cost-effective than \name up until $N \approx 80$ with prompt caching, as opposed to $N \approx 60$ without prompt caching.

However, prompt caching has several limitations. First, many providers evict cache entries after a short period of time, around 5-10 minutes. Thus, all $N$ questions must be generated within a set time frame to benefit. Moreover, many open-source model providers do not include prompt caching as a feature (as of the time of writing). Therefore, while we present the benefits that prompt caching may provide \BaselineMT, we still demonstrate that \name is more cost effective at large scale.

\section{Related Work}
\label{sec:related-work}

\textbf{Automated question-generation} has evolved from early Seq2Seq models \cite{du-etal-2017-learning, Neural_QG} to transformer-based approaches \cite{attention_is_all_you_need}. Models like BERT \cite{devlin-etal-2019-bert}, T5 \cite{t5}, BART \cite{lewis-etal-2020-bart}, and GPT-3 \cite{gpt_3} have significantly improved question generation \cite{chan-fan-2019-bert, li-etal-2021-addressing-semantic}. However, reliance on labeled datasets such as SQuAD \cite{rajpurkar-etal-2016-squad} and HotpotQA \cite{yang-etal-2018-hotpotqa} limits generalizability to other domains.

Researchers have explored LLMs for question generation~\cite{knowledge_base_prompting, reading_comprehension_language_llm, code_QG, mcq_computing, MIT_law}. However, these efforts have focused on generating questions from short, domain-specific context. Our work mitigates this limitation and generates high-quality questions from long documents.


Prior methods for \textbf{automated evaluation using LLMs} use metrics like ROUGE \cite{lin-2004-rouge} and BLEU \cite{papineni-etal-2002-bleu}, but these often misalign with humans~\cite{guo2024survey-neural-question-gen}. Some papers fine-tune small models for specific metrics \cite{zhu2023judgelm, wang2024pandalm}, but they face scalability issues, annotation reliance, or poor generalizability \cite{zhu2023judgelm}. Recent work uses  LLMs like GPT-4o as evaluators \cite{zheng2023judging, lin2023llmevalunifiedmultidimensionalautomatic}. While they achieve good human alignment, they focus on multi-turn conversations, a different context from ours.

For multiple-choice question generation, small models like BART and T5 assess relevance and usability \cite{moon-etal-2024-generative, raina2022multiplechoicequestiongenerationautomated} but require ground-truth data, limiting scalability. Others use LLM judges to rate relevance, coverage, and fluency on a 1-5 Likert scale \cite{microsoft_agriculture}. 
We adopt a similar approach with GPT-4o on a 1-4 scale. 
%for better human alignment. 
LLM judges can introduce positional \cite{zheng2024large, wang-etal-2024-large-language-models-fair}, egocentric \cite{koo-etal-2024-benchmarking}, and misinformation biases \cite{chen-etal-2024-humans, koo-etal-2024-benchmarking}.
%, which require mitigation.


\textbf{Retrieval-Augmented Generation (RAG)} enhances language model accuracy by retrieving relevant information to ground responses and reduce hallucinations~\cite{lewis2020retrieval, shuster2021retrieval, colbertv2, gottumukkala2022investigating}. Advances like dense passage retrieval~\cite{karpukhin2020dense} and late interaction models~\cite{colbert} improve efficiency. \name's pipeline uses recent advances in information retrieval models to fetch the most relevant context for question generation.

\section{Conclusion and Future Work}
\label{sec:concl}

\name uses LLMs and RAG in a concept-driven, three-stage framework to generate multiple-choice quizzes that assess deep understanding of large documents. Evaluations with \totalevaluators experts on \totaldocuments papers and dissertations show that, among those with a preference, \name outperforms a direct-prompting LLM baseline by 6.5$\times$ for dissertations and 1.5$\times$ for papers. Additionally, as document length increases, \name's advantages in question quality and cost efficiency become more pronounced.


We now discuss several avenues for future work.
While \name generates conceptual questions that test depth of understanding, few of them require mathematical analysis, logical reasoning, or creative thinking. \name produces quiz sessions, but we have not yet evaluated session quality. Currently, \name has not utilized human feedback to improve, which could be done using direct-preference optimization (DPO)~\cite{dpo}, Kahneman-Twersky Optimization (KTO) \cite{kto}, or reinforcement learning with human feedback (RLHF) \cite{rlhf}. To help learners, \name should adapt the difficulty of questions to the learner's answering accuracy and the time to answer questions. 


Our attempts to align AI-generated evaluations with human expert judgments have been unsuccessful. Further research is necessary to improve AI judges in educational contexts.
Finally, validating \name's domain-independence requires testing across a broader spectrum of fields. 
\section{Limitation}
The use of 3D-printed PLA for structural components improves improving ease of assembly and reduces weight and cost, yet it causes deformation under heavy load, which can diminish end-effector precision. Using metal, such as aluminum, would remedy this problem. Additionally, \robot relies on integrated joint relative encoders, requiring manual initialization in a fixed joint configuration each time the system is powered on. Using absolute joint encoders could significantly improve accuracy and ease of use, although it would increase the overall cost. 

%Reliance on commercially available actuators simplifies integration but imposes constraints on control frequency and customization, further limiting the potential for tailored performance improvements.

% The 6 DoF configuration provides sufficient mobility for most tasks; however, certain bimanual operations could benefit from an additional degree of freedom to handle complex joint constraints more effectively. Furthermore, the limited torque density of commercially available proprioceptive actuators restricts the payload and torque output, making the system less suitability for handling heavier loads or high-torque applications. 

The 6 DoF configuration of the arm provides sufficient mobility for single-arm manipulation tasks, yet it shows a limitation in certain bimanual manipulation problems. Specifically, when \robot holds onto a rigid object with both hands, each arm loses 1 DoF because the hands are fixed to the object during grasping. This leads to an underactuated kinematic chain which has a limited mobility in 3D space. We can achieve more mobility by letting the object slip inside the grippers, yet this renders the grasp less robust and simulation difficult. Therefore, we anticipate that designing a lightweight 3 DoF wrist in place of the current 2 DoF wrist allows a more diverse repertoire of manipulation in bimanual tasks.

Finally, the limited torque density of commercially available proprioceptive actuators restricts the performance. Currently, all of our actuators feature a 1:10 gear ratio, so \robot can handle up to 2.5 kg of payload. To handle a heavier object and manipulate it with higher torque, we expect the actuator to have 1:20$\sim$30 gear ratio, but it is difficult to find an off-the-shelf product that meets our requirements. Customizing the actuator to increase the torque density while minimizing the weight will enable \robot to move faster and handle more diverse objects.

%These constraints highlight opportunities for improvement in future iterations, including alternative materials for enhanced rigidity, custom actuator designs for higher control precision and torque density, the adoption of absolute joint encoders, and optimized configurations to balance dexterity and weight.



\section*{Acknowledgments}
We thank all the expert evaluators for their time, insights, and feedback. This work was funded in part by Quanta Computer, Inc. under the AIR Project.

\newpage
% \bibliography{refs}
% This must be in the first 5 lines to tell arXiv to use pdfLaTeX, which is strongly recommended.
% This must be in the first 5 lines to tell arXiv to use pdfLaTeX, which is strongly recommended.
\pdfoutput=1
% In particular, the hyperref package requires pdfLaTeX in order to break URLs across lines.

\documentclass[11pt]{article}

% Change "review" to "final" to generate the final (sometimes called camera-ready) version.
% Change to "preprint" to generate a non-anonymous version with page numbers.
\usepackage[final]{acl}
% \usepackage{acl}

% \usepackage[subtle, title=normal]{savetrees}


% Standard package includes
\usepackage{times}
\usepackage{latexsym}
\usepackage{xspace}
\usepackage{enumitem}
\usepackage{tabto}
\usepackage{subcaption}
\usepackage{tabularx}
\usepackage{array}
\usepackage{placeins}

% For proper rendering and hyphenation of words containing Latin characters (including in bib files)
\usepackage[T1]{fontenc}
% For Vietnamese characters
% \usepackage[T5]{fontenc}
% See https://www.latex-project.org/help/documentation/encguide.pdf for other character sets

% This assumes your files are encoded as UTF8
\usepackage[utf8]{inputenc}
\usepackage{csquotes}

% This is not strictly necessary, and may be commented out,
% but it will improve the layout of the manuscript,
% and will typically save some space.
\usepackage{microtype}

% This is also not strictly necessary, and may be commented out.
% However, it will improve the aesthetics of text in
% the typewriter font.
\usepackage{inconsolata}

%Including images in your LaTeX document requires adding
%additional package(s)
\usepackage{graphicx}
\usepackage{multirow}
\usepackage{amsmath}
\usepackage{amssymb}
\usepackage{mathtools}
\usepackage{amsthm}
\usepackage{caption} 
\captionsetup{aboveskip=8pt, belowskip=8pt}
\usepackage{tcolorbox}
\usepackage{aliascnt}
\usepackage{xcolor}
\usepackage{tikz}
\usepackage{colortbl}
\usepackage{pgf} % Ensure this package is included in your preamble
\usepackage{pgfmath}
\usepackage{ragged2e}


\makeatletter
\renewcommand{\sectionautorefname}{\S\@gobble}
\makeatother % Changes 'subs
\makeatletter
\renewcommand{\subsectionautorefname}{\S\@gobble}
\makeatother % Changes 'subs
\makeatletter
\renewcommand{\subsubsectionautorefname}{\S\@gobble} % Changes 'subs
\makeatother
\renewcommand{\equationautorefname}{Eqn} % Changes 'subs
\renewcommand{\figureautorefname}{Fig.} % Changes 'Figure' to 'Fig.'
\renewcommand{\tableautorefname}{Table} % Keeps 'Table' as is or change as needed

\def\compactify{\itemsep=0pt \topsep=0pt \partopsep=0pt \parsep=0pt}
\let\latexusecounter=\usecounter
\newenvironment{CompactEnumerate}
  {\def\usecounter{\compactify\latexusecounter}
   \begin{enumerate}}
  {\end{enumerate}\let\usecounter=\latexusecounter}



\tcbset{
    colframe=gray!40, 
    colback=gray!5,   
    coltitle=black,   
    fonttitle=\bfseries,
    sharp corners,
    boxrule=0.5mm,  
    width=\columnwidth,
    left=0.1mm,        
    right=0.1mm,      
    toptitle=0.1mm,
    bottomtitle=0.1mm,
    title=Question Generation Prompt
}
% If the title and author information does not fit in the area allocated, uncomment the following
%
%\setlength\titlebox{<dim>}
%
% and set <dim> to something 5cm or larger.

\title{\name: Scalable Concept-Driven Question Generation\\to Enhance Human Learning}

% Author information can be set in various styles:
% For several authors from the same institution:
% \author{Author 1 \and ... \and Author n \\
%         Address line \\ ... \\ Address line}
% if the names do not fit well on one line use
%         Author 1 \\ {\bf Author 2} \\ ... \\ {\bf Author n} \\
% For authors from different institutions:
% \author{Author 1 \\ Address line \\  ... \\ Address line
%         \And  ... \And
%         Author n \\ Address line \\ ... \\ Address line}
% To start a separate ``row'' of authors use \AND, as in
% \author{Author 1 \\ Address line \\  ... \\ Address line
%         \AND
%         Author 2 \\ Address line \\ ... \\ Address line \And
%         Author 3 \\ Address line \\ ... \\ Address line}

\author{
 \textbf{Kimia Noorbakhsh\textsuperscript{*}}, 
 \textbf{Joseph Chandler\textsuperscript{*}}, 
 \textbf{Pantea Karimi\textsuperscript{*}}, 
 \\
 \textbf{Mohammad Alizadeh}, 
 \textbf{Hari Balakrishnan}
\\
\\
 M.I.T. Computer Science and Artificial Intelligence Lab (CSAIL)
}


% \renewcommand{\thefootnote}{\textsuperscript{*}}
% \footnotetext{Equal contribution.}

%##########################Custom commands ###############
\definecolor{customblue}{HTML}{DAE8FC}
\definecolor{customred}{HTML}{F8CECC}
\definecolor{customgreen}{HTML}{D5E8D4}
\definecolor{custompurple}{HTML}{E1D5E7}
\definecolor{customorange}{HTML}{FFE6CC}


\newcommand*\circled[2]{\tikz[baseline=(char.base)]{
            \node[shape=circle, line width=0.75pt, draw=black, fill=#2, inner sep=1pt] 
            (char) {\textcolor{black}{\small\textsf{#1}}};}}


\newcommand{\bluecircle}{\raisebox{0pt}{\protect\tikz[baseline=(char.base)]{\protect\node[shape=circle,draw, fill=customblue, minimum size=1.8mm, inner sep=1pt] (char) {\footnotesize 1};}}}
\newcommand{\redcircle}{\raisebox{0pt}{\protect\tikz[baseline=(char.base)]{\protect\node[shape=circle,draw, fill=customred, minimum size=1.8mm, inner sep=1pt] (char) {\footnotesize 2};}}}
\newcommand{\greencircle}{\raisebox{0pt}{\protect\tikz[baseline=(char.base)]{\protect\node[shape=circle,draw, fill=customgreen, minimum size=1.8mm, inner sep=1pt] (char) {\footnotesize 3};}}}
\newcommand{\purplecircle}{\raisebox{0pt}{\protect\tikz[baseline=(char.base)]{\protect\node[shape=circle,draw, fill=custompurple, minimum size=1.8mm, inner sep=1pt] (char) {\footnotesize 4};}}}
\newcommand{\orangecircle}{\raisebox{0pt}{\protect\tikz[baseline=(char.base)]{\protect\node[shape=circle,draw, fill=customorange, minimum size=1.8mm, inner sep=1pt] (char) {\footnotesize 5};}}}



\newcommand*\sfboxed[1]{\tikz[baseline=(char.base)]{
            \node[shape=rectangle,line width=0.75pt, draw=black,inner sep=2pt, rounded corners=2pt] (char) {\textcolor{black}{\small\textsf{#1}}};}}
\newcommand{\NewPara}[1]{\noindent{\bf #1}}

\newcommand{\TheSystem}{Savaal\xspace}
\newcommand{\Baseline}{Direct\xspace}
\newcommand{\BaselineMT}{Direct\xspace}
\newcommand{\name}{Savaal\xspace}
\newcommand{\arxiv}{arXiv\xspace}
\newcommand{\gpt}{GPT-4o}

\begin{document}
\maketitle
\def\thefootnote{*}\footnotetext{These authors contributed equally to this work.}\def\thefootnote{\arabic{footnote}}
% Scatter plot PhD

\newcommand{\hitresponse}{38\%\xspace}

\newcommand{\understandingpreferration}{6.5$\times$\xspace}
\newcommand{\choicespreferration}{3$\times$\xspace}
\newcommand{\usabilitypreferration}{2$\times$\xspace}

\newcommand{\savaalunderstandingpreferrationpercent}{61.9\%\xspace}
\newcommand{\baselineunderstandingpreferrationpercent}{9.5\%\xspace}
\newcommand{\sameunderstandingpreferrationpercent}{28.6\%\xspace}

\newcommand{\savaalchoicespreferrationpercent}{57.1\%\xspace}
\newcommand{\baselinechoicespreferrationpercent}{19.0\%\xspace}
\newcommand{\samechoicespreferrationpercent}{23.9\%\xspace}

\newcommand{\savaalusabilitypreferrationpercent}{47.6\%\xspace}
\newcommand{\baselineusabilitypreferrationpercent}{23.8\%\xspace}
\newcommand{\sameusabilitypreferrationpercent}{28.6\%\xspace}

% Weighted average difference
\newcommand{\phdunderstandingWAD}{17\%\xspace}
\newcommand{\phdchoicesWAD}{10\%\xspace}
\newcommand{\phdusabilityWAD}{11.4\%\xspace}

% PhD


\newcommand{\baselinephdunderstandingvalue}{29.0}
\newcommand{\savaalphdunderstandingvalue}{11.9}

\newcommand{\baselinephdchoicesvalue}{39.0}
\newcommand{\savaalphdchoicesvalue}{29.0}


\newcommand{\baselinephdusabilityvalue}{32.9}
\newcommand{\savaalphdusabilityvalue}{21.4}


% Paper 
\newcommand{\baselinepaperunderstandingvalue}{16.7}
\newcommand{\savaalpaperunderstandingvalue}{10.9}

\newcommand{\baselinepaperchoicesvalue}{22.1}
\newcommand{\savaalpaperchoicesvalue}{21.8}


\newcommand{\baselinepaperusabilityvalue}{15.3}
\newcommand{\savaalpaperusabilityvalue}{13.8}



%%%%%%%%%%% CHNAGEABLES UP %%%%%%%%%%%

% Percent disagree macros
\newcommand{\savaalphdusability}{\savaalphdusabilityvalue\%\xspace}
\newcommand{\baselinephdusability}{\baselinephdusabilityvalue\%\xspace}
\newcommand{\phdusabilityreductionvalue}{%
  \pgfmathparse{round(\baselinephdusabilityvalue/\savaalphdusabilityvalue*100 )/100}%
  \pgfmathprintnumber[fixed, precision=1]{\pgfmathresult}%
}

\newcommand{\phdusabilityreduction}{\phdusabilityreductionvalue$\times$\xspace}

\newcommand{\savaalphdchoice}{\savaalphdchoicevalue\%\xspace}
\newcommand{\baselinephdchoice}{\baselinephdchoicevalue\%\xspace}
\newcommand{\phdchoicereductionvalue}{%
  \pgfmathparse{round(\baselinephdchoicevalue/\savaalphdchoicevalue*100 )/100}%
  \pgfmathprintnumber[fixed, precision=1]{\pgfmathresult}%
}

\newcommand{\phdchoicereduction}{\phdchoicereductionvalue$\times$\xspace}

\newcommand{\savaalphdunderstanding}{\savaalphdunderstandingvalue\%\xspace}
\newcommand{\baselinephdunderstanding}{\baselinephdunderstandingvalue\%\xspace}
\newcommand{\phdunderstandingreductionvalue}{%
  \pgfmathparse{round(\baselinephdunderstandingvalue/\savaalphdunderstandingvalue*100 )/100}%
  \pgfmathprintnumber[fixed, precision=1]{\pgfmathresult}%
}
\newcommand{\phdunderstandingreduction}{\phdunderstandingreductionvalue$\times$\xspace}


\newcommand{\savaalphdchoices}{\savaalphdchoicesvalue\%\xspace}
\newcommand{\baselinephdchoices}{\baselinephdchoicesvalue\%\xspace}
\newcommand{\phdchoicesreductionvalue}{%
  \pgfmathparse{round(\baselinephdchoicesvalue/\savaalphdchoicesvalue*100 )/100}%
  \pgfmathprintnumber[fixed, precision=1]{\pgfmathresult}%
}
\newcommand{\phdchoicesreduction}{\phdchoicesreductionvalue$\times$\xspace}

\newcommand{\savaalpaperusability}{\savaalpaperusabilityvalue\%\xspace}
\newcommand{\baselinepaperusability}{\baselinepaperusabilityvalue\%\xspace}
\newcommand{\paperusabilityreductionvalue}{%
  \pgfmathparse{round(\baselinepaperusabilityvalue/\savaalpaperusabilityvalue*100 )/100}%
  \pgfmathprintnumber[fixed, precision=1]{\pgfmathresult}%
}
\newcommand{\paperusabilityreduction}{\paperusabilityreductionvalue$\times$\xspace}

\newcommand{\savaalpaperunderstanding}{\savaalpaperunderstandingvalue\%\xspace}
\newcommand{\baselinepaperunderstanding}{\baselinepaperunderstandingvalue\%\xspace}
\newcommand{\paperunderstandingreductionvalue}{%
  \pgfmathparse{round(\baselinepaperunderstandingvalue/\savaalpaperunderstandingvalue*100 )/100}%
  \pgfmathprintnumber[fixed, precision=1]{\pgfmathresult}%
}

\newcommand{\paperunderstandingreduction}{\paperunderstandingreductionvalue$\times$\xspace}

\newcommand{\savaalpaperchoices}{\savaalpaperchoicesvalue\%\xspace}
\newcommand{\baselinepaperchoices}{\baselinepaperchoicesvalue\%\xspace}
\newcommand{\paperchoicesreductionvalue}{%
  \pgfmathparse{round(\baselinepaperchoicesvalue/\savaalpaperchoicesvalue*100 )/100}%
  \pgfmathprintnumber[fixed, precision=1]{\pgfmathresult}%
}

\newcommand{\paperchoicesreduction}{\paperchoicesreductionvalue$\times$\xspace}

% Num evaluators macros
\newcommand{\numemails}{200\xspace}

\newcommand{\numpaperevaluatorsvalue}{55}
\newcommand{\numphdevaluatorsvalue}{21}

\newcommand{\numpapersvalue}{50}
\newcommand{\numphdvalue}{21}

\newcommand{\numloogle}{48\xspace}
\newcommand{\numfields}{5\xspace}

% Scale of dataset macros
\newcommand{\avgphdpagesvalue}{142}
\newcommand{\avgphdpages}{\avgphdpagesvalue\xspace}
\newcommand{\avgpaperpagesvalue}{19}
\newcommand{\avgpaperpages}{\avgpaperpagesvalue\xspace}


% Math relations
\newcommand{\numphd}{\numphdvalue\xspace}
\newcommand{\numpapers}{\numpapersvalue\xspace}

\newcommand{\totaldocuments}{\pgfmathparse{int(\numphdvalue+\numpapersvalue)}\pgfmathresult\xspace}%

\newcommand{\numpaperevaluators}{\numpaperevaluatorsvalue\xspace}
\newcommand{\numphdevaluators}{\numphdevaluatorsvalue\xspace}

\newcommand{\totalevaluatorsvalue}{\pgfmathparse{int(\numpaperevaluatorsvalue+\numphdevaluatorsvalue)}\pgfmathresult}

\newcommand{\numpaperquestionsvalue}{\pgfmathparse{int(\numpaperevaluatorsvalue*20)}\pgfmathresult}


\newcommand{\numphdquestionsvalue}{\pgfmathparse{int(\numphdevaluatorsvalue*20)}\pgfmathresult}

\newcommand{\totalevaluators}{\totalevaluatorsvalue\xspace}
\newcommand{\numpaperquestions}{\numpaperquestionsvalue\xspace}
\newcommand{\numphdquestions}{\numphdquestionsvalue\xspace}


\newcommand{\numtotalhumanquestionvalue}{\pgfmathparse{int(\numphdevaluatorsvalue*20 + \numpaperevaluatorsvalue*20)}\pgfmathresult}

\newcommand{\numtotalhumanquestion}
{\numtotalhumanquestionvalue\xspace}


\newcommand{\savaalcostreduction}{1.39$\times$\xspace}
\newcommand{\directcostinflation}{1.64$\times$\xspace}

\begin{abstract}

Hypotheses are central to information acquisition, decision-making, and discovery. However, many real-world hypotheses are abstract, high-level statements that are difficult to validate directly. 
This challenge is further intensified by the rise of hypothesis generation from Large Language Models (LLMs), which are prone to hallucination and produce hypotheses in volumes that make manual validation impractical. Here we propose \mname, an agentic framework for rigorous automated validation of free-form hypotheses. 
Guided by Karl Popper's principle of falsification, \mname validates a hypothesis using LLM agents that design and execute falsification experiments targeting its measurable implications. A novel sequential testing framework ensures strict Type-I error control while actively gathering evidence from diverse observations, whether drawn from existing data or newly conducted procedures.
We demonstrate \mname on six domains including biology, economics, and sociology. \mname delivers robust error control, high power, and scalability. Furthermore, compared to human scientists, \mname achieved comparable performance in validating complex biological hypotheses while reducing time by 10 folds, providing a scalable, rigorous solution for hypothesis validation. \mname is freely available at \url{https://github.com/snap-stanford/POPPER}.




\end{abstract}

\section{Introduction}
\label{sec:intro}

Many people learn new material effectively by taking quizzes. Answering questions not only assesses knowledge, but also  reinforces learning by strengthening correct responses and revealing gaps in understanding. A major challenge in the 21st century is the rapid expansion of knowledge across fields like science, technology, medicine, law, finance, and more. AI tools are accelerating this growth, making it increasingly difficult for students, researchers, and professionals---from engineers to salespeople---to stay current. The need to learn efficiently and at scale has never been greater.


One response is to rely on AI for answers, effectively outsourcing expertise. While sometimes necessary, this does little to improve human understanding. Instead, we advocate using AI to enhance {\em our} ability to learn and master new material. %, and our work aims to advance this goal.


Programs like ChatGPT, Gemini, Claude, NotebookLM, Perplexity, and DeepSeek built atop large language models (LLMs) have made remarkable strides in summarization and question-answering. However, less attention has been given to {\em question generation}, specifically, creating high-quality questions that test human understanding and mastery of knowledge. That is the focus of this paper.

Anyone who has made an exam knows how difficult and time-consuming it is to make a good set of questions. Our goal is to produce questions automatically with three objectives:
\begin{CompactEnumerate}
\item {\em Scalability}: Generating questions across vast document corpora, such as rapidly evolving research fields or enterprise knowledge bases.
\item {\em Depth of understanding}: Producing questions beyond memorization and the superficial, requiring conceptual reasoning, synthesis, and analysis.
\item {\em Domain-independence}: Creating high-quality questions across diverse fields, including new material absent in an LLM’s pre-training data.
\end{CompactEnumerate}


Prior work on question generation has produced a small number of questions from short passages, but has not demonstrated scalability~\citep{du-etal-2017-learning, Neural_QG, chan-fan-2019-bert, li-etal-2021-addressing-semantic, knowledge_base_prompting, reading_comprehension_language_llm, code_QG, mcq_mult_sentence}. Our results (\autoref{sec:eval}) show that even well-engineered prompts to an LLM produce poor, repetitive questions on large text contexts (tens to hundreds of pages), highlighting the scalability challenge.
 

We present \textbf{\name}, a scalable question generation system for large documents. Savaal uses a three-stage pipeline. The first stage extracts and ranks the key concepts in a corpus of documents\footnote{We use ``document'' to also refer to the corpus of documents used to generate a quiz.} using a map-reduce computation. The second stage retrieves relevant passages corresponding to each concept with an efficient vector embedding retrieval model such as ColBERT~\cite{colbert}. Finally, the third stage prompts an LLM to generate questions for each ranked concept using the retrieved passages as context.

This approach scales well because each LLM computation is confined to a distinct, self-contained task while operating within a manageable context size. By first identifying core concepts and later synthesizing questions from all relevant passages, \name ensures that the generated questions are both targeted and conceptually rich, requiring deeper understanding by linking a given concept across different sections of a document.

We compare \name to a direct-prompting baseline (\Baseline) using \totalevaluators human expert evaluators (the primary authors of \numpapers recent conference papers and \numphd PhD dissertations in subfields of computer science and aeronautics) on \numtotalhumanquestion questions. We also evaluate each paper, as well as 48 arXiv papers, using an LLM as an AI judge.
%and also using an LLM judge on \numloogle papers from the Loogle \cite{loogle} dataset and \arxiv. 
%These evaluations span \numfields distinct fields. 
We find that: 
\begin{CompactEnumerate}
    \item On \numphdquestions questions from \numphdevaluators large documents (dissertations with average \avgphdpages pages), experts reported that \baselinephdunderstanding of \Baseline's questions {\em did not} test understanding, compared to \savaalphdunderstanding of \name, a \phdunderstandingreduction improvement. 
    They reported that \baselinephdchoices of \Baseline's questions lacked good choice quality, compared to \name's \savaalphdchoices, improving by \phdchoicesreduction. They found \baselinephdusability of \Baseline's questions {\em unusable} in a quiz, compared to \savaalphdusability of \name's questions, a \phdusabilityreduction reduction. Moreover, among experts with a preference, \understandingpreferration more favored \name over baseline in understanding, \choicespreferration in choice quality, and \usabilitypreferration in usability.


    \item Even on shorter documents, experts rated \name better in terms of depth of understanding and usability. On \numpaperquestions questions from \numpapers conference papers, \numpaperevaluators experts reported that \baselinepaperunderstanding of baseline's questions {\em did not} test understanding, compared to \savaalpaperunderstanding of \name, a \paperunderstandingreduction improvement. 
    

    
    \item \name is less expensive than \Baseline as the number of questions grows: \Baseline's cost for 100 questions generated from the dissertations is \directcostinflation higher than \name (\$0.47 vs. \$0.77 on average per document).

    
    \item There is a large gap between AI judgments and human evaluations. Despite several attempts to align the AI judge to human responses, scores remained misaligned.% with human expert evaluations.

\end{CompactEnumerate}



\section{Why is Generating Good Questions Hard?}
\label{sec:insights}

Our goal is to enhance human learning from large documents spanning dozens to hundreds of pages by generating multiple-choice questions. Multiple-choice questions are widely used in assessments, are easy to use by learners, and are easy to grade. The task involves generating a set of clear questions, each with four  choices and a correct answer.

High-quality questions assess {\em depth of understanding}, requiring conceptual reasoning and plausible choices (distractors) that challenge the learner. Beyond clarity and precision, our notion of a good question is one that could appear in an advanced quiz on the material as judged by a human expert. While this paper focuses on generating individual high-quality questions, effective quiz sessions should ensure {\em concept coverage} and {\em adapting the difficulty} to prior answers in the session, both avenues for future work.


The main challenge in scalable question generation using LLMs is selecting an appropriate context to use with LLM prompts. We examine four potential strategies: (i) using the full document corpus, (ii) dividing the corpus into sections, (iii) summarizing the corpus, and (iv) using content selection classifiers~\citep{context_Steuer, Context_diverse_hadifar}. Although each strategy has merits, we show that each strategy fails on at least one of our key objectives: {\em scalability}, {\em depth of understanding}, or {\em domain-independence}.

\newcommand{\questionbox}[1]{%
    \colorbox{customblue}{\parbox{0.97\linewidth}{\vspace{1pt}\textbf{#1}\vspace{1pt}}}
}

\begin{table*}[t]
\centering
\renewcommand{\arraystretch}{1.0} % Adjust row height for better readability
\begin{tabular}{|p{0.2\textwidth}|m{0.5\textwidth}|m{0.2\textwidth}|} 
\hline
\rowcolor{gray!20} % Light gray background for header row
\textbf{Context} & \textbf{Generated Question} & \textbf{Issue} \\ \hline
\footnotesize
\sfboxed{1} \textbf{Entire Document} & 
\footnotesize
\questionbox{What is the primary benefit of using the Adam optimizer in training the Transformer model?}
A. It reduces the need for dropout regularization. \newline
\textbf{B.} It adapts the learning rate based on the training step, improving convergence. \newline
C. It eliminates the need for positional encodings. \newline
D. It simplifies the model architecture by reducing the number of layers. 
&
\footnotesize
\textbf{Too general}: The question is about a basic property of the Adam optimizer rather than the key ideas of the paper.

$\Rightarrow$ Does not test depth of understanding
\\ \hline
\footnotesize
\sfboxed{2} \textbf{Document Section}& 
\footnotesize
\questionbox{In evaluating the performance and efficiency of the Transformer (big) model on the WMT 2014 English-to-French translation task, which of the following factors most significantly contributes to its ability to outperform previous models at a reduced training cost?}
A. The use of a dropout rate of 0.1 instead of 0.3, which enhances model regularization and reduces overfitting.\newline
B. The implementation of beam search with a beam size of 4 and a length penalty $\alpha$ = 0.6, which optimizes the translation output quality.\newline
\textbf{C.} The averaging of the last 20 checkpoints, which stabilizes the model's performance and improves translation accuracy.\newline
D. The reduction in training time to less than 1/4 of the previous state-of-the-art model, which directly correlates with improved BLEU scores.&
\footnotesize
\textbf{Irrelevant detail:} Because the method looks at one section at a time, it fixates on minutiae and irrelevant details (e.g., “averaging the last 20 checkpoints”) that may seem important in isolation, but are not.

$\Rightarrow$ Does not test depth of understanding
\\ \hline
\footnotesize
\sfboxed{3} \textbf{Document Summary} &
\footnotesize
\questionbox{How does the Transformer model address the challenge of learning dependencies between distant positions in sequences compared to models like ConvS2S and ByteNet?}
A. By using convolutional layers to capture long-range dependencies\newline
B. By increasing the number of layers in the encoder and decoder stacks\newline
C. By employing a recurrent neural network to process sequences\newline
\textbf{D.} By reducing the number of operations to a constant using self-attention mechanisms"
& 
\footnotesize
\textbf{Missing context:} The summary mentions ``...The Transformer model addresses this by reducing the number of operations to a constant, using self-attention mechanisms.'' which led the LLM design this incomplete question.


$\Rightarrow$ Leads to inaccurate questions
\\ \hline
\end{tabular}
\caption{Examples from the ``Attention Is All You Need'' paper \citep{attention_is_all_you_need} using three different context selection methods. The questions are drawn from three separate 20-question quizzes, each generated using a different method via OpenAI's API \citep{openai_api} with the \texttt{gpt-4o} model.}
\label{tab:bad-examples-attention}
\vspace{-10 pt}
\end{table*}


\subsection{Using the Entire Document Corpus}
\label{sec:insights-whole-context}

One approach is to provide the entire document as context to an LLM for quiz generation. However, this method has two major drawbacks.
First, as prior research shows~\citep{lost-in-the-middle}, LLMs allocate attention unevenly across long documents, focusing more on the beginning and end while largely neglecting the middle. 

Second, LLMs struggle to capture dependencies between different sections of a long document~\citep{loogle}, leading to superficial questions and missing key concepts. When we prompted OpenAI's \texttt{gpt-4o} model with the full text of the ``Attention Is All You Need'' paper~\citep{attention_is_all_you_need}, many of the 20 generated questions overlooked key ideas. See Example \sfboxed{1} in \autoref{tab:bad-examples-attention} for a question, which is not relevant to the paper's key ideas.

We found that LLMs struggle to follow instructions when the context length is large~\cite{gao2024insights}. For example, we instruct the LLM not to repeat questions. While it avoids repetition when generating a few questions, larger batches (e.g., 20 questions) often contain duplicates. 


\subsection{Using Document Sections}
\label{sec:insights-section-context}

An alternative is to split the document into sections, generate a limited number of questions per section, and later combine them into a quiz. While this method mitigates long-context issues, it introduces {\em context fragmentation}: the LLM cannot connect concepts spanning multiple sections. It often misses deeper connections that can assess stronger conceptual understanding. For example, key insights in a paper’s Algorithm or Methods section may be essential for understanding its Results, but treating these sections independently leads to disjointed, narrow questions.

Another issue is {\em uneven importance weighting}. Not all sections contribute equally to the document’s  ideas, yet a naïve section-based approach may overemphasize minor details while missing key concepts. Example \sfboxed{2} in \autoref{tab:bad-examples-attention} shows how this can generate irrelevant memorization questions.


\subsection{Summarization}
\label{sec:insights-summary}

Providing a {\em document summary} as context offers another way to streamline question generation. While LLMs are effective at summarization, summaries often lack critical details, leading to vague or incomplete questions. More concerning, summaries can introduce hallucinations~\citep{llm_hallucination}, distorting or misrepresenting causal relationships and fabricating details, further degrading question quality.

Example \sfboxed{3} in \autoref{tab:bad-examples-attention} illustrates how summarization can result in misleading or imprecise questions. Here, the summary includes a statement about using self-attention to ``reduce the number of operations to a constant'', but omits that this refers to {\em sequential} operations and maximum path length (Sec. 4 of \citep{attention_is_all_you_need}), leading to an inaccurate question. 



\subsection{Content Selection Classifiers}

Some methods attempt to select relevant content for question generation, often using trained models to identify key passages~\citep{context_Steuer, Context_diverse_hadifar}. However, these approaches typically require domain-specific training data (e.g., pre-existing question-answer pairs), making them {\em domain-dependent}. Such approaches are frequently limited in scope, making them neither reliable nor generalizable to diverse domains. 

\begin{figure*}[!t]
\centering
    \includegraphics[width=1\linewidth]{FIG/savaal.drawio.pdf}
    \caption{\name's Pipeline. \bluecircle\ \name extracts main ideas from sections of the document in parallel, \redcircle\ combines them into a succinct list, and \greencircle\ ranks them in order of importance. Next, \purplecircle\ \name fetches relevant passages from the document using a vector-based retrieval model. Finally, \orangecircle\ given a main idea and fetched passages, \name generates questions.}
    \label{fig:savaal-workflow}
\vspace{-10 pt}
\end{figure*}

\section{\name's Question-Generation Pipeline}
\label{sec:pipeline}


To address challenges of \autoref{sec:insights}, we propose a novel three-stage pipeline: \emph{main idea extraction}, \emph{relevant passage retrieval}, and \emph{question generation}. \autoref{fig:savaal-workflow} shows Savaal's workflow. The idea is to generate questions targeted at key explicitly determined concepts and to retrieve passages relevant to the concept from the source document.
% to generate questions.


\subsection{Extracting Main Ideas}
\label{sec:pipeline-main-idea}
This stage extracts succinct main ideas from different document chapters. This is done in a map-combine-reduce fashion~\cite{langchain_mapreduce}. First, we use GROBID~\citep{GROBID} to parse and segment documents into distinct sections.


In the map stage, \circled{1}{customblue}, we use an LLM to extract the main ideas for each section individually. These extracted main ideas are aggregated and deduplicated in the combine stage, \circled{2}{customred}, into a single, cohesive list of the paper’s main ideas. If the combined output exceeds a predefined length threshold (set to the maximum token window of the LLM), the reduce stage collapses the list further for brevity and clarity. The result is a curated list of main ideas, including main idea titles and their short descriptions (see \autoref{subsubsec:example_main_idea} for examples). The same (or a different) LLM then ranks the main ideas based on their importance in the ranking stage in \circled{3}{customgreen} (see \autoref{subsec:appendix_prompts} for the prompts).

Initially, we attempted to extract the main ideas for the entire document in one shot. However, as noted in \autoref{sec:insights-whole-context}, as the context length grew, this became less effective. We found that using map-reduce extracted main ideas that were more detailed and useful for question generation, particularly on large documents.



\subsection{Retrieving Relevant Passages}
\label{sec:pipeline-retrieval}

Because the main ideas in \autoref{sec:pipeline-main-idea} are concise, they lack sufficient content to generate a question. As discussed in \autoref{sec:insights-summary}, asking an LLM to generate questions based on a concept alone (a main idea or even a summary) has shortcomings. To overcome this problem, \name retrieves relevant text segments directly from the original document to provide granular content for generating a question and to ensure that the questions are grounded in truth.


\name's retriever uses ColBERT, a late-interaction retrieval method~\citep{colbert, colbertv2}, to find the most relevant passages for each main idea (stage \circled{4}{custompurple}).
% integrated in the RAGatouille \footnote{\url{https://github.com/AnswerDotAI/RAGatouille}} library.
For each ranked main idea in \circled{3}{customgreen}, we retrieve the top $k$ passages as added context for the next stage ($k=3$ in our experiments).

We chose ColBERT for its state-of-the-art performance and wide adoption, but any high-performing retrieval method could be used. We also tried using the LLM to identify passages related to a main idea, but as in \autoref{sec:insights-whole-context} and \autoref{sec:pipeline-main-idea}, it struggled with large context sizes.


\subsection{Generating Questions and Choices}
\label{sec:pipeline-QG}


After retrieving the passages for each main idea, stage \circled{5}{customorange} instructs an LLM to generate questions. To create $N$ questions from $M$ ideas, we generate $N/M$ questions per idea.\footnote{We use only the top $N$ ranked main ideas if $N < M$.}  The prompt (\autoref{fig:question_generation}) includes the main idea and its retrieved passages.


Although LLMs often produce good questions, generating good {\em choices} is more challenging. Poorly designed choices can make the correct answer too obvious or, worse, introduce ambiguity or multiple correct options. We experimented with many prompt variations to improve choice quality, yielding mixed results. We also tested a separate ``choice refinement'' stage, where an LLM was specifically instructed to improve the answer choices for a given question. This prompt included detailed constraints, such as ensuring alignment with the question's intent (e.g., a question about benefits should not include limitations as choices; see \autoref{appendix:choice-refine}).
Although this additional step produced more challenging choices, we found that it caused excessive ambiguity and was less preferred by human expert evaluators. Therefore, \name does not include a choice refinement stage. Instead, its question-generation prompt explicitly emphasizes that the choices should be ``plausible distractors''.

Finally, we observed {\em positional biases} in the placement of the correct choice, corroborating prior findings~\cite{pezeshkpour2023large}. For example, in a set of 1000 questions from 50 papers (20 per paper) generated by \texttt{GPT-4o}, choice B was correct 73.3\% of the time! Thus, we randomize the choices to eliminate this bias.

\section{Evaluation}
\label{sec:eval}

We evaluated \name on \totaldocuments documents using both human experts and an AI judge. We used \texttt{GPT-4o} via the OpenAI API as our primary LLM. We also evaluated \texttt{Meta-Llama-3.3-70B-Instruct} (\autoref{subsec:ablation-model}). All models are set to temperature 0.0 for all experiments, with default settings for all other parameters. \name is model-agnostic and is compatible with current LLMs. We implemented our pipeline using LangChain~\cite{langchain} and traced our experiments in Weights \& Biases~\cite{wandb}.
%any commercial or open-source LLM.

\subsection{Datasets}
\label{sec:evaluation-data}
%We gathered three datasets for evaluation:
\begin{itemize}[topsep=0pt, itemsep=0pt, leftmargin=*]
    \item \textbf{PhD dissertations}: \numphd long documents in Aerospace, Machine Learning, Networks, Systems, and Databases (\autoref{tab:human-eval-dataset-stats}).
    \item \textbf{Conference papers}: \numpapers papers from conferences in CS and Aeronautics in 2023 and 2024.
    \item \textbf{Diverse \arxiv papers}: \numloogle papers from CS, Physics, Mathematics, Economics, and Biology (\autoref{tab:benchmark-stats}). 
\end{itemize}

\begin{table}[h]
\centering
\renewcommand{\arraystretch}{1} % Increase row height for better readability
\setlength{\tabcolsep}{1pt} % Adjust column spacing for better fit
% \begin{small}
\begin{tabular}{|l|c|c|}
\hline
\small  \textbf{Statistic} & \small  \textbf{Conference Papers} & \small \textbf{Dissertations} \\
\hline
\small \textbf{No. Documents} & \small \numpapers & \small \numphd \\ 
\hline
\small \textbf{Avg. Words} & \small 10,354 & \small 26,511 \\ 
\hline
\small \textbf{Avg. Pages} & \small \avgpaperpages & \small \avgphdpages \\
\hline
\end{tabular}
\caption{Statistics for the number of words in the conference papers and PhD dissertations.}
\label{tab:human-eval-dataset-stats}
% \end{small}
\end{table}



\subsection{Methods Compared}
\label{sec:evaluation-baselines}
We compare \name to \Baseline, a direct-prompting baseline (\autoref{sec:insights-whole-context}) that provides the entire document to the LLM with a detailed prompt to generate $N$ multiple-choice questions (\autoref{fig:baseline_question_generation_prompt}). We found that when $N$ exceeds $\approx$ 20, \Baseline fails to produce $N$ distinct questions. Since broad concept coverage requires generating a large pool of questions and sampling for shorter quizzes, we generate $N > 20$ questions in batches of $b=20$ using an additional prompt (\autoref{fig:baseline_large_question_generation_prompt}). We use this {\em multi-turn method} for \Baseline on longer documents. 


We evaluate other methods using the AI judge: Summary (\autoref{sec:insights-summary}) and Single-Prompt Savaal, which condenses Savaal's idea extraction, retrieval, and question generation into a single prompt (\autoref{subsec:ablation-methods}).


\subsection{Evaluation Criteria}
\label{sec:evaluation-metrics}

Evaluating the quality of questions is challenging because it involves subjective human judgment~\cite{fu2024qgeval}. We primarily rely on human evaluations but also use \texttt{GPT-4o} as an AI judge~\cite{naismith2023automated} to expand the scope of our evaluation to more approaches, documents, and criteria. 

% Exempt ID: E-6417

\paragraph{Human experts:} We generated 10 multiple-choice questions from Savaal and 10 from \Baseline for each of the \numphd PhD dissertations and \numpapers conference papers. We contacted the primary authors to evaluate the quality of questions via a secure web-based feedback form.\footnote{\emph{MIT} Institutional Review Board exempted this study (Exemption Number: E-6417). All the personnel were certified, and participants were over 18 years of age and provided informed consent before participating.} We asked each expert to rate their questions on clarity, depth of understanding\footnote{Used interchangeably with understanding.}, and quality of choices using a four-point scale: \emph{Disagree}, \emph{Somewhat Disagree}, \emph{Somewhat Agree}, and \emph{Agree}. They also assessed usability by answering: ``Would I use this question on a graduate-level quiz?'' with options: {\em Yes}, {\em Yes with small changes}, and {\em No}. The questions were randomly mixed and the evaluators were blind to their source. In all, \totalevaluators experts participated (\autoref{subsec:appendix_human_eval_conduct}).
 

\label{sec:metrics-auto}
\paragraph{AI judge:} We prompted \texttt{GPT-4o} at temperature 0.0 to score each question on a 1–4 scale (\autoref{subsubsec:eval-prompts}) on Depth of Understanding, Quality of Choices, Clarity, Usability, Difficulty, Cognitive Level, and Engagement (\autoref{subsec:ablation-metrics}). Our evaluation prompts provide detailed guidelines than those given to humans, including explicit criteria for each rating (\autoref{subsubsec:eval-prompts}).



\subsection{Results with Human Experts}
\label{sec:evaluation-results}

\begin{figure}[!t]
    \centering
    \begin{subfigure}[b]{0.9\linewidth}
        \centering
        \includegraphics[width=1\linewidth]{FIG/thesis_only_disagree_all_metrics_bar_charts.pdf}
        \caption{PhD dissertations: \numphdquestions questions, \numphdevaluators experts.}
        \label{fig:human-eval-disagree-phd}
    \end{subfigure}
    \hfill
    \begin{subfigure}[b]{0.9\linewidth}
        \centering
        \includegraphics[width=1\linewidth]{FIG/no_dedup_no_refine_final_only_disagree_all_metrics_bar_charts.pdf}
        \caption{Conference papers: \numpaperquestions questions, \numpaperevaluators experts.}
        \label{fig:human-eval-disagree-paper}
    \end{subfigure}
    \vspace{-10 pt}
    \caption{ Summary of human evaluation: The charts show the percentage and standard error of respondents who {\em Disagree} or {\em Somewhat Disagree} with questions on understanding, choice quality, and usability. {\bf Lower values indicate better performance.}}
    \label{fig:human-eval-disagree}
\vspace{-20 pt}
\end{figure}

\begin{figure*}[!t]
    \centering
    \begin{subfigure}[b]{0.32\linewidth}
        \centering
        \includegraphics[width=\linewidth]{FIG/thesis_understanding_num_AGREE_scatter.pdf}
        \caption{Depth of understanding: 61.9\% prefer \name, 9.5\% \Baseline.}
        \label{fig:thesis-scatter-understanding}
    \end{subfigure}
    \hfill
    \begin{subfigure}[b]{0.32\linewidth}
        \centering
        \includegraphics[width=\linewidth]{FIG/thesis_quality_of_choices_num_AGREE_scatter.pdf}
        \caption{Quality of choices: 57.1\% prefer \name, 19\% \Baseline.}
        \label{fig:thesis-scatter-choices}
    \end{subfigure}
    \hfill
    \begin{subfigure}[b]{0.32\linewidth}
        \centering
        \includegraphics[width=\linewidth]{FIG/thesis_overall_quality_num_AGREE_scatter.pdf}
        \caption{Usability: 47.6\% prefer \name, 23.8\% \Baseline.}
        \label{fig:thesis-scatter-overall}
    \end{subfigure}
    
    \caption{Expert preferences for \numphdevaluators PhD dissertations. Each point shows the number of \emph{Agree}s or \emph{Somewhat Agree}s in a 10-question quiz for each of \name and \Baseline. The majority of experts prefer \name to \Baseline on depth of understanding, quality of choices, and usability on long documents (experts above $y=x$ prefer \name).}
    \label{fig:human-eval-scatter}
    \vspace{-10 pt}
\end{figure*}


\label{sec:evaluation-human}

\autoref{fig:human-eval-disagree} summarizes the results of our expert human evaluation on PhD dissertations and papers. We show here the negative sentiment of the experts, i.e., the percentage of questions that experts responded with \emph{Disagree} or \emph{Somewhat Disagree} for each criterion (see \autoref{fig:human-phd-breakdown} and \autoref{fig:human_paper_breakdown} for the full breakdown). 

For the \numphdquestions questions from \numphd PhD dissertations (\autoref{fig:human-eval-disagree-phd}), the experts responded that \baselinephdunderstanding of \Baseline's questions {\em did not test understanding}; by contrast, only \savaalphdunderstanding of \name's questions did not, a  \phdunderstandingreduction reduction in negative sentiment. They also rated \baselinephdusability of \Baseline's questions as {\em unusable in a quiz}, versus \savaalphdusability for \name, a \phdusabilityreduction reduction.

For  conference papers (\autoref{fig:human-eval-disagree-paper}), on \numpaperquestions questions, \numpaperevaluators experts\footnote{Some papers had multiple expert respondents.} found that \savaalpaperunderstanding of \name's questions {\em did not} test understanding, versus \baselinepaperunderstanding for \Baseline, a \paperunderstandingreduction improvement. They also rated \baselinepaperusability of \Baseline's questions as {\em unusable}, versus \savaalpaperusability for \name.

The experts agreed or somewhat agreed that over 90\% of the questions in both \Baseline and \name had clarity (not shown in the figure). This result is unsurprising because LLMs can be prompted to generate coherent and unambiguous text. 

For PhD dissertations, \autoref{fig:human-eval-scatter} shows how each of the \numphdevaluators experts scored \name vs. \Baseline on the metrics for the PhD dissertations. The $x$ and $y$ axes show number of \emph{Agree} or \emph{Somewhat Agree} for \Baseline and \name, respectively. Each point represents one expert evaluator. 

We observe that \savaalunderstandingpreferrationpercent favor \name over \Baseline for understanding (\autoref{fig:thesis-scatter-understanding}), whereas only \baselineunderstandingpreferrationpercent (\understandingpreferration fewer) prefer \Baseline over \name (\sameunderstandingpreferrationpercent rate the two systems the same). For choice quality, \savaalchoicespreferrationpercent prefer \name compared to \baselinechoicespreferrationpercent for \Baseline (\choicespreferration more, see \autoref{fig:thesis-scatter-choices}), while for usability \savaalusabilitypreferrationpercent prefer \name compared to \baselineusabilitypreferrationpercent for \Baseline (\usabilitypreferration more, see \autoref{fig:thesis-scatter-overall}). 

The data in \autoref{fig:human-eval-scatter} also shows that, on average, expert evaluators rated \emph{Agree} or \emph{Somewhat Agree} for more questions in \name quizzes than \Baseline: \phdunderstandingWAD more for understanding, \phdchoicesWAD more for quality of choices, and \phdusabilityWAD more for usability.

\autoref{fig:human_paper_breakdown} shows the breakdown of expert responses for \numpaperquestions questions from the conference papers. On these shorter documents, experts slightly prefer \name over \Baseline in terms of depth of understanding. They reported that 16.7\% of \TheSystem's questions {\em did not} test understanding, compared to 10.9\% for \Baseline. Experts rated the two methods similarly for choice quality and usability. As in the results for Ph.D. dissertations (\autoref{fig:human-auto-correlation}), the \texttt{GPT-4o} scores (\autoref{fig:AI_paper_breakdown}) correlated poorly with expert evaluations.


\autoref{fig:paper-human-eval-scatter} shows how each of the \numpaperevaluators experts scored \name vs. \Baseline. The $x$-axis shows the number of \emph{Agree} or \emph{Somewhat Agree} for \Baseline, and the $y$-axis shows the same for \name. Each point represents one expert evaluator. Among evaluators with a preference, 1.5$\times$ more experts favor \TheSystem over \Baseline in understanding (34.5\% for \name vs 21.8\% for \Baseline, \autoref{fig:paper-scatter-understanding}). Experts do not exhibit a strong preference between \name and \Baseline for choice quality (\autoref{fig:paper-scatter-choices}) or usability (\autoref{fig:paper-scatter-overall}). The average relative increase in the Agree score for \TheSystem compared to \Baseline is 5.8\% for understanding, 4\% for quality of choices, and 1.5\% for usability.
% , meaning that on average, experts like at least one more question in \name's quizzes compared to \Baseline.


\begin{figure*}[h]
    \centering
    \begin{subfigure}[b]{0.32\linewidth}
        \centering
        \includegraphics[width=\linewidth]{FIG/no_dedup_no_refine_final_understanding_num_AGREE_scatter.pdf}
        \caption{Depth of understanding: 34.5\% prefer \name, 21.8\% prefer \Baseline.}
        \label{fig:paper-scatter-understanding}
    \end{subfigure}
    \hfill
    \begin{subfigure}[b]{0.32\linewidth}
        \centering
        \includegraphics[width=\linewidth]{FIG/no_dedup_no_refine_final_quality_of_choices_num_AGREE_scatter.pdf}
        \caption{Quality of choices: no specific preference exhibited.}
        \label{fig:paper-scatter-choices}
    \end{subfigure}
    \hfill
    \begin{subfigure}[b]{0.32\linewidth}
        \centering
        \includegraphics[width=\linewidth]{FIG/no_dedup_no_refine_final_overall_quality_num_AGREE_scatter.pdf}
        \caption{Usability: no specific preference exhibited.}
        \label{fig:paper-scatter-overall}
    \end{subfigure}
    
    \caption{Human expert preferences for \numpaperevaluators experts on short conference papers. Each point shows the number of \emph{Agree}s in a 10-question quiz for \name and \Baseline respectively. More experts prefer \name to \Baseline on the depth of understanding. Experts don't exhibit any preference between the quality of choices and usability on short documents (experts above $y=x$ prefer \name).}
    \label{fig:paper-human-eval-scatter}
\end{figure*}


\subsection{Results with an AI Judge}
\label{sec:evaluation-auto}


We used an AI judge to scale evaluations across more documents and criteria. We first examined its alignment with human experts by having \texttt{GPT-4o} evaluate the same \numphdquestions questions from the expert-reviewed dissertations dataset. 

\autoref{fig:human-auto-correlation} compares the AI judge with human experts. The AI judge rarely assigns \emph{Disagree} or \emph{Somewhat Disagree} for understanding and usability and slightly favors \name, giving it 28.6\% Agree rating in comparison to 14.3\% Agree ratings for \Baseline for understanding. However, for quality of choices, it rates both schemes poorly, with only 9.6\% \emph{Agree} or \emph{Somewhat Agree} for \name and 19\% for \Baseline.

We observed similar trends in the \numpaperquestions questions from the conference-paper dataset (\autoref{fig:paper_breakdown}), where the AI judge again slightly preferred \name but remained misaligned with human expert evaluations. For completeness, we also present AI judge results on the Diverse \arxiv dataset in \autoref{subsec:ablation}.

%\paragraph{Limitations of the AI judge.} 
Our takeaway is that our \texttt{GPT-4o} AI judge was unaligned with human expert judgments (see \autoref{fig:auto-correlation-ai} vs. \autoref{fig:human-phd-breakdown}). Despite our extensive efforts in prompt engineering to maximize alignment---including using the prompt optimizer program in DSPy~\citep{khattab2024dspy}---AI-human correlation did not improve. Our experience calls into question the wisdom of using only AI judges in research studies. 




\begin{figure}[t]
    \centering
    \begin{subfigure}[b]{1\linewidth}
        \centering
        \includegraphics[width=1\linewidth]{FIG/thesis_no_combine_all_metrics_bar_charts.pdf}
        \caption{Breakdown of human expert scores on PhD dissertations.}
        \label{fig:human-phd-breakdown}
    \end{subfigure}  
    \hfill
    \begin{subfigure}[b]{1\linewidth}
        \centering
        \includegraphics[width=1\linewidth]{FIG/FINAL_PLOTS_AUTO/thesis_human_auto_all_metrics_bar_charts.pdf}
        \caption{Breakdown of GPT-4o AI judge scores on PhD dissertations.}
        \label{fig:auto-correlation-ai}
    \end{subfigure}
    \caption{Score distribution for  \protect\numphdquestions questions from dissertations: GPT-4o as a judge does not align with humans for assessing the metrics.}
    \label{fig:human-auto-correlation}
    \vspace{-20 pt}
\end{figure}


\begin{figure}[h]
    \centering
    \begin{subfigure}{\linewidth}
        \centering
        \includegraphics[width=\linewidth]{FIG/no_dedup_no_refine_final_no_combine_all_metrics_bar_charts.pdf}
        \caption{Breakdown of human expert scores on conference papers.}
        \label{fig:human_paper_breakdown}
    \end{subfigure}
    \hfill
    \begin{subfigure}{\linewidth}
        \centering
        \includegraphics[width=\linewidth]{FIG/FINAL_PLOTS_AUTO/papers_human_auto_all_metrics_bar_charts.pdf}
        \caption{Breakdown of GPT-4o scores on conference papers.}
        \label{fig:AI_paper_breakdown}
    \end{subfigure}
    \caption{Score distribution for  \protect \numpaperquestions questions from conference papers.}
    \label{fig:paper_breakdown}
    \vspace{-10 pt}
\end{figure}





\subsection{Cost Scalability}
\label{sec:scalability-case}

\autoref{fig:cost-scalability} compares the costs of \name and \BaselineMT on the dissertations. While \name incurs a higher one-time cost to generate the concepts, it becomes less expensive when generating more questions. At $N = 60$ questions, \name has the same cost as \BaselineMT; when $N$ grows to 100 questions, \BaselineMT is \directcostinflation more expensive. 

% Details are in \autoref{appendix:costs}.

\begin{figure}[h]
    \centering
    \includegraphics[width=0.8\linewidth]{FIG/phd_cost_comparison.pdf}
    \caption{Average cost comparison of \BaselineMT and \name when generating questions from \numphd PhD dissertations. \name becomes less expensive as $N$ grows. We calculated costs by tracing prompt and completion tokens with OpenAI's February 2025 API pricing.}
    \label{fig:cost-scalability}
    \vspace{-20 pt}
\end{figure}

\input{cost_detail}
\section{Related Work}
\label{sec:related-work}

\textbf{Automated question-generation} has evolved from early Seq2Seq models \cite{du-etal-2017-learning, Neural_QG} to transformer-based approaches \cite{attention_is_all_you_need}. Models like BERT \cite{devlin-etal-2019-bert}, T5 \cite{t5}, BART \cite{lewis-etal-2020-bart}, and GPT-3 \cite{gpt_3} have significantly improved question generation \cite{chan-fan-2019-bert, li-etal-2021-addressing-semantic}. However, reliance on labeled datasets such as SQuAD \cite{rajpurkar-etal-2016-squad} and HotpotQA \cite{yang-etal-2018-hotpotqa} limits generalizability to other domains.

Researchers have explored LLMs for question generation~\cite{knowledge_base_prompting, reading_comprehension_language_llm, code_QG, mcq_computing, MIT_law}. However, these efforts have focused on generating questions from short, domain-specific context. Our work mitigates this limitation and generates high-quality questions from long documents.


Prior methods for \textbf{automated evaluation using LLMs} use metrics like ROUGE \cite{lin-2004-rouge} and BLEU \cite{papineni-etal-2002-bleu}, but these often misalign with humans~\cite{guo2024survey-neural-question-gen}. Some papers fine-tune small models for specific metrics \cite{zhu2023judgelm, wang2024pandalm}, but they face scalability issues, annotation reliance, or poor generalizability \cite{zhu2023judgelm}. Recent work uses  LLMs like GPT-4o as evaluators \cite{zheng2023judging, lin2023llmevalunifiedmultidimensionalautomatic}. While they achieve good human alignment, they focus on multi-turn conversations, a different context from ours.

For multiple-choice question generation, small models like BART and T5 assess relevance and usability \cite{moon-etal-2024-generative, raina2022multiplechoicequestiongenerationautomated} but require ground-truth data, limiting scalability. Others use LLM judges to rate relevance, coverage, and fluency on a 1-5 Likert scale \cite{microsoft_agriculture}. 
We adopt a similar approach with GPT-4o on a 1-4 scale. 
%for better human alignment. 
LLM judges can introduce positional \cite{zheng2024large, wang-etal-2024-large-language-models-fair}, egocentric \cite{koo-etal-2024-benchmarking}, and misinformation biases \cite{chen-etal-2024-humans, koo-etal-2024-benchmarking}.
%, which require mitigation.


\textbf{Retrieval-Augmented Generation (RAG)} enhances language model accuracy by retrieving relevant information to ground responses and reduce hallucinations~\cite{lewis2020retrieval, shuster2021retrieval, colbertv2, gottumukkala2022investigating}. Advances like dense passage retrieval~\cite{karpukhin2020dense} and late interaction models~\cite{colbert} improve efficiency. \name's pipeline uses recent advances in information retrieval models to fetch the most relevant context for question generation.

\section{Conclusion and Future Work}
\label{sec:concl}

\name uses LLMs and RAG in a concept-driven, three-stage framework to generate multiple-choice quizzes that assess deep understanding of large documents. Evaluations with \totalevaluators experts on \totaldocuments papers and dissertations show that, among those with a preference, \name outperforms a direct-prompting LLM baseline by 6.5$\times$ for dissertations and 1.5$\times$ for papers. Additionally, as document length increases, \name's advantages in question quality and cost efficiency become more pronounced.


We now discuss several avenues for future work.
While \name generates conceptual questions that test depth of understanding, few of them require mathematical analysis, logical reasoning, or creative thinking. \name produces quiz sessions, but we have not yet evaluated session quality. Currently, \name has not utilized human feedback to improve, which could be done using direct-preference optimization (DPO)~\cite{dpo}, Kahneman-Twersky Optimization (KTO) \cite{kto}, or reinforcement learning with human feedback (RLHF) \cite{rlhf}. To help learners, \name should adapt the difficulty of questions to the learner's answering accuracy and the time to answer questions. 


Our attempts to align AI-generated evaluations with human expert judgments have been unsuccessful. Further research is necessary to improve AI judges in educational contexts.
Finally, validating \name's domain-independence requires testing across a broader spectrum of fields. 
\section{Limitation}
The use of 3D-printed PLA for structural components improves improving ease of assembly and reduces weight and cost, yet it causes deformation under heavy load, which can diminish end-effector precision. Using metal, such as aluminum, would remedy this problem. Additionally, \robot relies on integrated joint relative encoders, requiring manual initialization in a fixed joint configuration each time the system is powered on. Using absolute joint encoders could significantly improve accuracy and ease of use, although it would increase the overall cost. 

%Reliance on commercially available actuators simplifies integration but imposes constraints on control frequency and customization, further limiting the potential for tailored performance improvements.

% The 6 DoF configuration provides sufficient mobility for most tasks; however, certain bimanual operations could benefit from an additional degree of freedom to handle complex joint constraints more effectively. Furthermore, the limited torque density of commercially available proprioceptive actuators restricts the payload and torque output, making the system less suitability for handling heavier loads or high-torque applications. 

The 6 DoF configuration of the arm provides sufficient mobility for single-arm manipulation tasks, yet it shows a limitation in certain bimanual manipulation problems. Specifically, when \robot holds onto a rigid object with both hands, each arm loses 1 DoF because the hands are fixed to the object during grasping. This leads to an underactuated kinematic chain which has a limited mobility in 3D space. We can achieve more mobility by letting the object slip inside the grippers, yet this renders the grasp less robust and simulation difficult. Therefore, we anticipate that designing a lightweight 3 DoF wrist in place of the current 2 DoF wrist allows a more diverse repertoire of manipulation in bimanual tasks.

Finally, the limited torque density of commercially available proprioceptive actuators restricts the performance. Currently, all of our actuators feature a 1:10 gear ratio, so \robot can handle up to 2.5 kg of payload. To handle a heavier object and manipulate it with higher torque, we expect the actuator to have 1:20$\sim$30 gear ratio, but it is difficult to find an off-the-shelf product that meets our requirements. Customizing the actuator to increase the torque density while minimizing the weight will enable \robot to move faster and handle more diverse objects.

%These constraints highlight opportunities for improvement in future iterations, including alternative materials for enhanced rigidity, custom actuator designs for higher control precision and torque density, the adoption of absolute joint encoders, and optimized configurations to balance dexterity and weight.



\section*{Acknowledgments}
We thank all the expert evaluators for their time, insights, and feedback. This work was funded in part by Quanta Computer, Inc. under the AIR Project.

\newpage
% \bibliography{refs}
% This must be in the first 5 lines to tell arXiv to use pdfLaTeX, which is strongly recommended.
% This must be in the first 5 lines to tell arXiv to use pdfLaTeX, which is strongly recommended.
\pdfoutput=1
% In particular, the hyperref package requires pdfLaTeX in order to break URLs across lines.

\documentclass[11pt]{article}

% Change "review" to "final" to generate the final (sometimes called camera-ready) version.
% Change to "preprint" to generate a non-anonymous version with page numbers.
\usepackage[final]{acl}
% \usepackage{acl}

% \usepackage[subtle, title=normal]{savetrees}


% Standard package includes
\usepackage{times}
\usepackage{latexsym}
\usepackage{xspace}
\usepackage{enumitem}
\usepackage{tabto}
\usepackage{subcaption}
\usepackage{tabularx}
\usepackage{array}
\usepackage{placeins}

% For proper rendering and hyphenation of words containing Latin characters (including in bib files)
\usepackage[T1]{fontenc}
% For Vietnamese characters
% \usepackage[T5]{fontenc}
% See https://www.latex-project.org/help/documentation/encguide.pdf for other character sets

% This assumes your files are encoded as UTF8
\usepackage[utf8]{inputenc}
\usepackage{csquotes}

% This is not strictly necessary, and may be commented out,
% but it will improve the layout of the manuscript,
% and will typically save some space.
\usepackage{microtype}

% This is also not strictly necessary, and may be commented out.
% However, it will improve the aesthetics of text in
% the typewriter font.
\usepackage{inconsolata}

%Including images in your LaTeX document requires adding
%additional package(s)
\usepackage{graphicx}
\usepackage{multirow}
\usepackage{amsmath}
\usepackage{amssymb}
\usepackage{mathtools}
\usepackage{amsthm}
\usepackage{caption} 
\captionsetup{aboveskip=8pt, belowskip=8pt}
\usepackage{tcolorbox}
\usepackage{aliascnt}
\usepackage{xcolor}
\usepackage{tikz}
\usepackage{colortbl}
\usepackage{pgf} % Ensure this package is included in your preamble
\usepackage{pgfmath}
\usepackage{ragged2e}


\makeatletter
\renewcommand{\sectionautorefname}{\S\@gobble}
\makeatother % Changes 'subs
\makeatletter
\renewcommand{\subsectionautorefname}{\S\@gobble}
\makeatother % Changes 'subs
\makeatletter
\renewcommand{\subsubsectionautorefname}{\S\@gobble} % Changes 'subs
\makeatother
\renewcommand{\equationautorefname}{Eqn} % Changes 'subs
\renewcommand{\figureautorefname}{Fig.} % Changes 'Figure' to 'Fig.'
\renewcommand{\tableautorefname}{Table} % Keeps 'Table' as is or change as needed

\def\compactify{\itemsep=0pt \topsep=0pt \partopsep=0pt \parsep=0pt}
\let\latexusecounter=\usecounter
\newenvironment{CompactEnumerate}
  {\def\usecounter{\compactify\latexusecounter}
   \begin{enumerate}}
  {\end{enumerate}\let\usecounter=\latexusecounter}



\tcbset{
    colframe=gray!40, 
    colback=gray!5,   
    coltitle=black,   
    fonttitle=\bfseries,
    sharp corners,
    boxrule=0.5mm,  
    width=\columnwidth,
    left=0.1mm,        
    right=0.1mm,      
    toptitle=0.1mm,
    bottomtitle=0.1mm,
    title=Question Generation Prompt
}
% If the title and author information does not fit in the area allocated, uncomment the following
%
%\setlength\titlebox{<dim>}
%
% and set <dim> to something 5cm or larger.

\title{\name: Scalable Concept-Driven Question Generation\\to Enhance Human Learning}

% Author information can be set in various styles:
% For several authors from the same institution:
% \author{Author 1 \and ... \and Author n \\
%         Address line \\ ... \\ Address line}
% if the names do not fit well on one line use
%         Author 1 \\ {\bf Author 2} \\ ... \\ {\bf Author n} \\
% For authors from different institutions:
% \author{Author 1 \\ Address line \\  ... \\ Address line
%         \And  ... \And
%         Author n \\ Address line \\ ... \\ Address line}
% To start a separate ``row'' of authors use \AND, as in
% \author{Author 1 \\ Address line \\  ... \\ Address line
%         \AND
%         Author 2 \\ Address line \\ ... \\ Address line \And
%         Author 3 \\ Address line \\ ... \\ Address line}

\author{
 \textbf{Kimia Noorbakhsh\textsuperscript{*}}, 
 \textbf{Joseph Chandler\textsuperscript{*}}, 
 \textbf{Pantea Karimi\textsuperscript{*}}, 
 \\
 \textbf{Mohammad Alizadeh}, 
 \textbf{Hari Balakrishnan}
\\
\\
 M.I.T. Computer Science and Artificial Intelligence Lab (CSAIL)
}


% \renewcommand{\thefootnote}{\textsuperscript{*}}
% \footnotetext{Equal contribution.}

%##########################Custom commands ###############
\definecolor{customblue}{HTML}{DAE8FC}
\definecolor{customred}{HTML}{F8CECC}
\definecolor{customgreen}{HTML}{D5E8D4}
\definecolor{custompurple}{HTML}{E1D5E7}
\definecolor{customorange}{HTML}{FFE6CC}


\newcommand*\circled[2]{\tikz[baseline=(char.base)]{
            \node[shape=circle, line width=0.75pt, draw=black, fill=#2, inner sep=1pt] 
            (char) {\textcolor{black}{\small\textsf{#1}}};}}


\newcommand{\bluecircle}{\raisebox{0pt}{\protect\tikz[baseline=(char.base)]{\protect\node[shape=circle,draw, fill=customblue, minimum size=1.8mm, inner sep=1pt] (char) {\footnotesize 1};}}}
\newcommand{\redcircle}{\raisebox{0pt}{\protect\tikz[baseline=(char.base)]{\protect\node[shape=circle,draw, fill=customred, minimum size=1.8mm, inner sep=1pt] (char) {\footnotesize 2};}}}
\newcommand{\greencircle}{\raisebox{0pt}{\protect\tikz[baseline=(char.base)]{\protect\node[shape=circle,draw, fill=customgreen, minimum size=1.8mm, inner sep=1pt] (char) {\footnotesize 3};}}}
\newcommand{\purplecircle}{\raisebox{0pt}{\protect\tikz[baseline=(char.base)]{\protect\node[shape=circle,draw, fill=custompurple, minimum size=1.8mm, inner sep=1pt] (char) {\footnotesize 4};}}}
\newcommand{\orangecircle}{\raisebox{0pt}{\protect\tikz[baseline=(char.base)]{\protect\node[shape=circle,draw, fill=customorange, minimum size=1.8mm, inner sep=1pt] (char) {\footnotesize 5};}}}



\newcommand*\sfboxed[1]{\tikz[baseline=(char.base)]{
            \node[shape=rectangle,line width=0.75pt, draw=black,inner sep=2pt, rounded corners=2pt] (char) {\textcolor{black}{\small\textsf{#1}}};}}
\newcommand{\NewPara}[1]{\noindent{\bf #1}}

\newcommand{\TheSystem}{Savaal\xspace}
\newcommand{\Baseline}{Direct\xspace}
\newcommand{\BaselineMT}{Direct\xspace}
\newcommand{\name}{Savaal\xspace}
\newcommand{\arxiv}{arXiv\xspace}
\newcommand{\gpt}{GPT-4o}

\begin{document}
\maketitle
\def\thefootnote{*}\footnotetext{These authors contributed equally to this work.}\def\thefootnote{\arabic{footnote}}
\input{010values}
\begin{abstract}
\input{000abstract}
\end{abstract}

\input{010newintro}
\input{020insights}
\input{030pipeline}
\input{040evaluation}
\input{050related_work}
\input{060conclusions}
\input{limitations}

\section*{Acknowledgments}
We thank all the expert evaluators for their time, insights, and feedback. This work was funded in part by Quanta Computer, Inc. under the AIR Project.

\newpage
% \bibliography{refs}
\input{savaal_acl.bbl}
\input{070appendix}

\end{document}

\section{Formal Definition of Counterfactual Stability}\label{app:stability}

Counterfactual stability is a desirable property of SCMs~\citep{oberst2019counterfactual} that has previously been used in the context of autoregressive generation of LLMs~\cite{chatzi2024counterfactual}. In the following, we provide its formal definition along with a simple example to explain the intuition behind it. Throughout this section, $\PP^{\Ccal \,;\, do(\cdot)}$ denotes the probability of the interventional distribution entailed by an SCM $\Ccal$ under an intervention $do(\cdot)$.
% and $do(\cdot)$ denotes interventions to random variables. 
Moreover, $\PP^{\Ccal \,|\, \star \,;\, do(\cdot)}$ denotes the probability of the counterfactual distribution entailed by an SCM $\Ccal$ under an intervention $do(\cdot)$ given that an observed event $\star$ has already occurred.

\begin{definition}
    A sampling mechanism defined by $f_T$ and $P_U$ satisfies counterfactual stability if for all LLMs $m,m'\in \Mcal$, $i\in\{1,2,\ldots,K\}$ and tokens $t_1,t_2 \in V$ with $t_1\neq t_2$, the condition 
    \begin{equation}\label{eq:stability_condition}
        \frac{\PP^{\Ccal \,;\, do(M=m')}[T_i=t_1 \given D_i]}{\PP^{\Ccal \,;\, do(M=m)}[T_i=t_1 \given D_i]} \geq \frac{\PP^{\Ccal \,;\, do(M=m')}[T_i=t_2 \given D_i]}{\PP^{\Ccal \,;\, do(M=m)}[T_i=t_2 \given D_i]}
    \end{equation}
    implies that $\PP^{\Ccal \,|\, D_i,M=m,T_i=t_1 \,;\, do(M=m')}[T_i=t_2]=0$.
\end{definition}

The property of counterfactual stability has an intuitive interpretation that can be best understood via a simple example. Assume that the vocabulary contains $2$ tokens ``A'' and ``B'' and, using LLM $m$, the next-token distribution at a time step $i$ assigns values $0.6$, $0.4$ to the two tokens, respectively. Moreover, the realized noise value $\ub_i$ is such that the token ``A'' is sampled.
% , that is, $\text{``A''}=f_T(d_i, u_i)$.
Now, consider that, while keeping the noise value $\ub_i$ fixed, we change the LLM to $m'$, resulting in a next-token distribution that assigns values $0.7$, $0.3$ to the two tokens, respectively. Counterfactual stability ensures that, since the noise value $\ub_i$ led to ``A'' being sampled under $m$ at $0.6$ to $0.4$ odds, the same value
cannot lead to ``B'' being sampled under $m'$ where its relative odds are lower (\ie, $0.3$ to $0.7$).
% also leads to ``A'' being sampled under $m'$ where its relative odds are higher (\ie, $0.7$ to $0.3$).

\section{Proofs} \label{sec:proofs}

\subsection{Proof of Proposition~\ref{prop:variance}}
 We can rewrite the variance of the difference in scores under independent generation in terms of the variance of the difference in scores under coupled generation as follows:
 %
    \begin{align*}
        \Var[R_m(\Ub,S_q)-R_{m'}(\Ub',S_q)] ] 
        &= \Var[R_m(\Ub,S_q)-R_{m'}(\Ub,S_q) + R_{m'}(\Ub,S_q) -R_{m'}(\Ub',S_q)] ] \nonumber
        \\ &= \Var[R_m(\Ub,S_q)-R_{m'}(\Ub,S_q)] + \Var[R_{m'}(\Ub,S_q)-R_{m'}(\Ub',S_q)] \nn
            \\ &+ 2 \cdot \Cov[R_m(\Ub,S_q)-R_{m'}(\Ub,S_q), R_{m'}(\Ub,S_q)-R_{m'}(\Ub',S_q) ]. 
    \end{align*}
%
For the variance of the difference in scores for the same LLM under independent noise values, we have that
%
    \begin{align*}
        \Var[R_{m'}(\Ub,S_q)-R_{m'}(\Ub',S_q)]
        & \overset{(a)}{=} \EE[(R_{m'}(\Ub,S_q)-R_{m'}(\Ub',S_q))^2] -
         \EE[R_{m'}(\Ub,S_q)-R_{m'}(\Ub',S_q)]^2 % \label{eq:defvar1}
        \\& \overset{(b)}{=} \EE[R_{m'}(\Ub,S_q)^2-2\cdot R_{m'}(\Ub,S_q)R_{m'}(\Ub',S_q) + R_{m'}(\Ub',S_q)^2] % \label{eq:zeroterm}
        \\& \overset{(c)}{=} 2 \cdot \EE[R_{m'}(\Ub,S_q)^2] - 2 \cdot \EE[R_{m'}(\Ub,S_q)R_{m'}(\Ub',S_q)], % \label{eq:rewrite}
    \end{align*}
%
where (a) holds by the definition of variance, (b) is due to the subtraction term being $0$, and (c) is due to the linearity of expectation.
%
Further, for the covariance of the difference in scores under independent generation and the difference in scores under coupled generation, we have that
%
\begingroup
\allowdisplaybreaks
\begin{align*}
    \Cov[R_m(\Ub,S_q)-R_{m'}&(\Ub,S_q), R_{m'}(\Ub,S_q)-R_{m'}(\Ub',S_q)]\nonumber
    \\ % \begin{split}\label{eq:covdef}
    &\overset{(a)}{=} \EE[\left(R_m(\Ub,S_q)-R_{m'}(\Ub,S_q)\right) \cdot \left( R_{m'}(\Ub,S_q)-R_{m'}(\Ub',S_q) \right)] 
    \\ &\quad \quad - \EE[R_m(\Ub,S_q)-R_{m'}(\Ub,S_q)]\cdot \EE[R_{m'}(\Ub,S_q)-R_{m'}(\Ub',S_q)]
    % \end{split}
    \\ % \begin{split}\label{eq:zeroexp}
    &\overset{(b)}{=} \EE[R_m(\Ub,S_q)R_{m'}(\Ub,S_q)] - \EE[R_m(\Ub,S_q)R_{m'}(\Ub',S_q)]
    \\ &\quad \quad - \EE[R_{m'}(\Ub,S_q)R_{m'}(\Ub,S_q)] + \EE[R_{m'}(\Ub,S_q)R_{m'}(\Ub',S_q)] 
    \\ % \end{split}
    & \overset{(c)}{=} \Cov[R_m(\Ub,S_q),R_{m'}(\Ub,S_q)] - \EE[R_{m'}(\Ub,S_q)^2] + \EE[R_{m'}(\Ub,S_q)R_{m'}(\Ub',S_q)]] % \label{eq:covfinal}
\end{align*}
\endgroup
where (a) and (c) hold by the definition of covariance and (b) is due to the last term being zero and by the expansion of the first term.

Putting all the above results together, it follows that
%
\begin{align*}
    \Var[R_m(\Ub,S_q)-R_{m'}(\Ub',S_q)] ]
            % \\ \begin{split}
        & = \Var[R_m(\Ub,S_q)-R_{m'}(\Ub,S_q)] 
        + 2\cdot \Cov\left[R_m(\Ub,S_q),R_{m'}(\Ub,S_q)\right] \\
         & \quad+ 2\cdot \EE[R_{m'}(\Ub,S_q)R_{m'}(\Ub',S_q)] 
         - 2\cdot \EE[R_{m'}(\Ub,S_q)^2] 
         + 2 \cdot \EE[R_{m'}(\Ub,S_q)^2] \\
         & \quad - 2 \cdot \EE[R_{m'}(\Ub,S_q)R_{m'}(\Ub',S_q)]
       %  \end{split}
        \\&= \Var[R_m(\Ub,S_q)-R_{m'}(\Ub,S_q)] 
        + 2\cdot \Cov[R_m(\Ub,S_q),R_{m'}(\Ub,S_q)] 
\end{align*}
which concludes the proof.


\subsection{Proof of Proposition~\ref{prop:var_stability}}

Due to Proposition~\ref{prop:variance}, to show that Eq.~\ref{eq:binary_var} holds, it suffices to show that the covariance between the scores of the different LLMs under coupled generation is non-negative, \ie, $\Cov[R_m(\Ub,S_q),R_{m'}(\Ub,S_q)]\geq 0$.

To this end, we first rewrite the covariance as
    \begin{multline}
            \Cov[R_m(\Ub,S_q),R_{m'}(\Ub,S_q)] = \PP[R_m(\Ub,S_q)=1, R_{m'}(\Ub,S_q)=1] -\PP[R_m(\Ub,S_q)=1]\cdot \PP[R_{m'}(\Ub,S_q)=1] \\ \label{eq:covariance_rewrite}
            =\sum_{s_q} \PP[S_q=s_q] \cdot \left( \PP[R_m(\Ub,s_q)=1, R_{m'}(\Ub,s_q)=1]-\PP[R_m(\Ub,s_q)=1]\cdot \PP[R_{m'}(\Ub,s_q)=1] \right) 
        % \end{split}
    \end{multline}
%
Next, we note that the event $R_m(\Ub,s_q)=1$ is equivalent to LLM $m$ sampling the ground truth token for prompt $s_q$. 
%
Without loss of generality, assume $t_1$ is the ground truth token, \ie, $\textsc{c}(s_q)=t_1$. 
%
Then, since only tokens $\{t_1, t_2\}$ have positive probability under $m$ and $m'$, it must hold that either (i) one LLM assigns a greater probability to $t_1$ and the other LLM assigns a greater probability to $t_2$, 
%
or (ii) both LLMs assign the same probabilities. 
%
Further, since the sampling mechanism defined by $f_T$ and $P_U$ satisfies counterfactual stability, we have that the condition in Eq.~\ref{eq:stability_condition} holds in both (i) and (ii) and, under coupled generation, the LLM with greater (or equal) probability for $t_1$ will always sample $t_1$ when the LLM with lower (or equal) probability does. 
%
This implies that
%
\begin{equation}
    \PP[R_m(\Ub,s_q)=1, R_{m'}(\Ub,s_q)=1] = \min\{ \PP[R_m(\Ub,s_q)=1], \PP[R_{m'}(\Ub,s_q)=1]\}
\end{equation}
%
Finally, since it holds that
%
\begin{equation}
    \min\{ \PP[R_m(\Ub,s_q)=1], \PP[R_{m'}(\Ub,s_q)=1]\} \geq \PP[R_m(\Ub,s_q)=1] \PP[R_{m'}(\Ub,s_q)=1]
\end{equation}
because $\PP[R_m(\Ub,s_q)=1] \in (0,1)$ and $\PP[R_{m'}(\Ub,s_q)=1]\in(0,1)$ by assumption,
% (with strict inequality holding if in $(0,1)$),
we can conclude from Eq.~\ref{eq:covariance_rewrite} that
    \begin{equation}
        \begin{split}
            &\Cov[R_m(\Ub,S_q),R_{m'}(\Ub,S_q)] > 0.
        \end{split}
    \end{equation}
%
% Additionally, we can also conclude that, if $\PP[R_m(\Ub,s_q)=1] \in (0,1)$ 
% and $\PP[R_{m'}(\Ub,s_q)=1]\in (0,1)$, the above inequality is strict.
    
\subsection{Proof of Proposition~\ref{prop:var_gumbel}}
%
Using Proposition~\ref{prop:variance}, we have that
%
    \begin{align*}
            \Cov[R_m(\Ub,S_q),R_{m'}(\Ub,S_q)] &= \EE[R_m(\Ub,S_q)\cdot R_{m'}(\Ub,S_q)]-\EE[R_m(\Ub,S_q)]\cdot  \EE[R_{m'}(\Ub,S_q)]\\
            &= \underbrace{\PP[R_m(\Ub,S_q)=1, R_{m'}(\Ub,S_q)=1
            ]}_{(i)}-\underbrace{\PP[R_m(\Ub,S_q)=1) \cdot \PP[R_{m'}(\Ub,S_q)=1]}_{(ii)}.
    \end{align*}
%
In the remainder of the proof, we will bound each term (i) and (ii) separately and, since $|\textsc{c}(s_q)|=1$ for all $s_q \sim P_{Q}$, assume without loss of generality that the correct token is single-token sequence $t_1$.

To bound the term (ii), first note that, using the definition of the Gumbel-Max SCM, we have that, for each $k \in \{2, \ldots, |V|\}$, it holds that
%
\begin{align*}
            R_{m}(\Ub,s_q)=1 &\iff U_1 + \log ( [f_D(s_q,m)]_{t_1} ) \geq U_k + \log ( [f_D(s_q, m)]_{t_k} ), \\
            R_{m'}(\Ub,s_q)=1 &\iff U_1 + \log ( [f_D(s_q,m')]_{t_1} ) \geq U_k + \log ( [f_D(s_q, m')]_{t_k} ).
\end{align*}
%
Next, let $\varepsilon^*>0$ be an arbitrary constant that we will determine later such that
%
\begin{equation} \label{eq:bound}
    |\log ( [f_D(S_q,m)]_{t_k}) -\log ([f_D(S_q,m')]_{t_k}) |\leq \varepsilon^*,
\end{equation}
%
and note that since, by assumption, $D_{t_k} > 0$ for all $k \in \{1, \ldots, |V|\}$, any bound on the absolute difference of log-probabilities $|\log ( [f_D(S_q,m)]_{t_k}) -\log ([f_D(S_q,m')]_{t_k}) |$ uniformly implies a bound on the difference of probabilities $|[f_D(S_q,m)]_{t_k} -[f_D(S_q,m')]_{t_k} |$ and vice versa. 
%
For simplicity, we prove the result in the log-domain.

Now, using the bound defined by Eq.~\ref{eq:bound}, we have that
%
\begin{multline*}
        \bigcap_{k\neq1} \left\{ U_1 + \log ( [f_D(S_q, m')]_{t_1} ) \geq U_k + \log ( [f_D(S_q, m')]_{t_k} ) \right\}\\
            \subset \bigcap_{k\neq1} \left\{U_1 + \log ( [f_D(S_q, m)]_{t_1} )+\varepsilon^* \geq U_k + \log ( [f_D(S_q, m)]_{t_k} )-\varepsilon^* \right\},
\end{multline*}
%
and we can then bound the term (ii) as follows:
    \begin{align*}
            \PP[R_m(\Ub,S_q)=1] &\cdot \PP[R_{m'}(\Ub,S_q)=1] \\
            &=
            \PP[\cap_{k\neq1} \{U_1 + \log ( [f_D(S_q,m)]_{t_1} ) \geq U_k + \log ( [f_D(S_q,m)]_{t_k} )\}]\\
            & \qquad \times \PP[\cap_{k\neq1}\{ U_1 + \log ( [f_D(S_q,m')]_{t_1} ) \geq U_k + \log ( [f_D(S_q,m')]_{t_k} ) \}]\\
            &\leq \PP[\cap_{k\neq1} \{U_1 + \log ( [f_D(S_q,m)]_{t_1} ) \geq U_k + \log ( [f_D(S_q,m)]_{t_k} )\}]\\
            & \qquad \times \PP[\cap_{k\neq1} \{U_1 + \log ( [f_D(S_q,m)]_{t_1} )+\varepsilon^* \geq U_k + \log ( [f_D(S_q,m)]_{t_k} )-\varepsilon^*\}].
    \end{align*}
%
To bound the term (i), first note that, using the bound defined by Eq.~\ref{eq:bound}, we have that
    \begin{multline*}
        \bigcap_{k\neq1} \left\{U_1 + \log ( [f_D(S_q,m')]_{t_1} ) \geq U_k + \log ( [f_D(S_q,m')]_{t_k} ) \right\}\\
         \supset \bigcap_{k\neq1} \left\{U_1 + \log ( [f_D(S_q,m)]_{t_1} )-\varepsilon^* \geq U_k + \log ( [f_D(S_q,m)]_{t_k} )+\varepsilon^* \right\}.
    \end{multline*} 
%        
Thus, we can bound the term (i) as follows:
%
    \begin{align*}
        \PP[R_m(\Ub,S_q)=1, R_{m'}(\Ub,S_q)=1] &=\PP\Big[\cap_{k\neq1} \{U_1 + \log ( [f_D(S_q,m)]_{t_1} ) \geq U_k + \log ([f_D(S_q,m)]_{t_k} )\}\\
            & \qquad \cap \{ U_1 + \log ( [f_D(S_q,m')]_{t_1} ) \geq U_k + \log ( [f_D(S_q,m')]_{t_k} ) \}\Big] \\
            & \geq \PP\Big[\cap_{k\neq1} \{U_1 + \log ( [f_D(S_q,m)]_{t_1} ) \geq U_k + \log ( [f_D(S_q,m)]_{t_k} )\}\\
            & \qquad \cap \{ U_1 + \log ( [f_D(S_q,m)]_{t_1} ) \geq U_k + \log ( [f_D(S_q,m)]_{t_k} ) +2\varepsilon^* \}\Big] \\
            & \overset{(a)}{=} \sum_{s_q} \PP[S_q=s_q]\cdot \PP[\cap_{k\neq1} \{ U_1 + \log ( [f_D(S_q,m)]_{t_1} ) \\
            & \geq U_k + \log ( [f_D(S_q,m)]_{t_k} ) +2\varepsilon^* \}],
    \end{align*}
where (a) follows from the fact that
%
\begin{multline*}
      \left\{ U_1 + \log ( [f_D(S_q,m)]_{t_1} ) \geq U_k + \log ( [f_D(S_q,m)]_{t_k} ) +2\varepsilon^* \right\} \\
      \subset \left\{ U_1 + \log ( [f_D(S_q,m)]_{t_1} ) \geq U_k + \log ( [f_D(S_q,m)]_{t_k} ) \right\}.
\end{multline*}
%
Now, note that, for $k \in \{2, \ldots, |V|\}$, the variable $X_k \equiv U_1-U_k \sim \text{Logistic}(0,1)$ (for $k=1$, define $X_k \equiv 0$). 
%
Therefore, we can rewrite the bound for (i) as
%  
\begin{multline*}
      \PP[R_m(\Ub,S_q)=1, R_{m'}(\Ub,S_q)=1]\\
      \geq\sum_{s_q} \PP[S_q=s_q] \cdot \prod_{k\neq1}\cdot \PP[ \{ X_k \geq \log ( [f_D(S_q,m)]_{t_k} )-\log ( [f_D(S_q,m)]_{t_1} ) +2\varepsilon^* \}]
\end{multline*}
%
and we can rewrite the bound for (ii) as
%
\begin{multline*}
    \PP[R_m(\Ub,S_q)=1] \PP[R_{m'}(\Ub,S_q)=1]\leq \\
    \sum_{s_q}\PP[S_q=s_q]\cdot 
            \left\{
            \prod_{k\neq1}
            \PP[ \{ X_k \geq \log ( [f_D(S_q,m)]_{t_k} )-\log ( [f_D(S_q,m)]_{t_k} ) - 2\varepsilon^* \} ]
            \right\}\\
            \times \PP[\cap_{k\neq1} \{X_k \geq \log ( [f_D(S_q,m)]_{t_k} )-\log ( [f_D(S_q,m)]_{t_k} )\}].
\end{multline*}
%
As a consequence, to prove that $\PP[R_m(\Ub,S_q)=1, R_{m'}(\Ub,S_q)=1] > \PP[R_m(\Ub,S_q)=1] \PP[R_{m'}(\Ub,S_q)=1]$, it suffices to show that
    \begin{multline}\label{eq:condition-covariance-binary-scores}
            \sum_{s_q}P[S_q=s_q] \prod_{k\neq1}\cdot \PP[ \{ X_k \geq \log ( [f_D(S_q,m)]_{t_k} )-\log ( [f_D(S_q,m)]_{t_1} ) +2\varepsilon^* \}]\\
            > \sum_{s_q}\PP[S_q=s_q] \prod_{k\neq1}\cdot \PP[ \{ X_k \geq \log ( [f_D(S_q,m)]_{t_k} )-\log ( [f_D(S_q,m)]_{t_1} ) - 2\varepsilon^* \}]
            \\
            \times \PP[\cap_{k\neq1} \{X_k \geq \log ( [f_D(S_q,m)]_{t_k} )-\log ( [f_D(S_q,m)]_{t_1} )\}]
    \end{multline}
%
To do so, note that Eq.~\ref{eq:condition-covariance-binary-scores} holds trivially for $\varepsilon^*=0$ since
%
\begin{equation*}
      \PP[\cap_{k\neq1} \{X_k \geq \log ( [f_D(S_q,m)]_{t_k} )-\log ( [f_D(S_q,m)]_{t_1} )\}] < 1,
\end{equation*}
%
which is a fixed term independent of $m'$. Since all terms in Eq.~\ref{eq:condition-covariance-binary-scores} are continuous in $\varepsilon^*$, there exists $\varepsilon^*(m) >0 $, possibly dependent of $m$ but independent of $m'$, such that Eq.~\ref{eq:condition-covariance-binary-scores} holds if
\begin{equation*}
    \sup_{s_q} \norm{\log(f_D(s_q,m))-\log(f_D(s_q, m'))}_\infty < \varepsilon^*(m).
\end{equation*}
Since by assumption $D_t>0$ for all $t\in V$, there exists $\varepsilon(m)>0$ in probability space such that Eq.~\ref{eq:condition-covariance-binary-scores} holds if
\begin{equation*}
    \sup_{s_q} \norm{f_D(s_q,m)-f_D(s_q, m')}_\infty < \varepsilon(m).
\end{equation*}
%
This concludes the proof.

\subsection{Proof of Proposition~\ref{prop:gap_win_rates_stability}}

% We first compute the win-rates under coupled autoregressive generation.
%
Under coupled autoregressive generation, if the LLM $m$ samples the preferred token $t_+$, then the LLM $m'$ must also sample $t_{+}$ because $t_{+}$ is more likely under $m'$ than under $m$ and the sampling mechanism defined by $f_T$ and $P_U$ satisfies counterfactual stability. 
%
This implies that the win-rate achieved by $m$ against $m'$ is
%
    \begin{equation}\label{eq:prop_greater_shared}
            \EE_{\Ub\sim P_{\Ub}} [\one\{R_m(\Ub,s_q)>R_{m'}(\Ub,s_q)\}] =
            \PP[f_T(f_D(s_q, m),\Ub)= t_+, f_T(f_D(s_q, m'),\Ub)= t_-]=0 
    \end{equation}
and that
    \begin{equation}\label{eq:prop_equal_probs+}
            \PP[f_T(f_D(s_q, m),\Ub)= t_+, f_T(f_D(s_q, m'),\Ub)= t_+]= \PP[f_T(f_D(s_q, m),\Ub)= t_+] =p_m.
    \end{equation}
%
Using the same reasoning, if the LLM $m'$ samples the non-preferred token $t_{-}$, then, $m$ must also sample $t_{-}$ because $t_{-}$ is more likely under $m$ than under $m'$. This implies that
%
    \begin{equation}\label{eq:prop_equal_probs-}
            \PP[f_T(f_D(s_q, m),\Ub)= t_-, f_T(f_D(s_q, m'),\Ub)= t_-]= \PP[f_T(f_D(s_q, m'),\Ub)= t_-] = 1-p_{m'}
    \end{equation} 
%
Then, from Eq.~\ref{eq:prop_equal_probs+} and Eq.~\ref{eq:prop_equal_probs-}, we can conclude that
%
    \begin{equation}\label{eq:prop_equal_shared}
        \begin{aligned}
            \EE_{\Ub\sim P_{\Ub}} [\one\{R_m(\Ub,s_q)=R_{m'}(\Ub,s_q)\}] = p_m + (1-p_{m'})
        \end{aligned}
    \end{equation}
%
Finally, from Eq.~\ref{eq:prop_greater_shared} and Eq.~\ref{eq:prop_equal_shared}, we can conclude that the win-rate achieved by $m'$ against $m$ is
%
\begin{multline*}
    \EE_{\Ub\sim P_{\Ub}} [\one\{R_m(\Ub,s_q)<R_{m'}(\Ub,s_q)\}] \\
        = 1 - \EE_{\Ub\sim P_{\Ub}} [\one\{R_m(\Ub,s_q)>R_{m'}(\Ub,s_q)\}] -\EE_{\Ub\sim P_{\Ub}} [\one\{R_m(\Ub,s_q)=R_{m'}(\Ub,s_q)\}] = p_{m'}-p_m.
\end{multline*}

% Next, we compute the win-rates under independent autoregressive generation.
%
Under independent autoregressive generation, the LLMs $m$ and $m'$ sample tokens independently from each other, \ie, $f_T(f_D(s_q, m),\Ub) \perp f_T(f_D(s_q, m'),\Ub')$. 
%
Thus, we can factorize all joint probabilities when computing the win-rates and obtain
%
\begin{align*}\nonumber
    % \begin{split}
        \EE_{\Ub, \Ub'\sim P_{\Ub}} [\one\{R_m(\Ub,s_q)>R_{m'}(\Ub',s_q)\}] &=\PP[f_T(f_D(s_q, m),\Ub)= t_+] \cdot \PP[f_T(f_D(s_q, m'),\Ub')= t_-] \\ &=
        p_m \cdot (1-p_{m'})
    % \end{split}
\end{align*}
and
\begin{equation}\nonumber
    \begin{split}
        \EE_{\Ub, \Ub'\sim P_{\Ub}} & [\one\{R_m(\Ub,s_q)<R_{m'}(\Ub',s_q)\}] 
         = p_{m'} \cdot (1-p_m).
        % we do not write it the = rates in the proposition 
        %\\
        % \text{and} \quad \EE_{\Ub\sim P_{\Ub}, \Ub'\sim P_{\Ub}, S_q\sim P_{Q}} & [\one\{R_m(\Ub,S_q)=R_{m'}(\Ub',S_q)\}] 
        %  = p_m(p_m+\delta) + (1-p_m-\delta)(1-p_m)
    \end{split}
\end{equation}


\subsection{Proof of Proposition~\ref{prop:gumbel_ties}}

We follow the notations and technique of Proposition~\ref{prop:var_gumbel}.
Fix query $s_q$ and consider first the case of independent autoregressive generation. Since each LLM can only assign a non-zero probability to single-token sequences, we have:
\begin{equation*}
    \begin{split}
        \PP[R_m(\Ub, s_q)=R_{m'}(\Ub', s_q)]&=\sum_{k=1}^{|V|}\PP[f_T(f_D(s_q,m),\Ub)=t_k] \PP[f_T(f_D(s_q,m'),\Ub)=t_k]\\
        &<\sum_{k=1}^{|V|}\PP[f_T(f_D(s_q,m),\Ub)=t_k],
    \end{split}
\end{equation*}
%
In the case of coupled autoregressive generation, since
\begin{equation*}
    \PP\left[\{f_T(f_D(s_q,m),\Ub)=t_k\} \cap \{f_T(f_D(s_q,m),\Ub)=t_j \}\right]=0,\; i\neq j,
\end{equation*}
%
we obtain:
\begin{equation*}
    \begin{split}
        &\PP[R_m(\Ub, s_q)=R_{m'}(\Ub, s_q)]\\
        &=\PP\left[\cup_i \{ f_T(f_D(s_q,m),\Ub)=t_k, f_T(f_D(s_q,m'),\Ub)=t_k \} \right]\\
        &=\sum_k\PP[\{f_T(f_D(s_q,m),\Ub)=t_k, f_T(f_D(s_q,m'),\Ub)=t_k \}]\\
        &=\sum_k \PP[f_T(f_D(s_q,m),\Ub)=t_k]\PP[f_T(f_D(s_q,m'),\Ub)=t_k|f_T(f_D(s_q,m),\Ub)=t_k].\\
    \end{split}
\end{equation*}
We now follow \cite{9729603} and expand the posterior Gumbels, $\PP[f_T(f_D(s_q,m'),\Ub)=t_k|f_T(f_D(s_q,m),\Ub)=t_k]$, as truncated Gumbel distributions. In particular, we leverage the fact that
\begin{equation}\label{eq:max of gumbels}
    \max_{t\in V}\{U_t+\log([f_D(s_q,\bullet)]_{t})\} \sim \text{Gumbel}(0,1),
\end{equation}
and that a Gumbel distribution, with parameter $\log (\theta)$, truncated at $b\sim \text{Gumbel}(0,1)$  can be sampled as
\begin{equation}\label{eq: truncated gumbel}
    -\log(\exp(-b)-\log(\eta)/\theta),\; \eta \sim U(0,1).
\end{equation}
Furthermore, by assumption, $D_{t_k} > 0$ for all $k \in \{1, \ldots, |V|\}$, so that any bound on the absolute difference of log-probabilities $|\log ( [f_D(s_q,m)]_{t_k}) -\log ([f_D(s_q,m')]_{t_k}) |$ uniformly implies a bound on the difference of probabilities $|[f_D(s_q,m)]_{t_k} -[f_D(s_q,m')]_{t_k} |$ and vice versa. Using the bound 
\begin{equation*}
    |\log([f_D(s_q,m)]_{t_k})-\log([f_D(s_q,m')]_{t_k})|\leq \varepsilon^*
\end{equation*}
and the Gumbel properties in Eq.~\ref{eq:max of gumbels} and Eq.~\ref{eq: truncated gumbel}, we obtain:
%
\begin{align}
        &\PP[R_m(\Ub, s_q)=R_{m'}(\Ub, s_q)] \nn \\
        &=\sum_k \PP[f_T(f_D(s_q,m),\Ub)=t_k] \nn \\
            & \qquad \quad \times \PP\Bigg[
            \bigcap_k \Big\{
            \log( [f_D(s_1, m')]_{t_k}) - \log([f_D(s_1, m)]_{t_k})-\log(-\log (\eta_k)) \nn \\
            & \qquad \qquad \qquad \qquad \geq \log([f_D(s_1, m')]_{t_j}) - \log([f_D(s_1, m)]_{t_j})-\log(-\log(\eta_k) -\log(\eta_j)/[f_D(s_1,m')]_{t_j})
            \Big\}
            \Bigg] \nn \\
            &\geq
            \sum_k \PP[f_T(f_D(s_q,m),\Ub)=t_k] \nn \\
            &\qquad \quad \times\PP\left[
            \cap_k \{
            -\log(-\log (\eta_k)) \geq -2\varepsilon^* -\log(-\log(\eta_k) -\log(\eta_j)/[f_D(s_1,m')]_{t_j})
            \}
            \right] \label{eq:final-bound}
\end{align}
where $\eta_k\sim \text{U}(0,1)$ are independently distributed uniform random variables.
%
Now, note that the claim holds for $\varepsilon^*=0$ since, in that case, we have that
%
\begin{equation*}
   \PP\left[
        \bigcap_k \left\{
        -\log(-\log (\eta_k)) \geq -\log(-\log(\eta_k) -\log(\eta_k)/[f_D(s_1,m')]_{t_k})
        \right\} \right]=1,
\end{equation*}
%
using that $x\mapsto -\log(x)$ is strictly decreasing. 
%
Since all terms in Eq.~\ref{eq:final-bound} are continuous in $\varepsilon^*$, there exists $\varepsilon^*(m) >0 $, possibly dependent on $m$ but independent of $m'$, such that 
\begin{equation}\label{eq:prop5 claim prob}
    \PP[R_m(\Ub, s_q)=R_{m'}(\Ub, s_q)] > \PP[R_m(\Ub, s_q)=R_{m'}(\Ub', s_q)]
\end{equation}
holds if
\begin{equation*}
    \sup_{s_q} \norm{\log(f_D(s_q,m))-\log(f_D(s_q, m'))}_\infty < \varepsilon^*(m).
\end{equation*}
Since by assumption $D_t>0$ for all $t\in V$, there exists $\varepsilon(m)>0$ in probability space such that Eq.~\ref{eq:prop5 claim prob} holds if

\begin{equation*}
    \sup_{s_q} \norm{f_D(s_q,m)-f_D(s_q, m')}_\infty < \varepsilon(m).
\end{equation*}
%
This concludes the proof.






\subsection{Calculation of average win-rates in the example used in Sections~\ref{sec:intro} and~\ref{sec:pairwise}} \label{app:example-ranking}
%
In this section, we provide detailed calculations of the win-rates for the example in Sections~\ref{sec:intro} and~\ref{sec:pairwise}. Recall that in this example, we are given three LLMs $m_1$, $m_2$ and $m_3$, and we need to rank them according to their ability to answer correctly two types of input prompts, $q$ and $q'$, picked uniformly at random.
%
We assume that the true probability that each LLM answers correctly each type of input prompt is given by:
%
\begin{table}[H]
    \centering
    \begin{tabular}{lccc}
        \toprule
         & $m_1$ & $m_2$ & $m_3$ \\
        \midrule
        $q$           & $p_1=0.4$             & $p_2=0.48$           & $p_3=0.5$             \\
        $q'$           & $p'_1=1$           & $p'_2=0.9$         & $p'_3=0.89$              \\
        \bottomrule
 \end{tabular}
 \vspace{-8mm}
 \caption*{}
 \end{table}

Using Proposition~\ref{prop:gap_win_rates_stability}, 
% under the assumption of counterfactual stability, 
the win-rates under independent autoregressive generation are given, for each LLM $m_k$, by:
\begin{equation}
    \begin{split}
       \frac{1}{2}\sum_{j\neq k}\EE_{\Ub, \Ub' \sim P_{\Ub}, S_q\sim P_Q}[\one\{R_{m_k}(\Ub, S_q)>R_{m_j}(\Ub', S_q)\}] & =\frac{\sum_{j\neq k}p_k(1-p_j) + \sum_{j\neq k}p'_k(1-p'_j)}{4}.
    \end{split}
\end{equation}
%
Substituting the numerical values we obtain:

 \begin{equation}
    \begin{split}
       \frac{1}{2}\sum_{j\neq 1}\EE_{\Ub, \Ub' \sim P_{\Ub}, S_q\sim P_Q}[\one\{R_{m_1}(\Ub, S_q)>R_{m_j}(\Ub', S_q)\}] & =0.1545,\\
       \frac{1}{2}\sum_{j\neq 2}\EE_{\Ub, \Ub' \sim P_{\Ub}, S_q\sim P_Q}[\one\{R_{m_2}(\Ub, S_q)>R_{m_j}(\Ub', S_q)\}] & =0.15675,\\
       \frac{1}{2}\sum_{j\neq 3}\EE_{\Ub, \Ub' \sim P_{\Ub}, S_q\sim P_Q}[\one\{R_{m_3}(\Ub, S_q)>R_{m_j}(\Ub', S_q)\}] & =0.16225\\
    \end{split}
\end{equation}

Similarly, using Proposition~\ref{prop:gap_win_rates_stability}, the win-rates using coupled autoregressive generation can be written, for each LLM $m_k$, as:
\begin{equation}
    \begin{split}
       \frac{1}{2}\sum_{j\neq k}\EE_{\Ub \sim P_{\Ub}, S_q\sim P_Q}[\one\{R_{m_k}(\Ub, S_q)>R_{m_j}(\Ub, S_q)\}] & =\frac{\sum_{j\neq k}(p_k-p_j)_{+} + \sum_{j\neq k}(p'_k-p'_j)_{+}}{4},
    \end{split}
\end{equation}
where $(\bullet)_{+}=\max(0, \bullet)$ denotes the positive part. Substituting the numerical values we obtain:

 \begin{align*}
       \frac{1}{2}\sum_{j\neq 1}\EE_{\Ub \sim P_{\Ub}, S_q\sim P_Q}[\one\{R_{m_1}(\Ub, S_q)>R_{m_j}(\Ub, S_q)\}] & = 
       % \frac{(p_1-p_2)_{+} +(p_1-p_3)_{+} +(p'_1-p'_2)_{+} +(p'_1-p'_3)_{+}}{4}
       0.0525, \\
       \frac{1}{2}\sum_{j\neq 2}\EE_{\Ub \sim P_{\Ub}, S_q\sim P_Q}[\one\{R_{m_2}(\Ub, S_q)>R_{m_j}(\Ub, S_q)\}] &=
       % \frac{(p_2-p_1)_{+} +(p_2-p_3)_{+} +(p'_2-p'_1)_{+} +(p'_2-p'_3)_{+}}{4}=
       0.0225, \\
       \frac{1}{2}\sum_{j\neq 3}\EE_{\Ub \sim P_{\Ub}, S_q\sim P_Q}[\one\{R_{m_3}(\Ub, S_q)>R_{m_j}(\Ub, S_q)\}] &=
       % \frac{(p_3-p_2)_{+} +(p_3-p_1)_{+} +(p'_3-p'_2)_{+} +(p'_3-p'_1)_{+}}{4}=
       0.03.
\end{align*}






\newpage

\section{Additional Experimental Details} \label{app:add_exp}
\input{072experimentaldetails}

\clearpage
\newpage

% \section{Additional Experimental Results} \label{app:exp_results}
\input{073additional-experiments}


\end{document}

\section{Formal Definition of Counterfactual Stability}\label{app:stability}

Counterfactual stability is a desirable property of SCMs~\citep{oberst2019counterfactual} that has previously been used in the context of autoregressive generation of LLMs~\cite{chatzi2024counterfactual}. In the following, we provide its formal definition along with a simple example to explain the intuition behind it. Throughout this section, $\PP^{\Ccal \,;\, do(\cdot)}$ denotes the probability of the interventional distribution entailed by an SCM $\Ccal$ under an intervention $do(\cdot)$.
% and $do(\cdot)$ denotes interventions to random variables. 
Moreover, $\PP^{\Ccal \,|\, \star \,;\, do(\cdot)}$ denotes the probability of the counterfactual distribution entailed by an SCM $\Ccal$ under an intervention $do(\cdot)$ given that an observed event $\star$ has already occurred.

\begin{definition}
    A sampling mechanism defined by $f_T$ and $P_U$ satisfies counterfactual stability if for all LLMs $m,m'\in \Mcal$, $i\in\{1,2,\ldots,K\}$ and tokens $t_1,t_2 \in V$ with $t_1\neq t_2$, the condition 
    \begin{equation}\label{eq:stability_condition}
        \frac{\PP^{\Ccal \,;\, do(M=m')}[T_i=t_1 \given D_i]}{\PP^{\Ccal \,;\, do(M=m)}[T_i=t_1 \given D_i]} \geq \frac{\PP^{\Ccal \,;\, do(M=m')}[T_i=t_2 \given D_i]}{\PP^{\Ccal \,;\, do(M=m)}[T_i=t_2 \given D_i]}
    \end{equation}
    implies that $\PP^{\Ccal \,|\, D_i,M=m,T_i=t_1 \,;\, do(M=m')}[T_i=t_2]=0$.
\end{definition}

The property of counterfactual stability has an intuitive interpretation that can be best understood via a simple example. Assume that the vocabulary contains $2$ tokens ``A'' and ``B'' and, using LLM $m$, the next-token distribution at a time step $i$ assigns values $0.6$, $0.4$ to the two tokens, respectively. Moreover, the realized noise value $\ub_i$ is such that the token ``A'' is sampled.
% , that is, $\text{``A''}=f_T(d_i, u_i)$.
Now, consider that, while keeping the noise value $\ub_i$ fixed, we change the LLM to $m'$, resulting in a next-token distribution that assigns values $0.7$, $0.3$ to the two tokens, respectively. Counterfactual stability ensures that, since the noise value $\ub_i$ led to ``A'' being sampled under $m$ at $0.6$ to $0.4$ odds, the same value
cannot lead to ``B'' being sampled under $m'$ where its relative odds are lower (\ie, $0.3$ to $0.7$).
% also leads to ``A'' being sampled under $m'$ where its relative odds are higher (\ie, $0.7$ to $0.3$).

\section{Proofs} \label{sec:proofs}

\subsection{Proof of Proposition~\ref{prop:variance}}
 We can rewrite the variance of the difference in scores under independent generation in terms of the variance of the difference in scores under coupled generation as follows:
 %
    \begin{align*}
        \Var[R_m(\Ub,S_q)-R_{m'}(\Ub',S_q)] ] 
        &= \Var[R_m(\Ub,S_q)-R_{m'}(\Ub,S_q) + R_{m'}(\Ub,S_q) -R_{m'}(\Ub',S_q)] ] \nonumber
        \\ &= \Var[R_m(\Ub,S_q)-R_{m'}(\Ub,S_q)] + \Var[R_{m'}(\Ub,S_q)-R_{m'}(\Ub',S_q)] \nn
            \\ &+ 2 \cdot \Cov[R_m(\Ub,S_q)-R_{m'}(\Ub,S_q), R_{m'}(\Ub,S_q)-R_{m'}(\Ub',S_q) ]. 
    \end{align*}
%
For the variance of the difference in scores for the same LLM under independent noise values, we have that
%
    \begin{align*}
        \Var[R_{m'}(\Ub,S_q)-R_{m'}(\Ub',S_q)]
        & \overset{(a)}{=} \EE[(R_{m'}(\Ub,S_q)-R_{m'}(\Ub',S_q))^2] -
         \EE[R_{m'}(\Ub,S_q)-R_{m'}(\Ub',S_q)]^2 % \label{eq:defvar1}
        \\& \overset{(b)}{=} \EE[R_{m'}(\Ub,S_q)^2-2\cdot R_{m'}(\Ub,S_q)R_{m'}(\Ub',S_q) + R_{m'}(\Ub',S_q)^2] % \label{eq:zeroterm}
        \\& \overset{(c)}{=} 2 \cdot \EE[R_{m'}(\Ub,S_q)^2] - 2 \cdot \EE[R_{m'}(\Ub,S_q)R_{m'}(\Ub',S_q)], % \label{eq:rewrite}
    \end{align*}
%
where (a) holds by the definition of variance, (b) is due to the subtraction term being $0$, and (c) is due to the linearity of expectation.
%
Further, for the covariance of the difference in scores under independent generation and the difference in scores under coupled generation, we have that
%
\begingroup
\allowdisplaybreaks
\begin{align*}
    \Cov[R_m(\Ub,S_q)-R_{m'}&(\Ub,S_q), R_{m'}(\Ub,S_q)-R_{m'}(\Ub',S_q)]\nonumber
    \\ % \begin{split}\label{eq:covdef}
    &\overset{(a)}{=} \EE[\left(R_m(\Ub,S_q)-R_{m'}(\Ub,S_q)\right) \cdot \left( R_{m'}(\Ub,S_q)-R_{m'}(\Ub',S_q) \right)] 
    \\ &\quad \quad - \EE[R_m(\Ub,S_q)-R_{m'}(\Ub,S_q)]\cdot \EE[R_{m'}(\Ub,S_q)-R_{m'}(\Ub',S_q)]
    % \end{split}
    \\ % \begin{split}\label{eq:zeroexp}
    &\overset{(b)}{=} \EE[R_m(\Ub,S_q)R_{m'}(\Ub,S_q)] - \EE[R_m(\Ub,S_q)R_{m'}(\Ub',S_q)]
    \\ &\quad \quad - \EE[R_{m'}(\Ub,S_q)R_{m'}(\Ub,S_q)] + \EE[R_{m'}(\Ub,S_q)R_{m'}(\Ub',S_q)] 
    \\ % \end{split}
    & \overset{(c)}{=} \Cov[R_m(\Ub,S_q),R_{m'}(\Ub,S_q)] - \EE[R_{m'}(\Ub,S_q)^2] + \EE[R_{m'}(\Ub,S_q)R_{m'}(\Ub',S_q)]] % \label{eq:covfinal}
\end{align*}
\endgroup
where (a) and (c) hold by the definition of covariance and (b) is due to the last term being zero and by the expansion of the first term.

Putting all the above results together, it follows that
%
\begin{align*}
    \Var[R_m(\Ub,S_q)-R_{m'}(\Ub',S_q)] ]
            % \\ \begin{split}
        & = \Var[R_m(\Ub,S_q)-R_{m'}(\Ub,S_q)] 
        + 2\cdot \Cov\left[R_m(\Ub,S_q),R_{m'}(\Ub,S_q)\right] \\
         & \quad+ 2\cdot \EE[R_{m'}(\Ub,S_q)R_{m'}(\Ub',S_q)] 
         - 2\cdot \EE[R_{m'}(\Ub,S_q)^2] 
         + 2 \cdot \EE[R_{m'}(\Ub,S_q)^2] \\
         & \quad - 2 \cdot \EE[R_{m'}(\Ub,S_q)R_{m'}(\Ub',S_q)]
       %  \end{split}
        \\&= \Var[R_m(\Ub,S_q)-R_{m'}(\Ub,S_q)] 
        + 2\cdot \Cov[R_m(\Ub,S_q),R_{m'}(\Ub,S_q)] 
\end{align*}
which concludes the proof.


\subsection{Proof of Proposition~\ref{prop:var_stability}}

Due to Proposition~\ref{prop:variance}, to show that Eq.~\ref{eq:binary_var} holds, it suffices to show that the covariance between the scores of the different LLMs under coupled generation is non-negative, \ie, $\Cov[R_m(\Ub,S_q),R_{m'}(\Ub,S_q)]\geq 0$.

To this end, we first rewrite the covariance as
    \begin{multline}
            \Cov[R_m(\Ub,S_q),R_{m'}(\Ub,S_q)] = \PP[R_m(\Ub,S_q)=1, R_{m'}(\Ub,S_q)=1] -\PP[R_m(\Ub,S_q)=1]\cdot \PP[R_{m'}(\Ub,S_q)=1] \\ \label{eq:covariance_rewrite}
            =\sum_{s_q} \PP[S_q=s_q] \cdot \left( \PP[R_m(\Ub,s_q)=1, R_{m'}(\Ub,s_q)=1]-\PP[R_m(\Ub,s_q)=1]\cdot \PP[R_{m'}(\Ub,s_q)=1] \right) 
        % \end{split}
    \end{multline}
%
Next, we note that the event $R_m(\Ub,s_q)=1$ is equivalent to LLM $m$ sampling the ground truth token for prompt $s_q$. 
%
Without loss of generality, assume $t_1$ is the ground truth token, \ie, $\textsc{c}(s_q)=t_1$. 
%
Then, since only tokens $\{t_1, t_2\}$ have positive probability under $m$ and $m'$, it must hold that either (i) one LLM assigns a greater probability to $t_1$ and the other LLM assigns a greater probability to $t_2$, 
%
or (ii) both LLMs assign the same probabilities. 
%
Further, since the sampling mechanism defined by $f_T$ and $P_U$ satisfies counterfactual stability, we have that the condition in Eq.~\ref{eq:stability_condition} holds in both (i) and (ii) and, under coupled generation, the LLM with greater (or equal) probability for $t_1$ will always sample $t_1$ when the LLM with lower (or equal) probability does. 
%
This implies that
%
\begin{equation}
    \PP[R_m(\Ub,s_q)=1, R_{m'}(\Ub,s_q)=1] = \min\{ \PP[R_m(\Ub,s_q)=1], \PP[R_{m'}(\Ub,s_q)=1]\}
\end{equation}
%
Finally, since it holds that
%
\begin{equation}
    \min\{ \PP[R_m(\Ub,s_q)=1], \PP[R_{m'}(\Ub,s_q)=1]\} \geq \PP[R_m(\Ub,s_q)=1] \PP[R_{m'}(\Ub,s_q)=1]
\end{equation}
because $\PP[R_m(\Ub,s_q)=1] \in (0,1)$ and $\PP[R_{m'}(\Ub,s_q)=1]\in(0,1)$ by assumption,
% (with strict inequality holding if in $(0,1)$),
we can conclude from Eq.~\ref{eq:covariance_rewrite} that
    \begin{equation}
        \begin{split}
            &\Cov[R_m(\Ub,S_q),R_{m'}(\Ub,S_q)] > 0.
        \end{split}
    \end{equation}
%
% Additionally, we can also conclude that, if $\PP[R_m(\Ub,s_q)=1] \in (0,1)$ 
% and $\PP[R_{m'}(\Ub,s_q)=1]\in (0,1)$, the above inequality is strict.
    
\subsection{Proof of Proposition~\ref{prop:var_gumbel}}
%
Using Proposition~\ref{prop:variance}, we have that
%
    \begin{align*}
            \Cov[R_m(\Ub,S_q),R_{m'}(\Ub,S_q)] &= \EE[R_m(\Ub,S_q)\cdot R_{m'}(\Ub,S_q)]-\EE[R_m(\Ub,S_q)]\cdot  \EE[R_{m'}(\Ub,S_q)]\\
            &= \underbrace{\PP[R_m(\Ub,S_q)=1, R_{m'}(\Ub,S_q)=1
            ]}_{(i)}-\underbrace{\PP[R_m(\Ub,S_q)=1) \cdot \PP[R_{m'}(\Ub,S_q)=1]}_{(ii)}.
    \end{align*}
%
In the remainder of the proof, we will bound each term (i) and (ii) separately and, since $|\textsc{c}(s_q)|=1$ for all $s_q \sim P_{Q}$, assume without loss of generality that the correct token is single-token sequence $t_1$.

To bound the term (ii), first note that, using the definition of the Gumbel-Max SCM, we have that, for each $k \in \{2, \ldots, |V|\}$, it holds that
%
\begin{align*}
            R_{m}(\Ub,s_q)=1 &\iff U_1 + \log ( [f_D(s_q,m)]_{t_1} ) \geq U_k + \log ( [f_D(s_q, m)]_{t_k} ), \\
            R_{m'}(\Ub,s_q)=1 &\iff U_1 + \log ( [f_D(s_q,m')]_{t_1} ) \geq U_k + \log ( [f_D(s_q, m')]_{t_k} ).
\end{align*}
%
Next, let $\varepsilon^*>0$ be an arbitrary constant that we will determine later such that
%
\begin{equation} \label{eq:bound}
    |\log ( [f_D(S_q,m)]_{t_k}) -\log ([f_D(S_q,m')]_{t_k}) |\leq \varepsilon^*,
\end{equation}
%
and note that since, by assumption, $D_{t_k} > 0$ for all $k \in \{1, \ldots, |V|\}$, any bound on the absolute difference of log-probabilities $|\log ( [f_D(S_q,m)]_{t_k}) -\log ([f_D(S_q,m')]_{t_k}) |$ uniformly implies a bound on the difference of probabilities $|[f_D(S_q,m)]_{t_k} -[f_D(S_q,m')]_{t_k} |$ and vice versa. 
%
For simplicity, we prove the result in the log-domain.

Now, using the bound defined by Eq.~\ref{eq:bound}, we have that
%
\begin{multline*}
        \bigcap_{k\neq1} \left\{ U_1 + \log ( [f_D(S_q, m')]_{t_1} ) \geq U_k + \log ( [f_D(S_q, m')]_{t_k} ) \right\}\\
            \subset \bigcap_{k\neq1} \left\{U_1 + \log ( [f_D(S_q, m)]_{t_1} )+\varepsilon^* \geq U_k + \log ( [f_D(S_q, m)]_{t_k} )-\varepsilon^* \right\},
\end{multline*}
%
and we can then bound the term (ii) as follows:
    \begin{align*}
            \PP[R_m(\Ub,S_q)=1] &\cdot \PP[R_{m'}(\Ub,S_q)=1] \\
            &=
            \PP[\cap_{k\neq1} \{U_1 + \log ( [f_D(S_q,m)]_{t_1} ) \geq U_k + \log ( [f_D(S_q,m)]_{t_k} )\}]\\
            & \qquad \times \PP[\cap_{k\neq1}\{ U_1 + \log ( [f_D(S_q,m')]_{t_1} ) \geq U_k + \log ( [f_D(S_q,m')]_{t_k} ) \}]\\
            &\leq \PP[\cap_{k\neq1} \{U_1 + \log ( [f_D(S_q,m)]_{t_1} ) \geq U_k + \log ( [f_D(S_q,m)]_{t_k} )\}]\\
            & \qquad \times \PP[\cap_{k\neq1} \{U_1 + \log ( [f_D(S_q,m)]_{t_1} )+\varepsilon^* \geq U_k + \log ( [f_D(S_q,m)]_{t_k} )-\varepsilon^*\}].
    \end{align*}
%
To bound the term (i), first note that, using the bound defined by Eq.~\ref{eq:bound}, we have that
    \begin{multline*}
        \bigcap_{k\neq1} \left\{U_1 + \log ( [f_D(S_q,m')]_{t_1} ) \geq U_k + \log ( [f_D(S_q,m')]_{t_k} ) \right\}\\
         \supset \bigcap_{k\neq1} \left\{U_1 + \log ( [f_D(S_q,m)]_{t_1} )-\varepsilon^* \geq U_k + \log ( [f_D(S_q,m)]_{t_k} )+\varepsilon^* \right\}.
    \end{multline*} 
%        
Thus, we can bound the term (i) as follows:
%
    \begin{align*}
        \PP[R_m(\Ub,S_q)=1, R_{m'}(\Ub,S_q)=1] &=\PP\Big[\cap_{k\neq1} \{U_1 + \log ( [f_D(S_q,m)]_{t_1} ) \geq U_k + \log ([f_D(S_q,m)]_{t_k} )\}\\
            & \qquad \cap \{ U_1 + \log ( [f_D(S_q,m')]_{t_1} ) \geq U_k + \log ( [f_D(S_q,m')]_{t_k} ) \}\Big] \\
            & \geq \PP\Big[\cap_{k\neq1} \{U_1 + \log ( [f_D(S_q,m)]_{t_1} ) \geq U_k + \log ( [f_D(S_q,m)]_{t_k} )\}\\
            & \qquad \cap \{ U_1 + \log ( [f_D(S_q,m)]_{t_1} ) \geq U_k + \log ( [f_D(S_q,m)]_{t_k} ) +2\varepsilon^* \}\Big] \\
            & \overset{(a)}{=} \sum_{s_q} \PP[S_q=s_q]\cdot \PP[\cap_{k\neq1} \{ U_1 + \log ( [f_D(S_q,m)]_{t_1} ) \\
            & \geq U_k + \log ( [f_D(S_q,m)]_{t_k} ) +2\varepsilon^* \}],
    \end{align*}
where (a) follows from the fact that
%
\begin{multline*}
      \left\{ U_1 + \log ( [f_D(S_q,m)]_{t_1} ) \geq U_k + \log ( [f_D(S_q,m)]_{t_k} ) +2\varepsilon^* \right\} \\
      \subset \left\{ U_1 + \log ( [f_D(S_q,m)]_{t_1} ) \geq U_k + \log ( [f_D(S_q,m)]_{t_k} ) \right\}.
\end{multline*}
%
Now, note that, for $k \in \{2, \ldots, |V|\}$, the variable $X_k \equiv U_1-U_k \sim \text{Logistic}(0,1)$ (for $k=1$, define $X_k \equiv 0$). 
%
Therefore, we can rewrite the bound for (i) as
%  
\begin{multline*}
      \PP[R_m(\Ub,S_q)=1, R_{m'}(\Ub,S_q)=1]\\
      \geq\sum_{s_q} \PP[S_q=s_q] \cdot \prod_{k\neq1}\cdot \PP[ \{ X_k \geq \log ( [f_D(S_q,m)]_{t_k} )-\log ( [f_D(S_q,m)]_{t_1} ) +2\varepsilon^* \}]
\end{multline*}
%
and we can rewrite the bound for (ii) as
%
\begin{multline*}
    \PP[R_m(\Ub,S_q)=1] \PP[R_{m'}(\Ub,S_q)=1]\leq \\
    \sum_{s_q}\PP[S_q=s_q]\cdot 
            \left\{
            \prod_{k\neq1}
            \PP[ \{ X_k \geq \log ( [f_D(S_q,m)]_{t_k} )-\log ( [f_D(S_q,m)]_{t_k} ) - 2\varepsilon^* \} ]
            \right\}\\
            \times \PP[\cap_{k\neq1} \{X_k \geq \log ( [f_D(S_q,m)]_{t_k} )-\log ( [f_D(S_q,m)]_{t_k} )\}].
\end{multline*}
%
As a consequence, to prove that $\PP[R_m(\Ub,S_q)=1, R_{m'}(\Ub,S_q)=1] > \PP[R_m(\Ub,S_q)=1] \PP[R_{m'}(\Ub,S_q)=1]$, it suffices to show that
    \begin{multline}\label{eq:condition-covariance-binary-scores}
            \sum_{s_q}P[S_q=s_q] \prod_{k\neq1}\cdot \PP[ \{ X_k \geq \log ( [f_D(S_q,m)]_{t_k} )-\log ( [f_D(S_q,m)]_{t_1} ) +2\varepsilon^* \}]\\
            > \sum_{s_q}\PP[S_q=s_q] \prod_{k\neq1}\cdot \PP[ \{ X_k \geq \log ( [f_D(S_q,m)]_{t_k} )-\log ( [f_D(S_q,m)]_{t_1} ) - 2\varepsilon^* \}]
            \\
            \times \PP[\cap_{k\neq1} \{X_k \geq \log ( [f_D(S_q,m)]_{t_k} )-\log ( [f_D(S_q,m)]_{t_1} )\}]
    \end{multline}
%
To do so, note that Eq.~\ref{eq:condition-covariance-binary-scores} holds trivially for $\varepsilon^*=0$ since
%
\begin{equation*}
      \PP[\cap_{k\neq1} \{X_k \geq \log ( [f_D(S_q,m)]_{t_k} )-\log ( [f_D(S_q,m)]_{t_1} )\}] < 1,
\end{equation*}
%
which is a fixed term independent of $m'$. Since all terms in Eq.~\ref{eq:condition-covariance-binary-scores} are continuous in $\varepsilon^*$, there exists $\varepsilon^*(m) >0 $, possibly dependent of $m$ but independent of $m'$, such that Eq.~\ref{eq:condition-covariance-binary-scores} holds if
\begin{equation*}
    \sup_{s_q} \norm{\log(f_D(s_q,m))-\log(f_D(s_q, m'))}_\infty < \varepsilon^*(m).
\end{equation*}
Since by assumption $D_t>0$ for all $t\in V$, there exists $\varepsilon(m)>0$ in probability space such that Eq.~\ref{eq:condition-covariance-binary-scores} holds if
\begin{equation*}
    \sup_{s_q} \norm{f_D(s_q,m)-f_D(s_q, m')}_\infty < \varepsilon(m).
\end{equation*}
%
This concludes the proof.

\subsection{Proof of Proposition~\ref{prop:gap_win_rates_stability}}

% We first compute the win-rates under coupled autoregressive generation.
%
Under coupled autoregressive generation, if the LLM $m$ samples the preferred token $t_+$, then the LLM $m'$ must also sample $t_{+}$ because $t_{+}$ is more likely under $m'$ than under $m$ and the sampling mechanism defined by $f_T$ and $P_U$ satisfies counterfactual stability. 
%
This implies that the win-rate achieved by $m$ against $m'$ is
%
    \begin{equation}\label{eq:prop_greater_shared}
            \EE_{\Ub\sim P_{\Ub}} [\one\{R_m(\Ub,s_q)>R_{m'}(\Ub,s_q)\}] =
            \PP[f_T(f_D(s_q, m),\Ub)= t_+, f_T(f_D(s_q, m'),\Ub)= t_-]=0 
    \end{equation}
and that
    \begin{equation}\label{eq:prop_equal_probs+}
            \PP[f_T(f_D(s_q, m),\Ub)= t_+, f_T(f_D(s_q, m'),\Ub)= t_+]= \PP[f_T(f_D(s_q, m),\Ub)= t_+] =p_m.
    \end{equation}
%
Using the same reasoning, if the LLM $m'$ samples the non-preferred token $t_{-}$, then, $m$ must also sample $t_{-}$ because $t_{-}$ is more likely under $m$ than under $m'$. This implies that
%
    \begin{equation}\label{eq:prop_equal_probs-}
            \PP[f_T(f_D(s_q, m),\Ub)= t_-, f_T(f_D(s_q, m'),\Ub)= t_-]= \PP[f_T(f_D(s_q, m'),\Ub)= t_-] = 1-p_{m'}
    \end{equation} 
%
Then, from Eq.~\ref{eq:prop_equal_probs+} and Eq.~\ref{eq:prop_equal_probs-}, we can conclude that
%
    \begin{equation}\label{eq:prop_equal_shared}
        \begin{aligned}
            \EE_{\Ub\sim P_{\Ub}} [\one\{R_m(\Ub,s_q)=R_{m'}(\Ub,s_q)\}] = p_m + (1-p_{m'})
        \end{aligned}
    \end{equation}
%
Finally, from Eq.~\ref{eq:prop_greater_shared} and Eq.~\ref{eq:prop_equal_shared}, we can conclude that the win-rate achieved by $m'$ against $m$ is
%
\begin{multline*}
    \EE_{\Ub\sim P_{\Ub}} [\one\{R_m(\Ub,s_q)<R_{m'}(\Ub,s_q)\}] \\
        = 1 - \EE_{\Ub\sim P_{\Ub}} [\one\{R_m(\Ub,s_q)>R_{m'}(\Ub,s_q)\}] -\EE_{\Ub\sim P_{\Ub}} [\one\{R_m(\Ub,s_q)=R_{m'}(\Ub,s_q)\}] = p_{m'}-p_m.
\end{multline*}

% Next, we compute the win-rates under independent autoregressive generation.
%
Under independent autoregressive generation, the LLMs $m$ and $m'$ sample tokens independently from each other, \ie, $f_T(f_D(s_q, m),\Ub) \perp f_T(f_D(s_q, m'),\Ub')$. 
%
Thus, we can factorize all joint probabilities when computing the win-rates and obtain
%
\begin{align*}\nonumber
    % \begin{split}
        \EE_{\Ub, \Ub'\sim P_{\Ub}} [\one\{R_m(\Ub,s_q)>R_{m'}(\Ub',s_q)\}] &=\PP[f_T(f_D(s_q, m),\Ub)= t_+] \cdot \PP[f_T(f_D(s_q, m'),\Ub')= t_-] \\ &=
        p_m \cdot (1-p_{m'})
    % \end{split}
\end{align*}
and
\begin{equation}\nonumber
    \begin{split}
        \EE_{\Ub, \Ub'\sim P_{\Ub}} & [\one\{R_m(\Ub,s_q)<R_{m'}(\Ub',s_q)\}] 
         = p_{m'} \cdot (1-p_m).
        % we do not write it the = rates in the proposition 
        %\\
        % \text{and} \quad \EE_{\Ub\sim P_{\Ub}, \Ub'\sim P_{\Ub}, S_q\sim P_{Q}} & [\one\{R_m(\Ub,S_q)=R_{m'}(\Ub',S_q)\}] 
        %  = p_m(p_m+\delta) + (1-p_m-\delta)(1-p_m)
    \end{split}
\end{equation}


\subsection{Proof of Proposition~\ref{prop:gumbel_ties}}

We follow the notations and technique of Proposition~\ref{prop:var_gumbel}.
Fix query $s_q$ and consider first the case of independent autoregressive generation. Since each LLM can only assign a non-zero probability to single-token sequences, we have:
\begin{equation*}
    \begin{split}
        \PP[R_m(\Ub, s_q)=R_{m'}(\Ub', s_q)]&=\sum_{k=1}^{|V|}\PP[f_T(f_D(s_q,m),\Ub)=t_k] \PP[f_T(f_D(s_q,m'),\Ub)=t_k]\\
        &<\sum_{k=1}^{|V|}\PP[f_T(f_D(s_q,m),\Ub)=t_k],
    \end{split}
\end{equation*}
%
In the case of coupled autoregressive generation, since
\begin{equation*}
    \PP\left[\{f_T(f_D(s_q,m),\Ub)=t_k\} \cap \{f_T(f_D(s_q,m),\Ub)=t_j \}\right]=0,\; i\neq j,
\end{equation*}
%
we obtain:
\begin{equation*}
    \begin{split}
        &\PP[R_m(\Ub, s_q)=R_{m'}(\Ub, s_q)]\\
        &=\PP\left[\cup_i \{ f_T(f_D(s_q,m),\Ub)=t_k, f_T(f_D(s_q,m'),\Ub)=t_k \} \right]\\
        &=\sum_k\PP[\{f_T(f_D(s_q,m),\Ub)=t_k, f_T(f_D(s_q,m'),\Ub)=t_k \}]\\
        &=\sum_k \PP[f_T(f_D(s_q,m),\Ub)=t_k]\PP[f_T(f_D(s_q,m'),\Ub)=t_k|f_T(f_D(s_q,m),\Ub)=t_k].\\
    \end{split}
\end{equation*}
We now follow \cite{9729603} and expand the posterior Gumbels, $\PP[f_T(f_D(s_q,m'),\Ub)=t_k|f_T(f_D(s_q,m),\Ub)=t_k]$, as truncated Gumbel distributions. In particular, we leverage the fact that
\begin{equation}\label{eq:max of gumbels}
    \max_{t\in V}\{U_t+\log([f_D(s_q,\bullet)]_{t})\} \sim \text{Gumbel}(0,1),
\end{equation}
and that a Gumbel distribution, with parameter $\log (\theta)$, truncated at $b\sim \text{Gumbel}(0,1)$  can be sampled as
\begin{equation}\label{eq: truncated gumbel}
    -\log(\exp(-b)-\log(\eta)/\theta),\; \eta \sim U(0,1).
\end{equation}
Furthermore, by assumption, $D_{t_k} > 0$ for all $k \in \{1, \ldots, |V|\}$, so that any bound on the absolute difference of log-probabilities $|\log ( [f_D(s_q,m)]_{t_k}) -\log ([f_D(s_q,m')]_{t_k}) |$ uniformly implies a bound on the difference of probabilities $|[f_D(s_q,m)]_{t_k} -[f_D(s_q,m')]_{t_k} |$ and vice versa. Using the bound 
\begin{equation*}
    |\log([f_D(s_q,m)]_{t_k})-\log([f_D(s_q,m')]_{t_k})|\leq \varepsilon^*
\end{equation*}
and the Gumbel properties in Eq.~\ref{eq:max of gumbels} and Eq.~\ref{eq: truncated gumbel}, we obtain:
%
\begin{align}
        &\PP[R_m(\Ub, s_q)=R_{m'}(\Ub, s_q)] \nn \\
        &=\sum_k \PP[f_T(f_D(s_q,m),\Ub)=t_k] \nn \\
            & \qquad \quad \times \PP\Bigg[
            \bigcap_k \Big\{
            \log( [f_D(s_1, m')]_{t_k}) - \log([f_D(s_1, m)]_{t_k})-\log(-\log (\eta_k)) \nn \\
            & \qquad \qquad \qquad \qquad \geq \log([f_D(s_1, m')]_{t_j}) - \log([f_D(s_1, m)]_{t_j})-\log(-\log(\eta_k) -\log(\eta_j)/[f_D(s_1,m')]_{t_j})
            \Big\}
            \Bigg] \nn \\
            &\geq
            \sum_k \PP[f_T(f_D(s_q,m),\Ub)=t_k] \nn \\
            &\qquad \quad \times\PP\left[
            \cap_k \{
            -\log(-\log (\eta_k)) \geq -2\varepsilon^* -\log(-\log(\eta_k) -\log(\eta_j)/[f_D(s_1,m')]_{t_j})
            \}
            \right] \label{eq:final-bound}
\end{align}
where $\eta_k\sim \text{U}(0,1)$ are independently distributed uniform random variables.
%
Now, note that the claim holds for $\varepsilon^*=0$ since, in that case, we have that
%
\begin{equation*}
   \PP\left[
        \bigcap_k \left\{
        -\log(-\log (\eta_k)) \geq -\log(-\log(\eta_k) -\log(\eta_k)/[f_D(s_1,m')]_{t_k})
        \right\} \right]=1,
\end{equation*}
%
using that $x\mapsto -\log(x)$ is strictly decreasing. 
%
Since all terms in Eq.~\ref{eq:final-bound} are continuous in $\varepsilon^*$, there exists $\varepsilon^*(m) >0 $, possibly dependent on $m$ but independent of $m'$, such that 
\begin{equation}\label{eq:prop5 claim prob}
    \PP[R_m(\Ub, s_q)=R_{m'}(\Ub, s_q)] > \PP[R_m(\Ub, s_q)=R_{m'}(\Ub', s_q)]
\end{equation}
holds if
\begin{equation*}
    \sup_{s_q} \norm{\log(f_D(s_q,m))-\log(f_D(s_q, m'))}_\infty < \varepsilon^*(m).
\end{equation*}
Since by assumption $D_t>0$ for all $t\in V$, there exists $\varepsilon(m)>0$ in probability space such that Eq.~\ref{eq:prop5 claim prob} holds if

\begin{equation*}
    \sup_{s_q} \norm{f_D(s_q,m)-f_D(s_q, m')}_\infty < \varepsilon(m).
\end{equation*}
%
This concludes the proof.






\subsection{Calculation of average win-rates in the example used in Sections~\ref{sec:intro} and~\ref{sec:pairwise}} \label{app:example-ranking}
%
In this section, we provide detailed calculations of the win-rates for the example in Sections~\ref{sec:intro} and~\ref{sec:pairwise}. Recall that in this example, we are given three LLMs $m_1$, $m_2$ and $m_3$, and we need to rank them according to their ability to answer correctly two types of input prompts, $q$ and $q'$, picked uniformly at random.
%
We assume that the true probability that each LLM answers correctly each type of input prompt is given by:
%
\begin{table}[H]
    \centering
    \begin{tabular}{lccc}
        \toprule
         & $m_1$ & $m_2$ & $m_3$ \\
        \midrule
        $q$           & $p_1=0.4$             & $p_2=0.48$           & $p_3=0.5$             \\
        $q'$           & $p'_1=1$           & $p'_2=0.9$         & $p'_3=0.89$              \\
        \bottomrule
 \end{tabular}
 \vspace{-8mm}
 \caption*{}
 \end{table}

Using Proposition~\ref{prop:gap_win_rates_stability}, 
% under the assumption of counterfactual stability, 
the win-rates under independent autoregressive generation are given, for each LLM $m_k$, by:
\begin{equation}
    \begin{split}
       \frac{1}{2}\sum_{j\neq k}\EE_{\Ub, \Ub' \sim P_{\Ub}, S_q\sim P_Q}[\one\{R_{m_k}(\Ub, S_q)>R_{m_j}(\Ub', S_q)\}] & =\frac{\sum_{j\neq k}p_k(1-p_j) + \sum_{j\neq k}p'_k(1-p'_j)}{4}.
    \end{split}
\end{equation}
%
Substituting the numerical values we obtain:

 \begin{equation}
    \begin{split}
       \frac{1}{2}\sum_{j\neq 1}\EE_{\Ub, \Ub' \sim P_{\Ub}, S_q\sim P_Q}[\one\{R_{m_1}(\Ub, S_q)>R_{m_j}(\Ub', S_q)\}] & =0.1545,\\
       \frac{1}{2}\sum_{j\neq 2}\EE_{\Ub, \Ub' \sim P_{\Ub}, S_q\sim P_Q}[\one\{R_{m_2}(\Ub, S_q)>R_{m_j}(\Ub', S_q)\}] & =0.15675,\\
       \frac{1}{2}\sum_{j\neq 3}\EE_{\Ub, \Ub' \sim P_{\Ub}, S_q\sim P_Q}[\one\{R_{m_3}(\Ub, S_q)>R_{m_j}(\Ub', S_q)\}] & =0.16225\\
    \end{split}
\end{equation}

Similarly, using Proposition~\ref{prop:gap_win_rates_stability}, the win-rates using coupled autoregressive generation can be written, for each LLM $m_k$, as:
\begin{equation}
    \begin{split}
       \frac{1}{2}\sum_{j\neq k}\EE_{\Ub \sim P_{\Ub}, S_q\sim P_Q}[\one\{R_{m_k}(\Ub, S_q)>R_{m_j}(\Ub, S_q)\}] & =\frac{\sum_{j\neq k}(p_k-p_j)_{+} + \sum_{j\neq k}(p'_k-p'_j)_{+}}{4},
    \end{split}
\end{equation}
where $(\bullet)_{+}=\max(0, \bullet)$ denotes the positive part. Substituting the numerical values we obtain:

 \begin{align*}
       \frac{1}{2}\sum_{j\neq 1}\EE_{\Ub \sim P_{\Ub}, S_q\sim P_Q}[\one\{R_{m_1}(\Ub, S_q)>R_{m_j}(\Ub, S_q)\}] & = 
       % \frac{(p_1-p_2)_{+} +(p_1-p_3)_{+} +(p'_1-p'_2)_{+} +(p'_1-p'_3)_{+}}{4}
       0.0525, \\
       \frac{1}{2}\sum_{j\neq 2}\EE_{\Ub \sim P_{\Ub}, S_q\sim P_Q}[\one\{R_{m_2}(\Ub, S_q)>R_{m_j}(\Ub, S_q)\}] &=
       % \frac{(p_2-p_1)_{+} +(p_2-p_3)_{+} +(p'_2-p'_1)_{+} +(p'_2-p'_3)_{+}}{4}=
       0.0225, \\
       \frac{1}{2}\sum_{j\neq 3}\EE_{\Ub \sim P_{\Ub}, S_q\sim P_Q}[\one\{R_{m_3}(\Ub, S_q)>R_{m_j}(\Ub, S_q)\}] &=
       % \frac{(p_3-p_2)_{+} +(p_3-p_1)_{+} +(p'_3-p'_2)_{+} +(p'_3-p'_1)_{+}}{4}=
       0.03.
\end{align*}






\newpage

\section{Additional Experimental Details} \label{app:add_exp}
\vspace{-2mm}

\xhdr{Hardware setup} Our experiments are executed on a compute server equipped with 2 $\times$ Intel Xeon Gold 5317 CPU, $1{,}024$ GB main memory, and $2$ $\times$ A100 Nvidia Tesla GPU ($80$ GB, Ampere Architecture). In each experiment a single Nvidia A100 GPU is used.


\xhdr{Datasets} 
As a benchmark dataset, we use Measuring Massive Multitask Language Understanding dataset (MMLU)~\cite{hendrycks2021measuring} consisting of $14{,}042$ questions covering $52$ diverse knowledge areas with each question offering four possible choices indexed from A to D, and a ground-truth answer. For pairwise comparison tasks, we use the first $500$ questions from the LMSYS-Chat-1M dataset~\cite{zheng2024lmsys}.

\xhdr{Models} In our experiments, we 
% evaluate popular large language models from the \texttt{Llama} family sharing 
% the same token vocabulary.  
%Specifically, we 
use \texttt{Llama-3.1-8B-Instruct}, its quantized variants \texttt{Llama-3.1-8B-Instruct-\{AWQ-INT4, bnb-4bit, bnb-8bit\}} and \texttt{Llama-3.2-\{1B, 3B\}-Instruct} models. The models are obtained from \texttt{Hugging Face}, and the quantised LLM variants \texttt{Llama-3.1-8B-Instruct-\{bnb-4bit, bnb-8bit\}} are built using the \texttt{bitsandbytes} library~\cite{bitsnbytes2024bits}.

\xhdr{Prompts} To instruct LLMs for generating output, we use the system prompt in Table~\ref{app:sys_prompt_2} for the MMLU dataset and Table~\ref{app:sys_prompt_3} for the LMSYS-Chat-1M dataset. Further, to perform pairwise comparisons of outputs of different LLMs, we use the system prompt in Table~\ref{app:sys_prompt_1}, which is adapted from~\cite{chiang2024chatbot}, to prompt the strong LLM.

\begin{table}[h]
\centering
\begin{tcolorbox}[
    colframe=white,      % Border color
    colback=gray!14,     % Background color
    boxrule=0.5mm,       % Border thickness
    arc=4mm,             % Rounded corners
    left=3mm,            % Left margin
    right=3mm,           % Right margin
    top=3mm,             % Top margin
    bottom=3mm           % Bottom margin
]
\begin{tabular}{ m{15.2cm} }
\rowcolor{gray!14} 
    \textbf{System:} Please act as an impartial judge and evaluate the quality of the responses provided by two AI assistants to the user prompt displayed below. Your job is to evaluate which assistant's answer is better. When evaluating the assistants' answers, compare both assistants' answers. You must identify and correct any mistakes or inaccurate information. Then consider if the assistant's answers are helpful, relevant, and concise. Helpful means the answer correctly responds to the prompt or follows the instructions. Note when user prompt has any ambiguity or more than one interpretation, it is more helpful and appropriate to ask for clarifications or more information from the user than providing an answer based on assumptions. Relevant means all parts of the response closely connect or are appropriate to what is being asked. Concise means the response is clear and not verbose or excessive. Then consider the creativity and novelty of the assistant's answers when needed. Finally, identify any missing important information in the assistants' answers that would be beneficial to include when responding to the user prompt. do not provide any justification or explanation for your response. You must output only one of the following choices as your final verdict: 

    \vspace{2mm}
    `A' if the response of assistant A is better
    
    \vspace{1mm}
    `B' if the response of assistant B is better
    
    \vspace{1mm}
    `Tie' if the responses are tied
\end{tabular}
\end{tcolorbox}
\vspace{-3mm}
\caption{\label{app:system_prompt}System prompt used for obtaining pairwise preferences using \texttt{GPT-4o-2024-11-20} as the judge.}
\label{app:sys_prompt_1}
\end{table}


\vspace{-3mm}

\begin{table}[h]
\centering
\begin{tcolorbox}[
    colframe=white,      % Border color
    colback=gray!14,     % Background color
    boxrule=0.5mm,       % Border thickness
    arc=4mm,             % Rounded corners
    left=3mm,            % Left margin
    right=3mm,           % Right margin
    top=3mm,             % Top margin
    bottom=3mm           % Bottom margin
]
\begin{tabular}{ m{15.2cm} }
    \rowcolor{gray!14}
    \textbf{System:} You will be given multiple choice questions. Please reply with a single character `A', `B', `C', or `D' only. DO NOT explain your reply.
\end{tabular}
\end{tcolorbox}
\vspace{-3mm}
\caption{System prompt used for the MMLU dataset.}
\label{app:sys_prompt_2}
\end{table}


\vspace{-3mm}

\begin{table}[h]
\centering
\begin{tcolorbox}[
    colframe=white,      % Border color
    colback=gray!14,     % Background color
    boxrule=0.5mm,       % Border thickness
    arc=4mm,             % Rounded corners
    left=3mm,            % Left margin
    right=3mm,           % Right margin
    top=3mm,             % Top margin
    bottom=3mm           % Bottom margin
]
\begin{tabular}{ m{15.2cm} }
    \rowcolor{gray!14}
    \textbf{System:} Keep your responses short and to the point.
\end{tabular}
\end{tcolorbox}
\vspace{-3mm}
\caption{System prompt used for the LMSYS Chatbot Arena dataset.}
\label{app:sys_prompt_3}
\end{table}






\clearpage
\newpage

% \section{Additional Experimental Results} \label{app:exp_results}
%In this section, we present experimental results supplementing those from Section~\ref{sec:experiments}, first for the MMLU dataset, and then the LMSYS-Chat-1M dataset.

\section{Additional Experimental Results on the MMLU Dataset} \label{app:mmlu}
% Here, we present experimental results supplementing the results from 
% Section~\ref{sec:experiments}, comparing different pairs of models in 
% Figure~\ref{fig:mmlu-1B-vs-3B-models} and different knowledge areas in 
% Figure~\ref{fig:mmlu-1B-vs-3B-areas}. 

\begin{figure}[h]
\centering
\begin{tabular}{c c c}
    \multicolumn{3}{c}{\texttt{Llama-3.2-1B-Instruct} vs. \texttt{Llama-3.2-3B-Instruct}}\\
    \includegraphics[width=0.27\linewidth]{./figures/mmlu/covariance/college_computer_science/cov_Llama-3.2-3B-Instruct_Llama-3.2-1B-Instruct_kde.pdf} &
    \includegraphics[width=0.27\linewidth]{./figures/mmlu/variance/college_computer_science/var_Llama-3.2-3B-Instruct_Llama-3.2-1B-Instruct_kde.pdf} &
    \includegraphics[width=0.27\linewidth]{./figures/mmlu/error/college_computer_science/error_Llama-3.2-3B-Instruct_Llama-3.2-1B-Instruct.pdf} \\ \\
%
    \multicolumn{3}{c}{\texttt{Llama-3.2-1B-Instruct} vs. \texttt{Llama-3.1-8B-Instruct}}\\
    \includegraphics[width=0.27\linewidth]{./figures/mmlu/covariance/college_computer_science/cov_Llama-3.2-1B-Instruct_Llama-3.1-8B-Instruct_kde.pdf} &
    \includegraphics[width=0.27\linewidth]{./figures/mmlu/variance/college_computer_science/var_Llama-3.1-8B-Instruct_Llama-3.2-1B-Instruct_kde.pdf} &
    \includegraphics[width=0.27\linewidth]{./figures/mmlu/error/college_computer_science/error_Llama-3.1-8B-Instruct_Llama-3.2-1B-Instruct.pdf} \\ \\
%
    \multicolumn{3}{c}{\texttt{Llama-3.2-3B-Instruct} vs. \texttt{Llama-3.1-8B-Instruct}}\\
    \includegraphics[width=0.27\linewidth]{./figures/mmlu/covariance/college_computer_science/cov_Llama-3.1-8B-Instruct_Llama-3.2-3B-Instruct_kde.pdf} &
    \includegraphics[width=0.27\linewidth]{./figures/mmlu/variance/college_computer_science/var_Llama-3.1-8B-Instruct_Llama-3.2-3B-Instruct_kde.pdf} &
    \includegraphics[width=0.27\linewidth]{./figures/mmlu/error/college_computer_science/error_Llama-3.1-8B-Instruct_Llama-3.2-3B-Instruct.pdf} \\ \\
%
    (a) Score covariance & (b) Variance of the score difference & (c) Estimation error vs. \# samples \\ 
    
\end{tabular}
    \caption{\textbf{Comparison between three pairs of LLMs on multiple-choice questions from the ``college computer science'' knowledge area of the MMLU dataset.}
    Panels in column (a) show the kernel density estimate (KDE) of the covariance between the scores of the two LLMs on each question under coupled generation; the dashed lines correspond to average values. Panels in column (b) show the KDE of the variance of the difference between the scores of the LLMs on each question under coupled and independent generation; the highlighted points correspond to median values. Panels in column (c) show the absolute error in the estimation of the expected difference between the scores of the LLMs against the number of samples; for each point on the x-axis, we perform $1{,}000$ sub-samplings and shaded areas correspond to $95\%$ confidence intervals.}
    \label{fig:mmlu-1B-vs-3B-models}
\end{figure}


\begin{figure}[h]
\centering
\begin{tabular}{c c c}
%
    \multicolumn{3}{c}{\textbf{College chemistry}} \\
    \includegraphics[width=0.27\linewidth]{./figures/mmlu/covariance/college_chemistry/cov_Llama-3.2-3B-Instruct_Llama-3.2-1B-Instruct_kde.pdf} &
    \includegraphics[width=0.27\linewidth]{./figures/mmlu/variance/college_chemistry/var_Llama-3.2-3B-Instruct_Llama-3.2-1B-Instruct_kde.pdf} &
    \includegraphics[width=0.27\linewidth]{./figures/mmlu/error/college_chemistry/error_Llama-3.2-3B-Instruct_Llama-3.2-1B-Instruct.pdf} \\
%
    \multicolumn{3}{c}{\textbf{Professional accounting}} \\
    \includegraphics[width=0.27\linewidth]{./figures/mmlu/covariance/professional_accounting/cov_Llama-3.2-3B-Instruct_Llama-3.2-1B-Instruct_kde.pdf} &
    \includegraphics[width=0.27\linewidth]{./figures/mmlu/variance/professional_accounting/var_Llama-3.2-3B-Instruct_Llama-3.2-1B-Instruct_kde.pdf} &
    \includegraphics[width=0.27\linewidth]{./figures/mmlu/error/professional_accounting/error_Llama-3.2-3B-Instruct_Llama-3.2-1B-Instruct.pdf} \\
% 
    \multicolumn{3}{c}{\textbf{Professional law}}\\
    \includegraphics[width=0.27\linewidth]{./figures/mmlu/covariance/professional_law/cov_Llama-3.2-3B-Instruct_Llama-3.2-1B-Instruct_kde.pdf} &
    \includegraphics[width=0.27\linewidth]{./figures/mmlu/variance/professional_law/var_Llama-3.2-3B-Instruct_Llama-3.2-1B-Instruct_kde.pdf} &
    \includegraphics[width=0.27\linewidth]{./figures/mmlu/error/professional_law/error_Llama-3.2-3B-Instruct_Llama-3.2-1B-Instruct.pdf} \\ \\
%
    \multicolumn{3}{c}{\textbf{Professional medicine}} \\   
    \includegraphics[width=0.27\linewidth]{./figures/mmlu/covariance/professional_medicine/cov_Llama-3.2-3B-Instruct_Llama-3.2-1B-Instruct_kde.pdf} &
    \includegraphics[width=0.27\linewidth]{./figures/mmlu/variance/professional_medicine/var_Llama-3.2-3B-Instruct_Llama-3.2-1B-Instruct_kde.pdf} &
    \includegraphics[width=0.27\linewidth]{./figures/mmlu/error/professional_medicine/error_Llama-3.2-3B-Instruct_Llama-3.2-1B-Instruct.pdf} \\
%
    (a) Score covariance & (b) Variance of the score difference & (c) Estimation error vs. \# samples \\ 
\end{tabular}
    \caption{\textbf{Comparison between \texttt{Llama-3.2-1B-Instruct} and \texttt{Llama-3.2-3B-Instruct} on multiple-choice questions from four knowledge areas of the MMLU dataset.}
    Panels in column (a) show the kernel density estimate (KDE) of the covariance between the scores of the two LLMs on each question under coupled generation; the dashed lines correspond to average values. Panels in column (b) show the KDE of the variance of the difference between the scores of the LLMs on each question under coupled and independent generation; the highlighted points correspond to median values. Panels in column (c) show the absolute error in the estimation of the expected difference between the scores of the LLMs against the number of samples; for each point on the x-axis, we perform $1{,}000$ sub-samplings and shaded areas correspond to $95\%$ confidence intervals. We observe qualitatively similar results for other knowledge areas.}
    \label{fig:mmlu-1B-vs-3B-areas}
\end{figure}

\clearpage
\newpage

\section{Additional Experimental Results on the LMSYS-Chat-1M Dataset} \label{app:lmsys}
% Here, we report the empirical win-rate of each LLM against other LLMs using 
% coupled and independent generation in Figure~\ref{fig:lmsys-all-llms}.
%
\begin{figure}[h]
\centering
\includegraphics[width=0.72\linewidth]{figures/lmsys/all.pdf}
%
\caption{
\textbf{Empirical win-rate of each LLM against other LLMs 
on questions from the LMSYS-Chat-1M dataset.} 
%
Empirical estimate of the win-rate under coupled autoregressive generation as given by Eq.~\ref{eq:coupled-generation-win-rates} and under independent generation generation as given by  Eq.~\ref{eq:independent-generation-win-rates}. 
%
Each empirical win-rate is computed using pairwise comparisons between the outputs of each LLM and any other LLM over $500$ questions with $10$ (different) random seeds.
%
The error bars correspond to $95\%$ confidence intervals.
%
For each pair of empirical win-rates, we conduct a two-tailed test, to test the hypothesis that the empirical win-rates are the same; (\fourstars, \threestars, \twostars, \onestar) indicate $p$-values ($<0.0001$, $<0.001$, $<0.01$, $<0.05$), respectively.
}
\label{fig:lmsys-all-llms}
\end{figure}


\end{document}

\section{Formal Definition of Counterfactual Stability}\label{app:stability}

Counterfactual stability is a desirable property of SCMs~\citep{oberst2019counterfactual} that has previously been used in the context of autoregressive generation of LLMs~\cite{chatzi2024counterfactual}. In the following, we provide its formal definition along with a simple example to explain the intuition behind it. Throughout this section, $\PP^{\Ccal \,;\, do(\cdot)}$ denotes the probability of the interventional distribution entailed by an SCM $\Ccal$ under an intervention $do(\cdot)$.
% and $do(\cdot)$ denotes interventions to random variables. 
Moreover, $\PP^{\Ccal \,|\, \star \,;\, do(\cdot)}$ denotes the probability of the counterfactual distribution entailed by an SCM $\Ccal$ under an intervention $do(\cdot)$ given that an observed event $\star$ has already occurred.

\begin{definition}
    A sampling mechanism defined by $f_T$ and $P_U$ satisfies counterfactual stability if for all LLMs $m,m'\in \Mcal$, $i\in\{1,2,\ldots,K\}$ and tokens $t_1,t_2 \in V$ with $t_1\neq t_2$, the condition 
    \begin{equation}\label{eq:stability_condition}
        \frac{\PP^{\Ccal \,;\, do(M=m')}[T_i=t_1 \given D_i]}{\PP^{\Ccal \,;\, do(M=m)}[T_i=t_1 \given D_i]} \geq \frac{\PP^{\Ccal \,;\, do(M=m')}[T_i=t_2 \given D_i]}{\PP^{\Ccal \,;\, do(M=m)}[T_i=t_2 \given D_i]}
    \end{equation}
    implies that $\PP^{\Ccal \,|\, D_i,M=m,T_i=t_1 \,;\, do(M=m')}[T_i=t_2]=0$.
\end{definition}

The property of counterfactual stability has an intuitive interpretation that can be best understood via a simple example. Assume that the vocabulary contains $2$ tokens ``A'' and ``B'' and, using LLM $m$, the next-token distribution at a time step $i$ assigns values $0.6$, $0.4$ to the two tokens, respectively. Moreover, the realized noise value $\ub_i$ is such that the token ``A'' is sampled.
% , that is, $\text{``A''}=f_T(d_i, u_i)$.
Now, consider that, while keeping the noise value $\ub_i$ fixed, we change the LLM to $m'$, resulting in a next-token distribution that assigns values $0.7$, $0.3$ to the two tokens, respectively. Counterfactual stability ensures that, since the noise value $\ub_i$ led to ``A'' being sampled under $m$ at $0.6$ to $0.4$ odds, the same value
cannot lead to ``B'' being sampled under $m'$ where its relative odds are lower (\ie, $0.3$ to $0.7$).
% also leads to ``A'' being sampled under $m'$ where its relative odds are higher (\ie, $0.7$ to $0.3$).

\section{Proofs} \label{sec:proofs}

\subsection{Proof of Proposition~\ref{prop:variance}}
 We can rewrite the variance of the difference in scores under independent generation in terms of the variance of the difference in scores under coupled generation as follows:
 %
    \begin{align*}
        \Var[R_m(\Ub,S_q)-R_{m'}(\Ub',S_q)] ] 
        &= \Var[R_m(\Ub,S_q)-R_{m'}(\Ub,S_q) + R_{m'}(\Ub,S_q) -R_{m'}(\Ub',S_q)] ] \nonumber
        \\ &= \Var[R_m(\Ub,S_q)-R_{m'}(\Ub,S_q)] + \Var[R_{m'}(\Ub,S_q)-R_{m'}(\Ub',S_q)] \nn
            \\ &+ 2 \cdot \Cov[R_m(\Ub,S_q)-R_{m'}(\Ub,S_q), R_{m'}(\Ub,S_q)-R_{m'}(\Ub',S_q) ]. 
    \end{align*}
%
For the variance of the difference in scores for the same LLM under independent noise values, we have that
%
    \begin{align*}
        \Var[R_{m'}(\Ub,S_q)-R_{m'}(\Ub',S_q)]
        & \overset{(a)}{=} \EE[(R_{m'}(\Ub,S_q)-R_{m'}(\Ub',S_q))^2] -
         \EE[R_{m'}(\Ub,S_q)-R_{m'}(\Ub',S_q)]^2 % \label{eq:defvar1}
        \\& \overset{(b)}{=} \EE[R_{m'}(\Ub,S_q)^2-2\cdot R_{m'}(\Ub,S_q)R_{m'}(\Ub',S_q) + R_{m'}(\Ub',S_q)^2] % \label{eq:zeroterm}
        \\& \overset{(c)}{=} 2 \cdot \EE[R_{m'}(\Ub,S_q)^2] - 2 \cdot \EE[R_{m'}(\Ub,S_q)R_{m'}(\Ub',S_q)], % \label{eq:rewrite}
    \end{align*}
%
where (a) holds by the definition of variance, (b) is due to the subtraction term being $0$, and (c) is due to the linearity of expectation.
%
Further, for the covariance of the difference in scores under independent generation and the difference in scores under coupled generation, we have that
%
\begingroup
\allowdisplaybreaks
\begin{align*}
    \Cov[R_m(\Ub,S_q)-R_{m'}&(\Ub,S_q), R_{m'}(\Ub,S_q)-R_{m'}(\Ub',S_q)]\nonumber
    \\ % \begin{split}\label{eq:covdef}
    &\overset{(a)}{=} \EE[\left(R_m(\Ub,S_q)-R_{m'}(\Ub,S_q)\right) \cdot \left( R_{m'}(\Ub,S_q)-R_{m'}(\Ub',S_q) \right)] 
    \\ &\quad \quad - \EE[R_m(\Ub,S_q)-R_{m'}(\Ub,S_q)]\cdot \EE[R_{m'}(\Ub,S_q)-R_{m'}(\Ub',S_q)]
    % \end{split}
    \\ % \begin{split}\label{eq:zeroexp}
    &\overset{(b)}{=} \EE[R_m(\Ub,S_q)R_{m'}(\Ub,S_q)] - \EE[R_m(\Ub,S_q)R_{m'}(\Ub',S_q)]
    \\ &\quad \quad - \EE[R_{m'}(\Ub,S_q)R_{m'}(\Ub,S_q)] + \EE[R_{m'}(\Ub,S_q)R_{m'}(\Ub',S_q)] 
    \\ % \end{split}
    & \overset{(c)}{=} \Cov[R_m(\Ub,S_q),R_{m'}(\Ub,S_q)] - \EE[R_{m'}(\Ub,S_q)^2] + \EE[R_{m'}(\Ub,S_q)R_{m'}(\Ub',S_q)]] % \label{eq:covfinal}
\end{align*}
\endgroup
where (a) and (c) hold by the definition of covariance and (b) is due to the last term being zero and by the expansion of the first term.

Putting all the above results together, it follows that
%
\begin{align*}
    \Var[R_m(\Ub,S_q)-R_{m'}(\Ub',S_q)] ]
            % \\ \begin{split}
        & = \Var[R_m(\Ub,S_q)-R_{m'}(\Ub,S_q)] 
        + 2\cdot \Cov\left[R_m(\Ub,S_q),R_{m'}(\Ub,S_q)\right] \\
         & \quad+ 2\cdot \EE[R_{m'}(\Ub,S_q)R_{m'}(\Ub',S_q)] 
         - 2\cdot \EE[R_{m'}(\Ub,S_q)^2] 
         + 2 \cdot \EE[R_{m'}(\Ub,S_q)^2] \\
         & \quad - 2 \cdot \EE[R_{m'}(\Ub,S_q)R_{m'}(\Ub',S_q)]
       %  \end{split}
        \\&= \Var[R_m(\Ub,S_q)-R_{m'}(\Ub,S_q)] 
        + 2\cdot \Cov[R_m(\Ub,S_q),R_{m'}(\Ub,S_q)] 
\end{align*}
which concludes the proof.


\subsection{Proof of Proposition~\ref{prop:var_stability}}

Due to Proposition~\ref{prop:variance}, to show that Eq.~\ref{eq:binary_var} holds, it suffices to show that the covariance between the scores of the different LLMs under coupled generation is non-negative, \ie, $\Cov[R_m(\Ub,S_q),R_{m'}(\Ub,S_q)]\geq 0$.

To this end, we first rewrite the covariance as
    \begin{multline}
            \Cov[R_m(\Ub,S_q),R_{m'}(\Ub,S_q)] = \PP[R_m(\Ub,S_q)=1, R_{m'}(\Ub,S_q)=1] -\PP[R_m(\Ub,S_q)=1]\cdot \PP[R_{m'}(\Ub,S_q)=1] \\ \label{eq:covariance_rewrite}
            =\sum_{s_q} \PP[S_q=s_q] \cdot \left( \PP[R_m(\Ub,s_q)=1, R_{m'}(\Ub,s_q)=1]-\PP[R_m(\Ub,s_q)=1]\cdot \PP[R_{m'}(\Ub,s_q)=1] \right) 
        % \end{split}
    \end{multline}
%
Next, we note that the event $R_m(\Ub,s_q)=1$ is equivalent to LLM $m$ sampling the ground truth token for prompt $s_q$. 
%
Without loss of generality, assume $t_1$ is the ground truth token, \ie, $\textsc{c}(s_q)=t_1$. 
%
Then, since only tokens $\{t_1, t_2\}$ have positive probability under $m$ and $m'$, it must hold that either (i) one LLM assigns a greater probability to $t_1$ and the other LLM assigns a greater probability to $t_2$, 
%
or (ii) both LLMs assign the same probabilities. 
%
Further, since the sampling mechanism defined by $f_T$ and $P_U$ satisfies counterfactual stability, we have that the condition in Eq.~\ref{eq:stability_condition} holds in both (i) and (ii) and, under coupled generation, the LLM with greater (or equal) probability for $t_1$ will always sample $t_1$ when the LLM with lower (or equal) probability does. 
%
This implies that
%
\begin{equation}
    \PP[R_m(\Ub,s_q)=1, R_{m'}(\Ub,s_q)=1] = \min\{ \PP[R_m(\Ub,s_q)=1], \PP[R_{m'}(\Ub,s_q)=1]\}
\end{equation}
%
Finally, since it holds that
%
\begin{equation}
    \min\{ \PP[R_m(\Ub,s_q)=1], \PP[R_{m'}(\Ub,s_q)=1]\} \geq \PP[R_m(\Ub,s_q)=1] \PP[R_{m'}(\Ub,s_q)=1]
\end{equation}
because $\PP[R_m(\Ub,s_q)=1] \in (0,1)$ and $\PP[R_{m'}(\Ub,s_q)=1]\in(0,1)$ by assumption,
% (with strict inequality holding if in $(0,1)$),
we can conclude from Eq.~\ref{eq:covariance_rewrite} that
    \begin{equation}
        \begin{split}
            &\Cov[R_m(\Ub,S_q),R_{m'}(\Ub,S_q)] > 0.
        \end{split}
    \end{equation}
%
% Additionally, we can also conclude that, if $\PP[R_m(\Ub,s_q)=1] \in (0,1)$ 
% and $\PP[R_{m'}(\Ub,s_q)=1]\in (0,1)$, the above inequality is strict.
    
\subsection{Proof of Proposition~\ref{prop:var_gumbel}}
%
Using Proposition~\ref{prop:variance}, we have that
%
    \begin{align*}
            \Cov[R_m(\Ub,S_q),R_{m'}(\Ub,S_q)] &= \EE[R_m(\Ub,S_q)\cdot R_{m'}(\Ub,S_q)]-\EE[R_m(\Ub,S_q)]\cdot  \EE[R_{m'}(\Ub,S_q)]\\
            &= \underbrace{\PP[R_m(\Ub,S_q)=1, R_{m'}(\Ub,S_q)=1
            ]}_{(i)}-\underbrace{\PP[R_m(\Ub,S_q)=1) \cdot \PP[R_{m'}(\Ub,S_q)=1]}_{(ii)}.
    \end{align*}
%
In the remainder of the proof, we will bound each term (i) and (ii) separately and, since $|\textsc{c}(s_q)|=1$ for all $s_q \sim P_{Q}$, assume without loss of generality that the correct token is single-token sequence $t_1$.

To bound the term (ii), first note that, using the definition of the Gumbel-Max SCM, we have that, for each $k \in \{2, \ldots, |V|\}$, it holds that
%
\begin{align*}
            R_{m}(\Ub,s_q)=1 &\iff U_1 + \log ( [f_D(s_q,m)]_{t_1} ) \geq U_k + \log ( [f_D(s_q, m)]_{t_k} ), \\
            R_{m'}(\Ub,s_q)=1 &\iff U_1 + \log ( [f_D(s_q,m')]_{t_1} ) \geq U_k + \log ( [f_D(s_q, m')]_{t_k} ).
\end{align*}
%
Next, let $\varepsilon^*>0$ be an arbitrary constant that we will determine later such that
%
\begin{equation} \label{eq:bound}
    |\log ( [f_D(S_q,m)]_{t_k}) -\log ([f_D(S_q,m')]_{t_k}) |\leq \varepsilon^*,
\end{equation}
%
and note that since, by assumption, $D_{t_k} > 0$ for all $k \in \{1, \ldots, |V|\}$, any bound on the absolute difference of log-probabilities $|\log ( [f_D(S_q,m)]_{t_k}) -\log ([f_D(S_q,m')]_{t_k}) |$ uniformly implies a bound on the difference of probabilities $|[f_D(S_q,m)]_{t_k} -[f_D(S_q,m')]_{t_k} |$ and vice versa. 
%
For simplicity, we prove the result in the log-domain.

Now, using the bound defined by Eq.~\ref{eq:bound}, we have that
%
\begin{multline*}
        \bigcap_{k\neq1} \left\{ U_1 + \log ( [f_D(S_q, m')]_{t_1} ) \geq U_k + \log ( [f_D(S_q, m')]_{t_k} ) \right\}\\
            \subset \bigcap_{k\neq1} \left\{U_1 + \log ( [f_D(S_q, m)]_{t_1} )+\varepsilon^* \geq U_k + \log ( [f_D(S_q, m)]_{t_k} )-\varepsilon^* \right\},
\end{multline*}
%
and we can then bound the term (ii) as follows:
    \begin{align*}
            \PP[R_m(\Ub,S_q)=1] &\cdot \PP[R_{m'}(\Ub,S_q)=1] \\
            &=
            \PP[\cap_{k\neq1} \{U_1 + \log ( [f_D(S_q,m)]_{t_1} ) \geq U_k + \log ( [f_D(S_q,m)]_{t_k} )\}]\\
            & \qquad \times \PP[\cap_{k\neq1}\{ U_1 + \log ( [f_D(S_q,m')]_{t_1} ) \geq U_k + \log ( [f_D(S_q,m')]_{t_k} ) \}]\\
            &\leq \PP[\cap_{k\neq1} \{U_1 + \log ( [f_D(S_q,m)]_{t_1} ) \geq U_k + \log ( [f_D(S_q,m)]_{t_k} )\}]\\
            & \qquad \times \PP[\cap_{k\neq1} \{U_1 + \log ( [f_D(S_q,m)]_{t_1} )+\varepsilon^* \geq U_k + \log ( [f_D(S_q,m)]_{t_k} )-\varepsilon^*\}].
    \end{align*}
%
To bound the term (i), first note that, using the bound defined by Eq.~\ref{eq:bound}, we have that
    \begin{multline*}
        \bigcap_{k\neq1} \left\{U_1 + \log ( [f_D(S_q,m')]_{t_1} ) \geq U_k + \log ( [f_D(S_q,m')]_{t_k} ) \right\}\\
         \supset \bigcap_{k\neq1} \left\{U_1 + \log ( [f_D(S_q,m)]_{t_1} )-\varepsilon^* \geq U_k + \log ( [f_D(S_q,m)]_{t_k} )+\varepsilon^* \right\}.
    \end{multline*} 
%        
Thus, we can bound the term (i) as follows:
%
    \begin{align*}
        \PP[R_m(\Ub,S_q)=1, R_{m'}(\Ub,S_q)=1] &=\PP\Big[\cap_{k\neq1} \{U_1 + \log ( [f_D(S_q,m)]_{t_1} ) \geq U_k + \log ([f_D(S_q,m)]_{t_k} )\}\\
            & \qquad \cap \{ U_1 + \log ( [f_D(S_q,m')]_{t_1} ) \geq U_k + \log ( [f_D(S_q,m')]_{t_k} ) \}\Big] \\
            & \geq \PP\Big[\cap_{k\neq1} \{U_1 + \log ( [f_D(S_q,m)]_{t_1} ) \geq U_k + \log ( [f_D(S_q,m)]_{t_k} )\}\\
            & \qquad \cap \{ U_1 + \log ( [f_D(S_q,m)]_{t_1} ) \geq U_k + \log ( [f_D(S_q,m)]_{t_k} ) +2\varepsilon^* \}\Big] \\
            & \overset{(a)}{=} \sum_{s_q} \PP[S_q=s_q]\cdot \PP[\cap_{k\neq1} \{ U_1 + \log ( [f_D(S_q,m)]_{t_1} ) \\
            & \geq U_k + \log ( [f_D(S_q,m)]_{t_k} ) +2\varepsilon^* \}],
    \end{align*}
where (a) follows from the fact that
%
\begin{multline*}
      \left\{ U_1 + \log ( [f_D(S_q,m)]_{t_1} ) \geq U_k + \log ( [f_D(S_q,m)]_{t_k} ) +2\varepsilon^* \right\} \\
      \subset \left\{ U_1 + \log ( [f_D(S_q,m)]_{t_1} ) \geq U_k + \log ( [f_D(S_q,m)]_{t_k} ) \right\}.
\end{multline*}
%
Now, note that, for $k \in \{2, \ldots, |V|\}$, the variable $X_k \equiv U_1-U_k \sim \text{Logistic}(0,1)$ (for $k=1$, define $X_k \equiv 0$). 
%
Therefore, we can rewrite the bound for (i) as
%  
\begin{multline*}
      \PP[R_m(\Ub,S_q)=1, R_{m'}(\Ub,S_q)=1]\\
      \geq\sum_{s_q} \PP[S_q=s_q] \cdot \prod_{k\neq1}\cdot \PP[ \{ X_k \geq \log ( [f_D(S_q,m)]_{t_k} )-\log ( [f_D(S_q,m)]_{t_1} ) +2\varepsilon^* \}]
\end{multline*}
%
and we can rewrite the bound for (ii) as
%
\begin{multline*}
    \PP[R_m(\Ub,S_q)=1] \PP[R_{m'}(\Ub,S_q)=1]\leq \\
    \sum_{s_q}\PP[S_q=s_q]\cdot 
            \left\{
            \prod_{k\neq1}
            \PP[ \{ X_k \geq \log ( [f_D(S_q,m)]_{t_k} )-\log ( [f_D(S_q,m)]_{t_k} ) - 2\varepsilon^* \} ]
            \right\}\\
            \times \PP[\cap_{k\neq1} \{X_k \geq \log ( [f_D(S_q,m)]_{t_k} )-\log ( [f_D(S_q,m)]_{t_k} )\}].
\end{multline*}
%
As a consequence, to prove that $\PP[R_m(\Ub,S_q)=1, R_{m'}(\Ub,S_q)=1] > \PP[R_m(\Ub,S_q)=1] \PP[R_{m'}(\Ub,S_q)=1]$, it suffices to show that
    \begin{multline}\label{eq:condition-covariance-binary-scores}
            \sum_{s_q}P[S_q=s_q] \prod_{k\neq1}\cdot \PP[ \{ X_k \geq \log ( [f_D(S_q,m)]_{t_k} )-\log ( [f_D(S_q,m)]_{t_1} ) +2\varepsilon^* \}]\\
            > \sum_{s_q}\PP[S_q=s_q] \prod_{k\neq1}\cdot \PP[ \{ X_k \geq \log ( [f_D(S_q,m)]_{t_k} )-\log ( [f_D(S_q,m)]_{t_1} ) - 2\varepsilon^* \}]
            \\
            \times \PP[\cap_{k\neq1} \{X_k \geq \log ( [f_D(S_q,m)]_{t_k} )-\log ( [f_D(S_q,m)]_{t_1} )\}]
    \end{multline}
%
To do so, note that Eq.~\ref{eq:condition-covariance-binary-scores} holds trivially for $\varepsilon^*=0$ since
%
\begin{equation*}
      \PP[\cap_{k\neq1} \{X_k \geq \log ( [f_D(S_q,m)]_{t_k} )-\log ( [f_D(S_q,m)]_{t_1} )\}] < 1,
\end{equation*}
%
which is a fixed term independent of $m'$. Since all terms in Eq.~\ref{eq:condition-covariance-binary-scores} are continuous in $\varepsilon^*$, there exists $\varepsilon^*(m) >0 $, possibly dependent of $m$ but independent of $m'$, such that Eq.~\ref{eq:condition-covariance-binary-scores} holds if
\begin{equation*}
    \sup_{s_q} \norm{\log(f_D(s_q,m))-\log(f_D(s_q, m'))}_\infty < \varepsilon^*(m).
\end{equation*}
Since by assumption $D_t>0$ for all $t\in V$, there exists $\varepsilon(m)>0$ in probability space such that Eq.~\ref{eq:condition-covariance-binary-scores} holds if
\begin{equation*}
    \sup_{s_q} \norm{f_D(s_q,m)-f_D(s_q, m')}_\infty < \varepsilon(m).
\end{equation*}
%
This concludes the proof.

\subsection{Proof of Proposition~\ref{prop:gap_win_rates_stability}}

% We first compute the win-rates under coupled autoregressive generation.
%
Under coupled autoregressive generation, if the LLM $m$ samples the preferred token $t_+$, then the LLM $m'$ must also sample $t_{+}$ because $t_{+}$ is more likely under $m'$ than under $m$ and the sampling mechanism defined by $f_T$ and $P_U$ satisfies counterfactual stability. 
%
This implies that the win-rate achieved by $m$ against $m'$ is
%
    \begin{equation}\label{eq:prop_greater_shared}
            \EE_{\Ub\sim P_{\Ub}} [\one\{R_m(\Ub,s_q)>R_{m'}(\Ub,s_q)\}] =
            \PP[f_T(f_D(s_q, m),\Ub)= t_+, f_T(f_D(s_q, m'),\Ub)= t_-]=0 
    \end{equation}
and that
    \begin{equation}\label{eq:prop_equal_probs+}
            \PP[f_T(f_D(s_q, m),\Ub)= t_+, f_T(f_D(s_q, m'),\Ub)= t_+]= \PP[f_T(f_D(s_q, m),\Ub)= t_+] =p_m.
    \end{equation}
%
Using the same reasoning, if the LLM $m'$ samples the non-preferred token $t_{-}$, then, $m$ must also sample $t_{-}$ because $t_{-}$ is more likely under $m$ than under $m'$. This implies that
%
    \begin{equation}\label{eq:prop_equal_probs-}
            \PP[f_T(f_D(s_q, m),\Ub)= t_-, f_T(f_D(s_q, m'),\Ub)= t_-]= \PP[f_T(f_D(s_q, m'),\Ub)= t_-] = 1-p_{m'}
    \end{equation} 
%
Then, from Eq.~\ref{eq:prop_equal_probs+} and Eq.~\ref{eq:prop_equal_probs-}, we can conclude that
%
    \begin{equation}\label{eq:prop_equal_shared}
        \begin{aligned}
            \EE_{\Ub\sim P_{\Ub}} [\one\{R_m(\Ub,s_q)=R_{m'}(\Ub,s_q)\}] = p_m + (1-p_{m'})
        \end{aligned}
    \end{equation}
%
Finally, from Eq.~\ref{eq:prop_greater_shared} and Eq.~\ref{eq:prop_equal_shared}, we can conclude that the win-rate achieved by $m'$ against $m$ is
%
\begin{multline*}
    \EE_{\Ub\sim P_{\Ub}} [\one\{R_m(\Ub,s_q)<R_{m'}(\Ub,s_q)\}] \\
        = 1 - \EE_{\Ub\sim P_{\Ub}} [\one\{R_m(\Ub,s_q)>R_{m'}(\Ub,s_q)\}] -\EE_{\Ub\sim P_{\Ub}} [\one\{R_m(\Ub,s_q)=R_{m'}(\Ub,s_q)\}] = p_{m'}-p_m.
\end{multline*}

% Next, we compute the win-rates under independent autoregressive generation.
%
Under independent autoregressive generation, the LLMs $m$ and $m'$ sample tokens independently from each other, \ie, $f_T(f_D(s_q, m),\Ub) \perp f_T(f_D(s_q, m'),\Ub')$. 
%
Thus, we can factorize all joint probabilities when computing the win-rates and obtain
%
\begin{align*}\nonumber
    % \begin{split}
        \EE_{\Ub, \Ub'\sim P_{\Ub}} [\one\{R_m(\Ub,s_q)>R_{m'}(\Ub',s_q)\}] &=\PP[f_T(f_D(s_q, m),\Ub)= t_+] \cdot \PP[f_T(f_D(s_q, m'),\Ub')= t_-] \\ &=
        p_m \cdot (1-p_{m'})
    % \end{split}
\end{align*}
and
\begin{equation}\nonumber
    \begin{split}
        \EE_{\Ub, \Ub'\sim P_{\Ub}} & [\one\{R_m(\Ub,s_q)<R_{m'}(\Ub',s_q)\}] 
         = p_{m'} \cdot (1-p_m).
        % we do not write it the = rates in the proposition 
        %\\
        % \text{and} \quad \EE_{\Ub\sim P_{\Ub}, \Ub'\sim P_{\Ub}, S_q\sim P_{Q}} & [\one\{R_m(\Ub,S_q)=R_{m'}(\Ub',S_q)\}] 
        %  = p_m(p_m+\delta) + (1-p_m-\delta)(1-p_m)
    \end{split}
\end{equation}


\subsection{Proof of Proposition~\ref{prop:gumbel_ties}}

We follow the notations and technique of Proposition~\ref{prop:var_gumbel}.
Fix query $s_q$ and consider first the case of independent autoregressive generation. Since each LLM can only assign a non-zero probability to single-token sequences, we have:
\begin{equation*}
    \begin{split}
        \PP[R_m(\Ub, s_q)=R_{m'}(\Ub', s_q)]&=\sum_{k=1}^{|V|}\PP[f_T(f_D(s_q,m),\Ub)=t_k] \PP[f_T(f_D(s_q,m'),\Ub)=t_k]\\
        &<\sum_{k=1}^{|V|}\PP[f_T(f_D(s_q,m),\Ub)=t_k],
    \end{split}
\end{equation*}
%
In the case of coupled autoregressive generation, since
\begin{equation*}
    \PP\left[\{f_T(f_D(s_q,m),\Ub)=t_k\} \cap \{f_T(f_D(s_q,m),\Ub)=t_j \}\right]=0,\; i\neq j,
\end{equation*}
%
we obtain:
\begin{equation*}
    \begin{split}
        &\PP[R_m(\Ub, s_q)=R_{m'}(\Ub, s_q)]\\
        &=\PP\left[\cup_i \{ f_T(f_D(s_q,m),\Ub)=t_k, f_T(f_D(s_q,m'),\Ub)=t_k \} \right]\\
        &=\sum_k\PP[\{f_T(f_D(s_q,m),\Ub)=t_k, f_T(f_D(s_q,m'),\Ub)=t_k \}]\\
        &=\sum_k \PP[f_T(f_D(s_q,m),\Ub)=t_k]\PP[f_T(f_D(s_q,m'),\Ub)=t_k|f_T(f_D(s_q,m),\Ub)=t_k].\\
    \end{split}
\end{equation*}
We now follow \cite{9729603} and expand the posterior Gumbels, $\PP[f_T(f_D(s_q,m'),\Ub)=t_k|f_T(f_D(s_q,m),\Ub)=t_k]$, as truncated Gumbel distributions. In particular, we leverage the fact that
\begin{equation}\label{eq:max of gumbels}
    \max_{t\in V}\{U_t+\log([f_D(s_q,\bullet)]_{t})\} \sim \text{Gumbel}(0,1),
\end{equation}
and that a Gumbel distribution, with parameter $\log (\theta)$, truncated at $b\sim \text{Gumbel}(0,1)$  can be sampled as
\begin{equation}\label{eq: truncated gumbel}
    -\log(\exp(-b)-\log(\eta)/\theta),\; \eta \sim U(0,1).
\end{equation}
Furthermore, by assumption, $D_{t_k} > 0$ for all $k \in \{1, \ldots, |V|\}$, so that any bound on the absolute difference of log-probabilities $|\log ( [f_D(s_q,m)]_{t_k}) -\log ([f_D(s_q,m')]_{t_k}) |$ uniformly implies a bound on the difference of probabilities $|[f_D(s_q,m)]_{t_k} -[f_D(s_q,m')]_{t_k} |$ and vice versa. Using the bound 
\begin{equation*}
    |\log([f_D(s_q,m)]_{t_k})-\log([f_D(s_q,m')]_{t_k})|\leq \varepsilon^*
\end{equation*}
and the Gumbel properties in Eq.~\ref{eq:max of gumbels} and Eq.~\ref{eq: truncated gumbel}, we obtain:
%
\begin{align}
        &\PP[R_m(\Ub, s_q)=R_{m'}(\Ub, s_q)] \nn \\
        &=\sum_k \PP[f_T(f_D(s_q,m),\Ub)=t_k] \nn \\
            & \qquad \quad \times \PP\Bigg[
            \bigcap_k \Big\{
            \log( [f_D(s_1, m')]_{t_k}) - \log([f_D(s_1, m)]_{t_k})-\log(-\log (\eta_k)) \nn \\
            & \qquad \qquad \qquad \qquad \geq \log([f_D(s_1, m')]_{t_j}) - \log([f_D(s_1, m)]_{t_j})-\log(-\log(\eta_k) -\log(\eta_j)/[f_D(s_1,m')]_{t_j})
            \Big\}
            \Bigg] \nn \\
            &\geq
            \sum_k \PP[f_T(f_D(s_q,m),\Ub)=t_k] \nn \\
            &\qquad \quad \times\PP\left[
            \cap_k \{
            -\log(-\log (\eta_k)) \geq -2\varepsilon^* -\log(-\log(\eta_k) -\log(\eta_j)/[f_D(s_1,m')]_{t_j})
            \}
            \right] \label{eq:final-bound}
\end{align}
where $\eta_k\sim \text{U}(0,1)$ are independently distributed uniform random variables.
%
Now, note that the claim holds for $\varepsilon^*=0$ since, in that case, we have that
%
\begin{equation*}
   \PP\left[
        \bigcap_k \left\{
        -\log(-\log (\eta_k)) \geq -\log(-\log(\eta_k) -\log(\eta_k)/[f_D(s_1,m')]_{t_k})
        \right\} \right]=1,
\end{equation*}
%
using that $x\mapsto -\log(x)$ is strictly decreasing. 
%
Since all terms in Eq.~\ref{eq:final-bound} are continuous in $\varepsilon^*$, there exists $\varepsilon^*(m) >0 $, possibly dependent on $m$ but independent of $m'$, such that 
\begin{equation}\label{eq:prop5 claim prob}
    \PP[R_m(\Ub, s_q)=R_{m'}(\Ub, s_q)] > \PP[R_m(\Ub, s_q)=R_{m'}(\Ub', s_q)]
\end{equation}
holds if
\begin{equation*}
    \sup_{s_q} \norm{\log(f_D(s_q,m))-\log(f_D(s_q, m'))}_\infty < \varepsilon^*(m).
\end{equation*}
Since by assumption $D_t>0$ for all $t\in V$, there exists $\varepsilon(m)>0$ in probability space such that Eq.~\ref{eq:prop5 claim prob} holds if

\begin{equation*}
    \sup_{s_q} \norm{f_D(s_q,m)-f_D(s_q, m')}_\infty < \varepsilon(m).
\end{equation*}
%
This concludes the proof.






\subsection{Calculation of average win-rates in the example used in Sections~\ref{sec:intro} and~\ref{sec:pairwise}} \label{app:example-ranking}
%
In this section, we provide detailed calculations of the win-rates for the example in Sections~\ref{sec:intro} and~\ref{sec:pairwise}. Recall that in this example, we are given three LLMs $m_1$, $m_2$ and $m_3$, and we need to rank them according to their ability to answer correctly two types of input prompts, $q$ and $q'$, picked uniformly at random.
%
We assume that the true probability that each LLM answers correctly each type of input prompt is given by:
%
\begin{table}[H]
    \centering
    \begin{tabular}{lccc}
        \toprule
         & $m_1$ & $m_2$ & $m_3$ \\
        \midrule
        $q$           & $p_1=0.4$             & $p_2=0.48$           & $p_3=0.5$             \\
        $q'$           & $p'_1=1$           & $p'_2=0.9$         & $p'_3=0.89$              \\
        \bottomrule
 \end{tabular}
 \vspace{-8mm}
 \caption*{}
 \end{table}

Using Proposition~\ref{prop:gap_win_rates_stability}, 
% under the assumption of counterfactual stability, 
the win-rates under independent autoregressive generation are given, for each LLM $m_k$, by:
\begin{equation}
    \begin{split}
       \frac{1}{2}\sum_{j\neq k}\EE_{\Ub, \Ub' \sim P_{\Ub}, S_q\sim P_Q}[\one\{R_{m_k}(\Ub, S_q)>R_{m_j}(\Ub', S_q)\}] & =\frac{\sum_{j\neq k}p_k(1-p_j) + \sum_{j\neq k}p'_k(1-p'_j)}{4}.
    \end{split}
\end{equation}
%
Substituting the numerical values we obtain:

 \begin{equation}
    \begin{split}
       \frac{1}{2}\sum_{j\neq 1}\EE_{\Ub, \Ub' \sim P_{\Ub}, S_q\sim P_Q}[\one\{R_{m_1}(\Ub, S_q)>R_{m_j}(\Ub', S_q)\}] & =0.1545,\\
       \frac{1}{2}\sum_{j\neq 2}\EE_{\Ub, \Ub' \sim P_{\Ub}, S_q\sim P_Q}[\one\{R_{m_2}(\Ub, S_q)>R_{m_j}(\Ub', S_q)\}] & =0.15675,\\
       \frac{1}{2}\sum_{j\neq 3}\EE_{\Ub, \Ub' \sim P_{\Ub}, S_q\sim P_Q}[\one\{R_{m_3}(\Ub, S_q)>R_{m_j}(\Ub', S_q)\}] & =0.16225\\
    \end{split}
\end{equation}

Similarly, using Proposition~\ref{prop:gap_win_rates_stability}, the win-rates using coupled autoregressive generation can be written, for each LLM $m_k$, as:
\begin{equation}
    \begin{split}
       \frac{1}{2}\sum_{j\neq k}\EE_{\Ub \sim P_{\Ub}, S_q\sim P_Q}[\one\{R_{m_k}(\Ub, S_q)>R_{m_j}(\Ub, S_q)\}] & =\frac{\sum_{j\neq k}(p_k-p_j)_{+} + \sum_{j\neq k}(p'_k-p'_j)_{+}}{4},
    \end{split}
\end{equation}
where $(\bullet)_{+}=\max(0, \bullet)$ denotes the positive part. Substituting the numerical values we obtain:

 \begin{align*}
       \frac{1}{2}\sum_{j\neq 1}\EE_{\Ub \sim P_{\Ub}, S_q\sim P_Q}[\one\{R_{m_1}(\Ub, S_q)>R_{m_j}(\Ub, S_q)\}] & = 
       % \frac{(p_1-p_2)_{+} +(p_1-p_3)_{+} +(p'_1-p'_2)_{+} +(p'_1-p'_3)_{+}}{4}
       0.0525, \\
       \frac{1}{2}\sum_{j\neq 2}\EE_{\Ub \sim P_{\Ub}, S_q\sim P_Q}[\one\{R_{m_2}(\Ub, S_q)>R_{m_j}(\Ub, S_q)\}] &=
       % \frac{(p_2-p_1)_{+} +(p_2-p_3)_{+} +(p'_2-p'_1)_{+} +(p'_2-p'_3)_{+}}{4}=
       0.0225, \\
       \frac{1}{2}\sum_{j\neq 3}\EE_{\Ub \sim P_{\Ub}, S_q\sim P_Q}[\one\{R_{m_3}(\Ub, S_q)>R_{m_j}(\Ub, S_q)\}] &=
       % \frac{(p_3-p_2)_{+} +(p_3-p_1)_{+} +(p'_3-p'_2)_{+} +(p'_3-p'_1)_{+}}{4}=
       0.03.
\end{align*}






\newpage

\section{Additional Experimental Details} \label{app:add_exp}
\vspace{-2mm}

\xhdr{Hardware setup} Our experiments are executed on a compute server equipped with 2 $\times$ Intel Xeon Gold 5317 CPU, $1{,}024$ GB main memory, and $2$ $\times$ A100 Nvidia Tesla GPU ($80$ GB, Ampere Architecture). In each experiment a single Nvidia A100 GPU is used.


\xhdr{Datasets} 
As a benchmark dataset, we use Measuring Massive Multitask Language Understanding dataset (MMLU)~\cite{hendrycks2021measuring} consisting of $14{,}042$ questions covering $52$ diverse knowledge areas with each question offering four possible choices indexed from A to D, and a ground-truth answer. For pairwise comparison tasks, we use the first $500$ questions from the LMSYS-Chat-1M dataset~\cite{zheng2024lmsys}.

\xhdr{Models} In our experiments, we 
% evaluate popular large language models from the \texttt{Llama} family sharing 
% the same token vocabulary.  
%Specifically, we 
use \texttt{Llama-3.1-8B-Instruct}, its quantized variants \texttt{Llama-3.1-8B-Instruct-\{AWQ-INT4, bnb-4bit, bnb-8bit\}} and \texttt{Llama-3.2-\{1B, 3B\}-Instruct} models. The models are obtained from \texttt{Hugging Face}, and the quantised LLM variants \texttt{Llama-3.1-8B-Instruct-\{bnb-4bit, bnb-8bit\}} are built using the \texttt{bitsandbytes} library~\cite{bitsnbytes2024bits}.

\xhdr{Prompts} To instruct LLMs for generating output, we use the system prompt in Table~\ref{app:sys_prompt_2} for the MMLU dataset and Table~\ref{app:sys_prompt_3} for the LMSYS-Chat-1M dataset. Further, to perform pairwise comparisons of outputs of different LLMs, we use the system prompt in Table~\ref{app:sys_prompt_1}, which is adapted from~\cite{chiang2024chatbot}, to prompt the strong LLM.

\begin{table}[h]
\centering
\begin{tcolorbox}[
    colframe=white,      % Border color
    colback=gray!14,     % Background color
    boxrule=0.5mm,       % Border thickness
    arc=4mm,             % Rounded corners
    left=3mm,            % Left margin
    right=3mm,           % Right margin
    top=3mm,             % Top margin
    bottom=3mm           % Bottom margin
]
\begin{tabular}{ m{15.2cm} }
\rowcolor{gray!14} 
    \textbf{System:} Please act as an impartial judge and evaluate the quality of the responses provided by two AI assistants to the user prompt displayed below. Your job is to evaluate which assistant's answer is better. When evaluating the assistants' answers, compare both assistants' answers. You must identify and correct any mistakes or inaccurate information. Then consider if the assistant's answers are helpful, relevant, and concise. Helpful means the answer correctly responds to the prompt or follows the instructions. Note when user prompt has any ambiguity or more than one interpretation, it is more helpful and appropriate to ask for clarifications or more information from the user than providing an answer based on assumptions. Relevant means all parts of the response closely connect or are appropriate to what is being asked. Concise means the response is clear and not verbose or excessive. Then consider the creativity and novelty of the assistant's answers when needed. Finally, identify any missing important information in the assistants' answers that would be beneficial to include when responding to the user prompt. do not provide any justification or explanation for your response. You must output only one of the following choices as your final verdict: 

    \vspace{2mm}
    `A' if the response of assistant A is better
    
    \vspace{1mm}
    `B' if the response of assistant B is better
    
    \vspace{1mm}
    `Tie' if the responses are tied
\end{tabular}
\end{tcolorbox}
\vspace{-3mm}
\caption{\label{app:system_prompt}System prompt used for obtaining pairwise preferences using \texttt{GPT-4o-2024-11-20} as the judge.}
\label{app:sys_prompt_1}
\end{table}


\vspace{-3mm}

\begin{table}[h]
\centering
\begin{tcolorbox}[
    colframe=white,      % Border color
    colback=gray!14,     % Background color
    boxrule=0.5mm,       % Border thickness
    arc=4mm,             % Rounded corners
    left=3mm,            % Left margin
    right=3mm,           % Right margin
    top=3mm,             % Top margin
    bottom=3mm           % Bottom margin
]
\begin{tabular}{ m{15.2cm} }
    \rowcolor{gray!14}
    \textbf{System:} You will be given multiple choice questions. Please reply with a single character `A', `B', `C', or `D' only. DO NOT explain your reply.
\end{tabular}
\end{tcolorbox}
\vspace{-3mm}
\caption{System prompt used for the MMLU dataset.}
\label{app:sys_prompt_2}
\end{table}


\vspace{-3mm}

\begin{table}[h]
\centering
\begin{tcolorbox}[
    colframe=white,      % Border color
    colback=gray!14,     % Background color
    boxrule=0.5mm,       % Border thickness
    arc=4mm,             % Rounded corners
    left=3mm,            % Left margin
    right=3mm,           % Right margin
    top=3mm,             % Top margin
    bottom=3mm           % Bottom margin
]
\begin{tabular}{ m{15.2cm} }
    \rowcolor{gray!14}
    \textbf{System:} Keep your responses short and to the point.
\end{tabular}
\end{tcolorbox}
\vspace{-3mm}
\caption{System prompt used for the LMSYS Chatbot Arena dataset.}
\label{app:sys_prompt_3}
\end{table}






\clearpage
\newpage

% \section{Additional Experimental Results} \label{app:exp_results}
%In this section, we present experimental results supplementing those from Section~\ref{sec:experiments}, first for the MMLU dataset, and then the LMSYS-Chat-1M dataset.

\section{Additional Experimental Results on the MMLU Dataset} \label{app:mmlu}
% Here, we present experimental results supplementing the results from 
% Section~\ref{sec:experiments}, comparing different pairs of models in 
% Figure~\ref{fig:mmlu-1B-vs-3B-models} and different knowledge areas in 
% Figure~\ref{fig:mmlu-1B-vs-3B-areas}. 

\begin{figure}[h]
\centering
\begin{tabular}{c c c}
    \multicolumn{3}{c}{\texttt{Llama-3.2-1B-Instruct} vs. \texttt{Llama-3.2-3B-Instruct}}\\
    \includegraphics[width=0.27\linewidth]{./figures/mmlu/covariance/college_computer_science/cov_Llama-3.2-3B-Instruct_Llama-3.2-1B-Instruct_kde.pdf} &
    \includegraphics[width=0.27\linewidth]{./figures/mmlu/variance/college_computer_science/var_Llama-3.2-3B-Instruct_Llama-3.2-1B-Instruct_kde.pdf} &
    \includegraphics[width=0.27\linewidth]{./figures/mmlu/error/college_computer_science/error_Llama-3.2-3B-Instruct_Llama-3.2-1B-Instruct.pdf} \\ \\
%
    \multicolumn{3}{c}{\texttt{Llama-3.2-1B-Instruct} vs. \texttt{Llama-3.1-8B-Instruct}}\\
    \includegraphics[width=0.27\linewidth]{./figures/mmlu/covariance/college_computer_science/cov_Llama-3.2-1B-Instruct_Llama-3.1-8B-Instruct_kde.pdf} &
    \includegraphics[width=0.27\linewidth]{./figures/mmlu/variance/college_computer_science/var_Llama-3.1-8B-Instruct_Llama-3.2-1B-Instruct_kde.pdf} &
    \includegraphics[width=0.27\linewidth]{./figures/mmlu/error/college_computer_science/error_Llama-3.1-8B-Instruct_Llama-3.2-1B-Instruct.pdf} \\ \\
%
    \multicolumn{3}{c}{\texttt{Llama-3.2-3B-Instruct} vs. \texttt{Llama-3.1-8B-Instruct}}\\
    \includegraphics[width=0.27\linewidth]{./figures/mmlu/covariance/college_computer_science/cov_Llama-3.1-8B-Instruct_Llama-3.2-3B-Instruct_kde.pdf} &
    \includegraphics[width=0.27\linewidth]{./figures/mmlu/variance/college_computer_science/var_Llama-3.1-8B-Instruct_Llama-3.2-3B-Instruct_kde.pdf} &
    \includegraphics[width=0.27\linewidth]{./figures/mmlu/error/college_computer_science/error_Llama-3.1-8B-Instruct_Llama-3.2-3B-Instruct.pdf} \\ \\
%
    (a) Score covariance & (b) Variance of the score difference & (c) Estimation error vs. \# samples \\ 
    
\end{tabular}
    \caption{\textbf{Comparison between three pairs of LLMs on multiple-choice questions from the ``college computer science'' knowledge area of the MMLU dataset.}
    Panels in column (a) show the kernel density estimate (KDE) of the covariance between the scores of the two LLMs on each question under coupled generation; the dashed lines correspond to average values. Panels in column (b) show the KDE of the variance of the difference between the scores of the LLMs on each question under coupled and independent generation; the highlighted points correspond to median values. Panels in column (c) show the absolute error in the estimation of the expected difference between the scores of the LLMs against the number of samples; for each point on the x-axis, we perform $1{,}000$ sub-samplings and shaded areas correspond to $95\%$ confidence intervals.}
    \label{fig:mmlu-1B-vs-3B-models}
\end{figure}


\begin{figure}[h]
\centering
\begin{tabular}{c c c}
%
    \multicolumn{3}{c}{\textbf{College chemistry}} \\
    \includegraphics[width=0.27\linewidth]{./figures/mmlu/covariance/college_chemistry/cov_Llama-3.2-3B-Instruct_Llama-3.2-1B-Instruct_kde.pdf} &
    \includegraphics[width=0.27\linewidth]{./figures/mmlu/variance/college_chemistry/var_Llama-3.2-3B-Instruct_Llama-3.2-1B-Instruct_kde.pdf} &
    \includegraphics[width=0.27\linewidth]{./figures/mmlu/error/college_chemistry/error_Llama-3.2-3B-Instruct_Llama-3.2-1B-Instruct.pdf} \\
%
    \multicolumn{3}{c}{\textbf{Professional accounting}} \\
    \includegraphics[width=0.27\linewidth]{./figures/mmlu/covariance/professional_accounting/cov_Llama-3.2-3B-Instruct_Llama-3.2-1B-Instruct_kde.pdf} &
    \includegraphics[width=0.27\linewidth]{./figures/mmlu/variance/professional_accounting/var_Llama-3.2-3B-Instruct_Llama-3.2-1B-Instruct_kde.pdf} &
    \includegraphics[width=0.27\linewidth]{./figures/mmlu/error/professional_accounting/error_Llama-3.2-3B-Instruct_Llama-3.2-1B-Instruct.pdf} \\
% 
    \multicolumn{3}{c}{\textbf{Professional law}}\\
    \includegraphics[width=0.27\linewidth]{./figures/mmlu/covariance/professional_law/cov_Llama-3.2-3B-Instruct_Llama-3.2-1B-Instruct_kde.pdf} &
    \includegraphics[width=0.27\linewidth]{./figures/mmlu/variance/professional_law/var_Llama-3.2-3B-Instruct_Llama-3.2-1B-Instruct_kde.pdf} &
    \includegraphics[width=0.27\linewidth]{./figures/mmlu/error/professional_law/error_Llama-3.2-3B-Instruct_Llama-3.2-1B-Instruct.pdf} \\ \\
%
    \multicolumn{3}{c}{\textbf{Professional medicine}} \\   
    \includegraphics[width=0.27\linewidth]{./figures/mmlu/covariance/professional_medicine/cov_Llama-3.2-3B-Instruct_Llama-3.2-1B-Instruct_kde.pdf} &
    \includegraphics[width=0.27\linewidth]{./figures/mmlu/variance/professional_medicine/var_Llama-3.2-3B-Instruct_Llama-3.2-1B-Instruct_kde.pdf} &
    \includegraphics[width=0.27\linewidth]{./figures/mmlu/error/professional_medicine/error_Llama-3.2-3B-Instruct_Llama-3.2-1B-Instruct.pdf} \\
%
    (a) Score covariance & (b) Variance of the score difference & (c) Estimation error vs. \# samples \\ 
\end{tabular}
    \caption{\textbf{Comparison between \texttt{Llama-3.2-1B-Instruct} and \texttt{Llama-3.2-3B-Instruct} on multiple-choice questions from four knowledge areas of the MMLU dataset.}
    Panels in column (a) show the kernel density estimate (KDE) of the covariance between the scores of the two LLMs on each question under coupled generation; the dashed lines correspond to average values. Panels in column (b) show the KDE of the variance of the difference between the scores of the LLMs on each question under coupled and independent generation; the highlighted points correspond to median values. Panels in column (c) show the absolute error in the estimation of the expected difference between the scores of the LLMs against the number of samples; for each point on the x-axis, we perform $1{,}000$ sub-samplings and shaded areas correspond to $95\%$ confidence intervals. We observe qualitatively similar results for other knowledge areas.}
    \label{fig:mmlu-1B-vs-3B-areas}
\end{figure}

\clearpage
\newpage

\section{Additional Experimental Results on the LMSYS-Chat-1M Dataset} \label{app:lmsys}
% Here, we report the empirical win-rate of each LLM against other LLMs using 
% coupled and independent generation in Figure~\ref{fig:lmsys-all-llms}.
%
\begin{figure}[h]
\centering
\includegraphics[width=0.72\linewidth]{figures/lmsys/all.pdf}
%
\caption{
\textbf{Empirical win-rate of each LLM against other LLMs 
on questions from the LMSYS-Chat-1M dataset.} 
%
Empirical estimate of the win-rate under coupled autoregressive generation as given by Eq.~\ref{eq:coupled-generation-win-rates} and under independent generation generation as given by  Eq.~\ref{eq:independent-generation-win-rates}. 
%
Each empirical win-rate is computed using pairwise comparisons between the outputs of each LLM and any other LLM over $500$ questions with $10$ (different) random seeds.
%
The error bars correspond to $95\%$ confidence intervals.
%
For each pair of empirical win-rates, we conduct a two-tailed test, to test the hypothesis that the empirical win-rates are the same; (\fourstars, \threestars, \twostars, \onestar) indicate $p$-values ($<0.0001$, $<0.001$, $<0.01$, $<0.05$), respectively.
}
\label{fig:lmsys-all-llms}
\end{figure}


\end{document}
