
\subsubsection{Question Generation Prompts}
\label{subsubsec:question-generation-prompts}

\autoref{fig:baseline_question_generation_prompt} presents the \Baseline question generation prompt. \BaselineMT builds upon this by generating additional unique questions, as shown in \autoref{fig:baseline_large_question_generation_prompt}. Similarly, \autoref{fig:question_generation} introduces the \name question generation prompt, used in step \circled{5}{customorange} of \autoref{fig:savaal-workflow}, which closely resembles the \Baseline prompt. Beyond question generation, \autoref{fig:map_prompt} depicts the map prompt from step \circled{1}{customblue}, while \autoref{fig:combine_prompt} and \autoref{fig:reduce_prompt} (step \circled{2}{customred}) extend this by consolidating multiple concept maps into a comprehensive summary. Finally, \autoref{fig:main_idea_ranking_prompt} illustrates the ranking prompt used in step \circled{3}{custompurple} of \autoref{fig:savaal-workflow}.

%%%%%%%%%%%%%%%%%%%%%% Question Generation Prompt %%%%%%%%%%%%%%%%%%%%%%

\begin{figure}[h]
\centering
\begin{tcolorbox}[width=1\columnwidth, fontupper=\footnotesize,title=\Baseline Question Generation Prompt]
\RaggedRight
\obeylines
\textbf{Instructions}: \\
{
Based on the following context, create $\{\text{num\_questions}\}$ multiple-choice questions that require deep understanding, critical thinking, and detailed analysis.  
The questions should go beyond mere factual recall, involving higher-order thinking skills like analysis, synthesis, and evaluation.

\textbf{Provide four answer choices for each question:}
\begin{itemize}[label=-,nosep]
    \item The choices should start with \textbf{A.}, \textbf{B.}, \textbf{C.}, and \textbf{D.}
    \item \textbf{One correct answer.}
    \item \textbf{Three plausible distractors} that are:
    \begin{itemize}[label=$\bullet$,nosep]
        \item Contextually appropriate.
        \item Relevant to the content.
        \item Reflect common misunderstandings or errors without introducing contradictory or irrelevant information.
    \end{itemize}
\end{itemize}
\vspace{3mm}
\textbf{Note:} The questions should focus on one concept and not be overly long.  
\textbf{DO NOT} ask multiple questions in one.
}
\vspace{3mm}
\vspace{3mm}
\textbf{Context}: \\
$\{\text{context}\}$ \\

\end{tcolorbox}
\caption{\Baseline Question Generation Prompt.}
\label{fig:baseline_question_generation_prompt}
\end{figure}

%%%%%%%%%%%%%%%%%%%%%% Question Generation %%%%%%%%%%%%%%%%%%%%%%

 \begin{figure}[h]
\centering
\begin{tcolorbox}[width=1\columnwidth, fontupper=\footnotesize,title= Savaal Question Generation Prompt]
\RaggedRight
\obeylines
\textbf{Instructions}: \\
{Based on the following main idea and its relevant passages, create $\{\text{num\_questions}\}$ \\ multiple-choice questions that require deep understanding, critical thinking, and detailed analysis. The questions should go beyond mere factual recall, involving higher-order thinking skills like analysis, synthesis, and evaluation. \\
Do not use the phrases "main idea" or "passages" in the question statement. Instead, directly address the content or concepts described. \\
Provide four answer choices for each question: \\
\begin{itemize}[label=-,nosep]
    \item The choices should start with A., B., C., and D.
    \item One correct answer.
    \item \textbf{Three plausible distractors} that are contextually appropriate, relevant to the content, and reflect common misunderstandings or errors without introducing contradictory or irrelevant information.
\end{itemize}

\vspace{3mm}
\textbf{Note}: The questions should be focused on one concept and not very long, DO NOT ask multiple questions in one.} \\
\vspace{3mm}
\vspace{3mm}
\textbf{Main Idea}: \\
$\{\text{main\_idea}\}$ \\
\vspace{3mm}
\textbf{Passages}: \\
$\{\text{passages}\}$ \\


\end{tcolorbox}
\centering
\caption{The question generation prompt in \autoref{fig:savaal-workflow}.}
\label{fig:question_generation}
\end{figure}

% \vspace{-20 pt}
%%%%%%%%%%%%%%%%%%%%%% MAP %%%%%%%%%%%%%%%%%%%%%%

\begin{figure}[h]
\centering
\begin{tcolorbox}[width=1\columnwidth, fontupper=\footnotesize,title=Map Prompt]
\RaggedRight
\obeylines
\textbf{Instructions}: \\
{You are an expert educator specializing in creating detailed concept maps from academic texts. Given the following excerpt from a longer document, extract the main ideas, detailed concepts, and supporting details that are critical to understanding the material. \vspace{3mm}
Focus on identifying:
\begin{itemize}[label=-,nosep]
    \item Key concepts or terms introduced in the text.
    \item Definitions or explanations of these concepts.
    \item Relationships between concepts.
    \item Any examples or applications mentioned.
\end{itemize}
\vspace{3mm}
\vspace{3mm}
Use clear, bullet-point summaries, organized by topic. Here is the excerpt:}
\vspace{3mm}
\vspace{3mm}
\textbf{Context}: \\
$\{\text{context}\}$ \\

Respond with a structured list of detailed main ideas and concepts.
\end{tcolorbox}
\caption{\centering The map prompt in \autoref{fig:savaal-workflow}.}
\label{fig:map_prompt}
\end{figure}

%%%%%%%%%%%%%%%%%%%%%% Combine %%%%%%%%%%%%%%%%%%%%%%
\begin{figure}[h]
\centering
\begin{tcolorbox}[width=1\columnwidth, fontupper=\footnotesize,title=Combine Prompt]
\RaggedRight
\obeylines
\textbf{Instructions}: \\
{You are combining multiple concept maps into a single, comprehensive summary while retaining all key ideas and details. Below are several lists of main ideas and concepts extracted from a larger document.%\\ \\
\vspace{3mm}
Your task is to:
\begin{enumerate}[nosep]
    \item Merge these lists into a single structured list, removing redundancies while keeping all unique and detailed information.
    \item Ensure all main ideas, relationships, and examples are preserved and clearly organized.
\end{enumerate}
\vspace{3mm}
\vspace{3mm}
Here are the concept maps to combine:} \\
\vspace{3mm}
\vspace{3mm}
\textbf{Context}: \\
$\{\text{context}\}$ \\

Respond with the consolidated and organized list of main ideas and concepts.
\end{tcolorbox}
\caption{The combine prompt in \autoref{fig:savaal-workflow}.}
\label{fig:combine_prompt}
\end{figure}

%%%%%%%%%%%%%%%%%%%%%% Reduce %%%%%%%%%%%%%%%%%%%%%%

\begin{figure}[h]
\centering
\begin{tcolorbox}[width=1\columnwidth, fontupper=\footnotesize,title=Reduce Prompt]
\RaggedRight
\obeylines
\textbf{Instructions}: \\
{You are reducing sets of detailed concept maps, a concise yet comprehensive list of important concepts, generated by extracting concepts from a document and potentially combining subsets of them that are relevant to each other. \\
The goal is to create a structured resource that fully captures the essence of the material for testing and teaching purposes.
\vspace{3mm}
Your task is to:
\begin{itemize}[label=-,nosep]
    \item Identify the most critical concepts from the detailed concept map.
    \item Provide a full-sentence summary for each concept that explains its significance, its relationship to other concepts, and any relevant examples or applications.
    \item Ensure that the summaries are clear, self-contained, and detailed enough to aid in understanding without requiring additional context.
    \item If necessary, combine related concepts into a single summary. Some of the concept maps have broader headings that can be used to guide this process.
\end{itemize}
\vspace{3mm}
\vspace{3mm}
Here is the detailed concept map:} \\
\vspace{3mm}
\vspace{3mm}
\textbf{Context}: \\
$\{\text{context}\}$ \\

Respond with a structured list where each important concept is followed by its full-sentence, detailed summary. For example:
\begin{enumerate}[nosep]
    \item Concept Name: [Detailed full-sentence summary explaining the concept, its relevance, and any examples or applications.]
    \item Another Concept: [Detailed full-sentence summary explaining this concept, its connections to other ideas, and its role in understanding the material.]
\end{enumerate}

Continue in this format for all important concepts.
\end{tcolorbox}
\caption{The reduce prompt in \autoref{fig:savaal-workflow}.}
\label{fig:reduce_prompt}
\end{figure}



%%%%%%%%%%%%%%%%%%%%%% Main Idea Rank %%%%%%%%%%%%%%%%%%%%%%
\begin{figure}[h]
\centering
\begin{tcolorbox}[width=1\columnwidth, fontupper=\footnotesize,title=Ranking Main Ideas]
\RaggedRight
\obeylines
\textbf{Instructions}: \\
{Given the following groups of main ideas extracted from a text, rank them in order of importance, with the most important main idea receiving a rank of 1 and lower ranks for less important ideas. \\
Focus on the most important aspects of the text and the main ideas that are critical to understanding the material. While sometimes important, background information or less critical ideas should be ranked lower. \vspace{3mm}
\textbf{When ranking}:
\begin{itemize}[label=-,nosep]
    \item \textbf{Assign a unique number to each main idea, starting from 1}.
    \item \textbf{Ensure that the most important main idea is ranked first}.
    \item \textbf{Rank the main ideas based on their relevance and significance}.
\end{itemize}}
\vspace{3mm}
Example: \\
\tabto{1cm} Input: [Main Idea 1, Main Idea 2, Main Idea 3] \\
\tabto{1cm} Output: [2, 1, 3] \\
\vspace{3mm}
\vspace{3mm}
\textbf{Main Ideas}: \\
$\{\text{main\_ideas}\}$ \\
\end{tcolorbox}
\caption{The main idea ranking prompt.}
\label{fig:main_idea_ranking_prompt}
\end{figure}



\begin{figure}[h]
\centering
\begin{tcolorbox}[width=1\columnwidth, fontupper=\footnotesize,title=\BaselineMT Additional Question Generation Prompt]
\RaggedRight
\obeylines
\textbf{Instructions}: \\
{
Now, please create $\{\text{num\_questions}\}$ \textbf{additional} multiple-choice questions that require deep understanding, critical thinking, and detailed analysis.  
The questions should go beyond mere factual recall, involving higher-order thinking skills like analysis, synthesis, and evaluation.

\textbf{Provide four answer choices for each question:}
\begin{itemize}[label=-,nosep]
    \item The choices should start with \textbf{A.}, \textbf{B.}, \textbf{C.}, and \textbf{D.}
    \item \textbf{One correct answer.}
    \item \textbf{Three plausible distractors} that are:
    \begin{itemize}[label=$\bullet$,nosep]
        \item Contextually appropriate.
        \item Relevant to the content.
        \item Reflect common misunderstandings or errors without introducing contradictory or irrelevant information.
    \end{itemize}
\end{itemize}
\vspace{3mm}
\textbf{Note:} The questions should focus on one concept and not be overly long.  
\textbf{Note:} The questions should be different from the ones generated in the previous step.
}
\vspace{3mm}
\vspace{3mm}
\textbf{Context}: \\
$\{\text{context}\}$ \\

\end{tcolorbox}
\caption{\BaselineMT Additional Question Generation Prompt.}
\label{fig:baseline_large_question_generation_prompt}
\end{figure}


%%%%%%%%%%%%%%%%%%%%%% Question Structure %%%%%%%%%%%%%%%%%%%%%%



\subsubsection{Evaluation Prompts}
\label{subsubsec:eval-prompts}
The AI evaluation framework consists of six metrics designed to assess multiple-choice questions based on different dimensions. The understanding prompt (\autoref{fig:understanding_prompt}) measures the depth of conceptual understanding required to answer the question. The quality of choices prompt (\autoref{fig:choices_quality_prompt}) evaluates the plausibility of the distractors. The clarity evaluation prompt (\autoref{fig:clarity_prompt}) determines the ambiguity level of the question. The difficulty evaluation prompt (\autoref{fig:difficulty_prompt}) categorizes questions based on their complexity and required cognitive effort. The cognitive level evaluation prompt (\autoref{fig:cognitive_level_prompt}) aligns questions with Bloom's taxonomy~\cite{anderson2001taxonomy}, assessing their level from simple recall to higher-order thinking. Finally, the engagement evaluation prompt (\autoref{fig:engagement_prompt}) measures how stimulating and thought-provoking a question is. Each prompt assigns a score from 1 to 4, ensuring a structured and objective analysis of question quality. We map these numerical scores of 4 to 1 to the qualitative scores of ``Agree'', ``Somewhat Agree'', ``Somewhat Disagree'', and ``Disagree'' for comparison with human evaluation.

%%%%%%%%%%%%%%%%%%%%%% Understanding Prompt %%%%%%%%%%%%%%%%%%%%%%

\begin{figure}[h]
\centering
\begin{tcolorbox}[width=1\columnwidth, fontupper=\footnotesize,title=Understanding Evaluation Prompt]
\RaggedRight
\obeylines

{For the following multiple-choice question:\\
-----------\\
Question: $\{\text{question}\}$\\
\vspace{3mm}
Options: $\{\text{options}\}$\\
\vspace{3mm}
Answer: $\{\text{answer}\}$\\
-----------\\
Please answer the following:\\

Please carefully read the multiple-choice question, the options, and the correct answer.
Rate the understanding level of the question on a scale of 1 to 4 based on the following criteria:
\begin{itemize}[label=-,nosep]
\item \textbf{Score 4} if the question tests a deep understanding of a concept, requiring integration and application of ideas.
\item \textbf{Score 3} if the question tests understanding of a concept but is more straightforward, requiring less integration or application.
\item \textbf{Score 2} if the question largely depends on recall but includes some context-specific details that require a conceptual understanding.
\item \textbf{Score 1} if the question primarily tests memorization of facts or details with minimal to no application of concepts.
\end{itemize}


Please output only a score between 1 and 4.
}

\end{tcolorbox}
\caption{Understanding prompt.}
\label{fig:understanding_prompt}
\end{figure}



%%%%%%%%%%%%%%%%%%%%%% Quality of Choices  Prompt %%%%%%%%%%%%%%%%%%%%%%

\begin{figure}[h]
\centering
\begin{tcolorbox}[width=1\columnwidth, fontupper=\footnotesize,title=Quality of Choices  Evaluation Prompt]
\RaggedRight
\obeylines
{For the following multiple-choice question:\\
-----------\\
Question: $\{\text{question}\}$\\
\vspace{3mm}
Options: $\{\text{options}\}$\\
\vspace{3mm}
Answer: $\{\text{answer}\}$\\
-----------\\
Please answer the following:\\
Please carefully read the multiple-choice question, the options, and the correct answer.  
Rate the quality of choices in the question on a scale of 1 to 4 based on the following criteria:
\begin{itemize}[label=-,nosep]
    \item \textbf{Score 4} if it is challenging to eliminate any incorrect choice due to well-crafted distractors that are plausible, unambiguous, and relevant to the question.
    \item \textbf{Score 3} if incorrect choices can be somewhat challenging to eliminate, requiring a good understanding of the material, but they are less sophisticated.
    \item \textbf{Score 2} if most incorrect choices are fairly easy to eliminate, with perhaps one plausible distractor.
    \item \textbf{Score 1} if incorrect choices are very easy to eliminate, often due to being obviously incorrect or irrelevant.
\end{itemize}

Please output only a score between 1 and 4.
}
\end{tcolorbox}
\caption{Quality of Choices  Evaluation Prompt.}
\label{fig:choices_quality_prompt}
\end{figure}


%%%%%%%%%%%%%%%%%%%%%% Clarity Evaluation Prompt %%%%%%%%%%%%%%%%%%%%%%

\begin{figure}[h]
\centering
\begin{tcolorbox}[width=1\columnwidth, fontupper=\footnotesize,title=Clarity Evaluation Prompt]
\RaggedRight
\obeylines
{For the following multiple-choice question:\\
-----------\\
Question: $\{\text{question}\}$\\
\vspace{3mm}
Options: $\{\text{options}\}$\\
\vspace{3mm}
Answer: $\{\text{answer}\}$\\
-----------\\
Please answer the following:\\

Please carefully read the multiple-choice question, the options, and the correct answer.  
Rate the clarity level of the question on a scale of 1 to 4 based on the following criteria:
\begin{itemize}[label=-,nosep]
    \item \textbf{Score 4} if the question is completely clear and unambiguous.
    \item \textbf{Score 3} if the question is mostly clear, but may have some ambiguity.
    \item \textbf{Score 2} if the question has notable ambiguity that could confuse the reader.
    \item \textbf{Score 1} if the question is highly confusing or unclear.
\end{itemize}

Please output only a score between 1 and 4.
}

\end{tcolorbox}
\caption{Clarity Evaluation Prompt.}
\label{fig:clarity_prompt}
\end{figure}


%%%%%%%%%%%%%%%%%%%%%% Difficulty Evaluation Prompt %%%%%%%%%%%%%%%%%%%%%%

\begin{figure}[h]
\centering
\begin{tcolorbox}[width=1\columnwidth, fontupper=\footnotesize,title=Difficulty Evaluation Prompt]
\RaggedRight
\obeylines

{For the following multiple-choice question:\\
-----------\\
Question: $\{\text{question}\}$\\
\vspace{3mm}
Options: $\{\text{options}\}$\\
\vspace{3mm}
Answer: $\{\text{answer}\}$\\
-----------\\
Please answer the following:\\

Please carefully read the multiple-choice question, the options, and the correct answer.  
Rate the difficulty level of the question on a scale of 1 to 4 based on the following criteria:
\begin{itemize}[label=-,nosep]
    \item \textbf{Score 4} if the question is very challenging, requiring deep understanding and advanced conceptual application.
    \item \textbf{Score 3} if the question is moderately difficult, requiring understanding and some conceptual application.
    \item \textbf{Score 2} if the question is relatively easy and mainly requires recall or basic understanding.
    \item \textbf{Score 1} if the question is very easy and can be answered without specific knowledge.
\end{itemize}

Please output only a score between 1 and 4.
}

\end{tcolorbox}
\caption{Difficulty Evaluation Prompt.}
\label{fig:difficulty_prompt}
\end{figure}



%%%%%%%%%%%%%%%%%%%%%% Cognitive Level Evaluation Prompt %%%%%%%%%%%%%%%%%%%%%%

\begin{figure}[h]
\centering
\begin{tcolorbox}[width=1\columnwidth, fontupper=\footnotesize,title=Cognitive Level Evaluation Prompt]
\RaggedRight
\obeylines

{For the following multiple-choice question:\\
-----------\\
Question: $\{\text{question}\}$\\
\vspace{3mm}
Options: $\{\text{options}\}$\\
\vspace{3mm}
Answer: $\{\text{answer}\}$\\
-----------\\
Please answer the following:\\

Please carefully read the multiple-choice question, the options, and the correct answer.  
Rate the cognitive level of the question based on Bloom's taxonomy on a scale of 1 to 4 based on the following criteria:
\begin{itemize}[label=-,nosep]
    \item \textbf{Score 4} if the question requires higher-level thinking (e.g., analysis, synthesis, or evaluation).
    \item \textbf{Score 3} if the question requires application or understanding of concepts.
    \item \textbf{Score 2} if the question requires basic understanding or recall.
    \item \textbf{Score 1} if the question only tests rote memorization with minimal understanding.
\end{itemize}

Please output only a score between 1 and 4.
}

\end{tcolorbox}
\caption{Cognitive Level Evaluation Prompt.}
\label{fig:cognitive_level_prompt}
\end{figure}


%%%%%%%%%%%%%%%%%%%%%% Engagement Evaluation Prompt %%%%%%%%%%%%%%%%%%%%%%

\begin{figure}[h]
\centering
\begin{tcolorbox}[width=1\columnwidth, fontupper=\footnotesize,title=Engagement Evaluation Prompt]
\RaggedRight
\obeylines
{For the following multiple-choice question:\\
-----------\\
Question: $\{\text{question}\}$\\
\vspace{3mm}
Options: $\{\text{options}\}$\\
\vspace{3mm}
Answer: $\{\text{answer}\}$\\
-----------\\
Please answer the following:\\

Please carefully read the multiple-choice question, the options, and the correct answer.  
Rate the engagement level of the question on a scale from 1 to 4 based on the following criteria:
\begin{itemize}[label=-,nosep]
    \item \textbf{Score 4} if the question is highly engaging and thought-provoking.
    \item \textbf{Score 3} if the question is engaging but not particularly unique or thought-provoking.
    \item \textbf{Score 2} if the question is somewhat engaging but fairly straightforward.
    \item \textbf{Score 1} if the question is uninteresting or not engaging.
\end{itemize}

Please output only a score between 1 and 4.
}

\end{tcolorbox}
\caption{Engagement Evaluation Prompt.}
\label{fig:engagement_prompt}
\end{figure}
