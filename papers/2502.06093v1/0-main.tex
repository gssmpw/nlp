%outline of paper
%--------------------outline------------------------%
%https://docs.qq.com/doc/DS1Z1VkRoS1dQYk5X
%------------------------------------------------%

%Todo of NeuralSubFab
%--------------------todo------------------------%
%https://docs.qq.com/sheet/DQkljcWJORWJWYm94?tab=BB08J2
%--------------------todo------------------------%



%\documentclass[acmtog]{formats/acmart} 
%\documentclass[acmtog]{formats/acmart}

% \documentclass[acmtog,anonymous,review]{acmart}
\documentclass[acmtog]{acmart}
 % \acmSubmissionID{512}

\usepackage{booktabs} % For formal tables
\citestyle{acmauthoryear}
\setcitestyle{square}
\usepackage{ifthen}
\usepackage{enumitem}
\usepackage{hyperref}
\usepackage{algorithm}
%\usepackage{algorithmicx}
%\usepackage{algcompatible}
\usepackage[noend]{algpseudocode}
\usepackage{hyperxmp}
%\usepackage{algorithm2e}
\usepackage{syntax}
\usepackage{amsfonts}
\usepackage{listings}
\usepackage{fancyvrb}
\usepackage{wrapfig}
\usepackage{graphicx}
\usepackage{subfigure}
\usepackage{xspace}
\usepackage{colortbl}
\usepackage{gensymb}
\usepackage{verbatim}
\usepackage{colortbl}
\usepackage{booktabs}
\usepackage{multirow}
\usepackage{cleveref}
\usepackage{pifont}
\usepackage{circledsteps}
% \usepackage{colortbl} % 支持表格着色
% \usepackage{xcolor}   % 提供更多颜色选项

\algdef{SE}[DOWHILE]{Do}{doWhile}{\algorithmicdo}[1]{\algorithmicwhile\ #1}%

\definecolor{mygreen}{rgb}{0,0.6,0}                         
\definecolor{mygray}{rgb}{0.95,0.95,0.95}
\definecolor{codebg}{rgb}{0.95, 0.95, 0.95}  
\lstset{                                                                        
  language=Caml,                                                                
  basicstyle=\tiny\ttfamily,                                                   
  frame=single,                                                                 
  numberstyle=\tiny\color{mygray},                                              
  backgroundcolor=\color{codebg},                                              
  numbers=left,                                                                 
  escapeinside={/*}{*/},                                                        
  morekeywords={Setup_Tracksaw, Tracksaw, Setup_Chopsaw, Chopsaw, Setup_Drill, Drill, Return, Box, Make_Stock, Make_Sketch, Query, Support, Constraint, Geometry, Line, PointOnObject, Make_Hole, Make_Cut, Ref},
  tabsize=1,                                                                    
  keywordstyle=\color{blue},                                                    
  numbersep=5pt,                                                                
  rulesep=0pt,                                                                  
  framexleftmargin=2mm                                                          
}  
 
%\lstset{style=mystyle}
\newcommand{\etal}{et al.}

\newcommand{\llu}[1]{{\color{red} LL: #1}}
\newcommand{\hs}[1]{{\color{cyan} HZ: #1}}
\newcommand{\fc}[1]{{\color{brown} FC: #1}}
\newcommand{\lhc}[1]{{\color{olive} HC: #1}}
\newcommand{\yx}[1]{{\color{red} YX: #1}}
\newcommand{\zm}[1]{{\color{orange} #1}}
\newcommand{\jw}[1]{{\color{green}#1}}
\newcommand{\jwc}[1]{{\color{mygreen}[JW: #1]}}

% 将注释隐藏
% \renewcommand{\fc}[1]{}
\let\Bbbk\relax
\usepackage{amsthm,amsmath,amssymb}
\usepackage{mathrsfs}
\usepackage[normalem]{ulem}
% switch comments to eliminate revision tag
\newcommand{\revision}[1]{\textcolor{blue}{#1}}
\newcommand{\revisiontodo}[1]{\textcolor{red}{#1}}
\newcommand{\revisionII}[1]{\textcolor{black}{#1}}
\newcommand{\revisionIII}[1]{\textcolor{black}{#1}}

%\newcommand{\revisiontodo}[2]{#2}

\newcommand{\zdiff}[2]{{\textbf{\color{blue}{#1}} \color{orange}{(suggested to replace ``\textit{#2}'')}}}

\newcommand{\todo}[1]{{\color{blue}{#1}}}
\newcommand{\old}[1]{{\color{red}{#1}}}
\newcommand{\inchsign}{^{\prime\prime}}  
\renewcommand{\grammarlabel}[2]{#1\hfill#2} 
\definecolor{mypink}{rgb}{.99,.91,.95}


\begin{document}

\setcopyright{acmlicensed}
\acmJournal{TOG}
%\acmYear{XXX} \acmVolume{XX} \acmNumber{XX} \acmArticle{XX} \acmMonth{XX} %\acmPrice{15.00}\acmDOI{10.1145/3618324}

\title{DeepMill: Neural Accessibility Learning for Subtractive Manufacturing}

% DeepMill: Accessibility Learning for Subtractive Manufacturing

%“DeepMill: Learning Accessibility for Complex Subtractive Manufacturing Geometries”
%DeepMill: Neural Accessibility Learning for Subtractive Manufacturing”
%DeepMill: Accessibility Learning for Multi-Axis Subtractive Manufacturing”

%“DeepMill: Neural Learning Accessibility for Complex Multi-Axis Subtractive Manufacturing Geometries”


%DeepMill: Neural Accessibility Learning for Subtractive Manufacturing
%DeepMill: Neural Accessibility Learning
%Neural Subtractive Manufacturability Analysis

\author{Fanchao Zhong}
\authornote{Equal contribution}
\email{fanchaoz98@gmail.com}
\affiliation{%
  \institution{Shandong University}
  \city{Qingdao}
  \country{China}
}

\author{Yang Wang}
\authornotemark[1]
\email{1766897491wy@gmail.com}
\affiliation{%
  \institution{Shandong University}
  \city{Qingdao}
  \country{China}
}

\author{Peng-Shuai Wang}
\email{wangps@hotmail.com}
\affiliation{%
  \institution{Peking University}
  \city{Peking}
  \country{China}
}

\author{Lin Lu}
\email{lulin.linda@gmail.com}
\affiliation{%
  \institution{Shandong University}
  \city{Qingdao}
  \country{China}
}

\author{Haisen Zhao}
\authornote{corresponding author}
\email{haisenzhao@sdu.edu.cn}
\affiliation{%
  \institution{Shandong University}
  \city{Qingdao}
  \country{China}
}

%The analysis of accessibility in subtractive manufacturing is crucial as it directly affects manufacturing efficiency and costs. Ignoring or misjudging accessibility can create non-manufacturable areas in workpieces, leading to expensive rework and longer production times. Our study addresses non-manufacturability stemming from tool inaccessibility by introducing a deep-learning network designed to predict non-manufacturable regions in arbitrary models with accuracy and in real-time. It can also detect occluded regions for necessary design changes. Our network is well-suited for diverse tool sizes and intricate geometries. To overcome challenges such as tool collisions and limited data, we employ cutter-aware octree-based neural network. Experiments indicate our network achieves 96.3$\%$ accuracy and 90.0$\%$ F1 score in identifying non-manufacturable and occluded regions while only 0.17$\%$ of the original time is needed for relatively complex geometries. In comparison to traditional techniques, it enhances efficiency, retains accuracy, and adapts well to both freeform and CAD models, with robust tool-size generalization.

%

\begin{abstract}
Manufacturability is vital for product design and production, with accessibility being a key element, especially in subtractive manufacturing. 
Traditional methods for geometric accessibility analysis are time-consuming and struggle with scalability, while existing deep learning approaches in manufacturability analysis often neglect geometric challenges in accessibility and are limited to specific model types.
In this paper, we introduce DeepMill, the first neural framework designed to accurately and efficiently predict inaccessible and occlusion regions under varying machining tool parameters, applicable to both CAD and freeform models.
To address the challenges posed by cutter collisions and the lack of extensive training datasets, we construct a cutter-aware dual-head octree-based convolutional neural network (O-CNN) and generate an inaccessible and occlusion regions analysis dataset with a variety of cutter sizes for network training.
Experiments demonstrate that DeepMill achieves 94.7\% accuracy in predicting inaccessible regions and 88.7\% accuracy in identifying occlusion regions, with an average processing time of 0.04 seconds for complex geometries.
Based on the outcomes, DeepMill implicitly captures both local and global geometric features, as well as the complex interactions between cutters and intricate 3D models.
\end{abstract}


\ccsdesc[500]{Computing methodologies~Shape modeling}
\ccsdesc[300]{Computing methodologies~Graphics systems and interfaces}

\acmJournal{TOG}

\keywords{Subtractive manufacturing, Accessibility Analysis, Manufacturability Analysis}

\begin{figure*}[!tbp]
    \centering
    \includegraphics[width=0.9\textwidth,page=2]{./figures/qwen_tpp.pdf}
    \caption{Language-TPP processes tokenized event information including event type, description, and time through the QwenLM decoder. The decoder autoregressively generates information about the next event through next-token prediction, while the event intensity is modeled using the hidden state corresponding to the last token.}
    \label{fig:architecture}
\end{figure*}

\newcommand{\thought}[1]{{\color[rgb]{0.2,0.39,0.66}(#1)}}
\newcommand{\todo}[1]{{\color[rgb]{1.0,0.0,0.0}(#1)}}
\newcommand{\hsh}[1]{{\color{green!50!black} Henrik: #1}}
\newcommand{\st}[1]{{\color{red!50!black} Sebastian: #1}}

\newcommand{\ulm}[1]{_{\scaleto{\mathrm{#1}}{3pt}}}
\newcommand\at[2]{\left.#1\right|_{#2}}











\newtheorem{assumption}{Assumption}

\DeclareMathOperator*{\argmax}{arg\,max}
\DeclareMathOperator*{\argmin}{arg\,min}

\newcommand{\swname}[1]{\texttt{#1}}
\newcommand{\ie}{i\/.\/e\/.,\/~}
\newcommand{\eg}{e\/.\/g\/.,\/~}
\newcommand{\cf}{cf\/.\/~}

\newcommand{\fig}{Fig\/.\/~}
\newcommand{\defn}{Def\/.\/~}
\newcommand{\sect}{Sec\/.\/~}
\newcommand{\tabl}{Tab\/.\/~}
\newcommand{\algo}{Algorithm~}
\newcommand{\theo}{Theorem~}

\newcommand{\bnnl}{3 hidden layers}
\newcommand{\bnnn}{50 neurons}
\newcommand{\bnna}{tanh activations}

\newcommand{\capt}[1]{\mdseries{\emph{#1}}}

\newcommand{\videolink}{at \url{https://youtu.be/_d7AqTRjz6g}}
\newcommand{\codelink}{\url{https://github.com/wheelbot/mini-wheelbot}}

\newcommand{\fakepar}[1]{\vspace{0mm}\noindent\textbf{#1.}}

\newcommand{\needref}{\textcolor{red}{[REF]}}

\newcommand{\plotfontsize}{9pt}

\maketitle


\section{Introduction}
\label{sec:intro}

Foundational models (FMs)~\cite{zhang2024data, zhou2023comprehensive} have shown remarkable progress in the healthcare domain, enabling professional-like assessment of disease diagnosis, treatment decision-making, and monitoring~\cite{zhang2023text, wang2022medclip, lu2023mi-zero}. 
Examples include LLaVA-Med~\cite{li2023llava}, Med-PaLM Multimodal~\cite{tu2024towards}, and Med-Flamingo~\cite{moor2023med}, have demonstrated their capacity on question answering, medical image analysis, and report generation.
These studies follow a predominant top-down model development strategy that requires upstream developers to collect data and train models for downstream tasks. 
Consequently, the developed model capabilities are heavily dependent on the training data, limiting their generalization performance in diverse clinical scenarios. 
For instance, Med-Gemini~\cite{yang2024advancing} reveals promising general capabilities in report generation while it lags behind state-of-the-art (SoTA) models on classification tasks, especially for out-of-domain applications. 
This indicates that while the generalizability of the foundation model is promising, more solutions are expected to meet the various specialized clinical needs.

To address these challenges, multi-center data centralization becomes essential to enhance model capacity and robustness across varied clinical scenarios~\cite{rajpurkar2022ai}. 
Centralizing distributed data can significantly improve model training and inference performance.
However, the process of medical data storage, transfer, and aggregation among centers requires extra efforts to ensure data security and system interoperability~\cite{bradford2020international}.
Moreover, a growing concern for patient privacy makes large-scale multi-center data sharing particularly challenging. 
While efforts like federated learning~\cite{wen2023survey, li2020review} can achieve good model performance on local data, the need for synchronized system coordination presents significant challenges, as clients are unable to update asynchronously. This limitation greatly restricts the practical capability of such approaches.
As a result, without a flexible collaboration, medical community still struggles to fully utilize the isolated data and local computation resources for comprehensive medical AI model development. 
To address this dilemma, open-source platforms encourage public data sharing and knowledge integration~\cite{markiewicz2021openneuro, zenodo}.
However, these platforms focus solely on raw data sharing while seldom providing collaborative model training or cooperation between different institutions.
Recently, collaborative learning has emerged as a viable approach for enhancing multi-model robustness~\cite{boulemtafes2020review}. 
For instance, software-like model development~\cite{raffel2023building} mimics software engineering practices by introducing structured workflows, enabling merging, version control, and continuous model integration.
Under this design, model ability can be strengthened with incremental knowledge updates similar to the version updating in software development. 

Although collaborative learning provides a multi-model collaboration, two key challenges remain in the leakage of raw data during collaboration~\cite{huang2023lorahub} and the synchronization of multiple collaborators~\cite{mcmahan2017communication} in the medical AI community. It is still challenging to integrate decentralized, privacy-sensitive data across institutions, leading to under-utilized insights and fragmented knowledge sharing~\cite{kaissis2020secure, rajpurkar2022ai, abdullah2021ethics}.
 To address these challenges, inspired by the collaborative software development, we propose \textbf{Med}ical \textbf{Fo}undation Models Me\textbf{rg}ing (\textbf{MedForge}), a cooperative workflow enabling continuously community-driven foundation model (FM) development.
MedForge enables a lightweight manner for individual centers to share their knowledge among multiple centers, minimizing the burden of data transmission and integration while enhancing model robustness.
Meanwhile, MedForge facilitates asynchronous and flexible collaboration, allowing individual centers to continuously update and improve medical FMs without the need for real-time synchronization.
Similar to open-source software development, MedForge incrementally updates medical knowledge and follows a sustainable model development scheme. 
This key design emphasizes a bottom-up construction of a multi-task medical FM, allowing downstream users to collaboratively build, refine, and update the upstream model according to their local resources. Our major contributions of MedForge are as below: 
\begin{enumerate}
    \item[$\bullet$] We introduce a collaborative workflow to promote the merging scheme of open-source software development. Our proposed MedForge allows distributed clinical centers to asynchronously contribute to comprehensive medical model construction while reducing transmitting costs among centers and avoiding the leakage of raw data, thus enhancing the utilization of private resources in the healthcare system. 
    \item[$\bullet$] We propose two effective knowledge-merging strategies for the asynchronous branch contribution. The MedForge-Fusion strategy updates the plugin module parameters of the main model during the merging phase, whereas the MedForge-Mixture strategy integrates the output of the plugin module by memorizing each contributor's coefficient. These strategies make MedForge more flexible and versatile. MedForge-Fusion is friendly to implement, while the MedForge-Mixture offers better performance and robustness.
    \item[$\bullet$]  We comprehensively evaluate model merging strategies to accumulate medical knowledge among multiple branch plugin modules. MedForge yields superior performance on medical classification tasks compared to other collaborative baselines across multiple datasets. We demonstrate the robustness of MedForge by shuffling the task order and evaluating various configurations of plugin modules and dataset distillation methods.
\end{enumerate}



\section{Related Work}

\subsection{Advancements in AI and Agentic Workflows for Code Generation}

Since the training process of GPT-3.5 incorporated a substantial amount of code data to enhance the logical reasoning capabilities of language models \cite{chenEvaluatingLargeLanguage2021}, code generation has become closely intertwined with language modeling. With the emergence of models that place a stronger emphasis on reasoning, these capabilities continue to evolve. According to the SWE-bench benchmark, which simulates human programmers' problem-solving workflows, AI programming performance increased from below 2\% in December 2023 \cite{jimenez2024swebench} to over 60\% by February 2025\footnote{https://www.swebench.com/}.

However, simply reinforcing the reasoning ability of language models primarily advances lower-level software development tasks such as auto-completion and refactoring. To enhance automation in real-world software and system development, researchers have introduced various agentic workflows, including OpenHands \cite{openhands}, an open-source coding agent designed for end-to-end development, and Agent Company, which simulates the operation of a software company \cite{xu2024theagentcompany}. Nonetheless, as of February 2025, even the most sophisticated agentic workflows remain unable to fully realize end-to-end programming\footnote{https://www.swebench.com/}, let alone organization-level agency\footnote{https://the-agent-company.com/}. 

Within code generation and system development, front-end code generation—such as website development—often demonstrates stronger performance than back-end development. Research in this domain has examined reconstructing HTML/CSS structures from UI screenshots using computer vision techniques \cite{soseliaLearningUItoCodeReverse2023}, implementing hierarchical decomposition strategies for interface elements to optimize UI code generation \cite{wanAutomaticallyGeneratingUI2024}, and improving model specialization through domain-specific fine-tuning for UI generation \cite{wuUICoderFinetuningLarge2024}. To systematically evaluate front-end code generation, specialized benchmarks have been developed to assess the quality of HTML, CSS, and JavaScript implementations \cite{siDesign2CodeHowFar2024}. To investigate the societal impact of this notable improvement in AI programming capabilities, we focus on the task of website generation, where current AI systems are relatively close to achieving near end-to-end automation.

\subsection{Beyond Templates: AI-Powered, User-Centric UI}

With the continuing development of AI-driven user interface (UI) generation, users increasingly seek more personalized and diverse expressions rather than relying solely on conventional template reuse. Recent advances have led to adaptive UI generation systems like FrameKit, which allows end users to manually design keyframes and generate multiple interface variants \cite{wu_framekit_2024}. PromptInfuser goes a step further by enabling runtime dynamic input and generation of UI content \cite{petridisPromptInfuserHowTightly2024}. In this context, AI tools have been shown to offer inspiration for professional designers \cite{luBridgingGapUX2022}. For instance, DesignAID \cite{cai_designaid_2023} demonstrates that generative AI can provide conceptual directions and stimulate creativity at early design stages. Misty supports remixing concepts by allowing users to blend example images with the current UI, thereby enabling flexible conceptual exploration \cite{luMistyUIPrototyping2024}.

Beyond offering inspiration, AI can also provide real-time design feedback to guide iterative refinement and error correction \cite{duan_towards_2023}, such as handling CSS styling in simple websites and optimizing specific UI components \cite{liUsingLLMsCustomize2023}. It is capable of evaluating UI quality and relevance, offering suggestions at various design stages \cite{wuUIClipDatadrivenModel2024}, and even detecting potential development or UI issues in advance \cite{petridisPromptInfuserHowTightly2024}. Automated heuristic evaluations generated by AI can provide more precise assessments and recommendations, thereby streamlining the iterative process \cite{duanGeneratingAutomaticFeedback2024}. When combined with traditional heuristic rules, AI has been shown to increase the effectiveness of UI error detection and correction \cite{lu_ai_2024}. Integrating prototype-checking techniques into the UI generation workflow can further enhance automatic repair capabilities \cite{xiaoPrototype2CodeEndtoendFrontend2024}.

\subsection{Improving the Creative Workflow with AI}

In many creativity workflows, a prolonged progression from ideation, prototyping, and development to iteration is required \cite{palaniEvolvingRolesWorkflows2024}. Those creative processes are frequently constrained by multiple intricate steps that limit users' expressive capabilities. For example, the complexity and associated costs of developing a personal website often deter individuals from undertaking this process, prompting many to resort to standardized website templates for personal websites. However, GenAI can assist with the creativity workflow from various angles \cite{wanItFeltHaving2024,palaniEvolvingRolesWorkflows2024,longNotJustNovelty2024}. First, GenAI such as text-to-image generation can reduce the time needed to produce high-fidelity outcomes. This enables creators to focus on refining the gap between the high-fidelity results and their envisioned expectations, rather than expending effort on how to achieve high fidelity in the first place \cite{edwardsSketch2PrototypeRapidConceptual2024}. Besides, AI lowers the cost of experimenting with new ideas, thereby minimizing the psychological barriers to conducting trial and error with unconventional concepts \cite{palaniEvolvingRolesWorkflows2024}. When users are uncertain about what they want or have only a broad concept lacking specific details, AI can offer inspiration \cite{rickSupermindIdeatorExploring2023}. Moreover, AI can facilitate parallel prototyping by presenting multiple design directions simultaneously, allowing creators to compare and refine a range of diverse design solutions \cite{dowParallelPrototypingLeads2010}.
\goodname~enhances safety of LLM inputs and outputs while improving their quality. Specifically, it achieves two goals, 1) all user inputs are safe, contextually grounded, and effectively processed, such that the inputs to the LLMs are of high-quality and informative; and 2) the output generated by the LLMs are evaluated and enhanced, such that the outputs passed to users can be both relevant and of high quality. 
The pipeline can be partitioned into two parts, including 
1) processing before LLM inference that enhances user queries, and 2) processing after LLM inference that detects undesired content and handle them properly. We overview our pipeline in Figure~\ref{fig: system_overview}.


\noindent\underline{\textit{Pre-inference processing. }}
Before sending user queries to LLMs, \goodname~detects if there are any safety issues in the queries with \detection~and ground the queries with context knowledge with \grounding. 
\detection~monitors user inputs to identify and reject queries that might be unsafe. The monitoring includes typical safety checks, including toxicity, stereotypes, threats, obscenities, prompt injection attacks, etc. Any form of unsafe content will lead to the queries being rejected. 
Inputs that pass this initial safety check are grounded with context with \grounding, where the user query is contextualized and enhanced with relevant knowledge retrieved from the vector data storage. By equipping the query with some context knowledge, the LLM can do inference with enriched information, thus can reduce hallucinations when generating responses. The details of \detection~ and \grounding~will be introduced in \S\ref{sec:safety_detector} and \S\ref{sec:grounding}, respectively.




\noindent\underline{\textit{Post-inference processing. }}
Upon LLM finishing inference, \detection~detects safety issues in the LLM outputs, specifically, hallucinations. This is because LLM applications typically leverages well-developed LLMs or APIs, such as LLaMA~\citep{touvron2023llama} and ChatGPT API~\citep{openai-data-paper}, which are generally safe and less likely to generate toxic or other unsafe content, while hallucinations occur frequently. \detection~identifies hallucinations and provides reasons for the hallucinations, such that \goodname~can utilize the reasoning for later refinement of the LLM outputs. To achieve goal, \goodname~employs a text generation model to generate explainable results, and adjusts the loss function during training to ensure the model to produce classification results. 
After \detection~finishes detection, \fixing~fixes the problematic content or aligns the outputs with some rule-based wrappers to meet user expectations. 
If the outputs are difficult to fix, e.g., hallucinated responses, 
\fixing~will call a fixing model to fix the answers. Details about \fixing~can be found in \S\ref{sec:fixing}.


\section{Method}


In this work, we propose a method to achieve 3D-aware 2D representations and enable 3D reconstruction in the latent space. We base our method on the widely used Variational Autoencoder (VAE) from Latent Diffusion models \citep{metzer2022latent}. To enhance the 3D awareness of both encoder and decoder of the VAE, we present a three-stage pipeline as illustrated in Fig. \ref{fig:pipeline}. The first stage focuses on improving the 3D awaresness of the VAE's encoder through a novel correspondence-aware constraint on the latent space, making the 2D representations follow the geometry consistency (Sec.~\ref{subsec: Epipolar-aware Autoencoding}); The second stage builds a latent radiance field (LRF) to represent 3D scenes from the 3D-aware 2D representations (Sec.~\ref{subsec: Latent Radiance Fields}); The third stage further introduces a VAE-Radiance Field (VAE-RF) alignment method to boost the reconstruction performance (Sec.~\ref{subsec: Radiance Field-Guided Image Decoding}). In together, our LRF enables 3D reconstruction on the 2D latent space instead of the image space. It can render high-quality and photorealistic novel views, even for the unbounded scenes (Sec. \ref{sec: exp}). More details of our method are discussed in the following sections.


\begin{figure}[!t]
    \centering
    \includegraphics[width=\linewidth]{figures/method.png}
    \vspace{-1em}
    \caption{An illustration of  our pipeline for creating a latent radiance field in conjunction with 3D-aware 2D representation fine-tuning. 
    Firstly in Stage-I, we inject 3D awareness into the VAE’s encoder through applying a novel correspondence consistency constraint on the latent space, making the 2D representations follow the geometry consistency. Then in Stage-II, we create the latent radiance field (LRF) to represent 3D scenes based on the 3D-aware 2D representations. Finally in Stage-III, we introduce a VAE-Radiance Field alignment method to enhance the performance of image decoding from the  rendered latent space.
}
\vspace{.5em}
    \label{fig:pipeline}
\end{figure}

\subsection{Correspondece-aware Autoencoding}
\label{subsec: Epipolar-aware Autoencoding}
The first stage of our method is incorporating the geometry-awareness into the autoencoding process. Given $K$ muilt-view images $\mathcal{I}=\left\{\boldsymbol{I}_i\right\}_{i=1}^K,\left(\boldsymbol{I}_i \in \mathbb{R}^{H \times W \times 3}\right)$, the VAE encoder extracts 2D representations $\mathcal{Z}=\left\{\boldsymbol{Z}_i\right\}_{i=1}^K,\left(\boldsymbol{Z}_i \in \mathbb{R}^{H' \times W' \times 4}\right)$ in a low-dimensional latent space while the semantic information can be preserved effectively. However, as shown in Fig. \ref{fig: exp_recon}, most of existing NVS frameworks fail to reconstruct the photo-realistic images from the rendered latents.
It is mainly because the VAE encoding process significantly damages the multi-view consistency within the original image space, since the latent space presents massive high-frequency noises to compress the original RGB space into a compact latent space (see Fig. \ref{fig: encoder}). 
This brings severe challenges for reconstructing the 2D latent representations in the 3D space. 




\noindent\textbf{Correspondence consistency on the latent space.}
To address the above issue and enable effective latent 3D reconstruction, we are inspired by the multi-view correspondence consistency which serves as the foundation principle for modeling the natural 3D world. Specifically, points $\boldsymbol{x}_i \in \mathbb{R}^{2}$ in image $\boldsymbol{I}_i$ and points $\boldsymbol{x}_j \in \mathbb{R}^{2}$ in another image $\boldsymbol{I}_j$ are considered correspondences if they are connected by the fundamental matrix $\boldsymbol{F}_{ij} \in \mathbb{R}^{3 \times 3}$, satisfying the multi-view geometry constraint~\citep{schoenberger2016sfm}:
\begin{equation}
\boldsymbol{x}_{j}^\top \boldsymbol{F}_{ij} \boldsymbol{x}_i = 0.
\label{eq:fundamental}
\end{equation}
Eq. \ref{eq:fundamental} tells that a pair of correspondence points on the image space should be close to each other, so that the consistent geometry can be ensured during the optimization in the 3D space; otherwise, the artifacts and redundant geometry representation due to the local optimal will damage the quality of the 3D reconstruction and novel view synthesize. 
Motivated by this, we propose an computationally efficient strategy that incorporates the correspondence consistency into the autoencoder training. 
Specifically, a set of multi-view images $\mathcal{I}=\left\{\boldsymbol{I}_i\right\}_{i=1}^K,\left(\boldsymbol{I}_i \in \mathbb{R}^{H \times W \times 3}\right)$ are fed into the autoencoder to extract the latent representations  $\mathcal{Z}=\left\{\boldsymbol{Z}_i\right\}_{i=1}^K,\left(\boldsymbol{Z}_i \in \mathbb{R}^{H' \times W '\times 4}\right)$, and the correspondence consistency loss on the latent space is computed by 
% \textcolor{red}{Give the defination of j and N, and this loss should be step loss instead of total images loss}
\begin{equation}
\mathcal{L}_{\text{corres}} =  \sum_{i=1}^{K} \sum_{j \in \mathcal{K}(i)} \lambda_{ij} \left\| \boldsymbol{z}_i - \boldsymbol{z}_j \right\|_1.
\end{equation}
where $\boldsymbol{z}_i$ refers to the the latent pixel in the $\boldsymbol{Z}_i$ and $\boldsymbol{z}_i$ is the corresponding latent pixel in the neighbouring latent  $\boldsymbol{Z}_j$.
$\mathcal{L}_{\text{corres}}$ ensures that the encoded features follow the correspondence consistency derived from the multi-view images, where $\lambda_{ij}$ is the weight based on the average pose error (APE) calculated from the Frobenius norm between the two camera poses of images $\boldsymbol{I}_i$ and $\boldsymbol{I}_j$ to weight the accurate pose distance to represent the view-dependant latent codes. The detail of calculating $\lambda_{ij}$ can be found in Appendix \ref{subsec: APE details}
By injecting the latent correspondence consistency into the standard VAE training, our VAE training objective is: 
\begin{equation} 
\mathcal{L}_\text{StageI} =\mathcal{L}_\text{VAE} + \lambda_{1}\mathcal{L}_{\text{corres}} + \lambda_{2}\mathcal{L}_{\text{reg}}.
\label{eq:encoder}
\end{equation}

$\mathcal{L}_\text{VAE}$ is original VAE traning objective for VAE in Eq. \ref{eq:vae}. 
$\mathcal{L}_{\text{reg}} = -\text{KL}\left( q(\boldsymbol{Z}|\boldsymbol{X}) \parallel q_{\text{original}}(\boldsymbol{Z}|\boldsymbol{X}) \right)$ enforces the fine-tuned 2D representations being close to those of the pre-trained VAE, preserving the representation capability of the finet-tuned autoencoder.  This new learning objective ensures that the compact latent space of VAE preserves the multi-view geometric consistency, such that it is compatible with existing NVS frameworks such as 3DGS.



\textbf{Insight into latent correspondence consistency.} 
The maximum degree of the spherical harmonics is always set as 3 in NVS methods for the efficiency and robustness in the modeling the view-dependant information. To be more specific, the lower degree terms is aim to mostly capture low-frequency information such as albedo for the scene while the higher degrees are tended to model the high-frequency, view dependent information such as the lightning. For the latent space, the latent code can be considered as the combination of the base value and high frequency noise. Due to such a compact representation, the amount of the noise can be greatly increase compared to the RGB space, creating more difficulties for the SH coefficients to model the information from different views. When maximum degree is fixed, it is easier for SH coefficients to reach the global optimal instead of locally over-fitting. Fortunately, with our $\mathcal{L}_{\text{corres}}$, the high frequency noise can be effectively removed while the high-quality image generative ability can still be preserved, leading to a more stable process of the optimization and consistent geometry representation. Fig. \ref{fig: encoder} shows that the correspondence-aware encoding can significantly remove the high frequency noises in the 2D latent space and the visualization of applying Fast Fourier transform also showing less high-frequency noise in latent space achieved by our encoder,  resulting an effective approach to lifting the 2D features into the 3D latent fields.

\begin{figure}[!t]
    \centering
    \begin{tikzpicture}
     

        \node[anchor=south west, inner sep=0] (image1) at (0,0) {\includegraphics[width=1.0\textwidth]{figures/fft.png}};
        
       
        \node[anchor=south] at (1.3, 2.0) {\small Image};               
        \node[anchor=south] at (4.15, 2.0) {\small VAE latent};         
        \node[anchor=south] at (7.0,  2.0) {\small Finetuned latent};               
        \node[anchor=south] at (9.8,  2.0) {\small VAE latent FFT};
         \node[anchor=south] at (12.55,   2.0) {\small Finetuned latent FFT};
    \end{tikzpicture}
    \vspace{-1em}
    \caption{A visualization of latent spaces of original and our fine-tuned VAEs. Our method ensures an accurate geometry in the latent space while removing a certain amount of high-frequency noises.}
\label{fig: encoder}
\end{figure}



\subsection{Latent Radiance Field}
\label{subsec: Latent Radiance Fields}



Based on the 3D-aware 2D representation fine-tuning discussed in Sec.~\ref{subsec: Epipolar-aware Autoencoding}, we create 3D representations directly in the 2D latent space of VAE, namely the latent radiance field (LRF). We take 3DGS \citep{kerbl3Dgaussians} as an example of radiance field representations to discuss our LRF.  

By following 3DGS, a set of latent 3D Gaussians is formulated as
\begin{equation}
    \mathcal{G} = \{(\bm{\mu}, \mathbf{s}, \mathbf{R}, \alpha, \mathbf{SH}_{f})_j)\}_{1\leq j \leq M} \textnormal{,}
\end{equation}
where $\bm{\mu} \in \mathbb{R}^3$ is the 3D mean of the Gaussian, $\mathbf{S} = \textnormal{diag}(\mathbf{s}) \in \mathbb{R}^{3\times 3}$ is the Gaussian scale, $\mathbf{R}\in \mathbb{R}^{3\times 3}$ its orientation, $\alpha \in \mathbb{R}$ a per-Gaussian opacity, and $\mathbf{SH}_{f}$ models the view-dependant latent in the 3D latent space. By following the differentiable rasterization process of 3DGS, we rasterize the 2D latent representations using point-based $\alpha$-blending as follows:
\begin{equation}
\mathbf{Z} = \sum_{i\in \mathcal{N}}\mathbf{z}_{i}\alpha _{i}\prod_{j=1}^{i-1}(1-\alpha _{i}),
\end{equation}
where $\mathcal{N}$ is a set of ordered Gaussians overlapping the pixel, $\mathbf{z}_{i}\in \mathbb{R}^{dim}$
is the view-dependent latent code of each Gaussian, where $\mathbf{dim}$ is the number of the latent dimension of the feature. and $\alpha _{i}$ is given by evaluating a
2D Gaussian with covariance $\mathbf{\Sigma}$ multiplied with a
learned per-point opacity. 
Let  $\mathcal{I}=\left\{\boldsymbol{I}_i\right\}_{i=1}^K,\left(\boldsymbol{I}_i \in \mathbb{R}^{H \times W \times 3}\right)$ be a set of multi-view images of a scene with corresponding camera parameters. Let $\mathcal{Z}=\left\{\boldsymbol{Z}_i\right\}_{i=1}^K,\left(\boldsymbol{Z}_i \in \mathbb{R}^{H \times W \times 3}\right)$ be a corresponding set of latents from the VAE encoder. The rasterization function $r$ renders a set of latent Gaussians into a 2D latent representation according to the camera pose $\mathbf{P}_{i}$. Then, we optimize the latent Gaussian parameters, to optimally represent
latent $\mathcal{Z}$:
\begin{equation}
    \hat{\mathcal{G}} = \argmin_{\{(\bm{\mu}, \mathbf{s}, \mathbf{R}, \alpha, \mathbf{SH}_{f}\}} \sum_{i=1}^N \mathcal{L}^f(r(\mathcal{G}, \mathbf{P}_{i}),\mathbf{Z}_i) \textnormal{,}
\end{equation}
where $\mathcal{L}^f$ is a pixel-wise $l_{1}$ loss combined with a D-SSIM term. Notably, we do not need to impose additional geometric consistency constraints introduced by previous literature~\citep{yue2024improving,kobayashi2022distilledfeaturefields,zhou2024feature}, as our correspondence-aware autoencoder fine-tuning ensures geometrically consistent 2D representations in the 3D space. Therefore, our LRF reconstructs the 2D latent representations as a radiance field representation directly, and enables an efficient rendering of the 2D latent representations for novel views.

\subsection{VAE-Radiance Field Alignment} \label{subsec: Radiance Field-Guided Image Decoding}
Although the correspoondence-aware autoencoding introduced in Sec.~\ref{subsec: Epipolar-aware Autoencoding} improves the 3D consistency of VAE latent space, the LRF distribution $\boldsymbol{p}(z_{\text{NVS}})$ are still shifted from the VAE latent distribution $\boldsymbol{p}(z_{\text{VAE}})$ due to the non-linearity in neural rendering, resulting in performance decrease when we decode LRF rendering results back to images through the VAE decoder. 

We further propose to fine-tune the VAE decoder under the radiance field guidance to address this issue. With the LRF built in Sec. \ref{subsec: Latent Radiance Fields}, we can reconstruct LRFs from a large amount of scenes to generate a latent-image paired dataset. This dataset consists of the 2D latent representations $\mathcal{Z}=\left\{\boldsymbol{Z}_i\right\}_{i=1}^K,\left(\boldsymbol{Z}_i \in \mathbb{R}^{H' \times W' \times 4}\right)$ rendered by LRFs and the corresponding ground truth images $\mathcal{I}=\left\{\boldsymbol{I}_i\right\}_{i=1}^K,\left(\boldsymbol{I}_i \in \mathbb{R}^{H \times W \times 3}\right)$. Notably, we also include the training views of LRFs in this dataset, since a key feature of existing NVS methods is to overfit the training views. 
The training objective of our VAE-RF alignment decoder fine-tuning is:
\begin{equation} 
\mathcal{L}_\text{StageIII}=  \lambda_{\text{train}} \left\|D(Z_{\text{train}}) - I_{\text{train}} \right\|_1 + \lambda_{\text{novel}} \left\|D(Z_{\text{novel}}) - I_{\text{novel}}\right\|_1,
\label{eq:decoder}
\end{equation} 
where $D(\cdot)$ is the decoder, $Z_{\text{train}}$ and $Z_{\text{novel}}$  are the latent codes of the training views and novel views, respectively. $I$ refer to the corresponding ground truth images. $\lambda_{\text{novel}}$ and $\lambda_{\text{novel}}$ are the weighting coefficient that balances the contributions of the training and novel views. Both of the weights are set to $0.5$ to ensure that the decoder learns not only to decode effectively from the training views but also to generalize and perform well on the novel views.
Eq. \ref{eq:decoder} effectively minimizes the distribution mismatch between the VAE latent space and the LRF rendering space. After decoder fine-tuning, high-quality images can be reconstructed from the LRF rendering of either training or novel views. The fine-tuned autoencoder enhances 3D reconstruction and generation by providing a geometry-aware 2D latent space as well as a radiance field-compatible autoencoder.




\begin{figure*}[!h]
    \centering
    \begin{subfigure}[b]{0.8\linewidth}
        \centering
        \includegraphics[width=0.45\linewidth]{images/residual/text/CIReVL_Recall5.png}
        \hfil
        \includegraphics[width=0.45\linewidth]{images/residual/text/pic2word_recall5.png}
        \caption{\textbf{PDV-T}: Impact of $\alpha$ scaling on composed text embeddings}
        \label{fig:residual_text_sub}
    \end{subfigure}
    
    \begin{subfigure}[b]{0.8\linewidth}
        \centering
        \includegraphics[width=0.45\linewidth]{images/residual/image/CIReVL_Recall5.png}
        \hfil
        \includegraphics[width=0.45\linewidth]{images/residual/image/pic2word_recall5.png}
        \caption{\textbf{PDV-I}: Impact of $\alpha$ scaling on composed image embeddings}
        \label{fig:residual_image_sub}
    \end{subfigure}
    
    \begin{subfigure}[b]{0.8\linewidth}
        \centering
        \includegraphics[width=0.45\linewidth]{images/residual/fusion/CIReVL_Recall5.png}
        \hfil
        \includegraphics[width=0.45\linewidth]{images/residual/fusion/pic2word_recall5.png}
        \caption{\textbf{PDV-F}: Impact of varying $\beta$ with on composed fused embeddings}
        \label{fig:residual_fusion_sub}
    \end{subfigure}
    \caption{Impact of changing $\alpha$/$\beta$ on Recall@5 performance across different PDV applications. For each row, results are shown for the CIReVL (left) and Pic2Word (right) baseline methods.}
    \label{fig:residual_all}
\end{figure*}

\section{Experiments} 
\label{sec:exp}
\noindent\textbf{Implementation Details.} We utilize the official implementations of four ZS-CIR baseline methods: CIReVL\footnote{https://github.com/ExplainableML/Vision\_by\_Language} and LDRE \footnote{https://github.com/yzy-bupt/LDRE} as representative caption-based feature extraction approaches and Pic2Word\footnote{https://github.com/google-research/composed\_image\_retrieval} and SEARLE\footnote{https://github.com/miccunifi/SEARLE} as representative pseudo tokenization-based methods. All feature extraction processes follow the original implementations provided by these baseline methods. However, to calculate $\Delta_{PDV}$, we need text embeddings without prompts, which are not provided in the original implementations. For CIReVL and LDRE, we obtain these embeddings by passing the generated image captions directly to CLIP. For Pic2Word and SEARL, we construct the base text embedding by passing the phrase ``a photo of $\langle$token$\rangle$" to CLIP, where $\langle$token$\rangle$ represents the extracted image token obtained via text inversion.

\noindent\textbf{Datasets and Base Vision-Language Models.} Following previous work, we evaluated our method on a suite of datasets including Fashion-IQ \cite{wu2021fashion}, CIRR \cite{liu2021image} and CIRCO \cite{baldrati2023zero}. Our proposed method is a plug-and-play approach requiring no additional training, leveraging only pre-trained models. For feature extraction, we use three CLIP variants: ViT-B/32, ViT-L/14, and ViT-G/14, and used the same pre-trained weights as used by the baseline methods. For image tokenization, we employ the pre-trained Pic2Word model. 

\subsection{Effect of Using the PDV}
We now explore the impact of the three proposed uses of the PDV: Using the PDV to augment text queries (PDV-T, see Sec. \ref{sec:exp1}), using the PDV to augment image queries (PDV-I, see Sec. \ref{sec:exp2}), and using the PDV in queries that fuse image and text data (PDV-F, see Sec. \ref{sec:exp3}).

\begin{table*}
	\footnotesize
	\centering
	\begin{tabular}{l|l|c|c|c|cccccccc}
		\hline
		\textbf{Fashion-IQ} & & & & & \multicolumn{2}{c}{\textbf{Shirt}} & \multicolumn{2}{c}{\textbf{Dress}} & \multicolumn{2}{c}{\textbf{Toptee}} & \multicolumn{2}{c}{\textbf{Average}} \\ \hline
		Backbone & Method& $\beta$ & $\alpha_{I}$& $\alpha_{T}$ & R@10 & R@50 & R@10 & R@50 & R@10 & R@50 & R@10 & R@50 \\
		\hline
		\multirow{6}{*}{ViT-B/32} %
		& SEARLE & - & - & - & 24.14 & 41.81 & 18.39 & 38.08 & 25.91 & 47.02 & 22.81 & 42.30 \\
		& SEARLE + \textbf{PDV-F} & 0.9 & 1.1 & 0.9 & \hli{24.83} & 41.71 & \hli{20.13} & \hli{41.40} & \hli{25.96} & \hli{47.17}  & \hli{23.64} & \hli{43.43} \\
		& CIReVL \textdagger &- & -& -& 28.36 & 47.84 & 25.29 & 46.36 & 31.21 & 53.85 & 28.29 & 49.35 \\
		& CIReVL + \textbf{PDV-F} & 0.75 & 1.4 & 1.4 & \hlb{32.88} & \hlb{52.80} & \hlb{32.67} & \hlb{54.49} & \hlb{38.91} & \hlb{61.81} & \hlb{34.82} & \hlb{56.37} \\
		& LDRE \textdagger & - & - & - & 27.38 & 46.27 & 19.97 & 41.84 & 27.07 & 48.78 & 24.81 & 45.63 \\
		& SEIZE \textdagger & - & - & - & \underline{29.38} & \underline{47.97} & \underline{25.37} & \underline{46.84} & \underline{32.07} & \underline{54.78} & \underline{28.94} & \underline{49.86} \\
		\hline
		\multirow{8}{*}{ViT-L/14} & Pic2Word & & & & 25.96 & 43.52 & 19.63 & 40.90 & 27.28 & 47.83 & 24.29 & 44.08 \\
		& Pic2Word + \textbf{PV-F} & 0.8 & 1.0 & 1.0 & \hli{28.21} & \hli{44.55} & \hli{20.92} & \hli{42.24} & \hli{29.02} & \hli{48.90}& \hli{26.05} & \hli{45.23}\\
		& SEARLE & - & - & - & 26.84 & 45.19 & 20.08 & 42.19 & 28.40 & 49.62 & 25.11 & 45.67 \\
		& SEARLE +\textbf{PDV-F} & 0.8 & 1.2 & 1.0 & \hli{28.66} & \hli{46.76} & \hli{23.60} & \hli{46.41} & \hli{31.00} & \hli{52.32} & \hli{27.75} & \hli{48.50} \\
		& CIReVL \textdagger & & & & 29.49 & 47.40 & 24.79 & 44.76 & 31.36 & 53.65 & 28.55 & 48.57 \\
		
		& CIReVL + \textbf{PDV-F} & 0.55 & 1 & 1.3 & \hlb{37.78} & \hlb{54.22} & \hlb{33.61} & \hlb{56.07} & \hlb{41.61} & \hlb{62.16} & \hlb{37.67} & \hlb{57.48} \\
		& LinCIR & - & - & - & 29.10 & 46.81 & 20.92 & 42.44 & 28.81 & 50.18 & 26.82 & 46.49 \\
        & SEIZE & -& -& -& \underline{33.04} & \underline{53.22} & \underline{30.93} & \underline{50.76} & \underline{35.57} & \underline{58.64} & \underline{33.18} & \underline{54.21} \\
		\hline
        \multirow{6}{*}{ViT-G/14} & Pic2Word  & - & - & - & 33.17 & 50.39 & 25.43 & 47.65 & 35.24 & 57.62 & 31.28 & 51.89\\
         & SEARLE  & - & - & - & 36.46 & 55.35 & 28.16 & 50.32 & 39.83 & 61.45 & 34.81 & 55.71\\
		  & CIReVL \textdagger & -& -& -& 33.71 & 51.42 & 27.07 & 49.53 & 35.80 & 56.14 & 32.19 & 52.36 \\
		& CIReVL + \textbf{PV-F} & 0.6 & 1.4 & 1.4 & \hli{41.90} & \hli{58.19} & \hlb{40.70} & \hlb{62.82} & \underline{\hli{48.09}}& \hli{67.77}& \underline{\hli{43.56}}& \hli{62.93}\\
        & LinCIR & - & - & - & \textbf{46.76} & \underline{65.11} & 38.08& 60.88& \textbf{50.48}& \underline{71.09}& \textbf{45.11} & \underline{65.69}\\
        & SEIZE & - & - & - & \underline{43.60} & \textbf{65.42}& \underline{39.61} & \underline{61.02} & 45.94& \textbf{71.12}& 43.05& \textbf{65.85}\\
		\hline
	\end{tabular}
	\caption{Average recall for different methods on Fashion-IQ validation dataset. \textdagger~denotes that numbers are taken from the original paper.}
	\label{tab:fashion_iq_results}
\end{table*}


\begin{table*}
	\centering
	\footnotesize
	\setlength{\tabcolsep}{4pt}
	\begin{tabular}{ll|c|c|c|cccc|cccc|ccc}
		\hline
		\multicolumn{2}{c|}{\textbf{Dataset}} & & & &  \multicolumn{4}{c|}{\textbf{CIRCO}} & \multicolumn{7}{c}{\textbf{CIRR}} \\
		\hline
		\multicolumn{2}{c|}{Metric} & & & & \multicolumn{4}{c|}{mAP@k} & \multicolumn{4}{c|}{Recall@k} &\multicolumn{3}{c}{$R_s$@k} \\
		\cline{3-16}
		Arch & Method & $\beta$ & $\alpha_I$ & $\alpha_T$ & k=5 & k=10 & k=25 & k=50 & k=1 & k=5 & k=10 & k=50 & k=1 & k=2 & k=3 \\
		\hline
		\multirow{8}{*}{ViT-B/32} 
		& PALAVRA\cite{cohen2022my} \textdagger & -& -& -& 4.61 & 5.32 & 6.33 & 6.80 & 16.62 & 43.49 & 58.51 & 83.95 & 41.61 & 65.30 & 80.94 \\
		& SEARLE \textdagger & -& -&- & 9.35 & 9.94 & 11.13 & 11.84 & 24.00 & 53.42 & 66.82 
		& 89.78 & 54.89 & 76.60 & 88.19 \\
		& SEARLE + \textbf{PDV-F} & 0.9 & 1.4 & 1.2 & \hli{9.99} & \hli{10.50}  & \hli{11.70} & \hli{12.40} & \hli{24.53} & \hli{53.71} & \hli{67.33} & \hli{89.81} & \hli{56.94} & \hli{78.05} & \hli{88.99} \\
		&CIReVL \textdagger & - & - & -& 14.94 & 15.42 & 17.00 & 17.82 & 23.94 & 52.51 & 66.00 & 86.95 & 60.17 & 80.05 & 90.19 \\
		& CIReVL + \textbf{PDV-F} & 0.75 & 1.4 & 1.2 & \hlb{19.90} & \hlb{20.61} & \hlb{22.64} & \hlb{23.52} & \hlb{33.25} & \hlb{64.15} & \hlb{75.23} & \hlb{92.43} & \hlb{65.81} &\underline{\hli{83.76}} &\underline{\hli{92.10}} \\
		& LDRE & -& -& -& 17.81 & 18.04 & 19.73 & 20.67 & 25.69 & 55.52 & 68.77 & 89.86 & 60.10 & 80.58 & 91.04 \\
		& LDRE + \textbf{PDV-F} & 0.75 & 1.4 & 1.4 & \hli{17.80} & \hli{18.78} & \hli{20.61} & \hli{21.56} & \underline{\hli{29.30}} & \underline{\hli{60.39}} & \underline{\hli{72.51}} & \underline{\hli{91.42}} & \hli{63.06} & \hli{82.36} & \hli{91.54} \\
        & SEIZE & -&- &- & \underline{19.04} & \underline{19.64} & \underline{21.55}& \underline{22.49}& 27.47 & 57.42& 70.17 & - & \underline{65.59} & \textbf{84.48}& \textbf{92.77} \\
 		\hline
		\multirow{10}{*}{ViT-L/14}
		& Pic2Word & -& -& -& 6.81 & 7.49 & 8.51 & 9.07 & 23.69 & 51.32 & 63.66 & 86.21 & 53.61 & 74.34 & 87.28 \\
		& Pic2Word + \textbf{PDV-F} & 0.85 & 1.2 & 1.0 & \hli{7.74} &  \hli{8.67} & \hli{9.77} & \hli{10.37} & \hli{23.90} & \hli{51.95} & \hli{64.63} & \hli{87.04} & \hli{53.16}  & \hli{74.07} & \hli{87.08}\\
		& SEARLE \textdagger & - & - & - & 11.68 & 12.73 & 14.33 & 15.12 & 24.24 & 52.48 & 66.29 & 88.84 & 53.76 & 75.01 & 88.19 \\
		& SEARLE + \textbf{PDV-F} & 0.85 & 1.4 & 1.2 & \hli{12.58} & \hli{13.57} & \hli{15.30} & \hli{16.07} & \hli{25.64} & \hli{53.61} & \hli{66.58} & \hli{88.55} & \hli{55.83} & \hli{76.48} & \hli{88.53} \\
		& CIReVL \textdagger & -& -& -& 18.57 & 19.01 & 20.89 & 21.80 & 24.55 & 52.31 & 64.92 & 86.34 & 59.54 & 79.88 & 89.69 \\
		& CIReVL + \textbf{PDV-F} & 0.75 & 1.4 & 1.2 & \hlb{25.67} & \hlb{26.61} & \underline{\hli{28.81}} & \hlb{29.95} & \hlb{36.24} & \hlb{66.17} & \hlb{76.96} & \hlb{92.29} & \hlb{68.07} & \hlb{85.35} & \hlb{93.47} \\
		& LDRE & -& -& -& 22.32 & 23.75 & 25.97 & 27.03 & 26.68 &55.45  & 67.49 & 88.65 & 60.39 & 80.53 & 90.15 \\
		& LDRE + \textbf{PDV-F} & 0.75 & 1.4 & 1.4 & \hli{25.23} & \hli{26.52} & \hlb{28.94} & \hlb{29.95} & \underline{\hli{30.16}} & \underline{\hli{59.98}} & \underline{\hli{71.90}} & \underline{\hli{90.87}} & \hli{63.66} & \hli{82.87} & \hli{91.57} \\

        & LinCIR & - & - & - &12.59 &13.58 &15.00 &15.85 &25.04 &53.25 &66.68 & - &57.11 &77.37 &88.89\\
        & SEIZE & -& -& -& 24.98 & 25.82 &28.24 &\underline{29.35}& 28.65 &57.16& 69.23& - &\underline{66.22} &\underline{84.05} &\underline{92.34} \\
        

        
		\hline
		\multirow{7}{*}{ViT-G/14} & CIReVL \textdagger & -& -& -& 26.77 & 27.59 & 29.96 & 31.03 & 34.65 & 64.29 & 75.06 & 91.66 & 67.95 & 84.87 & 93.21 \\

		& CIReVL + \textbf{PDV-F} & 0.75 & 1.4 & 1.2 & \hli{30.02} & \hli{31.46} & \hli{34.01} & \hli{35.08} & \hli{38.15} &\hli{67.93} & \hli{77.90} & \hli{92.77} & \hli{69.37} & \hli{85.37} & \hli{93.45}  \\
		
		& LDRE & -& -& -& \underline{33.30} & \underline{34.32} & \underline{37.17} & \underline{38.27} & 37.40 & 66.96 & 78.17 & 93.66 & 68.84 & 85.64 & 93.90 \\
		& LDRE + \textbf{PDV-F} & 0.75 & 1.4 & 1.4 & \hlb{34.88} & \hlb{36.41} & \hlb{39.12} & \hlb{40.23} & \hlb{42.51} & \hlb{72.22} & \hlb{81.71} & \hlb{94.94} & \underline{\hli{72.39}} & \underline{\hli{88.34}} & \underline{\hli{94.80}} \\
        & SEARLE & - & - & - & 13.20 &13.85 &15.32 &16.04 & 34.80 & 64.07 & 75.11 &-&68.72 &84.70 &93.23 \\
        & LinCIR & - & - & - & 19.71 &21.01 &23.13 &24.18 &35.25 &64.72 &76.05 & - &63.35 &82.22 &91.98 \\
        & SEIZE & -& -& -& 32.46 & 33.77 &36.46 &37.55 &\underline{38.87} & \underline{69.42} & \underline{79.42} & -&\textbf{74.15} & \textbf{89.23} & \textbf{95.71} \\
		\hline
	\end{tabular}
	\caption{Performance comparison on CIRCO and CIRR test datasets. As in previous works, for CIRCO, mAP@k is reported, while for CIRR both Recall@k and $R_s$@k metrics are used. \textdagger~denotes that numbers are taken from the original paper.}
	\label{tab:circo_cirr_results}
\end{table*}

\noindent{\textbf{Analysing the PDV for Text (PDV-T)}}
\label{sec:exp1}
To investigate how scaling the prompt vector, $\Delta_{PDV}$, affects retrieval performance with composed text embeddings, we conducted experiments using two zero-shot approaches (CIReVL and Pic2Word) with different backbone networks across three datasets. We evaluated the performance by varying the scaling parameter, $\alpha$ (Eq. \ref{eqn:text_embedding}), from -0.5 to 3 by an interval of 0.1.

The results are presented in Figure \ref{fig:residual_text_sub}. To account for scale variations across different experiments, we report relative recall values, where a baseline of zero is established at $\alpha=1$. As shown in Figure \ref{fig:residual_text_sub}, varying $\alpha$ leads to significant changes in relative recall performance\footnote{See supplementary material for Recall@10 and Recall@50 figures}. Our analysis reveals method-specific patterns across datasets. With CIReVL, increasing $\alpha$ improves relative recall on both FashionIQ and CIRCO datasets. In contrast, Pic2Word shows no significant improvement on FashionIQ and CIRR when varying $\alpha$, while CIRCO's performance improves when $\alpha$ is reduced to 0.8-1.0. This divergent behavior is fundamentally linked to each method's ability to generate an accurate $\Delta_{PDV}$. As demonstrated in Tables \ref{tab:fashion_iq_results} and \ref{tab:circo_cirr_results}, CIReVL consistently outperforms Pic2Word across various benchmarks, indicating its superior ability to generate a more accuraute composed query, and thus a more accurate $\Delta_{PDV}$. Consequently, increasing $\alpha$ yields greater benefits for CIReVL compared to Pic2Word.

We visualize the top-5 retrieval results using CIReVL with a ViT-B-32 backbone across three datasets (one reference image from each) under varying $\alpha$ values, as shown in Figure \ref{fig:residual_qual}\red{a}. As $\alpha$ increases, the retrieved results show stronger alignment with the prompt. Conversely, when $\alpha$ exceeds 1, the results include semantically related but unseen variations, while $\alpha$ values below 0.5 yields results opposite to the prompt's intent. For instance, ``brighter blue and sleeveless" retrieves ``dark blue with sleeves," ``plain background" yields ``natural/dark background," and ``young boy" returns ``adult" images.





\noindent{\textbf{Analysing the PDV for Image (PDV-I)}}
\label{sec:exp2}
To evaluate whether $\Delta_{PDV}$ enhances the retrieval performance of image embeddings, we conducted experiments following the protocol described in Section~\ref{sec:exp1}. We modified image embeddings by adding $\Delta_{PDV}$ scaled with $\alpha$ values ranging from -0.5 to 2.0, where $\alpha=0$ represents the original image-only embeddings. As shown in Figure \ref{fig:residual_image_sub}, Recall@K exhibits a positive correlation with $\alpha$ for values below 1. This upward trend continues until $\alpha=2.0$ for CIReVL, while Pic2Word's performance peaks when $\alpha$ reaches 1.4.  The performance of PDV-I was evaluated on the CIRR and CIRCO datasets by comparing it with other visual embedding-based methods, as detailed in Table \ref{tab:circo_cirr_results_pdv-I}. The results reveal that PDV-I achieved marginal improvements over existing approaches.

Following the methodology in Section~\ref{sec:exp1}, we conduct similar visualizations, with results shown in Figure \ref{fig:residual_qual}\red{b}. As with PDV-T, increasing $\alpha$ leads to stronger alignment between retrieved results and the prompt. When $\alpha$ exceeds 0.5, the results exhibit semantic relationships to the query, while $\alpha$ values below 0.5 yield results opposing the prompt's intent.
Notably, PDV-I's top retrievals demonstrate higher visual similarity to reference images compared to PDV-F, as evidenced by the preserved design elements in the clothing item (left) and laptop (middle). This characteristic is particularly valuable for applications include fashion search \cite{wu2021fashion} and logo retrieval \cite{tursun2019component}, where visual similarity plays a crucial role.



\begin{figure*}[!tbh]
	\centering
	\includegraphics[width=0.825\linewidth]{images/qualitative/PV_qual_all_mini.pdf}
	\caption{Visualisation of the impact of $\alpha$/$\beta$ scaling on top-5 retrieval results. CIReVL with ViT-B-32 Clip model is the baseline method used. Representative examples with prompts from three datasets: FashionIQ (left), CIRR (middle), and CIRCO (right) are shown at the top. \textbf{\textcolor{boxgreen}{Green}} and \textbf{\textcolor{boxblue}{blue}} bounding boxes indicate true positives and near-true positives, respectively.}
	\label{fig:residual_qual}
	
\end{figure*}

\noindent{\textbf{Analysing PDV Fusion (PDV-F)}}
\label{sec:exp3}
Finally, we evaluate the effectiveness of fusing image and text-composed embeddings by varying the fusion parameter, $\beta$, from 0 to 1 while maintaining $\alpha=1$
for both PDV-I and PDV-F. At $\beta=0$, the model relies solely on composed image embeddings, while at $\beta=1$, it uses only composed text embeddings. As shown in Figure \ref{fig:residual_fusion_sub}, the fusion of both embeddings consistently outperforms using either embedding type alone. Optimal retrieval performance is typically achieved when $\beta$ is between 0.4 and 0.8.

We similarly visualize the top-5 retrieved results across different $\beta$ values. As shown in Figure \ref{fig:residual_qual}\red{c}, when $\beta$ is small, the retrieved results maintain high visual similarity to the reference image. Conversely, as $\beta$ exceeds 0.5, the results demonstrate stronger semantic alignment with the prompt.



\subsection{ZS-CIR Benchmark Comparison}






\begin{table*}
	\centering
	\footnotesize
	\setlength{\tabcolsep}{4pt}
	\begin{tabular}{l|l|c|cccc|cccc|ccc}
		\hline
		\multicolumn{2}{c|}{\textbf{Dataset}} & & \multicolumn{4}{c|}{\textbf{CIRCO}} & \multicolumn{7}{c}{\textbf{CIRR}} \\
		\hline
		& Metric & & \multicolumn{4}{c|}{mAP@k} & \multicolumn{4}{c|}{Recall@k} & \multicolumn{3}{c}{$R_s$@k} \\
		\cline{2-14}
		Arch & Method & $\alpha_I$ & k=5 & k=10 & k=25 & k=50 & k=1 & k=5 & k=10 & k=50 & k=1 & k=2 & k=3 \\
		\hline
		\multirow{6}{*}{ViT-B/32} 
		& Image-only \textdagger & - & 1.34 & 1.60 & 2.12 & 2.41 & 6.89 & 22.99 & 33.68 & 59.23 & 21.04 & 41.04 & 60.31 \\
		& Text-only \textdagger & - & 2.56 & 2.67 & 2.98 & 3.18 & 21.81 & 45.22 & 57.42 & 81.01 & 62.24 & 81.13 & 90.70 \\
		& Image + Text \textdagger & - & 2.65 & 3.25 & 4.14 & 4.54 & 11.71 & 35.06 & 48.94 & 77.49 & 32.77 & 56.89 & 74.96 \\
		& SEARLE + \textbf{PDV-I} & 1.5 & 4.77 & 5.23  & 6.31 & 6.82 & 16.65 & 42.53 & 55.16 & 81.42 & 44.68 & 67.78 & 82.94\\
		& CIReVL + \textbf{PDV-I} & 2.0 & \textbf{10.29 }& \textbf{10.80} & \textbf{12.23} & \textbf{12.93} & \textbf{27.18} & \textbf{56.53} & \textbf{67.76} & \textbf{87.64} & \textbf{59.81} & \textbf{79.59} & \textbf{90.15}\\
		& LDRE + \textbf{PDV-I} & 2.0 & 8.00 & 8.88 & 10.06 & 10.72 & 23.37 & 51.21 & 63.69 & 85.57 & 55.57 & 76.63 & 88.15\\
		\hline
	\end{tabular}
	\caption{PDV-I performance on CIRCO and CIRR test datasets. Note that the image-only approach utilizes the visual embedding of the reference image, whereas the text-only approach employs the text embedding of the prompt.}
	\label{tab:circo_cirr_results_pdv-I}
\end{table*}

We evaluated PDV-F alongside four baseline approaches (CIReVL, LDRE, Pic2Word, and SEARLE) across three benchmarks. Notably, CIReVL was tested with three different backbones on three datasets, as its models and intermediate results are publicly available. However, for the remaining methods, we conducted partial evaluations due to limited open-source availability or restricted support.

The numerical results are presented in Tables \ref{tab:fashion_iq_results} and \ref{tab:circo_cirr_results}.
On the FashionIQ benchmark, PDV-F yields substantial improvements for all baseline approaches, with CIReVL showing particularly strong gains that scale with backbone size. Similarly, all methods demonstrate significant performance improvements on CIRCO and CIRR datasets. Notably, CIReVL achieves larger improvements compared to other methods, with the most substantial gains observed when using small and medium backbone architectures. Our PDV-F implementation within the CIReVL framework consistently outperformed other state-of-the-art methods, including LinCIR and SEIZE, across most evaluation metrics. Similar to SEIZE, PDV-F offers the advantage of being entirely training-free; however, unlike SEIZE, it does not significantly increase feature extraction computational costs. While LinCIR demonstrates exceptional inference speed, it lacks the training-free nature of our approach, requiring dedicated model training before deployment.  






\section{Results and Discussion}
\label{sec05}

In this section, we present the results, discuss them, and make some conclusions about the experiments.

With a slightly realistic scenario, the experiments present some interesting results. Figures \ref{fig:hist_score} and \ref{fig:hist_gen} show respectively histograms of (a) the final score after the full training process and (b) the number of generations the process took. Notice that most runs just stopped at 20 generations (maximum) and could not improve further, as Figure \ref{fig:hist_gen} suggests. Despite that, as can be seen in Fig. \ref{fig:hist_score}, more than 80\% of the runs ended with a score of 2 or less, meaning at most two wrong device activations on 260 interactions. 

\begin{figure*}
        \centering
        \begin{subfigure}[b]{0.475\textwidth}
            \centering
            \includegraphics[width=0.8\textwidth]{imgs/results/results_2/histogram_of_score_.png}
            \caption[]%
            {{\small Histogram of \textit{score} achieved on experiment runs. Notice that the lesser, the better.}}    
            \label{fig:hist_score}
        \end{subfigure}
        \hfill
        \begin{subfigure}[b]{0.475\textwidth}  
            \centering 
            \includegraphics[width=0.8\textwidth]{imgs/results/results_2/histogram_of_generations_.png}
            \caption[]%
            {{\small Histogram of \textit{generations} needed to achieve the lower score. Here, most runs needed the maximum number of generations}}    
            \label{fig:hist_gen}
        \end{subfigure}
        \caption[]
        {\small Histograms of the lowest score and generations needed to achieve that on each experiment run.} 
        \label{fig:hist_metrics}
\end{figure*}

Figures \ref{fig:hist_beh} and \ref{fig:hist_per} reflect the number of Behavioral and Perceptual Codelets respectively to achieve the best result in each run. As we can see in Fig. \ref{fig:hist_beh}, most runs needed 13 Behavioral Codelets, one for each Motor Codelet/actuation device. Fig \ref{fig:corr_behavior} shows the correlation between the number of behavioral codelets and score. The system response is better (lower score) as more Behavioral Codelets are used.
The number of Perceptual Codelets, on the other hand, shows approximate normal distributions with a slight bias to the right, meaning that the embedding may vary and the output still be good. This bias is reflected in the slight negative correlation between the number of Perceptual Codelets and Score (smaller than Behavioral).

\begin{figure*}
        \centering
        \begin{subfigure}[b]{0.475\textwidth}
            \centering
            \includegraphics[width=0.8\textwidth]{imgs/results/results_2/histogram_of_behaviorals_x.png}
            \caption[]%
            {{\small Histogram of the number of Behavioral Codelets needed to achieve the lowest score. Most runs needed 13, the same number of Motor Codelets (and actuators).}}    
            \label{fig:hist_beh}
        \end{subfigure}
        \hfill
        \begin{subfigure}[b]{0.475\textwidth}  
            \centering 
            \includegraphics[width=0.8\textwidth]{imgs/results/results_2/histogram_of_perceptuals_x.png}
            \caption[]%
            {{\small Histogram of the number of Perceptual Codelets needed to achieve the lowest score.}}    
            \label{fig:hist_per}
        \end{subfigure}
        \caption[]
        {\small Histogram of the number of ''internal'' Codelets needed to achieve the lowest score on each run.} 
        \label{fig:needed_codelets_2}
\end{figure*}


\begin{figure*}
        \centering
        \begin{subfigure}[b]{0.475\textwidth}
            \centering
            \includegraphics[width=0.8\textwidth]{imgs/results/results_2/correlation_correlation_of_score_with_behaviorals.png}
            \caption[]%
            {{\small Correlation between number of Behavioral Codelets and Score}}    
            \label{fig:corr_behavior}
        \end{subfigure}
        \hfill
        \begin{subfigure}[b]{0.475\textwidth}  
            \centering 
            \includegraphics[width=0.8\textwidth]{imgs/results/results_2/correlation_correlation_of_score_with_perceptuals.png}
            \caption[]%
            {{\small Correlation between number of Perceptual Codelets and Score}}    
            \label{fig:corr_per}
        \end{subfigure}
        \caption[]
        {\small Correlation between the number of ''internal'' Codelets and Score} 
        \label{fig:corr}
\end{figure*}



Table \ref{table:exp2} shows some statistics taken from the experiment. Notice that, while the individual number of Perceptual and Behavioral Codelets may go as low as 2, the combined ''Internal'' Codelets need a higher number to present satisfactory results.

\begin{table}[]
\centering
\caption{metrics on Experiments}
\label{table:exp2}
\begin{tabular}{@{}llllll@{}}
\toprule
              & mean    & median & std   & min & max \\ \midrule
score         & 1.438   & 1.0    & 1.894 & 0   & 26  \\
generations   & 13.6784 & 20.0   & 8.905 & 0   & 20  \\
n perceptuals & 9.5326  & 10.0   & 2.213 & 2   & 15  \\
n behaviorals & 11.2674 & 13.0   & 2.478 & 2   & 13  \\
n internals   & 20.8    & 22.0   & 3.998 & 7   & 28  \\ \bottomrule
\end{tabular}
\end{table}

\subsection{Conclusion and Future Works}

This paper has presented a pioneering approach to creating a Cognitive Twin by leveraging a distributed cognitive system in conjunction with an evolution strategy. Our work stands as a significant contribution to the field of cognitive computing by demonstrating the feasibility of orchestrating a multitude of simple physical and virtual devices to mimic a person's interaction behaviors. This achievement not only offers a practical application of distributed cognitive systems but also introduces a novel methodology for cognitive twin development, emphasizing the role of evolution strategies in optimizing system topology for more accurate behavior emulation.

In revisiting the themes introduced at the outset, our research seamlessly integrates the foundational principles of cognitive systems, Cyber-Physical Systems (CPS), and Systems of Systems (SoS) with contemporary advancements in artificial intelligence. By doing so, we have illustrated a comprehensive framework that not only addresses the complexities of human behavior simulation but also opens new avenues for automation, human-like agent creation, and in-depth behavioral analysis.

Comparatively, our approach distinguishes itself from established cognitive architectures such as ACT-R and SOAR, and the Standard Model of Mind, by emphasizing distributed processing and adaptability. While ACT-R and SOAR offer rich insights into cognitive processes through detailed psychological models, our model excels in harnessing distributed, interconnected devices to capture the multifaceted nature of human cognition. Similarly, the Standard Model of Mind provides a foundational framework for understanding cognitive functions. Yet, our work extends this understanding into the practical domain of CPS and distributed systems, offering a unique perspective on cognitive replication and interaction dynamics.

In conclusion, our research not only underscores the potential of distributed cognitive systems in creating sophisticated cognitive twins but also highlights the importance of evolutionary strategies in refining these systems. By drawing parallels and distinguishing our work from established cognitive architectures like ACT-R, SOAR, and the Standard Model of Mind, we contribute a novel perspective to the ongoing discourse on cognitive modeling and simulation. 



Future work will focus on further refining the distributed cognitive system and exploring its integration with other AI paradigms and models. This research sets the stage for developing more sophisticated Cognitive Twins capable of performing complex tasks with minimal human intervention. By continuing to build on this foundation, future studies can enhance the fidelity and applicability of Cognitive Twins, making them tools in the field of cognitive computing.
    

\section{Conclusion}
We introduced \methodname, an effective training framework defending against MIAs for LLMs. The extensive experiments demonstrate its robustness in protecting privacy while maintaining strong language modeling performance across various datasets and architectures. Although our study focuses on fine-tuning due to computational constraints, \methodname can be seamlessly applied to large-scale pretraining, as done in prior selective pretraining work~\cite{lin2024not}. By categorizing tokens and treating them appropriately, \methodname opens a novel pathway for MIA defense. Future work can explore improved token selection strategies and multi-objective training approaches.

\bibliographystyle{formats/ACM-Reference-Format}
\bibliography{neural} 

\begin{figure*}[htbp]
\vspace{-5pt}
\centering
  \includegraphics[width=0.99\linewidth]{images/gallery1.png}
\vspace{-5pt}
\caption{The gallery of DeepMill prediction results. On the left are CAD shapes, and on the right are freeform shapes. For each row of shapes, the first and third columns show the inaccessible and occlusion regions predicted by DeepMill. In the second and fourth columns, darker shades represent under-predicted areas, while lighter shades indicate over-predicted areas.}
\label{fig:gallery1}    
\end{figure*}

\begin{figure*}[htbp]
% \vspace{-5pt}
\centering
  \includegraphics[width=0.99\linewidth]{images/gallery2.png}
\vspace{-5pt}
\caption{Demonstration of DeepMill prediction results with extreme size of cutter. After adding the dataset generated with extreme cutters to the training set, DeepMill was able to extrapolate its prediction capability to cases involving extreme cutters.}
\label{fig:gallery2}    
\end{figure*}

\begin{figure*}[htbp]
%\vspace{-5pt}
\centering
  \includegraphics[width=0.99\linewidth]{images/Complex-models.png}
	% \vspace{-15pt}
\vspace{-5pt}
\caption{Testing of DeepMill on complex shapes. For each mesh, the left side shows the prediction results from DeepMill, while the right side displays the differences compared to the geometric method.}
\label{fig:Complex-models}    
\end{figure*}

\begin{figure*}[htbp]
\centering
  \includegraphics[width=0.99\linewidth]{images/Comparison-cutter-module.png}
	% \vspace{-15pt}
\vspace{-5pt}
\caption{Comparison of cutter module concatenation methods. The left and right show the prediction accuracy of inaccessible points and the F1 score of occlusion regions for different concatenation methods on the same test set. Our approach performs the best in both measures.}
\label{fig:Comparison-cutter-module}    
\end{figure*}


\begin{figure}[htbp]
\vspace{-5pt}
\centering
  \includegraphics[width=1.0\linewidth]{images/tool.png}
% \vspace{-10pt}
\vspace{-20pt}
\caption{Illustration of the effect of cutter length on inaccessible regions. Generally, longer cutters lead to fewer inaccessible regions.}
\label{fig:different-cutter}    
\end{figure}


\begin{figure}[htbp]
\vspace{-5pt}
\centering
  \includegraphics[width=1.0\linewidth]{images/line.png}
	% \vspace{-15pt}
\vspace{-20pt}
\caption{Comparison with SAGE. DeepMill shows significantly better prediction capabilities for inaccessible and occlusion regions compared to SAGE.}
\label{fig:Ablation-study}    
\end{figure}


\begin{figure}[htbp]
\vspace{-5pt}
\centering
  \includegraphics[width=1.0\linewidth]{images/symmetry.png}
\leftline{ \footnotesize  \hspace{0.11\linewidth}
            (a)  \hspace{0.18\linewidth}
            (b)   \hspace{0.205\linewidth}
            (c)  \hspace{0.18\linewidth}
            (d)  }
% \vspace{-15pt}
\vspace{-15pt}
\caption{Geometric symmetry illustration. (a) Non-axisymmetric cutter sampling causes asymmetric inaccessible regions (b). (c) Axisymmetric method has uneven distribution. (d) DeepMill combines both, yielding more symmetrical inaccessible regions.}
% \caption{Illustration of geometric symmetry. (a) The non-axisymmetric cutter direction uniform sampling method we used, causing inaccessible regions of geometrically symmetric shapes to lack symmetry (b). (c) The axisymmetric sampling method, however, has uneven distribution. (d) DeepMill combines the advantages of both, resulting in a more symmetrical distribution of inaccessible regions for symmetric geometries.}
\label{fig:symmetry}    
\end{figure}


\begin{figure}[htbp]
\centering
  \includegraphics[width=1.0\linewidth]{images/volume-accessibility.png}
	% \vspace{-15pt}
\vspace{-7pt}
\caption{Illustration of accessibility analysis within the volume. The red points represent inaccessible sampling points. On the top, the results predicted by DeepMill are shown, and on the bottom, the results obtained by the geometric method are displayed.}
\label{fig:volume-accessibility}    
\end{figure}





\clearpage











\end{document}
