%outline of paper
%--------------------outline------------------------%
%https://docs.qq.com/doc/DS1Z1VkRoS1dQYk5X
%------------------------------------------------%

%Todo of NeuralSubFab
%--------------------todo------------------------%
%https://docs.qq.com/sheet/DQkljcWJORWJWYm94?tab=BB08J2
%--------------------todo------------------------%



%\documentclass[acmtog]{formats/acmart} 
%\documentclass[acmtog]{formats/acmart}

% \documentclass[acmtog,anonymous,review]{acmart}
\documentclass[acmtog]{acmart}
 % \acmSubmissionID{512}

\usepackage{booktabs} % For formal tables
\citestyle{acmauthoryear}
\setcitestyle{square}
\usepackage{ifthen}
\usepackage{enumitem}
\usepackage{hyperref}
\usepackage{algorithm}
%\usepackage{algorithmicx}
%\usepackage{algcompatible}
\usepackage[noend]{algpseudocode}
\usepackage{hyperxmp}
%\usepackage{algorithm2e}
\usepackage{syntax}
\usepackage{amsfonts}
\usepackage{listings}
\usepackage{fancyvrb}
\usepackage{wrapfig}
\usepackage{graphicx}
\usepackage{subfigure}
\usepackage{xspace}
\usepackage{colortbl}
\usepackage{gensymb}
\usepackage{verbatim}
\usepackage{colortbl}
\usepackage{booktabs}
\usepackage{multirow}
\usepackage{cleveref}
\usepackage{pifont}
\usepackage{circledsteps}
% \usepackage{colortbl} % 支持表格着色
% \usepackage{xcolor}   % 提供更多颜色选项

\algdef{SE}[DOWHILE]{Do}{doWhile}{\algorithmicdo}[1]{\algorithmicwhile\ #1}%

\definecolor{mygreen}{rgb}{0,0.6,0}                         
\definecolor{mygray}{rgb}{0.95,0.95,0.95}
\definecolor{codebg}{rgb}{0.95, 0.95, 0.95}  
\lstset{                                                                        
  language=Caml,                                                                
  basicstyle=\tiny\ttfamily,                                                   
  frame=single,                                                                 
  numberstyle=\tiny\color{mygray},                                              
  backgroundcolor=\color{codebg},                                              
  numbers=left,                                                                 
  escapeinside={/*}{*/},                                                        
  morekeywords={Setup_Tracksaw, Tracksaw, Setup_Chopsaw, Chopsaw, Setup_Drill, Drill, Return, Box, Make_Stock, Make_Sketch, Query, Support, Constraint, Geometry, Line, PointOnObject, Make_Hole, Make_Cut, Ref},
  tabsize=1,                                                                    
  keywordstyle=\color{blue},                                                    
  numbersep=5pt,                                                                
  rulesep=0pt,                                                                  
  framexleftmargin=2mm                                                          
}  
 
%\lstset{style=mystyle}
\newcommand{\etal}{et al.}

\newcommand{\llu}[1]{{\color{red} LL: #1}}
\newcommand{\hs}[1]{{\color{cyan} HZ: #1}}
\newcommand{\fc}[1]{{\color{brown} FC: #1}}
\newcommand{\lhc}[1]{{\color{olive} HC: #1}}
\newcommand{\yx}[1]{{\color{red} YX: #1}}
\newcommand{\zm}[1]{{\color{orange} #1}}
\newcommand{\jw}[1]{{\color{green}#1}}
\newcommand{\jwc}[1]{{\color{mygreen}[JW: #1]}}

% 将注释隐藏
% \renewcommand{\fc}[1]{}
\let\Bbbk\relax
\usepackage{amsthm,amsmath,amssymb}
\usepackage{mathrsfs}
\usepackage[normalem]{ulem}
% switch comments to eliminate revision tag
\newcommand{\revision}[1]{\textcolor{blue}{#1}}
\newcommand{\revisiontodo}[1]{\textcolor{red}{#1}}
\newcommand{\revisionII}[1]{\textcolor{black}{#1}}
\newcommand{\revisionIII}[1]{\textcolor{black}{#1}}

%\newcommand{\revisiontodo}[2]{#2}

\newcommand{\zdiff}[2]{{\textbf{\color{blue}{#1}} \color{orange}{(suggested to replace ``\textit{#2}'')}}}

\newcommand{\todo}[1]{{\color{blue}{#1}}}
\newcommand{\old}[1]{{\color{red}{#1}}}
\newcommand{\inchsign}{^{\prime\prime}}  
\renewcommand{\grammarlabel}[2]{#1\hfill#2} 
\definecolor{mypink}{rgb}{.99,.91,.95}


\begin{document}

\setcopyright{acmlicensed}
\acmJournal{TOG}
%\acmYear{XXX} \acmVolume{XX} \acmNumber{XX} \acmArticle{XX} \acmMonth{XX} %\acmPrice{15.00}\acmDOI{10.1145/3618324}

\title{DeepMill: Neural Accessibility Learning for Subtractive Manufacturing}

% DeepMill: Accessibility Learning for Subtractive Manufacturing

%“DeepMill: Learning Accessibility for Complex Subtractive Manufacturing Geometries”
%DeepMill: Neural Accessibility Learning for Subtractive Manufacturing”
%DeepMill: Accessibility Learning for Multi-Axis Subtractive Manufacturing”

%“DeepMill: Neural Learning Accessibility for Complex Multi-Axis Subtractive Manufacturing Geometries”


%DeepMill: Neural Accessibility Learning for Subtractive Manufacturing
%DeepMill: Neural Accessibility Learning
%Neural Subtractive Manufacturability Analysis

\author{Fanchao Zhong}
\authornote{Equal contribution}
\email{fanchaoz98@gmail.com}
\affiliation{%
  \institution{Shandong University}
  \city{Qingdao}
  \country{China}
}

\author{Yang Wang}
\authornotemark[1]
\email{1766897491wy@gmail.com}
\affiliation{%
  \institution{Shandong University}
  \city{Qingdao}
  \country{China}
}

\author{Peng-Shuai Wang}
\email{wangps@hotmail.com}
\affiliation{%
  \institution{Peking University}
  \city{Peking}
  \country{China}
}

\author{Lin Lu}
\email{lulin.linda@gmail.com}
\affiliation{%
  \institution{Shandong University}
  \city{Qingdao}
  \country{China}
}

\author{Haisen Zhao}
\authornote{corresponding author}
\email{haisenzhao@sdu.edu.cn}
\affiliation{%
  \institution{Shandong University}
  \city{Qingdao}
  \country{China}
}

%The analysis of accessibility in subtractive manufacturing is crucial as it directly affects manufacturing efficiency and costs. Ignoring or misjudging accessibility can create non-manufacturable areas in workpieces, leading to expensive rework and longer production times. Our study addresses non-manufacturability stemming from tool inaccessibility by introducing a deep-learning network designed to predict non-manufacturable regions in arbitrary models with accuracy and in real-time. It can also detect occluded regions for necessary design changes. Our network is well-suited for diverse tool sizes and intricate geometries. To overcome challenges such as tool collisions and limited data, we employ cutter-aware octree-based neural network. Experiments indicate our network achieves 96.3$\%$ accuracy and 90.0$\%$ F1 score in identifying non-manufacturable and occluded regions while only 0.17$\%$ of the original time is needed for relatively complex geometries. In comparison to traditional techniques, it enhances efficiency, retains accuracy, and adapts well to both freeform and CAD models, with robust tool-size generalization.

%

\begin{abstract}
Manufacturability is vital for product design and production, with accessibility being a key element, especially in subtractive manufacturing. 
Traditional methods for geometric accessibility analysis are time-consuming and struggle with scalability, while existing deep learning approaches in manufacturability analysis often neglect geometric challenges in accessibility and are limited to specific model types.
In this paper, we introduce DeepMill, the first neural framework designed to accurately and efficiently predict inaccessible and occlusion regions under varying machining tool parameters, applicable to both CAD and freeform models.
To address the challenges posed by cutter collisions and the lack of extensive training datasets, we construct a cutter-aware dual-head octree-based convolutional neural network (O-CNN) and generate an inaccessible and occlusion regions analysis dataset with a variety of cutter sizes for network training.
Experiments demonstrate that DeepMill achieves 94.7\% accuracy in predicting inaccessible regions and 88.7\% accuracy in identifying occlusion regions, with an average processing time of 0.04 seconds for complex geometries.
Based on the outcomes, DeepMill implicitly captures both local and global geometric features, as well as the complex interactions between cutters and intricate 3D models.
\end{abstract}


\ccsdesc[500]{Computing methodologies~Shape modeling}
\ccsdesc[300]{Computing methodologies~Graphics systems and interfaces}

\acmJournal{TOG}

\keywords{Subtractive manufacturing, Accessibility Analysis, Manufacturability Analysis}

\begin{teaserfigure}
%\vspace{-5pt}
  \includegraphics[width=1.0\linewidth]{images/teaser.jpg}
	\vspace{-15pt}
\caption {Demonstration of the human-robot interaction process in SymbioSim. In this system, human interacts with a virtual robot in 3D space through AR. The robot perceives human actions and generates real-time responses. Feedback from the human experience is continuously collected to optimize the robot’s model, improving its performance. Concurrently, humans gradually develop a deeper understanding and trust in the robot. Ultimately, SymbioSim fosters bidirectional learning and adaptation, with the potential to promote human-robot symbiosis.}
  \label{fig:teaser}    
\end{teaserfigure}


%
\setlength\unitlength{1mm}
\newcommand{\twodots}{\mathinner {\ldotp \ldotp}}
% bb font symbols
\newcommand{\Rho}{\mathrm{P}}
\newcommand{\Tau}{\mathrm{T}}

\newfont{\bbb}{msbm10 scaled 700}
\newcommand{\CCC}{\mbox{\bbb C}}

\newfont{\bb}{msbm10 scaled 1100}
\newcommand{\CC}{\mbox{\bb C}}
\newcommand{\PP}{\mbox{\bb P}}
\newcommand{\RR}{\mbox{\bb R}}
\newcommand{\QQ}{\mbox{\bb Q}}
\newcommand{\ZZ}{\mbox{\bb Z}}
\newcommand{\FF}{\mbox{\bb F}}
\newcommand{\GG}{\mbox{\bb G}}
\newcommand{\EE}{\mbox{\bb E}}
\newcommand{\NN}{\mbox{\bb N}}
\newcommand{\KK}{\mbox{\bb K}}
\newcommand{\HH}{\mbox{\bb H}}
\newcommand{\SSS}{\mbox{\bb S}}
\newcommand{\UU}{\mbox{\bb U}}
\newcommand{\VV}{\mbox{\bb V}}


\newcommand{\yy}{\mathbbm{y}}
\newcommand{\xx}{\mathbbm{x}}
\newcommand{\zz}{\mathbbm{z}}
\newcommand{\sss}{\mathbbm{s}}
\newcommand{\rr}{\mathbbm{r}}
\newcommand{\pp}{\mathbbm{p}}
\newcommand{\qq}{\mathbbm{q}}
\newcommand{\ww}{\mathbbm{w}}
\newcommand{\hh}{\mathbbm{h}}
\newcommand{\vvv}{\mathbbm{v}}

% Vectors

\newcommand{\av}{{\bf a}}
\newcommand{\bv}{{\bf b}}
\newcommand{\cv}{{\bf c}}
\newcommand{\dv}{{\bf d}}
\newcommand{\ev}{{\bf e}}
\newcommand{\fv}{{\bf f}}
\newcommand{\gv}{{\bf g}}
\newcommand{\hv}{{\bf h}}
\newcommand{\iv}{{\bf i}}
\newcommand{\jv}{{\bf j}}
\newcommand{\kv}{{\bf k}}
\newcommand{\lv}{{\bf l}}
\newcommand{\mv}{{\bf m}}
\newcommand{\nv}{{\bf n}}
\newcommand{\ov}{{\bf o}}
\newcommand{\pv}{{\bf p}}
\newcommand{\qv}{{\bf q}}
\newcommand{\rv}{{\bf r}}
\newcommand{\sv}{{\bf s}}
\newcommand{\tv}{{\bf t}}
\newcommand{\uv}{{\bf u}}
\newcommand{\wv}{{\bf w}}
\newcommand{\vv}{{\bf v}}
\newcommand{\xv}{{\bf x}}
\newcommand{\yv}{{\bf y}}
\newcommand{\zv}{{\bf z}}
\newcommand{\zerov}{{\bf 0}}
\newcommand{\onev}{{\bf 1}}

% Matrices

\newcommand{\Am}{{\bf A}}
\newcommand{\Bm}{{\bf B}}
\newcommand{\Cm}{{\bf C}}
\newcommand{\Dm}{{\bf D}}
\newcommand{\Em}{{\bf E}}
\newcommand{\Fm}{{\bf F}}
\newcommand{\Gm}{{\bf G}}
\newcommand{\Hm}{{\bf H}}
\newcommand{\Id}{{\bf I}}
\newcommand{\Jm}{{\bf J}}
\newcommand{\Km}{{\bf K}}
\newcommand{\Lm}{{\bf L}}
\newcommand{\Mm}{{\bf M}}
\newcommand{\Nm}{{\bf N}}
\newcommand{\Om}{{\bf O}}
\newcommand{\Pm}{{\bf P}}
\newcommand{\Qm}{{\bf Q}}
\newcommand{\Rm}{{\bf R}}
\newcommand{\Sm}{{\bf S}}
\newcommand{\Tm}{{\bf T}}
\newcommand{\Um}{{\bf U}}
\newcommand{\Wm}{{\bf W}}
\newcommand{\Vm}{{\bf V}}
\newcommand{\Xm}{{\bf X}}
\newcommand{\Ym}{{\bf Y}}
\newcommand{\Zm}{{\bf Z}}

% Calligraphic

\newcommand{\Ac}{{\cal A}}
\newcommand{\Bc}{{\cal B}}
\newcommand{\Cc}{{\cal C}}
\newcommand{\Dc}{{\cal D}}
\newcommand{\Ec}{{\cal E}}
\newcommand{\Fc}{{\cal F}}
\newcommand{\Gc}{{\cal G}}
\newcommand{\Hc}{{\cal H}}
\newcommand{\Ic}{{\cal I}}
\newcommand{\Jc}{{\cal J}}
\newcommand{\Kc}{{\cal K}}
\newcommand{\Lc}{{\cal L}}
\newcommand{\Mc}{{\cal M}}
\newcommand{\Nc}{{\cal N}}
\newcommand{\nc}{{\cal n}}
\newcommand{\Oc}{{\cal O}}
\newcommand{\Pc}{{\cal P}}
\newcommand{\Qc}{{\cal Q}}
\newcommand{\Rc}{{\cal R}}
\newcommand{\Sc}{{\cal S}}
\newcommand{\Tc}{{\cal T}}
\newcommand{\Uc}{{\cal U}}
\newcommand{\Wc}{{\cal W}}
\newcommand{\Vc}{{\cal V}}
\newcommand{\Xc}{{\cal X}}
\newcommand{\Yc}{{\cal Y}}
\newcommand{\Zc}{{\cal Z}}

% Bold greek letters

\newcommand{\alphav}{\hbox{\boldmath$\alpha$}}
\newcommand{\betav}{\hbox{\boldmath$\beta$}}
\newcommand{\gammav}{\hbox{\boldmath$\gamma$}}
\newcommand{\deltav}{\hbox{\boldmath$\delta$}}
\newcommand{\etav}{\hbox{\boldmath$\eta$}}
\newcommand{\lambdav}{\hbox{\boldmath$\lambda$}}
\newcommand{\epsilonv}{\hbox{\boldmath$\epsilon$}}
\newcommand{\nuv}{\hbox{\boldmath$\nu$}}
\newcommand{\muv}{\hbox{\boldmath$\mu$}}
\newcommand{\zetav}{\hbox{\boldmath$\zeta$}}
\newcommand{\phiv}{\hbox{\boldmath$\phi$}}
\newcommand{\psiv}{\hbox{\boldmath$\psi$}}
\newcommand{\thetav}{\hbox{\boldmath$\theta$}}
\newcommand{\tauv}{\hbox{\boldmath$\tau$}}
\newcommand{\omegav}{\hbox{\boldmath$\omega$}}
\newcommand{\xiv}{\hbox{\boldmath$\xi$}}
\newcommand{\sigmav}{\hbox{\boldmath$\sigma$}}
\newcommand{\piv}{\hbox{\boldmath$\pi$}}
\newcommand{\rhov}{\hbox{\boldmath$\rho$}}
\newcommand{\upsilonv}{\hbox{\boldmath$\upsilon$}}

\newcommand{\Gammam}{\hbox{\boldmath$\Gamma$}}
\newcommand{\Lambdam}{\hbox{\boldmath$\Lambda$}}
\newcommand{\Deltam}{\hbox{\boldmath$\Delta$}}
\newcommand{\Sigmam}{\hbox{\boldmath$\Sigma$}}
\newcommand{\Phim}{\hbox{\boldmath$\Phi$}}
\newcommand{\Pim}{\hbox{\boldmath$\Pi$}}
\newcommand{\Psim}{\hbox{\boldmath$\Psi$}}
\newcommand{\Thetam}{\hbox{\boldmath$\Theta$}}
\newcommand{\Omegam}{\hbox{\boldmath$\Omega$}}
\newcommand{\Xim}{\hbox{\boldmath$\Xi$}}


% Sans Serif small case

\newcommand{\Gsf}{{\sf G}}

\newcommand{\asf}{{\sf a}}
\newcommand{\bsf}{{\sf b}}
\newcommand{\csf}{{\sf c}}
\newcommand{\dsf}{{\sf d}}
\newcommand{\esf}{{\sf e}}
\newcommand{\fsf}{{\sf f}}
\newcommand{\gsf}{{\sf g}}
\newcommand{\hsf}{{\sf h}}
\newcommand{\isf}{{\sf i}}
\newcommand{\jsf}{{\sf j}}
\newcommand{\ksf}{{\sf k}}
\newcommand{\lsf}{{\sf l}}
\newcommand{\msf}{{\sf m}}
\newcommand{\nsf}{{\sf n}}
\newcommand{\osf}{{\sf o}}
\newcommand{\psf}{{\sf p}}
\newcommand{\qsf}{{\sf q}}
\newcommand{\rsf}{{\sf r}}
\newcommand{\ssf}{{\sf s}}
\newcommand{\tsf}{{\sf t}}
\newcommand{\usf}{{\sf u}}
\newcommand{\wsf}{{\sf w}}
\newcommand{\vsf}{{\sf v}}
\newcommand{\xsf}{{\sf x}}
\newcommand{\ysf}{{\sf y}}
\newcommand{\zsf}{{\sf z}}


% mixed symbols

\newcommand{\sinc}{{\hbox{sinc}}}
\newcommand{\diag}{{\hbox{diag}}}
\renewcommand{\det}{{\hbox{det}}}
\newcommand{\trace}{{\hbox{tr}}}
\newcommand{\sign}{{\hbox{sign}}}
\renewcommand{\arg}{{\hbox{arg}}}
\newcommand{\var}{{\hbox{var}}}
\newcommand{\cov}{{\hbox{cov}}}
\newcommand{\Ei}{{\rm E}_{\rm i}}
\renewcommand{\Re}{{\rm Re}}
\renewcommand{\Im}{{\rm Im}}
\newcommand{\eqdef}{\stackrel{\Delta}{=}}
\newcommand{\defines}{{\,\,\stackrel{\scriptscriptstyle \bigtriangleup}{=}\,\,}}
\newcommand{\<}{\left\langle}
\renewcommand{\>}{\right\rangle}
\newcommand{\herm}{{\sf H}}
\newcommand{\trasp}{{\sf T}}
\newcommand{\transp}{{\sf T}}
\renewcommand{\vec}{{\rm vec}}
\newcommand{\Psf}{{\sf P}}
\newcommand{\SINR}{{\sf SINR}}
\newcommand{\SNR}{{\sf SNR}}
\newcommand{\MMSE}{{\sf MMSE}}
\newcommand{\REF}{{\RED [REF]}}

% Markov chain
\usepackage{stmaryrd} % for \mkv 
\newcommand{\mkv}{-\!\!\!\!\minuso\!\!\!\!-}

% Colors

\newcommand{\RED}{\color[rgb]{1.00,0.10,0.10}}
\newcommand{\BLUE}{\color[rgb]{0,0,0.90}}
\newcommand{\GREEN}{\color[rgb]{0,0.80,0.20}}

%%%%%%%%%%%%%%%%%%%%%%%%%%%%%%%%%%%%%%%%%%
\usepackage{hyperref}
\hypersetup{
    bookmarks=true,         % show bookmarks bar?
    unicode=false,          % non-Latin characters in AcrobatÕs bookmarks
    pdftoolbar=true,        % show AcrobatÕs toolbar?
    pdfmenubar=true,        % show AcrobatÕs menu?
    pdffitwindow=false,     % window fit to page when opened
    pdfstartview={FitH},    % fits the width of the page to the window
%    pdftitle={My title},    % title
%    pdfauthor={Author},     % author
%    pdfsubject={Subject},   % subject of the document
%    pdfcreator={Creator},   % creator of the document
%    pdfproducer={Producer}, % producer of the document
%    pdfkeywords={keyword1} {key2} {key3}, % list of keywords
    pdfnewwindow=true,      % links in new window
    colorlinks=true,       % false: boxed links; true: colored links
    linkcolor=red,          % color of internal links (change box color with linkbordercolor)
    citecolor=green,        % color of links to bibliography
    filecolor=blue,      % color of file links
    urlcolor=blue           % color of external links
}
%%%%%%%%%%%%%%%%%%%%%%%%%%%%%%%%%%%%%%%%%%%


\maketitle


%!TEX root = gcn.tex
\section{Introduction}
Graphs, representing structural data and topology, are widely used across various domains, such as social networks and merchandising transactions.
Graph convolutional networks (GCN)~\cite{iclr/KipfW17} have significantly enhanced model training on these interconnected nodes.
However, these graphs often contain sensitive information that should not be leaked to untrusted parties.
For example, companies may analyze sensitive demographic and behavioral data about users for applications ranging from targeted advertising to personalized medicine.
Given the data-centric nature and analytical power of GCN training, addressing these privacy concerns is imperative.

Secure multi-party computation (MPC)~\cite{crypto/ChaumDG87,crypto/ChenC06,eurocrypt/CiampiRSW22} is a critical tool for privacy-preserving machine learning, enabling mutually distrustful parties to collaboratively train models with privacy protection over inputs and (intermediate) computations.
While research advances (\eg,~\cite{ccs/RatheeRKCGRS20,uss/NgC21,sp21/TanKTW,uss/WatsonWP22,icml/Keller022,ccs/ABY318,folkerts2023redsec}) support secure training on convolutional neural networks (CNNs) efficiently, private GCN training with MPC over graphs remains challenging.

Graph convolutional layers in GCNs involve multiplications with a (normalized) adjacency matrix containing $\numedge$ non-zero values in a $\numnode \times \numnode$ matrix for a graph with $\numnode$ nodes and $\numedge$ edges.
The graphs are typically sparse but large.
One could use the standard Beaver-triple-based protocol to securely perform these sparse matrix multiplications by treating graph convolution as ordinary dense matrix multiplication.
However, this approach incurs $O(\numnode^2)$ communication and memory costs due to computations on irrelevant nodes.
%
Integrating existing cryptographic advances, the initial effort of SecGNN~\cite{tsc/WangZJ23,nips/RanXLWQW23} requires heavy communication or computational overhead.
Recently, CoGNN~\cite{ccs/ZouLSLXX24} optimizes the overhead in terms of  horizontal data partitioning, proposing a semi-honest secure framework.
Research for secure GCN over vertical data  remains nascent.

Current MPC studies, for GCN or not, have primarily targeted settings where participants own different data samples, \ie, horizontally partitioned data~\cite{ccs/ZouLSLXX24}.
MPC specialized for scenarios where parties hold different types of features~\cite{tkde/LiuKZPHYOZY24,icml/CastigliaZ0KBP23,nips/Wang0ZLWL23} is rare.
This paper studies $2$-party secure GCN training for these vertical partition cases, where one party holds private graph topology (\eg, edges) while the other owns private node features.
For instance, LinkedIn holds private social relationships between users, while banks own users' private bank statements.
Such real-world graph structures underpin the relevance of our focus.
To our knowledge, no prior work tackles secure GCN training in this context, which is crucial for cross-silo collaboration.


To realize secure GCN over vertically split data, we tailor MPC protocols for sparse graph convolution, which fundamentally involves sparse (adjacency) matrix multiplication.
Recent studies have begun exploring MPC protocols for sparse matrix multiplication (SMM).
ROOM~\cite{ccs/SchoppmannG0P19}, a seminal work on SMM, requires foreknowledge of sparsity types: whether the input matrices are row-sparse or column-sparse.
Unfortunately, GCN typically trains on graphs with arbitrary sparsity, where nodes have varying degrees and no specific sparsity constraints.
Moreover, the adjacency matrix in GCN often contains a self-loop operation represented by adding the identity matrix, which is neither row- nor column-sparse.
Araki~\etal~\cite{ccs/Araki0OPRT21} avoid this limitation in their scalable, secure graph analysis work, yet it does not cover vertical partition.

% and related primitives
To bridge this gap, we propose a secure sparse matrix multiplication protocol, \osmm, achieving \emph{accurate, efficient, and secure GCN training over vertical data} for the first time.

\subsection{New Techniques for Sparse Matrices}
The cost of evaluating a GCN layer is dominated by SMM in the form of $\adjmat\feamat$, where $\adjmat$ is a sparse adjacency matrix of a (directed) graph $\graph$ and $\feamat$ is a dense matrix of node features.
For unrelated nodes, which often constitute a substantial portion, the element-wise products $0\cdot x$ are always zero.
Our efficient MPC design 
avoids unnecessary secure computation over unrelated nodes by focusing on computing non-zero results while concealing the sparse topology.
We achieve this~by:
1) decomposing the sparse matrix $\adjmat$ into a product of matrices (\S\ref{sec::sgc}), including permutation and binary diagonal matrices, that can \emph{faithfully} represent the original graph topology;
2) devising specialized protocols (\S\ref{sec::smm_protocol}) for efficiently multiplying the structured matrices while hiding sparsity topology.


 
\subsubsection{Sparse Matrix Decomposition}
We decompose adjacency matrix $\adjmat$ of $\graph$ into two bipartite graphs: one represented by sparse matrix $\adjout$, linking the out-degree nodes to edges, the other 
by sparse matrix $\adjin$,
linking edges to in-degree nodes.

%\ie, we decompose $\adjmat$ into $\adjout \adjin$, where $\adjout$ and $\adjin$ are sparse matrices representing these connections.
%linking out-degree nodes to edges and edges to in-degree nodes of $\graph$, respectively.

We then permute the columns of $\adjout$ and the rows of $\adjin$ so that the permuted matrices $\adjout'$ and $\adjin'$ have non-zero positions with \emph{monotonically non-decreasing} row and column indices.
A permutation $\sigma$ is used to preserve the edge topology, leading to an initial decomposition of $\adjmat = \adjout'\sigma \adjin'$.
This is further refined into a sequence of \emph{linear transformations}, 
which can be efficiently computed by our MPC protocols for 
\emph{oblivious permutation}
%($\Pi_{\ssp}$) 
and \emph{oblivious selection-multiplication}.
% ($\Pi_\SM$)
\iffalse
Our approach leverages bipartite graph representation and the monotonicity of non-zero positions to decompose a general sparse matrix into linear transformations, enhancing the efficiency of our MPC protocols.
\fi
Our decomposition approach is not limited to GCNs but also general~SMM 
by 
%simply 
treating them 
as adjacency matrices.
%of a graph.
%Since any sparse matrix can be viewed 

%allowing the same technique to be applied.

 
\subsubsection{New Protocols for Linear Transformations}
\emph{Oblivious permutation} (OP) is a two-party protocol taking a private permutation $\sigma$ and a private vector $\xvec$ from the two parties, respectively, and generating a secret share $\l\sigma \xvec\r$ between them.
Our OP protocol employs correlated randomnesses generated in an input-independent offline phase to mask $\sigma$ and $\xvec$ for secure computations on intermediate results, requiring only $1$ round in the online phase (\cf, $\ge 2$ in previous works~\cite{ccs/AsharovHIKNPTT22, ccs/Araki0OPRT21}).

Another crucial two-party protocol in our work is \emph{oblivious selection-multiplication} (OSM).
It takes a private bit~$s$ from a party and secret share $\l x\r$ of an arithmetic number~$x$ owned by the two parties as input and generates secret share $\l sx\r$.
%between them.
%Like our OP protocol, o
Our $1$-round OSM protocol also uses pre-computed randomnesses to mask $s$ and $x$.
%for secure computations.
Compared to the Beaver-triple-based~\cite{crypto/Beaver91a} and oblivious-transfer (OT)-based approaches~\cite{pkc/Tzeng02}, our protocol saves ${\sim}50\%$ of online communication while having the same offline communication and round complexities.

By decomposing the sparse matrix into linear transformations and applying our specialized protocols, our \osmm protocol
%($\prosmm$) 
reduces the complexity of evaluating $\numnode \times \numnode$ sparse matrices with $\numedge$ non-zero values from $O(\numnode^2)$ to $O(\numedge)$.

%(\S\ref{sec::secgcn})
\subsection{\cgnn: Secure GCN made Efficient}
Supported by our new sparsity techniques, we build \cgnn, 
a two-party computation (2PC) framework for GCN inference and training over vertical
%ly split
data.
Our contributions include:

1) We are the first to explore sparsity over vertically split, secret-shared data in MPC, enabling decompositions of sparse matrices with arbitrary sparsity and isolating computations that can be performed in plaintext without sacrificing privacy.

2) We propose two efficient $2$PC primitives for OP and OSM, both optimally single-round.
Combined with our sparse matrix decomposition approach, our \osmm protocol ($\prosmm$) achieves constant-round communication costs of $O(\numedge)$, reducing memory requirements and avoiding out-of-memory errors for large matrices.
In practice, it saves $99\%+$ communication
%(Table~\ref{table:comm_smm}) 
and reduces ${\sim}72\%$ memory usage over large $(5000\times5000)$ matrices compared with using Beaver triples.
%(Table~\ref{table:mem_smm_sparse}) ${\sim}16\%$-

3) We build an end-to-end secure GCN framework for inference and training over vertically split data, maintaining accuracy on par with plaintext computations.
We will open-source our evaluation code for research and deployment.

To evaluate the performance of $\cgnn$, we conducted extensive experiments over three standard graph datasets (Cora~\cite{aim/SenNBGGE08}, Citeseer~\cite{dl/GilesBL98}, and Pubmed~\cite{ijcnlp/DernoncourtL17}),
reporting communication, memory usage, accuracy, and running time under varying network conditions, along with an ablation study with or without \osmm.
Below, we highlight our key achievements.

\textit{Communication (\S\ref{sec::comm_compare_gcn}).}
$\cgnn$ saves communication by $50$-$80\%$.
(\cf,~CoGNN~\cite{ccs/KotiKPG24}, OblivGNN~\cite{uss/XuL0AYY24}).

\textit{Memory usage (\S\ref{sec::smmmemory}).}
\cgnn alleviates out-of-memory problems of using %the standard 
Beaver-triples~\cite{crypto/Beaver91a} for large datasets.

\textit{Accuracy (\S\ref{sec::acc_compare_gcn}).}
$\cgnn$ achieves inference and training accuracy comparable to plaintext counterparts.
%training accuracy $\{76\%$, $65.1\%$, $75.2\%\}$ comparable to $\{75.7\%$, $65.4\%$, $74.5\%\}$ in plaintext.

{\textit{Computational efficiency (\S\ref{sec::time_net}).}} 
%If the network is worse in bandwidth and better in latency, $\cgnn$ shows more benefits.
$\cgnn$ is faster by $6$-$45\%$ in inference and $28$-$95\%$ in training across various networks and excels in narrow-bandwidth and low-latency~ones.

{\textit{Impact of \osmm (\S\ref{sec:ablation}).}}
Our \osmm protocol shows a $10$-$42\times$ speed-up for $5000\times 5000$ matrices and saves $10$-2$1\%$ memory for ``small'' datasets and up to $90\%$+ for larger ones.

\section{Related Work}

\subsection{Advancements in AI and Agentic Workflows for Code Generation}

Since the training process of GPT-3.5 incorporated a substantial amount of code data to enhance the logical reasoning capabilities of language models \cite{chenEvaluatingLargeLanguage2021}, code generation has become closely intertwined with language modeling. With the emergence of models that place a stronger emphasis on reasoning, these capabilities continue to evolve. According to the SWE-bench benchmark, which simulates human programmers' problem-solving workflows, AI programming performance increased from below 2\% in December 2023 \cite{jimenez2024swebench} to over 60\% by February 2025\footnote{https://www.swebench.com/}.

However, simply reinforcing the reasoning ability of language models primarily advances lower-level software development tasks such as auto-completion and refactoring. To enhance automation in real-world software and system development, researchers have introduced various agentic workflows, including OpenHands \cite{openhands}, an open-source coding agent designed for end-to-end development, and Agent Company, which simulates the operation of a software company \cite{xu2024theagentcompany}. Nonetheless, as of February 2025, even the most sophisticated agentic workflows remain unable to fully realize end-to-end programming\footnote{https://www.swebench.com/}, let alone organization-level agency\footnote{https://the-agent-company.com/}. 

Within code generation and system development, front-end code generation—such as website development—often demonstrates stronger performance than back-end development. Research in this domain has examined reconstructing HTML/CSS structures from UI screenshots using computer vision techniques \cite{soseliaLearningUItoCodeReverse2023}, implementing hierarchical decomposition strategies for interface elements to optimize UI code generation \cite{wanAutomaticallyGeneratingUI2024}, and improving model specialization through domain-specific fine-tuning for UI generation \cite{wuUICoderFinetuningLarge2024}. To systematically evaluate front-end code generation, specialized benchmarks have been developed to assess the quality of HTML, CSS, and JavaScript implementations \cite{siDesign2CodeHowFar2024}. To investigate the societal impact of this notable improvement in AI programming capabilities, we focus on the task of website generation, where current AI systems are relatively close to achieving near end-to-end automation.

\subsection{Beyond Templates: AI-Powered, User-Centric UI}

With the continuing development of AI-driven user interface (UI) generation, users increasingly seek more personalized and diverse expressions rather than relying solely on conventional template reuse. Recent advances have led to adaptive UI generation systems like FrameKit, which allows end users to manually design keyframes and generate multiple interface variants \cite{wu_framekit_2024}. PromptInfuser goes a step further by enabling runtime dynamic input and generation of UI content \cite{petridisPromptInfuserHowTightly2024}. In this context, AI tools have been shown to offer inspiration for professional designers \cite{luBridgingGapUX2022}. For instance, DesignAID \cite{cai_designaid_2023} demonstrates that generative AI can provide conceptual directions and stimulate creativity at early design stages. Misty supports remixing concepts by allowing users to blend example images with the current UI, thereby enabling flexible conceptual exploration \cite{luMistyUIPrototyping2024}.

Beyond offering inspiration, AI can also provide real-time design feedback to guide iterative refinement and error correction \cite{duan_towards_2023}, such as handling CSS styling in simple websites and optimizing specific UI components \cite{liUsingLLMsCustomize2023}. It is capable of evaluating UI quality and relevance, offering suggestions at various design stages \cite{wuUIClipDatadrivenModel2024}, and even detecting potential development or UI issues in advance \cite{petridisPromptInfuserHowTightly2024}. Automated heuristic evaluations generated by AI can provide more precise assessments and recommendations, thereby streamlining the iterative process \cite{duanGeneratingAutomaticFeedback2024}. When combined with traditional heuristic rules, AI has been shown to increase the effectiveness of UI error detection and correction \cite{lu_ai_2024}. Integrating prototype-checking techniques into the UI generation workflow can further enhance automatic repair capabilities \cite{xiaoPrototype2CodeEndtoendFrontend2024}.

\subsection{Improving the Creative Workflow with AI}

In many creativity workflows, a prolonged progression from ideation, prototyping, and development to iteration is required \cite{palaniEvolvingRolesWorkflows2024}. Those creative processes are frequently constrained by multiple intricate steps that limit users' expressive capabilities. For example, the complexity and associated costs of developing a personal website often deter individuals from undertaking this process, prompting many to resort to standardized website templates for personal websites. However, GenAI can assist with the creativity workflow from various angles \cite{wanItFeltHaving2024,palaniEvolvingRolesWorkflows2024,longNotJustNovelty2024}. First, GenAI such as text-to-image generation can reduce the time needed to produce high-fidelity outcomes. This enables creators to focus on refining the gap between the high-fidelity results and their envisioned expectations, rather than expending effort on how to achieve high fidelity in the first place \cite{edwardsSketch2PrototypeRapidConceptual2024}. Besides, AI lowers the cost of experimenting with new ideas, thereby minimizing the psychological barriers to conducting trial and error with unconventional concepts \cite{palaniEvolvingRolesWorkflows2024}. When users are uncertain about what they want or have only a broad concept lacking specific details, AI can offer inspiration \cite{rickSupermindIdeatorExploring2023}. Moreover, AI can facilitate parallel prototyping by presenting multiple design directions simultaneously, allowing creators to compare and refine a range of diverse design solutions \cite{dowParallelPrototypingLeads2010}.


% \vspace{-1em}
\section{Design Overview}
\label{sec:overview}
% \vspace{-0.2em}

In this section, we first present our new HBD architecture \sys{} guided by the design principles outlined above. We then provide an overview of its key components.


\begin{figure*}[ht]
    \centering
    \begin{subfigure}[b]{0.45\textwidth}
        \centering
        \includegraphics[height=80pt]{figs/design/transceiver.pdf}
        \caption{Components of OCS transceivers.}
        \label{figure:design:transceiver:component}
    \end{subfigure}
    \hspace{10pt}
    \begin{subfigure}[b]{0.45\textwidth}
        \centering
        \includegraphics[height=80pt]{figs/design/phase-shifter-v2.pdf}
        \caption{Zoom into OCS MZI switch matrix.}
        \label{figure:design:transceiver:ocs}
    \end{subfigure}
    \vspace{-10pt}
    \caption{Design of OCS Transceivers. The core component is OCS integrated in transceivers.}
    \label{figure:design:transceiver}
    \vspace{-15pt}
\end{figure*}

\para{Transceiver-centric HBD architecture}.
As discussed in \S\ref{sec:background:hbd} and summarized in \tabref{tab:hbd-compare}, existing architectures face a fundamental tradeoff among scalability, cost, and fault isolation. The GPU-centric architecture offers high scalability and low cost connectivity but suffers from a large fault explosion radius. In contrast, the switch-centric architecture improves fault isolation by leveraging centralized switches to confine failures to the node level. However, this comes at the cost of reduced scalability and higher connection overhead. The GPU-switch hybrid architecture takes a middle-ground approach but still suffers from significant fault propagation. As a result, no existing architecture fully meets all requirements.

Our key insight is that \textit{connectivity and dynamic switching can be unified at the transceiver level} using Optical Circuit Switching (OCS). By embedding OCS within each transceiver, we can enable reconfigurable point-to-multipoint connectivity, effectively combining both connectivity and switching at the optical layer. This represents a fundamental departure from conventional designs, where transceivers are limited to static point-to-point links and rely on high-radix switches for dynamic switching. We refer to this novel design as the \textit{transceiver-centric HBD architecture}. 

We realize this design with \sys{}, which has three key components as shown in \figref{fig:overview}.


\para{Design 1: Silicon Photonics based OCS transceiver (OCSTrx) (\secref{sec:design:docs}).} 
To enable large-scale deployment, we require a low-cost, low-power transceiver with Optical Circuit Switching (OCS) support. Unlike prior high-radix switches solutions that rely on MEMS-based switching~\cite{urata2022missionapollo, mem-optical-switches}, we leverage advances in Silicon Photonics (SiPh), which offer a simpler structure, lower cost, and reduced power consumption—making them well-suited for commercial transceivers.

Our SiPh-based OCS transceiver (OCSTrx), shown on the left of \figref{fig:overview}, provides two types of communication paths: i) \textit{Cross-lane loopback path (path 3)}, enabling direct GPU-to-GPU communication within the node, which can be used to construct dynamic size topologies; ii) \textit{Dual external paths (path 1\&2)}, connecting to external nodes. All these paths utilize time-division bandwidth allocation, featuring sub-1ms switching latency. With this capability, our \ocstrx \xspace allows dynamic reallocation of full GPU bandwidth to an active external path rather than splitting bandwidth across multiple paths. This eliminates redundant link waste—for instance, activating one external path completely disables the other, ensuring efficient bandwidth utilization.


\para{Design 2: Reconfigurable K-Hop Ring topology (\secref{section:design:topology}).}
With \ocstrx~ that provides reconfigurable connections at the transceiver, the next challenge is designing the topology. A naive starting point is the the full-mesh topology~\cite{fullmesh} which can provide full connectivity among all nodes using \ocstrx~. However, full-mesh design requires $O(N^2)$ links, inducing prohibitive complexity and cost. To reduce costs while maintaining near-ideal fault tolerance and performance, we prune the full-mesh topology into a K-Hop Ring topology based on traffic locality and fault non-locality (Details in~\S\ref{section:design:topology}). Combining the reconfigurability of \ocstrx{}, we propose a \textit{reconfigurable K-Hop Ring topology}, shown in the middle of \figref{fig:overview}, which consists of two key parts:

i) \textit{Intra-node topology:} dynamic GPU-granular ring construction is enabled by activating loopback paths. For example, while $N_1$-$N_3$ physically form a line topology, activating loopback paths creates a ring between $N_1$'s GPUs (1–4) and $N_3$'s GPUs (1–4). This mechanism allows for the construction of arbitrary-sized rings at any location, supporting optimal TP group sizes for different models while effectively minimizing resource fragmentation.

ii) \textit{Inter-node fault isolation: } dual external paths connect to primary and secondary neighbors (e.g., 2-Hop Ring). When a node fails (e.g., $N_2$), its neighbor ($N_1$) activates the backup path ($N_1$-$N_3$) to bypass the fault while maintaining full bandwidth, approaching node-level fault explosion radius. \S\ref{section:design:topology} generalizes this design to $K>2$.


\para{Design 3: HBD-DCN Orchestration Algorithm (\secref{sec:design:orch}).}
Designing an optimal HBD topology is crucial, but end-to-end training performance also depends on the efficient coordination between HBD and DCN. For instance, improper orchestration of TP groups can cause DP traffic to span across ToRs, resulting in DCN congestion. However, existing methods lack the ability to jointly optimize HBD and DCN coordination to alleviate congestion and enhance communication efficiency.
To address this, we propose the HBD-DCN Orchestrator, as shown on the right side of \figref{fig:overview}. The orchestrator takes three inputs: the user-defined job scale and parallelism strategy, the DCN topology and traffic pattern, and the real-time HBD fault pattern. It then generates the TP placement scheme, which maximizes GPU utilization and minimizes cross-ToR communication within the DCN.




\section{Method}

In Fig. \ref{fig:overview}, we illustrate two major stages of MedForge for collaborative model development, including feature branch development (Sec~\ref{branch}) and model merging (Sec~\ref{forging}). In the feature branch development, individual contributors (i.e., medical centers) could make individual knowledge contributions asynchronously. Our MedForge allows each contributor to develop their own plugin module and distilled data locally without the need to share any private data. In the model merging stage, MedForge enables multi-task knowledge integration by merging the well-prepared plugin module asynchronously. This key integration process is guided by the distilled dataset produced by individual branch contributors, resulting in a generalizable model that performs strongly among multiple tasks.


\subsection{Preliminary}
\label{pre}
In MedForge, the development of a multi-capability model relies on the multi-center and multi-task knowledge introduced by branch plugin modules and the distilled datasets.
The relationship between the main base model and branch plugin modules in our proposed MedForge is conceptually similar to the relationship between the main repository and its branches in collaborative software version control platforms (e.g., GitHub~\cite{github}). 
To facilitate plugin module training on branches and model merging, we use the parameter-efficient finetuning (PEFT) technique~\cite{hu2021lora} for integrating knowledge from individual contributors into the branch plugin modules. 

\subsubsection{Parameter-efficient Finetuning}
Compared to resource-intensive full-parameter finetuning, parameter-efficient finetuning (PEFT) only updates a small fraction of the pretrained model parameters to reduce computational costs and accelerate training on specific tasks. These benefits are particularly crucial in medical scenarios where computational resources are often limited.
As the representative PEFT technique, LoRA (Low-Rank Adaptation)~\cite{hu2021lora} is widely utilized in resource-constrained downstream finetuning scenarios. In our MedForge, each contributor trains a lightweight LoRA on a specific task as the branch plugin module. LoRA decomposes the weight matrices of the target layer into two low-rank matrices to represent the update made to the main model when adapting to downstream tasks. If the target weight matrix is $W_0 \in R^{d \times k}$, during the adaptation, the updated weight matrix can be represented as $W_0+\Delta W=W_0+B A$, where $B \in \mathbb{R}^{d \times r}, A \in \mathbb{R}^{r \times k}$ are the low-rank matrices with rank $r \ll  \min (d, k)$ and $AB$ constitute the LoRA module. 



\subsubsection{Dataset Distillation}
Dataset distillation~\cite{wang2018dataset, yu2023dataset, lei2023comprehensive} is particularly valuable for medicine scenarios that have limited storage capabilities, restricted transmitting bandwidth, and high concerns for data privacy~\cite{li2024dataset}. 
We leverage the power of dataset distillation to synthesize a small-scale distilled dataset from the original data.

The distilled datasets serve as the training set in the subsequent merging stage to allow multi-center knowledge integration. Models trained on this distilled dataset maintain comparable performance to those trained on the original dataset (\ref{tab:main_res}). Moreover, the distinctive visual characteristics among images of the raw dataset are blurred (see \ref{fig:overview}(a)), which alleviates the potential patient information leakage. 

To perform dataset distillation, we define the original dataset as $\mathcal{T}=\{x_i,y_i\}^N_{i=1}$ and the model parameters as $\theta$. The dataset distillation aims to synthesize a distilled dataset ${\mathcal{S}=\{{s_i},\tilde{y_i}\}^M_{i=1}}$ with a much smaller scale (${M \ll N}$), while models trained on $\mathcal{S}$ can show similar performance as models trained on $\mathcal{T}$. 
This process is achieved by narrowing the performance gap between the real dataset $\mathcal{T}$ and the synthesized dataset $\mathcal{S}$. In MedForge, we utilize the distribution matching (DM)~\cite{zhao2023dataset}, which increases data distribution similarity between the synthesized distilled data and the real dataset
The distribution similarity between the real and synthesized dataset is evaluated through the empirical estimate of the Maximum Mean Discrepancy (MMD)~\cite{gretton2012kernel}:
\begin{equation}
\mathbb{E}_{\boldsymbol{\vartheta} \sim P_{\vartheta}}\left\|\frac{1}{|\mathcal{T}|} \sum_{i=1}^{|\mathcal{T}|} \psi_{\boldsymbol{\vartheta}}\left(\boldsymbol{x}_i\right)-\frac{1}{|\mathcal{S}|} \sum_{j=1}^{|\mathcal{S}|} \psi_{\boldsymbol{\vartheta}}\left(\boldsymbol{s}_j\right)\right\|^2
\end{equation}

where $P_\vartheta$ is the distribution of network parameters, $\psi_{\boldsymbol{\vartheta}}$ is a feature extractor. Then the distillation loss $\mathcal{L}_{DM}$ is:
\begin{equation}\scalebox{0.9}{$
\mathcal{L}_{\mathrm{DM}}(\mathcal{T},\mathcal{S},\psi_{\boldsymbol{\vartheta}})=\sum_{c=0}^{C-1}\left\|\frac{1}{\left|\mathcal{T}_c\right|} \sum_{\mathbf{x} \in \mathcal{T}_c} \psi(\mathbf{x})-\frac{1}{\left|\mathcal{S}_c\right|} \sum_{\mathbf{s} \in \mathcal{S}_c} \psi(\mathbf{s})\right\|^2$}
\end{equation}

We also applied the Differentiable Siamese Augmentation (DSA) strategy~\cite{zhao2021dataset} in the training process of distilled data to enhance the quality of the distilled data. DSA could ensure the distilled dataset is representative of the original data by exploiting information in real data with various transformations. The distilled images extract invariant and critical features from these augmented real images to ensure the distilled dataset remains representative.
\begin{figure}[t]
    \centering
    \includegraphics[width=\linewidth]{assets/img/model_arch.png}
    \caption{\textbf{Main model architecture.} We adopt CLIP as the base module and attach LoRA modules to the visual encoder and visual projection as the plugin module. During all the procedures, only the plugin modules are tuned while the rest are frozen. We get the classification result by comparing the cosine similarity of the visual and text embeddings.}
    \label{fig:model_arch}
\end{figure}

\subsection{Feature Branch Development}
\label{branch}
In the feature branch development stage, the branch contributors are responsible for providing the locally trained branch plugin modules and the distilled data to the MedForge platform, as shown in Fig~\ref{fig:overview} (a).
In collaborative software development, contributors work on individual feature branches, push their changes to the main platform, and later merge the changes into the main branch to update the repository with new features. Inspired by such collaborative workflow, branch contributors in MedForge follow similar preparations before the merging stage, enabling the integration of diverse branch knowledge into the main branch while effectively utilizing local resources.

MedForge consistently keeps a base module and a forge item as the main branch. The base module preserves generative knowledge of the foundation model pretrained on natural image datasets (i.e., ImageNet~\cite{deng2009imagenet}), while the forge item contains model merging information that guides the integration of feature branch knowledge (i.e., a merged plugin module or the merging coefficients assigned to plugin modules). 
Similar to individual software developers working in their own branches, each branch contributor (e.g., individual medical centers) trains a task-specific plugin module using their private data to introduce feature branch knowledge into the main branch. These branch plugin modules are then committed and pushed to update the forge items of the main branch in the merging stage, thus enhancing the model's multi-task capabilities.


\begin{figure*}
    \centering
    \includegraphics[width=\textwidth]{assets/img/fusion.png}
    \caption{\textbf{The detailed methodology of the proposed Fusion.} Branch contributors can asynchronously commit and push their branch plugin modules and the distilled datasets. the plugin modules will then be weighted fused to the current main plugin module.}

    \label{fig:merge}
\end{figure*}


Regarding model architecture, MedForge contains a base module and a plugin module (Fig ~\ref{fig:model_arch}). The base module is pretrained on general datasets (e.g., ImageNet) and remains the model parameters frozen in all processes and branches (main and feature branches) to avoid catastrophic forgetting of foundational knowledge acquired from pretraining. Meanwhile, the plugin module is adaptable for knowledge integration and can be flexibly added or removed from the base module, allowing updates without affecting the base model. In our study, we use the pretrained CLIP~\cite{radford2021learning} model as the base module. For the language encoder and projection layer of the CLIP model, all the parameters are frozen, which enables us to directly leverage the language capability of the original CLIP model. For the visual encoder, we apply LoRA on weight matrices of query ($W_q$) and value ($W_v$), following the previous study~\cite{hu2021lora}. To better adapt the model to downstream visual tasks, we apply the LoRA technique to both the visual encoder and the visual projection, and these LoRA modules perform as the plugin module. During the training, only the plugin module (LoRA modules) participates in parameter updates, while the base module (the original CLIP model) remains unchanged. 

In addition to the plugin modules, the feature branch contributors also develop a distilled dataset based on their private local data, which encapsulates essential patterns and features, serving as the foundation for training the merging coefficients in the subsequent merging stage~\ref{forging}. Compared to previous model merging approaches that rely on whole datasets or few-shot sampling, distilled data is lightweight and representative, mitigating the privacy risks associated with sharing raw data. 
We illustrate our distillation procedure in Algorithm~\ref{algorithm:alg1}. In each distillation step, the synthesized data $\mathcal{S}$ will be updated by minimizing $\mathcal{L}_{DM}$.
\begin{algorithmic}[1]
    \STATE \textbf{Input:} A list of clauses $C$
    \STATE \textbf{Output:} List of primary outputs $PO$, primary inputs $PI$, intermediate variables $IV$, and Boolearn expressions $BE$
    \STATE $SC$ = [] \COMMENT{List of sub-clauses}
    \FOR{$i = 1$ to length($C$)}
        % \IF{$C[i] \cap SC = \emptyset$}
        %     \STATE Append \text{Simplify}(\text{FindBooleanExpression}([], $SC$)) to $BE$
        %     %\COMMENT{Simplify Boolean expression}
        %     \FOR{each item $w$ in $SC$}
        %         \IF{$w \notin IV$ and $w \neq v$}
        %             \STATE Append $w$ to $PI$
        %         \ENDIF
        %     \ENDFOR
        %     \STATE $SC$ = []
        % \ELSE
            \STATE Append $C[i]$ to $SC$
            \FOR{each item $v$ in $SC$}
                \IF{$v \notin PI$ and $v \notin IV$}
                    \STATE $f \gets \text{FindBooleanExpression}(v, SC)$ %\COMMENT{Find Boolean expression for $v$}
                    \STATE $g \gets \text{FindBooleanExpression}(\neg v, SC)$ %\COMMENT{Find Boolean expression for $\neg v$}
                    \IF{$f = \neg g$}
                        \STATE Append \text{Simplify}($f$) to $BE$ %\COMMENT{Simplify Boolean expression}
                        \IF{$f = True$ or $f = False$}
                            \STATE Append $v$ to $PO$
                        \ELSE
                            \STATE Append $v$ to $IV$
                        \ENDIF
                        \FOR{each item $w$ in $SC$}
                            \IF{$w \notin IV$ and $w \neq v$}
                                \STATE Append $w$ to $PI$
                            \ENDIF
                        \ENDFOR
                        \STATE SC = []
                        \STATE \textbf{break}
                    \ENDIF
                \ENDIF    
            \ENDFOR
        % \ENDIF
    \ENDFOR
    \STATE \textbf{Return} $PO, PI, IV, BE$
    \vspace{-0.65cm}
\end{algorithmic}



\subsection{MedForge Merging Stage}
\label{forging}
Following the feature branch development stage illustrated in Fig~\ref{fig:overview} (a), branch contributors push and merge their branch plugin modules along with the corresponding distilled dataset into the main branch, as shown in Fig~\ref{fig:overview} (b). Our MedForge allows an incremental capability accumulation from branches to construct a comprehensive medical model that can handle multiple tasks.

In the merging stage, the $i^{th}$ branch contributor is assigned a coefficient $w'_i$ for the contribution of merging, while the coefficient for the current main branch is $w_i$. By adaptively adjusting the value of coefficients, the main branch can balance and coordinate updates from different contributors, ultimately enhancing the overall performance of the model across multiple tasks.
The optimization of the coefficients is done by minimizing the cross-entropy loss for classification based on the distilled datasets. We also add $L1$ regularization to the loss to regulate the weights to avoid outlier coefficient values (e.g., extremely large or small coefficient values)~\cite{huang2023lorahub}. During optimization, following~\cite{huang2023lorahub}, we utilize Shiwa algorithm~\cite{liu2020versatile} to enable model merging under gradient-free conditions, with lower computational and time costs. The optimizer selector~\cite{liu2020versatile} automatically chooses the most suitable optimization method for coefficient optimization. 

In the following sections, we introduce the two merging methods used in our MedForge: Fusion and Mixture. In MedForge-Fusion, the parameters of the branch plugin modules are fused into the main branch after each round of the merging stage. For MedForge-Mixture, the outputs of the branch modules are weighted and summed based on their respective coefficients rather than directly applying the weighted sum to the model parameters. This largely preserves the internal parameter structure of each branch module.

\paragraph{MedForge-Fusion}
In MedForge, forge items are utilized to facilitate the integration of branch knowledge into the main branch.
For MedForge-Fusion, the forge item refers to adaptable main plugin modules. When the $i^{th}$ branch contributor pushes its branch plugin module $\theta'_i=A'_iB'_i$ to the main branch, the current main plugin module $\theta_{i-1}=A_{i-1}B_{i-1}$ will be updated to $\theta_{i}=A_{i}B_{i}$. The parameters of the branch and the current main plugin modules are weighted with coefficients and added to fuse a new version. The $A_i$, $B_i$ are the low-rank matrices composing the LoRA module $\theta_i$. The detailed fusion process can be represented as:
\begin{equation}
\theta_{i}=(w_i A_{i-1}+w'_i A'_i)(w_i B_{i-1}+ w'_i B'_i)
\end{equation}
Where $w_i$ is the coefficient assigned to the current main branch, while $w'_i$ is the coefficient assigned to the branch contributor. After this round of merging, the resulting plugin module $\theta_{i}$ is the updated version of main forging item, thus the main model is able to obtain new capacity introduced by the current branch contributor. When new contributors push their plugin modules and distilled datasets, the main branch can be incrementally updated through merging stages, and the optimization of the coefficients is guided by distilled data.
As shown in Fig.~\ref{fig:merge}, though multiple contributors commit their branch plugin modules and distilled datasets at different times, they can flexibly merge their plugin modules with the current main branch. After each merging round, the plugin module of the main branch will be updated, and thus the version iteration has been achieved.
\begin{figure*}[t]
    \centering
    \includegraphics[width=\textwidth]{assets/img/mixture.png}
    \caption{\textbf{The detailed methodology of the proposed Mixture.} Branch contributors can asynchronously commit and push their branch plugin modules and the distilled datasets. the outputs of different plugin modules will be weighted aggregated. The weights of each merging step will be saved.}

    \label{fig:mixmerge}
\end{figure*}


\paragraph{MedForge-Mixture}
To further improve the model merging performance, inspired by~\cite{zhao2024loraretriever}, we also propose medForge-mixture. For MedForge-Mixture, the forge items refer to the optimized coefficients.
As shown in Fig.~\ref{fig:mixmerge}, for MedForge-Mixture, the coefficient of each branch contributor is acquired and optimized based on distilled datasets. Then the outputs of plugin modules will be weighted combined with these coefficients to get the merged output. 

For each merging round, with branch contributor $i$, the branch coefficient is $w'_i$, the main coefficient is $w_i$, the branch plugin module is $\theta'_i=A'_iB'_i$, and the current main plugin module is $\theta_i=A_iB_i$. With the input $x$, the resulted MedForge-Mixture output can be represented as:
\begin{equation}
y_{i}=w_i A_{i-1} B_{i-1} x+w'_i A'_i B'_i x
\end{equation}

In this way, MedForge encourages additional contributors as the workflow supports continuous incremental knowledge updates.

Overall, both MedForge merging strategies greatly improve the communication efficiency among contributors. We use this design to build a multi-task medical foundation model that enhances the full utilization of resources in the medical community. For the MedForge-Fusion strategy, the main plugin module is updated after each merging round, thus avoiding storing the previous plugin modules and saving space. Meanwhile, the MedForge-Mixture strategy avoids directly updating the parameters of each plugin module, thus preserving their original structure and preventing the introduction of additional noise, which enhances the robustness and stability of the models.

% \section{Discussion}

% \begin{figure}
  \centering
  \includegraphics[width=\linewidth]{figures/per_frame_boxplot.png}
  
  \caption{\label{fig:frame-boxplot} Comparison of the distribution of F1 scores across all frames for each model.}
\end{figure}
% \subsection{Model Performance}

% \subsubsection{Out-of-Domain Performance}


% \begin{table}
    \centering
    \begin{tabularx}{\linewidth}{Xcccc}
        \hline
        \textbf{Model} & \textbf{All} & \textbf{Amb} \\ 
        \hline
        % Qwen 2.5-7B     & 0.755 & 0.665 & 0.707 & 0.547 \\ % no candidates @ fe
        % Qwen 2.5-7B     & 0.668 & 0.665 & 0.666 & 0.500 \\ % cand @ fe 
        % Phi-4           & 0.798 & 0.717 & 0.756 & 0.607 \\ % no candidates @ fe
        % Phi-4           & 0.719 & 0.717 & 0.718 & 0.560 \\ % cand @ fe
        % Qwen 2.5-7B     & 91.76 & 90.95 \\ % cand @ fe 
        Phi-4                           & 0.375 & 0.262 \\ % Not finetuned
        % $\text{Phi-4}_{cand}$ w/o LF    & 0.927 & 0.918 \\ % Finetuned on candidates
        $\text{Phi-4}_{cand}$ w/o LF    & 0.882 & \textbf{0.862} \\ % Finetuned on candidates
        $\text{Phi-4}_{cand}$ w/ LF     & 0.894 & \textbf{0.862} \\ % Finetuned on candidates
        % $\text{Phi-4}_{cand}$ w/ LF     & \textbf{0.931} & \textbf{0.918} \\ % Finetuned on candidates
        \hline
        KAF-SPA             & 0.912 & 0.776 \\
        KGFI                & 0.924 & 0.844 \\
        CoFFTEA             & \textbf{0.926} & 0.850 \\
        \hline
    \end{tabularx}
    \caption{Results on frame identification using frame element predictions.}
    \label{tab:candidate_frame}
\end{table}
% \subsection{Frame Identification}
% Previous work~\cite{devasier-etal-2024-robust} explored the possibility of filtering candidate targets produced by matching potential lexical units using a frame identification model. To build upon this idea towards a single-step frame-semantic parsing method, we explore the potential of frame elements being used to filter out candidate targets. In this approach, no ground-truth frame inputs are given. This also removes the bias from the model assuming the input always has at least one frame element.

% We represent the LLM instructions using the JSON-exist representation as it performed the best in Table~\ref{tab:representation_performance}. We used Phi-4 for this experiment as it had a very high performance-to-size ratio, as shown in Table~\ref{tab:candidate_frame}. \todo{should run this on qwen-72b} We found that directly using the model performed poorly, likely due to bias in the model learning that each input contains the given frame. To address this, we fine-tuned the LLM using candidates from the training set and found a significant improvement in performance. \todo{add candidates examples}

% Performance on par with CoFFTEA, the previous-best frame identification system.
% Maybe qwen 72b will perform better.
\section{Conclusion }
This paper introduces the Latent Radiance Field (LRF), which to our knowledge, is the first work to construct radiance field representations directly in the 2D latent space for 3D reconstruction. We present a novel framework for incorporating 3D awareness into 2D representation learning, featuring a correspondence-aware autoencoding method and a VAE-Radiance Field (VAE-RF) alignment strategy to bridge the domain gap between the 2D latent space and the natural 3D space, thereby significantly enhancing the visual quality of our LRF.
Future work will focus on incorporating our method with more compact 3D representations, efficient NVS, few-shot NVS in latent space, as well as exploring its application with potential 3D latent diffusion models.


\bibliographystyle{formats/ACM-Reference-Format}
\bibliography{neural} 

\begin{figure*}[htbp]
\vspace{-5pt}
\centering
  \includegraphics[width=0.99\linewidth]{images/gallery1.png}
\vspace{-5pt}
\caption{The gallery of DeepMill prediction results. On the left are CAD shapes, and on the right are freeform shapes. For each row of shapes, the first and third columns show the inaccessible and occlusion regions predicted by DeepMill. In the second and fourth columns, darker shades represent under-predicted areas, while lighter shades indicate over-predicted areas.}
\label{fig:gallery1}    
\end{figure*}

\begin{figure*}[htbp]
% \vspace{-5pt}
\centering
  \includegraphics[width=0.99\linewidth]{images/gallery2.png}
\vspace{-5pt}
\caption{Demonstration of DeepMill prediction results with extreme size of cutter. After adding the dataset generated with extreme cutters to the training set, DeepMill was able to extrapolate its prediction capability to cases involving extreme cutters.}
\label{fig:gallery2}    
\end{figure*}

\begin{figure*}[htbp]
%\vspace{-5pt}
\centering
  \includegraphics[width=0.99\linewidth]{images/Complex-models.png}
	% \vspace{-15pt}
\vspace{-5pt}
\caption{Testing of DeepMill on complex shapes. For each mesh, the left side shows the prediction results from DeepMill, while the right side displays the differences compared to the geometric method.}
\label{fig:Complex-models}    
\end{figure*}

\begin{figure*}[htbp]
\centering
  \includegraphics[width=0.99\linewidth]{images/Comparison-cutter-module.png}
	% \vspace{-15pt}
\vspace{-5pt}
\caption{Comparison of cutter module concatenation methods. The left and right show the prediction accuracy of inaccessible points and the F1 score of occlusion regions for different concatenation methods on the same test set. Our approach performs the best in both measures.}
\label{fig:Comparison-cutter-module}    
\end{figure*}


\begin{figure}[htbp]
\vspace{-5pt}
\centering
  \includegraphics[width=1.0\linewidth]{images/tool.png}
% \vspace{-10pt}
\vspace{-20pt}
\caption{Illustration of the effect of cutter length on inaccessible regions. Generally, longer cutters lead to fewer inaccessible regions.}
\label{fig:different-cutter}    
\end{figure}


\begin{figure}[htbp]
\vspace{-5pt}
\centering
  \includegraphics[width=1.0\linewidth]{images/line.png}
	% \vspace{-15pt}
\vspace{-20pt}
\caption{Comparison with SAGE. DeepMill shows significantly better prediction capabilities for inaccessible and occlusion regions compared to SAGE.}
\label{fig:Ablation-study}    
\end{figure}


\begin{figure}[htbp]
\vspace{-5pt}
\centering
  \includegraphics[width=1.0\linewidth]{images/symmetry.png}
\leftline{ \footnotesize  \hspace{0.11\linewidth}
            (a)  \hspace{0.18\linewidth}
            (b)   \hspace{0.205\linewidth}
            (c)  \hspace{0.18\linewidth}
            (d)  }
% \vspace{-15pt}
\vspace{-15pt}
\caption{Geometric symmetry illustration. (a) Non-axisymmetric cutter sampling causes asymmetric inaccessible regions (b). (c) Axisymmetric method has uneven distribution. (d) DeepMill combines both, yielding more symmetrical inaccessible regions.}
% \caption{Illustration of geometric symmetry. (a) The non-axisymmetric cutter direction uniform sampling method we used, causing inaccessible regions of geometrically symmetric shapes to lack symmetry (b). (c) The axisymmetric sampling method, however, has uneven distribution. (d) DeepMill combines the advantages of both, resulting in a more symmetrical distribution of inaccessible regions for symmetric geometries.}
\label{fig:symmetry}    
\end{figure}


\begin{figure}[htbp]
\centering
  \includegraphics[width=1.0\linewidth]{images/volume-accessibility.png}
	% \vspace{-15pt}
\vspace{-7pt}
\caption{Illustration of accessibility analysis within the volume. The red points represent inaccessible sampling points. On the top, the results predicted by DeepMill are shown, and on the bottom, the results obtained by the geometric method are displayed.}
\label{fig:volume-accessibility}    
\end{figure}





\clearpage











\end{document}
