\begin{figure*}[htbp]
\vspace{-5pt}
\centering
  \includegraphics[width=0.99\linewidth]{images/gallery1.png}
\vspace{-5pt}
\caption{The gallery of DeepMill prediction results. On the left are CAD shapes, and on the right are freeform shapes. For each row of shapes, the first and third columns show the inaccessible and occlusion regions predicted by DeepMill. In the second and fourth columns, darker shades represent under-predicted areas, while lighter shades indicate over-predicted areas.}
\label{fig:gallery1}    
\end{figure*}

\begin{figure*}[htbp]
% \vspace{-5pt}
\centering
  \includegraphics[width=0.99\linewidth]{images/gallery2.png}
\vspace{-5pt}
\caption{Demonstration of DeepMill prediction results with extreme size of cutter. After adding the dataset generated with extreme cutters to the training set, DeepMill was able to extrapolate its prediction capability to cases involving extreme cutters.}
\label{fig:gallery2}    
\end{figure*}

\begin{figure*}[htbp]
%\vspace{-5pt}
\centering
  \includegraphics[width=0.99\linewidth]{images/Complex-models.png}
	% \vspace{-15pt}
\vspace{-5pt}
\caption{Testing of DeepMill on complex shapes. For each mesh, the left side shows the prediction results from DeepMill, while the right side displays the differences compared to the geometric method.}
\label{fig:Complex-models}    
\end{figure*}

\begin{figure*}[htbp]
\centering
  \includegraphics[width=0.99\linewidth]{images/Comparison-cutter-module.png}
	% \vspace{-15pt}
\vspace{-5pt}
\caption{Comparison of cutter module concatenation methods. The left and right show the prediction accuracy of inaccessible points and the F1 score of occlusion regions for different concatenation methods on the same test set. Our approach performs the best in both measures.}
\label{fig:Comparison-cutter-module}    
\end{figure*}


\begin{figure}[htbp]
\vspace{-5pt}
\centering
  \includegraphics[width=1.0\linewidth]{images/tool.png}
% \vspace{-10pt}
\vspace{-20pt}
\caption{Illustration of the effect of cutter length on inaccessible regions. Generally, longer cutters lead to fewer inaccessible regions.}
\label{fig:different-cutter}    
\end{figure}


\begin{figure}[htbp]
\vspace{-5pt}
\centering
  \includegraphics[width=1.0\linewidth]{images/line.png}
	% \vspace{-15pt}
\vspace{-20pt}
\caption{Comparison with SAGE. DeepMill shows significantly better prediction capabilities for inaccessible and occlusion regions compared to SAGE.}
\label{fig:Ablation-study}    
\end{figure}


\begin{figure}[htbp]
\vspace{-5pt}
\centering
  \includegraphics[width=1.0\linewidth]{images/symmetry.png}
\leftline{ \footnotesize  \hspace{0.11\linewidth}
            (a)  \hspace{0.18\linewidth}
            (b)   \hspace{0.205\linewidth}
            (c)  \hspace{0.18\linewidth}
            (d)  }
% \vspace{-15pt}
\vspace{-15pt}
\caption{Geometric symmetry illustration. (a) Non-axisymmetric cutter sampling causes asymmetric inaccessible regions (b). (c) Axisymmetric method has uneven distribution. (d) DeepMill combines both, yielding more symmetrical inaccessible regions.}
% \caption{Illustration of geometric symmetry. (a) The non-axisymmetric cutter direction uniform sampling method we used, causing inaccessible regions of geometrically symmetric shapes to lack symmetry (b). (c) The axisymmetric sampling method, however, has uneven distribution. (d) DeepMill combines the advantages of both, resulting in a more symmetrical distribution of inaccessible regions for symmetric geometries.}
\label{fig:symmetry}    
\end{figure}


\begin{figure}[htbp]
\centering
  \includegraphics[width=1.0\linewidth]{images/volume-accessibility.png}
	% \vspace{-15pt}
\vspace{-7pt}
\caption{Illustration of accessibility analysis within the volume. The red points represent inaccessible sampling points. On the top, the results predicted by DeepMill are shown, and on the bottom, the results obtained by the geometric method are displayed.}
\label{fig:volume-accessibility}    
\end{figure}





\clearpage







