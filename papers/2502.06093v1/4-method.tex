\section{Geometric method for dataset generation}
\label{sec:Overview}
In this section, we introduce a rapid geometric approach to generate datasets with labels for inaccessible and occlusion regions.

\begin{figure}[t]

\centering
\includegraphics[width=1.0\linewidth]{images/collision-detection.png}
\leftline{ \footnotesize  \hspace{0.08\linewidth}
            (a)  \hspace{0.21\linewidth}
            (b)  \hspace{0.22\linewidth}
            (c)  \hspace{0.21\linewidth}
            (d)}
% \vspace{-5pt}
    \caption{Illustration of inaccessible point detection. Orange represents the cutter, and the gray points represent sampled Voronoi sites. (a) A ball-end cutter can be simplified using four parameters. Note that above cutter is a non-accessible shaft space, and $PD$ is set to infinity. (b) For collision detection with red points, the mesh is first rotated, and points are quickly filtered by checking whether they lie within the detection box (red) of radius FR+$\sigma$, which eliminates most points far from the cutter. $\sigma$ is set to 5 in our experiments. (c) A finer collision check is performed for the points inside the box. (d) To prevent the cutter from penetrating the mesh without detection, the spacing between adjacent sites must be smaller than the cutter’s ball-end radius ($CR$).}
\label{fig:collision-detection}    
\end{figure}

\subsection{Voronoi-based accessibility analysis}
\label{sec:acc}
We use the subtractive collision detection method from \cite{zhong2023vasco} to gather accessibility training data, as it's efficient. 
We introduced a slight modification to their method by incorporating a detection box for pre-detection, enabling faster calculation.
Details of the method are outlined below.

\paragraph{Voronoi-based sampling.} The inset figure demonstrates the use of \begin{wrapfigure}[6]{r}{0.18\textwidth}
\vspace{-14pt}
\hspace{-20pt}
\centering
\includegraphics[width=0.19\textwidth]{images/vorono.png}
\end{wrapfigure} Lloyd's Voronoi relaxation for uniform sampling on the surface of $M$, with each Voronoi cell represented by its site $s_i\in S$, where $S = \{ s_1, s_2, \dots, s_n \}$. The surface of $M$ can be simplified as $M = \cup_{i=1}^n s_i$.

\paragraph{Ball-end cutter} 
In finishing machining of CNC, a ball-end cutter is commonly employed to finish the surface, modeled as a hemisphere combined with two differently sized cylinders.
\autoref{fig:collision-detection}(a) illustrates the cross-section of the cutter, which can be characterized using four parameters: two radii ($CR$ and $FR$) and two heights ($CH$ and $FH$). To ensure successful manufacturing, it is essential to guarantee the cutter $\mathcal{C}$ does not collide with $M$ in any direction $d_k$:
\begin{equation} 
\forall s_i \in S, \forall d_k \in D, s_i \cap \mathcal{C}(d_k) = \varnothing
\end{equation}

\paragraph{Inaccessible points.} 
For each $s_i \in S$, collision detection is performed with other sites, $\forall s_j\in S$ $(j \neq i)$, using the method from \cite{zhong2023vasco}. We first uniformly sample cutter directions $D = \{d_1, d_2, \dots, d_m\}$ using the Fibonacci Sphere sampling method~\cite{vorobiev2002fibonacci} on the upper Gaussian hemisphere and then rotate $M$ contrarily along each direction. To accelerate computation, a cylinder with a radius of $FR + \sigma$ is added as a detecting box before collision detection, allowing only points within the cylinder to undergo finer detection, as shown in \autoref{fig:collision-detection}(b).
Next, as illustrated in \autoref{fig:collision-detection}(c), each $s_j$ is rapidly evaluated for collision by calculating its horizontal distance from the center of $\mathcal{C}$. If the Z-coordinate of $s_j$ exceeds $CR + CH + FH$, it is immediately classified as colliding with the infinitely large shaft space (called global collision). 
After the traversal, if $s_i$ collides with at least one $s_j$ in all cutter directions, it is classified as an inaccessible point:
\begin{equation} 
s_i \leftarrow l_I \iff \forall d_k \in D, |S \cap \mathcal{C}(d_k)| > 0
\end{equation}
Compared to the triangle-facet-based approach~\cite{dhaliwal2003algorithms}, which involves collision detection between the cutter’s cylindrical surface and the triangular mesh, the proposed method is based on discrete sampling points (sites of Voronoi cells), significantly improving computational efficiency.
Even so, this method still has a worst-case complexity of $O(mn^2)$. Besides, it is crucial to ensure that the shortest edge length of the smallest Voronoi cell is greater than $2*CR$ to prevent the cutter from passing through the cell without detection, as illustrated in \autoref{fig:collision-detection}(d).

\begin{figure}[t]

\centering
\includegraphics[width=1.0\linewidth]{images/Occlusion-analyze.png}
\leftline{ \footnotesize  \hspace{0.08\linewidth}
            (a)  \hspace{0.225\linewidth}
            (b)  \hspace{0.225\linewidth}
            (c)  \hspace{0.195\linewidth}
            (d)}
% \vspace{-5pt}
\caption{Illustration of occlusion point calculation. (a)$\sim$(c) Perform collision detection for three inaccessible points, recording the points that collide with the cutter in each cutter direction and counting the total number of collisions for each point. (d) The top 10$\%$ of points with the highest total collision counts are labeled as occlusion points ($l_O$).}
  \label{fig:Occlusion-analyze}    
\end{figure}

\subsection{Occlusion analysis}
To further assist designers in modifying the "culprit" causing inaccessible points, we compute the "occlusion factor" $\beta_i$ for each $s_i \in S$ to quantify the severity of its occlusion for the inaccessible points:
\begin{equation} 
\beta_i = \sum_{s_j \in S}\sum_{d_k \in D} \left(
\begin{cases} 
1  \text{, if } s_j \text{ is inaccessible and } s_i \text{ occludes } \text{$s_j$ in} \; d_k \\
0  \text{, otherwise}
\end{cases} \right)
\end{equation}
The top 10$\%$ of $s_i$ with the highest $\beta_i$ values are defined as "occlusion points" ($s_i \leftarrow l_O$). As shown in \autoref{fig:Occlusion-analyze}, the points in the upper region of the 2D shape are marked as "occlusion points."



