
\section{Introduction}
\label{sec:intro}

\begin{teaserfigure}
%\vspace{-5pt}
  \includegraphics[width=1.0\linewidth]{images/teaser.jpg}
	\vspace{-15pt}
\caption {Demonstration of the human-robot interaction process in SymbioSim. In this system, human interacts with a virtual robot in 3D space through AR. The robot perceives human actions and generates real-time responses. Feedback from the human experience is continuously collected to optimize the robot’s model, improving its performance. Concurrently, humans gradually develop a deeper understanding and trust in the robot. Ultimately, SymbioSim fosters bidirectional learning and adaptation, with the potential to promote human-robot symbiosis.}
  \label{fig:teaser}    
\end{teaserfigure}

Manufacturability is a fundamental concept in product design and production~\cite{shukor2009manufacturability,gupta1997automated}, referring to the ease and efficiency with which a design can be transformed into a physical product, while accounting for factors such as material constraints, cutter capabilities, and production costs~\cite{joshi1988graph,li2006machinability,hoefer2017automated}.
Ensuring manufacturability early in the design process is critical to avoiding costly revisions, delays, and inefficiencies, thereby optimizing production timelines and minimizing resource waste.

Accessibility is a key aspect of manufacturability~\cite{elber1994accessibility,spyridi1990accessibility}.
In subtractive manufacturing, accessibility pertains to whether all surfaces and features of a part can be reached by machining tools during production~\cite{zhang2020manufacturability}. Accessibility issues arise when certain features are difficult to access, such as deep holes, internal cavities, or overhanging geometries. Addressing these challenges through accessibility analysis is essential for identifying potential machining difficulties early, enabling designers to adjust part geometry or select appropriate cutters to optimize the manufacturing process. Beyond subtractive manufacturing, \accessAna also plays a critical role in decisions such as setup planning~\cite{zhong2023vasco}, cutter selection~\cite{athawale2010topsis}, cutter orientation adjustment~\cite{mahdavi2020vdac}, and tool path planning~\cite{balasubramaniam2003collision}, contributing to overall production efficiency and cost-effectiveness.



Traditional methods for geometric accessibility analysis, which emerged in the 1990s, primarily rely on geometric and computational techniques to evaluate cutter accessibility in multi-axis CNC machining. These methods, while foundational, are often time-consuming, especially for geometrically complex parts, with analysis of intricate designs taking hours—unacceptable in fast-paced design environments that require rapid iteration~\cite{dai2018support}. 
Although early approaches advanced from basic visibility analysis to more precise accessibility evaluations, they struggle with scalability and high computational overhead when applied to high-resolution 3D models. These limitations highlight the need for faster, more scalable methods capable of handling complex geometries and diverse cutter parameters.

In recent years, the advent of deep learning techniques has opened new possibilities for improving computational efficiency in manufacturability analysis.
Several studies have explored the use of deep learning models to predict non-manufacturable regions~\cite{kerbrat2011new,chen2020manufacturability,ghadai2018learning}, enhancing performance by reducing processing time. 
However, most of these efforts focus primarily on process-related issues, such as process planning and collision detection~\cite{chen2020manufacturability}, while neglecting the crucial geometric challenges inherent in accessibility analysis.
Moreover, these methods often rely on feature-based CAD models~\cite{yan2023automated,balu2020orthogonal}, which limits their applicability to freeform or highly complex product designs.

This paper presents DeepMill, the first neural framework, to the best of our knowledge, specifically designed for predicting non-manufacturable regions in arbitrary models, including freeform shapes, with high accuracy and efficiency. Unlike previous approaches, DeepMill focuses specifically on cutter accessibility, identifying regions where cutter collisions occur due to geometric constraints such as occlusion. 
We propose to utilize a neural network capable of real-time predictions for both non-manufacturable regions and the occlusion regions causing these issues, providing designers with actionable insights to quickly iterate and refine their designs. 
DeepMill demonstrates exceptional generalization across various cutter sizes and complex geometries, making it suitable for a wide range of design contexts.


One of the key challenges in accessibility analysis is the complexity of cutter collisions, which can be both local and global in nature. Factors such as cutter rotation and size affect accessibility, requiring methods that efficiently learn and represent these underlying geometric features. 
Moreover, the scarcity of extensive training datasets for these specific tasks has hindered the creation of robust models.


%
To overcome these challenges, we propose utilizing octree-based convolutional neural network (O-CNN) to efficiently capture both local and global geometric features, while embedding cutter modules to capture intricate interactions between cutters and complex 3D surfaces.
This approach enables our network to handle both CAD and freeform models, providing a scalable and flexible solution to the manufacturability analysis problem.

Additionally, we created the first inaccessible and occlusion regions analysis dataset with diverse cutter parameters for training DeepMill and generated multiple test set categories, addressing the challenges of data scarcity.

In summary, DeepMill offers a significant advancement in both computational efficiency and accuracy. 
%Through extensive testing on multiple datasets, we demonstrate that our network outperforms traditional methods in terms of prediction accuracy and computational speed, particularly for geometrically complex parts.
Experiments indicate DeepMill achieves 94.7$\%$ and 88.7$\%$ accuracy on average in identifying inaccessible and occlusion regions, with an average processing time of only 0.04 seconds for complex geometries.
Our model is adaptable to a wide range of cutter sizes, ensuring its applicability across diverse design contexts. 
Additionally, we introduce a new dataset to support further research in this area and facilitate the development of more robust manufacturability analysis cutters.

% \hs{FC, say more technique contributions}
% \hs{FC, describe the key feature of our dataset}




\begin{comment}


%datasets
%Accessibility prediction using machine learning relies heavily on high-quality datasets. These datasets often require detailed annotations that account for the spatial relationship between tools and workpieces. Common strategies include generating synthetic data through simulations, labeling accessibility regions based on geometric properties, or collecting real-world data from machining processes.

%A significant challenge in this domain lies in balancing accuracy and efficiency. Accessibility prediction models must handle diverse tool geometries, complex part surfaces, and varying machining constraints without compromising computational performance. Moreover, incorporating physical and mechanical constraints—such as collision avoidance and material removal feasibility—into the learning process is critical to ensure practical applicability.


5. Challenges in Tool Accessibility Prediction
Identify challenges specific to tool accessibility prediction, such as:
Representing diverse tool geometries and orientations.
Capturing intricate interactions between tools and complex 3D surfaces.
Balancing prediction accuracy with computational efficiency.
Set the stage for how your work addresses these challenges.

\end{comment}
