\section{Results and Discussions}
\label{sec:results}

\subsection{Data processing and datasets}
\paragraph{Data processing and cleaning} 
We performed the accessibility and occlusion analysis described in \autoref{sec:Overview} to construct the training and test sets, uniformly sampled 150 cutter directions on the upper hemisphere. We selected various CAD shapes from the ABC dataset \cite{Koch2019CVPR} and freeform shapes from Thingi10K \cite{Thingi10K}. After cleaning non-manifold, non-watertight, and multi-component assemblies, we retained high-quality shapes. To avoid invalid data from shapes significantly smaller than the cutter, we extracted each shape's bounding box, ensuring its minimum edge length was at least 80 mm. For each shape, the coordinates of Voronoi cell sites, normals, and corresponding inaccessible and occlusion labels were recorded in the training and test sets.

\paragraph{Datasets}
In the training set, we randomly generated the four cutter parameters for over 5K CAD shapes within specified ranges: $CR \in [1, 2]$, $FR \in [5, 100]$, $CH \in [0.1, 10.1]$, and $FH \in [0.1, 10.1]$. Each shape had an average of approximately 7K mesh vertices. 
For the test sets, we selected 1K CAD shapes distinct from the training set and created two sets through remeshing, containing 7K and 15K mesh vertices, respectively. Similarly, we generated two test sets for freeform shapes. 
To further validate DeepMill's generalization ability, we curated a complex dataset of 500 shapes with over 100K vertices from the ABC dataset. Additionally, we varied cutter size parameters and generated multiple test sets based on CAD shapes to demonstrate DeepMill's adaptability to different cutter sizes.

\subsection{Implementation}
All experiments were conducted on a desktop computer equipped with an Intel Core i7-11700F CPU running at 2.5 GHz, 16 GB of memory, and an RTX 3090 GPU with 24 GB of memory. The source code will be released after the paper is published. The octree depth was set to 5, with encoder channels configured as [32, 32, 64, 128, 256] and decoder channels as [256, 256, 128, 96, 96]. The cutter module channels were set to [4, 32, 256]. We used the stochastic gradient descent (SGD) optimizer for training, starting with an initial learning rate of 1.0, which was adjusted using the Cosine Annealing method. The network was trained for 1500 epochs with a batch size of 128 for both training and testing. All input points were normalized to the unit cube $[-1, 1]^3$, and data augmentation methods from~\cite{choy20194d}, including random mirroring and elastic deformations, were applied.


\begin{table}[t]
	\begin{minipage}{1.0\linewidth}
		\centering
		\caption{Statistics of DeepMill and geometric method on various datasets. The values in ( ) indicate the average number of mesh vertices in the dataset. $\text{Acc}_i$ and $\text{Acc}_o$ represent the prediction accuracy for inaccessible points and occlusion points, respectively. $\text{F1}_i$ and $\text{F1}_o$ represent their F1-scores. $T_i$ and $T_o$ denote the calculation time (second) for inaccessible and occlusion points by geometric method, while $T$ represents the total time. Due to the excessive computation time required by geometric method to generate occlusion points for complex models, we do not perform statistics on them.} 
		\label{table:table-different-data}
            \setlength{\tabcolsep}{2pt}
		\resizebox{1\textwidth}{!}{
			\begin{tabular}{l|ccccc|ccc} 
				\hline
				&  \multicolumn{5}{c|}{\textbf{DeepMill}}  & \multicolumn{3}{c}{\small{\textbf{Geometric}}} \\ 
                \hline
                \textbf{DATASET} & $\textbf{Acc}_i$ & $\textbf{F1}_i$ & $\textbf{Acc}_o$ & $\textbf{F1}_o$ &$T$ & $T_i$ & $T_o$ & $T$ \\
                \hline 
                CAD(7K) 	&96.3$\%$ & 97.2$\%$ & 98.3$\%$  & 89.4$\%$ & \cellcolor{orange!25}0.01  & 4.0 & 25.1 & 29.1 \\ 
                CAD(15K) 	& \cellcolor{orange!25}96.3$\%$ & \cellcolor{orange!25}97.3$\%$ & \cellcolor{orange!25}98.3$\%$ & \cellcolor{orange!25}90.0$\%$ & \cellcolor{orange!25}0.01  & 17.3 & 207.4 & 224.7 \\ 
                Freeform(7K) 	& 92.8$\%$ & 93.7$\%$ & 97.5$\%$ & 86.5$\%$ & \cellcolor{orange!25}0.01  & 5.4 & 41.3 & 46.7 \\ 
                Freeform(15K) 	& 93.2$\%$ & 93.3$\%$ & 98.0$\%$ & 88.7$\%$ & \cellcolor{orange!25}0.02  & 19.8 & 373.1 & 392.9 \\ 
                \small{Complex(10W+)}  	& 90.5$\%$ & 90.0$\%$ & $\backslash$ & $\backslash$ & \cellcolor{orange!25}0.04  & 137.0 & $\backslash$ & $\backslash$ \\ 
                \hline
			\end{tabular}
		}
	\end{minipage}
\end{table}


\subsection{Evaluation}
\paragraph{Efficiency and Generalization}
DeepMill was trained on a dataset of over 5K CAD shapes.
We evaluated its efficiency on datasets containing diverse shapes, as shown in \autoref{table:table-different-data}.
We map the Voronoi sites onto the triangles for better visualization.
\autoref{fig:teaser} and \autoref{fig:gallery1} illustrate plenty of examples of inaccessible and occlusion regions predicted by DeepMill with various cutter sizes.


In general, DeepMill maintains high prediction accuracy, with most errors occurring near the boundaries of inaccessible and occlusion regions, which minimally affects the overall distribution.
Additionally, as the mesh resolution increases (e.g., 7K vs. 15K Freeform shapes), finer geometric details further improve DeepMill's prediction accuracy, and the higher resolution amplifies the time advantage of DeepMill over traditional geometric methods.
% Regarding cutter influence, short cutters expand inaccessible regions and increase global collisions, causing frequent collisions with the mesh top, where occlusion regions are concentrated.\fc{say long cutter}

% In contrast, long cutters mainly induce localized collisions, making occlusion regions more prevalent in geometrically complex regions.


For different datasets, DeepMill exhibits varying performance:
1) On CAD datasets with geometric styles similar to the training set, DeepMill achieves up to 96.3$\%$ accuracy for predicting inaccessible regions. Since occlusion regions are derived from inaccessible regions, they present greater prediction challenges.
2) For freeform shapes, which feature fewer sharp or weak geometric characteristics and differ significantly from CAD shapes, the network maintains high prediction accuracy (92.8$\%$, 86.5$\%$), demonstrating its ability to effectively learn the geometric relationship between shapes and cutters.
3) The Complex dataset consists of meshes with a large number of triangular facets (over 100K), as shown in \autoref{fig:Complex-models}. The intricate geometries increase prediction difficulty, and for unconventional structures that diverge significantly from the training set, DeepMill show reduced accuracy (e.g., top-right corner of \autoref{fig:Complex-models}).




\paragraph{Computation time}
DeepMill offers significant advantages over traditional geometric methods, achieving real-time predictions. For inaccessible and occlusion regions, DeepMill requires only \textbf{0.004$\%$} of the total time needed by geometric methods on CAD shapes with 15K mesh vertices. For more complex shapes, the time is reduced to \textbf{0.029$\%$} for inaccessible analysis only. Moreover, if finer cutter direction sampling is used in geometric methods, the time efficiency of DeepMill becomes even more pronounced.



\subsection{Ablation and Comparisons}
\begin{table}[t]
	\begin{minipage}{1.0\linewidth}
		\centering

		\caption{Comparison statistics of the cutter module in DeepMill for test sets with different cutter size. The baseline refers to the control group without the cutter module. $Avg_i$ and $Avg_f$ represent the average values of accuracy and F1-score, respectively. In the Uniform test set, the random range of cutter parameters is the same as in the training set. In the Short set, the ranges of $CH$, $FR$, and $FH$ are [0.1, 0.2], [80, 100], and [0.1, 0.2], respectively. In the Long set, the ranges are [10, 10.1], [5, 5.1], and [10, 10.1]. In the Extreme set, the ranges are [20, 20.1], [5, 5.1], and [20, 20.1].}
		\label{table:table-time}
            \setlength{\tabcolsep}{3pt}
		\resizebox{1\textwidth}{!}{
			\begin{tabular}{l|l|ccccccc}
				\hline
				\textbf{Cutter} & \textbf{Decoder}   &  \textbf{$\text{Acc}_i$} & \textbf{$\text{F1}_i$} & \textbf{$\text{Acc}_o$} & \textbf{$\text{F1}_o$} & \textbf{$\text{Avg}_{acc}$} & \textbf{$\text{Avg}_{f1}$} \\ \hline 
                
                \multirow{2}{*}{Uniform}
                &Baseline 	& 0.932 & 0.949 & 0.976  & 0.856 & 0.954 & 0.903\\ 
                &Our	& \cellcolor{orange!25}0.963 & 0.972 & 0.983 & \cellcolor{orange!25}0.894 & 0.973 & 0.933\\
                \hline
                \multirow{2}{*}{Short}
                &Baseline 	& 0.898 & 0.916 & 0.972 & 0.838 & 0.935 & 0.877\\ 
                &Our	& \cellcolor{orange!25}0.962 & 0.967 & 0.981 & \cellcolor{orange!25}0.896 & 0.972 & 0.932\\ 
                \hline
                \multirow{2}{*}{Long}
                &Baseline 	& 0.869 & 0.909 & 0.930 & 0.576 & 0.900 & 0.743\\ 
                &Our 	& \cellcolor{orange!25}0.940 & 0.960 & 0.961 & \cellcolor{orange!25}0.763 & 0.951 & 0.862\\ 
                \hline
                \multirow{2}{*}{Extreme}
                &Baseline & 0.809 & 0.873 &  0.901 & 0.388 & 0.855 & 0.631\\ 
                &Our	& \cellcolor{orange!25}0.928 & 0.937 & 0.975 & \cellcolor{orange!25}0.537 &  0.952 & 0.737\\ 
                \hline
			\end{tabular}
		}
	\end{minipage}
\end{table}



\paragraph{Ablation study for cutter module}
To demonstrate the effectiveness of the cutter module, we compared it with the baseline model without the cutter module. As shown in \autoref{table:table-time}, DeepMill with the cutter module significantly outperforms the baseline across test sets with various cutter parameter ranges. 
This improvement is especially noticeable when using short and long cutters, as the baseline model tends to predict using an "average-sized cutter," resulting in a larger accuracy gap compared to DeepMill.
It is worth noting that some regions of the shape are inherently "cutter-independent," meaning their accessibility remains unchanged regardless of cutter size. The baseline successfully learns these regions, maintaining reasonable prediction accuracy. 
% On the other hand, tests with extreme cutter parameters further validate the network’s ability to understand cutter collision patterns effectively. 
On the other hand, since shorter cutter sizes were used in the training set, the ratio of local collisions to global collisions is relatively small. As a result, DeepMill performs less accurately in predicting occlusion regions caused by extensive local collisions from extremely long cutters.
To demonstrate it, we added 3K+ CAD shapes with extreme cutter sizes to the training set for calculating inaccessible and occlusion regions. The accuracy and F1-score for the two regions improved to $95.8\%$ and $87.8\%$, respectively. \autoref{fig:gallery2} shows several examples.


\paragraph{Comparison of cutter module positions}
We compared the impact of adding the cutter module at different positions in the decoder on the prediction results. As shown in \autoref{fig:Comparison-cutter-module}, adding the module to each layer, rather than only to the first or last layer of the decoder, enables the network to better learn the influence of cutter parameters on both local and global geometry.

\paragraph{Comparison of different cutter sizes}
\autoref{fig:different-cutter} illustrates the effect of cutter length on inaccessible regions. Short cutter is more prone to collide with the shaft above cutter, resulting in larger inaccessible regions. On the other hand, longer cutter is less likely to cause collisions, allowing for more accessible regions.

\paragraph{Comparison with other network}
\autoref{fig:Ablation-study} presents a comparison between DeepMill and the GraphSAGE baseline. DeepMill significantly outperforms GraphSAGE, demonstrating its superior ability to capture complex geometric interactions. In contrast, GraphSAGE struggles to achieve satisfactory accuracy improvements, highlighting its limited capability to effectively learn collision relationships between shapes and cutters.
% \paragraph{loss test}
% As shown in \autoref{??}, we compared three different loss functions with varying weights for $\lambda_1$ and $\lambda_2$ in \autoref{??}. Assigning a larger weight to the loss for occluded regions helps the network prioritize predicting occluded areas correctly. However, since occluded regions are highly dependent on unreachable regions, placing excessive emphasis on them makes the network more prone to overfitting. In contrast, a balanced weight promotes iterative learning between the two, leading to more robust predictions.

\subsection{Discussion and Extension}

\paragraph{Geometric symmetry}
The geometric method in the dataset uses Fibonacci sphere sampling for evenly distributed cutter directions, as shown in \autoref{fig:symmetry} (a). However, the directions lack axial symmetry, causing asymmetric inaccessibility distributions for symmetric shapes, as shown in \autoref{fig:symmetry} (b).
Another sampling method based on spherical coordinates, shown in \autoref{fig:symmetry} (c), achieves symmetric direction distribution but suffers from uneven spacing, easily missing directions.
Surprisingly, DeepMill combines the strengths of both methods, learning to symmetrically adjust predictions from uniformly distributed directions. It produces more symmetric and reasonable inaccessibility regions when predicting geometrically symmetric shapes, as illustrated in \autoref{fig:symmetry} (d).

\paragraph{Volume accessibility analysis}
DeepMill can also be applied in accessibility analysis for other machining methods. For instance, rough machining removes the blank material layer by layer using a mill cutter, which is typically the pre-process for the finishing process. The accessibility analysis focuses on the interior volume of the blank. 
We use the same collision detection method to detect the interior sampling points of the bounding box of the input mesh to generate datasets. As shown in \autoref{fig:volume-accessibility}, after training on the new dataset of over 5K CAD shapes, DeepMill can accurately predict the inaccessible regions within the volume.
Compared to surface-based accessibility analysis, DeepMill can more easily predict accessibility within the volume, achieving an accuracy of up to $97.9\%$. Volume sampling uses voxelization, where adjacency resembles pixels in 2D, making it more suitable for 3D convolution operations.






