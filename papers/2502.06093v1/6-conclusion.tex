
\section{Conclusion and Future work}
\label{sec:conclusion}
This paper introduces DeepMill, a deep learning framework that improves cutter accessibility and manufacturability analysis for complex designs. Utilizing octree-based convolutional neural network (O-CNN), DeepMill efficiently predicts inaccessible regions and occlusions across various cutter sizes, overcoming the scalability and computational limitations of traditional methods. Its real-time predictions enable faster design iterations and enhance production efficiency. Additionally, the new dataset introduced supports further research and development of robust manufacturability analysis cutters, making DeepMill a significant advancement in the field. 
Extensive testing and comparisons have demonstrated DeepMill's powerful cutter-aware prediction ability.


%\hs{missing limitations}

%模型的预测是在静态几何形状下进行的,未考虑加工过程中的动态因素,比如 1)制造过程中模型的orientation可能会被调整  2)由于应力释放或热变形,工件几何可能发生偏移或弯曲


% \paragraph{Future work}


Based on the challenges and opportunities outlined, several directions for future work are proposed. First, to enhance the prediction performance of our current network, integrating an attention mechanism will be a promising approach. 
%This will enable the model to more effectively focus on critical features in the data, improving both accuracy and robustness. 
Second, incorporating geometric prior knowledge, such as symmetry, similarity, and topological properties, will allow the network to better capture the underlying structures.
%, leading to more precise and reliable predictions
%
Another promising avenue involves incorporating cutters that work with irregular shapes in subtractive manufacturing, which will broaden the applicability of our framework to a wider range of real-world scenarios.
Additionally, exploring the downstream applications of our current framework, such as path planning and model correction from non-accessible to accessible models, would be an exciting direction for further development.

Exploring learning-based methods in digital design and manufacturing, integrating both subtractive and additive aspects, will drive intelligent solutions to complex challenges.

% Crucially, it is essential to delve deeper into the use of learning-based methods in digital design and manufacturing. This involves integrating manufacturability aspects from both subtractive and additive manufacturing. These efforts will help push forward the limits of intelligent manufacturing and design, providing innovative strategies to tackle complex real-world issues.



% \begin{comment}    
% To further improve the prediction performance,
% add attention mechanism to our current network,
% encode geometry prior knowledge to the network. including symmetry and XXX.
% Another direction is including more tools with irregular shapes in additive manufacturing, and XXX domain.
% Furthermore, explore the downstream application of current frameworks is also interesting, including path planning, model correction from non-accessible to accessible model.
% More important, it's worth to go deep along this direction of apply learning manufacting for digital design and manufacturing, including other manufacturability considerations in subtractive or additive domain.
% \end{comment}
    





