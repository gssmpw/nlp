\begin{figure}[t]

\centering
\includegraphics[width=1.0\linewidth]{images/collision-detection.png}
\leftline{ \footnotesize  \hspace{0.08\linewidth}
            (a)  \hspace{0.21\linewidth}
            (b)  \hspace{0.22\linewidth}
            (c)  \hspace{0.21\linewidth}
            (d)}
% \vspace{-5pt}
    \caption{Illustration of inaccessible point detection. Orange represents the cutter, and the gray points represent sampled Voronoi sites. (a) A ball-end cutter can be simplified using four parameters. Note that above cutter is a non-accessible shaft space, and $PD$ is set to infinity. (b) For collision detection with red points, the mesh is first rotated, and points are quickly filtered by checking whether they lie within the detection box (red) of radius FR+$\sigma$, which eliminates most points far from the cutter. $\sigma$ is set to 5 in our experiments. (c) A finer collision check is performed for the points inside the box. (d) To prevent the cutter from penetrating the mesh without detection, the spacing between adjacent sites must be smaller than the cutter’s ball-end radius ($CR$).}
\label{fig:collision-detection}    
\end{figure}