\section{METHODOLOGY}
This section first outlines the problem and goal of the network, followed by an introduction to our octree-based network, DeepMill, designed for predicting inaccessible and occlusion regions on the surface of the input mesh $M$. We then provide a detailed analysis of its advantages over GNN-based methods.


\subsection{Problem formulation}
The detection of inaccessibility and occlusion regions, which segments these areas from the surface of $M$, is formulated as a 3D geometric segmentation problem. 
To facilitate the computation, we use points to represent its local region.
The goal of inaccessibility detection is to identify inaccessible points (labeled as $l_I$) on $M$ where the cutter cannot reach without collision, while occlusion detection aims to locate points causing the most severe occlusion for the inaccessible points. 
In this paper, we classify the top 10$\%$ of points with the highest occlusion severity as "occlusion points" (labeled as $l_O$).
Due to the severe imbalance in data distribution, we use the F1-score to evaluate the occlusion points.
Unlike standard visibility problems, the shape and size of the cutter $\mathcal{C}$ directly influence the segmentation results. Consequently, this task is formulated as a dual-task binary segmentation problem with a cutter-aware objective, expressed as:
\begin{equation}
\label{eq:weightSet}
\begin{aligned}
Max(accuracy(l_I)+F1(l_O)\;|\;\mathcal{C}).
\end{aligned}
\end{equation}

\begin{figure}[t]
%\vspace{-5pt}
\centering
\includegraphics[width=1.0\linewidth]{images/OCNN.png}
% \vspace{-15pt}
\caption{DeepMill's network architecture. The input mesh is converted into a point cloud with normals, with each point corresponding to a Voronoi cell's site. Features are progressively extracted through the encoder, and the decoder, embedded with a cutter module (CM), restores spatial resolution. Both encoder and decoder are stacked with several Octree-based residual blocks. Finally, each site is subjected to dual-task binary segmentation through two header layers. Red and green represent the inaccessible and occlusion regions, respectively.}
\label{fig:Cutter-embended}    
\end{figure}


\subsection{DeepMill}
DeepMill's main components include the encoder and the decoder, which are well-suited for segmentation task. Additional cutter modules and prediction head are added to further adapt to our problem. \autoref{fig:Cutter-embended} shows the network architecture of DeepMill.

\paragraph{O-CNN with U-Net architecture.}
Unlike more complex network architectures, such as the Hierarchical Graph Neural Network~\cite{huang5065158hierarchical}, the O-CNN-based U-Net architecture uses a concise representation—point clouds and their normals—as input. The output consists of two predicted labels for each point. 
The U-Net architecture~\cite{ronneberger2015u} is composed of an encoder and a decoder, both of which are stacked with multiple Octree-based residual blocks~\cite{wang2017cnn}, with skip connections between the encoder and decoder. 
The encoder progressively extracts multi-scale features from the input 3D data, while the decoder gradually restores spatial resolution and reconstructs accurate predictions through the use of skip connections. 
DeepMill adopts this architecture, and the benefits of this approach will be discussed in \autoref{sec:analysis}.

 
\paragraph{Cutter embeding}
To enable the network to learn the impact of the cutter on inaccessibility and occlusion points, we embed cutter modules into the network. Compared to the encoder, the decoder is closer to the network's final decision-making region, and concatenating the cutter features at this stage minimizes interference with the network's early learning of geometric shapes. We validate this hypothesis in subsequent experiments. Furthermore, considering that the cutter causes collisions in both local regions and distant global regions (collide with the shaft space above the cutter) of $M$, we embed cutter modules at every layer of the decoder to help the network better learn the collision patterns between the cutter and $M$ at different scales.

In detail, we encode the four shape parameters $\{CR, CH, FR, FH\}$ of the cutter into a vector $\mathbf{V} = [v_1, v_2, v_3, v_4]^T$ and pass it through four fully connected cutter modules. As shown on the right side of \autoref{fig:Cutter-embended}, each cutter module consists of two 'Linear-ReLU-BN-Dropout' sub-blocks, where the 4-dimensional vector $\mathbf{V}$ is transformed into a 256-dimensional cutter feature. These features are then concatenated into each layer of the decoder in the U-Net architecture.
\begin{equation} 
f'_i = f_i \oplus f_i^c, \quad i = 1, 2, 3, 4
\end{equation}
where $f_i$ indicates the output feature of the $i$-th layers in the decoder and $f_i^c$ is the output feature of the $i$-th cutter module. 



\paragraph{Dual-head segmentation} 
To predict inaccessible points and occlusion points separately, we use two fully connected header layers to predict these two types of labels. Since occlusion points are calculated based on inaccessible points, and both labels are computed using the same geometric algorithm during collision detection, there is a strong geometric correlation between the two labels. Therefore, before passing through the header layers, their features are fully shared. The predicted results $\hat{y}^i_j$ for $s_i \in S$ are denoted as:
\begin{equation} 
\hat{y}^i_j = \text{header}_j(f'_4), j=1,2
\end{equation}
% In the results section, we compare this approach with an earlier method that separates the features, validating our approach.


\paragraph{Architecture details} 
As shown on the left side of \autoref{fig:Cutter-embended}, each octree-based residual block consists of two 'Convolution + BN + ReLU' sub-blocks, connected by residual connections~\cite{he2016deep}. Batch normalization (BN) is applied to reduce internal covariate shift~\cite{ioffe2015batch}, while the ReLU activation function $(f : x \in \mathbb{R} \longmapsto max(0,x))$ is used to activate the output.

In the encoder, the input point cloud undergoes multiple octree-based 3D convolution operations through several octree-based residual blocks, generating feature maps at different levels to capture multi-scale geometric features for hierarchical representation. Unlike traditional 3D-CNN convolutions~\cite{maturana2015voxnet}, the octree structure marks non-empty nodes at the current depth, representing regions containing point clouds, and applies convolutions only to these nodes. 
The depth of the octree gradually decreases, and high-resolution child node features are aggregated into their corresponding parent nodes.


In the decoder, the global feature map is progressively processed through deconvolution for feature upsampling and spatial detail recovery, with cutter feature fusion enhancing the modeling of inaccessibility and occlusion effects.
As the depth of octree increases, features are progressively passed down to the high-resolution child nodes.
Output-guided skip connections~\cite{wang2020deep} are used to transfer features from the encoder to the decoder, excluding sparse regions. If the octree node corresponding to a feature output from a block is empty, the skip connection is not applied.


\paragraph{Loss function}
During network optimization, we use cross-entropy loss function to compute the loss for inaccessible and occlusion points separately, denoted as $\mathcal{L}_I$ and $\mathcal{L}_O$. The total loss function is defined as:
\begin{equation}
\mathcal{L} =  \mathcal{L}_I(\hat{y}_1, y_1) +  \mathcal{L}_O(\hat{y}_2, y_2)
\end{equation}
where $y_1$ and $y_2$ denote the ground truth labels calculated using the geometric method mentioned in \autoref{sec:Overview}.


\begin{figure}[t]
%\vspace{-5pt}
\centering
  \includegraphics[width=1.0\linewidth]{images/sage.png}
	% \vspace{-15pt}
\caption{Comparisons with GraphSAGE. $M$ is converted into a graph, with nodes representing mesh vertices and edges representing topological connections of them. Similar to O-CNN, initial node features include vertex coordinates and normals. Node features are propagated and updated through successive convolutions on neighboring nodes. }
  \label{fig:different-methods}    
\end{figure}

\subsection{Analysis for different networks}
\label{sec:analysis}
To demonstrate the simplicity and effectiveness of DeepMill, we constructed GNN-based framework similar to the approach in \cite{huang5065158hierarchical} for comparison. As shown in \autoref{fig:different-methods}, we used the classic GraphSAGE model \cite{hamilton2017inductive}. 
The network generates node embeddings by sampling neighboring nodes and aggregating their features, addressing the computational bottleneck of traditional GNNs on large-scale graphs. Compared to GCN~\cite{kipf2016semi} or GAT~\cite{velivckovic2017graph}, it is better suited for handling graphs constructed from high-resolution meshes in accessibility analysis.
GraphSAGE seems capable of preserving the mesh's topology, thus learning the geometric relationship between the cutter and $M$ more effectively. However, there are significant global collisions between the cutter and $M$, and occlusion points are often topologically distant from inaccessible points, making it difficult for graph convolutions to efficiently capture this relationship. 

% Similarly, as shown in \autoref{fig:different-methods} (b), even if we capture global and local features using O-CNN and use high-dimensional features as the initial node features for the graph, the graph convolution operation, which emphasizes propagating topological information over deeply understanding geometric relationships, makes it difficult for the graph to effectively leverage these features.
% \vspace{-10pt}
In contrast, O-CNN-based U-Net architecture efficiently processes sparse 3D data through the octree structure, avoiding redundant computations. In accessibility analysis, CAD shapes often contain numerous holes, grooves, etc., and O-CNN effectively captures these key sparse geometric features. 
Additionally, the multi-scale convolution operation based on 3D space extracts shape features at different scales and is more effective at capturing collision relationships between distant positions on the mesh.
Furthermore, U-Net’s encoder-decoder structure with skip connections enables it to capture both local features and global context, preserving the detailed geometric insights essential for predicting inaccessible points.
% In contrast, O-CNN[??] based U-net Architecture efficiently processes sparse 3D data through the octree structure, avoiding redundant computations and extracting shape features at different scales. In subtractive accessibility analysis, CAD workpieces often have a large number of holes, holes, grooves, O-CNN can effectively capture key areas of these sparse geometric shapes while avoiding the computation of irrelevant regions. Its local convolution operations enable features extracted at different scales to better reflect the accessibility of complex geometric areas, especially during subtractive finishing processes. 
% U-Net can simultaneously capture local features and global contextual information. Its encoder-decoder structure and skip connections help retain detailed information, which is crucial for predicting inaccessible areas.
% Additionally, U-Net is flexible in handling complex and irregular shapes, making it well-suited to three-dimensional geometries. We replace each fully connected (FC) module in U-Net with an O-CNN block to more efficiently learn the input geometric shapes.
