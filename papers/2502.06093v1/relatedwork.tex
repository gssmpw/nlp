\section{Related Work}
\label{sec:related}

 

The rapid advancements in artificial intelligence (AI) have significantly propelled solutions in digital geometric design and manufacturing~\cite{abdelaal2024ai}. 
AI applications in additive manufacturing~\cite{wang2020smart, zhang2024machine} and subtractive manufacturing~\cite{manikanta2024machine, soori2023machine} have been extensively reviewed.
Cutting-edge research similarly advances computer graphics, encompassing assembly planning~\cite{jones2021automate}, LLM-centric design and manufacturing~\cite{makatura2024can1,makatura2024can2}, 3D printing path optimization~\cite{huang2024learning, liu2024neural}, and feedback-based 3D printing control~\cite{piovarci2022closed}.
%
These advancements underscore a significant trend towards AI-based automation and optimization in design and manufacturing processes.
This paper primarily examines the \manuAna using a learning method, specifically focusing on the \accessAna for subtractive manufacturing~\cite{gupta1997automated, hoefer2017automated}, which is also the main content of this section.



\paragraph{Traditional Manufacturability Analysis}

% In the past, designer
%commercial software: DFMXpress in Solidworks, VAYO, DFMPro.
%Traditional \manuAna for subtractive manufacturing predominantly relies on geometric and analytical techniques, which can be broadly categorized into two main approaches. The first is the feature-based method, which involves extracting machining features as a prerequisite for manufacturability analysis~\cite{han2000manufacturing}. Key challenges in this approach center around feature recognition, addressed through various techniques, such as graph-based methods for machining part analysis~\cite{joshi1988graph}, volumetric decomposition methods that partition the input into convex hull volumes as machining features~\cite{kailash2001volume,kim1990convex}, and hint-based approaches~\cite{regli1995geometric}. The second category comprises feature-less methods, which focus on analyzing the surface representation of the model to determine manufacturability, making them well-suited for arbitrary geometries without requiring feature recognition. This approach tackles challenges in surface representation using techniques like slicing-based methods to map machinable ranges~\cite{li2006machinability} and octree decomposition~\cite{kerbrat2011new}.

Manufacturability of subtractive manufacturing is defined as four characteristics: visibility, reachability, accessibility, and setup complexity \cite{gupta1997automated, hoefer2017automated}.
%
Traditional techniques of \manuAna primarily relies on two approaches: feature-based and feature-less methods~\cite{zhang2020manufacturability}. Feature-based methods extract machining features as a prerequisite, using techniques like graph-based analysis~\cite{joshi1988graph}, volumetric decomposition~\cite{kailash2001volume,kim1990convex}, and hint-based approaches~\cite{regli1995geometric}. Feature-less methods, on the other hand, analyze surface representations to assess manufacturability, employing techniques such as slicing for machinable range mapping~\cite{li2006machinability} and octree decomposition~\cite{kerbrat2011new}.


%\hs{why learning???}

Aligned with these conventional studies, where accessibility is the main evaluation metric for both feature-rich and feature-agnostic methods~\cite{zhang2020manufacturability}, this paper primarily focuses on assessing accessibility.
Unlike traditional approaches, we investigate the use of neural networks for \accessAna. 


\paragraph{Learning based Manufacturability Analysis}

Early efforts to integrate machine learning into \manuAna have demonstrated potential in addressing the limitations of traditional methods. Examples include heuristic rule-based scoring~\cite{kerbrat2011new,joshi2017geometric} and process planning for Quasi-rotational components~\cite{chen2020manufacturability}. Recent advancements leverage deep learning, such as autoencoder-based generative models for feature matching~\cite{yan2023automated}, 3D-CNNs with orthogonal distance fields (ODF) for manufacturability prediction~\cite{balu2020orthogonal}, and enhanced B-rep structures with surface normal data for feature recognition~\cite{ghadai2018learning}. Hierarchical graph neural networks have also been applied to analyze B-rep topology and UV network geometry for multi-level learning~\cite{huang5065158hierarchical}.


A significant limitation of the above studies lies in the challenge of applying learning methods to more intricate aspects, such as predicting cutter accessibility for freeform shapes.
To the best of our knowledge, no prior research has introduced a learning-based \accessAna method applicable to freeform and complex geometries. Our work bridges this gap by enabling real-time prediction of non-reachable and occlusion areas while supporting general cutting tools.


\paragraph{Geometric Accessibility Analysis}

% 应用进入intro, method留下
%but noted the computational expense of Minkowski operations

% Determining accessibility remains a significant challenge in multi-axis CNC machining, with numerous methods and algorithms proposed since the 1990s. 
% Early research on accessibility in multi-axis CNC machining explored various geometric and computational methods. Woo’s spherical visibility map laid the foundation for visibility analysis on 3D surfaces~\cite{woo1994visibility,elber1994accessibility}, while Spyridi and Requicha introduced global and local accessibility concepts for CMM~\cite{spyridi1990accessibility}. Other approaches utilized NURBS surfaces for interference calculations~\cite{lee19952}, configuration space mapping for cutter range feasibility~\cite{choi1997c}, and effective cutter radius evaluations for end mills~\cite{vafaeesefa1998accessibility}.
% To simplify calculations, methods such as sampling-based orientation analysis~\cite{dhaliwal2003algorithms,zhao2018dscarver,mahdavi2020vdac}, plane projections for blisk machining~\cite{chen2015collision}, and boundary-based range construction~\cite{liang2016accessible} have been widely adopted. 
% Advanced techniques include Gaussian spherical mapping~\cite{liu2020sequence} and ray-tracing-based metrics for assessing five-axis milling manufacturability~\cite{chen2021design}. 
Determining accessibility in multi-axis CNC machining remains a significant challenge, with numerous methods proposed since the 1990s. Early research focused on geometric and computational approaches, with Woo’s spherical visibility map laying the foundation for 3D surface visibility analysis~\cite{woo1994visibility,elber1994accessibility}, and Spyridi and Requicha introducing global and local accessibility concepts for CMM~\cite{spyridi1990accessibility}. Other methods explored NURBS surfaces for interference calculations~\cite{lee19952}, configuration space mapping for cutter range feasibility~\cite{choi1997c}, and effective cutter radius evaluations for end mills~\cite{vafaeesefa1998accessibility}.
To simplify calculations, methods such as sampling-based orientation analysis~\cite{dhaliwal2003algorithms,zhao2018dscarver,mahdavi2020vdac}, plane projections for blisk machining~\cite{chen2015collision}, and boundary-based range construction~\cite{liang2016accessible} have been widely adopted. More advanced methods include Gaussian spherical mapping~\cite{liu2020sequence} and ray-tracing metrics for assessing five-axis milling manufacturability~\cite{chen2021design}.




These approaches reflect the progression from basic visibility analysis to more efficient and precise accessibility evaluations in complex machining scenarios. However, when applied to intricate or high-resolution 3D models, they often face scalability challenges and significant computational overhead due to the time-intensive nature of strictly geometric methods~\cite{dai2018support}. Therefore, developing fast and scalable methods capable of handling complex geometries and diverse cutter parameters is highly meaningful.


\paragraph{Spatial Analysis Learning}

Early works extended deep learning methods to 3D voxels~\cite{Maturana2015,Wu2015} for spatial analysis.
However, voxel-based approaches are computationally expensive and memory-intensive, making them unsuitable for high-resolution 3D data.
To address these limitations, sparse voxel-based CNNs leverage octrees~\cite{Wang2017} or hash tables~\cite{Graham2018,Choy2019} to confine computation to sparsely occupied voxels, significantly improving efficiency.
Point-based neural networks~\cite{Qi2017a,Qi2017,Li2018} eliminate the need for voxelization by directly processing point clouds, offering an alternative solution.
Recently, transformers have also been applied to 3D data, demonstrating promising results~\cite{Guo2021,Zhao2021,Wang2023}.
In this work, we adopt O-CNN~\cite{Wang2017} for \accessAna due to its efficiency and strong performance across a range of 3D tasks.




\begin{comment}
% for introduction

%Traditional \manuAna for additive manufacturing focuses on various aspects of the process, with specific considerations in three-axis additive manufacturing, where manufacturability primarily emphasizes avoiding excessively narrow printing contours. Key techniques include slicing-based methods for detecting thin structures~\cite{chen2010manufactruability} and feature recognition approaches utilizing heat kernel signatures~\cite{shi2018manufacturability}. A comprehensive survey on the manufacturability analysis of metal laser-based powder bed fusion additive manufacturing further highlights advancements and challenges in the field~\cite{zhang2020manufacturability}.


\paragraph{Design for Manufacturing}


topology optimization~\cite{lazarov2016length}
topology optimization (visibility map)~\cite{chen2016topology}
topology optimization~\cite{wang2021structural}

design for manufacturing (DFM)~\cite{patterson2021generation,hoefer2017automated,patterson2024manufacturability}

Mold part design(fictitious physical models for manufacturability)~\cite{sato2017manufacturability}

Robot Manipulation and Grasping~\cite{xu2023neural}

========================
%Accessibility is crucial in additive manufacturing (3D printing) as well~\cite{zhang2020manufacturability}, where the tool (printer nozzle) must be able to access all the regions of the build surface, which is important in toolpath generation ~\cite{}, multi-axis 3d printing~\cite{wu2019general}, and sequence planning in 3d printing~\cite{zhong2023continuous}.
%Researchers have investigated methods to predict whether a nozzle can reach specific points of the model, considering overhangs ~\cite{}, support structures ~\cite{}, and the material deposition process ~\cite{}.
========================


\paragraph{Visibility Analysis}

In manufacturing, accessibility refers to whether a tool (or machine) can reach certain areas of a workpiece for tasks such as machining, cutting, or assembly.
It involves determining if the tool can physically interact with or reach specific parts of the workpiece without collisions, interference, or excessive tool movement.
Accessibility is related to visibility in the geometry domain~\cite{zhang2020manufacturability}, which  refers to the ability to see a part of a 3D object from a specific viewpoint or a set of viewpoints. This concept is primarily concerned with whether a point on the surface of an object is visible from a given position, considering obstacles, occlusions, and geometric configurations.

Fast and Accurate Visibility Computation in a 3D Urban Environment
~\cite{gal2012fast}
Deep Learning-Based Atmospheric Visibility Detection~\cite{qu2024deep}
Learning-Based~\cite{uyanik2021machine}

Early work, such as the spherical visibility map, defined visibility on 3D surfaces using hemispheres and their intersections~\cite{chen1992computational,woo1994visibility,elber1994accessibility}. 
Spyridi and Requicha introduced global and local accessibility concepts for coordinate measuring machines (CMM)~\cite{spyridi1990accessibility}, while NURBS surfaces were utilized to compute global tool interference~\cite{lee19952}. 
Configuration space methods have been developed to map obstacles and machine limits into a 2D space for feasible tool range calculations~\cite{choi1997c}. 
Vafaeesefa and ElMaraghy calculated effective tool radii for torus-shaped end mills to enhance local accessibility~\cite{vafaeesefa1998accessibility}, 
while Spitz and Requicha accounted for stylus tip radii and truncated visibility cones for CMM probes~\cite{spitz2000accessibility}.
Sampling-based methods, such as unit sphere partitioning or model face sampling, have been widely applied to determine access orientations in multi-axis NC machining~\cite{suh1995process,dhaliwal2003algorithms}. 
Visibility-based methods captured tool-accessible ranges in 3D Cartesian space~\cite{balasubramaniam2003collision}, while highly precise algorithms exploited the axial symmetry of rotational tools to detect collisions~\cite{ilushin2005precise}. 
Fast algorithms updated accessible regions along smooth tool paths~\cite{wang2007automatic}, and plane-based methods simplified accessible region calculations for blisk machining by projecting them onto 2D reference planes~\cite{chen2015collision}. 
Boundary-based methods further streamlined range construction by simplifying obstacle surfaces into boundary curves~\cite{liang2016accessible}, and dynamic obstacle-growing environments were tackled using optimized machinability algorithms~\cite{chen2018optimized}. 
Recent approaches evaluate tool accessibility using Gaussian spherical mapping~\cite{liu2020sequence} and projection-based strategies~\cite{liu2020sequence}. Chen et al.~\cite{chen2021design} introduced a novel ray-tracing-based metric for five-axis milling manufacturability, analyzing intersections between tool ray samples and part geometry to assess 3D geometric accessibility in part designs.

    
\end{comment}

%Despite their success, adapting these architectures for tool \accessAna poses unique challenges. Neural networks must represent dynamic tool orientations and capture intricate interactions between tools and complex 3D surfaces. Additionally, ensuring the interpretability of predictions and integrating these models with existing manufacturing workflows remain open problems.






\begin{comment}
    Early attempts to integrate machine learning into \manuAna have shown potential in overcoming the limitations of traditional methods. 
For instance, heuristic rules are typically manually configured for scoring~\cite{kerbrat2011new,joshi2017geometric}, while manufacturability analysis for Quasi-rotational components with columnar features depends on pre-arranged process planning schemes~\cite{chen2020manufacturability}. 
Recent innovations include employing auto-encoder-driven deep generative models for machining feature matching~\cite{yan2023automated} and utilizing orthogonal distance fields (ODF) with 3D-CNN to forecast manufacturability~\cite{balu2020orthogonal}. 
Additional noteworthy initiatives involve machining feature recognition via 3D convolutional neural networks (3D-CNNs)~\cite{zhang2018featurenet,peddireddy2021identifying}, enhancing B-rep structures with surface normal data to detect features such as holes~\cite{ghadai2018learning}, and leveraging 3D-CNNs for manufacturability classification~\cite{ghadai2018learning}. Furthermore, hierarchical graph neural networks have been applied to represent topological interconnections of B-rep structures and use UV network geometric information for multi-level learning~\cite{huang5065158hierarchical}.
\end{comment}