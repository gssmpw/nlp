\section{Experiment}
\subsection{Settings and Implement Details}

\paragraph{Implementation details}
Our method is based on the pre-trained state-of-the-art open-source latent video diffusion model, CogVideoX-5B~\cite{cogvideox}. Notice that, we only modify the latent input of the diffusion model, our method might also work on any newly designed text-to-video latent diffusion models~\cite{mochi, hunyuanvideo}, without training.
% In our experiments, by inputting a single prompt, we can generate longer and more coherent videos. 
Each video has a resolution of 480x720, and the inference step is set to 50 following a standard DDIM sampling strategy. Other parameters are the same as the default settings of CogVideoX. To evaluate the proposed methods, we choose 140 prompts from VBench \cite{vbench} and EvalCrafter~\cite{evalcrafter} and use GPT~\cite{chatgpt} to expand them into more detailed descriptions. All the experiments are conducted on a single NVIDIA H100 GPU. Since we only add a temporal latent shift in each step denoising, the proposed method has a similar inference speed compared with direct generation.

\paragraph{Baseline}
Since there is no previous work for open-domain looping video generation from a text description, we majorly compare two generative interpolation methods and one method from the community. The first generative interpolation method is \textit{Svd-Interp.} from Generative Image Inbetween~\cite{wang2024generative}, which is trained on the stable video diffusion model~\cite{svd} for frame interpolation. The other generative interpolation is \textit{CogX-Interp.}\footnote{\url{https://github.com/feizc/CogvideX-Interpolation}}, which is also trained from the image-to-video model of the CogVideoX for frame interpolation. 
To compare, we consider the first frame of our generated results for the starting and ending key frames of the interpolation. Notice that these two methods are based on larger-scale training for frame interpolation. Our method generates the looping video from the text description directly.
\textit{Latent Mix} is a method to achieve this looping video, which has been reported on Github\footnote{https://github.com/THUDM/CogVideo/issues/149}, we compare with this method directly.


% Additionally, we compare with the \textbf{\textit{Naïve}} approach, which simply concatenates short videos generated directly without any additional operations.
% Please add the following required packages to your document preamble:
% \usepackage[normalem]{ulem}
% \useunder{\uline}{\ul}{}
\begin{table}[]
\caption{Quantitative experimental results for different methods under the numerical evaluation metrics. * for the interpolation-based method, we utilize our generated first frame for the start and end keyframe, thus the MSE between the two frames is the oracle value. }
\vspace{-1em}
\begin{tabular}{lccccc}
\toprule
\multicolumn{1}{c}{\textbf{}} & MSE$\downarrow$ & FVD$\downarrow$           & \begin{tabular}[c]{@{}c@{}}CLIP\end{tabular}$\uparrow$ & \begin{tabular}[c]{@{}c@{}}Motion \\ Smooth\end{tabular}$\uparrow$ & \begin{tabular}[c]{@{}c@{}}Dynamic\\ Score\end{tabular}$\uparrow$ \\ \hline
Svd-Interp.*    & {\color{gray}{18.30}}    &5.66     &32.08     &0.9950       & 0.0667             \\
CogX-Interp.*  & {\color{gray}{15.59}}    &28.60    &31.88     &0.9830       &0.3333        \\ \hline
CogVideoX      &66.89    &56.02    &32.19     &0.9738       &0.7056             \\
Latent Mix     &45.17    &60.02    &31.99     &0.9749       &\textbf{0.7273}          \\ 
Ours           &\textbf{25.43}    &\textbf{40.78}    &\textbf{32.24}     &\textbf{0.9850}       &0.4722            \\ \bottomrule
\end{tabular}
\vspace{-1em}
\label{tab:looping}
\end{table}

\paragraph{Evaluation Metrics}
We report the MSE of the first frame and the last frame in the generated videos due to the looping video has the same first frame and the end frame. For the overall video quality, we utilize the widely used FVD~\cite{fvd} and CLIP Score~\cite{clip-score} for comparison. Besides, we give the overall video smoothness and dynamic score of the whole video from VBench~\cite{vbench}.

\begin{figure*}[t]
\centering
\includegraphics[width=\textwidth]{figures/Single_compare/looping1.pdf}
\vspace{-2em}
\caption{Compare with other methods. We give the first frame, the intermediate frame, and the last frame for comparison. Notice that, both Svd-Interp. and Cog-Interp. are frame-interpolation methods, we manually give the same start frame and end frame as key-frames.}
\label{fig:single_compare}
\end{figure*}


\subsection{Comparison with Other Methods}
As introduced before, since current cinemagraph methods can not work on open-domain looping video generation, we compare our method with the state-of-the-art generative frame interpolation methods introduced in the baseline section. As shown in Fig.~\ref{fig:single_compare}, the baseline interpolation methods may produce still results or generate content that is far away from the start frame and the end frame. The latent mix method blending the initial and final latent may result in artifacts in the end frame. Differently, the proposed method can generate the same start and end frames without noticeable differences. Due to the page limitation, we give more examples in Fig.~\ref{fig:additional} and the supplementary video.

As for the numerical comparison, as shown in Tab.~\ref{tab:looping}, the proposed method shows a better visual quality and text-video alignment than previous methods. Besides, we also achieve a relatively higher score with both motion smoothness, video dynamic, and the MSE between the first frame and the last frame, which shows the advantage of the proposed methods. We argue that although the latent mix method gives much dynamic video, the generated content might not be a looping one according to the MSE between the first and the last frame. 
Evaluating the looping videos using current automatic evaluation metrics is also difficult, so we conduct a subjective user study to prove the proposed method's effectiveness further. 
In detail, we invite 23 participants to rank ten questions across three aspects, totaling 690 opinions under five different methods. Each participant will be asked to rank the overall visual quality of the video, the consistency of the video frames, and how dynamic the video is, on a scale of 5 to 1. Finally, we calculate the average score of these opinions. As shown in Table~\ref{tab:user_study}, our method outperforms others in visual quality, temporal quality, and video dynamic.



\section{User evaluation with frequent users of mobile ASR: Lab study and online survey }
To evaluate the usability of our approach, we decided to conduct an in-person lab evaluation of the SpeechCompass phone case and the speech-to-text application (described in Section~\ref{subsection:app}), with frequent users of mobile transcription technology. We first conducted a large-scale online pilot study to inform the design of the in-person lab evaluation, which we conducted with eight deaf or hard-of-hearing participants, set up to mimic a realistic conversation scenario. 

\begin{figure*}
  \centering
  \includegraphics[width=0.75\linewidth]{images/second_study.pdf}
  \caption{Participants' preferences for different visualization techniques in the online survey. A) Results indicating how valuable the specific indicator would be for the user. B) Preferences for the specific indicators for speech direction.} 
  \label{fig:user_preferences_online} 
\end{figure*}


\subsection{Large-scale, online survey (n=494)} In this survey, we use screenshots of our interactive UI prototypes to solicit initial user
feedback on the potential for our proposed approach, to guide the design of a more realistic in-person lab study.

The study was conducted using the same Google Surveys deployment and screening methodology as for the foundational study, detailed in Section 3. The participants were shown different UI renderings and were asked to rate them. The large-scale online survey could only show static images of the interfaces, due to limitations of the survey tool. Out of 985 respondents we focus our analysis on the 494 participants who use captioning technology multiple times per week or more frequently. 

As shown in Figure~\ref{fig:user_preferences_online}A, the colored text was found to be valuable by 60\% of participants. Glyph indicators for speech direction, which included arrow and circle+line indicators, were found valuable by 70\%. The Edge indicator and the mini map had a less positive reception. 

To better understand which glyph indicators were favored, we also asked targeted questions about them, as shown in Figure~\ref{fig:user_preferences_online}B. \emph{Circle + line} was preferred by 13.1\% more respondents than the \emph{highlight box} (45.1\% vs 32.0\%), and the \emph{arrow} was preferred by 21.9\% more respondents than the \emph{circle + line} (51.2\% vs 29.3\%).


\subsection{Lab study (n=8)}
\alex{explain and emphasize intention}
We recruited 8 participants from our institution who were frequent users of captioning technology. Five were female, three were male, and all were deaf or hard of hearing. One participant was 25--34 years old, four were 34--44, one was 45--53, and two were 65+ (we are only allowed to collect age ranges at our institution). 


% setup: https://docs.google.com/document/d/1akr5HVMgJb8Kd9KaEZJcdXn2S0IbHhd8JdBPTE0TiA0/edit?usp=sharing
The study took place in a quiet lab over approximately 60 minutes and used the phone-case prototype (Figure~\ref{fig:pcb_design}) with our mobile ASR application (Figure~\ref{fig:phone_interfaces}). First, the participant was introduced to the technology, prototype, and the purpose of the study. Then, the participant was asked to fill out a background survey, which included demographic questions and their current use and experienced challenges with transcription technology. Afterward, the participant was introduced to different visualization scenarios with the SpeechCompass application. The participant used the SpeechCompass transcription while sitting between the two experimenters, as they all sat around a small table with the SpeechCompass phone case in the center. In each of the seven conditions, which ran for 5 minutes, the experimenters sat across from each other and had short conversations about different topics. The participants were instructed to turn off hearing aid devices if they used any, and were asked to use the SoundCompass UI and transcript to follow the conversation. The experimenters' casual conversations included topics like weekend plans, hobbies, and the weather. The seven conditions, which used the ASR, diarization, and localization functionality for different visualization techniques, are shown in Figure~\ref{fig:ui_options} and presented with more UI context in Figure~\ref{fig:phone_interfaces}. The conditions were:
\begin{enumerate}
    \item \textbf{Transcription only}. The transcribed text is shown in white on a black background. 
    
    \item \textbf{Edge indicator}. A circle (``dot'') that moves around the edge of the screen to point to the currently active speaker. The color of the dot changes based on the direction. 
    
    \item \textbf{Arrow indicator}. A glyph using a colored arrow next to a white text block. The glyph points in the direction of the associated speech. 
    
    \item \textbf{Circle + line indicator}. A glyph using a circle with a directional line next to a white text block. The glyph points in the direction of the speech associated with the text. 
    
    \item \textbf{Mini map}. A colored circle with a smaller circle (``dot'') moves around its edge to point to the currently active speaker. The color of the dot changes based on the direction. 
    
    \item \textbf{Colored text}. The text is colored based on the direction that the associated speech was coming from. 
    
    \item \textbf{Everything on}. All indicators are turned on (except the Circle + line, as it couldn't be used simultaneously with the arrow). 
\end{enumerate}

%five isolated visualization techniques, baseline with just text transcription (no speaker information), and with all visualization turned one. Minimap was shown with an arrow, since we envisioned it would be combined with other techniques. 
After participants had completed all conditions, they filled out a form that asked them to rate how desirable each of the five visual indicator styles (\textit{Edge indicator}, \textit{Arrow}, \textit{Circle  + line}, \textit{Colored map}, and \textit{Colored text}) were on a 7-point Likert scale, from \emph{-3: Strongly dislike} to \emph{+3: Strongly like}. Finally, they were asked to rate the overall value of directional feedback to the transcription experience, how strongly they would recommend these features to users of mobile captioning, and whether they had any general free-form feedback about SpeechCompass. 

\begin{figure*}
  \centering
  \includegraphics[width=0.65\linewidth]{images/study_setup.png}
  \caption{Examples of seven visualization scenarios that participants experienced in the in-person study.} 
  \label{fig:ui_options} 
\end{figure*}

%After running the scenarios, participants filled out the second part of the survey, which asked them to rate each scenario and overall impression on a scale from -3:strongly dislike to +3:strongly like. Finally, the participants filled out free form feedback about the study. 

\begin{figure*}
  \centering
  \includegraphics[width=0.65\linewidth]{images/box_plot_in_person_study_results.png}
  \caption{Boxplots of results of the in-person study. A) Participants' preferences for different visualization techniques. B) Overall opinions about augmented mobile ASR application.\alex{love these plots -- maybe to B you could also add the question about multi-people conversations as the leftmost, since it is also on same scale?} } 
  \label{fig:user_preferences} 
\end{figure*}

\subsection{Results}
Mobile transcription apps (e.g., Android Live Transcribe) were the most used communication technology for the participants. Specifically, three used them multiple times per day, one used them daily, three used them multiple times per week, and one used them rarely. 

75\% of participants frequently experienced the scenario where multiple people would get mixed up in the transcript (two multiple times per day, two daily, two multiple times per week). All participants agreed that it was challenging to participate in conversations when speech was combined from multiple people. 
%Similarly to the online survey, we asked participants to select the biggest challenges they experienced in their use of transcription technology (same options as in Figure~\ref{fig: survey-challenges}). where the majority (6/8) selected \textit{"Have to look away from the person I am talking to"}.  
\\

A Kruskal-Wallis (KW) test found a significant effect
on participant preferences for visualization techniques (P=.014).
The post-hoc pair-wise analyses using the Wilcoxon test with Bonferroni correction did, however, not show statistical significance between any pairs.
Of the five visual indicator styles that participants experienced, \emph{Colored text} was the most well-received (mean ($\bar{x})=2.625$), as it was rated positively by all the participants. %, with six strong like (+3), one like (+2), and one slight like (+1). 
The \emph{Arrow} indicator was also well-received ($\bar{x}=1.125$), with six positive, one negative, and one neutral participant.
%(one strong like (+3), three like (+2) and one slight like (+1)) and one dislike (-1) and one neutral (0)). 
Several participants noted that \emph{Arrow} and \emph{Colored text} worked well together: \emph{"Arrows + color seem to be most easier way to indicate the direction." (P2)} and \emph{"The combination of the colored text with the arrow was the most effective for me." (P7)}.

The other indicator styles received more mixed feedback. The feedback for both \emph{Edge indicator} ($\bar{x}=0.25$) and \emph{Circle + line} ($\bar{x}=-0.125$) was split between four negative and four positive participants. 
Some participants were concerned that \emph{Edge indicator} was distracting and not sufficiently discreet: \textit{"I do prefer the tool be as discrete as possible and would perhaps choose to avoid bright colored things moving around since this would be eye-catching and this kind of attention is often undesired" (P3)} and 
\textit{"Indicator moving around the edge was distracting and causing a bit of eye strain" (P2)}.
On the other hand, another participant found this style particularly useful: \textit{"the color dot moving to the speaker direction worked REALLY well" (P1)}. 
For \emph{Circle + line}, some participants struggled with its legibility: \textit{"If the analog direction indicators were larger (and translucent, or set behind)" (P8)} and \textit{"The lines in a circle were a bit slower and not as accurate (buggy)" (P5)}.
The \emph{Mini map} was rated positively by five participants and negatively by three. The most favorable participant stated: \emph{"this is also great for environmental awareness for those with single-sided hearing or no hearing at all." (P3)} and a participant who disliked the \emph{Edge indicator} commented: \emph{"steady map in the corner worked a bit better (P5)"}.

Overall, all participants agreed with the value of directional feedback ($\bar{x}=2.88$, seven Strongly agree:+3 and one Agree:+2) and would recommend these features to other users of captioning technology ($\bar{x}=2.63$, five Strongly agree:+3 and three Agree:+2): \textit{"I really liked that almost immediately I could tell that there was a speaker change, so that as soon as the text started to show up, I could better contextualize that text as attributed to a new speaker." (P1)}, \textit{"I'm very happy to see this tool being developed, it's a great addition to other speech recognition tools!" (P3)}, and \textit{"This prototype is definitely a life changer and I strongly believe that it will improve the quality of access to communication with speakers for many users" (P6)}.

\subsection{Discussion}
Consistent with the large-scale survey, the value of the diarization and localization features was immediate to all users. The participants were asked if directional guidance would be valuable in their mobile transcription experience. All eight users agreed. Also, all eight users would recommend this feature to mobile captioning users. 

While the large-scale survey helped inform our testing and exclude conditions (e.g., \emph{Highlight box}), the lab study allowed us to more rigorously evaluate the techniques in a realistic scenario. This difference became significant for the \emph{Edge indicator} and \emph{Mini map}, where issues, such as discreetness and distracting aspects, became evident during live usage. 

The results suggest that the combination of \textit{Colored text} and \textit{Arrow} would meet the preferences of most users, thanks to the balance of directional encoding and clarity. The arrow has redundant benefits too, since colored text might not always be reliably visible depending on lighting and screen conditions (e.g., strong sunlight, or dim display) and might also not be usable for colorblind users. The mixed feedback for other techniques indicates that the interface may also benefit from mechanisms that would allow users to customize the visualization style. Such customization could also apply to rendering properties, such as color, transparency, and line thickness, as some participants found \textit{Circle + line} particularly difficult to interpret. In both the large-scale survey and the in-person lab study, the \textit{Arrow} was preferred over \textit{Circle + line}. Through more customization options and extended usage in their daily lives, participants will be able to provide more nuanced feedback about these techniques. 


% Edge indicator and mini map had a less positive reception. However, they were rated more positively than those in the in-person study. Since participants didn't experience the working prototype, the discreet and distracting aspects that were observed in the in-person study were not captured. 

% In both online and in-person study, the arrow directional glyph was preferred to circle+line.



% This dichotomy demonstrates that users should be given a way to customize their experience. For example, the edge indicator received strong likes and dislikes from different participants. 


% This indicates that the interface designers should make the directional glyphs as easy to read as possible.


% The results of the online survey followed what was observed in the in-person study. Edge indicator and mini map had a less positive reception. However, they were rated more positively than those in the in-person study. Since participants didn't experience the working prototype, the discreet and distracting aspects that were observed in the in-person study were not captured. In both online and in-person study, the arrow directional glyph was preferred to circle+line.

% As indicated in the survey, the value of the diarization and localization features was immediate to all users. The participants were asked if directional feedback is valuable in their mobile transcription experience. All eight users agreed. Also, all eight users would recommend this feature to mobile captioning users. 


% \textit{"I really liked that almost immediately I could tell that there was a speaker change, so that as soon as the text started to show up, I could better contextualize that text as attributed to a new speaker." (P1)}

% P3
% Arrows + color seem to be most easier way to indicate the direction.
% \emph{"Arrows + color seem to be most easier way to indicate the direction." (P2)}
% P4
% \textit{"I'm very happy to see this tool being developed, it's a great addition to other speech recognition tools!" (P3)
% }
% \textit{"it was great to see so many options being offered" (P3)
% }
% P6 
% \textit{"This prototype is definitely a life changer and I strongly beleve that it will improve the quality of access to communication with speakers for many users" (P6)}

% P8
% The combination of the colored text with the arrow was the most effective for me.

% \emph{"The combination of the colored text with the arrow was the most effective for me." (P7)}



\subsection{Ablation Studies}
We have given the example in Fig.~\ref{fig:vae} to validate the effectiveness of the proposed frame-invariance latent decoding. Here, we give more ablation studies on the ROPE-Interp. and the skip step in our latent shifting. When performing a latent shift, we can shift the latent $s$ step for denoising, where a small step will be similar to the original inference. As shown in Fig.~\ref{fig:ablation}, when shifting the latent 6 steps in each denoising, the generated content is in a balance of the generated content and the motion. Differently, a small skip will show obversely artifacts. 


\begin{table*}
  [t]
  \centering
  \resizebox{\textwidth}{!}{%
  \begin{tabular}{cccccccccccc}
    \toprule \multicolumn{2}{c}{Components}                                                             & \multicolumn{5}{c}{Re-executability Rate (\%)} & \multicolumn{5}{c}{Readability (\#)} \\
    \cmidrule(lr){1-2} \cmidrule(lr){3-7} \cmidrule(lr){8-12}        \hspace{8pt}\labelemoji\hspace{8pt}                                                                & \hspace{8pt}\toolemoji\hspace{8pt}                                      & O0                                 & O1             & O2             & O3             & AVG            & O0             & O1             & O2             & O3             & AVG            \\
    \hline
    \rowcolor[rgb]{0.93,0.93,0.93}\multicolumn{12}{c}{\textbf{Initialize with LLM4Decompile-End-6.7B~\citep{llm4decompile}}}   \\
    \xmark                                                                                              & \xmark                                    & 69.51                              & 46.95          & 50.61          & 46.34          & 53.35          & 3.98 & 3.41 & 3.44 & 3.38 & 3.55 \\
    \cmark                                                                                              & \xmark                                    & 75.61                              & 50.61          & 50.00          & 50.00          & 56.55          & 4.01 & 3.44 & 3.39 & \textbf{3.49} & 3.58 \\
    \xmark                                                                                              & \cmark                                    & 83.54                     & \textbf{56.10}          & 51.22          & 50.61 & 60.37 & 4.05 & 3.51 & 3.51 & 3.42 & 3.62 \\
    \cmark                                                                                              & \cmark                                    & \textbf{85.37}                            & \textbf{56.10}                     & \textbf{51.83} & \textbf{52.43}          & \textbf{61.43} & \textbf{4.13} & \textbf{3.60} & \textbf{3.54} & \textbf{3.49} & \textbf{3.69} \\

    \rowcolor[rgb]{0.93,0.93,0.93}\multicolumn{12}{c}{\textbf{Initialize with Deepseek-Coder-6.7B-base~\citep{deepseekcoder}}} \\
    \xmark                                                                                              & \xmark                                    & 59.15                              & 35.98          & 39.02          & 37.80          & 42.99          & 3.71 & 3.05 & 3.16 & 3.05 & 3.24 \\
    \cmark                                                                                              & \xmark                                    & 66.46                              & 41.46          & 38.41          & 36.59          & 45.73          & 3.76 & 3.17 & \textbf{3.21} & 3.08 & 3.31 \\
    \xmark                                                                                              & \cmark                                    & 70.73                              & 39.63          & 39.02          & 40.24          & 47.41          & 3.90 & 3.17 & 3.08 & 3.11 & 3.31 \\
    \cmark                                                                                              & \cmark                                    & \textbf{79.88}                     & \textbf{45.73} & \textbf{43.90} & \textbf{42.68} & \textbf{53.05} & \textbf{3.96} & \textbf{3.21} & 3.18 & \textbf{3.19} & \textbf{3.38} \\
    \bottomrule
  \end{tabular}%
  }
  \caption{The ablation study of different methods across four optimization levels
  (O0, O1, O2, O3), as well as their average scores (AVG). The results in bold represent the optimal performance. The ~\labelemoji~ and ~\toolemoji~ means Relabedling and Function Call. \textbf{Bold} denotes the best performance.}
  \label{tab:ablation}
\end{table*}


We also conduct experiments on the RoPE interpolation. In the method, we give two different ways to utilize the interpolated RoPE. As shown in Fig.~\ref{fig:rope}, the fixed RoPE-Interp performs well in our longer video looping generation, allowing each frame to be treated as the first frame during video generation, thereby achieving better looping results.
% which is more similar to the original diffusion model denoising, where the inputted latent has the same position as the model training.


\begin{figure}[!t]
\centering
\includegraphics[width=\columnwidth]{figures/Ablation/ablation_RoPE2025_1_23.pdf}
\vspace{-2em}
\caption{\textbf{Ablation study on RoPE-Interp.} Under the implementation of latent shifting, different RoPE strategies can have a significant impact on the content of video generation.}
\label{fig:rope}
\end{figure}


% \begin{figure}[!t]
\centering
\includegraphics[width=\columnwidth]{figures/Compare with inference time/cmp_with_inference_time2025-1-18.PDF}
\vspace{-2em}
\caption{Time Efficiency.}
\vspace{-1em}
\label{fig:compare_with_inference_time}
\end{figure}
% \noindent\textbf{Time Efficiency  }
% We also conducted temporal efficiency experiments under CogVideoX, as shown in Figure~\ref{fig:compare_with_inference_time}, where the size of the legend also represents the length of different video frames. It can be seen that as the video frames continue to grow, the inference time of various methods is increasing. The Naïve method represents the baseline, that is, the lowest cost time. For Ditctrl, the latent mixing strategy infers too many intermediate latents for mixing, leading to a rapid increase in time cost as the frame length changes. However, our proposed method is only slightly higher than the baseline and much less than the time of Ditfctrl, proving its high temporal efficiency.


\subsection{Applications on Longer Video Generation}


% Since the proposed latent shifting can naturally work for longer video generation, we also compare our method for longer video generation, including \textit{Gen-L-Video}~\cite{gen-l-video} achieves video smoothness by blending the overlapping regions of latents. \textit{FreeNoise} \cite{freenoise} maintains subject consistency using a shuffled sequence of latents. \textit{FIFO}~\cite{fifo} generates each frame iteratively using a queue of latents at different noise levels. \textit{DitCtrl}~\cite{ditctrl} preserves video transitions through a KV-sharing mechanism and latent mixing strategies. 

% We conducted a visual comparison with the aforementioned baseline methods. 

Since the proposed latent shifting can naturally work for longer video generation, we also compare our method on longer video generation, where these baselines have been introduced in Section ~\ref{longer_video_generation} for details. 
% We conducted a visual comparison with the recent state-of-the-art  in the related work, please refer to Section ~\ref{longer_video_generation} for details.
% Each method has its own drawbacks,
As shown in Figure~\ref{fig:single_compare}, directly increasing the size of the latent causes video quality collapse. \textit{Gen-L-Video}~\cite{gen-l-video} produces overly smooth transitions in the background and excessive changes in the direction of the seagull. \textit{FreeNoise}~\cite{freenoise} tends to keep the seagull's orientation constant, the static nature of the image caused by latent shuffling is immediately apparent, and the phenomenon of the seagull having three legs also occurs. Although \textit{FIFO}~\cite{fifo} achieves better motion changes and video coherence, the issues of the seagull changing direction twice in a row and having three legs persist. \textit{DiTCtrl}~\cite{ditctrl} improves the seagull's orientation issue, but still has problems with the defective generation of the seagull's head in the first frame and the three-legged issue. In contrast, the proposed method maintains the seagull's orientation while ensuring coherent video motion. It does not exhibit the issue of the seagull having three legs, thereby achieving superior long video generation.
We give the full comparison in the supplementary video. As for the numerical comparison, we conduct the experiments on the same prompts of our looping video generation and calculate the main numerical results in Tab.~\ref{tab:single} utilizing the well-known metrics from previous studies~\cite{ditctrl,freenoise}.


% Please add the following required packages to your document preamble:
% \usepackage[normalem]{ulem}
% \useunder{\uline}{\ul}{}
\begin{table}[]
\caption{Comparing with other longer video generation methods.}
\vspace{-1em}
\begin{tabular}{lccc}
\toprule
\multicolumn{1}{c}{\textbf{}} &  FVD$\downarrow$          & \begin{tabular}[c]{@{}c@{}}CLIP \\ Score\end{tabular}$\uparrow$ & \begin{tabular}[c]{@{}c@{}}Motion \\ Smooth\end{tabular}$\uparrow $\\ \hline
Gen-L-Video~\cite{gen-l-video}     & 38.15          & 29.57         & \textbf{98.86\%}         \\
FreeNoise~\cite{freenoise}    & 33.56          & \underline{32.34}        & 97.48\%                                                       \\
FIFO~\cite{fifo}    & 41.25   & 32.15       & 96.83\%                                                       \\
DiTCtrl~\cite{ditctrl}    & \underline{31.64}    & 32.13    & 97.89\%                                                       \\ \hline
Ours     & \textbf{29.89} & \textbf{32.43}     &\underline{98.04\%}                                                   \\ \bottomrule
\label{tab:single}
\end{tabular}
\vspace{-2em}
\end{table}

% \subsubsection{Multi-prompt long video generation}
% \subsubsection{Visual Comparison}
% \subsubsection{Quantitative Results}

% \subsubsection{longer video generation}
% \subsubsection{Visual Comparison}
% \subsubsection{Quantitative Results}



\subsection{Limitations}
Since our method is a training-free method based on the pre-trained video diffusion model, our motion prior might be influenced by the pre-trained video diffusion model. As shown in Fig.~\ref{fig:limitation}, we give the results of the successive frame of the generated illustration video. However, the generated dress might not be consistent in the generated results and does not show obvious movement. We argue that this is because of the issues of the motion prior in the pre-trained video diffusion model we use. A better latent diffusion model~\cite{hunyuanvideo, sora} might work better.


One limitation of this study is that it only evaluated LLaVA as the target Vision Language Model (VLM), which may limit the generalizability of the findings to other models. Additionally, the alignment of visual attention heatmaps for non-existing objects was not assessed, indicating that further analysis is needed in this area. 

Moreover, the experiments were conducted solely using the MSCOCO dataset, and future work should expand the evaluation to include additional datasets to ensure the robustness and broader applicability of the results. Furthermore, since datasets that contain both questions and corresponding answers alongside matching segmentation data, which can be used to evaluate object hallucination, are scarce, it may be necessary to develop such datasets.




