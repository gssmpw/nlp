\subsection{Data \label{sec:data}}
Our analysis utilizes the BIOSCAN-5M dataset\footnote{The BIOSCAN-5M dataset contains 5.15\,M arthropod records, each with associated an image and DNA barcode sequence. Although the images are different for each record, the same barcode can occur across multiple records, hence there are fewer than 5\,M unique barcodes.
%However, since multiple images can map to the same DNA barcode sequence, the dataset contains approximately 2.4\,M unique DNA barcodes.
}, a comprehensive collection of 2.4\,M unique DNA barcodes organized into three distinct partitions: {\it (i) Pretrain}: Contains 2.28\,M unique DNA barcodes from unclassified specimens, used for self-supervised pretraining. {\it (ii) Seen}: Encompasses DNA barcodes with validated scientific species names, split into training (118\,k barcodes), validation (6.6\,k barcodes), and test (18.4\,k barcodes) subsets for closed-world evaluation tasks. {\it (iii) Unseen}: Contains novel species with reliable placeholder taxonomic labels, distributed across reference key (12.2\,k barcodes), validation (2.4\,k barcodes), and test (3.4\,k barcodes) subsets for open-world evaluation tasks.
For each sample in \textit{unseen}, its species does not appear in \textit{seen}, but its genus does appear.
%
% The dataset partitioning ensures species-level isolation between the {\it Seen} and the {\it Unseen} partitions, with the test sets also incorporating a flattened species distribution to mitigate taxonomic imbalance.
This structure enables the evaluation of both closed-world classification and open-world species identification capabilities.