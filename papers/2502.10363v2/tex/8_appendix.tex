\section*{Appendix}

\subsection{Reward Functions}
\label{sec:append_reward}
The reward functions we used during the training are shown in Table~\ref{tab:reward}, which mainly comes from~\cite{agarwal2023legged, fu2021minimizing, kumar2021rma, margolis2023walk, long2024learninghumanoid}. The corresponding symbols and their descriptions are provided in Table~\ref{tab:symbol}.
\vspace{-0.1cm}

\renewcommand{\arraystretch}{1.3}
\begin{table}[ht!]
\caption{\textbf{Reward components and weights in Stage I.} Penalty rewards prevent undesired behaviors for sim-to-real transfer, regularization refines motion, and task rewards ensure successful getting up or rolling over.}
\label{tab:reward_stage1}
\centering
\resizebox{\linewidth}{!}{
\begin{tabular}{l c c}
\toprule[0.95pt]
{\scshape Term} & {\scshape Expression} & {\scshape Weight} \\
\midrule[0.6pt]
\multicolumn{3}{l}{\textit{\textbf{Penalty:}}} \\
\midrule[0.6pt]
Torque limits & $ \mathds{1}({\torque \notin [\bs{\tau}_{\min}, \bs{\tau}_{\max} ]}) $ & -0.1 \\
DoF position limits & $ \mathds{1}({\dofpos \notin [\bs{q}_{\min}, \bs{q}_{\max} ]}) $ & -5 \\
Energy & $ \lVert\ \boldsymbol{\tau} \odot \dot{\mathbf{q}} \rVert $ & -1e-4 \\
Termination & $\mathds{1}_\text{termination}$ & -500 \\
\midrule[0.6pt]
\multicolumn{3}{l}{\textit{\textbf{Regularization:}}} \\
\midrule[0.6pt]
DoF acceleration & $\lVert \dofacc \rVert_2$ & -1e-7 \\
DoF velocity & $\lVert \dofvel \rVert_2^2$ & -1e-4 \\
Action rate & $ \lVert \bs{a}_t \rVert_2^2 $ & -0.1 \\
Torque & $\lVert\torque\rVert$ & -6e-7 \\
DoF position error & $\mathds{1} (h_{\text{base}} \geq 0.8) \cdot \exp \left( -0.05 \|\dofpos - \dofpos^{\text{default}}| \right) $ & -0.75 \\
Angular velocity & $\lVert\omega^2\rVert$ & -0.1 \\
Base velocity & $\lVert \bs{v}^2\rVert$ & -0.1 \\
Foot slip & $ \mathds{1} (\bF^{\text{feet}}_{z} > 5.0) \cdot \sqrt{\|\bs{v}_z^{\text{feet}} \|} $ & -1 \\
\midrule[0.6pt]
\multicolumn{3}{l}{\textit{\textbf{Getting-Up Task Rewards:}}} \\
\midrule[0.6pt]
Base height exp & $\exp(\bh^{base}) - 1$ & 5 \\
Head height exp & $\exp(\bh^{head}) - 1$ & 5 \\
$\Delta$ base height & $\mathds{1} (\bh_{t}^{base} > \bh^{base}_{t-1})$ & 1 \\
Feet contact forces reward & $\mathds{1} (\lVert \bF_{t}^{feet} \rVert > \lVert \bF^{\text{feet}}_{t-1}) \rVert$ & 1 \\
Standing on feet reward & $\mathds{1} \big((\lVert \bF^{\text{feet}} \rVert > 0\big)\& \big(\bh^{\text{feet}} < 0.2)\big)$ & 2.5 \\
Body upright reward & $\exp(-\mathbf{g}^{\text{base}}_z)$ & 0.25 \\
Feet height reward & $\exp (-10 \cdot \bh^{\text{feet}}) $ & 2.5 \\
Feet distance reward & \makecell{
$\frac{1}{2} \Big( \exp(-100 \left| \max(\bs{d}_{\text{feet}} - \bs{d}_{\min}, -0.5) \right|)$ \\ 
$+ \exp(-100 \left| \max(\bs{d}_{\text{feet}} - \bs{d}_{\max}, 0) \right|) \Big)$
} & 2 \\
Foot orientation & $ \sqrt{\lVert G_{xy}^{\text{feet}} \rVert} $ & -0.5 \\ 
Soft body symmetry penalty & $ \left\| \mathbf{a}_{\text{left}} - \mathbf{a}_{\text{right}} \right\| $ & -1.0 \\  
Soft waist symmetry penalty & $ \left\| \mathbf{a}^{\text{waist}}\right\| $ & -1.0 \\

\midrule[0.6pt]
\multicolumn{3}{l}{\textit{\textbf{Rolling-Over Task Rewards:}}} \\
\midrule[0.6pt]
Base Gravity Exp & \makecell{
$ \frac{1}{2} \Big( \exp \big( -0.01 (1 - \cos \theta_{\text{left}}) \big) +$ \\ 
$\exp \big( -0.01 (1 - \cos \theta_{\text{right}}) \big) \Big),$ \\ 
$\cos \theta = \frac{\mathbf{g}^{\text{knee}} \cdot \mathbf{g}_{\text{target}}}{\|\mathbf{g}^{\text{base}}\| \|\mathbf{g}_{\text{base}}\|}$
} & 8 \\
Knee Gravity Exp & \makecell{
$ \frac{1}{2} \Big( \exp \big( -0.01 (1 - \cos \theta_{\text{left}}) \big) +$ \\ 
$\exp \big( -0.01 (1 - \cos \theta_{\text{right}}) \big) \Big),$ \\ 
$\cos \theta = \frac{\mathbf{g}^{\text{knee}} \cdot \mathbf{g}_{\text{target}}}{\|\mathbf{g}^{\text{base}}\| \|\mathbf{g}_{\text{base}}\|}$
} & 8 \\
Feet distance reward & \makecell{
$\frac{1}{2} \Big( \exp(-100 \left| \max(\bs{d}_{\text{feet}} - \bs{d}_{\min}, -0.5) \right|)$ \\ 
$+ \exp(-100 \left| \max(\bs{d}_{\text{feet}} - \bs{d}_{\max}, 0) \right|) \Big)$
} & 2 \\
Feet height reward & $\exp (-10 \cdot \bh^{\text{feet}}) $ & 2.5 \\
\bottomrule[0.95pt] 
\end{tabular}
}
\end{table}
\vspace{-0.35cm}

\newtheorem{fact}{Fact}
\newtheorem{observation}{Observation}
\def\header{\vspace{0.8mm} \noindent}

\def\review{\vspace{3mm} \noindent}

\def\algocapup{\vspace{-4mm}}
\def\algocapdown{\vspace{-4mm}}
\def\tblcapup{\vspace{0mm}}
\def\tblcapdown{\vspace{1mm}}
\def\tbldown{\vspace{-0mm}}
\def\figcapup{\vspace{-2mm}}
\def\figcapdown{\vspace{-2mm}}
\def\theoremtop{\vspace{1mm}}
\def\theoremdown{\vspace{1mm}}

\newcommand{\pushright}[1]{\ifmeasuring@#1\else\omit\hfill$\displaystyle#1$\fi\ignorespaces}
\newcommand{\pushleft}[1]{\ifmeasuring@#1\else\omit$\displaystyle#1$\hfill\fi\ignorespaces}
\newcommand{\vit}[1]{{\color{red} #1}}
\newcommand{\try}[1]{{\color{blue} [ #1 ]}}
\newcommand{\redrev}[1]{{\color{red} #1}}
%\def\pi{\mathrm{PPR}}
%\def\diagcapup{\vspace{-2mm}}
%\def\legenddown{\vspace{-2mm}}
%\def\diagramdown{\vspace{-2mm}}
%\def\extrashrink{\vspace{-2mm}}
%\def\done{\hspace*{\fill} {$\square$}}
%\def\mathdone{\pushright{\square}}
%\def\done{\vspace{-0.5mm} \hspace*{\blank} {$\square$}}

\def\la{\langle}
\def\ra{\rangle}

\newcommand{\eqn}[1]{{Equation~(\ref{#1})}}
\newcommand{\reqn}[1]{{Equation~(#1)}}
\newcommand{\ineqn}[1]{{Inequality~(\ref{#1})}}
\newcommand{\rineqn}[1]{{Inequality~(#1)}}
\newcommand{\alg}[1]{{Algorithm~\ref{#1}}}
\newcommand{\ralg}[1]{{Algorithm~{#1}}}
\newcommand{\thm}[1]{{Theorem~\ref{#1}}}
\newcommand{\term}[1]{{Term~(\ref{#1})}}
\newcommand{\nodef}[1]{{{\bf{x}}_{#1}}}
\newcommand{\edgef}[2]{{{\bf{e}}_{#1,#2}}}
\newcommand{\mem}[0]{{\bf mem}}


\def\appG{\hat{G}}
\def\aggfunc{{\mbox{AGGR}}}
\def\memfunc{\mbox{MEM}}
\def\emb{{\bf emb}}
\def\CoNeifunc{{\mbox{CO-REL}}}
\def\rel{{\bf rel}}
\def\Ex{\mathrm{E}}
\def\Var{\mathrm{Var}}
\def\Cov{\mathrm{Cov}}
\newcommand{\lutodo}[1]{{\color{blue} [Lu todo: #1]}}
\newcommand{\jietodo}[1]{{\color{yellow} [Jie todo: #1]}}
\newcommand{\pingtodo}[1]{{\color{orange} [Ping todo: #1]}}
\newcommand{\notsure}[1]{{\color{red} [#1]}}
\def\n{n}
\def\d{\bar{d}}

%\def\I{\mathcal{I}}
%\def\Pcal{\mathcal{I}}
%\def\s{\mathbf{s}}
%\def\t{\mathbf{t}}
% methods
\def\G{\mathcal{G}}
\def\e{\mathbf{e}}
\def\V{\mathcal{V}}
\def\E{\mathcal{E}}
\def\loss{l}
\def\adj{\mathbf{A}}
\def\tadj{\tilde{\mathbf{A}}}
\def\deg{\mathbf{D}}
\def\tdeg{\tilde{\mathbf{D}}}
\def\feat{\mathbf{X}}
\def\featvec{\mathbf{x}}
\def\R{\mathbb{R}}
\def\Rsd{\mathbf{R}}
\def\rsum{\r_{\rm sum}}
\def\W{\mathbf{W}}
\def\w{\mathbf{w}}
\def\prop{\mathbf{P}}
\def\CloseFloat{\setlength{\textfloatsep}{2.5mm}}
\def\OpenFloat{\setlength{\textfloatsep}{16pt}}
\def\step{L}
\def\newemb{\mathbf{Z}'}
\def\Q{\boldsymbol{q}}
\def\q{\boldsymbol{q}}
\def\appemb{\mathbf{\hat{Z}}}
\def\appnewemb{\mathbf{\hat{Z}'}}
\def\embvec{\mathbf{z}}
\def\appembvec{\mathbf{\hat{z}}}
\def\newprop{\mathbf{P}'}
\def\Loss{\mathcal{L}}
\def\Lossb{\mathcal{L}_\mathbf{b}}
\def\appLoss{\hat{\mathcal{L}}}
\def\appLossb{\hat{\mathcal{L}}_\mathbf{b}}
\def\appD{\hat{\mathcal{D}}}
\def\I{\mathcal{I}}
\def\b{\mathbf{b}}
\def\ldeg{\mathbf{d}}
\def\rmax{r_{\max}}
\def\T{\boldsymbol{T}}
\def\t{\boldsymbol{t}}
\def\AL{\mathcal{A}}
\def\appAL{\mathcal{A}}
\def\AUL{\mathcal{M}}
\def\D{\mathcal{D}}
\def\Diag{\mathbf{D}}
\def\P{\mathbf{P}}
\def\H{\mathcal{H}}
\def\Hes{\mathbf{H}}
\def\Y{\mathbf{Y}}
\def\y{\mathbf{y}}
\def\appHes{\hat{\Hes}}
\def\nei{\mathcal{N}}
\def\one{\mathbf{1}}
\def\u{\mathbf{u}}
% \def\v{\mathbf{v}}
\def\w{\mathbf{w}}
\def\optw{\mathbf{w}^\star}
\def\U{\mathbf{U}}
\def\W{\mathbf{W}}
\def\A{\mathbf{A}}
\def\B{\mathbf{B}}
\def\C{\mathbf{C}}
\def\zero{\mathbf{0}}
\def\T{\mathbf{T}}
\def\appT{\hat{\mathbf{T}}}
\def\apppi{{\hat{\boldsymbol{\pi}}}}
\def\ppi{{\boldsymbol{\pi}}}
\def\r{\boldsymbol{r}}
\def\Res{\boldsymbol{R}}

\newenvironment{proofm}
{\par\medskip\indent{\bf\upshape Proof}\hspace{0.5em}\ignorespaces}
{\hfill\par\medskip}

%%% Local Variables:
%%% mode: latex
%%% TeX-master: "paper"
%%% End:

\vspace{-0.1cm}

\subsection{Terrain Curriculum}
\label{sec:append_curriculum}
The training terrains using curriculum comprises \textit{Stones Everywhere}, \textit{Stepping Stones}, and \textit{Balancing Beams}. The \textit{Stones Everywhere} terrain spans an area of $8m\times8m$, while both \textit{Stepping Stones} and \textit{Balancing Beams} are $2m$ in width and $8m$ in length, with single-direction commands. The depth of gaps relative to the ground is set to $1.0m$, and all stones and beams exhibit height variations within $\pm 0.05m$. The depth tolerance threshold, $\epsilon$, is set to $-0.1 m$.

We define terrain difficulty levels ranging from 0 to 8, denoted as $l$. The specific terrain curriculum at each difficulty level are as follows:
\begin{itemize}
    \item \textit{Stones Everywhere}: The stone size is $\max\{0.25, 1.5(1-0.1l)\}$, and the stone distance is $0.05\lceil l/2\rceil$.
    \item \textit{Stepping Stones}: The stone sizes follow the sequence $[0.8, 0.65, 0.5, 0.4, 0.35, 0.3, 0.25, 0.2, 0.2]$, with a maximum stone distance of $0.1 + 0.05l$.
    \item \textit{Balancing Beams}: The stone size is $0.3-0.05\lfloor l/3\rfloor$, with the stone distance in $x$-direction $0.4 - 0.05l$, and in $y$-direction $[0.2, 0.2, 0.2, 0.25, 0.3, 0.35, 0.35, 0.4, 0.2]$. At the highest difficulty level, the terrain forms a single continuous balancing beam.
\end{itemize}

\subsection{Domain Randomization}
\label{sec:append_random}
\begin{table}[h]
    \centering
    \caption{Domain Randomization Setting}
    \begin{tabular}{ll}
    \toprule[1.0pt]
    \textbf{Term} & \textbf{Value}\\
    
    \midrule[0.8pt]
    \multicolumn{2}{c}{\textbf{Observations}} \\ [0.3ex]
    angular velocity noise & $\mathcal{U}(-0.5, 0.5)$ rad/s \\ % [0.05ex]
    joint position noise & $\mathcal{U}(-0.05, 0.05)$ rad/s \\ % [0.05ex]
    joint velocity noise & $\mathcal{U}(-2.0, 2.0)$ rad/s \\ % [0.05ex]
    projected gravity noise & $\mathcal{U}(-0.05, 0.05)$ rad/s \\ % [0.05ex]
    
    \midrule[0.5pt] 
    \multicolumn{2}{c}{\textbf{Humanoid Physical Properties}} \\ [0.3ex]
    actuator offset & $\mathcal{U}(-0.05, 0.05)$ rad \\ % [0.05ex]
    motor strength noise & $\mathcal{U}(0.9, 1.1)$ \\ % [0.05ex]
    payload mass & $\mathcal{U}(-2.0, 2.0)$ kg \\ % [0.05ex]
    center of mass displacement & $\mathcal{U}(-0.05, 0.05)$ m \\ % [0.05ex]
    Kp, Kd noise factor & $\mathcal{U}(0.85, 1.15)$ \\ % [0.05ex]

    \midrule[0.5pt] 
    \multicolumn{2}{c}{\textbf{Terrain Dynamics}} \\ [0.3ex]
    friction factor & $\mathcal{U}(0.4, 1.0)$ \\ % [0.05ex]
    restitution factor & $\mathcal{U}(0.0, 1.0)$ \\ % [0.05ex]
    terrain height noise & $\mathcal{U}(-0.02, 0.02)$ m \\ % [0.05ex]

    \midrule[0.5pt] 
    \multicolumn{2}{c}{\textbf{Elevation Map}} \\ [0.3ex]
    vertical offset & $\mathcal{U}(-0.03, 0.03)$ m \\ % [0.05ex]
    vertical noise & $\mathcal{U}(-0.03, 0.03)$ m \\ % [0.05ex]
    map roll, pitch rotation noise & $\mathcal{U}(-0.03, 0.03)$ m \\ % [0.05ex]
    map yaw rotation noise & $\mathcal{U}(-0.2, 0.2)$ rad \\ % [0.05ex]
    foothold extension probability & $0.6$ \\ % [0.05ex]
    map repeat probability & $0.2$ \\ % [0.05ex]
    
    \bottomrule[1.0pt]
    \end{tabular}
    \label{tab:domain_random}
\end{table}
\vspace{-0.15cm}

\subsection{Hyperparameters}
\vspace{-0.15cm}
\begin{table}[ht]\small
\centering
\caption{Hyperparameter Study on Poisson equation.}
\label{table:hp}
\renewcommand\arraystretch{0.6}
\begin{sc}
    \renewcommand{\multirowsetup}{\centering}
    % \setlength{\tabcolsep}{4.7pt}
    \resizebox{0.8\linewidth}{!}{
    \begin{tabular}{c|c|c}
       \toprule
       Type & Configuration & Iteration \\ 
       \midrule
       \multirow{5}{*}{Feature Width} & 8 & 226 \\
       & 16 & 224 \\
       & \textbf{32} & \textbf{184} \\
       & 64 & 308 \\
       & 128 & 356 \\
       \midrule
       \multirow{5}{*}{Pre Ite} & \textbf{1} & \textbf{184} \\
       & 2 & 217 \\
       & 3 & 268 \\
       & 4 & 225 \\
       & 5 & 225 \\
       \midrule
       \multirow{5}{*}{Post Ite} & \textbf{1} & \textbf{184} \\
       & 2 & 219 \\
       & 3 & 223 \\
       & 4 & 218 \\
       & 5 & 254 \\
       \midrule
       \multirow{5}{*}{Num C} & 8 & 216 \\
       & 16 & 234 \\
       & 32 & 205 \\
       & 64 & 233 \\
       & \textbf{128} & \textbf{184} \\
       \midrule
       \multirow{5}{*}{Num Heads} & 1 & 229 \\
       & 2 & 251 \\
       & \textbf{4} & \textbf{184} \\
       & 8 & 196 \\
       & 16 & 302 \\
       \bottomrule
    \end{tabular}}
\end{sc}
\end{table}