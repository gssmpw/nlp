\section{Conclusion}

In this paper, we proposed a novel framework, \beamdojo, which enables humanoid robots to traverse with agility and robustness on sparse foothold terrains such as stepping stones and balance beams, and generalize to a wider range of challenging terrains (e.g., gaps, balancing beams). The key conclusions are summarized as follows:

\begin{itemize}
    \item \textbf{Accuracy of Foot Placement:} We introduced a foothold reward for polygonal feet, which is proportional to the contact area between the footstep and the safe foothold region. This continuous reward effectively encourages precise foot placements.
    
    \item \textbf{Training Efficiency and Effectiveness:} By incorporating a two-stage reinforcement learning training process, \beamdojo enables full trial-and-error exploration. Additionally, the double-head critic significantly enhances the learning of sparse foothold rewards, regularizes gait patterns, and facilitates long-distance foot placement planning.
    
    \item \textbf{Agility and Robustness in the Real World:} Our experiments demonstrate that \beamdojo empowers humanoid robots to exhibit agility and achieve a high success rate in real-world scenarios. The robots maintain stable walking even under substantial external disturbances and the inevitable sway of beams in real world. Notably, by leveraging LiDAR-based mapping, we have achieved stable backward walking, a challenge typically encountered with depth cameras.
\end{itemize}
