\begin{tcolorbox}[gray_box, title = {{Synthetic Data Example Type II-I: Verify correctness}}]\tiny
[Input]
\begin{verbatim}
A library has a total of $50$ books, each of which is categorized as a biography, novel, or self-help book. The library's patrons 
borrow $30$ books in one week, and the ratio of borrowed biographies to borrowed novels to borrowed self-help books is $2:3:5$. 
If the library has an equal number of biographies, novels, and self-help books, how many self-help books were borrowed in that week?
\end{verbatim}

\hdashrule[0.5ex]{\linewidth}{0.1mm}{0.1mm} % Line width, dash length, gap length

[Output]
\begin{verbatim}
First, we determine the number of biographies, novels, and self_help books in the library. Since the library has an equal number of 
each type of book and the total number of books is 50, we can divide 50 by 3:

\[ \text{Number of biographies} = \text{Number of novels} = \text{Number of self_help books} = \frac{50}{3} \]

However, since the number of books must be an integer and the problem states that the library has an equal number of each type of 
book, we need to re-evaluate the problem statement. The problem states that the library has an equal number of biographies, novels, 
and self_help books, but the ratio of borrowed 
biographies to borrowed novels to borrowed self_help books is \(2:3:5\). This means the total number of books borrowed must be a 
multiple of \(2+3+5=10\). Since 30 books were borrowed, it is 
indeed a multiple of 10.

<|reflect|>
Let me stop here to verify. The calculation of the number of each type of book in the library is correct, but the focus should be
on the ratio of borrowed books, which is consistent with the given total of 30 borrowed books.<|continue|>
Let \( b \), \( n \), and \( s \) represent the number of biographies, novels, and self_help books borrowed respectively. According 
to the given ratio \(2:3:5\), we can write:
\[ b = 2x \]
\[ n = 3x \]
\[ s = 5x \]<|continue|>
The total number of books borrowed is 30:
\[ b + n + s = 30 \]<|continue|>
Substitute the expressions for \( b \), \( n \), and \( s \) into the equation:
\[ 2x + 3x + 5x = 30 \]<|continue|>
Combine like terms:
\[ 10x = 30 \]<|continue|>
Solve for \( x \):
\[ x = \frac{30}{10} = 3 \]<|continue|>
Find the number of self_help books borrowed:
\[ s = 5x = 5 \times 3 = 15 \]

Therefore, the number of self_help books borrowed in that week is:
\[ \boxed{15} \]
\end{verbatim}

\end{tcolorbox}