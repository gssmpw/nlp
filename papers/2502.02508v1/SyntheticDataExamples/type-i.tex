\begin{tcolorbox}[gray_box, title = {{Synthetic Data Example Type I: No reflection}}]\tiny
[Input]
\begin{verbatim}
Five couples plan to take their boats on a trip down a river. Each boat can carry 4 people. What is the smallest number of boats
that can accommodate all five couples?
\end{verbatim}

\hdashrule[0.5ex]{\linewidth}{0.1mm}{0.1mm} % Line width, dash length, gap length

[Output]
\begin{verbatim}
First, we determine the total number of people. Since there are five couples, and each couple consists of 2 people, the total number
of people is:

\[ 5 \times 2 = 10 \]<|continue|>
Next, we need to figure out how many boats are required to carry all 10 people. Each boat can carry 4 people. To find the minimum
number of boats, we divide the total number of people by the capacity of each boat:

\[ \frac{10}{4} = 2.5 \]<|continue|>
Since we cannot have half a boat, we need to round up to the next whole number. Therefore, we need 3 boats to accommodate all 
10 people.

To verify, let's consider how the people can be distributed among the boats:
- Boat 1: 4 people
- Boat 2: 4 people
- Boat 3: 2 people

All 10 people are accounted for, and no boat exceeds its capacity.

Thus, the smallest number of boats that can accommodate all five couples is:

\[ \boxed{3} \]
\end{verbatim}
\end{tcolorbox}