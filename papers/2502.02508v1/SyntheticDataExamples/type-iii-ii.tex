\begin{tcolorbox}[gray_box, title = {{Synthetic Data Example Type III-II: Correct mistake multiple times}}]\tiny
[Input]
\begin{verbatim}
In $\triangle ABC$, $a$, $b$, $c$ are the sides opposite to the angles $A$, $B$, $C$ respectively. Given $c=4$, $\tan A=3$, 
$\cos C= \frac{ \sqrt{5}}{5} $, find the area of $\triangle ABC$.
\end{verbatim}

\hdashrule[0.5ex]{\linewidth}{0.1mm}{0.1mm} % Line width, dash length, gap length

[Output]
\begin{verbatim}
First, we find \(\sin C\) using the Pythagorean identity \(\sin^2 C + \cos^2 C = 1\):

\[
\sin^2 C = 1 - \cos^2 C = 1 - \left(\frac{\sqrt{5}}{5}\right)^2 = 1 - \frac{5}{25} = \frac{20}{25} = \frac{4}{5}
\]

Since \(C\) is an angle in a triangle, \(\sin C\) is positive:

\[
\sin C = \frac{2}{5}
\]<|continue|>
Next, we find \(\sin A\) using the given \(\tan A = 3\). Recall that \(\tan A = \frac{\sin A}{\cos A}\). Let \(\sin A = 3k\) and 
\(\cos A = k\). Using the Pythagorean identity \(\sin^2 A + \cos^2 A = 1\):

\[
(3k)^2 + k^2 = 1 \implies 9k^2 + k^2 = 1 \implies 10k^2 = 1 \implies k^2 = \frac{1}{10} \implies k = \frac{1}{\sqrt{10}} = 
\frac{\sqrt{10}}{10}
\]

Thus, \(\sin A = 3k = 3 \cdot \frac{\sqrt{10}}{10} = \frac{3\sqrt{10}}{10}\) and \(\cos A = k = \frac{\sqrt{10}}
{10}\).<|continue|>
We use the Law of Sines to find \(a\):

\[
\frac{a}{\sin A} = \frac{c}{\sin C} \implies \frac{a}{\frac{3\sqrt{10}}{10}} = \frac{4}{\frac{2}{5}} \implies \frac{a}
{\frac{3\sqrt{10}}{10}} = 4 \cdot \frac{5}{2} = 10 \implies a = 10 \cdot \frac{3\sqrt{10}}{10} = 3\sqrt{10}
\]

<|reflect|>
Wait a minute. The calculation of \(\sin C\) is incorrect because \(\sin C\) should be \(\frac{2}{\sqrt{5}}\) instead of \
(\frac{2}{5}\).<|explore|>
This time, Use the correct value of \(\sin C = \frac{2}{\sqrt{5}}\) in subsequent calculations.<|continue|>
Correct the calculation for \(\sin C\):

\[
\sin C = \frac{2}{\sqrt{5}}
\]<|continue|>
Use the correct value of \(\sin C\) in the Law of Sines to find \(a\):

\[
\frac{a}{\sin A} = \frac{c}{\sin C} \implies \frac{a}{\frac{3\sqrt{10}}{10}} = \frac{4}{\frac{2}{\sqrt{5}}} \implies 
\frac{a}{\frac{3\sqrt{10}}{10}} = 4 \cdot \frac{\sqrt{5}}{2} = 2\sqrt{5} \implies a = 2\sqrt{5} \cdot \frac{3\sqrt{10}}{10} = 
\frac{6\sqrt{50}}{10} = \frac{6 \cdot 5\sqrt{2}}{10} = 3\sqrt{2}
\]<|continue|>
 Use the formula for the area of a triangle \( \text{Area} = \frac{1}{2}ab \sin C \):

\[
\text{Area} = \frac{1}{2} \cdot 3\sqrt{2} \cdot 4 \cdot \frac{2}{\sqrt{5}} = \frac{1}{2} \cdot 12\sqrt{2} \cdot \frac{2}
{\sqrt{5}} = \frac{1}{2} \cdot \frac{24\sqrt{2}}{\sqrt{5}} = 12 \cdot \frac{\sqrt{2}}{\sqrt{5}} = 12 \cdot \sqrt{\frac{2}{5}} = 
12 \cdot \frac{\sqrt{10}}{5} = \frac{12\sqrt{10}}{5}
\]

Therefore, the area of \(\triangle ABC\) is: \(\boxed{\frac{12\sqrt{10}}{5}}\).<|reflect|>
Let's reassess my previous conclusion. The mistake is in the calculation of the area, where the value of \(\sin B\) should 
have been used instead of \(\sin C\).<|explore|>
To find a better solution, Use the correct value of \(\sin B = \frac{\sqrt{2}}{2}\) in the area formula: \(\text{Area} = 
\frac{1}{2} \cdot 3\sqrt{2} \cdot 4 \cdot \frac{\sqrt{2}}{2} = 6\).<|continue|>
correct the calculation of the area using the correct value of \(\sin B = \frac{\sqrt{2}}{2}$: \(\text{ Area} = \frac{1}{2} 
\cdot 3\sqrt{2} \cdot 4 \cdot \frac{\sqrt{2}}{2} = 6\).

Therefore, the area of \(\triangle ABC\) is: \(\boxed{6}\).
\end{verbatim}

\end{tcolorbox}