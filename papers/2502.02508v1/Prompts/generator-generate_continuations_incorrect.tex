\begin{tcolorbox}[brown_box, title = {{Prompt Template 1.2.2 --- Generator generates continuations for incorrect partial solutions}}]\tiny
\begin{verbatim}
<|im_start|>system
Your task is to continue writing a solution to a problem. Given a problem, an incorrect partial solution where the last step is 
wrong, along with a revision feedback and a hint for next steps, you should continue the solution using the hint. Do not directly 
copy contents under "## Incorrect partial solution with revision feedback and hint for next steps". Think step by step using
English and adhere to the following format.

Step: [Brief explanation and calculations]

Step: [Brief explanation and calculations]

...

**Please do not use other language than English.**

Your solution should conclude with "Therefore, the final answer is: \(\boxed{answer}\)", where [answer] is the final number or 
expression that solves the problem. All steps in the solutions should be brief and always start with "Step:" and end with two 
line breaks. Each subsequent step should not explicitly refer to previous steps, and the steps should constitute a coherent
progression towards the final answer.<|im_end|>
<|im_start|>user
## Problem
Devanti wears her hair in various styles. A sleek bun is tied in top of the head, a half on the other with a long extension hanging
down to the shoulders. To prepare for his next photo shoot, Devanti decides to ward off a few days of the daily grays until it 
reaches there full potential. Research shows that one-fourth of the gray hairs are deposited on the first day, and the second day 
is when half the remaining hair has disappeared,  leaving Devanti with 15 gray hairs on her full head. Calculate the total number
of the gray hairs that Devanti was able to tadpoles from her hair.

## Incorrect partial solution with revision feedback and hint for next steps

Step: Let \( x \) be the total number of gray hairs Devanti had initially.

Step: On the first day, one-fourth of the gray hairs are deposited on the head. Therefore, the number of gray hairs remaining 
after the first day is:
\[
x - \frac{1}{4}x = \frac{3}{4}x
\]
Wait. I subtracted one-fourth of x instead of adding it, but gray hairs are deposited (added) on the first day. Alternatively, 
I should add one-fourth of x to find the total number of gray hairs after the first day.<|im_end|>
<|im_start|>assistant
## Continued solution

Step: Let me correct the first day's calculation by adding one-fourth of the gray hairs.

\[
\text{Initial gray hairs} = x
\]
\[
\text{Gray hairs deposited on the first day} = \frac{1}{4}x
\]
\[
\text{Total gray hairs after the first day} = x + \frac{1}{4}x = \frac{5}{4}x
\]

Step: Then, calculate the number of gray hairs remaining after the second day, where half of them disappear.

\[
\text{Gray hairs after the second day} = \frac{1}{2} \times \frac{5}{4}x = \frac{5}{8}x
\]

Step: Next, set up the equation based on the information that 15 gray hairs remain after the second day.

\[
\frac{5}{8}x = 15
\]

Step: Finally, let's solve for \( x \) to find the initial number of gray hairs.

\[
x = 15 \times \frac{8}{5} = 24
\]

Therefore, the final answer is: \(\boxed{24}\).<|im_end|>
<|im_start|>user
## Problem
<<<instruction>>>

## Incorrect partial solution with revision feedback and hint for next steps

<<<trajectory>>><|im_end|>
<|im_start|>assistant
## Continued solution

Step:
\end{verbatim}
\end{tcolorbox}