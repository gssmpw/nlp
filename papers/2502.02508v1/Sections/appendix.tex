\begingroup
\hypersetup{linkcolor=darkblue} % Set link color
\addtocontents{toc}{\protect\StartAppendixEntries}
\listofatoc
\endgroup


\section{Satori's Demo Examples}  \label{sec:demo}
\begin{figure*}[htbp]
    \centering
     \includegraphics[width=1.0\textwidth]
     {Figures/math-4.pdf}
     \caption{\textbf{Math Domain Example.} Satori verifies the correctness of the intermediate steps and proceeds to the next reasoning step.}
\end{figure*}
\vspace{-2em}
\begin{figure*}[htbp]
    \centering
     \includegraphics[width=1.0\textwidth]
     {Figures/math-1.pdf}
     \caption{\textbf{Math Domain Example.} Satori identifies the mistakes in the previous solution and proposes an alternative correct solution.}
\end{figure*}
\vspace{-2em}
\begin{figure*}[htbp]
    \centering
     \includegraphics[width=1.0\textwidth]
     {Figures/math-2.pdf}
     \caption{\textbf{Math Domain Example.} Satori verifies the correctness of previous solution and initiates a different solution.}
\end{figure*}
\vspace{-2em}
\begin{figure*}[htbp]
    \centering
     \includegraphics[width=1.0\textwidth]
     {Figures/math-5.pdf}
     \caption{\textbf{Math Domain Example.} Satori verifies the correctness of previous solution and further explores a simpler solution.}
\end{figure*}
\vspace{-2em}
\begin{figure*}[htbp]
    \centering
     \includegraphics[width=1.0\textwidth]
     {Figures/math-3.pdf}
     \caption{\textbf{Math Domain Example.} 1) Satori verifies the correctness of intermediate steps in early stage. 2) Satori realizes that the pervious solution is actually erroneous and then proposes an alternative correct solution.}
\end{figure*}
\vspace{-2em}
\begin{figure*}[htbp]
    \centering
     \includegraphics[width=1.0\textwidth]
     {Figures/logic.pdf}
     \caption{\textbf{Out-of-domain Example.} 1) Satori identifies the potential mistakes in intermediate steps and initiates another solution. 2) Satori realizes that the pervious solution is still erroneous and then proposes an alternative correct solution.}
\end{figure*}
\vspace{-2em}
\begin{figure*}[htbp]
    \centering
     \includegraphics[width=1.0\textwidth]
     {Figures/commonsense.pdf}
     \caption{\textbf{Out-of-domain Example.} Satori identifies the potential mistakes in intermediate steps and initiates another correct solution.}
\end{figure*}
\vspace{-2em}
\begin{figure*}[htbp]
    \centering
     \includegraphics[width=1.0\textwidth]
     {Figures/chemistry.pdf}
     \caption{\textbf{Out-of-domain Example.} 1) Satori verifies the correctness of intermediate steps in early stage. 2) Satori realizes that the pervious solution is actually erroneous and then proposes an alternative correct solution.}
\end{figure*}
\vspace{-2em}
\begin{figure*}[htbp]
    \centering
     \includegraphics[width=1.0\textwidth]
     {Figures/economics.pdf}
     \caption{\textbf{Out-of-domain Example.} Satori engages in multiple self-reflection processes during intermediate reasoning steps.}
\end{figure*}
\vspace{-2em}
\begin{figure*}[htbp]
    \centering
     \includegraphics[width=1.0\textwidth]
     {Figures/code.pdf}
     \caption{\textbf{Out-of-domain Example.} 1) Satori verifies the correctness of intermediate steps in early stage. 2) Satori realizes that the pervious solution is actually erroneous and then proposes an alternative correct solution.}
\end{figure*}
\vspace{-2em}
\begin{figure*}[htbp]
    \centering
     \includegraphics[width=1.0\textwidth]
     {Figures/table.pdf}
      \caption{\textbf{Out-of-domain Example.} Satori identifies the mistakes in previous solution and proposes an alternative correct solution.}
\end{figure*}


\clearpage
\putsec{related}{Related Work}

\noindent \textbf{Efficient Radiance Field Rendering.}
%
The introduction of Neural Radiance Fields (NeRF)~\cite{mil:sri20} has
generated significant interest in efficient 3D scene representation and
rendering for radiance fields.
%
Over the past years, there has been a large amount of research aimed at
accelerating NeRFs through algorithmic or software
optimizations~\cite{mul:eva22,fri:yu22,che:fun23,sun:sun22}, and the
development of hardware
accelerators~\cite{lee:cho23,li:li23,son:wen23,mub:kan23,fen:liu24}.
%
The state-of-the-art method, 3D Gaussian splatting~\cite{ker:kop23}, has
further fueled interest in accelerating radiance field
rendering~\cite{rad:ste24,lee:lee24,nie:stu24,lee:rho24,ham:mel24} as it
employs rasterization primitives that can be rendered much faster than NeRFs.
%
However, previous research focused on software graphics rendering on
programmable cores or building dedicated hardware accelerators. In contrast,
\name{} investigates the potential of efficient radiance field rendering while
utilizing fixed-function units in graphics hardware.
%
To our knowledge, this is the first work that assesses the performance
implications of rendering Gaussian-based radiance fields on the hardware
graphics pipeline with software and hardware optimizations.

%%%%%%%%%%%%%%%%%%%%%%%%%%%%%%%%%%%%%%%%%%%%%%%%%%%%%%%%%%%%%%%%%%%%%%%%%%
\myparagraph{Enhancing Graphics Rendering Hardware.}
%
The performance advantage of executing graphics rendering on either
programmable shader cores or fixed-function units varies depending on the
rendering methods and hardware designs.
%
Previous studies have explored the performance implication of graphics hardware
design by developing simulation infrastructures for graphics
workloads~\cite{bar:gon06,gub:aam19,tin:sax23,arn:par13}.
%
Additionally, several studies have aimed to improve the performance of
special-purpose hardware such as ray tracing units in graphics
hardware~\cite{cho:now23,liu:cha21} and proposed hardware accelerators for
graphics applications~\cite{lu:hua17,ram:gri09}.
%
In contrast to these works, which primarily evaluate traditional graphics
workloads, our work focuses on improving the performance of volume rendering
workloads, such as Gaussian splatting, which require blending a huge number of
fragments per pixel.

%%%%%%%%%%%%%%%%%%%%%%%%%%%%%%%%%%%%%%%%%%%%%%%%%%%%%%%%%%%%%%%%%%%%%%%%%%
%
In the context of multi-sample anti-aliasing, prior work proposed reducing the
amount of redundant shading by merging fragments from adjacent triangles in a
mesh at the quad granularity~\cite{fat:bou10}.
%
While both our work and quad-fragment merging (QFM)~\cite{fat:bou10} aim to
reduce operations by merging quads, our proposed technique differs from QFM in
many aspects.
%
Our method aims to blend \emph{overlapping primitives} along the depth
direction and applies to quads from any primitive. In contrast, QFM merges quad
fragments from small (e.g., pixel-sized) triangles that \emph{share} an edge
(i.e., \emph{connected}, \emph{non-overlapping} triangles).
%
As such, QFM is not applicable to the scenes consisting of a number of
unconnected transparent triangles, such as those in 3D Gaussian splatting.
%
In addition, our method computes the \emph{exact} color for each pixel by
offloading blending operations from ROPs to shader units, whereas QFM
\emph{approximates} pixel colors by using the color from one triangle when
multiple triangles are merged into a single quad.

 \label{app:related}

\clearpage
\section{Details about Data Synthesis Framework} \label{sec:synthesis}

\begin{figure*}[!t]
    \centering
     \includegraphics[width=0.8\textwidth]
     {Figures/synthesis.pdf}
\caption{\textbf{Demonstration Trajectories Synthesis.} First, multiple initial reasoning trajectories are sampled from the generator and sent to critic to ask for feedback. The critic model identifies the mistake for trajectories with incorrect final answers and proposes an alternative solution. For trajectories with correct final answers, the critic model provides verification of its correctness. Based on the feedback, the generator self-refines its previous trajectories, and the incorrect trajectories are sent to the critic again for additional feedback with maximum $m$ iterations. At each step, those trajectories with successful refinements are preserved and finally, a reward model rates and collects high-quality demonstration trajectories to form the synthetic dataset $\Dc_{\text{syn}}$.}
\label{fig:synthesis}
\end{figure*}

\paragraph{Sample Initial Trajectories.}
The details of data synthesis framework are illustrated in Figure~\ref{fig:synthesis}. Given an input problem $\boldsymbol{x}\in \Dc$, we begin by sampling the generator $\pi_g$ to generate $K$ initial reasoning trajectories. For each trajectory $\boldsymbol{\Tilde{y}} = [\boldsymbol{\Tilde{y}}_{1}, \boldsymbol{\Tilde{y}}_{2}, \ldots, \boldsymbol{\Tilde{y}}_{L}] \sim \pi_g(\cdot | \boldsymbol{x})$, we evaluate whether the final answer $\boldsymbol{\Tilde{y}}_{L}$ matches the ground-truth answer $\boldsymbol{y^*}$. Based on the evaluation, the generated trajectories are divided into two subsets according to their correctness, which are then processed differently in subsequent steps.

\paragraph{Critic and Refinement.}
For those incorrect trajectories, the critic $\pi_c$ provides feedback to help the generator address its flaws. Specifically, the critic, given the ground-truth solution, identifies the first erroneous step $\boldsymbol{\Tilde{y}}_{l}$ and generates a summary $\boldsymbol{\Tilde{y}}_{l+1}$ of the mistake as a reflection, along with a exploration direction (hint), $\boldsymbol{\Tilde{y}}_{l+2}$, i.e., $[\boldsymbol{\Tilde{y}}_{l+1}, \boldsymbol{\Tilde{y}}_{l+2}] \sim \pi_c(\cdot | \boldsymbol{x}, \boldsymbol{\Tilde{y}}_{1}, \ldots, \boldsymbol{\Tilde{y}}_{l}; \boldsymbol{y^*})$. Next, we ask the generator $\pi_g$ to self-refine its current trajectory based on the feedback provided by the critic, performing a conditional generation of the remaining reasoning steps, $[\boldsymbol{\Tilde{y}}_{l+3}, \ldots, \boldsymbol{\Tilde{y}}_{L}] \sim \pi_g(\cdot | \boldsymbol{x}, \boldsymbol{\Tilde{y}}_{1}, \ldots, \boldsymbol{\Tilde{y}}_{l}; \boldsymbol{\Tilde{y}}_{l+1}, \boldsymbol{\Tilde{y}}_{l+2})$. 

For correct trajectories, the critic $\pi_c$ focuses on verifying the correctness of the generator’s reasoning. A random intermediate reasoning step $\boldsymbol{\Tilde{y}}_{l}$ is selected, and the critic provides a summary explaining why the preceding steps are progressing toward the correct solution, i.e., $\boldsymbol{\Tilde{y}}_{l+1} \sim \pi_c(\cdot | \boldsymbol{x}, \boldsymbol{\Tilde{y}}_{1}, \ldots, \boldsymbol{\Tilde{y}}_{l}; \boldsymbol{y^*}), \text{where} \ \boldsymbol{\Tilde{y}}_{l+1}$. Similarly, the generator continues from the current solution, generating the subsequent steps as $[\boldsymbol{\Tilde{y}}_{l+2}, \ldots, \boldsymbol{\Tilde{y}}_{L}] \sim \pi_g(\cdot | \boldsymbol{x}, \boldsymbol{\Tilde{y}}_{1}, \ldots, \boldsymbol{\Tilde{y}}_{l}; \boldsymbol{\Tilde{y}}_{l+1})$.

Finally, we check whether the final answer $\boldsymbol{\Tilde{y}}_{L}$ aligns with $\boldsymbol{y^*}$. The above procedure is repeated iteratively, up to a maximum of $m$ iterations, until the generator produces the correct final answer. All feedback and refinements are then contaminated to synthesize the final demonstration trajectories. Additionally, the trajectories are post-processed by inserting meta-action tokens at the beginning of each reasoning step to indicate its meta-action type.

\paragraph{Trajectory Filtering.}
The above procedure may yield multiple demonstration trajectories for each problem $\boldsymbol{x}$. We then select the top-$k$ ($k<K$) trajectories based on the reward scores $r = \pi_r(\boldsymbol{\Tilde{y}}, \boldsymbol{x})$ assigned by the reward model $\pi_r$. This approach allows us to construct a diverse synthetic dataset $\Dc_{\text{syn}}$ containing high-quality demonstration trajectories, including (1) short-cut COAT paths that boil down CoT reasoning paths and (2) more complex COAT paths involving multiple rounds of self-reflection and exploration of alternative solutions.


\clearpage
\section{Experimental Setup} 
\label{app:exp-details}

\subsection{Data Processing}
\paragraph{Data Source.}
We construct our training dataset by combining two open-source synthetic datasets: OpenMathInstruct2~\cite{openmathinstruct2} and NuminaMath-COT~\cite{numina}. After a careful review of the synthetic data, we identify and remove invalid questions to improve data reliability.
To further enhance the quality of the dataset, we adopt the mutual consistency filtering method inspired by rStar~\cite{qi2024mutual}, which removes examples with inconsistent answers provided by different models. Specifically, we utilize QwQ~\cite{qwq-32b-preview} to relabel the questions and compared the newly generated answers with the original answers from the source datasets. Only examples with consistent answers were retained. Additionally, we apply de-duplication tools from~\cite{Stack} to eliminate redundant examples. Through these filtering processes, we finalized a high-quality dataset with approximately 550K samples in total.

\paragraph{Multi-agent COAT Data Synthesis.}
For the multi-agent demonstration trajectory synthesis framework, we utilize three models: Qwen-2.5-Math-7B-Instruct as the generator, Llama-3.1-70B-Instruct as the critic, and Skywork-Reward-Llama-3.1-8B-v0.2 as the outcome reward model. For the generator, we set the temperature to 0.3 and the maximum generation token limit to 2048. First, the generator samples $K=100$ initial solutions for each problem, dividing the generated solutions into three subsets: correct, incorrect, and invalid (those that fail to produce a final answer). Invalid solutions are discarded, and we randomly select four correct solutions and four incorrect solutions. Next, we set sampling temperature of critic to 0.2 and a maximum token limit of 256, and let critic provides feedback on these selected solutions, allowing the generator to perform conditional generation. This iterative process is repeated for a maximum of $m=2$ iterations, potentially resulting in up to $8 \times 2 = 16$ demonstration trajectories.

During the generation process, various situations require different prompt templates:
\begin{enumerate} \item The generator produces an initial solution.
\item The initial solution is correct, and the critic verifies its correctness.
\item The initial solution is incorrect, and the critic identifies mistakes.
\item The generator generates continuations after the critic verifies the correctness of its correct initial solution.
\item The generator generates continuations after the critic identifies mistakes in its incorrect initial solution.
\item The generator fails to solve the problem after refinement, and the critic provides an additional feedback to identify errors in the generator's second attempt.
\end{enumerate}
The prompt templates for these situations are detailed in Appendix~\ref{subsec:template}.

Among the synthetic trajectories, we categorize them into the following types:
\begin{enumerate} \item \textbf{Type-I:} Synthetic trajectories without critic feedback, i.e., no reflection actions.
\item \textbf{Type-II-I:} Synthetic trajectories that include an intermediate reflection action to verify the correctness of previous reasoning steps.
\item \textbf{Type-II-II:} Synthetic trajectories that include 1) an intermediate reflection action to verify the correctness of previous reasoning steps, and 2) a second reflection action to correct mistakes in the previous solution, followed by an explore action to propose an alternative solution.
\item \textbf{Type-III-I:} Synthetic trajectories that include a reflection action to correct mistakes in the previous solution and an explore action to propose an alternative solution.
\item \textbf{Type-III-II:} Synthetic trajectories that include two rounds of self-reflection and self-explore.
\end{enumerate}
Examples of these five types of synthetic trajectories are provided in Appendix~\ref{subsec:example-synthetic}. Finally, the outcome reward model is applied to select the top-1 (k=1) sample of each type from the 16 demonstration trajectories based on the reward score, if such a type exists.

\newpage
\subsubsection{Prompt Templates} \label{subsec:template}
\begin{tcolorbox}[brown_box, title = {{Prompt Template 1.1 --- Generator generates initial solution}}]\tiny
\begin{verbatim}
<|im_start|>user
Solve the following math problem efficiently and clearly.
Please reason step by step, and put your final answer within \boxed{}.
Problem: <<<instruction>>><|im_end|>
<|im_start|>assistant
\end{verbatim}
\end{tcolorbox}

\vspace{-1.5em}
\begin{tcolorbox}[brown_box, title = {{Prompt Template 1.2.1 --- Generator generates continuations for correct partial solutions}}]\tiny
\begin{verbatim}
<|im_start|>system
Your task is to continue writing a solution to a problem. Given a problem, a correct partial solution, along with a sanity check
of previous steps, you should continue the current solution to solve the problem. Think step by step using English and adhere 
to the following format.

Step: [Brief explanation and calculations]

Step: [Brief explanation and calculations]

...

**Please do not use other language than English.**

Your solution should conclude with "Therefore, the final answer is: \(\boxed{answer}\)", where [answer] is the final number
or expression that solves the problem. All steps in the solutions should be brief and always start with "Step:" and end with two
line breaks. Each subsequent step should not explicitly refer to previous steps, and the steps should constitute a coherent 
progression towards the final answer.<|im_end|>
<|im_start|>user
## Problem
$$\frac{16^m}{4^n}=2$$
Find $m+n$.

## Partial solution with a sanity check

Step: First, we express 16 and 4 as powers of 2:
\[ 16 = 2^4 \]
\[ 4 = 2^2 \]

Step: Substitute these expressions into the given equation:
\[ \frac{(2^4)^m}{(2^2)^n} = 2 \]

Step: Use the power of a power property \((a^m)^n = a^{mn}\) to simplify the exponents:
\[ \frac{2^{4m}}{2^{2n}} = 2 \]

Step: Use the property of exponents \(\frac{a^m}{a^n} = a^{m-n}\) to combine the exponents:
\[ 2^{4m-2n} = 2^1 \]

Step: Since the bases are the same, we can set the exponents equal to each other:
\[ 4m - 2n = 1 \]
I correctly simplified the exponents and equated \(4m - 2n\) to 1, which is essential to find the relationship between \(m\)
and \(n\).<|im_end|>
<|im_start|>assistant
## Continued solution

Step: To find \( m + n \), we need to solve this equation for \( m \) and \( n \). First, divide the entire equation by 2 to 
simplify: \[ 2m - n = \frac{1}{2} \]

Step: Add \( n \) to both sides to isolate \( 2m \):
\[ 2m = n + \frac{1}{2} \]

Step: Add \( m \) to both sides to find \( m + n \):
\[ m + n = \frac{1}{2} + \frac{1}{2} \]

Step: Simplify the right side:
\[ m + n = 1 \]

Therefore, the value of \( m + n \) is:
\[ \boxed{1} \]<|im_end|>
<|im_start|>user
## Problem
<<<instruction>>>

## Partial solution with a sanity check

<<<trajectory>>><|im_end|>
<|im_start|>assistant
## Continued solution

Step:
\end{verbatim}
\end{tcolorbox}
\vspace{-1.5em}
\begin{tcolorbox}[brown_box, title = {{Prompt Template 1.2.2 --- Generator generates continuations for incorrect partial solutions}}]\tiny
\begin{verbatim}
<|im_start|>system
Your task is to continue writing a solution to a problem. Given a problem, an incorrect partial solution where the last step is 
wrong, along with a revision feedback and a hint for next steps, you should continue the solution using the hint. Do not directly 
copy contents under "## Incorrect partial solution with revision feedback and hint for next steps". Think step by step using
English and adhere to the following format.

Step: [Brief explanation and calculations]

Step: [Brief explanation and calculations]

...

**Please do not use other language than English.**

Your solution should conclude with "Therefore, the final answer is: \(\boxed{answer}\)", where [answer] is the final number or 
expression that solves the problem. All steps in the solutions should be brief and always start with "Step:" and end with two 
line breaks. Each subsequent step should not explicitly refer to previous steps, and the steps should constitute a coherent
progression towards the final answer.<|im_end|>
<|im_start|>user
## Problem
Devanti wears her hair in various styles. A sleek bun is tied in top of the head, a half on the other with a long extension hanging
down to the shoulders. To prepare for his next photo shoot, Devanti decides to ward off a few days of the daily grays until it 
reaches there full potential. Research shows that one-fourth of the gray hairs are deposited on the first day, and the second day 
is when half the remaining hair has disappeared,  leaving Devanti with 15 gray hairs on her full head. Calculate the total number
of the gray hairs that Devanti was able to tadpoles from her hair.

## Incorrect partial solution with revision feedback and hint for next steps

Step: Let \( x \) be the total number of gray hairs Devanti had initially.

Step: On the first day, one-fourth of the gray hairs are deposited on the head. Therefore, the number of gray hairs remaining 
after the first day is:
\[
x - \frac{1}{4}x = \frac{3}{4}x
\]
Wait. I subtracted one-fourth of x instead of adding it, but gray hairs are deposited (added) on the first day. Alternatively, 
I should add one-fourth of x to find the total number of gray hairs after the first day.<|im_end|>
<|im_start|>assistant
## Continued solution

Step: Let me correct the first day's calculation by adding one-fourth of the gray hairs.

\[
\text{Initial gray hairs} = x
\]
\[
\text{Gray hairs deposited on the first day} = \frac{1}{4}x
\]
\[
\text{Total gray hairs after the first day} = x + \frac{1}{4}x = \frac{5}{4}x
\]

Step: Then, calculate the number of gray hairs remaining after the second day, where half of them disappear.

\[
\text{Gray hairs after the second day} = \frac{1}{2} \times \frac{5}{4}x = \frac{5}{8}x
\]

Step: Next, set up the equation based on the information that 15 gray hairs remain after the second day.

\[
\frac{5}{8}x = 15
\]

Step: Finally, let's solve for \( x \) to find the initial number of gray hairs.

\[
x = 15 \times \frac{8}{5} = 24
\]

Therefore, the final answer is: \(\boxed{24}\).<|im_end|>
<|im_start|>user
## Problem
<<<instruction>>>

## Incorrect partial solution with revision feedback and hint for next steps

<<<trajectory>>><|im_end|>
<|im_start|>assistant
## Continued solution

Step:
\end{verbatim}
\end{tcolorbox}
\vspace{-1.5em}
\begin{tcolorbox}[blue_box, title = {{Prompt Template 2.1 --- Critic verifies correctness}}]\tiny
\begin{verbatim}
## General Guidelines
You are a student. Your task is to carefully review your own correct partial solution to a math problem, and adhere to the 
following guidelines:

1. Verify the correctness of your own solution and explain your reason: "Verify: [brief explanation of why you are correct 
with one sentence]"

3. You are provided with the question, the ground truth solution, and your step-by-step partial solution.

4. Your response should not include phrases like "ground truth solution".

5. Your response should be exactly in the following format:
Verify: [brief explanation of why you are correct
with one sentence]

## Test Example
### Question
<<<question>>>

### Ground truth solution
<<<gt_solution>>>

### Your partial solution
<<<student_solution>>>

### Your review

\end{verbatim}
\end{tcolorbox}
\vspace{-1.5em}
\begin{tcolorbox}[blue_box, title = {{Prompt Template 2.2.1 --- Critic corrects mistakes}}]\tiny
\begin{verbatim}
## General Guidelines
You are a student. Your task is to carefully review your own solution to a math problem, and adhere to the following guidelines:

1. Directly point out the first potentially incorrect step you find and explain your reason: "In Step <id>: [brief explanation
of the mistake with one sentence]"

2. After this, suggest an alternative step that you should have taken to correct the currect incorrect step: "Alternatively:
[your suggested step with one sentence]"

3. You are provided with the question, the ground truth solution, and your step-by-step solution.

4. The alternative step you propose should not include phrases like "ground truth solution".

5. Your response should be exactly in the following format:
In Step <id>: [brief explanation of the mistake in this step, with one sentence]

Alternatively: [your suggested new step, with one sentence]

## Test Example
### Question
<<<question>>>

### Ground truth solution
<<<gt_solution>>>

### Your incorrect solution
<<<student_solution>>>

### Your review

\end{verbatim}
\end{tcolorbox}
\vspace{-1.5em}
\begin{tcolorbox}[blue_box, title = {{Prompt Template 2.2.2 --- Critic corrects mistakes for a second round}}]\tiny
\begin{verbatim}
You are a student in a math class. You are collaborating with a partner to solve math problems. 
First, your partner provided a partial solution to a math problem and identified a mistake. 
Then, you continue your partner's partial solution and finish solving the problem. 
Now, your task is to carefully review your continued solution, adhering to the following guidelines.

## Guidelines
1. "Your partner's partial solution" includes your partner's partial solution to the problem. It might be long and 
contain several substeps. "Your continued solution" includes your completion of the problem based on your partner's partial solution.

2. Your task is NOT to check the correctness of "Your partner's partial solution", but to directly point out the first potentially
incorrect step in "Your continued solution", and explain the reason why you made the mistake: "In Step <id>: [brief explanation of 
the mistake in your own solution with one sentence]".

3. After this, suggest an alternative step that you should have taken to correct your incorrect step: "Alternatively: 
[your suggested step with one sentence]"

4. You are provided with the question, the ground truth solution, and your partner's partial solution, and your continued solution.

5. The alternative step you propose SHOULD NOT explicitly include phrases like "ground truth solution", "as shown in the ground 
truth solution", etc.

6. You should only focus on your own solution. DO NOT refer to any mistake in your partner's partial solution.

7. Your response should be exactly in the following format:
In Step <id>: [brief explanation of the mistake in your own solution with one sentence]

Alternatively: [your suggested new step with one sentence]

## Test Example
### Question
<<<question>>>

### Ground truth solution
<<<gt_solution>>>

### Your partner's partial solution
<<<partner_solution>>>

### Your continued solution
<<<student_solution>>>

### Your review

\end{verbatim}
\end{tcolorbox}

\subsubsection{Synthetic Data Examples} \label{subsec:example-synthetic}
\begin{tcolorbox}[gray_box, title = {{Synthetic Data Example Type I: No reflection}}]\tiny
[Input]
\begin{verbatim}
Five couples plan to take their boats on a trip down a river. Each boat can carry 4 people. What is the smallest number of boats
that can accommodate all five couples?
\end{verbatim}

\hdashrule[0.5ex]{\linewidth}{0.1mm}{0.1mm} % Line width, dash length, gap length

[Output]
\begin{verbatim}
First, we determine the total number of people. Since there are five couples, and each couple consists of 2 people, the total number
of people is:

\[ 5 \times 2 = 10 \]<|continue|>
Next, we need to figure out how many boats are required to carry all 10 people. Each boat can carry 4 people. To find the minimum
number of boats, we divide the total number of people by the capacity of each boat:

\[ \frac{10}{4} = 2.5 \]<|continue|>
Since we cannot have half a boat, we need to round up to the next whole number. Therefore, we need 3 boats to accommodate all 
10 people.

To verify, let's consider how the people can be distributed among the boats:
- Boat 1: 4 people
- Boat 2: 4 people
- Boat 3: 2 people

All 10 people are accounted for, and no boat exceeds its capacity.

Thus, the smallest number of boats that can accommodate all five couples is:

\[ \boxed{3} \]
\end{verbatim}
\end{tcolorbox}
\begin{tcolorbox}[gray_box, title = {{Synthetic Data Example Type II-I: Verify correctness}}]\tiny
[Input]
\begin{verbatim}
A library has a total of $50$ books, each of which is categorized as a biography, novel, or self-help book. The library's patrons 
borrow $30$ books in one week, and the ratio of borrowed biographies to borrowed novels to borrowed self-help books is $2:3:5$. 
If the library has an equal number of biographies, novels, and self-help books, how many self-help books were borrowed in that week?
\end{verbatim}

\hdashrule[0.5ex]{\linewidth}{0.1mm}{0.1mm} % Line width, dash length, gap length

[Output]
\begin{verbatim}
First, we determine the number of biographies, novels, and self_help books in the library. Since the library has an equal number of 
each type of book and the total number of books is 50, we can divide 50 by 3:

\[ \text{Number of biographies} = \text{Number of novels} = \text{Number of self_help books} = \frac{50}{3} \]

However, since the number of books must be an integer and the problem states that the library has an equal number of each type of 
book, we need to re-evaluate the problem statement. The problem states that the library has an equal number of biographies, novels, 
and self_help books, but the ratio of borrowed 
biographies to borrowed novels to borrowed self_help books is \(2:3:5\). This means the total number of books borrowed must be a 
multiple of \(2+3+5=10\). Since 30 books were borrowed, it is 
indeed a multiple of 10.

<|reflect|>
Let me stop here to verify. The calculation of the number of each type of book in the library is correct, but the focus should be
on the ratio of borrowed books, which is consistent with the given total of 30 borrowed books.<|continue|>
Let \( b \), \( n \), and \( s \) represent the number of biographies, novels, and self_help books borrowed respectively. According 
to the given ratio \(2:3:5\), we can write:
\[ b = 2x \]
\[ n = 3x \]
\[ s = 5x \]<|continue|>
The total number of books borrowed is 30:
\[ b + n + s = 30 \]<|continue|>
Substitute the expressions for \( b \), \( n \), and \( s \) into the equation:
\[ 2x + 3x + 5x = 30 \]<|continue|>
Combine like terms:
\[ 10x = 30 \]<|continue|>
Solve for \( x \):
\[ x = \frac{30}{10} = 3 \]<|continue|>
Find the number of self_help books borrowed:
\[ s = 5x = 5 \times 3 = 15 \]

Therefore, the number of self_help books borrowed in that week is:
\[ \boxed{15} \]
\end{verbatim}

\end{tcolorbox}
\begin{tcolorbox}[gray_box, title = {{Synthetic Data Example Type II-II: Verify correctness then correct mistake}}]\tiny
[Input]
\begin{verbatim}
Consider the set of all positive integers whose digits, when read from left to right, form a valid English word. For example, 
325 is a member of this set because "three two five" is a valid English phrase. What is the largest member of this set that is 
also a perfect square?
\end{verbatim}

\hdashrule[0.5ex]{\linewidth}{0.1mm}{0.1mm} % Line width, dash length, gap length

[Output]
\begin{verbatim}
To find the largest member of the set of positive integers whose digits form a valid English word and is also a perfect 
square, we need to consider the constraints of the problem. The number must be a perfect square and its digits must form a valid 
English word.

<|reflect|>
I want to double-check my calculations. The step correctly identifies the two main constraints of the problem: the number 
must be a perfect square and its digits must form a valid English word.<|continue|>
We start by checking the largest perfect squares in descending order until we find one whose digits form a valid English word.
<|continue|>
 Let's check \( 9801 \):
\[ 9801 = 99^2 \]
The digits are 9, 8, 0, 1. "Nine eight zero one" is a valid English phrase.

Since \( 9801 \) meets both conditions (it is a perfect square and its digits form a valid English word), it is the largest 
such number.

Therefore, the largest member of the set that is also a perfect square is:
\[ \boxed{9801} \]<|reflect|>
But wait. The digits 9, 8, 0, 1 do not form a valid English word because "zero" is not a valid word in this context.
<|explore|>
I will now explore another method. We should check the next largest perfect square, which is 961, and verify if its digits 
form a valid English word.<|continue|>
Let's check \( 961 \):
\[ 961 = 31^2 \]
The digits are 9, 6, 1. "Nine six one" is a valid English phrase.

Since \( 961 \) meets both conditions (it is a perfect square and its digits form a valid English word), it is the largest 
such number.

Therefore, the largest member of the set that is also a perfect square is:
\[ \boxed{961} \]
\end{verbatim}

\end{tcolorbox}
\begin{tcolorbox}[gray_box, title = {{Synthetic Data Example Type III-I: Correct mistake}}]\tiny
[Input]
\begin{verbatim}
Given $A=\left\{x\in \mathbb{N}^* \mid -x^{2}+3x\geqslant 0\right\}$ and $B=\left\{x \mid \log_{\frac{1}{2}}x\leqslant 
0\right\}$, then $A\cap B=$

A: $\left[3,+\infty \right)$

B: $\left[0,1\right]$

C: $\left[1,3\right]$

D: $\left\{1,2,3\right\}$
\end{verbatim}

\hdashrule[0.5ex]{\linewidth}{0.1mm}{0.1mm} % Line width, dash length, gap length

[Output]
\begin{verbatim}
First, we solve the inequality for set \( A \):
\[ -x^2 + 3x \geq 0 \]<|continue|>
Factor out the common term:
\[ x(-x + 3) \geq 0 \]<|continue|>
Determine the critical points by setting each factor to zero:
\[ x = 0 \quad \text{or} \quad -x + 3 = 0 \implies x = 3 \]<|continue|>
The critical points divide the real number line into three intervals: \( (-\infty, 0) \), \( (0, 3) \), and \( (3, \infty) 
\). We test a point from each interval in the inequality \( x(-x + 3) \geq 0 \).

- For \( x \in (-\infty, 0) \), choose \( x = -1 \):
  \[ (-1)(-1 + 3) = (-1)(2) = -2 \quad (\text{not} \geq 0) \]

- For \( x \in (0, 3) \), choose \( x = 1 \):
  \[ (1)(1 + 3) = (1)(2) = 2 \quad (\geq 0) \]

- For \( x \in (3, \infty) \), choose \( x = 4 \):
  \[ (4)(4 - 3) = (4)(1) = 4 \quad (\geq 0) \]<|continue|>
 Since the inequality is non-strict (\(\geq\)), we include the critical points \( x = 0 \) and \( x = 3 \). Therefore, the 
 solution to the inequality is:
\[ A = [0, 3] \]<|continue|>
Next, we solve the inequality for set \( B \):
\[ \log_{\frac{1}{2}} x \leq 0 \]<|continue|>
Recall that the base of the logarithm is \(\frac{1}{2}\), which is less than 1. For a logarithm with a base less than 1, the 
inequality \(\log_{\frac{1}{2}} x \leq 0\) implies:
\[ x \geq 1 \]

<|reflect|>
I may have overlooked something. The solution to the inequality incorrectly includes the critical point \( x = 0 \) because \( x 
\in \mathbb{N}^* \) (the set of positive natural numbers), and \( 0 \) is not a positive natural number.<|explore|>
To find a better solution, The solution to the inequality should be \( A = \left\{1, 2, 3\right\} \) since \( x \) must be 
a natural number and \( 0 < x \leqslant 3 \).<|continue|>
Correct the solution for set \( A \) considering \( x \in \mathbb{N}^* \):

\[ A = \{1, 2, 3\} \]<|continue|>
Set \( B \) remains the same:

\[ B = [1, \infty) \]<|continue|>
Find the intersection \( A \cap B \):

\[ A \cap B = \{1, 2, 3\} \cap [1, \infty) = \{1, 2, 3\} \]

Therefore, the final answer is:
\[ \boxed{D} \]

\end{verbatim}

\end{tcolorbox}
\begin{tcolorbox}[gray_box, title = {{Synthetic Data Example Type III-II: Correct mistake multiple times}}]\tiny
[Input]
\begin{verbatim}
In $\triangle ABC$, $a$, $b$, $c$ are the sides opposite to the angles $A$, $B$, $C$ respectively. Given $c=4$, $\tan A=3$, 
$\cos C= \frac{ \sqrt{5}}{5} $, find the area of $\triangle ABC$.
\end{verbatim}

\hdashrule[0.5ex]{\linewidth}{0.1mm}{0.1mm} % Line width, dash length, gap length

[Output]
\begin{verbatim}
First, we find \(\sin C\) using the Pythagorean identity \(\sin^2 C + \cos^2 C = 1\):

\[
\sin^2 C = 1 - \cos^2 C = 1 - \left(\frac{\sqrt{5}}{5}\right)^2 = 1 - \frac{5}{25} = \frac{20}{25} = \frac{4}{5}
\]

Since \(C\) is an angle in a triangle, \(\sin C\) is positive:

\[
\sin C = \frac{2}{5}
\]<|continue|>
Next, we find \(\sin A\) using the given \(\tan A = 3\). Recall that \(\tan A = \frac{\sin A}{\cos A}\). Let \(\sin A = 3k\) and 
\(\cos A = k\). Using the Pythagorean identity \(\sin^2 A + \cos^2 A = 1\):

\[
(3k)^2 + k^2 = 1 \implies 9k^2 + k^2 = 1 \implies 10k^2 = 1 \implies k^2 = \frac{1}{10} \implies k = \frac{1}{\sqrt{10}} = 
\frac{\sqrt{10}}{10}
\]

Thus, \(\sin A = 3k = 3 \cdot \frac{\sqrt{10}}{10} = \frac{3\sqrt{10}}{10}\) and \(\cos A = k = \frac{\sqrt{10}}
{10}\).<|continue|>
We use the Law of Sines to find \(a\):

\[
\frac{a}{\sin A} = \frac{c}{\sin C} \implies \frac{a}{\frac{3\sqrt{10}}{10}} = \frac{4}{\frac{2}{5}} \implies \frac{a}
{\frac{3\sqrt{10}}{10}} = 4 \cdot \frac{5}{2} = 10 \implies a = 10 \cdot \frac{3\sqrt{10}}{10} = 3\sqrt{10}
\]

<|reflect|>
Wait a minute. The calculation of \(\sin C\) is incorrect because \(\sin C\) should be \(\frac{2}{\sqrt{5}}\) instead of \
(\frac{2}{5}\).<|explore|>
This time, Use the correct value of \(\sin C = \frac{2}{\sqrt{5}}\) in subsequent calculations.<|continue|>
Correct the calculation for \(\sin C\):

\[
\sin C = \frac{2}{\sqrt{5}}
\]<|continue|>
Use the correct value of \(\sin C\) in the Law of Sines to find \(a\):

\[
\frac{a}{\sin A} = \frac{c}{\sin C} \implies \frac{a}{\frac{3\sqrt{10}}{10}} = \frac{4}{\frac{2}{\sqrt{5}}} \implies 
\frac{a}{\frac{3\sqrt{10}}{10}} = 4 \cdot \frac{\sqrt{5}}{2} = 2\sqrt{5} \implies a = 2\sqrt{5} \cdot \frac{3\sqrt{10}}{10} = 
\frac{6\sqrt{50}}{10} = \frac{6 \cdot 5\sqrt{2}}{10} = 3\sqrt{2}
\]<|continue|>
 Use the formula for the area of a triangle \( \text{Area} = \frac{1}{2}ab \sin C \):

\[
\text{Area} = \frac{1}{2} \cdot 3\sqrt{2} \cdot 4 \cdot \frac{2}{\sqrt{5}} = \frac{1}{2} \cdot 12\sqrt{2} \cdot \frac{2}
{\sqrt{5}} = \frac{1}{2} \cdot \frac{24\sqrt{2}}{\sqrt{5}} = 12 \cdot \frac{\sqrt{2}}{\sqrt{5}} = 12 \cdot \sqrt{\frac{2}{5}} = 
12 \cdot \frac{\sqrt{10}}{5} = \frac{12\sqrt{10}}{5}
\]

Therefore, the area of \(\triangle ABC\) is: \(\boxed{\frac{12\sqrt{10}}{5}}\).<|reflect|>
Let's reassess my previous conclusion. The mistake is in the calculation of the area, where the value of \(\sin B\) should 
have been used instead of \(\sin C\).<|explore|>
To find a better solution, Use the correct value of \(\sin B = \frac{\sqrt{2}}{2}\) in the area formula: \(\text{Area} = 
\frac{1}{2} \cdot 3\sqrt{2} \cdot 4 \cdot \frac{\sqrt{2}}{2} = 6\).<|continue|>
correct the calculation of the area using the correct value of \(\sin B = \frac{\sqrt{2}}{2}$: \(\text{ Area} = \frac{1}{2} 
\cdot 3\sqrt{2} \cdot 4 \cdot \frac{\sqrt{2}}{2} = 6\).

Therefore, the area of \(\triangle ABC\) is: \(\boxed{6}\).
\end{verbatim}

\end{tcolorbox}

\clearpage
\subsection{Format Tuning}

\paragraph{Training Details.}
We perform supervised fine-tuning (SFT) on the Qwen-2.5-Math-7B base model using our synthetic COAT dataset with 10K unique questions. The SFT prompt template is illustrated below. We utilize a cosine learning rate scheduler with an initial learning rate of 2e-5. The batch size is set to 128, the maximum sequence length is 4096, and the model is trained for a maximum of two epochs. We add the following special tokens \texttt{<|continue|>},\texttt{<|reflect|>},\texttt{<|explore|>} into the vocabulary. All experiments are implemented using the LLaMA-Factory framework~\cite{llamafactory}.


\begin{tcolorbox}[green_box, title = {{Prompt Template 3: SFT and RL}}]
\label{box:prompt_template_3}
\begin{verbatim}
<|im_start|>user
Solve the following math problem efficiently and clearly.
Please reason step by step, and put your final answerwithin \boxed{}.
Problem: <<<your instruction>>>
<|im_start|>assistant
\end{verbatim}
\label{fig:sft_template}
\end{tcolorbox}

\subsection{Reinforcement Learning}
\label{subsec:rl_details}

\paragraph{ORM Training.}
To construct the preference data for our ORM models, we utilize our format-tuned model, Satori-Qwen-7B-FT, to generate trajectories. Starting with our filtered training dataset of 550K unique questions, we follow these steps: (1) allow the FT model to sample eight solutions for each question; (2) evaluate the correctness of these solutions and label them accordingly; and (3) select only those questions that contain correct and incorrect solutions. For these selected questions, we construct preference data by pairing correct solutions with their corresponding incorrect ones, resulting in a preference dataset of approximately 300K unique questions.

For each problem $\boldsymbol{x}$, we allow $\pi_{\theta}$ to randomly generate multiple reasoning trajectories, constructing a dataset $\Dc_r$ with positive and negative pairs of trajectories. We select trajectories with the correct final answer as positive trajectories $\boldsymbol{\Tilde{y}}^{+}$ and trajectories with incorrect final answer as negative trajectories $\boldsymbol{\Tilde{y}}^{-}$. Assuming the Bradley-Terry (BT) preference model, we optimize the reward model $r_\psi(x, \Tilde{y})$ through negative log-likelihood,
    \begin{align}
    \mathcal{L}_{rm}(\psi) \defeq -\mathbb{E}_{(\boldsymbol{x}, \boldsymbol{\Tilde{y}}^{+}, \boldsymbol{\Tilde{y}}^{-}) \sim \Dc_r} \left[ \log \left( \sigma \left( r_\psi(\boldsymbol{x}, \boldsymbol{\Tilde{y}}^{+}) - r_\psi(x, \boldsymbol{\Tilde{y}}^{-}) -\tau \right) \right) \nonumber \right]
    \end{align}
where $\tau$ denotes a target reward margin. In practice, we observe that setting $\tau>0$ improves the performance of the reward model.

\paragraph{RL Data.}
Our RL training dataset consists of 300K unique questions from the preference dataset. This ensures that the questions are neither too easy (where the FT model always produces correct solutions) nor too difficult (where the FT model never succeeds). This encourages policy to learn through trial and error during RL training. To further guide the model to learn self-reflection capabilities, we apply the proposed RAE technique, augmenting input problems with restart buffers, i.e., intermediate reasoning steps collected from the FT model. These intermediate steps are extracted from the preference dataset, and for each question, we randomly select one correct and one incorrect trajectory, applying the back-track technique for up to $T=2$ steps.

\paragraph{Training Details.}
For both ORM and RL training, we implement our experiments using the OpenRLHF framework~\cite{hu2024openrlhf}. For ORM training, we employ a cosine learning rate scheduler with an initial learning rate of 2e-6. The batch size is set to 128, the maximum sequence length to 4096, and the model is trained for two epochs. As the objective function, we use PairWiseLoss~\cite{christiano2017deep} with a margin of $\tau=2$. For evaluation, we select the optimal ORM model checkpoint based on RM@8 performance, measured using the SFT model on a held-out validation dataset. Specifically, we allow the FT model to sample eight trajectories and let ORM select the best trajectory according to the highest reward score. The RM@8 accuracy is then computed based on the selected trajectories.

For RL training, we use the PPO algorithm~\cite{ppo}. The critic model is initialized from our ORM model, while the actor model is initialized from our FT model. We optimize the models using a cosine learning rate scheduler, setting the learning rate to 2e-7 for the actor model and 5e-6 for the critic model. During PPO training, we sample one trajectory per prompt. The training batch size is set to 128, while the rollout batch size is 1024. Both the number of epochs and episodes are set to 1. The maximum sequence length for prompts and generations is fixed at 2048. Additional parameter settings include a KL coefficient of 0.0, a sampling temperature of 0.6, and a bonus scale of $r_{\text{bonus}}=0.5$.

\paragraph{Second-round Self-improvement.}
We begin with a set of 240K unique questions, also used in the distillation experiments shown in Table~\ref{fig:distill}. The policy of the first round of RL training serves as a teacher model to generate synthetic reasoning trajectories. Among these 240K questions and corresponding trajectories, we filter the data based on question difficulty, selecting the most challenging 180K samples for distillation. This process results in a new fine-tuned (FT) model checkpoint, obtained from supervised fine-tuning (SFT) on these 180K trajectories. Since the new FT model has been trained on numerous high-quality trajectories, including reflection actions distilled from the teacher model, we do not apply restart and exploration (RAE) techniques in the second round of RL training to further encourage reflection. Additionally, we increase the sampling temperature from 0.6 to 1.2, generating eight samples per prompt to encourage more aggressive exploration to push the performance limit.

\subsection{Evaluation Details.}
\label{app:eval}
For each model, we use the same zero-shot CoT prompt template to obtain results on all test datasets. For Satori and all its variants, we use Prompt Template 3 (Appendix \ref{box:prompt_template_3}). We set the temperature to 0 (greedy decoding) for every model, and collect pass@1 accuracies. Details of each test dataset are as follows.

\textbf{MATH500} \cite{prm800k} is a subset of MATH \cite{MATH} of uniformly sampled 500 test problems. The distribution of difficulty levels and subjects in MATH500 was shown to be representative of the entire MATH test set.

\textbf{GSM8K} \cite{GSM8K} is a math dataset that consists of 8.5K high-quality, linguistically diverse grade-school math word problems designed for multi-step reasoning (2 to 8 steps). Solutions involve elementary arithmetic operations and require no concepts beyond early algebra. Its test set contains 1319 unique problems.

\textbf{OlympiadBench} \cite{olympiadbench} is a bilingual, multimodal scientific benchmark with 8,476 Olympiad-level math and physics problems, including those from the Chinese college entrance exam. We use the open-ended, text-only math competition subset, containing 674 problems in total. 

\textbf{AMC2023} and \textbf{AIME2024} contain 40 text-only problems from American Mathematics Competitions 2023 and 30 text-only problems from American Invitational Mathematics Examination 2024, respectively.

\textbf{FOLIO} \cite{folio} is a human-annotated dataset designed to evaluate complex logical reasoning in natural language, featuring 1,430 unique conclusions paired with 487 sets of premises, all validated with first-order logic (FOL) annotations. Its test set contains 203 unique problems.

\textbf{BoardgameQA (BGQA)} \cite{bgqa} is a logical reasoning dataset designed to evaluate language models' ability to reason with contradictory information using defeasible reasoning, where conflicts are resolved based on source preferences (e.g., credibility or recency). Its test set contains 15K unique problems.

\textbf{CRUXEval} \cite{cruxeval} is a benchmark for evaluating code reasoning, understanding, and execution, featuring 800 Python functions (3-13 lines) with input-output pairs for input and output prediction tasks. Given a function snippet and an input example, LLMs are tasked to generate the corresponding outputs. Its test set contains 800 unique problems.

\textbf{StrategyQA} \cite{strategyqa} is a question-answering benchmark designed for multi-hop reasoning where the necessary reasoning steps are implicit and must be inferred using a strategy. Each of the 2,780 examples includes a strategy question, its step-by-step decomposition, and supporting Wikipedia evidence.

\textbf{TableBench} \cite{tablebench} is a tabular reasoning benchmark designed to evaluate LLMs on real-world tabular data challenges, covering 18 fields across four major TableQA categories: Fact checking, numerical reasoning, data analysis, and code generation for visualization. We test all models on fact checking and numerical reasoning subsets for simplicity of answer validation, resulting in 491 unique problems.

\textbf{MMLUProSTEM} is a subset of MMLU-Pro \cite{mmlupro}. MMLU-Pro is an enhanced benchmark designed to extend MMLU \cite{mmlu} by incorporating more reasoning-focused questions, expanding answer choices from four to ten, and removing trivial or noisy items. We select six STEM subsets: physics, chemistry, computer science, engineering, biology, and economics (we remove the math subset as it belongs to in-domain tasks). Finally, we obtain 5371 unique problems in total.


\section{Additional Results} \label{app:results}

\subsection{Ablation on Reflection Bonus}
\begin{table}[h]
  \begin{center}
  \footnotesize
  \captionsetup{font=small}
  \caption{\textbf{Ablation Study on Reflection Bonus.}}
  \setlength{\tabcolsep}{1.3pt}
  \begin{tabular}{cccccccccc}
    \toprule
    \textbf{Bonus Scale} &\textbf{GSM8K} & \textbf{MATH500}  &  \textbf{Olym.} & \textbf{AMC2023} & \textbf{AIME2024} \\
    \midrule
    \textbf{0.0} & 93.6 & 84.4  & 48.9 & 62.5 & 16.7 \\
    \textbf{0.5 (default)} &93.2 & 85.6 &46.6 & 67.5 &20.0   \\
    \bottomrule
  \end{tabular}
  \label{table:reflect-bonus}
  \end{center}
\end{table}
During RL training, we introduce a reflection bonus to facilitate the policy to learn self-reflection capabilities. The default value of the reflection bonus is set to $r_{\text{reflect}}=0.5$. To analyze its impact on performance, we also evaluate the model with the reflection bonus set to $r_{\text{reflect}}=0$. The results are presented in Table~\ref{table:reflect-bonus}. We observe that the performance slightly degrades on challenging benchmark AMC2023 and AIME2024 when set $r_{\text{reflect}}=0$ compared to $r_{\text{reflect}}=0.5$.


\subsection{Offline Restart Buffer v.s. Online Restart Buffer}
Complementary to the reflection bonus, the restart buffer is designed to enhance the policy's self-reflection capabilities by augmenting the initial states with a diverse set of intermediate states. This includes trajectories processed from both correct and incorrect reasoning paths, which are then categorized into positive ($\Dc^+_{\text{restart}}$) and negative ($\Dc^-_{\text{restart}}$) restart buffers, as described in Section~\ref{subsec:rl}.

In addition to constructing the restart buffer offline, we also explore an online restart buffer approach. Specifically, after each PPO episode, we use the updated policy to construct the restart buffer and collect rollouts from this buffer to optimize the policy, iteratively repeating this process. However, this approach is suboptimal. During PPO training, we observe that the majority of sampled trajectories are correct, leading to a significant imbalance between correct and incorrect intermediate states in the online restart buffer. As a result, the model fail to adequately learn from incorrect paths, which are essential for incentivize self-reflection actions.

To overcome this limitation, we opt for an offline restart buffer approach to mitigate the bias introduced by online collection. Offline sampling ensures a balanced inclusion of intermediate states from both correct and incorrect trajectories.