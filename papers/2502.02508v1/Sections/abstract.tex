\begin{abstract}
Large language models (LLMs) have demonstrated remarkable reasoning capabilities across diverse domains. Recent studies have shown that increasing test-time computation enhances LLMs' reasoning capabilities. This typically involves extensive sampling at inference time guided by an external LLM verifier, resulting in a two-player system. Despite external guidance, the effectiveness of this system demonstrates the potential of a single LLM to tackle complex tasks. Thus, we pose a new research problem: \textit{Can we internalize the searching capabilities to fundamentally enhance the reasoning abilities of a single LLM?} This work explores an orthogonal direction focusing on post-training LLMs for autoregressive searching (\textit{i.e.,} an extended reasoning process with self-reflection and self-exploration of new strategies). To achieve this, we propose the Chain-of-Action-Thought (COAT) reasoning and a two-stage training paradigm: 1) a small-scale format tuning stage to internalize the COAT reasoning format and 2) a large-scale self-improvement stage leveraging reinforcement learning. Our approach results in Satori, a 7B LLM trained on open-source models and data. Extensive empirical evaluations demonstrate that Satori achieves state-of-the-art performance on mathematical reasoning benchmarks while exhibits strong generalization to out-of-domain tasks. Code, data, and models will be fully open-sourced\footnote{ \url{https://satori-reasoning.github.io/}}.

\end{abstract}


