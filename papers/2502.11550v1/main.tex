\documentclass[journal]{IEEEtran}



\usepackage{booktabs}
\usepackage{xcolor}
\usepackage{color}
\usepackage{bbding}
\usepackage{multirow}
\usepackage{amsthm}
\usepackage{amsmath}
\usepackage{pifont}
\interdisplaylinepenalty=2500
\usepackage{amssymb}
\usepackage{yhmath}
\bibliographystyle{IEEEtran}
\usepackage{graphicx}
\usepackage[linesnumbered,ruled,vlined]{algorithm2e}
\SetKwRepeat{Do}{do}{while}
%\usepackage{algorithmic}
\usepackage{algorithmicx}
\SetAlFnt{\footnotesize}%设置算法中的字体小于原来文本字体
\newcommand{\paratitle}[1]{\vspace{0.5ex}\noindent \underline{\textbf{#1}}}
\usepackage{bm}
\usepackage{setspace}
\usepackage{tabularx}
\usepackage{threeparttable}
%\usepackage{fixltx2e}
\usepackage[caption=false,font=footnotesize]{subfig}%不使用caption会使得标题全部靠左 subfig自动调用caption包 IEEE建议不使用且靠左
\graphicspath{{figs/}}
\usepackage{cite} 
\usepackage{tikz}
\usepackage[utf8]{inputenc}
\newtheorem{definition}{Definition}
\newtheorem{theorem}{Theorem}
\usepackage{caption}
\usepackage{mathrsfs}
\usepackage{textcomp}
\usepackage{graphicx}
\usepackage{float} 
%\usepackage{subfigure}
%\usepackage{subcaption}
\usepackage{subfig}
\usepackage{booktabs,makecell,multirow}
\usepackage{titlesec}
\usepackage{xspace}
% 自定义 section 编号为阿拉伯数字
\renewcommand\thesection{\arabic{section}}

% 自定义 subsection 编号为大写字母后跟点
\renewcommand{\thesubsection}{\Alph{subsection}.}

% 自定义 subsubsection 编号为 1)
\renewcommand{\thesubsubsection}{\arabic{subsubsection})}

% 自定义 section 格式为正体
\titleformat{\section}
  {\normalfont\Large\bfseries}
  {\thesection}{1em}{}

% 自定义 subsection 格式为正体
\titleformat{\subsection}
  {\normalfont\normalsize\bfseries}
  {\thesubsection}{1em}{}

% 自定义 subsubsection 格式为正体,并且编号形式为 1)
\titleformat{\subsubsection}
  {\normalfont\normalsize\bfseries}
  {\thesubsubsection}{1em}{\normalfont}
%\renewcommand\thesubsectiondis{\thesection.\arabic{subsection}}
% Some very useful LaTeX packages include:
% (uncomment the ones you want to load)
\usepackage{hyperref}
\newcommand{\yes}{\textcolor{ForestGreen}{\faCheck}}
\newcommand{\no}{\textcolor{Maroon}{\faTimes}}
\newcommand{\add}[1]{{\color{blue}#1}}
\newcommand{\MH}[1]{\textbf{\textcolor{red}{MH:}~\textcolor{blue}{#1}}}
\newcommand{\TrinityI}{Trinity-\uppercase\expandafter{\romannumeral1}\xspace}
\newcommand{\TrinityII}{Trinity-\uppercase\expandafter{\romannumeral2}\xspace}







% *** GRAPHICS RELATED PACKAGES ***
%
\ifCLASSINFOpdf
  % \usepackage[pdftex]{graphicx}
  % declare the path(s) where your graphic files are
  % \graphicspath{{../pdf/}{../jpeg/}}
  % and their extensions so you won't have to specify these with
  % every instance of \includegraphics
  % \DeclareGraphicsExtensions{.pdf,.jpeg,.png}
\else
  % or other class option (dvipsone, dvipdf, if not using dvips). graphicx
  % will default to the driver specified in the system graphics.cfg if no
  % driver is specified.
  % \usepackage[dvips]{graphicx}
  % declare the path(s) where your graphic files are
  % \graphicspath{{../eps/}}
  % and their extensions so you won't have to specify these with
  % every instance of \includegraphics
  % \DeclareGraphicsExtensions{.eps}
\fi
% graphicx was written by David Carlisle and Sebastian Rahtz. It is
% required if you want graphics, photos, etc. graphicx.sty is already
% installed on most LaTeX systems. The latest version and documentation
% can be obtained at: 
% http://www.ctan.org/pkg/graphicx
% Another good source of documentation is "Using Imported Graphics in
% LaTeX2e" by Keith Reckdahl which can be found at:
% http://www.ctan.org/pkg/epslatex
%
% latex, and pdflatex in dvi mode, support graphics in encapsulated
% postscript (.eps) format. pdflatex in pdf mode supports graphics
% in .pdf, .jpeg, .png and .mps (metapost) formats. Users should ensure
% that all non-photo figures use a vector format (.eps, .pdf, .mps) and
% not a bitmapped formats (.jpeg, .png). The IEEE frowns on bitmapped formats
% which can result in "jaggedy"/blurry rendering of lines and letters as
% well as large increases in file sizes.
%
% You can find documentation about the pdfTeX application at:
% http://www.tug.org/applications/pdftex





% *** MATH PACKAGES ***
%
%\usepackage{amsmath}
% A popular package from the American Mathematical Society that provides
% many useful and powerful commands for dealing with mathematics.
%
% Note that the amsmath package sets \interdisplaylinepenalty to 10000
% thus preventing page breaks from occurring within multiline equations. Use:
%\interdisplaylinepenalty=2500
% after loading amsmath to restore such page breaks as IEEEtran.cls normally
% does. amsmath.sty is already installed on most LaTeX systems. The latest
% version and documentation can be obtained at:
% http://www.ctan.org/pkg/amsmath





% *** SPECIALIZED LIST PACKAGES ***
%
%\usepackage{algorithmic}
% algorithmic.sty was written by Peter Williams and Rogerio Brito.
% This package provides an algorithmic environment fo describing algorithms.
% You can use the algorithmic environment in-text or within a figure
% environment to provide for a floating algorithm. Do NOT use the algorithm
% floating environment provided by algorithm.sty (by the same authors) or
% algorithm2e.sty (by Christophe Fiorio) as the IEEE does not use dedicated
% algorithm float types and packages that provide these will not provide
% correct IEEE style captions. The latest version and documentation of
% algorithmic.sty can be obtained at:
% http://www.ctan.org/pkg/algorithms
% Also of interest may be the (relatively newer and more customizable)
% algorithmicx.sty package by Szasz Janos:
% http://www.ctan.org/pkg/algorithmicx




% *** ALIGNMENT PACKAGES ***
%
%\usepackage{array}
% Frank Mittelbach's and David Carlisle's array.sty patches and improves
% the standard LaTeX2e array and tabular environments to provide better
% appearance and additional user controls. As the default LaTeX2e table
% generation code is lacking to the point of almost being broken with
% respect to the quality of the end results, all users are strongly
% advised to use an enhanced (at the very least that provided by array.sty)
% set of table tools. array.sty is already installed on most systems. The
% latest version and documentation can be obtained at:
% http://www.ctan.org/pkg/array


% IEEEtran contains the IEEEeqnarray family of commands that can be used to
% generate multiline equations as well as matrices, tables, etc., of high
% quality.




% *** SUBFIGURE PACKAGES ***
%\ifCLASSOPTIONcompsoc
%  \usepackage[caption=false,font=normalsize,labelfont=sf,textfont=sf]{subfig}
%\else
%  \usepackage[caption=false,font=footnotesize]{subfig}
%\fi
% subfig.sty, written by Steven Douglas Cochran, is the modern replacement
% for subfigure.sty, the latter of which is no longer maintained and is
% incompatible with some LaTeX packages including fixltx2e. However,
% subfig.sty requires and automatically loads Axel Sommerfeldt's caption.sty
% which will override IEEEtran.cls' handling of captions and this will result
% in non-IEEE style figure/table captions. To prevent this problem, be sure
% and invoke subfig.sty's "caption=false" package option (available since
% subfig.sty version 1.3, 2005/06/28) as this is will preserve IEEEtran.cls
% handling of captions.
% Note that the Computer Society format requires a larger sans serif font
% than the serif footnote size font used in traditional IEEE formatting
% and thus the need to invoke different subfig.sty package options depending
% on whether compsoc mode has been enabled.
%
% The latest version and documentation of subfig.sty can be obtained at:
% http://www.ctan.org/pkg/subfig




% *** FLOAT PACKAGES ***
%
%\usepackage{fixltx2e}
% fixltx2e, the successor to the earlier fix2col.sty, was written by
% Frank Mittelbach and David Carlisle. This package corrects a few problems
% in the LaTeX2e kernel, the most notable of which is that in current
% LaTeX2e releases, the ordering of single and double column floats is not
% guaranteed to be preserved. Thus, an unpatched LaTeX2e can allow a
% single column figure to be placed prior to an earlier double column
% figure.
% Be aware that LaTeX2e kernels dated 2015 and later have fixltx2e.sty's
% corrections already built into the system in which case a warning will
% be issued if an attempt is made to load fixltx2e.sty as it is no longer
% needed.
% The latest version and documentation can be found at:
% http://www.ctan.org/pkg/fixltx2e


%\usepackage{stfloats}
% stfloats.sty was written by Sigitas Tolusis. This package gives LaTeX2e
% the ability to do double column floats at the bottom of the page as well
% as the top. (e.g., "\begin{figure*}[!b]" is not normally possible in
% LaTeX2e). It also provides a command:
%\fnbelowfloat
% to enable the placement of footnotes below bottom floats (the standard
% LaTeX2e kernel puts them above bottom floats). This is an invasive package
% which rewrites many portions of the LaTeX2e float routines. It may not work
% with other packages that modify the LaTeX2e float routines. The latest
% version and documentation can be obtained at:
% http://www.ctan.org/pkg/stfloats
% Do not use the stfloats baselinefloat ability as the IEEE does not allow
% \baselineskip to stretch. Authors submitting work to the IEEE should note
% that the IEEE rarely uses double column equations and that authors should try
% to avoid such use. Do not be tempted to use the cuted.sty or midfloat.sty
% packages (also by Sigitas Tolusis) as the IEEE does not format its papers in
% such ways.
% Do not attempt to use stfloats with fixltx2e as they are incompatible.
% Instead, use Morten Hogholm'a dblfloatfix which combines the features
% of both fixltx2e and stfloats:
%
% \usepackage{dblfloatfix}https://www.overleaf.com/project/625e7352c2e48337b2350bfc
% The latest version can be found at:
% http://www.ctan.org/pkg/dblfloatfix




%\ifCLASSOPTIONcaptionsoff
%  \usepackage[nomarkers]{endfloat}
% \let\MYoriglatexcaption\caption
% \renewcommand{\caption}[2][\relax]{\MYoriglatexcaption[#2]{#2}}
%\fi
% endfloat.sty was written by James Darrell McCauley, Jeff Goldberg and 
% Axel Sommerfeldt. This package may be useful when used in conjunction with 
% IEEEtran.cls'  captionsoff option. Some IEEE journals/societies require that
% submissions have lists of figures/tables at the end of the paper and that
% figures/tables without any captions are placed on a page by themselves at
% the end of the document. If needed, the draftcls IEEEtran class option or
% \CLASSINPUTbaselinestretch interface can be used to increase the line
% spacing as well. Be sure and use the nomarkers option of endfloat to
% prevent endfloat from "marking" where the figures would have been placed
% in the text. The two hack lines of code above are a slight modification of
% that suggested by in the endfloat docs (section 8.4.1) to ensure that
% the full captions always appear in the list of figures/tables - even if
% the user used the short optional argument of \caption[]{}.
% IEEE papers do not typically make use of \caption[]'s optional argument,
% so this should not be an issue. A similar trick can be used to disable
% captions of packages such as subfig.sty that lack options to turn off
% the subcaptions:
% For subfig.sty:
% \let\MYorigsubfloat\subfloat
% \renewcommand{\subfloat}[2][\relax]{\MYorigsubfloat[]{#2}}
% However, the above trick will not work if both optional arguments of
% the \subfloat command are used. Furthermore, there needs to be a
% description of each subfigure *somewhere* and endfloat does not add
% subfigure captions to its list of figures. Thus, the best approach is to
% avoid the use of subfigure captions (many IEEE journals avoid them anyway)
% and instead reference/explain all the subfigures within the main caption.
% The latest version of endfloat.sty and its documentation can obtained at:
% http://www.ctan.org/pkg/endfloat
%
% The IEEEtran \ifCLASSOPTIONcaptionsoff conditional can also be used
% later in the document, say, to conditionally put the References on a 
% page by themselves.




% *** PDF, URL AND HYPERLINK PACKAGES ***
%
%\usepackage{url}
% url.sty was written by Donald Arseneau. It provides better support for
% handling and breaking URLs. url.sty is already installed on most LaTeX
% systems. The latest version and documentation can be obtained at:
% http://www.ctan.org/pkg/url
% Basically, \url{my_url_here}.




% *** Do not adjust lengths that control margins, column widths, etc. ***
% *** Do not use packages that alter fonts (such as pslatex).         ***
% There should be no need to do such things with IEEEtran.cls V1.6 and later.
% (Unless specifically asked to do so by the journal or conference you plan
% to submit to, of course. )


% correct bad hyphenation here
\hyphenation{op-tical net-works semi-conduc-tor}



\begin{document}
%
% paper title
% Titles are generally capitalized except for words such as a, an, and, as,
% at, but, by, for, in, nor, of, on, or, the, to and up, which are usually
% not capitalized unless they are the first or last word of the title.
% Linebreaks \\ can be used within to get better formatting as desired.
% Do not put math or special symbols in the title.
\title{Trinity: A Scalable and Forward-Secure DSSE for Spatio-Temporal Range Query}
%
%
% author names and IEEE memberships
% note positions of commas and nonbreaking spaces ( ~ ) LaTeX will not break
% a structure at a ~ so this keeps an author's name from being broken across
% two lines.
% use \thanks{} to gain access to the first footnote area
% a separate \thanks must be used for each paragraph as LaTeX2e's \thanks
% was not built to handle multiple paragraphs
%
\iffalse 
\author{Zhijun Li}
\authornote{Both authors contributed equally to this research.}
\email{richikun2014@gmail.com}
\orcid{0000-0001-7736-6818}
\author{Jianfeng Ma}
\authornotemark[1]
\email{jfma@mail.xidian.edu.cn}
\affiliation{
	\institution{School of Cyber Engineering, Xidian University}
	\streetaddress{Taibai NO.2}
	\city{Xi'an}
	\state{Shaanxi}
	\postcode{710071}
}

\author{Xindi Ma}
\affiliation{
	\institution{School of Cyber Engineering, Xidian University}
	\streetaddress{Taibai NO.2}
	\city{Xi'an}
	\state{Shaanxi}
	\email{ma@mail.xidian.edu.cn}
\fi
\author{Zhijun~Li, %~\IEEEmembership{Member,~IEEE,}
        Kuizhi~Liu, %~\IEEEmembership{Member,~IEEE,}
        Minghui~Xu,~\IEEEmembership{Member,~IEEE,}
        Xiangyu~Wang,~\IEEEmembership{Member,~IEEE,}
        Yinbin~Miao,~\IEEEmembership{Member,~IEEE,}
        Jianfeng~Ma,~\IEEEmembership{Member,~IEEE,}
        and
        Xiuzhen~Cheng,~\IEEEmembership{Fellow,~IEEE,}
     
    %~\IEEEmembership{Life~Fellow,~IEEE}% <-this % stops a space
\thanks{Zhijun Li, Minghui Xu,  and Xiuzhen Cheng are with the School of Computer Science and Technology, Shandong University, China, Qingdao 266237, China. (e-mail: richikun2014@gmail.com, mhxu@sdu.edu.cn,  xzcheng@sdu.edu.cn).}
%\thanks{ is with the School of Cyber Engineering, Xidian University, Xi'an 710071, China; e-mail: jfma@mail.xidian.edu.cn} <-this % stops a space
\thanks{Kuizhi Liu, Xiangyu Wang, Yinbing Miao, and Jianfeng Ma are with the School of Cyber Engineering, Xidian University, Xi'an 710071, China. (e-mail: ighxiy@163.com, wangxiangyu01@xidian.edu.cn, ybmiao@xidian.edu.cn, jfma@mail.xidian.edu.cn).}
\thanks{Corresponding author: Minghui Xu}

%\thanks{D. Wu is with the Chongqing University of Posts and Telecommunications, Chongqing 400065, China, and also with the University of Massachusetts at Dartmouth, Dartmouth, MA 02747, USA (e-mail: wudapengphd@gmail.com).}
}

% note the % following the last \IEEEmembership and also \thanks - \thanks{Manuscript received August 25, 2020; revised August 26, 2020.}
% these prevent an unwanted space from occurring between the last author name
% and the end of the author line. i.e., if you had this:
% 
% \author{....lastname \thanks{...} \thanks{...} }
%                     ^------------^------------^----Do not want these spaces!
%
% a space would be appended to the last name and could cause every name on that
% line to be shifted left slightly. This is one of those "LaTeX things". For
% instance, "\textbf{A} \textbf{B}" will typeset as "A B" not "AB". To get
% "AB" then you have to do: "\textbf{A}\textbf{B}"
% \thanks is no different in this regard, so shield the last } of each \thanks
% that ends a line with a % and do not let a space in before the next \thanks.
% Spaces after \IEEEmembership other than the last one are OK (and needed) as
% you are supposed to have spaces between the names. For what it is worth,
% this is a minor point as most people would not even notice if the said evil
% space somehow managed to creep in.



% The paper headers
\markboth{IEEE TRANSACTIONS ON INFORMATION FORENSICS AND SECURITY}%
{Li \MakeLowercase{\textit{$et\ al.$}}: Bare Demo of IEEEtran.cls for IEEE Journals}
% The only time the second header will appear is for the odd numbered pages
% after the title page when using the twoside option.
% 
% *** Note that you probably will NOT want to include the author's ***
% *** name in the headers of peer review papers.                   ***
% You can use \ifCLASSOPTIONpeerreview for conditional compilation here if
% you desire.




% If you want to put a publisher's ID mark on the page you can do it like
% this:
%\IEEEpubid{0000--0000/00\$00.00~\copyright~2015 IEEE}
% Remember, if you use this you must call \IEEEpubidadjcol in the second
% column for its text to clear the IEEEpubid mark.



% use for special paper notices
%\IEEEspecialpapernotice{(Invited Paper)}




% make the title area
\maketitle

% As a general rule, do not put math, special symbols or citations
% in the abstract or keywords.The rapid growth of e-Health has spread globally. Thus, in spite of the great developing potential of this technology, there is widespread concern that patient privacy issues may stunted its growth. 
\begin{abstract}
Cloud-based outsourced Location-based services have profound impacts on various aspects of people's lives but bring security concerns. Existing spatio-temporal data secure retrieval schemes have significant shortcomings regarding dynamic updates, either compromising privacy through leakage during updates (forward insecurity) or incurring excessively high update costs that hinder practical application. 
Under these circumstances, we first propose a basic filter-based spatio-temporal range query scheme \TrinityI that supports low-cost dynamic updates and automatic expansion. Furthermore, to improve security, reduce storage cost, and false positives, we propose a forward secure and verifiable scheme \TrinityII that simultaneously minimizes storage overhead. A formal security analysis proves that \TrinityI and \TrinityII are Indistinguishable under Selective Chosen-Plaintext Attack (IND-SCPA). Finally, extensive experiments demonstrate that our design \TrinityII significantly reduces storage requirements by 80\%, enables data retrieval at the 1 million-record level in just 0.01 seconds, and achieves 10 $\times$ update efficiency than state-of-art.
 
\end{abstract}

% Note that keywords are not normally used for peerreview papers.
\begin{IEEEkeywords}
Location-based services, spatio-temporal data, dynamic update, forward-secure.
\end{IEEEkeywords}






% For peer review papers, you can put extra information on the cover
% page as needed:
% \ifCLASSOPTIONpeerreview
% \begin{center} \bfseries EDICS Category: 3-BBND \end{center}
% \fi
%
% For peerreview papers, this IEEEtran command inserts a page break and
% creates the second title. It will be ignored for other modes.
\IEEEpeerreviewmaketitle



\section{Introduction}
% The very first letter is a 2 line initial drop letter followed
% by the rest of the first word in caps.
% 
% form to use if the first word consists of a single letter:
% \IEEEPARstart{A}{demo} file is ....
% 
% form to use if you need the single drop letter followed by
% normal text (unknown if ever used by the IEEE):
% \IEEEPARstart{A}{}demo file is ....
% 
% Some journals put the first two words in caps:
% \IEEEPARstart{T}{his demo} file is ....
% 
% Here we have the typical use of a "T" for an initial drop letter
% and "HIS" in caps to complete the first word.
\IEEEPARstart {T}{he} popularity of smart devices has accelerated Location-Based Services (LBS), which retrieve and sort information based on user location and requests, then provide recommendations to users. 
% For example, navigation or food delivery apps offer services to billions of people daily, and collect spatio-temporal data. 
% This paper aims to explore how to enhance data retrieval processes within spatio-temporal databases while addressing existing challenges related to querying efficiency and accuracy. 
In traditional spatial LBS, users can ask if a stationary object sits within a specific area. However, how do users track the movement patterns of dynamic objects such as vehicles? That is where spatio-temporal LBS comes in, stepping up from mere spatial snapshots to track and query objects across both space and time. 

For spatio-temporal LBS service providers, outsourcing spatio-temporal data to the cloud is a practical method way of improving the service quality and reducing local operating costs. However, outsourcing data to third parties inevitably brings security and privacy issues. For example, in 2018, fitness tracking app Strava published a global heat map that visualized the activity of its users, inadvertently revealing the locations and activity time of military personnel at sensitive facilities around the world. Therefore, Dynamic Searchable Symmetric Encryption (DSSE) \cite{stefanov2014practical} has become a treatment to this concern, as it allows clients with specific keys to query and update ciphertext without revealing sensitive data. However, almost all DSSE schemes allow for certain information leakage in exchange for efficiency \cite{cao2019protecting,huang2021netr,zhu2021privacy,wang2022quickn,guo2022search,yang2022lightweight,miao2023efficient}. Therefore, attackers can potentially disclose the contents of past queries by inserting new documents, as the server is capable of recognizing matches between these newly inserted documents and previous search queries \cite{zhang2016all}. To avoid such leakage, several studies \cite{kermanshahi2020geometric,li2021secure,wang2022forward,li2023enabling} have focused on introducing forward security to prevent such leaks that occur during DSSE updates.% Kermanshahi Kermanshahi $et\ al.$ \cite{kermanshahi2020geometric} introduced the first spatial data DSSE scheme with forward security, utilizing semi-homomorphic encryption in their approach. Li $et\ al.$ \cite{li2021secure} proposed a forward-secure spatial data retrieval scheme by expiring prior tokens with each update. Wang $et\ al.$ \cite{wang2022forward} implemented a homomorphic encrypted bitmap-based scheme for spatial data protection.
%As for forward security, Kermanshahi $et\ al.$ \cite{kermanshahi2020geometric} introduced the first spatial data DSSE scheme supporting forward security. But this scheme requires semi homomorphic encryption of all ciphertexts every time it is updated. Li $et\ al.$ \cite{li2021secure} proposed a spatial data retrieval scheme based on R tree, which supports efficient addition operation. However, this solution still has deletion problems and scalability challenges(when adding operations that exceed the database size). Wang $et\ al.$ \cite{wang2022forward} later proposed a scheme based on homomorphic encrypted bitmap that achieved faster update operations, although it still faced scalability challenges.
%While filter-based retrieval methods offer advantages in privacy protection and data indexing due to their rapid retrieval capabilities, they face inherent challenges including false positives, dynamic update difficulties, and fixed capacity constraints.过滤器缺点


When applying DSSE to spatio-temporal scenarios, scalability becomes an critical issue. Due to the spatio-temporal nature of the data, the size of the database grows continuously, making scalability a critical concern in system design. 
% Since the data set expands with incoming data streams, there are several situations that require data deletion operations. These situations include the closure of shops and the updates to delivery zones. Therefore, an efficient mechanism supporting data deletion, while maintaining database scalability, is essential for practical spatio-temporal data management.
The scalability of DSSE schemes largely depends on the index structures, including tree-based index structures \cite{kermanshahi2020geometric,etemad2018efficient} and linear index structures \cite{wang2017fastgeo,dou2024dynamic}. Tree-based spatio-temporal range query schemes inherently face scalability challenges \cite{li2019efficient,zheng2020practical}. When data volume grows, we have to frequently expand tree structures by adjusting tree height and reorganizing nodes, which poses scalability limitations \cite{wang2014maple}. 
% Because spatial data scenarios typically deal with static datasets that undergo infrequent updates, focusing primarily on analyzing existing data collections. Therefore, the limitation on scalability is acceptable in spatial data scenarios but not spatio-temporal scenarios \cite{cao2015scalable}. For instance, Uber continuously receives vehicle trajectory tracking points as time goes by. 
% This dynamic nature necessitates an index structure capable of flexible scalability for processing spatio-temporal data \cite{cao2019protecting}. 

Compared to tree-based schemes, filter-based ones are more scalable based on an array-based index structure \cite{miao2023efficient}. Furthermore, the filters demonstrate superior space efficiency compared to the R-trees \cite{cui2019geo}. These advantages have led to the increasing adoption of filters in spatio-temporal range query schemes. Filter-based DSSE for spatio-temporal range query work by transforming high-dimensional spatio-temporal data into one-dimensional representations stored in filters. By leveraging a filter, which maps elements to multiple locations using hash functions, queries can be executed through simple bit operations with low computational complexity \cite{cui2019geo}. This approach effectively utilizes the filters' fast lookup capability to efficiently determine element existence and process spatio-temporal range queries. Many DSSE schemes are developed based on bloom filter  \cite{cui2019geo,wang2021enabling,wang2022forward,zhang2022efficient,miao2023efficient}, there still remain challenges that need to be addressed further.  %Cui $et\ al.$ \cite{cui2019geo} pioneered the first secure Boolean geometric range search scheme by bloom filter. Wang $et\ al.$ \cite{wang2021enabling} employed bloom filter hierarchical tree, significantly improving retrieval speed.Recent works by Zhang $et\ al.$ \cite{zhang2022efficient} and Miao $et\ al.$ \cite{miao2023efficient} employ similar non-scalable structure that require substantial storage overhead to maintain low false positive rates. However, filter based solutions also have their own limitations, which present several challenges that need to be addressed.

\subsection{Challenges} 
%For example, Miao $et\ al.$ \cite{miao2023efficient} use a bloom filter to carry out the spatial range query, but also poses challenges with deletion. Yang $et\ al.$ \cite{yang2022lightweight} constructs a lightweight spatial query scheme with geohash, this technique also leads to deletion disability.
%As the leaf nodes are full and have reached their maximum capacity, a split is required. After completion of the insertion and splitting of the R tree, the changes must be propagated upward to adjust the entries of the relevant parent nodes until reaching the root node. If the root node also splits, the height of the tree will increase, resulting in a deeper tree structure that accommodates the new hierarchical configuration.

\textbf{Challenge~I: How to construct a filter that enables flexible and efficient element deletion.} Existing range query schemes \cite{cui2019geo,wang2021enabling,zhu2021privacy,zhang2022efficient} based on the bloom filter do not support deletion, because the bloom filter uses a single-digit set to represent the existence of an element. Each bit of the bloom filter is not exclusive, and multiple elements may share the same bit \cite{almeida2007scalable}.
% When an element is added, multiple hash functions map it to different positions in the bit group and set the bits at these positions to 1. However, once a bit is set to 1, it cannot be determined which element caused it \cite{almeida2007scalable}. In other words, each bit of the Bloom filter is not exclusive, and multiple elements may share the same bit. 
Therefore, if a bit corresponding to an element is deleted, it may negatively affect other elements that share the same bit \cite{pandey2021vector}. For example, if the two elements “seafood" and “Japanese cuisine" happen to overlap in some bit positions. Setting the bit position corresponding to the “seafood" element to 0 for deletion can cause the “Japanese cuisine" element to be accidentally deleted. Although there are some none-filter DSSE schemes \cite{kermanshahi2020geometric,wang2022forward} that support deletion, our experiment results show that their efficiency remains a concern and they do not apply to filter-based schemes.
% As a result, bloom filter-based spatio-temporal range query schemes show a limitation when dealing with data deletion scenarios.
 %, such as removing incorrect location records, discontinued service areas, or historical data for storage optimization.At present, most privacy preserving spatio-temporal data retrieval schemes \cite{cao2019protecting,wang2021enabling,huang2021netr,yang2022lightweight,miao2023efficient} cannot achieve forward security \cite{bost2016ovarphiovarsigma}, which means they are vulnerable to file injection attack \cite{zhang2016all}. To solve this problem, Kermanshahi $et\ al.$ \cite{kermanshahi2020geometric} propose a DSSE scheme that achieves forward security.

\textbf{Challenge~II: How can the scalability of forward-secure filters be improved.} Existing forward-secure range query filters \cite{wang2021enabling,zhu2021privacy,wang2022forward} face scalability problems. As the amount of spatio-temporal data increases over time, the probability of hash collisions increases, which affects the query efficiency and false positive rate \cite{zhang2022efficient}. Especially when the number of data entries in the database reaches its maximum capacity, performing the add operation will cause the database to crash or rebuild. 
% Specifically, a bloom filter consists of a fixed size $n$-bit array and $k$ hash functions. When an element is added, the hash function calculates multiple positions and sets the bits of these positions to 1. 
% The calculation process for this position is to modulo the array size $n$ with $k$ hash values. This design makes the bloom filter very efficient in inserting elements, but it also means that 
Moreover, the bit group size of a filter is static and cannot be modified once set \cite{mullin1983second}. Expanding the size would require remapping existing data, which is impractical \cite{almeida2007scalable}. While initially overprovisioning filter capacity might seem viable, it results in substantial wasted space. Additionally, assuming consistent data input is unrealistic. Periods of inactivity would further amplify space inefficiency.
%As the leaf nodes are full and have reached their maximum capacity, a split is required. After completion of the insertion and splitting of the R tree, the changes must be propagated upward to adjust the entries of the relevant parent nodes until reaching the root node. If the root node also splits, the height of the tree will increase, resulting in a deeper tree structure that accommodates the new hierarchical configuration.However, filter-based schemes have better scalability. 

\textbf{Challenge~IIIs: How to overcome the trade-off between saving storage and minimizing false positives.} Filter-based schemes \cite{li2022adaptively,li2023vrfms,tong2023verifiable} need free space to keep a low false positive rate (FPR). For example, the FFR of widely sued bloom filters is given by $\epsilon = \left ( 1-e^{-k\cdot \frac{n}{m} } \right ) ^{k} $, where $k$ is the number of hash functions, $m$ is the length of the filter, and $n$ is the number of inserted elements \cite{fan1998summary}. 
%
To achieve a practical FPR of 0.01$\%$, a balance between accuracy and efficiency is sought \cite{almeida2007scalable}. This configuration inherently sets the ratio $\frac{m}{n} =\frac{k} {\ln 2}$, where $k=-\frac{\ln p }{\ln 2}$. This fixed ratio imposes a fundamental design constraint, requiring $\frac{m}{n}\geq 20$ to maintain the desired FPR. In other words, at least 19 times the number of inserted elements in free space is necessary. Overcoming this trade-off is crucial for improving filter-based DSSE.


% \par \textbf{Forward security.} Leaving BRQ out in the cold, forward-secure DSSE schemes need an upgrade to embrace the full spectrum of spatio-temporal querying.
\iffalse \begin{table*}[htbp]
\centering
\caption{Comparison With Prior Works }
\label{comparisions}
\setlength{\tabcolsep}{3.5mm}{
\begin{tabular}{|c|c|c|c|c|c|}
\hline
& GRS-\uppercase\expandafter{\romannumeral 2}\cite{kermanshahi2020geometric} & $\mathsf {DSSE}_{\mathsf {SKQ}}$\cite{wang2022forward} & SKSE-\uppercase\expandafter{\romannumeral 2}\cite{wang2021enabling} & Trinity-\uppercase\expandafter{\romannumeral 1} & Trinity-\uppercase\expandafter{\romannumeral 2}\\
\hline
Cryptographic Primitive & ASHE & ASHE & HVE & SHVE & SHVE\\
\hline
Dynamic Update & \checkmark & \checkmark & \ding{53} &  \checkmark & \checkmark\\
\hline
Forward Security & \checkmark & \checkmark & \ding{53} & \ding{53} & \checkmark\\
\hline
High Efficiency & \ding{53} & \checkmark & \checkmark & \checkmark & \checkmark\\
\hline
Spatio-Temporal & \ding{53} & \ding{53} & \ding{53} & \checkmark & \checkmark\\
\hline
\end{tabular}}
\begin{tablenotes}
			\footnotesize
			\item[ ] \textbf{Notes.} ASHE stands for Additive Symmetric Homomorphic Encryption. ASPE stands for Asymmetric Scalar Product-Preservation Encryption. HVE stands for Hidden Vector Encryption. SHVE stands for Symmetric-key Hidden Vector Encryption.
		\end{tablenotes}
\end{table*}

\begin{table*}[htbp]
\centering
\caption{Comparison With Prior Works}
\label{comparisions}
\setlength{\tabcolsep}{3.5mm}{
\begin{tabular}{|l|c|c|c|c|c|c|}
\hline
Schemes & \makecell{Cryptographic\\Primitive} & \makecell{Dynamic\\Update} & \makecell{Forward\\Security} & \makecell{High\\Efficiency} & \makecell{Spatio-\\Temporal} & \makecell{Security \\Model}\\
\hline
GRS-\uppercase\expandafter{\romannumeral 2}\cite{kermanshahi2020geometric} & ASHE & \checkmark & \checkmark & \ding{53} & \ding{53}& \ IND-CPA \\
\hline
$\mathsf{DSSE}_{\mathsf{SKQ}}$\cite{wang2022forward} & ASHE & \checkmark & \checkmark & \checkmark & \ding{53} & \ IND-CPA \\
\hline
SKSE-\uppercase\expandafter{\romannumeral 2}\cite{wang2021enabling} & HVE & \ding{53} & \ding{53} & \checkmark & \ding{53} & \ IND-SCPA \\
\hline
Trinity-\uppercase\expandafter{\romannumeral 1} & SHVE & \checkmark & \ding{53} & \checkmark & \checkmark & \ IND-SCPA\\
\hline
Trinity-\uppercase\expandafter{\romannumeral 2} & SHVE & \checkmark & \checkmark & \checkmark & \checkmark & \ IND-SCPA\\
\hline
\end{tabular}}
\begin{tablenotes}
\footnotesize
\item \textbf{Notes.} ASHE stands for Additive Symmetric Homomorphic Encryption. ASPE stands for Asymmetric Scalar Product-Preservation Encryption. HVE stands for Hidden Vector Encryption. SHVE stands for Symmetric-key Hidden Vector Encryption.
\end{tablenotes}
\end{table*}
\fi
\begin{table*}[htbp]
\centering
\caption{Comparison With Prior Works}
\label{comparisions}
\setlength{\tabcolsep}{3mm}
\begin{threeparttable}
\begin{tabular}{lccccccc}
\toprule
Schemes & \makecell{Cryptographic\\Primitive} & \makecell{Dynamic\\Update} & \makecell{Forward\\Security} & \makecell{High\\Efficiency} & \makecell{Spatio-\\Temporal} & \makecell{Scalability} & \makecell{Security \\Model}\\
\midrule

$\mathsf{DSSE}_{\mathsf{SKQ}}$\cite{wang2022forward} & ASHE & \checkmark & \checkmark & \checkmark & \ding{53}& \ding{53} & IND-CPA \\
SKSE-\uppercase\expandafter{\romannumeral 2}\cite{wang2021enabling} & HVE & \ding{53} & \ding{53} & \checkmark & \ding{53}& \ding{53} & IND-SCPA \\
GRS-\uppercase\expandafter{\romannumeral 2}\cite{kermanshahi2020geometric} & ASHE & \checkmark & \checkmark & \ding{53} & \ding{53}& \ding{53} & IND-CPA \\
Trinity-\uppercase\expandafter{\romannumeral 1} & SHVE & \checkmark & \ding{53} & \checkmark & \checkmark & \ding{53}& IND-SCPA \\
Trinity-\uppercase\expandafter{\romannumeral 2} & SHVE & \checkmark & \checkmark & \checkmark & \checkmark & \checkmark &IND-SCPA \\
\bottomrule
\end{tabular}
\begin{tablenotes}
\footnotesize
\item \textbf{Notes.} ASHE stands for Additive Symmetric Homomorphic Encryption. ASPE stands for Asymmetric Scalar Product-Preservation Encryption. HVE stands for Hidden Vector Encryption. SHVE stands for Symmetric-key Hidden Vector Encryption.
\end{tablenotes}
\end{threeparttable}
\end{table*}

% Efficiency:
% 1. dynamic - deletion and addition.
% 2. effiency -  quotient filter
% 3. scalable - quotient filter + technique1
% 4. accuracy - verfication + technique1
% Security: 
% 5. adaptive security and forward security - 5.1 SHVE for adaptive security + 5.2 salt for forward security


% We propose a scalable solution that decreases the data density within the filter, effectively reducing the occurrence of false positives and optimizing query speed. Additionally, we introduce a roar bitmap-based verification feature that completely resolves the false positive issue while significantly saving on storage costs. Our Trinity framework enables rapid query performance on ciphertexts, capable of handling hundreds of millions of queries per second. The key contributions encapsulated in our research are as follows:
% that reduces update costs using a quotient filter
%In this work, we introduce a novel DSSE system that facilitates scalability, and efficient querying of encrypted spatio-temporal data. We attain secure dynamic range queries for spatio-temporal data through the integration of quotient filter technology, Hilbert curve mapping, and Symmetric-Key Hidden Vector Encryption (SHVE). Furthermore, we integrate Constrained Pseudo-Random Function (CPRF) into our encryption methodology to ensure a forward-secure encryption process. Specifically, we reduce the false positive rate by automatically expanding filters and implementing verification. 
\textbf{Our Contributions}. We introduce two novel DSSE schemes tailored for spatio-temporal range queries.
\begin{itemize}
    \item \textbf{\TrinityI}. A fundamental DSSE construction that supports dynamic and efficient element updates, including addition and deletion. \TrinityI offers a performance boost over state-of-the-art methods, achieving up to a 20$\times$ speedup. And \TrinityI adaptively expands its index structure, so we can keep a negligible false positive rate and low latency on search and update for the filter. Besides, our fundamental DSSE construction is secure against IND-SCPA. 
    \item \textbf{\TrinityII}. Building upon \TrinityI, this scheme provides additional high accuracy, and forward security. \TrinityII offers a verification function that strikes a balance between storage efficiency and low false positives. It reduces storage requirements by 80\% while enabling data retrieval at the 1 million-record level in just 0.01 seconds, and offers enhanced forward security against file injection attacks. \TrinityII still offers a performance boost over state-of-the-art methods, achieving up to a 10$\times$ speedup on update latency.
\end{itemize}
    % \item Our design significantly reduces storage requirements by 80\% while enabling data retrieval at the million-record level in just 0.01 seconds. Additionally, it provides enhanced security against IND-SCPA and file injection attacks. Performance evaluations demonstrate that our scheme excels in both update efficiency and storage cost, showcasing its clear advantages over existing solutions.
%In empirical tests involving millions of data points, Trinity shows an impressive retrieval time of just 0.01 seconds, and 80\% space saving.
%\par To ensure a more impartial and lucid presentation, we list Trinity with previous works in Table~\ref{comparisions}. To the best of our knowledge, Trinity is the first forward-secure spatio-temporal range query scheme that accommodates dynamic updates.


\hfill 
\section{Related Work}
In this section, we introduce some related work. To ensure a more impartial and lucid presentation, we list Trinity with previous works in Table~\ref{comparisions}.
\par  \textbf{DSSE.} DSSE is a cryptographic technique that enables clients to securely outsource encrypted databases to servers while still maintaining the ability to perform search and update operations on encrypted data \cite{stefanov2014practical,cash2015leakage,kim2017forward, 10621113, etemad2018efficient,zuo2020forward,wang2022forward}. While DSSE provides efficient access to encrypted databases, it inevitably incurs some information leakage, particularly during update operations, where sensitive data can potentially be exposed, thus compromising privacy. The concept of forward security, introduced by Stefanov $et\ al.$ \cite{stefanov2014practical}, aims to address this issue by ensuring that updates to the encrypted database do not reveal information about previous queries or data. Zhang $et\ al.$ \cite{zhang2016all} further highlighted the vulnerability of DSSE to leakage attack, specifically file injection tactics, where an attacker strategically inserts a limited number of documents into the encrypted database to deduce search queries. Following this finding, Bost \cite{bost2016ovarphiovarsigma} provided a formal definition of forward security and proposed a concrete scheme that incorporates forward security in DSSE. Since then, several forward-secure DSSE schemes have been proposed \cite{etemad2018efficient,song2018forward,li2021towards,guo2023forward}, with a focus on improving security, efficiency, and functional diversity.

%These schemes aim to minimize information leakage during updates while maintaining the ability to efficiently search and update encrypted data. They utilize various techniques to achieve forward security, such as re-encryption mechanisms \cite{bost2017forward,song2018forward}, oblivious RAM \cite{wu2022forward,wu2023obi}, multi-set hash functions \cite{zhang2019towards}, and puncturable encryption\cite{green2015forward,bost2017forward,sun2018practical}. The ongoing research in this area is essential for developing secure and practical DSSE solutions that can be deployed in real-world applications, such as cloud storage and data sharing platforms.

 %Striking a balance between efficiency and security has been a central challenge in cryptography. Within LBS systems, there is a need to guarantee not only the absence of significant security loopholes but also the usability of the proposed solutions. However, the aforementioned approaches fall short of meeting these criteria in at least one respect.

\par \textbf{Filter-based DSSE.} 
The common drawback of filter-based search DSSE schemes \cite{wang2021enabling, zhu2021privacy,li2022adaptively, zhang2022efficient, miao2023efficient, li2023vrfms, tong2023verifiable}, particularly bloom filter-based schemes, is that they cannot support deletion. These filters are designed to efficiently test membership in a set while allowing for false positive results, meaning they can indicate that an element is present in the set when it is not. This characteristic is inherent in their probabilistic nature, which prioritizes space efficiency over absolute accuracy. Wang $et\ al.$ \cite{wang2015circular} introduced a pioneering privacy-preserving circular range search scheme; however, the search token size scales with the square of the search radius R, posing scalability challenges. To address these issues, Wang $et\ al.$ \cite{wang2021enabling} ingeniously leveraged the Hilbert curve, effectively reducing queries of the two-dimensional spatial range to one-dimensional searches, thus significantly enhancing retrieval speed. Although this approach represents a notable improvement, it still lacks support for dynamic updates and is limited to a fixed data scale. Similarly, the non-scalable structure employed in \cite{zhang2022efficient, miao2023efficient} requires significant space overhead to maintain a low False Positive Rate (FPR), further highlighting the limitations of these methods.
 
%\par Therefore, the privacy preserving search mechanisms \cite{li2021forward,wang2021enabling,yang2022lightweight,kermanshahi2022geometric,wang2022forward,miao2023efficient,li2023enabling} for outsourced spatio-temporal data are essential.  In this situation, Cui $et\ al.$ \cite{cui2019geo} contributed a privacy-preserving Boolean Range Query (BRQ) solution, yet it was found to be vulnerable to Ciphertext-Only Attacks (COA) \cite{li2019insecurity} due to the use of Asymmetric Scalar Product-Preservation Encryption (ASPE) \cite{wong2009secure}.To address this security issue, Yang $et\ al.$ \cite{yang2022lightweight} proposed an enhanced ASPE scheme that achieves Indistinguishability under Chosen Plaintext Attack (IND-CPA). However, their approach compromised on search efficiency. Wang $et\ al.$ \cite{wang2021enabling} then improved search speed by incorporating a Bloom filter hierarchical tree, but their solution was limited to single-user scenarios due to its symmetric key setting and focused solely on spatial queries, neglecting the temporal dimension. Cao $et\ al.$ \cite{cao2019protecting} offered a differential privacy-based scheme to protect spatio-temporal data, but this approach may have impacted the accuracy of the returned results. Huang $et\ al.$ \cite{huang2021netr} presented a mixed cryptographic primitives-based scheme for privacy-preserving spatio-temporal Location-Based Services (LBS), but its computational costs were high due to the use of RSA and Attribute-Based Encryption. Li $et\ al.$ \cite{li2023enabling} developed an efficient privacy-preserving spatio-temporal LBS scheme capable of retrieving million-level data in milliseconds. However, the structure of their R tree database made it difficult to modify the size of the encrypted database. Unfortunately, none of the aforementioned privacy-preserving spatio-temporal LBS schemes fully support dynamic updates, which is crucial in real-world LBS applications where data, such as active Instagram users' location information, is constantly being updated.

\textbf{None-filter DSSE.} Balancing efficiency and security has consistently presented a challenge in privacy-preserving LBS \cite{kerschbaum2014optimal,grubbs2017leakage,wang2022quickn,guo2022search}. Various effective privacy-preserving approaches have been proposed, including the use of Order-Revealing Encryption (ORE) \cite{kerschbaum2014optimal}, and a variant of Order-Preserving Encryption (OPE) \cite{wang2022quickn}. However, recent research \cite{grubbs2017leakage} has highlighted severe security vulnerabilities in property-preserving encryption schemes such as OPE and ORE. Cui $et\ al.$ \cite{cui2019geo} contributed a privacy-preserving Boolean Range Query (BRQ) solution, which was later found to be vulnerable to Ciphertext-Only Attacks (COA) \cite{li2019insecurity} due to the use of Asymmetric Scalar Product-Preservation Encryption (ASPE) \cite{wong2009secure}.
To address this security issue, Yang $et\ al.$ \cite{yang2022lightweight} proposed an enhanced ASPE scheme that achieves Indistinguishability under Chosen Plaintext Attack (IND-CPA). However, their approach compromised on search efficiency. Kermanshahi \cite{kermanshahi2020geometric} proposed a forward-secure DSSE for spatial range query, but the efficiency of their solution is unacceptable. Wang $et\ al.$ \cite{wang2021enabling} then improved search speed by incorporating a Bloom filter hierarchical tree, but their solution was limited to single-user scenarios due to its symmetric key setting and focused solely on spatial queries, neglecting the temporal dimension.  Li $et\ al.$ \cite{li2023enabling} developed an efficient privacy-preserving spatio-temporal LBS scheme capable of retrieving million-level data in milliseconds.
%Cao $et\ al.$ \cite{cao2019protecting} offered a differential privacy-based scheme to protect spatio-temporal data, though this approach impacted the accuracy of the returned results. Huang $et\ al.$ \cite{huang2021netr} presented a mixed cryptographic primitives-based scheme for privacy-preserving spatio-temporal Location-Based Services (LBS), but its computational costs were high due to the use of RSA and Attribute-Based Encryption. Although polynomial fitting techniques in conjunction with Attribute-based Searchable Encryption (ASPE) \cite{xu2018enabling} have facilitated efficient single-round trip LBS, ASPE-based solutions have proven insufficiently robust against real-world security threats. Notably, privacy-preserving LBS often necessitate the compromise of disclosing certain information to achieve efficiency trade-offs. Additionally, file injection attacks can exploit Keyword-Pair Result Pattern (KPRP) leaks \cite{cash2013highly} to uncover all keywords within conjunctive searches.The DSSE scheme based on Oblivious RAM (ORAM) \cite{gui2019encrypted} offers resilience against file injection attacks. Nevertheless, ORAM introduces multiple round-trip communications and computational overhead, diminishing its practicality in real-world applications. Homomorphic encryption \cite{kermanshahi2020geometric} represents an ideal cryptographic primitive capable of securely facilitating computations followed by comparison operations. However, the practicality of HE still requires enhancement, currently suitable only for small-scale datasets.

%Filter-based retrieval methods have gained popularity due to their quick retrieval capabilities, as evidenced in works such as . However, these methods inherently grapple with drawbacks, including the potential for false positives, difficulties in handling dynamic updates, and constraints imposed by a fixed data capacity.

\section{Preliminaries}
Here, we introduce some crypto primitives and data structures widely used in Trinity including Hilbert curve, bloom filter, and SHVE.
\subsection{Hilbert Curve}
A Hilbert curve is a type of continuous factal space-filling curve that fills a $d$-dimensional area \cite{hilbert1935stetige}. For a $d$-dimensional space, each dimension is uniformly divided into $2^h$ segments, where $h$ denotes the order of the Hilbert curve. The Hilbert curve is constructed by recursively dividing an area into $2^d$ smaller areas and connecting the centers of these smaller areas in a specific order. With the Hilbert curve, any $d$-dimensional spatial range query can be converted into a range query in one-dimensional space of $2^{dh}$ consecutive areas, each represented by a $dh$-bit value. 

% We provide an example of the Hilbert curve in Fig.~\ref{hilbert}, a two-dimensional square can be transformed into a one-dimensional Hilbert curve ($d=2, h=3$) with a query range of $2^{6}=64$. The green region corresponds to the query range $R_1=[2,3]$, and the blue region corresponds to the query range $R_2=\{[34,35],[46,47]\}$. 

% The Hilbert curve is constructed by recursively dividing a square into four smaller squares and connecting the centers of these smaller squares in a specific order. 

% \begin{figure}[htbp]
% 	\centering
% 	\includegraphics[scale=0.35]{hilbert.pdf}
% 	\caption{An example of Hilbert curve ($d=2, h=3$)}
% 	\label{hilbert}
% \end{figure}
%\subsection{Prefix Membership Verification}

\subsection{Quotient Filter}
\par A quotient filter is a variant of the Bloom filter. It can quickly detect whether an element is in a set, indicating that the data definitely do not exist or may possibly exist. The quotient filter employs a space-efficient hash table mechanism, dividing the $p$-bit fingerprint of an element into two distinct segments. The first segment, known as the quotient, consists of the $q$ most significant bits and is used to rapidly determine the “target position" of the element within the filter. The second segment, referred to as the remainder, encapsulates the $r = p - q$ least significant bits and serves to differentiate elements with the same quotient. The original slot where the hash output point's quotient resides is called the \texttt{canonical slot}. A sequence of consecutive slots containing remainders with the same quotient is termed a \texttt{run}. A \texttt{cluster} in a quotient filter is a contiguous sequence of \texttt{run}s, commencing with the first \texttt{run} whose initial fingerprint occupies its \texttt{canonical slot}, and extending until an unoccupied slot is encountered or the another \texttt{run} occupies its \texttt{canonical slot} is identified. Each slot contains a 3-bit counter.%Multiple consecutive \texttt{run}s are referred to as a \texttt{cluster}, with the first \texttt{run} starting at the position corresponding to the quotient, while subsequent \texttt{runs} do not necessarily start at positions corresponding to the quotient.
\begin{itemize}
	\item \textbf {Is\_ occupied}. Set to 1 if a slot is occupied correctly, that is, a quotient value corresponds to the slot index.
	\item \textbf {Is\_ continuation}. Set to 1 when a slot is not the start of the \texttt{run}. 
    % It happens when the quotient of the adjacent slot on the left is the same as it.
	\item \textbf {Is\_ shifted}. Set to 1 when this slot is not the start of the \texttt{cluster}. It happens when there is an offset between the position where the remainder is stored and the index represented by the quotient associated with the remainder.
\end{itemize}
%\par Where \texttt{run} refers to the remainder of a continuous record with the same quotient; The \texttt{cluster} refers to multiple consecutive \texttt{run}, and the first \texttt{run} starts from the position corresponding to the quotient, while the subsequent \texttt{run} do not start from the position corresponding to the quotient.     The quotient table on the right of the figure indicates the quotient $f_q$ and remainder $f_r$ for each fingerprint $f$.

\begin{figure}[!ht]
	\centering
	\includegraphics[scale=0.3]{quotientfilter.pdf}
	\caption{An example of quotient filter}
	\label{quotient filter}
\end{figure}

We provide an example of the quotient filter in Fig.~\ref{quotient filter}. The upper portion depicts a simplified view, where eight fingerprints are categorized into buckets based on their quotient (blue numbers). For example, the first fingerprint 000$\big|$00010, with the quotient of 000, is placed in slot \textcircled{0}. The second fingerprint 000$\big|$00011 ideally belongs to slot \textcircled{0}, but slot \textcircled{0} is occupied, so we shift it right to slot \textcircled{1} (the actual slot of the second fingerprint).
%
The lower portion shows the actual storage mechanism. Each bucket holds a 3-bit counter and a remainder. In this example, 000$\big|$00010, and 000$\big|$00011 share the same quotient, forming a \texttt{run}. The eight fingerprints finally constitute three \texttt{run}s and one \texttt{cluster}.



% [010$\big|$00001, 010$\big|$00010, 010$\big|$00011], [101$\big|$00001, 101$\big|$00011] are \texttt{run}s, too. Value 010$\big|$00001 is stored in its canonical position, marking the inception of a \texttt{cluster}. Despite the fact that 011$\big|$00011 should have been stored in bucket 011, it is propelled forward by 010$\big|$00010 and 010$\big|$00011 to bucket 101. Values {[010$\big|$00001, 010$\big|$00010, 010$\big|$00011], [101$\big|$00001, 101$\big|$00011]} collectively constitute a \texttt{cluster}. To make it clear, 011$\big|$00011 in slot \textcircled{5} is not the run itself, but provides information pointing to the actual run (located in the \textcircled{6} and \textcircled{7} slots) and the entry point. As a marker or index, it indicates the position of the shifted run, but it does not constitute a part of the run itself.



\subsection{Symmetric-key Hidden Vector Encryption}

Symmetric-key Hidden Vector Encryption (SHVE) is a lightweight crypto primitive based on Hidden Vector Encryption (HVE). Like HVE, it supports conjunctive, equality , and subset queries on encrypted data. Let $\Gamma$ be a finite field and ‘$\ast$’ be a wildcard symbol not belong to $\Gamma$, $\Gamma_{\ast}=\Gamma \bigcup \{\ast \}$. The SHVE is defined in the subsequent algorithm:
\begin{itemize}
	\item \textbf{SHVE.Setup$(\{0,1\}^{\lambda })$}: On input a security parameter ${\lambda }$ , it returns a master secret key $msk$ and the message space $\mathcal{M}$.

\item \textbf{SHVE.KeyGen$\left(msk,{\bf V}\in \Gamma_{*}^{d}\right)$}: On input a master secret key $msk$ and a predicate vector $\displaystyle\mathbf{v}=~\{v_{1},\ldots,v_{d}\}~\in~\Gamma_{*}^{d}$, it returns the $s$ as a decryption key.
\item \textbf{SHVE.Enc$(msk,\mu \in \mathcal{M},\mathbf{ind} \in \Gamma^{d})$}: On input an index vector $ind$, a master key $msk$ and a message $\mu$, it returns the ciphertext $c$ associated with $(\mathbf{ind},\mu)$.

\item \textbf{SHVE.Query$\left(\mathbf{c},\mathbf{s}\right)$}: On input a ciphertext $c$ associated with $(\mathbf{ind},\mu)$ and a decryption key $s$ corresponding to the predicate vector $v$, it returns $\mu$ if ${\cal P}_{\mathrm{v}}^{\mathrm{SHVE}}({\bf x})=1$; else returns null.
\end{itemize}


\section{Problem Formulation}
In this section, we illustrate the system model and the threat model considered in this paper.
%\subsection{System Model}
\iffalse
\begin{figure}[htbp]
	\centering
	\includegraphics[scale=0.30]{systemmodel.pdf}
	\caption{System model}
	\label{system model}
\end{figure}
\fi
\subsection{System Model} Trinity involves three entities: Cloud Server (CS), Data Owner (DO) and Data Users (DU).
\begin{itemize}
	\item \textsf{Cloud server}. The CS is assumed to have sufficient computing and storage capabilities. The primary responsibilities of the CS include storing the ciphertext and secure index provided by the DO. Upon receiving a search token from a DU, the CS processes the token and returns the results to the DU.
	\item \textsf{Data owner}. The DO encrypts the files and generates the corresponding security index, which is uploaded to the CS. Additionally, the DO generates a secret key and securely distributes it to the authorized DU through a secure channel.  
	\item \textsf{Data users}. The DU obtains the key from the DO and creates search tokens corresponding to the spatio-temporal objects being queried. These tokens are then sent to the CS. DU receives the results from the CS after the search is completed.
	
\end{itemize}
% \par The DO generates ciphertext and secure token, and send them to the CS (step \textcircled{1}). The DU sends a request to the DO, and the DO dispatches SK to the DU (step \textcircled{2}). The DU then generates search tokens and sends them to the CS (step \textcircled{3}). Once the CS receives search tokens from the DU, it returns the results to the DU (step \textcircled{4}).
% Based on the model described above, we now describe the definition of Trinity. 
A spatio-temporal database ${\bf DB}$ is defined as: ${\bf DB}=\{{\bf p}_{i},\ {\bf ind}_{i}\}_{i=1}^{N}$, ${\bf p}_{i}=(x_i,y_i,z_i)$ is a space-time node, ${\bf ind}_{i}$ is a file identifier, and $N$ is the size of the quotient filter.
% Note that the structure we use here is a filter, which leads to a very small probability of false positives.
\begin{definition}
\label{3}
For an encrypted spatio-temporal database ${\bf EDB}$ and a search token ST, Trinity is to search a subset $\{{\bf ind}_{i}^{\ast}\}_{i=1}^{c}$ such that $\forall 1\leq i\leq c$, $\mathcal{P}_{i}^{*}\in\mathcal{R}_{\mathcal{Q}}$, where $\mathcal{P}_{i}^{*}$ represents the minimum set of prefix elements that collectively cover the
entire range query $\mathcal{R}_{\mathcal{Q}}$.



\end{definition}
\begin{definition}
\label{4}
The Trinity is a DSSE scheme consisting three algorithms described as follows.
\begin{itemize}
	\item \textup{\textbf{Setup}$(\{0,1\}^{\lambda })\rightarrow(K_{\Sigma},\sigma;\mathrm{EDB})$}: On input a security parameter ${\lambda }$, it returns a secret key set $K_{\Sigma}$, a state $\sigma$ and the encrypted database $\mathrm{EDB}$.

\item \textup{\textbf{Search}} $(\mathrm{Q},K_{\Sigma},\sigma;\mathrm{EDB})\rightarrow \mathrm{R}$: On input a query $\mathrm{Q}$, a secret key set $K_{\Sigma}$, a state $\sigma$ and the encrypted database $\mathrm{EDB}$, the client sends a search request to the server, and the server returns the result $\mathrm{R}$ after searching all over the $\mathrm{EDB}$.
\item \textup{\textbf{Update}}$(K_{\Sigma},O,up,\sigma;\mathrm{EDB})\rightarrow (K_{\Sigma},\sigma^{\prime};\mathrm{EDB}^{\prime})$: On input $(K_{\Sigma},O,up,\sigma;\mathrm{EDB})$, where $O=\{p,ind\}$ and $up\in\{add,del\}$. $up$, $add$ and $del$ denote update, addition, and deletion, respectively. The server adds or deletes the object $O$ from $\mathrm{EDB}$.
\end{itemize}

\end{definition}



\subsection{Threat Model} 
% We assume that DO and DU are trusted. CS is honest-but-curious, which honestly executes the protocol but curiously collects information related to the encrypted data. For the security of DSSE (e.g., Trinity), an adversarial CS $\mathcal {A}$ should not get any useful information from encrypted databases and queries. 
% %
% Such DSSE security is captured by using the game between the real world and the ideal world. Let $\mathcal {L}=\lbrace \mathcal{L}^{Stp},\mathcal {L}^{Srch},\mathcal{L}^{Updt}\rbrace$ be the leakage function that the adversary captures from \textbf{Setup}, \textbf{Search}, and \textbf{Update}. If $\mathcal {A}$ cannot distinguish the outputs of the real game
% $Game_{\mathcal{A}}^{\mathcal{R}}(\lambda)$ and the ideal game $Game_{\mathcal{A}}^{\mathcal{S}}(\lambda)$, then we can say that there is no leakage beyond the leakage function \cite{bost2016ovarphiovarsigma}. The games can be formally defined as followed;

We assume that the Data Owner (DO) and Data User (DU) are trustworthy. The Cloud Server (CS), however, is considered honest-but-curious, meaning it faithfully executes the protocol but may attempt to extract information from encrypted data. 
%
To ensure the security of the DSSE system (e.g., Trinity), the adversary, represented as $\mathcal{A}$, should not be able to gain any meaningful information from the encrypted databases or queries. This security property is formalized through a real-world vs. ideal-world game-based approach. 
%
Let $\mathcal{L} = \{\mathcal{L}^{Stp}, \mathcal{L}^{Srch}, \mathcal{L}^{Updt}\}$ denote the leakage function that captures the information an adversary can potentially obtain from the \textbf{Setup}, \textbf{Search}, and \textbf{Update} phases. The formal definitions of these games are as follows:

\begin{itemize}
 \item The real game $Game_{\mathcal{A}}^{\mathcal{R}}(\lambda)$: $\mathcal{A}$ selects a database DB and generates the EDB via \textup{\textbf{Setup}}.  Subsequently, $\mathcal{A}$ adaptively runs \textup{\textbf{Search}} or \textup{\textbf{Update}}. Throughout the experiment, $\mathcal{A}$ maintains full observational access to the real operational transcripts. Upon conclusion of the adaptive query phase, $\mathcal{A}$ outputs $b \in \lbrace 0,1\rbrace$.
 \item The ideal game $Game_{\mathcal{A}}^{\mathcal{S}}(\lambda)$: $\mathcal{A}$ selects a database DB and generates the EDB via simulator $\mathcal {S}(\mathcal {L}^{{Stp}}(\mathsf {DB}))$.  Subsequently, $\mathcal{A}$ adaptively runs $\mathcal {S}(\mathcal {L}^{{Srch}})$ or $\mathcal {S}(\mathcal {L}^{{Updt}})$. Throughout the experiment, $\mathcal{A}$ maintains full observational access to the simulated operational transcripts. Upon conclusion of the adaptive query phase, $\mathcal{A}$ outputs $b \in \lbrace 0,1\rbrace$.
 \par If there is an efficient simulator $\mathcal{S}$ such that:

\end{itemize}
If $\mathcal{A}$ cannot distinguish between the real-world game $Game_{\mathcal{A}}^{\mathcal{R}}(\lambda)$ and the ideal-world game $Game_{\mathcal{A}}^{\mathcal{S}}(\lambda)$, then the system is considered secure, meaning no information beyond the specified leakage function is leaked. 

\begin{definition}
\label{def1}
A DSSE is $\mathcal {L}$-adaptively secure \cite{curtmola2006searchable} if for any probabilistic polynomial-time (PPT) adversary $\mathcal{A}$, there is an efficient simulator $\mathcal{S}$:
 \begin{equation}
    \nonumber
  \vert Pr[Game_{\mathcal{A}}^{\mathcal{R}}(\lambda)\!\!=\!\!1]-Pr[Game_{\mathcal{A}}^{\mathcal{S}}(\lambda)=1]\vert \leq \mathsf{negl}(\lambda). 
    \end{equation}
\end{definition}

Besides, forward security plays a pivotal role in safeguarding DSSE schemes from leakage-abuse attacks. The DSSE scheme is considered 'forward-secure' if there is no connection between an update of encrypted data and any previously performed search results. Formally speaking, 
\begin{definition}
\label{def2}
\par  A $\mathcal {L}$-adaptively secure DSSE is forward-secure \cite{bost2016ovarphiovarsigma} if the update leakage function $\mathcal {L}^{Updt}$ is defined as follows,
	\begin{equation}
    \nonumber
  \mathcal {L}^{Updt}(op,in=(p,ind))=\mathcal {L}^{\prime }(up,in=(ind,c)) ,
    \end{equation}
   
   where $op$ denotes the operation like addition or deletion, $in$ denotes the input, $ind$ represents the document identifier, and $c$ is the number of update files.
 \end{definition}

\iffalse
\begin{definition}
	Trinity maintains content-private if the leakage functions $\mathcal {L}^{Srch}$, $\mathcal {L}^{Updt}$ can be expressed as:
 \begin{equation}
    \begin{split}
		\nonumber
   \mathcal { L }^{Updt}(up, ind) =\mathcal { L^{\prime } }(up). \\
  \mathcal { L }^{Srch}(Q) = \mathcal { L^{\prime \prime } }(p). 
    \end{split}
	\end{equation}
 \par where $\mathcal { L^{\prime } }$ and $\mathcal { L^{\prime \prime }}$ are stateless.
	\begin{itemize}
		\item Content privacy serves as a mechanism that limits the server's access to information regarding modifications to file IDs during search queries. As a result, update queries cannot reveal any details about the updated files.
		
	\end{itemize}
\end{definition}
\fi





\subsection{Design Goals} 
The design goals of Trinity are described as follows.
\begin{itemize}
	\item \textbf{Dynamic.} The proposed scheme is designed to be dynamically configurable, enabling document addition and deletion operations.
	\item \textbf{Update-efficient.} The proposed scheme aims to achieve more efficient updates compared to existing forward-secure schemes.
    
    \item \textbf{Scalable.} The proposed scheme aims to be more scalable, which means it can efficiently expand its capacity for a continuous stream of data.  \item \textbf{Verifiable.} The proposed scheme aims to be verifiable, which means there would be no false positives in results and no storage waste to keep false positives minimized.  
        %\item \textbf{Accurate and storage-saving.} The proposed scheme aims to minimize false positives while saving  storage costs.  
        \item \textbf{Privacy- preserving.} The proposed scheme aims to be $\mathcal {L}$-adaptively secure and forward-secure. The system should safeguard sensitive information such as file collections, indexes, and background information of keywords from unauthorized access. Furthermore, our scheme adheres to forward security principles, ensuring that the CS remains oblivious to any association between recent updates and prior search results.
\end{itemize}





\section{Trinity Schemes}

In this section, we first introduce a scalable, update-efficient spatio-temporal range search scheme, \TrinityI, that builds on quotient filter, Hilbert curve and SHVE. Subsequently, we propose a forward-secure, and verifiable scheme Trinity-\uppercase\expandafter{\romannumeral2}. Trinity-\uppercase\expandafter{\romannumeral1} is faster than Trinity-\uppercase\expandafter{\romannumeral2} in search and update latency, but it wastes more storage and lacks forward security.

\subsection{\TrinityI: Basic Trinity Construction} 
\subsubsection{Technique Overview}
We treat spatio-temporal data as three-dimensional data and use the Hilbert curve to reduce it to a one-dimensional form. \iffalse To retrieve objects within regions $R_1$ or $R_2$, a DU constructs query ranges $R_1$ = [4, 7] and $R_2$ = [10, 15], respectively.\fi This technique effectively translates spatio-temporal range queries in a multi-dimensional space into multiple one-dimensional range queries. We denote the Hilbert curve of a point p or a range R as $\mathcal{H}(p)$ and $\mathcal{H}(R)$, respectively. Although it is feasible to encode all elements within a specified range into a QF and test for the presence of an encoded space-time node, the QF's size scales linearly with the number of elements. Notably, range queries typically encompass significantly more elements than individual data objects. To optimize QF size, we must minimize the number of elements involved in range queries. To address this challenge, we employ a prefix membership verification technique introduced by Liu $et\ al.$ \cite{liu2010privacy}.
  

 Previously mentioned bloom filter-based schemes \cite{li2022adaptively, zhang2022efficient, miao2023efficient, li2023vrfms, tong2023verifiable} suffer from several challenges: deletion capability, scalable structure, and minimizing false positives. In response to these limitations, our approach leverages the quotient filter, and scalable bloom filter technology to resolve those challenges. The quotient filter employs a fingerprint-based storage mechanism, where each element is represented by a unique fingerprint comprising quotient and remainder components, thereby facilitating precise element identification and efficient deletion operations. 
\iffalse \begin{figure}[htbp]
	
	\centering
	\includegraphics[scale=0.23]{SBF.pdf}
	\caption{Scalable bloom filter}
	\label{SBF}
\end{figure}


 Due to the nature of spatio-temporal data, our retrieval is typically range-based rather than exact. Therefore, we need to support both attribute matching and range retrieval. So, SHVE is selected due to its capability to execute attribute matching queries on encrypted data coupled with wildcard pattern matching support. Alternative cryptographic primitives, including secure $k$-NN \cite{wong2009secure} and homomorphic encryption \cite{canetti2017chosen}, are excluded from consideration due to their inherent limitations in efficiency and security. \fi 

The system's performance inversely correlates with data volume: as data volume increases, both insertion and retrieval operations become less efficient, while false positive rates increase. So we adjust the structure of the quotient filter to make it scalable, i.e. when the amount of inserted data approaches a certain threshold (affecting performance or false positive rate is high), it automatically expands dynamically. We simply borrow one bit from the remainder into the quotient instead of rehashing all elements for the expanding. %we allow QF to expand when the capacity is insufficient or a false positive cannot be guaranteed. When the actual number of elements inserted into the filter exceeds the expected amount, the performance of the filter can be negatively impacted. To address this issue, we propose the following approach: create a new filter that is twice the size of the original filter; rehash all elements from the original filter into the new filter; and replace the original filter with the new filter.
 
 To ascertain the subset relationship between encrypted vector representations, we leverage SHVE for performing privacy-preserving set membership queries within a cryptographically secured computational domain \cite{liang2023privacy}. Specifically, our scheme uses multi-threaded computing technology to speed up SHVE.




\subsubsection{Details of Basic Trinity Construction} 
In this section, we will briefly introduce \textbf{Setup} and \textbf{Search}, focusing on dynamic \textbf{Update}. Here dynamic \textbf{Update} refers to the ability to efficiently modify the data structure that supports the encryption scheme, allowing for the addition and deletion of elements without requiring a complete re-encryption of the dataset. Specially, our scheme uses the murmur hash function and multi-threaded computing SHVE.

\begin{table}
\centering
\label{Notation}
\caption{Notation Description for Trinity}
\resizebox{\linewidth}{!}{
\label{notation}
\begin{tabular}{cl}
\hline
Notations& Descriptions\\
\hline
%$\vec p,\vec q$& Extended vectors\\
$QF$& Quotient Filter\\
$\xi, \xi^{\prime}$& random values between (0, 1)\\
$\check{p}$& spatio-temporal object vector\\
$Q$& search query\\
$N$& size of Quotient Filter\\
$OT_{c}$& order token corresponding to the current count\\
$e_{c+1}$& salted token \\
${M_{1\_{c}}},{M_{2\_{c}}}$& salted matrices\\
$p_i$& encoded spatio-temporal data\\
$O$& space-time node\\
$ind_i$&  the i-th file identifier.\\
$H()$& hash function\\
$\mathcal{H}()$& Hilbert curve encode\\
$P()$& prefix of the nodes\\
$\mathcal{P}()$& prefix of the range\\
$e_{i}$& fingerprint of file identifier\\

${{F_{Q}}}$& quotient of fingerprint\\
${{F_{R}}}$& remainder of fingerprint\\
%$s$& the starting index of the \texttt{run} associated with the quotient.\\
%$sp$& subscript of the last element of s\\
%\texttt{prev}& the value of the slot corresponding to $s$ before addition\\
%\texttt{curr}& the value of the slot corresponding to $s$ after addition\\
%\texttt{next}&  the value of the slot corresponding to $sp$\\
%\texttt{next}& the original value of $s$\\
%$highbits$& High bits of divisor\\
%$clr$& clear\\
%$N$& Root node\\
$ST_{non}$& Search token for non-leaf nodes\\
$ST_{L}$& Search token for leaf nodes\\
\hline
\end{tabular}
}
\end{table}
\par \textbf{Setup$(\{0,1\}^{\lambda })\rightarrow(K_{\Sigma},\sigma;\mathrm{EDB})$:} Given a security parameter $\lambda$ and a database DB, it returns a master key $msk$, $t$ hash functions $H=\{{H}_{i}(\cdot)\}_{i=1}^{t}$, and an encrypted database EDB. And DO share the secret key set $K_{\Sigma}=\{msk,H\}$ with the DU. For each space-time node $O_i=\{p_i,ind_i\}$, encoding the spatio-temporal data. \begin{figure}[!h]
	\centering
	\includegraphics[scale=0.39]{prefix.pdf}
	\caption{An example of encoded Hilbert curve with prefixes }
	\label{prefix}
\end{figure}
For a given $\omega$-bits data item $X=a_{1}~a_{2}\cdot\cdot\cdot a_{\omega}$, its prefix family is defined as a set of $\omega +1$ elements: ${P}(X)=\{a_{1}\,a_{2}\cdot\cdot\cdot a_{\omega},a_{1}\,a_{2}\cdot\cdot\cdot a_{\omega-1}{*},\cdot\cdot\cdot,a_{1}{*}\cdot\cdot\cdot{*},\ast\ast\cdot\cdot\cdot\ast\}$, where the $i$-th prefix element is $a_{1}\ \ a_{2}\cdot\cdot\cdot a_{\omega-i+1}\ *\cdot\cdot\cdot*$. Given a range $[x_{min}, y_{max} ]$, the query $\mathcal{P}\bigl(\left[x_{m i n},\,x_{m a x}\right]\bigr)$ represents the minimum set of prefix elements that collectively cover the entire range. An item X belongs to the range $[x_{min}, y_{max} ]$ if and only if the intersection of its prefix family $P(X)$ and the query prefix set $\mathcal{P}\bigl(\left[x_{m i n},\,x_{m a x}\right]\bigr)$ is non-empty: $X\in[x_{m i n},x_{m a x}] \Leftrightarrow P(X)\cap \mathcal{P}([x_{m i n},x_{m a x}])\neq \emptyset$. 
\begin{algorithm}
\caption{Setup ($1^\lambda$)}
\label{alg:Setup}
\KwIn{$(\{0,1\}^{\lambda }),\mathrm{DB})$}
\KwOut{$(K_{\Sigma},\sigma;\mathrm{EDB})$}
$\mathrm{Setup(1^{\lambda})}\rightarrow msk;$
\;
Randomly generates $H=\{{H}_{i}(\cdot)\}_{i=1}^{t}$\;
$K_{\Sigma}=\{msk,H\}$\;
 \For {{\rm space-time node} $O_i=\{p_i,ind_i\}$ }{
 $H({P}(\mathcal{H}(p_{i})))\rightarrow e_{i}$\;
 $e_{i}\rightarrow ({{F_{Q}}},{{F_{R}}})$\;

 ${\rm QF}\{H,({{F_{Q}}},{{F_{R}}})\}\rightarrow {\rm QF_i}$\;
 SHVE.Enc$(msk,\textbf{“True"},{\rm QF_i})\rightarrow \rm{C_{SHVE}}(QF_i)$\;
 $\{\rm{C_{SHVE}}({\rm QF_{i}}),ind_i\}_{i=1}^{N}\rightarrow $EDB\;

}

  \textbf{return} $(K_{\Sigma},\sigma;\mathrm{EDB})$ \;
\end{algorithm}
Fig.~\ref{prefix} demonstrates examples of spatio-temporal range queries over a spatio-temporal database utilizing our spatio-temporal data and query encoding methodology. Consider a spatio-temporal database holding four space-time nodes, each encoded via the Hilbert curve and represented as a prefix family. To retrieve objects within region $R_1$, the DU initially encodes the query range as $\{[4, 7]\}$ and subsequently generates the corresponding query prefixes as $\{ 01**\}$. Since $001***$ belongs to the prefix family of ${P}(\mathcal{H}({O}_{1})))$, we conclude that $O_1$ resides within $R_1$. To query objects within $R_2$, the DU similarly encodes the query range as $\{[10, 15]\}$ and constructs the corresponding query prefixes as $\{101*, 111*\}$. We determine that $O_3$ and $O_4$ belong to $R_2$ based on the presence of $101* \in {P}(\mathcal{H}({O}_{3})))$ and $111* \in {P}(\mathcal{H}({O}_{4})))$. We create a quotient filter and store the fingerprints ${{H}}({p}_{i})$ of the space-time node in a trinity form $({{H}}({p}_{i}),{{F_{Q}}},{{F_{R}}})$. Finally, we encrypt all the elements within the quotient filter using $msk$. A formal representation of the \textbf{Setup} phase is displayed in Algorithm~\ref{alg:Setup}.

\par\textbf{Search $(K_{\Sigma},\mathrm{Q},\sigma;\mathrm{EDB})\rightarrow \mathrm{R}$}: Given a query $\mathrm{Q}$, a secret key set $K_{\Sigma}$, a state $\sigma$, and the encrypted database $\mathrm{EDB}$, the DU sends a search request to the server. The CS then returns the result $\mathrm{R}$ after searching  through the $\mathrm{EDB}$. To search for a value in a Quotient Filter, DU first calculates the fingerprint $e_q$, and divides it as quotient $F_Q$ and remainder $F_R$, and encrypts them with $msk$. Then Du sends the search token $\rm{S_{SHVE}}({\rm QF_Q})$ to the CS. The CS checks if the target slot corresponding to ${T\_F_{Q}}$ is occupied. If it is occupied, find the start index of the \texttt{run} corresponding to the quotient $F_Q$. The CS continues the loop when the current slot is a continuation of \texttt{run}. The CS starts from the starting position $s$, compares the remainder one by one. If the remainder is equal to ${{F_{R}}}$, return true. If the remainder is greater than ${{F_{R}}}$, return false. If there is no matched remainder in \texttt{run}, return false. A formal representation of the \textbf{Search} phase is displayed in Algorithm~\ref{alg:Search}.


\begin{algorithm}
 \caption{Search}
 \label{alg:Search}
 \KwIn{$(K_{\Sigma},\mathrm{Q},\sigma;\mathrm{EDB})$}
 \KwOut{$ \mathrm{R} $}
 \textbf{DU:} \\
  $H(\mathcal{P}(\mathcal{H}(R)))\rightarrow e_q$,
 $e_{q}\rightarrow (F_{Q},F_{R})$\;
 ${\rm QF}\{H,(F_{Q},F_{R})\}\rightarrow {\rm {\rm QF_Q}}$\;
 SHVE.KeyGen$(msk,{\rm QF_Q})\rightarrow \rm{S_{SHVE}}({\rm QF_Q})$\;
Send $\rm{S_{SHVE}}({\rm QF_Q})$ to the CS\;
 \textbf{CS:} \\
   \If{${\rm is\_occupied}(T\_F_{Q}) == \textup{false}$} {
            $\Return\ \textup{false}$ \ $/\ast$ {Checks\ if\ the\ element\ at\ the\ quotient\ index\ is\ occupied. If not, it returns false immediately.} $\ast/$\\
        } 
 $ s = {\rm find\_run\_index}({\rm QF}, {{F_Q}}) $;\ $/\ast$ {Finds the starting index of the \texttt{run} associated with the quotient.} $\ast/$\\


\Do{$({\rm is\_continuation}({\rm get\_elem}({\rm QF}, s)))$}{$rem = {\rm get\_remainder}({\rm get\_elem}({\rm QF}, s))$\;
\If  {$rem == {{F_R}}$}
	{  
	$ \Return \ \textup{true}$\;
	}

	\Else{$rem > {{F_R}} $\;
	$ \Return \ \textup{false}$\;
	}
  $s = {\rm incr} ({\rm QF},s)$$/\ast$ {Increments the index to access the next element in the \texttt{run}.}$\ast/$}	 
 $ \Return \ \textup{false}$\;
 
\end{algorithm}

\par \textbf{Update$(K_{\Sigma},O,up,\sigma;\mathrm{EDB})\rightarrow (K_{\Sigma},\sigma^{\prime};\mathrm{EDB}^{\prime})$}: Given $(K_{\Sigma},O,up,\sigma;\mathrm{EDB})$, the CS adds or deletes the object $O$ from the $\mathrm{EDB}$. For both addition and deletion, DO encodes the updated data and uses a hash function to obtain the updated fingerprint $e_a $ or $
e_d$ (for addition or deletion). DO then computes the quotient and remainder of the fingerprint. Do encrypt the quotient and remainder and sends them to the server for preparation to be inserted into the quotient filter. 
\begin{algorithm}
\caption{Update\_Addition}
\label{alg:Addition}
\KwIn{$(K_{\Sigma},O,add,\sigma;\mathrm{EDB})$}
\KwOut{$(K_{\Sigma},\sigma^{\prime};\mathrm{EDB}^{\prime})$}
 \textbf{DO:} \\
  $H({P}(\mathcal{H}(p_{a})))\rightarrow e_{a}$,
$e_{a}\rightarrow ({{F_{Q}}},{{F_{R}}})$\;
 ${\rm QF}\{H,(F_{Q},F_{R})\}\rightarrow {\rm QF_A}$\;
 SHVE.KeyGen$(msk,{\rm QF_A})\rightarrow \rm{S_{SHVE}}({\rm QF_A})$\;
Send $\rm{S_{SHVE}}({\rm QF_A})$ to the CS\;
 \textbf{CS:} \\
${\rm prev} = {\rm get\_elem}({\rm QF}, s)$ \;
${\rm empty} = {\rm is\_empty\_element}({\rm prev}) $\;

\Do{$(!{\rm empty})$}{
    \If{$(!{\rm empty})$}{
        ${\rm prev} = {\rm set\_shifted}({\rm prev})$ \;
        \If{$({\rm is\_occupied\_element}({\rm prev}))$}{
            ${\rm curr} = {\rm set\_occupied\_element}({\rm curr})$ \;
            ${\rm prev} = {\rm clr\_occupied\_element}({\rm prev})$ \;
        }
    }
    ${\rm set\_element}({\rm QF}, s, {\rm curr}) $\;
    ${\rm curr} ={\rm prev}$ ,
    $s = {\rm incr}({\rm QF}, s) $,
    ${\rm prev} = {\rm get\_elem}({\rm QF}, s)$ \;
    ${\rm empty} = {\rm is\_empty\_element}({\rm prev}) $\;
}
\If{${\rm(QF_{i}\_entries \ge \frac{1}{20}QF_{i}\_max\_size)}$}{
    $\mathrm{i} ++$;\ $/\ast$ {Expand the original ${\rm QF_{i}}$ to ${\rm QF_{i+1}}$ when ${\rm QF_{i}}$ exceeds 5\% of the capacity.} $\ast/$\\
}


$T\_{{F_{Q}}}$ = {\rm get\_elem}(QF, ${{F_{Q}}}$) \;
entry = (${{F_{R}}}$ $\ll$ 3) \& $ \sim $7 \;

\If{$({\rm is\_empty\_element}({T\_{F_{Q}}}))$}{
    set\_elem(QF, ${{F_{Q}}}$, ${\rm set\_occupied}(entry))$ \;
     $\mathrm{QF\_entries} ++$\;
    \Return true
}

\If{$(!{\rm is\_occupied}({T\_{F_{Q}}})) $}{
    set\_elem(QF, ${{F_{Q}}}$, set\_occupied(${T\_{F_{Q}}}$)) \;
}

${\rm start} = {\rm find\_run\_index}({\rm QF}, {{F_{Q}}})$ \;
$s = {\rm start} $\;


\If{$s != {{F_{Q}}}$}{
    ${\rm entry} = {\rm set\_shifted}({\rm entry})$ \;
}

${\rm insert\_into}({\rm QF}, s, {\rm entry})$ \;
 $\mathrm{QF\_entries} ++$\;
\Return true
\end{algorithm}



\par As for addition, let us define \texttt{prev} as the element currently at index $s$, and \texttt{curr} as the element that will be written to index $s$. If the slot corresponding to the quotient is empty, we add it directly and end. We set the \texttt{is\_occupied} metadata bit and find the starting position of the \texttt{run} corresponding to ${{F_{Q}}}$. Note that the \texttt{is\_occupied} metadata bit in ${{F_{Q}}}$ must be marked. If the slot is not empty, the CS executes the loop: set the previous slot as \texttt{is\_shifted}, if the previous slot is set as \texttt{is\_occupied}, set the current slot \texttt{curr} as \texttt{is\_occupied}, and clear the metadata \texttt{is\_occupied} of the previous slot \texttt{prev}. If the \texttt{is\_occupied} in ${T}_{{F_{Q}}}$ is not marked, then we return the expected starting position of the \texttt{run} (because the \texttt{run} corresponding to ${{F_{Q}}}$ does not exist). Otherwise, we return the starting position of the \texttt{run} (because the \texttt{run} corresponding to ${{F_{Q}}}$ exists). If the \texttt{run} corresponding to ${{F_{Q}}}$ exists, we must determine the specific insertion position to maintain the order of \texttt{run} after insertion. The result is in the variable $s$. If $s$ is the starting position of the \texttt{run}, we must set \texttt{is\_continuation} at the starting position since, after adding the fingerprint at $s$, this element becomes part of the continuation of the \texttt{run}; otherwise, we must set \texttt{is\_continuation} in the element being added. We determine whether the insertion position is a canonical slot. If not, then we set \texttt{is\_shifted} in \texttt{insert\_into} and then move elements one by one.

When the number of entries in ${\rm QF_{i}}$ exceeds 5\% of its maximum size, a new ${\rm QF_{i+1}}$ is created (usually twice the size), the value of quotient $q$ increases by 1 (due to capacity doubling), and the value of remainder $r$ remains unchanged.
The total number of bits in the new filter is $p = q_{new} + r$. For example, the quotient and remainder of fingerprint 110$\big|$0101 are 110 and 0101. In this old filter, $p = 7, r = 4, q = p - r = 3$. As for the new filter that doubles the size, the fingerprint and remainder stay unchanged. The quotient $q_{new} = q+1=3 + 1 = 4$. So the new quotient is taken as the highest four digits 1100, and the remainder is still 0101. A formal representation of the \textbf{Addition} phase is displayed in Algorithm~\ref{alg:Addition}.
%\par Next, we introduce \texttt{insert\_into}.  In each loop, we get the element currently corresponding to the index $s$ to \texttt{prev}. We determine whether \texttt{prev} is empty. If not empty, then we need to set the \texttt{is\_shifted} of \texttt{prev} (moving elements to the right will always cause a shift), and the \texttt{is\_occupied} remains at the quotient. We write \texttt{curr} to the slot corresponding to index $s$. A formal representation of the \textbf{Addition} phase is displayed in Algorithm~\ref{alg:Addition}.

\begin{algorithm}
\caption{Update\_Deletion}
\label{alg:Deletion}
\KwIn{$(K_{\Sigma},O,del,\sigma;\mathrm{EDB})$}
\KwOut{$(K_{\Sigma},\sigma^{\prime};\mathrm{EDB}^{\prime})$}
 \textbf{DO:} \;
  $H({P}(\mathcal{H}(p_{d}))) \rightarrow e_{d}$,
  $e_{d}\rightarrow ({{F_{Q}}},{{F_{R}}})$\;
  ${\rm QF} \{H,(F_{Q},F_{R})\} \rightarrow {\rm QF_D}$\;
  SHVE.KeyGen$(msk,{\rm QF_D}) \rightarrow \rm{S_{SHVE}}({\rm QF_D})$\;
  Send $\rm{S_{SHVE}}({\rm QF_D})$ to the CS\;

 \textbf{CS:} \\
 ${\rm get\_elem}({\rm QF}, s) \rightarrow {\rm curr}$,
 ${\rm incr}({\rm QF}, s) \rightarrow sp$,
 $s \rightarrow {\rm orig}$\;

 \While{$(\textup{true})$}{
    %${\rm get\_elem}({\rm QF}, s) \rightarrow {\rm next}$\;
    ${\rm is\_occupied}({\rm curr}) \rightarrow {\rm curr\_occupied}$\;

    \If{$({\rm is\_empty\_element}({\rm next}) \parallel {\rm is\_cluster\_start}({\rm next}) \parallel sp == {\rm orig})$}{
        ${\rm set\_elem}({\rm QF}, s, 0)$\;
        \textbf{return}\;
    } 
    \Else {
        %${\rm next} \rightarrow {\rm updated\_next}$\;
        \If{$({\rm is\_run\_start}({\rm next}))$}{
            \Do{$(!{\rm is\_occupied}({\rm {\rm get\_elem}({\rm QF}, {\rm F_{Q}}))})$}{
                ${\rm incr}({\rm QF}, {\rm F_{Q}}) \rightarrow {\rm F_{Q}}$\;
            }
            \If{$({\rm curr\_occupied} \&\& \ {\rm F_{Q}} == s)$}{
                ${\rm clr\_shifted}({\rm next}) \rightarrow {\rm updated\_next}$\;
            }
        }
        ${\rm set\_elem}({\rm QF}, s, {\rm curr\_occupied} ? {\rm set\_occupied}({\rm updated\_next}) : {\rm clr\_occupied}({\rm updated\_next}))$\;
        $sp \rightarrow s$\;
        ${\rm incr}({\rm QF}, sp) \rightarrow sp$\;
        ${\rm next} \rightarrow {\rm curr}$\;
    }
}
   
\If{$(!{\rm is\_occupied}(T_{F_Q})) || (!{\rm QF} \rightarrow {\rm QF_{entries}})$}{
     \Return true\;
}

${\rm start} = {\rm find\_run\_index}({\rm QF}, F_{Q})$\;
$s = {\rm start}$\;

\Do{$({\rm is\_continuation}({\rm get\_elem}({\rm QF}, s)))$}{
    $rem = {\rm get\_remainder}({\rm get\_elem}({\rm QF}, s))$\;
    
    \If{$rem == F_{R}$}{
        \textbf{break}\;
    }
    \ElseIf{$rem > F_{R}$}{
         \Return true\;
    }
    
    $s = {\rm incr}({\rm QF}, s)$\;
}

\If{$(rem != F_{R})$}{
    \Return true\;
}

delete\_entry$({\rm QF}, s, F_{Q})$\;

% Step 11
 $\mathrm{QF\_entries} --$\;
\Return true

\end{algorithm}
\par As for deletion, $sp$ represents the index of the entry after $s$, \texttt{curr} represents the value of the slot corresponding to $s$, \texttt{next} represents the value of the slot corresponding to $sp$, and \texttt{next} represents the original value of $s$. The CS retrieves the current element from position $s$ in the QF and obtains the next increment position of $s$, saving the original starting position. Then it enters an infinite loop to process entry sliding. First, it checks if the current slot \texttt{curr} is occupied. If the next slot is empty or marks the start of a cluster, the current slot \texttt{curr} is set as empty. Otherwise, the CS prepares to update the next slot \texttt{next}. If the next slot \texttt{next} marks the start of a run, the CS finds the next occupied quotient $F_{Q}$ and increments the quotient's position. If the current slot \texttt{curr} is occupied and the quotient equals the current position, the \texttt{is\_shifted} flag is cleared.

The CS keeps the slot \texttt{is\_occupied} and updates the position pointer 
$sp$, incrementing both the pointer 
$sp$ and the current slot \texttt{curr}. If the start of a run for the quotient is found at position $s$, the CS traverses consecutive slots to retrieve the remainder of the current slot \texttt{curr}. If a matching remainder is $rem$ found, the traversal terminates. If the remainder $rem$ is greater than the target remainder $F_{R}$, the CS returns true, indicating that no match was found. The CS then moves $s$ to the next slot and continues until the target remainder is found. Once the target remainder is found, the CS deletes the entry. A formal representation of the \textbf{Deletion} phase is displayed in Algorithm~\ref{alg:Deletion}.


%first the CS  retrieves the current element from the position $s$ of the QF, obtains the next incremental position of $s$, and saves the original starting position.  if the \texttt{is\_occupied} bit is not set for the slot corresponding to the entry, then no further steps are required. Locate the start of the \texttt{run} and find the entry to be deleted. If the entry to be deleted is the start of the \texttt{run} and the next entry does not belong to the \texttt{run}, then the \texttt{is\_occupied} bit for the \texttt{run} must be cleared. We delete the entry and shift all subsequent entries forward. If the entry that was just deleted was the start of the \texttt{run}, the new \texttt{run} that starts after the deletion may need to have its \texttt{is\_continuation} and \texttt{is\_shifted} bits cleared. 

%The \texttt{delete\_entry()} function works as follows: During deletion, 

\iffalse The following steps are performed in the loop, depending on the type of \texttt{next}:
\begin{itemize}
	
\item If \texttt{next} is the start of a \texttt{cluster} or an empty slot, or if \texttt{next} has already looped back to \texttt{next}, then no further processing is required. The slot corresponding to $s$ is cleared and the function exits.

\item If \texttt{next} is the start of a \texttt{run} that does not start at the beginning of a \texttt{cluster} (or if \texttt{next} is the start of a shifted \texttt{run}), then the cursor moves forward one position. The \texttt{is\_shifted} bit may need to be cleared, so the quotient of the \texttt{run} must be calculated before making a decision. Additionally, the \texttt{is\_occupied} bit must be moved forward.
\item If \texttt{next} is a subsequent element of a \texttt{run}, then the cursor moves forward one position. The \texttt{is\_continuation} and \texttt{is\_shifted} bits are still set. Additionally, the \texttt{is\_occupied} bit is kept at the quotient position.
\end{itemize}
\par A formal representation of the \textbf{Deletion} phase is displayed in Algorithm~\ref{alg:Deletion}.
\fi


\subsection{Trinity-\uppercase\expandafter{\romannumeral2}: Trinity with Improved Security and Accuracy} 

\subsubsection{Technique Overview}
The basic Trinity we proposed earlier is efficient enough, but as we said before, we want a secure and accurate and dynamic Trinity. So first, we use “salts" and CPRF techniques to avoid leakage from addition. To achieve this, Trinity employs a technique where hashed order tokens, denoted as $OT_i$, are used to introduce unique “salts" to the encrypted data points. This salting process is streamlined by utilizing a CPRF, which effectively minimizes bandwidth usage. \iffalse While it's straightforward to compute $OT_{c+1}$ from $OT_c$ using $\mathbb{G}(ST, 0)$, reversing this process requires access to  
$\bar{\mathbb{G}}(msk, C_{c+1})$. \fi Upon a new node being added to the EDB (specifically, the $(i+1)$th addition), a special ordering token, $OT_i$, is generated by the DO using the secret key $K$. This token is then used to “salts" the space-time node, enhancing its security. When a search query is initiated, the DUs provide both the ST and the query itself to the CS. The CS, in turn, leverages the ST to compute all necessary ordering tokens, subsequently reversing the salting process to unveil the original encrypted points. 



We utilize the quotient filter data structure, a variant of the bloom filter. Like the bloom filter, the quotient filter also exhibits a false positive rate. And the FPR is an inherent challenge in filters, stemming from the design philosophy of the filter. It consists of a fixed-size binary bit array and a series of random mapping functions (hash functions). The core idea is to use multiple different hash functions to address conflicts. Due to the issue of hash collisions (where two different elements may map to the same value after applying a hash function), it introduces multiple hash functions to reduce collisions. If any hash function determines that an element is not in the set, then the element is definitely absent. The element likely exists only when all hash functions unanimously indicate its presence.

\begin{equation}
		\nonumber
		\textup{FPR}=\left ( 1-e^{-k\cdot \frac{n}{m} } \right ) ^{k},
	\end{equation}
 where $k$ is the number of hash functions, $n$ is the number of elements, $m$ is the length of the filter, $e$ is natural constant.
 \par Since false positives are inevitable in filter structures, we must implement a verification method to validate results. For ensuring the accuracy of results, we maintain a dedicated verify token for each file, linked to its unique ind. Bitmaps, known for their simplicity and efficiency, employ a binary bit array to represent information. The most prevalent bitmap indexing technique involves associating each bit with a specific element's position: a 1 signifies the element's presence within a set, while a 0 indicates its absence. Then we compress verify token with Roaring Bitmaps \cite{Chambi2014BetterBP} and encrypt it. Upon executing a search request, the verify tokens are presented to the DU alongside the other results. Having obtained these tokens, the DU decrypts and authenticates the results using decrypted verify token. The verification process not only reduces FPR but also decreases the quotient filter size, resulting in surprisingly significant space cost savings.
\subsubsection{Details of Trinity-\uppercase\expandafter{\romannumeral2} Construction} 
In this section, we will briefly introduce \textbf{Setup} and \textbf{Search}, focusing on \textbf{Addition} and \textbf{Verification}. \textbf{Deletion} phase in Trinity-\uppercase\expandafter{\romannumeral1} and Trinity-\uppercase\expandafter{\romannumeral2} is the same, so it will not be repeated here

\par \textbf{Setup$(\{0,1\}^{\lambda },0\rightarrow c)\rightarrow(K_{\Sigma},\sigma;\mathrm{EDB},OT_c)$:}
Given a security parameter $\lambda$, a update counter $c$ and a database DB, it returns a master key $msk$, $t$ hash functions $H=\{{H}_{i}(\cdot)\}_{i=1}^{t}$, an order token $OT_c$ and an encrypted database EDB. One of the key difference between Trinity-\uppercase\expandafter{\romannumeral1} and Trinity-\uppercase\expandafter{\romannumeral2} is the introduction of counter c, which tracks the number of updates. And the DO generates $OT_i$ by $\mathbb{G}$ with the input of $msk$ and $c$. Within the salting process, $OT_c$ is itself “salted" using a hash function H. Subsequently, the DU incorporates this salt into the original fingerprints, $e_{i}$, resulting in the generation of salted fingerprints, $e_{i}^c$.
And rest are the same to the Trinity-\uppercase\expandafter{\romannumeral1}. A formal representation of the \textbf{Setup} phase is displayed in Algorithm~\ref{alg:setup}.
\begin{algorithm}
\caption{Setup ($1^\lambda$)}
\label{alg:setup}
\KwIn{$(\{0,1\}^{\lambda }),c,\mathrm{DB})$}
\KwOut{$(K_{\Sigma},\sigma;OT_c,\mathrm{EDB})$}
$\mathrm{Setup(1^{\lambda})}\rightarrow msk$
\;
Randomly generates $H=\{{H}_{i}(\cdot)\}_{i=1}^{t}$\;
$K_{\Sigma}=\{msk,H\}$,
$\mathbb{G}(msk,c)\rightarrow OT_{c}$,
$H(K, OT_{c})\rightarrow e^{c}$\;
 \For {{\rm space-time node} $O_i=\{p_i,ind_i\}$ }{
 $H({P}(\mathcal{H}(p_{i})))\rightarrow e_{i}$,
$e_{i}\oplus e^{c}\rightarrow e_{i}^{c}$,
 $e_{i}^{c} \rightarrow ({F_{Q}^{c}},{F_{R}^{c}})$\;
 ${\rm QF}\{H,{{F_{Q}^{c}}},{{F_{R}^{c}}})\}\rightarrow {\rm QF_i}$\;
 SHVE.Enc$(msk,\textbf{“True"},{\rm QF_i})\rightarrow \rm{C_{SHVE}}({\rm QF_i})$\;
 $\{\rm{C_{SHVE}}({\rm QF_{i}}),ind_i\}_{i=1}^{N}\rightarrow $EDB\;

}

  \textbf{return} $(K_{\Sigma},\sigma;\mathrm{EDB},OT_c)$ \;
\end{algorithm}

\begin{algorithm}
 \caption{Search}
 \label{alg:search}
 \KwIn{$(K_{\Sigma},\mathrm{Q},\sigma;\mathrm{EDB})$}
 \KwOut{$ \mathrm{R} $}
 \textbf{DU:} \\
$\bar{\mathbb{G}}(msk,c)\rightarrow ST$\;

Send $(ST,K,\rm{S_{SHVE}}({\rm QF_Q}))$ to the CS\;
 \textbf{CS:} \\
 $ \mathbb{G}(ST,c)\rightarrow OT_c$,
 $H(K, OT_{c})\rightarrow e^{c}$,
  $H(\mathcal{P}(\mathcal{H}(R)))\rightarrow e_q$\;
$e_{q}\oplus e^{c}\rightarrow e_{q}^{c}$,
  $e_{q}^{c} \rightarrow ({F_{Q}^{c}},{F_{R}^{c}})$,
 ${\rm QF}\{H,F_{Q}^{c},F_{R}^{c})\}\rightarrow {\rm QF_Q}$\;
 SHVE.KeyGen$(msk,{\rm QF_Q})\rightarrow \rm{S_{SHVE}}({\rm QF_Q})$\;
   \If{${\rm is\_occupied}(T\_F_{Q}) == \textup{false}$} {
            $\Return\ \textup{false}$ \ $/\ast$ {Check if the element at the quotient index is occupied. Return false if unoccupied.} $\ast/$\\
        } 
 $ s = {\rm find\_run\_index}({\rm QF}, {{F_Q}}) $;\ $/\ast$ {Finds the starting index of the \texttt{run} associated with the quotient.} $\ast/$\\


\Do{$({\rm is\_continuation}({\rm get\_elem}({\rm QF}, s)))$}{$rem = {\rm get\_remainder}({\rm get\_elem}({\rm QF}, s))$\;
\If  {$rem == {{F_R}}$}
	{  
	$ \Return \ \textup{true}$\;
	}

	\Else{$rem > {{F_R}} $\;
	$ \Return \ \textup{false}$\;
	}
  $s = {\rm incr} ({\rm QF},s)$$/\ast$ {Increments the index to move to the next element in the \texttt{run}.}$\ast/$}	 
 $ \Return \ \textup{false}$\;
 
\end{algorithm}
\par \textbf{Search $(K_{\Sigma},\mathrm{Q},\sigma;\mathrm{EDB})\rightarrow \mathrm{R}$:}
Trinity-\uppercase\expandafter{\romannumeral1} and Trinity-\uppercase\expandafter{\romannumeral2} are highly similar in the search phase, with the only difference being that Trinity-\uppercase\expandafter{\romannumeral2} employs $K$ and $e_{q}^{c}$ for salting desalting. Search token $e_{q}^{c}$ are like $e_{i}^{c}$ but input with a range query instead of space-time node. After the search, all added nodes in QF.Cache are desalted and transferred to ${\rm QF_i}$. A formal representation of the \textbf{Search} phase is displayed in Algorithm~\ref{alg:search}.
\begin{algorithm}
\caption{Addition}
\label{alg:addition}
\KwIn{$(K_{\Sigma},O,add,\sigma;\mathrm{EDB})$}
\KwOut{$(K_{\Sigma},\sigma^{\prime};\mathrm{EDB}^{\prime})$}
 \textbf{DO:} \\
$\mathbb{G}(msk,c)\rightarrow OT_{c}$,

$H(K, OT_{c})\rightarrow e^{c}$,
  $H({P}(\mathcal{H}(p_{a})))\rightarrow e_{a}$\;
 $e_{a}\oplus e^{c}\rightarrow e_{a}^{c}$,
 $e_{a}^{c} \rightarrow ({F_{Q}^{c}},{F_{R}^{c}})$\;
Send $(c,F_{Q}^{c},F_{R}^{c})$ to the CS\;
 \textbf{CS:} \\
 $(c,F_{Q}^{c},F_{R}^{c})\rightarrow $QF.Cache\;

${\rm prev} = {\rm get\_elem}({\rm QF_{i}}, s)$ \;
${\rm empty} = {\rm is\_empty\_element}({\rm prev}) $\;

\Do{$(!{\rm empty})$}{
    \If{$(!{\rm empty})$}{
        ${\rm prev} = {\rm set\_shifted}({\rm prev})$ \;
        \If{$({\rm is\_occupied\_element}({\rm prev}))$}{
            ${\rm curr} = {\rm set\_occupied\_element}({\rm curr})$ \;
            ${\rm prev} = {\rm clr\_occupied\_element}({\rm prev})$ \;
        }
    }
    ${\rm set\_element}({\rm QF}, s, {\rm curr}) $\;
    ${\rm curr} ={\rm prev}$ ,
    $s = {\rm incr}({\rm QF}, s) $,
    ${\rm prev} = {\rm get\_elem}({\rm QF}, s)$ \;
    ${\rm empty} = {\rm is\_empty\_element}({\rm prev}) $\;
}
%\If{${\rm(QF_{i}\_entries \ge QF_{i}\_max\_size)}$}{
 %   \Return false}
\If{${\rm(QF_{i}\_entries \ge \frac{1}{5}QF_{i}\_max\_size)}$}{
    $\mathrm{i} ++$;\ $/\ast$ {Expand the original ${\rm QF_{i}}$ to ${\rm QF_{i+1}}$ when ${\rm QF_{i}}$ exceeds half capacity.} $\ast/$\\
}
$T\_{{F_{Q}}}$ = get\_elem(${\rm QF_{i}}, {{F_{Q}^{c}}}$) \;
entry = (${{F_{R}^{c}}}$ $\ll$ 3) \& $ \sim $7 \;

\If{$({\rm is\_empty\_element}({T\_{F_{Q}}}))$}{
    set\_elem$({\rm QF_{i}}, {{F_{Q}^{c}}}, {\rm set\_occupied}({\rm entry}))$ \;
     $\mathrm{QF\_entries} ++$\;
    \Return true
}

\If{$(!{\rm is\_occupied}({T\_{F_{Q}}})) $}{
    ${\rm set\_elem}({\rm QF_{i}}, {{F_{Q}^{c}}}, {\rm set\_occupied}({T\_{F_{Q}}}))$ \;
}

${\rm start} = {\rm find\_run\_index}({\rm QF_{i}}, {{F_{Q}^{c}}})$ \;
$s = {\rm start} $\;
\iffalse
\If{$({\rm is\_occupied}({T\_{F_{Q}}})) $}{
    \Do{$({\rm is\_continuation}({\rm get\_elem}({\rm QF_{i}}, s)))$}{
       $ rem = {\rm get\_remainder}({\rm get\_elem}({\rm QF_{i}}, s))$ \;
        \If{rem == ${{F_{R}^{c}}}$}{
            \Return true
        } \ElseIf{rem  $>{{F_{R}^{c}}}$}{
            break
        }
        $s = {\rm incr}({\rm QF_{i}}, s)$
    }
    
    \If{$s == {\rm start}$}{
        ${\rm old\_head} = {\rm get\_elem}({\rm QF_{i}}, {\rm start})$ \;
       $ {\rm set\_elem}({\rm QF_{i}}, {\rm start},$ ${\rm set\_continuation}({\rm old\_head})) $\;
    } \Else{
        ${\rm entry} = {\rm set\_continuation}({\rm entry})$ \;
    }
}
\fi
\If{$s != {{F_{Q}^{c}}}$}{
    ${\rm entry} = {\rm set\_shifted}({\rm entry})$ \;
}

${\rm insert\_into}({\rm QF_{i}}, s, {\rm entry})$ \;
$\mathrm{QF\_entries} ++$\;
\textbf{return} true\;
\end{algorithm}

\par \textbf{Update $(K_{\Sigma},O,add,\sigma;\mathrm{EDB})\rightarrow (K_{\Sigma},\sigma^{\prime};\mathrm{EDB}^{\prime})$:} Since delete operation in Trinity-\uppercase\expandafter{\romannumeral1} and Trinity-\uppercase\expandafter{\romannumeral2} are exactly the same, so here we discuss add operation only. Every time the DO wants to add a space-time node $p=(x,y,z)$ to EDB, first, the DO increases the value of the counter to the match node O. Then the DO starts the “salted" process like \textbf{Setup} phase. Next, the DO sends the salted token into the QF.Cache of the server like in Fig.~\ref{quotient SBF}. If the number of entries in QF exceeds 20\% of its maximum size, expand the QF twice as the original size. The salted values $(e_{a}^{c},F_{Q}^{c},F_{R}^{c})$ are desalted and sent to ${\rm QF_{i}}$ after the next search. 

 The remaining operations are identical to those in Trinity-\uppercase\expandafter{\romannumeral1}. A formal representation of the \textbf{Addition} phase is displayed in Algorithm~\ref{alg:addition}.
\begin{figure}[htbp]
	
	\centering
	\includegraphics[scale=0.20]{quotientSBF.pdf}
	\caption{An example of addition}
	\label{quotient SBF}
\end{figure}
\par \textbf{Verification:}
  As mentioned earlier in technique overview, in order to maintain a low false positive rate of about 0.01$\%$, it is customary to set the ratio $N/m$ at 20, and the parameter $t$ at 14. This implies that for a database with a capacity of 5 million space-time nodes, a filter size of 100 million would be required. Such a configuration results in a considerable waste of storage space.But with the help of verification, we do not require such low FPR, we only have to keep a basic accuracy of quotient filter that doesn't affect normal search function. Based on verification, we set FPR$\leq 6\% $, and $t=-ln(p)/ln(2)$, we have 4.05. And assume we have 1 millions elements in the filter, we should set the length of filter as $N=mt/ln(2)$, the number is 5.85 millions. So we set $N/m=6$, $t=4$, and the FPR is 5.56$\%$. Consequently, both storage and computational costs decrease precipitously. Next, I will formally describe the Trinity-\uppercase\expandafter{\romannumeral2}. Verification Process, which consists of two distinct phases: Setup and Search.
\begin{itemize}
\item \textbf{Setup}: An additional verify array is generated for each space-time node ${V}(\mathcal{H}(p_i))\rightarrow va_i$. An extra secret key $sk$ is generated to encrypt verify arrays. Then we compress the verify array $\textbf{comp}(vt_i)\rightarrow vt_{i}^{\prime}$ and encrypt it $\textbf{Enc}(va_{i}^{\prime})\rightarrow vt_{i}$. Finally, we store $vt_{i}$ associated with the $ind_i$. The rest procedure are the same to the Trinity-\uppercase\expandafter{\romannumeral2}. 
\item \textbf{Search}: The CS sends verify token $vt_i$ along with other results to the DU. Once receiving results, the DU decrypts the $vt_i$ and verifies the token according to the query, if it matches, keep the results; if not, remove the index.
\end{itemize}
\section{Security Analysis}
This section delves into security analysis of our scheme. We begin by defining a leakage function, and then proposing a rigorous proof of the construction's security.

\par  The leakage function, denoted as $\mathcal{L}$, encompasses leakage from three distinct phases: setup, search, and update operations.
$\mathcal {L}=\lbrace \mathcal{L}^{Stp},\mathcal {L}^{Srch},\mathcal{L}^{Updt}\rbrace$. Meanwhile, the leakage function can also be explained as follows:
%\par Search pattern: The search pattern, denoted as ${sp}$, exposes instances where identical keywords are retrieved using distinct search tokens.
\par Access pattern: The access pattern, denoted as ${ap}$, signifies the specific locations of documents that align with the query.
\par Size pattern: The size pattern, denoted as $\mathbb{S}$, reflects in the number of objects $m$.
%\par Update History $\textsf {UpHist}$: The update history function $\textsf {UpHist}$ records all object modifications as a collection of tuples, each structured as $(u, e_{i}, e_{i}^c)$, where $u$ signifies the timestamp of the update.
\begin{theorem}\label{leakage}
	
	Define the leakage function $\mathcal{L}_1$ of Trinity-{\uppercase\expandafter{\romannumeral1}}, if it can be described as following: 
	\begin{equation}
		\nonumber
		\mathcal{L}_{\uppercase\expandafter{\romannumeral1}}= \{\mathbb{S}(DB),ap\},
	\end{equation}
 
	Trinity-{\uppercase\expandafter{\romannumeral1}} is Indistinguishability under Selective Chosen-Plaintext Attack (IND-SCPA) secure if SHVE is IND-SCPA secure.
	
\end{theorem}
\begin{table*}[htbp]\footnotesize
	\caption{Complexity comparisons}
	\centering
	\label{Comparison}
	%\setlength{\tabcolsep}{6.6mm}
 \resizebox{\linewidth}{!}{
	\begin{threeparttable}
		{	\begin{tabular}{ | c | c | c | c | c | c |}
				\hline	
				
				\multicolumn{1}{|c|}{ \multirow{2}*{$Scheme$} }& \multicolumn{2}{c|}{Computation} & \multicolumn{2}{c|}{Communication}& \multirow{2}*{Storage size}\\
				\cline{2-5} 
				$ $&$Search$&$Update$&$Search$&$Update$&\\
               
				%\hline
				%ETSQ*& $\mathcal {O}(k \cdot \log_{ }{N})$& $\mathcal {O}(2k \cdot \log_{ }{N})$&$\mathcal {O}(k \cdot \lambda)$& $\lambda \cdot \mathcal {O}(k \cdot \log N+\log c)$&$\mathcal {O}(2N)$ \\
                \hline
				GRS-\uppercase\expandafter{\romannumeral 2}& $\mathcal {O}(\log_{ }{R}\cdot N)$& $\mathcal {O}(2^t N)$&$\mathcal {O}(\log_{ }{R}\cdot N)$& $\mathcal {O}(2^t N)$&$\mathcal {O}(2^t N)$ \\
               \hline
				$\mathsf {DSSE}_{\mathsf {SKQ}}$ &$\mathcal {O}(a_{w}(|R|+k))$& $\mathcal {O}(\log N+\log m)$&$\mathcal {O}((1+k)l)$& $\mathcal {O}((\log N+\log m)(2\lambda +l))$&$\mathcal {O}(a_{w}(\log N+\log m)(2\lambda +l))$ \\
                \hline
				SKSE-{\uppercase\expandafter{\romannumeral2}} &$\mathcal {O}(\overline{x}k m\cdot \log m)$& $N/A$&$  \mathcal {O}(\overline{x}k\lambda m\cdot \log m )$& $N/A$&$\mathcal {O}(m\lambda N)$ \\
                \hline
                Trinity-{\uppercase\expandafter{\romannumeral1}} &$\mathcal {O}(k m\cdot \log m)$& \textcolor{red}{$\mathcal {O}(k)$}&$  \mathcal {O}(k\lambda m\cdot \log m )$& $\mathcal {O}(k \lambda)$&$\mathcal {O}(m\lambda N)$ \\
                \hline
                Trinity-{\uppercase\expandafter{\romannumeral2}} &$\mathcal {O}(k (m\cdot \log m+\log c))$& \textcolor{red}{$\mathcal {O}(k)$}&$ \mathcal {O}(k\lambda (m\cdot \log m+\log c))$& $\mathcal {O}(k \lambda)$&$\mathcal {O}(m\lambda N)$ \\
				\hline
				%SES-ESTD&$\mathcal {O}(k \cdot \log_{ }{N} )$& $\mathcal {O}(\log_{ }{N})$&$k \cdot \mathcal {O}(12+|r|+|\varepsilon|+|\varepsilon|^2+I)$& $\mathcal {O}(1)$&$\mathcal {O}(2N)$ \\
				
				
				
				%\hline	
				
		\end{tabular}}
		
		\begin{tablenotes}
			\footnotesize
			\item[ ] \textbf{Notes.} $N$ is the number of data points in the database(\textit{i.e.} the number of entries), $M$ is the number of keywords in the database, $m$ is the size of  database, $|U|$ is the size of each ciphertext,  $\lambda$ is a security parameter, $k$ is the number of returned results range, $|k|$ is the size of key $K$, $|\varepsilon|$ is the size of random vector $\varepsilon$, $\overline{x}$ is the average number of the nodes at each level that traversed by the CS, $c$ is the node number of GGM tree and $R$ is search range radius. $2^t$ is the size of the binary tree.
		\end{tablenotes}
	\end{threeparttable}
 }
\end{table*}


\par \begin{proof} The proof of Theorem \ref{leakage} is established through a simulation-based approach. Assuming that the adversary $\mathcal{A}$ cannot differentiate between the output of the real game and the ideal game, we can conclude that no leakage exists beyond $\mathcal{L}_1$. All spatio-temporal data are encrypted by SHVE which is proved in \cite{li2021secure}.
The security of Trinity-{\uppercase\expandafter{\romannumeral1}} is based on the security of SHVE. So the proof of Theorem \ref{leakage} leverages a simulation as follows:
\begin{itemize}
	\item \textbf{Setup}: The adversary $\mathcal{A}$ transmits a chosen database, denoted as ${\bf DB}=\{{\bf p}_{i},\ {\bf ind}_{i}\}_{i=1}^{N}$, to the challenger $\mathcal{C}$. The challenger $\mathcal{C}$ generates both a set of $t$ random hash functions H and a secret key $msk$, maintaining the secrecy of $K_{\Sigma}=\{msk,H\}$. 
\item \textbf{Phase 1}: The adversary $\mathcal{A}$ proceeds to adaptively select a series of queries, each represented as $Q_j$, where $j \in [q_1]$. In response to each query, $\mathcal{C}$ encodes the range query $Q_j$ into ${P}(\mathcal{H}(Q_j)))$. Subsequently, for each prefix element $pe_k$ within ${P}(\mathcal{H}(Q_j)))$ (where $1 \leq k \leq \beta$), $\mathcal{C}$ constructs a quotient filter comprises $({{H}}({p}_{i}),{{F_{Q}}},{{F_{R}}})$, and then encrypts it with SHVE, with $'0'$ positions being substituted with $'\star'$ and $'1'$ positions remaining unaltered. Finally, $\mathcal{C}$ transmits the search token $ST_j$ to $\mathcal{A}$.
\item \textbf{Challenge}: $\mathcal{C}$ randomly selects a bit, denoted as $b$, from the set $\{0, 1\}$. For each space-time node $O_i$, $\mathcal{C}$ constructs a corresponding quotient filter, $QF\{H,(e_{i},{{F_{Q}}},{{F_{R}}})\}\rightarrow {\rm QF_i}$. Subsequently, $\mathcal{C}$ offers $\mathcal{A}$ with the challenge encrypted database EDB, constructed by SHVE.Enc$(msk,{\rm QF_i})\rightarrow \rm{C_{SHVE}}({\rm QF_i})$ for $i \in [N]$.
\item \textbf{Phase 2}: $\mathcal{A}$ repeats the \textbf{Phase 1} procedure and receives $ST_j$ for $q_1 +1\leq j\leq q_2$.
\item \textbf{Guess}: $\mathcal{A}$ takes a guess of $b$.
\end{itemize}

\par Due to the utilization of SHVE for encryption within our Trinity-{\uppercase\expandafter{\romannumeral1}}, the indistinguishability of indexes and trapdoors within the Trinity-{\uppercase\expandafter{\romannumeral1}} directly inherits the indistinguishability property of SHVE. Notably, the security game associated with our Trinity-{\uppercase\expandafter{\romannumeral1}} is essentially simulated by $q^{\prime}$ instances of SHVE, where $q^{\prime}$ denotes the total number of SHVE instances engaged in the game. As a result, the ability of adversary $\mathcal{A}^{\prime}$ to distinguish between two SHVE encrypted values would directly enable it to distinguish between indexes and trapdoors within the Trinity-{\uppercase\expandafter{\romannumeral1}}. This relationship can be succinctly expressed as follows:
\begin{equation}
	\nonumber
 \begin{array}{l}{{\mathbf{Adv}_{\mathrm{Trinity-{\uppercase\expandafter{\romannumeral1}}},\mathcal{A}}^{\mathrm{IND-SCPA}}(1^{\lambda})\leq\mathbf{Adv}_{\mathrm{SHVE},\mathcal{A}^{\prime}}^{\mathrm{IND-SCPA}}(1^{\lambda})}} {{\leq q^{\prime}\cdot\mathrm{negl}(\lambda)}}\end{array}.
\end{equation}
\par Since Trinity-{\uppercase\expandafter{\romannumeral2}} also utilizes SHVE as its encryption tool, its security can be proven using the same approach. 
\end{proof}
\par We formally define forward security for Trinity-{\uppercase\expandafter{\romannumeral2}}, where CPRF and “salts" work in tandem to achieve this property. This combination effectively thwarts adaptive file injection attacks, as rigorously proven by the following demonstration.

\par \textbf{Forward Security.} 
Forward security prevents information leakage during the addition of new data in dynamic updates. For clarity, we adopt the precise definition from \cite{bost2016ovarphiovarsigma}.
\begin{theorem}
	
If the function $\mathcal{L}^{FS}=\{\mathcal{L}^{Updt}\}$ is defined as follows, Trinity-{\uppercase\expandafter{\romannumeral2}} can be considered as forward-secure: 
	\begin{equation}
    \begin{split}
		\nonumber
   \mathcal{L}^{Updt}(op, in)= \mathcal{L}^{\prime}(op, {(ind_i, c)}). \\
  %\mathcal {L}^{Srch}=\lbrace sp,ap,\mathbb{S},\textsf {UpHist}\rbrace. 
    \end{split}
	\end{equation}

	The function $\mathcal{L}^{\prime}$ is stateless and $c$ represents the number of updated keywords for the updated file $ind_i$.
	
\end{theorem}
 \begin{proof}  
 First, spatio-temporal data is encrypted locally by the DO using SHVE before being transmitted to the CS. Then, Trinity-{\uppercase\expandafter{\romannumeral2}} uses SHVE to ensure document security while keeping the key private. In this way, the master key $msk$ of SHVE is held only by the DO, thus preventing unauthorized access by the CS. Indices undergo encryption locally before being uploaded to the CS for security and confidentiality. The quotient filter stores only fingerprints, quotients and remainders. And ST generation encrypts each range query using SHVE, CPRF and “salts", ensuring confidentiality. the DO incorporates “salts" alongside SHVE when adding nodes to the quotient filter, effectively preventing the CS from inferring keyword information from past search tokens.
\end{proof}



\section{Performance Evaluation}
We implemented Trinity in C++\footnote{https://github.com/eulermachine/trinity/}, with SHVE utilizing multi-threading technology, CPRF implemented via HMAC256, and Murmur serving as the hash function. In this section, a comprehensive comparison is made between the construction performance, volume size, token generation efficiency, search capability, addition performance, and deletion performance of Trinity-\uppercase\expandafter{\romannumeral 1} and Trinity-\uppercase\expandafter{\romannumeral 2}, as compared to those of SKSE-\uppercase\expandafter{\romannumeral 2} \cite{wang2021enabling}, GRS-\uppercase\expandafter{\romannumeral 2} \cite{kermanshahi2020geometric}, and $\mathsf {DSSE}_{\mathsf {SKQ}}$ \cite{wang2022forward}. We conducted experiments on an Ubuntu 20.04.5 LTS system using an Intel Xeon Gold 6226R processor. This processor has a base clock speed of 2.90 GHz. The system featured 64*8=512 GB of RAM. Network bandwidth used during these experiments was set at 100 Mbps.
\subsection{Theoretical Analysis} Table~\ref{Comparison} provides a comprehensive comparison of the time and space complexity of  GRS-\uppercase\expandafter{\romannumeral 2}, $\mathsf {DSSE}_{\mathsf {SKQ}}$, SKSE-\uppercase\expandafter{\romannumeral 2}, Trinity-\uppercase\expandafter{\romannumeral 1} and Trinity-\uppercase\expandafter{\romannumeral 2}. Since we use quotient filter as search structure and there are k queries sent by the DUs, the Trinity-\uppercase\expandafter{\romannumeral 1}'s computation complexity of search is $k m\cdot \log m$. So is computation complexity of update. As for communication complexity of search, it is $\mathcal {O}(k\lambda m\cdot \log m )$. The communication complexity of update is $\mathcal {O}(k\lambda m)$. Trinity-\uppercase\expandafter{\romannumeral 2} differs in two aspects: the use of “salts" and QF.Cache. The employed GGM-tree requires $\mathcal {O}(\log c )$ computation time to retrieve ST or OT. So we add $\mathcal {O}(\log c )$ and $\mathcal {O}(\lambda \log c )$ to the computation complexity and communication complexity of search in Trinity-\uppercase\expandafter{\romannumeral 2}. And since we set up a QF.Cache, there is an extra size of EDB, but we set the size of QF.Cache to $\mathcal {O}(\log m)$, which is negligible.


\subsection{Datasets} We use a spatio-temporal dataset from Yelp to evaluate the performance. We selected 1,007,016 timestamped locations, with some locations appearing multiple times at different timestamps. In the context of fine-grained spatiotemporal data, we refer to the logistics transportation scenario and set the spatial granularity to 10 meters and the temporal granularity to 5 minutes\cite{liang2019urbanfm}.
\begin{figure}[htbp]
\centering{
\subfloat[EDB setup latency]{\includegraphics[width=.48\columnwidth]{build.pdf}\label{fig:5_build} }
\subfloat[EDB size]{\includegraphics[width=.50\columnwidth]{EDB.pdf}\label{fig:5_EDB} }

}
\caption{Trinity-\uppercase\expandafter{\romannumeral 1} vs Trinity-\uppercase\expandafter{\romannumeral 2},SKSE-\uppercase\expandafter{\romannumeral 2},GRS-\uppercase\expandafter{\romannumeral 2} and $\mathsf {DSSE}_{\mathsf {SKQ}}$ setup performance}
\label{fig:5_comp} 
\end{figure}

\subsection{Setup Performance}
\par \textbf{Setup latency.} The setup latency of Trinity-\uppercase\expandafter{\romannumeral 1} and Trinity-\uppercase\expandafter{\romannumeral 2} remains stable and increases linearly, as do the setup latency of GRS-\uppercase\expandafter{\romannumeral 2}, $\mathsf {DSSE}_{\mathsf {SKQ}}$, and SKSE-\uppercase\expandafter{\romannumeral 2}. For instance, as shown in Figure~\ref{fig:5_comp}\subref{fig:5_build}, Trinity-\uppercase\expandafter{\romannumeral 1} takes about 11 seconds to build an EDB for 100,000 data points, which is significantly less than the time taken by Trinity-\uppercase\expandafter{\romannumeral 2}, which takes 20 seconds. Meanwhile, SKSE-\uppercase\expandafter{\romannumeral 2}, GRS-\uppercase\expandafter{\romannumeral 2}, and $\mathsf {DSSE}_{\mathsf {SKQ}}$ currently take 26 seconds, 75 seconds, and 50 seconds respectively to construct an EDB of 100,000 data points. To ensure accuracy, each method was tested 100 times, with the average value serving as the final result.

\par \textbf{EDB size.}  The size of Trinity-\uppercase\expandafter{\romannumeral 1} and Trinity-\uppercase\expandafter{\romannumeral 2} expands proportionally to the number of data points, as does the size of SKSE-\uppercase\expandafter{\romannumeral 2}, GRS-\uppercase\expandafter{\romannumeral 2} and $\mathsf {DSSE}_{\mathsf {SKQ}}$. As illustrated in Figure~\ref{fig:5_comp}\subref{fig:5_EDB}, when handling 100,000 data points, Trinity-\uppercase\expandafter{\romannumeral 1} and $\mathsf {DSSE}_{\mathsf {SKQ}}$ occupy EDB sizes of 1.456 GB and 1.136 GB respectively, which are relatively similar and notably five times smaller than the EDB sizes consumed by GRS-\uppercase\expandafter{\romannumeral 2} (6.395 GB) and SKSE-\uppercase\expandafter{\romannumeral 2} (6.424 GB). However, Trinity-\uppercase\expandafter{\romannumeral 2} boasts a much more compact EDB size of 0.30146 GB due to its efficient roar bitmap architecture, which dramatically reduces storage cost.
\subsection{Search Performance}
\par \textbf{Token generation performance.} In evaluating the token generation cost of Trinity-\uppercase\expandafter{\romannumeral 1} and Trinity-\uppercase\expandafter{\romannumeral 2}, we compare their token generation times with those of Figure~\ref{fig:6_comp}\subref{fig:5_enc} for SKSE-\uppercase\expandafter{\romannumeral 2}, GRS-\uppercase\expandafter{\romannumeral 2}, and $\mathsf {DSSE}_{\mathsf {SKQ}}$. The evaluation tested different numbers of entries, ranging from 10,000 to 100,000 space-time nodes. Trinity-\uppercase\expandafter{\romannumeral 1} and Trinity-\uppercase\expandafter{\romannumeral 2} exhibit minimal differences in token generation cost, with respective times of 19.748 ms and 20.198 ms per token generation.
In terms of encryption methods, SKSE-\uppercase\expandafter{\romannumeral 2} employs HVE, while GRS-\uppercase\expandafter{\romannumeral 2} and $\mathsf {DSSE}_{\mathsf {SKQ}}$ utilize ASHE. Their corresponding token generation times are 92.024 ms, 3658.275 ms, and 2347.0956 ms, respectively. To ensure accuracy, each method was tested 100 times, with the average value serving as the final result.
\begin{figure}[htbp]
\centering{
\subfloat[Token generation latency]{\includegraphics[width=.52\columnwidth]{enc.pdf}\label{fig:5_enc} }
\subfloat[Search latency]{\includegraphics[width=.47\columnwidth]{search.pdf}\label{fig:5_search} }
}
\caption{Trinity-\uppercase\expandafter{\romannumeral 1} vs Trinity-\uppercase\expandafter{\romannumeral 2},SKSE-\uppercase\expandafter{\romannumeral 2},GRS-\uppercase\expandafter{\romannumeral 2} and $\mathsf {DSSE}_{\mathsf {SKQ}}$ search performance}
\label{fig:6_comp} 
\end{figure}

\par \textbf{Search latency.} In evaluating the efficiency of Trinity-\uppercase\expandafter{\romannumeral 1} and Trinity-\uppercase\expandafter{\romannumeral 2}, we compare their search times to those of SKSE-\uppercase\expandafter{\romannumeral 2}, GRS-\uppercase\expandafter{\romannumeral 2}, and $\mathsf {DSSE}_{\mathsf {SKQ}}$, as depicted in Figure~\ref{fig:6_comp}\subref{fig:5_search}. The number of entries, which refers to space-time objects, ranges from 10,000 to 100,000 in increments of 10,000. The search range is set to 100 $m^2$ in 5 minutes. As the number of entries increases, the search time cost per entry in Trinity-\uppercase\expandafter{\romannumeral 2} increases once the number exceeds 10\% of its current capacity. This results in a change from a search time cost per entry of 14.262 ns when there are 10,000 entries to 27.994 ns when there are 100,000 entries.

In contrast, the average search time per entry for Trinity-\uppercase\expandafter{\romannumeral 1}, SKSE-\uppercase\expandafter{\romannumeral 2}, GRS-\uppercase\expandafter{\romannumeral 2}, and $\mathsf {DSSE}_{\mathsf {SKQ}}$ is respectively 10.62 ns, 110.1 ns, 53.162 ns, and 40.1 \textmu s. Consequently, it can be inferred that Trinity-\uppercase\expandafter{\romannumeral 1} outperforms Trinity-\uppercase\expandafter{\romannumeral 2}, being 2.63 times faster when there are 100,000 entries. Additionally, Trinity-\uppercase\expandafter{\romannumeral 2} is also 4.21 times faster than SKSE-\uppercase\expandafter{\romannumeral 2}, 1.9 times faster than GRS-\uppercase\expandafter{\romannumeral 2}, and 1.426 times faster than $\mathsf {DSSE}_{\mathsf {SKQ}}$.
\begin{figure}[htbp]
\centering{

\subfloat[Addition performance]{\includegraphics[width=0.50\columnwidth]{update.pdf}\label{fig:5_update} }
\subfloat[Deletion performance]{\includegraphics[width=0.50\columnwidth]{delete.pdf}\label{fig:5_delete} }
}
\caption{Trinity-\uppercase\expandafter{\romannumeral 1} vs Trinity-\uppercase\expandafter{\romannumeral 2} , GRS-\uppercase\expandafter{\romannumeral 2} and $\mathsf {DSSE}_{\mathsf {SKQ}}$ update performance}
\label{fig:5_com} 
\end{figure}
\subsection{Update Performance}

\par \textbf{Addition performance.}  The addition time for Trinity-\uppercase\expandafter{\romannumeral 1}, Trinity-\uppercase\expandafter{\romannumeral 2}  and $\mathsf {DSSE}_{\mathsf {SKQ}}$ increases linearly, while for GRS-\uppercase\expandafter{\romannumeral 2} the addition time increases as a logarithmic function. However, as shown in Figure~\ref{fig:5_com}\subref{fig:5_update}, Trinity-\uppercase\expandafter{\romannumeral 1}, Trinity-\uppercase\expandafter{\romannumeral 2}  and $\mathsf {DSSE}_{\mathsf {SKQ}}$  at 100,000 points have an addition time of 10.09 ms, 21.59 ms and 282.77 ms respectively, which is negligible compared with the 105,301 ms of GRS-\uppercase\expandafter{\romannumeral 2}. This is because each addition in GRS-\uppercase\expandafter{\romannumeral 2} is also an update of the entire binary tree. Also, SKSE-\uppercase\expandafter{\romannumeral 2} will not be discussed here because it does not support dynamic updates.

\par \textbf{Deletion performance.} Similar to the add operation, the deletion time for Trinity-\uppercase\expandafter{\romannumeral 1}, Trinity-\uppercase\expandafter{\romannumeral 2}  and $\mathsf {DSSE}_{\mathsf {SKQ}}$ increases linearly, while for for GRS-\uppercase \expandafter{\romannumeral 2}, the deletion time increases as logarithmically. As shown in Figure~\ref{fig:5_com}\subref{fig:5_delete}, Trinity-\uppercase\expandafter{\romannumeral 1}, Trinity-\uppercase\expandafter{\romannumeral 2}  and $\mathsf {DSSE}_{\mathsf {SKQ}}$ have deletion times of 10.44 ms, 22.91 ms and 243.53 ms at 100,000 points, respectively, which are negligible compared to the 105,277 ms of GRS-\uppercase\expandafter{\romannumeral 2}. This is because each deletion in GRS-\uppercase\expandafter{\romannumeral 2} is also an update of the entire binary tree. Again, SKSE-\uppercase\expandafter{\romannumeral 2} will not be discussed here because it does not support dynamic updates.

\section{Conclusion}\label{sec:conclusion}

We propose a novel spatio-temporal data DSSE scheme \TrinityI that supports efficient dynamic updates, and automatic scalability. \TrinityI utilizes Hilbert curves and quotient filters to achieve spatio-temporal range query, implementing IND-SCPA security based on SHVE technology. Additionally, \TrinityI can perform millions of data retrievals within just a few milliseconds. Our solution simultaneously addresses the issues of low query efficiency, scalability challenges, and lack of deletion support, making it more suitable for tasks in spatio-temporal scenarios. But \TrinityI is storage expensive and vulnerable to file injection attacks. So we propose a \TrinityII to solve those questions. Our \TrinityII is forward-secure and storage-saving. Our \TrinityII saves 80\% storage cost than \TrinityI, and eliminate false positive by verification. However, the reduced storage cost also results in a smaller capacity for the QF, making hash collisions more likely. Consequently, compared to \TrinityI, \TrinityII exhibits increased query latency and update latency. However, experimental data consistently demonstrate that our proposed solution \TrinityII, outperforms existing solutions in terms of query efficiency, update efficiency, and storage cost. 

\TrinityII only provides forward security to protect against attack from \cite{zhang2016all,cash2015leakage} 
 caused by the add operation. While the leakage caused by the delete operation is not taken into account, and the backward security for the spatio-temporal data security retrieval scheme requires further improvement.
\section*{Acknowledgements}

This study was partially supported by the National Key R\&D Program of China (No.2022YFB4501000), the National Natural Science Foundation of China (No.62232010, 62302266, 62202364, U23A20302, U24A20244), Shandong Science Fund for Excellent Young Scholars (No.2023HWYQ-008), and Shandong Science Fund for Key Fundamental Research Project (ZR2022ZD02), the fellowship of China National Postdoctoral Program for Innovation Talents (No. BX20230279), the China Postdoctoral Science Foundation (No. 2024M752534), and the Key Research and Development Program of Shaanxi (No.2024GX-YBXM-075).

% Can use something like this to put references on a page
% by themselves when using endfloat and the captionsoff option.
\ifCLASSOPTIONcaptionsoff
  \newpage
\fi



% trigger a \newpage just before the given reference
% number - used to balance the columns on the last page
% adjust value as needed - may need to be readjusted if
% the document is modified later
%\IEEEtriggeratref{8}
% The "triggered" command can be changed if desired:
%\IEEEtriggercmd{\enlargethispage{-5in}}

% references section

% can use a bibliography generated by BibTeX as a .bbl file
% BibTeX documentation can be easily obtained at:
% http://mirror.ctan.org/biblio/bibtex/contrib/doc/
% The IEEEtran BibTeX style support page is at:
% http://www.michaelshell.org/tex/ieeetran/bibtex/
%\bibliographystyle{IEEEtran}
% argument is your BibTeX string definitions and bibliography database(s)
%\bibliography{IEEEabrv,../bib/paper}
%
% <OR> manually copy in the resultant .bbl file
% set second argument of \begin to the number of references
% (used to reserve space for the reference number labels box)
%\begin{thebibliography}{1}
%
%\bibitem{IEEEhowto:kopka}
%H.~Kopka and P.~W. Daly, \emph{A Guide to \LaTeX}, 3rd~ed.\hskip 1em plus
%  0.5em minus 0.4em\relax Harlow, England: Addison-Wesley, 1999.
%
%\end{thebibliography}
%\bibliographystyle{IEEEtran}
% argument is your BibTeX string definitions and bibliography database(s)
%\bibliography{mybib}
% Generated by IEEEtran.bst, version: 1.14 (2015/08/26)
\begin{thebibliography}{10}
\providecommand{\url}[1]{#1}
\csname url@samestyle\endcsname
\providecommand{\newblock}{\relax}
\providecommand{\bibinfo}[2]{#2}
\providecommand{\BIBentrySTDinterwordspacing}{\spaceskip=0pt\relax}
\providecommand{\BIBentryALTinterwordstretchfactor}{4}
\providecommand{\BIBentryALTinterwordspacing}{\spaceskip=\fontdimen2\font plus
\BIBentryALTinterwordstretchfactor\fontdimen3\font minus \fontdimen4\font\relax}
\providecommand{\BIBforeignlanguage}[2]{{%
\expandafter\ifx\csname l@#1\endcsname\relax
\typeout{** WARNING: IEEEtran.bst: No hyphenation pattern has been}%
\typeout{** loaded for the language `#1'. Using the pattern for}%
\typeout{** the default language instead.}%
\else
\language=\csname l@#1\endcsname
\fi
#2}}
\providecommand{\BIBdecl}{\relax}
\BIBdecl
\bibitem{stefanov2014practical}
E.~Stefanov, C.~Papamanthou, and E.~Shi, ``Practical dynamic searchable encryption with small leakage.'' in \emph{Proc. ISOC Network and Distributed System Security Symposium (NDSS'14)}, vol.~71, 2014, pp. 72--75.

\bibitem{cao2019protecting}
Y.~Cao, Y.~Xiao, L.~Xiong, L.~Bai, and M.~Yoshikawa, ``Protecting spatiotemporal event privacy in continuous location-based services,'' \emph{IEEE Transactions on Knowledge and Data Engineering}, vol.~33, no.~8, pp. 3141--3154, 2019.

\bibitem{huang2021netr}
X.~Huang, Y.~Gao, X.~Gao, and G.~Chen, ``Netr-tree: An efficient framework for social-based time-aware spatial keyword query,'' in \emph{Proc. IEEE Conference on Web Services (ICWS'21)}.\hskip 1em plus 0.5em minus 0.4em\relax IEEE, 2021, pp. 198--207.

\bibitem{zhu2021privacy}
X.~Zhu, E.~Ayday, and R.~Vitenberg, ``A privacy-preserving framework for outsourcing location-based services to the cloud,'' \emph{IEEE Transactions on Dependable and Secure Computing}, vol.~18, no.~1, pp. 384--399, 2021.

\bibitem{wang2022quickn}
B.~Wang, Y.~Hou, and M.~Li, ``Quickn: Practical and secure nearest neighbor search on encrypted large-scale data,'' \emph{IEEE Transactions on Cloud Computing}, vol.~10, no.~3, pp. 2066--2078, 2022.

\bibitem{guo2022search}
R.~Guo, B.~Qin, Y.~Wu, H.~Chen, and C.~Li, ``Search geometric ranges efficiently as keywords over encrypted spatial data,'' \emph{High-Confidence Computing}, vol.~2, no.~2, p. 100058, 2022.

\bibitem{yang2022lightweight}
Y.~Yang, Y.~Miao, K.-K.~R. Choo, and R.~H. Deng, ``Lightweight privacy-preserving spatial keyword query over encrypted cloud data,'' in \emph{Proc. IEEE Conference on Distributed Computing Systems (ICDCS'22)}.\hskip 1em plus 0.5em minus 0.4em\relax IEEE, 2022, pp. 392--402.

\bibitem{miao2023efficient}
Y.~Miao, Y.~Yang, X.~Li, Z.~Liu, H.~Li, K.-K.~R. Choo, and R.~H. Deng, ``Efficient privacy-preserving spatial range query over outsourced encrypted data,'' \emph{IEEE Transactions on Information Forensics and Security}, 2023.

\bibitem{zhang2016all}
Y.~Zhang, J.~Katz, and C.~Papamanthou, ``All your queries are belong to us: The power of file-injection attacks on searchable encryption,'' in \emph{Proc. {USENIX} Security Symposium ({USENIX} Security'16)}.\hskip 1em plus 0.5em minus 0.4em\relax {USENIX}, 2016, pp. 707--720.

\bibitem{kermanshahi2020geometric}
S.~K. Kermanshahi, S.-F. Sun, J.~K. Liu, R.~Steinfeld, S.~Nepal, W.~F. Lau, and M.~H.~A. Au, ``Geometric range search on encrypted data with forward/backward security,'' \emph{IEEE Transactions on Dependable and Secure Computing}, vol.~19, no.~1, pp. 698--716, 2020.

\bibitem{li2021secure}
J.~Li, J.~Ma, Y.~Miao, F.~Yang, X.~Liu, and K.-K.~R. Choo, ``Secure semantic-aware search over dynamic spatial data in vanets,'' \emph{IEEE Transactions on Vehicular Technology}, vol.~70, no.~9, pp. 8912--8925, 2021.

\bibitem{wang2022forward}
X.~Wang, J.~Ma, X.~Liu, Y.~Miao, Y.~Liu, and R.~H. Deng, ``Forward/backward and content private dsse for spatial keyword queries,'' \emph{IEEE Transactions on Dependable and Secure Computing}, 2022.

\bibitem{li2023enabling}
Z.~Li, J.~Ma, Y.~Miao, X.~Wang, J.~Li, and C.~Xu, ``Enabling efficient privacy-preserving spatio-temporal location-based services for smart cities,'' \emph{IEEE Internet of Things Journal}, 2023.

\bibitem{etemad2018efficient}
M.~Etemad, A.~K{\"u}p{\c{c}}{\"u}, C.~Papamanthou, and D.~Evans, ``Efficient dynamic searchable encryption with forward privacy,'' \emph{Proc. Privacy Enhancing Technologies}, vol.~1, pp. 5--20, 2018.

\bibitem{wang2017fastgeo}
B.~Wang, M.~Li, and L.~Xiong, ``Fastgeo: Efficient geometric range queries on encrypted spatial data,'' \emph{IEEE transactions on dependable and secure computing}, vol.~16, no.~2, pp. 245--258, 2017.

\bibitem{dou2024dynamic}
H.~Dou, Z.~Dan, P.~Xu, W.~Wang, S.~Xu, T.~Chen, and H.~Jin, ``Dynamic searchable symmetric encryption with strong security and robustness,'' \emph{IEEE Transactions on Information Forensics and Security}, 2024.

\bibitem{li2019efficient}
X.~Li, Y.~Zhu, J.~Wang, and J.~Zhang, ``Efficient and secure multi-dimensional geometric range query over encrypted data in cloud,'' \emph{Journal of Parallel and Distributed Computing}, vol. 131, pp. 44--54, 2019.

\bibitem{zheng2020practical}
Z.~Zheng, Z.~Cao, and J.~Shen, ``Practical and secure circular range search on private spatial data,'' in \emph{Proc. IEEE International Conference on Trust, Security and Privacy in Computing and Communications (TrustCom'20)}.\hskip 1em plus 0.5em minus 0.4em\relax IEEE, 2020, pp. 639--645.

\bibitem{wang2014maple}
B.~Wang, Y.~Hou, M.~Li, H.~Wang, and H.~Li, ``Maple: Scalable multi-dimensional range search over encrypted cloud data with tree-based index,'' in \emph{Proc. ACM symposium on Information, computer and communications security (ASIACCS'14)}, 2014, pp. 111--122.

\bibitem{cui2019geo}
N.~Cui, J.~Li, X.~Yang, B.~Wang, M.~Reynolds, and Y.~Xiang, ``When geo-text meets security: privacy-preserving boolean spatial keyword queries,'' in \emph{Proc. IEEE Conference on Data Engineering (ICDE'19)}.\hskip 1em plus 0.5em minus 0.4em\relax IEEE, 2019, pp. 1046--1057.

\bibitem{wang2021enabling}
X.~Wang, J.~Ma, F.~Li, X.~Liu, Y.~Miao, and R.~H. Deng, ``Enabling efficient spatial keyword queries on encrypted data with strong security guarantees,'' \emph{IEEE Transactions on Information Forensics and Security}, vol.~16, pp. 4909--4923, 2021.

\bibitem{zhang2022efficient}
S.~Zhang, S.~Ray, R.~Lu, Y.~Guan, Y.~Zheng, and J.~Shao, ``Efficient and privacy-preserving spatial keyword similarity query over encrypted data,'' \emph{IEEE Transactions on Dependable and Secure Computing}, 2022.

\bibitem{almeida2007scalable}
P.~S. Almeida, C.~Baquero, N.~Pregui{\c{c}}a, and D.~Hutchison, ``Scalable bloom filters,'' \emph{Information Processing Letters}, vol. 101, no.~6, pp. 255--261, 2007.

\bibitem{pandey2021vector}
P.~Pandey, A.~Conway, J.~Durie, M.~A. Bender, M.~Farach-Colton, and R.~Johnson, ``Vector quotient filters: Overcoming the time/space trade-off in filter design,'' in \emph{Proc. ACM SIGMOD International Conference on Management of data (SIGMOD'21)}, 2021, pp. 1386--1399.

\bibitem{mullin1983second}
J.~K. Mullin, ``A second look at bloom filters,'' \emph{Communications of the ACM}, vol.~26, no.~8, pp. 570--571, 1983.

\bibitem{li2022adaptively}
R.~Li and A.~X. Liu, ``Adaptively secure and fast processing of conjunctive queries over encrypted data,'' \emph{IEEE Transactions on Knowledge and Data Engineering}, vol.~34, no.~4, pp. 1588--1602, 2022.

\bibitem{li2023vrfms}
X.~Li, Q.~Tong, J.~Zhao, Y.~Miao, S.~Ma, J.~Weng, J.~Ma, and K.-K.~R. Choo, ``Vrfms: verifiable ranked fuzzy multi-keyword search over encrypted data,'' \emph{IEEE Transactions on Services Computing}, vol.~16, no.~1, pp. 698--710, 2023.

\bibitem{tong2023verifiable}
Q.~Tong, Y.~Miao, J.~Weng, X.~Liu, K.-K.~R. Choo, and R.~H. Deng, ``Verifiable fuzzy multi-keyword search over encrypted data with adaptive security,'' \emph{IEEE Transactions on Knowledge and Data Engineering}, vol.~35, no.~5, pp. 5386--5399, 2023.

\bibitem{fan1998summary}
L.~Fan, P.~Cao, J.~Almeida, and A.~Z. Broder, ``Summary cache: A scalable wide-area web cache sharing protocol,'' \emph{ACM SIGCOMM Computer Communication Review}, vol.~28, no.~4, pp. 254--265, 1998.

\bibitem{cash2015leakage}
D.~Cash, P.~Grubbs, J.~Perry, and T.~Ristenpart, ``Leakage-abuse attacks against searchable encryption,'' in \emph{Proc. ACM SIGSAC conference on computer and communications security (CCS'15)}.\hskip 1em plus 0.5em minus 0.4em\relax ACM, 2015, pp. 668--679.

\bibitem{kim2017forward}
K.~S. Kim, M.~Kim, D.~Lee, J.~H. Park, and W.-H. Kim, ``Forward secure dynamic searchable symmetric encryption with efficient updates,'' in \emph{Proc. ACM SIGSAC Conference on Computer and Communications Security (CCS'16)}.\hskip 1em plus 0.5em minus 0.4em\relax ACM, 2017, pp. 1449--1463.

\bibitem{10621113}
M.~Xu, J.~Zhang, H.~Guo, X.~Cheng, D.~Yu, Q.~Hu, Y.~Li, and Y.~Wu, ``Filedes: A secure, scalable and succinct decentralized encrypted storage network,'' in \emph{IEEE INFOCOM 2024 - IEEE Conference on Computer Communications}, 2024, pp. 261--270.

\bibitem{zuo2020forward}
C.~Zuo, S.~Sun, J.~K. Liu, J.~Shao, J.~Pieprzyk, and L.~Xu, ``Forward and backward private dsse for range queries,'' \emph{IEEE Transactions on Dependable and Secure Computing}, 2020.

\bibitem{bost2016ovarphiovarsigma}
R.~Bost, ``$\sum$o$\varphi$o$\varsigma$: Forward secure searchable encryption,'' in \emph{Proc. ACM SIGSAC Conference on Computer and Communications Security (CCS'16)}.\hskip 1em plus 0.5em minus 0.4em\relax ACM, 2016, pp. 1143--1154.

\bibitem{song2018forward}
X.~Song, C.~Dong, D.~Yuan, Q.~Xu, and M.~Zhao, ``Forward private searchable symmetric encryption with optimized i/o efficiency,'' \emph{IEEE Transactions on Dependable and Secure Computing}, pp. 1--1, 2018.

\bibitem{li2021towards}
F.~Li, J.~Ma, Y.~Miao, L.~Zhiquan, K.-K.~R. Choo, X.~Liu, and R.~Deng, ``Towards efficient verifiable boolean search over encrypted cloud data,'' \emph{IEEE Transactions on Cloud Computing}, 2021.

\bibitem{guo2023forward}
C.~Guo, W.~Li, X.~Tang, K.-K.~R. Choo, and Y.~Liu, ``Forward private verifiable dynamic searchable symmetric encryption with efficient conjunctive query,'' \emph{IEEE Transactions on Dependable and Secure Computing}, 2023.

\bibitem{wang2015circular}
B.~Wang, M.~Li, H.~Wang, and H.~Li, ``Circular range search on encrypted spatial data,'' in \emph{Proc. 2015 IEEE Conference on Communications and Network Security (CNS)}.\hskip 1em plus 0.5em minus 0.4em\relax IEEE, 2015, pp. 182--190.

\bibitem{kerschbaum2014optimal}
F.~Kerschbaum and A.~Schr{\"o}pfer, ``Optimal average-complexity ideal-security order-preserving encryption,'' in \emph{Proc. ACM SIGSAC Conference on Computer and Communications Security (CCS'14)}, 2014, pp. 275--286.

\bibitem{grubbs2017leakage}
P.~Grubbs, K.~Sekniqi, V.~Bindschaedler, M.~Naveed, and T.~Ristenpart, ``Leakage-abuse attacks against order-revealing encryption,'' in \emph{Proc. IEEE Symposium on Security and Privacy (S\&P'17)}.\hskip 1em plus 0.5em minus 0.4em\relax IEEE, 2017, pp. 655--672.

\bibitem{li2019insecurity}
R.~Li, A.~X. Liu, Y.~Liu, H.~Xu, and H.~Yuan, ``Insecurity and hardness of nearest neighbor queries over encrypted data,'' in \emph{Proc. IEEE Conference on Data Engineering (ICDE'19)}, 2019, pp. 1614--1617.

\bibitem{wong2009secure}
W.~K. Wong, D.~W.-l. Cheung, B.~Kao, and N.~Mamoulis, ``Secure knn computation on encrypted databases,'' in \emph{Proc. ACM SIGMOD International Conference on Management of data (SIGMOD'09)}, 2009, pp. 139--152.

\bibitem{hilbert1935stetige}
D.~Hilbert and D.~Hilbert, ``{\"U}ber die stetige abbildung einer linie auf ein fl{\"a}chenst{\"u}ck,'' \emph{Dritter Band: Analysis{\textperiodcentered} Grundlagen der Mathematik{\textperiodcentered} Physik Verschiedenes: Nebst Einer Lebensgeschichte}, pp. 1--2, 1935.

\bibitem{curtmola2006searchable}
R.~Curtmola, J.~Garay, S.~Kamara, and R.~Ostrovsky, ``Searchable symmetric encryption: improved definitions and efficient constructions,'' in \emph{Proc. ACM conference on Computer and communications security (CCS'06)}, 2006, pp. 79--88.

\bibitem{liu2010privacy}
A.~X. Liu and F.~Chen, ``Privacy preserving collaborative enforcement of firewall policies in virtual private networks,'' \emph{IEEE Transactions on Parallel and Distributed Systems}, vol.~22, no.~5, pp. 887--895, 2010.

\bibitem{liang2023privacy}
Y.~Liang, J.~Ma, Y.~Miao, D.~Kuang, X.~Meng, and R.~H. Deng, ``Privacy-preserving bloom filter-based keyword search over large encrypted cloud data,'' \emph{IEEE Transactions on Computers}, vol.~72, no.~11, pp. 3086--3098, 2023.

\bibitem{Chambi2014BetterBP}
S.~Chambi, D.~Lemire, O.~Kaser, and R.~Godin, ``Better bitmap performance with roaring bitmaps,'' \emph{Software: Practice and Experience}, vol.~46, pp. 709 -- 719, 2014.

\bibitem{liang2019urbanfm}
Y.~Liang, K.~Ouyang, L.~Jing, S.~Ruan, Y.~Liu, J.~Zhang, D.~S. Rosenblum, and Y.~Zheng, ``Urbanfm: Inferring fine-grained urban flows,'' in \emph{Proc. ACM SIGKDD international conference on knowledge discovery \& data mining (KDD'19)}, 2019, pp. 3132--3142.

\end{thebibliography}

% 
% If you have an EPS/PDF photo (graphicx package needed) extra braces are
% needed around the contents of the optional argument to biography to prevent
% the LaTeX parser from getting confused when it sees the complicated
% \includegraphics command within an optional argument. (You could create
% your own custom macro containing the \includegraphics command to make things
% simpler here.)
%\begin{IEEEbiography}[{\includegraphics[width=1in,height=1.25in,clip,keepaspectratio]{mshell}}]{Michael Shell}
% or if you just want to reserve a space for a photo:

%\begin{IEEEbiography}{Zhijun Li}
%Biography text here.
%\end{IEEEbiography}
%
%% if you will not have a photo at all:
%\begin{IEEEbiographynophoto}{Jianfeng Ma}
%Biography text here.
%\end{IEEEbiographynophoto}

% insert where needed to balance the two columns on the last page with
% biographies
%\newpage
\begin{IEEEbiography}[{\includegraphics[width=1in,height=1.25in,clip,keepaspectratio]{fig/lzj.jpg}}]{Zhijun Li}
	received the B.E. degree in school of software engineering from Dalian University of Technology, Dalian, China, in 2018, and the Ph.D degree in Cyberspace security from Xidian University, Xi'an, China, in 2024. He is currently a postdoctoral researcher with the School of Computer Science and Technology, Shandong University, China. His current research interests include cloud computing security, edge computing security, and applied cryptography.
\end{IEEEbiography}
\vspace{-4em}
\begin{IEEEbiography}[{\includegraphics[width=1in,height=1.25in,clip,keepaspectratio]{fig/lkz.jpg}}]{Kuizhi Liu} received the B.S. degree from Nankai University, Tianjin, China, in 2021. He is currently a Ph.D. candidate in School of Cyber Engineering, Xidian University. His current research interests include cloud computing security, database security and data applied cryptography.
\end{IEEEbiography}
\vspace{-4em}
\begin{IEEEbiography}[{\includegraphics[width=1in,height=1.25in,clip,keepaspectratio]{fig/xmh.jpg}}]	{Minghui Xu (Member, IEEE)}
 received the BS degree in physics from Beijing Normal University, Beijing, China, in 2018, and the PhD degree in computer science from George Washington University, Washington DC, USA, in 2021. He is currently an associate professor with the School of Computer Science and Technology, Shandong University, China. His research focuses on blockchain, distributed computing, and applied cryptography.
\end{IEEEbiography}
\vspace{-4em}
\begin{IEEEbiography}[{\includegraphics[width=1in,height=1.25in,clip,keepaspectratio]{fig/wxy.png}}]	{Xiangyu Wang}
received the B.E. degree and the Ph.D. degree from Xidian University, Xi'an, China, in 2017 and 2021, respectively. He is currently an associate professor with School of Cyber Engineering, Xidian University, Xi'an, China. He received the Outstanding Doctoral Dissertation Award from the China Institute of Communications in 2022. His research interests include big data security, cloud security, and applied cryptography.
\end{IEEEbiography}
\vspace{-4em}
\begin{IEEEbiography}[{\includegraphics[width=1in,height=1.25in,clip,keepaspectratio]{fig/myb.jpg}}]{Yinbin Miao}
	(Member, IEEE) received the Ph.D. degree from the Department of Telecommunication Engineering, Xidian University, Xi’an, China, in 2016. He was a Post-Doctoral Fellow with Nanyang Technological University, Singapore, from 2018 to 2019. He is currently a Lecturer with the Department of Cyber Engineering, Xidian University. His current research interests include information security and applied cryptography.
\end{IEEEbiography}
\vspace{-4em}
\begin{IEEEbiography}[{\includegraphics[width=1in,height=1.25in,clip,keepaspectratio]{fig/mjf.jpg}}]{Jianfeng Ma}
	(Member, IEEE) received the Ph.D. degree in computer software and telecommunication engineering from Xidian University, Xi’an, China, in 1995. He was a Research Fellow with Nanyang Technological University, Singapore, from 1999 to 2001. He is currently a Professor and a Ph.D. Supervisor with the Department of Cyber Engineering, Xidian University. His current research interests include information and network security, wireless and mobile computing systems, and computer networks.
\end{IEEEbiography}
\vspace{-4em}
\begin{IEEEbiography}[{\includegraphics[width=1in,height=1.25in,clip,keepaspectratio]{fig/xzc.jpg}}]{Xiuzhen Cheng}
	(Fellow, IEEE) Xiuzhen Cheng (Fellow, IEEE) received the MS and PhD degrees in computer science from the University of Minnesota – Twin Cities, in 2000 and 2002, respectively. She is a professor with the School of Computer Science and Technology, Shandong University. Her current research interests include wireless and mobile security, cyber physical systems, wireless and mobile computing, sensor networking, and algorithm design and analysis. She has served on the editorial boards of several technical journals and the technical program committees of various professional conferences/workshops. She also has chaired several international conferences. She worked as a program director for the US National Science Foundation (NSF) from April to October in 2006 (full time), and from April 2008 to May 2010 (part time). She received the NSF CAREER Award in 2004. She is a member of ACM.
\end{IEEEbiography}
\vspace{-4em}

%\begin{IEEEbiography}[{\includegraphics[width=1in,height=1.25in,clip,keepaspectratio]{fig/wdp.jpg}}]{Dapeng Wu} received the M.S. degree in communication and information systems from the Chongqing University of Posts and Telecommunications, Chongqing, China, in 2006, and the Ph.D.degree from the Beijing University of Posts and Telecommunications, Beijing, China, in 2009. His research interests include ubiquitous networks, IP QoS architecture, network reliability, and performance evaluation in communication systems.
%\end{IEEEbiography}
%\begin{IEEEbiographynophoto}{Yinbin Miao}
%Biography text here.
%\end{IEEEbiographynophoto}
%
%\begin{IEEEbiographynophoto}{Ximeng Liu}
%	Biography text here.
%\end{IEEEbiographynophoto}
% You can push biographies down or up by placing
% a \vfill before or after them. The appropriate
% use of \vfill depends on what kind of text is
% on the last page and whether or not the columns
% are being equalized.

%\vfill

% Can be used to pull up biographies so that the bottom of the last one
% is flush with the other column.
%\enlargethispage{-5in}



% that's all folks
\end{document}


