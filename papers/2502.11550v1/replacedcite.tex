\section{Related Work}
In this section, we introduce some related work. To ensure a more impartial and lucid presentation, we list Trinity with previous works in Table~\ref{comparisions}.
\par  \textbf{DSSE.} DSSE is a cryptographic technique that enables clients to securely outsource encrypted databases to servers while still maintaining the ability to perform search and update operations on encrypted data ____. While DSSE provides efficient access to encrypted databases, it inevitably incurs some information leakage, particularly during update operations, where sensitive data can potentially be exposed, thus compromising privacy. The concept of forward security, introduced by Stefanov $et\ al.$ ____, aims to address this issue by ensuring that updates to the encrypted database do not reveal information about previous queries or data. Zhang $et\ al.$ ____ further highlighted the vulnerability of DSSE to leakage attack, specifically file injection tactics, where an attacker strategically inserts a limited number of documents into the encrypted database to deduce search queries. Following this finding, Bost ____ provided a formal definition of forward security and proposed a concrete scheme that incorporates forward security in DSSE. Since then, several forward-secure DSSE schemes have been proposed ____, with a focus on improving security, efficiency, and functional diversity.

%These schemes aim to minimize information leakage during updates while maintaining the ability to efficiently search and update encrypted data. They utilize various techniques to achieve forward security, such as re-encryption mechanisms ____, oblivious RAM ____, multi-set hash functions ____, and puncturable encryption____. The ongoing research in this area is essential for developing secure and practical DSSE solutions that can be deployed in real-world applications, such as cloud storage and data sharing platforms.

 %Striking a balance between efficiency and security has been a central challenge in cryptography. Within LBS systems, there is a need to guarantee not only the absence of significant security loopholes but also the usability of the proposed solutions. However, the aforementioned approaches fall short of meeting these criteria in at least one respect.

\par \textbf{Filter-based DSSE.} 
The common drawback of filter-based search DSSE schemes ____, particularly bloom filter-based schemes, is that they cannot support deletion. These filters are designed to efficiently test membership in a set while allowing for false positive results, meaning they can indicate that an element is present in the set when it is not. This characteristic is inherent in their probabilistic nature, which prioritizes space efficiency over absolute accuracy. Wang $et\ al.$ ____ introduced a pioneering privacy-preserving circular range search scheme; however, the search token size scales with the square of the search radius R, posing scalability challenges. To address these issues, Wang $et\ al.$ ____ ingeniously leveraged the Hilbert curve, effectively reducing queries of the two-dimensional spatial range to one-dimensional searches, thus significantly enhancing retrieval speed. Although this approach represents a notable improvement, it still lacks support for dynamic updates and is limited to a fixed data scale. Similarly, the non-scalable structure employed in ____ requires significant space overhead to maintain a low False Positive Rate (FPR), further highlighting the limitations of these methods.
 
%\par Therefore, the privacy preserving search mechanisms ____ for outsourced spatio-temporal data are essential.  In this situation, Cui $et\ al.$ ____ contributed a privacy-preserving Boolean Range Query (BRQ) solution, yet it was found to be vulnerable to Ciphertext-Only Attacks (COA) ____ due to the use of Asymmetric Scalar Product-Preservation Encryption (ASPE) ____.To address this security issue, Yang $et\ al.$ ____ proposed an enhanced ASPE scheme that achieves Indistinguishability under Chosen Plaintext Attack (IND-CPA). However, their approach compromised on search efficiency. Wang $et\ al.$ ____ then improved search speed by incorporating a Bloom filter hierarchical tree, but their solution was limited to single-user scenarios due to its symmetric key setting and focused solely on spatial queries, neglecting the temporal dimension. Cao $et\ al.$ ____ offered a differential privacy-based scheme to protect spatio-temporal data, but this approach may have impacted the accuracy of the returned results. Huang $et\ al.$ ____ presented a mixed cryptographic primitives-based scheme for privacy-preserving spatio-temporal Location-Based Services (LBS), but its computational costs were high due to the use of RSA and Attribute-Based Encryption. Li $et\ al.$ ____ developed an efficient privacy-preserving spatio-temporal LBS scheme capable of retrieving million-level data in milliseconds. However, the structure of their R tree database made it difficult to modify the size of the encrypted database. Unfortunately, none of the aforementioned privacy-preserving spatio-temporal LBS schemes fully support dynamic updates, which is crucial in real-world LBS applications where data, such as active Instagram users' location information, is constantly being updated.

\textbf{None-filter DSSE.} Balancing efficiency and security has consistently presented a challenge in privacy-preserving LBS ____. Various effective privacy-preserving approaches have been proposed, including the use of Order-Revealing Encryption (ORE) ____, and a variant of Order-Preserving Encryption (OPE) ____. However, recent research ____ has highlighted severe security vulnerabilities in property-preserving encryption schemes such as OPE and ORE. Cui $et\ al.$ ____ contributed a privacy-preserving Boolean Range Query (BRQ) solution, which was later found to be vulnerable to Ciphertext-Only Attacks (COA) ____ due to the use of Asymmetric Scalar Product-Preservation Encryption (ASPE) ____.
To address this security issue, Yang $et\ al.$ ____ proposed an enhanced ASPE scheme that achieves Indistinguishability under Chosen Plaintext Attack (IND-CPA). However, their approach compromised on search efficiency. Kermanshahi ____ proposed a forward-secure DSSE for spatial range query, but the efficiency of their solution is unacceptable. Wang $et\ al.$ ____ then improved search speed by incorporating a Bloom filter hierarchical tree, but their solution was limited to single-user scenarios due to its symmetric key setting and focused solely on spatial queries, neglecting the temporal dimension.  Li $et\ al.$ ____ developed an efficient privacy-preserving spatio-temporal LBS scheme capable of retrieving million-level data in milliseconds.
%Cao $et\ al.$ ____ offered a differential privacy-based scheme to protect spatio-temporal data, though this approach impacted the accuracy of the returned results. Huang $et\ al.$ ____ presented a mixed cryptographic primitives-based scheme for privacy-preserving spatio-temporal Location-Based Services (LBS), but its computational costs were high due to the use of RSA and Attribute-Based Encryption. Although polynomial fitting techniques in conjunction with Attribute-based Searchable Encryption (ASPE) ____ have facilitated efficient single-round trip LBS, ASPE-based solutions have proven insufficiently robust against real-world security threats. Notably, privacy-preserving LBS often necessitate the compromise of disclosing certain information to achieve efficiency trade-offs. Additionally, file injection attacks can exploit Keyword-Pair Result Pattern (KPRP) leaks ____ to uncover all keywords within conjunctive searches.The DSSE scheme based on Oblivious RAM (ORAM) ____ offers resilience against file injection attacks. Nevertheless, ORAM introduces multiple round-trip communications and computational overhead, diminishing its practicality in real-world applications. Homomorphic encryption ____ represents an ideal cryptographic primitive capable of securely facilitating computations followed by comparison operations. However, the practicality of HE still requires enhancement, currently suitable only for small-scale datasets.

%Filter-based retrieval methods have gained popularity due to their quick retrieval capabilities, as evidenced in works such as . However, these methods inherently grapple with drawbacks, including the potential for false positives, difficulties in handling dynamic updates, and constraints imposed by a fixed data capacity.