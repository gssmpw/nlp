
\section{CONCLUSION}

This study examines how robotic failures affect human gaze dynamics and perceptions during collaborative tasks, offering insights into using gaze as a failure indicator to assist in repair. The findings reveal that executional failures lead to more gaze shifts toward the robot, indicating user confusion, while decisional failures result in lower entropy in gaze transitions among areas of interest. 
%Whether the failure occurs at the beginning or the end of the interaction, it can affect both gaze behaviour and users' perception of the robot. 
Failures at the beginning of the interaction lead to more randomness in gaze shifts across AoIs. The timing of the failure during the task also affects users' gaze distribution across AoIs. Finally, acknowledgement of failure does not seem to affect gaze behaviour or users' perception. 
Our work contributes to a better understanding of how gaze behaviour can be leveraged in HRC to design more effective and reliable human-robot interaction systems.