
\subsection{Robot Failures}
We designed the robot to fail during each task in its interaction with the participants. These failures varied based on their type, timing, and whether the robot acknowledged its failure or not.

\subsubsection{\textbf{Types of Failures}}

The types of failures in our experiment represent typical robot malfunctions that may occur during interactions and are commonly reported in HRI. In this research, the robot will simulate two distinct types of failures: 1) Executional failure (EF) and 2) Decisional failure (DF).

%Timing issue
EF can be categorised as a technical failure, specifically timing and ordering \cite{honig_understanding_2018}. In this scenario, the robot pauses for 15 seconds just before picking up an object, while keeping the object within its end effector. After the 15-second pause, the robot will resume and complete the task of picking up and placing the object. This type of failure aligns with previous research \cite{kontogiorgos_embodiment_2020, wachowiak_analysing_2022}.

%Performing the wrong action
DF can also categorized as a technical failure, where the robot performs the correct action incorrectly \cite{honig_understanding_2018}. In this scenario, after picking up an object, the robot will mistakenly move to the location designated for a different object, place it and pause for 5 seconds. While still holding the object, the robot will then lift the object again and place it in the correct location. This type of failure aligns with previous research, in which the robot attempts to perform the correct action but executes it incorrectly \cite{inceoglu_fino-net_2021, stiber_using_2023}.


The procedure for picking up and placing objects during failure events is identical to the procedure when no failure occurs, indicating that the robot shows no signs of committing a failure beforehand. The only distinction in the EF is a pause, which increases the total time for the pick-and-place task by 15 seconds. In the DF, the robot moves its arm to the wrong location, goes down, and comes back up, resulting in an overall increase of 16.5 seconds to the motion.


\subsubsection{\textbf{Timing of Failures}}

The literature suggests that the timing of a failure—whether it occurs at the beginning or the end of an interaction—can affect a person’s perception of the robot differently. 
In this research, we aim to investigate how the timing of a failure impacts both gaze behaviour and user perceptions. Specifically, the robot may fail either at the beginning of the collaboration when placing its first piece, or towards the end of the interaction when placing its third piece.
 
\subsubsection{\textbf{Acknowledgement of Failures}}
A fault confessed is half redressed. Guided by this principle, we explored how the robot’s ability to acknowledge its mistakes influences participants’ perception and gaze patterns in subsequent failures. We designed two distinct scenarios. In one scenario, the robot demonstrates awareness of its mistakes by acknowledging each failure immediately after they occur. After a DF failure, it says, ``Sorry, I made a mistake.'' and after an EF, it says, ``Sorry for the delay.'' In the other scenario, the robot does not declare any of its failures.  In both scenarios, the robot performs physical repairs.
