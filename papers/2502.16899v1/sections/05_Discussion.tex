\section{DISCUSSION}  

% In this work, we compared behavioural responses to robot failures, examining how individuals reacted to these failures and their subsequent perceptions of the robot. The failures in our experiment varied in type, timing, and the robot's acknowledgement of the failure. Our findings indicate differences in both user gaze and perception of the robot, specifically in feelings of anxiety, and perceptions of the robot being skilled and sensible. Based on these results, we will discuss the impact of two failure types, two timings, and two levels of the robot's acknowledgement of the failure on user gaze and perception.

This study compared behavioural responses to robot failures, focusing on how individuals reacted and perceived the robot. Failures varied by type, timing, and acknowledgement. The findings revealed that robot failures affect user gaze and perceptions. These findings are discussed further in the following section.

\subsection{Behavioural Response}

To address the first research question, we analysed user gaze behaviour in multiple ways: the number of gaze shifts, gaze distribution during puzzle-solving, and gaze entropy based on transition matrices. These measures allowed us to examine how the type and timing of failures, as well as whether the robot acknowledged its failure, influenced user gaze patterns and whether gaze behaviour varied across different failure scenarios. Our results showed that user gaze is a reliable indicator of robot failures. When the robot made a failure, participants exhibited more frequent gaze shifts between different AoIs, likely due to confusion and an attempt to understand what was happening. This finding is similar to the results of Kontogiorgos et al. \cite{kontogiorgos_embodiment_2020}, who found that people tend to gaze more at the robot when it makes a mistake. The literature suggests that different types of failures influence user perceptions of the robot \cite{morales_interaction_2019}, and our findings support this by showing that users exhibit distinct gaze behaviours in response to various failure types. For example, when the failure was executional, the number of gaze shifts towards the robot was significantly higher compared to when the failure was decisional. Moreover, during executional failures, the proportion of time spent looking at the robot was much higher compared to decisional failures. It is crucial for the robot to recognize the type of failure it has made so that it can determine the appropriate strategy for recovery and regain the user's trust.


The timing of the failure is also crucial for the robot, as it requires different approaches for recovery and repair. In our research, while the timing of the failure—whether at the start or end of the interaction—did not significantly affect gaze shifts, it did influence gaze transition matrices, and gaze distribution across AoIs. Failures at the beginning of the interaction led to higher median gaze transition values, indicating more randomness early on. Additionally, participants' focus on the Tangram figure was more when the failure occurred at the beginning of the interaction compared to later ones, while their focus on the robot's body or end effector was more during late failures than early ones. 


% \subsubsection{Failure Acknowledgement}

In our research, after committing a failure, the robot could either acknowledge the failure and then continue its action, or proceed without acknowledgement. We could not find significant differences in users' gaze behaviour when the robot acknowledged its failure and when it did not. As the literature suggests \cite{esterwood_you_2021, karli_what_2023, wachowiak_when_2024}, there are other verbal approaches to failure recovery, such as promises and technical explanations, which might influence users' gaze differently. Verbal failure recovery is important for robots, as it demonstrates an awareness of mistakes. This, in turn, can make the robot appear more intelligent and encourage users to engage with it more.


% \subsubsection{Anticipatory Gaze Behavior}
Our study also explored changes in users' anticipatory gaze behaviour during the task and its potential role in assisting the robot to recover from failures. Participants frequently anticipated the placement of the object before the robot executed the action, even when the robot made an error. This anticipatory gaze behaviour could serve as a valuable cue for the robot to detect its failures and initiate appropriate recovery strategies. However, we observed a decrease in participants' anticipatory gaze behaviour as the number of tasks increased. This decline may indicate reduced engagement over time, with participants being more actively collaborative at the beginning of the interaction. It also suggests that users' gaze behaviour might change throughout the interaction. These findings highlight the dynamic nature of gaze behaviour throughout the interaction.


\subsection{Subjective Measures}

To address the second research question, we examined user perceptions of the robot in three areas: perceived intelligence, sense of safety, and trust during failures. The analysis revealed how these measures varied with the type and timing of failure and whether the robot acknowledged its mistake.

The results of the subjective evaluation revealed that users' perceptions of the robot's intelligence and safety were not significantly influenced by the type of failure. However, users exhibited higher levels of trust in the robot during executional failures compared to decisional failures, suggesting that placing an object in an incorrect location reduces trust more than making an incorrect decision. Additionally, we observed interesting findings regarding the timing of the robot's failures. When failures occurred early in the interaction, users rated the robot as more intelligent and trustworthy compared to failures that occurred later. For the measure of "Sensible," this difference was statistically significant. These findings are consistent with previous research by Morales et al. \cite{morales_interaction_2019} and Lucas et al. \cite{lucas_getting_2018}. Interestingly, users reported feeling more relaxed when failures occurred later in the interaction, aligning with results from Desai et al. \cite{desai_impact_2013} and Rossi et al. \cite{rossi_how_2017}.

When the robot acknowledged its failures, users perceived it as slightly more intelligent and trustworthy but also experienced increased anxiety. This finding may be explained by the robot’s consistent physical repair actions a few seconds after each failure. When the robot did not explicitly acknowledge its failures, users might not have interpreted these actions as errors, reducing their perception of failure events.

\subsection{Limitations and Future Work}
 There were instances where participants were preoccupied with determining the placement of their next piece, which occasionally led them to overlook the robot's movements. However, these occurrences were minimal. Another limitation is the restriction to only two types of failure and whether the robot acknowledges its failure or not. The effect size in our study was medium; however, to obtain more robust results, a larger sample size would be beneficial.
 %Additionally, focusing solely on participants' gaze behaviour may not provide a comprehensive measure of failure detection. Incorporating other non-verbal cues, such as gestures or facial expressions, alongside gaze, could improve the accuracy of failure detection. 
 Furthermore, for safety reasons, the robot's arm movement was slowed and the experimenter was in the room, which may have influenced participants' perceptions. 
 Future research could address these limitations by exploring a broader range of failure types and incorporating explanatory feedback from the robot.



