\subsection{Experiment}



 Participants were first briefed by an experimenter on how to interact with the robot and the goals of the tasks. They then completed a demographics questionnaire, providing their age and gender. Following this, participants were seated at a table opposite the robot and asked to wear eye-tracking glasses during the experiment. As per the experimental conditions (Table \ref{table:participant_data}), during each puzzle, the robot correctly picked and placed three pieces but intentionally made an error with one piece. After each puzzle, participants completed a questionnaire assessing their perception of the robot's performance during that specific puzzle. Participants were unaware that these errors were pre-programmed. This cycle was repeated for all four puzzles. The experimental procedure is illustrated in Figure \ref{fig:Experimentdiagram}.

The experimenter initiated the robot's turns and intervened when participants made mistakes by triggering the robot's verbal response. This ensured that the robot began its turn immediately after the participant's turn, maintaining a consistent time gap across all participants. The experimenter was seated on the opposite side of the table, near the robot, to ensure safety and to press the emergency button in case of an actual malfunction. For consistency, the same experimenter conducted all sessions and operated the robot throughout the study.




The experiment was conducted in a laboratory on the University of Melbourne campus. The duration of solving each puzzle together with the robot was about 191.12s ± 35.40s. After each puzzle, the experimenter asked the participant to complete a survey and prepared the table for the next puzzle. The gap between each puzzle was about 93.88s ± 38.81s.


A mixed experimental design was used, with failure types (executional and decisional) and failure timings (early and late) as within-subjects factors, and failure acknowledgement as a between-subjects factor. To minimise order effects, the within-subjects factors were counterbalanced using a four-condition balanced Latin Square. Each factor was systematically integrated into the puzzles. The first thirteen participants experienced the failure acknowledgement, while the second fourteen did not. 



\begin{table}[h!]
\centering
\resizebox{0.48\textwidth}{!}{
\begin{tabular}{ c|c c c c c}

\textbf{Participant ID}  & \textbf{Puzzle 1} & \textbf{Puzzle 2} & \textbf{Puzzle 3} & \textbf{Puzzle 4} & \textbf{Acknowledgment} \\ \hline
1  & EF (Early) & EF (Late) & DF (Late) & DF (Early) & Yes \\ 
2  & EF (Late) & DF (Early) & EF (Early) & DF (Late) & Yes\\ 
3  & DF (Early) & DF (Late) & EF (Late) & EF (Early) & Yes\\ 
4  & DF (Late) & EF (Early) & DF (Early) & EF (Late) & Yes\\
5  & EF (Early) & EF (Late) & DF (Late) & DF (Early) & Yes \\
...  &  ... &  ... & ... & ... &  ... \\ 
14  & EF (Early) & EF (Late) & DF (Late) & DF (Early) & No \\ 
...  &  ... &  ... & ... & ... &  ... \\
\end{tabular}

}
\caption{Order of failure type and timing across puzzles with acknowledgement of failure}
\label{table:participant_data}
\end{table}


