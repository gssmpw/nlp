\section{RESULTS}

%We separate the results into two sections: Behavioural Response and Questionnaire Analysis. 

%In the behaviouraul Response section, we specifically compare the gaze response, focusing on two components. First, we examine gaze behaviour when a failure occurred, considering the effects of failure type, timing, and whether the robot’s declaration influenced the gaze response. Second, we compare gaze behaviour during failure events to non-failure events, also assessing the effect of the robot’s declaration. In the Questionnaire Analysis section, we present an analysis of the ratings from the questionnaires.

\subsection{Behavioural Response}


In this section, we address the first research question by analysing participants' gaze behaviour using the measures outlined in section \ref{gaze_measures}. Our analysis focuses on how these metrics vary during failure situations. Additionally,  we investigate the anticipatory capability of participants' gaze about the placement of objects.

% In this section, we analyse participants' gaze behaviour through multiple approaches. First, we examine the total number of gaze shifts across all AOIs, as well as those specifically directed towards the robot. Second, we evaluate the proportion of gaze directed towards various AOIs during the tasks. Third, we conduct an analysis of gaze transitions using transition matrices. Finally, we investigate the anticipatory capability of participants' gaze about the placement of objects.




\subsubsection{Gaze Shift}

 We conducted a two-way ANOVA to compare gaze patterns during failure versus non-failure robot actions, with having failure type as a within-subjects and the acknowledgement as a between-subjects. For the number of gaze shifts across all AoIs, the results indicated a significant main effect for the factor of failure type ($F(2,48)=15.16$; $p<.001$; $\eta^2=0.39$). Bonferroni-corrected pairwise t-tests revealed significant differences between each type of failure and no failure (NF). For the number of gaze shifts toward the robot’s body (i.e., the robot’s face and torso), the results again showed significant main effects for the factor of failure type ($F(2,48)=21.48$; $p<.001$; $\eta^2=0.47$). Bonferroni-corrected pairwise t-tests indicated significant differences between EF and DF, as well as between EF and NF. Figure \ref{GazeShifts2Conditions} illustrates the average values for each condition.


Subsequently, a three-way ANOVA was conducted to analyse gaze patterns during failure durations, focusing on the effects of failure type and timing as within-subjects factors, and acknowledgement as a between-subjects factor. For the number of gaze shifts across all AoIs, the results showed no significant effects for any of the factors. However, for the number of gaze shifts toward the robot’s body, the results revealed a significant effect of failure type ($F(1,24)=17.780$; $p<.001$; $\eta^2=0.43$). Figure \ref{GazeShifts3Conditions} shows the average values for each condition.

%Next, we conducted a two-way ANOVA to compare gaze patterns during failure versus non-failure robot actions. For the total number of gaze shifts, the results indicated a significant main effect for the factor of failure type ($F(2,48)=15.16$; $p<0.001$; $\eta^2=0.39$). Bonferroni-corrected pairwise tests revealed significant differences between each type of failure and NF. For the number of shifts toward the robot’s body, the results again showed significant main effects for the factor of failure ($F(2,48)=21.48$; $p<0.001$; $\eta^2=0.47$). Bonferroni-corrected pairwise tests indicated significant differences between EF and NF, as well as between EF and DF. Figure \ref{GazeShifts2Conditions} illustrates the average values for each condition, along with their respective SEM.


\begin{figure}[t]
  \centering
  \begin{subfigure}[t]{0.48\linewidth}
    \centering
    \includegraphics[width=\linewidth]{Photos/Presentation_plot1.pdf}

    \label{GazeShiftsAllAoIs}
  \end{subfigure}
  \hfill
  \begin{subfigure}[t]{0.48\linewidth}
    \centering
    \includegraphics[width=\linewidth]{Photos/Presentation_plot2.pdf}

    \label{GazeShiftsRobotBody}
  \end{subfigure}
  \caption{The average number of gaze shifts across all AoIs (left) and toward the robot body (right) across three different failure situations, with or without the robot acknowledging its failure. Error bars represent the standard error of the mean. Significance levels, based on adjusted p-values, are denoted as follows: ** for $p < .01$, and **** for $p < .0001$.}
  \label{GazeShifts2Conditions}
\end{figure}

% \begin{figure}[t]
%   \centering
%   \includegraphics[width=\linewidth]{Photos/combined_plot2.pdf} 
%   \caption{The average number of gaze shifts across all AoIs and the average number of gaze shifts toward the robot body across three different failure situations, with or without the robot acknowledging its failure. Error bars represent the standard error of the mean.}
%   \label{GazeShifts2Conditions}
% \end{figure}



\begin{figure}[t]
  \centering
  \begin{subfigure}[t]{0.48\linewidth}
    \centering
    \includegraphics[width=\linewidth]{Photos/Presentation_plot4.pdf}
  \end{subfigure}
  \hfill
  \begin{subfigure}[t]{0.48\linewidth}
    \centering
    \includegraphics[width=\linewidth]{Photos/Presentation_plot5.pdf}
  \end{subfigure}
  \caption{The average number of gaze shifts across all AoIs and the average number of gaze shifts toward the robot body, comparing failure type, failure timing, and the robot’s acknowledgement of its failure. Error bars represent the standard error of the mean. Significance levels are denoted as follows: *** for $p < .001$.}
  \label{GazeShifts3Conditions}
\end{figure}

% \begin{figure}[t]
%   \centering
%   \includegraphics[width=\linewidth]{Photos/combined_plot.pdf} 
%   \caption{The average number of gaze shifts across all AoIs and the average number of gaze shifts toward the robot body, comparing failure type, failure timing, and the robot’s acknowledgement of its failure. Error bars represent the standard error of the mean.}
%   \label{GazeShifts3Conditions}
% \end{figure}

\begin{comment}
    

\begin{figure}[t]
  \centering
  \includegraphics[width=\linewidth]{Photos/combined_plot5.png} 
  \caption{transition entropy and stationary entropy across three different failure situations and two types of declarations.}
  \label{TSE2conditions}
\end{figure}

\begin{figure}[t!]
  \centering
  \includegraphics[width=\linewidth]{Photos/combined_plot6.png} 
  \caption{Shannon entropy for task and end-effector across three different failure situations and two types of declarations.}
  \label{SETE2conditions}
\end{figure}

\end{comment}


\subsubsection{Gaze Distribution}


In this section, we compare the proportion of gaze directed toward three AoIs: the robot’s end effector, the Tangram figure, and the robot’s body, during each task while the robot is performing its actions.

First, we compare the proportion of gaze directed during failure events to that during non-failure events. The results of the two-way ANOVA revealed significant differences in failure type but no differences in acknowledgement across all measures. Specifically, significant differences were observed for the end effector ($F(2,48)=6.13$; $p = .009$; $\eta^2=0.20$), the Tangram figure ($F(2,48)=17.71$; $p<.001$; $\eta^2=0.42$), and the robot's body ($F(2,48)=14.35$; $p<.001$; $\eta^2=0.37$). Bonferroni-corrected pairwise tests indicated significant differences between EF and NF, as well as between EF and DF for all measures. Figure \ref{proportions2conditions} shows the average values for each measure.

We subsequently conducted a three-way ANOVA with failure type and timing as within-subjects factors, and acknowledgement as a between-subjects factor. The results, as presented in Table \ref{gazedis}, demonstrated significant differences in failure type and timing across the end effector, Tangram figure, and Robot body. Notably, for the Robot body, we also observed significant interactions between acknowledgement and timing.
%Post-hoc analysis showed that when the robot acknowledges its failure, there are differences in timing, and when the timing is early, there are differences in acknowledgement.
%% My attempt: 
Our analysis showed that participants looked at the Tangram figure more when the failure occurred early in the interaction compared to late failures, while they focused more on the robot's body and end effector during late failures than early ones.




\begin{table*}[tbh!]
\centering
\resizebox{0.9\textwidth}{!}{
\begin{tabular}{c||c||c c c c c c c}

Scale & Measure & Type & Timing & Acknowledgement & [Type*Timing] & [Type*Acknowledgement] & [Timing*Acknowledgement] & [Type*Timing*Acknowledgement] \\ \hline
End Effector & df & $$(1,24)$$ & $(1,24)$ & $(1,24)$ & $(1,24)$ & $(1,24)$ & $(1,24)$ & $(1,24)$ \\ 
 & F value & 7.79 & 4.49 & 0.31 & $<$0.001 & 3.85 & 0.55 & 0.75 \\
& p value & \textbf{.010} & \textbf{.447} & .586 & .980 & .061 & .465 & .396 \\ 
& $\eta^2$  & 0.24 & 0.16 & 0.01 & $<$0.0001 & 0.14 & 0.02 & 0.03 \\ \hline
Tangram figure & df & $(1,24)$ & $(1,24)$ & $(1,24)$ & $(1,24)$ & $(1,24)$ & $(1,24)$ & $(1,24)$ \\ 
& F value & 25.86 & 4.85 & 2.02 & 0.03 & 4.09 & 0.80 & 2.09 \\
& p value & \textbf{$<$.001} & \textbf{.038} & .168 & .868 & .054 & .379 & .161 \\
& $\eta^2$  & 0.52 & 0.17 & 0.08 & $<$0.01 & 0.15 & 0.03 & 0.08 \\  \hline
Robot Body & df & $(1,24)$ & $(1,24)$ & $(1,24)$ & $(1,24)$ & $(1,24)$ & $(1,24)$ & $(1,24)$ \\ 
 & F value & 23.01 & 5.29 & 1.20 & 2.80 & 0.46 & 5.71 & 1.34 \\
& p value & \textbf{$<$.001} & \textbf{.030} & .284 & .108 & .502 & \textbf{.025} & .258 \\ 
& $\eta^2$  & 0.49 & 0.18 & 0.05 & 0.10 & 0.02 & 0.19 & 0.05 \\ \hline


\end{tabular}
}
\caption{Results of the three-way mixed ANOVA for gaze distribution}
\label{gazedis}
\end{table*} 



%Figure 4 illustrates the average value for each condition, along with their respective SEM.




%Bonferroni-corrected pairwise tests indicated significant differences between EF and NF conditions, as well as between no-failure and DF conditions for transition entropy. For stationary entropy and Shannon Entropy of the end-effector, significant differences were found between EF and DF conditions, as well as between no-failure and DF conditions. Additionally, for stationary entropy, significant differences were observed between no-failure and DF conditions. The transition matrices are shown in Figure \ref{Transition_matrices}.



\subsubsection{Gaze Transition Matrix}

Based on the AoIs, we created transition matrices, focusing exclusively on transitions between different AoIs while excluding self-repeating transitions. We then compared the transition matrices using transition entropy and stationary entropy.
%, and Shannon Entropy, with a particular focus on the end-effector and task AOIs.

%Based on the AOIs, we designed transition matrices for both components of our behavioural analysis. In this context, we focus solely on transitions between different AOIs, excluding self-repeating transitions. We compare the transition matrices using transition entropy, stationary entropy, and Shannon Entropy specifically for the end-effector and task AOIs. 

We conducted a two-way ANOVA to compare transition matrices during failure versus non-failure robot actions. The results showed significant differences in failure type for both entropies. Specifically, significant differences were observed for transition entropy ($F(2,48)=13.90$; $p < .001$; $\eta^2=0.37$), and stationary entropy ($F(2,48)=11.01$; $p < .001$; $\eta^2=0.31$). No significant differences were found for acknowledgement. Further, Bonferroni-corrected pairwise t-tests indicated significant differences in transition entropy between EF and NF,
%($p.adj < 0.05$)
 and between NF and DF.
%($p.adj < 0.0001$)
For stationary entropy, significant differences were observed between EF and DF,
%($p.adj < 0.001$)
and between NF and DF.
%($p.adj < 0.001$)
The mean values for transition entropy indicate that NF has the highest value, while DF has the lowest. In contrast, for stationary entropy, EF has the highest value, and DF the lowest. In all conditions where the robot acknowledges its failure, both entropy values are lower.  
The transition matrices are shown in Figure \ref{Transition_matrices}.

%We conducted a two-way ANOVA to compare transition matrices during failure versus non-failure robot actions. The results showed significant differences in failure type for all entropies. Specifically, significant differences were observed for transition entropy ($F(2,48)=13.90$; $p < 0.001$; $\eta^2=0.37$), stationary entropy ($F(2,48)=11.01$; $p < 0.001$; $\eta^2=0.31$), Shannon Entropy for the end-effector ($F(2,48)=17.71$; $p < 0.001$; $\eta^2=0.42$), and Shannon Entropy for the task ($F(2,48)=14.35$; $p < 0.001$; $\eta^2=0.37$). No significant differences were found for declaration.

%Bonferroni-corrected pairwise t-tests revealed significant differences in various entropy measures across different conditions. For transition entropy, significant differences were found between EF and NF ($p.adj < 0.05$), as well as between NF and DF ($p.adj < 0.0001$). In the case of stationary entropy, significant differences were observed between EF and DF ($p.adj < 0.001$), and between NF and DF ($p.adj < 0.001$). For Shannon Entropy related to the end-effector, significant differences emerged between EF and DF ($p.adj < 0.05$) and between NF and DF ($p.adj < 0.05$). Additionally, for Shannon Entropy related to the task, a significant difference was detected between NF and DF ($p.adj < 0.0001$).





Subsequently, we performed the Wilcoxon Signed-Rank Test for failure type and timing, and the Wilcoxon Rank-Sum Test for acknowledgement. The results (Table \ref{TranstionEntropy}) indicate significant differences in failure type for both entropies. Additionally, significant differences in timing were observed for stationary entropy. No significant differences were found for acknowledgement. The median values show that when the failure type is DF, the timing is late, or the robot acknowledges its failure, both entropy values are lower.



\begin{figure}[t!]
  \centering
  \includegraphics[width=\linewidth]{Photos/Presentation_plot3.pdf} 
  \caption{The average proportion of participant gazes directed at the end effector (EE), robot body/face (RB), and Tangram figure (T) during puzzle solving, across three different failure situations, with or without the robot acknowledging its failure. Error bars represent the standard error of the mean. Significance levels, based on adjusted p-values, are denoted as follows: * for $p < .05$, ** for $p < .01$, and *** for $p < .001$.}
  \label{proportions2conditions}
\end{figure}



\begin{table}[!h]
\centering
\resizebox{0.45\textwidth}{!}{
\begin{tabular}{c||c||c c c}

Scale & Measure & Type & Timing & Acknowledgement \\ \hline
Transition Entropy & N & 52 & 52 & 52 \\ 
 & W & 1007 & 864 & 1583 \\ 
& p-value & \textbf{.004} & .112 & .134  \\ 
 \hline
Stationary Entropy & N & 52 & 52 & 52 \\ 
 & W & 1197 & 1035 & 1522 \\ 
& p-value & \textbf{$<$.001} & \textbf{.002} & .270 \\ 
 \hline

\end{tabular}
}

\caption{Results of the Wilcoxon tests for the entropy of the transition matrices, where N represents the sample size for each condition and W is the test statistic.}
\label{TranstionEntropy}
\end{table}


\begin{figure*}[t!]
  \centering
  \includegraphics[width=0.9\linewidth]{Photos/heatmaps_decnodec.pdf} 
  \caption{The transition matrices for three different interaction scenarios—NF, EF, and DF—are presented, both for cases where the robot acknowledges its failure and where it does not. The vertical axis represents the current states, while the horizontal axis represents the next states. 'EE' stands for End Effector, 'T' for Tangram figure, 'RB' for Robot Body, 'Exp.' for Experimenter, 'PR' for Pieces (Robot), and 'PP' for Pieces (Participant). The transition matrices are displayed as heat maps.}
  \label{Transition_matrices}
\end{figure*}







\subsubsection{Goal Anticipation by Gaze Analysis}


In this section, we examine the proportion of time participants spent looking at the correct goal location for object placement within the Tangram figure, compared to the total time spent looking inside the Tangram figure. The purpose of this analysis is to determine whether participants exhibited anticipatory gaze behaviour to assist the robot in recovering from its failures. Specifically, we assessed the average percentage of time participants looked at the goal, as well as the average number of gaze shifts towards the goal during each puzzle and failure period.
The duration considered for each puzzle spanned from the moment the robot's end effector was positioned above the object it intended to pick up until it was positioned above the designated placement location for that object. During failure periods, we focused on the time from when the robot initiated a failure until it began its repair.

The results indicate that as participants progressed through the puzzles, the proportion of time spent looking at the goal decreased. Specifically, in Puzzle 1, participants looked at the goal for 47\% ($\pm$14\%) of the time, followed by 44\% ($\pm$13\%) in Puzzle 2, 38\% ($\pm$16\%) in Puzzle 3, and 22\% ($\pm$9\%) in Puzzle 4. Additionally, the total number of gaze shifts towards the goal also almost decreased as participants advanced through the puzzles. The average number of gaze shifts per piece was 10.33 ($\pm$5.35) in Puzzle 1, 10.65 ($\pm$4.57) in Puzzle 2, 7.66 ($\pm$4.79) in Puzzle 3, and 5.47 ($\pm$3.18) in Puzzle 4.

% When analysing the failure periods, the results show that when it was an executional failure (EF), participants spent 35\% (±26\%) of their task-related gaze time looking at the goal. This percentage was higher -- 41\% (±15\%), when it was a decisional failure (DF). Participants exhibited an average of 3.42 (±3.36) gaze shifts for EF per puzzle and substantially higher average of gaze shifts for DF ( 14.84, ±7.11).

When analysing the failure periods, the results show that participants spent 35\% ($\pm$26\%) of their task-related gaze time looking at the goal during EF. This percentage was higher, at 41\% ($\pm$15\%), during DF. Additionally, participants exhibited an average of 3.42 ($\pm3.36$) gaze shifts for each EF, compared to a substantially higher average of 14.84 ($\pm7.11$) gaze shifts for each DF.



\subsection{Subjective Measures}
In this section, we address the second research question by analysing participants' subjective behaviour using the measures outlined in section \ref{Subjective_Measures}. To achieve this, we conducted a three-way ANOVA for each subjective scale to examine the effects of failure type, timing, and acknowledgement. For the Competent scale, significant interaction effects were found for type*timing ($F$(1,24)$ = 6.79$, $p = .016$, $\eta^2 = 0.22$). The Sensible scale showed a significant main effect of timing ($F$(1,24)$ = 5.79$, $p = .024$, $\eta^2 = 0.19$). In the Anxious/Relaxed (Self) scale, significant main effects were observed for timing ($F$(1,24)$ = 7.80$, $p = .010$, $\eta^2 = 0.24$) and acknowledgement ($F$(1,24)$ = 5.50$, $p = .027$, $\eta^2 = 0.18$). The Predictable scale had a significant interaction effect for type*acknowledgement ($F$(1,25)$ = 5.38$, $p = .029$, $\eta^2 = 0.18$). The Skilled scale showed a significant main effect of type ($F$(1,25)$ = 4.98$, $p = .035$, $\eta^2 = 0.17$). Finally, the Capable scale revealed a significant three-way interaction of type*timing*acknowledgement ($F$(1,25)$ = 6.99$, $p = .014$, $\eta^2 = 0.22$). The results showed that participants rated the robot higher on measures of perceived intelligence and trust in the questionnaire when the failure was executional, occurred early, or when the robot acknowledged its failure. However, for feelings of safety, ratings were higher when the failure occurred late and the robot did not acknowledge it. More information can be found in the Appendix.

