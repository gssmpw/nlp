\subsection{Participants}

\begin{figure*}[t!]
  \centering
  \includegraphics[width=0.8\linewidth]{Photos/The_experimental_flow.pdf} 
  \caption{Experimental diagram showing the process where participants first complete a pre-task questionnaire, followed by collaboratively solving a Tangram puzzle four times.}
  \label{fig:Experimentdiagram}
\end{figure*}


We conducted \textit{a priori} power analysis to calculate the sample size for our experiment using \textit{G*Power}~\cite{faul_gpower_2007}. The calculation was based on a medium effect size of $f=0.25$, an alpha level of 0.05, and a power of 0.8. As a result, we determined a minimum of 24 participants was required; however, we recruited 27 participants (16 females, 10 males, and 1 non-binary) via our university recruitment website. Participants primarily consisted of students and university staff, none of whom had previous experience working with robots. They were compensated with a gift voucher for their participation. The average age was ranging between 18y.o. and 34y.o.  (M = 23.26, SD = 4.3). Due to some technical issue, one participant's gaze data was not recorded, and another participant did not complete some questions in one of the after-task questionnaires. Participants signed a consent form before participation and were made aware that their gaze data was recorded during the experiment. At the end of the experiment, participants were informed that the study aimed to observe their responses to the robot's failures.