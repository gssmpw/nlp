\documentclass[conference]{IEEEtran}
\IEEEoverridecommandlockouts
% The preceding line is only needed to identify funding in the first footnote. If that is unneeded, please comment it out.
\usepackage{cite}
\usepackage{amsmath,amssymb,amsfonts}
\usepackage{algorithmic}
\usepackage{graphicx}
\usepackage{textcomp}
\usepackage{xcolor}
\usepackage{comment}
\usepackage{hyperref}
\usepackage{threeparttable}
\usepackage[font=footnotesize]{caption}
\usepackage{afterpage}
\usepackage{float}
\usepackage{balance}
\usepackage{subcaption} % Add this in your preamble if not already included

\usepackage[moderate,tracking=normal]{savetrees}
%\usepackage[subtle,tracking=normal]{savetrees}


 
\def\BibTeX{{\rm B\kern-.05em{\sc i\kern-.025em b}\kern-.08em
    T\kern-.1667em\lower.7ex\hbox{E}\kern-.125emX}}
\begin{document}

%\title{Investigating Human Gaze Dynamics in Response to Robotic Failure in Human-Robot Collaboration
%}
\title{Gazing at Failure: Investigating Human Gaze in Response to Robot Failure in Collaborative Tasks
}


    
\author{\IEEEauthorblockN{Ramtin Tabatabaei}
\IEEEauthorblockA{\textit{The University of Melbourne} \\
Melbourne, Australia \\
stabatabaeim@student.unimelb.edu.au}
\and

\IEEEauthorblockN{Vassilis Kostakos}
\IEEEauthorblockA{\textit{The University of Melbourne} \\
Melbourne, Australia \\
vassilis.kostakos@unimelb.edu.au}
\and

\IEEEauthorblockN{Wafa Johal}
\IEEEauthorblockA{\textit{The University of Melbourne} \\
Melbourne, Australia \\
wafa.johal@unimelb.edu.au}}



% \author{\IEEEauthorblockN{Anonymous Author(s)}
% }



\maketitle

\begin{abstract}
%Robots are prone to making errors during collaborative tasks with human users, which can negatively impact user perception of robot. 
Robots are prone to making errors, which can negatively impact their credibility as teammates during collaborative tasks with human users.
Detecting and recovering from these failures is crucial for maintaining effective level of trust from users. However, robots may fail without being aware of it. One way to detect such failures could be by analysing humans' non-verbal behaviours and reactions to failures. This study investigates how human gaze dynamics can signal a robot's failure and examines how different types of failures affect people's perception of robot. We conducted a user study with 27 participants collaborating with a robotic mobile manipulator to solve tangram puzzles. The robot was programmed to experience two types of failures ---executional and decisional--- occurring either at the beginning or end of the task, with or without acknowledgement of the failure. Our findings reveal that the type and timing of the robot's failure significantly affect participants' gaze behaviour and perception of the robot. Specifically, executional failures led to more gaze shifts and increased focus on the robot, while decisional failures resulted in lower entropy in gaze transitions among areas of interest, particularly when the failure occurred at the end of the task.
These results highlight that gaze can serve as a reliable indicator of robot failures and their types, and could also be used to predict the appropriate recovery actions.

\end{abstract}


\begin{IEEEkeywords}
Robot Failures, Gaze Dynamics, Human-Robot Collaboration
\end{IEEEkeywords}


\section{Introduction}

% \textcolor{red}{Still on working}

% \textcolor{red}{add label for each section}


Robot learning relies on diverse and high-quality data to learn complex behaviors \cite{aldaco2024aloha, wang2024dexcap}.
Recent studies highlight that models trained on datasets with greater complexity and variation in the domain tend to generalize more effectively across broader scenarios \cite{mann2020language, radford2021learning, gao2024efficient}.
% However, creating such diverse datasets in the real world presents significant challenges.
% Modifying physical environments and adjusting robot hardware settings require considerable time, effort, and financial resources.
% In contrast, simulation environments offer a flexible and efficient alternative.
% Simulations allow for the creation and modification of digital environments with a wide range of object shapes, weights, materials, lighting, textures, friction coefficients, and so on to incorporate domain randomization,
% which helps improve the robustness of models when deployed in real-world conditions.
% These environments can be easily adjusted and reset, enabling faster iterations and data collection.
% Additionally, simulations provide the ability to consistently reproduce scenarios, which is essential for benchmarking and model evaluation.
% Another advantage of simulations is their flexibility in sensor integration. Sensors such as cameras, LiDARs, and tactile sensors can be added or repositioned without the physical limitations present in real-world setups. Simulations also eliminate the risk of damaging expensive hardware during edge-case experiments, making them an ideal platform for testing rare or dangerous scenarios that are impractical to explore in real life.
By leveraging immersive perspectives and interactions, Extended Reality\footnote{Extended Reality is an umbrella term to refer to Augmented Reality, Mixed Reality, and Virtual Reality \cite{wikipediaExtendedReality}}
(XR)
is a promising candidate for efficient and intuitive large scale data collection \cite{jiang2024comprehensive, arcade}
% With the demand for collecting data, XR provides a promising approach for humans to teach robots by offering users an immersive experience.
in simulation \cite{jiang2024comprehensive, arcade, dexhub-park} and real-world scenarios \cite{openteach, opentelevision}.
However, reusing and reproducing current XR approaches for robot data collection for new settings and scenarios is complicated and requires significant effort.
% are difficult to reuse and reproduce system makes it hard to reuse and reproduce in another data collection pipeline.
This bottleneck arises from three main limitations of current XR data collection and interaction frameworks: \textit{asset limitation}, \textit{simulator limitation}, and \textit{device limitation}.
% \textcolor{red}{ASSIGN THESE CITATION PROPERLY:}
% \textcolor{red}{list them by time order???}
% of collecting data by using XR have three main limitations.
Current approaches suffering from \textit{asset limitation} \cite{arclfd, jiang2024comprehensive, arcade, george2025openvr, vicarios}
% Firstly, recent works \cite{jiang2024comprehensive, arcade, dexhub-park}
can only use predefined robot models and task scenes. Configuring new tasks requires significant effort, since each new object or model must be specifically integrated into the XR application.
% and it takes too much effort to configure new tasks in their systems since they cannot spawn arbitrary models in the XR application.
The vast majority of application are developed for specific simulators or real-world scenarios. This \textit{simulator limitation} \cite{mosbach2022accelerating, lipton2017baxter, dexhub-park, arcade}
% Secondly, existing systems are limited to a single simulation platform or real-world scenarios.
significantly reduces reusability and makes adaptation to new simulation platforms challenging.
Additionally, most current XR frameworks are designed for a specific version of a single XR headset, leading to a \textit{device limitation} 
\cite{lipton2017baxter, armada, openteach, meng2023virtual}.
% and there is no work working on the extendability of transferring to a new headsets as far as we know.
To the best of our knowledge, no existing work has explored the extensibility or transferability of their framework to different headsets.
These limitations hamper reproducibility and broader contributions of XR based data collection and interaction to the research community.
% as each research group typically has its own data collection pipeline.
% In addition to these main limitations, existing XR systems are not well suited for managing multiple robot systems,
% as they are often designed for single-operator use.

In addition to these main limitations, existing XR systems are often designed for single-operator use, prohibiting collaborative data collection.
At the same time, controlling multiple robots at once can be very difficult for a single operator,
making data collection in multi-robot scenarios particularly challenging \cite{orun2019effect}.
Although there are some works using collaborative data collection in the context of tele-operation \cite{tung2021learning, Qin2023AnyTeleopAG},
there is no XR-based data collection system supporting collaborative data collection.
This limitation highlights the need for more advanced XR solutions that can better support multi-robot and multi-user scenarios.
% \textcolor{red}{more papers about collaborative data collection}

To address all of these issues, we propose \textbf{IRIS},
an \textbf{I}mmersive \textbf{R}obot \textbf{I}nteraction \textbf{S}ystem.
This general system supports various simulators, benchmarks and real-world scenarios.
It is easily extensible to new simulators and XR headsets.
IRIS achieves generalization across six dimensions:
% \begin{itemize}
%     \item \textit{Cross-scene} : diverse object models;
%     \item \textit{Cross-embodiment}: diverse robot models;
%     \item \textit{Cross-simulator}: 
%     \item \textit{Cross-reality}: fd
%     \item \textit{Cross-platform}: fd
%     \item \textit{Cross-users}: fd
% \end{itemize}
\textbf{Cross-Scene}, \textbf{Cross-Embodiment}, \textbf{Cross-Simulator}, \textbf{Cross-Reality}, \textbf{Cross-Platform}, and \textbf{Cross-User}.

\textbf{Cross-Scene} and \textbf{Cross-Embodiment} allow the system to handle arbitrary objects and robots in the simulation,
eliminating restrictions about predefined models in XR applications.
IRIS achieves these generalizations by introducing a unified scene specification, representing all objects,
including robots, as data structures with meshes, materials, and textures.
The unified scene specification is transmitted to the XR application to create and visualize an identical scene.
By treating robots as standard objects, the system simplifies XR integration,
allowing researchers to work with various robots without special robot-specific configurations.
\textbf{Cross-Simulator} ensures compatibility with various simulation engines.
IRIS simplifies adaptation by parsing simulated scenes into the unified scene specification, eliminating the need for XR application modifications when switching simulators.
New simulators can be integrated by creating a parser to convert their scenes into the unified format.
This flexibility is demonstrated by IRIS’ support for Mujoco \cite{todorov2012mujoco}, IsaacSim \cite{mittal2023orbit}, CoppeliaSim \cite{coppeliaSim}, and even the recent Genesis \cite{Genesis} simulator.
\textbf{Cross-Reality} enables the system to function seamlessly in both virtual simulations and real-world applications.
IRIS enables real-world data collection through camera-based point cloud visualization.
\textbf{Cross-Platform} allows for compatibility across various XR devices.
Since XR device APIs differ significantly, making a single codebase impractical, IRIS XR application decouples its modules to maximize code reuse.
This application, developed by Unity \cite{unity3dUnityManual}, separates scene visualization and interaction, allowing developers to integrate new headsets by reusing the visualization code and only implementing input handling for hand, head, and motion controller tracking.
IRIS provides an implementation of the XR application in the Unity framework, allowing for a straightforward deployment to any device that supports Unity. 
So far, IRIS was successfully deployed to the Meta Quest 3 and HoloLens 2.
Finally, the \textbf{Cross-User} ability allows multiple users to interact within a shared scene.
IRIS achieves this ability by introducing a protocol to establish the communication between multiple XR headsets and the simulation or real-world scenarios.
Additionally, IRIS leverages spatial anchors to support the alignment of virtual scenes from all deployed XR headsets.
% To make an seamless user experience for robot learning data collection,
% IRIS also tested in three different robot control interface
% Furthermore, to demonstrate the extensibility of our approach, we have implemented a robot-world pipeline for real robot data collection, ensuring that the system can be used in both simulated and real-world environments.
The Immersive Robot Interaction System makes the following contributions\\
\textbf{(1) A unified scene specification} that is compatible with multiple robot simulators. It enables various XR headsets to visualize and interact with simulated objects and robots, providing an immersive experience while ensuring straightforward reusability and reproducibility.\\
\textbf{(2) A collaborative data collection framework} designed for XR environments. The framework facilitates enhanced robot data acquisition.\\
\textbf{(3) A user study} demonstrating that IRIS significantly improves data collection efficiency and intuitiveness compared to the LIBERO baseline.

% \begin{table*}[t]
%     \centering
%     \begin{tabular}{lccccccc}
%         \toprule
%         & \makecell{Physical\\Interaction}
%         & \makecell{XR\\Enabled}
%         & \makecell{Free\\View}
%         & \makecell{Multiple\\Robots}
%         & \makecell{Robot\\Control}
%         % Force Feedback???
%         & \makecell{Soft Object\\Supported}
%         & \makecell{Collaborative\\Data} \\
%         \midrule
%         ARC-LfD \cite{arclfd}                              & Real        & \cmark & \xmark & \xmark & Joint              & \xmark & \xmark \\
%         DART \cite{dexhub-park}                            & Sim         & \cmark & \cmark & \cmark & Cartesian          & \xmark & \xmark \\
%         \citet{jiang2024comprehensive}                     & Sim         & \cmark & \xmark & \xmark & Joint \& Cartesian & \xmark & \xmark \\
%         \citet{mosbach2022accelerating}                    & Sim         & \cmark & \cmark & \xmark & Cartesian          & \xmark & \xmark \\
%         ARCADE \cite{arcade}                               & Real        & \cmark & \cmark & \xmark & Cartesian          & \xmark & \xmark \\
%         Holo-Dex \cite{holodex}                            & Real        & \cmark & \xmark & \cmark & Cartesian          & \cmark & \xmark \\
%         ARMADA \cite{armada}                               & Real        & \cmark & \xmark & \cmark & Cartesian          & \cmark & \xmark \\
%         Open-TeleVision \cite{opentelevision}              & Real        & \cmark & \cmark & \cmark & Cartesian          & \cmark & \xmark \\
%         OPEN TEACH \cite{openteach}                        & Real        & \cmark & \xmark & \cmark & Cartesian          & \cmark & \cmark \\
%         GELLO \cite{wu2023gello}                           & Real        & \xmark & \cmark & \cmark & Joint              & \cmark & \xmark \\
%         DexCap \cite{wang2024dexcap}                       & Real        & \xmark & \cmark & \xmark & Cartesian          & \cmark & \xmark \\
%         AnyTeleop \cite{Qin2023AnyTeleopAG}                & Real        & \xmark & \xmark & \cmark & Cartesian          & \cmark & \cmark \\
%         Vicarios \cite{vicarios}                           & Real        & \cmark & \xmark & \xmark & Cartesian          & \cmark & \xmark \\     
%         Augmented Visual Cues \cite{augmentedvisualcues}   & Real        & \cmark & \cmark & \xmark & Cartesian          & \xmark & \xmark \\ 
%         \citet{wang2024robotic}                            & Real        & \cmark & \cmark & \xmark & Cartesian          & \cmark & \xmark \\
%         Bunny-VisionPro \cite{bunnyvisionpro}              & Real        & \cmark & \cmark & \cmark & Cartesian          & \cmark & \xmark \\
%         IMMERTWIN \cite{immertwin}                         & Real        & \cmark & \cmark & \cmark & Cartesian          & \xmark & \xmark \\
%         \citet{meng2023virtual}                            & Sim \& Real & \cmark & \cmark & \xmark & Cartesian          & \xmark & \xmark \\
%         Shared Control Framework \cite{sharedctlframework} & Real        & \cmark & \cmark & \cmark & Cartesian          & \xmark & \xmark \\
%         OpenVR \cite{openvr}                               & Real        & \cmark & \cmark & \xmark & Cartesian          & \xmark & \xmark \\
%         \citet{digitaltwinmr}                              & Real        & \cmark & \cmark & \xmark & Cartesian          & \cmark & \xmark \\
        
%         \midrule
%         \textbf{Ours} & Sim \& Real & \cmark & \cmark & \cmark & Joint \& Cartesian  & \cmark & \cmark \\
%         \bottomrule
%     \end{tabular}
%     \caption{This is a cross-column table with automatic line breaking.}
%     \label{tab:cross-column}
% \end{table*}

% \begin{table*}[t]
%     \centering
%     \begin{tabular}{lccccccc}
%         \toprule
%         & \makecell{Cross-Embodiment}
%         & \makecell{Cross-Scene}
%         & \makecell{Cross-Simulator}
%         & \makecell{Cross-Reality}
%         & \makecell{Cross-Platform}
%         & \makecell{Cross-User} \\
%         \midrule
%         ARC-LfD \cite{arclfd}                              & \xmark & \xmark & \xmark & \xmark & \xmark & \xmark \\
%         DART \cite{dexhub-park}                            & \cmark & \cmark & \xmark & \xmark & \xmark & \xmark \\
%         \citet{jiang2024comprehensive}                     & \xmark & \cmark & \xmark & \xmark & \xmark & \xmark \\
%         \citet{mosbach2022accelerating}                    & \xmark & \cmark & \xmark & \xmark & \xmark & \xmark \\
%         ARCADE \cite{arcade}                               & \xmark & \xmark & \xmark & \xmark & \xmark & \xmark \\
%         Holo-Dex \cite{holodex}                            & \cmark & \xmark & \xmark & \xmark & \xmark & \xmark \\
%         ARMADA \cite{armada}                               & \cmark & \xmark & \xmark & \xmark & \xmark & \xmark \\
%         Open-TeleVision \cite{opentelevision}              & \cmark & \xmark & \xmark & \xmark & \cmark & \xmark \\
%         OPEN TEACH \cite{openteach}                        & \cmark & \xmark & \xmark & \xmark & \xmark & \cmark \\
%         GELLO \cite{wu2023gello}                           & \cmark & \xmark & \xmark & \xmark & \xmark & \xmark \\
%         DexCap \cite{wang2024dexcap}                       & \xmark & \xmark & \xmark & \xmark & \xmark & \xmark \\
%         AnyTeleop \cite{Qin2023AnyTeleopAG}                & \cmark & \cmark & \cmark & \cmark & \xmark & \cmark \\
%         Vicarios \cite{vicarios}                           & \xmark & \xmark & \xmark & \xmark & \xmark & \xmark \\     
%         Augmented Visual Cues \cite{augmentedvisualcues}   & \xmark & \xmark & \xmark & \xmark & \xmark & \xmark \\ 
%         \citet{wang2024robotic}                            & \xmark & \xmark & \xmark & \xmark & \xmark & \xmark \\
%         Bunny-VisionPro \cite{bunnyvisionpro}              & \cmark & \xmark & \xmark & \xmark & \xmark & \xmark \\
%         IMMERTWIN \cite{immertwin}                         & \cmark & \xmark & \xmark & \xmark & \xmark & \xmark \\
%         \citet{meng2023virtual}                            & \xmark & \cmark & \xmark & \cmark & \xmark & \xmark \\
%         \citet{sharedctlframework}                         & \cmark & \xmark & \xmark & \xmark & \xmark & \xmark \\
%         OpenVR \cite{george2025openvr}                               & \xmark & \xmark & \xmark & \xmark & \xmark & \xmark \\
%         \citet{digitaltwinmr}                              & \xmark & \xmark & \xmark & \xmark & \xmark & \xmark \\
        
%         \midrule
%         \textbf{Ours} & \cmark & \cmark & \cmark & \cmark & \cmark & \cmark \\
%         \bottomrule
%     \end{tabular}
%     \caption{This is a cross-column table with automatic line breaking.}
% \end{table*}

% \begin{table*}[t]
%     \centering
%     \begin{tabular}{lccccccc}
%         \toprule
%         & \makecell{Cross-Scene}
%         & \makecell{Cross-Embodiment}
%         & \makecell{Cross-Simulator}
%         & \makecell{Cross-Reality}
%         & \makecell{Cross-Platform}
%         & \makecell{Cross-User}
%         & \makecell{Control Space} \\
%         \midrule
%         % Vicarios \cite{vicarios}                           & \xmark & \xmark & \xmark & \xmark & \xmark & \xmark \\     
%         % Augmented Visual Cues \cite{augmentedvisualcues}   & \xmark & \xmark & \xmark & \xmark & \xmark & \xmark \\ 
%         % OpenVR \cite{george2025openvr}                     & \xmark & \xmark & \xmark & \xmark & \xmark & \xmark \\
%         \citet{digitaltwinmr}                              & \xmark & \xmark & \xmark & \xmark & \xmark & \xmark &  \\
%         ARC-LfD \cite{arclfd}                              & \xmark & \xmark & \xmark & \xmark & \xmark & \xmark &  \\
%         \citet{sharedctlframework}                         & \cmark & \xmark & \xmark & \xmark & \xmark & \xmark &  \\
%         \citet{jiang2024comprehensive}                     & \cmark & \xmark & \xmark & \xmark & \xmark & \xmark &  \\
%         \citet{mosbach2022accelerating}                    & \cmark & \xmark & \xmark & \xmark & \xmark & \xmark & \\
%         Holo-Dex \cite{holodex}                            & \cmark & \xmark & \xmark & \xmark & \xmark & \xmark & \\
%         ARCADE \cite{arcade}                               & \cmark & \cmark & \xmark & \xmark & \xmark & \xmark & \\
%         DART \cite{dexhub-park}                            & Limited & Limited & Mujoco & Sim & Vision Pro & \xmark &  Cartesian\\
%         ARMADA \cite{armada}                               & \cmark & \cmark & \xmark & \xmark & \xmark & \xmark & \\
%         \citet{meng2023virtual}                            & \cmark & \cmark & \xmark & \cmark & \xmark & \xmark & \\
%         % GELLO \cite{wu2023gello}                           & \cmark & \xmark & \xmark & \xmark & \xmark & \xmark \\
%         % DexCap \cite{wang2024dexcap}                       & \xmark & \xmark & \xmark & \xmark & \xmark & \xmark \\
%         % AnyTeleop \cite{Qin2023AnyTeleopAG}                & \cmark & \cmark & \cmark & \cmark & \xmark & \cmark \\
%         % \citet{wang2024robotic}                            & \xmark & \xmark & \xmark & \xmark & \xmark & \xmark \\
%         Bunny-VisionPro \cite{bunnyvisionpro}              & \cmark & \cmark & \xmark & \xmark & \xmark & \xmark & \\
%         IMMERTWIN \cite{immertwin}                         & \cmark & \cmark & \xmark & \xmark & \xmark & \xmark & \\
%         Open-TeleVision \cite{opentelevision}              & \cmark & \cmark & \xmark & \xmark & \cmark & \xmark & \\
%         \citet{szczurek2023multimodal}                     & \xmark & \xmark & \xmark & Real & \xmark & \cmark & \\
%         OPEN TEACH \cite{openteach}                        & \cmark & \cmark & \xmark & \xmark & \xmark & \cmark & \\
%         \midrule
%         \textbf{Ours} & \cmark & \cmark & \cmark & \cmark & \cmark & \cmark \\
%         \bottomrule
%     \end{tabular}
%     \caption{TODO, Bruce: this table can be further optimized.}
% \end{table*}

\definecolor{goodgreen}{HTML}{228833}
\definecolor{goodred}{HTML}{EE6677}
\definecolor{goodgray}{HTML}{BBBBBB}

\begin{table*}[t]
    \centering
    \begin{adjustbox}{max width=\textwidth}
    \renewcommand{\arraystretch}{1.2}    
    \begin{tabular}{lccccccc}
        \toprule
        & \makecell{Cross-Scene}
        & \makecell{Cross-Embodiment}
        & \makecell{Cross-Simulator}
        & \makecell{Cross-Reality}
        & \makecell{Cross-Platform}
        & \makecell{Cross-User}
        & \makecell{Control Space} \\
        \midrule
        % Vicarios \cite{vicarios}                           & \xmark & \xmark & \xmark & \xmark & \xmark & \xmark \\     
        % Augmented Visual Cues \cite{augmentedvisualcues}   & \xmark & \xmark & \xmark & \xmark & \xmark & \xmark \\ 
        % OpenVR \cite{george2025openvr}                     & \xmark & \xmark & \xmark & \xmark & \xmark & \xmark \\
        \citet{digitaltwinmr}                              & \textcolor{goodred}{Limited}     & \textcolor{goodred}{Single Robot} & \textcolor{goodred}{Unity}    & \textcolor{goodred}{Real}          & \textcolor{goodred}{Meta Quest 2} & \textcolor{goodgray}{N/A} & \textcolor{goodred}{Cartesian} \\
        ARC-LfD \cite{arclfd}                              & \textcolor{goodgray}{N/A}        & \textcolor{goodred}{Single Robot} & \textcolor{goodgray}{N/A}     & \textcolor{goodred}{Real}          & \textcolor{goodred}{HoloLens}     & \textcolor{goodgray}{N/A} & \textcolor{goodred}{Cartesian} \\
        \citet{sharedctlframework}                         & \textcolor{goodred}{Limited}     & \textcolor{goodred}{Single Robot} & \textcolor{goodgray}{N/A}     & \textcolor{goodred}{Real}          & \textcolor{goodred}{HTC Vive Pro} & \textcolor{goodgray}{N/A} & \textcolor{goodred}{Cartesian} \\
        \citet{jiang2024comprehensive}                     & \textcolor{goodred}{Limited}     & \textcolor{goodred}{Single Robot} & \textcolor{goodgray}{N/A}     & \textcolor{goodred}{Real}          & \textcolor{goodred}{HoloLens 2}   & \textcolor{goodgray}{N/A} & \textcolor{goodgreen}{Joint \& Cartesian} \\
        \citet{mosbach2022accelerating}                    & \textcolor{goodgreen}{Available} & \textcolor{goodred}{Single Robot} & \textcolor{goodred}{IsaacGym} & \textcolor{goodred}{Sim}           & \textcolor{goodred}{Vive}         & \textcolor{goodgray}{N/A} & \textcolor{goodgreen}{Joint \& Cartesian} \\
        Holo-Dex \cite{holodex}                            & \textcolor{goodgray}{N/A}        & \textcolor{goodred}{Single Robot} & \textcolor{goodgray}{N/A}     & \textcolor{goodred}{Real}          & \textcolor{goodred}{Meta Quest 2} & \textcolor{goodgray}{N/A} & \textcolor{goodred}{Joint} \\
        ARCADE \cite{arcade}                               & \textcolor{goodgray}{N/A}        & \textcolor{goodred}{Single Robot} & \textcolor{goodgray}{N/A}     & \textcolor{goodred}{Real}          & \textcolor{goodred}{HoloLens 2}   & \textcolor{goodgray}{N/A} & \textcolor{goodred}{Cartesian} \\
        DART \cite{dexhub-park}                            & \textcolor{goodred}{Limited}     & \textcolor{goodred}{Limited}      & \textcolor{goodred}{Mujoco}   & \textcolor{goodred}{Sim}           & \textcolor{goodred}{Vision Pro}   & \textcolor{goodgray}{N/A} & \textcolor{goodred}{Cartesian} \\
        ARMADA \cite{armada}                               & \textcolor{goodgray}{N/A}        & \textcolor{goodred}{Limited}      & \textcolor{goodgray}{N/A}     & \textcolor{goodred}{Real}          & \textcolor{goodred}{Vision Pro}   & \textcolor{goodgray}{N/A} & \textcolor{goodred}{Cartesian} \\
        \citet{meng2023virtual}                            & \textcolor{goodred}{Limited}     & \textcolor{goodred}{Single Robot} & \textcolor{goodred}{PhysX}   & \textcolor{goodgreen}{Sim \& Real} & \textcolor{goodred}{HoloLens 2}   & \textcolor{goodgray}{N/A} & \textcolor{goodred}{Cartesian} \\
        % GELLO \cite{wu2023gello}                           & \cmark & \xmark & \xmark & \xmark & \xmark & \xmark \\
        % DexCap \cite{wang2024dexcap}                       & \xmark & \xmark & \xmark & \xmark & \xmark & \xmark \\
        % AnyTeleop \cite{Qin2023AnyTeleopAG}                & \cmark & \cmark & \cmark & \cmark & \xmark & \cmark \\
        % \citet{wang2024robotic}                            & \xmark & \xmark & \xmark & \xmark & \xmark & \xmark \\
        Bunny-VisionPro \cite{bunnyvisionpro}              & \textcolor{goodgray}{N/A}        & \textcolor{goodred}{Single Robot} & \textcolor{goodgray}{N/A}     & \textcolor{goodred}{Real}          & \textcolor{goodred}{Vision Pro}   & \textcolor{goodgray}{N/A} & \textcolor{goodred}{Cartesian} \\
        IMMERTWIN \cite{immertwin}                         & \textcolor{goodgray}{N/A}        & \textcolor{goodred}{Limited}      & \textcolor{goodgray}{N/A}     & \textcolor{goodred}{Real}          & \textcolor{goodred}{HTC Vive}     & \textcolor{goodgray}{N/A} & \textcolor{goodred}{Cartesian} \\
        Open-TeleVision \cite{opentelevision}              & \textcolor{goodgray}{N/A}        & \textcolor{goodred}{Limited}      & \textcolor{goodgray}{N/A}     & \textcolor{goodred}{Real}          & \textcolor{goodgreen}{Meta Quest, Vision Pro} & \textcolor{goodgray}{N/A} & \textcolor{goodred}{Cartesian} \\
        \citet{szczurek2023multimodal}                     & \textcolor{goodgray}{N/A}        & \textcolor{goodred}{Limited}      & \textcolor{goodgray}{N/A}     & \textcolor{goodred}{Real}          & \textcolor{goodred}{HoloLens 2}   & \textcolor{goodgreen}{Available} & \textcolor{goodred}{Joint \& Cartesian} \\
        OPEN TEACH \cite{openteach}                        & \textcolor{goodgray}{N/A}        & \textcolor{goodgreen}{Available}  & \textcolor{goodgray}{N/A}     & \textcolor{goodred}{Real}          & \textcolor{goodred}{Meta Quest 3} & \textcolor{goodred}{N/A} & \textcolor{goodgreen}{Joint \& Cartesian} \\
        \midrule
        \textbf{Ours}                                      & \textcolor{goodgreen}{Available} & \textcolor{goodgreen}{Available}  & \textcolor{goodgreen}{Mujoco, CoppeliaSim, IsaacSim} & \textcolor{goodgreen}{Sim \& Real} & \textcolor{goodgreen}{Meta Quest 3, HoloLens 2} & \textcolor{goodgreen}{Available} & \textcolor{goodgreen}{Joint \& Cartesian} \\
        \bottomrule
        \end{tabular}
    \end{adjustbox}
    \caption{Comparison of XR-based system for robots. IRIS is compared with related works in different dimensions.}
\end{table*}


\section{RELATED WORKS}

Research has shown that users display common instinctive social signals during robot errors, distinguishing these situations from error-free scenarios. These signals include gaze behaviour \cite{kontogiorgos_embodiment_2020, peacock_gaze_2022, kontogiorgos_systematic_2021, ramtin_hri_2025}, facial expressions \cite{mirnig_impact_2015, stiber_modeling_2022, kontogiorgos_systematic_2021, wachowiak_time_2024}, verbalisation \cite{kontogiorgos_embodiment_2020, mirnig_impact_2015, kontogiorgos_systematic_2021}, and body movements \cite{mirnig_impact_2015, trung_head_2017, wachowiak_time_2024}. For example,  Peacock et al. \cite{peacock_gaze_2022} observed that gaze initially increases in motion during failures and then stabilises as users address the issue. Stiber et al. \cite{stiber_using_2023} identified heightened activity in facial muscles, such as smiling and brow lowering, during robot errors. Similarly, Kontogiorgos et al. \cite{kontogiorgos_systematic_2021, kontogiorgos_embodiment_2020} reported increased spoken words, longer utterances, and more gaze shifts toward the robot, reflecting greater user engagement during failures.

Several studies have explored machine-learning approaches to detect failures in human-robot interactions using various behavioural and physiological cues. For example, Peacock et al. \cite{peacock_gaze_2022} trained logistic regression models on gaze dynamics to detect failures, achieving accurate detection a few seconds after the errors occurred. Similarly, Kontogiorgos et al. \cite{kontogiorgos_systematic_2021, kontogiorgos_behavioural_2020} developed machine learning models, including XGBoost and Random Forest, that utilised multimodal behaviours—such as linguistic, facial, and acoustic features—to achieve high accuracy in distinguishing failure scenarios from non-failure scenarios in a verbal guidance scenario. Separately, Stiber et al. \cite{stiber_modeling_2022} trained Multi-Layer Perceptron (MLP) models using action units (AUs) derived from facial reactions, showing that AUs are effective if users provide timely and observable responses to robot errors. Since not all users exhibit clear facial or verbal reactions to failures, this highlights a gap in the literature. This research aims to address this gap by designing classifier models based on user gaze during a collaborative task, evaluating their performance relative to the time elapsed after a robot failure, and assessing their effectiveness in real-time settings.
\section{Methodology}

\subsection{Tasks Description}

The experiment consists of four distinct tasks, in which one participant and a robot collaboratively solve Tangram puzzles. In each task, participants were required to create a unique shape using Tangram pieces. The sequence of shapes to solve is Rocket, Rabbit, Turtle, and Cat. We chose puzzles of similar difficulty to ensure the difficulty would not affect participants’ perception and behaviour towards the robot.

Each Tangram puzzle consisted of seven pieces. The robot handled four pieces (two small triangles, a square, and a parallelogram), while the participant had to place correctly three pieces (a medium-sized triangle and two large triangles). The Tangram pieces were 3D printed, and when assembled, formed a square of $200\,\text{mm}$ in side. Each piece was $20\,\text{mm}$ high. Besides, an cube ($32\,\text{mm} \times 32\,\text{mm} \times 40\,\text{mm}$) was attached on top of each piece to act as a handle and to facilitate the robot's ability to pick up the pieces. The puzzles' silhouettes were printed in black on A2 white paper, slightly larger than the Tangram pieces to avoid the need for very precise placement, with approximately 1 cm clearance on each side. These papers were fixed to the table, and the participant and the robot had to place each of their pieces in the correct position. The participant was asked to move a piece only after the robot had completed its action.

The robot always placed the first piece. To reduce confusion about when the participant should place their piece, the robot said: ``Now it is your turn.'', after placing each piece, except for the last one, when it said: ``Now, let’s solve the next puzzle.'' If the participant placed a piece incorrectly, the robot responded,``You have placed the object in the wrong location.''

The robot's pieces were placed next to the paper and near the robot, as shown in Figure \ref{fig:Robot}. In each puzzle, the arrangement of the pieces varied from the previous one, and the robot first determined the placement and orientation of each piece before picking it up. To facilitate this process, an ArUco marker was attached to the top of each piece, allowing the robot to accurately locate them. The Tiago robot, programmed using ROS1, then utilised the tf library to transform the pose of the desired object to the coordinate frame associated with its arm, and subsequently employed inverse kinematics to move its arm to the correct location. The robot's head movements were pre-programmed to approximately mimic human gaze behaviour. During its turn, the robot maintained its gaze on the Tangram piece while picking it up and placing it. Once the robot finished placing a piece, it started looking at the participant.




\subsection{Robot Failures}
We designed the robot to fail during each task in its interaction with the participants. These failures varied based on their type, timing, and whether the robot acknowledged its failure or not.

\subsubsection{\textbf{Types of Failures}}

The types of failures in our experiment represent typical robot malfunctions that may occur during interactions and are commonly reported in HRI. In this research, the robot will simulate two distinct types of failures: 1) Executional failure (EF) and 2) Decisional failure (DF).

%Timing issue
EF can be categorised as a technical failure, specifically timing and ordering \cite{honig_understanding_2018}. In this scenario, the robot pauses for 15 seconds just before picking up an object, while keeping the object within its end effector. After the 15-second pause, the robot will resume and complete the task of picking up and placing the object. This type of failure aligns with previous research \cite{kontogiorgos_embodiment_2020, wachowiak_analysing_2022}.

%Performing the wrong action
DF can also categorized as a technical failure, where the robot performs the correct action incorrectly \cite{honig_understanding_2018}. In this scenario, after picking up an object, the robot will mistakenly move to the location designated for a different object, place it and pause for 5 seconds. While still holding the object, the robot will then lift the object again and place it in the correct location. This type of failure aligns with previous research, in which the robot attempts to perform the correct action but executes it incorrectly \cite{inceoglu_fino-net_2021, stiber_using_2023}.


The procedure for picking up and placing objects during failure events is identical to the procedure when no failure occurs, indicating that the robot shows no signs of committing a failure beforehand. The only distinction in the EF is a pause, which increases the total time for the pick-and-place task by 15 seconds. In the DF, the robot moves its arm to the wrong location, goes down, and comes back up, resulting in an overall increase of 16.5 seconds to the motion.


\subsubsection{\textbf{Timing of Failures}}

The literature suggests that the timing of a failure—whether it occurs at the beginning or the end of an interaction—can affect a person’s perception of the robot differently. 
In this research, we aim to investigate how the timing of a failure impacts both gaze behaviour and user perceptions. Specifically, the robot may fail either at the beginning of the collaboration when placing its first piece, or towards the end of the interaction when placing its third piece.
 
\subsubsection{\textbf{Acknowledgement of Failures}}
A fault confessed is half redressed. Guided by this principle, we explored how the robot’s ability to acknowledge its mistakes influences participants’ perception and gaze patterns in subsequent failures. We designed two distinct scenarios. In one scenario, the robot demonstrates awareness of its mistakes by acknowledging each failure immediately after they occur. After a DF failure, it says, ``Sorry, I made a mistake.'' and after an EF, it says, ``Sorry for the delay.'' In the other scenario, the robot does not declare any of its failures.  In both scenarios, the robot performs physical repairs.

\subsection{Participants}

\begin{figure*}[t!]
  \centering
  \includegraphics[width=0.8\linewidth]{Photos/The_experimental_flow.pdf} 
  \caption{Experimental diagram showing the process where participants first complete a pre-task questionnaire, followed by collaboratively solving a Tangram puzzle four times.}
  \label{fig:Experimentdiagram}
\end{figure*}


We conducted \textit{a priori} power analysis to calculate the sample size for our experiment using \textit{G*Power}~\cite{faul_gpower_2007}. The calculation was based on a medium effect size of $f=0.25$, an alpha level of 0.05, and a power of 0.8. As a result, we determined a minimum of 24 participants was required; however, we recruited 27 participants (16 females, 10 males, and 1 non-binary) via our university recruitment website. Participants primarily consisted of students and university staff, none of whom had previous experience working with robots. They were compensated with a gift voucher for their participation. The average age was ranging between 18y.o. and 34y.o.  (M = 23.26, SD = 4.3). Due to some technical issue, one participant's gaze data was not recorded, and another participant did not complete some questions in one of the after-task questionnaires. Participants signed a consent form before participation and were made aware that their gaze data was recorded during the experiment. At the end of the experiment, participants were informed that the study aimed to observe their responses to the robot's failures.
\subsection{Experiment}



 Participants were first briefed by an experimenter on how to interact with the robot and the goals of the tasks. They then completed a demographics questionnaire, providing their age and gender. Following this, participants were seated at a table opposite the robot and asked to wear eye-tracking glasses during the experiment. As per the experimental conditions (Table \ref{table:participant_data}), during each puzzle, the robot correctly picked and placed three pieces but intentionally made an error with one piece. After each puzzle, participants completed a questionnaire assessing their perception of the robot's performance during that specific puzzle. Participants were unaware that these errors were pre-programmed. This cycle was repeated for all four puzzles. The experimental procedure is illustrated in Figure \ref{fig:Experimentdiagram}.

The experimenter initiated the robot's turns and intervened when participants made mistakes by triggering the robot's verbal response. This ensured that the robot began its turn immediately after the participant's turn, maintaining a consistent time gap across all participants. The experimenter was seated on the opposite side of the table, near the robot, to ensure safety and to press the emergency button in case of an actual malfunction. For consistency, the same experimenter conducted all sessions and operated the robot throughout the study.




The experiment was conducted in a laboratory on the University of Melbourne campus. The duration of solving each puzzle together with the robot was about 191.12s ± 35.40s. After each puzzle, the experimenter asked the participant to complete a survey and prepared the table for the next puzzle. The gap between each puzzle was about 93.88s ± 38.81s.


A mixed experimental design was used, with failure types (executional and decisional) and failure timings (early and late) as within-subjects factors, and failure acknowledgement as a between-subjects factor. To minimise order effects, the within-subjects factors were counterbalanced using a four-condition balanced Latin Square. Each factor was systematically integrated into the puzzles. The first thirteen participants experienced the failure acknowledgement, while the second fourteen did not. 



\begin{table}[h!]
\centering
\resizebox{0.48\textwidth}{!}{
\begin{tabular}{ c|c c c c c}

\textbf{Participant ID}  & \textbf{Puzzle 1} & \textbf{Puzzle 2} & \textbf{Puzzle 3} & \textbf{Puzzle 4} & \textbf{Acknowledgment} \\ \hline
1  & EF (Early) & EF (Late) & DF (Late) & DF (Early) & Yes \\ 
2  & EF (Late) & DF (Early) & EF (Early) & DF (Late) & Yes\\ 
3  & DF (Early) & DF (Late) & EF (Late) & EF (Early) & Yes\\ 
4  & DF (Late) & EF (Early) & DF (Early) & EF (Late) & Yes\\
5  & EF (Early) & EF (Late) & DF (Late) & DF (Early) & Yes \\
...  &  ... &  ... & ... & ... &  ... \\ 
14  & EF (Early) & EF (Late) & DF (Late) & DF (Early) & No \\ 
...  &  ... &  ... & ... & ... &  ... \\
\end{tabular}

}
\caption{Order of failure type and timing across puzzles with acknowledgement of failure}
\label{table:participant_data}
\end{table}



\subsection{Measures}
\subsubsection{Objective Gaze Measures}\label{gaze_measures}

For each puzzle and each piece, we recorded the robot’s current action— such as moving above the target object, and lowering to pick up the object—along with whether a failure occurred and the type of failure, all based on Unix time. We recorded users’ gaze data during the whole experiment.


Gaze data was collected during the tasks as participants collaborated with the Tiago robot to solve the puzzles. In our experiment, the gaze data during the robot's turn was particularly important, from the moment it started moving until it completed its turn. Data was captured using Neon Eye Tracking Glasses from Pupil Labs. The gaze data included the participant’s field of view image frame along with the x and y coordinates of their gaze within that frame. This data was recorded in real-time on a computer. The gaze data was captured at a rate of 30 Hz for both the image frames and gaze coordinates.


To facilitate the identification of participants' areas of interest (AoIs), we attached ArUco markers near the areas of interest. The AoIs in our experiment included the robot body (comprising the robot's face and torso), the Tangram figure, the end effector, the robot’s pieces, the participant’s pieces, and the experimenter. These areas of interest are illustrated in Figure \ref{fig:Robot}.

%one on the head of the robot, one on the body, one on the end effector, one near the robot pieces’ place on the left side of the task, one on the top of the task, one on the left side of the task, one on the left side of the participant’s pieces, and one on the right side of the participant’s pieces.

% We quantified the proportion of time each area of interest was observed during the task, the total number of gaze shifts, and the total number of shifts towards each area of interest. Utilising the gaze data, we developed transition matrices to facilitate comparative analysis across different conditions \cite{krejtz_gaze_2015, ebeid_analyzing_2019}.


We calculated several gaze-related measures to analyse user behaviour during the interaction. These metrics included: (1) the number of gaze shifts toward the robot body, (2) the number of gaze shifts across all AoIs, (3) the proportional distribution of gaze directed toward the robot body, the Tangram figure, and the robot’s end effector, and (4) transition and stationary entropy derived from gaze transition matrices \cite{krejtz_gaze_2015, ebeid_analyzing_2019}. Each of these measures captures different aspects of gaze behaviour. The number of gaze shifts reflects the frequency of visual transitions between specific areas, providing insight into user engagement and focus dynamics. The proportional distribution of gaze indicates how much time users spent looking at each AoI, offering a measure of relative visual attention. Transition entropy quantifies the unpredictability of gaze transitions between AoIs, while stationary entropy measures the overall distribution of gaze within the AoIs, highlighting how scattered or concentrated the gaze behaviour was during the task.


The gaze measures were calculated during a specific time window for both failure and non-failure conditions: from the moment the robot began moving to pick up an object until it placed the object and returned to its initial position.
%For each puzzle task, the robot correctly placed three of its own pieces and introduced one intentional failure. 
Since failure timing is not applicable in non-failure conditions, the analysis of these measures was conducted in two ways. First, we analysed the data by failure type (no failure, executional failure, decisional failure) and acknowledgement (yes vs. no). Second, we analysed it by failure type (executional failure, decisional failure), timing (early vs. late), and acknowledgement (yes vs. no).


\subsubsection{Subjective Measures}\label{Subjective_Measures}

After each puzzle, participants rated their perceptions of the robot's behaviour in terms of perceived intelligence, perceived safety, and performance trust. Perceived intelligence and safety were measured using items from the Godspeed questionnaire \cite{bartneck_measurement_2009}, while performance trust was assessed using items from the Multi-Dimensional Measure of Trust (MDMT) questionnaire \cite{ullman_mdmt_2023}.

To evaluate the level of intelligence participants attributed to the robot, we used three items from the Godspeed questionnaire: “Incompetent/Competent,” “Irresponsible/Responsible,” and “Foolish/Sensible.” For perceived safety, we included one item from the Godspeed questionnaire: “Anxious/Relaxed.” To assess performance trust across various robot failures, we utilised the “performance trust” dimension from the MDMT. This included two items from the Reliable subscale (“Reliable” and “Predictable”) and two items from the Competent subscale (“Skilled” and “Capable”).

The analysis of these measures was conducted based on failure type (executional failure, decisional failure), timing (early vs. late), and acknowledgement (yes vs. no).

%The items taken from the Godspeed questionnaire are presented as semantic differential 5-point scales, while the items from the MDMT are presented as semantic differential 7-point scales.
\begin{comment}
    
\subsubsection{Perceived Intelligence}

To measure the level of intelligence that participants attribute to our robots, we extracted three items from the Godspeed questionnaire. These items are presented as semantic differential 5-point scales: “Incompetent/Competent,” “Irresponsible/Responsible,” “Foolish/Sensible.”

\subsubsection{Perceived Safety}

To measure the level of safety that participants feel towards our robots, we extracted one item from the Godspeed questionnaire. These item is presented as semantic differential 5-point scales: “Anxious/Relaxed,”

\subsubsection{Trust}

The literature has previously identified trust as a critical factor that can be adversely impacted by robot failures. To assess and compare levels of trust across various robot failures, we utilized the "performance trust" dimension from the MDMT. Additionally, we incorporated two items from the Reliable subscale (“Reliable” and “Predictable”) and two items from the Competent subscale (“Skilled” and “Capable”). All questions were rated using a 7-point Likert scale.


\end{comment}
% \input{sections/03_6_Data Analysis}

\section{Results}
\label{sec:results}
% 
Figures~\ref{fig:LLM}-\ref{fig:DC} present results from applying counterfactual cross-validation (Algorithm~\ref{alg:C-CV}) across six benchmark scenarios detailed in \S\ref{sec:Benchmark_Toolbox}. Below, we outline our implementation approach and key findings.

For the LLM-based social network model, we conducted 10 distinct runs, constrained by OpenAI API limitations; the current simulation, with $N=1000$ and $T=30$, required approximately 100,000 GPT-3.5 API calls to generate experimental and ground truth results. Each run employs a unique treatment allocation following a staggered rollout design across three stages with $\Vec{\expr} = (0.2, 0.5, 0.8)$, each spanning 10 periods. This design implies that on average 20\% of units received the intervention in the initial 10 periods, followed by an additional 30\% in the subsequent 10 periods, and so forth.

For the remaining five experiments, we conducted 100 independent runs for each setting, utilizing a fresh treatment allocation for each run through a staggered rollout design. The design comprises four equal-length stages with treatment probabilities $\Vec{\expr} = (0, 0.2, 0.4, 0.6)$. In each figure's leftmost panel, we display the temporal evolution of outcomes through their mean and standard deviation, along with the 95th percentile across runs.

\begin{figure}
    \centering
    \includegraphics[width=1\linewidth]{plots/LLM.pdf}
    \caption{LLM-based social network with $N=1,000$ agents.}
    \label{fig:LLM}
\end{figure}

\begin{figure}
    \centering
    \includegraphics[width=1\linewidth]{plots/BAM1.pdf}
    \caption{Belief adoption model with Krupina network with $N=3,366$ users.}
    \label{fig:BAM1}
\end{figure}

\begin{figure}
    \centering
    \includegraphics[width=1\linewidth]{plots/BAM2.pdf}
    \caption{Belief adoption model with Topolcany network with $N=18,246$ users.}
    \label{fig:BAM2}
\end{figure}

\begin{figure}
    \centering
    \includegraphics[width=1\linewidth]{plots/BAM3.pdf}
    \caption{Belief adoption model with Zilina network with $N=42,971$ users.}
    \label{fig:BAM3}
\end{figure}

\begin{figure}
    \centering
    \includegraphics[width=1\linewidth]{plots/Auction.pdf}
    \caption{Ascending auction model with $N=500$ objects.}
    \label{fig:auction}
\end{figure}

The second panels of Figures~\ref{fig:LLM}-\ref{fig:DC} display the box plot of the average total treatment effect (TTE) across multiple time periods. The TTE contrasts the counterfactual of all units under treatment against all units under control:
% 
\begin{align}
    \label{eq:TTE}
    \text{TTE} :=
    \frac{1}{LN} \sum_{t=T-L+1}^T \sum_{i=1}^\UN 
    \left[
    \outcomeDW{\mathbf{1}}{i}{t}
    -
    \outcomeDW{\mathbf{0}}{i}{t}
    \right].
\end{align}
% 
In each setting, we carefully select $L$ so that the TTE in \eqref{eq:TTE} covers all periods with nonzero treatment probability, ensuring our benchmark estimators remain meaningful. The results compare ground truth (GT) values against estimates obtained from three methods: our proposed causal message-passing approach (CMP), the difference-in-means estimator (DM), and the Horvitz-Thompson estimator (HT)%
% 
\footnote{Difference-in-means (DM) and Horvitz-Thompson (HT) are expressed as:
\begin{align*}
%
\DIME :=  \frac{1}{L} \sum_{t=T-L+1}^T 
\Big(
\frac{\sum_{i=1}^N\outcomeD{}{i}{t}\treatment{i}{t}}{\sum_{i=1}^N\treatment{i}{t}} - \frac{\sum_{i=1}^N\outcomeD{}{i}{t}(1-\treatment{i}{t})}{\sum_{i=1}^N(1-\treatment{i}{t})} \Big),
%
\quad\quad
% 
\HTE :=  \frac{1}{LN} \sum_{t=T-L+1}^T \sum_{i=1}^N \left( \frac{\outcomeD{}{i}{t} \treatment{i}{t}}{\E[\treatment{i}{t}]} - \frac{\outcomeD{}{i}{t} (1 - \treatment{i}{t})}{\E[1 - \treatment{i}{t}]} \right).
%
\end{align*}}
%
\citep{savje2021average}. Finally, the rightmost two panels in each figure display the CFE under the ground truth and CMP estimates for all-control and all-treatment conditions, along with their respective 95th percentiles.

In implementing Algorithm~\ref{alg:C-CV}, we employ five validation batches ($b_v=5$). 
To select candidate estimators, we begin with a base model where each outcome is expressed as a linear function of two components: the sample mean of outcomes from the previous round and the current treatment allocation means. We then systematically modify this model by incorporating additional first-order and higher-order terms. The configurations also span batch counts from 200 to 2000 and batch sizes ranging from 0.1 to 20 percent of the population size. To estimate parameters, we employ Ridge regression with penalty parameters logarithmically spaced from $10^{-4}$ to $10^{4}$. These parameters are comprehensively combined to generate a diverse set of potential estimators, with time blocks aligned to experimental stages. For example, when $T=40$ and the design $\Vec{\expr} = (0, 0.2, 0.4, 0.6)$ with equal length blocks is used, 
$\tblockList$ has four elements, one corresponding to each block with a fixed treatment probability.
Then, the selection process incorporates both domain knowledge and observed data characteristics. For instance, the pronounced temporal patterns evident in the left panels of Figures~\ref{fig:NYC_taxi}-\ref{fig:DC}, observed in the New York City taxi model, exercise encouragement program, and data center model, necessitate estimators with detrending steps (see Remark~\ref{rem:proprocessing}). Computational efficiency is maintained by constraining the estimator search space based on the experimental context.

Overall, our framework demonstrates robust performance across all six scenarios, successfully estimating counterfactual evolutions despite strong seasonality patterns and without requiring information about the underlying interference network. The proposed method achieves significantly better performance than both DM and HT estimators, even in settings with subtle treatment effects. As illustrated in Figures~\ref{fig:LLM}-\ref{fig:NYC_taxi}, CMP yields estimates with both smaller bias and variance in different scenarios. The effectiveness of our method is particularly evident in the challenging scenarios presented in Figures~\ref{fig:BAM1}-\ref{fig:auction} and \ref{fig:DC}, where conventional estimators struggle to reliably determine even the direction of treatment effects. These comprehensive results establish our framework's capability to deliver precise estimates of counterfactual evolutions and treatment effects across diverse experimental settings.

\begin{figure}
    \centering
    \includegraphics[width=1\linewidth]{plots/NYC_taxi.pdf}
    \caption{New York City Taxi model with $N=18,768$ Routes.}
    \label{fig:NYC_taxi}
\end{figure}

\begin{figure}
    \centering
    \includegraphics[width=1\linewidth]{plots/EEP.pdf}
    \caption{Exercise encouragement program with $N=30,162$ users.}
    \label{fig:EP}
\end{figure}

\begin{figure}
    \centering
    \includegraphics[width=1\linewidth]{plots/DC_N1k.pdf}
    \caption{Data Center model with $N=1,000$ servers.}
    \label{fig:DC}
\end{figure}

\begin{remark}
    \label{rem:batch_generation}
    % 
    Selecting a predetermined number of batches for a given batch size $n^\batch$ presents a significant computational challenge, particularly in large-scale problems with time-varying treatment allocations across units. For staggered rollout designs, we implement a heuristic approach while deferring comprehensive analysis to future research. Our heuristic consists of three steps. First, we order units by their treatment duration, defined as the number of time periods under treatment. Second, we select two blocks of size $n^\batch$—one that slides through the ordered list to cover all treatment durations, and another chosen randomly to ensure sufficient between-batch variation. Third, we select individual units from these merged blocks with equal probability to generate batches with average size $n^\batch$. This procedure maintains computational efficiency while ensuring batches with diverse treatment allocations with high probability.
\end{remark}

\section{DISCUSSION}  

% In this work, we compared behavioural responses to robot failures, examining how individuals reacted to these failures and their subsequent perceptions of the robot. The failures in our experiment varied in type, timing, and the robot's acknowledgement of the failure. Our findings indicate differences in both user gaze and perception of the robot, specifically in feelings of anxiety, and perceptions of the robot being skilled and sensible. Based on these results, we will discuss the impact of two failure types, two timings, and two levels of the robot's acknowledgement of the failure on user gaze and perception.

This study compared behavioural responses to robot failures, focusing on how individuals reacted and perceived the robot. Failures varied by type, timing, and acknowledgement. The findings revealed that robot failures affect user gaze and perceptions. These findings are discussed further in the following section.

\subsection{Behavioural Response}

To address the first research question, we analysed user gaze behaviour in multiple ways: the number of gaze shifts, gaze distribution during puzzle-solving, and gaze entropy based on transition matrices. These measures allowed us to examine how the type and timing of failures, as well as whether the robot acknowledged its failure, influenced user gaze patterns and whether gaze behaviour varied across different failure scenarios. Our results showed that user gaze is a reliable indicator of robot failures. When the robot made a failure, participants exhibited more frequent gaze shifts between different AoIs, likely due to confusion and an attempt to understand what was happening. This finding is similar to the results of Kontogiorgos et al. \cite{kontogiorgos_embodiment_2020}, who found that people tend to gaze more at the robot when it makes a mistake. The literature suggests that different types of failures influence user perceptions of the robot \cite{morales_interaction_2019}, and our findings support this by showing that users exhibit distinct gaze behaviours in response to various failure types. For example, when the failure was executional, the number of gaze shifts towards the robot was significantly higher compared to when the failure was decisional. Moreover, during executional failures, the proportion of time spent looking at the robot was much higher compared to decisional failures. It is crucial for the robot to recognize the type of failure it has made so that it can determine the appropriate strategy for recovery and regain the user's trust.


The timing of the failure is also crucial for the robot, as it requires different approaches for recovery and repair. In our research, while the timing of the failure—whether at the start or end of the interaction—did not significantly affect gaze shifts, it did influence gaze transition matrices, and gaze distribution across AoIs. Failures at the beginning of the interaction led to higher median gaze transition values, indicating more randomness early on. Additionally, participants' focus on the Tangram figure was more when the failure occurred at the beginning of the interaction compared to later ones, while their focus on the robot's body or end effector was more during late failures than early ones. 


% \subsubsection{Failure Acknowledgement}

In our research, after committing a failure, the robot could either acknowledge the failure and then continue its action, or proceed without acknowledgement. We could not find significant differences in users' gaze behaviour when the robot acknowledged its failure and when it did not. As the literature suggests \cite{esterwood_you_2021, karli_what_2023, wachowiak_when_2024}, there are other verbal approaches to failure recovery, such as promises and technical explanations, which might influence users' gaze differently. Verbal failure recovery is important for robots, as it demonstrates an awareness of mistakes. This, in turn, can make the robot appear more intelligent and encourage users to engage with it more.


% \subsubsection{Anticipatory Gaze Behavior}
Our study also explored changes in users' anticipatory gaze behaviour during the task and its potential role in assisting the robot to recover from failures. Participants frequently anticipated the placement of the object before the robot executed the action, even when the robot made an error. This anticipatory gaze behaviour could serve as a valuable cue for the robot to detect its failures and initiate appropriate recovery strategies. However, we observed a decrease in participants' anticipatory gaze behaviour as the number of tasks increased. This decline may indicate reduced engagement over time, with participants being more actively collaborative at the beginning of the interaction. It also suggests that users' gaze behaviour might change throughout the interaction. These findings highlight the dynamic nature of gaze behaviour throughout the interaction.


\subsection{Subjective Measures}

To address the second research question, we examined user perceptions of the robot in three areas: perceived intelligence, sense of safety, and trust during failures. The analysis revealed how these measures varied with the type and timing of failure and whether the robot acknowledged its mistake.

The results of the subjective evaluation revealed that users' perceptions of the robot's intelligence and safety were not significantly influenced by the type of failure. However, users exhibited higher levels of trust in the robot during executional failures compared to decisional failures, suggesting that placing an object in an incorrect location reduces trust more than making an incorrect decision. Additionally, we observed interesting findings regarding the timing of the robot's failures. When failures occurred early in the interaction, users rated the robot as more intelligent and trustworthy compared to failures that occurred later. For the measure of "Sensible," this difference was statistically significant. These findings are consistent with previous research by Morales et al. \cite{morales_interaction_2019} and Lucas et al. \cite{lucas_getting_2018}. Interestingly, users reported feeling more relaxed when failures occurred later in the interaction, aligning with results from Desai et al. \cite{desai_impact_2013} and Rossi et al. \cite{rossi_how_2017}.

When the robot acknowledged its failures, users perceived it as slightly more intelligent and trustworthy but also experienced increased anxiety. This finding may be explained by the robot’s consistent physical repair actions a few seconds after each failure. When the robot did not explicitly acknowledge its failures, users might not have interpreted these actions as errors, reducing their perception of failure events.

\subsection{Limitations and Future Work}
 There were instances where participants were preoccupied with determining the placement of their next piece, which occasionally led them to overlook the robot's movements. However, these occurrences were minimal. Another limitation is the restriction to only two types of failure and whether the robot acknowledges its failure or not. The effect size in our study was medium; however, to obtain more robust results, a larger sample size would be beneficial.
 %Additionally, focusing solely on participants' gaze behaviour may not provide a comprehensive measure of failure detection. Incorporating other non-verbal cues, such as gestures or facial expressions, alongside gaze, could improve the accuracy of failure detection. 
 Furthermore, for safety reasons, the robot's arm movement was slowed and the experimenter was in the room, which may have influenced participants' perceptions. 
 Future research could address these limitations by exploring a broader range of failure types and incorporating explanatory feedback from the robot.





\vspace{-3.5mm}
\section{Conclusion}
\vspace{-2mm}

In summary, we introduce a first-order proximal algorithm to solve the perspective relaxation of cardinality-constrained GLM problems.
By leveraging the problem’s unique mathematical structure, we design a customized PAVA to efficiently evaluate the proximal operator, ensuring scalability to high-dimensional settings.
Further acceleration is achieved through an efficient value-based restart strategy and compatibility with GPUs, which collectively enhance convergence rates and computational speed.
Extensive empirical results demonstrate that our method outperforms state-of-the-art solvers by 1-2 orders of magnitude, establishing it as a practical, high-performance component for integration into next-generation MIP solvers.





%\section*{Acknowledgment}


\bibliographystyle{ieeetr}
\balance
\bibliography{references}


%\section{Appendix A: Additional Information}
\clearpage
\onecolumn
\appendix\label{App}

Results of the three-way mixed ANOVA for subjective measures

% In the appendix section:
\begin{table}[h!]
\centering
\resizebox{0.9\textwidth}{!}{
\begin{tabular}{c||c||c c c c c c c}

Scale  & Measure & Type & Timing & Acknowledgement & [Type*Timing] & [Type*Acknowledgement] & [Timing*Acknowledgement] & [Type*Timing*Acknowledgement] \\ \hline
Competent & df & $$(1,24)$$ & $(1,24)$ & $(1,24)$ & $(1,24)$ & $(1,24)$ & $(1,24)$ & $(1,24)$ \\ 
 & F value & 1.51 & 0.03 & 0.70 & 6.79 & 2.49 & 0.03 & 2.75 \\ 
& p value & .231 & .855 & .413 & \textbf{.016} & .127 & .855 & .110 \\ 
& $\eta^2$    & 0.06 & $<$0.01 & 0.03 & 0.22 & 0.09 & $<$0.01 & 0.10 \\ \hline
Sensible & df & $(1,24)$ & $(1,24)$ & $(1,24)$ & $(1,24)$ & $(1,24)$ & $(1,24)$ & $(1,24)$ \\ 
 & F value & 2.63 & 5.79 & 0.53 & 0.44 & 2.62 & 1.79 & 2.41 \\ 
& p value & .118 & \textbf{.024} & .475 & .512 & .118 & .194 & .134 \\ 
& $\eta^2$    & 0.10 & 0.19 & 0.02 & 0.02 & 0.10 & 0.07 & 0.09 \\  \hline
Responsible & df & $(1,24)$ & $(1,24)$ & $(1,24)$ & $(1,24)$ & $(1,24)$ & $(1,24)$ & $(1,24)$ \\ 
 & F value & 0.58 & 0.04 & 1.74 & 0.03 & 0.21 & 0.34 & 0.29 \\ 
& p value & .453 & .848 & .200 & .859 & .651 & .567 & .595 \\ 
& $\eta^2$    & 0.02 & $<$0.01 & 0.07 & $<$0.01 & $<$0.01 & 0.01 & 0.01 \\  \hline
Anxious/Relaxed & df & $(1,24)$ & $(1,24)$ & $(1,24)$ & $(1,24)$ & $(1,24)$ & $(1,24)$ & $(1,24)$ \\ 
(Self) & F value & 0.07 & 7.80 & 2.23 & 0.26 & 5.50 & 0.84 & 0.01 \\ 
& p value & .792 & \textbf{.010} & .148 & .613 & \textbf{.027} & .369 & .905 \\ 
& $\eta^2$  & $<$0.01 & 0.24 & 0.08 & 0.01 & 0.18 & 0.03 & $<$0.001 \\ \hline

Reliable & df & $(1,25)$ & $(1,25)$ & $(1,25)$ & $(1,25)$ & $(1,25)$ & $(1,25)$ & $(1,25)$ \\ 
 & F value & 1.61 & 0.83 & 1.28 & 1.50 & 1.00 & 0.35 & 0.99 \\ 
& p value & .216 & .370 & .268 & .233 & .328 & .561 & .330 \\ 
& $\eta^2$   & 0.06 & 0.03 & 0.05 & 0.06 & 0.04 & 0.01 & 0.04 \\  \hline

Predictable & df & $(1,25)$ & $(1,25)$ & $(1,25)$ & $(1,25)$ & $(1,25)$ & $(1,25)$ & $(1,25)$ \\ 
& F value & 2.98 & 1.43 & 0.94 & 3.97 & 5.38 & 0.26 & 3.97 \\ 
& p value & .097 & .243 & .340 & .057 & \textbf{.029} & .616 & .057 \\ 
& $\eta^2$  & 0.11 & 0.05 & 0.04 & 0.14 & 0.18 & 0.01 & 0.14 \\ \hline

Skilled  & df & $(1,25)$ & $(1,25)$ & $(1,25)$ & $(1,25)$ & $(1,25)$ & $(1,25)$ & $(1,25)$ \\ 
& F value & 4.98 & 2.93 & 0.43 & 0.13 & 1.65 & 0.11 & 3.46 \\ 
& p value & \textbf{.035} & .099 & .516 & .719 & .210 & .741 & .075 \\ 
& $\eta^2$    & 0.17 & 0.11 & 0.02 & $<$0.01 & 0.06 & $<$0.01 & 0.12 \\  \hline

Capable & df & $(1,25)$ & $(1,25)$ & $(1,25)$ & $(1,25)$ & $(1,25)$ & $(1,25)$ & $(1,25)$ \\ 
& F value & 1.71 & 3.02 & 0.80 & 1.15 & 0.002 & 0.41 & 6.99 \\ 
& p value & .203 & .095 & .380 & .293 & .962 & .528 & \textbf{.014} \\ 
& $\eta^2$    & 0.06 & 0.11 & 0.03 & 0.04 & $<$0.001 & 0.02 & 0.22 \\ \hline

\end{tabular}
}
%\caption{Results of the three-way mixed ANOVA for subjective measures}
\label{subjective}
\end{table} 



\end{document}
