\documentclass[conference]{IEEEtran}
\IEEEoverridecommandlockouts
% The preceding line is only needed to identify funding in the first footnote. If that is unneeded, please comment it out.
\usepackage{cite}
\usepackage{amsmath,amssymb,amsfonts}
\usepackage{algorithmic}
\usepackage{graphicx}
\usepackage{textcomp}
\usepackage{xcolor}
\usepackage{comment}
\usepackage{hyperref}
\usepackage{threeparttable}
\usepackage[font=footnotesize]{caption}
\usepackage{afterpage}
\usepackage{float}
\usepackage{balance}
\usepackage{subcaption} % Add this in your preamble if not already included

\usepackage[moderate,tracking=normal]{savetrees}
%\usepackage[subtle,tracking=normal]{savetrees}


 
\def\BibTeX{{\rm B\kern-.05em{\sc i\kern-.025em b}\kern-.08em
    T\kern-.1667em\lower.7ex\hbox{E}\kern-.125emX}}
\begin{document}

%\title{Investigating Human Gaze Dynamics in Response to Robotic Failure in Human-Robot Collaboration
%}
\title{Gazing at Failure: Investigating Human Gaze in Response to Robot Failure in Collaborative Tasks
}


    
\author{\IEEEauthorblockN{Ramtin Tabatabaei}
\IEEEauthorblockA{\textit{The University of Melbourne} \\
Melbourne, Australia \\
stabatabaeim@student.unimelb.edu.au}
\and

\IEEEauthorblockN{Vassilis Kostakos}
\IEEEauthorblockA{\textit{The University of Melbourne} \\
Melbourne, Australia \\
vassilis.kostakos@unimelb.edu.au}
\and

\IEEEauthorblockN{Wafa Johal}
\IEEEauthorblockA{\textit{The University of Melbourne} \\
Melbourne, Australia \\
wafa.johal@unimelb.edu.au}}



% \author{\IEEEauthorblockN{Anonymous Author(s)}
% }



\maketitle

\begin{abstract}
%Robots are prone to making errors during collaborative tasks with human users, which can negatively impact user perception of robot. 
Robots are prone to making errors, which can negatively impact their credibility as teammates during collaborative tasks with human users.
Detecting and recovering from these failures is crucial for maintaining effective level of trust from users. However, robots may fail without being aware of it. One way to detect such failures could be by analysing humans' non-verbal behaviours and reactions to failures. This study investigates how human gaze dynamics can signal a robot's failure and examines how different types of failures affect people's perception of robot. We conducted a user study with 27 participants collaborating with a robotic mobile manipulator to solve tangram puzzles. The robot was programmed to experience two types of failures ---executional and decisional--- occurring either at the beginning or end of the task, with or without acknowledgement of the failure. Our findings reveal that the type and timing of the robot's failure significantly affect participants' gaze behaviour and perception of the robot. Specifically, executional failures led to more gaze shifts and increased focus on the robot, while decisional failures resulted in lower entropy in gaze transitions among areas of interest, particularly when the failure occurred at the end of the task.
These results highlight that gaze can serve as a reliable indicator of robot failures and their types, and could also be used to predict the appropriate recovery actions.

\end{abstract}


\begin{IEEEkeywords}
Robot Failures, Gaze Dynamics, Human-Robot Collaboration
\end{IEEEkeywords}



\section{Introduction}

\begin{figure*}
    \centering
    \includegraphics[width=\textwidth]{figures/Introduction.pdf}
    \caption{Showing the novel problem statement applied to traffic prediction use case. Multiple unstructured observations from the past are used to reconstruct a hidden traffic state from which a full traffic state is forecast with a set of query locations. }
    \label{fig:intro}
\end{figure*}

% Was sagen denn die anderen warum Traffic Prediction gut ist? 
Forecasting the traffic in the near future is an important task for city management.
Data from the near past is used to predict future traffic states with spatio-temporal Graph Neural Networks \cite{bui22}.
Accurate prediction provides the opportunity to optimize traffic flow, reduce traffic jams and increase air quality \cite{Po19}.

% Wieso ist Sparsity in allen Dimensionen wichtig.
While traffic prediction relies on the availability of data from traffic sensors, there exists a plethora of reasons why sensors may stop working temporarily, such as simple errors, energy saving, or overloaded communication systems.
Considering small- or medium-sized cities, the coverage of sensors may be low because the sensors are too expensive or not available.
Also, the sensors are typically static and do not adapt to changes in the traffic flow (e.g. caused by a construction site), which motivates moving sensors that for example could be mounted on cars. 
However, both missing and moving sensors introduce sparsity, since measurements may not be available for all locations at all times.
This sparsity must be explicitly addressed in traffic prediction for a realistic application scenario, which is illustrated in figure \ref{fig:intro}.
From one hour of data on Sunday morning, only few observations of the traffic state are available at each timestep.
The number of observations may differ throughout the observed time and the observation itself can be distributed arbitrarily in the city. 
We assume a relatively low number of sensors to account for resource saving and sensor failure in our proposed framework SUSTeR.
The task is to predict the dense traffic state one timestep after the observations at all possible sensor locations.
We study this problem on the traffic dataset Metr-LA and PEMS-BAY to test our assumption that only a fraction of the sensor values would be enough for good predictions.
By modifying an existing traffic dataset, we are able to compare our results from very sparse observations to the bottom line with all information available.
A successful study will provide insights in how sensors in new cities can be reduced before installing them and further mobile sensors would save more resources and are able to adapt to new traffic situations.
We argue that in order to be adaptable to other cities and changes in traffic flows, prior information like the road network should be neglected and just the sparse observations considered.
This comes with the added benefit of making our solution applicable in regions where no openly available road network is maintained or pathways change frequently (e.g. flood areas, animal observations). 


The aforementioned problem is novel and more challenging than the commonly considered traffic prediction problem, since there exist very few observations in each input sample.
Current works for the traffic prediction problem do not consider any missing values. \cite{Li2021, Shao22}
A common method among state of the art approaches is the usage of Graph Neural Networks on graphs that model the sensor network \cite{bui22}.
The values of a sensor are applied to the same graph node for each timestep which prohibits any non-stationary sensors . 
With fixed sensor locations, the resulting sensor network is highly correlated with the road network.
Streets connecting two intersections with sensors should be also an interesting point for correlations in the sensor network.
However, variable observations and high temporal sparsity rates can not be modeled adequately in a static network.
We show in our experiments that the road network has only a small influence on the traffic predictions.

Besides the traffic prediction for future timesteps, some works explore the field of traffic speed imputation \cite{Cini22, Cuza22} where missing sensor values are predicted.
But the amount of missing values is assumed to be at most 80\%, which on average are still over 40 given sensors in each timestep in the Metr-LA dataset with a total of 207 sensors.
We consider up to 99.9\% missing values which are on average 2.4 observations in each timestep that are used as input.
Such high sparsity rates drastically decrease the chance that multiple values are present in one input sample from the same sensor location, which makes it challenging to recognize and learn temporal correlations for each location on its own.

High sparsity rates (>95\%) result in few sensor values, but if a reconstruction of the traffic state would be possible, we question if spatio-temporal graphs require nodes for each sensor.
In SUSTeR we utilize only a small amount of graph nodes for the encoding of information and do not relate such nodes to the sensor network.
We call this the hidden graph (see figure \ref{fig:intro}), which is still able to reconstruct the complete traffic state.
Due to the reduced number of nodes SUSTeR achieves faster runtimes, as shown in the experiments.
This hidden graph is not embedded directly in the spatial domain, which is why the assignment of observations, as well as the querying of the future traffic, is done with an encoder and a decoder, implemented as neural networks.
The decoding from the hidden graph to future values depends on a set of query locations.
Figure \ref{fig:intro} shows the query locations as given from outside and in combination with the reconstructed traffic state the future values are predicted.

To construct the hidden graph we encode observations from each timestep into from multiple graphs, one for each timestep. 
The graphs are created in a residual style and information is added to the node embeddings from the previous timesteps.
We choose this method to incorporate all timesteps equally into the hidden state because the redundant information along the past is non-existing for high sparsity rates.
From the sequence of graphs where our framework inserted the observations step by step we apply STGCN \cite{Yu18}, an algorithm for traffic prediction to find and learn the spatio-temporal correlations on our small number of graph nodes.
The first future timestep of the STGCN is our hidden graph in which the traffic state is reconstructed. 

% Recent work has an implicit embedding of the graph nodes into the spatial domain as the assignment from the sensor to graph node is fixed one by one.
% Because the graph has the same structure as the road network spatio-temporal correlations can be learned between those sensors.
% We reduce the number of nodes and use a non-linear assignment learned data-driven from the observations.

We find in the experiments that SUSTeR outperforms the plain STGCN and modern traffic prediction frameworks like D2STGNN for high sparsity rates $(\geq 99\%)$.
This is equivalent to only $0.2$ to $2.4$ observation for each timestep on average.
SUSTeR uses fewer parameters than the baselines and can train faster and with less training data.
Our main contributions can be summarized as follows:
\begin{itemize}
    \item We introduce a sparse and unstructured variant of the traffic prediction problem with sparsity in all dimensions. The sensors report only a fraction of their values and are arbitrarily distributed in the spatial domain.
    \item We propose SUSTeR, a framework around the STGCN architecture, which maps sparse observations onto a dense hidden graph to reconstruct the complete traffic state.
    Our code is available at github.\footnote{https://github.com/ywoelker/SUSTeR}
    \item We conducts experiments that show that SUSTeR outperforms the baselines in very sparse situations ($\geq 95\%$) and has a competitive performance in low sparsity rates.
    % \item SUSTeR trains a third faster than the next competitor.
\end{itemize}

\section{RELATED WORKS}

Research has shown that users display common instinctive social signals during robot errors, distinguishing these situations from error-free scenarios. These signals include gaze behaviour \cite{kontogiorgos_embodiment_2020, peacock_gaze_2022, kontogiorgos_systematic_2021, ramtin_hri_2025}, facial expressions \cite{mirnig_impact_2015, stiber_modeling_2022, kontogiorgos_systematic_2021, wachowiak_time_2024}, verbalisation \cite{kontogiorgos_embodiment_2020, mirnig_impact_2015, kontogiorgos_systematic_2021}, and body movements \cite{mirnig_impact_2015, trung_head_2017, wachowiak_time_2024}. For example,  Peacock et al. \cite{peacock_gaze_2022} observed that gaze initially increases in motion during failures and then stabilises as users address the issue. Stiber et al. \cite{stiber_using_2023} identified heightened activity in facial muscles, such as smiling and brow lowering, during robot errors. Similarly, Kontogiorgos et al. \cite{kontogiorgos_systematic_2021, kontogiorgos_embodiment_2020} reported increased spoken words, longer utterances, and more gaze shifts toward the robot, reflecting greater user engagement during failures.

Several studies have explored machine-learning approaches to detect failures in human-robot interactions using various behavioural and physiological cues. For example, Peacock et al. \cite{peacock_gaze_2022} trained logistic regression models on gaze dynamics to detect failures, achieving accurate detection a few seconds after the errors occurred. Similarly, Kontogiorgos et al. \cite{kontogiorgos_systematic_2021, kontogiorgos_behavioural_2020} developed machine learning models, including XGBoost and Random Forest, that utilised multimodal behaviours—such as linguistic, facial, and acoustic features—to achieve high accuracy in distinguishing failure scenarios from non-failure scenarios in a verbal guidance scenario. Separately, Stiber et al. \cite{stiber_modeling_2022} trained Multi-Layer Perceptron (MLP) models using action units (AUs) derived from facial reactions, showing that AUs are effective if users provide timely and observable responses to robot errors. Since not all users exhibit clear facial or verbal reactions to failures, this highlights a gap in the literature. This research aims to address this gap by designing classifier models based on user gaze during a collaborative task, evaluating their performance relative to the time elapsed after a robot failure, and assessing their effectiveness in real-time settings.
\section{Methodology}

\subsection{Tasks Description}

The experiment consists of four distinct tasks, in which one participant and a robot collaboratively solve Tangram puzzles. In each task, participants were required to create a unique shape using Tangram pieces. The sequence of shapes to solve is Rocket, Rabbit, Turtle, and Cat. We chose puzzles of similar difficulty to ensure the difficulty would not affect participants’ perception and behaviour towards the robot.

Each Tangram puzzle consisted of seven pieces. The robot handled four pieces (two small triangles, a square, and a parallelogram), while the participant had to place correctly three pieces (a medium-sized triangle and two large triangles). The Tangram pieces were 3D printed, and when assembled, formed a square of $200\,\text{mm}$ in side. Each piece was $20\,\text{mm}$ high. Besides, an cube ($32\,\text{mm} \times 32\,\text{mm} \times 40\,\text{mm}$) was attached on top of each piece to act as a handle and to facilitate the robot's ability to pick up the pieces. The puzzles' silhouettes were printed in black on A2 white paper, slightly larger than the Tangram pieces to avoid the need for very precise placement, with approximately 1 cm clearance on each side. These papers were fixed to the table, and the participant and the robot had to place each of their pieces in the correct position. The participant was asked to move a piece only after the robot had completed its action.

The robot always placed the first piece. To reduce confusion about when the participant should place their piece, the robot said: ``Now it is your turn.'', after placing each piece, except for the last one, when it said: ``Now, let’s solve the next puzzle.'' If the participant placed a piece incorrectly, the robot responded,``You have placed the object in the wrong location.''

The robot's pieces were placed next to the paper and near the robot, as shown in Figure \ref{fig:Robot}. In each puzzle, the arrangement of the pieces varied from the previous one, and the robot first determined the placement and orientation of each piece before picking it up. To facilitate this process, an ArUco marker was attached to the top of each piece, allowing the robot to accurately locate them. The Tiago robot, programmed using ROS1, then utilised the tf library to transform the pose of the desired object to the coordinate frame associated with its arm, and subsequently employed inverse kinematics to move its arm to the correct location. The robot's head movements were pre-programmed to approximately mimic human gaze behaviour. During its turn, the robot maintained its gaze on the Tangram piece while picking it up and placing it. Once the robot finished placing a piece, it started looking at the participant.




\subsection{Robot Failures}
We designed the robot to fail during each task in its interaction with the participants. These failures varied based on their type, timing, and whether the robot acknowledged its failure or not.

\subsubsection{\textbf{Types of Failures}}

The types of failures in our experiment represent typical robot malfunctions that may occur during interactions and are commonly reported in HRI. In this research, the robot will simulate two distinct types of failures: 1) Executional failure (EF) and 2) Decisional failure (DF).

%Timing issue
EF can be categorised as a technical failure, specifically timing and ordering \cite{honig_understanding_2018}. In this scenario, the robot pauses for 15 seconds just before picking up an object, while keeping the object within its end effector. After the 15-second pause, the robot will resume and complete the task of picking up and placing the object. This type of failure aligns with previous research \cite{kontogiorgos_embodiment_2020, wachowiak_analysing_2022}.

%Performing the wrong action
DF can also categorized as a technical failure, where the robot performs the correct action incorrectly \cite{honig_understanding_2018}. In this scenario, after picking up an object, the robot will mistakenly move to the location designated for a different object, place it and pause for 5 seconds. While still holding the object, the robot will then lift the object again and place it in the correct location. This type of failure aligns with previous research, in which the robot attempts to perform the correct action but executes it incorrectly \cite{inceoglu_fino-net_2021, stiber_using_2023}.


The procedure for picking up and placing objects during failure events is identical to the procedure when no failure occurs, indicating that the robot shows no signs of committing a failure beforehand. The only distinction in the EF is a pause, which increases the total time for the pick-and-place task by 15 seconds. In the DF, the robot moves its arm to the wrong location, goes down, and comes back up, resulting in an overall increase of 16.5 seconds to the motion.


\subsubsection{\textbf{Timing of Failures}}

The literature suggests that the timing of a failure—whether it occurs at the beginning or the end of an interaction—can affect a person’s perception of the robot differently. 
In this research, we aim to investigate how the timing of a failure impacts both gaze behaviour and user perceptions. Specifically, the robot may fail either at the beginning of the collaboration when placing its first piece, or towards the end of the interaction when placing its third piece.
 
\subsubsection{\textbf{Acknowledgement of Failures}}
A fault confessed is half redressed. Guided by this principle, we explored how the robot’s ability to acknowledge its mistakes influences participants’ perception and gaze patterns in subsequent failures. We designed two distinct scenarios. In one scenario, the robot demonstrates awareness of its mistakes by acknowledging each failure immediately after they occur. After a DF failure, it says, ``Sorry, I made a mistake.'' and after an EF, it says, ``Sorry for the delay.'' In the other scenario, the robot does not declare any of its failures.  In both scenarios, the robot performs physical repairs.

\subsection{Participants}

\begin{figure*}[t!]
  \centering
  \includegraphics[width=0.8\linewidth]{Photos/The_experimental_flow.pdf} 
  \caption{Experimental diagram showing the process where participants first complete a pre-task questionnaire, followed by collaboratively solving a Tangram puzzle four times.}
  \label{fig:Experimentdiagram}
\end{figure*}


We conducted \textit{a priori} power analysis to calculate the sample size for our experiment using \textit{G*Power}~\cite{faul_gpower_2007}. The calculation was based on a medium effect size of $f=0.25$, an alpha level of 0.05, and a power of 0.8. As a result, we determined a minimum of 24 participants was required; however, we recruited 27 participants (16 females, 10 males, and 1 non-binary) via our university recruitment website. Participants primarily consisted of students and university staff, none of whom had previous experience working with robots. They were compensated with a gift voucher for their participation. The average age was ranging between 18y.o. and 34y.o.  (M = 23.26, SD = 4.3). Due to some technical issue, one participant's gaze data was not recorded, and another participant did not complete some questions in one of the after-task questionnaires. Participants signed a consent form before participation and were made aware that their gaze data was recorded during the experiment. At the end of the experiment, participants were informed that the study aimed to observe their responses to the robot's failures.
\subsection{Experiment}



 Participants were first briefed by an experimenter on how to interact with the robot and the goals of the tasks. They then completed a demographics questionnaire, providing their age and gender. Following this, participants were seated at a table opposite the robot and asked to wear eye-tracking glasses during the experiment. As per the experimental conditions (Table \ref{table:participant_data}), during each puzzle, the robot correctly picked and placed three pieces but intentionally made an error with one piece. After each puzzle, participants completed a questionnaire assessing their perception of the robot's performance during that specific puzzle. Participants were unaware that these errors were pre-programmed. This cycle was repeated for all four puzzles. The experimental procedure is illustrated in Figure \ref{fig:Experimentdiagram}.

The experimenter initiated the robot's turns and intervened when participants made mistakes by triggering the robot's verbal response. This ensured that the robot began its turn immediately after the participant's turn, maintaining a consistent time gap across all participants. The experimenter was seated on the opposite side of the table, near the robot, to ensure safety and to press the emergency button in case of an actual malfunction. For consistency, the same experimenter conducted all sessions and operated the robot throughout the study.




The experiment was conducted in a laboratory on the University of Melbourne campus. The duration of solving each puzzle together with the robot was about 191.12s ± 35.40s. After each puzzle, the experimenter asked the participant to complete a survey and prepared the table for the next puzzle. The gap between each puzzle was about 93.88s ± 38.81s.


A mixed experimental design was used, with failure types (executional and decisional) and failure timings (early and late) as within-subjects factors, and failure acknowledgement as a between-subjects factor. To minimise order effects, the within-subjects factors were counterbalanced using a four-condition balanced Latin Square. Each factor was systematically integrated into the puzzles. The first thirteen participants experienced the failure acknowledgement, while the second fourteen did not. 



\begin{table}[h!]
\centering
\resizebox{0.48\textwidth}{!}{
\begin{tabular}{ c|c c c c c}

\textbf{Participant ID}  & \textbf{Puzzle 1} & \textbf{Puzzle 2} & \textbf{Puzzle 3} & \textbf{Puzzle 4} & \textbf{Acknowledgment} \\ \hline
1  & EF (Early) & EF (Late) & DF (Late) & DF (Early) & Yes \\ 
2  & EF (Late) & DF (Early) & EF (Early) & DF (Late) & Yes\\ 
3  & DF (Early) & DF (Late) & EF (Late) & EF (Early) & Yes\\ 
4  & DF (Late) & EF (Early) & DF (Early) & EF (Late) & Yes\\
5  & EF (Early) & EF (Late) & DF (Late) & DF (Early) & Yes \\
...  &  ... &  ... & ... & ... &  ... \\ 
14  & EF (Early) & EF (Late) & DF (Late) & DF (Early) & No \\ 
...  &  ... &  ... & ... & ... &  ... \\
\end{tabular}

}
\caption{Order of failure type and timing across puzzles with acknowledgement of failure}
\label{table:participant_data}
\end{table}



\subsection{Measures}
\subsubsection{Objective Gaze Measures}\label{gaze_measures}

For each puzzle and each piece, we recorded the robot’s current action— such as moving above the target object, and lowering to pick up the object—along with whether a failure occurred and the type of failure, all based on Unix time. We recorded users’ gaze data during the whole experiment.


Gaze data was collected during the tasks as participants collaborated with the Tiago robot to solve the puzzles. In our experiment, the gaze data during the robot's turn was particularly important, from the moment it started moving until it completed its turn. Data was captured using Neon Eye Tracking Glasses from Pupil Labs. The gaze data included the participant’s field of view image frame along with the x and y coordinates of their gaze within that frame. This data was recorded in real-time on a computer. The gaze data was captured at a rate of 30 Hz for both the image frames and gaze coordinates.


To facilitate the identification of participants' areas of interest (AoIs), we attached ArUco markers near the areas of interest. The AoIs in our experiment included the robot body (comprising the robot's face and torso), the Tangram figure, the end effector, the robot’s pieces, the participant’s pieces, and the experimenter. These areas of interest are illustrated in Figure \ref{fig:Robot}.

%one on the head of the robot, one on the body, one on the end effector, one near the robot pieces’ place on the left side of the task, one on the top of the task, one on the left side of the task, one on the left side of the participant’s pieces, and one on the right side of the participant’s pieces.

% We quantified the proportion of time each area of interest was observed during the task, the total number of gaze shifts, and the total number of shifts towards each area of interest. Utilising the gaze data, we developed transition matrices to facilitate comparative analysis across different conditions \cite{krejtz_gaze_2015, ebeid_analyzing_2019}.


We calculated several gaze-related measures to analyse user behaviour during the interaction. These metrics included: (1) the number of gaze shifts toward the robot body, (2) the number of gaze shifts across all AoIs, (3) the proportional distribution of gaze directed toward the robot body, the Tangram figure, and the robot’s end effector, and (4) transition and stationary entropy derived from gaze transition matrices \cite{krejtz_gaze_2015, ebeid_analyzing_2019}. Each of these measures captures different aspects of gaze behaviour. The number of gaze shifts reflects the frequency of visual transitions between specific areas, providing insight into user engagement and focus dynamics. The proportional distribution of gaze indicates how much time users spent looking at each AoI, offering a measure of relative visual attention. Transition entropy quantifies the unpredictability of gaze transitions between AoIs, while stationary entropy measures the overall distribution of gaze within the AoIs, highlighting how scattered or concentrated the gaze behaviour was during the task.


The gaze measures were calculated during a specific time window for both failure and non-failure conditions: from the moment the robot began moving to pick up an object until it placed the object and returned to its initial position.
%For each puzzle task, the robot correctly placed three of its own pieces and introduced one intentional failure. 
Since failure timing is not applicable in non-failure conditions, the analysis of these measures was conducted in two ways. First, we analysed the data by failure type (no failure, executional failure, decisional failure) and acknowledgement (yes vs. no). Second, we analysed it by failure type (executional failure, decisional failure), timing (early vs. late), and acknowledgement (yes vs. no).


\subsubsection{Subjective Measures}\label{Subjective_Measures}

After each puzzle, participants rated their perceptions of the robot's behaviour in terms of perceived intelligence, perceived safety, and performance trust. Perceived intelligence and safety were measured using items from the Godspeed questionnaire \cite{bartneck_measurement_2009}, while performance trust was assessed using items from the Multi-Dimensional Measure of Trust (MDMT) questionnaire \cite{ullman_mdmt_2023}.

To evaluate the level of intelligence participants attributed to the robot, we used three items from the Godspeed questionnaire: “Incompetent/Competent,” “Irresponsible/Responsible,” and “Foolish/Sensible.” For perceived safety, we included one item from the Godspeed questionnaire: “Anxious/Relaxed.” To assess performance trust across various robot failures, we utilised the “performance trust” dimension from the MDMT. This included two items from the Reliable subscale (“Reliable” and “Predictable”) and two items from the Competent subscale (“Skilled” and “Capable”).

The analysis of these measures was conducted based on failure type (executional failure, decisional failure), timing (early vs. late), and acknowledgement (yes vs. no).

%The items taken from the Godspeed questionnaire are presented as semantic differential 5-point scales, while the items from the MDMT are presented as semantic differential 7-point scales.
\begin{comment}
    
\subsubsection{Perceived Intelligence}

To measure the level of intelligence that participants attribute to our robots, we extracted three items from the Godspeed questionnaire. These items are presented as semantic differential 5-point scales: “Incompetent/Competent,” “Irresponsible/Responsible,” “Foolish/Sensible.”

\subsubsection{Perceived Safety}

To measure the level of safety that participants feel towards our robots, we extracted one item from the Godspeed questionnaire. These item is presented as semantic differential 5-point scales: “Anxious/Relaxed,”

\subsubsection{Trust}

The literature has previously identified trust as a critical factor that can be adversely impacted by robot failures. To assess and compare levels of trust across various robot failures, we utilized the "performance trust" dimension from the MDMT. Additionally, we incorporated two items from the Reliable subscale (“Reliable” and “Predictable”) and two items from the Competent subscale (“Skilled” and “Capable”). All questions were rated using a 7-point Likert scale.


\end{comment}
% \input{sections/03_6_Data Analysis}

\section{Results}
We identified key contexts of parental involvement, perceptions of AI-generated content, preferences for AI-assisted content creation, and collaborative patterns in shared interactions with a robot. We present the findings based on our research questions as follows.

%We identified four key areas that inform the design of an AI-assisted educational robot to support parental involvement in young children's learning activities. We present our results based on the study phases as follows: (1) \textit{Phase 1: Parent Contextual Needs and Scenarios}: understanding the real-life contexts and challenges parents face; (2) \textit{Phase 2.1: Parent Perspectives on AI-Generated Learning Content}: capturing parents' attitudes and concerns regarding the use of AI in generating educational content; (3) \textit{Phase 2.2: Parent Use of the LLM-Assisted Content Supervision Mechanism}: examining how parents review, edit, and supervise AI-generated content; and (4) \textit{Parent Use of the Robot Involvement Adjustment Mechanism}: exploring how parents delegate roles between themselves and the robot during learning activities.

\subsection{(RQ1) What contexts do parents encounter when involving in young children's learning activities?}

Each participant provided examples for the eight scenarios representing unique combinations of the three two-dimensional factors using the \texttt{SET-scenario cards}. We present P2's scenarios as an example in Table~\ref{tab:scneario-example}. The full set of scenario examples from all participants is documented as supplementary materials.\footnote{SET Scenarios: \url{https://osf.io/zfksg/?view_only=b59bd41287f543ce82ab85950aaf004f}} Beyond capturing the contexts shared by each parent, we analyzed these examples to identify and summarize key contextual patterns for the three factors of parental involvement: \textit{skills}, \textit{energy}, and \textit{time}.

\subsubsection{\textbf{Skill:} Parents face challenges in pedagogical skills, particularly with advanced or unfamiliar concepts.}
Parents mentioned several \textit{skills} in supporting their child's (1) intellectual, (2) pedagogical, and (3) social-emotional development, highlighting key challenges across these areas. While some parents (7/20) reported low confidence in \textit{intellectual} activities, especially advanced STEM topics (P1, P7, P12, P14, P17–19), many (16/20) felt confident in literacy (\textit{e.g.,} reading, spelling; P1–6, P8, P9, P11, P13, P14, P19, P20) and basic STEM (P6, P8–10, P12–15). Confidence often stemmed from personal expertise or interests, consistent with the Hoover-Dempsey and Sandler (HDS) framework \cite{green2007parents}. For example, P15, a physicist, felt confident teaching physics-related activities. The majority of parents (14/20) struggled with \textit{pedagogical} skills, such as explaining concepts (P7, P8, P13, P17, P18), answering or formulating questions (P3, P4, P6, P7, P9, P13), identifying developmental benchmarks (P4, P6, P10, P11), and allowing their child to learn from mistakes (P2, P12, P19). A smaller group of parents (7/20) expressed confidence in these areas, particularly explaining concepts (P14, P18, P19) and answering questions (P4, P5, P10, P14). \textit{Social-emotional} skills presented additional challenges. Some parents (6/20) struggled with teaching emotion regulation (P2, P17), behavioral management (P5, P15, P20), and interpersonal conflict resolution (P3, P15). Others (5/20) lacked confidence in encouraging participation in learning activities (P5, P11) or maintaining patience during learning support (P4, P10, P18, P19). Conversely, several parents (10/20) felt confident teaching emotion regulation (P1, P3, P10–13, P16, P18–20) and norms of polite communication (P7, P12, P16, P20).

\subsubsection{\textbf{Energy:} Parents' motivation depends on their physical and emotional status as well as the child's willingness to learn.}
%\paragraph{Parents' motivation depends on their physical and emotional status, time, and the child's willingness to learn.}
Parents suggested that their motivation to facilitate learning activities was affected by (1) physical status, (2) emotional status, and (3) time. Commenting on their \textit{physical status}, most parents (16/20) indicated low motivation when they need rest due to feeling ``\textit{hungry},'' ``\textit{sick},'' or ``\textit{tired}'' (P1--11, P14--17, P19), and many parents (9/20) reported being highly motivated when they are ``\textit{well rested}'' or after having ``\textit{a really good meal}'' (P1--4, P6, P7, P9, P11, P16). Regarding \textit{emotional state}, many parents (9/20) lacked motivation when they needed a mental break or ``\textit{me time}'' if they felt emotionally exhausted (P3, P7, P11--14, P17--19) or after spending time with their child (P5, P8, P20). In addition, some parents (7/20) lost motivation if their child appeared to be disinterested (P11, P12, P14) or poorly behaved (P4, P15, P17, P20). In contrast, many parents (16/20) were motivated when their child needed support (P12, P15), expressed interest and invited the parent to participate (P1, P4, P7, P8, P10, P12–14, P16–19), or is well behaved and ready to learn (P2, P5, P13, P18, P19, P20). Some parents (6/20) were highly motivated when they wanted to connect with their child (P3, P14, P15) or when they were personally interested in the activity (P5, P10, P12, P14).


\subsubsection{\textbf{Time:} Parents' availability depended on work, chores, other family members.}
Parents discussed (1) work and commitment, (2) household chores, and (3) family needs as factors that determined whether they had time, \textit{i.e.,} availability and presence, to facilitate learning activities. Most parents (19/20) were not available when they needed to be at \textit{work} (P1--4, P6, P7, P10--14, P16--19) and had other personal or professional engagements (P4, P5, P9, P11, P15, p20). Many parents (17/20) stated that \textit{household chores}, such as laundry, meal preparation, and cleaning, also determined their availability to be with their child (P1--6, P9--17, P20). Although some parents involved their child in chores (P1, P2, P7, P10, P15, P17), not all chores were seen as being appropriate or safe for children. Parents' availability also depended on the ability of other family members to provide support (P2--4, P6--8, P16--20), \textit{e.g.,} when a spouse helped with chores or an older child watches a younger sibling. Parents had less time if other family members needed them (P3, P7, P8, P12, P13, P16--18, P20), \textit{e.g.,} when a younger child is crying or a family member is sick. Finally, parents described their availability using specific time frames, \textit{e.g.,} ``\textit{weekday mornings} (P5, P8, P19),'' ``\textit{weekdays after dinner and before bedtime} (P5, P10, P18),'' ``\textit{anytime on weekends} (P1, P3, P4, P6, P8, P10, P16),'' or ``\textit{unstructured time} (P2, P11, P14, P16, P17, P20).'' They often structured their time and consider themselves available when they are physically present with their child (P1, P3--7, P14--16), such as during grocery shopping, car rides, or trips to the park together.





\subsection{(RQ2) How do parents perceive AI-generated content for young children?}\label{sec-result-2}

Parents showed mixed attitudes toward AI-generated learning content for young children. They discussed their perceived benefits and risks and envisioned ways to mitigate their concerns.

\subsubsection{Mixed Attitude towards AI-generated content}
Parents expressed a range of attitudes towards allowing AI to generate content for young children, ranging from skepticism and concern (P2--6, P10, P13--15) to open-minded caution (P9, P8, P16, P19, P20), acceptance (P6, P7, P11, P17, P18) and, in some cases, neutral (P1, P12). Parents who were \textit{\textbf{skeptical and concerned}} questioned whether AI-generated content met quality and safety standards, \textit{e.g.,} P3 questioned, ``\textit{Who's generating the content? Where is it getting the content from? Is it good? Is it safe?}'' On the other hand, parents who hold an attitude of \textit{\textbf{open-minded caution}} recognize the risks of using AI-generated content but feel open to use it under specific conditions. P16 highlighted model training, stating, ``\textit{I wouldn't be against it if the people training it were proficient in what the AI is teaching.}'' Similarly, P20 emphasized personal oversight, explaining, ``\textit{I can do my own evaluation to determine whether or not I think the content is good regardless of who it came from.}'' Furthermore, parents who have an attitude of \textit{\textbf{acceptance}} assume people who created the system have already ensure the appropriateness for children, \textit{e.g.,} P7 stated, ``\textit{I'm assuming because it's AI, there would be more research behind it.  So I would be okay with it.}'' Finally, parents who hold a \textit{\textbf{neutral}} attitude typically don't have much experience with AI and therefore feel unsure about their attitude for AI-generated content, \textit{e.g.,} P1 had ``\textit{not even thought about it until before this study.}''

\subsubsection{Perceived Benefits and Risks}
Parents identified several benefits of AI-generated content for young children. Some parents (P2, P4, P8, P10, P11, P16) highlighted AI's potential in \textit{\textbf{adaptability}} to adjust learning content to their child's evolving developmental needs, \textit{e.g.,} P11 expected AI to help ``\textit{adjust content as the child grows.}'' In addition, parents (P2, P3, P4, P16, P19) discussed \textit{\textbf{customization}}, illustrating that ``\textit{one of the big benefits would be to create material that are related to his[child's] interests and things that would be motivating to him[child]} (P4).'' Parents (P6, P7, P8, P10, P12, P17) also emphasized \textit{\textbf{efficiency}} of AI, explaining ``\textit{because it[AI] can access a huge amount of information very fast} (P12),'' enabling a ``\textit{quicker way to learn or to see something} (P7).'' Moreoever, a few parents (P1, P11, P18, P20) noted AI's potential to foster \textit{\textbf{affordability}}, suggesting that AI-generated content could enhance the scalability and accessibility of learning resources, making ``\textit{more learning materials available, more variety available} (P1),'' and making things ``\textit{cheaper and more accessible for people} (P20).'' Finally, a few parents (P14, P15) expected easier \textit{\textbf{pedagogical integration}} with AI, enabling parents to ``\textit{teach children things that sometimes parents don't know because not all parents know everything} (P14).''

Meanwhile, parents described their perceived risks of AI-generated content for children. Most parents (P1--3, P5, P11--15, P17, P19) were concerned about \textit{\textbf{age-inappropriateness}} of the content, which could be ``\textit{violent and don’t match family values} (P1),'' ``\textit{physically harmful and sexually inappropriate} (P3),'' and ``\textit{stuff about body image and certain people being better than other people} (P5).'' In addition many parents (P2, P3, P9, P14, P16, P17, P18) expressed concerns about the \textit{\textbf{inaccuracy}} of the information presented through AI-generated content, worrying that AI could provide ``\textit{factually inaccurate}'' learning materials or content that might imply theories that are ``\textit{misframed or misconstructed} (P2).'' Moreover, parents (P2--4, P14, P15) raised concerns about the \textit{\textbf{training data quality}} for AI models. P2 emphasized transparency stating, ``\textit{I'd want to know a lot more about where that training data came from or who supervised that learning process}.'' A few parents (P6, P7) expressed concerns about children's \textit{\textbf{over dependence}} on AI instead of developing their own cognitive abilities. P6 worried that constant use of AI could discourage critical thinking, stating, ``\textit{if they have a question, instead of thinking through the question, they just ask AI, not using their own brain}.'' Finally, two parents shared concerns over \textit{\textbf{message dilution}}, where AI oversimplifies complex ideas and diminishes their original intent. P15 worried that AI might dilute sociopolitical issues, such as racial diversity and gender identity. Similarly, P20 emphasized concern about whether the core message being conveyed to the child aligns with parental values, stating ``\textit{I'm more concerned about the message the book is trying to impart on the child}.

\subsubsection{Envisioned Risk Mitigation Methods}
Parents described what methods they envisioned to address their concerns. First, some parents (P3, P5, P11, P12, P14, P16) stressed the need to enable \textit{\textbf{parental review and verification}}. For example, P5 stated ``\textit{I would read it to make sure that it was actually something I wanted to read with her}.'' In addition, a few parents (P2, P17, P19) expressed that \textit{\textbf{social and public validation}} could also enhance their trust in AI-generated content, \textit{e.g.,} P2 described that ``\textit{if a thousand people used it...and endorsed this model, that would give me more confidence in it}.'' Moreover, some parents (P2, P9, P15) discussed \textit{\textbf{model and data transparency}}, emphasizing the need to understand how AI models are trained. As explained by P9, ``\textit{being able to know exactly what's going on or how it works...would make me feel more secure about what my child is learning}.'' Lastly, a few parents (P1, P15, P18) highlighted the importance of \textbf{expert involvement} in creating AI-generated content. For instance, P1 emphasized the need for oversight by ``\textit{people with a background in human development.}''

\subsection{(RQ3) How would parents prefer to collaborate with LLM on supervising content creation under different contexts?}

\begin{figure*}[b]
\includegraphics[width=\textwidth]{figures/figure-result-03-hho.pdf}
   \vspace{-6pt}
  \caption{Summary of parent's use of LLM-assisted content supervision mechanism: (1) content evaluation criteria, (2) use pattern, (3) perceived value.}
  \label{fig:result-03}
   \vspace{-6pt}
\end{figure*}

We found three main themes for parent-AI collaboration on content creation using the \textit{editor interface}: (1) \textit{Content evaluation and criteria}, referring to what parents pay attention to when reviewing and revising LLM-generated content. (2) \textit{Contextual usage patterns}, describing how parents envision using the LLM-powered interface in various contexts. (3) \textit{Perceived value and benefits}, covering what values parents believe LLM brings.

\subsubsection{Theme 1: Content evaluation and criteria}
Parents focus on balancing \textit{difficulty} and \textit{variety} of concepts as well as ensuring the \textit{quality} of questions when reviewing, regenerating, and revising LLM-generated content for young children.

\textit{\textbf{Parents aim to give their children the right level of challenge while reinforcing skills they can confidently accomplish.}} Many (8/20) avoided overly easy questions to prevent boredom but strategically included them at the start or after difficult questions to build confidence. As P2 explained, ``\textit{I want her to get the answers and then have it get increasingly difficult as she goes so she doesn't get discouraged at the beginning}.'' Meanwhile, most parents (11/20) valued challenges that stretch their child's abilities without overwhelming them. P12 concerned that ``\textit{underestimating her would be damaging for her},'' while P20 expressed interest in seeing how his child would handle harder concepts, saying, ``\textit{I'm actually really interested to see if she can answer.}'' Finally, parents (7/20) were also cautious of content that might be too advanced, \textit{e.g.,}``\textit{she[child] doesn't know uppercase or lowercase yet, so that doesn't mean anything to her} (P20).'' Additionally, \textit{\textbf{parents aim to maintain engagement by introducing diverse concepts and question types throughout the activity.}} Many parents (11/20) expressed concerns over repetitive content and preferred diverse topics to challenge their child differently. For instance, P20 changed the concept of a question to ``addition'', explaining, ``\textit{I just made the last question a `how many,' so this one I want a different concept}.'' Finally, \textit{\textbf{parents evaluate the quality of LLM-generated learning content based on standards} such as question clarity and coherence (9/20), wording precision (6/20), visual clarity (5/20), and cognitive load (P12, P20).} P12 raised issues with wording, stating, ``\textit{I don't think she's going to fully understand front legs versus back legs when it's a front view},'' while P6 expressed concerns about visual clarity: ``\textit{from the pictures, you can't really tell how many bugs with black bodies are flying in the air}.'' P20 also reflected on cognitive load, saying, ``\textit{I think it's just too long, too much information for her to process}.'' 

\subsubsection{Theme 2: Contextual usage patterns}
We discussed parents' preferences and behaviors when collaborating with LLM under two main contexts: (1) when parents have limited time or energy and (2) when they have sufficient time and energy.

\textit{When parents have limited time or energy}, most were still \textit{willing} to invest minimal effort (P4, P6--9, P11, P12, P15, P18), often opting to \textit{\textbf{skim through the LLM-generated content with minor self-editing}}. For example, P9 shared, ``\textit{I might skip quickly, skim through it, make sure there isn't anything that I feel is not appropriate}.'' This approach allows involvement with minimal time commitment. However, some parents (P4, P11, P15) emphasized that the \textit{\textbf{LLM output must be high-quality enough to require minimal editing}}, otherwise they may not use it at all. P11 explained, ``\textit{The more that stuff can be in really good shape before it gets to parents, the more we can minimize how much work we have to do ahead of time}.''

Some parents were \textit{unwilling} to invest effort when time or energy was limited. They preferred to either \textit{\textbf{reuse previously reviewed activities}} (P8, P11, P12) or directly \textit{\textbf{use LLM-generated content without review}} (P5, P6, P9, P10, P18, P20). As P8 explained, ``\textit{if I don't have time, I would have to be using something he's already done before, so I don't have to supervise it},'' while P10 noted, ``\textit{if the AI-generated questions were enough to keep him engaged, then it would be worth it}.'' A few parents (P1, P7) preferred to \textit{\textbf{avoid using the system entirely}}, as they feel uncomfortable leaving their child engaged with the content without supervision. As P1 explained, ``\textit{if I'm either physically or mentally not present. It's just not happening}.''

\textit{When parents have sufficient time and energy}, most of them (P6–P10, P12, P18, P20) choose to \textit{\textbf{review and edit the content in detail, even customizing questions}} to better supervise and personalize learning for their child. P9 shared, ``\textit{If I had more time and motivation, I would take the time to do it myself. I enjoy writing, so I'd probably spend time customizing the content}.'' Similarly, P12 noted, ``\textit{If I had all the time, I would go through and be picky with the wording and content of the questions}.''

In contrast, some parents (P1, P2, P7, P11, P12, P18) still prefer to \textit{\textbf{skim through the content with minor editing}}, as they found the detailed process too effortful even when time allowed, but they cannot fully trust LLM or themselves to come up with good questions. For example, P11 shared, ``\textit{I would probably scroll through and try to do as little editing as possible},'' while P7 expressed doubt, stating, ``\textit{I don't know that I would come up with better questions than this one from AI}.'' A few parents opted to \textit{\textbf{avoid using the system entirely}}, preferring to spend their time on other activities (P15) or relying on their ability to engage their child without the system (P4, P5). For instance, P15 shared, ``\textit{I would rather spend that time playing an imaginative game with her than spending time designing this,}'' P5 similarly expressed confidence, saying, ``\textit{I think I can and do ask her questions about stuff we read}.''

\subsubsection{Theme 3: Perceived value and benefits}
We found that parents perceive the value of the system to include not only \textit{content supervision}, but also \textit{content co-creation with LLM} and \textit{parent empowerment through pedagogical insights}.

First, and unsurprisingly, most parents suggested that the system allows them to \textit{\textbf{supervise the learning content generated by LLM}}. For example, P2, while feeling skeptical about trusting AI, noted, ``\textit{I don't know what AI model was used, still, I can confirm everything myself},'' reflecting the value parents place on maintaining oversight of the content presented to their children. Second, some parents (P6–10, P12, P18, P20) appreciated that the system allows them to \textit{\textbf{co-create personalized learning content with the LLM}} for their child without having to start from scratch. For example, P18 appreciated the ability to adapt the content to their child's needs, saying, ``\textit{tailoring it to her difficulty levels and seeing the ability to modify the content alleviates some concerns}.'' P10 highlighted how the LLM creates a draft to work from, stating, ``\textit{I do appreciate the concepts and the kinds of questions that it [LLM] provides, and how it has that template there}.'' This flexibility allowed parents to easily modify content while leveraging the assistance from LLM. Third, some parents found that the system \textit{\textbf{empowered parents with pedagogical insights}}. As many parents do not possess formal pedagogical knowledge--such as understanding how to effectively teach their child--they often struggle with determining what questions to ask or which concepts are age-appropriate. Since parents brought up the same value after interacting with the robot as well, we discuss this value more in-depth in Section \ref{sec-6.4.2}.

\subsection{(RQ4) How would parents prefer to collaborate with an AI-assisted robot to engage in learning activities with their children under different contexts?}

\begin{figure*}[b!]
\includegraphics[width=\textwidth]{figures/figure-result-04-hho.pdf}
   \vspace{-6pt}
  \caption{Summary of parent's use of robot involvement adjustment mechanisms: (1) usage pattern, (2) parenting education.}
  \label{fig:result-03}
   \vspace{-6pt}
\end{figure*}

We identified two major themes in the use of parent-robot collaboration mechanisms (\textit{i.e.,} \textit{mode-switching} and \textit{role-delegation}) within the \textit{activity interface}: (1) \textit{Contextual mode utilization}, referring to how parents adjust their involvement based on varying time and energy levels, and (2) \textit{Perceived educational impact on parenting}, highlighting how parents value the process for enhancing their skills and knowledge in parenting.

\subsubsection{Theme 1: Contextual mode utilization}

We discussed parents' preferences when collaborating with the AI-assisted robot across four contexts: (1) sufficient energy and time, (2) sufficient time but low energy, (3) sufficient energy but limited time, and (4) low energy and time. The impact of parental skill is discussed in specific cases.

(1) \textit{Parents have sufficient energy and time}: many parents (P2, P5–7, P9, P15, P18) preferred the \textit{\textbf{parent takeover mode}}, where they facilitate activities themselves while using LLM-generated content as a resource. For example, P18 shared, ``\textit{if I'm feeling motivated, I'd probably take over, but still look at some AI-generated questions to prompt me or remind me of things to ask or do with her}.'' Similarly, P15 noted, ``\textit{with full energy and time, I would use the parent-only mode because I want to interact with her and give her all my attention}.'' Parents valued the ability to take full control while using LLM-generated content for supplemental support when they have sufficient energy and time.

In addition, some parents (P2, P8–12, P15, P20) envisioned using \textit{\textbf{collaboration modes}}--where both the parent and the robot share responsibilities (\textit{i.e.,} parent-led or robot-led mode)--with the parents' \textit{skills} in specific areas relative to the robot playing a critical role in determining the pattern of role delegation. Parents often chose to involve the robot when they felt it could enhance their child's engagement especially in high-stakes tasks like quizzes. For example, P15 noted, ``\textit{I would use the robot for quizzes as a playful element to keep her engaged}.'' Similarly, P2 highlighted the objectivity of the robot in quizzing: ``\textit{I like the idea of reading her the book and then a neutral third party gets to test her on it}.'' On the other hand, parents took on specific roles when they believed their involvement would benefit their child more. P11 shared, ``\textit{I would let the robot read and ask questions but step in if he wasn't understanding or needed guidance},'' while P12 emphasized the emotional aspect of teaching: ``\textit{I can explain in a way that she understands, whereas the robot might come across as too harsh}.''

(2) \textit{Parents have sufficient time but lack energy}: some parents (P3, P7, P10--12, P15, P18, P20) opted for \textit{\textbf{collaboration modes}}, with their involvement influenced by their motivation levels and partially by their \textit{skill} relative to the robot. For example, P10 noted, ``\textit{when I'm not motivated, having the robot do the quiz takes some heat off me}.'' Similarly, P9 mentioned, ``\textit{I'd probably read the book, but have the robot do everything else}.'' Additionally, some parents (P1, P2, P4, P6–8, P15) chose to use \textit{\textbf{robot takeover mode}}--where the robot facilitates everything--while they remained nearby to supervise. For instance, P15 noted, ``\textit{I'd be around, but I wouldn't physically do much because I'm not feeling well}.'' Similarly, P2 noted, ``\textit{If I'm not motivated, I could see myself handing it all over to the robot}.''

(3) \textit{Parents have sufficient energy but lack time}: many parents (P2, P3, P5, P7--11, P15) opted for the \textit{\textbf{robot takeover mode}}--where the robot facilitates everything--while adjusting their usage based on \textit{how much they trust LLM}. Parents with higher trust allowed their child to use the content directly without review (P3, P5, P7, P9, P10). For example, P7 mentioned, ``\textit{If I'm trying to take a walk, I might do the robot takeover, then I can physically be gone}.'' In contrast, parents with less trust preferred to supervise while multitasking (P2, P7, P8), review content beforehand using the editor (P2, P8), or use the system only if the LLM model met high-quality standards (P11, P15). For example, P2 shared, ``\textit{If I'm not there, I wouldn't want them to do it, unless I had used the editor to review},'' while P7 described a multitasking scenario: ``\textit{I could be working from home while the robot takes over, and I'm nearby to supervise}.'' Moreover, a few parents (P2, P12) chose to \textit{\textbf{avoid using the system entirely}} due to their lack of trust in using LLM-generated content directly and insufficient time to review it, or because the system design did not support independent use for young children (P4, P12, P20). For instance, P2 noted, ``\textit{If I'm absent, I don't know if I'd want them to do any of this},'' while P12 stated, ``\textit{I know my daughter is sensitive, and if [the questions are too hard and] the robot keeps telling her she's wrong, she might take it personally and give up}.''

(4) \textit{Parents lack both energy and time}: Some parents chose \textit{\textbf{robot takeover mode}}, adjusting their usage based on their trust in LLM-generated content. Others \textit{\textbf{avoided using the system entirely}} due to low trust and insufficient time to review (P2, P12), or because the system design did not support independent use by young children (P4, P12, P20). Refer to the previous case—parents with sufficient energy but lacking time—as the usage patterns and contextual reasons are very similar.

\begin{figure*}[!t]
  \includegraphics[width=\textwidth]{figures/figure-quant-hho.pdf}
  \caption{Parent perception on child's math and literacy ability before and after the reading session. The result suggested that parents adjusted their perception after observing their child doing the activity and they tend to underestimate them, especially for advanced math concepts and phonological awareness concept for literacy. The horizontal lines represents significance from the Wilcoxon Signed-Ranked Test: $p < .01^{**}$, $p < .05^{*}$.}
  \label{fig:quant-result}
   \vspace*{-10pt}
\end{figure*}

\subsubsection{Theme 2: Perceived Educational Impact on Parenting} \label{sec-6.4.2}

Supported by mixed-method data, many parents thought \texttt{PAiREd} has value in parenting education, providing them with pedagogical strategies and giving them opportunities to observe their child's proficiency level systematically through observation. If the system provides a comprehensive framework and ample ideas, parents may not decrease their involvement in an activity just because they don't have the pedagogical skill; in fact, they may even increase their involvement. In addition, parents will be able to observe and adjust their understanding about what their child can do or cannot do, instead of under- or over- estimate their child's ability.

Several parents (P1, P4, P6, P7, P10--12, P18) appreciated that the system \textit{\textbf{offered ideas they might not have considered on their own}}, providing new topics to explore with their child. For instance, P1 emphasized, ``\textit{I hadn't even thought of all the different types of concepts},'' and P10 highlighted that the system ``\textit{gives more of a structure…even the drop down list of concepts is insightful, offering lenses I wouldn't normally consider when reading}.'' Others (P2, P6--9, P11, P12) valued that the LLM \textit{\textbf{generated example questions for each concept}}, allowing them to start with ready-made content without worrying whether their own questions reflected the intended learning goals. For example, P8 mentioned, ``\textit{What's nice about the AI-generated ones is that you can specifically choose a variety of concepts, whereas creating them on your own, you don't always know what the concepts are}.'' Additionally, some parents noted that the system allowed them to systematically select questions they were unsure their child could answer, which \textit{\textbf{provided a structured way to observe and assess their child's proficiency level}}. For example, P7 remarked, ``\textit{it gives her the opportunity to show me things she knows that I otherwise wouldn't have asked},'' while P8 shared, ``\textit{I was curious to see how he does if he doesn't know how to answer this, rather than just setting him up to succeed}.''

Before and after parent-child pairs engage in the activity, we asked parents to rate their perception on their child's math and literacy abilities. Our quantitative results suggest that parents adjusted their understanding of their child's proficiency in certain concepts after reading together. Specifically, \textit{\textbf{parents tended to underestimate what their child can or cannot do, especially with more advanced math concepts}} (Math-L3: $p < .01^{**}$, Math-L4: $p < .01^{**}$) and the phonological awareness concept in literacy ($p < .05^{*}$). Figure \ref{fig:quant-result} summarizes the significance results from the Wilcoxon Signed-Rank Test.

%A One-Way ANOVA revealed significant differences between levels (L1-L4), necessitating separate analyses. Subsequent Repeated Measures ANOVAs for each level showed significant improvement in Math L4 post-intervention. Post-hoc pairwise tests, using both parametric (Paired T-Tests) and non-parametric (Wilcoxon Signed-Rank) methods to ensure robustness, confirmed significance in Literacy L2, Math L3, and Math L4. This multi-layered approach—combining One-Way ANOVA, Repeated Measures ANOVA, and diverse post-hoc tests—ensured level independence, accounted for within-subject variability, and provided a comprehensive, robust understanding of level-specific improvements while minimising misinterpretation risks.







\section{Conclusion}

This paper introduces a variant of the multivariate time series traffic prediction problem with a focus on highly sparse and unstructured observations.
To address this problem we propose SUSTeR, a framework which handles sparse unstructured observations by creating hidden graphs in a residual fashion, which are then used with a conventional spatio-temporal GNN.
SUSTeR achieves better predictions for high sparsity (80\% - 99.9\% missing data) than existing baselines and remains competitive in denser settings or even when using only half the amount of the training data.
In addition, its training is considerably faster than the next-best competitor due to a smaller model size.

% We conduct experiments on a unstructured and sparse version of the traffic dataset Metr-LA and compare the performance of SUSTeR with traffic prediction baselines.
% The consideration of the sparsity within SUSTeR outperforms other approaches at sparsity rates $\geq$99\%.
% Experiments were performed up to a sparsity with only 2.4 observations within a sample where without missing data such a sample contains 12$\times$207 values.
% Further, the ablation studies explore the influence of our design choices and show the robustness of our framework.


\section{Future Work}

We plan to explore the interpretability within SUSTeR to obtain an intuitive understanding of the graph nodes within the hidden graph.
Small design choices are made within SUSTeR to make this possible, from observations that are not relying on each other in the same timestep, variable amounts of observations, a learnable assignment function from the observation to the hidden node, and an explicit learned laplacian matrix. 
The problem of sparse unstructured observations, which should be reconstructed into a hidden state, is present in many other domains.
In particular ocean data is a very promising application field for SUSTeR where sparse ARGO\footnote{https://argo.ucsd.edu} observations would perfectly match the problem definition to predict ocean states. 
There, observations are typically spatially and temporally sparse - comparable to the highest dropout rate in this paper - and observations are non-stationary and change their position freely.
We see SUSTeR as a bridge of the well-studied spatio-temporal mining methods into a new area of domains, in which such methods previously were not applicable.


\section{Conclusion}\label{sec:conclusion}
%This work explores the impact of grid-connected and wireless measurement setups on capacitive human body communication, revealing significant differences in both channel \revise{gain} and frequency behavior. 
While conventional data acquisition setups are effective for quantifying the forward path loss, which depends on the conductive properties of the human body, they substantially alter the return path behavior by artificially modifying the capacitive coupling to earth ground.
Therefore, a wireless, wearable-sized data acquisition system is essential for quantitatively evaluating the full \ac{HBC} communication channel in a realistic environment with minimal measurement interference. 
To address this challenge, this work introduces \textit{BodySense}, an evaluation platform for human body communication that is fully wireless, compact enough for wearable applications, and designed for extendability.
To validate the proposed system, the measured channel gains of a classical, grid-connected setup and a wireless setup have been determined for distances of \qty{10}{\centi\meter}, \qty{30}{\centi\meter}, and \qty{50}{\centi\meter} between transmitter and receiver for a frequency range between \qty{4}{\mega\hertz} and \qty{64}{\mega\hertz}.
A comparison between the two scenarios yields an average overestimation of \qty{18.15}{\db} over all investigated distances for the classical case, highlighting the importance of evaluating capacitive \ac{HBC} in realistic conditions.
When comparing the energy consumption of capacitive \ac{HBC} with \ac{BLE}, we achieved results comparable to state-of-the-art \ac{BLE} frontends. 
This demonstrates its potential as a promising alternative to conventional \ac{RF} links, offering opportunities to further enhance the overall energy efficiency of wearable devices and move closer to the realization of battery-free, body-worn sensor nodes.



%This paper proposes \textit{Bodysense}, a fully wireless, wearable-sized system designed to accurately evaluate capacitive human body communication. Experimental evaluation has revealed significant differences in both channel loss and frequency behavior. This paper demonstrated that while conventional data acquisition setups are effective for quantifying the forward path loss, which depends on the conductive properties of the human body, they substantially alter the return path behavior by artificially modifying the capacitive coupling to earth ground. Thus, the proposed wearable-sized data acquisition system is essential for quantitatively evaluating the full \ac{HBC} communication channel in a realistic environment with minimal measurement interference. 
%To address this issue, this paper presents \textit{Bodysense}, a fully wireless, wearable-sized, and extendable evaluation platform for human body communication.
%To validate the proposed system, the measured channel gains of a classical, grid-connected setup and a wireless setup have been determined for distances of \qty{10}{\centi\meter}, \qty{30}{\centi\meter}, and \qty{50}{\centi\meter} between transmitter and receiver for a frequency range between \qty{4}{\mega\hertz} and \qty{64}{\mega\hertz}.
%A comparison between the two scenarios yields an average overestimation of \qty{18.15}{\db} over all investigated distances for the classical case, highlighting the importance of evaluating capacitive \ac{HBC} with a measurement setup that is similar or ideally identical to the envisaged use case.






%\section*{Acknowledgment}


\bibliographystyle{ieeetr}
\balance
\bibliography{references}


%\section{Appendix A: Additional Information}
\clearpage
\onecolumn
\appendix\label{App}

Results of the three-way mixed ANOVA for subjective measures

% In the appendix section:
\begin{table}[h!]
\centering
\resizebox{0.9\textwidth}{!}{
\begin{tabular}{c||c||c c c c c c c}

Scale  & Measure & Type & Timing & Acknowledgement & [Type*Timing] & [Type*Acknowledgement] & [Timing*Acknowledgement] & [Type*Timing*Acknowledgement] \\ \hline
Competent & df & $$(1,24)$$ & $(1,24)$ & $(1,24)$ & $(1,24)$ & $(1,24)$ & $(1,24)$ & $(1,24)$ \\ 
 & F value & 1.51 & 0.03 & 0.70 & 6.79 & 2.49 & 0.03 & 2.75 \\ 
& p value & .231 & .855 & .413 & \textbf{.016} & .127 & .855 & .110 \\ 
& $\eta^2$    & 0.06 & $<$0.01 & 0.03 & 0.22 & 0.09 & $<$0.01 & 0.10 \\ \hline
Sensible & df & $(1,24)$ & $(1,24)$ & $(1,24)$ & $(1,24)$ & $(1,24)$ & $(1,24)$ & $(1,24)$ \\ 
 & F value & 2.63 & 5.79 & 0.53 & 0.44 & 2.62 & 1.79 & 2.41 \\ 
& p value & .118 & \textbf{.024} & .475 & .512 & .118 & .194 & .134 \\ 
& $\eta^2$    & 0.10 & 0.19 & 0.02 & 0.02 & 0.10 & 0.07 & 0.09 \\  \hline
Responsible & df & $(1,24)$ & $(1,24)$ & $(1,24)$ & $(1,24)$ & $(1,24)$ & $(1,24)$ & $(1,24)$ \\ 
 & F value & 0.58 & 0.04 & 1.74 & 0.03 & 0.21 & 0.34 & 0.29 \\ 
& p value & .453 & .848 & .200 & .859 & .651 & .567 & .595 \\ 
& $\eta^2$    & 0.02 & $<$0.01 & 0.07 & $<$0.01 & $<$0.01 & 0.01 & 0.01 \\  \hline
Anxious/Relaxed & df & $(1,24)$ & $(1,24)$ & $(1,24)$ & $(1,24)$ & $(1,24)$ & $(1,24)$ & $(1,24)$ \\ 
(Self) & F value & 0.07 & 7.80 & 2.23 & 0.26 & 5.50 & 0.84 & 0.01 \\ 
& p value & .792 & \textbf{.010} & .148 & .613 & \textbf{.027} & .369 & .905 \\ 
& $\eta^2$  & $<$0.01 & 0.24 & 0.08 & 0.01 & 0.18 & 0.03 & $<$0.001 \\ \hline

Reliable & df & $(1,25)$ & $(1,25)$ & $(1,25)$ & $(1,25)$ & $(1,25)$ & $(1,25)$ & $(1,25)$ \\ 
 & F value & 1.61 & 0.83 & 1.28 & 1.50 & 1.00 & 0.35 & 0.99 \\ 
& p value & .216 & .370 & .268 & .233 & .328 & .561 & .330 \\ 
& $\eta^2$   & 0.06 & 0.03 & 0.05 & 0.06 & 0.04 & 0.01 & 0.04 \\  \hline

Predictable & df & $(1,25)$ & $(1,25)$ & $(1,25)$ & $(1,25)$ & $(1,25)$ & $(1,25)$ & $(1,25)$ \\ 
& F value & 2.98 & 1.43 & 0.94 & 3.97 & 5.38 & 0.26 & 3.97 \\ 
& p value & .097 & .243 & .340 & .057 & \textbf{.029} & .616 & .057 \\ 
& $\eta^2$  & 0.11 & 0.05 & 0.04 & 0.14 & 0.18 & 0.01 & 0.14 \\ \hline

Skilled  & df & $(1,25)$ & $(1,25)$ & $(1,25)$ & $(1,25)$ & $(1,25)$ & $(1,25)$ & $(1,25)$ \\ 
& F value & 4.98 & 2.93 & 0.43 & 0.13 & 1.65 & 0.11 & 3.46 \\ 
& p value & \textbf{.035} & .099 & .516 & .719 & .210 & .741 & .075 \\ 
& $\eta^2$    & 0.17 & 0.11 & 0.02 & $<$0.01 & 0.06 & $<$0.01 & 0.12 \\  \hline

Capable & df & $(1,25)$ & $(1,25)$ & $(1,25)$ & $(1,25)$ & $(1,25)$ & $(1,25)$ & $(1,25)$ \\ 
& F value & 1.71 & 3.02 & 0.80 & 1.15 & 0.002 & 0.41 & 6.99 \\ 
& p value & .203 & .095 & .380 & .293 & .962 & .528 & \textbf{.014} \\ 
& $\eta^2$    & 0.06 & 0.11 & 0.03 & 0.04 & $<$0.001 & 0.02 & 0.22 \\ \hline

\end{tabular}
}
%\caption{Results of the three-way mixed ANOVA for subjective measures}
\label{subjective}
\end{table} 



\end{document}
