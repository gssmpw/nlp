In this paper, we present simulations in a finite but large dimension ($d=10,000$). We artificially generate samples from the model $y=\left \langle \mathbf{x}, \left ( \mathbf{v}^* \right )^{\odot 2}  \right \rangle+\xi $, where $\mathbf{x}\sim \mathcal{N} \left ( \mathbf{0},\mathbf{H}   \right ) $, $\mathbf{H}=\mathrm{diag}\left \{ i^{-\alpha } \right \} $, $ \mathbf{w} ^*_i=i^{-\frac{\beta-\alpha }{4} }$, and $\xi\sim \mathcal{N} \left (0,1   \right )$ is independent of $\mathbf{x}$. In our simulations, given a total of $T$ iteration, we assume that Algorithm~\ref{SGD} can access $T$ independent samples $\left \{ \left ( \mathbf{x}_i,y_i  \right )  \right \} _{i=1}^T$ generated by the above model. $h$ in Algorithm~\ref{SGD} is set to $\frac{T}{\log_2(T)} $. We numerically approximate the expected error by averaging the results of 100 independent repetitions of the experiment.
In the following, we detail the specific experimental settings and present the results obtained for each scenario.
\begin{itemize}
    \item \textbf{Figure~\ref{simulation-appendix} (a):} We compare the curve of mean error of SGD against the number of iteration steps for both linear and quadratic models, under the setting $\alpha=3$, $\beta=2$ and $T=500$.  The results show that the quadratic model exhibits a phase of diminishing error, while the linear model demonstrates a continuous, steady decrease in error.
    \item \textbf{Figure~\ref{simulation-appendix} (b):} We compare the curve of mean error of SGD against the number of iteration steps for both linear and quadratic models, under the setting $\alpha=2.5$, $\beta=1.5$ and $T=500$.  The results show that the quadratic model exhibits a phase of diminishing error, while the linear model demonstrates a continuous, steady decrease in error.
    \item \textbf{Figure~\ref{simulation-appendix} (c):} We compare the curve of mean error of SGD against the number of sample size for both linear and quadratic models, under the setting $\alpha=3$, $\beta=2$ and $T$ ranging from $1000$ to $5000$. The results indicate that the quadratic model outperforms the linear model and exhibits convergence behavior that is closer to the theoretical algorithm rate.
    \item \textbf{Figure~\ref{simulation-appendix} (d):} We compare the curve of mean error of SGD against the number of sample size for both linear and quadratic models, under the setting $\alpha=2.5$, $\beta=1.5$ and $T$ ranging from $1000$ to $5000$. The results indicate that the quadratic model outperforms the linear model and exhibits convergence behavior that is closer to the theoretical algorithm rate.
    \item \textbf{Figure~\ref{simulation-appendix} (e):} We compare the curve of mean error of SGD against the number of sample size for quadratic models with model size $M=10,30,50,100,200$, under the setting $\alpha=3$, $\beta=2$ and $T$ ranging from $1$ to $10000$. 
    The results show that for a fixed $M$, when $T$ is small, the convergence rate approaches the rate observed as $M\to \infty$. As $T$ increases sufficiently, the convergence rate stabilizes. Increasing $M$ results in an increase in the value of at which this stabilization occurs, which is consistent with the scaling law.
    \item \textbf{Figure~\ref{simulation-appendix} (f):} We compare the curve of mean error of SGD against the number of sample size for quadratic models with model size $M=10,30,50,100,200$, under the setting $\alpha=2.5$, $\beta=1.5$ and $T$ ranging from $1$ to $10000$. The results exhibit similar patterns to those observed in the previous figure.
\end{itemize}


\begin{figure}[t]\label{simulation-appendix}
\centering


\subfigure[\scriptsize{Quadratic v.s. Linear Model}]{
\begin{minipage}[t]{0.33\linewidth}
\centering
\includegraphics[width=2.2 in]{phase2.pdf}
\end{minipage}%
}%
\subfigure[\scriptsize{Quadratic v.s. Linear Model }]{
\begin{minipage}[t]{0.33\linewidth}
\centering
\includegraphics[width=2.2 in]{phase.pdf}
\end{minipage}%
}%
\subfigure[\scriptsize{Empirical v.s. Theoretical Results}]{
\begin{minipage}[t]{0.33\linewidth}
\centering
\includegraphics[width=2.2 in]{lin_vs_qua.pdf}
\end{minipage}%
}%

\subfigure[\scriptsize{Empirical v.s. Theoretical Results}]{
\begin{minipage}[t]{0.33\linewidth}
\centering
\includegraphics[width=2.2 in]{lin_vs_qua2.pdf}
\end{minipage}%
}%
\subfigure[\scriptsize{Scaling Law }]{
\begin{minipage}[t]{0.33\linewidth}
\centering
\includegraphics[width=2.2 in]{scaling_law.pdf}
\end{minipage}%
}%
\subfigure[\scriptsize{Scaling Law}]{
\begin{minipage}[t]{0.33\linewidth}
\centering
\includegraphics[width=2.2 in]{scaling_law2.pdf}
\end{minipage}%
}%
\centering
\caption{Numerical simulation results.}
\end{figure}
