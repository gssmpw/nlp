\section{Architecture}
\label{sec:architecture}

\begin{figure}
    \centering
    \includegraphics[width=0.92\linewidth]{graphs/arch.pdf}
    \caption{
        \sysname{} proposes an LLM-based no-code application development framework using FaaS for infrastructure abstraction.
        The prompt constructor combines a user's application description with a system prompt for an LLM that generates application code.
        The function deployer uses that code to deploy a FaaS function on a FaaS platform.
    }
    \label{fig:arch}
\end{figure}

LLMs are excellent tools for transforming natural language software descriptions into executable code, but are by themselves unable to deploy and operate that code for users.
FaaS platforms can deploy small pieces of code as scalable, managed applications.
With \sysname{}, we propose combining these two technologies into an end-to-end no-code application development platform.
Our goal is to let non-technical users, i.e., individuals without experience in software development or operation, provide application descriptions in natural language and build fully-managed applications from those descriptions.
Examples for such applications can be found, e.g., in the context of smart home automation, simple extensions of enterprise applications, or custom information aggregation from news websites, social media, and web APIs~\cite{paper_bermbach2020_webapibenchmarking2}.
To support such applications, we design \sysname{} as shown in \cref{fig:arch}.

\sysname{} comprises three main components: an LLM for generating user-specified code, a FaaS platform for efficient function deployment, and a bridge that orchestrates prompt construction and function deployment.
Users provide their natural language application descriptions to \sysname{}, which combines them with a static system prompt in a \emph{prompt constructor}.
This structured prompt instructs an LLM to generate code based on the natural language description, including, e.g., details on programming language, application context, API references, and runtime environment.
\sysname{} then parses the LLM's answer for code in a \emph{function deployer}.
This generated code is deployed on the FaaS platform which abstracts the underlying application infrastructure complexities by providing containerized, auto-scaling environments for on-demand execution.
