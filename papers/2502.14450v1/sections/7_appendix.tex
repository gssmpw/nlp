\appendix
\section{Questionnaire Design}
\label{sec:appendix:questionnaire}

To determine how well \sysname{} performs, we evaluate it in a smart home scenario with four automation ideas in varying levels of complexity, i.e., remote device control (\emph{easy}), scheduled plans (\emph{medium}), comfort homes (\emph{advanced}), and energy-efficient homes (\emph{complex}).
These automation ideas increase in complexity, covering a range of common smart home needs.


We design a questionnaire to gather users' natural language descriptions of these four smart home automation ideas.
The resulting descriptions serve as the primary input for the LLM to generate code snippets, which will then be deployed to FaaS platform.


Our questionnaire contains:
\begin{itemize}
    \item an introduction that includes the questionnaire objective, participants information, and data processing statement,
    \item the context of questions to familiarize participants with project context before proceeding with questions, i.e., a smart home layout and a list of available smart devices along with their functionality (\cref{tab:device-functions}),
    \item four questions about smart home automation ideas (our tasks) for participants to provide their natural language descriptions.
\end{itemize}

We avoid leading questions and design our questions to be sufficiently general to not guide answers while being clear enough for our participants to understand them.
Our questions for task descriptions are arranged in increasing order of complexity to enhance understanding.
We also collect users' professional occupations to confirm that they have no relevant technical expertise.
Before sending our questionnaire to users, we conduct test runs with five additional participants.



\begin{table}
    \renewcommand{\arraystretch}{1.5}
    \caption{Smart devices and their functionality}
    \label{tab:device-functions}
    \centering
    \resizebox{\columnwidth}{!}{%
        \begin{tabular}{p{1.47cm}p{4.9cm}}
            \toprule
            \textbf{Smart Device} & \textbf{Functionality}                                           \\ \midrule
            Light                 & Turn on/off, set brightness level (low, medium or high)          \\
            Cleaning robot        & Turn on/off, run a cleaning routine                              \\
            Music player          & Turn on/off, play playlist (`morning vibes', `daily news', etc.) \\
            
            Coffee machine        & Turn on/off, make a coffee (Americano, cappuccino, etc.)         \\
            
            Heater                & Turn on/off, set a temperature (e.g., 25°C)                      \\
            
            Air conditioner       & Turn on/off, set a temperature (e.g., 25°C)                      \\
            Humidifier            & Turn on/off                                                      \\
            
            Curtain               & Turn on/off                                                      \\
            
            TV                    & Turn on/off                                                      \\
            
            Socket                & Turn on/off                                                      \\
            
            Door                  & Open/close                                                       \\
            
            Window                & Open/close                                                       \\
            Temperature sensor    & Read temperature (°C)                                            \\
            Humidity sensor       & Read relative humidity (\%)                                      \\
            Light sensor          & Read light intensity (strong, normal, or weak)                   \\
            Smoke sensor          & Read smoke levels (safe or warning)                              \\ \bottomrule
        \end{tabular}%
    }
\end{table}

Our tasks are as follows:

\begin{enumerate}
    \item \emph{Easy}: turn on/off a specific device in the smart home in response to an event
    \item \emph{Medium}: routinely turn on/off a set of devices at a specific time of day
    \item \emph{Advanced}: automatically adjust a specific device in the smart home in response to a sensor reading
    \item \emph{Complex}: automatically turn on/off a set of devices in response to a sensor reading
\end{enumerate}



\section{Prompt Design}
\label{sec:appendix:prompt}

\sysname{} combines natural language application descriptions with system prompts to guide the LLM to generate code for a FaaS platform.
Our system prompt has three components:

\begin{itemize}
    \item Contextual information and key elements needed for the LLM to generate a response, e.g., the directive to generate FaaS code
    \item The user's natural language description
    \item Reference materials that the LLM should use, e.g., API descriptions
\end{itemize}

\begin{listing}
    \begin{minted}[baselinestretch=1.1,breaklines]{markdown}
# Preparation
Hi, I want you to provide me a 'function.py' file for my current smart home project based on my given functional description. Four code files in my project, i.e., sensor.py, actuator.py, home_plan.py and config.py, are in the 'home' folder. The required 'function.py' should locate in the 'functions' folder, function.py should contain main function. I will first give you the functional description, then give you the 4 python source code.

# Functional Description
<!-- INSERT HERE -->

# Source Code
## sensor.py
...
## actuator.py
...
## home_plan.py
...
## config.py
...
\end{minted}
    \caption{
        \sysname{} Prompt: The prompt explicitly requires a Python function designed to address specific user needs within a smart home scenario, with references offering foundational implementations, APIs, and related methods for seamless integration.
    }
    \label{fig:appendix:llm4faas-prompt}
\end{listing}

We show the prompt used in our experiments (\cref{sec:eva}) in \cref{fig:appendix:llm4faas-prompt}.
We provide source code for a smart home library that can interact with our smart home and that is available for the LLM to use to read sensors or talk to devices.

Our baseline prompt further includes the instruction to build a full Python application (rather than a FaaS function) and provide a single command that is to be executed from a Linux command line.
