\section{Background}
\label{sec:background}

\sysname{} relies on two main components: LLMs for code generation from natural language descriptions and FaaS for application deployment and operation.

\subsection{Large Language Models (LLMs) in Software Engineering}
\label{sec:background:llms}
LLMs offer advancements in software engineering, addressing challenges in code understanding and generation, and improving developer productivity.
An LLM is a type of deep learning model trained on extensive text corpora to understand, generate, and manipulate human language. 
LLMs excel in comprehending and addressing various natural language tasks, including text generation and translation as well as code generation, modification, and verification, which has made them a valuable tool for developers~\cite{liu2024your,vaithilingam2022expectation,ni2023lever,weisz2021perfection,xu2022ide,jin2024can}.
While the advanced natural language interpretation ability of LLMs are also a promising avenue for the involvement of individuals without programming skills in software development~\cite{bernsteiner2022citizen,smith2020unleashing,corradini2021floware}, operational concerns remain a considerable barrier.


\subsection{Function-as-a-Service (FaaS)}
\label{sec:background:faas}

FaaS is flexible, fine-grained infrastructure abstraction that minimizes operational overhead and allows developers to focus on building and optimizing application functionality, while an underlying FaaS platform, e.g., managed by a cloud provider, manages and runs individual functions in response to specific events or requests~\cite{baldini2017serverless, mcgrath2017serverless}.
With features such as scale-to-zero, event-driven architecture, and fine-grained control, the FaaS paradigm is well-suited for software development~\cite{gupta2023integration,macia2023serverless,kjorveziroski2021iot,gadepalli2019challenges,wen2021empirical,wolski2019cspot}. % serverless papers about iot application development
By automatically scaling functionality on demand, FaaS platform can quickly respond to changing user needs in real time.
FaaS allows developers to concentrate on functions without burden of infrastructure configuration and maintenance, which accelerates the development process, reduces time-to-market, and empowers rapid iteration on key functionality.
