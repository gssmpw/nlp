\section{Related Work}
\label{sec:rw}

The advent of LLMs has lead to a variety of research exploring its use in low-code and FaaS software development.
Liu et al.~\cite{liu2024empirical} compare traditional and LLM-based low-code programming by analyzing the similarities and differences of developer discussions on Stack Overflow over the past three years.
They show that LLM-based low-code programming is applicable to a broader range of scenarios and problem domains that its traditional counterpart.

Gao et al.~\cite{gao2024chatiot} propose a zero-code, token-efficient approach for generating \emph{trigger-action programs} based on user requirements for smart home automation.
However, in the absence of FaaS abstractions, their method requires a complex division of the TAP generation process into multiple subtasks, along with the design of a specialized conversational agent for each one.
This is a burden that keeps non-technical users from following their approach in practice.
Furthermore, there remains a gap between generating trigger-action programs and delivering a functional application to end users which we close through the FaaS platform.

Buchmann et al.~\cite{buchmann2024white} envision the use of white-box LLMs for low-code engineering tailored for citizen developers.
In their approach, citizen developers design the initial domain model and provide optional natural language annotations for domain rules.
The annotations guide LLM to refine Java code generated from the initial model.
However, this approach requires users to have prior knowledge of the domain model, limiting its accessibility for non-technical end users.
Moreover, achieving accurate code refinement involves iterative annotation rephrasing and repetition.

Hagel et al.~\cite{hagel2024turning} propose transforming DSL-based low-code platforms into no-code platforms by leveraging natural language descriptions to generate models.
These models are then applied within the DSL-based low-code platform for application customization, significantly reducing task completion time compared to manual modeling.
Their study involves technical participants working under supervision to ensure correct model generation.
However, this reliance on supervision and technical expertise limits the accessibility for non-technical users.
It also raises concerns about the accuracy of models generated by unsupervised users.
Additionally, the approach is specific to DSL-based low-code platforms, limiting its broader applicability.

A number of research publications has also considered integrating LLMs with FaaS programming.
For example, Esashi et al.~\cite{esashi2024action} use LLMs to automatically generate FaaS workflows, which can be used to assist cloud developers in configuring FaaS applications.
While similar in approach to \sysname{}, they target technical users that already have experience in software development, especially in the context of FaaS.

Eskandani et al.~\cite{eskandani2024towards} outline a vision for AI-powered software systems, theoretically emphasizing using LLMs solving FaaS challenges, e.g., cold starts, statelessness, and vendor lock-in.
Kathiriya et al. ~\cite{kathiriyaserverless} propose a conceptual architecture for banking applications that leverages LLMs to identify and perform basic data formatting tasks.
These two works offer a theoretical perspective on the concepts outlined in our paper and do not provide an implementation or experimental evaluation.
To the best of our knowledge, \sysname{} is the first no-code application development system based on FaaS that allows non-technical users to build applications from natural language descriptions.
