\section{Introduction}
\label{sec:introduction}


Large language models (LLMs) have shown remarkable capabilities in processing natural language and generating corresponding code, thus bridging the gap between non-technical users and the technical world~\cite{liu2024your,vaithilingam2022expectation,ni2023lever,weisz2021perfection,xu2022ide,jin2024can,smith2020unleashing}.
However, while non-technical users can use LLMs to generate code for their desired functionality, they typically lack the expertise to properly deploy and run the generated code.
For most people, managing servers, configuring services, or even using the command line are high barriers to operating applications.

We believe that the Function-as-a-Service (FaaS) paradigm and its no-ops principle can help:
FaaS platforms offer a scalable, event-driven, and fine-grained infrastructure abstraction~\cite{gupta2023integration,macia2023serverless,paper_bermbach2021_cloud_engineering,kjorveziroski2021iot,gadepalli2019challenges,wen2021empirical,wolski2019cspot,pfandzelter2023serverless,schirmer2023nightshift,wang2023lotus,malekabbasi2024geofaas}.
By decoupling functionality from infrastructure management, FaaS aligns with the principle of separation of concerns in application development, allowing developers to focus on business logic rather than operational concerns.

We propose combining the capabilities of LLMs with the abstractions provided by FaaS to enable non-technical users to build and operate their own custom applications based solely on natural-language prompts.
In this paper, we present \sysname{}, a novel no-code application development approach for end-users.
\sysname{} leverages (i)~the natural language processing capabilities of LLMs to transform user requirements into code snippets and (ii)~FaaS abstractions to streamline code generation and accelerate and simplify application development.
In this way, \sysname{} enables both application customization and development efficiency.

We make the following contributions:

\begin{itemize}
    \item We introduce \sysname{}, a no-code application development approach based on LLMs and FaaS (\cref{sec:architecture}).
    \item We implement a proof-of-concept prototype of \sysname{} using the \emph{GPT-4o}~\cite{openai2024} LLM and the \emph{tinyFaaS}~\cite{paper_pfandzelter2020_tinyfaas} lightweight FaaS platform, demonstrating the feasibility of our approach (\cref{sec:eva:poc}).
    \item We collect a dataset of prompts on application descriptions from non-technical users (\cref{sec:eva:study-design}).
    \item Using our dataset, we assess the efficacy of \sysname{} compared to running LLM-generated code outside FaaS, showing that \sysname{} can reliably build and deploy code in 71.47\% of cases, up from 43.48\% in a baseline without FaaS (\cref{sec:eva:results}).
\end{itemize}

We make all artifacts and datasets used to produce this paper available publicly.\footnote{https://github.com/Mhwwww/llm4faas-python}
