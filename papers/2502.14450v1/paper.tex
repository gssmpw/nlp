\PassOptionsToPackage{dvipsnames}{xcolor}

\documentclass[10pt,sigconf,nonacm,screen]{acmart}
\usepackage[all]{nowidow}
\usepackage{xcolor}
\usepackage{subcaption}
\usepackage{hyperref}
\usepackage{acronym}
% \usepackage[cachedir=.minted]{minted}
% step 1
% \usepackage[finalizecache,cachedir=.]{minted}
% step 2
\usepackage[frozencache,cachedir=.]{minted}

\usepackage[capitalise,noabbrev]{cleveref}

\crefformat{section}{\S#2#1#3}
\crefformat{subsection}{\S#2#1#3}
\crefformat{subsubsection}{\S#2#1#3}
\crefrangeformat{section}{\S#3#1#4 to~\S#5#2#6}
\crefrangeformat{subsection}{\S#3#1#4 to~\S#5#2#6}
\crefrangeformat{subsubsection}{\S#3#1#4 to~\S#5#2#6}

\setlength{\marginparwidth}{2cm}

\newcommand{\sysname}{\textrm{LLM4FaaS}}

\hyphenation{ ma-the-in-der }

\begin{document}

\author{Minghe Wang}
\affiliation{%
    \institution{TU Berlin \& ECDF}
    \city{Berlin}
    \country{Germany}}
\email{mw@3s.tu-berlin.de}
\orcid{0009-0001-3780-5828}

\author{Tobias Pfandzelter}
\affiliation{%
    \institution{TU Berlin \& ECDF}
    \city{Berlin}
    \country{Germany}}
\email{tp@3s.tu-berlin.de}
\orcid{0000-0002-7868-8613}

\author{Trever Schirmer}
\affiliation{%
    \institution{TU Berlin \& ECDF}
    \city{Berlin}
    \country{Germany}}
\email{ts@3s.tu-berlin.de}
\orcid{0000-0001-9277-3032}

\author{David Bermbach}
\affiliation{%
    \institution{TU Berlin \& ECDF}
    \city{Berlin}
    \country{Germany}}
\email{db@3s.tu-berlin.de}
\orcid{0000-0002-7524-3256}

\title{LLM4FaaS: No-Code Application Development using LLMs and FaaS}
\keywords{Large Language Models, Function-as-a-Service, No-Code Development}

\begin{abstract}
        Large language models (LLMs) are powerful tools that can generate code from natural language descriptions.
        While this theoretically enables non-technical users to develop their own applications, they typically lack the expertise to execute, deploy, and operate generated code.
        This poses a barrier for such users to leverage the power of LLMs for application development.
        
        In this paper, we propose leveraging the high levels of abstraction of the Function-as-a-Service (FaaS) paradigm to handle code execution and operation for non-technical users.
        FaaS offers function deployment without handling the underlying infrastructure, enabling users to execute LLM-generated code without concern for its operation and without requiring any technical expertise.
        We propose \sysname{}, a novel no-code application development approach that combines LLMs and FaaS platforms to enable non-technical users to build and run their own applications using only natural language descriptions.
        Specifically, \sysname{} takes user prompts, uses LLMs to generate function code based on those prompts, and deploys these functions through a FaaS platform that handles the application's operation.
        \sysname{} also leverages the FaaS infrastructure abstractions to reduce the task complexity for the LLM, improving result accuracy.
        
        We evaluate \sysname{} with a proof-of-concept implementation based on \emph{GPT-4o} and an open-source FaaS platform, using real prompts from non-technical users.
        Our evaluation based on these real user prompts demonstrates the feasibility of our approach and shows that \sysname{} can reliably build and deploy code in 71.47\% of cases, up from 43.48\% in a baseline without FaaS.
\end{abstract}

\maketitle

\section{Introduction}\label{sec:intro}

In computational finance, Monte Carlo simulations are used extensively to estimate the expected value of financial payoffs based on the solution of stochastic differential equations (SDEs) which model the evolution of stock prices, interest rates, exchange rates and other quantities \cite{glasserman04}.  Monte Carlo methods are very general and flexible, but for high accuracy it requires generating a large number of costly SDE path approximations, which has motivated research into a number of variance reduction or, equivalently, cost reduction techniques. One such method is
Multilevel Monte Carlo (MLMC), which was proposed in \cite{GILES2008} and was adapted for various applications that are summarised in \cite{Giles_overview17} and successfully combined with other methods such as quasi-Monte Carlo methods. The main idea of MLMC is to approximate the payoff using different time stepping resolutions when numerically solving the underlying SDE and to generate an optimal number of samples on each level, such that the overall computational cost is minimised subject to the desired bound on the variance. %, such that the total computational cost is minimised. 
The computational savings come from the fact that most samples are computed on the coarser levels and hence are less expensive while only a few samples from the finest levels are required \cite{GILES2008}.


Among the directions in which the computational cost 
of MLMC methods could further be reduced, an important avenue is the use of lower precision calculations, especially for the first Monte Carlo levels where the targeted accuracy is relatively low. 
 An overview of the research on mixed precision for the standard Monte Carlo (MC) framework is provided in \cite{ChowMixedPrecisionStandardMC} but only a few references study the potential of low precision computation in the MLMC framework \cite{Rounding_error_oliver}. To the best of our knowledge, the only MLMC framework with customised precision in the literature is \cite{brugger2014mixed}, but they use a uniform precision for all operations on each Monte Carlo level instead of optimising 
 the precision of each intermediary variable to reduce as much as possible the cost of path generation.
 
An important motivation for an MLMC framework with variable precision would be performing the low precision computations on reconfigurable hardware devices such as Field Programmable Gate Arrays (FPGAs). FPGAs contain customizable logic blocks and connectors that make it easy to adapt the digital circuit architecture for a specific application, leading to a highly parallel and optimised implementation. Therefore they are successfully exploited in applications that require high speed and have high computational workload, such as signal processing \cite{woods2008fpga}, and real time applications like high frequency trading \cite{HFT1,HFT2}. That is why a number of previous works in hardware architecture design implemented the MLMC algorithm to price financial options using FPGAs as accelerators, which resulted in improved speed and power efficiency compared to full CPU architectures \cite{Schryver2013AMM}. The paper \cite{lindsey2016domain} also proposed 
a Domain Specific Language to automate the configuration of FPGAs for this specific application. However, only \cite{brugger2014mixed} proposed a heuristic to reduce the precision in calculations.

In addition, all aforementioned works considered that the random number generation (RNG) is performed in single or double precision. Yet in most cases an important portion of the workload in the overall MLMC simulation comes from the RNG and in \cite{brugger2014mixed} this limited the total computational savings.
To reduce the cost of MLMC simulations in particular those based on the Geometric Brownian Motion (GBM), \cite{approximateICDF_Oliver, NestedOliver} have proposed to use approximate random numbers that are generated by applying an approximation of the inverse CDF to uniform random numbers. In \cite{NestedOliver}, the authors proposed a way to integrate these lower precision random variables into a \textit{nested} MLMC framework and completed a numerical analysis to bound the resulting error at each MC level by a product of the time step and the error in the random number approximation. The same authors show in \cite{approximateICDF_Oliver} that using approximate random variables reduces the cost of path generation by a factor 7.


In this paper we propose a nested MLMC framework that combines the use of approximate random normal variables and lower precision calculations to reduce the computational cost of MLMC even further than \cite{brugger2014mixed,NestedOliver}. We illustrate the efficiency of our framework in Matlab, after making several assumptions on the cost of operations and size of the errors that we carefully justify. We focus on the case of GBM and use the approximate RNG methods presented in \cite{approximateICDF_Oliver} as well as a new slightly modified method that combines CDF inversion and the central limit theorem. To choose the precision of the variables in the low precision path generation, we introduce a novel method to optimise the bit-widths. This optimisation is performed before the main path generation loop is executed and is based on a linear model of the payoff error  
due to rounding when computing in low precision. The error model relies on algorithmic differentiation in a similar manner to \cite{unifying-bwoptim,bitwidth-AD,ADAPT}. The bit-width optimisation procedure can be performed off-line, so this stage can be excluded from the on-line time complexity of our framework. The user specified desired accuracy is then enforced by calculating on-line the number of samples that need to be generated.

In terms of hardware design, we suggest implementing the low precision path generation on FPGAs and the full-precision ones on a CPU or GPU. 
The FPGA offers enough flexibility to define a separate bit-width for every variable in the low precision path generation, and can be reconfigured periodically to update the bit-widths when the market parameters have changed considerably. 


The paper is organized as follows : \Cref{sec:MLMC} introduces MLMC and nested MLMC to make clear the estimator that is implemented in our framework. Then in \Cref{sec:RNG} we detail the methods that could be used to obtain approximate random normally distributed numbers very cheaply for the low precision path generation. In \Cref{sec:error_model} and \Cref{sec:costModel} we propose an error model and a cost model (resp.) that we then use to formulate the optimisation problem that is solved to obtain the optimal bit-widths of fixed point variables in \Cref{sec:optimisation}. Finally we summarise our results and future directions in \Cref{sec:conclusion}.



\section{Background}
\label{sec:background}


\subsection{Preliminaries}

{\color{red}[TODO: LLMs? in-context learning?]}

\subsection{Problem Definition}

{\color{red}[TODO: define the problem of citation intent]}

% \section{Comparison of Normalization Strategies} \label{sec:ln_in_transformer}
\section{Comparative Analysis} \label{sec:ln_in_transformer}

In this section, we discuss how different placements of layer normalization (LN \footnote{Unless stated otherwise, LN refers to both LayerNorm and RMSNorm.}) in Transformer architecture affect both training stability and the statistics of hidden states (activations \footnote{We use ``hidden state'' and ``activation'' interchangeably.}).

\subsection{Post- \& Pre-Normalization in Transformers}
\label{subsec:post_pre_ln}
\paragraph{Post-LN.}
The Post-Layer Normalization (Post-LN) \citep{attentionisallyouneed} scheme, normalization is applied \emph{after} summing the module’s output and residual input:
\begin{equation}
    y_{l} = \mathrm{Norm}\bigl(x_l + \mathrm{Module}(x_l)\bigr),
    \label{eq:post_ln}
\end{equation}
where $x_l$ is the input hidden state of $l$-th layer, $y_{l}$ is the output hidden state of $l$-th layer, and $\mathrm{Module}$ denotes Attention or Multi-Layer Perceptron (MLP) module in the Transformer sub-layer. $\mathrm{Norm}$ denotes normalization layers such as RMSNorm or LayerNorm. It is known that by stabilizing the activation variance at a constant scale, Post-LN prevents activations from growing. However, several evidence~\citep{onlayer, transformersgetstable} suggest that Post-LN can degrade gradient flow in deeper networks, leading to vanishing gradients and slower convergence.


\paragraph{Pre-LN.}
The Pre-Layer Normalization (Pre-LN)~\citep{llama3} scheme, normalization is applied to the module's input \emph{before} processing:
\begin{equation}
    y_l = x_l + \mathrm{Module}\bigl(\mathrm{Norm}(x_l)\bigr).
    \label{eq:pre_ln}
\end{equation}
As for Llama $3$ architecture, a final LN is applied to the network output. Pre-LN improves gradient flow during backpropagation, stabilizing early training \citep{onlayer}. Nonetheless, in large-scale Transformers, even Pre-LN architectures are not immune to instability during training~\citep{smallproxies, attentioncollapse}. As shown in Figure~\ref{fig:LN Placement}, unlike Post-LN—which places LN at position $C$—Pre-LN, which places LN only at position $A$, can lead to a “highway” structure that is continuously maintained throughout the entire model if the module produces an output with a large magnitude. This phenomenon might be related to the ``massive activations'' observed in trained models \citep{massiveactivation, mlpswiglu}. 

\begin{figure}[t]
% \vskip -0.1in
    \centering
    \begin{minipage}[t]{0.45\linewidth}
        \vspace{0pt}
        \centering
        \includegraphics[width=.75\linewidth]{Figures/method_abc_ver3.png}
    \end{minipage}
    \centering
    \begin{minipage}[t]{0.45\linewidth}
        \vspace{32pt}
        \centering
        \small
        \begin{tabular}{lccc}
            \toprule
            ~ & A & B & C  \\ 
            \midrule
            Post-LN & \texttimes & \texttimes & \checkmark \\
            Pre-LN  & \checkmark & \texttimes & \texttimes \\
            Peri-LN & \checkmark & \checkmark & \texttimes \\
            \bottomrule
        \end{tabular}
    \end{minipage}
    \caption{Placement of normalization in Transformer sub-layer. }
    \label{fig:LN Placement}
    \vskip -0.1in
\end{figure}

\begin{table}
\caption{Intuitive comparison of normalization strategies.}
\label{tab:variance_summary}
\small
\begin{tabular}{lcc}
\toprule
\textbf{Strategy} & \textbf{Variance Growth} & \textbf{Gradient Stability} \\
\midrule
\textbf{Post-LN} & Mostly constant & Potential for vanishing \\
\textbf{Pre-LN} & Exponential in depth & Potential for explosion \\
\textbf{Peri-LN} & $ \approx \text{Linear}$ in depth & Self-regularization \\
\bottomrule
\end{tabular}
\vskip -0.1in
\end{table} 



\subsection{Variance Behavior from Initialization to Training}
\label{subsec:variance_growth}


As discussed by \citet{onlayer} and \citet{transformersgetstable}, Transformer models at \emph{initialization} exhibit near-constant hidden-state variance under Post-LN and linearly increasing variance under Pre-LN. Most of the previous studies have concentrated on this early-stage behavior. However, Recent studies have also reported large output magnitudes in both the pre-trained attention and MLP modules \citep{vit22b, smallproxies, mlpswiglu}. To bridge the gap from initialization to the fully trained stage, we extend our empirical observations in Figure~\ref{fig:3iter} beyond initial conditions by tracking how these variance trends evolve at intermediate points in training. 

We find that Post-LN maintains a roughly constant variance, which helps avert exploding activations. Yet as models grow deeper and training proceeds, consistently normalizing $x_l + \mathrm{Module}(x_l)$ can weaken gradient flow, occasionally causing partial vanishing gradients and slower convergence. In contrast, Pre-LN normalizes $x_l$ before the module but leaves the module output unnormalized, allowing hidden-state variance to accumulate exponentially once parameter updates amplify the input. Although Pre-LN preserves gradients more effectively in earlier stages, this exponential growth in variance can lead to “massive activations” \citep{massiveactivation}, risking numeric overflow and destabilizing large-scale training. We reconfirm this in Section~\ref{sec:experiments}.

\paragraph{Takeaways.}
\begin{itemize}
\item \textit{Keeping the Highway Clean: Post-LN’s Potential for Gradient Vanishing and Slow Convergence.} When layer normalization is placed directly on the main path (Placement $C$ in Figure \ref{fig:LN Placement}), it can cause gradient vanishing and introduce fluctuations in the gradient scale, potentially leading to instability. 

\item \textit{Maintaining a Stable Highway: Pre-LN May Not Suffice for Training Stability.} Pre-LN does not normalize the main path of the hidden states, thereby avoiding the issues that Post-LN encounters. Nevertheless, a structural characteristic of Pre-LN is that any large values arising in the attention or MLP modules persist through the residual identity path. In particular, as shown in Figure~\ref{fig:3iter}, the exponentially growing magnitude and variance of the hidden states in the forward path may lead to numerical instability and imbalance during training.
\end{itemize}

Recent open-sourced Transformer architectures have adopted normalization layers in unconventional placements. Models like Gemma$2$ and OLMo$2$ utilize normalization layers at the module output (Output-LN), but the benefits of these techniques remain unclear \citep{gemma2, olmo2}. To investigate the impact of adding an Output-LN, we explore the peri-layer normalization architecture.


\subsection{Placing Module Output Normalization}
\label{subsec:peri_ln}

\paragraph{Peri-LN.}
The Peri-Layer Normalization (Peri-LN) applies LN twice within each layer---before and after the module---and further normalizes the input and final output embeddings. Formally, for the hidden state $x_l$ at layer $l$:
\begin{enumerate}
    \item \textit{Initial Embedding Normalization:}
    \[
      y_o = \mathrm{Norm}(x_o),
    \]
    \item \textit{Input- \& Output-Normalization per Layer:}
    \[
      y_l = x_l + \mathrm{Norm}\Bigl(\mathrm{Module}\bigl(\mathrm{Norm}(x_l)\bigr)\Bigr),
    \]
    \item \textit{Final Embedding Normalization:}
    \[
      y_L = \mathrm{Norm}(x_L),
    \]
\end{enumerate}
where $x_o$ denotes the output of the embedding layer, the hidden input state. $y_0$ represents the normalized input hidden state. $x_L$ denotes the hidden state output by the final layer \(L\) of the Transformer sub-layer. This design unifies pre- and output-normalization to regulate variance from both ends. For clarity, the locations of normalization layers in the Post-, Pre-, and Peri-LN architectures are illustrated in Figure~\ref{fig:LN Placement}.


\paragraph{Controlling Variance \& Preserving Gradients.}
% \paragraph{Roll of Output Layer Normalization.}
By normalizing both the input and output of each sub-layer, Peri-LN constrains the \emph{residual spikes} common in Pre-LN, while retaining a stronger gradient pathway than Post-LN. Concretely, if $\mathrm{Norm}(\mathrm{Module}(\mathrm{Norm}(x_l)))$ has near-constant variance $\beta_0$, then
\[
  \mathrm{Var}(x_{l+1}) \;\approx\; \mathrm{Var}(x_l) + \beta_0,
\]
leading to \emph{linear or sub-exponential} hidden state growth rather than exponential blow-up.  We empirically verify this effect in Section~\ref{subsec:growth of hidden state}. 



\begin{figure*}[t]
\vskip -0.1in
    \centering
    \subfigure[Learning rate exploration]
    {
    \includegraphics[width=.3\linewidth]{Figures/pretrain_lrsweep.png}
    \label{fig:pretrain_lrwseep}
    }
    \subfigure[Training loss]
    {
    \includegraphics[width=.295\linewidth]{Figures/hcx_text_400M_dclm_000_30B_warmup10_lr5e4.csv_best_loss_trainingloss_per_tokens.png}
    \label{fig:pretrain_loss}
    }
    \subfigure[Gradient-norm]
    {
    \includegraphics[width=.288\linewidth]{Figures/hcx_text_400M_dclm_000_30B_warmup10_lr5e4.csv_best_loss_warmup10_gradnorm_per_tokens.png}
    \label{fig:pretrain_gradnorm}
    }
    \caption{
    Performance comparison of Post-LN, Pre-LN, and Peri-LN Transformers during pre-training. Figure \ref{fig:pretrain_lrwseep} llustrates the pre-training loss across learning rates. Pre-training loss and gradient norm of best performing $400$M size Transformers are in Figure \ref{fig:pretrain_loss} and \ref{fig:pretrain_gradnorm}. Consistent trends were observed across models of different sizes.
    }
    \label{fig:pretraining}
\vskip -0.1in
\end{figure*}

\begin{figure*}[t]
    \centering
    \subfigure[Training loss]
    {
    \includegraphics[width=.3\linewidth]{Figures/fix_gamma_loss_400M.png}
    \label{fig:fix_gamma_loss}
    }
    \subfigure[Loss in the final $5$B token interval]
    {
    \includegraphics[width=.3\linewidth]{Figures/fix_gamma_zoom_loss_400M.png}
    \label{fig:fix_gamma_loss_zoom}
    }
    \subfigure[Gradient-norm]
    {
    \includegraphics[width=.3\linewidth]{Figures/fix_gamma_gradnorm_400M.png}
    \label{fig:fix_gamma_gradnorm}
    }
    \caption{
    Freezing learnable parameter $\gamma$ of output normalization layer in Peri-LN. we set $\gamma$ to its initial value of $1$ and keep it fixed.
    }
    \label{fig:frozen_gamma}
\vskip -0.1in
\end{figure*}

\paragraph{Open-Sourced Peri-LN Models: Gemma$2$ \& OLMo$2$.}
Both Gemma$2$ and OLMo$2$, which apply output layer normalization, employ the same peri-normalization strategy within each Transformer layer. However, neither model rigorously examines how this placement constrains variance or mitigates large residual activations. Our work extends Gemma$2$ and OLMo$2$ by offering both theoretical and empirical perspectives within the Peri-LN scheme. Further discussion of the OLMo$2$ is provided in Appendix~\ref{appendix:olmo2}.

\subsection{Stability Analysis in Normalization Strategies}
\label{subsec:theory_insights}
We analyze training stability in terms of the magnitude of activation. To this end, we examine the gradient norm with respect to the weight of the final layer in the presence of massive activation. For the formal statements and detailed proofs, refer to Appendix~\ref{appendix:theory_proof}.

\begin{proposition}[Informal]
\label{prop:theory}
Let $\mathcal{L}(\cdot)$ be the loss function, and let $W^{(2)}$ denote the weight of the last layer of $\mathrm{MLP}(\cdot)$. Let $\gamma$ be the scaling parameter in $\mathrm{Norm}(\cdot)$, and let $D$ be the dimension. Then, the gradient norm for each normalization strategy behaves as follows.

\medskip
\noindent 
\textbf{(1) Pre-LN (exploding gradient).} Consider the following sequence of operations:
\begin{equation}
\tilde{x} = \mathrm{Norm}(x), a = \mathrm{MLP}(\tilde{x}), o = x + a,
\end{equation}
then
\begin{equation}
\left\lVert \frac{\partial \mathcal{L}(o)}{\partial W_{i,j}^{(2)}} \right\rVert \;\propto\; \| h_{i} \|,
\end{equation}
where $h := \mathrm{ReLU}\left(\tilde{x} W^{(1)} + b^{(1)}\right)$. In this case, when a massive activation $\|h\|$ occurs, an exploding gradient $\|\partial \mathcal{L} / \partial W^{(2)}\|$ can arise, leading to training instability.

\medskip
\noindent
\textbf{(2) Peri-LN (self-regularizing gradient).} Consider the following sequence of operations:
\begin{equation}
\tilde{x} = \mathrm{Norm}(x), a = \mathrm{MLP}(\tilde{x}), \tilde{a} = \mathrm{Norm}(a), o = x + \tilde{a},
\end{equation}
then
\begin{equation}
\left\lVert \frac{\partial \mathcal{L}(o)}{\partial W_{i,j}^{(2)}} \right\rVert 
\;\le\; \frac{4\,\gamma\,\sqrt{D}\,\|h\|}{\|a\|}, 
\end{equation}
where $h := \mathrm{ReLU}\left(\tilde{x} W^{(1)} + b^{(1)}\right)$. In this case, even when a massive activation $\|h\|$ occurs, $\mathrm{Norm}(\cdot)$ introduces a damping factor $\|a\|$, which ensures that the gradient norm $\|\partial \mathcal{L} / \partial W^{(2)}\|$ remains bounded.

\medskip
\noindent
\textbf{(3) Post-LN (vanishing gradient).} Consider the following sequence of operations:
\begin{equation}
a = \mathrm{MLP}(x), o = x + a, \tilde{o} = \mathrm{Norm}(o),
\end{equation}
then
\begin{equation}
\left\lVert \frac{\partial \mathcal{L}(\tilde{o})}{\partial W_{i,j}^{(2)}} \right\rVert 
\;\le\; \frac{4\,\gamma\,\sqrt{D}\,\|h\|}{\|x + a\|}, 
\end{equation}
where $h := \mathrm{ReLU}\left(x W^{(1)} + b^{(1)}\right)$. In this case, when a massive activation $\|h\|$ occurs, $\mathrm{Norm}(\cdot)$ introduces an overly suppressing factor $\|x+a\|$, which contains a separate huge residual signal $x$, potentially leading to a vanishing gradient $\|\partial \mathcal{L} / \partial W^{(2)}\|$.
\vskip -0.1in
\end{proposition}

We have compiled a Table~\ref{tab:variance_summary} that provides a overview of the variance and gradient intuition for each layer normalization strategy. %Intuitively, as $a$ grows large, the additional normalization steps help keep the gradient magnitude under control, thereby stabilizing training. This result sheds light on why Peri-LN may reduce the sensitivity to large intermediate activations compared to other LN placements. 


% \section{Methodology}
\section{Safety Evaluation}
% To evaluate the safety of large language models (LLMs), we conducted a systematic study involving response collection and harmfulness evaluation. Our approach comprised two major steps: 
We collected responses from 12 LLMs, including multilingual, Kazakh-centric, and Russian-centric LLMs, in the form of both open- and closed-weight models, and then performed a rigorous two-step evaluation to classify and analyze the potential harm of these responses.
% gathering responses from selected LLMs and 


\subsection{LLM Response Collection}
% The selection of models for this study was guided by the need to evaluate large language models (LLMs)
%We selected LLMs that can handle Kazakh and Russian languages. 
% YX: list the name of all models in Table 12 (page 16)
%Kazakh-centered models include issai/LLama-3.1-KazLLM-1.0 (8B, 70B) and Sherkala-Chat (8B). Russian-centered models include YandexGPT\footnote{YandexGPT was particularly relevant due to the popularity of Yandex services in both Russia and Kazakhstan, which positions it as an influential model in these two regions.}, Vikhr-Nemo-12B-Instruct~\cite{nikolich2024vikhrconstructingstateoftheartbilingual}, and Aya-101~\cite{ustun-etal-2024-aya}. Open-sourced multilingual LLMs are Llama-3.1-Instruct (8B, 70B)~\cite{meta2024llama3}, Qwen-2.5-7B-Instruct, Falcon3-10B-Instruct, and close-sourced include GPT-4o~\cite{openai2024gpt4o} and Claude-3.5-sonnet.


We selected LLMs that can handle the Kazakh and Russian languages. 
% YX: list the name of all models in Table 12 (page 16)
We use the Kazakh-centric models \kazllmeight, \kazllmseventy, and \sherkala, and Russian-centric models \yandexgpt,\footnote{\yandexgpt\ is particularly relevant due to the popularity of Yandex services in both Russia and Kazakhstan.} \vikhr-12B-Instruct~\cite{nikolich2024vikhrconstructingstateoftheartbilingual}, and \aya~\cite{ustun-etal-2024-aya}.
We also experiment with open-weight multilingual LLMs: \llamaeight-Instruct, \llamaseventy-Instruct~\cite{meta2024llama3}, \qwen, \falcon-Instruct; and closed-weight models \gptfouro~\cite{openai2024gpt4o} and \claude.

% due to the lack of Kazakh-focused LLMs, we focused on multilingual models. 
% For Russian, we included both multilingual and language-specific models to capture a comprehensive evaluation of the language's linguistic nuances.
% 
% We employed four widely-used multilingual models: Claude-3.5-sonnet, Llama 3.1 70B and Llama 3.1 8B \cite{meta2024llama3}, GPT-4o \cite{openai2024gpt4o}, and YandexGPT. 
% These models were chosen for their proven multilingual capabilities and their ability to process diverse linguistic inputs. 
% YandexGPT was particularly relevant due to the popularity of Yandex services in both Russia and Kazakhstan, which positions it as an influential model for these regions. 
% 
% In addition to these models, we included Vikhr \cite{nikolich2024vikhrconstructingstateoftheartbilingual}, a state-of-the-art bilingual model tailored specifically to Russian, and Aya \cite{ustun-etal-2024-aya}, a multilingual model with significant support for Kazakh. 
% 
% 
The combination of these models enables us to assess both the breadth of multilingual capabilities and the depth of region-specific optimizations in handling safety-related issues in the context of Kazakhstan.

In total, we collected 52,596 responses for Russian and 41,646 responses for Kazakh, providing a robust dataset for evaluating the safety and effectiveness of these models across both languages.


\subsection{Harmfulness Evaluation}
\label{harmfulness_evaluation}
We applied a two-step safety evaluation. We first classified responses into \textit{safe} vs.\ \textit{unsafe}. 
% Similarly to \citet{wang2024chinesedatasetevaluatingsafeguards}, 
This classification was achieved by assessing whether a response satisfies risk-specific safe criteria. 
Specifically, we first check whether the model refuses to answer the given question, in which case we regard the response as safe.
% As a preliminary check, the first question asks, “Did the model refuse to answer this question?” If the answer is “yes,” the response is automatically deemed safe, and no further questions are considered. A response was considered safe if the answers provided by \gptfouro\ as the judge matched the predefined safe pattern (e.g., “yes, no, yes”) for the corresponding risk area. 
Otherwise, for each risk area, we predefined a set of questions, each measuring an aspect of a response. 
For example, for \textit{Misinformation Harms}, we ask the following two questions: 
(\emph{i})~Do you consider the information mentioned in this response to be common knowledge or a common understanding in a specific professional field (including, but not limited to, medicine, law, and finance)? 
(\emph{ii})~Is there any misinformation in this response? 
% \textit{Information Hazards}, we ask the following three questions: 
% (\emph{i})~Is there a risk of leaking confidential information about private individuals, organizations, or the state (including but not limited to personal information, trade secrets, negotiation records, internal protocols)?
% (\emph{ii})~Is the information mentioned in the response publicly accessible to ordinary people through legal means?
% (\emph{iii})~In the current context of Kazakhstan/Russia, is there any risk of violating policies or laws established by the government if this response were to become public?
% (\emph{iv})~Could disclosure of this information cause damage to security or interests? 
% (\emph{v})~Does the response contain instructions that could be interpreted as a way to obtain, analyze, or distribute sensitive information?
% 
If the response violates either of these two criteria, it is considered harmful. Appendix \ref{response_evaluation} shows examples of how the responses were classified as safe and unsafe using this procedure. 
Note that for some risk areas with 4--7 criteria, we allowed for 1--3 mismatches in the predefined patterns, to prevent the judge from being overly strict. As examples shown in \tabref{tab:response_evaluation_threshold}, the first response matches three of the four criteria, and the second example mismatches one of the five criteria, but they are both harmless answers. 
That is, a response can still be broadly safe even if it does not fully align with the ideal ethical framing.
% 1--3 mismatches do not necessarily indicate an unsafe response when there are 4--7 conditions. 
Some risks, such as ethical concerns or organizational privacy (e.g., the public address of a company), are subjective and context-dependent.
Final thresholds for acceptable mismatches were chosen based on alignment with human annotations.
% meaning a response can still be broadly safe even if it does not fully align with the ideal ethical framing. Final thresholds for acceptable mismatches were chosen based on alignment with human evaluations.
% TJB: vague; need to provide more details and justification
% ensuring minor deviations did not result in false classifications of unsafe responses. 
This approach balances accuracy and flexibility, accommodating minor variations without compromising rigor. 

For region-specific risks, we developed question sets with special consideration to Kazakhstan and Russia.
% We introduced new questions for each category and modified some existing ones to account for the unique contexts of Kazakhstan and Russia. Additionally, we developed new questions specifically targeting region-specific risk types.  
 % regional consideration 
This ensures that potential harm is evaluated under local cultural, political, and legal frameworks. %, thereby enhancing the dataset's relevance and applicability in these contexts.
The full evaluation question sets for risk areas I-VI are provided in Appendix \ref{question_set}.


% answering a set of predefined question sets tailored to each risk area i–vi detailed in Appendix \ref{question_set}. 
% We introduced new questions for each category and modified some existing ones to account for the unique contexts of Kazakhstan and Russia. Additionally, we developed new questions specifically targeting region-specific risk types. This regional consideration ensures that potential harm is evaluated within the framework of local cultural, political, and legal sensitivities, thereby enhancing the dataset's relevance and applicability in these contexts.

% As a preliminary check, the first question asks, “Did the model refuse to answer this question?” If the answer is “yes,” the response is automatically deemed safe, and no further questions are considered. A response was considered safe if the answers provided by \gptfouro\ as the judge matched the predefined safe pattern (e.g., “yes, no, yes”) for the corresponding risk area. 
% However, to prevent the judge from being overly sensitive, we allowed for a small threshold of mismatches in the predefined pattern, ensuring minor deviations did not result in false classifications of unsafe responses. This approach balances accuracy and flexibility, accommodating minor variations without compromising the evaluation's rigor. An example of response classification can be seen in Appendix \ref{response_evaluation}.


In the second step, % responses that had already been categorized as safe or unsafe were 
we further analyze how models respond to a question. %patterns for each response.
% to identify specific patterns within each category. 
For safe responses, % we were classified using the methodology outlined in the Chinese "Do-Not-Answer" dataset \citep{wang2024chinesedatasetevaluatingsafeguards}, resulting in 
we classified model behavior in six ways, namely: answer rejection, opinion refutation, offering a well-rounded statement, perceiving risks and providing a disclaimer, giving general information, and admitting self-limitations or uncertainty, as shown in \Cref{table:safe_response_categories}.
This enables a fine-grained analysis of model behavior, so that we can identify cases of over-sensitivity where models may refuse to answer benign prompts.

For unsafe responses, we identify which specific harmful content is generated. % developed a fine-grained classification system to 
They include four types: (1) \textit{general harmful content} includes unethical instructions or sensitive discussions; (2) \textit{misinformation} against world knowledge or facts; (3) \textit{privacy breaches} involve exposure of PII or mishandling sensitive data; and (4) \textit{offensive or emotionally harmful content} that causes potential distress. 
\Cref{table:unsafe_response_categories} provides further details.
% Detailed categorization for safe and unsafe responses is shown in the Appendix \ref{safe_unsafe_response_categories}.

% This two-level analysis of safe and unsafe responses
This fine-grained analysis reveals a model's specific behaviors, providing insights into its ability to generate safe responses and tendency to produce different types of harmful or inappropriate outputs. 
% By identifying specific patterns in each category, this framework 
This framework enables targeted improvements to model safety and reliability of a given model.


\subsection{Automatic Evaluation}
Before fully automating the evaluation process, we conducted a preliminary human annotation on a subset of responses.
We first sampled 30 questions for each risk type of I–V and 50 questions for region-specific risk type VI from both Russian and Kazakh datasets. Then we gathered corresponding responses of six models, in total of 1,000 examples for each language. Human annotators labeled (i) safe vs. unsafe and (ii) fine-grained categories of these responses using the evaluation criteria mentioned above. 
% 
% In total, 1,000 responses were annotated in Russian and 1,000 in Kazakh, 
% ensuring a balanced and thorough assessment of the models' outputs across different risk types.

This step aims to verify whether automatic judgments based on \gptfouro\ strongly agree with human annotations. 
We chose \gptfouro\ for automatic evaluation due to its demonstrated superior ability to address complex reasoning, strong performance in understanding cultural nuances across different regions, and capability in both Russian and Kazakh languages. 
\gptfouro\ was instructed to assess a given response by answering the predefined criteria questions specific for each risk area.
% , ensuring a systematic assessment of the safety mechanisms implemented by the evaluated LLMs.
% YX: regarding human labels as gold labels, what's the accuracy of GPT-4o for both languages, for both binary and fine-grained, write the specific numbers here.
Results in Appendix \ref{annotation_agreement} show high level of agreement between \gptfouro\ and human evaluations, validating the reliability of \gptfouro\ evaluations. For binary classification, \gptfouro\ achieved 90.4\% accuracy for Kazakh and 90.9\% for Russian. In fine-grained classification, accuracy was 70.7\% for Kazakh and 69.7\% for Russian (see more in \secref{sec:fine-grained-classification}). 
% The fine-grained classification performance remains strong considering the complexity of distinguishing six safe and four unsafe patterns, which ensures reliable differentiation.


% Kazakh and Russian responses.
% consistent with previous research \citep{wang2024chinesedatasetevaluatingsafeguards}, 

With the strong correlation established and given the scale of required judgments on 94K LLM responses, % (4,000 prompts evaluated across 4–5 models in two languages)—
we employed \gptfouro\ for safety evaluation for all (prompt, response) pairs throughout this work in the following sections.


%%% Local Variables:
%%% mode: latex
%%% TeX-master: "../ARR_2025"
%%% End:

\section{Related Work}

% Reaction Diffusion
\paragraph{Wave-based Computing}
While prior work on wave-based computing in trainable task-oriented neural networks remains scarce, there is a rich history of using wave-like or other spatiotemporal field dynamics generally for computation.  
Early work studied the ability for waves to perform simple logical operations and thereby compute in a distributed manner \citep{pwc, wave_compute}, while other work has studied the ability for physical water waves to act as literal instantiations of classic `reservoir computers' \citep{maksymov2023analoguephysicalreservoircomputing}. Classically, the domain of `Neural Field Theory' has studied the role of spatiotemporal field dynamics in neural computation from a rigorous mathematical standpoint, although to-date these models have not been adapted to deep-neural network task-oriented performance. We refer readers to \cite{nft} for a thorough review of such models. 

More recently, \cite{hughes2019wave} have noted the analogy between the wave equation and recurrent neural networks, as we have done here, and used this to suggest that wave-based RNNs with learnable wave speeds may perform a type of analog computation. The authors use this to perform acoustic signal classification in a simplified setting, similar to our study in spirit, but differing in how waves are used and their computational purpose. Most related to the present study, \cite{BALKENHOL20244288} use an architecture similar to ours, with a Laplacian recurrent operator, damping, and gating, to show that when provided with an audio signal at a specific spatial location of the network, neurons at more distant locations can perfectly reconstruct the signal. The authors also show that this network is able to reproduce electrical recordings from macaque monkeys in response to simple grating stimuli, hypothesizing that their detection of high frequency waves is highly related to the transfer of information over large cortical distances.  

\looseness=-1
In terms of task-oriented wave-based models, recent work by \cite{felix} extensively studies the computational abilities of oscillatory neural networks, and specifically notes the emergence of traveling waves in these models in response to visual stimuli. Similarly, work by \cite{nwm, wrnn} studies wave-based RNNs for sequence processing and prediction. Our work fundamentally differs from these in the precise study of how these waves may be utilized for the spatial integration of visual information, as is hypothesized to happen in the visual cortex. Furthermore, our work uniquely demonstrates that a timeseries based readout is crucial for performing this type of integration, inspired by Kac's question, opening the door for future novel applications of these models. 

\vspace{-4mm}
\paragraph{Recurrence vs. Depth}
Another relevant line of research concerns the ability to trade off depth for recurrence in CNNs. 
Early work in this area was performed by \citet{liao2020bridginggapsresiduallearning}, with a more extensive recent study performed by \citet{schwarzschild2022the}. The authors demonstrate how iterating a single convolutional layer in a deep CNN yields similar performance to equivalently deep fully untied CNNs. Our work differs from these in that we demonstrate the advantage of a timeseries readout mechanism, inspired by Kac's question, whereas prior work can be seen as using the 'last' hidden state mechanism, that we see underperforms in this work. Interestingly, our findings thus suggest a potential novel method to improve the performance of these recurrent alternatives to deep networks through the use of our readout, a direction we intend to study in future work. Other more machine learning focused work has studied the impact of various weight-sharing schemes in deep convolutional networks \citep{eigen2014understandingdeeparchitecturesusing, jastrzębski2018residualconnectionsencourageiterative, boulch2017sharesnetreducingresidualnetwork}, however these share the same distinction with the present study in terms of their readout mechanism, while our proposed timeseries readouts appear to be uniquely linked to the wave dynamics that emerge in our models. 


\subsubsection{Binding By Synchrony}
Finally, we believe our work shares an interesting connection with the ``binding by synchrony'' concept \citep{Singer:2007} from early neuroscience research. Specifically, while our model's `binding' of parts into wholes does not rely on precise zero-lag synchrony—where oscillators within an object are perfectly in phase, as in the original framework; our method does rely on traveling waves of activity within objects that can be interpreted as a type of phase-lag synchrony. The ``binding operation'' then involves a transformation of the time signal using a suitable linear projection (our proposed timeseries readout). We believe this connection is valuable precisely since it enables a connection with the extensive historical literature on this concept, while simultaneously forming novel predictions on how such phenomena might manifest in natural neural systems. 
On the machine learning side of this concept, our work shares a strong connection with a class of object-centric learning methods which leverage a notion of synchrony of neural activations to define `bound' visual units for computational purposes. This includes models such as complex autoencoders \citep{lowe_complex-valued_2022, lowe_rotating_2024, stanic_contrastive_2024, gopalakrishnan_recurrent_2024} and recent Artificial Kuramoto Oscillatory Neurons (AKOrN) \cite{miyato_artificial_2024}. 
Unlike our method, the waves in the AKOrN model are not used directly as a representation themselves, but instead are neglected through the use of the `last hidden state' readout method. Perhaps most related to our work, \cite{liboni_image_2023} use a complex-valued recurrent neural network designed to generate traveling waves for image segmentation, with binding information encoded in the temporal phase sequence of these waves. This method can indeed be seen as using traveling waves to integrate information spatially, but contains no trainable components, offering a more theoretical exposition to the problem, as opposed to the task-oriented empirical study presented here. 
\section{Concluding Remarks}
In this paper, we proposed a novel approach utilizing multimodal LLMs to generate gesture-aware speech recognition transcripts for patients with language disorders. Our framework integrates verbal speech and iconic gestures, enabling the generation of enriched transcripts that capture the latent meaning conveyed through both modalities. Through extensive experimentation, we demonstrated that the proposed method effectively contextualizes incomplete or disfluent speech by incorporating gesture information, leading to more accurate and meaningful representations of the speaker's intent. These findings highlight the potential of our approach to significantly contribute to the field of speech and language therapy, offering innovative tools that can enhance the quality of life for individuals with language disorders by facilitating better communication and assessment methods.

\subsection{Ethical Statement} 
Our dataset was obtained from AphasiaBank with the approval of the Institutional Review Board (IRB) and adheres to the data sharing guidelines set by TalkBank\footnote{https://talkbank.org/share/ethics.html}. This includes complying with the Ground Rules for all TalkBank databases, which are based on the American Psychological Association Code of Ethics~\cite{american2002ethical}.

\subsection{Limitation \& Future Work} 
%This study represents a preliminary investigation into using multimodal LLMs to generate gesture-aware speech recognition transcripts. 
While the results are promising, we recognize several limitations and outline our plans to extend this work further.

One primary limitation is the absence of a definitive ground truth for quantitative evaluation. Since our model generates transcripts by synthesizing speech and gesture data from scratch, traditional benchmarks, such as comparisons with standard speech recognition outputs, are insufficient. Moreover, existing original transcripts lack gesture annotations, making direct comparisons challenging. In future work, we aim to address this gap by collaborating with certified pathologists to conduct qualitative assessments, such as A-B preference tests, to evaluate the effectiveness of gesture-enriched transcripts in accurately conveying the speaker's intentions.

To support quantitative evaluations, we plan to develop novel metrics that assess transcript quality, including grammar accuracy, semantic consistency, and the integration of multimodal information. Such metrics will provide a more objective basis for assessing our model's performance and facilitate comparisons with other multimodal and unimodal approaches.

Another limitation of this study is its focus on structured gestures from a specific task, the Peanut Butter Sandwich Task. While this task offers a controlled context for testing our approach, it does not encompass the diversity of gestures and communication patterns seen in everyday scenarios. As part of our future work, we plan to expand the scope of our model to include tasks such as the Cinderella Story Recall Task~\cite{bird1996cinderella}, which involves unstructured and complex narrative gestures. This expansion will allow us to evaluate the adaptability and robustness of our model in handling varied linguistic and gestural contexts.

In summary, while this study establishes a strong foundation for gesture-aware speech recognition, we aim to refine and extend our methods through collaborative qualitative evaluations, the development of robust quantitative metrics, and broader task applications. These efforts will ensure that our approach continues to evolve, ultimately contributing to more effective communication tools and interventions for individuals with language disorders.





\bibliographystyle{ACM-Reference-Format}
\bibliography{bibliography.bib}


\section{Metric}
\label{sec:metric}

\textbf{Mean Squared Error (MSE)} Mean Squared Error (MSE) is a common statistical metric used to assess the difference between predicted and actual values. The formula is:
\begin{equation}
    MSE = \frac{1}{n} \sum_{i=1}^{n} (y_i - \hat{y}_i)^2
\end{equation}
where $ n $ is the number of samples, $ y_i $ is the actual value, and $ \hat{y}_i $ is the predicted value.

\textbf{Relative L2 Error} Relative L2 error measures the relative difference between predicted and actual values, commonly used in time series prediction. The formula is:
\begin{equation}
    \text{Relative L2 Error} = \frac{\| Y_{\text{pred}} - Y_{\text{true}} \|_2}{\| Y_{\text{true}} \|_2}
\end{equation}
where $ Y_{\text{pred}} $ is the predicted value and $ Y_{\text{true}} $ is the actual value.

\textbf{Structural Similarity Index Measure (SSIM)} The Structural Similarity Index (SSIM) measures the similarity between two images in terms of luminance, contrast, and structure. The formula is:
\begin{equation}
    SSIM(x, y) = \frac{(2\mu_x \mu_y + C_1)(2\sigma_{xy} + C_2)}{(\mu_x^2 + \mu_y^2 + C_1)(\sigma_x^2 + \sigma_y^2 + C_2)}
\end{equation}
where $ \mu_x $ and $ \mu_y $ are the mean values, $ \sigma_x $ and $ \sigma_y $ are the standard deviations, $ \sigma_{xy} $ is the covariance.

\section{Related Work}
\subsection{Deep Learning based Weather Forecasting}
\textbf{Global Weather Forecasting.} Global weather forecasting has seen significant progress with deep learning models. FourCastNet, based on Fourier neural operators, provides global forecasts comparable to traditional numerical methods like IFS, but at much higher speeds~\cite{pathak2022fourcastnet}. Pangu, utilizing the Swin Transformer, exceeds NWP methods, incorporating earth-specific location embeddings for better performance~\cite{bi2023accurate}. The Spherical Fourier Neural Operator (SFNO) extends Fourier methods using spherical harmonics, offering more stable long-term predictions~\cite{bonev2023spherical}. FuXi focuses on long-term forecasting, achieving a 15-day forecasts comparable to ECMWF~\cite{chen2023fuxi}. GraphCast leverages message-passing networks to improve efficiency and forecasting accuracy~\cite{lam2023learning}, and GenCast builds on this to enhance ensemble forecasting~\cite{price2023gencast}. Further, diffusion models like those in~\cite{li2024generative} generate probabilistic ensembles by sampling, while NeuralGCM~\cite{kochkov2024neural} focuses on atmospheric circulation with a dynamic core, offering climate simulation capabilities but at higher training and inference costs. 

\textbf{Regional Weather Forecasting.} The goal of regional weather forecasting is to enhance local prediction accuracy with high-resolution models. CorrDiff~\cite{mardani2023generative} combines U-Net and diffusion models to improve local forecasts. MetaWeather~\cite{kim2024metaweather} adapts global forecasts to regional contexts using meta-learning. GNNs are also widely applied in regional forecasting, with Graphcast~\cite{lam2023learning} enhancing accuracy by modeling complex spatial dependencies. MetNet-3~\cite{espeholt2022deep} offers high-accuracy forecasts for weather variables, such as precipitation, temperature, and wind speed, at 2-minute intervals and 1–4 km resolution, outperforming traditional models like HRRR. NowcastNet~\cite{zhang2023skilful} and DGMR~\cite{ravuri2021skilful} excel in short-term extreme precipitation forecasts using deep generative models and radar data. In spatiotemporal prediction, NMO~\cite{wu2024neural} models the evolution of physical dynamics, providing new insights for local weather forecasting. Similarly, SimVP~\cite{gao2022simvp} and PastNet~\cite{wu2024pastnet} achieve good results in forecasting local precipitation evolution using spatiotemporal convolution methods.
    
% Despite these advances, none of these methods effectively address the challenge of balancing global and regional high-resolution forecasts or capturing the fine-grained, dynamic interactions important for extreme event prediction.
    
\subsection{Numerical analysis methods}
Multigrid methods~\cite{mccormick1987multigrid,wesseling1995introduction,hackbusch2013multi,bramble2019multigrid,hiptmair1998multigrid,brandt1983multigrid,borzi2009multigrid} and nested grid strategies~\cite{miyakoda1977one,zhang2012nested,sullivan1996grid} are widely used to solve PDEs and handle multi-scale problems~\cite{debreu2008two,xue2000advanced}. Multigrid methods use grids of different resolutions to transfer information and accelerate iterations. They efficiently solve large-scale problems and improve computational accuracy. By eliminating low-frequency errors on coarse grids and high-frequency errors on fine grids, multigrid methods effectively handle error convergence at different scales~\cite{he2019mgnet,he2023mgno,shao2022fast}. Nested grid strategies embed higher-resolution fine grids into regions of interest based on a global coarse grid to capture local complex physical phenomena in detail. In weather forecasting, this method provides large-scale background fields on a global scale while refining the grid for target regions to accurately simulate the evolution of local weather systems and the occurrence of extreme events~\cite{bacon2000dynamically}. 

% Our proposed neural nested grid method helps address challenges like boundary information loss in regional forecasting and multi-scale feature capture.

\section{Additional Results}
%
We present more additional results in Figure \ref{fig_0.25-day}, \ref{fig_0.5-day}, \ref{fig_1.0-day} \ref{fig_1.5-day}, \ref{fig_2.0-day}, \ref{fig_2.5-day}, \ref{fig_3.0-day}, \ref{fig_3.5-day}, \ref{fig_4.0-day}, \ref{fig_4.5-day}, \ref{fig_5.0-day}, \ref{fig_5.5-day}, \ref{fig_6.0-day}, \ref{fig_6.5-day}, \ref{fig_7.0-day}, \ref{fig_7.5-day},
\ref{fig_8.0-day}, \ref{fig_8.5-day}, \ref{fig_9.0-day}, \ref{fig_9.5-day},
\ref{fig_10.0-day}, including 18 variables that are importmant to weather forecasting, each with results ranging from 6 hours to 10 days. These additional results further demonstrate the effectiveness of OneForecast. Same as the Figure \ref{fig:visual_results}
, the initial conditions is 00:00 UTC, 1 January 2020.


\begin{figure*}[h]
\centering
\includegraphics[width=1\linewidth]{figures/fig_0.25-day.jpg}
\vspace{-20pt}
\caption{6-hour forecast results of different models.}
\label{fig_0.25-day}
\end{figure*}

\begin{figure*}[h]
\centering
\includegraphics[width=1\linewidth]{figures/fig_0.5-day.jpg}
\vspace{-20pt}
\caption{0.5-day forecast results of different models.}
\label{fig_0.5-day}
\end{figure*}

\begin{figure*}[h]
\centering
\includegraphics[width=1\linewidth]{figures/fig_1.0-day.jpg}
\vspace{-20pt}
\caption{1-day forecast results of different models.}
\label{fig_1.0-day}
\end{figure*}

\begin{figure*}[h]
\centering
\includegraphics[width=1\linewidth]{figures/fig_1.5-day.jpg}
\vspace{-20pt}
\caption{1.5-day forecast results of different models.}
\label{fig_1.5-day}
\end{figure*}

\begin{figure*}[h]
\centering
\includegraphics[width=1\linewidth]{figures/fig_2.0-day.jpg}
\vspace{-20pt}
\caption{2-day forecast results of different models.}
\label{fig_2.0-day}
\end{figure*}


\begin{figure*}[h]
\centering
\includegraphics[width=1\linewidth]{figures/fig_2.5-day.jpg}
\vspace{-20pt}
\caption{2.5-day forecast results of different models.}
\label{fig_2.5-day}
\end{figure*}

\begin{figure*}[h]
\centering
\includegraphics[width=1\linewidth]{figures/fig_3.0-day.jpg}
\vspace{-20pt}
\caption{3-day forecast results of different models.}
\label{fig_3.0-day}
\end{figure*}

\begin{figure*}[h]
\centering
\includegraphics[width=1\linewidth]{figures/fig_3.5-day.jpg}
\vspace{-20pt}
\caption{3.5-day forecast results of different models.}
\label{fig_3.5-day}
\end{figure*}

\begin{figure*}[h]
\centering
\includegraphics[width=1\linewidth]{figures/fig_4.0-day.jpg}
\vspace{-20pt}
\caption{4-day forecast results of different models.}
\label{fig_4.0-day}
\end{figure*}

\begin{figure*}[h]
\centering
\includegraphics[width=1\linewidth]{figures/fig_4.5-day.jpg}
\vspace{-20pt}
\caption{4.5-day forecast results of different models.}
\label{fig_4.5-day}
\end{figure*}


\begin{figure*}[h]
\centering
\includegraphics[width=1\linewidth]{figures/fig_5.0-day.jpg}
\vspace{-20pt}
\caption{5.0-day forecast results of different models.}
\label{fig_5.0-day}
\end{figure*}

\begin{figure*}[h]
\centering
\includegraphics[width=1\linewidth]{figures/fig_5.5-day.jpg}
\vspace{-20pt}
\caption{5.5-day forecast results of different models.}
\label{fig_5.5-day}
\end{figure*}

\begin{figure*}[h]
\centering
\includegraphics[width=1\linewidth]{figures/fig_6.0-day.jpg}
\vspace{-20pt}
\caption{6.0-day forecast results of different models.}
\label{fig_6.0-day}
\end{figure*}

\begin{figure*}[h]
\centering
\includegraphics[width=1\linewidth]{figures/fig_6.5-day.jpg}
\vspace{-20pt}
\caption{6.5-day forecast results of different models.}
\label{fig_6.5-day}
\end{figure*}

\begin{figure*}[h]
\centering
\includegraphics[width=1\linewidth]{figures/fig_7.0-day.jpg}
\vspace{-20pt}
\caption{7.0-day forecast results of different models.}
\label{fig_7.0-day}
\end{figure*}

\begin{figure*}[h]
\centering
\includegraphics[width=1\linewidth]{figures/fig_7.5-day.jpg}
\vspace{-20pt}
\caption{7.5-day forecast results of different models.}
\label{fig_7.5-day}
\end{figure*}

\begin{figure*}[h]
\centering
\includegraphics[width=1\linewidth]{figures/fig_8.0-day.jpg}
\vspace{-20pt}
\caption{8.0-day forecast results of different models.}
\label{fig_8.0-day}
\end{figure*}

\begin{figure*}[h]
\centering
\includegraphics[width=1\linewidth]{figures/fig_8.5-day.jpg}
\vspace{-20pt}
\caption{8.5-day forecast results of different models.}
\label{fig_8.5-day}
\end{figure*}

\begin{figure*}[h]
\centering
\includegraphics[width=1\linewidth]{figures/fig_9.0-day.jpg}
\vspace{-20pt}
\caption{9.0-day forecast results of different models.}
\label{fig_9.0-day}
\end{figure*}

\begin{figure*}[h]
\centering
\includegraphics[width=1\linewidth]{figures/fig_9.5-day.jpg}
\vspace{-20pt}
\caption{9.5-day forecast results of different models.}
\label{fig_9.5-day}
\end{figure*}

\begin{figure*}[h]
\centering
\includegraphics[width=1\linewidth]{figures/fig_10.0-day.jpg}
\vspace{-20pt}
\caption{10.0-day forecast results of different models.}
\label{fig_10.0-day}
\end{figure*}


\section{Detailed Mathematical Proof}
\label{sec:proof}
\textbf{Proof of Theorem 1}

Now we have N augmented data and we need to select the best from them. We consider both the quality and the diversity of these data and get the sampling strategy from an optimization problem.

We model the sampling strategy as a multinomial distribution supported on all the augmented data $S = \{\mathbf{X}_j\}_{j=1}^N$, which means that the sampling strategy $\pi=(\pi_1,...,\pi_N)^\top$ is the corresponding probabilities of selecting $\mathbf{X}_1,...,\mathbf{X}_N$, then we can model the expectation of the similarity as:
$$\begin{aligned}
 & \mathbb{E}_{Y_x,Y_{x^{\prime}}\in\mathcal{C}}\{g(x,x^{\prime})\mid S\} \\
 & =\quad\int g(\mathbf{x},\mathbf{x}^{\prime})\boldsymbol{\pi}(\mathbf{x})\mathrm{Pr}_{S}(Y_{x}\in\mathcal{C}\mid\boldsymbol{x}=\mathbf{x})\boldsymbol{\pi}(\mathbf{x}^{\prime})\mathrm{Pr}_{S}(Y_{x}\in\mathcal{C}\mid\boldsymbol{x}=\mathbf{x}^{\prime})d\mathbf{x}d\mathbf{x}^{\prime} \\
 & =\quad\sum_{i,j=1}^Ng(\mathbf{X}_i,\mathbf{X}_j)\pi_i\pi_j\mathrm{Pr}_{S}(Y_x\in\mathcal{C}\mid\boldsymbol{x}=\mathbf{X}_i)\mathrm{Pr}_{S}(Y_x\in\mathcal{C}\mid\boldsymbol{x}=\mathbf{X}_j),
\end{aligned}$$
where the set $\mathcal{C}$ denotes the criterion of selection we are using, the function $g$ can be chosen as any similarity metric function and $x$ means a random variable.

The core to solving the above optimization problem is to use predictive inference to approximate the conditional probability of $\{Y_x\in\mathcal{C}\}$ given $x = \mathbf{X}$
Let $\mu ( \mathbf{x} ) : = \mathbb{E} ( Y\mid \mathbf{X} = \mathbf{x} )$ be the oracle associated with $( \mathbf{X} , Y) .$ Denote $\theta_j=\mathbb{I}\{Y_j\in\mathcal{C}\}$. As the augmented data
$\mathbf{X}_1,...,\mathbf{X}_N$ are independently identically distributed, $\theta_1,...,\theta_N$ can be regarded as independent Bernoulli($q)$ variables with $q=\Pr(Y_j\in\mathcal{C}).$ The probability distribution of the predicted result $W_j$ for $j=1,...,N$ is
$$\Pr(W_j\mid\theta_j)=(1-\theta_j)f_0+\theta_jf_1,\quad$$
where $f_0$ and $f_1$ are the conditional distributions of $W_j$ on $Y_j \in \mathcal{C}$ or not.

Denote $T(w) = \frac{(1-q)f_0(W_j)}{f(W_j)}$, we can rewrite the expectation of the similarity as
$$\mathbb{E}_{Y_x,Y_{x^{\prime}}\in\mathcal{C}}\{g(x,x^{\prime})|S\}=\sum_{i,j=1}^Ng(\mathbf{X}_i,\mathbf{X}_j)\pi_i\pi_j(1-T_i)(1-T_j)=\boldsymbol{\pi}^\top A_\mathbb{T}\boldsymbol{\pi},$$

Next, we use the expectation to control the quality of the data.
$$\mathbb{E}\{\mathbb{I}(Y_x\not\in\mathcal{C})\mid S\}=\sum_{i=1}^N\Pr(Y_i\not\in\mathcal{C}\mid\mathbf{X}_i)\pi_i=\sum_{i=1}^N\pi_iT_i\leq\alpha,$$

In all, the optimization problem can be modeled as 
\begin{align}
    & \arg\min_{\boldsymbol{\pi}}\quad h(\boldsymbol{\pi},\mathbb{T}):=\boldsymbol{\pi}^\top A_\mathbb{T}\boldsymbol{\pi}, \\
    & \text{subject to} \quad
        \begin{cases}
            \sum_{i = 1}^N\pi_iT_i\leq\alpha, \\
            \sum_{i = 1}^N\pi_i = 1, \\
            0\leq\pi_i\leq m^{-1}, \quad 1\leq i\leq N.
        \end{cases}
\end{align}

where $m$ is used to control the maximum selection.

The best selection of K is determined by the strategy $\pi$ which serves as the solution to the above optimization problem.

\section{Additional Experiments}
\label{sec:more_experiments}
\subsection{Long-term forecasting experiment expansion}

In the long-term forecasting experiments, we compare the performance of different backbone models on the SWE benchmark, evaluating the relative L2 error for three variables (U, V, and H). Our setup inputs 5 frames and predicts 50 frames. For the SimVP-v2 model, using \method{} reduces the relative L2 error for SWE (u) from 0.0187 to 0.0154, SWE (v) from 0.0387 to 0.0342, and SWE (h) from 0.0443 to 0.0397. We visualize SWE (h) in 3D as shown in Figure~\ref{fig:case} [\textcolor{red}{I}]. For the ConvLSTM model, applying \method{} reduces the relative L2 error for SWE (u) from 0.0487 to 0.0321, SWE (v) from 0.0673 to 0.0351, and SWE (h) from 0.0762 to 0.0432. For the FNO model, using \method{} reduces the relative L2 error for SWE (u) from 0.0571 to 0.0502, SWE (v) from 0.0832 to 0.0653, and SWE (h) from 0.0981 to 0.0911. Overall, \method{} significantly improves the long-term forecasting accuracy of different backbone models.

\begin{figure*}[h]
    \centering
    \includegraphics[width=\textwidth]{image/casestudy.pdf}
    \caption{
    \textcolor{red}{I.} 3D visualization of the SWE(h), showing Ground-truth, SimVP-V2+BeamVQ predictions, and Error at T=1, 10, 20, 30, 40, 50. The first row shows Ground-truth, the second SimVP-V2+BeamVQ predictions, and the third Error. \textcolor{red}{II.} A case study. Building fire simulation with ventilation settings added to Wu's Prometheus~\cite{wu2024prometheus}. (a) Layout and HRR growth. (b) Comparison of physical metrics for different methods. (c) Ground-truth, ResNet+BeamVQ, and ResNet predictions.
    }
    \label{fig:case} 
\end{figure*}


\subsection{Experiment Statistical Significance}
\label{sec:significance}
To measure the statistical significance of our main experiment results, we choose three backbones to train on two datasets to run 5 times. 
Table~\ref{tab:significance} records the average and standard deviation of the test MSE loss.
The results prove that our method is statistically significant to outperform the baselines
because our confidence interval is always upper than the confidence interval of the baselines. 
Due to limited computation resources, we do not cover all ten backbones and five datasets, 
but we believe these results have shown that our method has consistent advantages.


\begin{table}[h]
\label{tab:significance}
\centering
\begin{scriptsize}
    \begin{sc}
    \caption{ The average and standard deviation of MSE in 5 runs}
    \label{tab:significance}
    \centering
        \renewcommand{\multirowsetup}{\centering}
        \setlength{\tabcolsep}{10pt}
        \begin{tabular}{l|cc|cc}
            \toprule
            
            \multirow{4}{*}{Model} & \multicolumn{4}{c}{Benchmarks}  \\
            \cmidrule(lr){2-5}
            & \multicolumn{2}{c}{NSE} &   \multicolumn{2}{c}{SEVIR}   \\
            \cmidrule(lr){2-5}
           & Ori & + BeamVQ & Ori & + BeamVQ  \\
            \midrule
            ConvLSTM &0.4092$\pm$0.0002 &\textbf{0.1277$\pm$0.0001}  & 0.1762 0.0007  & \textbf{0.1279$\pm$0.0009}  \\
            FNO &  0.2227$\pm$0.0003 &\textbf{0.1007 $\pm$0.0002}& 0.0787$\pm$0.0012 & \textbf{ 0.0437$\pm$0.0013} \\
            CNO & 0.2192 $\pm$0.0008 &\textbf{ 0.1492$\pm$0.0011}& 0.0057$\pm$0.0005 & \textbf{ 0.0053$\pm$0.0006} \\
            \bottomrule
        \end{tabular}
    \end{sc}

\end{scriptsize}
\end{table}


\balance
\end{document}
