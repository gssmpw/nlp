\PassOptionsToPackage{dvipsnames}{xcolor}

\documentclass[10pt,sigconf,nonacm,screen]{acmart}
\usepackage[all]{nowidow}
\usepackage{xcolor}
\usepackage{subcaption}
\usepackage{hyperref}
\usepackage{acronym}
% \usepackage[cachedir=.minted]{minted}
% step 1
% \usepackage[finalizecache,cachedir=.]{minted}
% step 2
\usepackage[frozencache,cachedir=.]{minted}

\usepackage[capitalise,noabbrev]{cleveref}

\crefformat{section}{\S#2#1#3}
\crefformat{subsection}{\S#2#1#3}
\crefformat{subsubsection}{\S#2#1#3}
\crefrangeformat{section}{\S#3#1#4 to~\S#5#2#6}
\crefrangeformat{subsection}{\S#3#1#4 to~\S#5#2#6}
\crefrangeformat{subsubsection}{\S#3#1#4 to~\S#5#2#6}

\setlength{\marginparwidth}{2cm}

\newcommand{\sysname}{\textrm{LLM4FaaS}}

\hyphenation{ ma-the-in-der }

\begin{document}

\author{Minghe Wang}
\affiliation{%
    \institution{TU Berlin \& ECDF}
    \city{Berlin}
    \country{Germany}}
\email{mw@3s.tu-berlin.de}
\orcid{0009-0001-3780-5828}

\author{Tobias Pfandzelter}
\affiliation{%
    \institution{TU Berlin \& ECDF}
    \city{Berlin}
    \country{Germany}}
\email{tp@3s.tu-berlin.de}
\orcid{0000-0002-7868-8613}

\author{Trever Schirmer}
\affiliation{%
    \institution{TU Berlin \& ECDF}
    \city{Berlin}
    \country{Germany}}
\email{ts@3s.tu-berlin.de}
\orcid{0000-0001-9277-3032}

\author{David Bermbach}
\affiliation{%
    \institution{TU Berlin \& ECDF}
    \city{Berlin}
    \country{Germany}}
\email{db@3s.tu-berlin.de}
\orcid{0000-0002-7524-3256}

\title{LLM4FaaS: No-Code Application Development using LLMs and FaaS}
\keywords{Large Language Models, Function-as-a-Service, No-Code Development}

\begin{abstract}
        Large language models (LLMs) are powerful tools that can generate code from natural language descriptions.
        While this theoretically enables non-technical users to develop their own applications, they typically lack the expertise to execute, deploy, and operate generated code.
        This poses a barrier for such users to leverage the power of LLMs for application development.
        
        In this paper, we propose leveraging the high levels of abstraction of the Function-as-a-Service (FaaS) paradigm to handle code execution and operation for non-technical users.
        FaaS offers function deployment without handling the underlying infrastructure, enabling users to execute LLM-generated code without concern for its operation and without requiring any technical expertise.
        We propose \sysname{}, a novel no-code application development approach that combines LLMs and FaaS platforms to enable non-technical users to build and run their own applications using only natural language descriptions.
        Specifically, \sysname{} takes user prompts, uses LLMs to generate function code based on those prompts, and deploys these functions through a FaaS platform that handles the application's operation.
        \sysname{} also leverages the FaaS infrastructure abstractions to reduce the task complexity for the LLM, improving result accuracy.
        
        We evaluate \sysname{} with a proof-of-concept implementation based on \emph{GPT-4o} and an open-source FaaS platform, using real prompts from non-technical users.
        Our evaluation based on these real user prompts demonstrates the feasibility of our approach and shows that \sysname{} can reliably build and deploy code in 71.47\% of cases, up from 43.48\% in a baseline without FaaS.
\end{abstract}

\maketitle

\section{Introduction}

Video generation has garnered significant attention owing to its transformative potential across a wide range of applications, such media content creation~\citep{polyak2024movie}, advertising~\citep{zhang2024virbo,bacher2021advert}, video games~\citep{yang2024playable,valevski2024diffusion, oasis2024}, and world model simulators~\citep{ha2018world, videoworldsimulators2024, agarwal2025cosmos}. Benefiting from advanced generative algorithms~\citep{goodfellow2014generative, ho2020denoising, liu2023flow, lipman2023flow}, scalable model architectures~\citep{vaswani2017attention, peebles2023scalable}, vast amounts of internet-sourced data~\citep{chen2024panda, nan2024openvid, ju2024miradata}, and ongoing expansion of computing capabilities~\citep{nvidia2022h100, nvidia2023dgxgh200, nvidia2024h200nvl}, remarkable advancements have been achieved in the field of video generation~\citep{ho2022video, ho2022imagen, singer2023makeavideo, blattmann2023align, videoworldsimulators2024, kuaishou2024klingai, yang2024cogvideox, jin2024pyramidal, polyak2024movie, kong2024hunyuanvideo, ji2024prompt}.


In this work, we present \textbf{\ours}, a family of rectified flow~\citep{lipman2023flow, liu2023flow} transformer models designed for joint image and video generation, establishing a pathway toward industry-grade performance. This report centers on four key components: data curation, model architecture design, flow formulation, and training infrastructure optimization—each rigorously refined to meet the demands of high-quality, large-scale video generation.


\begin{figure}[ht]
    \centering
    \begin{subfigure}[b]{0.82\linewidth}
        \centering
        \includegraphics[width=\linewidth]{figures/t2i_1024.pdf}
        \caption{Text-to-Image Samples}\label{fig:main-demo-t2i}
    \end{subfigure}
    \vfill
    \begin{subfigure}[b]{0.82\linewidth}
        \centering
        \includegraphics[width=\linewidth]{figures/t2v_samples.pdf}
        \caption{Text-to-Video Samples}\label{fig:main-demo-t2v}
    \end{subfigure}
\caption{\textbf{Generated samples from \ours.} Key components are highlighted in \textcolor{red}{\textbf{RED}}.}\label{fig:main-demo}
\end{figure}


First, we present a comprehensive data processing pipeline designed to construct large-scale, high-quality image and video-text datasets. The pipeline integrates multiple advanced techniques, including video and image filtering based on aesthetic scores, OCR-driven content analysis, and subjective evaluations, to ensure exceptional visual and contextual quality. Furthermore, we employ multimodal large language models~(MLLMs)~\citep{yuan2025tarsier2} to generate dense and contextually aligned captions, which are subsequently refined using an additional large language model~(LLM)~\citep{yang2024qwen2} to enhance their accuracy, fluency, and descriptive richness. As a result, we have curated a robust training dataset comprising approximately 36M video-text pairs and 160M image-text pairs, which are proven sufficient for training industry-level generative models.

Secondly, we take a pioneering step by applying rectified flow formulation~\citep{lipman2023flow} for joint image and video generation, implemented through the \ours model family, which comprises Transformer architectures with 2B and 8B parameters. At its core, the \ours framework employs a 3D joint image-video variational autoencoder (VAE) to compress image and video inputs into a shared latent space, facilitating unified representation. This shared latent space is coupled with a full-attention~\citep{vaswani2017attention} mechanism, enabling seamless joint training of image and video. This architecture delivers high-quality, coherent outputs across both images and videos, establishing a unified framework for visual generation tasks.


Furthermore, to support the training of \ours at scale, we have developed a robust infrastructure tailored for large-scale model training. Our approach incorporates advanced parallelism strategies~\citep{jacobs2023deepspeed, pytorch_fsdp} to manage memory efficiently during long-context training. Additionally, we employ ByteCheckpoint~\citep{wan2024bytecheckpoint} for high-performance checkpointing and integrate fault-tolerant mechanisms from MegaScale~\citep{jiang2024megascale} to ensure stability and scalability across large GPU clusters. These optimizations enable \ours to handle the computational and data challenges of generative modeling with exceptional efficiency and reliability.


We evaluate \ours on both text-to-image and text-to-video benchmarks to highlight its competitive advantages. For text-to-image generation, \ours-T2I demonstrates strong performance across multiple benchmarks, including T2I-CompBench~\citep{huang2023t2i-compbench}, GenEval~\citep{ghosh2024geneval}, and DPG-Bench~\citep{hu2024ella_dbgbench}, excelling in both visual quality and text-image alignment. In text-to-video benchmarks, \ours-T2V achieves state-of-the-art performance on the UCF-101~\citep{ucf101} zero-shot generation task. Additionally, \ours-T2V attains an impressive score of \textbf{84.85} on VBench~\citep{huang2024vbench}, securing the top position on the leaderboard (as of 2025-01-25) and surpassing several leading commercial text-to-video models. Qualitative results, illustrated in \Cref{fig:main-demo}, further demonstrate the superior quality of the generated media samples. These findings underscore \ours's effectiveness in multi-modal generation and its potential as a high-performing solution for both research and commercial applications.
\section{Background} \label{section:LLM}

% \subsection{Large Language Model (LLM)}   

Figure~\ref{fig:LLaMA_model}(a) shows that a decoder-only LLM initially processes a user prompt in the “prefill” stage and subsequently generates tokens sequentially during the “decoding” stage.
Both stages contain an input embedding layer, multiple decoder transformer blocks, an output embedding layer, and a sampling layer.
Figure~\ref{fig:LLaMA_model}(b) demonstrates that the decoder transformer blocks consist of a self attention and a feed-forward network (FFN) layer, each paired with residual connection and normalization layers. 

% Differentiate between encoder/decoder, explain why operation intensity is low, explain the different parts of a transformer block. Discuss Table II here. 

% Explain the architecture with Llama2-70B.

% \begin{table}[thb]
% \renewcommand\arraystretch{1.05}
% \centering
% % \vspace{-5mm}
%     \caption{ML Model Parameter Size and Operational Intensity}
%     \vspace{-2mm}
%     \small
%     \label{tab:ML Model Parameter Size and Operational Intensity}    
%     \scalebox{0.95}{
%         \begin{tabular}{|c|c|c|c|c|}
%             \hline
%             & Llama2 & BLOOM & BERT & ResNet \\
%             Model & (70B) & (176B) & & 152 \\
%             \hline
%             Parameter Size (GB) & 140 & 352 & 0.17 & 0.16 \\
%             \hline
%             Op Intensity (Ops/Byte) & 1 & 1 & 282 & 346 \\
%             \hline
%           \end{tabular}
%     }
% \vspace{-3mm}
% \end{table}

% {\fontsize{8pt}{11pt}\selectfont 8pt font size test Memory Requirement}

\begin{figure}[t]
    \centering
    \includegraphics[width=8cm]{Figure/LLaMA_model_new_new.pdf}
    \caption{(a) Prefill stage encodes prompt tokens in parallel. Decoding stage generates output tokens sequentially.
    (b) LLM contains N$\times$ decoder transformer blocks. 
    (c) Llama2 model architecture.}
    \label{fig:LLaMA_model}
\end{figure}

Figure~\ref{fig:LLaMA_model}(c) demonstrates the Llama2~\cite{touvron2023llama} model architecture as a representative LLM.
% The self attention layer requires three GEMVs\footnote{GEMVs in multi-head attention~\cite{attention}, narrow GEMMs in grouped-query attention~\cite{gqa}.} to generate query, key and value vectors.
In the self-attention layer, query, key and value vectors are generated by multiplying input vector to corresponding weight matrices.
These matrices are segmented into multiple heads, representing different semantic dimensions.
The query and key vectors go though Rotary Positional Embedding (RoPE) to encode the relative positional information~\cite{rope-paper}.
Within each head, the generated key and value vectors are appended to their caches.
The query vector is multiplied by the key cache to produce a score vector.
After the Softmax operation, the score vector is multiplied by the value cache to yield the output vector.
The output vectors from all heads are concatenated and multiplied by output weight matrix, resulting in a vector that undergoes residual connection and Root Mean Square layer Normalization (RMSNorm)~\cite{rmsnorm-paper}.
The residual connection adds up the input and output vectors of a layer to avoid vanishing gradient~\cite{he2016deep}.
The FFN layer begins with two parallel fully connections, followed by a Sigmoid Linear Unit (SiLU), and ends with another fully connection.
\section{Architecture}
\label{sec:architecture}

\begin{figure}
    \centering
    \includegraphics[width=0.92\linewidth]{graphs/arch.pdf}
    \caption{
        \sysname{} proposes an LLM-based no-code application development framework using FaaS for infrastructure abstraction.
        The prompt constructor combines a user's application description with a system prompt for an LLM that generates application code.
        The function deployer uses that code to deploy a FaaS function on a FaaS platform.
    }
    \label{fig:arch}
\end{figure}

LLMs are excellent tools for transforming natural language software descriptions into executable code, but are by themselves unable to deploy and operate that code for users.
FaaS platforms can deploy small pieces of code as scalable, managed applications.
With \sysname{}, we propose combining these two technologies into an end-to-end no-code application development platform.
Our goal is to let non-technical users, i.e., individuals without experience in software development or operation, provide application descriptions in natural language and build fully-managed applications from those descriptions.
Examples for such applications can be found, e.g., in the context of smart home automation, simple extensions of enterprise applications, or custom information aggregation from news websites, social media, and web APIs~\cite{paper_bermbach2020_webapibenchmarking2}.
To support such applications, we design \sysname{} as shown in \cref{fig:arch}.

\sysname{} comprises three main components: an LLM for generating user-specified code, a FaaS platform for efficient function deployment, and a bridge that orchestrates prompt construction and function deployment.
Users provide their natural language application descriptions to \sysname{}, which combines them with a static system prompt in a \emph{prompt constructor}.
This structured prompt instructs an LLM to generate code based on the natural language description, including, e.g., details on programming language, application context, API references, and runtime environment.
\sysname{} then parses the LLM's answer for code in a \emph{function deployer}.
This generated code is deployed on the FaaS platform which abstracts the underlying application infrastructure complexities by providing containerized, auto-scaling environments for on-demand execution.


\section{Experiments}
\label{sec:experiments}



\begin{figure*}[t]
    \centering
    \includegraphics[width=1\linewidth]{images/Environments.pdf} 
    % \vspace{-20pt}
    \captionsetup{
    width=\textwidth,
    font=Smallfont,
    labelfont=Smallfont,
    textfont=Smallfont
    }
    \captionsetup{
    width=\textwidth,
    font=Smallfont,
    labelfont=Smallfont,
    textfont=Smallfont
    }
    \caption{Four different real-world experiment environments.}
    \label{fig:environments}
    % \vspace{-6pt}
\end{figure*}

\begin{figure*}[t]
    \centering
    \captionsetup{
    width=\textwidth,
    font=Smallfont,
    labelfont=Smallfont,
    textfont=Smallfont
    }
    % Top-left subfigure
    \begin{subfigure}[b]{0.45\textwidth}
        \centering
        \includegraphics[width=\textwidth]{images/fig_office.pdf}
        \caption{Office}
        \label{fig:subfig1}
    \end{subfigure}
    \hspace{0.02\textwidth}
    % Top-right subfigure
    \begin{subfigure}[b]{0.45\textwidth}
        \centering
        \includegraphics[width=\textwidth]{images/fig_apt.pdf}
        \caption{Apartment}
        \label{fig:subfig2}
    \end{subfigure}

    \vskip\baselineskip

    % Bottom-left subfigure
    \begin{subfigure}[b]{0.45\textwidth}
        \centering
        \includegraphics[width=\textwidth]{images/fig_outdoor.pdf}
        \caption{Outdoor}
        \label{fig:subfig3}
    \end{subfigure}
    \hspace{0.02\textwidth}
    % Bottom-right subfigure
    \begin{subfigure}[b]{0.45\textwidth}
        \centering
        \includegraphics[width=\textwidth]{images/fig_hallway.pdf}
        \caption{Hallway}
        \label{fig:subfig4}
    \end{subfigure}

    \caption{Top-down view of the trajectories comparison on the value maps with the detection results across the four different environments.}
    \label{fig:value_map}
\end{figure*}


\begin{table*}[ht]
\captionsetup{
    width=\textwidth,
    font=Smallfont,
    labelfont=Smallfont,
    textfont=Smallfont
    }
\caption{Vision-language navigation performance in 4 unseen environments (SR and SPL).}
\label{SOTAResults}
\centering
\begin{tabular}{lcccc|cccc}
\toprule
\multirow{2}{*}{\textbf{Method}} & \multicolumn{4}{c}{\textbf{SR (\%)}} & \multicolumn{4}{c}{\textbf{SPL}} \\
\cmidrule(lr){2-5} \cmidrule(lr){6-9}
 & Hallway & Office & Apartment & Outdoor & Hallway & Office & Apartment & Outdoor \\
\midrule
\textbf{Frontier Exploration}  
  & 40.0 & 41.7 & 55.6 & 33.3  
  & 0.239 & 0.317 & 0.363 & 0.189 \\

\textbf{VLFM} \cite{yokoyama2024vlfm}                 
  & 53.3 & 75.0 & 66.7 & 44.4  
  & 0.366 & 0.556 & 0.412 & 0.308 \\

\textbf{VL-Nav w/o IBTP}      
  & 66.7 & 83.3 & \underline{70.2} & \underline{55.6}  
  & 0.593 & 0.738 & 0.615 & \underline{0.573} \\

\textbf{VL-Nav w/o curiosity}      
  & \underline{73.3} & \underline{86.3} & 66.7 & \underline{55.6}  
  & \underline{0.612} & \underline{0.743} & \underline{0.631} & 0.498 \\

\textbf{VL-Nav}               
  & \textbf{86.7} & \textbf{91.7} & \textbf{88.9} & \textbf{77.8}  
  & \textbf{0.672} & \textbf{0.812} & \textbf{0.733} & \textbf{0.637} \\

\bottomrule
\end{tabular}
\end{table*}








\subsection{Experimental Setting}
\label{sec:experimental_setting}

We evaluate our approach in real-robot experiments against five methods: (1) classical frontier-based exploration, (2) VLFM \cite{yokoyama2024vlfm}, (3) VLNav without instance-based target points, (4) VLNav without curiosity terms, and (5) the full VLNav configuration. Because the original VLFM relies on BLIP-2 \cite{li2023blip}, which is too computationally heavy for real-time edge deployment, we use the YOLO-World \cite{cheng2024yolo} model instead to generate per-observation similarity scores for VLFM. Each method is tested under the same conditions to ensure a fair comparison of performance.

\paragraph{Environments:}
We consider four distinct environments (shown in \fref{fig:environments}), each with a specific combination of semantic complexity (\textit{High}, \textit{Medium}, or \textit{Low}) and size (\textit{Big}, \textit{Mid}, or \textit{Small}). Concretely, we use a Hallway (\textit{Medium \& Big}), an Office (\textit{High \& Mid}), an Outdoor area (\textit{Low \& Big}), and an Apartment (\textit{High \& Small}). In each environment, we evaluate five methods using three language prompts, yielding a diverse range of spatial layouts and semantic challenges. This setup provides a rigorous assessment of each method’s adaptability.

\paragraph{Language-Described Instance:}
We define nine distinct, uncommon human-described instances to serve as target objects or persons during navigation. Examples include phrases such as “tall white trash bin,” “there seems to be a man in white,” “find a man in gray,” “there seems to be a black chair,” “tall white board,” and “there seems to be a fold chair.” The variety in these descriptions ensures that the robot must rely on vision-language understanding to accurately locate these targets.

\noindent\textbf{Robots and Sensor Setup:} 
All experiments are conducted using a four-wheel Rover equipped with a Livox Mid-360 LiDAR. The LiDAR is tilted by approximately 23 degrees to the front to achieve a $\pm 30$ degrees vertical FOV coverage closely aligned with the forward camera’s view. An Intel RealSense D455 RGB-D camera, tilted upward by 7 degrees to detect taller objects, provides visual observation, though its depth data are not used for positioning or mapping. LiDAR measurements are a primary source of mapping and localization due to their higher accuracy. The whole VL-Nav system runs on an NVIDIA Jetson Orin NX on-board computer.




\subsection{Main Results}
\label{sec:main_results}

We validate the proposed VL-Nav system in real-robot experiments across four distinct environments (\textit{Hallway}, \textit{Office}, \textit{Apartment}, and \textit{Outdoor}), each featuring different semantic levels and sizes. Building on the motivation articulated in~\sref{sec:intro}, we focus on evaluating VL-Nav’s ability to (1) interpret fine-grained vision-language features and conduct robust VLN, (2) explore efficiently in unfamiliar spaces across various environments, and (3) run in real-time on resource-constrained platforms. \fref{fig:value_map} presents a top-down comparison of trajectories and detection results on the value map.

\begin{figure*}[t]
    \centering
    \includegraphics[width=1\linewidth]{images/result_plot.pdf} 
    % \vspace{-20pt}
    \captionsetup{
    width=\textwidth,
    font=Smallfont,
    labelfont=Smallfont,
    textfont=Smallfont
    }
    \caption{Plots of performance in different environments sizes and semantic comlexities.}
    \label{fig:results}
    % \vspace{-6pt}
\end{figure*}

\paragraph{Overall Performance:}
As reported in Table~\ref{SOTAResults}, our full \textbf{VL-Nav} consistently obtains the highest Success Rate (SR) and Success weighted by Path Length (SPL) across all four environments. In particular, VL-Nav outperforms classical exploration by a large margin, confirming the advantage of integrating CVL spatial reasoning with partial frontier-based search rather than relying solely on geometric exploration.

\paragraph{Effect of Instance-Based Target Points (IBTP):}
We note a marked improvement when enabling IBTP: the variant without IBTP lags behind, particularly in complex domains like the \textit{Apartment} and \textit{Office}. As discussed in \sref{sec:method}, IBTP allows VL-Nav to pursue and verify tentative detections with confidence above a threshold, mirroring human search behavior. This pragmatic mechanism prevents ignoring possible matches to the target description and reduces overall travel distance to confirm or discard candidate objects.

\paragraph{Curiosity Contributions:}
The \emph{curiosity Score} is also significant to VL-Nav’s performance. It merges two key components:
\begin{itemize}
    \item \textbf{Distance Weighting}: Preventing easily select very far way goals to reduce travel time and energy consumption which is extremely important for the efficiency (metrics SPL) in the large-size environments.
    \item \textbf{Unknown-Area Weighting}: Rewards navigation toward regions that yield more information.
\end{itemize}
Our ablations reveal that removing the distance-scoring element (\textit{VL-Nav w/o curiosity}) degrades both SR and SPL, particularly in the more cluttered environments. Meanwhile, dropping the instance-based target points (IBTP) similarly lowers performance, reflecting how each piece of CVL addresses a complementary aspect of semantic navigation.

\paragraph{Comparison to VLFM:}
Although the VLFM approach \cite{yokoyama2024vlfm} harnesses vision-language similarity value, it lacks the pixel-wise vision-language features, instance-based target points verification mechanism, and CVL-based spatial reasoning. Consequently, VL-Nav surpasses VLFM in both SR and SPL by effectively combining the pixel-wise vision language features and the curiosity cues via the CVL spatial reasoning. These gains are especially pronounced in semantic complex (\textit{Apartment}) and open-area (\textit{Outdoor}) environments, underscoring how our CVL spatial reasoning enhance vision-language navigation in complex settings and scenarios.



\paragraph{Summary of Findings:}
In conclusion, the experimental results confirm that VL-Nav delivers superior vision-language navigation across diverse, unseen real-world environments. By fusing frontier-based target points detection, instance-based target points, and the CVL spatial reasoning for goal selection, VL-Nav balances semantic awareness and exploration efficiency. The system’s robust performance, even in large or cluttered domains, highlights its potential as a practical solution for zero-shot vision-language navigation on low-power robots.

% !TEX root = ../main.tex

\section{Related work}
\label{sec:related_work}

\subsection{Visual unsupervised anomaly localization}

% In recent years the creation of the MVTec AD benchmark~\cite{mvtec} has given impetus to the development of new methods for visual unsupervised anomaly detection and localization. We review several main approaches which have representatives among top-5 methods on the localization track of the MVTec AD leaderboard
% The MVTec AD benchmark~\cite{mvtec}, developed in recent years, has been instrumental in propelling research towards new methods in visual unsupervised anomaly detection and localization.
In this section, we review several key approaches, each represented among the top five methods on the localization track of the MVTec AD benchmark~\cite{mvtec}, developed to stir progress in visual unsupervised anomaly detection and localization. 
% \footnote{\url{https://paperswithcode.com/sota/anomaly-detection-on-mvtec-ad}}.
% \paragraph{Synthetic anomalies} In unsupervised setting, real anomalies are either not present or not labeled in the training images. Some methods~\cite{memseg,mood_top1}, however, propose synthetic procedures that corrupt random regions in the images and train a segmentation model to predict the corrupted regions' masks.

\paragraph{Synthetic anomalies.} In unsupervised settings, real anomalies are typically absent or unlabeled in training images. To simulate anomalies, researchers synthetically corrupt random regions by replacing them with noise, random patterns from a special set~\cite{memseg}, or parts of other training images~\cite{mood_top1}. A segmentation model is trained to predict binary masks of corrupted regions, providing well-calibrated anomaly scores for individual pixels. While straightforward to train, these models may overfit to synthetic anomalies and struggle with real ones.
% . Unlabeled real anomalies in training images cannot be included in the binary masks, leading the model to predict zero scores for these regions and resulting in false negatives.

% One limitation of this approach is that the models may overfit to synthetic anomalies and generalize poorly to real anomalies. Another limitation is that training images may contain real anomalies which are unlabeled and cannot be included in the training binary masks. Thus, segmentation model is trained to predict zero scores for these regions which leads to false negatives.

% \paragraph{Reconstruction-based} Reconstruction-based methods build a generative model that takes an image $x$ as input and generates its normal (anomaly-free) version $\hat{x}$. Then anomaly scores are obtained as pixel-wise reconstruction errors between $x$ and $\hat{x}$. SotA methods from this family, e.g. DRAEM~\cite{draem}, DiffusionAD~\cite{diffusionad}, POUTA~\cite{pouta}, present a combination of reconstruction-based and synthetic-based approaches. First, they train a generative model to reconstruct synthetically corrupted image regions. Then, they train a segmentation model that takes a corrupted image and its reconstructed version as input and predicts the mask of the corrupted regions.

\paragraph{Reconstruction-based.} 
% In reconstruction-based methods, anomaly scores are obtained as reconstruction errors between the input image $x$ and generated normal (anomaly-free) counterpart $\hat{x}$.
% Reconstruction-based methods build a generative model that takes an image $x$ as input and generates its normal (anomaly-free) version $\hat{x}$. Then anomaly scores are obtained as reconstruction errors between $x$ and $\hat{x}$.
Trained solely on normal images, reconstruction-based approaches~\cite{autoencoder, vae, fanogan}, poorly reconstruct anomalous regions, allowing pixel-wise or feature-wise discrepancies to serve as anomaly scores. Later generative approaches~\cite{draem, diffusionad, pouta} integrate synthetic anomalies. The limitation stemming from anomaly-free train set assumption still persists -- if anomalous images are present, the model may learn to reconstruct anomalies as well as normal regions, undermining the ability to detect anomalies through differences between $x$ and $\hat{x}$.
% Early approaches, such as Autoencoders~\cite{autoencoder} and Variational Autoencoders~\cite{vae}, are trained solely on normal images. During inference, these models poorly reconstruct anomalous regions, allowing pixel-wise squared errors ${(x - \hat{x})^2}$ to serve as anomaly scores. Methods like f-AnoGAN~\cite{fanogan} enhance this by training W-GAN~\cite{wgan} $g$ to generate normal images and an encoder $f$ to map images to the GAN's latent space, ensuring ${\hat{x} = g(f(x)) \approx x}$. Anomalies are detected using a weighted average of reconstruction errors in pixel space and discrepancies in feature maps from GAN discriminator.

% State-of-the-art methods such as DRAEM~\cite{draem}, DiffusionAD~\cite{diffusionad}, and POUTA~\cite{pouta} integrate synthetic anomalies into the reconstruction process. They first train a generative model (autoencoder / diffusion model) to reconstruct synthetically corrupted regions. Then, they train a segmentation model that takes both the corrupted image and its reconstruction as input to predict masks of the corrupted regions.

% A major limitation of reconstruction-based methods is the assumption that the training set contains only normal images. If anomalous images are present, the generative model may learn to reconstruct anomalies as well as normal regions, undermining the ability to detect anomalies through differences between $x$ and $\hat{x}$.

% The earliest methods from this family are based on Autoencoder~\cite{autoencoder} or Variational Autoencoder~\cite{vae}, which are trained on anomaly-free images. At the inference stage, when it takes an image $x$ with anomalies it is intended to badly reconstruct the anomalous regions in $\hat{x}$, so that pixel-wise squared errors $(x - \hat{x})^2$ can be used as anomaly scores.

% Another method, f-AnoGAN~\cite{fanogan} at the first step trains W-GAN~\cite{wgan}, consisting of generator $g$ and discriminator $d$, to generate anomaly-free images $x \sim g(z)$ from latent variables $z \sim \mathcal{N}(0, I)$. Then, at the second step, it trains encoder $f$ to map anomaly-free images $x$ to the GAN's latent space, s.t. $\hat{x} = g(f(x)) \approx x$. At the inference stage, when $x$ is anomalous image, generator is assumed to generate its anomaly-free version $\hat{x}$, as it is trained only on normal images. Anomaly score are then obtained as a weighted average of reconstruction errors $(x - \hat{x})^2$ in pixel space and squared differences $(\varphi_d(x) - \varphi_d(x'))^2$ between feature maps $\varphi_d(x)$ and $\varphi_d(x')$ taken intermediate layers of GAN discriminator $d$.

% The SotA reconstruction-based methods, e.g. DRAEM~\cite{draem}, DiffusionAD~\cite{diffusionad}, POUTA~\cite{pouta}, present a combination with the approach based on synthetic anomalies. First, they train a generative model, e.g. autoencoder~\cite{draem,pouta} or diffusion model~\cite{diffusionad}, to reconstruct synthetically corrupted image regions. Then, they train a segmentation model that takes a corrupted image and its reconstructed version as input and predicts the mask of the corrupted regions.

% The main limitation of reconstruction-based methods is that they assume that training set does not contain anomalous images. Otherwise, generative model may learn to reconstruct anomalous regions as well as normal ones, which does not allow to detect anomalies by comparison of $x$ and $\hat{x}$.

\paragraph{Density-based.} Density-based methods for anomaly detection model the distribution of the training image patterns. As modeling of the joint distribution of raw pixel values is infeasible, these methods usually model the marginal or conditional distribution of pixel-wise deep feature vectors.

Some methods~\cite{ttr, pni} perform a non-parametric density estimation using memory banks. More scalable flow-based methods~\cite{fastflow,cflow,msflow}, leverage normalizing flows to assign low likelihoods to anomalies. From this family, we selected MSFlow as a representative baseline, because it is simpler than PNI, and yields similar top-5 results on the MVTec AD. 


\subsection{Medical unsupervised anomaly localization}
While there's no standard benchmark for pathology localization on CT images, MOOD~\cite{mood} offers a relevant benchmark with synthetic target anomalies. Unfortunately, at the time of preparing this work, the benchmark is closed for submissions, preventing us from evaluating our method on it. We include the top-performing method from MOOD~\cite{mood_top1} in our comparison, that relies on synthetic anomalies.

Other recognized methods for anomaly localization in medical images are reconstruction-based: variants of AE / VAE~\cite{autoencoder, dylov}, f-AnoGAN~\cite{fanogan}, and diffusion-based~\cite{latent_diffusion}. These approaches highly rely on the fact that the the training set consists of normal images only. However, it is challenging and costly to collect a large dataset of CT images of normal patients. While these methods work acceptable in the domain of 2D medical images and MRI, the capabilities of the methods have not been fully explored in a more complex CT data domain. We have adapted these methods to 3D.

\paragraph{Summary}
Our findings provide significant insights into the influence of correctness, explanations, and refinement on evaluation accuracy and user trust in AI-based planners. 
In particular, the findings are three-fold: 
(1) The \textbf{correctness} of the generated plans is the most significant factor that impacts the evaluation accuracy and user trust in the planners. As the PDDL solver is more capable of generating correct plans, it achieves the highest evaluation accuracy and trust. 
(2) The \textbf{explanation} component of the LLM planner improves evaluation accuracy, as LLM+Expl achieves higher accuracy than LLM alone. Despite this improvement, LLM+Expl minimally impacts user trust. However, alternative explanation methods may influence user trust differently from the manually generated explanations used in our approach.
% On the other hand, explanations may help refine the trust of the planner to a more appropriate level by indicating planner shortcomings.
(3) The \textbf{refinement} procedure in the LLM planner does not lead to a significant improvement in evaluation accuracy; however, it exhibits a positive influence on user trust that may indicate an overtrust in some situations.
% This finding is aligned with prior works showing that iterative refinements based on user feedback would increase user trust~\cite{kunkel2019let, sebo2019don}.
Finally, the propensity-to-trust analysis identifies correctness as the primary determinant of user trust, whereas explanations provided limited improvement in scenarios where the planner's accuracy is diminished.

% In conclusion, our results indicate that the planner's correctness is the dominant factor for both evaluation accuracy and user trust. Therefore, selecting high-quality training data and optimizing the training procedure of AI-based planners to improve planning correctness is the top priority. Once the AI planner achieves a similar correctness level to traditional graph-search planners, strengthening its capability to explain and refine plans will further improve user trust compared to traditional planners.

\paragraph{Future Research} Future steps in this research include expanding user studies with larger sample sizes to improve generalizability and including additional planning problems per session for a more comprehensive evaluation. Next, we will explore alternative methods for generating plan explanations beyond manual creation to identify approaches that more effectively enhance user trust. 
Additionally, we will examine user trust by employing multiple LLM-based planners with varying levels of planning accuracy to better understand the interplay between planning correctness and user trust. 
Furthermore, we aim to enable real-time user-planner interaction, allowing users to provide feedback and refine plans collaboratively, thereby fostering a more dynamic and user-centric planning process.


\bibliographystyle{ACM-Reference-Format}
\bibliography{bibliography.bib}

\section{Secure Token Pruning Protocols}
\label{app:a}
We detail the encrypted token pruning protocols $\Pi_{prune}$ in Figure \ref{fig:protocol-prune} and $\Pi_{mask}$ in Figure \ref{fig:protocol-mask} in this section.

%Optionally include supplemental material (complete proofs, additional experiments and plots) in appendix.
%All such materials \textbf{SHOULD be included in the main submission.}
\begin{figure}[h]
%vspace{-0.2in}
\begin{protocolbox}
\noindent
\textbf{Parties:} Server $P_0$, Client $P_1$.

\textbf{Input:} $P_0$ and $P_1$ holds $\{ \left \langle Att \right \rangle_{0}^{h}, \left \langle Att \right \rangle_{1}^{h}\}_{h=0}^{H-1} \in \mathbb{Z}_{2^{\ell}}^{n\times n}$ and $\left \langle x \right \rangle_{0}, \left \langle x \right \rangle_{1} \in \mathbb{Z}_{2^{\ell}}^{n\times D}$ respectively, where H is the number of heads, n is the number of input tokens and D is the embedding dimension of tokens. Additionally, $P_1$ holds a threshold $\theta \in \mathbb{Z}_{2^{\ell}}$.

\textbf{Output:} $P_0$ and $P_1$ get $\left \langle y \right \rangle_{0}, \left \langle y \right \rangle_{1} \in \mathbb{Z}_{2^{\ell}}^{n'\times D}$, respectively, where $y=\mathsf{Prune}(x)$ and $n'$ is the number of remaining tokens.

\noindent\rule{13.2cm}{0.1pt} % This creates the horizontal line
\textbf{Protocol:}
\begin{enumerate}[label=\arabic*:, leftmargin=*]
    \item For $h \in [H]$, $P_0$ and $P_1$ compute locally with input $\left \langle Att \right \rangle^{h}$, and learn the importance score in each head $\left \langle s \right \rangle^{h} \in \mathbb{Z}_{2^{\ell}}^{n} $, where $\left \langle s \right \rangle^{h}[j] = \frac{1}{n} \sum_{i=0}^{n-1} \left \langle Att \right \rangle^{h}[i,j]$.
    \item $P_0$ and $P_1$ compute locally with input $\{ \left \langle s \right \rangle^{i} \in \mathbb{Z}_{2^{\ell}}^{n}  \}_{i=0}^{H-1}$, and learn the final importance score $\left \langle S \right \rangle \in \mathbb{Z}_{2^{\ell}}^{n}$ for each token, where  $\left \langle S \right \rangle[i] = \frac{1}{H} \sum_{h=0}^{H-1} \left \langle s \right \rangle^{h}[i]$.
    \item  For $i \in [n]$, $P_0$ and $P_1$ invoke $\Pi_{CMP}$ with inputs  $\left \langle S \right \rangle$ and $ \theta $, and learn  $\left \langle M \right \rangle \in \mathbb{Z}_{2^{\ell}}^{n}$, such that$\left \langle M \right \rangle[i] = \Pi_{CMP}(\left \langle S \right \rangle[i] - \theta) $, where: \\
    $M[i] = \begin{cases}
        1  &\text{if}\ S[i] > \theta, \\
        0  &\text{otherwise}.
            \end{cases} $
    % \item If the pruning location is insensitive, $P_0$ and $P_1$ learn real mask $M$ instead of shares $\left \langle M \right \rangle$. $P_0$ and $P_1$ compute $\left \langle y \right \rangle$ with input $\left \langle x \right \rangle$ and $M$, where  $\left \langle x \right \rangle[i]$ is pruned if $M[i]$ is $0$.
    \item $P_0$ and $P_1$ invoke $\Pi_{mask}$ with inputs  $\left \langle x \right \rangle$ and pruning mask $\left \langle M \right \rangle$, and set outputs as $\left \langle y \right \rangle$.
\end{enumerate}
\end{protocolbox}
\setlength{\abovecaptionskip}{-1pt} % Reduces space above the caption
\caption{Secure Token Pruning Protocol $\Pi_{prune}$.}
\label{fig:protocol-prune}
\end{figure}




\begin{figure}[h]
\begin{protocolbox}
\noindent
\textbf{Parties:} Server $P_0$, Client $P_1$.

\textbf{Input:} $P_0$ and $P_1$ hold $\left \langle x \right \rangle_{0}, \left \langle x \right \rangle_{1} \in \mathbb{Z}_{2^{\ell}}^{n\times D}$ and  $\left \langle M \right \rangle_{0}, \left \langle M \right \rangle_{1} \in \mathbb{Z}_{2^{\ell}}^{n}$, respectively, where n is the number of input tokens and D is the embedding dimension of tokens.

\textbf{Output:} $P_0$ and $P_1$ get $\left \langle y \right \rangle_{0}, \left \langle y \right \rangle_{1} \in \mathbb{Z}_{2^{\ell}}^{n'\times D}$, respectively, where $y=\mathsf{Prune}(x)$ and $n'$ is the number of remaining tokens.

\noindent\rule{13.2cm}{0.1pt} % This creates the horizontal line
\textbf{Protocol:}
\begin{enumerate}[label=\arabic*:, leftmargin=*]
    \item For $i \in [n]$, $P_0$ and $P_1$ set $\left \langle M \right \rangle$ to the MSB of $\left \langle x \right \rangle$ and learn the masked tokens $\left \langle \Bar{x} \right \rangle \in Z_{2^{\ell}}^{n\times D}$, where
    $\left \langle \Bar{x}[i] \right \rangle = \left \langle x[i] \right \rangle + (\left \langle M[i] \right \rangle << f)$ and $f$ is the fixed-point precision.
    \item $P_0$ and $P_1$ compute the sum of $\{\Pi_{B2A}(\left \langle M \right \rangle[i]) \}_{i=0}^{n-1}$, and learn the number of remaining tokens $n'$ and the number of tokens to be pruned $m$, where $m = n-n'$.
    \item For $k\in[m]$, for $i\in[n-k-1]$, $P_0$ and $P_1$ invoke $\Pi_{msb}$ to learn the highest bit of $\left \langle \Bar{x}[i] \right \rangle$, where $b=\mathsf{MSB}(\Bar{x}[i])$. With the highest bit of $\Bar{x}[i]$, $P_0$ and $P_1$ perform a oblivious swap between $\Bar{x}[i]$ and $\Bar{x}[i+1]$:
    $\begin{cases}
        \Tilde{x}[i] = b\cdot \Bar{x}[i] + (1-b)\cdot \Bar{x}[i+1] \\
        \Tilde{x}[i+1] = b\cdot \Bar{x}[i+1] + (1-b)\cdot \Bar{x}[i]
    \end{cases} $ \\
    $P_0$ and $P_1$ learn the swapped token sequence $\left \langle \Tilde{x} \right \rangle$.
    \item $P_0$ and $P_1$ truncate $\left \langle \Tilde{x} \right \rangle$ locally by keeping the first $n'$ tokens, clear current MSB (all remaining token has $1$ on the MSB), and set outputs as $\left \langle y \right \rangle$.
\end{enumerate}
\end{protocolbox}
\setlength{\abovecaptionskip}{-1pt} % Reduces space above the caption
\caption{Secure Mask Protocol $\Pi_{mask}$.}
\label{fig:protocol-mask}
%\vspace{-0.2in}
\end{figure}

% \begin{wrapfigure}{r}{0.35\textwidth}  % 'r' for right, and the width of the figure area
%   \centering
%   \includegraphics[width=0.35\textwidth]{figures/msb.pdf}
%   \caption{Runtime of $\Pi_{prune}$ and $\Pi_{mask}$ in different layers. We compare different secure pruning strategies based on the BERT Base model.}
%   \label{fig:msb}
%   \vspace{-0.1in}
% \end{wrapfigure}

% \begin{figure}[h]  % 'r' for right, and the width of the figure area
%   \centering
%   \includegraphics[width=0.4\textwidth]{figures/msb.pdf}
%   \caption{Runtime of $\Pi_{prune}$ and $\Pi_{mask}$ in different layers. We compare different secure pruning strategies based on the BERT Base model.}
%   \label{fig:msb}
%   % \vspace{-0.1in}
% \end{figure}

\textbf{Complexity of $\Pi_{mask}$.} The complexity of the proposed $\Pi_{mask}$ mainly depends on the number of oblivious swaps. To prune $m$ tokens out of $n$ input tokens, $O(mn)$ swaps are needed. Since token pruning is performed progressively, only a small number of tokens are pruned at each layer, which makes $\Pi_{mask}$ efficient during runtime. Specifically, for a BERT base model with 128 input tokens, the pruning protocol only takes $\sim0.9$s on average in each layer. An alternative approach is to invoke an oblivious sort algorithm~\citep{bogdanov2014swap2,pang2023bolt} on $\left \langle \Bar{x} \right \rangle$. However, this approach is less efficient because it blindly sort the whole token sequence without considering $m$. That is, even if only $1$ token needs to be pruned, $O(nlog^{2}n)\sim O(n^2)$ oblivious swaps are needed, where as the proposed $\Pi_{mask}$ only need $O(n)$ swaps. More generally, for an $\ell$-layer Transformer with a total of $m$ tokens pruned, the overall time complexity using the sort strategy would be $O(\ell n^2)$ while using the swap strategy remains an overall complexity of $O(mn).$ Specifically, using the sort strategy to prune tokens in one BERT Base model layer can take up to $3.8\sim4.5$ s depending on the sorting algorithm used. In contrast, using the swap strategy only needs $0.5$ s. Moreover, alternative to our MSB strategy, one can also swap the encrypted mask along with the encrypted token sequence. However, we find that this doubles the number of swaps needed, and thus is less efficient the our MSB strategy, as is shown in Figure \ref{fig:msb}.

\section{Existing Protocols}
\label{app:protocol}
\noindent\textbf{Existing Protocols Used in Our Private Inference.}  In our private inference framework, we reuse several existing cryptographic protocols for basic computations. $\Pi_{MatMul}$ \citep{pang2023bolt} processes two ASS matrices and outputs their product in SS form. For non-linear computations, protocols $\Pi_{SoftMax}, \Pi_{GELU}$, and $\Pi_{LayerNorm}$\citep{lu2023bumblebee, pang2023bolt} take a secret shared tensor and return the result of non-linear functions in ASS. Basic protocols from~\citep{rathee2020cryptflow2, rathee2021sirnn} are also utilized. $\Pi_{CMP}$\citep{EzPC}, for example, inputs ASS values and outputs a secret shared comparison result, while $\Pi_{B2A}$\citep{EzPC} converts secret shared Boolean values into their corresponding arithmetic values.

\section{Polynomial Reduction for Non-linear Functions}
\label{app:b}
The $\mathsf{SoftMax}$ and $\mathsf{GELU}$ functions can be approximated with polynomials. High-degree polynomials~\citep{lu2023bumblebee, pang2023bolt} can achieve the same accuracy as the LUT-based methods~\cite{hao2022iron-iron}. While these polynomial approximations are more efficient than look-up tables, they can still incur considerable overheads. Reducing the high-degree polynomials to the low-degree ones for the less important tokens can imporve efficiency without compromising accuracy. The $\mathsf{SoftMax}$ function is applied to each row of an attention map. If a token is to be reduced, the corresponding row will be computed using the low-degree polynomial approximations. Otherwise, the corresponding row will be computed accurately via a high-degree one. That is if $M_{\beta}'[i] = 1$, $P_0$ and $P_1$ uses high-degree polynomials to compute the $\mathsf{SoftMax}$ function on token $x[i]$:
\begin{equation}
\mathsf{SoftMax}_{i}(x) = \frac{e^{x_i}}{\sum_{j\in [d]}e^{x_j}}
\end{equation}
where $x$ is a input vector of length $d$ and the exponential function is computed via a polynomial approximation. For the $\mathsf{SoftMax}$ protocol, we adopt a similar strategy as~\citep{kim2021ibert, hao2022iron-iron}, where we evaluate on the normalized inputs $\mathsf{SoftMax}(x-max_{i\in [d]}x_i)$. Different from~\citep{hao2022iron-iron}, we did not used the binary tree to find max value in the given vector. Instead, we traverse through the vector to find the max value. This is because each attention map is computed independently and the binary tree cannot be re-used. If $M_{\beta}[i] = 0$, $P_0$ and $P_1$ will approximate the $\mathsf{SoftMax}$ function with low-degree polynomial approximations. We detail how $\mathsf{SoftMax}$ can be approximated as follows:
\begin{equation}
\label{eq:app softmax}
\mathsf{ApproxSoftMax}_{i}(x) = \frac{\mathsf{ApproxExp}(x_i)}{\sum_{j\in [d]}\mathsf{ApproxExp}(x_j)}
\end{equation}
\begin{equation}
\mathsf{ApproxExp}(x)=\begin{cases}
    0  &\text{if}\ x \leq T \\
    (1+ \frac{x}{2^n})^{2^n} &\text{if}\ x \in [T,0]\\
\end{cases}
\end{equation}
where the $2^n$-degree Taylor series is used to approximate the exponential function and $T$ is the clipping boundary. The value $n$ and $T$ determines the accuracy of above approximation. With $n=6$ and $T=-13$, the approximation can achieve an average error within $2^{-10}$~\citep{lu2023bumblebee}. For low-degree polynomial approximation, $n=3$ is used in the Taylor series.

Similarly, $P_0$ or $P_1$ can decide whether or not to approximate the $\mathsf{GELU}$ function for each token. If $M_{\beta}[i] = 1$, $P_0$ and $P_1$ use high-degree polynomials~\citep{lu2023bumblebee} to compute the $\mathsf{GELU}$ function on token $x[i]$ with high-degree polynomial:
% \begin{equation}
% \mathsf{GELU}(x) = 0.5x(1+\mathsf{Tanh}(\sqrt{2/\pi}(x+0.044715x^3)))
% \end{equation}
% where the $\mathsf{Tanh}$ and square root function are computed via a OT-based lookup-table.

\begin{equation}
\label{eq:app gelu}
\mathsf{ApproxGELU}(x)=\begin{cases}
    0  &\text{if}\ x \leq -5 \\
    P^3(x), &\text{if}\ -5 < x \leq -1.97 \\
    P^6(x), &\text{if}\ -1.97 < x \leq 3  \\
    x, &\text{if}\ x >3 \\
\end{cases}
\end{equation}
where $P^3(x)$ and $P^6(x)$ are degree-3 and degree-6 polynomials respectively. The detailed coefficient for the polynomial is: 
\begin{equation*}
    P^3(x) = -0.50540312 -  0.42226581x - 0.11807613x^2 - 0.01103413x^3
\end{equation*}
, and
\begin{equation*}
    P^6(x) = 0.00852632 + 0.5x + 0.36032927x^2 - 0.03768820x^4 + 0.00180675x^6
\end{equation*}

For BOLT baseline, we use another high-degree polynomial to compute the $\mathsf{GELU}$ function.

\begin{equation}
\label{eq:app gelu}
\mathsf{ApproxGELU}(x)=\begin{cases}
    0  &\text{if}\ x < -2.7 \\
    P^4(x), &\text{if}\   |x| \leq 2.7 \\
    x, &\text{if}\ x >2.7 \\
\end{cases}
\end{equation}
We use the same coefficients for $P^4(x)$ as BOLT~\citep{pang2023bolt}.

\begin{figure}[h]
 % \vspace{-0.1in}
    \centering
    \includegraphics[width=1\linewidth]{figures/bumble.pdf}
    % \captionsetup{skip=2pt}
    % \vspace{-0.1in}
    \caption{Comparison with prior works on the BERT model. The input has 128 tokens.}
    \label{fig:bumble}
\end{figure}

If $M_{\beta}'[i] = 0$, $P_0$ and $P_1$ will use low-degree 
polynomial approximation to compute the $\mathsf{GELU}$ function instead. Encrypted polynomial reduction leverages low-degree polynomials to compute non-linear functions for less important tokens. For the $\mathsf{GELU}$ function, the following degree-$2$ polynomial~\cite{kim2021ibert} is used:
\begin{equation*}
\mathsf{ApproxGELU}(x)=\begin{cases}
    0  &\text{if}\ x <  -1.7626 \\
    0.5x+0.28367x^2, &\text{if}\ x \leq |1.7626| \\
    x, &\text{if}\ x > 1.7626\\
\end{cases}
\end{equation*}


\section{Comparison with More Related Works.}
\label{app:c}
\textbf{Other 2PC frameworks.} The primary focus of CipherPrune is to accelerate the private Transformer inference in the 2PC setting. As shown in Figure \ref{fig:bumble}, CipherPrune can be easily extended to other 2PC private inference frameworks like BumbleBee~\citep{lu2023bumblebee}. We compare CipherPrune with BumbleBee and IRON on BERT models. We test the performance in the same LAN setting as BumbleBee with 1 Gbps bandwidth and 0.5 ms of ping time. CipherPrune achieves more than $\sim 60 \times$ speed up over BOLT and $4.3\times$ speed up over BumbleBee.

\begin{figure}[t]
 % \vspace{-0.1in}
    \centering
    \includegraphics[width=1\linewidth]{figures/pumab.pdf}
    % \captionsetup{skip=2pt}
    % \vspace{-0.1in}
    \caption{Comparison with MPCFormer and PUMA on the BERT models. The input has 128 tokens.}
    \label{fig:pumab}
\end{figure}

\begin{figure}[h]
 % \vspace{-0.1in}
    \centering
    \includegraphics[width=1\linewidth]{figures/pumag.pdf}
    % \captionsetup{skip=2pt}
    % \vspace{-0.1in}
    \caption{Comparison with MPCFormer and PUMA on the GPT2 models. The input has 128 tokens. The polynomial reduction is not used.}
    \label{fig:pumag}
\end{figure}

\textbf{Extension to 3PC frameworks.} Additionally, we highlight that CipherPrune can be also extended to the 3PC frameworks like MPCFormer~\citep{li2022mpcformer} and PUMA~\citep{dong2023puma}. This is because CipherPrune is built upon basic primitives like comparison and Boolean-to-Arithmetic conversion. We compare CipherPrune with MPCFormer and PUMA on both the BERT and GPT2 models. CipherPrune has a $6.6\sim9.4\times$ speed up over MPCFormer and $2.8\sim4.6\times$ speed up over PUMA on the BERT-Large and GPT2-Large models.


\section{Communication Reduction in SoftMax and GELU.}
\label{app:e}

\begin{figure}[h]
    \centering
    \includegraphics[width=0.9\linewidth]{figures/layerwise.pdf}
    \caption{Toy example of two successive Transformer layers. In layer$_i$, the SoftMax and Prune protocol have $n$ input tokens. The number of input tokens is reduced to $n'$ for the Linear layers, LayerNorm and GELU in layer$_i$ and SoftMax in layer$_{i+1}$.}
    \label{fig:layer}
\end{figure}

\begin{table*}[h]
\captionsetup{skip=2pt}
\centering
\scriptsize
\caption{Communication cost (in MB) of the SoftMax and GELU protocol in each Transformer layer.}
\begin{tblr}{
    colspec = {c |c c c c c c c c c c c c},
    row{1} = {font=\bfseries},
    row{2-Z} = {rowsep=1pt},
    % row{4} = {bg=LightBlue},
    colsep = 2.5pt,
    }
\hline
\textbf{Layer Index} & \textbf{0}  & \textbf{1}  & \textbf{2} & \textbf{3} & \textbf{4} & \textbf{5} & \textbf{6} & \textbf{7} & \textbf{8} & \textbf{9} & \textbf{10} & \textbf{11} \\
\hline
Softmax & 642.19 & 642.19 & 642.19 & 642.19 & 642.19 & 642.19 & 642.19 & 642.19 & 642.19 & 642.19 & 642.19 & 642.19 \\
Pruned Softmax & 642.19 & 129.58 & 127.89 & 119.73 & 97.04 & 71.52 & 43.92 & 21.50 & 10.67 & 6.16 & 4.65 & 4.03 \\
\hline
GELU & 698.84 & 698.84 & 698.84 & 698.84 & 698.84 & 698.84 & 698.84 & 698.84 & 698.84 & 698.84 & 698.84 & 698.84\\
Pruned GELU  & 325.10 & 317.18 & 313.43 & 275.94 & 236.95 & 191.96 & 135.02 & 88.34 & 46.68 & 16.50 & 5.58 & 5.58\\
\hline
\end{tblr}
\label{tab:layer}
\end{table*}

{
In Figure \ref{fig:layer}, we illustrate why CipherPrune can reduce the communication overhead of both  SoftMax and GELU. Suppose there are $n$ tokens in $layer_i$. Then, the SoftMax protocol in the attention module has a complexity of $O(n^2)$. CipherPrune's token pruning protocol is invoked to select $n'$ tokens out of all $n$ tokens, where $m=n-n'$ is the number of tokens that are removed. The overhead of the GELU function in $layer_i$, i.e., the current layer, has only $O(n')$ complexity (which should be $O(n)$ without token pruning). The complexity of the SoftMax function in $layer_{i+1}$, i.e., the following layer, is reduced to $O(n'^2)$ (which should be $O(n^2)$ without token pruning). The SoftMax protocol has quadratic complexity with respect to the token number and the GELU protocol has linear complexity. Therefore, CipherPrune can reduce the overhead of both the GELU protocol and the SoftMax protocols by reducing the number of tokens. In Table \ref{tab:layer}, we provide detailed layer-wise communication cost of the GELU and the SoftMax protocol. Compared to the unpruned baseline, CipherPrune can effectively reduce the overhead of the GELU and the SoftMax protocols layer by layer.
}

\section{Analysis on Layer-wise redundancy.}
\label{app:f}

\begin{figure}[h]
    \centering
    \includegraphics[width=0.9\linewidth]{figures/layertime0.pdf}
    \caption{The number of pruned tokens and pruning protocol runtime in different layers in the BERT Base model. The results are averaged across 128 QNLI samples.}
    \label{fig:layertime}
\end{figure}

{
In Figure \ref{fig:layertime}, we present the number of pruned tokens and the runtime of the pruning protocol for each layer in the BERT Base model. The number of pruned tokens per layer was averaged across 128 QNLI samples, while the pruning protocol runtime was measured over 10 independent runs. The mean token count for the QNLI samples is 48.5. During inference with BERT Base, input sequences with fewer tokens are padded to 128 tokens using padding tokens. Consistent with prior token pruning methods in plaintext~\citep{goyal2020power}, a significant number of padding tokens are removed at layer 0.  At layer 0, the number of pruned tokens is primarily influenced by the number of padding tokens rather than token-level redundancy.
%In Figure \ref{fig:layertime}, we demonstrate the number of pruned tokens and the pruning protocol runtime in each layer in the BERT Base model. We averaged the number of pruned tokens in each layer across 128 QNLI samples and then tested the pruning protocol runtime in 10 independent runs. The mean token number of the QNLI samples is 48.5. During inference with BERT Base, input sequences with small token number are padded to 128 tokens with padding tokens. Similar to prior token pruning methods in the plaintext~\citep{goyal2020power}, a large number of padding tokens can be removed at layer 0. We remark that token-level redundancy builds progressively throughout inference~\citep{goyal2020power, kim2022LTP}. The number of pruned tokens in layer 0 mostly depends on the number of padding tokens instead of token-level redundancy.
}

{
%As shown in Figure \ref{fig:layertime}, more tokens are removed in the intermediate layers, e.g., layer $4$ to layer $7$. This suggests there is more redundant information in these intermediate layers. 
In CipherPrune, tokens are removed progressively, and once removed, they are excluded from computations in subsequent layers. Consequently, token pruning in earlier layers affects computations in later layers, whereas token pruning in later layers does not impact earlier layers. As a result, even if layers 4 and 7 remove the same number of tokens, layer 7 processes fewer tokens overall, as illustrated in Figure \ref{fig:layertime}. Specifically, 8 tokens are removed in both layer $4$ and layer $7$, but the runtime of the pruning protocol in layer $4$ is $\sim2.4\times$ longer than that in  layer $7$.
}

\section{Related Works}
\label{app:g}

{
In response to the success of Transformers and the need to safeguard data privacy, various private Transformer Inferences~\citep{chen2022thex,zheng2023primer,hao2022iron-iron,li2022mpcformer, lu2023bumblebee, luo2024secformer, pang2023bolt}  are proposed. To efficiently run private Transformer inferences, multiple cryptographic primitives are used in a popular hybrid HE/MPC method IRON~\citep{hao2022iron-iron}, i.e., in a Transformer, HE and SS are used for linear layers, and SS and OT are adopted for nonlinear layers. IRON and BumbleBee~\citep{lu2023bumblebee} focus on optimizing linear general matrix multiplications; SecFormer~\cite{luo2024secformer} improves the non-linear operations like the exponential function with polynomial approximation; BOLT~\citep{pang2023bolt} introduces the baby-step giant-step (BSGS) algorithm to reduce the number of HE rotations, proposes a word elimination (W.E.) technique, and uses polynomial approximation for non-linear operations, ultimately achieving state-of-the-art (SOTA) performance.
}

{Other than above hybrid HE/MPC methods, there are also works exploring privacy-preserving Transformer inference using only HE~\citep{zimerman2023converting, zhang2024nonin}. The first HE-based private Transformer inference work~\citep{zimerman2023converting} replaces \mysoftmax function with a scaled-ReLU function. Since the scaled-ReLU function can be approximated with low-degree polynomials more easily, it can be computed more efficiently using only HE operations. A range-loss term is needed during training to reduce the polynomial degree while maintaining high accuracy. A training-free HE-based private Transformer inference was proposed~\citep{zhang2024nonin}, where non-linear operations are approximated by high-degree polynomials. The HE-based methods need frequent bootstrapping, especially when using high-degree polynomials, thus often incurring higher overhead than the hybrid HE/MPC methods in practice.
}


\balance
\end{document}
