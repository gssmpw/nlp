

\section{Overview}
\label{sec:label}

\methodname guides an LLM through the process of developing a library of procedural functions that matches an input design intent.
In our problem framing, we assume that a user has a procedural modeling domain in mind (e.g., a particular category of shapes).
The user will communicate their design intent to our system, which is then tasked with producing a fully realized library of abstraction functions that meet our desiderata: (a) they should generalize, (b) they should be interpretable, and (c) they should produce plausible outputs. 

Our system receives a number of benefits from the prior knowledge encoded in LLMs.
Since LLMs have been trained extensively on human-written code, they are able to author functions with meaningful names and parameters.
This exposes an interface that a person can easily work with and understand.
However, we also find that LLMs are prone to hallucinate, generating mismatches from `real' distributions of shapes (e.g.,  collections of 3D assets).

To overcome this issue, we guide and ground the LLM outputs under the supervision of the user provided design intent, consisting of a textual description and a set of seed shapes. 
Textual descriptions of desired function properties help constrain the interface design, prompting the semantic prior of the LLM to attune towards a particular modeling task.
Each seed set we consider is composed of twenty 3D shapes with part-level semantic segmentations and textured renders.
Our system validates the plausibility of its productions by searching for function implementations and applications that can explain sub-structures in these exemplars.

In the following, we explain how~\methodname~solves this problem.
In Section~\ref{sec:lib_design}, we describe how we convert design intent into a fully realized library of abstraction functions.
In Section~\ref{sec:lib_usage}, we describe how we can expand the usage of this library beyond the seed set by training a recognition network on synthetic data. 
