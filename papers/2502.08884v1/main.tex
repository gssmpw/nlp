\documentclass[acmtog, nonacm]{acmart}

\AtBeginDocument{%
  \providecommand\BibTeX{{%
    \normalfont B\kern-0.5em{\scshape i\kern-0.25em b}\kern-0.8em\TeX}}}

\acmJournal{TOG}

\usepackage{booktabs} % For formal tables
\usepackage{savesym}
\savesymbol{zifour@default}
\savesymbol{zifour@scaled}
\usepackage{enumerate}
\usepackage{multirow} 
\usepackage[normalem]{ulem}
\usepackage{inconsolata}

\usepackage[ruled]{algorithm2e} % For algorithms
\renewcommand{\algorithmcfname}{ALGORITHM}

\SetAlFnt{\small}
\SetAlCapFnt{\small}
\SetAlCapNameFnt{\small}
\SetAlCapHSkip{0pt}

\citestyle{acmauthoryear}

\setlength{\abovecaptionskip}{5pt plus 2pt minus 2pt}


%
\setlength\unitlength{1mm}
\newcommand{\twodots}{\mathinner {\ldotp \ldotp}}
% bb font symbols
\newcommand{\Rho}{\mathrm{P}}
\newcommand{\Tau}{\mathrm{T}}

\newfont{\bbb}{msbm10 scaled 700}
\newcommand{\CCC}{\mbox{\bbb C}}

\newfont{\bb}{msbm10 scaled 1100}
\newcommand{\CC}{\mbox{\bb C}}
\newcommand{\PP}{\mbox{\bb P}}
\newcommand{\RR}{\mbox{\bb R}}
\newcommand{\QQ}{\mbox{\bb Q}}
\newcommand{\ZZ}{\mbox{\bb Z}}
\newcommand{\FF}{\mbox{\bb F}}
\newcommand{\GG}{\mbox{\bb G}}
\newcommand{\EE}{\mbox{\bb E}}
\newcommand{\NN}{\mbox{\bb N}}
\newcommand{\KK}{\mbox{\bb K}}
\newcommand{\HH}{\mbox{\bb H}}
\newcommand{\SSS}{\mbox{\bb S}}
\newcommand{\UU}{\mbox{\bb U}}
\newcommand{\VV}{\mbox{\bb V}}


\newcommand{\yy}{\mathbbm{y}}
\newcommand{\xx}{\mathbbm{x}}
\newcommand{\zz}{\mathbbm{z}}
\newcommand{\sss}{\mathbbm{s}}
\newcommand{\rr}{\mathbbm{r}}
\newcommand{\pp}{\mathbbm{p}}
\newcommand{\qq}{\mathbbm{q}}
\newcommand{\ww}{\mathbbm{w}}
\newcommand{\hh}{\mathbbm{h}}
\newcommand{\vvv}{\mathbbm{v}}

% Vectors

\newcommand{\av}{{\bf a}}
\newcommand{\bv}{{\bf b}}
\newcommand{\cv}{{\bf c}}
\newcommand{\dv}{{\bf d}}
\newcommand{\ev}{{\bf e}}
\newcommand{\fv}{{\bf f}}
\newcommand{\gv}{{\bf g}}
\newcommand{\hv}{{\bf h}}
\newcommand{\iv}{{\bf i}}
\newcommand{\jv}{{\bf j}}
\newcommand{\kv}{{\bf k}}
\newcommand{\lv}{{\bf l}}
\newcommand{\mv}{{\bf m}}
\newcommand{\nv}{{\bf n}}
\newcommand{\ov}{{\bf o}}
\newcommand{\pv}{{\bf p}}
\newcommand{\qv}{{\bf q}}
\newcommand{\rv}{{\bf r}}
\newcommand{\sv}{{\bf s}}
\newcommand{\tv}{{\bf t}}
\newcommand{\uv}{{\bf u}}
\newcommand{\wv}{{\bf w}}
\newcommand{\vv}{{\bf v}}
\newcommand{\xv}{{\bf x}}
\newcommand{\yv}{{\bf y}}
\newcommand{\zv}{{\bf z}}
\newcommand{\zerov}{{\bf 0}}
\newcommand{\onev}{{\bf 1}}

% Matrices

\newcommand{\Am}{{\bf A}}
\newcommand{\Bm}{{\bf B}}
\newcommand{\Cm}{{\bf C}}
\newcommand{\Dm}{{\bf D}}
\newcommand{\Em}{{\bf E}}
\newcommand{\Fm}{{\bf F}}
\newcommand{\Gm}{{\bf G}}
\newcommand{\Hm}{{\bf H}}
\newcommand{\Id}{{\bf I}}
\newcommand{\Jm}{{\bf J}}
\newcommand{\Km}{{\bf K}}
\newcommand{\Lm}{{\bf L}}
\newcommand{\Mm}{{\bf M}}
\newcommand{\Nm}{{\bf N}}
\newcommand{\Om}{{\bf O}}
\newcommand{\Pm}{{\bf P}}
\newcommand{\Qm}{{\bf Q}}
\newcommand{\Rm}{{\bf R}}
\newcommand{\Sm}{{\bf S}}
\newcommand{\Tm}{{\bf T}}
\newcommand{\Um}{{\bf U}}
\newcommand{\Wm}{{\bf W}}
\newcommand{\Vm}{{\bf V}}
\newcommand{\Xm}{{\bf X}}
\newcommand{\Ym}{{\bf Y}}
\newcommand{\Zm}{{\bf Z}}

% Calligraphic

\newcommand{\Ac}{{\cal A}}
\newcommand{\Bc}{{\cal B}}
\newcommand{\Cc}{{\cal C}}
\newcommand{\Dc}{{\cal D}}
\newcommand{\Ec}{{\cal E}}
\newcommand{\Fc}{{\cal F}}
\newcommand{\Gc}{{\cal G}}
\newcommand{\Hc}{{\cal H}}
\newcommand{\Ic}{{\cal I}}
\newcommand{\Jc}{{\cal J}}
\newcommand{\Kc}{{\cal K}}
\newcommand{\Lc}{{\cal L}}
\newcommand{\Mc}{{\cal M}}
\newcommand{\Nc}{{\cal N}}
\newcommand{\nc}{{\cal n}}
\newcommand{\Oc}{{\cal O}}
\newcommand{\Pc}{{\cal P}}
\newcommand{\Qc}{{\cal Q}}
\newcommand{\Rc}{{\cal R}}
\newcommand{\Sc}{{\cal S}}
\newcommand{\Tc}{{\cal T}}
\newcommand{\Uc}{{\cal U}}
\newcommand{\Wc}{{\cal W}}
\newcommand{\Vc}{{\cal V}}
\newcommand{\Xc}{{\cal X}}
\newcommand{\Yc}{{\cal Y}}
\newcommand{\Zc}{{\cal Z}}

% Bold greek letters

\newcommand{\alphav}{\hbox{\boldmath$\alpha$}}
\newcommand{\betav}{\hbox{\boldmath$\beta$}}
\newcommand{\gammav}{\hbox{\boldmath$\gamma$}}
\newcommand{\deltav}{\hbox{\boldmath$\delta$}}
\newcommand{\etav}{\hbox{\boldmath$\eta$}}
\newcommand{\lambdav}{\hbox{\boldmath$\lambda$}}
\newcommand{\epsilonv}{\hbox{\boldmath$\epsilon$}}
\newcommand{\nuv}{\hbox{\boldmath$\nu$}}
\newcommand{\muv}{\hbox{\boldmath$\mu$}}
\newcommand{\zetav}{\hbox{\boldmath$\zeta$}}
\newcommand{\phiv}{\hbox{\boldmath$\phi$}}
\newcommand{\psiv}{\hbox{\boldmath$\psi$}}
\newcommand{\thetav}{\hbox{\boldmath$\theta$}}
\newcommand{\tauv}{\hbox{\boldmath$\tau$}}
\newcommand{\omegav}{\hbox{\boldmath$\omega$}}
\newcommand{\xiv}{\hbox{\boldmath$\xi$}}
\newcommand{\sigmav}{\hbox{\boldmath$\sigma$}}
\newcommand{\piv}{\hbox{\boldmath$\pi$}}
\newcommand{\rhov}{\hbox{\boldmath$\rho$}}
\newcommand{\upsilonv}{\hbox{\boldmath$\upsilon$}}

\newcommand{\Gammam}{\hbox{\boldmath$\Gamma$}}
\newcommand{\Lambdam}{\hbox{\boldmath$\Lambda$}}
\newcommand{\Deltam}{\hbox{\boldmath$\Delta$}}
\newcommand{\Sigmam}{\hbox{\boldmath$\Sigma$}}
\newcommand{\Phim}{\hbox{\boldmath$\Phi$}}
\newcommand{\Pim}{\hbox{\boldmath$\Pi$}}
\newcommand{\Psim}{\hbox{\boldmath$\Psi$}}
\newcommand{\Thetam}{\hbox{\boldmath$\Theta$}}
\newcommand{\Omegam}{\hbox{\boldmath$\Omega$}}
\newcommand{\Xim}{\hbox{\boldmath$\Xi$}}


% Sans Serif small case

\newcommand{\Gsf}{{\sf G}}

\newcommand{\asf}{{\sf a}}
\newcommand{\bsf}{{\sf b}}
\newcommand{\csf}{{\sf c}}
\newcommand{\dsf}{{\sf d}}
\newcommand{\esf}{{\sf e}}
\newcommand{\fsf}{{\sf f}}
\newcommand{\gsf}{{\sf g}}
\newcommand{\hsf}{{\sf h}}
\newcommand{\isf}{{\sf i}}
\newcommand{\jsf}{{\sf j}}
\newcommand{\ksf}{{\sf k}}
\newcommand{\lsf}{{\sf l}}
\newcommand{\msf}{{\sf m}}
\newcommand{\nsf}{{\sf n}}
\newcommand{\osf}{{\sf o}}
\newcommand{\psf}{{\sf p}}
\newcommand{\qsf}{{\sf q}}
\newcommand{\rsf}{{\sf r}}
\newcommand{\ssf}{{\sf s}}
\newcommand{\tsf}{{\sf t}}
\newcommand{\usf}{{\sf u}}
\newcommand{\wsf}{{\sf w}}
\newcommand{\vsf}{{\sf v}}
\newcommand{\xsf}{{\sf x}}
\newcommand{\ysf}{{\sf y}}
\newcommand{\zsf}{{\sf z}}


% mixed symbols

\newcommand{\sinc}{{\hbox{sinc}}}
\newcommand{\diag}{{\hbox{diag}}}
\renewcommand{\det}{{\hbox{det}}}
\newcommand{\trace}{{\hbox{tr}}}
\newcommand{\sign}{{\hbox{sign}}}
\renewcommand{\arg}{{\hbox{arg}}}
\newcommand{\var}{{\hbox{var}}}
\newcommand{\cov}{{\hbox{cov}}}
\newcommand{\Ei}{{\rm E}_{\rm i}}
\renewcommand{\Re}{{\rm Re}}
\renewcommand{\Im}{{\rm Im}}
\newcommand{\eqdef}{\stackrel{\Delta}{=}}
\newcommand{\defines}{{\,\,\stackrel{\scriptscriptstyle \bigtriangleup}{=}\,\,}}
\newcommand{\<}{\left\langle}
\renewcommand{\>}{\right\rangle}
\newcommand{\herm}{{\sf H}}
\newcommand{\trasp}{{\sf T}}
\newcommand{\transp}{{\sf T}}
\renewcommand{\vec}{{\rm vec}}
\newcommand{\Psf}{{\sf P}}
\newcommand{\SINR}{{\sf SINR}}
\newcommand{\SNR}{{\sf SNR}}
\newcommand{\MMSE}{{\sf MMSE}}
\newcommand{\REF}{{\RED [REF]}}

% Markov chain
\usepackage{stmaryrd} % for \mkv 
\newcommand{\mkv}{-\!\!\!\!\minuso\!\!\!\!-}

% Colors

\newcommand{\RED}{\color[rgb]{1.00,0.10,0.10}}
\newcommand{\BLUE}{\color[rgb]{0,0,0.90}}
\newcommand{\GREEN}{\color[rgb]{0,0.80,0.20}}

%%%%%%%%%%%%%%%%%%%%%%%%%%%%%%%%%%%%%%%%%%
\usepackage{hyperref}
\hypersetup{
    bookmarks=true,         % show bookmarks bar?
    unicode=false,          % non-Latin characters in AcrobatÕs bookmarks
    pdftoolbar=true,        % show AcrobatÕs toolbar?
    pdfmenubar=true,        % show AcrobatÕs menu?
    pdffitwindow=false,     % window fit to page when opened
    pdfstartview={FitH},    % fits the width of the page to the window
%    pdftitle={My title},    % title
%    pdfauthor={Author},     % author
%    pdfsubject={Subject},   % subject of the document
%    pdfcreator={Creator},   % creator of the document
%    pdfproducer={Producer}, % producer of the document
%    pdfkeywords={keyword1} {key2} {key3}, % list of keywords
    pdfnewwindow=true,      % links in new window
    colorlinks=true,       % false: boxed links; true: colored links
    linkcolor=red,          % color of internal links (change box color with linkbordercolor)
    citecolor=green,        % color of links to bibliography
    filecolor=blue,      % color of file links
    urlcolor=blue           % color of external links
}
%%%%%%%%%%%%%%%%%%%%%%%%%%%%%%%%%%%%%%%%%%%


\begin{document}

\title[\methodname: designing a library of procedural 3D shape abstractions with Large Language Models]{\methodname: Designing a library of procedural 3D shape abstractions \\with Large Language Models}


\author{R. Kenny Jones}
\email{russell_jones@brown.edu}
\affiliation{%
    \institution{Brown University}
    \country{USA}
}

\author{Paul Guerrero}
\email{guerrero@adobe.com}
\affiliation{%
    \institution{Adobe Research}
    \country{United Kingdom}
}

\author{Niloy J. Mitra}
\email{n.mitra@cs.ucl.ac.uk}
\affiliation{%
    \institution{University College London and Adobe Research}
    \country{United Kingdom}
}

\author{Daniel Ritchie}
\email{daniel\_ritchie@brown.edu}
\affiliation{%
    \institution{Brown University}
    \country{USA}
}

\begin{abstract}

During the early stages of interface design, designers need to produce multiple sketches to explore a design space.  Design tools often fail to support this critical stage, because they insist on specifying more details than necessary. Although recent advances in generative AI have raised hopes of solving this issue, in practice they fail because expressing loose ideas in a prompt is impractical. In this paper, we propose a diffusion-based approach to the low-effort generation of interface sketches. It breaks new ground by allowing flexible control of the generation process via three types of inputs: A) prompts, B) wireframes, and C) visual flows. The designer can provide any combination of these as input at any level of detail, and will get a diverse gallery of low-fidelity solutions in response. The unique benefit is that large design spaces can be explored rapidly with very little effort in input-specification. We present qualitative results for various combinations of input specifications. Additionally, we demonstrate that our model aligns more accurately with these specifications than other models. 

% OLD ABSTRACT
%When sketching Graphical User Interfaces (GUIs), designers need to explore several aspects of visual design simultaneously, such as how to guide the user’s attention to the right aspects of the design while making the intended functionality visible. Although current Large Language Models (LLMs) can generate GUIs, they do not offer the finer level of control necessary for this kind of exploration. To address this, we propose a diffusion-based model with multi-modal conditional generation. In practice, our model optionally takes semantic segmentation, prompt guidance, and flow direction to generate multiple GUIs that are aligned with the input design specifications. It produces multiple examples. We demonstrate that our approach outperforms baseline methods in producing desirable GUIs and meets the desired visual flow.

% Designing visually engaging Graphical User Interfaces (GUIs) is a challenge in HCI research. Effective GUI design must balance visual properties, like color and positioning, with user behaviors to ensure GUIs easy to comprehend and guide attention to critical elements. Modern GUIs, with their complex combinations of text, images, and interactive components, make it difficult to maintain a coherent visual flow during design.
% Although current Large Language Models (LLMs) can generate GUIs, they often lack the fine control necessary for ensuring a coherent visual flow. To address this, we propose a diffusion-based model that effectively handles multi-modal conditional generation. Our model takes semantic segmentation, optional prompt guidance, and ordered viewing elements to generate high-fidelity GUIs that are aligned with the input design specifications.
% We demonstrate that our approach outperforms baseline methods in producing desirable GUIs and meets the desired visual flow. Moreover, a user study involving XX designers indicates that our model enhances the efficiency of the GUI design ideation process and provides designers with greater control compared to existing methods.    



% %%%%%%%%%%%%%%%%%%%%%%%%%%%%%%%%%%%%%%%%%%%%%%%%%%%%%%
% % Writing Clinic Comments:
% %%%%%%%%%%%%%%%%%%%%%%%%%%%%%%%%%%%%%%%%%%%%%%%%%%%%%%
% % Define: Effective UI design
% % Motivate GANs and write in full form.
% % LLMs vs ControlNet vs GANs
% % Say something about the Figma plugin?
% % Write the work is novel or what has been done before
% % What is desirable UI and how to evalutate that?
% % Visual Flow - main theme (center around it)
% % Re-Title: use word Flow!
% % Use ControlNet++ & SPADE for abstract.
% % Write about input/output. 
% % Why better than previous work?
% %%%%%%%%%%%%%%%%%%%%%%%%%%%%%%%%%%%%%%%%%%%%%%%%%%%%%

% % v2:
% % \noindent \textcolor{red}{\textbf{NEW Abstract!} (Post Writing Clinic 1 - 25-Jun)}

% % \noindent \textcolor{red}{----------------------------------------------------------------------}

% % \noindent Designing user interfaces (UIs) is a time-consuming process, particularly for novice designers. 
% % Creating UI designs that are effective in market funneling or any other designer defined goal requires a good understanding of the visual flow to guide users' attention to UI elements in the desired order. 
% % While current Large Language Models (LLMs) can generate UIs from just prompts, they often lack finer pixel-precise control and fail to consider visual flow. 
% % In this work, we present a UI synthesis method that incorporates visual flow alongside prompts and semantic layouts. 
% % Our efficient approach uses a carefully designed Generative Adversarial Network (GAN) optimized for scenarios with limited data, making it more suitable than diffusion-based and large vision-language models.
% % We demonstrate that our method produces more "desirable" UIs according to the well-known contrast, repetition, alignment, and proximity principles of design. 
% % We further validate our method through comprehensive automatic non-reference, human-preference aligned network scoring and subjective human evaluations.
% % Finally, an evaluation with xx non-expert designers using our contributed Figma plugin shows that <method-name> improves the time-efficiency as well as the overall quality of the UI design development cycle.

% % \noindent \textcolor{red}{----------------------------------------------------------------------}


% \noindent \textcolor{blue}{\textbf{NEW Abstract!} (Pre Writing Clinic 9-July)}

% \noindent \textcolor{blue}{----------------------------------------------------------------------}

% \noindent Exploring different graphical user interface (GUI) design ideas is time-consuming, particularly for novice designers. 
% Given the segmentation masks, design requirement as prompt, and/or preferred visual flow, we aim to facilitate creative exploration for GUI design and generate different UI designs for inspiration.
% While current Vision Language Models (VLMs) can generate GUIs from just prompts, they often lack control over visual concepts and flow that are difficult to convey through language during the generation process. 
% In this work, we present FlowGenUI, a semantic map-guided GUI synthesis method that optionally incorporates visual flow information based on the user's choice alongside language prompts. 
% We demonstrate that our model not only creates more realistic GUIs but also creates "predictable" (how users pay attention to and order of looking at GUI elements) GUIs.
% Our approach uses Stable Diffusion (SD), a large paired image-text pretrained diffusion model with a rich latent space that we steer toward realistic GUIs using a trainable copy of SD's encoder for every condition (segmentation masks, prompts, and visual flow). 
% We further provide a semantic typography feature to create custom text-fonts and styles while also alleviating SD's inherent limitations in drawing coherent, meaningful and correct aspect-ratio text. 
% Finally, a subjective evaluation study of XX non-expert and expert designers demonstrates the efficiency and fidelity of our method.


% This process encourages creativity and prevents designers from falling into habitual patterns.


% ------------------------------------------------------------------
% Joongi Why is it important to create realistic GUI?
% I do not see how the Visual Flow given on the left hand side is reflected in the results on the right hand side. 
% I’d avoid making unsubstantiated claims about designers (falling into habitual patterns).
% The UIs you generate do not “align with users’ attention patterns” but rather try to control it (that’s what visual flow means)
% ------------------------------------------------------------------
% Comments - Writing Clinic - 9th July:
% Improve title. More names: FlowGen
% Figure 1: Use an inference time hand-drawn mask
% Figure 1: Show both workflows. Add a designer --> Input.
% Figure 1: Make them more diverse
% ------------------------------------------------------------------
% Designing graphical user interfaces (GUIs) requires human creativity and time. Designers often fall into habitual patterns, which can limit the exploration of new ideas. 
% To address this, we introduce FlowGenUI, a method that facilitates creative exploration and generates diverse GUI designs for inspiration. By using segmentation masks, design requirements as prompts, and/or selected visual flows, our approach enhances control over the visual concepts and flows during the generation process, which current Vision Language Models (VLMs) often lack.
% FlowGenUI uses Stable Diffusion (SD), a largely pretrained text-to-image diffusion model, and guides it to create realistic GUIs. 
% We achieve this by using a trainable copy of SD's encoder for each condition (segmentation masks, prompts, and visual flow). 
% This method enables the creation of more realistic and predictable GUIs that align with users' attention patterns and their preferred order of viewing elements.
% We also offer a semantic typography feature that creates custom text fonts and styles while addressing SD's limitations in generating coherent, meaningful, and correctly aspect-ratio text.
% Our approach's efficiency and fidelity are evaluated through a subjective user study involving XX designers. 
% The results demonstrate the effectiveness of FlowGenUI in generating high-quality GUI designs that meet user requirements and visual expectations.

% ---------------------------------------


%A critical and general issue remains while using such deep generative priors: creating coherent, meaningful and correct aspect-ratio text. 
%We tackle this issue within our framework and additionally provide a semantic typography feature to create custom text-fonts and styles. 


% %Creating UI designs that are effective in market funneling or any other designer-defined goal requires a good understanding of the visual flow to guide users' attention to UI elements in the desired order. 
% %While current largely pre-trained Vision Language Models (VLMs) can generate GUIs from just prompts, they often lack finer or pixel-precise control which can be crucial for many easy-to-understand visual concepts but difficult to convey through language. 
% % However, obtaining such pixe-level labels is an extremely expensive so we
% % For example - overlaying text on images with certain aspect ratios and two equally separated buttons 
% Additionally, all prior GUI generation work fails to consider visual flow information during the generation process. 
% We demonstrate that visual flow-informed generation not only creates more realistic and human-friendly GUIs but also creates "predictable" (how users pay attention to and order of looking at GUI elements) UIs that could be beneficial for designers for tasks like creating effective market funnels.
% In this work, we present a semantic map-guided GUI synthesis method that optionally incorporates visual flow information based on the user's choice alongside language prompts. 
% Our approach uses Stable Diffusion, a large (billions) paired image-text pretrained diffusion model with a rich latent space that we steer toward realistic GUIs using an ensemble of ControlNets. 
% % TODO: Mention it in 1 sentence:
% A critical and general issue remains while using such deep generative priors: creating coherent, meaningful and correct aspect-ratio text. 
% We tackle this issue within our framework and additionally provide a semantic typography feature to create custom text-fonts and styles. 
% To evaluate our method, we demonstrate that our method produces more "desirable" UIs according to the well-known contrast, repetition, alignment, and proximity principles of design. 
% % We further validate our method through comprehensive automatic non-reference and human-preference aligned scores. (TODO: Maybe Unskip if we get UIClip from Jason!)
% % TODO: Re-word this and only keep ideation cycles and time-efficiency.
% Finally, a subjective evaluation study of XX non-expert and expert designers demonstrates the efficiency and fidelity of our method.
% % improves the time-efficiency by quick iterations of the UI design ideation process.
% %Finally, an evaluation with xx non-expert designers using our contributed <method-name> improves the time-efficiency by quick iterations of the UI design ideation cycle.

%\noindent \textcolor{blue}{----------------------------------------------------------------------}


%In an evaluation with xx designers, we found that GenerativeLayout: 1) enhances designers' exploration by expanding the coverage of the design space, 2) reduces the time required for exploration, and 3) maintains a perceived level of control similar to that of manual exploration.



% Present-day graphical user interfaces (GUIs) exhibit diverse arrangements of text, graphics, and interactive elements such as buttons and menus, but representations of GUIs have not kept up. They do not encapsulate both semantic and visuo-spatial relationships among elements. %\color{red} 
% To seize machine learning's potential for GUIs more efficiently, \papername~ exploits graph neural networks to capture individual elements' properties and their semantic—visuo-spatial constraints in a layout. The learned representation demonstrated its effectiveness in multiple tasks, especially generating designs in a challenging GUI autocompletion task, which involved predicting the positions of remaining unplaced elements in a partially completed GUI. The new model's suggestions showed alignment and visual appeal superior to the baseline method and received higher subjective ratings for preference. 
% Furthermore, we demonstrate the practical benefits and efficiency advantages designers perceive when utilizing our model as an autocompletion plug-in.


% Overall pipeline: Maybe drop semantic typography / visual flow?
\end{abstract}

\begin{CCSXML}
<ccs2012>
<concept>
<concept_id>10010147.10010371.10010396</concept_id>
<concept_desc>Computing methodologies~Shape modeling</concept_desc>
<concept_significance>500</concept_significance>
</concept>
</ccs2012>
\end{CCSXML}
\ccsdesc[500]{Computing methodologies~Shape modeling}
\keywords{procedural modeling, shape analysis, shape abstraction, library learning, large language models, LLMs, semantics}


\begin{teaserfigure}
\centering
  \includegraphics[width=\linewidth]{figs/shapelib_teaser_v1.pdf}
  \caption{\methodname guides an LLM to design a library of procedural shape functions from a given set of (20) seed shapes and textual descriptions. Using an LLM prior makes the functions semantically interpretable and easy to edit, while aligning them with the seed shapes specializes the functions to a given domain and reduces LLM hallucinations. The library can be used to train a network for visual program induction that generalizes well beyond the seed shapes.
  }
  \label{fig:teaser}
\end{teaserfigure}


\maketitle

\section{Introduction}

Tutoring has long been recognized as one of the most effective methods for enhancing human learning outcomes and addressing educational disparities~\citep{hill2005effects}. 
By providing personalized guidance to students, intelligent tutoring systems (ITS) have proven to be nearly as effective as human tutors in fostering deep understanding and skill acquisition, with research showing comparable learning gains~\citep{vanlehn2011relative,rus2013recent}.
More recently, the advancement of large language models (LLMs) has offered unprecedented opportunities to replicate these benefits in tutoring agents~\citep{dan2023educhat,jin2024teach,chen2024empowering}, unlocking the enormous potential to solve knowledge-intensive tasks such as answering complex questions or clarifying concepts.


\begin{figure}[t!]
\centering
\includegraphics[width=1.0\linewidth]{Figs/Fig.intro.pdf}
\caption{An illustration of coding tutoring, where a tutor aims to proactively guide students toward completing a target coding task while adapting to students' varying levels of background knowledge. \vspace{-5pt}}
\label{fig:example}
\end{figure}

\begin{figure}[t!]
\centering
\includegraphics[width=1.0\linewidth]{Figs/Fig.scaling.pdf}
\caption{\textsc{Traver} with the trained verifier shows inference-time scaling for coding tutoring (detailed in \S\ref{sec:scaling_analysis}). \textbf{Left}: Performance vs. sampled candidate utterances per turn. \textbf{Right}: Performance vs. total tokens consumed per tutoring session. \vspace{-15pt}}
\label{fig:scale}
\end{figure}


Previous research has extensively explored tutoring in educational fields, including language learning~\cite{swartz2012intelligent,stasaski-etal-2020-cima}, math reasoning~\cite{demszky-hill-2023-ncte,macina-etal-2023-mathdial}, and scientific concept education~\cite{yuan-etal-2024-boosting,yang2024leveraging}. 
Most aim to enhance students' understanding of target knowledge by employing pedagogical strategies such as recommending exercises~\cite{deng2023towards} or selecting teaching examples~\cite{ross-andreas-2024-toward}. 
However, these approaches fall short in broader situations requiring both understanding and practical application of specific pieces of knowledge to solve real-world, goal-driven problems. 
Such scenarios demand tutors to proactively guide people toward completing targeted tasks (e.g., coding).
Furthermore, the tutoring outcomes are challenging to assess since targeted tasks can often be completed by open-ended solutions.



To bridge this gap, we introduce \textbf{coding tutoring}, a promising yet underexplored task for LLM agents.
As illustrated in Figure~\ref{fig:example}, the tutor is provided with a target coding task and task-specific knowledge (e.g., cross-file dependencies and reference solutions), while the student is given only the coding task. The tutor does not know the student's prior knowledge about the task.
Coding tutoring requires the tutor to proactively guide the student toward completing the target task through dialogue.
This is inherently a goal-oriented process where tutors guide students using task-specific knowledge to achieve predefined objectives. 
Effective tutoring requires personalization, as tutors must adapt their guidance and communication style to students with varying levels of prior knowledge. 


Developing effective tutoring agents is challenging because off-the-shelf LLMs lack grounding to task-specific knowledge and interaction context.
Specifically, tutoring requires \textit{epistemic grounding}~\citep{tsai2016concept}, where domain expertise and assessment can vary significantly, and \textit{communicative grounding}~\citep{chai2018language}, necessary for proactively adapting communications to students' current knowledge.
To address these challenges, we propose the \textbf{Tra}ce-and-\textbf{Ver}ify (\textbf{\model}) agent workflow for building effective LLM-powered coding tutors. 
Leveraging knowledge tracing (KT)~\citep{corbett1994knowledge,scarlatos2024exploring}, \model explicitly estimates a student's knowledge state at each turn, which drives the tutor agents to adapt their language to fill the gaps in task-specific knowledge during utterance generation. 
Drawing inspiration from value-guided search mechanisms~\citep{lightman2023let,wang2024math,zhang2024rest}, \model incorporates a turn-by-turn reward model as a verifier to rank candidate utterances. 
By sampling more candidate tutor utterances during inference (see Figure~\ref{fig:scale}), \model ensures the selection of optimal utterances that prioritize goal-driven guidance and advance the tutoring progression effectively. 
Furthermore, we present \textbf{Di}alogue for \textbf{C}oding \textbf{T}utoring (\textbf{\eval}), an automatic protocol designed to assess the performance of tutoring agents. 
\eval employs code generation tests and simulated students with varying levels of programming expertise for evaluation. While human evaluation remains the gold standard for assessing tutoring agents, its reliance on time-intensive and costly processes often hinders rapid iteration during development. 
By leveraging simulated students, \eval serves as an efficient and scalable proxy, enabling reproducible assessments and accelerated agent improvement prior to final human validation. 



Through extensive experiments, we show that agents developed by \model consistently demonstrate higher success rates in guiding students to complete target coding tasks compared to baseline methods. We present detailed ablation studies, human evaluations, and an inference time scaling analysis, highlighting the transferability and scalability of our tutoring agent workflow.

\section{Background and related work}
% 重点看Artistic data visualization: Beyond visual analytics 和Visualization criticism-the missing link between information visualization and art 的被引


This section reviews the background on artistic data visualization and previous research related to this topic.

\subsection{Artistic Data Visualization in Art History Context}
\label{ssec:contemporary}

Art history has been marked by transformative periods characterized by different aesthetic pursuits, materials, and methods. Since the time of Plato, imitation (or \textit{mimesis}, which views art as a mirror to the world around us) has been an important pursuit~\cite{pooke2021art}. Successful artworks, such as Greek sculptures and the convincing illusions of depth and space in Renaissance paintings, exemplify this tradition.
The advent of modern society and new technology, especially photography, posed a significant challenge to the notion of art as imitation~\cite{perry2004themes}. By the 1850s, modern art began to emerge with the core goal of transcending traditional forms and conventions. Movements like Post Impressionism, Expressionism, and Cubism revolutionized art through innovative uses of form (\eg color, texture, composition), moving art towards abstraction and experimentation. 
After World War II, the Cold War and the surge of various social problems heightened skepticism about the progress narrative of modernity and the superiority of the capitalist system, leading to the rise of postmodernism and the birth of contemporary art~\cite{hopkins2000after,harrison1992art}. One prominent feature of contemporary art is the absence of fixed standards or genres historically defined by the church or the academy. Postmodern design neither defines a cohesive set of aesthetic values nor restricts the range of media used~\cite{pooke2021art}, sparking novel practices such as installations, performances, lens-based media, videos, and land-based art~\cite{hopkins2000after}.
Meanwhile, artists have increasingly drawn energy from various philosophical and critical theories such as gender studies, psychoanalysis, Marxism, and post-structuralism~\cite{pooke2021art}. As a result, as described by Foster~\cite{foster1999recodings}, artists have increasingly become ``manipulators of signs and symbols... and the viewer an active reader of messages rather than a passive contemplator of the aesthetic''. Hopkins~\cite{hopkins2000after} described this shift as the ``death of the object'' and ``the move to conceptualism''. 
% Joseph Kosuth, one of the most important representatives of conceptual artists, also once said that “all art (after Duchamp) is conceptual (in nature) because art only exists conceptually”
% As argued by Danto~\cite{danto2015after}, traditional notions of aesthetics can no longer apply to contemporary art. ``“All there is at the end,” Danto wrote, “is theory, art having finally become vaporized in a dazzle of pure thought about itself, and remaining, as it were, solely as the object of its own theoretical consciousness.''
% The Anti-aesthetic (1983) has described these as ‘anti-aesthetic’ strategies – typified, as we have seen, by a conceptually driven approach to the art object and to the process of its production.

Emerging within the contemporary art historical context, data art has been significantly propelled by the advent of big data. An early milestone was Kynaston McShine's 1970 exhibition \textit{Information} at the Museum of Modern Art (MoMA). 
% In the exhibition catalogue, McShine wrote~\cite{information_moma}: ``Increasingly artists use mail, telegrams, telex machines, etc., for transmission of works themselves—photographs, films, documents—or of information about their activity.'' 
% The millennium era has witnessed substantial growth in this area.
In 2008, Google’s Data Arts Team was founded to explore the advancement of what creativity and technology can do together~\cite{google}.
% data artist Aaron Koblin
In 2012, Viégas and Wattenberg's \textit{Wind Map}, an artwork that visualizes real-time air movement, became the first web-based artwork to be included in MoMA's permanent collection~\cite{wind}.
Since 2013, the academic conference IEEE VIS has included an Arts Program (IEEE VISAP), showcasing artistic data visualizations through accepted papers and curated exhibitions. 
As noted by Barabási~\cite{dataism} (a Fellow of the American Physical Society and the head of a data art lab), data has become a vital medium for artists to deal with the complexities of our society: ``Humanity is facing a complexity explosion. We are confronted with too much data for any of us to make sense of...The traditional tools and mediums of art, be they canvas or chisel, are woefully inadequate for this task...today’s and tomorrow’s artists can embrace new tools and mediums that scale to the challenge, ensuring that their practice can continue to reflect our changing epistemology.''
% a physicist and head of a data art lab, has noted, 

% Artists are now exploring new mediums and methods, incorporating datasets, computer technology, and algorithms into their work.



\subsection{Research on Artistic Data Visualization}
\label{ssec:artisticvis}

Artistic data visualization is also referred to as artistic visualization, data art, or information art~\cite{holmquist2003informative,rodgers2011exploring,few,viegas2007artistic}. Despite the varying terminologies, there is a basic consensus that artistic data visualization must be art practice grounded in real data~\cite{viegas2007artistic}.
Since the early 2000s, a series of papers introduced innovative artistic systems for visualizing everyday data, such as museum visit routes and bus schedule information~\cite{skog2003between,holmquist2003informative,viegas2004artifacts}.
In 2007, Viégas and Wattenberg~\cite{viegas2007artistic} explicitly proposed the concept of \textit{artistic data visualization} and viewed it as a promising domain beyond visual analytics.
% and defined it as ``visualization of data done by artists with the intent of making art''. 
Kosara~\cite{kosara2007visualization} drew a spectrum of visualization design, positioning artistic visualization and pragmatic visualization at opposite ends of this spectrum to demonstrate that the goals of these two types of design often diverge. 
% advocating that analytical visualizations prioritize practicality, while artistic data visualizations emphasize sublime quality, that is, the capacity to inspire awe and grandeur and elicit profound emotional or intellectual responses. 
% In 2011, Rodgers and Bartram~\cite{rodgers2011exploring} utilized artistic data visualization to enhance residential energy use feedback. 
However, overall, research on this subject has been sparse. Among those relevant papers, most have focused on specific applications of artistic data visualization. 
%~\cite{rodgers2011exploring,schroeder2015visualization,perovich2020chemicals}
For instance, Rodgers and Bartram~\cite{rodgers2011exploring} utilized ambient artistic data visualization to enhance residential energy use feedback. Samsel~\etal~\cite{samsel2018art} transferred artistic styles from paintings into scientific visualization.
Artistic practice has also been observed in fields such as data physicalization~\cite{hornecker2023design,perovich2020chemicals,offenhuber2019data} and sonification~\cite{enge2024open}. For example, Hornecker~\etal~\cite{hornecker2023design} found that many artists are practicing transforming data into tangible artifacts or installations. Enge~\etal~\cite{enge2024open} discussed a set of representative artistic cases that combine sonification and visualization.
% dragicevic2020data
% Offenhuber~\cite{offenhuber2019data} created a set of artworks in urban settings that transform air quality data into situated displays, allowing people to encounter visualizations in their daily lives.

% But in contrast, empirical studies that describe the characteristics of artistic visualization and how they are created are extremely scarce. This scarcity forms a stark contrast to the increasingly rich and diverse practices by artists in the field.
% As for the difference between artistic data visualization and traditional visualizations for analytics, Vi{\'e}gas and Wattenberg~\cite{viegas2007artistic} thought that the most salient feature of artistic data visualizations is their forceful expression of viewpoints.
% In Ramirez~\cite{ramirez2008information}'s opinion, functional information visualizations are concerned with usability and performance while aesthetic information visualizations are concerned with visually attractive forms of representation design.
% Donath~\etal~\cite{donath2010data} presented a series of tools developed to integrate artistic expressions in generating unique and customized visualizations based on users' personal data, such as health monitoring data, album records, and e-mail records. 

On the other hand, some studies, while not focusing on artistic data visualization, have explored a key art-related concept: aesthetics. 
% ~\cite{moere2012evaluating,cawthon2007effect,lau2007towards} explored the aesthetics of visualization design in their research. They
For example, Moere~\etal~\cite{moere2012evaluating} compared analytical, magazine, and artistic visualization styles, noting that analytical styles enhance the discovery of analytical insights, while the other two induce more meaning-based insights. Cawthon~\etal~\cite{cawthon2007effect} asked participants to rank eleven visualization types on an aesthetic scale from ``ugly'' to ``beautiful'', finding that some visualizations (\eg sunburst) were perceived as more beautiful than others (\eg beam trees).
% Moere~\etal~\cite{moere2012evaluating} displayed data in three different styles (analytical style, magazine style, artistic style) and found that these styles led to different perceptions of usability and types of insights.
% More importantly, the authors found that the sunburst chart ranks the highest in aesthetics and is one of the top-performing visualizations in both efficiency and effectiveness, thus exemplifying the notion that "beautiful is indeed usable".
Factors such as embellishment~\cite{bateman2010useful}, colorfulness~\cite{harrison2015infographic}, and interaction~\cite{stoll2024investigating} have also been found to influence aesthetics. 
% borkin2013makes,tanahashi2012design
However, most of these studies have simplified aesthetics to hedonic features (\eg beauty), without delving into the nuanced connotations of aesthetics.
% most of these studies have simplified aesthetics to concepts like 'beauty,' 'preference,' or 'pleasing,' without exploring the deeper essence of aesthetics as the core of art.

The value of artistic data visualization to the visualization community is still in controversy. For instance, Few~\cite{few} argued for a clearer distinction between data art and data visualization; he highlighted the negative consequences when data art ``masquerades as data visualization'', such as making visualizations difficult to perceive. Willers~\cite{willers2014show} thought the artistic approach is ``unlikely be appreciated if the intention was for rapid decision making.''
% In an interview, American artist and technologist Harris commented that ``data can be made pretty by design, but this is a superficial prettiness, like a boring woman wearing too much makeup.''~\cite{harris2015beauty} 
To address these gaps, more empirical research needs to be conducted to explore how artistic data visualizations are designed, their underlying pursuits, and how they might inspire our community.




% Examining this field not only helps us understand the latest application of data visualization in various domains but also extends our understanding of the aesthetic and humanistic aspects of data visualization.
% there should be more theoretical investigation into artistic data visualization. 

% These three concepts emphasize placing or installing visualizations at physical places that people will encounter in their daily lives. 

% ~\cite{wang2019emotional}


% gap between art and science~\cite{judelman2004aesthetics}
% constructive visualization~\cite{huron2014constructive}
% data feminism~\cite{d2020data}
% critical infovis~\cite{dork2013critical}
% citizen data and participation~\cite{valkanova2015public}

% \x{Lee~\etal~\cite{lee2013sketchstory}, give users artistic freedom to create their own visualizations.}


% Aesthetics refers to the study of beauty, taste, and sensory perception and is deeply intertwined with art.
% a particular taste for or approach to what is pleasing to the senses and especially sight

% why shouldn't all charts be scatter plot~\cite{bertini2020shouldn}
% aesthetic model~\cite{lau2007towards}
% Aesthetics for Communicative Visualization : a Brief Review
% Stacked graphs--geometry \& aesthetics~\cite{byron2008stacked}
% storyline optimization~\cite{tanahashi2012design}
% graphic designers rate the attractiveness of non-standard and pictorial visualizations higher than standard and abstract ones, whereas the opposite is true for laypeople.~\cite{quispel2014would}
% evaluate aesthetics - wordcloud
% An Evaluation of Semantically Grouped Word Cloud Designs, tag cloud, wordle

% On the other hand, empirical studies conducted with designers have shown that functionality is not the only design goal of visualization. For example, Bigelow~\etal~\cite{bigelow2014reflections} found that designers would frequently fine-tune the non-data elements in their designs, and data encoding was even "a later consideration with respect to other visual elements of the infographic".
% Moere~\cite{moere2011role} - design
% Quispel~\etal~\cite{quispel2018aesthetics} found that for information designers, clarity and aesthetics are both important for making a design attractive.


\section{Overview}
\label{sec:label}

\methodname guides an LLM through the process of developing a library of procedural functions that matches an input design intent.
In our problem framing, we assume that a user has a procedural modeling domain in mind (e.g., a particular category of shapes).
The user will communicate their design intent to our system, which is then tasked with producing a fully realized library of abstraction functions that meet our desiderata: (a) they should generalize, (b) they should be interpretable, and (c) they should produce plausible outputs. 

Our system receives a number of benefits from the prior knowledge encoded in LLMs.
Since LLMs have been trained extensively on human-written code, they are able to author functions with meaningful names and parameters.
This exposes an interface that a person can easily work with and understand.
However, we also find that LLMs are prone to hallucinate, generating mismatches from `real' distributions of shapes (e.g.,  collections of 3D assets).

To overcome this issue, we guide and ground the LLM outputs under the supervision of the user provided design intent, consisting of a textual description and a set of seed shapes. 
Textual descriptions of desired function properties help constrain the interface design, prompting the semantic prior of the LLM to attune towards a particular modeling task.
Each seed set we consider is composed of twenty 3D shapes with part-level semantic segmentations and textured renders.
Our system validates the plausibility of its productions by searching for function implementations and applications that can explain sub-structures in these exemplars.

In the following, we explain how~\methodname~solves this problem.
In Section~\ref{sec:lib_design}, we describe how we convert design intent into a fully realized library of abstraction functions.
In Section~\ref{sec:lib_usage}, we describe how we can expand the usage of this library beyond the seed set by training a recognition network on synthetic data. 

\begin{figure*}[t!]
\centering
  \includegraphics[width=\linewidth]{figs/shapelib_method_v1.pdf}
   \caption{Method overview. We design a function library in four steps, starting from a user intent (light blue) that consists of function descriptions and a set of seed shapes. First, (a) we prompt an LLM to create function interfaces that define parameters and annotate the function's purpose. Then, (b) the LLM is prompted to propose multiple applications of the functions that reconstruct the seed shapes. Next, (c) we use this information to guide the LLM to propose multiple function implementations. The library is finalized with a validation step (d) that searches for pairs of  applications and implementations that best reconstruct the seed shapes. We can use the library to extend beyond the seed shapes by guiding the LLM to author a synthetic data generator with the library functions, and using the resulting paired data to train a recognition network for visual program induction.
   } 
  \label{fig:method_fig}
\end{figure*}


\section{Library Design}
\label{sec:lib_design}

\methodname~converts design intent into a library of functions through a series of steps, which we depict in Figure~\ref{fig:method_fig}.
The interface creation step converts function descriptions into a library interface (Section~\ref{sec:lib_interface}). 
The application proposal step identifies which library functions should model which seed set shapes (Section~\ref{sec:prop_apps}).
The implementation proposal step generates candidate function implementations (Section~\ref{sec:prop_impls}).
The library is then finalized with a validation step that checks combinations of proposed function applications and implementations against seed set examples (Section~\ref{sec:lib_validation}). 


\subsection{Interface Creation}
\label{sec:lib_interface}

\methodname~first converts user function descriptions into a library interface (Fig.~\ref{fig:method_fig}, a).
We prompt an LLM to produce a structured interface, where for each function it produces a typed signature and an accompanying doc-string.

We provide the LLM with two default classes: a `Part' class that creates primitives that abstract detailed geometry and a `CoordFrame' class that defines a local bounding volume.
Our prompt contains task instructions and in-context expert demonstrations sourced from different categories.
By default, we use axis-aligned cuboid primitives, though this design decision could be generalized by modifying prompt instructions and examples.

The LLM produces function signatures that expose parametric handles, e.g. the numbers of bars in a ladder back or the height of base runner.
Each function is instructed to take in a special first parameter, \textit{CF}, a `CoordFrame' that specifies the expected extents of the functions outputs. 
Functions are typed so that they return a List of `Part' objects.

Through our in-context examples and instructions, we prompt the doc-string to have a particular structure. 
First, it defines a \textit{description} field to explain the high-level goals of the function.
Then, it defines a \textit{parts} field, that specifies what parts should be produced depending on the input parameters.
Finally, it defines a \textit{parameter} field, that explains how they should affect the output structure.
This interface is then used to guide the library development.

\subsection{Proposing Function Applications}
\label{sec:prop_apps}

As LLMs are prone to hallucinate, we do not directly implement each function following the prior step. 
Instead, we would like to ground each function implementation by referencing structures from the seed set.
To find such references, we propose programs that apply library functions that explain exemplar shapes (Fig.~\ref{fig:method_fig}, b).

This step begins by sampling a shape from the seed set.
We ask a VLM to describe the parts that is sees from a render of the shape.
We also convert the 3D semantic part annotations into a list of labeled `Part' objects.
We combine these inputs together, and task an LLM with deciding what parts should be explained by which library functions (even though these functions lack implementations).
The LLM outputs this decision by authoring a `program()' function that proposes library function applications (along with parameters).
We ask the LLM to use a special `group\_parts' function when constructing this program, that consumes a list of input `Part' objects and returns a bounding `CoordFrame' object.
In this way, the `program' provides information about which parts of the input shape should be explained by which library functions.

As we later demonstrate empirically, the accuracy of individual LLM calls has a high variance which makes them hard to trust. 
Therefore, instead of finding a single program for each shape, we run this procedure K times for each shape in the seed set (K=5).


\subsection{Propose Function Implementations}
\label{sec:prop_impls}

\methodname~now has the information from the prior steps it needs to author good function implementations: typed signatures, doc-string guidance, and input-output examples.
These input-output example pairs can be automatically found from the proposed function applications.
From this input, we ask the LLM to complete the implementation of each function so that it matches the signature type, meets the doc-string specification, and respects the observed patterns present in the usage examples (Fig.~\ref{fig:method_fig}, c). 

Of note, we find that the LLM predictions in the previous application proposal step do a good job of identifying which functions should explain which parts, but do a much worse job at predicting parameter values. With this in mind, we mask out parameter values with a special token `?' in all input-output examples.
We do this for every parameter value, except for the first \textit{CF} `CoordFrame', as the correct value for this parameter can be found automatically with the `group\_parts' function.


Similar to previous step, we find that some implementations produced by the LLM produce better or worse matches against the input specification.
So for each function in our library, we propose K different ways that it could be implemented (K=4).

\subsection{Library Validation}
\label{sec:lib_validation}

At this point we are close to having a fully realized library.
From the prior steps we have (a) function doc-strings and signatures, (b) proposals of how the functions should be applied to explain groups of parts in seed-set shapes, and (c) proposals of how the function should be implemented.
This validation step is responsible for deciding which of these proposals are `good', and not just LLM hallucinations
(Fig~\ref{fig:method_fig}, d).

To make this decision, we search over pairs of proposed implementations and parameterizations, and record those that geometrically match structures present in the seed set shapes.
For each  proposed function implementation from (c) we check which of proposed part groups from (b) this implementation can explain.
Specifically, we try executing the function with the proposed parameterizations sourced from (b), calculate the observed error between the target parts and function output, and record the parameterization that achieves the best error.
Our error metric compares corner-to-corner distances between sets of geometric primitives, and mark function applications as invalid  if the paired structures are not similar enough (see the appendix for details).

At this point, for each group of parts from (b) we know which implementation from (c) best matches the observed part structure.
We keep the implementation that achieves the \textit{best} error across the \textit{most} part groups, and remove all others proposals.
If this \textit{best} implementation found valid applications across multiple seed set shapes, we update the library interface entry with its implementation logic. 
Otherwise, we remove the function entry from the interface.




\section{Using the Library for Program Synthesis}
\label{sec:lib_usage}

In Section~\ref{sec:lib_design}, we constructed a library of functions that have meaningful signatures and structured doc-strings.
Each function has an implementation that is capable of producing structures that capture patterns observed in the seed set, but a question remains: how can we use these functions to represent new shapes?

In this section, we describe our strategy for expanding library function usage beyond the seed set (Fig.~\ref{fig:method_fig}, \textit{right}).
To begin, we once again make use of the strong prior of LLMs by providing it with our library interface and asking it to design a procedure that uses the abstraction functions to randomly synthesize synthetic shapes.
Once we've developed this synthetic data sampler, we can use it to produce paired training data for a recognition network that learns how to solve an inverse task: given an input shape structure, write a program using the library functions that explain its parts.


\paragraph{Generating a synthetic shape sampler}

In this step, we design a prompt that describes the library we've developed, including the interface of each function  and examples of how to use it (sourced from the validation stage).
We give this prompt to an LLM and ask it to write a `sample\_shape' function that randomly produces new shapes using the provided abstractions.
Interestingly, we find that frontier LLMs are able to provide useful implementations of such a `sample\_shape' function.
A shown in Figure~\ref{fig:method_fig}, some of these random outputs produce good shape abstractions, while other random samples violate class semantics.
With this in mind, instead of attempting to get the LLM to perfect its implementation, we treat its output as a synthetic data generator for a recognition network. 
To broaden the coverage and variety of structures that these `sample\_shape' functions produce, we employ an iterative refinement loop that provides automatic feedback to the LLM.
This refinement procedure ensures that all functions and parameters in the library get used, and instructs the `sample\_shape' function to produce outputs spanning the observed structures from validation step (see appendix).


\paragraph{Training a recognition network}

Once we've improved the `sample\_shape' function through rounds of iterative refinement, we can use it to produce training data for a recognition network.
This network takes as input a shape represented as a set of unordered primitives (e.g., Cuboid dimensions and positions).
It outputs a program that uses library functions to reconstruct this input shape.
We implement this network as an autoregressive Transformer decoder~\cite{att_is_all} with a causal prefix mask over the input shape representation.
We train this network from scratch, streaming random samples from the synthetic data generator: each program we sample becomes a target output and we execute the program to find the corresponding input.
Once trained, we can use this network to find library function applications that explain shapes from outside of the starting seed set (Fig.~\ref{fig:method_fig}, \textit{right-bottom}).  
Our inference procedure prompts the network with an input set of unordered primitives and samples a large number of programs according the network's predicted distribution.
We try executing each program, and we record its complexity (the number of tokens it uses) and its geometric error against the input set. 
We choose the program that minimizes an objective that is a simple weighted combination of these two values.






























\section{Results}\label{sec:results}


\subsection{Analysis of Interaction Logs}\label{sec:results_interaction_logs}
We report our analyses of the interaction data. 

\begin{figure*}[t]
    \centering
    \includegraphics[width=\linewidth]{figures/interaction_logs_boxplots.pdf}
    \caption{Three measures of interaction behaviour: Task completion time \textit{(left)}, manual typing \textit{(centre)}, and writing speed \textit{(right)}. All AI features increased typing speed and reduced the time taken (both sig. for \modemail). They also reduced the number of keystrokes (sig. for \modeours{} and \modemail). If people made use of the optional \imppass{} feature \revision{(impr.)} in \modeours, this contributed to narrowing the gap between the otherwise sentence-level design of \modeours{} and the message-level design of \modemail{} (sig. for manual typing and writing speed). See \cref{sec:results_interaction_logs} for details.}
    \Description{This figure presents three box plots that measure interaction behaviour from the study across three different user interfaces: NoAI (manual mode), CDLR (AI-supported with and without the optional improvement pass feature), and MSG (AI-supported). The three metrics being compared are:
    Task Completion Time (Left):
    This plot shows how long it took participants to complete the task in minutes.
    Both CDLR (with and without improvement) and MSG show reduced task completion times compared to NoAI, with MSG showing the most significant reduction.
    Manual Typing (Center):
    This plot shows the number of keystrokes made by participants during the task.
    Both CDLR and MSG significantly reduced the number of keystrokes compared to NoAI, particularly when participants used the optional improvement pass feature in CDLR.
    Writing Speed (Right):
    This plot shows the writing speed of participants measured in characters per second.
    Both CDLR and MSG increased writing speed compared to NoAI, with MSG again showing the most notable improvement.
    Overall, the figure demonstrates that AI-supported interfaces (CDLR and MSG) led to faster task completion, fewer keystrokes, and increased writing speed. In particular, the optional improvement pass feature in CDLR helped narrow the performance gap between the sentence-level design of CDLR and the message-level design of MSG. For detailed statistical analysis, see the corresponding section of the paper (Section 6.1).}
    \label{fig:interaction_logs_boxplots}
\end{figure*}


\subsubsection{Task Completion Time}\label{sec:results_time}

On average, participants took \mins{3.20} (SD 2.51, median 2.46) to write an email manually.
With \modemail, this decreased to \mins{2.03} (SD 1.90, median 1.44), while \modeours{} reduced it to \mins{2.95} (SD 2.49, median 2.32). For \modeours, using the \imppass{} feature resulted in \mins{2.90} (SD 2.58, median 2.26), while not using it had \mins{3.06} (SD 2.29, median 2.57).
\cref{fig:interaction_logs_boxplots} (left) shows this as box plots.
These differences were significant as follows (\cref{tab:lmm_overview}, row 1): 
Participants finished replying significantly faster with \modemail{} than without AI (-\secs{70}). \modemail{} was also significantly faster than \modeours{} (-\secs{66}).



\subsubsection{Writing Speed}\label{sec:results_speed}

On average, participants wrote 2.05 characters per second without AI (SD 1.57, median 1.78).
Replying with \modemail{} had a mean of 7.21 (SD 7.90, median 4.89), while the mean speed with \modeours{} was 4.38 (SD 5.59, median 2.91). For \modeours, using the \imppass{} feature resulted in 5.03 (SD 6.47, median 3.23), while not using it had 2.91 (SD 2.00, median 2.31).
\cref{fig:interaction_logs_boxplots} (right) shows this as box plots.
These differences were significant as follows (\cref{tab:lmm_overview}, row 2): 
Compared to writing without AI, participants produced significantly more characters per second with \modemail{} (5.2 chars more per s) and if they used the \imppass{} feature in \modeours{} (2.5 chars more per s). The difference between \modemail{} and \modeours{} was also significant.




\subsubsection{Manual Typing}\label{sec:results_keystrokes}
Without AI, participants on average needed 321.69 keystrokes (SD 217.76, median 284), compared to 146.67 (SD 144.97, median 108) with \modemail, and 174.62 (SD 166.57, median 128) with \modeours. For \modeours, using the \imppass{} feature had a mean of 176.0 (SD 173.7, median 126), while not using it had 171.4 (SD 149.9, median 133.5).
\cref{fig:interaction_logs_boxplots} (centre) shows this as box plots.
These differences were significant (\cref{tab:lmm_overview}, row 3): People needed significantly fewer keystrokes with AI features than without them, and even significantly fewer with \modemail{} (\pct{58} decrease) than with \modeours{} (\pct{48} decrease). Using the \imppass{} feature in \modeours{} significantly reduced this further for that UI (\pct{5.9} decrease). In summary, all AI features significantly reduced manual typing.




\subsubsection{Interaction with \modeours}
We logged interactions specific to \modeours.
On average participants tapped on 2.64  (SD 2.89, median 2) sentences per email, that is, on \pct{30.36} (SD \pct{29.59}, median \pct{23.08}) of sentences in each email.
They replied to \pct{87.37} of tapped sentences. In \pct{83.27} of the cases, they did so by accepting a suggestion. %

Suggestions were paginated; most suggestions (\pct{67.15}) were accepted on the first page. Another \pct{19.50} and \pct{13.36} were accepted on the second and third page, respectively.
The majority of accepted suggestions (\pct{80.14}) were generated without an explicit prompt, and most (\pct{92.30}) were not edited afterwards.


\revision{On the first screen, \pct{69.05} of participants accepted a sentence suggestion at least once, and \pct{55.56} manually entered text for at least one local response. %
On the second screen, \pct{83.33} of participants accepted at least one email-level suggestion. Only \pct{1.59} (two participants) did not use any \modeours-specific features.}




On average, \mins{1.67} (\pct{57.23}) were spent on the first screen and \mins{1.25} (\pct{42.77}) on the second (\cref{fig:time_spent_on_screens_barplot}).
Participants used the \imppass{} feature for 287 emails (\pct{75.93}) and accepted an improved email for 274 emails (\pct{72.49}).
When the \imppass{} feature was used at least once, an improved email was requested on average 1.35 times (SD 0.95, median 1.00) and accepted 1.13 times (SD 0.47, median 1.00) per email.
The last accepted improved email was identical to the sent email in \pct{83.94} of all cases.
When participants made changes these had a mean edit distance of 72.73 (SD 97.10, median 37).


\subsubsection{Interaction with Full Email Generation (\modemail)}
With this UI, \pct{71.98} of the first generated replies were accepted; otherwise, a new generation was requested.
Users spent \mins{0.43} (\pct{21.20}) on the incoming email screen, \mins{1.31} (\pct{65.09}) on the generation view and \mins{0.28} (\pct{13.71}) on the editing screen (\cref{fig:time_spent_on_screens_barplot}).

\subsubsection{Workflow Analysis}\label{sec:results_workflows}


\begin{figure}
    \centering
    \includegraphics[width=\minof{\columnwidth}{0.66\textwidth}]{figures/sentence_based_workflow_scatterplot}
    \caption{Analysis of workflows with \modeourstxt: Each point is one email and its position is the state of the drafting process at the moment when the user switched from the first screen (\cref{fig:teaser}.1) to the second (\cref{fig:teaser}.2). Concretely, the x-axis shows normalised time (0-\pct{100}), i.e. temporal progression. The y-axis shows normalised length, i.e. draft progression. Note that y-values >\pct{100} are possible if an intermediate draft is longer than the final version. Colour and marker shape indicate if the \imppass{} feature was used or not. The figure reveals three clusters: \textit{(1) Bottom left} -- here, people skipped to the second screen and used the \imppass{} feature to generate a draft. \textit{(2) Top right} -- mostly drafting on the first screen, with light manual editing on the second. \textit{(3) In between} -- partly drafting on the first screen and finalising it with AI on the second one.}
    \Description{This figure displays a scatter plot analysing workflows with content-driven local responses, focusing on when participants switched from the first screen to the second screen during the email drafting process.
    X-axis (Progress in Time [\% total time]): This axis represents the normalised time (ranging from 0 to 100) indicating the temporal progression of the drafting process.
    Y-axis (Draft Progress [\% final length]): This axis represents the normalised length of the draft at the moment of switching screens. Values greater than 1 are possible if an intermediate draft was longer than the final version.
    Colour Coding: The colour of each point indicates whether the "improvement pass" feature was used:
    Blue points represent emails where the improvement pass feature was used.
    Orange points indicate emails where it was not used.
    The scatter plot reveals three distinct clusters of participant behaviour:
    Bottom Left Cluster:
    Participants in this group quickly moved to the second screen and utilised the improvement pass feature to generate a draft with minimal work on the first screen.
    Top Right Cluster:
    Participants in this group spent most of their time drafting on the first screen, with only light manual editing on the second screen.
    In Between Cluster:
    This group represents participants who partly drafted on the first screen and then finalised the draft with AI assistance on the second screen.
    This analysis highlights the different strategies participants used during the email drafting process, depending on their interaction with the interface and the improvement pass feature.}
    \label{fig:workflow_scatterplot}
\end{figure}

\begin{figure}
    \centering
    \includegraphics[width=\minof{\columnwidth}{0.75\textwidth}]{figures/time_spent_on_screens.pdf}
    \caption{Participants spent their time on different screens and thus different aspects. The figure shows the means for the time spent on screens that focus on reading the incoming email vs on screens that focus on responding (colour). Borders indicate which steps required AI (solid) or offered it optionally (dashed). For \textit{\modemanual}, users read the email, then spend most of the time writing the reply. For \textit{\modeours}, the local response screen (\cref{fig:teaser}.1) enables reading and responding in parallel (striped), followed by responding on the second screen (\cref{fig:teaser}.2), both with optional AI (sentence suggestions, \imppass{}). In contrast, \textit{\modemail} requires AI after the initial reading phase to generate the response, which can then be manually edited.}
    \Description{This figure displays a bar plot analysing times spent on each step in the answering process. The figure shows the means for the time spent on screens that focus on reading the incoming email vs on the screens that focus on responding. For NoAI, users read the email, then spend most of the time writing the reply. For CDLR, the local response screen enables reading and responding in parallel, followed by responding on the second screen, both with optional AI (sentence suggestions, improvement pass). In contrast, MSG requires AI after the initial reading phase to generate the response, which can then be manually edited.}
    \label{fig:time_spent_on_screens_barplot}
\end{figure}

For \modeours, we discovered three main workflows by plotting when people switched from the first screen (\cref{fig:teaser}A) to the second (\cref{fig:teaser}B). \cref{fig:workflow_scatterplot} reveals three clusters: (1) Sometimes people went straight to the second screen and used the \imppass{} feature to create a draft. (2) Alternatively, they spent most of their time drafting on the first screen, with light manual editing on the second. (3) Finally, people partially drafted on the first screen and finished it on the second screen, using AI.
We fitted a GMM\footnote{Gaussian Mixture Model with 3 components using \url{https://scikit-learn.org/}} to estimate the number of emails: Cluster 1 had 136, cluster 2 had 54, and cluster 3 had 188 emails.

We also examined the relationship of the incoming email's length and whether people skipped the local response screen without entering any text. This was significant (\cref{tab:lmm_overview2}, row 2): Each additional word (i.e. 5 additional characters) in the incoming email is associated with a \pct{2.46} decreased chance of skipping the local response step in \modeours{}. 

For \modemail, we found that in most cases (\pct{77.8} of emails) people sent the generated drafts without manually editing them further. When they indeed edited them (\pct{22.2} of emails), the mean edit distance between generated and edited version was 64.87 (SD 67.48, median 44.50). This corresponds to typing about twelve words~\cite{kristensson2014inviscid}.

We also analysed how people prompted with \modemail: In a majority of cases (\pct{82.01} of emails), participants entered a prompt right away. Otherwise, they generated text solely based on the information in the incoming email. In half of those cases (\pct{51.47}), participants did not accept the result and generated another. In comparison, such a regeneration was only needed in \pct{22.9} of the cases where participants entered a prompt. We observed a learning effect for some: \pct{40} of people started entering a prompt if they were not happy with the initial result generated without a prompt. %



\subsection{Perception of Interaction}\label{sec:results_perception}
We analysed participants' perception of the three UIs.


\subsubsection{In-app Questionnaire (Likert Data)}\label{sec:results_in_app}


\begin{figure*}[t]
    \centering
    \includegraphics[width=\linewidth]{figures/inapp_likert_items.pdf}
    \caption{Likert results on perception of the UIs and interaction, rated after each email task. Overall, participants rated the AI-supported UIs higher on speed, quality, and helpfulness compared to the manual mode. However, the latter was rated higher on control.}
    \Description{This figure presents bar charts displaying Likert scale results from participants' perceptions of different UIs and their interactions, rated after completing email tasks. 
    The figure is divided into four sections, each comparing three UIs: NoAI (manual mode), CDLR (AI-supported), and MSG (AI-supported).
    Top-Left Chart: The app interface was helpful
    The NoAI interface received mixed responses, with a significant portion of participants disagreeing or remaining neutral, and fewer strongly agreeing.
    Both CDLR and MSG interfaces were rated more positively, with a larger number of participants agreeing or strongly agreeing that the interfaces were helpful.
    Top-Right Chart: The app interface helped me reply to the email quickly
    The NoAI interface again had a more varied response, with some participants disagreeing or remaining neutral, while others agreed.
    CDLR and MSG interfaces were rated highly for helping users reply quickly, with the majority of participants agreeing or strongly agreeing.
    Bottom-Left Chart: The app interface helped me write a good reply
    Similar to the other charts, the NoAI interface had a mix of responses, with fewer participants strongly agreeing.
    CDLR and MSG interfaces were again rated highly, with most participants agreeing or strongly agreeing that these interfaces helped them write good replies.
    Bottom-Right Chart: I was in control of the content of my reply
    For this aspect, the NoAI interface was rated slightly higher, with more participants strongly agreeing that they felt in control of their reply content.
    Although CDLR and MSG interfaces were also rated positively, there was a slight decrease in the number of participants who strongly agreed compared to the NoAI interface.
    In summary, the figure shows that participants generally rated the AI-supported UIs (CDLR and MSG) higher in terms of speed, quality, and helpfulness. 
    However, the manual mode (NoAI) was rated slightly higher in terms of giving users a sense of control over their replies.}
    \label{fig:inapp_likert_items}
\end{figure*}

Participants rated four Likert items in the app after each email (\cref{fig:inapp_likert_items}). 
We found statistically significant \revision{effects of \ivmode{} on all four items -- speed (\artf{2}{996.11}{433.71}{<.0001}, \petasq{.46}), control (\artf{2}{996.11}{20.60}{<.0001}, \petasq{.04}), quality (\artf{2}{996.13}{466.99}{<.0001}, \petasq{.48}), and helpfulness (\artf{2}{996.1}{430.61}{<.0001}, \petasq{.46}).}
\revision{Concretely,} \modeours{} and \modemail{} were both rated significantly higher than \modemanual{} on speed \revision{(\modeours{}: \artc{996}{23.34}{<.0001}; \modemail{}: \artc{996}{27.25}{<.0001})}, quality \revision{(\modeours{}: \artc{996}{25.72}{<.0001}; \modemail{}: \artc{996}{27.18}{<.0001})}, and helpfulness \revision{(\modeours{}: \artc{996}{24.65}{<.0001}; \modemail{}: \artc{996}{26.14}{<.0001})}. They were both rated significantly lower than \modemanual{} on control \revision{(\modeours{}: \artc{996}{-5.88}{<.0001}; \modemail{}: \artc{996}{-5.17}{<.0001})}. 
The only significant difference between the UIs with AI was that \modemail{} was rated higher on speed than \modeours{} \revision{(\artc{996}{3.91}{=.0001})}. 


\subsubsection{In-App Feedback}
We reviewed the in-app feedback optionally provided after each reply.
For both AI modes it was overwhelmingly positive, such as: ``Im really enjoying this kind of AI help mode.'' (P60\oldId{P1351}, \modemail), ``It made work easy'' (P76\oldId{P1370}, \modemail), ``Very smooth process, good suggestions for each part.'' (P47\oldId{P1329}, \modeours), and ``This made my reply look way better.'' (P81\oldId{P1375}, \modeours).



The negative feedback was less homogeneous.
For \modemail, around half of these critiques highlighted difficulties in getting the AI to incorporate specific information, such as: ``the AI who seemed to resist wanting to offer access to my colleague'' (P25\oldId{P1298}); or ``had a bit of trouble trying to get the AI to properly acknowledge that \$200 was okay [...]''  (P37\oldId{P1312}). 
Related, P53\oldId{P1338} noted that ``Control in replying was lacking, It didn't give me many options to 'add' ideas of my own.''
People found ways to steer the system; P47\oldId{P1329} said that ``I had to adjust the prompt a few times to get the sort of reply that I was looking for, but it did generate a good reply overall and I was satisfied with the end result.''

Notably, issues with including specific information were rarely mentioned for \modeours.
Most of the negative feedback instead concerned the tone: ``This was too wordy for an informal email.'' (P56\oldId{P1341}), %
and ``The ai was helpful but it made the response feel slightly too formal and professional.'' (P49\oldId{P1333}) %

For the manual mode, people ``had no issues, [and] felt able to use the platform freely and there was no technical faults'' (P7\oldId{P1276}), and that ``It was just like normal email.'' (P60\oldId{P1351}). %


\subsubsection{Favourite Reply Support}\label{sec:results_fav_mode}
Only \pct{4} (5 people) preferred \modemanual, %
\pct{49.2} (62 people) favoured \modemail, and \pct{43.7} (55 people) preferred \modeours.
The remaining \pct{3.2} (4 people) did not pick a favourite.

Notably, the high-level code ``Efficiency'' occurred in \pct{56.45} of comments for \modemail{} and \pct{32.73} for \modeours{}.
``Quality'' in \pct{25.82} for \modemail{} and \pct{30.91} for \modeours{}.
``Control'' in \pct{8.07} for \modemail{} and \pct{29.09} for \modeours{}.
``Tailoring'' zero times for \modemail{} and \pct{5.46} for \modeours{}.
The remaining comments were assigned the code ``Others'' (e.g. P105\oldId{P1405} ``just liked the interface'' of \modemail{} and P76\oldId{P1370} favoured \modeours{} because it supported them in being creative).

Two out of the four people who did not select a favourite stated that they liked both depending on the ``context'' (P77\oldId{P1371}).
For instance, P19\oldId{P1290} explained that ``both have different advantages in different situations. Single prompt allows to produce a full email much faster so is handy when you are short of time but still want to respond. Sentence based provides the user the ability to create a much more tailored email which can cover all bases.''

The \pct{4} (5 people) who preferred \modemanual{} said they were ``used to it'' (P4\oldId{P1273}) or ``confident in [their] writing ability'' (P78\oldId{P1372}).
















\subsubsection{Summary}
People perceived AI features as helpful and preferred having them. They were divided about their favourite and perceived meaningful tradeoffs between the two designs with AI on control vs efficiency: While people felt in control with all UIs (\cref{sec:results_in_app}), when reflecting on their favourite, they mentioned control aspects relatively more frequently for \modeours{} than \modemail{} -- and vice versa for efficiency.




\subsection{Analysis of Emails}\label{sec:results_emails}
We analysed the content of the emails. \cref{fig:quality_boxplots} shows four box plots as an overview.

\begin{figure*}[t]
    \centering
    \includegraphics[width=\linewidth]{figures/quality_boxplots.pdf}
    \caption{Four measures of email characteristics from our study. The plots show email length \textit{(left)} and rate of spelling/grammar/punctuation errors \textit{(centre left)}. Moreover, we measured lexical diversity with the distinct-2 score \textit{(centre right)}, which is defined as an email's number of distinct bigrams divided by its number of words (higher = more diverse). Finally, we measure diversity between emails \textit{(right)}, based on the cosine similarity of vector embeddings (higher = less diverse). Overall, all AI features increased reply lengths, decreased error rates, and lowered diversity (all sig.).  \revision{For \modeours{} we further distinguish between emails where the improvement pass feature was used (impr.) and those where it was not (no impr.).}
    See \cref{sec:results_emails} for details.}
    \Description{This figure presents four box plots comparing different characteristics of emails across three user interfaces: NoAI (manual mode), CDLR (AI-supported with and without the optional improvement pass feature), and MSG (AI-supported). The metrics being compared are:
    Email Length (Left):
    This plot shows the length of the emails in characters.
    AI-supported interfaces (CDLR and MSG) led to longer emails compared to the NoAI interface, with MSG producing the longest emails.
    Error Rate (Center Left):
    This plot displays the rate of spelling, grammar, and punctuation errors relative to the number of characters in the email.
    Both CDLR and MSG reduced the error rate compared to NoAI, with slightly better performance for the improvement pass-enabled CDLR and MSG interfaces.
    Diversity within Emails (Center Right):
    The distinct-2 score measures the lexical diversity within individual emails by calculating the number of unique bigrams (word pairs) relative to the total word count. A higher score indicates greater diversity.
    NoAI emails had higher within-email diversity compared to CDLR and MSG, which showed similar scores indicating reduced lexical diversity.
    Diversity between Emails (Right):
    This plot measures the cosine similarity between vector embeddings of emails, with higher scores indicating less diversity between emails.
    Emails generated using CDLR and MSG were more similar to one another, as indicated by higher cosine similarity scores, compared to those written using the NoAI interface.
    Overall, the figure demonstrates that AI-supported interfaces (CDLR and MSG) increased email length, reduced error rates, and resulted in slightly less diversity both within and between emails. For detailed analysis, refer to the corresponding section in the paper (Section 6.2).
    }
    \label{fig:quality_boxplots}
\end{figure*}


\subsubsection{Email Lengths}\label{sec:results_lengths}

On average, emails written without AI were 302.5 characters long (SD 169.7, median 267).
\modemail{} resulted in 536.1 (SD 319.0, median 447.5), while \modeours{} had 483.0 (SD 285.4, median 382). For \modeours, using the \imppass{} feature had a mean length of 523.9 (SD 294.1, median 412.0), while not using it had 390.6 (SD 241.5, median 324.5).
These differences were significant (\cref{tab:lmm_overview}, row 4): People wrote significantly longer replies with the AI features than without them, and significantly more so with \modemail{} (\pct{77} increase) than with \modeours{} (\pct{27} increase). Using the \imppass{} feature in \modeours{} significantly increased this further for that UI (\pct{38} increase). In summary, all AI features significantly increased text lengths.


\subsubsection{Error Rates}\label{sec:results_errors}
We checked grammar and spelling with the language-tool-python\footnote{\url{https://pypi.org/project/language-tool-python/}} library.
Per email, we recorded the minimum of British English and American English spell checking to avoid penalising spelling differences. %
Manual writing had the highest mean error rate of .00375 errors per character.  Both AI versions were about half of that: \modemail{} had .00176 and \modeours{} had .00182. Using the \imppass{} feature in \modeours{} contributed to reducing errors (mean .00147 when using it vs .00260 when not). All these differences were significant (\cref{tab:lmm_overview}, row 5). In summary, all AI features significantly reduced error rates.

\subsubsection{Diversity Across Emails}\label{sec:results_email_similarity}
We analysed the semantic similarity between emails, following related work~\cite{padmakumar2024diversity}. We computed the cosine similarity of the vector embeddings of all pairs of emails written with the same mode and for the same briefing, using the Sentence Transformers library (SBERT\footnote{\url{https://sbert.net}, specifically \url{https://huggingface.co/sentence-transformers/all-MiniLM-L6-v2}})~\cite{reimers2019sbert}.
As expected from the literature, manual emails had the lowest mean pairwise similarity (.582), that is, they were the most diverse. \modemail{} had the highest similarity (.756), followed by \modeours{} (.726). 
Using the \imppass{} feature in \modeours{} contributed to the increase (.676 without using it vs .749 with it). 
All these differences were significant (\cref{tab:lmm_overview}, row 6). In summary, all AI features significantly reduced semantic diversity.


\subsubsection{Diversity Within Emails}\label{sec:results_lexical_diversity}
We analysed lexical diversity, as in related work~\cite{Fu2023sentencevsmessage}, with the distinct-2 metric, defined as the number of distinct bigrams divided by the total number of words.
Our results match the related work: Writing without AI had the highest mean lexical diversity (.950), \modemail{} lowered it to .930, and \modeours{} had .924.
Using the \imppass{} feature in \modeours{} (almost) closes the gap between \modeours{} and \modemail{} (.915 without it vs .927 with it). 
All these differences were significant (\cref{tab:lmm_overview}, row 7). In summary, lexical diversity was significantly affected by all AI features, in the way we would expect from related work~\cite{Fu2023sentencevsmessage}: Sentence-level generations decreased it more than message-level generations.







\subsubsection{Email Structure}\label{sec:results_structure}

Salutations were missing in \pct{8.5} of manually written replies and in \pct{10.6} with \modeours. All replies with \modemail{} had salutations. All replies with \modeours{} and the \imppass{} feature had a salutation.
Similarly, only one email with \modemail{} lacked a closing statement, compared to \pct{14.3} for \modemanual{} and \pct{9.0} for \modeours. Again, when the \imppass{} was used in \modeours, all emails ended with a closing signature.


\subsubsection{Briefing Conformity}\label{sec:results_briefing}
Each email reply task showed a briefing that asked participants to respond with certain information (see \cref{sec:procedure_email_tasks}). This allowed us to analyse if their emails conformed to this or not (see \cref{sec:quality_m}).
This varied across the UIs: With \modeours, \pct{23} of emails missed a key aspect of the study briefing, compared to \pct{18} for \modemail{} and \pct{13} for \modemanual.

The differences between \modeours{} and the other two UIs were significant (\cref{tab:lmm_overview2}, row 1). %
However, people's prompting behaviour had a larger impact here: Across \modeours{} and \modemail{}, generating a full reply without any own input (83 emails in the data) missed a key aspect of the briefing in half of the cases (\pct{49}). We return to this in the discussion (\cref{sec:discussion_methods})



\subsubsection{Subjective Assessment of Quality}\label{sec:result_quality}
All emails were read by (at least) two researchers. %
While length and structure varied (\cref{sec:results_lengths}, \cref{sec:results_structure}), we did not notice ``nonsense'' responses. Even those emails which did not conform to the briefings (\cref{sec:results_briefing}) reflected the general topic and most would have been believable replies. We noticed that sensible replies can vary drastically -- from short responses to elaborate, formal emails. The latter mostly coincided with message-level AI generation. We do not consider this an issue of quality since we did not specify a level of formality to follow. Overall, based on our subjective assessment, we thus concluded that reply quality was suitable across all UIs. 

\section{Conclusion}\label{sec:conclusion}
This work introduces a novel approach to TOT query elicitation, leveraging LLMs and human participants to move beyond the limitations of CQA-based datasets. Through system rank correlation and linguistic similarity validation, we demonstrate that LLM- and human-elicited queries can effectively support the simulated evaluation of TOT retrieval systems. Our findings highlight the potential for expanding TOT retrieval research into underrepresented domains while ensuring scalability and reproducibility. The released datasets and source code provide a foundation for future research, enabling further advancements in TOT retrieval evaluation and system development.


\bibliographystyle{ACM-Reference-Format}
\bibliography{main}

\appendix

% \section{Example Appendix}
% \label{sec:appendix}

% This is an appendix.

\section{Extracting data with \spike}\label{sec:appendix-spike}

We found two patterns of statements, which can convey a clear sentiment, and built queries upon these patterns to extract statements from \spike. Examples for all types of statements are presented in Table~\ref{tab:base-sentence}.


First, are statements in which the verb in the statements is a verb with clear sentiment, that often implies the sentiment of the entire statement. E.g., `wastes', `rejects', `fails' are negative verbs, while verbs like `enjoys', `succeeds', `empowers', conveys positive statements. 

The second pattern of statements that we found suitable for conveying a clear sentiment, are statements which describe some event/action, and its consequences, where often the adjective that describes the consequences holds information whether it is positive or negative. 

Next, we needed to label and filter them due to two main issues. First, we needed to handle the cases in which negation words appear in the statement and flips the sentiment. For example, a statement like ``We did not enjoy the show'' includes a positive verb (enjoy), but the negation flips its sentiment to be a negative statement. Another issue we encountered is that there are many statements which are irrelevant to our case, even though they match the positive/negative patterns, for example ``I couldn't sympathize with the shopping aspect of the book since I hate to shop .'' does not convey any clear sentiment, despite the use of the verb `hate'.



\begin{table*}[t]
\centering
\resizebox{\textwidth}{!}{
\begin{tabular}{ll}
\toprule
\textbf{Category} & \textbf{Example Sentence} \\
\midrule
Positive Verb & ``To my surprise I did \textbf{enjoy} the book and the characters .'' \\
Negative Verb & ``This dock has done nothing but provide frustration and \textbf{waste} a great deal of my time trying to get it to work properly .'' \\
\midrule
Positive Outcome & ``This bag provides \textbf{good} protection for my snare drum at a really \textbf{good} price .'' \\
Negative Outcome & ``For me , Aspartame causes \textbf{bad} memory loss and \textbf{nasty} gastrointestinal distress .'' \\
\bottomrule
\end{tabular}
}
\caption{Examples for base statements collected using \spike. The words that inflect the sentiment are in bold.}
\label{tab:base-sentence}

\end{table*}

\subsection{SPIKE Queries}
\begin{enumerate}
    \item :something :[{pos/neg verbs}]develops
    \item:something :[{pos/neg adjectives}]badly :[{cause synonym}]causes :something
\end{enumerate}

\subsection{Word Lists}

\paragraph{Positive verbs.} achieve, admire, affirm, appreciate, aspire, awe, bless, blossom, celebrate, cherish, comfort, contribute, delight, donate, elevate, empower, enchant, encourage, energize, engage, enjoy, enrich, enthuse, excel, fervor, flourish, fortify, glisten, glow, gratitude, grow, harmonize, heal, illuminate, innovate, inspire, invigorate, laugh, learn, liberate, love, motivate, nourish, nurture, praise, prosper, radiate, rally, refresh, rejoice, renew, revel, revere, revitalize, savor, shine, smile, soar, spark, sparkle, stimulate, strengthen, succeed, support, synergize, thrive, unite, uplift, volunteer, adore, amaze, boost, captivate, win.

\paragraph{Negative verbs.} abandon, abuse, accuse, alienate, begrudge, betray, bewilder, blame, collapse, complain, condemn, confuse, contradict, criticize, decay, deceive, decline, defeat, demoralize, deny, despair, destroy, deteriorate, devalue, discourage, discriminate, dishearten, dismantle, dismiss, dissolve, doubt, exploit, fail, falter, fear, frustrate, grieve, harass, hate, hurt, ignore, inhibit, intimidate, lose, mock, overlook, overwhelm, pollute, punish, regress, reject, repress, resent, sabotage, shatter, sicken, stifle, suffer, suffocate, suppress, terrorize, torment, undermine, violate, waste, weaken, whine, withdraw, withhold, worry.

\paragraph{Positive adjectives.}
admirable, lucky, enjoyable, magnificent, enthusiastic, marvelous, euphoric, amazing, excellent, exceptional, amused, excited, amusing, extraordinary, nice, noble, outstanding, appreciative, fabulous, overjoyed, astonishing, fantastic, benevolent, fortunate, pleasant, blissful, pleasurable, brilliant, positive, glad, prominent, good, proud, charming, cheerful, reliable, gracious, grateful, clever, great, happy, superb, superior, terrific, incredible, tremendous, inspirational, delighted, delightful, joyful, joyous, uplifting, wonderful, lovely.

\paragraph{Negative adjectives.}
sad, angry, upset, disgusting, boring, disappointing, frustrating, annoying, miserable, terrible, deppressing, unhappy, melancolic, heartbreaking, Furious, iritating, emberessing, horrible, stupid, unlucky, negative, bad.

\paragraph{``Causes'' synonym.} causes, creates, generates, prompts, produces, induces, yields, affects, invokes, effectuates, results, encourages, promotes, introduces, begets, engenders, occasions, develops, starts, contributes, initiates, inaugurates, establishes, begins, cultivates, acquires, provides, launches.



\section{Adding Framing}\label{sec:framing-prompts}

\begin{table*}[]
\resizebox{\textwidth}{!}{%
\begin{tabular}{@{}lll@{}}
\toprule
\textbf{Base Sentence} &
  \textbf{Base Sentiment} &
  \textbf{Opposite Framing Sentence} \\ \midrule
``To my surprise I did enjoy the book and the characters .'' &
  Positive &
  \begin{tabular}[c]{@{}l@{}}``To my surprise I did enjoy the book and the characters, \textbf{even though}\\ \textbf{it had a disappointing ending}. ''\end{tabular} \\ \midrule
\begin{tabular}[c]{@{}l@{}}``For me , Aspartame causes bad memory loss and nasty \\ gastrointestinal distress .''\end{tabular} &
  Negative &
  \begin{tabular}[c]{@{}l@{}}``For me, Aspartame causes bad memory loss and nasty gastrointestinal \\ distress, \textbf{but this has encouraged me to seek out healthier, natural} \\ \textbf{alternatives and cultivate a balanced diet} .''\end{tabular} \\ \bottomrule
\end{tabular}%
}
\caption{Sentences after framing. Positive sentences are added with negative framing, and vice-versa. The opposite framing is in bold.}
\label{tab:after-framing}
\end{table*}

Example for statements after framing are presented in in Table~\ref{tab:after-framing}.

\subsection{Framing Prompts}

\begin{enumerate}
    \item ``Here is an example of a base statement with a negative sentiment: I failed my math test today. Here is the same statement, after adding a positive framing: I failed my math test today, however I see it as an opportunity to learn and improve in the future. Here is a negative statement: <statement> Like the example, add a positive suffix or prefix to it. Don't change the original statement.''

    \item ``Here is an example of a base statement with a positive sentiment: I got an A on my math test. Here is the same statement, after adding a negative framing: I got an A on my math test. I think I spent too much time learning to it though. Here is a positive statement: <statement>. Like the example, add a negative suffix or prefix to it. Don't change the original statement.''
\end{enumerate}

\section{Annotation Platform}\label{sec:mturk-appendix}

We select a pool of 10 qualified workers who successfully passed our qualification test, which consisted of 20 base statements (unframed), for which annotators were expected to achieve perfect accuracy. The estimated hourly wage for the entire experiment was approximately 14USD per hour.

Screenshot of the annotation platform is presented in Figure~\ref{fig:annotation-platform}.

\begin{figure*}
    \centering
    \includegraphics[width=\linewidth]{images/annotation.png}
    \caption{Screenshot of the annotation platform.}
    \label{fig:annotation-platform}
\end{figure*}

\section{Models}\label{sec:appendix-models}

We ran the open models via together-ai API.\footnote{\url{https://www.together.ai}} 
The list of models we used are:
\begin{itemize}
    \item "google/gemma-2-9b-it"
    \item "google/gemma-2-27b-it"
    \item "mistralai/Mistral-7B-Instruct-v0.3"
    \item "mistralai/Mixtral-8x7B-Instruct-v0.1"
    \item "mistralai/Mixtral-8x22B-Instruct-v0.1"
    \item "meta-llama/Llama-3-8b-chat-hf"
    \item "meta-llama/Llama-3-70b-chat-hf"
\end{itemize}

For \gpt{}, we used the OpenAI api, with "gpt-4o-2024-08-06".\footnote{\url{https://platform.openai.com/docs/overview}}

\begin{figure*}[htbp]
    \centering
    % First Subfigure
    \begin{subfigure}{0.49\textwidth} % Adjust width as needed
        \centering
        \includegraphics[width=\textwidth]{images/orig_negative_models_distribution.png} % Replace with your image path
        \caption{Sentences that are \textbf{negative} in their original form.}
        \label{fig:negative-flip}
    \end{subfigure}
    % \hfill % Adds horizontal space between subfigures
    % Second Subfigure
    \begin{subfigure}{0.49\textwidth}
        \centering
        \includegraphics[width=\textwidth]{images/orig_positive_models_distribution.png} % Replace with your image path
        \caption{Sentences that are \textbf{positive} in their original form.}
        \label{fig:positive-flip}
    \end{subfigure}
    \caption{Proportion of sentences for which LLMs flipped sentiment, became neutral, or retained the original sentiment when presented with opposite sentiment framing. For example, this measures the percentage of sentences originally labeled as positive, that were labeled as negative after applying negative framing (and vice versa).
    }
    \label{fig:flip-proportion}
\end{figure*}
\begin{figure}
    \centering
    \includegraphics[width=\linewidth]{images/pairwise_correlation_matrix.png}
    \caption{Pairwise Pearson correlation coefficients between predictions from different models, indicating the degree of similarity in their behavior under opposite sentiment framing scenarios.}
    \label{fig:heatmap-models}
\end{figure}

\begin{figure}
    \centering
    \includegraphics[width=\linewidth]{images/humans_distribution.png}
    \caption{Proportions of sentences where annotators agreed on the extent of sentiment shift after applying opposite sentiment framing. The bars represent the percentage of sentences with 0 to 5 annotators agreeing on a sentiment shift.}
    \label{fig:humans-flip}
\end{figure}





\end{document}
