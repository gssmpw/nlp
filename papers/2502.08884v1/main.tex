\documentclass[acmtog, nonacm]{acmart}

\AtBeginDocument{%
  \providecommand\BibTeX{{%
    \normalfont B\kern-0.5em{\scshape i\kern-0.25em b}\kern-0.8em\TeX}}}

\acmJournal{TOG}

\usepackage{booktabs} % For formal tables
\usepackage{savesym}
\savesymbol{zifour@default}
\savesymbol{zifour@scaled}
\usepackage{enumerate}
\usepackage{multirow} 
\usepackage[normalem]{ulem}
\usepackage{inconsolata}

\usepackage[ruled]{algorithm2e} % For algorithms
\renewcommand{\algorithmcfname}{ALGORITHM}

\SetAlFnt{\small}
\SetAlCapFnt{\small}
\SetAlCapNameFnt{\small}
\SetAlCapHSkip{0pt}

\citestyle{acmauthoryear}

\setlength{\abovecaptionskip}{5pt plus 2pt minus 2pt}

\newcommand{\thought}[1]{{\color[rgb]{0.2,0.39,0.66}(#1)}}
\newcommand{\todo}[1]{{\color[rgb]{1.0,0.0,0.0}(#1)}}
\newcommand{\hsh}[1]{{\color{green!50!black} Henrik: #1}}
\newcommand{\st}[1]{{\color{red!50!black} Sebastian: #1}}

\newcommand{\ulm}[1]{_{\scaleto{\mathrm{#1}}{3pt}}}
\newcommand\at[2]{\left.#1\right|_{#2}}











\newtheorem{assumption}{Assumption}

\DeclareMathOperator*{\argmax}{arg\,max}
\DeclareMathOperator*{\argmin}{arg\,min}

\newcommand{\swname}[1]{\texttt{#1}}
\newcommand{\ie}{i\/.\/e\/.,\/~}
\newcommand{\eg}{e\/.\/g\/.,\/~}
\newcommand{\cf}{cf\/.\/~}

\newcommand{\fig}{Fig\/.\/~}
\newcommand{\defn}{Def\/.\/~}
\newcommand{\sect}{Sec\/.\/~}
\newcommand{\tabl}{Tab\/.\/~}
\newcommand{\algo}{Algorithm~}
\newcommand{\theo}{Theorem~}

\newcommand{\bnnl}{3 hidden layers}
\newcommand{\bnnn}{50 neurons}
\newcommand{\bnna}{tanh activations}

\newcommand{\capt}[1]{\mdseries{\emph{#1}}}

\newcommand{\videolink}{at \url{https://youtu.be/_d7AqTRjz6g}}
\newcommand{\codelink}{\url{https://github.com/wheelbot/mini-wheelbot}}

\newcommand{\fakepar}[1]{\vspace{0mm}\noindent\textbf{#1.}}

\newcommand{\needref}{\textcolor{red}{[REF]}}

\newcommand{\plotfontsize}{9pt}

\begin{document}

\title[\methodname: designing a library of procedural 3D shape abstractions with Large Language Models]{\methodname: Designing a library of procedural 3D shape abstractions \\with Large Language Models}


\author{R. Kenny Jones}
\email{russell_jones@brown.edu}
\affiliation{%
    \institution{Brown University}
    \country{USA}
}

\author{Paul Guerrero}
\email{guerrero@adobe.com}
\affiliation{%
    \institution{Adobe Research}
    \country{United Kingdom}
}

\author{Niloy J. Mitra}
\email{n.mitra@cs.ucl.ac.uk}
\affiliation{%
    \institution{University College London and Adobe Research}
    \country{United Kingdom}
}

\author{Daniel Ritchie}
\email{daniel\_ritchie@brown.edu}
\affiliation{%
    \institution{Brown University}
    \country{USA}
}

\begin{abstract}
Humor is a social binding agent. It is an act of creativity that can provoke emotional reactions on a broad range of topics. Humor has long been thought to be “too human” for AI to generate. However, humans are complex, and humor requires our complex set of skills: cognitive reasoning, social understanding, a broad base of knowledge, creative thinking, and audience understanding. We explore whether giving AI such skills enables it to write humor. We target one audience: Gen Z humor fans. We ask people to rate meme caption humor from three sources: highly upvoted human captions, 2) basic LLMs, and 3) LLMs captions with humor skills. We find that users like LLMs captions with humor skills more than basic LLMs and almost on par with top-rated humor written by people. We discuss how giving AI human-like skills can help it generate communication that resonates with people. 

\end{abstract}

\begin{CCSXML}
<ccs2012>
<concept>
<concept_id>10010147.10010371.10010396</concept_id>
<concept_desc>Computing methodologies~Shape modeling</concept_desc>
<concept_significance>500</concept_significance>
</concept>
</ccs2012>
\end{CCSXML}
\ccsdesc[500]{Computing methodologies~Shape modeling}
\keywords{procedural modeling, shape analysis, shape abstraction, library learning, large language models, LLMs, semantics}


\begin{teaserfigure}
\centering
  \includegraphics[width=\linewidth]{figs/shapelib_teaser_v1.pdf}
  \caption{\methodname guides an LLM to design a library of procedural shape functions from a given set of (20) seed shapes and textual descriptions. Using an LLM prior makes the functions semantically interpretable and easy to edit, while aligning them with the seed shapes specializes the functions to a given domain and reduces LLM hallucinations. The library can be used to train a network for visual program induction that generalizes well beyond the seed shapes.
  }
  \label{fig:teaser}
\end{teaserfigure}


\maketitle

%!TEX root=main.tex

\section{Introduction}
% Decision-makers, analysts, data scientists, and policymakers frequently rely on data to draw conclusions and extract insights. This data-driven approach helps them identify actionable recommendations aimed at influencing an outcome of interest, such as increasing product satisfaction or income levels or decreasing the likelihood of experiencing serious health conditions \cite{galhotra2022hyper,lakkaraju2016interpretable,agrawal1994fast}. 
\revc{Prescriptions, or actionable recommendations, are commonly generated across various fields to influence key outcomes such as improving product satisfaction, enhancing economic policies, or increasing business efficiency. 
%Decision- or policy-makers, analysts, data scientists, and 
Policymakers in government, decision-makers in businesses, and data scientists in various fields, often rely on data-driven approaches to identify 
%actionable recommendations 
potential actions to influence an outcome of interest, such as increasing income levels or loan approval rates}.
% , or decreasing the likelihood of experiencing serious health conditions. 
%
While association or prediction-based methods are extensively used in practice to draw useful insights from data, they typically identify correlations among variables and may fail to reveal the underlying causal factors, i.e., which actions may result in an improved outcome, needed for informed decision-making. 
%For recommendations to be truly impactful, there must be a clear  explanation that justifies why a particular decision is appropriate for a specific subpopulation~\cite{sun2021treatment,plecko2022causal}. 

\emph{Causal analysis} or {\em causal inference}, therefore, is considered one of the most important requirements to generate prescriptions that are {\em actionable} and aligned with human reasoning~\cite{imbens2024causal}. Causal inference, and in particular {\em observational studies} for causal inference on collected data (when controlled trials are impossible due to cost or ethical reasons), have been extensively studied in the statistics and artificial intelligence (AI) literature for several decades \cite{rubin2005causal, pearl2009causal}. Motivated by this foundational work on causal inference, the notion of causality has also influenced the field of database research. The causal models from AI have been extended to relational databases \cite{salimi2020causal},  and causality has been incorporated into various data management tasks such as finding responsibilities of inputs toward query answers ~\cite{meliou2010causality, meliou2009so, meliou2014causality}, explanations for query answers \cite{roy2014formal, DBLP:journals/pacmmod/YoungmannCGR24}, data discovery~\cite{galhotra2023metam,youngmann2023causal}, data cleaning~\cite{pirhadi2024otclean,salimi2019interventional}, hypothetical reasoning \cite{galhotra2022causal}, and large system diagnostics~\cite{markakis2024sawmill,causalsim,sage, gudmundsdottir2017demonstration}. 


\revc{If-then rules are generally considered interpretable by humans~\cite{lakkaraju2016interpretable,guidotti2018local,van2021evaluating,pradhan2022interpretable,chen2018optimization}.
We give a concrete example of the difference between association and causation in generating prescriptions or recommended actions in the form of if-then rules below}:
\begin{example}	%
\label{example:ex1} {\bf Importance of causal prescriptions:}
Consider the Stack Overflow (SO) annual developer survey
\cite{stackoverflowreport}, where respondents from around the world answer
questions about their jobs and demographics. A sample of the dataset \reva{with a subset of the
attributes (there are 20 attributes)} is presented in \cref{tab:data}.
%
Alice, a researcher in the United Nations (UN) finance department, is interested in discovering ways to increase the salaries of high-tech employees worldwide. She is looking for a set of actionable recommendations 
%(that we call a prescription rules) 
to raise the overall average salary.
%
Using association-based approaches~\cite{chen2018optimization,lakkaraju2016interpretable}, she may discover that individuals residing in the US who identify as straight or heterosexual tend to earn higher salaries (see \cref{exp:quality} for full details). However, this observation merely indicates a correlation: people living in the US, for example, generally earn more than those outside the country. Their comparatively higher salaries are primarily attributable to the country's economy and are unrelated to their sexual orientation. Thus, this observation cannot be used as a prescription rule to increase salary. 
Our causal analysis, on the other hand, reveals that individuals aged 25-34 with dependents would benefit from working as front-end developers.
This results in a \$44,009 annual salary increase on average. \reva{Another observation is that students should pursue an
undergraduate major in CS. %Computer Science (CS). 
This can boost their salary by \$22,174 per year} (see details in \cref{sec:casestudy}).
\end{example}

%It has been incorporated into various tasks including . 
%Causal interventions are often more relatable and easier to understand, as they offer insight into the underlying reasons behind the recommendations and allow unraveling complex cause-effect relationships that govern our world~\cite{pearl2009causality}. Furthermore, causal interventions often have long-lasting effects~\cite{imbens2024causal}.

%, making it essential that the prescribed actions are not only actionable but also 

%causally consistent. 

%Decision makings, in particular, high-stak

\cut{
In this work, {we study the problem of generating causal insights (referred to as \emph{prescription rules}), which serve as actionable recommendations} to improve an outcome of interest.
Recent works have introduced causality to the field of database research~\cite{meliou2010causality,  meliou2014causality,salimi2020causal,10.14778/3554821.3554902}. It has been incorporated into various tasks including data discovery~\cite{galhotra2023metam,youngmann2023causal}, data cleaning~\cite{pirhadi2024otclean,salimi2019interventional}, and large system diagnostics~\cite{markakis2024sawmill,causalsim,sage, gudmundsdottir2017demonstration}. 
We propose using causal inference to generate prescription rules that are both actionable and justifiable.
}

While generating prescriptions based on causal inference may help in robust decision-making, causal prescriptions that solely consider the betterment of an outcome (like salary) are not enough in practice. 
It is well-known that decision-making in many high-stake applications (like hiring policy, or policy for approving loans by banks) may lead to disparate societal or economic impact on different sub-populations. 
As a shocking example from a recent work called 
%For example, 
CauSumX~\cite{DBLP:journals/pacmmod/YoungmannCGR24} that generates a set of causal explanations for an aggregated view, the explanations generated %by CauSumX %recommendations which 
suggest that male individuals do a Bachelor's degree to increase their salary while %suggesting that 
being an unmarried woman 
%the recommendation for women includes getting married 
has the most adverse effect on salary
(borrowed directly 
from Fig.~19 in~\cite{youngmann2024summarizedcausalexplanationsaggregate}). 
%We demonstrate the advantage of using causal reasoning to generate actionable recommendations and the limitations of not considering fairness requirements in the following example. 
We explored this further in the context of generating prescriptions and observed that prescriptions that are not fairness-aware can generate unfair outcomes to some subpopulations which we refer to as the {\em protected group}. Examples include women, Black, Latino, or Native Americans, individuals with a disability, countries with a weaker economy, or other protected groups specific to an application. %Here is a concrete example:


% Understanding the causal factors behind these recommendations is crucial to ensuring that decisions lead to fair and equitable outcomes, particularly in sensitive applications where biased decisions can perpetuate or even exacerbate societal inequalities.
% While prior work has extensively explored techniques for association rule mining~\cite{kumbhare2014overview}, recent efforts have focused on deriving causal explanations for individual data points or entire datasets~\cite{salimi2018bias,youngmann2022explaining,ma2023xinsight}. Although some of these methods produce causally consistent insights, the absence of fairness considerations in the process can lead to unfair outcomes, further reinforcing existing biases. For example, CauSumX~\cite{DBLP:journals/pacmmod/YoungmannCGR24} generates causal recommendation which suggest male individuals to do a Bachelor's degree to increase salary while the recommendation for women include getting married (borrowed directly from Figure~19 in the paper~\cite{youngmann2024summarizedcausalexplanationsaggregate}). 





%\emph{Causal inference} has been thoroughly studied in AI and Statistics~\cite{pearl2009causal,rubin2005causal}. Causal analysis is a vital tool in determining the effect of a \emph{treatment} on an \emph{outcome}, and has been used in decision-making in medicine \cite{robins2000marginal}, economics \cite{banerjee2011poor}, biology \cite{shipley2016cause}, and in high-stakes areas such as identifying the root causes of failures in critical infrastructure systems to prevent catastrophic outcomes. Recent works have introduced causality to the field of database research~\cite{meliou2010causality,  meliou2014causality,salimi2020causal,10.14778/3554821.3554902}. It has been incorporated into various tasks including data discovery~\cite{galhotra2023metam,youngmann2023causal}, query result explanation~\cite{salimi2018bias,youngmann2022explaining,DBLP:journals/pacmmod/YoungmannCGR24}, and large system diagnostics~\cite{markakis2024sawmill,causalsim,sage, gudmundsdottir2017demonstration}. We propose leveraging causal inference to generate interpretable and justifiable insights (referred to as \emph{prescription rules}), which serve as actionable recommendations to improve an outcome of interest. Causal reasoning is considered one of the most important requirements,  to generate insights that are actionable and aligned with human reasoning.




\begin{table*}[]
\footnotesize
    \centering
    	\caption{\textnormal{A subset of the Stack Overflow dataset.}}
         \label{tab:data}
    	% \vspace{-4mm}
  			\begin{tabular}[b]{|l|l|l|c|l|l|c|l|c|}
  			
				%\multicolumn{9}{c}{\textbf{Users}}\\ 
				\hline

				\textbf{ID}
    
    % \textbf{Country}& \textbf{Continent} 
    
    &\textbf{Gender} &\textbf{Ethnicity}&
				\textbf{Age} &\textbf{Role} &
				\textbf{Education} &\textbf{Country}&\textbf{Undergrad Major}&\textbf{Salary}
				\\ \hline

				1 &Male&White&26&Data Scientist & PhD& US&Computer Science&180k\\
    		2 &Non-binary&White&32&QA developer & Bachelor's degree& US&Mechanical Eng.&83k\\

 3 &Male&South Asian&29&C-suite executive  & Bachelor's degree & India&Computer Science&24k\\

  % 4 &Female&South Asian&25&Back-end developer  & Master's degree & India&Mathematics&7.5k\\

  4 &Female&East Asian&21&Back-end developer & Bachelor's degree & China&Computer Science&19k\\
  

        % $\ldots$ &  $\ldots$&  $\ldots$&  $\ldots$&  $\ldots$&  $\ldots$&  $\ldots$&  $\ldots$&  $\ldots$&  $\ldots$&  $\ldots$\\
    \hline
			\end{tabular}
            \vspace{-5mm}
\end{table*}




\begin{example}	%
\label{example:ex2}
{\bf Importance of fair prescriptions:}
Continuing Example~\ref{example:ex1}, while those causal prescription rules are highly beneficial for the overall population, they are considerably less effective for individuals residing in countries with a low GDP (indicating a weaker economy). For this group, the average expected increase in salary is only approximately \$13,000 per year (in contrast to \$44,009 for the entire group). % \sr{add which rule 44k or 25k} 
Consequently, implementing these rules would exacerbate the disparity between those living in countries with strong economies and those in countries with weaker economies.
\end{example}




% Our objective is to generate a small set of prescription rules aimed at increasing (or decreasing) an outcome of interest. This is framed as an optimization problem where the goal is to select the fewest prescription rules that maximize utility (i.e., the expected increase or decrease in the outcome). However, 

The example above shows that focusing solely on maximizing utility (\revc{i.e., increasing income}) can result in a scenario where only some of the population receive significant improvement, while others experience no benefit (\revc{only a small benefit for individuals from countries with weaker economies in our example}). Additionally, even if a large portion of the population receives recommendations, a protected subpopulation might not share the benefits and, worse, their situation could deteriorate, exacerbating inequalities.

Examples~\ref{example:ex1} and \ref{example:ex2} show that it is crucial to provide recommendations that are (1) {\em causal} for the outcome (beyond associations),  and (2) also {\em fair or equitable} in terms of the outcome for both the protected and non-protected groups. While recent work in database research
has focused on deriving {\em causal explanations} for individual data points, aggregated view, or entire datasets~\cite{salimi2018bias,youngmann2022explaining,ma2023xinsight, DBLP:journals/pacmmod/YoungmannCGR24}, and in particular \cite{DBLP:journals/pacmmod/YoungmannCGR24} has considered generating a set of causal explanations for an aggregated view that resemble a ruleset, 
%Although some of these methods produce causally consistent insights, 
the absence of fairness considerations in generating these causal explanations can lead to unfair outcomes for the protected group.
%further reinforcing existing biases.


%\red{We, therefore, enable users to incorporate various \emph{coverage and fairness constraints} along with the overall objective of improving an outcome of interest. }

\medskip
\noindent
\textbf{Our contributions.~} 
Motivated by the dual goals of generating causal and fair prescriptions for the betterment of an outcome, we introduce a {\em fairness-aware framework leveraging causal reasoning for generating a set of actionable prescription rules (ruleset)} called \sysName\ (\underline{Fair} \underline{CA}usal \underline{P}rescription).
%
Following research on fairness in data management~\cite{stoyanovich2020responsible,galhotra2022causal}, we assume the existence of a \emph{protected subpopulation}, defined by an attribute such as gender or race for people, or GDP of a country. Motivated by the causal explanation rules for an aggregated view \cite{DBLP:journals/pacmmod/YoungmannCGR24}, each prescription rule in our ruleset applies to a sub-population defined by a {\em grouping attribute}, and prescribes a {\em treatment or intervention} to improve the {\em outcome} for this sub-population. Fairness constraints ensure that the expected utility of the protected population is {\em comparable} to the utility of the unprotected individuals. We borrow the notions of \emph{group and individual fairness} from the fairness literature but tailor them for prescription rules. In addition to the fairness constraints, our coverage constraints ensure that a substantial fraction of the population and protected subpopulation receives at least one recommendation. 
%We demonstrate how such constraints ensure that the generated rules apply to a large portion of the population and ensure fairness through the following example.

\begin{example}
\label{ex:intro_example_3}
Continuing Examples~\ref{example:ex1} and \ref{example:ex2}, Alice uses our proposed system, called \sysName, to impose fairness and coverage constraints to discover useful and equitable recommendations for increasing salaries worldwide. In particular,
Alice chooses to implement a coverage constraint to ensure that the selected rules apply to a significant portion of people worldwide, including a sufficiently large number of individuals from countries with low GDP (the protected group). She also imposes a fairness constraint to ensure that the expected gains for both protected and non-protected groups are comparable.
\reva{She discovers, for example, that for individuals with 6-8 years of coding experience (a subpopulation comprising 21\% of the entire dataset and 25\% of the protected group), pursuing a bachelor’s degree in computer science will increase the expected salary by $\$14.9k$ for protected and by $\$17.8k$ for non-protected}. (See \cref{sec:casestudy} for more details.) This prescription rule applies to a large portion of the population and ensures fairness by providing a similar expected gain for both protected and non-protected groups, and the allowed difference of outcomes between these two populations may be adjusted by choosing appropriate thresholds in the fairness definitions. 
\end{example}


\noindent
Our main contributions are as follows. \\
%\begin{itemize}[leftmargin=*,topsep=0pt]
{\bf (1)} We {\bf develop a framework that generates a set of prescription rules to enhance an outcome of interest (Section~\ref{sec:problem})}. A prescription rule consists of a \emph{grouping pattern} and an \emph{intervention pattern}, representing the target subpopulation and the actionable recommendation for that group, respectively. The strength of the {\em conditional causal effect} (Section~\ref{sec:background-causal}) of this intervention on the subgroup is used to measure the expected utility of a rule. Our objective is to identify the smallest set of rules that maximizes overall expected utility. We refer to this problem as the {\em \probName} problem.
We adopt several notions of fairness (individual vs. group, statistical parity vs. bounded group loss) from the literature to define the {\bf fairness constraints} for our problem. In addition, {\bf coverage constraints} (for individual rules or for a group) ensure that the solution for the \probName\ problem is applied to a sufficient number of individuals and to minimize inequalities. We show NP-hardness for different variants of the problems and properties (matroid) useful in our algorithms. 
%We establish several definitions for group and individual fairness constraints tailored for prescription rules.
\smallskip
    \par
    \noindent
{\bf (2)} We {\bf develop a general three-step algorithm named \sysName to solve the optimization problem of selecting a fair prescription ruleset (Section~\ref{sec:algo})}. The first step involves mining frequent grouping patterns using the Apriori algorithm~\cite{agrawal1994fast}. In the second step, we employ a lattice-based algorithm to find high utility and fair intervention patterns for grouping patterns identified in the previous step. Finally, the third step applies a greedy approach to determine a solution. \sysName\ can be easily adapted to accommodate all variants of the \probName\ problem.

\smallskip
\par
\noindent
{\bf (3) We provide a detailed  case study  (Section~\ref{sec:casestudy}) and experimental analysis (Section~\ref{sec:experiments}) to evaluate our framework and algorithms.}
The case study shows the qualitative difference of different variants of our problem for different choices of the fairness and coverage constraints. The experiments include two datasets, three baselines, and 18 variations of our problem with different constraints. Our evaluations suggest that fairness may come at the cost of expected
utility for everyone. However, without fairness constraints, we often observe a significant disparity between the protected and non-protected. We also observe that
achieving individual fairness is harder than group fairness,
as most high-utility or high-coverage rules are unfair. Lastly, we show that \sysName\ can generate  prescription rules over large datasets in a reasonable time. 

%\end{itemize}


%\paragraph*{Paper outline} 
We discuss related work in \cref{sec:related}, review background on causal inference in \Cref{sec:background-causal}, %and our problem formulation can be found in \cref{sec:problem}. Our algorithmic framework is presented in \cref{sec:algo}. A case study demonstrating the impact of different constraint configurations on the solution is given in \cref{exp:problem_variants}, and our experimental evaluation is detailed in \cref{sec:experiments}. Finally, we 
and discuss the limitations of our framework and future work in \cref{sec:conc}.

% \noindent
% \boxed{\parbox{\columnwidth}{$\bullet$ 
% For people with a professional degree, move to the United Kingdom
%  (coverage = 435 (20), coverage-protected = 20 (13), utility = 186855, utility-protected = 0.)\\
% $\bullet$ For graphic developers, move to the	United States
%  (coverage = 116 (29), coverage-protected = 8 (2), utility = 169431, utility-protected = 0).\\
% $\bullet$ For people who have no formal education, move to the United States
%  (coverage = 123 (34), coverage-protected = 7 (2), utility = 206742, utility-protected = 0).\\
% % \textcolor{red}{size = 38, length = 76, overlap = 64029181, utility = 1659307}\\
% \textcolor{blue}{overall coverage =674, expected utility = 187485
% coverage-protected = 35, expected utility-protected = 0}
% \sr{should mention protected group, and possibly not mention coverage in the intro or just intuitively like high coverage}
% }}


% Alice notes that although these rules result in a \$187,485 increase in the overall salary for those to whom they apply, they only affect a small fraction of the population, specifically 674 individuals. Additionally, although the expected salary increase is substantial, there is no expected increase in salary for non-males, a subpopulation of particular interest to Alice. In other words, applying these rules would result in no gain for non-males.
% \end{example}

% \begin{example}[Episode 2 - coverage and fairness constraints]
% Alice introduces coverage and fairness constraints to ensure that enough people will benefit from the rules and that they will be \emph{fair} with respect to non-males. Specifically, she demands that the benefit for a randomly chosen individual to whom one of the rules applies is nearly the same as the benefit for a randomly chosen individual who identifies as non-male and to whom one of the rules applies.

% After adding these constraints, \sysName\ recommends the following set of prescription rules:



% \noindent
% \boxed{\parbox{\columnwidth}{$\bullet$ 
% For people who have no formal education, move to the United States
%  (coverage = 123 (34), coverage-protected = 7 (2), utility = 206742, utility-protected = 0)\\
% $\bullet$ 
% For females, change role to	DevOps specialist (coverage = 2256 (47), coverage-protected = 2256 (47), utility = 90023, utility-protected = 90023).\\
% $\bullet$ For people with a Master's degree, move to the	United States
%  (coverage = 9097 (2222), coverage-protected = 642 (236), utility = 85390, utility-protected = 84201).\\
% % \textcolor{red}{size = 38, length = 76, overlap = 64029181, utility = 1659307}\\
% \textcolor{blue}{overall coverage =11476	
% , expected utility = 87601,
% coverage-protected = 2905, expected utility-protected = 88519}
% }} 







% \begin{figure}[t]
%         \centering
%         \begin{minipage}[b]{1.0\linewidth}
%             \small
%             \begin{tcolorbox}[colback=white]
%             \vspace{-2mm}
% $\bullet$ For backend developers, the treatment with the highest effect on salary is “Country = US” effect size = 78646
% \begin{itemize}
%     \item For non-male the effect is only: 59429
%     \item For male the effect is 80454
% \end{itemize}

% $\bullet$ For frontend developers, the treatment with the highest effect is :Formal Education = Bachelor's degree” effect size: 17340
% \begin{itemize}
%     \item For white the effect is 33464
%     \item For non-white the effect is 15320
% \end{itemize}


% $\bullet$ For people in Europe, the treatment with the highest effect on salary is “DevType = C-suite executive” effect size = 53254
% \begin{itemize}
%     \item For white the effect is 55112
%     \item For non-white 35249
% \end{itemize}



%             \vspace{-2mm}
%             \end{tcolorbox}
%         \end{minipage}%%
%          % \vspace{-4mm}
%         \caption{Set of prescription rules.}
%         \label{fig:so-explanation}
%     \end{figure}

\section{Background and related work}
% 重点看Artistic data visualization: Beyond visual analytics 和Visualization criticism-the missing link between information visualization and art 的被引


This section reviews the background on artistic data visualization and previous research related to this topic.

\subsection{Artistic Data Visualization in Art History Context}
\label{ssec:contemporary}

Art history has been marked by transformative periods characterized by different aesthetic pursuits, materials, and methods. Since the time of Plato, imitation (or \textit{mimesis}, which views art as a mirror to the world around us) has been an important pursuit~\cite{pooke2021art}. Successful artworks, such as Greek sculptures and the convincing illusions of depth and space in Renaissance paintings, exemplify this tradition.
The advent of modern society and new technology, especially photography, posed a significant challenge to the notion of art as imitation~\cite{perry2004themes}. By the 1850s, modern art began to emerge with the core goal of transcending traditional forms and conventions. Movements like Post Impressionism, Expressionism, and Cubism revolutionized art through innovative uses of form (\eg color, texture, composition), moving art towards abstraction and experimentation. 
After World War II, the Cold War and the surge of various social problems heightened skepticism about the progress narrative of modernity and the superiority of the capitalist system, leading to the rise of postmodernism and the birth of contemporary art~\cite{hopkins2000after,harrison1992art}. One prominent feature of contemporary art is the absence of fixed standards or genres historically defined by the church or the academy. Postmodern design neither defines a cohesive set of aesthetic values nor restricts the range of media used~\cite{pooke2021art}, sparking novel practices such as installations, performances, lens-based media, videos, and land-based art~\cite{hopkins2000after}.
Meanwhile, artists have increasingly drawn energy from various philosophical and critical theories such as gender studies, psychoanalysis, Marxism, and post-structuralism~\cite{pooke2021art}. As a result, as described by Foster~\cite{foster1999recodings}, artists have increasingly become ``manipulators of signs and symbols... and the viewer an active reader of messages rather than a passive contemplator of the aesthetic''. Hopkins~\cite{hopkins2000after} described this shift as the ``death of the object'' and ``the move to conceptualism''. 
% Joseph Kosuth, one of the most important representatives of conceptual artists, also once said that “all art (after Duchamp) is conceptual (in nature) because art only exists conceptually”
% As argued by Danto~\cite{danto2015after}, traditional notions of aesthetics can no longer apply to contemporary art. ``“All there is at the end,” Danto wrote, “is theory, art having finally become vaporized in a dazzle of pure thought about itself, and remaining, as it were, solely as the object of its own theoretical consciousness.''
% The Anti-aesthetic (1983) has described these as ‘anti-aesthetic’ strategies – typified, as we have seen, by a conceptually driven approach to the art object and to the process of its production.

Emerging within the contemporary art historical context, data art has been significantly propelled by the advent of big data. An early milestone was Kynaston McShine's 1970 exhibition \textit{Information} at the Museum of Modern Art (MoMA). 
% In the exhibition catalogue, McShine wrote~\cite{information_moma}: ``Increasingly artists use mail, telegrams, telex machines, etc., for transmission of works themselves—photographs, films, documents—or of information about their activity.'' 
% The millennium era has witnessed substantial growth in this area.
In 2008, Google’s Data Arts Team was founded to explore the advancement of what creativity and technology can do together~\cite{google}.
% data artist Aaron Koblin
In 2012, Viégas and Wattenberg's \textit{Wind Map}, an artwork that visualizes real-time air movement, became the first web-based artwork to be included in MoMA's permanent collection~\cite{wind}.
Since 2013, the academic conference IEEE VIS has included an Arts Program (IEEE VISAP), showcasing artistic data visualizations through accepted papers and curated exhibitions. 
As noted by Barabási~\cite{dataism} (a Fellow of the American Physical Society and the head of a data art lab), data has become a vital medium for artists to deal with the complexities of our society: ``Humanity is facing a complexity explosion. We are confronted with too much data for any of us to make sense of...The traditional tools and mediums of art, be they canvas or chisel, are woefully inadequate for this task...today’s and tomorrow’s artists can embrace new tools and mediums that scale to the challenge, ensuring that their practice can continue to reflect our changing epistemology.''
% a physicist and head of a data art lab, has noted, 

% Artists are now exploring new mediums and methods, incorporating datasets, computer technology, and algorithms into their work.



\subsection{Research on Artistic Data Visualization}
\label{ssec:artisticvis}

Artistic data visualization is also referred to as artistic visualization, data art, or information art~\cite{holmquist2003informative,rodgers2011exploring,few,viegas2007artistic}. Despite the varying terminologies, there is a basic consensus that artistic data visualization must be art practice grounded in real data~\cite{viegas2007artistic}.
Since the early 2000s, a series of papers introduced innovative artistic systems for visualizing everyday data, such as museum visit routes and bus schedule information~\cite{skog2003between,holmquist2003informative,viegas2004artifacts}.
In 2007, Viégas and Wattenberg~\cite{viegas2007artistic} explicitly proposed the concept of \textit{artistic data visualization} and viewed it as a promising domain beyond visual analytics.
% and defined it as ``visualization of data done by artists with the intent of making art''. 
Kosara~\cite{kosara2007visualization} drew a spectrum of visualization design, positioning artistic visualization and pragmatic visualization at opposite ends of this spectrum to demonstrate that the goals of these two types of design often diverge. 
% advocating that analytical visualizations prioritize practicality, while artistic data visualizations emphasize sublime quality, that is, the capacity to inspire awe and grandeur and elicit profound emotional or intellectual responses. 
% In 2011, Rodgers and Bartram~\cite{rodgers2011exploring} utilized artistic data visualization to enhance residential energy use feedback. 
However, overall, research on this subject has been sparse. Among those relevant papers, most have focused on specific applications of artistic data visualization. 
%~\cite{rodgers2011exploring,schroeder2015visualization,perovich2020chemicals}
For instance, Rodgers and Bartram~\cite{rodgers2011exploring} utilized ambient artistic data visualization to enhance residential energy use feedback. Samsel~\etal~\cite{samsel2018art} transferred artistic styles from paintings into scientific visualization.
Artistic practice has also been observed in fields such as data physicalization~\cite{hornecker2023design,perovich2020chemicals,offenhuber2019data} and sonification~\cite{enge2024open}. For example, Hornecker~\etal~\cite{hornecker2023design} found that many artists are practicing transforming data into tangible artifacts or installations. Enge~\etal~\cite{enge2024open} discussed a set of representative artistic cases that combine sonification and visualization.
% dragicevic2020data
% Offenhuber~\cite{offenhuber2019data} created a set of artworks in urban settings that transform air quality data into situated displays, allowing people to encounter visualizations in their daily lives.

% But in contrast, empirical studies that describe the characteristics of artistic visualization and how they are created are extremely scarce. This scarcity forms a stark contrast to the increasingly rich and diverse practices by artists in the field.
% As for the difference between artistic data visualization and traditional visualizations for analytics, Vi{\'e}gas and Wattenberg~\cite{viegas2007artistic} thought that the most salient feature of artistic data visualizations is their forceful expression of viewpoints.
% In Ramirez~\cite{ramirez2008information}'s opinion, functional information visualizations are concerned with usability and performance while aesthetic information visualizations are concerned with visually attractive forms of representation design.
% Donath~\etal~\cite{donath2010data} presented a series of tools developed to integrate artistic expressions in generating unique and customized visualizations based on users' personal data, such as health monitoring data, album records, and e-mail records. 

On the other hand, some studies, while not focusing on artistic data visualization, have explored a key art-related concept: aesthetics. 
% ~\cite{moere2012evaluating,cawthon2007effect,lau2007towards} explored the aesthetics of visualization design in their research. They
For example, Moere~\etal~\cite{moere2012evaluating} compared analytical, magazine, and artistic visualization styles, noting that analytical styles enhance the discovery of analytical insights, while the other two induce more meaning-based insights. Cawthon~\etal~\cite{cawthon2007effect} asked participants to rank eleven visualization types on an aesthetic scale from ``ugly'' to ``beautiful'', finding that some visualizations (\eg sunburst) were perceived as more beautiful than others (\eg beam trees).
% Moere~\etal~\cite{moere2012evaluating} displayed data in three different styles (analytical style, magazine style, artistic style) and found that these styles led to different perceptions of usability and types of insights.
% More importantly, the authors found that the sunburst chart ranks the highest in aesthetics and is one of the top-performing visualizations in both efficiency and effectiveness, thus exemplifying the notion that "beautiful is indeed usable".
Factors such as embellishment~\cite{bateman2010useful}, colorfulness~\cite{harrison2015infographic}, and interaction~\cite{stoll2024investigating} have also been found to influence aesthetics. 
% borkin2013makes,tanahashi2012design
However, most of these studies have simplified aesthetics to hedonic features (\eg beauty), without delving into the nuanced connotations of aesthetics.
% most of these studies have simplified aesthetics to concepts like 'beauty,' 'preference,' or 'pleasing,' without exploring the deeper essence of aesthetics as the core of art.

The value of artistic data visualization to the visualization community is still in controversy. For instance, Few~\cite{few} argued for a clearer distinction between data art and data visualization; he highlighted the negative consequences when data art ``masquerades as data visualization'', such as making visualizations difficult to perceive. Willers~\cite{willers2014show} thought the artistic approach is ``unlikely be appreciated if the intention was for rapid decision making.''
% In an interview, American artist and technologist Harris commented that ``data can be made pretty by design, but this is a superficial prettiness, like a boring woman wearing too much makeup.''~\cite{harris2015beauty} 
To address these gaps, more empirical research needs to be conducted to explore how artistic data visualizations are designed, their underlying pursuits, and how they might inspire our community.




% Examining this field not only helps us understand the latest application of data visualization in various domains but also extends our understanding of the aesthetic and humanistic aspects of data visualization.
% there should be more theoretical investigation into artistic data visualization. 

% These three concepts emphasize placing or installing visualizations at physical places that people will encounter in their daily lives. 

% ~\cite{wang2019emotional}


% gap between art and science~\cite{judelman2004aesthetics}
% constructive visualization~\cite{huron2014constructive}
% data feminism~\cite{d2020data}
% critical infovis~\cite{dork2013critical}
% citizen data and participation~\cite{valkanova2015public}

% \x{Lee~\etal~\cite{lee2013sketchstory}, give users artistic freedom to create their own visualizations.}


% Aesthetics refers to the study of beauty, taste, and sensory perception and is deeply intertwined with art.
% a particular taste for or approach to what is pleasing to the senses and especially sight

% why shouldn't all charts be scatter plot~\cite{bertini2020shouldn}
% aesthetic model~\cite{lau2007towards}
% Aesthetics for Communicative Visualization : a Brief Review
% Stacked graphs--geometry \& aesthetics~\cite{byron2008stacked}
% storyline optimization~\cite{tanahashi2012design}
% graphic designers rate the attractiveness of non-standard and pictorial visualizations higher than standard and abstract ones, whereas the opposite is true for laypeople.~\cite{quispel2014would}
% evaluate aesthetics - wordcloud
% An Evaluation of Semantically Grouped Word Cloud Designs, tag cloud, wordle

% On the other hand, empirical studies conducted with designers have shown that functionality is not the only design goal of visualization. For example, Bigelow~\etal~\cite{bigelow2014reflections} found that designers would frequently fine-tune the non-data elements in their designs, and data encoding was even "a later consideration with respect to other visual elements of the infographic".
% Moere~\cite{moere2011role} - design
% Quispel~\etal~\cite{quispel2018aesthetics} found that for information designers, clarity and aesthetics are both important for making a design attractive.
\section{Overview}
\label{sec:overview}

In this section, we use a few running examples to explain
the core ideas behind \rarust{} that integrates AARA with Rust's
borrow mechanisms.
%
\cref{sec:overview:Shared} shows how \rarust{} deals with \textbf{shared} borrows.
%
\cref{sec:overview:Mutable} shows how \rarust{} deals with \textbf{mutable} borrows.
%
\cref{sec:overview:Lattice} shows how \rarust{} deals with the aliasing problem.
%
Recall that we propose a lightweight design for \rarust{}, which assumes the programs already pass Rust's borrow checking so that \rarust{} works directly on \emph{well-borrowed} and \emph{well-typed} Rust programs.
%
Concretely, \rarust{} assumes that all borrows in the analyzed program have known lifetimes (the span they live) and satisfy the following properties:
\begin{itemize}
    \item multiple shared but no mutable borrows from the same piece of data are live at the same time; or
    \item no shared but at most one mutable borrow from the same piece of data are live at the same time.
\end{itemize}
We will show how \rarust{} exploits those properties to carry out AARA for Rust programs.

% With some running examples, we first informally present the core idea of our type inference system, which will be formally presented in other sections. Examples are for indicating critical problems to solve. Instead of showing only the solution we worked out, we will explain our design step by step, with some failed attempts, to clearly convey the reason why our system formulates as it is.

% We organize subsections as follow: We first analyze shared borrows with sharing \ref{sec:overview:Shared} and then mutable borrows with prophecy \ref{sec:overview:Mutable}; we finally conclude the lattice algebra of resource types and discuss weak update and how it introduces inaccuracy \ref{sec:overview:Lattice}.

\begin{figure}[t]
\centering
\footnotesize
\hrule
\begin{subfigure}[b]{0.52\textwidth}
\begin{lstlisting}[language=Rust, style=colouredRust]
fn iter_twice(l: &List) {
  // l : &list(4)
  iter(&*l); // share 4 as 2 + 2, &*l : &list(2)
  // l : &list(2)
  iter(&*l); // share 2 as 2 + 0, &*l : &list(2)
  // l : &list(0)
}
\end{lstlisting}
\caption{Shared Reborrowing}
\label{fig:ex-sharing}
\end{subfigure}
%
\begin{subfigure}[b]{0.46\textwidth}
\begin{lstlisting}[language=Rust, style=colouredRust]
fn update(l: &mut List) {
  iter(&*l);
  // l : &mut list(0)
  *l = Cons(3, Box::new(Nil));
  // l : &mut list(4)
  iter(&*l); iter(&*l);
}
\end{lstlisting}
\caption{Mutating A Mutable Borrow}
\label{fig:ex-mut-borrow}
\end{subfigure}
%
\hrule
%
\begin{subfigure}[b]{0.48\textwidth}
\begin{lstlisting}[language=Rust, style=colouredRust]
fn prophecy() {
  let mut l = Cons(3, Box::new(Nil));
  // l : list(p)
  let x = &mut l;
  // l : list(q),  x : &mut(list(p ), list(q))
  update(x);    // x : &mut(list(p'), list(q))
  /* drop(x) */ // p' >= q
}
\end{lstlisting}
\caption{Creating \& Dropping A Mutable Borrow}
\label{fig:ex-prophecy}
\end{subfigure}
%
\begin{subfigure}[b]{0.51\textwidth}
\begin{lstlisting}[language=Rust, style=colouredRust]
fn weak(b: bool, l1: &mut List, l2: &mut List) {
  let l = if b { 
    &mut *l1 // : &mut(list(p1), list(q1))
  } else { 
    &mut *l2 // : &mut(list(p2), list(q2))
  }; // : &mut(list(min(p1,p2)), list(max(q1,q2)))
  update(l);
}
\end{lstlisting}
\caption{Mutable Reborrowing \& Aliasing}
\label{fig:ex-weak}
\end{subfigure}
\hrule
\caption{Examples to Demonstrate How \rarust{} Works}
\label{fig:running-examples}
\end{figure}

\subsection{Dealing with Shared Borrows via Shared Potentials}
\label{sec:overview:Shared}

In \cref{sec:overview:AARA}, we demonstrate the key concepts of AARA via
the Rust program shown in \cref{fig:list-iteration}, which already features shared borrows.
%
However, this is not the end of the story: multiple shared borrows from the same memory location can exist simultaneously; for example, the function
\verb|iter_twice| shown in \cref{fig:ex-sharing} uses the reborrowing mechanism
to create two more shared borrows by \verb|&*l|.
%
If we still follow the methodology presented in \cref{sec:overview:AARA},
suppose that the function parameter \verb|l| has type $\&\kwd{list}(\alpha)$,
it would be unsound to type the two shared reborrows \verb|&*l| as $\&\kwd{list}(\alpha)$, because it would double the potentials stored in \verb|l|.

% The story of \cref{fig:list-iteration} does not end, because we actually do not follow the intuition : simply typing the shared borrow $\&t$ with $\& T$ when $t$ is typed with type $T$. Remember that there could be many shared borrows for one variable.

% \textbf{Sharing Operation:} 

Fortunately, prior research on AARA has proposed a notion of \emph{sharing} to allow multiple uses of a variable in a linear or affine type system~\cite{AARA-Poly-Multivar,AARA-Poly}. 
%
This is because AARA for functional programs needs shared potentials to pass value by reference.
%
The idea is to replace a variable $x$ of resource-annotated type $T$
with two fresh variables $x_1,x_2$ of types $T_1,T_2$, such that the potential function denoted by $T$ equals the sum of the potential functions denoted by $T_1,T_2$.
%
In RaRust, we reuse the sharing mechanism to handle shared borrows and prove it is sound with respect to the safe Rust semantics.
% 
In our setting, this indicates that we can replace a typing context
$x : \kwd{list}(\alpha)$ with $x_1 : \kwd{list}(\alpha_1), x_2:\kwd{list}(\alpha_2)$ such that $\alpha = \alpha_1 + \alpha_2$.

In \rarust{}, we adopt a more imperative design inspired by \emph{remainder contexts}~\cite{kn:Walker02,ICFP:KH21}.
%
We associate every program point with a typing context, and when a statement performs a shared (re)borrow, we split the potentials into two parts by splitting the resource-annotated type into two types as shown above: one becomes the type
of the shared (re)borrow, and the other is put back into the remainder context, i.e., the typing context after the statement.
%
For example, in \cref{fig:ex-sharing}, suppose that the function parameter \verb|l| has type $\&\kwd{list}(4)$, the first function call to \verb|iter|
performs a shared reborrow and we split the type $\&\kwd{list}(4)$ to $\&\kwd{list}(2)$ (for the function call) and $\&\kwd{list}(2)$ (for the remainder context).
%
The second function call also performs a shared reborrow, but this time the typing context indicates that \verb|l| has type $\&\kwd{list}(2)$, so we split it to $\&\kwd{list}(2)$ (for the function call) and $\&\kwd{list}(0)$.
%
Observing that the function \verb|iter| requires one unit of additional potentials, we derive the following signature for \verb|iter_twice|:
\[
\verb|iter_twice| : \kwd{fn}(\verb|l|:\&\kwd{list}(4)) \to () | 2 .
\]

% In original AARA, there exists the sharing operation for multiple usage of one variable, sharing $l:\kwd{list}(\alpha)$ as $l_1:\kwd{list}(\alpha_1)$ and $l_2:\kwd{list}(\alpha_2)$, with constraints $\alpha = \alpha_1+\alpha_2$ and that all potentials are non-negative, $\alpha_i\geq 0, i=1, 2$. In our calculus, we have a similar operation for shared borrows. When a shared borrow occurs, we will share a type as two parts, one for the borrow, another back to original, as is shown in the comment of Figure \ref{fig:ex-sharing}.

It is worth noting that Rust's shared borrows are \emph{immutable}.
However, \rarust{}'s analysis \emph{mutates} the resource-annotated types of shared borrows because shared borrows could consume resources when being accessed. 
%
This is safe unless the value of the borrow is mutated.
%
For example, a shared borrow \verb|l| points to a list of length 10 that carries 2 units of potentials per element; thus, the total potentials are 20 units.
%
We then create another shared borrow \verb|&*l| and split the potentials to let \verb|&*l| carry one unit of potentials per element.
%
Now suppose we mutate the value via the borrow \verb|l| to increment the list length by one.
%
The type of \verb|&*l| still indicates one unit of potentials per element, thus indicating 11 units of total potentials, but it only has 10 units.
%
To tackle the problem, \rarust{} exploits Rust's borrow mechanisms to render the reasoning sound:
mutable borrows and shared borrows from the same memory location cannot exist simultaneously.
%
Thus, if a Rust program mutates the list via a mutable reference \verb|l|, then the previous shared borrow \verb|&*l| must have ended its lifetime. 

% \textbf{Resource Characterization:}
% Until now, our approach seems nearly the same as original AARA. We need to point out our insight that the sharing operation characterizes the shared borrows, in the prospect of resource analysis. In purely functional world, this insight is just abstract nonsense, whereas in Rust's land, it is precious due to the distinction between shared and mutable borrows, handled by borrow mechanism. 

% \textbf{Updates on Shared Borrows:}
% Also, our version of sharing is more imperative. We will update the resource typing context along the checking, making it more similar to symbolic execution. We therefore need to define not only reading but also writing towards typing context.  One interesting observation is that even shared borrow can mutate corresponding type, but only decrease its resource. It indicates that the borrow mechanism is too coarse to differentiate value mutation and resource mutation. Resource sensitive languages, especially those for smart contracts, expects a more precise type system.

\subsection{Dealing with Mutable Borrows via Prophecy Potentials}
\label{sec:overview:Mutable}

It might seem straightforward to support mutable borrows in the approach sketched in \cref{sec:overview:Shared}.
%
For example, \cref{fig:ex-mut-borrow} implements a function \verb|update| that manipulates a mutable reference \verb|l|.
%
Suppose that the function parameter \verb|l| has type $\&\kwd{mut}~\kwd{list}(2)$.
%
For the first function call to \verb|iter| with a shared reborrow \verb|&*l|, we split the type to $\&\kwd{list}(2)$ and $\&\kwd{mut}~\kwd{list}(0)$.
%
The next assignment statement mutates the list stored in the location referenced by \verb|l|, so we mutate its type in the typing context accordingly.
%
To obtain enough potential for the remaining two function calls to \verb|iter|, we set the type of \verb|l| to $\&\kwd{mut}~\kwd{list}(4)$.
%
Because the new list \verb|*l| is a singleton list, the mutation itself consumes 4 units of potentials.
%
Similarly to the reasoning in \cref{sec:overview:Shared}, the potential is sufficient to perform the remaining two function calls, and the final remainder context is $\verb|l|: \&\kwd{mut}~\kwd{list}(0)$.
%
Noting that three calls to \verb|iter| need three units of additional potentials, we derive the following signature for \verb|update|:
\[
\verb|update|: \kwd{fn}(\verb|l|:\&\kwd{mut}~\kwd{list}(2)) \to () | 7 .
\]

A tricky issue arises when one considers creating and dropping mutable borrows.
%
\cref{fig:ex-prophecy} shows an example where the program creates a mutable borrow \verb|x| from a mutable list \verb|l|.
%
Note that it is no longer sound to split the potentials of \verb|l| into two parts and store one part in \verb|x|: the reason is stated already at the end of \cref{sec:overview:Shared}; that is, the program can later mutate the list \verb|l| through the mutable reference \verb|x|, making the remainder type of \verb|l| unsound.
%
Fortunately, Rust's borrow mechanisms ensure a good property that
at most, one mutable borrow from the same memory location can be live simultaneously, so in principle, it would be possible to track the change in the type of the mutable borrow \verb|x| and pass the change back to \verb|l| when \verb|x| gets dropped.
%
It is worth noting that our type system is aware of when a borrow $x$ gets dropped via an explicit statement $\kwd{drop}~ x$, which is generated according to lifetime constraints given by the Rust compiler.

% \textbf{Mutation might increase resource.}
% We start by comparing shared borrows with mutable borrows. According to borrow properties, when lifetime of shared borrows does not end, the value is immutable and the resource can only monotonically decrease(non strict). But if mutable borrow exists, the resource can increase at need. Figure \ref{fig:ex-mut-borrow} presents such an example. Before assignment, the mutable borrow has $0$ resource and the value is to remove, while after it, the borrow should have at least $4$ per \lstinline|Cons| to pay for two iterations. Note that the value, or specifically the length, of the list has changed, therefore the example is different from just iterating 3 times.

% \textbf{Mutable borrow is the true borrow to restore.}
% Because the mutable borrow might increase resource, i.e. $\alpha - \alpha_2 < 0$, the sharing technique $\alpha = \alpha_1 + \alpha_2$ with $\alpha_1 \geq 0$ mentioned in the previous subsection is no longer applicable.  The mutable borrow should take away the resource from original place, and give back when it drops. Mutable borrow is the true borrow because it really borrow the resource instead of sharing a part, and will finally restore. This is the main difference between mutable borrows and shared borrows from the aspect of resource analysis. 

% \textbf{To attempt to mutably borrow with places is doomed to fail.} 

% Mutable borrow needs to restore its resource when it drops. 

One idea might emerge immediately that the resource-annotated type of a mutable borrow keeps the location where it borrows from, as $\verb|l|:\&\kwd{mut}(\kwd{list}(\alpha), \verb|p|)$ with \verb|p| be the location such that \lstinline|l = &mut p|.
%
However, this design essentially embeds a pointer analysis in the type system, and one would soon find out that every mutable reference type needs to record a \emph{set} of possible locations.
%
\cref{fig:ex-weak} exemplifies this case and we will revisit this example in \cref{sec:overview:Lattice}.
%
Because those locations are usually local variables, it becomes unclear how to carry out inter-procedural analysis in a compositional way, and we certainly do not want function signatures to reveal local variables.

% It is a bad design because the mutable borrow type disastrously depends to places, or said term variables. The dependent type will definitely increase the complexity of analysis. We just enumerate some simple but non-trivial problems. 
% \begin{enumerate}
%     \item {\textbf{Parameters:}
%     When the mutable borrow types appear as types of functions' formal parameters, the interpretation of places is confusing, and like a placeholder. When analyzing recursive functions, it is hard to assume a set of places as signatures. And such a analysis will soon degenerate to point-to analysis and may diverge.
%     }
%     \item {\textbf{Subtyping:} 
%     Recall that AARA approach needs a resource subtyping relation. There exists research\cite{CapTypes} on subtyping over resource dependent types, while it is complex. 
%     }
%     \item{\textbf{Identification:}
%     Places are usually local variables. When analyzing inter-procedural, it will be a big problem to identify whether two places are the same.
%     }
% \end{enumerate}

Such an issue is not uncommon in the studies of advanced type systems or verification frameworks for Rust.
%
\citet{ProphecyInSepLogic} adapt \emph{prophecy variables}---which were originally proposed by \citet{LICS:AL88} to talk about future's program states during reasoning---to integrate separation logic with prophecies.
%
RustHorn~\cite{RustHorn} and RefinedRust~\cite{RefinedRust} also use prophecy variables in their verification frameworks.
%
The high-level idea of using prophecy variables to analyze Rust's mutable borrows is to record additional information which corresponds to the final value of a reference, i.e., the value when the reference gets dropped. 
%
However, prophecy variables usually record the final values, which are too heavy for AARA.

In \rarust{}, we propose \emph{prophecy potentials}, a novel adaption of prophecy variables to the AARA methodology to reason about the future's potential functions.
%
\cref{fig:ex-prophecy} shows how prophecy potentials work with mutable borrows.
%
We now represent the type of mutable reference as $\&\kwd{mut}(\tau_1,\tau_2)$, where $\tau_1$ denotes the current potential function and $\tau_2$ denotes the prophecy potential function, i.e., the expected potential function when the lifetime of the reference ends.
%
In \cref{fig:ex-prophecy}, suppose that the initial type of \verb|l| is $\kwd{list}(p)$ with some $p \ge 0$.
%
To create a mutable borrow \verb|x| from \verb|l|, we generate a prophecy type $\kwd{list}(q)$ with some $q \ge 0$, which indicates the final resource-annotated type of \verb|x| when \verb|x| gets dropped.
%
The mutable reference type of \verb|x| is initialized to $\&\kwd{mut}(\kwd{list}(p), \kwd{list}(q))$.
%
After the call to the function \verb|update|, the type of \verb|x| becomes $\&\kwd{mut}(\kwd{list}(p'), \kwd{list}(q))$.
%
Note that the prophecy type should remain unchanged.
%
When the mutable reference \verb|x| drops, \rarust{} emits a constraint that the potentials indicated by $\kwd{list}(p')$ are no less than the potentials indicated by the prophecy type $\kwd{list}(q)$, i.e., $p' \ge q$.
%
If \verb|l| would later be used again, we can start from the type $\kwd{list}(q)$.
%
In this way,
prophecy potentials enable \emph{compositional} reasoning about mutable borrows.

% \textbf{Prophecy variables characterize restoring.} 
% Among problems listed above, identification might be the most pathological, yet shedding light on global demands. We need a global staff to indicate where the borrows come from. Note that AARA utilizes linear programming, which contains lots of linear variables global to the analyzed program. Just as it is used to indicate future values in research \cite{RustHorn} in program verification, prophecy variables can be used to indicate the resource when the mutable borrow drops and restore. In such a meaning, we say that prophecy variables characterize the prophetic restoring at the time when the borrow occurs. 

% \textbf{Restoring captured as linear constraints.}
% Figure \ref{fig:ex-prophecy} shows how prophecy variables are correlated with places. When the mutable borrow occurs, resource type in the original place $l$ will be replaced with prophecy type $\kwd{list}(p)$, and mutable borrow takes away original type $\kwd{list}(q)$; when the borrow drops, type checker will generate linear constraints from subtyping relation $\kwd{list}(p) \preceq \kwd{list}(q')$, to ensure the prophetic is bound by the final resource $\kwd{list}(q')$. We explicitly point out that the restoring is captured as linear constraints, fully utilized by linear programming solvers.

% \textbf{Higher order borrows can be characterized by subtyping over mutable borrow types.}
% It is obviously that prophecy types are contravariant for subtyping, i.e.$\&^\kwd{m}(\tau_{\text{c}, 1}, \tau_{\text{p}, 1}) \preceq \&^\kwd{m}(\tau_{\text{c}, 2}, \tau_{\text{p}, 2})$ if and only if $\tau_{\text{c}, 1} \preceq \tau_{\text{c}, 2}, \tau_{\text{p}, 2} \preceq \tau_{\text{p}, 1}$. With subtyping over mutable borrow types, borrows of borrows, or said higher order borrows can be easily characterized by prophetic version of mutable borrow types. 

\subsection{Dealing with Aliasing via A Lattice of Resource-Annotated Types}
\label{sec:overview:Lattice}

A type system sometimes cannot precisely determine where a mutable reference is borrowed from.
%
For example, \cref{fig:ex-weak} uses a conditional expression to assign a mutable reference \verb|l| to a mutable reborrow from either \verb|l1| or \verb|l2|, depending on the runtime value of the Boolean-valued variable \verb|b|.
%
As the example shows, although Rust's borrow mechanisms enjoy some good properties that aid our design of \rarust{}, we still face the problem of \emph{aliasing}, as other static analyses of heap-manipulating programs would also face.

In our work, we can still exploit Rust's borrow mechanisms, which ensure that aliasing and mutation cannot happen at the same time.
%
Therefore, we only need to consider \emph{control-flow aliasing}; that is,
when the control-flow paths merge at a program point, we need to \emph{merge} the types---including the mutable reference types---of a variable from different paths.
%
The merging here needs to be conservative, similarly to the \emph{weak updates} usually seen in pointer analyses.
%
In \rarust{}, we formulate a subtyping relation among resource-annotated types to formalize the notion of ``conservative'' and then construct a \emph{lattice} of resource-annotated types based on subtyping to carry out merging.
%
For example, in \cref{fig:ex-weak}, suppose that the mutable reborrows \verb|&mut *l1| has type $\&\kwd{mut}(\kwd{list}(p_1),\kwd{list}(q_1))$ and \verb|&mut *l2| has type $\&\kwd{mut}(\kwd{list}(p_2),\kwd{list}(q_2))$.
%
Recall that $\kwd{list}(q_1)$ and $\kwd{list}(q_2)$ are prophecy types.
%
To obtain the type of the mutable reference \verb|l|, we need to merge the two types above.
%
Thinking about the merging conservatively, one can derive that
\verb|l| can hold potentials no more than the potentials indicated by
$\kwd{list}(p_1)$ and $\kwd{list}(p_2)$, so \verb|l|'s current potential type is
at most $\kwd{list}(\min(p_1,p_2))$.
%
Meanwhile, to ensure that the prophecies are sound no matter which branch is executed, \verb|l|'s prophecy potential type should be at least $\kwd{list}(\max(q_1,q_2))$.
%
In addition, because the type system cannot know which of \verb|l1| and \verb|l2| would be mutated later or which of them would remain unchanged, \rarust{} enforces that $p_1 \ge q_1$ and $p_2 \ge q_2$.

As illustrated above, \rarust{}'s current mechanism of handling aliasing compromises the precision of the resource analysis, mostly due to weak updates.
%
On the one hand, such precision loss is unavoidable---at least for \cref{fig:ex-weak}---due to insufficient information about runtime values during type checking.
%
On the other hand, recent work such as Flux~\cite{Flux} introduces \emph{strong references} to perform strong updates, and it is interesting future research to adapt them in \rarust{}.

% \textbf{Merging of Typing Context:} 
% Recall that resource typing context will be updated during type checking, for shared borrows, also for mutable borrows. Together with branching statements, it brings a new problem that after branching there would be more than one typing contexts to merge as one. With the help of resource subtyping, exactly the lattice algebra of resource types, we can use meet operation in algebra to merge contexts. 

% \textbf{Weak Update:} 
% Branching statements will introduce weak mutable borrows, those pointing to multiple possible places. Actually, it is also one reason why we give up mutable borrow with places. The updates on weak borrows are what called weak updates in literates of program verification. It is dangerous to update all possible places, because it might generate resource from the vacuum. As the example in Figure \ref{fig:ex-weak}, when updating $l$ increasingly, it is definitely wrong to increase resource at both $l1$ and $l2$, because in runtime, there always only one place to increase. To ensure soundness, our choice is to force non-increasing when merge mutable borrows, therefore additional non-increasing constraints introducing inaccuracy into our sound analysis. Besides, in presence of nondeterministic boolean values, or said static unknown values, weak updates and inaccuracy are unavoidable, in the sense of analysis. The inaccuracy of our sound analysis are mainly introduced by weak updates.
\begin{figure*}[t!]
\centering
  \includegraphics[width=\linewidth]{figs/shapelib_method_v1.pdf}
   \caption{Method overview. We design a function library in four steps, starting from a user intent (light blue) that consists of function descriptions and a set of seed shapes. First, (a) we prompt an LLM to create function interfaces that define parameters and annotate the function's purpose. Then, (b) the LLM is prompted to propose multiple applications of the functions that reconstruct the seed shapes. Next, (c) we use this information to guide the LLM to propose multiple function implementations. The library is finalized with a validation step (d) that searches for pairs of  applications and implementations that best reconstruct the seed shapes. We can use the library to extend beyond the seed shapes by guiding the LLM to author a synthetic data generator with the library functions, and using the resulting paired data to train a recognition network for visual program induction.
   } 
  \label{fig:method_fig}
\end{figure*}


\section{Library Design}
\label{sec:lib_design}

\methodname~converts design intent into a library of functions through a series of steps, which we depict in Figure~\ref{fig:method_fig}.
The interface creation step converts function descriptions into a library interface (Section~\ref{sec:lib_interface}). 
The application proposal step identifies which library functions should model which seed set shapes (Section~\ref{sec:prop_apps}).
The implementation proposal step generates candidate function implementations (Section~\ref{sec:prop_impls}).
The library is then finalized with a validation step that checks combinations of proposed function applications and implementations against seed set examples (Section~\ref{sec:lib_validation}). 


\subsection{Interface Creation}
\label{sec:lib_interface}

\methodname~first converts user function descriptions into a library interface (Fig.~\ref{fig:method_fig}, a).
We prompt an LLM to produce a structured interface, where for each function it produces a typed signature and an accompanying doc-string.

We provide the LLM with two default classes: a `Part' class that creates primitives that abstract detailed geometry and a `CoordFrame' class that defines a local bounding volume.
Our prompt contains task instructions and in-context expert demonstrations sourced from different categories.
By default, we use axis-aligned cuboid primitives, though this design decision could be generalized by modifying prompt instructions and examples.

The LLM produces function signatures that expose parametric handles, e.g. the numbers of bars in a ladder back or the height of base runner.
Each function is instructed to take in a special first parameter, \textit{CF}, a `CoordFrame' that specifies the expected extents of the functions outputs. 
Functions are typed so that they return a List of `Part' objects.

Through our in-context examples and instructions, we prompt the doc-string to have a particular structure. 
First, it defines a \textit{description} field to explain the high-level goals of the function.
Then, it defines a \textit{parts} field, that specifies what parts should be produced depending on the input parameters.
Finally, it defines a \textit{parameter} field, that explains how they should affect the output structure.
This interface is then used to guide the library development.

\subsection{Proposing Function Applications}
\label{sec:prop_apps}

As LLMs are prone to hallucinate, we do not directly implement each function following the prior step. 
Instead, we would like to ground each function implementation by referencing structures from the seed set.
To find such references, we propose programs that apply library functions that explain exemplar shapes (Fig.~\ref{fig:method_fig}, b).

This step begins by sampling a shape from the seed set.
We ask a VLM to describe the parts that is sees from a render of the shape.
We also convert the 3D semantic part annotations into a list of labeled `Part' objects.
We combine these inputs together, and task an LLM with deciding what parts should be explained by which library functions (even though these functions lack implementations).
The LLM outputs this decision by authoring a `program()' function that proposes library function applications (along with parameters).
We ask the LLM to use a special `group\_parts' function when constructing this program, that consumes a list of input `Part' objects and returns a bounding `CoordFrame' object.
In this way, the `program' provides information about which parts of the input shape should be explained by which library functions.

As we later demonstrate empirically, the accuracy of individual LLM calls has a high variance which makes them hard to trust. 
Therefore, instead of finding a single program for each shape, we run this procedure K times for each shape in the seed set (K=5).


\subsection{Propose Function Implementations}
\label{sec:prop_impls}

\methodname~now has the information from the prior steps it needs to author good function implementations: typed signatures, doc-string guidance, and input-output examples.
These input-output example pairs can be automatically found from the proposed function applications.
From this input, we ask the LLM to complete the implementation of each function so that it matches the signature type, meets the doc-string specification, and respects the observed patterns present in the usage examples (Fig.~\ref{fig:method_fig}, c). 

Of note, we find that the LLM predictions in the previous application proposal step do a good job of identifying which functions should explain which parts, but do a much worse job at predicting parameter values. With this in mind, we mask out parameter values with a special token `?' in all input-output examples.
We do this for every parameter value, except for the first \textit{CF} `CoordFrame', as the correct value for this parameter can be found automatically with the `group\_parts' function.


Similar to previous step, we find that some implementations produced by the LLM produce better or worse matches against the input specification.
So for each function in our library, we propose K different ways that it could be implemented (K=4).

\subsection{Library Validation}
\label{sec:lib_validation}

At this point we are close to having a fully realized library.
From the prior steps we have (a) function doc-strings and signatures, (b) proposals of how the functions should be applied to explain groups of parts in seed-set shapes, and (c) proposals of how the function should be implemented.
This validation step is responsible for deciding which of these proposals are `good', and not just LLM hallucinations
(Fig~\ref{fig:method_fig}, d).

To make this decision, we search over pairs of proposed implementations and parameterizations, and record those that geometrically match structures present in the seed set shapes.
For each  proposed function implementation from (c) we check which of proposed part groups from (b) this implementation can explain.
Specifically, we try executing the function with the proposed parameterizations sourced from (b), calculate the observed error between the target parts and function output, and record the parameterization that achieves the best error.
Our error metric compares corner-to-corner distances between sets of geometric primitives, and mark function applications as invalid  if the paired structures are not similar enough (see the appendix for details).

At this point, for each group of parts from (b) we know which implementation from (c) best matches the observed part structure.
We keep the implementation that achieves the \textit{best} error across the \textit{most} part groups, and remove all others proposals.
If this \textit{best} implementation found valid applications across multiple seed set shapes, we update the library interface entry with its implementation logic. 
Otherwise, we remove the function entry from the interface.




\section{Using the Library for Program Synthesis}
\label{sec:lib_usage}

In Section~\ref{sec:lib_design}, we constructed a library of functions that have meaningful signatures and structured doc-strings.
Each function has an implementation that is capable of producing structures that capture patterns observed in the seed set, but a question remains: how can we use these functions to represent new shapes?

In this section, we describe our strategy for expanding library function usage beyond the seed set (Fig.~\ref{fig:method_fig}, \textit{right}).
To begin, we once again make use of the strong prior of LLMs by providing it with our library interface and asking it to design a procedure that uses the abstraction functions to randomly synthesize synthetic shapes.
Once we've developed this synthetic data sampler, we can use it to produce paired training data for a recognition network that learns how to solve an inverse task: given an input shape structure, write a program using the library functions that explain its parts.


\paragraph{Generating a synthetic shape sampler}

In this step, we design a prompt that describes the library we've developed, including the interface of each function  and examples of how to use it (sourced from the validation stage).
We give this prompt to an LLM and ask it to write a `sample\_shape' function that randomly produces new shapes using the provided abstractions.
Interestingly, we find that frontier LLMs are able to provide useful implementations of such a `sample\_shape' function.
A shown in Figure~\ref{fig:method_fig}, some of these random outputs produce good shape abstractions, while other random samples violate class semantics.
With this in mind, instead of attempting to get the LLM to perfect its implementation, we treat its output as a synthetic data generator for a recognition network. 
To broaden the coverage and variety of structures that these `sample\_shape' functions produce, we employ an iterative refinement loop that provides automatic feedback to the LLM.
This refinement procedure ensures that all functions and parameters in the library get used, and instructs the `sample\_shape' function to produce outputs spanning the observed structures from validation step (see appendix).


\paragraph{Training a recognition network}

Once we've improved the `sample\_shape' function through rounds of iterative refinement, we can use it to produce training data for a recognition network.
This network takes as input a shape represented as a set of unordered primitives (e.g., Cuboid dimensions and positions).
It outputs a program that uses library functions to reconstruct this input shape.
We implement this network as an autoregressive Transformer decoder~\cite{att_is_all} with a causal prefix mask over the input shape representation.
We train this network from scratch, streaming random samples from the synthetic data generator: each program we sample becomes a target output and we execute the program to find the corresponding input.
Once trained, we can use this network to find library function applications that explain shapes from outside of the starting seed set (Fig.~\ref{fig:method_fig}, \textit{right-bottom}).  
Our inference procedure prompts the network with an input set of unordered primitives and samples a large number of programs according the network's predicted distribution.
We try executing each program, and we record its complexity (the number of tokens it uses) and its geometric error against the input set. 
We choose the program that minimizes an objective that is a simple weighted combination of these two values.






























\subsection{Qualitative Results}
After generating a large number of samples (\textgreater$200,000$), we apply quality checks to remove noisy generations, resulting in approximately $35,225$ samples. See \autoref{fig:counterfactual_examples}  for examples of counterfactual image generations and \autoref{fig:perturbed_examples} for parametric and source conflicts.

Two raters independently labeled a subset of $100$ quality-checked generations for each category of conflicts to determine if the new label (\retlabel or \updatedlabel) matches the perturbed image---see label quality ratings in \autoref{tab:dataset_description}. Counterfactuals have a higher quality rating (\textgreater90\%). Parametric (76\%) and source conflicts (82\%) produce more noisy generations which we attribute to the increased difficulty in replacing an object versus removing it. Raters only disagreed on a small fraction of samples (30/300), while a Cohen's Kappa of 0.45 reflects that disagreements happened only on lower quality generations \cite{delgado2019cohen}.
% However, since these disagreements are on the  ~\cite{delgado2019cohen} this results in a Cohen's Kappa of 0.45. 

%The low Kappa score is a consequence of class imbalance ~\cite{delgado2019cohen} which, for instance, can be seen from the high quality rating score (87-93\%) for counterfactual generations.


% \begin{figure*}
%     \centering
%     \vspace{-4mm}
%     \includegraphics[width=0.8\linewidth]{figures/results/finetuning_eval.pdf}
%     \vspace{-2mm}
%     \caption{Evaluation of baseline (-Base) and \segsub finetuned (-Ft) model accuracy on counterfactual and source conflicts (higher is better). Evaluation on original samples from VQAv2, OK-VQA, and WebQA datasets shows that finetuning does not result in performance regression on these tasks (except on WebQA two-image samples). Finetuned models outperform baselines across all types of knowledge conflict.}
%     \vspace{-4mm}
%     \label{fig:finetuning_results}
    
% \end{figure*}


\vspace{-2mm}
\paragraph{Parametric Conflicts}
While Phi3 model does benefit somewhat from finetuning (4\% drop in parametric response rate), Qwen2 and Llava are unaffected. Parametric response rates are low across the board ($\sim$20\%, \autoref{fig:parametric_effect}), showing that baseline models are already robust to conflicts between input sources and parametric memory.

\begin{figure}
    \centering
   
    \includegraphics[width=0.8\linewidth]{figures/results/parametric_plot.pdf}
    % \includegraphics[width=0.9\linewidth]{figures/results/perturbed_1_vs_2_image.pdf}
    \vspace{-4mm}
    \caption{Parametric effect analysis: how often does the model predict the original label for perturbed images? Lower is better, implying a reduced parametric effect.}
    \label{fig:parametric_effect}
    \vspace{-2mm}
\end{figure}


\subsection{Quantitative Results}
In \autoref{fig:finetuning_results} we find that baseline VLMs fail to acknowledge counterfactual conflicts (Counter) and source conflicts (Source). Finetuning mitigates this across every dataset. The resulting finetuned models (-Ft) outperform the baseline models (-Base) on perturbed samples. Finetuning has some benefit on the original samples (Original) for VQA and WebQA counterfactual sources, but a large performance regression is apparent for samples with source conflicts in WebQA.

%\begin{figure}[!ht]
%    \centering
%     \includegraphics[width=0.9\linewidth]{figures/results/perturbed_1_vs_2_image.pdf}
%     \caption{Finetuning on samples generated to induce parametric conflicts improves performance on original datasets for one image questions, but regresses it for the two image multihop case. \todo{update image or caption, currently inconsistent}}
%     \label{fig:one_two_image}
% \end{figure}


%Baseline models are capable of predicting the generated labels for samples generated to induce parametric conflicts (\autoref{fig:one_two_image}). Furthermore, finetuning on these augmented samples does not lead to a substantial improvement in model performance on the original labels. 

%As this perturbation category can have either one or two images per sample, we partition our evaluation into a single image set and a two image set. We find that for finetuned models, there is a gulf in performance between one and two image samples (\autoref{fig:one_two_image}). We attribute this performance gap to the increased difficulty in multihop reasoning. Rather than learning to determine if a two image sample has a valid answer, finetuned models over-predict the \retlabel, indicating a source conflict. Thus, performance is deteriorated for original two image samples from the WebQA dataset.



% As such, training models with images that differ only in the concept in question does not seem to improve performance, at least for the color and shape categories. We qualify this by the observation that questions concerning color and shape attributes are well represented in popular VQA datasets, including all three datasets in this study. As such, adding perturbed samples in these categories will have diminishing returns at best. 

% It is worth noting that these samples are still necessary to scaffold the model in learning the concept of knowledge conflicts so that the model cannot shortcut to predicting that every set of generated samples with 2 images is a conflict.

\vspace{-2mm}
\paragraph{Source Conflicts}
For WebQA samples with source conflicts, the finetuned models have extremely low accuracy on original samples. This is a result of the finetuned models failing to predict the old label and instead overpredicting the \retlabel when presented with two images. %We attribute this to the high amount of noise with knowledge conflict generations achieving a quality rating of only 82\% (\autoref{tab:dataset_description}), and the multihop nature of the knowledge conflict task. 
Interestingly, instead of generating an `acknowledgement' response, baseline models tend to predict one of the two incorrect answers---either the original label (for the unperturbed image) or \updatedlabel (for the perturbed image)---uniformly at random. 
% See \autoref{tab:webqa_conflicts} in the supplementary.


\begin{figure*}
    \centering
    \vspace{-4mm}
    \includegraphics[width=0.8\linewidth]{figures/results/finetuning_eval.pdf}
    \vspace{-2mm}
    \caption{Evaluation of baseline (-Base) and \segsub finetuned (-Ft) model accuracy on counterfactual and source conflicts (higher is better). Evaluation on original samples from VQAv2, OK-VQA, and WebQA datasets shows that finetuning does not result in performance regression on these tasks (except on WebQA two-image samples). Finetuned models outperform baselines across all types of knowledge conflict.}
    \vspace{-4mm}
    \label{fig:finetuning_results}
    
\end{figure*}



\vspace{-2mm}
\paragraph{Counterfactual Conflicts}
Baseline models perform poorly on counterfactual conflicts, with no model achieving more than 30\% accuracy. Since these models are not trained to return the \retlabel, we consider any admission of failure by the model as a \retlabel. These baseline models are sometimes able to determine when an image lacks the information required to answer a question, they are not robust to these samples. Finetuning on enables these models to identify counterfactual conflicts with high accuracy, without degrading performance on the original datasets. Additionally, finetuning provides a 5-10\% performance gain on the original samples from WebQA and VQA datasets.

\begin{figure}
    \centering
    \includegraphics[width=0.9\linewidth]{figures/results/context_scores.pdf}
    \vspace{-3mm}
    \caption{Decreased Accuracy on Counterfactual Conflicts in finetuned VLMs (and GPT4-o-mini) with Increasing Image Contextualization Scores. Baseline unsmoothed data is in the background.}
    \label{fig:context_scores}
    \vspace{-4mm}
\end{figure}


\vspace{-2mm}
\paragraph{Robustness of Counterfactual Conflicts}
We find that finetuned models are robust in detecting randomized counterfactual samples. They are not simply detecting images that have been modified by LaMa to remove objects. The finetuned Qwen2 model predicts \retlabel for 80\% of the randomized counterfactuals sampled from the WebQA dataset. \autoref{tab:natural_counterfactual_results} in the supplementary has further details.

\vspace{-2mm}
\paragraph{Parameter Size}
We find that performance improvements on the evaluation metrics derived from increasing model size have diminishing returns. There exists a gap in performance between SoTA models (i.e. GPT-4o-mini) and the finetuned models (see \autoref{fig:baseline_model_performance} in the supplementary).

\begin{figure}
    \centering
    \begin{subfigure}[b]{0.22\textwidth}
        \centering
        \includegraphics[width=\textwidth]{figures/segmentation/context/baseball_boy.jpeg}  % Replace with your figure file
        \caption{ChatGPT: "There doesn’t appear to be an object clearly visible in his hands."}
        \label{fig:bat_boy}
    \end{subfigure}%
    \hfill
    \begin{subfigure}[b]{0.22\textwidth}
        \centering
        \includegraphics[width=\textwidth]{figures/segmentation/context/bat_removed.png}  % Replace with your figure file
        \caption{ChatGPT: "The batsman in the image is holding a baseball bat as he prepares to swing."}
        \label{fig:bat_batsman}
    \end{subfigure}
    \caption{These counterfactual examples were generated by removing a baseball bat from two different VQA images. When asked 'what is he holding?', ChatGPT only hallucinates in the highly contextualized case (right).}
    \label{fig:baseball_example}
\end{figure}


\paragraph{Image-Question Contextualization}
Intuitively, image-question contextualization relates to contextual cues within an image that provides the models with clues to answer the question, as in \autoref{fig:baseball_example}. We find evidence for a link between image-question contextualization, as approximated by GPT-4o-mini, and accuracy on counterfactual samples. \autoref{fig:context_scores} reveals that models perform poorly in identifying a sample as counterfactual (i.e. lower accuracy of predicting \retlabel) and is more likely to hallucinate on heavily contextualized image question pairs. Interestingly, GPT-4o-mini hallucinates for all of the counterfactual examples given in \autoref{fig:counterfactual_examples}.

For a concrete example, see \autoref{fig:baseball_example}, where both counterfactual examples were generated by removing a baseball bat. Here, a poorly contextualized image question pair features a child standing in a field with the question "what is he holding?" (\autoref{fig:bat_boy}). The only contextual cues as to what the child might have been holding are the generic outdoor setting, and the child's body positioning. Contrasting this in the adjoining sample is a baseball player, adorned in a jersey with his player number printed on the back, in a stadium filled with sporting fans (\autoref{fig:bat_batsman}). ChatGPT recognizes that the child is holding nothing, but hallucinates a bat in the hands of the batsman. Alongside previous works that show a relationship between image context and object detection \citep{beery2018recognition}, these results indicate that contextual cues have a priming effect that induces hallucinations in VLMs for highly contextualized counterfactuals.
% \citep{beer}. 



% \begin{table}[]
%     \centering
%     \begin{tabular}{c|c}
%          &  \\
%          & 
%     \end{tabular}
%     \caption{Caption}
%     \label{tab:qualatitive_examples}
% \end{table}

\section{Conclusion}\label{sec:conclusion}
This work introduces a novel approach to TOT query elicitation, leveraging LLMs and human participants to move beyond the limitations of CQA-based datasets. Through system rank correlation and linguistic similarity validation, we demonstrate that LLM- and human-elicited queries can effectively support the simulated evaluation of TOT retrieval systems. Our findings highlight the potential for expanding TOT retrieval research into underrepresented domains while ensuring scalability and reproducibility. The released datasets and source code provide a foundation for future research, enabling further advancements in TOT retrieval evaluation and system development.


\bibliographystyle{ACM-Reference-Format}
\bibliography{main}

\appendix

\begin{figure*}[t!]
\centering
  \includegraphics[width=\linewidth]{figs/shapelib_qual_libs_v1.pdf}
   \caption{Examples of functions from the shape libraries discovered by \methodname. For each category, we show a function implementation, and a few example applications of the function. For each application, we show the full output shape, with parts corresponding to the function marked in the same color as the function name, and the function parameters. We can see that function applications are well-aligned with part semantics and that each function typically requires only a small set of parameters to represent a rich variety of part shapes.} 
  \label{fig:big_qual_libs}
\end{figure*}

\begin{figure*}[t!]
\centering
  \includegraphics[width=.85\linewidth]{figs/shapelib_qual_apps_v1.pdf}
   \caption{\methodname's abstraction functions provide a semantically aligned and interpretable interface that support downstream applications: text-based LLM editing and visual program induction from unstructured geometry. } 
  \label{fig:big_qual_apps}
\end{figure*}


\section{Additional Method Details}

\subsection{Objective Function }

When searching for programs that explain shapes, we need an objective function to guide the search. 
We take inspiration from prior approaches such as ~\cite{jones2023shapecoder}, and formulated an objective function as a weighted average of two terms.
One of these terms counts up the number of degrees of freedom in the program representation, for simplicity we treat every token in the program as a degree of freedom with the same weight (1.).
Another term ensures that the produced geometry does not deviate too far from the target structure. 
We calculate the geometric error (more on this in the next paragraph), and add that into our objective function with a weight of 10.

The geometric error function we use takes in two sets of unordered primitives. 
For every pair of primitives from the predicted to target set, we calculate the maximum minimum distance between any two corners from one primitive to the other. 
We then use a matching algorithm to assign a stable pairing between the two sets.
If any of the distances is above a threshold (0.25, where shapes are normalized to lie within the unit sphere), then we say that there is infinite geometric error.
Otherwise, the geometric error is an average of the maximum minimum corner distance (MMCD), calculated according to the best match.

\subsection{Network Design}

We implement all of our networks in PyTorch~\cite{paszke2017automatic}. 
All of our experiments are run on NVIDIA GeForce RTX 3090 graphic cards with 24GB of VRAM.
We use the Adam optimizer~\cite{Kingma2014AdamAM} with a learning rate of 1e-4.
We implement our recognition network as a Transformer decoder. 
Our network has 4 layers, 4 heads, model dim of 256, and a full feature dim of 1024.

This network has full attention over the conditioning information: each primitive in the input shape is quantized and treated as a discrete token.
We order the primitives according to their x-y-z positions, as we do not know how they should be ordered otherwise.
Programs are similarly tokenized, and our network is trained through teacher forcing. 
We use learned positional encodings, these cap the maximum sequence lengths and primitive amounts 
our network can reason over: 20 primitives and programs of up to length 64. 
We train with a batch size of 128.
For point cloud inputs, we replace the primitive token encodings with an embedding produced by a PointNet++~\cite{qi2017pointnet++} network. 
For voxel inputs, we replace the primitive token encodings with an embedding produced by a 3D-CNN. 
We train our networks for between 4-12 hours, depending on the category and task.

\subsection{Synthetic Data Sampler}

We perform two rounds of automated feedback for each `sample\_shape' function generated by the \textit{o1} LLM model.
This iterative approach aims to refine the sampler's outputs by addressing discrepancies and improving alignment with respect to seed set patterns. 
In each round of feedback, we evaluate the function by sampling a diverse set of shapes and assessing various aspects of its behavior. 
We examine whether all functions in the library were used, whether all parameter types were employed, and whether all output structures described in the function's documentation were produced. 
These checks are performed automatically.
Additionally, we analyze the structures generated by the sampled functions and determine their similarity to those observed during the validation stage. 
If significant deviations are detected, measured in the parameter space of each function, the sampler is instructed to update its logic to produce outputs closer to the expected structures.

\section{Additional Experimental Details  }

\subsection{Cost and Timing}

We provide detailed estimates for how expensive it is (from a time and API monetary expense perspective) to use our system to discover libraries of shape abstraction functions.
To produce 20 shape descriptions from images using gpt-4o: 10 cents and 1-2 minutes.
To create library interfaces from textual descriptions with o1mini: 25 cents, 2-4 minutes.
To propose function applications over (20) shapes with (1) o1mini call and (4) gpt-4o calls: \$2-3 
and 15-25 minutes.
To propose (4) implementations for each function with o1mini: \$2-4 and 15-30 minutes.
To propose a single program sampler with o1: 50 cents and 1 minute. In total, this amounts to \$5-8 and 30 minutes to 1 hour.

Notice that by default we use o1mini, but sometimes deviate based on our developmental experience. 
Making function applications without knowing function implementations is a `guess-based' exercise, so we are fine with the increased error rate that 4o produces in this step.
For the most complex tasks, like implementing a synthetic data sampler, we turn to o1 as we are able to provide enough task guidance and directives to make use of its `reasoning' capabilities.

\subsection{Data}

Collections of example shapes in the seed set are chosen by an expert user who has a design intent in mind (they also express this intent in natural language in the function descriptions).
Specifically, we have the user select 20 partNet shapes and put them in a list, and then we can automatically produce the rest of the structured data from the partNet annotations. 
Currently, we manually render associated ShapeNet meshes in MeshLab~\cite{meshlab}, but this could be easily relaxed for ease of use.

After we have selected these two shapes, we create separate `training' and 'validation' sets of shapes by randomly splitting up Partnet object instances.
We run all experiments over validation shapes, unless otherwise stated, and use the training shapes to get paired data for the visual program induction step that maps from unstructured geometry to a shape abstraction program.
The size of these train/val sets is 4000/1000 for chairs, 1216/400 for storage, 4000/1000 for tables, 434/400 for faucet, and 2625/656 for lamps.


\subsection{LLM-Direct Baseline}

The LLM-direct is an ablated version of our method that relies on only the prior of the LLM and the design intent of the expert user in the form of function descriptions.
We compare against it to validate the need for using the seed set of shapes alongside the natural language specification. 

This baseline, is equivalent to our method modulo a few critical changes. The interface creation step is exactly the same. 
After this step though, it immediately implements each function, without using any input/output guidance about how this function should constructed. 
As it has no seed set, it assumes that the LLM has perfectly implemented each function, and next advances to the synthetic sampler design stage where it prompts the LLM to produce a `sample\_shape' function from its constructed library.
Then, like the full~\methodname system, we can train a recognition network on data produced by this random sampling procedure.


\subsection{ShapeCoder}

In our comparisons against ShapeCoder we use the officially released implementation. 
The only change we make is removing the rotation operation from the base ShapeCoder language,
as we focus on structures of axis-aligned primitives in our experiments.
We develop ShapeCoder's library of abstraction over the same seed set of 20 shapes, which is much smaller than the large datasets used in the original ShapeCoder system (400 shapes).
Nevertheless, we find that ShapeCoder can generalize (in terms of compression, at least) fairly well even from these 20 shapes.

We experiment with discovering ShapeCoder libraries over a larger seed set of 400 shapes, and find that compression improves slightly on validation shapes, but not by a huge margin (Obj goes from 52.1 to 46.1, while the average library size grows from 19 to 24). 
Despite learning this library over a large collection of shapes, we still observe that this `ShapeCoder-400' variant does not find more semantically aligned function applications over validation structures.
In fact, its semantic entropy performance worsens (chair: 1.67 to 1.84, table: 1.578 to 2.16, storage: 2.07 to 2.08, lamp: 1.7 to 1.9, faucet: 2.1 to 2.3)
We view this result as lending our framing additional support: 
compression alone (even over a large dataset) is not enough to develop good shape abstraction libraries, top-down semantic guidance is also required.



\end{document}
