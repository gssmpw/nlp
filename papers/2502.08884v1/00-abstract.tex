

Procedural representations are desirable, versatile, and popular shape encodings. 
Authoring them, either manually or using data-driven procedures, remains challenging, as a well-designed procedural representation should be compact, intuitive, and easy to manipulate. 
A long-standing problem in shape analysis studies how to discover a reusable library of procedural functions, with semantically aligned exposed parameters, that can explain an entire shape family.
We present \methodname as the first method that leverages the
priors of frontier LLMs
to 
design a library of 3D shape abstraction functions.
Our system accepts two forms of design intent: 
text descriptions of functions to include in the library
and a seed set of exemplar shapes.
We discover procedural abstractions that match this design intent by proposing, and then validating, function applications and implementations.
The discovered shape functions in the library are not only expressive but also generalize beyond the seed set to a full family of shapes. 
We train a recognition network that learns to infer shape programs based on our library from different visual modalities (primitives, voxels, point clouds).
Our shape functions have parameters that are semantically interpretable and can be modified to produce plausible shape variations.
We show that this allows inferred programs to be successfully manipulated by an LLM given a text prompt. 
We evaluate \methodname on different datasets and show clear advantages over existing methods and alternative formulations.