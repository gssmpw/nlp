\section{Node contraction on an ~\ER~random graph}\label{apx:ER}
In \Cref{subsub:ER}, we show how when contracting random node sets of unequal size in a network generated from an \ER model, the resulting network is no longer \ER. This is not the case when contracting random node sets of equal size. 

\begin{proposition}
    Consider a random graph $G(V,E)$ generated by an \ER~model with $n$ nodes where the probability of an edge between two nodes existing occurs independently with probability $p$. Consider a corresponding unweighted household network $G'(V', E')$ constructed by randomly contracting $R$ disjoint node sets $H_1, H_2, \ldots, H_R$ of size $\ell$ where $n = R\ell$, according to the basic household contraction \Cref{eq:contraction}. Then $G'$ is also described by an \ER~random graph, $G'(R, 1 - (1-p)^{\ell^2})$.
\end{proposition}

\begin{proof}
    In the graph $G$, for any pair of nodes $v_i$ and $v_j$, edge $(v_i, v_j)$ does not exist with probability $(1-p)$. Now consider node sets $H_q$ and $H_r$. Then for node $v_i \in H_q$, there is probability $(1-p)^\ell$ that there are zero edges between node $v_i$ and any node in set $H_r$. Since there are $\ell$ nodes in set $H_q$, the total probability of no edge existing between any node in $H_q$ and any node in $H_r$ is $(1-p)^{\ell^2}$. 
    Because all edges in $G$ exist with independent probability, the probability of edges between any two sets $H_q$ and $H_r$ are also independent. For the graph $G'$ where any node $w_k$ is constructed by contracting nodes in set $H_k$, then for any pair of nodes $w_q$ and $w_r$ in $G'$, edge $(w_q, w_r)$ exists with probability $1 - (1-p)^{\ell^2}.$ Thus we can describe $G'$ as an \ER~random graph $G'(R, 1 - (1-p)^{\ell^2}).$ 
\end{proof}


\section{Detailed review of referenced network studies}\label{apx:emp_rev}


\paragraph{Diffusion of microfinance.} \cite{banerjee2013} is an exploration of how information about a microfinance loan (and subsequent uptake of a loan product) travels through villages in Karnataka, India. The study is interested in developing a model for diffusion and using this model to measure \textit{diffusion centrality} of individuals in the village. They compare the centrality of the village leaders who were first targeted with information about the microfinance loans with the eventual adoption of the loans in households across the village. Considering this study in the context of the characterizations we set above, we see that in this work: (i) the intervention (leaders being targeted) occurs at the individual level, the outcome (loan uptake) occurs at the household level, and the research question occurs at both the individual and household level; (ii) the network data is collected by taking the union of 12 separate individual relationship name generators and household networks are created by aggregating nodes and their individual edges; (iii) the authors study the household networks. In \Cref{sec:consequences}, we explore the individual and household level data that was collected across 72 villages. It is important to note that in this work, individuals in the same household are completely connected with one another\textemdash that is, all individuals living in the same house have an edge between one another. 

\paragraph{Iron-fortified salt.} \cite{alexander2022algorithms} considers the network of women living in shared apartment buildings in Mumbai, India and how information travels between them. For each of 50 buildings, women are asked to report other female heads of households in the building who they are connected to, across five different name generators. The researchers tell a subset of the women about a particular type of fortified salt available alongside its health benefits and track how information about the salt travels throughout the network and which non-targeted women end up buying the product and understanding its health benefits. Here, (i) the intervention and outcome happen at the individual level (female head of household), (ii) the network data is collected by taking the union of 5 separate name generators where only female heads of households were considered, (iii) the authors study the individual network of female heads of households. 

\paragraph{Maternal and child health.} In a large-scale experiment, \cite{airoldi2024} measure if there is more information spillover on social networks when using one-hop seeding\footnote{For the one-hop seeding, a household is selected at random, then an individual in the household is selected, one of their individual neighbours is selected at random, and then that person's household is targeted.}\citep{cohen2003efficient} as compared to other seeding strategies. Although we do not use any data from this work, we do use it as an example in \Cref{sec:rec} and \Cref{fig:ex_decision}, and thus introduce the experiment here. For this work, network data from 176 villages in Honduras was collected, and, under different targeting techniques, households were picked to receive 22 months of counseling about maternal and child health. Each month, a trained worker would visit the targeted household and provide 1-2 hours of education about maternal health. At both the beginning and end of the study, each household in the village took a comprehensive test meant to capture how their knowledge of maternal and child health outcomes changed throughout the study. Here, we see that (i) the intervention and outcome both happen at the household level, with the information being delivered to the entire household and the entire household's knowledge being tested;(ii) the network data is collected at the individual level by taking the union of three separate name generators; (iii) the network studied is, interestingly, an overlap of the individual and household level, where network metrics were reported on the household level but one-hop targeting was done on a mixture of the individual and household network, similar to the two-stage randomization discussed in \cite{basse2018analyzing}. Although the intervention took place at the household level, at a practical level there was no enforcement of every household member being present for the educational sessions, and the sessions were delivered to whichever household member was home at the time. 

\section{Further explorations into individual and household inversity} \label{apx:inversity}

In this section we expand on possible explanations for the observation noted in \cite{kumar2024friendship}: that the \textit{inversity} when computed for the individual and household networks from \cite{banerjee2013} substantively differs. As mentioned in \Cref{subsub:local}, we find that the difference in the inversity between the household and individual networks can be largely attributed to the fact that in the individual networks in the \cite{banerjee2013} dataset, individuals living in the same household are completely connected to one another. 

In fact, in \cite{kumar2024friendship}, they note that ``We can expect hub-based (or star) networks to have high inversity \dots In contrast, we expect that networks of clusters of various sizes will have low inversity.'' With this intuition in hand, and knowing that the individual networks are quite literally constructed with clusters of completely connected nodes---which get contracted when forming the household networks---the change in inversity which was surprising becomes highly expected. 

To isolate the impact of intrahousehold edges on inversity, we decompose the adjacency matrix so that we may vary the weight we give to intrahousehold connections in different contexts. This is the same decomposition we discuss in more detail in \Cref{subsec:inf_dif}.
\begin{align}
    \A^*_p = \A_{extra} + (1- p) \A_{intra},
\end{align}

\begin{figure}
    \centering
    \includegraphics[width=0.7\linewidth]{figures/inversity_p_mean.pdf}
    \caption{Expected inversity of the individual village networks with intrahousehold edges removed with probability $p$, with $p$ ranging from $0.1$ to $1$, corresponding to adjacency matrix $\A^*_p$. For each village we compute the mean inversity over 100 instances of intrahousehold edge removal, and plot the mean inversity over all villages with the shaded region capturing the 5\% and 95\% quantiles across the mean inversity for all villages.}
    \label{fig:inv_p_dots}
\end{figure}

In \Cref{fig:inv_p_dots} we plot the inversity of the individual village networks with intrahousehold edges removed with probability $p$, with $p$ ranging from 0.1 to 1. We see that inversity monotonically increases with more intrahousehold edges removed. We conclude the exploration into the impact of intrahousehold edges on inversity by considering that in \cite{kumar2024friendship}, the authors give the intuition that inversity is at its most negative when edges connect nodes of the same degree. This intuition gives more context to the impact that the complete subgraphs in the individual network have on the observed inversity\textemdash when individuals are completely connected with their household members, there is less degree variation across edges. We explore this impact further in \Cref{fig:prop_edges}, where we plot the inversity of the individual network against the proportion of the total edges in the network that exist between individuals in the same household. Indeed, we see that networks with a higher proportion of intrahousehold edges have a more negative inversity.

\begin{figure}
    \centering
    \includegraphics[width=0.7\linewidth]{figures/prop_edges.pdf}
    \caption{Individual network inversity (x-axis) against the proportion of the total edges in the network that can be attributed to edges within households (y-axis). Networks with a higher proportion of edges due to intrahousehold connections have a more negative inversity, due to less degree variation across edges.}
    \label{fig:prop_edges}
\end{figure}