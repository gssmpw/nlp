Data recording connections between people in communities and villages are collected and analyzed in various ways, most often as either networks of individuals or as networks of households. These two networks can differ in substantial ways. The methodological choice of \emph{which} network to study, therefore, is an important aspect in both study design and data analysis. In this work, we consider various key differences between household and individual social network structure, and ways in which the networks cannot be used interchangeably. In addition to formalizing the choices for representing each network, we explore the consequences of how the results of social network analysis change depending on the choice between studying the individual and household network---from determining whether networks are assortative or disassortative to the ranking of influence-maximizing nodes. As our main contribution, we draw upon related work to propose a set of systematic recommendations for determining the relevant network representation to study. Our recommendations include assessing a series of \textit{entitativity criteria} and relating these criteria to theories and observations about patterns and norms in social dynamics at the household level: notably, how information spreads within households and how power structures and gender roles affect this spread. We draw upon the definition of an \textit{illusion of entitativity} to identify cases wherein grouping people into households does not satisfy these criteria or adequately represent given cultural or experimental contexts. Given the widespread use of social network data for studying communities, there is broad impact in understanding which network to study and the consequences of that decision. We hope that this work gives guidance to practitioners and researchers collecting and studying social network data.