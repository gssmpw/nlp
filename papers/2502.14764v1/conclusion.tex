In this work we stress that there is a distinction\textemdash and a subsequent choice\textemdash between analyzing social networks at the individual and household level. In addition to emphasizing and formalizing this choice, we motivate their difference in the context of related work ranging from anthropology to political science to sociology. We explore the consequences of the choice between each network with range of empirical examples wherein analysis (local network metrics, information diffusion, and node centrality) of the network substantively change depending on which network the analysis is conducted on. 

We specifically draw upon the literature of \textit{household strategies} \citep[e.g.,][]{werner1998, wallace2002household, niehof2011conceptualizing}, which considers the role of household networks and critiques of studying household exchange relationships from an ethnographic framework. Notably, we rely on this literature for drawing our attention to the gendered labor that goes into creating and maintaining household relationships, as well as the fact that the concept and interpretation of a \textit{household} varies greatly across different cultures and economies. We employ the observations from this literature to consider that gender, power, and hierarchy should be evaluated in the context of an experiment to evaluate how edges in the resulting network should be weighted. To explore how gender impacts network structure, we construct male and female household networks from the networks in \cite{banerjee2013}, only connecting two households if a male in one is connected to a male in the other, or a female in one to a female in the other, respectively. We find that the centralities of each household differs across these gendered networks, highlighting one possible consequence of misspecifying the relevance of gender between households.

We provide a systematic recommendation for assessing which network should be analyzed in a given experimental and cultural setting grounded in the theoretical work of \cite{campbell1958common} and his criteria for \textit{entitativity}. We provide a dependent hierarchy of these entitativity criteria in the network setting by defining a decision tree based on their evaluation with respect to an experimental setting. We provide three examples for how to use this decision tree in the context of the work of \cite{banerjee2013}, \cite{alexander2022algorithms}, and \cite{airoldi2024}, ultimately recommending analysis on the weighted individual, gendered individual, and gendered household networks, respectively.

We hope that this work provides a clear opportunity for future research on networks to be more contextually-rigorous. In addition to providing specific recommendations for future network research scientists to employ, we hope that by emphasizing a difference between the individual and household networks a line of interesting research questions open. 
 
First, it is interesting to explore if the recommendations we make regarding the choice of network to study would substantively change the conclusions of the original experiments. For example, the work of \cite{banerjee2019} is focused on assessing the correlation between the individuals that villages \textit{nominate} as good gossips and the households with the highest diffusion centrality. The authors conclude that peoples' accurate perceptions of highly central people can be well-modeled with their model of diffusion centrality. It is interesting future work to then consider \textit{which network} these perceptions (and the spread of gossip) are occurring on\textemdash that is, if people's gossip nominees are even \textit{more} correlated with the individuals with highest diffusion centrality as measured on a more particular network, e.g., an individual network with edges weighted according to gender. 

In our discussion of the entitativity criterion of \textit{proximity} in \Cref{subsec:entitativity_rec}, we mentioned the additional difficulty in assessing this criterion in the case of longitudinal network studies or in cultures where the household unit is more ambiguously defined. Rigorous recommendations for how to define the boundaries of a household when its members do not remain consistent over time is an avenue for practical and impactful future work. Furthermore, as mentioned earlier, we did not consider \citeauthor{campbell1958common}'s criterion of assessing a household's \textit{resistance to intrusion}. In many respects, we can consider the very act of network data collection as an act of intrusion. Scientists have long considered impacts of \textit{the observer effect}: how the collection of data from a system impacts the system itself. With respect to social data, anthropologists center the consideration of how the observer effect can either invalidate, bias, or strengthen ethnographic research findings \citep{lecompte1982, monahan2010}. Considering how the intrusion of social network data collection impacts the communities being researched\textemdash and the subsequent data\textemdash and incorporating households' reactions to that intrusion in evaluating entitativity, is an interesting direction for future work. 

We hope that by detailing the choices, consequences, and recommendations for studying individual and household networks, this work serves to inspire a rigorous chain of questions about the data decisions that we make when we aim to use networks as tools for exploring experimental questions about communities and the social ties between the people within them.