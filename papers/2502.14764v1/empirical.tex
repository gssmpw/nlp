\subsection{Local network properties}\label{subsub:local}

For the empirical analyses which follow, we consider the collection of 72 networks of social support in southern India from \citet{banerjee2013}. In \cite{banerjee2013}, data is collected from all individuals in a household about other individuals in the village they get and give social support to and from. Then, this individual-level network data is aggregated at the household level according to the \textit{basic household contraction} rule we describe above. In the following sections, we thus compare metrics and experiments when considering the individual network or the household network. See \Cref{apx:emp_rev} for more details about the data and the contexts in which it was collected. 

We begin our empirical exploration by considering how two commonly-used local network metrics, degree assortativity and average clustering coefficient, are substantively different depending on whether they are measured on the individual or household network (see \Cref{fig:deg_clus}). Notably, the average clustering coefficient is relatively small when measured on the household networks, and relatively large on the individual networks. Given that clustering coefficient can be used to describe the social health of a community \citep[e.g.,][]{cartwright1956,bearman2004}, we see in the case of these social support networks that staggeringly different conclusions might be drawn depending on whether the individual or household network is analyzed. This further highlights the importance of choosing the most contextually relevant network for understanding and evaluating a community using network methods.

Similarly, over the same set of villages, networks aggregated at the household level largely display disassortative mixing, whereas the corresponding individual networks have positive degree assortativity. This observation echoes the work of \cite{newman2002assortative} and \cite{newman2003social} and the authors' explorations into differences in degree assortativity in social and biological networks. However, in their work they show that social networks are largely assortative and biological networks are largely disassortative. It is thus surprising, in the context of the \citeauthor{banerjee2013} networks, to find both assortativity and disassortativity on the same social networks represented at different granular representations. The assortativity of a network has implications for how a disease is sustained in a community and the resilience (or lack thereof) of a network to the removal of nodes \citep{newman2003social}. As such, we again see that distinct conclusions are drawn regarding these communities, depending on which network is considered further indicating that \textit{the choice of which network to study} has significant implications. 

A similarly peculiar observation to the above was recently made in \cite{kumar2024friendship}: that when analyzing the same village networks \citep{banerjee2013}, their proposed graph property \textit{inversity} (closely related to degree assortativity) was substantively different depending on whether or not it was measured on the household or individual network. In their work, \citeauthor{kumar2024friendship}~propose inversity as a correlation-based metric to characterize when different seeding strategies will be more effective based on the structure of the network\textemdash a positive inversity implies one strategy, and negative another. For more in-depth details, see \cite{kumar2024friendship}.

The surprising aspect of their finding, then, is that, although each network captures the same people living in the same village, different seeding strategies are determined to be more impactful if one analyzes the network at either the household or individual level. Notably, \citeauthor{kumar2024friendship}~write that the result of this finding is that one should consider different strategies dependent on whether or not the intervention is expected to occur at the individual or household level. We expand on their recommendation with particular principled evaluation criteria in \Cref{fig:dec_chart} and \Cref{sec:rec}.

\begin{figure}
    \centering
    \includegraphics[width=0.95\linewidth]{figures/assortativity_cc_vils.pdf}
    \caption{The household and individual networks from \cite{banerjee2013} have substantively different interpretations when considering both the degree assortativity and clustering coefficients.}
    \label{fig:deg_clus}
\end{figure}

To explore what might be causing the shift in inversity in these networks, we notice that within each household, all individuals within a household are completely connected to all other individuals in their household. That is, the network of individuals is essentially made up of many different cliques, loosely connected to other cliques (cliques amongst household members is also how we represent the individual network in \Cref{fig:hh_ind_diagram}). Surprisingly, if we remove all of the edges between individuals in the same household, we see that inversity measured on the individual networks more closely aligns with inversity on the household networks. In fact, we see that for the majority of the villages, the corresponding networks both have positive inversity (see \Cref{fig:inv_nohh}). This observation suggests that the flip in inversity that \citeauthor{kumar2024friendship}~observe can be attributed to the existence of cliques within households. We explore this reasoning further alongside a more in-depth discussion of inversity and additional numerical experiments in \Cref{apx:inversity}.


\begin{figure}
    \centering
    \includegraphics[width=0.8\linewidth]{figures/Inversity_replication_nohh.pdf}
    \caption{The inversity of the \cite{banerjee2013} networks at the individual (blue) and household (red) level. In grey, we plot the histogram of the inversity of the individual networks across villages when the intrahousehold edges are removed.}
    \label{fig:inv_nohh}
\end{figure}

\subsection{Spread of information over a network}\label{subsec:inf_dif}

We turn now to explore what differences we uncover if the spread of information is modeled on either the individual or household network. To address this question, we conduct an experiment on the individual and household networks from \cite{banerjee2013} to identify which nodes are \textit{influence maximizing}: which nodes when seeded with a piece of information will spread it to a maximal proportion of the network. The problem of influence maximization is relevant for efficiently and effectively distributing a limited resource with maximal impact on the network. On each network, we aim to identify the seed set of $K=10$ nodes which maximize the spread of information using the greedy strategy proposed in \cite{kempe2003maximizing}. 

To construct the set of influence maximizing nodes, we first run $1000$ \textit{independent cascade} simulations\textemdash where the spread of information is modeled as traveling across an edge with probability $p$\textemdash with each separate node as a single starting seed. We add to the seed set a single node which spreads information to the highest proportion of the network, on average, across the $1000$ simulations. We then iterate this process until the seed set has $K=10$ nodes, adding the node which, conditional on the nodes already in the seed set, reaches the highest proportion of nodes in the network across the simulations. \cite{kempe2003maximizing} showed that this greedy algorithm for assembling a seed set for an independent cascade is guaranteed to produce a set that reaches a population within a factor of $(1-1/e)$ of the optimal seed set (while also showing that the underlying optimization problem itself is NP-hard).

To relax the idea of completely connecting households in the individual network, we propose decomposing the individual adjacency matrix into the parts made up of \textit{intrahousehold} and \textit{extrahousehold} edges: edges which connect individuals in the same and in different households, respectively. 
\begin{align*}
    \A = \A_{extra} + \A_{intra},
\end{align*}
For individual nodes $i$ and $j$ belonging to households $H_k$ and $H_\ell$, respectively, then $\A_{extra}$ and $\A_{intra}$ are defined as,
\begin{align*}
    A_{extra, \, ij} = \begin{cases}
        0 \textrm{ for } H_k = H_\ell\\
        1 \textrm{ otherwise}
    \end{cases} A_{intra, \, ij} = \begin{cases}
        0 \textrm{ for } H_k \neq H_\ell\\
        1 \textrm{ otherwise.}
    \end{cases}
\end{align*}
Decomposing the individual adjacency matrix in this way allows us to consider the impact and appropriateness of intrahousehold edges in different contexts. Specifically, this decomposition allows us to vary the weight we give to intrahousehold connections:
\begin{align*}
    \A^*_p = \A_{extra} + (1- p) \A_{intra},
\end{align*}
for $p \in [0,1]$. Here, $p$ can allow for flexibility in the presence of relationships between people in the same household, by varying the probability that an intrahousehold connection exists in a given context. For example, small $p$ represents a high probability that information, once given to one household member, will spread to the other members of the household. By contrast, large $p$ would capture a situation where there is low probability of information spreading between household members. By making this distinction, we build upon the modeling recommendations of \cite{newman2003social}, wherein they consider how to project bipartite affiliation networks onto a corresponding social network. In the context of modeling the spread of information, writing adjacency matrices in this way can represent a rate of the spread of information within households and make a direct modeling connection to the work of, e.g., \cite{nickerson2008} who found that information does not deterministically reach all household members. 

In modeling the independent cascade on both individual and household networks we fix the probability of information passing between two households to be the same. For the purpose of our experiments, we model this probability with $q = 0.05$.\footnote{We chose this extrahousehold information passing probability to be relatively low so that information didn't completely saturate the networks in our experiments. In preliminary experiments with higher values of $q$, we found an even bigger difference between influence-maximizing seed sets, most probably because when information saturates the network there are many equivalently influence-maximizing sets, and comparison between sets becomes arbitrary.} For the individual networks we assume information passes along intrahousehold and extrahousehold edges at different rates: whereas we keep the probability of information passing between extrahousehold edges as $q=0.05$, we model the intrahousehold probability as $(1-p)=0.7$. Note that with $(1-p)=1$, independent cascade unfolds in the exact same way on the individual and household networks, and with $(1-p) = 0$ information can only pass along edges between individuals in different households. As such we choose $(1-p)=0.7$ to model an intermediate scenario, where information spreads within households, but not deterministically. This setup follows observations from, e.g., \cite{nickerson2008} and \cite{cheema2023canvassing} whereby intrahousehold transmission is high but not guaranteed (see \Cref{sec:background}). 

We construct the influence-maximizing seed sets for both the individual and household networks, $S_i$ and $S_h$, respectively, for each village. To compare the sets of nodes across individual and household networks, we map each seed in the individual sets to their corresponding households. In \Cref{fig:jac} we report the overlap between these pairs of sets across all of the villages. We see that for many of the villages, there is very little overlap between the sets of influence-maximizing households. This observation implies that a different set of households (or individuals) will be identified as influence-maximizing seeds, dependent on whether or not the household or individual network is studied. 

\begin{figure}
    \centering
    \includegraphics[width=0.6\linewidth]{figures/intersection_svill.pdf}
    \caption{The intersection between the sets of the 10 most influential households, found by greedily maximizing the average proportion of nodes reached over 1000 independent cascades on the household and individual networks from \cite{banerjee2013}. To compare sets, we map the set of the 10 influence maximizing individuals to their corresponding households.}
    \label{fig:jac}
\end{figure}

However, this observation does not imply that the \textit{impact} of choosing different seeds will be vastly different. There may be many local optima of very similar quality and furthermore, because we use a greedy algorithm for constructing both $S_i$ and $S_h$, the differences in output from this algorithm need not even necessarily reflect differences in the true optimal sets. Thus, we inspect the difference between the expected proportion of the network reached over 100 iterations of an independent cascade on both the household and individual networks, using either $S_i$ or $S_h$ as the seed set. 

When running the independent cascade on the household network with $S_i$ as the seed set, we directly map the individuals to their corresponding households to get the corresponding set of nodes on the household network, $S_{i\rightarrow h}$. When running the independent cascade on the individual network with $S_h$ as the seed set, we map the households in $S_h$ to a set of corresponding individuals, which we call $S_{h\rightarrow i}$. To do so, we map households to a random subset of the individuals who live in the household (where the subset is selected uniformly at random with $(1-p)=0.7$ to match our numerical experiments on the individual network). This mapping means that $S_{h\rightarrow i}$ is a much larger set than $S_i$. We then consider the difference between the expected proportion of the network reached when running 1000 independent cascades on either the household or individual network, using $S_i$ and $S_h$ (or their corresponding mappings, $S_{i\rightarrow h}$ and $S_{h \rightarrow i}$, respectively) as seed sets.

We find that the difference in the expected proportion of nodes reached by the independent cascade only varies by at most around 4\% of the network in both the experiments on the individual and household network. However, we notice that the difference is greater when we simulate the spread of information on the individual network. 

This observation provides perspective for the results of our analysis above: although the \textit{sets} of the most influential nodes are different depending on whether or not they are selected from the individual or household network, the \textit{impact} of this difference depends on what the set is being used for. If the set is identified to seed information in hopes that it will have maximal influence, then there may not be a significant difference between the proportion of the network reached. If the set is used to compare influence-maximizing nodes with a set of nodes chosen in some other way (e.g., according to their status as village leaders), then the conclusions will depend on the choice of which network they are selected from.

\subsection{Diffusion centrality and gendered networks}\label{subsec:gender}
As we discussed in \Cref{sec:background} and \Cref{sec:choices}, the relationships represented in networks of households can be supported, maintained, and connected by gendered labor and gendered relationships. Moreover, as we consider in \Cref{sec:rec}, some experimental frameworks consider the spread of information that is more relevant and accessible to certain genders. 

As a particular example of gendered networks, \cite{alexander2022algorithms} considers the network of women living in a shared apartment building in Mumbai, India. The researchers tell a subset women about a particular type of fortified salt available alongside its health benefits and track how information about the salt travels throughout the network and which non-targeted women end up buying the product. The work is an excellent example where the researchers collected and studied a network that represented the process and context of the experiment. We see that in this network representation, women\textemdash as the household members responsible for buying groceries\textemdash stand-in for their entire household, and the network only connects them to other women with the same responsibilities. As we will see in \Cref{subsec:rec_ex}, when we apply the recommendations we propose in \Cref{fig:dec_chart} to this experimental setting, we ultimately recommend studying the same network (see \Cref{fig:ex_decision}). 

In this section we consider the consequences of representing a household network with gendered edges, as we formalized in \Cref{eq:gendered_con}. Moving from contagion to centrality, we consider how the diffusion centrality \citep{banerjee2013, banerjee2019} of nodes changes depending on whether or not we weight household edges according to the gendered relationships between individuals in each household. 

Diffusion centrality, a variation of eigenvector centrality, can be interpreted as how extensively a piece of information spreads as a function of an initially informed node. Consider a directed, strongly connected network with corresponding weighted adjacency matrix $\boldsymbol{w}$ where $w_{ij}$ gives the relative probability that $i$ shares information with $j$. Then for a piece of information originating at node $i$, in each time period an informed node tells each neighbour $j$ the information with independent probability $w_{ij}$. If we consider that this process happens for $0 < T < \infty $ time periods, diffusion centrality is specifically defined as:
\begin{align}
    DC(\boldsymbol{w}, T):= \left(\sum_{t=1}^T (\boldsymbol{w})^t \right) \boldsymbol{1}.
\end{align}

To explore how the diffusion centrality of a household depends on whether or not we only consider gendered connections, we first subset the household networks from \cite{banerjee2013} to just the surveyed households, for which we have information about the gender of the individuals in each household. We find the diffusion centrality of each household in (the connected component of) the subsetted household network: $DC(\B, T) \vec{1} := DC_{B}$, where $\B$ is the adjacency matrix corresponding to the household network $G'$.
Next, we create two separate gendered networks: one where two households are connected only if a female from one house is connected to a female of the other house, $G'_{FF}$, and another where two households are connected only if a male from one house is connected to a male in the other house, $G'_{MM}$. We then find the diffusion centrality for each household in these networks, $DC_{B_{FF}}$ and $DC_{B_{MM}}$, respectively. 

Similar to the sensitivity analysis that \cite{banerjee2013} conduct \citep[see Table S7 in][]{banerjee2013} with respect to changes in $T$, we investigate how the use of gendered networks impacts the resulting diffusion centrality of each household by finding the correlations between the diffusion centralities, $DC_{B}, DC_{B_{FF}}$, and $DC_{B_{MM}}$. We visualize the Pearson correlation coefficient between each pair of vectors across all 72 villages in \Cref{fig:gender_corrs}. Notably, we see that the diffusion centrality on the male-to-male network is strongly positively correlated the diffusion centrality on the entire household network, implying that the spread of information on the household network is dominated by household relationships defined by male-to-male individual relationships. The diffusion centrality of households connected by considering only female-to-female relationships is also positively correlated with the household diffusion centrality, although we observe much more spread across villages. This observation indicates that the household network defined without considering gender does not necessarily capture the most relevant network, especially in the case where information may be more likely to spread across female-to-female connections. 

Overall we see that the households with the highest diffusion centrality vary across these gendered networks, highlighting the importance of understanding \textit{who} a piece of information is relevant to \citep[e.g., the women in each household who buy groceries, see][]{alexander2022algorithms} and \textit{what social norms} allow for communication about particular subject matter to take place \citep[e.g., sharing information about maternal health may be stigmatized as only relevant and shareable among women, see][]{berti2015adequacy}. Note that for the purpose of the experiments in this section we only considered weighting the household networks with possible extremes, zero or one, by either including an edge or not depending on the gender of individual connections between those households. However, this weight can be relaxed in contexts where information sharing is less gendered, such that, for example, household edges defined by only only male-to-male individual connections are weighted lower than household edges defined by female-to-female individual connections but are not completely absent.

Thus we conclude this section with another instance of how conclusions and interventions can meaningfully vary (in this case, by misidentifying the most central households in a village) depending on the choices of how to represent a network. We see that understanding how gender (or, as we discuss in \Cref{sec:rec}, other relevant similarities between individuals) plays a role in an experimental context is important for properly considering how people and their households are connected. 

\begin{figure}
    \centering
    \includegraphics[width=0.6\linewidth]{figures/DC_gender_corrs.pdf}
    \caption{The correlations between each pair of diffusion centrality vectors across all 72 villages. We observe that the diffusion centrality of households varies across these gendered networks. A village with high/low correlation corresponds to households having similar/different centralities across different network definitions.}
    \label{fig:gender_corrs}
\end{figure}

