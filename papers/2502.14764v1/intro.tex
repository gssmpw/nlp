Social networks represent how people are connected to one another, for example, through physical contact, friendship, or the lending and borrowing of goods. In many studies of social connections in communities, towns, or villages, social networks are typically collected and analyzed at one of two representational levels: individual or household. That is, when considering how people are connected with one another, we can consider which \textit{individuals} are connected to one another or which \textit{households} are connected to one another. There is therefore an important choice to be made in any study of such a social network: should the nodes represent individuals or households, and how should the relationships between the nodes be represented? In this work we focus on emphasizing and formalizing this choice, exploring possible consequences of it, and systematizing recommendations for how to make it.

Often \citep[e.g.,][]{banerjee2013, airoldi2024}, \textit{household network} data is collected by first collecting the \textit{individual network}, and then grouping those individuals (and their connections) together with those who live in the same household (see \Cref{fig:hh_ind_diagram}). When household networks are constructed in this way, two key assumptions are implicitly made: (i) that an individual's social connections are shared and can be utilized by every other individual in their household, and (ii) that households are connected through individual relationships. In practice, the impacts of these implicit assumptions are not considered and thus individually-driven processes and relationships are conflated with household ones. Given that both the household and individual networks represent the same \textit{idea}\textemdash how the same set of people are connected to one another\textemdash the distinctions between household and individual networks is nuanced. 

\begin{figure}
    \centering
    \includegraphics[width=0.8\linewidth]{figures/hhind_red.pdf}
    \caption{This work considers the choice of which network is the most appropriate to study in a given context, a choice which we present as consequential for meaningful empirical network analysis. In this diagram we show how individual networks (left, blue) are often translated to household networks (right, red). In what we refer to throughout as the \textit{individual network}, nodes are individuals and an edge between two individuals is collected through a survey. In what we refer to as the \textit{household network}, nodes are households and an edge between two households is usually determined by aggregating the relationships between individuals in that household. We specify this decision to represent household edges in this way as the \textit{basic household contraction rule} and propose alternate methods for defining edges between households in \Cref{subsubsec:cont_rules}. In the diagram here, we also represent individuals within the same household as completely connected to one another, an aspect of some individual network datasets which we discuss in more detail in \Cref{subsub:local}. We review how individual and household networks are collected and studied in practice in \Cref{tab:review}.}
    \label{fig:hh_ind_diagram}
\end{figure}

\begin{figure}
    \centering
    \includegraphics[width=.95\linewidth]{figures/flowchart_side.pdf}
    \caption{In this work we provide a systematic recommendation for determining whether the household or individual network should be studied given a particular context and experimental goal or intervention. The decision tree here poses a contextual evaluation of a set of entitativity criteria\textemdash \textit{proximity}, \textit{similarity}, \textit{common fate}, and \textit{internal diffusion}, which we discuss in detail in \Cref{sec:rec}\textemdash to determine an appropriate level of node aggregation, as well as to suggest how to weight edges. In \Cref{fig:ex_decision} we apply these recommendations to three separate examples to determine an appropriate network to analyze.}
    \label{fig:dec_chart}
\end{figure}

Plenty of social science research ranging from within sociology \citep{campbell1958common}, political science \citep{nickerson2008, auld2013inter, cheema2023canvassing}, anthropology \citep{werner1998, schmink1984household, niehof2011conceptualizing}, and network science \citep{kumar2024friendship} have theorized, measured, and observed that there are assumptions with consequences when studying individuals and their relationships interchangeably with their respective aggregates.

In this work, we contribute a bridging of insights and observations from disparate fields in an effort to systematize recommendations for researchers collecting, studying, or developing methods for social network data. We review and adopt insights from experimental work testing how political messages spread within households and detail relevant observations from ethnographic studies of household interactions. We adapt \citeauthor{campbell1958common}'s (\citeyear{campbell1958common}) theoretical \textit{entitativity criteria} for determining when a group of individuals can be reasonably studied as an aggregate, or when representing individuals as one, coherent household is inadequate in a given setting\textemdash an \textit{illusion}. By incorporating work from a broad range of disciplines, we can help researchers in making a rigorous choice for how to align their research question with an appropriate social network, so that their subsequent analysis and observations may be consistent with their research goals.

We explore the differences between household and individual networks through two approaches: through formalizing the choices in their representations and considering the consequences of those choices. To formalize the decision, we introduce notation for household and individual networks with their respective adjacency matrices and \textit{contraction rules} that map one to the other. We complement this decision with a discussion of how the individual and household networks can be further specified using weighted edges, including based on gendered connections, or more generally, the similarity of the nodes in specific contexts. 

We examine how different choices in this formalization can lead to substantially different conclusions based on various network metrics. Theoretically, we show a simple example showing that \textit{heterogeneous random node aggregations} in an \ER~random graph result in a network that is no longer \ER. We then consider how local metrics, the spread of information over a  network, and centrality metrics substantively differ on the individual and household village social support networks from \cite{banerjee2013}. Notably, we explore how analysis\textemdash and the corresponding conclusions about the social network\textemdash considerably differs between the individual and household networks, highlighting the importance of choosing the appropriate network to study in the context of a given research aim.

To assess when an individual or household network should be studied in a given context, we provide a systematic recommendation based on a series of \textit{entitativity criteria} \citep{campbell1958common}. We ground these recommendations in theories and experimental observations studying how individuals interact within households and as aggregates, as well as how gender and power interact in household networks to distinguish between the types of connections most relevant in a given setting \citep{werner1998}. We propose specific adaptations and extensions to the \textit{entitativity criteria} of \textit{proximity}, \textit{similarity}, \textit{common fate}, and \textit{internal diffusion} as they relate to studying networks in the context of interventions and experimental goals, which we organize as a decision tree in \Cref{fig:dec_chart}. In doing so, we relate the entitativity criteria to recommendations on how and when to collect and weight the relationships between individuals or households differently. To highlight examples of how to evaluate the criteria in different settings, we apply this set of recommendations to three different large-scale experimental network studies, \cite{banerjee2013}, \cite{alexander2022algorithms}, and \cite{airoldi2024}.

The structure of this work is as follows. 
In \Cref{sec:background} we briefly discuss the related work that we both build upon and directly use to make recommendations. In Sections \ref{sec:choices}, we formalize a set of possible ways to represent the household network given information about the individual network. Next, in \Cref{sec:consequences} we explore the impact of choosing either the household, individual, or a gendered network when studying a variety of different network metrics in empirical networks. In \Cref{sec:rec} we propose a set of systematic recommendations for how practitioners can decide which network is the most appropriate to study. We conclude in \Cref{sec:conclusion} with concluding remarks and suggestions for future work.



