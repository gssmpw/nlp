In this section, we first formalize the \textit{choice} to be made when studying a social network\textemdash broadly, to represent the network at either the individual or household level. The choice, in the language of networks, is twofold: what should the nodes represent, and what constitutes an edge between two nodes? In this work, we focus particularly on how these two questions are answered in the context of \textit{individual} and \textit{household} networks, although one could instead choose to consider aggregating individuals according to other affiliations, like club memberships, dorm rooms, or hobby groups. 

If the choice is to study the individual network, \textit{how} to represent edges may be as straightforward as relying on the union of multiple different name generator questions. However, determining how to represent a network of household relationships may be more complicated. We first consider how to represent the household network through different \textit{node contractions} on the individual network, and then turn to consider the choice to capture household relationships by relying on specific \textit{household generator} questions.

\subsection{Node contraction} \label{subsubsec:cont_rules}
Node contraction \citep{oxley2006matroid, bollobas2013modern} describes the graph operation when two nodes $v_i$ and $v_j$ in a graph are replaced by one node $w$, and every edge incident to either $v_i$ or $v_j$ becomes incident to $w$. As we reference more generally in this work, a \textit{set} of nodes can be contracted to one node $w$, and every edge incident to any node in the set becomes incident to $w$. 

\paragraph{Basic household contraction.} To make the above statement more precise, in the most basic case, for a graph $G = (V, E)$, where $V$ is a set of nodes and $E$ a set of edges between them, consider a set of node-contraction sets $H=\{H_1, H_2, \dots, H_R \}$ where each $H_r$ is a non-overlapping partition of $V$ such that $\sum_{r}^{R} \abs {H_r}  = N$, so each $v_k \in V$ belongs to some set $H_r$. Then the graph $G'$ which results from contracting these vertices is given by
\begin{equation} \label{eq:contraction}
    \begin{aligned}
    &G' = (V', E')\\
    &V' = \{ w_i \}_{i=1}^{R} \\
    &E' = \left\{ (w_i, w_j) \mid \exists \, v_k \in H_i \textrm{ and } v_\ell \in H_j \textrm{ such that } (v_k, v_\ell) \in E \right\}.
    \end{aligned}
\end{equation}
We note that if the original graph is weighted, directed, undirected, or contains loops, the graph resulting after contraction can be of the same type. For a simple graph, we extend the last condition above to be for $i \neq j$. 

In the literature, simple, unweighted household networks (either directed or undirected) are often constructed by contracting individual networks according to \cref{eq:contraction}\textemdash each individual is replaced by one node, and all of the edges incident to any individual become incident to the household node. However, we note that different \textit{types} of household networks can be constructed through node contraction rules. 

Below we define additional possible node contraction rules that, given an \acl{ind} network $G = (V, E)$ and a set of households and their corresponding members, $H = \{ H_1, H_2, \dots, H_R\}$, produce a \acl{hh} network $G' = (V', E')$. We note that there are many more contraction rules and that many of the below rules can be combined to define additional rules (e.g., one can define a \acl{hh} network by combining weighted and gendered contraction rules from below).


\paragraph{Weighted household contraction.} 
For node contraction on a weighted graph $G= (V, E, W)$, the weight of an edge, denoted $W(w_i, w_j)$ is equal to sum of the weights of the edges incident to any node within each corresponding contraction set. 
\begin{equation*} \label{eq:weighted_con}
    \begin{aligned}
    &G' = (V', E', W')\\
    &V' = \{ w_i \}_{i=1}^{R} \\
    &E' = \{ (w_i, w_j) \mid \exists \, v_k \in H_i \textrm{ and } v_\ell \in H_j \textrm{ such that } (v_k, v_\ell) \in E \},\\
    &W'(w_i, w_j) = \sum_{v_k \in H_i} \sum_{v_\ell \in H_j} W(v_k, v_\ell). 
\end{aligned}
\end{equation*}
The weight of an edge can also be normalized according to the proportion of \acl{inds} in $H_i$ who are connected to any \acl{ind} in $H_j$, capturing a general strength of the edge between the two households. 

\paragraph{Gendered household contraction.} For specified gender $a$, we can adapt \Cref{eq:contraction} such that an edge from household $H_i$ to $H_j$ exists if there is an edge from an \acl{ind} of gender $a$ in $H_i$ to an \acl{ind} of gender $a$ in $H_j$: 
\begin{equation*} \label{eq:gendered_con}
    \begin{aligned}
        E' = \{ (w_i, w_j) \mid \exists \, v_k \in H_i \textrm{ and } v_\ell \in H_j \textrm{ such that } (v_k, v_\ell) \in E    \textrm{ and } gender(v_k)=gender(v_\ell) = a\}.\\
    \end{aligned}
\end{equation*}
By considering that the presence of an edge between two households (or the strength of an edge) may be dependent on the gendered relationships between individuals in each household, we allow for the flexibility to account for potential gendered household labor, gendered sharing of information, and gendered access to relationships that has been observed and theorized in, e.g., \cite{werner1998}, \cite{berti2015adequacy}, and \cite{wallace2002household}. More generally, gender here can be replaced by other contextually-relevant identity markers. Looking ahead, we consider how gendered household networks impact the centrality of households in \Cref{subsec:gender}, and we consider how gendered differences and differences in power (with respect to household decision-making) relate to the \textit{similarity} of a set of nodes in a given context in \Cref{sec:rec} and show examples in \Cref{fig:ex_decision}. 

\paragraph{Household generator questions and layered household contraction.} 
Household networks are typically studied by aggregating information from the individual network level using the basic households rule in \Cref{eq:contraction} (see \Cref{tab:review}). However, the \textit{household strategies} literature we briefly reviewed in \Cref{sec:background} highlights that there is a key difference between individual relationships and those which entire households utilize, benefit from, and maintain. The contraction rules previously discussed are most suitable in contexts wherein it is appropriate to construct the household network \textit{from} the individual network\textemdash as opposed to constructing the household network from particular \textit{household generating questions}. This distinction is highlighted by contrasting the implications of two common name generator questions, \textit{``Who would you borrow kerosene or rice from?''} and \textit{``Who do you go to for medical advice?''} \citep[e.g.,][]{banerjee2013}. The first question may generate names of individuals who are stand-ins for an entire \textit{household} (if you go to X's house to ask for rice but X is not home, do you ask whoever answers the door?). The latter question, however, generates a particular \textit{individual} network, distinct from a \textit{shared household} one (the trust involved in asking X for advice about your health may not transfer to X's spouse).

We can consider that when the individual social network is collected using the union of $P$ different name generators it can also be represented as a \textit{multilayer network} $\mathcal{G} = \{G_1, G_2, \dots, G_P\}$ \citep[see, e.g.,][]{kivela2014multilayer}. As such, we may construct the household network through node contraction on the set of the most relevant household generating layer(s), $\mathcal{L}$, where
$G_L = (V_L, E_L)$ for each $L \in \mathcal{L}$:
\begin{equation*} \label{eq:layered_con}
    \begin{aligned}
    &G' = (V', E')\\
    &V' = \{ w_i \}_{i=1}^{R} \\
    &E' = \{ (w_i, w_j) \mid \exists \, (v_k, v_\ell) \in E_L \textrm{ for some }L \in \mathcal{L} \textrm{ and for } v_k \in H_i, v_\ell \in H_j  \}.
    \end{aligned}
\end{equation*}

\subsection{Choices in practice} \label{subsubsec:emp_choice}
To show how the choice to collect, represent, and study a social network at either the individual or household level is made in practice, we review a wide range of empirical studies on social networks in \Cref{tab:review}. We see that social networks are often collected by asking \textit{individuals} name generator questions about their social support systems (sometimes about multiple types of social support). Individuals and their connections are then grouped together with other individuals living in the same household to make a \textit{household} social network, through the ``basic household contraction'' rule discussed above, forming a household network that is the ultimate object of study. 

In reviewing each study we focus on identifying (i) whether the \textit{intervention} or \textit{experimental design} of the study occurs at an individual or household level, (ii) if the social network data is collected by asking individual or household questions, and (iii) if the network studied is the individual or household network. We delineate these characteristics to motivate our work and clearly highlight opportunities for rigorously aligning a research question with its corresponding social network data collection and analysis. In \Cref{sec:rec} and \Cref{fig:ex_decision} we focus on three relatively recent large scale experiments, \cite{banerjee2013}, \cite{banerjee2019}, and \cite{airoldi2024}, which we review in greater detail in \Cref{apx:emp_rev}.


%%%%%%%%%%%%%%%%%%%%%%%%%%%%%%%%%%%%%%%%
%          LANDSCAPE TABLE             %
%%%%%%%%%%%%%%%%%%%%%%%%%%%%%%%%%%%%%%%%
\afterpage{%
  \clearpage% Ensure the table starts on a fresh page
  \newgeometry{margin=.75cm} 
  \begin{landscape}
    \begin{table}
\begin{tabular}{|p{1.2in}|p{.9in}|p{1in}|p{2.2in}|p{4in}|}\hline
\vspace{.05cm}\textbf{Paper} &\vspace{.05cm} \textbf{Network collected} &\vspace{.015cm} \textbf{Network studied} &\vspace{.05cm} \textbf{Context} &\vspace{.05cm} \textbf{Intervention}\\ \hline
  \vspace{.5pt} \cite{alatas2012targeting} & \vspace{.5pt}  Household\tablefootnote{The household network is collected by interviewing one individual from each household. The person is asked to list other households they are related to, and then to list all social groups that each person in their household belongs to. To construct the household network, an edge connects any two households with individuals in the same social group.}& \vspace{.5pt}  Household & \vspace{.5pt}  The adoption of an agricultural technology across 631 villages in Indonesia & \vspace{.5pt} An individual is targeted with information about the agricultural technology. \\ \hline
 \vspace{.5pt}\cite{banerjee2013} &\vspace{.5pt}  Individual &\vspace{.5pt}  Household\tablefootnote{This paper notes the motivation for using the household network is because loan-participation occurs at the household level.} &\vspace{.5pt} Adoption of microfinance loans across 72 villages in Karnakata, India. & \vspace{.5pt} An individual is targeted with information about a microfinance loan \\ \hline
 \vspace{.5pt} \cite{cai2015social} & \vspace{.5pt} Individual\tablefootnote{The head of each household is asked to list their five closest friends. Individuals who are \textit{not} head of their household can be included in the individual network only if they are nominated, and may not nominate their own friends. Notably, in this study heads of households are ``almost exclusively male.''} &  \vspace{.5pt} Household &\vspace{.5pt} Adoption of weather insurance across 185 villages in rural China. &\vspace{.5pt}  The head of a randomly targeted household participates in an informational session about the weather insurance product. \\ \hline
 \vspace{.5pt} \cite{chami2017social} &\vspace{.5pt} Individual &\vspace{.5pt} Household\tablefootnote{This paper also notes the motivation behind their decision to study the household network: ``because community medicine distributors (CMDs) were trained to and have been shown to move from door to door to deliver medicines during MDA.''} &\vspace{.5pt} Fragmentation of the contact network by targeting households with deworming treatments across 17 villages in Uganda. &\vspace{.5pt} With different targeting techniques, a random household is visited by a medical team to administer a deworming treatment. The household is considered noncompliant if \textit{any} individual within the household refuses treatment. \\ \hline
\vspace{.5pt} \cite{banerjee2019} &\vspace{.5pt} Individual &\vspace{.5pt} Household &\vspace{.5pt} The spread of information in 71 villages in Karnakata, India. &\vspace{.5pt} The household corresponding to an individual selected either randomly or by community nomination is given information about a non-competitive raffle.\tablefootnote{In this same work, in a second phase of experiments on a different set of villages, the information is about vaccination clinics.}\\ \hline
\vspace{.5pt} \cite{alexander2022algorithms} &\vspace{.5pt} Individual &\vspace{.5pt} Individual (representing household) &\vspace{.5pt} Spillover effects of a health intervention under different seeding strategies in 50 residential buildings in Mumbai, India. &\vspace{.5pt} A subset of women are given information about the health benefits of iron-fortified salt and are given coupons to buy the salt and share with their networks. \\ \hline
\vspace{.5pt} \cite{airoldi2024} &\vspace{.5pt} Individual &\vspace{.5pt} Household &\vspace{.5pt} Spillover effects of education about maternal health in 176 rural villages in Honduras. &\vspace{.5pt} A healthworker comes to a random household at regular intervals over the course of 22 months to educate household members about maternal health. \\ \hline
\end{tabular}
\caption{An overview of major social network experiments in networks with household structure. For each study, we categorize the level the network is collected and studied at, as well as the level of the intervention and additional context.}\label{tab:review}
\vspace{-1.5in}
\end{table}
 % Include your table here
  \end{landscape}
  \restoregeometry
  \clearpage% Ensure the subsequent text starts on a new page
}
%%%%%%%%%%%%%%%%%%%%%%%%%%%%%%%%%%%%%%%%
%      End LANDSCAPE TABLE             %
%%%%%%%%%%%%%%%%%%%%%%%%%%%%%%%%%%%%%%%%


