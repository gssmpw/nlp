As we saw in \Cref{sec:consequences}, the insights from a given network analysis can substantively vary between individual and household networks. In order to help guide the decision of which network should be analyzed in a certain context, in this section we discuss specific recommendations based on the work we reviewed in \Cref{sec:background}. 

\subsection{Entitativity criteria} \label{subsec:entitativity_rec}
We first adapt \citeauthor{campbell1958common}'s criteria for entitativity\textemdash \textit{proximity}, \textit{similarity}, \textit{common fate}, and \textit{internal diffusion}\textemdash specifically to their potential use in assessing households as social entities. We also propose a dependent hierarchy of these entitativity criteria in \Cref{fig:dec_chart}, where we propose a decision tree for determining what network to study in a given context and how to weight edges or include intrahousehold edges. We note that while we do not adapt \citeauthor{campbell1958common}'s criterion of \textit{resistance to intrusion} here, we discuss doing so as a path for interesting future work in \Cref{sec:conclusion}.

\paragraph{Proximity.}
Proximity of individuals is determined by their physical closeness to one another. Although it may seem straightforward to assess this criteria for individuals in a household (because they all live in the same house), in some places and on certain timescales, household members may actually not be proximate. For example, assessing proximity must be approached with care in places where it’s common for one or more household members to leave the village/town for work for extended periods of time \citep[see][]{niehof2011conceptualizing}, or where the boundary of the household is more culturally nebulous \citep{yotebieng2018household}. Assessing this question in an experimental context depends both on the timescale of the intervention and on cultural norms regarding household membership. These considerations aside, an aggregate of individuals living in the same household necessarily meet this criterion. Assessing if the household meets the following entitativity criteria \textit{alongside} proximity considers whether or not the household boundary is what \citeauthor{campbell1958common} calls an \textit{illusion}\textemdash a boundary that satisfies one criterion but not others.

\paragraph{Similarity.} Similarity of individuals within their household unit should be particularly assessed with respect to an experimental question and/or intervention. Although not exhaustive, helpful dimensions along which to assess similarity are gender, power, and household roles. For example, if the experiment is focused on the spread of information, similarity must be assessed with respect to the content of the information that the individuals are exposed to and if the individuals' differences in gender, power, or household roles impact the sharing of information. Guiding questions to assess this criterion in a given context might be: 
\begin{itemize}
    \item \textit{Is there a gender-related norm, stigma, or significance regarding sharing this type of information?}
    \item \textit{Does one person in the household make financial decisions on behalf of their family that might be relevant to this experimental context?}
    \item \textit{Does the subject or outcome of the experiment lie primarily within the set of roles of one household member?}
\end{itemize}

\paragraph{Common fate.} This criterion addresses, within the specific context of the intervention and tracked outcomes for a particular study, whether or not individuals living in the same household share the same outcome. If the outcome and/or intervention occurs at an individual level, the individuals likely do not share a common fate. In \citeauthor{campbell1958common}'s introduction of this criteria he alludes to particles within a rock\textemdash if the rock is thrown, all the particles in the rock will be thrown together, and thus share a common fate. Indeed, the concept of common, or \textit{linked} fate in sociology has a long history, particularly in the context of considering racial, ethnic, and socioeconomic groups \citep{brewer2000superordinate, simien2005race}. In the context of individual and household networks, we consider individuals within a household as sharing a common fate when a decision or intervention implicates all of the household members (e.g., the decisions to take on a microfinance loan, install a new roof, or purchase a new household grocery item are all decisions in which the individuals within the same household share a common fate).

\paragraph{Internal diffusion.} This criterion considers whether information (or a behaviour, or a disease) spreads to all individuals in a household once one household member is exposed to it. This criterion is particularly relevant and important when assessing entitativity of households in experiments concerned with the spread of information (or disease dynamics), because modeling contagion on the household network necessarily implies that once a household is exposed to information, \textit{all individuals} within the household are as well. 

If a household network is the most appropriate network to study in a given scenario, particular attention should be given to \textit{which} questions are generating the network, as well as \textit{who} is being asked the questions. As we propose in \Cref{fig:dec_chart}, it is most important to capture \textit{household} connections as distinct from the aggregate of individual connections when there is no internal diffusion amongst household members. In this case, it is important to capture only the relationships between individuals which represent a broader connection between the households as a whole. We discussed this distinction in the discussion of \textit{household generator questions and layered household contraction} in \Cref{sec:choices}, wherein we pointed out the difference between the significance of the name generators, \textit{``Who do you borrow kerosene or rice from?''} and \textit{``Who do you go to for medical advice?''} Whereas the first question is still asked of \textit{individuals}, it represents a larger relationship between \textit{households}, and thus becomes the relevant relationship to consider in this scenario. 

\paragraph{Consistent metrics.} We add this criterion to \citeauthor{campbell1958common}'s set of criteria for use in the particular setting of analyzing networks, and is helpful in cases where determining entitativity might be non-obvious. This criterion is meant to assess whether or not the particular network metric of interest (e.g., an influence-maximizing seed set, clustering coefficient, inversity, or diffusion centrality) is qualitatively consistent regardless of which network is used. If it is not, we recommend analysis of the individual network. 

\subsection{Examples} \label{subsec:rec_ex}
In this section we consider three examples for how to assess each of the entitativity criteria in practice. To do so, we utilize the decision tree in \Cref{fig:dec_chart} to determine which network to study in the contexts of the experiments considered in \cite{banerjee2013}, \cite{alexander2022algorithms}, and \cite{airoldi2024}. We provide relevant context for each experiment in \Cref{apx:emp_rev} and evaluate each criteria in detail in \Cref{fig:ex_decision}. Notably, the networks we recommend studying differ from the networks that were actually studied in \cite{banerjee2013} and \cite{airoldi2024}. However, in the context of \cite{alexander2022algorithms}, our recommendation is aligned with the network studied in the original research, where the individual female network was studied.

For the setting and intervention context of \citet{banerjee2013}, in which an unweighted household network was studied in the original work, we instead recommend studying the individual network with weighted edges and with interhousehold edges only present if those connections were specifically reported. Although we determine that household members share a \textit{common fate} in this setting (the success of the intervention is tracked at the household level), the absence of \textit{similarity} and \textit{internal diffusion} between household members ultimately determines our recommendation to study a weighted individual network. In this context, it is not necessarily true that household members are similar in their financial decision-making roles \citep[e.g.,]{holvoet2005impact}, and thus we recommend weighting edges between individuals who do hold that primary household responsibility more heavily than between individuals who do not make financial decisions. As such, because not all household members necessarily share the same incentives or interest in this topic, we suggest that there is not necessarily internal diffusion in this context, and thus we recommend only connect individuals in the same household if those edges were individually reported.

In the context of \citet{airoldi2024}, where the intervention and outcome tracked is the spread of information about maternal health, we recommend studying a household network with edges weighted according to the gender of relationships between individuals in each household. This recommendation is in contrast to the unweighted household network studied in the original work. In this context, where outcomes are measured at the household level, we assess that household members share a \textit{common fate}. Furthermore, because the intervention occurs at the household level with health information presented to every household member, it is reasonable to assume that there is \textit{internal diffusion} and thus that households should be connected if any individual in one is connected to any individual in another \citep[this is how households are connected in ]{airoldi2024}. However, household members are not \textit{similar} in this context: there are likely gendered differences in how individuals discuss and share gendered health information \citep[see, e.g.,][for a discussion of how gender plays a role in family-planning decisions, discussions, and outcomes in Honduras]{berti2015adequacy}. As such, we recommend weighting the household network according to how many female-to-female connections there are between households. Under this recommendation, relationships between households where only men have relationships with other men will have a lower weight than relationships between households where women are connected to other women. This recommendation accounts for the relative differences in how gendered health information like maternal health is shared.

In each of the above experimental settings, a key goal of the work is to understand the spread of information over a network. As such, and as we saw above in \Cref{subsubsec:emp_choice}, the choice of \textit{how} to represent the network is necessary to properly understand the relevance of the conclusions which are drawn from network analysis. Our recommendations provide a principled approach for making this choice. Although out of scope for the present work, reconsidering the analysis of the above experiments with the networks we recommend, and exploring how the resulting conclusions differ is subject for interesting and important future work.


%%%%%%%%%%%%%%%%%%%%%%%%%%%%%%%%%%%%%%%%
%          LANDSCAPE TABLE             %
%%%%%%%%%%%%%%%%%%%%%%%%%%%%%%%%%%%%%%%%
\newgeometry{margin=2cm} 
\begin{landscape}
\begin{figure}
    \centering
    \includegraphics[width = \linewidth]
    {figures/ex_flowchart_dejavu.pdf}
    \caption{In this figure we show three examples \citep{banerjee2013, alexander2022algorithms, airoldi2024} of how to use the decision chart given by \Cref{fig:dec_chart} to determine what level of  network to use in a given context. For each example, we recommend a different type of network, with varying edge weights and intrahousehold connections. Our recommendations differ from the network analyzed in the original studies for both \citet{banerjee2013} and \citet{airoldi2024}.}
    \label{fig:ex_decision}
\end{figure}
\end{landscape}
\restoregeometry
%%%%%%%%%%%%%%%%%%%%%%%%%%%%%%%%%%%%%%%%
%      End LANDSCAPE TABLE             %
%%%%%%%%%%%%%%%%%%%%%%%%%%%%%%%%%%%%%%%%