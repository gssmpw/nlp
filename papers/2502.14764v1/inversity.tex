In this section we expand on possible explanations for the observation noted in \cite{kumar2024friendship}: that the \textit{inversity} when computed for the individual and household networks from \cite{banerjee2013} substantively differs. As mentioned in \Cref{subsub:local}, we find that the difference in the inversity between the household and individual networks can be largely attributed to the fact that in the individual networks in the \cite{banerjee2013} dataset, individuals living in the same household are completely connected to one another. 

In fact, in \cite{kumar2024friendship}, they note that ``We can expect hub-based (or star) networks to have high inversity \dots In contrast, we expect that networks of clusters of various sizes will have low inversity.'' With this intuition in hand, and knowing that the individual networks are quite literally constructed with clusters of completely connected nodes---which get contracted when forming the household networks---the change in inversity which was surprising becomes highly expected. 

To isolate the impact of intrahousehold edges on inversity, we decompose the adjacency matrix so that we may vary the weight we give to intrahousehold connections in different contexts. This is the same decomposition we discuss in more detail in \Cref{subsec:inf_dif}.
\begin{align}
    \A^*_p = \A_{extra} + (1- p) \A_{intra},
\end{align}

\begin{figure}
    \centering
    \includegraphics[width=0.7\linewidth]{figures/inversity_p_mean.pdf}
    \caption{Expected inversity of the individual village networks with intrahousehold edges removed with probability $p$, with $p$ ranging from $0.1$ to $1$, corresponding to adjacency matrix $\A^*_p$. For each village we compute the mean inversity over 100 instances of intrahousehold edge removal, and plot the mean inversity over all villages with the shaded region capturing the 5\% and 95\% quantiles across the mean inversity for all villages.}
    \label{fig:inv_p_dots}
\end{figure}

In \Cref{fig:inv_p_dots} we plot the inversity of the individual village networks with intrahousehold edges removed with probability $p$, with $p$ ranging from 0.1 to 1. We see that inversity monotonically increases with more intrahousehold edges removed. We conclude the exploration into the impact of intrahousehold edges on inversity by considering that in \cite{kumar2024friendship}, the authors give the intuition that inversity is at its most negative when edges connect nodes of the same degree. This intuition gives more context to the impact that the complete subgraphs in the individual network have on the observed inversity\textemdash when individuals are completely connected with their household members, there is less degree variation across edges. We explore this impact further in \Cref{fig:prop_edges}, where we plot the inversity of the individual network against the proportion of the total edges in the network that exist between individuals in the same household. Indeed, we see that networks with a higher proportion of intrahousehold edges have a more negative inversity.

\begin{figure}
    \centering
    \includegraphics[width=0.7\linewidth]{figures/prop_edges.pdf}
    \caption{Individual network inversity (x-axis) against the proportion of the total edges in the network that can be attributed to edges within households (y-axis). Networks with a higher proportion of edges due to intrahousehold connections have a more negative inversity, due to less degree variation across edges.}
    \label{fig:prop_edges}
\end{figure}