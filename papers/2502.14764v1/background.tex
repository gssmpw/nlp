The study of how individuals within social networks act and react in aggregate broadly spans political science, anthropology, and sociology. We briefly introduce this related work so that we may adapt it to make concrete recommendations the context of social networks in \Cref{sec:rec}. We note that the related work we consider here often spans or defies disciplinary boundaries. As considered in the network science literature, the abstract question we consider (of node-aggregation) is closely related to the literature on \textit{coarse-graining} \citep{itzkovitz2005coarse, gfeller2007spectral, kim2004geographical, klein2020emergence} where the goal is to explore \textit{how} to aggregate nodes while preserving some property of the network (e.g., degree distribution, spectral properties, properties of random walks). However, as we will see in this section, questions of how \textit{people} and their relationships can be considered in aggregates have been considered in many disparate literatures.

\subsection*{Name generators and multilayer networks}\label{subsec:namegen}
The question of how individuals are related through social networks has been extensively studied in the literature on \textit{name generators} \citep{campbell1991name}, which specifically studies the social network survey question(s) being asked and the impact the choice of question has on the structure of the resulting network. In the present work, we echo the scholarship's contextual focus on \textit{which relationships we care about} and \textit{how to ask the relevant question(s)} to capture those relationships. Furthermore, we complement the line of work by considering \textit{which nodes} (individuals or households) we care about representing in a given context. Here we also see the connection between the present work, name generators, and \textit{multilayer networks} \citep[e.g.,][]{kivela2014multilayer}. Succinctly, multilayer networks can capture relationships given by multiple name generator questions, by representing each \textit{type} of relationship in a different layer. In the context of household networks, multilayer networks allow for the possibility of multiple types of household connections\textemdash whether gendered, neighbourly, individual, or shared\textemdash to be represented.

\subsection*{Diffusion of voter information campaigns within the household}\label{subsec:gotv} 
How information spreads and is shared within households, particularly in the context of political messaging, has been extensively experimentally tested. 
In \citet{nickerson2008}, the authors target households with two registered voters and deliver a ``Get out the Vote'' message to whichever household member answers the door, ultimately finding that 60\% of the propensity to vote is passed to the other member of the household. In considering how this finding might be impacted by gender differences, \cite{cheema2023canvassing} studies the difference between targeting women and men with voter information interventions in Pakistan. They find that female voter turnout increases by 5.4 and 8.0 percentage points, respectively, when either the man or both man and woman are targeted, as compared with only targeting the woman in a household. \citet{ferrali2022registers}, however, finds that household heads (regardless of their gender) have a strong influence over whether other adults in their household register to vote. \citet{bhatti2017voter} complicates this and related findings by suggesting that, rather than the \textit{information} from voter education campaigns being shared with other household members, the spillover of increased voter participation can be attributed to social pressure exerted by the targeted household member. That is, their work suggests that pure information does not diffuse within a household, but rather personal opinions and social pressure.

These experimental findings suggest that information shared with one member of a household does not definitively imply that all members of the household will receive or be impacted by it \citep{nickerson2008}. Moreover, the \textit{impact} of information on a household can be vastly different, depending on the gender and hierarchical position of \textit{who} in the household receives it \citep{cheema2023canvassing, ferrali2022registers}. Furthermore, observing an outcome does not distinguish between information and social pressure diffusing between household members \citep{bhatti2017voter}. 

Broadly, this body of work cautions against an assumption that information reaching one member of a household guarantees that it reaches every other household member, an assumption implicit in how household networks are constructed in practice. Indeed, \textit{why}, \textit{how}, and \textit{how much} information is shared between household members is contextually dependent on hierarchy within the household, gender norms and expectations, and the type of information being shared. 

\subsection*{Household strategies}\label{subsec:housestrat}

The rich literature on \textit{household strategies}\textemdash which studies \textit{household networks} as forms of interaction and informal means of economic exchange\textemdash specifically distinguishes between individual relationships and household relationships. An observation key to our work is explicitly stated in \citeauthor{werner1998}'s ethnographic work on household strategies in Kazakhstan \citep{werner1998},  wherein she observes that it is ``wrong to assume that a household network equals the sum total of all household members' social bonds\dots In reality, a social bond established by one household member may or may not be at the disposal of other household members.'' Not all relationships that an individual has are accessible to all members of their household\textemdash the accessibility of particular connections may be dependent on context, gender, and power. A salient example of this claim is presented by \cite{werner1998}: a wife does not have direct access to her husband's professional connections, and her access and benefit of them must be mediated through him. 

Indeed, gender and power are intimately connected with \textit{how} household relationships are created and maintained, and the broader role of household networks is culturally, politically, and economically dependent \citep[see, e.g.,][]{werner1998, schmink1984household, wallace2002household, niehof2011conceptualizing, yotebieng2018household}.

\subsection*{Entitativity}\label{subsec:entity}

In her ethnographic evaluation of household strategies, \cite{werner1998} remarks that, ``individuals are social actors, but households are not.'' Questioning \textit{when} a group of people can be considered as a social entity itself leads us to the foundational work of \cite{campbell1958common}, an in-depth theoretical assessment of this very question. \citeauthor{campbell1958common} looks towards the biological sciences to propose several ``metrics'' for determining what he terms \textit{entitativity}\footnote{In reference to this clunky word, we echo \citeauthor{campbell1958common} in apologizing that ``the present writer regrets adding two suffix syllables to the word \textit{entitative}, already three fourths suffixes.''}: the degree to which a group of people can be considered one entity.\footnote{We note that \citeauthor{campbell1958common} exclusively uses the word ``metric'' colloquially with respect to rules for assessing whether a group is an entity or not, and not paired with any quantitative sense of a metric. To avoid confusion with quantitative metrics discussed in the present work, we instead use the word \textit{criteria} for Campbell's notion of ``metric.''}

The several criteria \citeauthor{campbell1958common} proposes are those assessing the \textit{proximity}, \textit{similarity}, \textit{common fate}, \textit{internal diffusion}, and \textit{reflection or resistance to intrusion} of the individuals within a potential aggregate. We consider the first four criteria in more detail and within the context of household networks in \Cref{sec:rec}, and discuss the fifth criterion as an interesting direction for future work in \Cref{sec:conclusion}.


Importantly, \citeauthor{campbell1958common} cautions against what he calls \textit{illusions} of entitativity which occur when solely relying on \textit{one} criteria to determine when an aggregate of people can be considered an entity. He comments on the importance of confirming boundaries of entitativity with multiple criteria, writing that \textit{illusions often rely on superficial boundaries.} It is in this sense that we aim to diagnose when it is an \textit{illusion} to treat households as social entities: if, in a given context, the boundary of the household unit is not validated by other entitativity criteria. 


We adapt the findings of this related work into a systematic recommendation for determining when the household or individual network should be collected and analyzed in a given setting. We organize our recommendations in  \Cref{fig:dec_chart}, where we propose a decision tree for determining what network to study in a given context. We discuss the evaluation criteria in detail in \Cref{sec:rec}.