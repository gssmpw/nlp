\subsection{Random graphs with random households}\label{subsub:ER}

To understand how node contraction can impact the structure of the resulting graph, we first explore an example of randomly contracting nodes in a random graph on $n$ nodes where an edge exists between any two nodes with independent probability $p$, called an \ER~random graph and denoted $G(n, p)$. In \Cref{apx:ER} we show that when households are of the exact same size, the resulting graph can be explained by the same \ER~generative process, but fewer nodes and with a different edge probability. This observation, however, no longer holds when we consider households of differing sizes.

\begin{proposition}
   Consider a random graph $G(V,E)$ generated by a $G(n,p)$. A corresponding unweighted household graph $G'(V',E')$ is constructed by constructing $m$ disjoint node sets $H_1, H_2, \dots, H_R$ of size $\ell_1, \ell_2, \dots \ell_R$ by choosing nodes uniformly at random, and contracting these sets into corresponding nodes $V' = \{w_r\}_{r=1}^R$, where $n = \sum_{r=1}^R\ell_r$. Then $G'$ is no longer described by an \ER~model, but an \text{inhomogeneous \ER~model} \citep{bollobas2007phase} where edges do not exist with the same probability across nodes. Instead, an edge $(w_q, w_r)$ exists independently with probability $1 - (1-p)^{\ell_q \ell_r }.$ The expected degree of node $w_k$ is $\mathbb{E}[deg(w_k)] = \sum_{j = 1}^R 1 - (1-p)^{\ell_k \ell_j}.$ For a simple graph, $j \neq k.$ 
\end{proposition}

\begin{proof}
    In the graph $G$, for any pair of nodes $v_i, v_j \in V$, the edge $(v_i, v_j)$ does not exist with independent probability $1-p$. Now consider node sets $H_q$ and $H_r$. Then for node $v_i \in H_q$, there is probability $(1-p)^{\ell_r}$ that there is no edge between node $v_i$ and any node in set $H_r$. Since there are $\ell_q$ nodes in set $H_q$, and each edge probability is independent, the total probability of no edge existing between any node in $H_q$ and any node in $H_r$ is $(1-p)^{\ell_q \ell_r }$. Thus for the graph $G'$ where any node $w_k$ is constructed by contracting nodes in set $H_k$, then for any pair of nodes $w_q$ and $w_r$ in $G'$, edge $(w_q, w_r)$ exists independently with probability $1 - (1-p)^{\ell_q \ell_r }.$ Thus we can describe $G'$ as a realization of an inhomogeneous \ER~model where the expected degree of node $w_k$ is $\mathbb{E}[deg(w_k)] = \sum_{j = 1}^R 1 - (1-p)^{\ell_k \ell_j}.$ If $G$ and $G'$ are simple graphs, we enforce $j \neq k$. 
\end{proof}

The point of this example is to show that contracting nodes can significantly alter the structure of the resulting graph, by changing the variation in expected degrees across nodes in the network. With this example in mind, we turn to analyze a set of empirical networks to see how metrics change under household node contraction.