\section{Experiments}
\label{sec:sec4}

To test our model, we set up batch experiments that simulate interactions of an agent (that could be, for example, a human) with some devices in a smart home. Those devices are sensors and actuators. As sensors, we have, for example, presence detectors and thermometers, that generate the relevant data as the agent has its normal behaviors. As actuators, we some devices that would require direct interaction, like light switches, from which we get the agent preferences. Our goal was to map the readings of sensors and the activation of the actuator devices. We did find works that used interaction-centered data like \cite{engelmann2016interaction}, but none of them had a publicly available dataset.

Specifically, we simulated the house and its devices as a server that listens to a specific port for information and change requests. A sensor may request the current readings of its specific device, and an actuator may set a device ''on'' or ''off''. 



As explained before, one of the main processes in the training of our Cognitive Twin is the Evolution Strategy, where we let a set of random choices of Perceptual and Behavioral Codelets be part of the agent. Each Perceptual differs from the other by the sensors it collects information: this may be any number between half of the available sensors and all of them, chosen randomly. Each Behavioral Codelet differs from others in the Motor Codelet (or Codelets) it feeds. 

Each individual is defined by an array of binary values representing the agent, so the mutation process is a "bit-flip" probability, and we choose not to use crossing-over methods. The selection method is "best five", meaning that we keep the best five individuals in each generation. The population at the beginning of each generation has twenty individuals and the initial population was randomly generated with each individual having a 20\% probability of '1' in each gene, which roughly reflects as a "mean participating number of Codelets" of 20\%.



The experiments were aimed to be relatively human-like, even if it is still synthetic. Based on the work of \cite{mendez2009simulating}, we generated data based on a transition probability matrix associated with interaction probability functions. So, our data is generated based on which room the agent would be in and with which devices it interacts. First, our base agent has some proprieties, like thermal comfort, that are the equivalent of a human having some preferences. These properties - and how their values are chosen - are:

\begin{itemize}
    \item \textbf{thermal\_comfort}: a uniformly random integer number in the interval [20, 30] that represents the temperature at which our base agent feels uncomfortable and turns AC on.
    \item \textbf{light\_comfort}: a uniformly random real number in the interval [0.4, 1] that represents the luminosity that our base agent feels uncomfortable and turns lights on.
    \item \textbf{voice\_prob}: a uniformly random real number in the interval [0, 1] that represents the probability that our base agent uses a voice-activated device.
    \item \textbf{coffee\_prob}: a uniformly random real number in the interval [0, 1] that represents the probability that our base agent uses the coffee machine, once in the room.
    \item \textbf{plant\_water\_th}: a uniformly random integer number in the interval [40, 100] that represents a threshold that a plant should be watered above.
\end{itemize}



\begin{figure}[bt]
\centering
	\includegraphics[width=1.0\columnwidth]{imgs/house_2.png}
\caption{House designed for the experiments. Notice this was used only to have a clear image when building an adjacency matrix}
\label{fig:house}
\end{figure}


Figure \ref{fig:house} illustrates the designed house. An agent could only move between consecutive rooms, as a human would do. This could be represented by an adjacency matrix. 
Considering that each non-null element of the adjacency matrix has an equal probability of being chosen, a transition matrix was built from it. This represents a ''room transition matrix'' and, in each room, the agent will have some interactions with room elements. These elements, being sensors and actuators, are shown in the list below alongside a brief explanation of how their states work.

\begin{itemize}
    \item \textbf{Existing sensors}
        \begin{itemize}
            \item  'presence': if the agent is on the room, it's '1'. Otherwise, is '0'
            \item  'temperature': a random integer number in [20, 30] interval
            \item  'luminosity': a random integer number in [0, 10] interval
            \item  'voice': a random binary choice with $p_1$ equal 'p\_voice'
            \item 'smoke\_detector': a random binary choice with probability of $p_1 = 0.1\%$ representing a fire occurrence
            \item 'shower\_temperature': a random integer number in [20, 30] interval
            \item 'humidity': a random integer number in [0, 10] interval
        \end{itemize}
    \item \textbf{Existing actuators}
        \begin{itemize}
            \item  'air\_conditioner': '1' if the 'temperature' sensor is lower than 'thermal\_comfort' and '0' otherwise 
            \item  'lights': '1' if the 'luminosity' sensor is lower than 'light\_comfort' and '0' otherwise   
            \item  'sound\_box': it's '1' if the 'voice' sensor was activated and '0' otherwise 
            \item 'coffee\_machine': a random binary choice with $p_1$ equal 'p\_coffee'
            \item 'anti\_fire': it's '1' if the ''smoke\_detector'' sensor was activated and '0' otherwise
            \item 'shower\_control': '1' if 'shower\_temperature' sensor is lower than 'thermal\_comfort' and '0' otherwise 
            \item 'plant\_watering': '1' if 'humidity' sensor is lower than 'plant\_water\_threshold' and '0' otherwise 
        \end{itemize}
\end{itemize}

After being in one specific state (one 'room'), all relevant sensors and actuator states change according to their behaviors. Lists of sensors and actuators each room has are shown below.
\begin{itemize}
    \item \textbf{living\_room}:
        \begin{itemize}
            \item sensors: 'presence', 'temperature', 'luminosity', 'voice'
            \item actuators: 'air\_conditioner', 'lights', 'sound\_box'
        \end{itemize}
    \item \textbf{bedroom}:
        \begin{itemize}
            \item sensors: 'presence', 'temperature', 'luminosity', 'voice'
            \item actuators: 'lights', 'sound\_box'
        \end{itemize}
    \item \textbf{kitchen}:
        \begin{itemize}
            \item sensors: 'presence', 'temperature', 'luminosity', 'smoke\_detector'
            \item actuators: 'lights', 'coffee\_machine', 'anti\_fire'
        \end{itemize}
    \item \textbf{bathroom}:
        \begin{itemize}
            \item sensors: 'presence', 'temperature', 'luminosity', 'shower\_temperature'
            \item actuators: 'lights', 'shower\_control'
        \end{itemize}
    \item \textbf{outside}:
        \begin{itemize}
            \item sensors: 'presence', 'luminosity', 'voice', 'humidity'
            \item actuators: 'lights', sound\_box', 'plant\_watering'
        \end{itemize}
\end{itemize}

For each run, we then took 400 samples based on a random walk using the transition matrix and the device interaction function shown above.
We ran a total number of 5000 experiments, in which we built slightly different base agents and different sample sets. This difference occurs in the randomly generated parameters reflecting ''preferences'', like \emph{thermal\_comfort}.



The experiments have a very specific main goal: to achieve an agent topology that presents the lowest (ideally zero) difference between expected output values and the actual ones. In the next section, we present the results and discuss them.
