



%****Cognitive Twin: A Cognitive Approach to Personalized Assistants
%Healthcare is a perfect example of an area where IoT virtual- ization also factors the human in, e.g., by creating digital twins of the human body that can be used for real-time remote-monitoring of physiological functions (Bruynseels, Santoni de Sio, and van den Hoven 2018). Beyond modeling the human as an organism, digital twins can expand to cover social dimensions of human life, which are characterized by our daily interactions with portable electronics, digital services, social networks, etc. Accordingly, in this paper we introduce the notion of cognitive (digital) twin, to highlight the key role that cogni- tive mechanisms play in modeling human decision making in the IoT digital space. Dinner Party Planning We have chosen as our proof-of-concept scenario to implement the planning of a dinner party. We chose this scenario for a number of reasons.

%***Cognitive Digital Twins for the Process Industry
%a discussion on an application of such an approach to a real-world problem from steel production process industry is presented in this section. The use case shows how various hurdles concerning asset maintenance and predictive controls from the process industry can be further improved from its current state. The steel production process typically has three stages. First, the scrap steel is collected and melted in an electric arc furnace. In the second stage, the molten melt is transferred to the ladles for secondary metallurgy. In the third and final step, the casting process, the molten steel is moulded to a desired shape. In the secondary metallurgy process, the molten metal


%***Cognitive Twin: A Personal Assistant Embedded in a Cognitive Architecture
%This paper presents an analysis of a cognitive twin, imple- mented in a cognitive architecture. The cognitive twin is in- tended to be a personal assistant that learns to make decisions  from your past behavior. In this proof-of-concept case, we have the cognitive twin select attendees to a party, based upon what it has learned (through ratings) about an agent’s social network. We evaluate two versions of a model with respect to rate of change in the social network, the noise in the rating data, and the sparsity of the data.

%In previous work (Somers, Oltramari, & Lebiere, 2020) we tested two versions of the cognitive twin, central and dis- tributed, against one another and against the simulation in their respective success at inviting guests to a ‘dinner party’. In this section we describe the dinner-party-planning scenario and the simulation.



%***Enhancing Cognition for Digital Twins
%“Built on the foundations of the IIoT and utilizing data analytics combined with cognitive technology, the emerging field of Cognitive Manufacturing is characterized by the vision and capacity to perceive changes in the production process and know how to respond to these dynamic variations with minimal human intervention. It does this by proposing improvements in processes and operations while suggesting alternatives to reduce cost and environmental impacts.” This had led to the recent development of the concept of Cognitive (Digital) Twins (CTs), which incorporate data-driven models produced via Data Analytics and Machine Learning to the DT artefact. In this paper, we introduce the notion of Enhanced  Cognitive Twins (ECTs), taking the notion of cognition a step further by providing CTs with the tools to actually “suggest alternatives to reduce cost and environmental impacts” as the WMF report states. This is implemented by incorporating optimization algorithms built to work both with numerical and data-driven models of production assets, thus facilitating decision-making under different objectives. The end-goal is to enable the realization of the Cognitive Factory as an ensemble of independent but intertwined ECTs, that are (i) able to self learn, and thus to effectively detect and react to anomalies and disruptions, but also to opportunities that may arise, (ii) enjoy a local or global view of operations and (iii) are capable forshort-, mid- and long-term optimization and reasoning.



%****Towards Actionable Cognitive Digital Twins for Manufacturing
%Shop-floor is the area of a manufacturing plant where production takes place.  It contains the machines required for production and is the place where work- ers operate them or manage the production process. Relevant Key Performance Indicators (KPIs) to the shop-floor are Operational Equipment Effectiveness (OEE)[14] and Overall Process Efficiency (OPE)[7] among others.


%Problem 1: anomaly detection: what anomalies do occur during produc- tion? How do they impact the existing production process?  
%– Problem 2: production planning: how do we re-schedule existing pro- duction plans based on factors such as early or late terminations, or lack of skilled workers? How do we mitigate potential issues that could affect operational up-time such as lack of required materials or skilled workers?
