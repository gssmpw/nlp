\section{Results and Discussion}
\label{sec05}

In this section, we present the results, discuss them, and make some conclusions about the experiments.

With a slightly realistic scenario, the experiments present some interesting results. Figures \ref{fig:hist_score} and \ref{fig:hist_gen} show respectively histograms of (a) the final score after the full training process and (b) the number of generations the process took. Notice that most runs just stopped at 20 generations (maximum) and could not improve further, as Figure \ref{fig:hist_gen} suggests. Despite that, as can be seen in Fig. \ref{fig:hist_score}, more than 80\% of the runs ended with a score of 2 or less, meaning at most two wrong device activations on 260 interactions. 

\begin{figure*}
        \centering
        \begin{subfigure}[b]{0.475\textwidth}
            \centering
            \includegraphics[width=0.8\textwidth]{imgs/results/results_2/histogram_of_score_.png}
            \caption[]%
            {{\small Histogram of \textit{score} achieved on experiment runs. Notice that the lesser, the better.}}    
            \label{fig:hist_score}
        \end{subfigure}
        \hfill
        \begin{subfigure}[b]{0.475\textwidth}  
            \centering 
            \includegraphics[width=0.8\textwidth]{imgs/results/results_2/histogram_of_generations_.png}
            \caption[]%
            {{\small Histogram of \textit{generations} needed to achieve the lower score. Here, most runs needed the maximum number of generations}}    
            \label{fig:hist_gen}
        \end{subfigure}
        \caption[]
        {\small Histograms of the lowest score and generations needed to achieve that on each experiment run.} 
        \label{fig:hist_metrics}
\end{figure*}

Figures \ref{fig:hist_beh} and \ref{fig:hist_per} reflect the number of Behavioral and Perceptual Codelets respectively to achieve the best result in each run. As we can see in Fig. \ref{fig:hist_beh}, most runs needed 13 Behavioral Codelets, one for each Motor Codelet/actuation device. Fig \ref{fig:corr_behavior} shows the correlation between the number of behavioral codelets and score. The system response is better (lower score) as more Behavioral Codelets are used.
The number of Perceptual Codelets, on the other hand, shows approximate normal distributions with a slight bias to the right, meaning that the embedding may vary and the output still be good. This bias is reflected in the slight negative correlation between the number of Perceptual Codelets and Score (smaller than Behavioral).

\begin{figure*}
        \centering
        \begin{subfigure}[b]{0.475\textwidth}
            \centering
            \includegraphics[width=0.8\textwidth]{imgs/results/results_2/histogram_of_behaviorals_x.png}
            \caption[]%
            {{\small Histogram of the number of Behavioral Codelets needed to achieve the lowest score. Most runs needed 13, the same number of Motor Codelets (and actuators).}}    
            \label{fig:hist_beh}
        \end{subfigure}
        \hfill
        \begin{subfigure}[b]{0.475\textwidth}  
            \centering 
            \includegraphics[width=0.8\textwidth]{imgs/results/results_2/histogram_of_perceptuals_x.png}
            \caption[]%
            {{\small Histogram of the number of Perceptual Codelets needed to achieve the lowest score.}}    
            \label{fig:hist_per}
        \end{subfigure}
        \caption[]
        {\small Histogram of the number of ''internal'' Codelets needed to achieve the lowest score on each run.} 
        \label{fig:needed_codelets_2}
\end{figure*}


\begin{figure*}
        \centering
        \begin{subfigure}[b]{0.475\textwidth}
            \centering
            \includegraphics[width=0.8\textwidth]{imgs/results/results_2/correlation_correlation_of_score_with_behaviorals.png}
            \caption[]%
            {{\small Correlation between number of Behavioral Codelets and Score}}    
            \label{fig:corr_behavior}
        \end{subfigure}
        \hfill
        \begin{subfigure}[b]{0.475\textwidth}  
            \centering 
            \includegraphics[width=0.8\textwidth]{imgs/results/results_2/correlation_correlation_of_score_with_perceptuals.png}
            \caption[]%
            {{\small Correlation between number of Perceptual Codelets and Score}}    
            \label{fig:corr_per}
        \end{subfigure}
        \caption[]
        {\small Correlation between the number of ''internal'' Codelets and Score} 
        \label{fig:corr}
\end{figure*}



Table \ref{table:exp2} shows some statistics taken from the experiment. Notice that, while the individual number of Perceptual and Behavioral Codelets may go as low as 2, the combined ''Internal'' Codelets need a higher number to present satisfactory results.

\begin{table}[]
\centering
\caption{metrics on Experiments}
\label{table:exp2}
\begin{tabular}{@{}llllll@{}}
\toprule
              & mean    & median & std   & min & max \\ \midrule
score         & 1.438   & 1.0    & 1.894 & 0   & 26  \\
generations   & 13.6784 & 20.0   & 8.905 & 0   & 20  \\
n perceptuals & 9.5326  & 10.0   & 2.213 & 2   & 15  \\
n behaviorals & 11.2674 & 13.0   & 2.478 & 2   & 13  \\
n internals   & 20.8    & 22.0   & 3.998 & 7   & 28  \\ \bottomrule
\end{tabular}
\end{table}

\subsection{Conclusion and Future Works}

This paper has presented a pioneering approach to creating a Cognitive Twin by leveraging a distributed cognitive system in conjunction with an evolution strategy. Our work stands as a significant contribution to the field of cognitive computing by demonstrating the feasibility of orchestrating a multitude of simple physical and virtual devices to mimic a person's interaction behaviors. This achievement not only offers a practical application of distributed cognitive systems but also introduces a novel methodology for cognitive twin development, emphasizing the role of evolution strategies in optimizing system topology for more accurate behavior emulation.

In revisiting the themes introduced at the outset, our research seamlessly integrates the foundational principles of cognitive systems, Cyber-Physical Systems (CPS), and Systems of Systems (SoS) with contemporary advancements in artificial intelligence. By doing so, we have illustrated a comprehensive framework that not only addresses the complexities of human behavior simulation but also opens new avenues for automation, human-like agent creation, and in-depth behavioral analysis.

Comparatively, our approach distinguishes itself from established cognitive architectures such as ACT-R and SOAR, and the Standard Model of Mind, by emphasizing distributed processing and adaptability. While ACT-R and SOAR offer rich insights into cognitive processes through detailed psychological models, our model excels in harnessing distributed, interconnected devices to capture the multifaceted nature of human cognition. Similarly, the Standard Model of Mind provides a foundational framework for understanding cognitive functions. Yet, our work extends this understanding into the practical domain of CPS and distributed systems, offering a unique perspective on cognitive replication and interaction dynamics.

In conclusion, our research not only underscores the potential of distributed cognitive systems in creating sophisticated cognitive twins but also highlights the importance of evolutionary strategies in refining these systems. By drawing parallels and distinguishing our work from established cognitive architectures like ACT-R, SOAR, and the Standard Model of Mind, we contribute a novel perspective to the ongoing discourse on cognitive modeling and simulation. 



Future work will focus on further refining the distributed cognitive system and exploring its integration with other AI paradigms and models. This research sets the stage for developing more sophisticated Cognitive Twins capable of performing complex tasks with minimal human intervention. By continuing to build on this foundation, future studies can enhance the fidelity and applicability of Cognitive Twins, making them tools in the field of cognitive computing.
    
