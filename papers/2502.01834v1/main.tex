%% 
%% Copyright 2019-2021 Elsevier Ltd
%% 
%% This file is part of the 'CAS Bundle'.
%% --------------------------------------
%% 
%% It may be distributed under the conditions of the LaTeX Project Public
%% License, either version 1.2 of this license or (at your option) any
%% later version.  The latest version of this license is in
%%    http://www.latex-project.org/lppl.txt
%% and version 1.2 or later is part of all distributions of LaTeX
%% version 1999/12/01 or later.
%% 
%% The list of all files belonging to the 'CAS Bundle' is
%% given in the file `manifest.txt'.
%% 
%% Template article for cas-dc documentclass for 
%% double column output.

\documentclass[a4paper,fleqn]{cas-dc}

% If the frontmatter runs over more than one page
% use the longmktitle option.

%\documentclass[a4paper,fleqn,longmktitle]{cas-dc}

%\usepackage[numbers]{natbib}
%\usepackage[authoryear]{natbib}
\usepackage[authoryear,longnamesfirst]{natbib}

%%%Author macros
\def\tsc#1{\csdef{#1}{\textsc{\lowercase{#1}}\xspace}}
\tsc{WGM}
\tsc{QE}
%%%

% Uncomment and use as if needed
\newtheorem{theorem}{Theorem}
%\newtheorem{lemma}[theorem]{Lemma}
%\newdefinition{rmk}{Remark}
%\newproof{pf}{Proof}
%\newproof{pot}{Proof of Theorem \ref{thm}}


% isso aqui permite os "definitions"
%\theoremstyle{definition}
\newtheorem{definition}{Definition}[section]

%\usepackage{cite}
\usepackage{amsmath,amssymb,amsfonts,url}
\usepackage{algorithmic}
\usepackage{graphicx}
\usepackage{textcomp}
\usepackage{xcolor}
%\usepackage[left=1in,right=1in,top=0.75in,bottom=1in]{geometry}
\usepackage{subcaption}
%\usepackage[redeflists]{IEEEtran}

\usepackage{booktabs}

%\newtheorem{theorem}{Theorem}

\begin{document}


\let\WriteBookmarks\relax
\def\floatpagepagefraction{1}
\def\textpagefraction{.001}

% Short title
\shorttitle{Building a Cognitive Twin Using a Distributed Cognitive System and an Evolution Strategy}    

% Short author
\shortauthors{Wandemberg Gibaut; Ricardo Gudwin}  

% Main title of the paper
\title [mode = title]{Building a Cognitive Twin Using a Distributed Cognitive System and an Evolution Strategy}  



\author[1]{Wandemberg Gibaut}
%\author[1]{Author1}

% Corresponding author indication
\cormark[1]

% Footnote of the first author
\fnmark[1]

% Email id of the first author
\ead{wgibaut@dca.fee.unicamp.br}

%\credit{<Credit authorship details>}

% Address/affiliation
\affiliation[1]{organization={University of Campinas (Unicamp)},
            %addressline={}, 
            city={Campinas},
          citysep={}, % Uncomment if no comma needed between city and postcode
            %postcode={}, 
            state={São Paulo},
            country={Brazil},
            orcid={ https://orcid.org/0000-0001-7322-5399}}

%\affiliation[1]{organization={org},
%            city={city},
%            state={state},
%            country={country}}%,
            
            
\author[2]{Ricardo Gudwin}
%\author[2]{Author2}

% Footnote of the second author
\fnmark[2]

% Email id of the second author
%\ead{}

% URL of the second author
\ead[url]{https://faculty.dca.fee.unicamp.br/gudwin/}
%\ead[url]{url}

% Credit authorship
%\credit{aaaa}

% Address/affiliation
\affiliation[2]{organization={University of Campinas (Unicamp)},
            city={Campinas},
            state={São Paulo},
            country={Brazil}}

%\affiliation[2]{organization={org},
%            city={city},
%            state={state},
%            country={country}}%,


% Corresponding author text
\cortext[1]{Corresponding author}

% Footnote text
\fntext[1]{}

% For a title note without a number/mark
%\nonumnote{}

% Here goes the abstract
\begin{abstract}
This work presents a technique to build interaction-based Cognitive Twins (a computational version of an external agent) using input-output training and an Evolution Strategy on top of a framework for distributed Cognitive Architectures. Here, we show that it's possible to orchestrate many simple physical and virtual devices to achieve good approximations of a person's interaction behavior by training the system in an end-to-end fashion and present performance metrics. The generated Cognitive Twin may later be used to automate tasks, generate more realistic human-like artificial agents or further investigate its behaviors.

\end{abstract}

% Use if graphical abstract is present
%\begin{graphicalabstract}
%\includegraphics{}
%\end{graphicalabstract}

% Research highlights
%\begin{highlights}
%\item We show a practical application of a distributed cognitive system
%\item A Cognitive Twin is possible by training systems end-to-end
%\item Evolution Strategy plays a role in defining the topology
%\end{highlights}

% Keywords
% Each keyword is seperated by \sep
\begin{keywords}
Cognitive Systems \sep  Artificial Intelligence \sep  Distributed Systems \sep Cognitive Twin \sep Internet of Things
\end{keywords}

\maketitle




\section{Introduction}
\label{sec:intro}

Foundational models (FMs)~\cite{zhang2024data, zhou2023comprehensive} have shown remarkable progress in the healthcare domain, enabling professional-like assessment of disease diagnosis, treatment decision-making, and monitoring~\cite{zhang2023text, wang2022medclip, lu2023mi-zero}. 
Examples include LLaVA-Med~\cite{li2023llava}, Med-PaLM Multimodal~\cite{tu2024towards}, and Med-Flamingo~\cite{moor2023med}, have demonstrated their capacity on question answering, medical image analysis, and report generation.
These studies follow a predominant top-down model development strategy that requires upstream developers to collect data and train models for downstream tasks. 
Consequently, the developed model capabilities are heavily dependent on the training data, limiting their generalization performance in diverse clinical scenarios. 
For instance, Med-Gemini~\cite{yang2024advancing} reveals promising general capabilities in report generation while it lags behind state-of-the-art (SoTA) models on classification tasks, especially for out-of-domain applications. 
This indicates that while the generalizability of the foundation model is promising, more solutions are expected to meet the various specialized clinical needs.

To address these challenges, multi-center data centralization becomes essential to enhance model capacity and robustness across varied clinical scenarios~\cite{rajpurkar2022ai}. 
Centralizing distributed data can significantly improve model training and inference performance.
However, the process of medical data storage, transfer, and aggregation among centers requires extra efforts to ensure data security and system interoperability~\cite{bradford2020international}.
Moreover, a growing concern for patient privacy makes large-scale multi-center data sharing particularly challenging. 
While efforts like federated learning~\cite{wen2023survey, li2020review} can achieve good model performance on local data, the need for synchronized system coordination presents significant challenges, as clients are unable to update asynchronously. This limitation greatly restricts the practical capability of such approaches.
As a result, without a flexible collaboration, medical community still struggles to fully utilize the isolated data and local computation resources for comprehensive medical AI model development. 
To address this dilemma, open-source platforms encourage public data sharing and knowledge integration~\cite{markiewicz2021openneuro, zenodo}.
However, these platforms focus solely on raw data sharing while seldom providing collaborative model training or cooperation between different institutions.
Recently, collaborative learning has emerged as a viable approach for enhancing multi-model robustness~\cite{boulemtafes2020review}. 
For instance, software-like model development~\cite{raffel2023building} mimics software engineering practices by introducing structured workflows, enabling merging, version control, and continuous model integration.
Under this design, model ability can be strengthened with incremental knowledge updates similar to the version updating in software development. 

Although collaborative learning provides a multi-model collaboration, two key challenges remain in the leakage of raw data during collaboration~\cite{huang2023lorahub} and the synchronization of multiple collaborators~\cite{mcmahan2017communication} in the medical AI community. It is still challenging to integrate decentralized, privacy-sensitive data across institutions, leading to under-utilized insights and fragmented knowledge sharing~\cite{kaissis2020secure, rajpurkar2022ai, abdullah2021ethics}.
 To address these challenges, inspired by the collaborative software development, we propose \textbf{Med}ical \textbf{Fo}undation Models Me\textbf{rg}ing (\textbf{MedForge}), a cooperative workflow enabling continuously community-driven foundation model (FM) development.
MedForge enables a lightweight manner for individual centers to share their knowledge among multiple centers, minimizing the burden of data transmission and integration while enhancing model robustness.
Meanwhile, MedForge facilitates asynchronous and flexible collaboration, allowing individual centers to continuously update and improve medical FMs without the need for real-time synchronization.
Similar to open-source software development, MedForge incrementally updates medical knowledge and follows a sustainable model development scheme. 
This key design emphasizes a bottom-up construction of a multi-task medical FM, allowing downstream users to collaboratively build, refine, and update the upstream model according to their local resources. Our major contributions of MedForge are as below: 
\begin{enumerate}
    \item[$\bullet$] We introduce a collaborative workflow to promote the merging scheme of open-source software development. Our proposed MedForge allows distributed clinical centers to asynchronously contribute to comprehensive medical model construction while reducing transmitting costs among centers and avoiding the leakage of raw data, thus enhancing the utilization of private resources in the healthcare system. 
    \item[$\bullet$] We propose two effective knowledge-merging strategies for the asynchronous branch contribution. The MedForge-Fusion strategy updates the plugin module parameters of the main model during the merging phase, whereas the MedForge-Mixture strategy integrates the output of the plugin module by memorizing each contributor's coefficient. These strategies make MedForge more flexible and versatile. MedForge-Fusion is friendly to implement, while the MedForge-Mixture offers better performance and robustness.
    \item[$\bullet$]  We comprehensively evaluate model merging strategies to accumulate medical knowledge among multiple branch plugin modules. MedForge yields superior performance on medical classification tasks compared to other collaborative baselines across multiple datasets. We demonstrate the robustness of MedForge by shuffling the task order and evaluating various configurations of plugin modules and dataset distillation methods.
\end{enumerate}



\section{The DCT}
\label{sec:sec2}

In this section, we will briefly explain our main tool. The DCT \citep{gibaut2020extending}, an acronym for Distributed Cognitive Toolkit, is a bare-bones toolkit to help the development of cognitive systems in a distributed, language-agnostic fashion. A cognitive agent created with DCT should be able to run across multiple physical (desktops, small computers like Raspberries, or microcontrollers like Arduino) or virtual devices (like Docker containers). It is a re-implementation of the ideas first seen in the CST main article \citep{Paraense201632}, as some features like being inherently single-device and being written in Java may be a shortcoming, sometimes. As expected, it also follows some theory lines like being Codelet-oriented, present in the Copycat architecture \citep{hofstadter1994copycat}. 

To better understand the DCT, one should first refer to the Cognitive Systems Toolkit (CST), the toolkit that came before. As its name suggests, CST is a toolkit for the development of cognitive architectures. Its purpose is to facilitate the creation of such systems the way the user wants, as long it respects its premises. In the core of CST are two basic entities that serve as building blocks: Codelets and Memories. A Codelet is a non-blocking, parallel process that runs continuously and represents a very specific piece of the cognition process of biological creatures. Likewise, a memory is a storage structure from which Codelets read and write information. A user may create any architecture that also follows a Codelet-oriented specification, based on already existing theories or new ones.

Many Cognitive Architectures like e.g. MECA \citep{gudwin2017multipurpose}, LIDA \citep{franklin2014lida}, and others may be seen as multi-agent systems. Following this paradigm, DCT conceives a distributed Cognitive System as a multi-agent system, where a standard protocol is used for the communication among the agents. Also, similar to its predecessor (CST), each agent in such a multi-agent system is built using Codelets and Memories.  The structure and functionality of a multi-agent system are fully compatible with distributed computing concepts and may be comfortably mapped to the Internet of Things: Sensory Codelets may be simply real-world sensor devices and Motor Codelets may be simply relays or actuators, while more complex Codelets may be embedded in microcontrollers or even software containers in the cloud. 


Our first prototype of DCT - and current software based on it - is written in \emph{Shell script} and \emph{Python}, and could be deployed in containers (like Docker) or across different devices, including Raspberries and Arduinos. Figure \ref{fig:multi_device} illustrates the idea: if the input/output conditions are satisfied, there is no need to execute the whole system on a single computer. The project repository can be accessed in: \href{https://github.com/wandgibaut/dct}{https://github.com/wandgibaut/dct}.


\begin{figure*}[hbt]
\centering
	\includegraphics[width=2.0\columnwidth]{imgs/Distributed_MECA_2.png}
\caption{Illustration of a multi-device, Codelet-oriented system as seen in \cite{gibaut2020extending}. Notice that, depending on how powerful is the device, it may run a single Codelet or multiple ones.}
\label{fig:multi_device}
\end{figure*}


\subsection{The DCT Architectural Overview}
\label{sec:dct_over}

From an architectural perspective, the DCT is composed of a set of \emph{Nodes} communicating to each other and integrating, as a whole, a functional system with cognitive capabilities. Theoretically, different subsets of the same collection of Nodes could even act as different systems. Here, the term \emph{Node} represents an entity (logical or physical), which works as a storage for groups of Codelets and/or Memories and is responsible for their operation and life cycles. Figure \ref{fig:multi_device} illustrates the idea of this non-homogeneity in device configurations. The subsection \ref{sec:dct_node} shows further details on this entity.

To communicate with each other, \emph{Nodes} follow a protocol regulating the interaction among them. Following CST specifications, Codelets only interact with Memories that is, a Codelet represents a block of computing unity, applying some process on data, but not holding it. This data storage is performed by Memories, which can be of different technologies. For this communication, DCT uses, canonically, \emph{json} formatted messages. This allows the use of a good range of technologies and simple sockets and databases like \emph{MongoDB} and \emph{Redis} are already supported by existing code. By following these directives, a user may use any language or technology suitable to a device in which the \emph{Node} is.


Formally, we can conceive a Cognitive System created with DCT in the following way:


\begin{definition}[A DCT Cognitive System]

Let $N$ be a set of \emph{Nodes}, where a \emph{Node} is an entity (logical or physical) that encapsulates one or more \emph{Codelets} and/or \emph{Memories} meant to be run under the supervision of a single \emph{Node Master} within an operational system.

To each \emph{Node}, there is an \emph{Interface} $I = \{MO, S\}$, where $MO$ is a subset of the \emph{Memories} implemented within a \emph{Node}, which will be accessed from other external \emph{Nodes} and $S$ is a Server that listens to a URI. This server $S$ should listen for requests and respond with \emph{json} formatted messages.


A Cognitive System created with DCT is defined by the interaction between the elements of $N$ following some Codelet-oriented Cognitive Theory, like MECA or LIDA.
\end{definition}






\subsection{The DCT Codelet structure}
\label{sec:dct_Codelet}


A DCT \emph{Codelet} is composed of a callable program file in a user-specified language that follows some guidelines and some configuration files that can be used to dynamically change some properties, like which \emph{Memories} it can access. Figure \ref{fig:Codelet} illustrates this structure. It is valid to note that, since it was first implemented, some improvements have been made in how a \emph{Codelet} works. The files that characterize a \emph{Codelet} are:


\begin{itemize}
\item A \verb!Codelet! compiled program or script, which runs until be ordered (by the \emph{Node Master}) to stop
\item The \emph{Codelet} configuration file (\emph{fields.json}). This file contains some information regarding the \emph{Codelet} behavior, like its inputs, and should be possible to dynamically change it.
\end{itemize}

Also, if needed for a problem-specific reason, additional files may be used (a \emph{.ini} file, for example). In this work, \emph{Codelet} is implemented in \emph{Python} language.


%%arrumar figura
\begin{figure}[bt]
\centering
	\includegraphics[width=1.0\columnwidth]{imgs/DCT_Codelet_3.png}
\caption{The concept of a DCT Codelet.}
\label{fig:Codelet}
\end{figure}

Following the original CST implementation, the \emph{Codelet} program should have two main functions: a  \verb!calculateActivation! and a \verb!proc!. The first one is used to calculate the current relevance of the \emph{Codelet} itself and may be used in different ways, but mainly with a threshold value to decide if it should execute its main function or skip it. The second one performs the core functionality and is the most important function to be defined by the user. This function represents the procedural code that the entity will periodically call at each time step. Also, it should be non-blocking, meaning that it should be possible to run other processes alongside a \emph{Codelet}.

The configuration file (\emph{fields.json}) contains a structure that is analog to the \emph{Codelet} class in CST, defining many important parameters, e.g. the input, output, and broadcast ports, from where the Codelet can communicate to its Memory objects. 


\subsection{The DCT Memory structure and default support}
\label{sec:dct_memory}

The other core structure, \emph{Memory}, is a generic term for the data structure that holds the information that \emph{Codelets} consume and/or process. Also, it contains some other meta-information, e.g. its name, URL, type, and an evaluation. As said before, this information is standardized as a \emph{json} structure. 

A \emph{Memory} should contain the following information:
 \begin{itemize}
     \item \emph{name}: String 
     \item \emph{IP/port}: String
     \item \emph{type}: String
     \item \emph{I}: String
     \item \emph{eval}: Double 
 \end{itemize}

\subsection{DCT Node}
\label{sec:dct_node}

In the DCT, a \emph{Node} is an abstraction for a physical or virtual device that contains an arbitrary number of \emph{Codelets} and/or \emph{Memories} and is supervised by a single \emph{Node Master}. This definition allows us to consider a computer to be a single \emph{Node} if all relevant entities run in the same environment, or to \textit{have} multiple \emph{Nodes} if each of them runs in a separated container with its \emph{Node Master}.
This \emph{Node Master} is responsible for starting, killing, adding, and removing Codelets and/or Memories, which are running through its supervision. Also, it should periodically check the health of its system, re-executing dead processes, and listen for external requests, like information requests or even requests to shut itself down.

Besides \emph{Codelets} and \emph{Memories}, a \emph{Node} should also implement an \emph{Interface} in which its internal entities may communicate with outside sources, e.g. a server with open sockets.


\section{An Evolutionary Cognitive Twin}
\label{sec:sec3}

Since we have already introduced some key concepts and the tool we're using, we can now discuss the proposed technique. Here, the main point is to build a Cognitive Twin using a vast amount of simple devices, orchestrated to work together as a single system, even if each device is a system of its own and may, theoretically, be a part of another system.

Here in this work, based on the definitions in section \ref{sec:sec1} and within the scope of what will be presented, the following definition will be presented:

\noindent\fbox{%
    \parbox{\columnwidth}{%
A \textit{Cognitive Twin} is a digital replica of the dynamics and cognitive - or just cognitive - processes of an intelligent physical system, usually aimed at a partial representation of a person. These cognitive processes refer to those identified in cognitive theories, such as perception, memory, behavior, adaptation, planning, learning, \textit{Reasoning} etc. The classification of an agent as \textit{Cognitive Twin} refers not only to the duplication of observable behavior of the virtual agent concerning the original but also to the possibility of in-depth investigation of the original individual through its copy.
    }%
}
\vspace{7mm}

The most fundamental idea here is the Codelet, already discussed in section \ref{sec:sec2}. We argue that sensors and actuators can be seen as Sensory and Motor Codelets, respectively. To make that consideration, we considered that both sensors and actuators are simple devices that do a very specific task, following the idea of Codelet, as seen in sections \ref{sec:sec1} and \ref{sec:sec2}. This consideration allows us to model our desired agent as a composition of simpler elements that interact with each other as needed.

Following that perspective, we postulate that the connections between sensors and actuators are given by some combination of elements that group and give them some sense, and elements that use that information to control actuators. This lets us use the concepts of Perceptual Codelets and Behavioral Codelets, respectively, largely used in other works that follow a Codelet-oriented Cognitive agency paradigm. Also, these premises fit in the concepts presented in section \ref{sec:sec1}: The System is composed of devices that are themselves systems connected through a network with an orchestration to build a Cognitive Agent that bridges virtual and physical domains.

But we have two main constraints: first, we sought to use simple, low computing power devices, and we do not have an infinite number of devices (with an infinite variety of input-output responses) to search for the best combination. We approach those constraints by having devices that can learn (or somehow adapt) and by making use of a heuristic that improves this search. It's also impractical to have those devices fully connected, as this communication overhead may degrade performance and/or be impossible if we deal with physical devices.

So, we propose to find the optimal configuration by having both explicit training on Codelets and an Evolution Strategy to find a suitable connection between Perceptual and Behavioral Codelets and between Behavioral and Motor Codelets. We'll discuss this in detail in the next subsections.


%propor o seguinte: a conexão eh encontrada com genetica e os nos perceptuais e comportamentais são decisiontrees

\subsection{Devices internal structure}
 First, we to define the internal structure of the devices we worked on. For the sake of simplicity, we used only virtual devices, as defined in section \ref{sec:sec2}, running Python Codelets on Docker. That allowed us to better manipulate some structures, like sending or requesting data from the master program to/from each Codelet and creating or destroying those virtual devices as we needed them. Even so, we kept the internal structures simple to draw a parallel with low-power devices. 
 
 As we mentioned before, we follow MECA Theory, which uses both \cite{Osman2004} theory of two, separated Cognition Systems working together and a \emph{Codelet}-oriented structure. Here we present a System 1 approach, which means that we'll be working only with four types of \emph{Codelets}: \emph{Sensory, Perceptual, Behavioral} and \emph{Motor Codelets}. We will briefly explain each of them.
 
 
 \subsubsection{Sensory Codelets}
 The simplest of them all, Sensory Codelets represent actual sensors, either physical or virtual. Like sensors, they are responsible for introducing the raw data into the system. For example, if we consider a human eye a sensor, the raw data is the light that enters the pupils. Here, it requests information about a specific attribute - temperature or luminosity, for example - of an environment. In this work, Sensory Codelets' internal structures will not be changed, simulating very simplistic devices, like a digital thermometer.
 
 \subsubsection{Perceptual Codelets}
 \label{sub:per}
 The subsequent structures in the information flow are Perceptual Codelets. These structures are responsible for aggregating the raw data that comes from Sensory Codelets in structures called Perceptions. In our human eye example, outputs of Perceptual Codelets would be \emph{depth}, \emph{objects}, relational properties (like distance from something), and so on. Notice that Perceptual Codelets, in a sense, represent how an agent experiences the world, as the information it could extract from data is heavily dependent on them. In this work, the internal structure of a Perceptual Codelet is represented by a Decision Tree Classifier, where the inputs are sensory data and the output is an integer value that represents a unique identifier (a token) of the input sensor's readings. Each Perceptual Codelet differs from another by the combination of its Sensor Codelets on the input. Those inputs are defined randomly with a uniform distribution to both quantity (a value between half and all the sensors) and which ones are picked up. %Figure \ref{fig:perceptualDT} illustrate the idea. 
  
  
 \subsubsection{Behavioral Codelets}
 Next, we have the Behavioral Codelets. The main purpose of these structures is to, based on previously structured information, activate one or more protocols to control what the agent should do, that is, to control one or more Motor Codelets. This Activation is usually encoded in a 0 to 1 \emph{float} value representing a Signal Strength, a way to measure how important that Behavior is to the current situation. Note that this so-called protocol may be anything from a simple heuristic to a whole Machine Learning method and the input of the Behavioral Codelet may include not only Perception but other information like those coming from a Motivational or even Emotional subsystem. Also, as a single Motor Codelet may have multiple Behavioral Codelets as input, those behaviors effectively compete to prevail and have their commands accepted.
   
   In this work, the Behavioral Codelets have also a Decision Tree Classifier as a method to decide what to send to its Motor Codelets based on Perception. 
   
 \subsubsection{Motor Codelets}
 The last basic structure here is the Motor Codelet. As mentioned before, this represents a direct parallel with an actuator, being physical or not. It simply responds to what was put as input and, through another Decision Tree, it sends a command to the corresponding entity in our virtual environment. This could be a direct association (''if this then that'') but, to make further usage of the code easier, we used a method that could accept more than one Behavioral without having to rewrite it completely.
 
 
 Figure \ref{fig:agent_struct} shows an overview of the structure of the topology of an agent. This concludes our overview of the main structures of our work. Next, we will detail about the Evolution Strategy.


  \begin{figure}[ht]
\centering
	\includegraphics[width=1.0\columnwidth]{imgs/cog_twin.drawio.png}
\caption{Graphical representation of an agent structure and its internal connections}
\label{fig:agent_struct}
\end{figure}

\subsection{The optimization}

The general process of building the architecture for our Cognitive Twin involves determining the connection topology between the different types of Nodes and adjusting the internal functions of each Codelet to reproduce the overall behavior of a primordial agent. This is a two-step offline optimization process, the first being an optimization of the connection topology between Nodes and the second being a conventional training process of the Codelets' internal Machine Learning models in each Node. The connections between the Nodes will be defined through an evolutionary strategy and, given the connection configurations of each individual of a given generation, supervised training methods will be used to minimize the error between the expected and obtained outputs.
In this work, the internal structures of the Sensory Codelets are not changed, simulating very simplistic devices, such as a digital thermometer. 

%We present details of each of these steps below.

\subsubsection{Evolution Strategy details}

In this part of the process, we want to, through an evolution process, define the best configuration of the connections between Perceptual and Behavioral Codelets and Behavioral and Motor Codelets. The connections between Sensorial and Perceptual Codelets are fixed and explained in subsection \ref{sub:per}.

To apply an Evolution Strategy, we need to do some definitions. First, we need to define our Individual encoding. Here, our Individual is a binary vector with the length of the number of total Perceptual Codelets plus the number of total Behavioral Codelets, where each index represents a specific Codelet. In that definition, a '1' represents that the corresponding Codelet is active on the Agent composition and a '0' means a non-connected Codelet. An example Individual is shown in figure \ref{fig:individual}.

\begin{figure}[hb]
\centering
	\includegraphics[width=1.0\columnwidth]{imgs/individual.drawio.png}
\caption{Example Individual for our Evolution Strategy Process. The Individual is encoded as an array of binary values, each one representing if a certain Perceptual or Behavioral Codelet is to be considered as part of the agent.}
\label{fig:individual}
\end{figure}

Second, we need to define a mutation method. In this work, we adopted a simple 'bit-flip' probability for mutation, meaning that each individual has a probability \emph{mut\_p} to be mutated and each of its genes has a probability \emph{ind\_m} to change its state from '0' to '1' and vice versa. So, if a '0' becomes a '1', that means we should take the corresponding Codelet into account when mounting the agent topology.

As we adopted the output of Motor Codelets as exclusively binary, we can choose the fitness evaluation method as a Hamming Distance between the expected outputs and the actual ones. With this choice, the lower the Score, the better the individual fitness.

As a selection method, we choose to keep the best five individuals in the population for the next generation. Also, we choose the overall best one to be cloned. This procedure gives the possibility of recovering from local minima.

%initial pop


Also, we choose not to have any mating process. This choice was completely arbitrary, as we foresaw that it would not cause significant changes and required an additional process that could make each iteration longer.


\subsubsection{The training process}

To build our distributed agent correctly, we need a training process to ensure it accurately maps the system's inputs to the expected outputs. This training is done a) by changing its topology, choosing which Perceptual and Behavioral Codelets are composing the agent, and b) by fitting the data through all individual components consistently. 

The main component of the optimization process is very straightforward: it is a simple - yet efficient - Evolution Strategy to define the agent topology. This process is graphically represented in figure \ref{fig:evol_strategy}. But our \emph{evaluation} method requires more attention. It is in this part that we try to perform an input-output mapping.


\begin{figure}[ht]
\centering
	\includegraphics[width=0.7\columnwidth]{imgs/evolution_strategy.drawio.png}
\caption{Evolution Strategy process diagram. Here we have a high-level representation of each step of the mentioned heuristic.}
\label{fig:evol_strategy}
\end{figure}

% population creation   evaluation    selection   cloning and mutation    do it again

First, we get our \emph{Individual} and reconfigure the Codelets connections properly, including cleaning all \emph{memories} and deciding which \emph{Behavioral Codelet} will feed each \emph{Motor Codelet}. We do that by writing a new \emph{fields.json} file and sending it to each \emph{Codelet}, and by forcing an empty value to the relevant \emph{Memories}, such as the \emph{motor-memories}, in each \emph{Motor Codelet}.

In the second part, we get our training inputs and send them to each relevant (the ones with a correspondent '1' in \emph{Individual} encoding) \emph{Perceptual Codelet} and send them a "train" signal, through a special \emph{Memory} on them with this sole purpose. This training process aims to create unique values - like tokens - that identify each observed combination of sensor readings. These readings refer solely to the sensors at the input of each \emph{Perceptual Codelet}. These tokens are integers corresponding to positions on an array with unique observations. For example, if ''[[0, 1, 0], [0, 1, 1], [0, 1, 0]]'' represents a set of training inputs, then ''[[0], [1], [0]]'' would be the outputs. Remember that each \emph{Perceptual Codelet} has its own sensor connections, so their responses differ one from another.


Now, the most sensitive part, we need to train \emph{Behavioral Codelets} properly considering the input-output response of each \emph{Perceptual Codelet} and respective \emph{Motor Codelet}. We do that in two steps: aggregating \emph{Perceptual Codelets} responses for each training input and mapping those responses to a \emph{Motor Codelet} input that would generate the desired output. These two sets (\emph{Perceptual} responses and \emph{Motor} input) represent the training we have on each \emph{Behavioral Codelet}.

The aggregation step is done by requesting the already trained model from \emph{Perceptual Codelets} and its respective input masks (representing which sensors feed them) and mounting a conjoined Perceptual output. This approach may also be useful for the data-sensitive task, as the system that collects/sends the information need not be the same that centralizes the Evolution Strategy.

Then, we need to get how each \emph{Motor Codelet} responds, either by getting a trained model or by getting an input-output set directly. We opt for the latter since \emph{Motor Codelets} represent actuators and, usually, they don't hold sensitive data. This part is problematic, as a \emph{Behavioral Codelet} may be assigned to two (or more) \emph{Motor Codelets} with fundamentally different behaviors. That could make it impossible to achieve correct model training. After collecting data, we send each input-output information to the respective \emph{Behavioral Codelet}.


Next, we are ready to evaluate the system's performance. One by one, we send the corresponding entry in the input test set to the environment server the system is sensing (we talk about this specificity in section \ref{sec:sec4}) and wait until the information propagates through a distributed agent, getting the system response after that (all \emph{Motor Codelets} outputs combined). This ''wait time'' is directly related to communication overheads and process concurrency if using a single computer.

Finally, we calculate the Hamming Distance between the expected values (test output) and the actual output. This will represent the \emph{Fitness} of the \emph{Individual}.




\begin{table*}[t!]
\centering
% \vspace{5pt}
\begin{small}
\begin{tabular}{l|c|c|c|c|c|c|c}
\toprule
\textbf{Method} & \textbf{Type} & \textbf{ToMi} & \textbf{BigToM} & \textbf{MMToM-QA} & \textbf{MuMA-ToM} & \textbf{Hi-ToM} & \textbf{All} \\
\midrule
SymbolicToM & Specific & \textbf{98.60} & - &  - & - & - & - \\
TimeToM & Specific & 87.80 & - &   - & - & - & - \\
% \textbf{96.00$^*$}
PercepToM & Specific & 82.90 & - & - & - & - & - \\
BIP-ALM & Specific & - & - & 76.70 & 33.90 & - & - \\
LIMP & Specific & - & - & - & 76.60 & - & - \\
\ours w/ Model Spec. & Specific & 88.80 & \textbf{86.75} & \textbf{79.83} & \textbf{84.00} & \textbf{74.00} & \textbf{82.68} \\
\midrule
Llama 3.1 70B & General & 72.00 & 77.83 & 43.83 & 55.78 & 35.00 & 47.41 \\
Gemini 2.0 Flash & General & 66.70 & 82.00 & 48.00 & 55.33 & 52.50 & 60.91\\
Gemini 2.0 Pro & General & 71.90 & 86.33 & 50.84 &  62.22 & 57.50 & 65.76 \\ 
GPT-4o & General & 77.00 & 82.42 & 44.00 & 63.55 & 50.00 & 63.39 \\
SimToM & General & 79.90 & 77.50 & 51.00 & 47.63 & 71.00 & 65.41\\ 
\ours & General & \textbf{88.30} & \textbf{86.92} & \textbf{75.50} & \textbf{81.44} & \textbf{72.50} & \textbf{80.93} \\
\bottomrule
\end{tabular}
\end{small}
\caption{Results of \ours and baselines on all benchmarks. There are two groups of methods: methods that require domain-specific knowledge (e.g., AutoToM w/ Model Spec.) or implementations (e.g., SymbolicToM) and methods that can be generally applied to any domain. ``-'' indicates that the domain-specific method is not applicable to the benchmark. The best results for each method type are highlighted in bold.}
\label{tab:results}
\vspace{-10pt}
\end{table*}



\section{Experiments}
\subsection{Experimental Settings}



We evaluated our method on multiple Theory of Mind benchmarks, including ToMi \citep{le2019revisiting}, BigToM \citep{gandhi2024understanding}, MMToM-QA \cite{jin2024mmtom}, MuMA-ToM \citep{shi2024muma}, and Hi-ToM \cite{he2023hi}. The diversity and complexity of these benchmarks pose significant reasoning challenges. For instance, MMToM-QA and MuMA-ToM incorporate both visual and textual input, while MuMA-ToM and Hi-ToM require higher-order inference. Additionally, MMToM-QA features exceptionally long contexts, and BigToM presents open-ended scenarios.



Besides the full \ours method, we additionally evaluated \ours given manually specified models (AutoToM w/ Model Spec.). 

We compared \ours against state-of-the-art baselines:
    \textbf{LLMs:} Llama 3.1 70B \citep{dubey2024llama}, Gemini 2.0 Flash, Gemini 2.0 Pro \cite{team2023gemini} and GPT-4o \cite{achiam2023gpt};
    
     \textbf{ToM prompting for LLMs:} SymbolicToM \cite{sclar2023minding}, SimToM \cite{wilf2023think}, TimeToM \cite{hou2024timetom}, and PercepToM \citep{jung2024perceptions};
 
  \textbf{Model-based inference:} BIP-ALM \cite{jin2024mmtom} and LIMP \cite{shi2024muma}.


For multimodal benchmarks, MMToM-QA and MuMA-ToM, we adopt the information fusion methods proposed by \citet{jin2024mmtom} and \citet{shi2024muma} to fuse information from visual and text inputs respectively. The fused information is in text form. We ensure that all methods use the same fused information as their input.


We use GPT-4o as the LLM backend for \ours and all ToM prompting and model-based inference baselines to ensure a fair comparison—except for TimeToM, which relies on GPT-4 and is not open-sourced.


\subsection{Results}
The main results are summarized in Table~\ref{tab:results}. Unlike \ours, many recent ToM baselines can only be applied to specific benchmarks. Among general methods, \ours achieves state-of-the-art results across all benchmarks. In particular, it outperforms its LLM backend, GPT-4o, by a large margin. This is because Bayesian inverse planning is more robust for inferring mental states given long contexts with complex environments and agent behavior. It is also more adept at recursive reasoning which is key to higher-order inference. Notably, \ours performs comparably to manually specified models, showing that automatic model discovery without domain knowledge is as effective as human-provided models. We provide additional results and qualitative examples in Appendix~\ref{sec:more_results}.


\subsection{Ablated Study}



\begin{figure}[t!]
  \centering
  \includegraphics[width=0.8\linewidth]{figures/comparison.pdf}
    \vspace{-10pt}
  \caption{Averaged performance and compute of the full \ours method (star) and the ablated methods (circles) on all benchmarks.}
  \label{fig:ablation}
  \vspace{-10pt}
\end{figure}


We evaluated the following variants of \ours for an ablation study: no hypothesis reduction (\textbf{w/o hypo. reduction}); always using POMDP (\textbf{w/ POMDP}); always using the initial model proposal without variable adjustment (\textbf{w/o variable adj.}); only considering the last timestep (\textbf{w/ last timestep}); and considering all timesteps without timestep adjustment (\textbf{w/ all timesteps}).

The results in Figure~\ref{fig:ablation} show that the full \ours method constructs a suitable BToM model, enabling rich ToM inferences while reducing compute. We analyze key model components below:

\textbf{Hypothesis reduction.}
Compared to the full method, \ours w/o hypo. reduction has a similar accuracy but consumes 53\% more tokens on average, demonstrating that hypothesis reduction optimizes efficiency without sacrificing performance.

\textbf{Variable adjustment.}
\ours dynamically identifies relevant variables for ToM inference, generalizing domain-specific BIP approaches to open-ended scenarios. Compared to its variant without variable adjustment, \ours improves performance with minimal additional compute. The variant that always uses POMDP performs well in scenarios aligned with the POMDP assumption (e.g., MMToM-QA) but generalizes poorly elsewhere and incurs much higher computational costs. %, leading to an 8.5% performance deficit.

\textbf{Timestep adjustment.}
By selecting relevant steps for inference, timestep adjustment enhances performance by focusing on essential information. In contrast, the variant using only the last timestep misses crucial details, significantly lowering performance. The variant incorporating all timesteps suffers from higher computational costs and reduced accuracy due to conditioning on unnecessary, potentially distracting information.



Full ablation results are provided in Appendix~\ref{sec:more_results_ablation}.

\section{Results and Discussion}
\label{sec05}

In this section, we present the results, discuss them, and make some conclusions about the experiments.

With a slightly realistic scenario, the experiments present some interesting results. Figures \ref{fig:hist_score} and \ref{fig:hist_gen} show respectively histograms of (a) the final score after the full training process and (b) the number of generations the process took. Notice that most runs just stopped at 20 generations (maximum) and could not improve further, as Figure \ref{fig:hist_gen} suggests. Despite that, as can be seen in Fig. \ref{fig:hist_score}, more than 80\% of the runs ended with a score of 2 or less, meaning at most two wrong device activations on 260 interactions. 

\begin{figure*}
        \centering
        \begin{subfigure}[b]{0.475\textwidth}
            \centering
            \includegraphics[width=0.8\textwidth]{imgs/results/results_2/histogram_of_score_.png}
            \caption[]%
            {{\small Histogram of \textit{score} achieved on experiment runs. Notice that the lesser, the better.}}    
            \label{fig:hist_score}
        \end{subfigure}
        \hfill
        \begin{subfigure}[b]{0.475\textwidth}  
            \centering 
            \includegraphics[width=0.8\textwidth]{imgs/results/results_2/histogram_of_generations_.png}
            \caption[]%
            {{\small Histogram of \textit{generations} needed to achieve the lower score. Here, most runs needed the maximum number of generations}}    
            \label{fig:hist_gen}
        \end{subfigure}
        \caption[]
        {\small Histograms of the lowest score and generations needed to achieve that on each experiment run.} 
        \label{fig:hist_metrics}
\end{figure*}

Figures \ref{fig:hist_beh} and \ref{fig:hist_per} reflect the number of Behavioral and Perceptual Codelets respectively to achieve the best result in each run. As we can see in Fig. \ref{fig:hist_beh}, most runs needed 13 Behavioral Codelets, one for each Motor Codelet/actuation device. Fig \ref{fig:corr_behavior} shows the correlation between the number of behavioral codelets and score. The system response is better (lower score) as more Behavioral Codelets are used.
The number of Perceptual Codelets, on the other hand, shows approximate normal distributions with a slight bias to the right, meaning that the embedding may vary and the output still be good. This bias is reflected in the slight negative correlation between the number of Perceptual Codelets and Score (smaller than Behavioral).

\begin{figure*}
        \centering
        \begin{subfigure}[b]{0.475\textwidth}
            \centering
            \includegraphics[width=0.8\textwidth]{imgs/results/results_2/histogram_of_behaviorals_x.png}
            \caption[]%
            {{\small Histogram of the number of Behavioral Codelets needed to achieve the lowest score. Most runs needed 13, the same number of Motor Codelets (and actuators).}}    
            \label{fig:hist_beh}
        \end{subfigure}
        \hfill
        \begin{subfigure}[b]{0.475\textwidth}  
            \centering 
            \includegraphics[width=0.8\textwidth]{imgs/results/results_2/histogram_of_perceptuals_x.png}
            \caption[]%
            {{\small Histogram of the number of Perceptual Codelets needed to achieve the lowest score.}}    
            \label{fig:hist_per}
        \end{subfigure}
        \caption[]
        {\small Histogram of the number of ''internal'' Codelets needed to achieve the lowest score on each run.} 
        \label{fig:needed_codelets_2}
\end{figure*}


\begin{figure*}
        \centering
        \begin{subfigure}[b]{0.475\textwidth}
            \centering
            \includegraphics[width=0.8\textwidth]{imgs/results/results_2/correlation_correlation_of_score_with_behaviorals.png}
            \caption[]%
            {{\small Correlation between number of Behavioral Codelets and Score}}    
            \label{fig:corr_behavior}
        \end{subfigure}
        \hfill
        \begin{subfigure}[b]{0.475\textwidth}  
            \centering 
            \includegraphics[width=0.8\textwidth]{imgs/results/results_2/correlation_correlation_of_score_with_perceptuals.png}
            \caption[]%
            {{\small Correlation between number of Perceptual Codelets and Score}}    
            \label{fig:corr_per}
        \end{subfigure}
        \caption[]
        {\small Correlation between the number of ''internal'' Codelets and Score} 
        \label{fig:corr}
\end{figure*}



Table \ref{table:exp2} shows some statistics taken from the experiment. Notice that, while the individual number of Perceptual and Behavioral Codelets may go as low as 2, the combined ''Internal'' Codelets need a higher number to present satisfactory results.

\begin{table}[]
\centering
\caption{metrics on Experiments}
\label{table:exp2}
\begin{tabular}{@{}llllll@{}}
\toprule
              & mean    & median & std   & min & max \\ \midrule
score         & 1.438   & 1.0    & 1.894 & 0   & 26  \\
generations   & 13.6784 & 20.0   & 8.905 & 0   & 20  \\
n perceptuals & 9.5326  & 10.0   & 2.213 & 2   & 15  \\
n behaviorals & 11.2674 & 13.0   & 2.478 & 2   & 13  \\
n internals   & 20.8    & 22.0   & 3.998 & 7   & 28  \\ \bottomrule
\end{tabular}
\end{table}

\subsection{Conclusion and Future Works}

This paper has presented a pioneering approach to creating a Cognitive Twin by leveraging a distributed cognitive system in conjunction with an evolution strategy. Our work stands as a significant contribution to the field of cognitive computing by demonstrating the feasibility of orchestrating a multitude of simple physical and virtual devices to mimic a person's interaction behaviors. This achievement not only offers a practical application of distributed cognitive systems but also introduces a novel methodology for cognitive twin development, emphasizing the role of evolution strategies in optimizing system topology for more accurate behavior emulation.

In revisiting the themes introduced at the outset, our research seamlessly integrates the foundational principles of cognitive systems, Cyber-Physical Systems (CPS), and Systems of Systems (SoS) with contemporary advancements in artificial intelligence. By doing so, we have illustrated a comprehensive framework that not only addresses the complexities of human behavior simulation but also opens new avenues for automation, human-like agent creation, and in-depth behavioral analysis.

Comparatively, our approach distinguishes itself from established cognitive architectures such as ACT-R and SOAR, and the Standard Model of Mind, by emphasizing distributed processing and adaptability. While ACT-R and SOAR offer rich insights into cognitive processes through detailed psychological models, our model excels in harnessing distributed, interconnected devices to capture the multifaceted nature of human cognition. Similarly, the Standard Model of Mind provides a foundational framework for understanding cognitive functions. Yet, our work extends this understanding into the practical domain of CPS and distributed systems, offering a unique perspective on cognitive replication and interaction dynamics.

In conclusion, our research not only underscores the potential of distributed cognitive systems in creating sophisticated cognitive twins but also highlights the importance of evolutionary strategies in refining these systems. By drawing parallels and distinguishing our work from established cognitive architectures like ACT-R, SOAR, and the Standard Model of Mind, we contribute a novel perspective to the ongoing discourse on cognitive modeling and simulation. 



Future work will focus on further refining the distributed cognitive system and exploring its integration with other AI paradigms and models. This research sets the stage for developing more sophisticated Cognitive Twins capable of performing complex tasks with minimal human intervention. By continuing to build on this foundation, future studies can enhance the fidelity and applicability of Cognitive Twins, making them tools in the field of cognitive computing.
    




\vspace{6pt}
\noindent\textbf{Declaration of Competing Interest}
The authors declare that they have no known competing financial
interests or personal relationships that could have appeared to influence
the work reported in this paper


\vspace{6pt}
\noindent\textbf{Acknowledgments:} This study was partially financed by the Coordenação de Aperfeiçoamento de Pessoal de Nível Superior - Brasil (CAPES) - Finance Code 001.

This project is part of the Hub for Artificial Intelligence and Cognitive Architectures (H.IAAC- Hub de Inteligência Artificial e Arquiteturas Cognitivas). We acknowledge the support of PPI-Softex/MCTI by grant 01245.013778/2020-21 through the Brazilian Federal Government



%% If you have bibdatabase file and want bibtex to generate the
%% bibitems, please use
%%
 \bibliographystyle{cas-model2-names} 
 \bibliography{main}



\end{document}

