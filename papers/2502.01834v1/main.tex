%% 
%% Copyright 2019-2021 Elsevier Ltd
%% 
%% This file is part of the 'CAS Bundle'.
%% --------------------------------------
%% 
%% It may be distributed under the conditions of the LaTeX Project Public
%% License, either version 1.2 of this license or (at your option) any
%% later version.  The latest version of this license is in
%%    http://www.latex-project.org/lppl.txt
%% and version 1.2 or later is part of all distributions of LaTeX
%% version 1999/12/01 or later.
%% 
%% The list of all files belonging to the 'CAS Bundle' is
%% given in the file `manifest.txt'.
%% 
%% Template article for cas-dc documentclass for 
%% double column output.

\documentclass[a4paper,fleqn]{cas-dc}

% If the frontmatter runs over more than one page
% use the longmktitle option.

%\documentclass[a4paper,fleqn,longmktitle]{cas-dc}

%\usepackage[numbers]{natbib}
%\usepackage[authoryear]{natbib}
\usepackage[authoryear,longnamesfirst]{natbib}

%%%Author macros
\def\tsc#1{\csdef{#1}{\textsc{\lowercase{#1}}\xspace}}
\tsc{WGM}
\tsc{QE}
%%%

% Uncomment and use as if needed
\newtheorem{theorem}{Theorem}
%\newtheorem{lemma}[theorem]{Lemma}
%\newdefinition{rmk}{Remark}
%\newproof{pf}{Proof}
%\newproof{pot}{Proof of Theorem \ref{thm}}


% isso aqui permite os "definitions"
%\theoremstyle{definition}
\newtheorem{definition}{Definition}[section]

%\usepackage{cite}
\usepackage{amsmath,amssymb,amsfonts,url}
\usepackage{algorithmic}
\usepackage{graphicx}
\usepackage{textcomp}
\usepackage{xcolor}
%\usepackage[left=1in,right=1in,top=0.75in,bottom=1in]{geometry}
\usepackage{subcaption}
%\usepackage[redeflists]{IEEEtran}

\usepackage{booktabs}

%\newtheorem{theorem}{Theorem}

\begin{document}


\let\WriteBookmarks\relax
\def\floatpagepagefraction{1}
\def\textpagefraction{.001}

% Short title
\shorttitle{Building a Cognitive Twin Using a Distributed Cognitive System and an Evolution Strategy}    

% Short author
\shortauthors{Wandemberg Gibaut; Ricardo Gudwin}  

% Main title of the paper
\title [mode = title]{Building a Cognitive Twin Using a Distributed Cognitive System and an Evolution Strategy}  



\author[1]{Wandemberg Gibaut}
%\author[1]{Author1}

% Corresponding author indication
\cormark[1]

% Footnote of the first author
\fnmark[1]

% Email id of the first author
\ead{wgibaut@dca.fee.unicamp.br}

%\credit{<Credit authorship details>}

% Address/affiliation
\affiliation[1]{organization={University of Campinas (Unicamp)},
            %addressline={}, 
            city={Campinas},
          citysep={}, % Uncomment if no comma needed between city and postcode
            %postcode={}, 
            state={São Paulo},
            country={Brazil},
            orcid={ https://orcid.org/0000-0001-7322-5399}}

%\affiliation[1]{organization={org},
%            city={city},
%            state={state},
%            country={country}}%,
            
            
\author[2]{Ricardo Gudwin}
%\author[2]{Author2}

% Footnote of the second author
\fnmark[2]

% Email id of the second author
%\ead{}

% URL of the second author
\ead[url]{https://faculty.dca.fee.unicamp.br/gudwin/}
%\ead[url]{url}

% Credit authorship
%\credit{aaaa}

% Address/affiliation
\affiliation[2]{organization={University of Campinas (Unicamp)},
            city={Campinas},
            state={São Paulo},
            country={Brazil}}

%\affiliation[2]{organization={org},
%            city={city},
%            state={state},
%            country={country}}%,


% Corresponding author text
\cortext[1]{Corresponding author}

% Footnote text
\fntext[1]{}

% For a title note without a number/mark
%\nonumnote{}

% Here goes the abstract
\begin{abstract}
This work presents a technique to build interaction-based Cognitive Twins (a computational version of an external agent) using input-output training and an Evolution Strategy on top of a framework for distributed Cognitive Architectures. Here, we show that it's possible to orchestrate many simple physical and virtual devices to achieve good approximations of a person's interaction behavior by training the system in an end-to-end fashion and present performance metrics. The generated Cognitive Twin may later be used to automate tasks, generate more realistic human-like artificial agents or further investigate its behaviors.

\end{abstract}

% Use if graphical abstract is present
%\begin{graphicalabstract}
%\includegraphics{}
%\end{graphicalabstract}

% Research highlights
%\begin{highlights}
%\item We show a practical application of a distributed cognitive system
%\item A Cognitive Twin is possible by training systems end-to-end
%\item Evolution Strategy plays a role in defining the topology
%\end{highlights}

% Keywords
% Each keyword is seperated by \sep
\begin{keywords}
Cognitive Systems \sep  Artificial Intelligence \sep  Distributed Systems \sep Cognitive Twin \sep Internet of Things
\end{keywords}

\maketitle




%!TEX root = gcn.tex
\section{Introduction}
Graphs, representing structural data and topology, are widely used across various domains, such as social networks and merchandising transactions.
Graph convolutional networks (GCN)~\cite{iclr/KipfW17} have significantly enhanced model training on these interconnected nodes.
However, these graphs often contain sensitive information that should not be leaked to untrusted parties.
For example, companies may analyze sensitive demographic and behavioral data about users for applications ranging from targeted advertising to personalized medicine.
Given the data-centric nature and analytical power of GCN training, addressing these privacy concerns is imperative.

Secure multi-party computation (MPC)~\cite{crypto/ChaumDG87,crypto/ChenC06,eurocrypt/CiampiRSW22} is a critical tool for privacy-preserving machine learning, enabling mutually distrustful parties to collaboratively train models with privacy protection over inputs and (intermediate) computations.
While research advances (\eg,~\cite{ccs/RatheeRKCGRS20,uss/NgC21,sp21/TanKTW,uss/WatsonWP22,icml/Keller022,ccs/ABY318,folkerts2023redsec}) support secure training on convolutional neural networks (CNNs) efficiently, private GCN training with MPC over graphs remains challenging.

Graph convolutional layers in GCNs involve multiplications with a (normalized) adjacency matrix containing $\numedge$ non-zero values in a $\numnode \times \numnode$ matrix for a graph with $\numnode$ nodes and $\numedge$ edges.
The graphs are typically sparse but large.
One could use the standard Beaver-triple-based protocol to securely perform these sparse matrix multiplications by treating graph convolution as ordinary dense matrix multiplication.
However, this approach incurs $O(\numnode^2)$ communication and memory costs due to computations on irrelevant nodes.
%
Integrating existing cryptographic advances, the initial effort of SecGNN~\cite{tsc/WangZJ23,nips/RanXLWQW23} requires heavy communication or computational overhead.
Recently, CoGNN~\cite{ccs/ZouLSLXX24} optimizes the overhead in terms of  horizontal data partitioning, proposing a semi-honest secure framework.
Research for secure GCN over vertical data  remains nascent.

Current MPC studies, for GCN or not, have primarily targeted settings where participants own different data samples, \ie, horizontally partitioned data~\cite{ccs/ZouLSLXX24}.
MPC specialized for scenarios where parties hold different types of features~\cite{tkde/LiuKZPHYOZY24,icml/CastigliaZ0KBP23,nips/Wang0ZLWL23} is rare.
This paper studies $2$-party secure GCN training for these vertical partition cases, where one party holds private graph topology (\eg, edges) while the other owns private node features.
For instance, LinkedIn holds private social relationships between users, while banks own users' private bank statements.
Such real-world graph structures underpin the relevance of our focus.
To our knowledge, no prior work tackles secure GCN training in this context, which is crucial for cross-silo collaboration.


To realize secure GCN over vertically split data, we tailor MPC protocols for sparse graph convolution, which fundamentally involves sparse (adjacency) matrix multiplication.
Recent studies have begun exploring MPC protocols for sparse matrix multiplication (SMM).
ROOM~\cite{ccs/SchoppmannG0P19}, a seminal work on SMM, requires foreknowledge of sparsity types: whether the input matrices are row-sparse or column-sparse.
Unfortunately, GCN typically trains on graphs with arbitrary sparsity, where nodes have varying degrees and no specific sparsity constraints.
Moreover, the adjacency matrix in GCN often contains a self-loop operation represented by adding the identity matrix, which is neither row- nor column-sparse.
Araki~\etal~\cite{ccs/Araki0OPRT21} avoid this limitation in their scalable, secure graph analysis work, yet it does not cover vertical partition.

% and related primitives
To bridge this gap, we propose a secure sparse matrix multiplication protocol, \osmm, achieving \emph{accurate, efficient, and secure GCN training over vertical data} for the first time.

\subsection{New Techniques for Sparse Matrices}
The cost of evaluating a GCN layer is dominated by SMM in the form of $\adjmat\feamat$, where $\adjmat$ is a sparse adjacency matrix of a (directed) graph $\graph$ and $\feamat$ is a dense matrix of node features.
For unrelated nodes, which often constitute a substantial portion, the element-wise products $0\cdot x$ are always zero.
Our efficient MPC design 
avoids unnecessary secure computation over unrelated nodes by focusing on computing non-zero results while concealing the sparse topology.
We achieve this~by:
1) decomposing the sparse matrix $\adjmat$ into a product of matrices (\S\ref{sec::sgc}), including permutation and binary diagonal matrices, that can \emph{faithfully} represent the original graph topology;
2) devising specialized protocols (\S\ref{sec::smm_protocol}) for efficiently multiplying the structured matrices while hiding sparsity topology.


 
\subsubsection{Sparse Matrix Decomposition}
We decompose adjacency matrix $\adjmat$ of $\graph$ into two bipartite graphs: one represented by sparse matrix $\adjout$, linking the out-degree nodes to edges, the other 
by sparse matrix $\adjin$,
linking edges to in-degree nodes.

%\ie, we decompose $\adjmat$ into $\adjout \adjin$, where $\adjout$ and $\adjin$ are sparse matrices representing these connections.
%linking out-degree nodes to edges and edges to in-degree nodes of $\graph$, respectively.

We then permute the columns of $\adjout$ and the rows of $\adjin$ so that the permuted matrices $\adjout'$ and $\adjin'$ have non-zero positions with \emph{monotonically non-decreasing} row and column indices.
A permutation $\sigma$ is used to preserve the edge topology, leading to an initial decomposition of $\adjmat = \adjout'\sigma \adjin'$.
This is further refined into a sequence of \emph{linear transformations}, 
which can be efficiently computed by our MPC protocols for 
\emph{oblivious permutation}
%($\Pi_{\ssp}$) 
and \emph{oblivious selection-multiplication}.
% ($\Pi_\SM$)
\iffalse
Our approach leverages bipartite graph representation and the monotonicity of non-zero positions to decompose a general sparse matrix into linear transformations, enhancing the efficiency of our MPC protocols.
\fi
Our decomposition approach is not limited to GCNs but also general~SMM 
by 
%simply 
treating them 
as adjacency matrices.
%of a graph.
%Since any sparse matrix can be viewed 

%allowing the same technique to be applied.

 
\subsubsection{New Protocols for Linear Transformations}
\emph{Oblivious permutation} (OP) is a two-party protocol taking a private permutation $\sigma$ and a private vector $\xvec$ from the two parties, respectively, and generating a secret share $\l\sigma \xvec\r$ between them.
Our OP protocol employs correlated randomnesses generated in an input-independent offline phase to mask $\sigma$ and $\xvec$ for secure computations on intermediate results, requiring only $1$ round in the online phase (\cf, $\ge 2$ in previous works~\cite{ccs/AsharovHIKNPTT22, ccs/Araki0OPRT21}).

Another crucial two-party protocol in our work is \emph{oblivious selection-multiplication} (OSM).
It takes a private bit~$s$ from a party and secret share $\l x\r$ of an arithmetic number~$x$ owned by the two parties as input and generates secret share $\l sx\r$.
%between them.
%Like our OP protocol, o
Our $1$-round OSM protocol also uses pre-computed randomnesses to mask $s$ and $x$.
%for secure computations.
Compared to the Beaver-triple-based~\cite{crypto/Beaver91a} and oblivious-transfer (OT)-based approaches~\cite{pkc/Tzeng02}, our protocol saves ${\sim}50\%$ of online communication while having the same offline communication and round complexities.

By decomposing the sparse matrix into linear transformations and applying our specialized protocols, our \osmm protocol
%($\prosmm$) 
reduces the complexity of evaluating $\numnode \times \numnode$ sparse matrices with $\numedge$ non-zero values from $O(\numnode^2)$ to $O(\numedge)$.

%(\S\ref{sec::secgcn})
\subsection{\cgnn: Secure GCN made Efficient}
Supported by our new sparsity techniques, we build \cgnn, 
a two-party computation (2PC) framework for GCN inference and training over vertical
%ly split
data.
Our contributions include:

1) We are the first to explore sparsity over vertically split, secret-shared data in MPC, enabling decompositions of sparse matrices with arbitrary sparsity and isolating computations that can be performed in plaintext without sacrificing privacy.

2) We propose two efficient $2$PC primitives for OP and OSM, both optimally single-round.
Combined with our sparse matrix decomposition approach, our \osmm protocol ($\prosmm$) achieves constant-round communication costs of $O(\numedge)$, reducing memory requirements and avoiding out-of-memory errors for large matrices.
In practice, it saves $99\%+$ communication
%(Table~\ref{table:comm_smm}) 
and reduces ${\sim}72\%$ memory usage over large $(5000\times5000)$ matrices compared with using Beaver triples.
%(Table~\ref{table:mem_smm_sparse}) ${\sim}16\%$-

3) We build an end-to-end secure GCN framework for inference and training over vertically split data, maintaining accuracy on par with plaintext computations.
We will open-source our evaluation code for research and deployment.

To evaluate the performance of $\cgnn$, we conducted extensive experiments over three standard graph datasets (Cora~\cite{aim/SenNBGGE08}, Citeseer~\cite{dl/GilesBL98}, and Pubmed~\cite{ijcnlp/DernoncourtL17}),
reporting communication, memory usage, accuracy, and running time under varying network conditions, along with an ablation study with or without \osmm.
Below, we highlight our key achievements.

\textit{Communication (\S\ref{sec::comm_compare_gcn}).}
$\cgnn$ saves communication by $50$-$80\%$.
(\cf,~CoGNN~\cite{ccs/KotiKPG24}, OblivGNN~\cite{uss/XuL0AYY24}).

\textit{Memory usage (\S\ref{sec::smmmemory}).}
\cgnn alleviates out-of-memory problems of using %the standard 
Beaver-triples~\cite{crypto/Beaver91a} for large datasets.

\textit{Accuracy (\S\ref{sec::acc_compare_gcn}).}
$\cgnn$ achieves inference and training accuracy comparable to plaintext counterparts.
%training accuracy $\{76\%$, $65.1\%$, $75.2\%\}$ comparable to $\{75.7\%$, $65.4\%$, $74.5\%\}$ in plaintext.

{\textit{Computational efficiency (\S\ref{sec::time_net}).}} 
%If the network is worse in bandwidth and better in latency, $\cgnn$ shows more benefits.
$\cgnn$ is faster by $6$-$45\%$ in inference and $28$-$95\%$ in training across various networks and excels in narrow-bandwidth and low-latency~ones.

{\textit{Impact of \osmm (\S\ref{sec:ablation}).}}
Our \osmm protocol shows a $10$-$42\times$ speed-up for $5000\times 5000$ matrices and saves $10$-2$1\%$ memory for ``small'' datasets and up to $90\%$+ for larger ones.

\section{The DCT}
\label{sec:sec2}

In this section, we will briefly explain our main tool. The DCT \citep{gibaut2020extending}, an acronym for Distributed Cognitive Toolkit, is a bare-bones toolkit to help the development of cognitive systems in a distributed, language-agnostic fashion. A cognitive agent created with DCT should be able to run across multiple physical (desktops, small computers like Raspberries, or microcontrollers like Arduino) or virtual devices (like Docker containers). It is a re-implementation of the ideas first seen in the CST main article \citep{Paraense201632}, as some features like being inherently single-device and being written in Java may be a shortcoming, sometimes. As expected, it also follows some theory lines like being Codelet-oriented, present in the Copycat architecture \citep{hofstadter1994copycat}. 

To better understand the DCT, one should first refer to the Cognitive Systems Toolkit (CST), the toolkit that came before. As its name suggests, CST is a toolkit for the development of cognitive architectures. Its purpose is to facilitate the creation of such systems the way the user wants, as long it respects its premises. In the core of CST are two basic entities that serve as building blocks: Codelets and Memories. A Codelet is a non-blocking, parallel process that runs continuously and represents a very specific piece of the cognition process of biological creatures. Likewise, a memory is a storage structure from which Codelets read and write information. A user may create any architecture that also follows a Codelet-oriented specification, based on already existing theories or new ones.

Many Cognitive Architectures like e.g. MECA \citep{gudwin2017multipurpose}, LIDA \citep{franklin2014lida}, and others may be seen as multi-agent systems. Following this paradigm, DCT conceives a distributed Cognitive System as a multi-agent system, where a standard protocol is used for the communication among the agents. Also, similar to its predecessor (CST), each agent in such a multi-agent system is built using Codelets and Memories.  The structure and functionality of a multi-agent system are fully compatible with distributed computing concepts and may be comfortably mapped to the Internet of Things: Sensory Codelets may be simply real-world sensor devices and Motor Codelets may be simply relays or actuators, while more complex Codelets may be embedded in microcontrollers or even software containers in the cloud. 


Our first prototype of DCT - and current software based on it - is written in \emph{Shell script} and \emph{Python}, and could be deployed in containers (like Docker) or across different devices, including Raspberries and Arduinos. Figure \ref{fig:multi_device} illustrates the idea: if the input/output conditions are satisfied, there is no need to execute the whole system on a single computer. The project repository can be accessed in: \href{https://github.com/wandgibaut/dct}{https://github.com/wandgibaut/dct}.


\begin{figure*}[hbt]
\centering
	\includegraphics[width=2.0\columnwidth]{imgs/Distributed_MECA_2.png}
\caption{Illustration of a multi-device, Codelet-oriented system as seen in \cite{gibaut2020extending}. Notice that, depending on how powerful is the device, it may run a single Codelet or multiple ones.}
\label{fig:multi_device}
\end{figure*}


\subsection{The DCT Architectural Overview}
\label{sec:dct_over}

From an architectural perspective, the DCT is composed of a set of \emph{Nodes} communicating to each other and integrating, as a whole, a functional system with cognitive capabilities. Theoretically, different subsets of the same collection of Nodes could even act as different systems. Here, the term \emph{Node} represents an entity (logical or physical), which works as a storage for groups of Codelets and/or Memories and is responsible for their operation and life cycles. Figure \ref{fig:multi_device} illustrates the idea of this non-homogeneity in device configurations. The subsection \ref{sec:dct_node} shows further details on this entity.

To communicate with each other, \emph{Nodes} follow a protocol regulating the interaction among them. Following CST specifications, Codelets only interact with Memories that is, a Codelet represents a block of computing unity, applying some process on data, but not holding it. This data storage is performed by Memories, which can be of different technologies. For this communication, DCT uses, canonically, \emph{json} formatted messages. This allows the use of a good range of technologies and simple sockets and databases like \emph{MongoDB} and \emph{Redis} are already supported by existing code. By following these directives, a user may use any language or technology suitable to a device in which the \emph{Node} is.


Formally, we can conceive a Cognitive System created with DCT in the following way:


\begin{definition}[A DCT Cognitive System]

Let $N$ be a set of \emph{Nodes}, where a \emph{Node} is an entity (logical or physical) that encapsulates one or more \emph{Codelets} and/or \emph{Memories} meant to be run under the supervision of a single \emph{Node Master} within an operational system.

To each \emph{Node}, there is an \emph{Interface} $I = \{MO, S\}$, where $MO$ is a subset of the \emph{Memories} implemented within a \emph{Node}, which will be accessed from other external \emph{Nodes} and $S$ is a Server that listens to a URI. This server $S$ should listen for requests and respond with \emph{json} formatted messages.


A Cognitive System created with DCT is defined by the interaction between the elements of $N$ following some Codelet-oriented Cognitive Theory, like MECA or LIDA.
\end{definition}






\subsection{The DCT Codelet structure}
\label{sec:dct_Codelet}


A DCT \emph{Codelet} is composed of a callable program file in a user-specified language that follows some guidelines and some configuration files that can be used to dynamically change some properties, like which \emph{Memories} it can access. Figure \ref{fig:Codelet} illustrates this structure. It is valid to note that, since it was first implemented, some improvements have been made in how a \emph{Codelet} works. The files that characterize a \emph{Codelet} are:


\begin{itemize}
\item A \verb!Codelet! compiled program or script, which runs until be ordered (by the \emph{Node Master}) to stop
\item The \emph{Codelet} configuration file (\emph{fields.json}). This file contains some information regarding the \emph{Codelet} behavior, like its inputs, and should be possible to dynamically change it.
\end{itemize}

Also, if needed for a problem-specific reason, additional files may be used (a \emph{.ini} file, for example). In this work, \emph{Codelet} is implemented in \emph{Python} language.


%%arrumar figura
\begin{figure}[bt]
\centering
	\includegraphics[width=1.0\columnwidth]{imgs/DCT_Codelet_3.png}
\caption{The concept of a DCT Codelet.}
\label{fig:Codelet}
\end{figure}

Following the original CST implementation, the \emph{Codelet} program should have two main functions: a  \verb!calculateActivation! and a \verb!proc!. The first one is used to calculate the current relevance of the \emph{Codelet} itself and may be used in different ways, but mainly with a threshold value to decide if it should execute its main function or skip it. The second one performs the core functionality and is the most important function to be defined by the user. This function represents the procedural code that the entity will periodically call at each time step. Also, it should be non-blocking, meaning that it should be possible to run other processes alongside a \emph{Codelet}.

The configuration file (\emph{fields.json}) contains a structure that is analog to the \emph{Codelet} class in CST, defining many important parameters, e.g. the input, output, and broadcast ports, from where the Codelet can communicate to its Memory objects. 


\subsection{The DCT Memory structure and default support}
\label{sec:dct_memory}

The other core structure, \emph{Memory}, is a generic term for the data structure that holds the information that \emph{Codelets} consume and/or process. Also, it contains some other meta-information, e.g. its name, URL, type, and an evaluation. As said before, this information is standardized as a \emph{json} structure. 

A \emph{Memory} should contain the following information:
 \begin{itemize}
     \item \emph{name}: String 
     \item \emph{IP/port}: String
     \item \emph{type}: String
     \item \emph{I}: String
     \item \emph{eval}: Double 
 \end{itemize}

\subsection{DCT Node}
\label{sec:dct_node}

In the DCT, a \emph{Node} is an abstraction for a physical or virtual device that contains an arbitrary number of \emph{Codelets} and/or \emph{Memories} and is supervised by a single \emph{Node Master}. This definition allows us to consider a computer to be a single \emph{Node} if all relevant entities run in the same environment, or to \textit{have} multiple \emph{Nodes} if each of them runs in a separated container with its \emph{Node Master}.
This \emph{Node Master} is responsible for starting, killing, adding, and removing Codelets and/or Memories, which are running through its supervision. Also, it should periodically check the health of its system, re-executing dead processes, and listen for external requests, like information requests or even requests to shut itself down.

Besides \emph{Codelets} and \emph{Memories}, a \emph{Node} should also implement an \emph{Interface} in which its internal entities may communicate with outside sources, e.g. a server with open sockets.


\section{An Evolutionary Cognitive Twin}
\label{sec:sec3}

Since we have already introduced some key concepts and the tool we're using, we can now discuss the proposed technique. Here, the main point is to build a Cognitive Twin using a vast amount of simple devices, orchestrated to work together as a single system, even if each device is a system of its own and may, theoretically, be a part of another system.

Here in this work, based on the definitions in section \ref{sec:sec1} and within the scope of what will be presented, the following definition will be presented:

\noindent\fbox{%
    \parbox{\columnwidth}{%
A \textit{Cognitive Twin} is a digital replica of the dynamics and cognitive - or just cognitive - processes of an intelligent physical system, usually aimed at a partial representation of a person. These cognitive processes refer to those identified in cognitive theories, such as perception, memory, behavior, adaptation, planning, learning, \textit{Reasoning} etc. The classification of an agent as \textit{Cognitive Twin} refers not only to the duplication of observable behavior of the virtual agent concerning the original but also to the possibility of in-depth investigation of the original individual through its copy.
    }%
}
\vspace{7mm}

The most fundamental idea here is the Codelet, already discussed in section \ref{sec:sec2}. We argue that sensors and actuators can be seen as Sensory and Motor Codelets, respectively. To make that consideration, we considered that both sensors and actuators are simple devices that do a very specific task, following the idea of Codelet, as seen in sections \ref{sec:sec1} and \ref{sec:sec2}. This consideration allows us to model our desired agent as a composition of simpler elements that interact with each other as needed.

Following that perspective, we postulate that the connections between sensors and actuators are given by some combination of elements that group and give them some sense, and elements that use that information to control actuators. This lets us use the concepts of Perceptual Codelets and Behavioral Codelets, respectively, largely used in other works that follow a Codelet-oriented Cognitive agency paradigm. Also, these premises fit in the concepts presented in section \ref{sec:sec1}: The System is composed of devices that are themselves systems connected through a network with an orchestration to build a Cognitive Agent that bridges virtual and physical domains.

But we have two main constraints: first, we sought to use simple, low computing power devices, and we do not have an infinite number of devices (with an infinite variety of input-output responses) to search for the best combination. We approach those constraints by having devices that can learn (or somehow adapt) and by making use of a heuristic that improves this search. It's also impractical to have those devices fully connected, as this communication overhead may degrade performance and/or be impossible if we deal with physical devices.

So, we propose to find the optimal configuration by having both explicit training on Codelets and an Evolution Strategy to find a suitable connection between Perceptual and Behavioral Codelets and between Behavioral and Motor Codelets. We'll discuss this in detail in the next subsections.


%propor o seguinte: a conexão eh encontrada com genetica e os nos perceptuais e comportamentais são decisiontrees

\subsection{Devices internal structure}
 First, we to define the internal structure of the devices we worked on. For the sake of simplicity, we used only virtual devices, as defined in section \ref{sec:sec2}, running Python Codelets on Docker. That allowed us to better manipulate some structures, like sending or requesting data from the master program to/from each Codelet and creating or destroying those virtual devices as we needed them. Even so, we kept the internal structures simple to draw a parallel with low-power devices. 
 
 As we mentioned before, we follow MECA Theory, which uses both \cite{Osman2004} theory of two, separated Cognition Systems working together and a \emph{Codelet}-oriented structure. Here we present a System 1 approach, which means that we'll be working only with four types of \emph{Codelets}: \emph{Sensory, Perceptual, Behavioral} and \emph{Motor Codelets}. We will briefly explain each of them.
 
 
 \subsubsection{Sensory Codelets}
 The simplest of them all, Sensory Codelets represent actual sensors, either physical or virtual. Like sensors, they are responsible for introducing the raw data into the system. For example, if we consider a human eye a sensor, the raw data is the light that enters the pupils. Here, it requests information about a specific attribute - temperature or luminosity, for example - of an environment. In this work, Sensory Codelets' internal structures will not be changed, simulating very simplistic devices, like a digital thermometer.
 
 \subsubsection{Perceptual Codelets}
 \label{sub:per}
 The subsequent structures in the information flow are Perceptual Codelets. These structures are responsible for aggregating the raw data that comes from Sensory Codelets in structures called Perceptions. In our human eye example, outputs of Perceptual Codelets would be \emph{depth}, \emph{objects}, relational properties (like distance from something), and so on. Notice that Perceptual Codelets, in a sense, represent how an agent experiences the world, as the information it could extract from data is heavily dependent on them. In this work, the internal structure of a Perceptual Codelet is represented by a Decision Tree Classifier, where the inputs are sensory data and the output is an integer value that represents a unique identifier (a token) of the input sensor's readings. Each Perceptual Codelet differs from another by the combination of its Sensor Codelets on the input. Those inputs are defined randomly with a uniform distribution to both quantity (a value between half and all the sensors) and which ones are picked up. %Figure \ref{fig:perceptualDT} illustrate the idea. 
  
  
 \subsubsection{Behavioral Codelets}
 Next, we have the Behavioral Codelets. The main purpose of these structures is to, based on previously structured information, activate one or more protocols to control what the agent should do, that is, to control one or more Motor Codelets. This Activation is usually encoded in a 0 to 1 \emph{float} value representing a Signal Strength, a way to measure how important that Behavior is to the current situation. Note that this so-called protocol may be anything from a simple heuristic to a whole Machine Learning method and the input of the Behavioral Codelet may include not only Perception but other information like those coming from a Motivational or even Emotional subsystem. Also, as a single Motor Codelet may have multiple Behavioral Codelets as input, those behaviors effectively compete to prevail and have their commands accepted.
   
   In this work, the Behavioral Codelets have also a Decision Tree Classifier as a method to decide what to send to its Motor Codelets based on Perception. 
   
 \subsubsection{Motor Codelets}
 The last basic structure here is the Motor Codelet. As mentioned before, this represents a direct parallel with an actuator, being physical or not. It simply responds to what was put as input and, through another Decision Tree, it sends a command to the corresponding entity in our virtual environment. This could be a direct association (''if this then that'') but, to make further usage of the code easier, we used a method that could accept more than one Behavioral without having to rewrite it completely.
 
 
 Figure \ref{fig:agent_struct} shows an overview of the structure of the topology of an agent. This concludes our overview of the main structures of our work. Next, we will detail about the Evolution Strategy.


  \begin{figure}[ht]
\centering
	\includegraphics[width=1.0\columnwidth]{imgs/cog_twin.drawio.png}
\caption{Graphical representation of an agent structure and its internal connections}
\label{fig:agent_struct}
\end{figure}

\subsection{The optimization}

The general process of building the architecture for our Cognitive Twin involves determining the connection topology between the different types of Nodes and adjusting the internal functions of each Codelet to reproduce the overall behavior of a primordial agent. This is a two-step offline optimization process, the first being an optimization of the connection topology between Nodes and the second being a conventional training process of the Codelets' internal Machine Learning models in each Node. The connections between the Nodes will be defined through an evolutionary strategy and, given the connection configurations of each individual of a given generation, supervised training methods will be used to minimize the error between the expected and obtained outputs.
In this work, the internal structures of the Sensory Codelets are not changed, simulating very simplistic devices, such as a digital thermometer. 

%We present details of each of these steps below.

\subsubsection{Evolution Strategy details}

In this part of the process, we want to, through an evolution process, define the best configuration of the connections between Perceptual and Behavioral Codelets and Behavioral and Motor Codelets. The connections between Sensorial and Perceptual Codelets are fixed and explained in subsection \ref{sub:per}.

To apply an Evolution Strategy, we need to do some definitions. First, we need to define our Individual encoding. Here, our Individual is a binary vector with the length of the number of total Perceptual Codelets plus the number of total Behavioral Codelets, where each index represents a specific Codelet. In that definition, a '1' represents that the corresponding Codelet is active on the Agent composition and a '0' means a non-connected Codelet. An example Individual is shown in figure \ref{fig:individual}.

\begin{figure}[hb]
\centering
	\includegraphics[width=1.0\columnwidth]{imgs/individual.drawio.png}
\caption{Example Individual for our Evolution Strategy Process. The Individual is encoded as an array of binary values, each one representing if a certain Perceptual or Behavioral Codelet is to be considered as part of the agent.}
\label{fig:individual}
\end{figure}

Second, we need to define a mutation method. In this work, we adopted a simple 'bit-flip' probability for mutation, meaning that each individual has a probability \emph{mut\_p} to be mutated and each of its genes has a probability \emph{ind\_m} to change its state from '0' to '1' and vice versa. So, if a '0' becomes a '1', that means we should take the corresponding Codelet into account when mounting the agent topology.

As we adopted the output of Motor Codelets as exclusively binary, we can choose the fitness evaluation method as a Hamming Distance between the expected outputs and the actual ones. With this choice, the lower the Score, the better the individual fitness.

As a selection method, we choose to keep the best five individuals in the population for the next generation. Also, we choose the overall best one to be cloned. This procedure gives the possibility of recovering from local minima.

%initial pop


Also, we choose not to have any mating process. This choice was completely arbitrary, as we foresaw that it would not cause significant changes and required an additional process that could make each iteration longer.


\subsubsection{The training process}

To build our distributed agent correctly, we need a training process to ensure it accurately maps the system's inputs to the expected outputs. This training is done a) by changing its topology, choosing which Perceptual and Behavioral Codelets are composing the agent, and b) by fitting the data through all individual components consistently. 

The main component of the optimization process is very straightforward: it is a simple - yet efficient - Evolution Strategy to define the agent topology. This process is graphically represented in figure \ref{fig:evol_strategy}. But our \emph{evaluation} method requires more attention. It is in this part that we try to perform an input-output mapping.


\begin{figure}[ht]
\centering
	\includegraphics[width=0.7\columnwidth]{imgs/evolution_strategy.drawio.png}
\caption{Evolution Strategy process diagram. Here we have a high-level representation of each step of the mentioned heuristic.}
\label{fig:evol_strategy}
\end{figure}

% population creation   evaluation    selection   cloning and mutation    do it again

First, we get our \emph{Individual} and reconfigure the Codelets connections properly, including cleaning all \emph{memories} and deciding which \emph{Behavioral Codelet} will feed each \emph{Motor Codelet}. We do that by writing a new \emph{fields.json} file and sending it to each \emph{Codelet}, and by forcing an empty value to the relevant \emph{Memories}, such as the \emph{motor-memories}, in each \emph{Motor Codelet}.

In the second part, we get our training inputs and send them to each relevant (the ones with a correspondent '1' in \emph{Individual} encoding) \emph{Perceptual Codelet} and send them a "train" signal, through a special \emph{Memory} on them with this sole purpose. This training process aims to create unique values - like tokens - that identify each observed combination of sensor readings. These readings refer solely to the sensors at the input of each \emph{Perceptual Codelet}. These tokens are integers corresponding to positions on an array with unique observations. For example, if ''[[0, 1, 0], [0, 1, 1], [0, 1, 0]]'' represents a set of training inputs, then ''[[0], [1], [0]]'' would be the outputs. Remember that each \emph{Perceptual Codelet} has its own sensor connections, so their responses differ one from another.


Now, the most sensitive part, we need to train \emph{Behavioral Codelets} properly considering the input-output response of each \emph{Perceptual Codelet} and respective \emph{Motor Codelet}. We do that in two steps: aggregating \emph{Perceptual Codelets} responses for each training input and mapping those responses to a \emph{Motor Codelet} input that would generate the desired output. These two sets (\emph{Perceptual} responses and \emph{Motor} input) represent the training we have on each \emph{Behavioral Codelet}.

The aggregation step is done by requesting the already trained model from \emph{Perceptual Codelets} and its respective input masks (representing which sensors feed them) and mounting a conjoined Perceptual output. This approach may also be useful for the data-sensitive task, as the system that collects/sends the information need not be the same that centralizes the Evolution Strategy.

Then, we need to get how each \emph{Motor Codelet} responds, either by getting a trained model or by getting an input-output set directly. We opt for the latter since \emph{Motor Codelets} represent actuators and, usually, they don't hold sensitive data. This part is problematic, as a \emph{Behavioral Codelet} may be assigned to two (or more) \emph{Motor Codelets} with fundamentally different behaviors. That could make it impossible to achieve correct model training. After collecting data, we send each input-output information to the respective \emph{Behavioral Codelet}.


Next, we are ready to evaluate the system's performance. One by one, we send the corresponding entry in the input test set to the environment server the system is sensing (we talk about this specificity in section \ref{sec:sec4}) and wait until the information propagates through a distributed agent, getting the system response after that (all \emph{Motor Codelets} outputs combined). This ''wait time'' is directly related to communication overheads and process concurrency if using a single computer.

Finally, we calculate the Hamming Distance between the expected values (test output) and the actual output. This will represent the \emph{Fitness} of the \emph{Individual}.




\section{Experiments}

\subsection{Experimental Setup}

\textbf{Datasets.} We use three categories from the Amazon Reviews dataset~\cite{mcauley2015image} for our experiments: ``Sports and Outdoors'' (\textbf{Sports}), ``Beauty'' (\textbf{Beauty}), and ``CDs and Vinyl'' (\textbf{CDs}). Each user’s historical reviews are considered ``actions'' and are sorted chronologically as action sequences, with earlier reviews appearing first. To evaluate the models, we adopt the widely used leave-last-out protocol~\cite{kang2018sasrec,zhao2022revisiting,rajput2023tiger}, where the last item and second-to-last item in each action sequence are used for testing and validation, respectively. More details about the datasets can be found in~\Cref{app:datasets}.

\textbf{Compared methods.} We compare the performance of ActionPiece with the following methods: (1)~ID-based sequential recommendation methods, including BERT4Rec~\cite{sun2019bert4rec}, and SASRec~\cite{kang2018sasrec}; (2)~feature-enhanced sequential recommendation methods, such as FDSA~\cite{zhang2019fdsa}, S$^3$-Rec~\cite{zhou2020s3}, and VQ-Rec~\cite{hou2023vqrec}; and (3)~generative recommendation methods, including P5-CID~\cite{hua2023p5cid}, TIGER~\cite{rajput2023tiger}, LMIndexer~\cite{jin2024lmindexer}, HSTU~\cite{zhai2024hstu}, and SPM-SID~\cite{singh2024spmsid}, each representing a different action tokenization method (\Cref{tab:act_tokenization}). A detailed description of these baselines is provided in~\Cref{appendix:baselines}.

\textbf{Evaluation settings.} Following~\citet{rajput2023tiger}, we use Recall@$K$ and NDCG@$K$ as metrics to evaluate the methods, where $K \in \{5, 10\}$. Model checkpoints with the best performance on the validation set are used for evaluation on the test set. We run the experiments with five random seeds and report the average metrics.

\textbf{Implementation details.} Please refer to~\Cref{appendix:implementation} for detailed implementation and hyperparameter settings.


\subsection{Overall Performance}

We compare ActionPiece with sequential recommendation and generative recommendation baselines, which use various action tokenization methods, across three public datasets. The results are shown in~\Cref{tab:performance}. 

For the compared methods, we observe that those using item features generally outperform item ID-only methods. This indicates that incorporating features enhances recommendation performance. Among the methods leveraging item features (``Feature + ID'' and ``Generative''), generative recommendation models achieve better performance. These results further confirm that injecting semantics into item indexing and optimizing at a sub-item level enables generative models to better use semantic information and improve recommendation performance. Among all the baselines, SPM-SID achieves the best results. By incorporating the SentencePiece model~\cite{kudo2018sentencepiece}, SPM-SID replaces popular semantic ID patterns within each item with new tokens, benefiting from a larger vocabulary.

\begin{table}[t!]
    \small
    \centering
	\caption{Ablation analysis of ActionPiece. The recommendation performance is measured using NDCG@$10$. The best performance is denoted in \textbf{bold} fonts.}
	\label{tab:ablation}
	\vskip 0.1in
% 	\setlength{\tabcolsep}{1mm}{
% \resizebox{2.1\columnwidth}{!}{
    \begin{tabular}{lccc}
	\toprule
	\multicolumn{1}{c}{\textbf{Variants}} & \textbf{Sports} & \textbf{Beauty} & \textbf{CDs} \\
	\midrule
	\midrule
    \multicolumn{4}{@{}c}{\textit{TIGER with larger vocabularies}} \\
    \midrule
    (1.1) TIGER\ -\ 1k ($4 \times 2^8$) & 0.0225 & 0.0384 & 0.0411 \\
    (1.2) TIGER-49k ($6 \times 2^{13}$) & 0.0162 & 0.0317 & 0.0338 \\
    (1.3) TIGER-66k ($4 \times 2^{14}$) & 0.0194 & N/A$^\dag$ & 0.0319 \\
    \midrule
    \multicolumn{4}{@{}c}{\textit{Vocabulary construction}} \\
    \midrule
    (2.1) \emph{w/o} tokenization & 0.0215 & 0.0389 & 0.0346 \\
    (2.2) \emph{w/o} context-aware & 0.0258 & 0.0416 & 0.0429 \\
    (2.3) \emph{w/o} weighted counting & 0.0257 & 0.0412 & 0.0435 \\
    \midrule
    \multicolumn{4}{@{}c}{\textit{Set permutation regularization}} \\
    \midrule
    (3.1) only for inference & 0.0192 & 0.0316 & 0.0329 \\
    (3.2) only for training & 0.0244 & 0.0387 & 0.0422 \\
    \midrule
    ActionPiece (40k) & \textbf{0.0264} & \textbf{0.0424} & \textbf{0.0451} \\
    \bottomrule
	\end{tabular}
	\vspace{0.05cm}
	\begin{flushleft}
        $^\dag$ not applicable as $2^{14}$ is larger than \#items in Beauty.
    \end{flushleft}
% 	}}
    \vskip -0.2in
\end{table}

\begin{figure*}[t!]
    \begin{center}
    \includegraphics[width=\linewidth]{fig/vocab_size.pdf}
    \vskip -0.1in
    \caption{Analysis of recommendation performance (NDCG@10, $\uparrow$) and average tokenized sequence length (NSL, $\downarrow$) \wrt vocabulary size across three datasets.
    % NSL refers to the normalized sequence length, calculated relative to the number of initial tokens.
    ``N/A’’ indicates that ActionPiece is not applied, \ie action sequences are represented solely by initial tokens.}
    \label{fig:vocab_size}
    \end{center}
    \vskip -0.2in
\end{figure*}

Our proposed ActionPiece consistently outperforms all baselines across three datasets, achieving a significant improvement in NDCG@$10$. It surpasses the best-performing baseline method by $6.00\%$ to $12.82\%$. Unlike existing methods, ActionPiece is the first context-aware action sequence tokenizer, \ie the same action can be tokenized into different tokens depending on its surrounding context. This allows ActionPiece to capture important sequence-level feature patterns that enhance recommendation performance.

% \begin{figure}[t]
% % \vskip 0.2in
% \begin{center}
% \centerline{\includegraphics[width=0.85\columnwidth]{fig/ndcg_vs_vocab_size.pdf}}
% \end{center}
% % \vskip -0.3in
% \vspace{-0.3in}
% \caption{Comparison of performance and vocabulary size (\#token for TIGER, SPM-SID, and ActionPiece; \#item for SASRec; and \#item+\#attribute for S$^3$-Rec) on ``Sports'' dataset.
% % By adjusting the vocabulary size, ActionPiece is shown to balance memory efficiency and recommendation performance.
% }
% \label{fig:intro}
% % \vskip -0.2in
% \vspace{-0.1in}
% \end{figure}


\subsection{Ablation Study}\label{sec:ablation}

We conduct ablation analyses in~\Cref{tab:ablation} to study how each proposed technique contributes to ActionPiece.\\
\hspace*{3mm} (1)~We increase the vocabulary size of TIGER, to determine whether the performance gain of ActionPiece is solely due to scaling up the number of tokens in the vocabulary. By increasing the number of semantic ID digits per item~($4 \rightarrow 6$) and the number of candidate semantic IDs per digit~($2^8 \rightarrow 2^{13}\ \text{or}\ 2^{14}$), we create two variants with vocabularies larger than ActionPiece. However, these TIGER variants perform worse than ActionPiece, and even the original TIGER with only $1024$ tokens. The experimental results suggest that scaling up the vocabulary size for generative recommendation models is challenging, consistent with the observations from~\citet{zhang2024moc}.\\
\hspace*{3mm} (2)~To evaluate the effectiveness of the proposed vocabulary construction techniques, we introduce the following variants: \emph{(2.1)~w/o tokenization}, which skips vocabulary construction, using item features directly as tokens; \emph{(2.2)~w/o context-aware}, which only considers co-occurrences and merges tokens within each action during vocabulary construction and segmentation; and \emph{(2.3)~w/o weighted counting}, which treats all token pairs equally rather than using the weights defined in~\Cref{eq:p_one_set,eq:p_two_sets}. The results indicate that removing any of these techniques reduces performance, demonstrating the importance of these methods for building a context-aware tokenizer.\\
\hspace*{3mm} (3)~To evaluate the effectiveness of SPR, we revert to naive segmentation, as described in~\Cref{subsubsec:segmentation}, during model training and inference, respectively. The results show that replacing SPR with naive segmentation in either training or inference degrades performance.

\begin{figure}[t!]
    \begin{center}
    \includegraphics[width=0.95\columnwidth]{fig/token_util.pdf}
    \vskip -0.1in
    \caption{Analysis of token utilization rate (\%) during model training \wrt segmentation strategy.
    % ``SPR'' denotes set permutation regularization.
    }
    \label{fig:token_util}
    \end{center}
    % \vskip -0.3in
    \vskip -0.3in
\end{figure}

\subsection{Further Analysis}

% In this section, we analyze the impact of key hyperparameters in vocabulary construction and segmentation.

\subsubsection{Performance and Efficiency \wrt Vocabulary Size}

Vocabulary size is a key hyperparameter for language tokenizers~\cite{meta2024llama3,dagan2024getting}. In this study, we investigate how adjusting vocabulary size affects the generative recommendation models. We use the normalized sequence length (NSL)~\cite{dagan2024getting} to measure the length of tokenized sequences, where a smaller NSL indicates fewer tokens per tokenized sequence. We experiment with vocabulary sizes in \{N/A, 5k, 10k, 20k, 30k, 40k\}, where ``N/A'' represents the direct use of item features as tokens. As shown in~\Cref{fig:vocab_size}, increasing the vocabulary size improves recommendation performance and reduces the tokenized sequence length. Conversely, reducing the vocabulary size lowers the number of model parameters, improving memory efficiency. This analysis demonstrates that adjusting vocabulary size enables a trade-off between model performance, sequence length, and memory efficiency.

\subsubsection{Token Utilization Rate \wrt Segmentation Strategy}\label{sec:token_utilization}

As described in~\Cref{subsubsec:training}, applying SPR augments the training corpus by producing multiple token sequences that share the same semantics. In~\Cref{tab:ablation}, we observe that incorporating SPR significantly improves recommendation performance. One possible reason is that SPR increases token utilization rates. To validate this assumption, we segment the action sequences in each training epoch using two strategies: naive segmentation and SPR. As shown in~\Cref{fig:token_util}, naive segmentation uses only $56.89\%$ of tokens for model training, limiting the model's ability to generalize to unseen action sequences. In contrast, SPR achieves a token utilization rate of $87.01\%$ after the first training epoch, with further increases as training progresses. These results demonstrate that the proposed SPR segmentation strategy improves the utilization of ActionPiece tokens, enabling better generalization and enhanced performance.


\subsubsection{Performance \wrt Inference-Time Ensembles}\label{sec:inference_time_ensemble}

As described in~\Cref{subsubsec:inference}, ActionPiece supports inference-time ensembling by using SPR segmentation. We vary the number of ensembled segments, $q$, in \{N/A, 1, 3, 5, 7\}, where ``N/A'' indicates using naive segmentation during model inference. As shown in~\Cref{fig:ensemble}, ensembling more tokenized sequences improves ActionPiece's recommendation performance. However, the performance gains slow down as $q$ increases to $5$ and $7$. Since a higher $q$ also increases the computational cost of inference, this creates a trade-off between performance and computational budget in practice.

\begin{figure}[t!]
    \begin{center}
    \includegraphics[width=\columnwidth]{fig/ensemble.pdf}
    \vskip -0.15in
    \caption{Analysis of performance (NDCG@10, $\uparrow$) \wrt the number of ensembled segments $q$ during model inference.}
    \label{fig:ensemble}
    \end{center}
    \vskip -0.25in
\end{figure}

\subsection{Case Study}\label{subsec:case}

To understand how GR models benefit from the unordered feature setting and context-aware action sequence tokenization, we present an illustrative example in~\Cref{fig:case}.

Each item in the action sequence is represented as a feature set, with each item consisting of five features. The features within an item do not require a specific order. The first step of tokenization leverages the unordered nature of the feature set and applies set permutation regularization~(\Cref{subsubsec:segmentation}). This process arranges each feature set into a specific permutation and iteratively groups features based on the constructed vocabulary~(\Cref{subsubsec:vocab_construct}). This results in different segments that convey the same semantics. Each segment is represented as a sequence of sets, where each set corresponds to a token in the vocabulary.

By examining the segments and their corresponding token sequences, we identify four types of tokens, as annotated in~\Cref{fig:case}: (1) a subset of features from a single item (token {\setlength{\fboxsep}{0pt}\colorbox{myblue}{14844}} corresponds to features {\setlength{\fboxsep}{0pt}\colorbox{myblue}{747}} and {\setlength{\fboxsep}{0pt}\colorbox{myblue}{923}} of the T-shirt); (2) a set containing a single feature (feature {\setlength{\fboxsep}{0pt}\colorbox{mygreen}{76}} of the socks); (3) all features of a single item (token {\setlength{\fboxsep}{0pt}\colorbox{myyellow}{7995}} corresponds to all features of the shorts); and (4) features from multiple items (\eg token {\setlength{\fboxsep}{0pt}\colorbox{myblue}{83}\colorbox{mygreen}{16}} includes feature {\setlength{\fboxsep}{0pt}\colorbox{myblue}{923}} from the T-shirt and feature {\setlength{\fboxsep}{0pt}\colorbox{mygreen}{679}} from the socks, while token {\setlength{\fboxsep}{0pt}\colorbox{mygreen}{19}\colorbox{myyellow}{895}} includes feature {\setlength{\fboxsep}{0pt}\colorbox{mygreen}{1100}} from the socks as well as features {\setlength{\fboxsep}{0pt}\colorbox{myyellow}{560}} and {\setlength{\fboxsep}{0pt}\colorbox{myyellow}{943}} from the shorts). Notably, the fourth type of token demonstrates that the features of one action can be segmented and grouped with features from adjacent actions. This results in different tokens for the same action depending on the surrounding context, showcasing the context-aware tokenization process of ActionPiece.


% \section{Discussion}

% \begin{figure}
  \centering
  \includegraphics[width=\linewidth]{figures/per_frame_boxplot.png}
  
  \caption{\label{fig:frame-boxplot} Comparison of the distribution of F1 scores across all frames for each model.}
\end{figure}
% \subsection{Model Performance}

% \subsubsection{Out-of-Domain Performance}


% \begin{table}
    \centering
    \begin{tabularx}{\linewidth}{Xcccc}
        \hline
        \textbf{Model} & \textbf{All} & \textbf{Amb} \\ 
        \hline
        % Qwen 2.5-7B     & 0.755 & 0.665 & 0.707 & 0.547 \\ % no candidates @ fe
        % Qwen 2.5-7B     & 0.668 & 0.665 & 0.666 & 0.500 \\ % cand @ fe 
        % Phi-4           & 0.798 & 0.717 & 0.756 & 0.607 \\ % no candidates @ fe
        % Phi-4           & 0.719 & 0.717 & 0.718 & 0.560 \\ % cand @ fe
        % Qwen 2.5-7B     & 91.76 & 90.95 \\ % cand @ fe 
        Phi-4                           & 0.375 & 0.262 \\ % Not finetuned
        % $\text{Phi-4}_{cand}$ w/o LF    & 0.927 & 0.918 \\ % Finetuned on candidates
        $\text{Phi-4}_{cand}$ w/o LF    & 0.882 & \textbf{0.862} \\ % Finetuned on candidates
        $\text{Phi-4}_{cand}$ w/ LF     & 0.894 & \textbf{0.862} \\ % Finetuned on candidates
        % $\text{Phi-4}_{cand}$ w/ LF     & \textbf{0.931} & \textbf{0.918} \\ % Finetuned on candidates
        \hline
        KAF-SPA             & 0.912 & 0.776 \\
        KGFI                & 0.924 & 0.844 \\
        CoFFTEA             & \textbf{0.926} & 0.850 \\
        \hline
    \end{tabularx}
    \caption{Results on frame identification using frame element predictions.}
    \label{tab:candidate_frame}
\end{table}
% \subsection{Frame Identification}
% Previous work~\cite{devasier-etal-2024-robust} explored the possibility of filtering candidate targets produced by matching potential lexical units using a frame identification model. To build upon this idea towards a single-step frame-semantic parsing method, we explore the potential of frame elements being used to filter out candidate targets. In this approach, no ground-truth frame inputs are given. This also removes the bias from the model assuming the input always has at least one frame element.

% We represent the LLM instructions using the JSON-exist representation as it performed the best in Table~\ref{tab:representation_performance}. We used Phi-4 for this experiment as it had a very high performance-to-size ratio, as shown in Table~\ref{tab:candidate_frame}. \todo{should run this on qwen-72b} We found that directly using the model performed poorly, likely due to bias in the model learning that each input contains the given frame. To address this, we fine-tuned the LLM using candidates from the training set and found a significant improvement in performance. \todo{add candidates examples}

% Performance on par with CoFFTEA, the previous-best frame identification system.
% Maybe qwen 72b will perform better.



\vspace{6pt}
\noindent\textbf{Declaration of Competing Interest}
The authors declare that they have no known competing financial
interests or personal relationships that could have appeared to influence
the work reported in this paper


\vspace{6pt}
\noindent\textbf{Acknowledgments:} This study was partially financed by the Coordenação de Aperfeiçoamento de Pessoal de Nível Superior - Brasil (CAPES) - Finance Code 001.

This project is part of the Hub for Artificial Intelligence and Cognitive Architectures (H.IAAC- Hub de Inteligência Artificial e Arquiteturas Cognitivas). We acknowledge the support of PPI-Softex/MCTI by grant 01245.013778/2020-21 through the Brazilian Federal Government



%% If you have bibdatabase file and want bibtex to generate the
%% bibitems, please use
%%
 \bibliographystyle{cas-model2-names} 
 \bibliography{main}



\end{document}

