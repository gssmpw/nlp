\section{Introduction}
\label{sec:sec1}

This work proposes an approach that uses an evolutionary algorithm along traditional Machine Learning methods to build a digital, distributed cognitive agent capable of emulating the potential actions (input-output behavior) of a user while allowing further analysis and experimentation – at a certain level – of its internal structures. We focus on the usage of simple devices and the automation of this building process, rather than manually designing the agent. To accomplish this, we used DCT \citep{gibaut2020extending}, a tool to build distributed cognitive architectures across multiple physical and virtual devices. We argue that this brings some advantages, like true, massively parallel processing, low computing power device usage, and the exploration of different ways in which each part of the system might interact with others. In this introduction, we provide a primer on cognitive systems (subsection \ref{cognitivesystems}), system of systems (subsection \ref{systemofsystems}), Cyber-Physical Systems (\ref{cyberphysicalsystems}), and related work on cognitive twins (subsection \ref{relatedwork}). In section \ref{sec:sec2}, we describe DCT, providing an architectural overview of DCT, its codelets, and memory structure, introducing important terminologies to be used later in this article. In section \ref{sec:sec3} we describe our vision on how to evolve a cognitive twin, and in section \ref{sec:sec4} we describe our experiments. Finally, in section \ref{sec05} we discuss our results, provide some conclusions, and indicate future works related to this research. 




\subsection{Cognitive Systems}
\label{cognitivesystems}

The research area of Cognitive Systems emerged in the early 90s, derived upon the research on Expert Systems, Production Systems, and Knowledge-Based Systems \citep{newell1961gps, simon1971human}. Investigating the origins of human knowledge, \cite{anderson1989theory} developed the ACT-R Cognitive Architecture \citep{anderson2004integrated,anderson2009can}. Approximately at the same time, based on the ideas presented by \cite{newell1989symbolic}, Laird released early versions of the SOAR cognitive architecture \citep{laird1996evolution,laird2012soar}. By the end of the 1990s, a large group of researchers involved in the \emph{Simulation of Adaptive Behavior} shaped the concept of Cognitive Architecture as an essential set of structures and processes necessary for the generation of a computational, cognitive model that could be used in several areas relating cognition and behavior. \cite{sloman2002architecture}, for example, discussed the ``architectural'' question of representing mental concepts, and \citet{sun2004desiderata} proposed a set of requirements needed to build a Cognitive Architecture. Next, \cite{sun2007importance} investigated issues and challenges in the development of such models. \cite{langley2009cognitive} analyzed several existing Cognitive Architectures, evaluating the overall progress in the research area. These studies indicated that the main advantage of characterizing a system as \emph{cognitive} is the possibility of obtaining a concrete \emph{framework} for cognitive modeling, in such a way that these models could be concretely tested. Such characterization might allow the definition of a set of structures, modules, and basic processes, creating a common language so that different researchers could explore their proposals for cognitive models.

Following this tradition, we understand a Cognitive System as a general-purpose control system inspired by scientific theories developed to explain the process of cognition in humans and animals. The idea of having an architecture is to decompose the cognitive phenomenon into a spectrum of capabilities, e.g. perception, attention, memory, reasoning, learning, behavior generation, and others. Furthermore, such architectures would also function as theoretical models of cognitive processes, allowing such theories to be tested and reused in different applications.

Cognitive architectures have been tested in several applications, from robot control to decision-making processes in intelligent agents, and substantially facilitate efforts to build intelligent artificial agents due to the specification of cognitive models inspired by human and animal cognition. 


The study of cognitive architectures has become quite popular more recently, although several questions remain open. Nevertheless, over the last 20 years, several Cognitive Architectures have been proposed \citep{goertzel2014brief, kotseruba2016review}. In 2010, Samsonovich \citep{samsonovich2010toward} developed a comparative table presenting a systematic review of many known and documented cognitive architectures.

\subsubsection{Digital and Cognitive Twins}

In the past few years, due to the rise of new technologies and the challenges of a hyper-connected digital world, there is a growing interest in digital copies of physical entities, which are usually referred to as \emph{Digital Twins}. Although there is no consensus on the definition of a Digital Twin, a good definition is presented by \cite{lim2019state}, as a "high fidelity virtual replica of the physical asset with real-time two-way communication for simulation purposes and decision-aiding features for product service enhancement". 
%In the past few years, due to the ubiquitous presence of computational devices and connectivity, there is a growing interest in 
There are many uses of a digital replica of an entity, and \cite{zhang2020towards} present a hierarchy that classifies Digital Twins from the simplest, descriptive type to the more complex ones. These are classified as Cognitive Digital Twins and can replicate, up to some level, human cognitive processes, acting with minimal or no human intervention. In this case, cognition means processes like self-awareness, dynamic knowledge acquisition, and management and decision-making capabilities in both digital and physical worlds, in an info-symbiotic manner.

Among other definitions, \cite{abburu2020cognitwin} present three levels of Digital Twins, being a Cognitive Twin (CT) defined as a hybrid, self-learning, and proactive system that optimizes its cognitive capabilities over time based on collected data and gained experience. It combines AI analytics techniques and expert knowledge to control the system's behavior, solving emerging problems.

In a more agent-centered vision, \cite{somers2020cognitive} understand a CT as a “digital reflection of the user, intended to make decisions and carry out tasks on the user's behalf”, to “highlight the key role that cognitive mechanisms play in modeling human decision-making in the IoT digital space”. There, they apply this concept as a way to model cognitive processes underlying the user’s decisions. In this same direction, \cite{du2020cognition} introduce a personal DT model of information-driven cognition (Cog-DT) as a “digital replica of a person’s cognitive process concerning information processing” including a VR platform and an adaptive UI.

As pointed out by \cite{abburu2020cognitive}, despite the existence of many definitions, there is an agreement that a Cognitive Twin is an extension of a Digital Twin with some cognitive capabilities, even though there is no consensus on what those capabilities might be. Each definition has its own set of capabilities, which better suit the area and application the definition came from.




\subsection{Systems of Systems}
\label{systemofsystems}

According to \cite{gudwin2016stateOfTheArt}, the term Systems of Systems (SoS, \emph{Systems of Systems}) began to appear in the 1990s and still today, continues to attract interest. Although being a widely discussed issue, the community has not yet converged to a common understanding of its meaning. According to  \citep{gorod2008system}, which tries to track the origin of the term, the first researcher to propose something related to the modern idea of an SoS was \cite{boulding1956sos}, who imagined SoS as a “gestalt” in theoretical construction creating a “spectrum of theories” greater than the sum of its parts. Later, \citet{jackson1984system} suggested the use of “SoS methodologies” as the interrelationship between different systems-based problem-solving methodologies in the field of operation research. Although the field has some pioneers, as shown above, it was only in 1989 that we find the first use of the term “system-of-systems” to describe an engineered technology system, in the Strategic Defense Initiative \citep{jacob1974logique, gpo1989restructuring, gorod2008system}.

From 1989 onward, modern research taking into account the term System of Systems was initiated. By that time, the concept of SoS was still very simple and poorly defined \citep{gorod2008system}. Among other definitions, Eisner \emph{et al.} \citep{eisner1991computer, eisner1993rcasse} defined an SoS as a set of systems (developed using a nominal systems engineering process) composed of interdependent systems combined to operate in a multifunctional solution, to achieve a common mission. \citet{shenhar1994new} defined an SoS as an ``array'' of systems, a large collection or network of systems working together to achieve a common goal. \citet{holland1995hidden}, in 1995, carried out studies on SoS from different perspectives. He described an SoS as an artificial complex adaptive system that, in continuous change, ``self-organizes'' with adaptive rules, to increase its complexity.
 In the same year, Maier proposed a characterization approach to differentiate a common system from a system of systems. According to him, an SoS must include ``operational interdependence of elements, managerial independence of elements, evolutionary development, emergent behavior, and geographic distribution''. Currently, Maier is one of the main contributors studying the field of SoS. In 1998, Maier proposed an SoS as a set of assembled components (which, in particular, are systems in themselves) exhibiting operational and managerial component interdependence \citep{maier1996architecting, maeir1998architecting}. \citet{kotov1997systems} and \citet{lukasik1998systems} then  addressed the issue of modeling an SoS. Further, \citet{boardman2006system} studied more than 40 SoS concepts, creating a common view on SoS. 

Following \citet{boardman2006system}, and also in consonance with others \citep{jamshidi2011system, gorod2014case, gorod2008system,dimario2006system,saunders2005system} there are 5 main characteristics shaping an SoS: Autonomy, Belonging, Connectivity, Diversity, and Emergence (or Evolution), usually referred as the ABCDE of SoS. Those characteristics may be described as follows:

\begin{itemize}
\item \textbf{Autonomy}: is the ability of an SoS component to make its own decisions independently of other systems present in its scope. Each system is free to make its own decisions. However, they cannot violate their performance issues and the SoS overall purpose. If a system violates the SoS overall purpose, it would be abandoned and, consequently, would make room for another system to join the SoS, instead of it.

\item \textbf{Belonging}: is the ability of systems to choose to belong to the SoS or not, based on their own needs and purpose. In other words, the system chooses to belong based on cost and benefit.

\item \textbf{Connectivity}: is the ability of an SoS component to maintain connections and exchange information with the other systems composing the SoS. There is a crucial problem here. How to ensure connectivity between systems, considering old legacy systems that do not change over time? How to add a new system to the platform? Researchers argue that it is necessary to define and build dynamic connectivity, creating interfaces and links to add and/or remove systems, in order to meet these needs. Also, an SoS might communicate with other systems autonomously and make connections in real-time.

\item \textbf{Diversity}: an SoS must be heterogeneous, i.e., its components might be of different kinds. If the SoS is diversified, it is more capable of working without becoming obsolete over time. Unlike common systems for which requirements should be defined even before they are designed, an SoS might (and should) update its requirements during runtime. With that, a heterogeneous SoS will remain stable for much longer, without having to be re-designed when new technologies become available. 

\item \textbf{Emergency}: is the ability to build new properties from developmental and evolutionary processes. With that, an unforeseen combination of system parts might result in a coalition that engenders a suitable interoperation of behaviors, which turns into an efficient solution to pursue the overall SoS goals. 
\end{itemize}

It is also important to note that an SoS cannot be developed based only on conventional systems engineering practices because an SoS has different characteristics, requiring a process of its own to deal with its unique particularities. 



\subsection{Cyber-Physical Systems}
\label{cyberphysicalsystems}

During the 90s, different research communities were exploring the development of ``embedded systems''. From the integration between dedicated computer hardware and the automatic control of physical processes, a new technology started to emerge: \emph{Cyber-Physical Systems} (CPS).
According to \cite{lee2011introduction}, a CPS refers to embedded computers and networks monitoring and controlling physical processes, usually comprising feedback loops where physical processes affect computations and vice versa.


The term ``Cyber-Physical Systems'' was originally coined by Helen Gill, from the National Science Foundation, in the United States in 2006. The prefix \emph{cyber} in CPS comes from ``Cybernetics'', a transdisciplinary approach, defined by Norbert Wiener in 1948, for exploring regulatory systems, their structures, constraints, and possibilities. In other words: the study of the effects of feedback in systems theory to implement some sort of control on the system's variables.


There are many definitions of CPS in the literature, but the most common definition among contributors is that a CPS has to do with exposing parts of the urban environment to the Internet, where in some way, computational systems use environmental information collected by sensory devices to develop the control of environmental assets such that some goals are achieved, and some performance indexes are optimized. Also, CPS are defined as integrated systems that provide computation, networking, and physical processes. Another definition might be: a system where the physical environment and software components are tightly interrelated, each of them operating on different spatial and temporal scales, performing different behavior modalities, and interacting with each other. \citep{conti2012looking, sha2009cyber, horvath2012cyber, lee2009computing, nsf10515, khaitan2015design}.

According to \cite{miclea2011dependability} and \cite{khaitan2015design}, a CPS presents a set of fundamental characteristics:

\begin{description}
  \item[$\cdot$ Feedback Mechanism:] All physical components have a \emph{cyber} capability (feedback loop).
  \item[$\cdot$ Automation:] High levels of automation.
  \item[$\cdot$ Scalability:] Networking at multiple scales.
  \item[$\cdot$ Integrability:] Integration at multiple temporal and spatial scales.
  \item[$\cdot$ Reconfigurable:] Configuring and reconfiguring in real-time, dynamically.
\end{description}

The term \emph{cyber} can also be referred to the \emph{cyberspace}, i.e., the Internet, or the cyber world. Events in the real world need to reflect in the cyber world, and consequently, the cyber world needs to communicate with the physical world, such that decisions can be passed to actuators. This communication should be performed in real time and with high precision. A CPS needs to coordinate different systems (computing devices and distributed sensors/actuators) to act. In this case, sensors and actuators work like bridges between the physical and the cyber worlds \citep{khaitan2015design}.

In mid-2014, NIST established the CPS Public Working Group (CPS PWG) which aimed at gathering experts to reason about key aspects of CPS. This reflects the enormous interest in the area from a long-term perspective. Also, in Europe, there is a strong concern for continually raising awareness about Cyber-Physical Systems, since these are seen as one of the most promising technologies attracting European funding lately \citep{haakansson2015reasoning}.


Cyber-Physical Systems are inherently heterogeneous since they combine physical dynamics with computational processes. This heterogeneity exists even within the physical and cyber domains. The physical domain may be multi-physics, combining, for example,  chemical, and biological processes, mechanical motion control, and human operators. The cyber domain may combine programming languages, software component models, networking technologies, and concurrency mechanisms.
Cyber-Physical Systems are intrinsically concurrent. The cyber and the physical subsystems coexist in time, but even within these subsystems, concurrent processes exist. Models of concurrency in the physical world are very different from models of concurrency in software (e.g. arbitrary interleaving of sequences of atomic actions) and also very different from models of concurrency in networks (e.g. asynchronous, partially ordered, discrete actions or clock-driven time slots) \citep{lee2006cyber}. To conciliate these divergent models of concurrency, ensuring interoperability among components with different models of concurrency is one of the core issues to be handled in CPS.


\subsection{Related Work}
\label{relatedwork}


In \cite{abburu2020cognitive}, the authors apply a Cognitive Digital Twin to a real-world problem concerning predictive control in a steel-processing industry. Somers, Oltramari, and Lebiere \citep{somerscognitive, somers2020cognitive} use a Cognitive Architecture to build a Cognitive Twin as a personal assistant that learns user behavior from past data on a user's social network. Then, as a proof-of-concept case, it uses this knowledge to select attendees to a party. 

\cite{eirinakis2020enhancing} present the concept of Enhanced Cognitive Twin, in which they use Data Analytics and Machine Learning to provide them with a tool to solve environmental challenges, like decision-making capabilities, autonomous detection of anomalies and opportunities, and long-term optimization of and reasoning.

\cite{rozanec2020towards} use Cognitive Digital Twins in a manufacturing scenario to improve Key Performance Indicators (KPIs) by detecting possible anomalies and predicting their impact on production and by planning some aspects of production itself, like rescheduling existing plans.

%fora de contexto talvez?
In the Symbiotic Autonomous Systems White Paper II \citep{initiative2018symbiotic}, the authors put Cognitive Digital Twins as an almost essential asset to manage the knowledge gap, which would greatly impact future education.


%Alguns trabalhos sobre estrategias evolutivas e arquiteturas cognitivas
 About using bio-inspired Evolutionary methods, \cite{zhang2020affective} applied a Classifier System in the context of Cognitive Robotics, building an emotional model for a robot. \cite{bellas2010multilevel} showed us a Cognitive Architecture that uses an evolutionary approach to provide an agent - hence a robot - with lifelong learning capabilities. That Architecture, the Multilevel Darwinist Brain, was also applied as a control method on two physical robots, Sony AIBO and Pioneer 2 \citep{bellas2006adaptive}.

To the best of our knowledge, we found no prior work on Cognitive Architectures and Evolution Strategy.






