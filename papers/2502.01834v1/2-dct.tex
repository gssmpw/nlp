\section{The DCT}
\label{sec:sec2}

In this section, we will briefly explain our main tool. The DCT \citep{gibaut2020extending}, an acronym for Distributed Cognitive Toolkit, is a bare-bones toolkit to help the development of cognitive systems in a distributed, language-agnostic fashion. A cognitive agent created with DCT should be able to run across multiple physical (desktops, small computers like Raspberries, or microcontrollers like Arduino) or virtual devices (like Docker containers). It is a re-implementation of the ideas first seen in the CST main article \citep{Paraense201632}, as some features like being inherently single-device and being written in Java may be a shortcoming, sometimes. As expected, it also follows some theory lines like being Codelet-oriented, present in the Copycat architecture \citep{hofstadter1994copycat}. 

To better understand the DCT, one should first refer to the Cognitive Systems Toolkit (CST), the toolkit that came before. As its name suggests, CST is a toolkit for the development of cognitive architectures. Its purpose is to facilitate the creation of such systems the way the user wants, as long it respects its premises. In the core of CST are two basic entities that serve as building blocks: Codelets and Memories. A Codelet is a non-blocking, parallel process that runs continuously and represents a very specific piece of the cognition process of biological creatures. Likewise, a memory is a storage structure from which Codelets read and write information. A user may create any architecture that also follows a Codelet-oriented specification, based on already existing theories or new ones.

Many Cognitive Architectures like e.g. MECA \citep{gudwin2017multipurpose}, LIDA \citep{franklin2014lida}, and others may be seen as multi-agent systems. Following this paradigm, DCT conceives a distributed Cognitive System as a multi-agent system, where a standard protocol is used for the communication among the agents. Also, similar to its predecessor (CST), each agent in such a multi-agent system is built using Codelets and Memories.  The structure and functionality of a multi-agent system are fully compatible with distributed computing concepts and may be comfortably mapped to the Internet of Things: Sensory Codelets may be simply real-world sensor devices and Motor Codelets may be simply relays or actuators, while more complex Codelets may be embedded in microcontrollers or even software containers in the cloud. 


Our first prototype of DCT - and current software based on it - is written in \emph{Shell script} and \emph{Python}, and could be deployed in containers (like Docker) or across different devices, including Raspberries and Arduinos. Figure \ref{fig:multi_device} illustrates the idea: if the input/output conditions are satisfied, there is no need to execute the whole system on a single computer. The project repository can be accessed in: \href{https://github.com/wandgibaut/dct}{https://github.com/wandgibaut/dct}.


\begin{figure*}[hbt]
\centering
	\includegraphics[width=2.0\columnwidth]{imgs/Distributed_MECA_2.png}
\caption{Illustration of a multi-device, Codelet-oriented system as seen in \cite{gibaut2020extending}. Notice that, depending on how powerful is the device, it may run a single Codelet or multiple ones.}
\label{fig:multi_device}
\end{figure*}


\subsection{The DCT Architectural Overview}
\label{sec:dct_over}

From an architectural perspective, the DCT is composed of a set of \emph{Nodes} communicating to each other and integrating, as a whole, a functional system with cognitive capabilities. Theoretically, different subsets of the same collection of Nodes could even act as different systems. Here, the term \emph{Node} represents an entity (logical or physical), which works as a storage for groups of Codelets and/or Memories and is responsible for their operation and life cycles. Figure \ref{fig:multi_device} illustrates the idea of this non-homogeneity in device configurations. The subsection \ref{sec:dct_node} shows further details on this entity.

To communicate with each other, \emph{Nodes} follow a protocol regulating the interaction among them. Following CST specifications, Codelets only interact with Memories that is, a Codelet represents a block of computing unity, applying some process on data, but not holding it. This data storage is performed by Memories, which can be of different technologies. For this communication, DCT uses, canonically, \emph{json} formatted messages. This allows the use of a good range of technologies and simple sockets and databases like \emph{MongoDB} and \emph{Redis} are already supported by existing code. By following these directives, a user may use any language or technology suitable to a device in which the \emph{Node} is.


Formally, we can conceive a Cognitive System created with DCT in the following way:


\begin{definition}[A DCT Cognitive System]

Let $N$ be a set of \emph{Nodes}, where a \emph{Node} is an entity (logical or physical) that encapsulates one or more \emph{Codelets} and/or \emph{Memories} meant to be run under the supervision of a single \emph{Node Master} within an operational system.

To each \emph{Node}, there is an \emph{Interface} $I = \{MO, S\}$, where $MO$ is a subset of the \emph{Memories} implemented within a \emph{Node}, which will be accessed from other external \emph{Nodes} and $S$ is a Server that listens to a URI. This server $S$ should listen for requests and respond with \emph{json} formatted messages.


A Cognitive System created with DCT is defined by the interaction between the elements of $N$ following some Codelet-oriented Cognitive Theory, like MECA or LIDA.
\end{definition}






\subsection{The DCT Codelet structure}
\label{sec:dct_Codelet}


A DCT \emph{Codelet} is composed of a callable program file in a user-specified language that follows some guidelines and some configuration files that can be used to dynamically change some properties, like which \emph{Memories} it can access. Figure \ref{fig:Codelet} illustrates this structure. It is valid to note that, since it was first implemented, some improvements have been made in how a \emph{Codelet} works. The files that characterize a \emph{Codelet} are:


\begin{itemize}
\item A \verb!Codelet! compiled program or script, which runs until be ordered (by the \emph{Node Master}) to stop
\item The \emph{Codelet} configuration file (\emph{fields.json}). This file contains some information regarding the \emph{Codelet} behavior, like its inputs, and should be possible to dynamically change it.
\end{itemize}

Also, if needed for a problem-specific reason, additional files may be used (a \emph{.ini} file, for example). In this work, \emph{Codelet} is implemented in \emph{Python} language.


%%arrumar figura
\begin{figure}[bt]
\centering
	\includegraphics[width=1.0\columnwidth]{imgs/DCT_Codelet_3.png}
\caption{The concept of a DCT Codelet.}
\label{fig:Codelet}
\end{figure}

Following the original CST implementation, the \emph{Codelet} program should have two main functions: a  \verb!calculateActivation! and a \verb!proc!. The first one is used to calculate the current relevance of the \emph{Codelet} itself and may be used in different ways, but mainly with a threshold value to decide if it should execute its main function or skip it. The second one performs the core functionality and is the most important function to be defined by the user. This function represents the procedural code that the entity will periodically call at each time step. Also, it should be non-blocking, meaning that it should be possible to run other processes alongside a \emph{Codelet}.

The configuration file (\emph{fields.json}) contains a structure that is analog to the \emph{Codelet} class in CST, defining many important parameters, e.g. the input, output, and broadcast ports, from where the Codelet can communicate to its Memory objects. 


\subsection{The DCT Memory structure and default support}
\label{sec:dct_memory}

The other core structure, \emph{Memory}, is a generic term for the data structure that holds the information that \emph{Codelets} consume and/or process. Also, it contains some other meta-information, e.g. its name, URL, type, and an evaluation. As said before, this information is standardized as a \emph{json} structure. 

A \emph{Memory} should contain the following information:
 \begin{itemize}
     \item \emph{name}: String 
     \item \emph{IP/port}: String
     \item \emph{type}: String
     \item \emph{I}: String
     \item \emph{eval}: Double 
 \end{itemize}

\subsection{DCT Node}
\label{sec:dct_node}

In the DCT, a \emph{Node} is an abstraction for a physical or virtual device that contains an arbitrary number of \emph{Codelets} and/or \emph{Memories} and is supervised by a single \emph{Node Master}. This definition allows us to consider a computer to be a single \emph{Node} if all relevant entities run in the same environment, or to \textit{have} multiple \emph{Nodes} if each of them runs in a separated container with its \emph{Node Master}.
This \emph{Node Master} is responsible for starting, killing, adding, and removing Codelets and/or Memories, which are running through its supervision. Also, it should periodically check the health of its system, re-executing dead processes, and listen for external requests, like information requests or even requests to shut itself down.

Besides \emph{Codelets} and \emph{Memories}, a \emph{Node} should also implement an \emph{Interface} in which its internal entities may communicate with outside sources, e.g. a server with open sockets.

