\section{Related Work}
Alpha factors, or mathematical expressions designed to predict future asset returns, have been a central focus in quantitative finance since Fama and French's pioneering work on their three-factor model~\cite{fama1992cross}. Traditional methods for alpha mining primarily rely on genetic programming (GP) for exploring the vast search space of factor formulation~\cite{lin2019revisiting, 10.1145/1830483.1830584,zhaofan2022genetic,patil2023ai}, yet they struggle to generate factors resistant to alpha decay. 
%
AlphaEvolve~\cite{alphaevolve} enriches traditional GP by incorporating parameter learning and matrix operations while maintaining GP's explicit formula structure. ~\cite{shen2023mining} strengthens GP with sparsity constraints during mutation, which guides the search toward alpha factors with lower complexity. 
%
Another branch of traditional alpha factor mining relies on Reinforcement learning (RL) to optimize factor formulation through policy learning~\cite{alphagen,shi2024alphaforgeframeworkdynamicallycombine,finrl}. AlphaGen~\cite{alphagen} mines formulaic alphas with deep reinforcement learning, using combination model performance as a reward signal to guide exploration within the alpha factor search space. Shi et al.~\cite{shi2024alphaforgeframeworkdynamicallycombine} propose an RL-based framework that simultaneously discovers and combines multiple alpha factors with fixed weights to form a unified signal. 
%
Building upon PPO~\cite{ppo}, an RL-based approach~\cite{zhao2024quantfactorreinforceminingsteady} is implemented that discards the critic network and introduces a reward-shaping mechanism aiming to generate more profitable and stable alphas for quantitative investment. 
%
\red{
However, these traditional paradigms face significant challenges in addressing alpha decay. Both GP and RL approaches tend to over-emphasize historical performance optimization, leading to either overly complex, overfitted factors or factors lacking economic rationale~\cite{shen2023mining, shi2024alphaforgeframeworkdynamicallycombine, zhao2024quantfactorreinforceminingsteady}. These limitations result in rapid alpha decay when factors are deployed in live markets.}

Recent advances in Large Language Models (LLMs) offer a promising direction to address the alpha decay challenge, as LLMs excel at capturing evolving patterns and incorporating domain knowledge to generate more resilient factors~\cite{wang2023aligninglargelanguagemodels, haluptzok2023language, zhu2024largelanguagemodelslearn, weng2023agent, sumers2024cognitivearchitectureslanguageagents}. While numerous studies have explored alpha mining using LLMs, the critical issue of alpha decay has received limited attention. AutoAlpha~\cite{zhang2020autoalphaefficienthierarchicalevolutionary} introduces an adaptive factor generation framework that continuously evolves alphas based on recent market conditions. LLMFactor~\cite{wang2024llmfactorextractingprofitablefactors} leverages knowledge-guided prompting to extract economically interpretable factors from financial news and historical data. FAMA~\cite{li-etal-2024-large-language} further advances this direction by dynamic factor combination and cross-sample selection, enabling performance across different market regimes. RD-Agent~\cite{yang2024collaborative} proposes a data-centric feedback loop that allows continuous factor adaptation to changing market conditions. However, existing approaches still suffer from alpha decay as they lack effective regularization mechanisms to prevent factors from overly relying on historical patterns and existing market knowledge.

% Recent advances in Large Language Models (LLMs) offer a promising direction to address these limitations, as LLMs have shown remarkable capabilities across multiple domains~\cite{wang2023aligninglargelanguagemodels, haluptzok2023language, zhu2024largelanguagemodelslearn, weng2023agent, sumers2024cognitivearchitectureslanguageagents}.
% %
% Recent works have explored various approaches to leverage LLMs for alpha mining. AutoAlpha ~\cite{zhang2020autoalphaefficienthierarchicalevolutionary} demonstrates the effectiveness of LLM-based formulaic alphas mining through comprehensive backtests in the Chinese stock market. LLMFactor~\cite{wang2024llmfactorextractingprofitablefactors} introduces a sequential knowledge-guided prompting framework that extracts interpretable factors from financial news and historical data. FAMA~\cite{li-etal-2024-large-language} advances this direction by incorporating cross-sample selection for factor diversification and chain-of-experience for efficient exploration while introducing dynamic factor combination strategies to address factor decay. RD-Agent introduces a data-centric loop to research and develop alpha factors autonomously~\cite{yang2024collaborative}. Despite these advances, challenges remain in adapting LLMs to dynamic market conditions, particularly maintaining factor effectiveness across different market regimes. 

%LLMs have demonstrated remarkable self-improvement capabilities through iterative learning in various domains such as gaming, programming, and mathematical problem solving ~\cite{wang2023aligninglargelanguagemodels, haluptzok2023language, zhu2024largelanguagemodelslearn}. 
%
% To harness this potential for domain-specific tasks, researchers have developed frameworks for autonomous LLM agents~\cite{weng2023agent, sumers2024cognitivearchitectureslanguageagents} that combine adaptive tool utilization~\cite{schick2023toolformer}, sophisticated memory systems and strategic planning mechanisms ~\cite{yao2023react, yao2023tree}. 
%
% Building upon the proven success of existing frameworks in real-world applications~\cite{autogpt}, RD-Agent~\cite{yang2024collaborative} innovatively introduces a Researcher-Developer workflow architecture that emulates real-world R\&D processes, aiming to develop robust and economically sound alpha factors through iterative refinement. 
%


% In this section, we briefly review existing research on multi-agent systems in finance and LLM-based alpha mining. These two areas are closely related to our framework, as multi-agent collaboration can enhance the decision-making processes in financial tasks, while alpha mining techniques leverage LLMs for generating new trading signals and strategies.

% \subsection{Foundation Models in Finance}
% Recent advances in foundation models, including large language models (LLMs) and multi-modal large language models (MLLMs) have spurred the development of multi-agent frameworks designed for financial applications. Such systems often feature different agents specialized in tasks like fundamental analysis, sentiment detection, and risk management, thereby replicating the collaborative nature of real-world trading teams.

% For instance, \textbf{FinRobot}~\cite{FinRobot} provides an open-source, multilayer architecture, which includes a Financial AI Agents layer that breaks down complex financial problems through a Chain-of-Thought (CoT) approach. Its platform-centric design streamlines the integration of different LLM agents to serve various analytical and decision-making functions.

% \textbf{FinAgent}~\cite{FinAgent} extends this idea to multimodal data, harnessing numerical, textual, and graphical information. Its dual-level reflection module enables rapid adaptation to market volatility, while a diversified memory retrieval mechanism and explicit integration of established trading principles enhance both the accuracy and explainability of trading actions.

% Researchers have also explored methods to emulate human-like memory and role structures within multi-agent systems. \textbf{FINMEM}~\cite{Finmem} proposes a layered memory approach, incorporating cognitive components such as profiling and decision-making to handle volatile financial data. This design allows the agent to continuously evolve its professional knowledge in real-time.

% Another representative framework, \textbf{TradingAgents}~\cite{TradingAgents}, brings together multiple LLM-driven agents with distinct specialties, from fundamental and sentiment analysis to technical trading strategies. By simulating dynamic debates and risk checks across these agents, the framework aims to improve trading performance. Similarly, \textbf{TradExpert}~\cite{TradExpert} adopts a Mixture-of-Experts paradigm, training separate LLMs on different data sources (e.g., news, market factors) and consolidating their outputs via a ``general expert'' agent. Beyond single-agent performance, \textbf{Enhancing LLM Trading Performance with Fact-Subjectivity Aware Reasoning}~\cite{FSReasoningAgent} delves into separating factual and subjective reasoning processes within a multi-agent environment, showcasing the benefits of each in bull and bear markets, respectively.

% Finally, the importance of multi-agent collaboration and LLM-based finance research has also been highlighted in survey works such as \textbf{Large Language Model Agent in Financial Trading: A Survey}~\cite{LLMSurvey}, which identifies the core architectural components and performance metrics in LLM-based trading agents, as well as challenges that remain unsolved.


% \subsection{LLM-Based Alpha Mining}
% Alpha factors (often referred to simply as ``alphas'') are essential to quantitative finance, designed to generate excess returns by exploiting specific market patterns, anomalies, or inefficiencies. Formulaic alpha, as a crucial subset of alphas, refers to a rule-based strategy in quantitative investment, relying on predefined mathematical models or algorithms. Formulaic alphas are typically created by search algorithms. Genetic programming~\cite{}, which evolves new alphas through structural and numerical mutations, Reinforcement learning~\cite{} dynamically optimizes trading strategies by learning from market interactions, maximizing cumulative returns through continuous exploration and exploitation of financial data. However, these approaches may lack the flexibility to fully encapsulate the creative insights of human researchers. 

% % , or machine learning models, such as LSTM, XGBoost, LightGBM. 

% %  mining typically involves manually crafted or , but

% % This paper focuses on formulaic alphas that refer to a rule-based strategy in quantitative investment, relying on predefined mathematical models or algorithms. Formulaic alphas are typically created .
% % While these approaches can capture complex market dynamics and improve predictive accuracy, they may also face challenges such as reduced interpretability and increased risk of overfitting. The goal of formulaic alphas is to provide a structured and repeatable framework for identifying and capitalizing on market opportunities.


% \textbf{Alpha-GPT}~\cite{AlphaGPT} addresses this gap by introducing a human-AI interactive alpha mining paradigm. Leveraging prompt engineering, it translates the domain expertise of quantitative researchers into new alpha factors that demonstrate both novelty and empirical effectiveness. Additionally, \textbf{Automate Strategy Finding with LLM in Quant investment}~\cite{AutomateStrategy} adopts a multi-agent architecture integrated with LLM-generated alpha signals from multimodal data. The proposed framework dynamically evaluates market conditions using ensemble learning, which leads to substantial gains in trading performance and stability.

% Overall, these works underscore the potential of LLMs to discover novel and profitable alpha factors, bridging human intuition with the modeling power of large-scale deep learning. They exemplify the ongoing efforts to refine alpha mining for more robust, data-driven decision-making in volatile financial environments.

% IC                                                   0.020820
% ICIR                                                 0.189262
% Rank IC                                              0.023697
% Rank ICIR                                            0.221166
% 1day.excess_return_without_cost.mean                 0.000584
% 1day.excess_return_without_cost.std                  0.005516
% 1day.excess_return_without_cost.annualized_return    0.138911
% 1day.excess_return_without_cost.information_ratio    1.632318
% 1day.excess_return_without_cost.max_drawdown        -0.134526
% 1day.excess_return_with_cost.mean                    0.000379
% 1day.excess_return_with_cost.std                     0.005516
% 1day.excess_return_with_cost.annualized_return       0.090280
% 1day.excess_return_with_cost.information_ratio       1.060993
% 1day.excess_return_with_cost.max_drawdown           -0.140035
% 1day.ffr                                             1.000000
% 1day.pa                                              0.000000
% 1day.pos                                             0.000000




\begin{figure*}[h]
\center
\includegraphics[width=0.8\textwidth]{figures/Overview.pdf}
\vspace{-10pt}
\caption{The autonomous workflow of AlphaAgent, where three agents work collectively to mine alphas that balance financial rationale, originality, and adaptability to evolving market conditions, counteract the risk of alpha decay in alpha mining tasks.}
%\caption{AlphaAgent's autonomous workflow for alpha mining, where three agents work collectively to mine alphas that balance financial rationale, originality, and adaptability to evolving market conditions, counteract the risk of alpha decay. } 
\label{fig:workflow}
\vspace{-10pt}
\end{figure*}