% Springer wants this
\usepackage[T1]{fontenc}

% Put hyperref here because springer has requirements about hyperref as well
\usepackage[hidelinks]{hyperref}
\hypersetup{
	colorlinks,
	urlcolor     = blue, %Colour for external hyperlinks
	linkcolor    = blue, %Colour of internal links
	citecolor    = red %Colour of citations
}

% Blurb required by Springer for proper font coloring
\usepackage{color}
\renewcommand\UrlFont{\color{blue}\rmfamily}
\urlstyle{rm}

% My packages

\usepackage[
  pass, % Make sure the default settings are not changed!
%  showframe,
]{geometry} 

% Crop whitespace away
% From: https://tex.stackexchange.com/questions/463871/cropping-white-margins-in-document-preview
%\usepackage[textwidth=\textwidth,textheight=\textheight,
%            paperwidth=\dimexpr\textwidth+2cm,
%            paperheight=\dimexpr\textheight+2cm]{geometry}

% Listings
\usepackage{listings}
\usepackage{chngcntr}
\usepackage{lstfiracode}
% Gobbles leading whitespace of lstlistings
\usepackage{lstautogobble}
\usepackage{mdframed}
% Remove all margins so mdframed can be used as a listings-compatible alternative for \fbox
\mdfsetup{
  skipabove=0pt,
  skipbelow=0pt,
  leftmargin=0pt,
  rightmargin=0pt,
  innerleftmargin=0pt,
  innerrightmargin=0pt,
  innertopmargin=0pt,
  innerbottommargin=0pt
}

\usepackage{booktabs}
\usepackage{siunitx}

\usepackage{mathtools}

\usepackage{syntax}
\usepackage{amsmath}
\usepackage{amsfonts} % For e.g. $\mathbb{Z}$
\usepackage[only, llbracket, rrbracket]{stmaryrd}
\usepackage{changepage}
\usepackage{lipsum}

\usepackage{lineno}
% \linenumbers

% Mathpartir, but with cite/labeling stuff added
\usepackage{mathpartir}
\usepackage{prftree}

% Drawing stuff
\usepackage{adjustbox}
\usepackage{tikz}
\usetikzlibrary{fit, positioning, arrows.meta, backgrounds, shadows}
\usepackage{tcolorbox}
\tcbuselibrary{skins}
\usepackage[disable]{todonotes}

% Symbol packages
\usepackage[misc]{ifsym}
\usepackage{fontawesome}

\usepackage{subcaption}
\usepackage{float}

\usepackage[inline, shortlabels]{enumitem}

% Only import {| |} symbols from styx

\makeatletter
\@ifpackageloaded{stix}{%
}{%
  \DeclareFontEncoding{LS2}{}{\noaccents@}
  \DeclareFontSubstitution{LS2}{stix}{m}{n}
  \DeclareSymbolFont{stix@largesymbols}{LS2}{stixex}{m}{n}
  \SetSymbolFont{stix@largesymbols}{bold}{LS2}{stixex}{b}{n}
  \DeclareMathDelimiter{\lBrace}{\mathopen} {stix@largesymbols}{"E8}%
                                            {stix@largesymbols}{"0E}
  \DeclareMathDelimiter{\rBrace}{\mathclose}{stix@largesymbols}{"E9}%
                                            {stix@largesymbols}{"0F}
}
\makeatother

% Hope this doesn't conflict with cleveref
%\usepackage{nameref}

% Must be last!
\usepackage[
  capitalise, % Capitalize everything, except lines (turned off below)
  nameinlink, % Make sure the "Section" in "Section 3" is clickable
%  noabbrev,   % Make sure nothing is abbreviated; prefer Figure over Fig.
]{cleveref}

\crefname{line}{line}{lines}
\crefname{page}{page}{pages}




%% Mathpartir stuff %%

\newcounter{inferrule}
\setcounter{inferrule}{1}
%\crefname{inferrule}{rule}{rules}
%%https://tex.stackexchange.com/questions/47351/can-i-redefine-a-command-to-contain-itself
%%https://tex.stackexchange.com/questions/4021/how-to-set-a-marker-counter-label-to-an-arbitrary-piece-of-text
%%https://tex.stackexchange.com/questions/672527/why-cref-doesnt-print-names-of-multiple-labels
%% Probably with the next package I can dynamically create a new command that cleveref can use to use to get the name of the rule
\let\oldinferrule\inferrule
%
%%https://tex.stackexchange.com/questions/369104/how-to-declare-dynamic-variable-name-in-latex
%
%% Make sure commandgenerator is in the aux 
\makeatletter
%% {Basename}{Postfix}{Number inputs}
%%\immediate\write\@auxout{%
%%\newcommand\commandgenerator[3]{%
%%  \expandafter\newcommand\csname #1#2\endcsname[#3]}
%%  \unexpanded{\global\@namedef{usevalue@#1}{#2}}%
%%}
%
% Open custom file: https://tex.stackexchange.com/questions/115932/on-the-basics-of-writing-to-reading-from-auxiliary-files-aux-toc-etc
%\newwrite\myfile
%\immediate\openout\myfile=\jobname.refrule
\newcommand*{\myspecialwrite}[2]{%
\immediate\write\@auxout{\unexpanded{\global\@namedef{#1}{#2}}}%
% use custom file
%\immediate\write\myfile{\unexpanded{\global\@namedef{#1}{#2}}}%
}
%
% This one had working numbering, but somehow latex kept using the section number instead of \theinferrule...
%\renewcommand*{\inferrule}[3][]{\oldinferrule[#1]{#2}{#3}}
%\newcommand*{\labelinferrule}[4]{\oldinferrule[#2]{#3}{#4}% First argument is now a label
%\label[inferrule]{#1}%
%%\immediate\write\@auxout{\commandgenerator{inferrulename}{\theinferrule}{#1}}
%%\immediate\write\@auxout{\unexpanded{\global\@namedef{inferrulename@\theinferrule}{#1}}}%
%% Ridiculous use of expandafter and expanded...
%%https://tex.stackexchange.com/questions/443949/fully-expand-macro-argument
%\expandafter\myspecialwrite\expandafter{\expanded{inferrulename@\theinferrule}}{#2}%
%}

\renewcommand*{\inferrule}[3][]{\oldinferrule[#1]{#2}{#3}}
\newcommand*{\labelinferrule}[4]{\oldinferrule[\refstepcounter{inferrule}#1]{#3}{#4}% First argument is now a label, second is label kind (paper or tr=technical report), third is additional arguments for underlying inferrule, and 3 is the actual body
%\label[inferrule]{#1}%
%\immediate\write\@auxout{\commandgenerator{inferrulename}{\theinferrule}{#1}}
%\immediate\write\@auxout{\unexpanded{\global\@namedef{inferrulename@\theinferrule}{#1}}}%
% Ridiculous use of expandafter and expanded...
%https://tex.stackexchange.com/questions/443949/fully-expand-macro-argument
\label{inferrulelabel@#1:#2}%
\expandafter\myspecialwrite\expandafter{\expanded{inferrulename@#1}}{#1}%
}

\crefformat{inferrule}{rule #2#1#3}

\newcommand*{\labelinferrulename}[1]{%
\ifcsname inferrulename@#1\endcsname {\mbox{\@nameuse{inferrulename@#1}}}\else{%
  \GenericWarning{[nonexistentrulename]}{Rule label "#1" does not exist}{}{}}\fi}
\newcommand*{\prettyrule}[1]{\textsc{\labelinferrulename{#1}}}
\newcommand*{\refrule}[2][paper]{{\hypersetup{allcolors=black}{\hyperref[inferrulelabel@#2:#1]{\prettyrule{#2}}}}}
\makeatother

% Time traveling variables
%https://tex.stackexchange.com/questions/652374/storing-variables-in-aux-files
% \@auxout is special
%https://tex.stackexchange.com/questions/115932/on-the-basics-of-writing-to-reading-from-auxiliary-files-aux-toc-etc

% This doesn't really do what I want, since you get both a top & side label.
% https://tex.stackexchange.com/questions/340788/cross-referencing-inference-rules

%%%%%%%%%%%%%%%%%%%%%%%%%%%%%%%
%%%% Mathpartir stuff done %%%%
%%%%%%%%%%%%%%%%%%%%%%%%%%%%%%%
