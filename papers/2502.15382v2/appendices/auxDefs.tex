\section{Auxiliary definitions}

\begin{figure}[t]
\begin{adjustwidth}{-0.15\paperwidth}{-0.15\paperwidth}
\begin{mathpar}
\inferrule[Sort]{}{
{
\begin{array}{cl}
\textsf{sort}(e) & =  e \\
\textsf{sort}(\texttt{$F$[$i$]}) & = F \\
\textsf{sort}(\texttt{$F$[$i$ := $E_{low}$ .. $E_{high}$]}) & = F
\end{array}
}
}
\end{mathpar}
\end{adjustwidth}
\caption{Definition of \textsf{sort} to approximate inequality between instances of $\alpha$}\label{fig:aux}
\end{figure}

\Cref{fig:aux} shows the auxiliary definition \textsf{sort}.
%Definition \textsc{Amp} allows composing a family of expressions with \kwand{}. When $E$ is an $E_{chor}$, we abbreviate $\alpha \in E$ to $\alpha \in \{\beta \mid \exists E'. \texttt{(\kwepexpr{} $\beta$; $E'$)} \in E\}$.
The function \textsf{sort} approximates inequalities on $\alpha$. This is useful for checking if it is possible for two $\alpha$ notations to be equal. For example, if $\textsf{sort}(\texttt{$F$[$i$]}) \neq \textsf{sort}(\texttt{$G$[$j$]})$, then \texttt{$F$[$i$]} and \texttt{$G$[$j$]} are also distinct.

We define functions $\textsf{pre}(m, E)$ and $\textsf{post}(m, E)$ axiomatically to return the pre-/postcondition of $m$. In addition, they replace any occurrence of \kwthis{} in the return value with $E$. If these functions are given a class $C$, they return the pre-/postcondition of the constructor of the class.