\begin{abstract}
Phishing continues to pose a significant cybersecurity threat.
While blocklists currently serve as a primary defense, due to their reactive, passive nature,  these delayed responses leave phishing websites operational long enough to harm potential victims.
It is essential to address this fundamental challenge at the root, particularly in phishing domains.
Domain registration presents a crucial intervention point, as domains serve as the primary gateway between users and websites.

We conduct a comprehensive longitudinal analysis of 690,502 unique phishing domains, spanning a 39-month period, to examine their characteristics and behavioral patterns throughout their lifecycle—from initial registration to detection and eventual deregistration. 
% We analyze 690,502 unique phishing domains collected over a 39-month period. 
We find that 66.1\% of the domains in our dataset are maliciously registered, leveraging cost-effective TLDs and targeting brands by mimicking their domain names under alternative TLDs (\eg, \cc{.top} and \cc{.tk}) instead of the TLDs under which the brand domains are registered (\eg, \cc{.com} and \cc{.ru}). 
We also observe minimal improvements in detection speed for maliciously registered domains compared to compromised domains. 
Detection times vary widely across blocklists, and phishing domains remain accessible for an average of 11.5 days after detection, prolonging their potential impact. 
Our systematic investigation uncovers key patterns from registration through detection to deregistration, which could be leveraged to enhance anti-phishing active defenses at the DNS level.


% Phishing remains a persistent and evolving cybersecurity threat. 
% While blocklists serve as a primary defense, their passive nature introduces critical prevention delays, allowing phishing websites to remain operational for extended periods and increasing risks to victims. 
% It is essential to address this fundamental challenge at the root, particularly phishing domains.
% Domains, serving as the gateway between users and websites, are strategically positioned to detect and block phishing attacks before they reach potential victims.
% To this end, we conduct a comprehensive longitudinal analysis spanning a 39-month period to define a method for detecting maliciously registered domains and to examine their characteristics and behavioral patterns throughout their lifecycle—from initial registration to detection and eventual deregistration. 
% This systematic investigation uncovers key patterns in attack strategies and highlights variations in domain lifespans across different TLDs and impersonated brands.

% Focusing on the DNS registration level, we analyze 690,502 unique phishing domains collected over a 39-month period. 
% Our findings reveal that 66.1\% of the domains in our dataset are maliciously registered, leveraging cost-effective TLDs and targeting brands by mimicking their domain names under alternative TLDs (\eg, \cc{.top} and \cc{.tk}) instead of the TLD under which the brand domain is registered (\eg, \cc{.com} and \cc{.ru}). 
% We also observe minimal improvements in detection speed for maliciously registered domains compared to compromised domains. 
% Detection times vary widely across blocklists, and phishing domains remain accessible for an average of 11.5 days after detection, prolonging their potential impact. 
% Our findings provide deeper insights into the DNS-based indicators that could be leveraged to enhance anti-phishing active defenses.
% Phishing remains a significant and evolving cybersecurity threat, as attackers constantly refine their tactics to maximize deception and evade detection. 
% While blocklisting serves as the primary defense against phishing attacks, its passive nature introduces critical detection delays, enabling malicious sites to remain operational for extended periods. 
% This extended vulnerability window increases the risk to potential victims and reduces the effectiveness of traditional defenses.

% Through a comprehensive longitudinal analysis (39 months), 
% we define a method to detect maliciously registered domains and investigate the characteristics and behavioral patterns of maliciously registered phishing domains throughout their lifecycle, from initial registration through detection to eventual deregistration.
% This systematic examination reveals crucial patterns in attack initialization and exposes variations in domain lifespans across different TLD and impersonated brands.

% Our research tackles phishing at its foundation --- the DNS registration level --- by analyzing 690,502 unique phishing domains over 3 years and 3 months.
% We define maliciously registered domains and examine their characteristics, focusing on TLD usage, targeted brands, and DNS records. 
% Our findings reveal that 66.1\% of the domains in our dataset are maliciously registered. These domains frequently leverage cost-effective TLDs and impersonate brands using alternative TLDs (\eg, .top and .tk) rather than the original brand’s TLD (\eg, .com and .ru).

% Additionally, we investigate the lifecycle of phishing domains from registration to detection and eventually deregistration. 
% Our analysis shows minimal improvements in detection speed for maliciously registered domains compared to compromised domains. 
% Detection times also vary significantly across blocklists, and phishing domains remain accessible for an average of 11.5 days even after being detected. 
% Our findings provide deeper insights into the DNS-based indicators that could be leveraged to enhance anti-phishing active defenses.
% Our research addresses this challenge at its root --- the DNS registration level.
% We analyze 690,502 unique phishing domains over 3 years and 3 months.
% Our study include defining maliciously registered domains and understanding characteristics of those of maliciously registered domains using their TLD, target brands, and DNS records.
% From our study, we find 66.1\% of domains from our dataset are maliciously registered domain.
% From those maliciously registered domains, we find that those domains use cheaper TLDs to register their domains.
% We also find that impersonated brands utilize different TLDs (\eg, .top and .tk) than original domain's TLD (\eg, .com and .ru).
% Furthermore, we investigate lifecycle of phishing domains from registration until detection further until deregistration.
% From our analysis, we find that there is only a minimal detection speed improvement in malicious domains when compared to compromised domains. Also, we find that detections times vary significantly between blocklists and phishing domains remain accessible even after detected by blocklists for an average of 11.5 days.
% We further analyze DNS records to uncover sophisticated evasion techniques employed by attackers, including dynamic DNS record modifications and geolocation-based response manipulation. 
% These tactics specifically target the limitations of current detection systems to delay blocklisting efforts. 
% Our findings provide deeper insights into the DNS-based indicators that could be leveraged to enhance anti-phishing active defenses at the DNS level.

% By identifying these DNS-based indicators and examining the temporal aspects of phishing campaigns, our study provides actionable insights for enhancing anti-phishing defenses.


% By highlighting the gaps in current detection methods, our study offers recommendations for reducing detection delays and strengthening the overall effectiveness of phishing mitigation strategies.

% Our findings highlight critical gaps in current detection methodologies and offer concrete recommendations for reducing detection latency and strengthening phishing mitigation strategies. This research contributes to the development of more proactive and effective countermeasures against evolving phishing threats.


% This paper explores the characteristics and behaviors of phishing domains, emphasizing maliciously registered domains.
% Through a comprehensive, longitudinal analysis, we examine the lifecycle of phishing domains, beginning with their registration and progressing through detection and eventual deregistration. By tracing these timelines, we aim to shed light on the starting points of phishing attacks and uncover lifespan variations based on the impersonated brands.
% In addition to examining domain lifespans, we analyze DNS records to identify sophisticated evasion strategies that attackers employ. These include rapid DNS record changes and location-based response manipulation, tactics designed to bypass detection systems and delay blocklisting. 
% Our findings provide deeper insights into the DNS-based indicators that could be leveraged to enhance anti-phishing defenses. 
% By highlighting the gaps in current detection methods, our study offers recommendations for reducing detection delays and strengthening the overall effectiveness of phishing mitigation strategies.
\end{abstract}