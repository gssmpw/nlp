% \section{DNS Characteristics of Phishing Domains}
\vspace{-3px}
\section{Identifying Maliciously Registered Domains}
\label{sec:def_mal_registered_domains}
We first define maliciously-registered domains and then devise a method to identify the ones
%maliciously-registered domains 
for phishing attacks.
We further analyze the bulk registrations of phishing domains.
% Then, we conduct a comparative analysis of maliciously-registered domains and compromised domains.

% \begin{table}[t]
% \caption{Maliciously-registered Domain. Each steps are taken after removing the Tranco 1M~\cite{Aresearc32:online} list of domains (total of 689,492).}
% \label{tab:maliciously-registered-domain-categories}
% \vspace{-5px}
% \resizebox{0.9\linewidth}{!}{ 
% \begin{NiceTabular}{l r r}
% \toprule
% \multicolumn{1}{c}{\textbf{Type}} & \multicolumn{1}{c}{\textbf{\# of URLs}} & \multicolumn{1}{c}{\textbf{\# of Domains$^*$}} \\
% \midrule
% (1) Brand Name in Domain & 709,694 & 247,699 (35.9\%) \\
% (2) Squatted Domain & 472,320 & 180,468 (26.2\%) \\
% (3) Random-looking Domain & 283,366 & 194,099 (28.2\%) \\
% (4) Bulk-registered Domain & 69,599 & 54,787 \xspace\xspace(7.9\%) \\
% \midrule
% Mal. Total${^\dagger}$ & 1,406,525 & 455,525 (66.1\%) \\
% % \midrule
% % Total & 2,178,088 & 689,492 \xspace(100\%) \\
% \midrule
% \multicolumn{3}{l}{${^\ast}$: Due to the overlap, total domains are over 100\%}
% % \\
% % \multicolumn{3}{l}{${^\dagger}$: Total \# of Maliciously-registered domain.}
% \end{NiceTabular}
% }
% \vspace{-10px}
% \end{table}
% \KL{redo the experiment with all blocklists}

% \subsection{Identifying Maliciously-Registered Domains}
% \KL{Using COMAR method: Squatted domains (in domain, subdomain, path), Levenshtein distance}
% \KL{Maliciously-registered domain: Squatted domain, brand in domain.}
\PP{Def. of Maliciously-registered Domains}
Phishing domains can be classified into two categories: maliciously registered domains and compromised domains. 
A maliciously registered domain is intentionally purchased by an attacker for malicious purposes.
% , often involving brand impersonation or other deceptive tactics. 
In contrast, a compromised domain is a legitimate domain originally used for benign purposes, but attackers exploit vulnerabilities of web servers and inject malicious content (\eg, phishing pages) into the benign servers. 
Detecting maliciously registered domains at an early stage is a critical step in preventing phishing attacks effectively.

% To determine whether a domain is maliciously registered, previous works~\cite{maroofi2020comar,Phishing18:online,de2021compromised,canali2013role,hao2016predator} show different approaches to finding maliciously registered domains.
% One of the most promising and detailed classification methods is a method from COMAR~\cite{maroofi2020comar}.
% Specifically, we leverage lexical features from the method where lexical feature provides 95\% of accuracy from previous work~\cite{maroofi2020comar}.
% From COMAR method~\cite{maroofi2020comar}, the lexical feature includes 9 features. 

\PP{Identification of Maliciously-registered Domains}
We utilize Tranco 1M domains~\cite{Aresearc32:online} as a reference to filter out both legitimate domains and web hosting (or website builder) service domains from our collected phishing domains.
For example, while `\cc{blogspot}.\cc{com}' is a legitimate blogging service, attackers may create subdomains, such as `\cc{usps}-\cc{tracking}-\cc{service}.\cc{blogspot}.\cc{com}' for phishing purposes. 
%By using the Tranco list as a reference, we are able to remove these hosting platform-based phishing domains from our analysis of maliciously registered phishing domains.
After removing the platform-based phishing domains, our list remains 689,492 domains. 
% \RM{also mentioned the number of removed domains?}
\looseness=-1



% \KL{add Tranco removal, what we have removed}
% Tranco: 7,745
Furthermore, we leverage the previous approaches~\cite{maroofi2020comar,Phishing18:online,de2021compromised,canali2013role,hao2016predator} on finding maliciously registered domains.
Especially, COMAR~\cite{CompareP38:online} demonstrated a 95\% accuracy using lexical features and registration timestamps in domains.
The method from COMAR includes nine lexical features (\ie, presence of a brand name in the domain name, path part of URL, and misspelled target brand name in the domain name).
By merging those features and additional features we discovered, we design our method to detect maliciously registered domains by using the following four steps: (1) brand name in domains, (2) squatted domains, (3) random-looking algorithmic domains, and (4) bulk registered domains.
As shown in~\autoref{tab:maliciously-registered-domain-categories}, each steps are taken after removing the Tranco 1M~\cite{Aresearc32:online} domains from the total list of domains.
% Previous works~\cite{maroofi2020comar,Phishing18:online,de2021compromised,canali2013role,hao2016predator} have explored various approaches to identifying maliciously registered domains. 
% One of the most detailed and effective classification methods is COMAR~\cite{maroofi2020comar}, which demonstrates a promising 95\% accuracy using lexical features. 
% Leveraging this approach, we focused on the lexical feature set described in COMAR, which consists of nine distinct features.
% 
% Through heuristic observations, 
% Specifically, we streamline these lexical features into two primary categories: brand names embedded in domains and squatted domains. 
% These categories effectively capture common patterns used by attackers to craft malicious domains. 
% Additionally, we identified two other prominent strategies that extend beyond the lexical features in COMAR: random-looking algorithmically generated domains and bulk-registered domains. 

% However, from our heuristic observation, we combined features into two categories: squatted domains and brand names in domains.
% Aside from two categories previously found, we also find that random-looking algorithmic domain and bulk registered domains are also widely used method to register malicious domains.

\begin{table}[t]
\caption{Maliciously-registered Domain. Each steps are taken after removing the Tranco Top 1M~\cite{Aresearc32:online} list of domains (total of 689,492).}
\label{tab:maliciously-registered-domain-categories}
\vspace{-10px}
\resizebox{0.9\linewidth}{!}{ 
\begin{NiceTabular}{l r r}
\toprule
\multicolumn{1}{c}{\textbf{Type}} & \multicolumn{1}{c}{\textbf{\# of URLs}} & \multicolumn{1}{c}{\textbf{\# of Domains$^*$}} \\
\midrule
(1) Brand Name in Domain & 709,694 & 247,699 (35.9\%) \\
(2) Squatted Domain & 472,320 & 180,468 (26.2\%) \\
(3) Random-looking Domain & 283,366 & 194,099 (28.2\%) \\
(4) Bulk-registered Domain & 69,599 & 54,787 \xspace\xspace(7.9\%) \\
\midrule
Mal. Total${^\dagger}$ & 1,406,525 & 455,525 (66.1\%) \\
% \midrule
% Total & 2,178,088 & 689,492 \xspace(100\%) \\
\midrule
\multicolumn{3}{l}{${^\ast}$: Due to the overlap, total domains are over 100\%}
% \\
% \multicolumn{3}{l}{${^\dagger}$: Total \# of Maliciously-registered domain.}
\end{NiceTabular}
}
\vspace{-15px}
\end{table}

\PP{(1) Brand Name in Domain}
The first approach to identifying maliciously registered domains involves detecting brand names within the domain or subdomain (\eg, \cc{usps}-\cc{security}.\cc{example}.\cc{com}, or \cc{www}.\cc{usps}-\cc{security}-\cc{login}.\cc{com}).
To establish a comprehensive baseline, we curate a list of the top 1,000 most targeted brands in our collected datasets (\ie, APWG), covering 97\% of the domains in our dataset. Domains or subdomains containing any of these brand names are flagged as part of this category.
Our analysis reveals that 33.9\% of domains in the dataset incorporate brand names in their domain or subdomain, highlighting the prevalence of this tactic among phishing attackers. Detailed results for maliciously registered domains across all categories are summarized in~\autoref{tab:maliciously-registered-domain-categories}.

% The first category of finding maliciously registered domains is finding a brand name in a domain or subdomain.
% We start our list of brands with the most targeted 1,000 brands which covers 97\% of domains from our dataset.
% For this case, we find domains if a brand name is in the domain or subdomain.
% As a result, from our dataset, we find that 33.9\% of domains are using brand name in their domain or subdomain.
% The result of the maliciously registered domain is shown in~\autoref{tab:maliciously-registered-domain-categories}.


\PP{(2) Squatted Domain}
% Having squatted domain (brand) names in domain / subdomain (Top 200’s squatted domains)
% 26.2\% (180,468/689,492)
The second category is one of the most common tactics used by phishing attackers: exploiting squatted domains. These domains incorporate a modified version of a brand name in the domain or subdomain, closely mimicking legitimate brand domains to deceive users. For example, a phishing website targeting \cc{facebook}.\cc{com} might use a squatted domain such as \cc{faceb\{\textbf{o}\}ook}.\cc{com} to trick victims into believing they are accessing an authentic website.
\looseness=-1

To identify potential squatted domains, we employ the dnstwist tool~\cite{elceefdn84:online}, which generates domain name variations using various squatting techniques and widely used in previous works~\cite{CompareP38:online,sharma2024securing,gorboe2022detection,kaushik2021exploring}. 
We apply this tool to the top 200 most targeted brand names, which account for 90\% of the domains in our dataset. 
This process generates 765,444 possible squatted domains based on techniques such as adding extra characters to the domain name (\eg, \cc{facebook\textbf{0}}.\cc{com} from \cc{facebook}.\cc{com}), modifying a single bit in the domain name (\eg, \cc{fa\textbf{a}ebook}.\cc{com}), replacing characters with visually similar alternatives (\eg, \cc{faceb\textbf{0}ok}.\cc{com}, where \cc{o} is replaced with the number, \cc{0}), and adding hyphens or extra prefixes (\eg, \cc{face\textbf{-}book}.\cc{com} or \cc{\textbf{d}facebook}.\cc{com}).
Our analysis shows that 26.2\% of domains in our dataset are squatted domains.
%use squatted domains in their domain name or subdomain. 
% This finding highlights the significant role that squatting techniques play in phishing campaigns and emphasizes the need for robust detection strategies to mitigate this threat effectively.

% The second category is one of the most common tactic for phishing attackers is to exploit the squatted domains.
% The squatted domains contain a squatted brand name in a domain or subdomain. 
% For example, a phishing website targeting `facebook.com' can have a squatted domain of `faceb\textbf{o}ook.com'.

% To find a possible \texttt{squatted domain}, we utilize dnstwist~\cite{elceefdn84:online} tool to generate a \texttt{squatted domain} from a top 200 most used target brand names.
% A top 200 most used target brand names covers 90\% of domains from our dataset.
% From 200 domains, we generate 765,444 possible squatted domains.
% A \texttt{squatted domain} includes adding a character (\ie, facebook0.com from facebook.com), bitsquatting which modifies one bit in the domain name (\ie, faaebook.com from facebook.com), homoglyph which swaps character with similar looking digits (\ie, faccb0ok.com from facebook.com where o is replaced with 0), and adding hyphen or character (\ie, face-book.com, dfacebook.com).
% As a result, we find 26.2\% of domains from our dataset is using squatted domain in their domain name or subdomain.



% A maliciously registered domain has three key features within a lexical feature: contain benign brand name in a domain or subdomain (\ie, squatted domain), randomly generated domain, 


\PP{(3) Random-looking Algorithmic Domain}
The random appearance of algorithmically generated domains makes them hard to detect~\cite{UsingAno20:online,Automati98:online}. 
%Domains can also be generated algorithmically to appear randomized, making them harder to detect~\cite{UsingAno20:online,Automati98:online}. 
Attackers exploit this trend by capitalizing on users’ tendencies not to scrutinize domain names closely before clicking on links, even when the domain looks suspicious. 
% This strategy allows attackers to bypass both human vigilance and automated detection, highlighting the need for improved detection mechanisms that can effectively analyze algorithmically generated domains.
To find random-looking algorithmic domains, we follow the approach in~\cite{Automati98:online} by matching domains with English word lists.
%We utilize~\cite{SCOWLCus71:online} to find English words from the list of our domains, and then we eliminate domains that have English words.
We use ~\cite{SCOWLCus71:online}, a word list containing 108,687 words,  to identify domains that include any English words.
%This English word list~\cite{SCOWLCus71:online} contains 108,687 words.
We apply this process after removing the brand in the domain and squatted domains, leaving a total of  194,099 domains.%after removing domains that contain English words.
%, we are left with 194,099 domains.

As shown in~\autoref{tab:maliciously-registered-domain-categories}, a significant portion of domains (28.2\%) are random-looking algorithmic domains.
While such domains may appear suspicious to a human ~\cite{Automati98:online}, automated detection tools often struggle to classify them as malicious due to their lack of clear patterns or recognizable features.
% \DK{stat here}

\PP{(4) Bulk Registration of Domain}
%Other than brand names in domains, squatted domains, and randomly generated domains, we  suggest that bulk-registered domains can be maliciously registered.
%Within maliciously registered domains, attackers can register in bulk by registering multiple malicious domains simultaneously to maximize profit with minimal effort~\cite{hao2013understanding}. 
Attackers often  register many malicious domains simultaneously through bulk registration to maximize profits with minimal effort~\cite{hao2013understanding}. 
A phishing campaign can involve
%For example, an attacker can 
registering multiple domains at the same time and deploying multiple webpages with different domains.
%this way when 
Even if one domain is blocklisted, an attacker can rely on others to continue the attack.
%another source to execute an attack.
Our method to find bulk registered domains includes three conditions that must all be met: registered at the same time, registered through the same registrar, and domain names are similar (using Levenshtein distance~\cite{korczynski2018cybercrime}).
% \KL{Bulk registration category: registered in same time, same registrar, Levenshtein distance}

% Bulk registration is another form of attackers maximizing their results with minimal effort~\cite{hao2013understanding}.
% An attacker can register multiple domains at the same time and deploy multiple webpages with different domains, this way when one domain is blocklisted, an attacker has another source to execute an attack.
% To detect bulk registered domains, we set three conditions as introduced in prior work~\cite{maroofi2020comar}: registered at the same time, registered by the same registrar, and using Levenshtein distance shown in~\cite{korczynski2018cybercrime} to determine whether registered domains are similar.
% Interestingly, bulk-registered domains are often created simultaneously through the same registrar as shown in~\autoref{tab:registrar-in-bulk}, yet blocklists detect them at different times. 
% Even worse, Alibaba Cloud allows users to register domains in bulk~\cite{DomainNa2:online} and advertises with lower price~\cite{NewUserD82:online}.
% By registering multiple domains in bulk, attackers can sustain their activities, as some domains remain active even after others are detected. Some registrars allow bulk registrations, but they could proactively implement measures to prevent attackers from maliciously registering phishing domains in bulk.

\begin{table}[!t]
\caption{Top 10 Registrar in Bulk Registered Domains. \cc{Alibaba} stands out as the registrar associated with the highest number of bulk-registered domains.}
\label{tab:registrar-in-bulk}
\vspace{-10px}
\resizebox{0.9\linewidth}{!}{ 
\begin{NiceTabular}{r l r c}
\toprule
\multicolumn{1}{c}{\textbf{Rank}} & \multicolumn{1}{c}{\textbf{Registrar}} & \multicolumn{1}{c}{\textbf{\# of Domains}} & \multicolumn{1}{c}{\textbf{Country}}\\
\midrule
1 & ALIBABA SGP.${^\ast}$~\cite{AlibabaC38:online} & 4,180 (7.6\%)& CN\\
2 & Alibaba (Wanwang)${^\dagger}$~\cite{NewUserD82:online} & 2,599 (4.7\%)& CN \\
3 & SAV.COM~\cite{Sav54:online} & 2,093 (3.8\%) & US\\
4 & GoDaddy.com~\cite{GoDaddy24:online} & 1,845 (3.4\%)& US\\
5 & Gname.com Pte.~\cite{GNAMEBuy32:online} & 1,560 (2.8\%)& SGP\\
6 & Alibaba Cloud${^\ddagger}$~\cite{AlibabaC38:online} & 1,352 (2.5\%)& CN \\
7 & NameSilo~\cite{LowCostD67:online} & 1,285 (2.3\%)& US\\
8 & Network Solutions~\cite{DomainNa10:online} & 623 (1.1\%)& US\\
9 & Dynadot Inc~\cite{BuyaDoma33:online} & 618 (1.1\%)& US\\
10 & Aceville Pte.~\cite{DNSPod44:online} & 604 (1.1\%)& SGP\\
\midrule
Total &  & 54,787 (100\%)& - \\
\midrule
% \multicolumn{4}{l}{${^\ast}$: ALIBABA.COM SINGAPORE E-COMMERCE PRIVATE LIMITED} \\
\multicolumn{4}{l}{${^\ast}$: ALIBABA.COM SINGAPORE E-COMMERCE PRIVATE} \\
\multicolumn{4}{l}{${^\dagger}$: Alibaba Cloud Computing Co., Ltd. (Wanwang)} \\
\multicolumn{4}{l}{${^\ddagger}$: Alibaba Cloud Computing Ltd. d/b/a HiChina (www.net.cn)}
\end{NiceTabular}
}
\vspace{-15px}
\end{table}

% \begin{figure}[!t]
% \centering
%     \begin{subfigure}{0.4\textwidth}
%         \hspace{5px}\includegraphics[width=\linewidth]{fig/TLD_by_year_caption_v1.pdf}
%         \vspace{-12px}
%         % \caption{Top 10 TLD by Year All.}
%     \end{subfigure}
% \centering
%     \begin{subfigure}{0.16\textwidth}
%         \includegraphics[width=\linewidth,height=6.9em]{fig/TLD_by_year_all_v1.pdf}
%         \vspace{-10px}
%         % \caption{Top 10 TLD by Year All.}
%         \caption{Top 10 (All).}
%         \label{fig:TLD_by_year_all}
%     \end{subfigure}
%     \begin{subfigure}{0.15\textwidth}
%         \includegraphics[width=\linewidth]{fig/TLD_by_year_mal_v1.pdf}
%         \vspace{-10px}
%         % \caption{Top 10 TLD by Year (Malicious).}
%         \caption{Top 10 (Mal.).}
%         \label{fig:TLD_by_year_mal}
%     \end{subfigure}
%     \begin{subfigure}{0.15\textwidth}
%         \includegraphics[width=\linewidth]{fig/TLD_by_year_comp_v1.pdf}
%         \vspace{-10px}
%         % \caption{Top 10 TLD by Year (Comp.).}
%         \caption{Top 10 (Comp.).}
%         \label{fig:TLD_by_year_comp}
%     \end{subfigure}
%     \vspace{-10px}
%     \caption{Top 10 TLD by Year. While \cc{.com} is the most used, \cc{.shop}, \cc{.cn} increase over the years.\KL{make lines thicker, try different colors}}
%     \label{fig:top10_TLD_by_year}
%     % \vspace{-10px}
% \end{figure}

\begin{table*}[t]
\caption{Top 10 Targeted Brands. 
Popular brands (\eg, USPS, OZON, Instagram) predominantly utilize \textcolor{magenta}{\textbf{.top}}, \textcolor{magenta}{\textbf{.tk}}, \textcolor{magenta}{\textbf{.ml}} than the origin of its brand (\eg, \textcolor{dkgreen}{\textbf{.com}}, \textcolor{dkgreen}{\textbf{.ru}}). 
}\vspace{-10px}
\label{tab:impersonated_brand}
\resizebox{0.98\linewidth}{!}{ 
% \begin{NiceTabular}{l c r r r l r r l r r}
% \begin{NiceTabular}{l c r r r l r r l r r}
\begin{tabular}{l c r r r l r r l r r}
\toprule
\multirow{2}{*}{\parbox{1cm}{\centering\textbf{Brand}}} & \multirow{2}{*}{\centering \textbf{Country}} & \multirow{2}{*}{\centering \textbf{Total}} & 
% \multicolumn{1}{c}{\multirow{2}{*}{\textbf{Malicious\\ domains}}} 
% \multicolumn{1}{c}{\multirow{2}{*}{\makecell{{\textbf{Malicious}}\\{\textbf{Domains}}}}}
\makecell{\multirow{2}{*}{\textbf{Malicious}} \\ \multirow{2}{*}{\textbf{Domains}}}
& 
% \multicolumn{1}{c}{\multirow{2}{*}{\textbf{Compromised}}}
\multicolumn{1}{c}{\multirow{2}{*}{\makecell{{\textbf{Compromised}}\\{\textbf{Domains}}}}}
& \multicolumn{3}{c}{\textbf{\makecell{Malicious Domains}}} & \multicolumn{3}{c}{\textbf{Compromised Domains}}\\
% \cmidrule(l){6-11}
\cmidrule(lr){6-8}\cmidrule(lr){9-11}
&
& 
& 
&
& \multicolumn{1}{c}{\textbf{TLD${^\ast}$}} & \multicolumn{1}{c}{\textbf{TLD Count}} & \multicolumn{1}{c}{\textbf{Unique${^\S}$}} & \multicolumn{1}{c}{\textbf{TLD${^\ast}$}} & \multicolumn{1}{c}{\textbf{TLD Count}} & \multicolumn{1}{c}{\textbf{Unique${^\S}$}}\\
\midrule
Facebook & US & 66,700 & 38,817 (58.2\%) & 27,227 (40.8\%) & .com & 11,485 (29.6\%) & 439 & .com & 10,764 (39.5\%) & 324 \\
USPS & US & 41,691 & 37,533 (90.0\%) & 4,109 \xspace\xspace(9.9\%) & \cellcolor{bubblegum}\textbf{.top} & 15,489 (41.3\%) & 259 & \cellcolor{bubblegum}\textbf{.top} & 1,835 (44.7\%) & 153 \\
Microsoft & US & 26,717 & 13,681 (51.2\%) & 12,438 (46.6\%) & .com & 5,759 (42.1\%) & 449 & .com & 6,358 (51.1\%) & 371 \\
DHL & GER${^\dagger}$ & 23,539 & 15,784 (67.1\%) & 7,277 (30.9\%) & .com & 5,741 (36.4\%) & 451 & .com & 3,612 (49.6\%) & 322 \\
OZON & RUS${^\dagger}$ & 18,513 & 10,248 (55.4\%) & 8,465 (45.7\%) & \cellcolor{bubblegum}\textbf{.tk} & 4,549 (44.4\%) & 34 & \cellcolor{bubblegum}\textbf{.tk} & 4,600 (54.3\%) & 17 \\
WhatsApp & US & 11,521 & 8,264 (71.7\%) & 3,163 (27.5\%) & .com & 2,363 (28.6\%) & 162 & .com & 1,198 (37.9\%) & 104 \\
Apple & US & 11,253 & 8,942 (79.5\%) & 2,056 (18.3\%) & .com & 3,385 (37.9\%) & 234 & .com & 918 (44.6\%) & 138 \\
Instagram & US & 11,181 & 7,337 (65.6\%) & 3,681 (32.9\%) & \cellcolor{bubblegum}\textbf{.ml} & 1,482 (20.2\%) & 212 & .com & 982 (26.7\%) & 165 \\
Naver & KOR${^\dagger}$ & 11,030 & 7,207 (65.3\%) & 3,725 (33.8\%) & .com & 2,506 (34.8\%) & 269 & .com & 1,549 (41.6\%) & 195 \\
Amazon & US & 9,473 & 7,390 (78.0\%) & 2,000 (21.1\%) & .com & 2,086 (28.2\%) & 192 & .com & 996 (49.8\%) & 110 \\
\midrule
\multicolumn{11}{l}{${^\ast}$: Most common TLD in brands. ${^\S}$: \# of unique TLD in brand. ${^\dagger}$: GER: Germany, RUS: Russia, KOR: Korea.}
% \multicolumn{11}{l}{${^\ddagger}$: Using other TLD (\eg, \textcolor{magenta}{\textbf{.top}}, \textcolor{magenta}{\textbf{.tk}}, \textcolor{magenta}{\textbf{.ml}}) than the origin of its brand (\eg, \textcolor{dkgreen}{\textbf{.com}}, \textcolor{dkgreen}{\textbf{.ru}})} 
\end{tabular}
}
\end{table*}


\begin{figure*}[!t]
\vspace{-5px}
\centering
        \begin{subfigure}{0.9\textwidth}
            % \hspace{5px}
            \includegraphics[width=\linewidth]{fig/Top10_brand_by_year_stacked_caption.pdf}
            % \vspace{-12px}
            % \caption{Top 10 TLD by Year All.}
        \end{subfigure}
\centering
    \begin{subfigure}{0.32\textwidth}
        \includegraphics[width=\linewidth,height=9em]{fig/Top10_brand_by_year_all_stacked.pdf}
        \vspace{-10px}
        \caption{Top 10 Brand by Year All.}
        \label{fig:Top10_brand_by_year_all}
    \end{subfigure}
    \begin{subfigure}{0.32\textwidth}
        \includegraphics[width=\linewidth]{fig/Top10_brand_by_year_mal_stacked.pdf}
        \vspace{-10px}
        \caption{Top 10 Brand by Year (Malicious).}
        \label{fig:Top10_brand_by_year_mal}
    \end{subfigure}
    \begin{subfigure}{0.32\textwidth}
        \includegraphics[width=\linewidth]{fig/Top10_brand_by_year_comp_stacked.pdf}
        \vspace{-10px}
        \caption{Top 10 Brand by Year (Comp.).}
        \label{fig:Top10_brand_by_year_comp}
    \end{subfigure}
    \vspace{-10px}
    \caption{Top 10 Brand by Year. USPS increases dramatically from 2022 to 2024, specifically in maliciously registered domains. On the other hand, Microsoft decreases in all domains, DHL increases in maliciously registered domains but decreases in the compromised domains.}
    \label{fig:top10_brand_by_year}
    \vspace{-10px}
\end{figure*}


% \DK{stat here: coverage}
Bulk-registered domains, often created simultaneously through the same registrar, account for 7.9\% of the domains in our dataset, as shown in~\autoref{tab:registrar-in-bulk}. While this percentage represents a smaller subset of the dataset, it carries significant implications.
\looseness=-1
% Identifying bulk-registered domains provides an opportunity to mitigate phishing attacks by restricting or disallowing bulk registrations, thereby disrupting attackers' ability to profit from this tactic.

% \DK{screenshot reference is missing.}
% To better understand this phenomenon, we examine which registrars facilitate bulk registration and identify those responsible for the highest number of bulk-registered domains, as shown in~\autoref{tab:registrar-in-bulk}. 
% Notably, \cc{Alibaba} \cc{Cloud}~\cite{AlibabaC38:online} frequently appears as a registrar for such domains, offering bulk registration services~\cite{DomainNa2:online} and even further, promoting them with discounted pricing~\cite{NewUserD82:online} as illustrated in \RM{??}~\autoref{fig:alibaba_bulk_registration}. 

Notably, \cc{Alibaba} \cc{Cloud}\cite{AlibabaC38:online} frequently serves as a registrar for these domains, offering bulk registration services\cite{DomainNa2:online}. 
Furthermore, it actively promotes bulk registrations through discounted pricing~\cite{NewUserD82:online}, as illustrated in~\autoref{apx:bulk_registration}.% ~\autoref{fig:alibaba_bulk_registration}. 
This combination of bulk registration functionality and discounted pricing likely lowers the barrier for registering multiple domains, making it an attractive option for attackers.
This practice enables attackers to sustain their operations by registering multiple domains in bulk, ensuring that some remain active even after others are detected. 
%While some registrars allow bulk registrations, adopting proactive measures, such as stricter verification or limits on bulk purchases, could significantly reduce the malicious use of bulk-registered domains.
Some registrars adopt proactive measures, such as stricter verification or limits on bulk purchases, which significantly reduce the malicious use of bulk-registered domains.

\looseness=-1
% Bulk-registered domains, which are often created simultaneously through the same registrar, account for 7.9\% of the domains in our dataset, as shown in~\autoref{tab:registrar-in-bulk}. 
% Although this percentage represents a smaller subset of the dataset, it carries significant implications. 
% Identifying domains registered in bulk offers an opportunity to mitigate phishing attacks by restricting or disallowing bulk registrations, thereby hindering attackers' ability to profit from this tactic.
% We then further look into which registrar may have allowed bulk register multiple domains, we show the most number of domains registered in~\autoref{fig:alibaba_bulk_registration}.
% Notably, Alibaba Cloud frequently appears as a registrar for bulk-registered domains, offering bulk registration services~\cite{DomainNa2:online} and promoting them with discounted pricing~\cite{NewUserD82:online} as shown in~\autoref{fig:alibaba_bulk_registration}. 
% This practice enables attackers to sustain their activities by registering multiple domains in bulk, ensuring some remain active even after others are detected. While some registrars allow bulk registrations, they could adopt proactive measures to prevent the malicious registration of phishing domains in bulk.

% Although the number of bulk-registered domains is relatively limited, registrars that permit bulk registrations can adopt preventive measures to deter attackers from registering malicious domains in bulk, thereby limiting their ability to maximize monetization.

% \rtbox{
% \textbf{Takeaway:} 
% Though few, registrars allowing bulk registrations can adopt measures to prevent attackers from registering malicious domains to maximize monetization.
% \KL{Fix}
% }



\PP{Manual Validation}
We randomly select 1,000 domains from our identified maliciously registered domains.
Then, we manually validate our method of identifying maliciously registered domains by examining the contents of the phishing domains.
% and DNS records.
Specifically, we utilize historical data from the Wayback Machine~\cite{WaybackM89:online} to identify domains that either lack historical snapshots or display content designed to mimic legitimate webpages.

Our analysis reveals that 72.3\% of the examined domains do not have any historical data in the Wayback Machine. Among the remaining 27.7\%, 14.8\% domains redirect to error pages, while the remaining 12.9\% of domains host malicious content pages. 
% These findings provide additional validation of our approach and highlight the characteristics of maliciously registered phishing domains.

% \DK{fill this..}

% This is based on our assumption that maliciously registered domains either 
% lack historical snapshots or exhibit content designed to mimic legitimate webpages. 


\rtbox{
\textbf{Takeaway:} 
We combined the existing method with our new method of identifying maliciously registered domains.
Maliciously registered domains are over half of phishing domains (66.1\%). 
% \RM{I dont understand the previous sentence. Do you mean Over half of the phishing domains are registered by attackers?}\KL{yes}
Phishing attackers often exploit bulk registration services, such as those offered by \cc{Alibaba} \cc{Cloud}. Notably, among registrars that provide bulk registration, \cc{Alibaba} emerges as the most frequently abused platform for registering domains in bulk.
}




% \subsection{Bulk Registration of Phishing Domains}
% Within maliciously registered domains, attackers can register in bulk. Other than squatted domains, brand names in domains, and randomly generated domains, we also suggest that bulk-registered domains are maliciously registered.
% Our method includes three steps: registered at the same time, registered through the same registrar, and domain names are similar (using Levenshtein distance).
% % \KL{Bulk registration category: registered in same time, same registrar, Levenshtein distance}
% A bulk registration is another form of attackers maximizing their results with minimal effort.
% An attacker can register multiple domains at the same time and deploy multiple webpages with different domains, this way when one domain is blocklisted, an attacker has another source to execute an attack.

% To detect bulk registered domains, we set three conditions as introduced in~\cite{maroofi2020comar}: registered at the same time, registered by the same registrar, and using Levenshtein distance shown in~\cite{korczynski2018cybercrime} to determine whether registered domains are similar.
% % Interestingly, bulk-registered domains are often created simultaneously through the same registrar as shown in~\autoref{tab:registrar-in-bulk}, yet blocklists detect them at different times. 
% % Even worse, Alibaba Cloud allows users to register domains in bulk~\cite{DomainNa2:online} and advertises with lower price~\cite{NewUserD82:online}.
% % By registering multiple domains in bulk, attackers can sustain their activities, as some domains remain active even after others are detected. Some registrars allow bulk registrations, but they could proactively implement measures to prevent attackers from maliciously registering phishing domains in bulk.

% Bulk-registered domains are often created simultaneously through the same registrar, as shown in~\autoref{tab:registrar-in-bulk}, yet blocklists detect them at varying times. 
% Notably, Alibaba Cloud frequently appears as a registrar for bulk-registered domains, offering bulk registration services~\cite{DomainNa2:online} and promoting them with discounted pricing~\cite{NewUserD82:online} as shown in~\autoref{fig:alibaba_bulk_registration}. 
% This practice enables attackers to sustain their activities by registering multiple domains in bulk, ensuring some remain active even after others are detected. While some registrars allow bulk registrations, they could adopt proactive measures to prevent the malicious registration of phishing domains in bulk.

% Although the number of bulk-registered domains is relatively limited, registrars that permit bulk registrations can adopt preventive measures to deter attackers from registering malicious domains in bulk, thereby limiting their ability to maximize monetization.

% \rtbox{
% \textbf{Takeaway:} 
% Though few, registrars allowing bulk registrations can adopt measures to prevent attackers from registering malicious domains to maximize monetization.
% \KL{Fix}
% }

% \subsection{Comparison between Domains}
% \KL{Need a better title}
% % \DK{Comparative analysis of maliciously registered domains and comppromised domains.}
% \DK{add how you analyzed this. In this section, only results are mentioned.}

% As shown in~\autoref{fig:registration_detection_all}, the detection rate for maliciously registered domains is not significantly higher than for compromised domains. 
% In the figure, the blue line represents compromised domains and the red line represents maliciously registered domains. 
% On average, maliciously registered domains are detected in 206.4 days, with a median detection time of 16.3 days. 
% In comparison, compromised domains have a longer average detection time of 332.1 days and a median detection time of 86 days. 
% These statistics highlight that while maliciously registered domains are detected faster on average, the improvement is not substantial.
% \DK{@Kiho, can you run stat on this?}
% As shown in~\autoref{fig:registration_detection_all}, the detection rate for maliciously registered domains is not significantly higher than for other domain types. 
% In~\autoref{fig:registration_detection_all}\KH{Maybe this reference is wrong.}\KL{changed color in figure2}, the blue line represents compromised domains, while the red line represents maliciously registered domains, with the black line indicating all domains. 
% malicious avg: 206.4
% malicious median: 16.3
% compromised avg: 332.1
% compromised median: 86

% The average detection times for compromised domains vary across categories: brand (11.9 days), squatted (7.8 days), and random (15.6 days). 
% Similarly, the average detection times for compromised domains are brand (190.8 days), squatted (188.8 days), and random (199.4 days). 
% In comparison, maliciously registered domains exhibit a higher median detection time of 86 days and an average detection time of 332.1 days.
% This result shows that while maliciously registered domains generally have longer detection times, the difference is not as substantial.

% \rtbox{
% \textbf{Takeaway:} 
% Maliciously registered domains are detected faster than compromised domains, but the difference is minimal, indicating a need for more effective detection strategies for both domain types.
% }

% \begin{figure}[!t]
%     \centering
%     % \includegraphics[width=1\linewidth]{fig/registration_detection_wide_all.pdf}
%     \includegraphics[width=1\linewidth]{fig/mal_vs_comp.pdf}
%     \vspace{-20px}
%     \caption{Days Between Registration and Detection.}
%     \label{fig:registration_detection_all}
% \end{figure}


