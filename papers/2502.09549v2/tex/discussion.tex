\section{Discussion}
\label{sec:discussion}
Based on our analysis, we outline limitations and provide recommendations to guide future research efforts.

\PP{Limitation}
During our verification step, a small number of domains may fall outside our defined malicious domain classification categories. 
While we conducted manual verification to ensure the accuracy of our results, it is still possible that a few domains can exhibit characteristics that do not align with our predefined criteria. 

\PP{Recommendation}
There have been various approaches to understanding how phishing attackers exploit domain registration systems and policies to register malicious domains. Previous research has proposed multiple strategies to address this issue, but the persistence of maliciously registered domains indicates that existing efforts remain insufficient. Several approaches have been discussed in prior work to prevent attackers from registering malicious domains:
\begin{itemize}[leftmargin=*, topsep=0pt, itemsep=0em]
    \item \textbf{Stricter Verification Processes}: Implementing enhanced registrant verification during domain registration, such as requiring government-issued identification or multi-factor authentication, to ensure the legitimacy of registrants.
    \item \textbf{Monitoring and Reporting Systems}: Developing real-time monitoring tools to detect suspicious registration patterns, such as bulk registrations or domains containing high-risk keywords, and establishing automated reporting mechanisms to notify registrars and relevant authorities.
    \item \textbf{Registrar Accountability}: Encouraging or mandating registrars to adopt anti-abuse policies, \eg,  proactive detection measures and swift suspension of flagged domains.\looseness=-1
    \item \textbf{Global Collaboration}: Promoting coordinated efforts between registries, registrars, security organizations, and governments to standardize policies and share intelligence on malicious registration practices.
    \item \textbf{Policy Enforcement for Low-Cost TLDs}: Strengthening oversight for TLDs with low registration costs, which are often exploited by attackers.
    % due to lenient policies. \RM{low cost= lenient?}\looseness=-1
\end{itemize}
% 1) Stricter Verification Processes: Implementing enhanced registrant verification during domain registration, such as requiring government-issued identification or multi-factor authentication, to ensure the legitimacy of registrants.
% 2) Monitoring and Reporting Systems: Developing real-time monitoring tools to detect suspicious registration patterns, such as bulk registrations or domains containing high-risk keywords, and establishing automated reporting mechanisms to notify registrars and relevant authorities.
% 3) Registrar Accountability: Encouraging or mandating registrars to adopt anti-abuse policies, including proactive detection measures and swift suspension of flagged domains.
% 4) Global Collaboration: Promoting coordinated efforts between registries, registrars, security organizations, and governments to standardize policies and share intelligence on malicious registration practices.
% 5) Policy Enforcement for Low-Cost TLDs: Strengthening oversight for TLDs with low registration costs, which are frequently exploited by attackers due to lenient policies.
However, due to the decentralized nature of domain registration systems and varying policies among registries and registrars, it is challenging to implement a generalized defense mechanism. Our analysis aims to reiterate these recommendations and emphasize the urgent need for domain registries and registrars to defend against malicious domains proactively. 
By adopting these measures, stakeholders can significantly reduce phishing attackers' exploitation of domain registration systems.

\PP{Ethics}
% This work does not raise any ethical issues.
Our methods emphasize ethical responsibility while upholding scientific rigor in analyzing real-world phishing domains. 
The data collection process including crawling DNS data and registration data (\eg, RDAP), strictly adheres to established ethical guidelines, utilizing phishing URLs sourced from blocklist feeds explicitly made available for research purposes.
\vspace{-10px}
% \RM{I do not see any critical ethics concern, except your crawling speed. I would just put `This work does not raise any ethical issues' for space.}
% Our research methodology prioritizes ethical responsibilities while maintaining scientific rigor in analyzing real-world phishing domains. We implemented key ethical safeguards: collecting data only from established blocklist feeds explicitly available for research purposes, securely storing and anonymizing all collected data, following responsible disclosure practices with registrars and hosting providers, and designing our crawling methods to minimize the impact on legitimate infrastructure. Our analysis methods avoided direct interaction with active phishing sites to prevent inadvertently supporting criminal operations. We conducted this research in compliance with institutional ethics guidelines and cybersecurity research best practices to benefit the broader security community by improving understanding of phishing infrastructure while protecting potential victims.