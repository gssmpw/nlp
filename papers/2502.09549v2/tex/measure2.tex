\section{Characteristics of Maliciously registered Domains}
\label{sec:mal-reg-char}

% \begin{figure*}[h]
% \centering
%     \begin{subfigure}{0.32\textwidth}
%         \includegraphics[width=\linewidth]{fig/TLD_by_year_all_02.pdf}
%         \vspace{-20px}
%         \caption{Top 10 TLD by Year All.}
%         \label{fig:TLD_by_year_all}
%     \end{subfigure}
%     \begin{subfigure}{0.32\textwidth}
%         \includegraphics[width=\linewidth]{fig/TLD_by_year_mal_02.pdf}
%         \vspace{-20px}
%         \caption{Top 10 TLD by Year (Malicious).}
%         \label{fig:TLD_by_year_mal}
%     \end{subfigure}
%     \begin{subfigure}{0.32\textwidth}
%         \includegraphics[width=\linewidth]{fig/TLD_by_year_comp_02.pdf}
%         \vspace{-20px}
%         \caption{Top 10 TLD by Year (Comp.).}
%         \label{fig:TLD_by_year_comp}
%     \end{subfigure}
%     \vspace{-10px}
%     \caption{Top 10 TLD by Year. While \cc{.com} is the most used, \cc{.shop}, \cc{.cn} increase over the years.}
%     \label{fig:top10_TLD_by_year}
%     \vspace{-10px}
% \end{figure*}

% \begin{table*}[!t]
\caption{Top 10 TLDs by Year. New-gTLD with lower registration costs is widely used in maliciously registered (\textcolor{magenta}{\textbf{.top}}, \textcolor{magenta}{\textbf{.xyz}}, \textcolor{magenta}{\textbf{.shop}}, \textcolor{magenta}{\textbf{.online}}) than compromised domains (\textcolor{dkgreen}{\textbf{.net}}, \textcolor{dkgreen}{\textbf{.info}}). Freenom usage decreases and (\eg, \textcolor{magenta}{\textbf{.cn}} increased in 2024 where \textcolor{dkgreen}{\textbf{.tk}} and \textcolor{dkgreen}{\textbf{.ml}} decreased)}
\label{tab:tld_by_year}
\vspace{-10px}
\resizebox{1\linewidth}{!}{ 
% \begin{NiceTabular}{lrrrrrrrrrrrrcc}
\begin{tabular}{lrrrrrrrrrrrrcc}
\toprule
% \multicolumn{1}{c}{\multirow{2}{*}{\textbf{Total}}}
\multicolumn{1}{c}{\multirow{2}{*}{\textbf{TLD}}} & \multicolumn{1}{c}{\multirow{2}{*}{\textbf{Total}}}
& \multicolumn{5}{c}{\textbf{\makecell[l]{Maliciously Registered Domains}}} & \multicolumn{5}{c}{\textbf{Compromised Domains}} & \multicolumn{1}{c}{\multirow{2}{20px}{\textbf{Price\\(USD)${^\ast}$}}} & \multicolumn{1}{c}{\multirow{2}{*}{\textbf{Types${^\dagger}$}}} & \multicolumn{1}{c}{\multirow{2}{*}{\textbf{Freenom}}}\\
% \cmidrule(l){3-12}
\cmidrule(lr){3-7}\cmidrule(lr){8-12}
% \multicolumn{1}{c}{\textbf{TLD}} & \multicolumn{1}{c}{\textbf{Total}}  
&
& \multicolumn{1}{c}{\textbf{2021}} & \multicolumn{1}{c}{\textbf{2022}} & \multicolumn{1}{c}{\textbf{2023}} & \multicolumn{1}{c}{\textbf{2024}} & \multicolumn{1}{c}{\textbf{Total}} & \multicolumn{1}{c}{\textbf{2021}} & \multicolumn{1}{c}{\textbf{2022}} & \multicolumn{1}{c}{\textbf{2023}} & \multicolumn{1}{c}{\textbf{2024}} & \multicolumn{1}{c}{\textbf{Total}} & & & \\
% \multicolumn{1}{c}{\textbf{Price}} & \multicolumn{1}{c}{\textbf{Types}} & \multicolumn{1}{c}{\textbf{Freenom}}\\
\midrule
.com & 218,267 & 19,359 & 53,795 & 42,376 & 26,949 & 142,479 & 10,837 & 30,051 & 20,813 & 14,087 & 75,788 & \$6 & gTLD & No \\
.top & 84,686 & 2,735 & 12,792 & 19,519 & 18,968 & 54,014 & 3,005 & 10,612 & 13,721 & 3,334 & 30,672 & \textbf{\$1} & new gTLD & No \\
.xyz & 37,698 & 3,545 & 10,520 & 5,396 & 3,624 & 23,085 & 2,985 & 8,310 & 2,732 & 586 & 14,613 & \textbf{\$1} & new gTLD & No \\
.shop & 30,065 & 764 & 2,790 & 5,905 & 11,081 & 20,540 & 628 & 1,532 & 2,550 & 4,815 & 9,525 & \textbf{\$1} & new gTLD & No \\
.cn & 24,708 & 1,639 & 9,234 & 2,045 & 7,060 & \cellcolor{bubblegum}\textbf{19,978} & 471 & 798 & 1,582 & 1,879 & \cellcolor{lightgreen}\textbf{4,730} & \$5 & ccTLD & No \\
.tk & 22,453 & 779 & 10,931 & 1,802 & \cellcolor{lightgreen}\textbf{52} & 13,564 & 213 & 7,001 & 1,623 & \cellcolor{lightgreen}\textbf{52} & 8,889 & \$7 & ccTLD & \cellcolor{light-gray}\textbf{Yes} \\
.online & 14,409 &679 & 3,609 & 3,084 & 1,271 & 8,643 & 254 & 3,123 & 1,741 & 648 & 5,766 & \textbf{\$1} & new gTLD & No \\
.ml & 14,154 &726 & 8,868 & 526 & \cellcolor{lightgreen}\textbf{5} & 10,125 & 251 & 3,388 & 390 & \cellcolor{lightgreen}\textbf{0} & 4,029 &\$12 & ccTLD & \cellcolor{light-gray}\textbf{Yes} \\
.net & 12,672 &1,302 & 3,217 & 2,309 & 1,260 & 8,088 & 691 & 1,987 & 1,233 & 673 & 4,584 &  \$10 & gTLD & No \\
.info & 10,619 &974 & 2,891 & 2,182 & 2,076 & 8,123 &  266 & 934 & 843 & 453 & 2,496 & \$2 & gTLD & No \\
\midrule
\multicolumn{15}{l}{${^\ast}$: Cost to register a domain in each TLD~\cite{CompareP38:online}. ${^\dagger}$: gTLD vs. ccTLD. Note that years 2021 and 2024 are not 12 months.}
\end{tabular}
}
\end{table*}




%maliciously registered domains,
We analyze DNS components of maliciously registered domains, including their targeted brands, TLDs, and DNS records, to gain insights into their characteristics.

\subsection{Targeted Brand}
We utilize target brand information from the APWG dataset.
In our analysis, we identify a diverse range of targeted brands, with \cc{Facebook} standing out as the most targeted brand, followed by \cc{USPS} as shown in~\autoref{tab:impersonated_brand}. 
These two brands alone account for a significant portion with 15.5\% (108,391 out of 697,237) of phishing domains, reflecting their widespread recognition and trust among users. 

\begin{table*}[!t]
\caption{Top 10 TLDs by Year. New-gTLD with lower registration costs is widely used in maliciously registered (\textcolor{magenta}{\textbf{.top}}, \textcolor{magenta}{\textbf{.xyz}}, \textcolor{magenta}{\textbf{.shop}}, \textcolor{magenta}{\textbf{.online}}) than compromised domains (\textcolor{dkgreen}{\textbf{.net}}, \textcolor{dkgreen}{\textbf{.info}}). Freenom usage decreases and (\eg, \textcolor{magenta}{\textbf{.cn}} increased in 2024 where \textcolor{dkgreen}{\textbf{.tk}} and \textcolor{dkgreen}{\textbf{.ml}} decreased)}
\label{tab:tld_by_year}
\vspace{-10px}
\resizebox{1\linewidth}{!}{ 
% \begin{NiceTabular}{lrrrrrrrrrrrrcc}
\begin{tabular}{lrrrrrrrrrrrrcc}
\toprule
% \multicolumn{1}{c}{\multirow{2}{*}{\textbf{Total}}}
\multicolumn{1}{c}{\multirow{2}{*}{\textbf{TLD}}} & \multicolumn{1}{c}{\multirow{2}{*}{\textbf{Total}}}
& \multicolumn{5}{c}{\textbf{\makecell[l]{Maliciously Registered Domains}}} & \multicolumn{5}{c}{\textbf{Compromised Domains}} & \multicolumn{1}{c}{\multirow{2}{20px}{\textbf{Price\\(USD)${^\ast}$}}} & \multicolumn{1}{c}{\multirow{2}{*}{\textbf{Types${^\dagger}$}}} & \multicolumn{1}{c}{\multirow{2}{*}{\textbf{Freenom}}}\\
% \cmidrule(l){3-12}
\cmidrule(lr){3-7}\cmidrule(lr){8-12}
% \multicolumn{1}{c}{\textbf{TLD}} & \multicolumn{1}{c}{\textbf{Total}}  
&
& \multicolumn{1}{c}{\textbf{2021}} & \multicolumn{1}{c}{\textbf{2022}} & \multicolumn{1}{c}{\textbf{2023}} & \multicolumn{1}{c}{\textbf{2024}} & \multicolumn{1}{c}{\textbf{Total}} & \multicolumn{1}{c}{\textbf{2021}} & \multicolumn{1}{c}{\textbf{2022}} & \multicolumn{1}{c}{\textbf{2023}} & \multicolumn{1}{c}{\textbf{2024}} & \multicolumn{1}{c}{\textbf{Total}} & & & \\
% \multicolumn{1}{c}{\textbf{Price}} & \multicolumn{1}{c}{\textbf{Types}} & \multicolumn{1}{c}{\textbf{Freenom}}\\
\midrule
.com & 218,267 & 19,359 & 53,795 & 42,376 & 26,949 & 142,479 & 10,837 & 30,051 & 20,813 & 14,087 & 75,788 & \$6 & gTLD & No \\
.top & 84,686 & 2,735 & 12,792 & 19,519 & 18,968 & 54,014 & 3,005 & 10,612 & 13,721 & 3,334 & 30,672 & \textbf{\$1} & new gTLD & No \\
.xyz & 37,698 & 3,545 & 10,520 & 5,396 & 3,624 & 23,085 & 2,985 & 8,310 & 2,732 & 586 & 14,613 & \textbf{\$1} & new gTLD & No \\
.shop & 30,065 & 764 & 2,790 & 5,905 & 11,081 & 20,540 & 628 & 1,532 & 2,550 & 4,815 & 9,525 & \textbf{\$1} & new gTLD & No \\
.cn & 24,708 & 1,639 & 9,234 & 2,045 & 7,060 & \cellcolor{bubblegum}\textbf{19,978} & 471 & 798 & 1,582 & 1,879 & \cellcolor{lightgreen}\textbf{4,730} & \$5 & ccTLD & No \\
.tk & 22,453 & 779 & 10,931 & 1,802 & \cellcolor{lightgreen}\textbf{52} & 13,564 & 213 & 7,001 & 1,623 & \cellcolor{lightgreen}\textbf{52} & 8,889 & \$7 & ccTLD & \cellcolor{light-gray}\textbf{Yes} \\
.online & 14,409 &679 & 3,609 & 3,084 & 1,271 & 8,643 & 254 & 3,123 & 1,741 & 648 & 5,766 & \textbf{\$1} & new gTLD & No \\
.ml & 14,154 &726 & 8,868 & 526 & \cellcolor{lightgreen}\textbf{5} & 10,125 & 251 & 3,388 & 390 & \cellcolor{lightgreen}\textbf{0} & 4,029 &\$12 & ccTLD & \cellcolor{light-gray}\textbf{Yes} \\
.net & 12,672 &1,302 & 3,217 & 2,309 & 1,260 & 8,088 & 691 & 1,987 & 1,233 & 673 & 4,584 &  \$10 & gTLD & No \\
.info & 10,619 &974 & 2,891 & 2,182 & 2,076 & 8,123 &  266 & 934 & 843 & 453 & 2,496 & \$2 & gTLD & No \\
\midrule
\multicolumn{15}{l}{${^\ast}$: Cost to register a domain in each TLD~\cite{CompareP38:online}. ${^\dagger}$: gTLD vs. ccTLD. Note that years 2021 and 2024 are not 12 months.}
\end{tabular}
}
\end{table*}

\begin{figure*}[!t]
    \vspace{-5px}
    \centering
        \begin{subfigure}{0.8\textwidth}
            % \hspace{5px}
            \includegraphics[width=\linewidth]{fig/TLD_by_year_stacked_caption.pdf}
            % \vspace{-12px}
            % \caption{Top 10 TLD by Year All.}
        \end{subfigure}
    \centering
    \begin{subfigure}{0.33\textwidth}
        \includegraphics[width=\linewidth,height=9em]{fig/TLD_by_year_all_stacked_v1.pdf}
        \vspace{-10px}
        \caption{Top 10 TLD by Year All.}
        \label{fig:TLD_by_year_all_stacked}
    \end{subfigure}
    \begin{subfigure}{0.32\textwidth}
        \includegraphics[width=\linewidth]{fig/TLD_by_year_mal_stacked_v1.pdf}
        \vspace{-10px}
        \caption{Top 10 TLD by Year (Malicious).}
        \label{fig:TLD_by_year_mal_stacked}
    \end{subfigure}
     \begin{subfigure}{0.32\textwidth}
        \includegraphics[width=\linewidth]{fig/TLD_by_year_comp_stacked_v1.pdf}
        \vspace{-10px}
        \caption{Top 10 TLD by Year (Comp.).}
        \label{fig:TLD_by_year_comp_stacked}
    \end{subfigure}
    \vspace{-10px}
    \caption{Top 10 TLD by Year. While \cc{.com} is the most used, \cc{.shop}, \cc{.cn} increase over the years.}
    \label{fig:top10_TLD_by_year}
    \vspace{-10px}
\end{figure*}


% \PP{rend of Targeted Brand}
% \KL{Need to check. general trend here.}
As shown in ~\autoref{fig:top10_brand_by_year}, a clear trend emerges among popular targeted brands. 
Notably, \cc{USPS} is the second most targeted brand, accounting for 6.0\% of phishing domains. 
Interestingly, while \cc{USPS}-targeted domains were minimal in 2021 and 2022, there has been a dramatic increase since 2023. 
This finding aligns with previous reports on phishing domain trends~\cite{Phishing18:online}. 
Conversely, \cc{Microsoft} shows an overall decline in targeting, with a more pronounced decrease observed in compromised domains, as illustrated in ~\autoref{fig:Top10_brand_by_year_comp}. 
Additionally, \cc{DHL}-targeted domains demonstrate an increasing trend in maliciously registered domains over the years, while showing a decline in compromised domains. 
% 
% As shown in ~\autoref{fig:top10_brand_by_year}, we see a prominent trend among popular brands.
% \cc{USPS} is the second most used targeted brand (6.0\%) however more interesting trend is that there was only minimal number of \cc{USPS} targeted brand domains in 2021 and 2022 however after 2023 the targeting \cc{USPS} has increased dramatically.
% This result is also aligns with previous reports in phishing domains~\cite{Phishing18:online}.
% On the other hand, \cc{Microsoft} decreases overall with more significant decrease in compromised domains as shown in~\autoref{fig:Top10_brand_by_year_comp}.
% Another observation is that the number of targeting \cc{DHL} increases on maliciously registered domains over the years however it decreases in the compromised domains.
% \DK{Logitudinal analysis}
% \KL{Add brand trend graph here (by year? or by month?)}
% This variety highlights how attackers prioritize brands that are universally trusted or frequently used for online transactions, where the likelihood of phishing success is higher. 





% Solely relying on lexical features limits the detection of maliciously registered domains, such as an attacker creating non-random and unpopular branded name domains.
% To overcome this limitation, we also leverage the DNS feature of the domain to determine whether a domain is registered for a malicious purpose.
% Also introduced in the COMAR~\cite{maroofi2020comar}, using a DNS feature also helps detect maliciously registered domains.
% First of all, having blocklisted domain name in the DNS record (\ie, NS, MX, TXT records).
% Secondly, from previous study~\cite{maroofi2020comar,docsapwg75:online}, phishing attackers only keep their domain for a short period of time due to the effort and money to maintain a domain if not used.
% Lastly, a self-resolving name server is another factor that can determine a maliciously registered domain (\eg, ns1.malicious.com for resolving malicious.com). 


% \begin{table*}[!t]
\caption{Top 10 TLDs by Year. New-gTLD with lower registration costs is widely used in maliciously registered (\textcolor{magenta}{\textbf{.top}}, \textcolor{magenta}{\textbf{.xyz}}, \textcolor{magenta}{\textbf{.shop}}, \textcolor{magenta}{\textbf{.online}}) than compromised domains (\textcolor{dkgreen}{\textbf{.net}}, \textcolor{dkgreen}{\textbf{.info}}). Freenom usage decreases and (\eg, \textcolor{magenta}{\textbf{.cn}} increased in 2024 where \textcolor{dkgreen}{\textbf{.tk}} and \textcolor{dkgreen}{\textbf{.ml}} decreased)}
\label{tab:tld_by_year}
\vspace{-10px}
\resizebox{1\linewidth}{!}{ 
% \begin{NiceTabular}{lrrrrrrrrrrrrcc}
\begin{tabular}{lrrrrrrrrrrrrcc}
\toprule
% \multicolumn{1}{c}{\multirow{2}{*}{\textbf{Total}}}
\multicolumn{1}{c}{\multirow{2}{*}{\textbf{TLD}}} & \multicolumn{1}{c}{\multirow{2}{*}{\textbf{Total}}}
& \multicolumn{5}{c}{\textbf{\makecell[l]{Maliciously Registered Domains}}} & \multicolumn{5}{c}{\textbf{Compromised Domains}} & \multicolumn{1}{c}{\multirow{2}{20px}{\textbf{Price\\(USD)${^\ast}$}}} & \multicolumn{1}{c}{\multirow{2}{*}{\textbf{Types${^\dagger}$}}} & \multicolumn{1}{c}{\multirow{2}{*}{\textbf{Freenom}}}\\
% \cmidrule(l){3-12}
\cmidrule(lr){3-7}\cmidrule(lr){8-12}
% \multicolumn{1}{c}{\textbf{TLD}} & \multicolumn{1}{c}{\textbf{Total}}  
&
& \multicolumn{1}{c}{\textbf{2021}} & \multicolumn{1}{c}{\textbf{2022}} & \multicolumn{1}{c}{\textbf{2023}} & \multicolumn{1}{c}{\textbf{2024}} & \multicolumn{1}{c}{\textbf{Total}} & \multicolumn{1}{c}{\textbf{2021}} & \multicolumn{1}{c}{\textbf{2022}} & \multicolumn{1}{c}{\textbf{2023}} & \multicolumn{1}{c}{\textbf{2024}} & \multicolumn{1}{c}{\textbf{Total}} & & & \\
% \multicolumn{1}{c}{\textbf{Price}} & \multicolumn{1}{c}{\textbf{Types}} & \multicolumn{1}{c}{\textbf{Freenom}}\\
\midrule
.com & 218,267 & 19,359 & 53,795 & 42,376 & 26,949 & 142,479 & 10,837 & 30,051 & 20,813 & 14,087 & 75,788 & \$6 & gTLD & No \\
.top & 84,686 & 2,735 & 12,792 & 19,519 & 18,968 & 54,014 & 3,005 & 10,612 & 13,721 & 3,334 & 30,672 & \textbf{\$1} & new gTLD & No \\
.xyz & 37,698 & 3,545 & 10,520 & 5,396 & 3,624 & 23,085 & 2,985 & 8,310 & 2,732 & 586 & 14,613 & \textbf{\$1} & new gTLD & No \\
.shop & 30,065 & 764 & 2,790 & 5,905 & 11,081 & 20,540 & 628 & 1,532 & 2,550 & 4,815 & 9,525 & \textbf{\$1} & new gTLD & No \\
.cn & 24,708 & 1,639 & 9,234 & 2,045 & 7,060 & \cellcolor{bubblegum}\textbf{19,978} & 471 & 798 & 1,582 & 1,879 & \cellcolor{lightgreen}\textbf{4,730} & \$5 & ccTLD & No \\
.tk & 22,453 & 779 & 10,931 & 1,802 & \cellcolor{lightgreen}\textbf{52} & 13,564 & 213 & 7,001 & 1,623 & \cellcolor{lightgreen}\textbf{52} & 8,889 & \$7 & ccTLD & \cellcolor{light-gray}\textbf{Yes} \\
.online & 14,409 &679 & 3,609 & 3,084 & 1,271 & 8,643 & 254 & 3,123 & 1,741 & 648 & 5,766 & \textbf{\$1} & new gTLD & No \\
.ml & 14,154 &726 & 8,868 & 526 & \cellcolor{lightgreen}\textbf{5} & 10,125 & 251 & 3,388 & 390 & \cellcolor{lightgreen}\textbf{0} & 4,029 &\$12 & ccTLD & \cellcolor{light-gray}\textbf{Yes} \\
.net & 12,672 &1,302 & 3,217 & 2,309 & 1,260 & 8,088 & 691 & 1,987 & 1,233 & 673 & 4,584 &  \$10 & gTLD & No \\
.info & 10,619 &974 & 2,891 & 2,182 & 2,076 & 8,123 &  266 & 934 & 843 & 453 & 2,496 & \$2 & gTLD & No \\
\midrule
\multicolumn{15}{l}{${^\ast}$: Cost to register a domain in each TLD~\cite{CompareP38:online}. ${^\dagger}$: gTLD vs. ccTLD. Note that years 2021 and 2024 are not 12 months.}
\end{tabular}
}
\end{table*}

% \begin{figure*}[!t]
%     \vspace{-5px}
%     \centering
%         \begin{subfigure}{0.8\textwidth}
%             % \hspace{5px}
%             \includegraphics[width=\linewidth]{fig/TLD_by_year_stacked_caption.pdf}
%             % \vspace{-12px}
%             % \caption{Top 10 TLD by Year All.}
%         \end{subfigure}
%     \centering
%     \begin{subfigure}{0.33\textwidth}
%         \includegraphics[width=\linewidth,height=9em]{fig/TLD_by_year_all_stacked_v1.pdf}
%         \vspace{-10px}
%         \caption{Top 10 TLD by Year All.}
%         \label{fig:TLD_by_year_all_stacked}
%     \end{subfigure}
%     \begin{subfigure}{0.32\textwidth}
%         \includegraphics[width=\linewidth]{fig/TLD_by_year_mal_stacked_v1.pdf}
%         \vspace{-10px}
%         \caption{Top 10 TLD by Year (Malicious).}
%         \label{fig:TLD_by_year_mal_stacked}
%     \end{subfigure}
%      \begin{subfigure}{0.32\textwidth}
%         \includegraphics[width=\linewidth]{fig/TLD_by_year_comp_stacked_v1.pdf}
%         \vspace{-10px}
%         \caption{Top 10 TLD by Year (Comp.).}
%         \label{fig:TLD_by_year_comp_stacked}
%     \end{subfigure}
%     \vspace{-10px}
%     \caption{Top 10 TLD by Year. While \cc{.com} is the most used, \cc{.shop}, \cc{.cn} increase over the years.}
%     \label{fig:top10_TLD_by_year}
%     \vspace{-10px}
% \end{figure*}

\begin{figure*}[!t]
    \centering
    \includegraphics[width=.98\linewidth]{fig/registration_by_brand_wide_02.pdf}
    \vspace{-15px}
    \caption{Days Between Registration and Detection by Top 10 Brand.
    }
    \label{fig:registration_by_brand}
    \vspace{-10px}
\end{figure*}

\subsection{Top-level domain (TLD)}
\label{sec:tld_characteristics}
% \KL{add comparison between all TLD vs maliciously-registered domains' TLD}
% TLD can play a role when registering a domain.
% Whether attacker aims to register through cheaper TLD, or maximize impersonation with brand (\ie, using same TLD as benign TLD is using).
% In one of the phishing report~\cite{Phishing18:online}, Freenom TLD was one of the largest player for phishing attackers to register malicious domains since they provide registration for free.
% However since a reports~\cite{} discovered that phishing attackers abuse this functionality to register malicious domains, Freenom TLD do not provide registration for free from early 2023.
% A report from Interisle~\cite{Phishing18:online} show that after Freenom stop providing registration for free, phishing domains in ccTLD has increased.
% We also want to compare how our analysis hold up with this result by analyzing TLD usage in phishing domains specific focus on maliciously registered domains.

We investigate the use of TLDs in phishing domains %utilize various TLDs 
and assess whether certain TLDs are disproportionately abused. 
Our analysis considers the varying registration costs across TLDs, which may influence attackers' choices and strategies.
% \DK{Analysis Plan} \KL{What we are analyzing in this section}


\PP{Motivation}
TLD choice plays a significant role in domain registration for phishing attacks. Attackers may opt for cheaper TLDs to minimize costs or strategically use the same TLD as the targeted brand to enhance impersonation (\eg, using \cc{.com} for brands that also use \cc{.com}). 
According to a phishing report~\cite{Phishing18:online}, Freenom TLDs were among the most commonly exploited by phishing attackers, as they offered free registrations. 
However, after reports revealed widespread abuse of this functionality for malicious domain registration, Freenom ceased offering free registrations in early 2023.
% \KH{Same policy change described multiple times with similar context \color{red} (B)}
Moreover, a subsequent report by Interisle~\cite{Phishing18:online} noted a shift, with phishing domains in ccTLDs increasing after Freenom's policy change. To examine whether our findings align with this trend, we analyze TLD usage in phishing domains, focusing specifically on maliciously registered domains to uncover patterns and their implications.



\PP{Result: Trend of TLD}
~\autoref{tab:tld_by_year} highlights significant trends in TLD usage across phishing domains, illustrating attackers’ preferences and the influence of policy changes. 
The \cc{.com} TLD dominates the landscape with 218,267 (31.3\% out of 697,237) total phishing domains, likely due to its credibility and widespread familiarity, which enhance its effectiveness for deception. 
Low-cost new gTLDs, such as \cc{.top} and \cc{.shop}, become prominent in our result, with 84,686 (12.1\%) and 37,698 (5.4\%) domains, respectively, reflecting attackers’ preference for inexpensive and lenient TLDs. 
As shown in~\autoref{tab:tld_by_year}, the lower registration costs of new gTLDs (with prices as low as \$1 in our dataset) may contribute to their increased exploitation by phishing domains.
% \RM{USD?}\KL{yes} 
% \DK{add domain specific price}

Freenom TLDs (\eg, \cc{.tk}) are heavily exploited in earlier years, but seen a dramatic decline, dropping from 10,931 domains in 2022 to merely 52 domains in 2024, after Freenom discontinued free registrations in 2023.
% \KH{Same policy change described multiple times with similar context \color{red} (B)}
This finding aligns with a previous report~\cite{Phishing18:online}.
% This shift underscores how stricter policies can significantly reduce abuse. 
Additionally, as shown in~\autoref{fig:top10_TLD_by_year}, the growing presence of \cc{.cn}, from 764 in 2021 to 7,060 in 2024 domains, signals a strategic adaptation by attackers to target TLDs with potentially weaker enforcement mechanisms~\cite{Phishing18:online}.
\looseness=-1

As illustrated in ~\autoref{fig:top10_TLD_by_year}, there is a notable increasing trend in the use of new gTLDs, particularly \cc{.top} and \cc{.shop}. 
Interestingly, the use of \cc{.top} in maliciously registered domains has steadily increased over the years, while its usage in compromised domains shows a decline in 2024. 
In contrast, \cc{.shop} demonstrates a consistent increase in usage across both maliciously registered and compromised domains.
% As illustrated in~\autoref{fig:top10_TLD_by_year}, we can also see the increasing trend of new gTLDs specifically \cc{.top}, \cc{.shop}.
% Interestingly, the usage of \cc{.top} in maliciously registered domains increase over years however in compromised domains, it decreases in 2024.
% On the other hand, \cc{.shop} shows constant increase in between maliciously registered and compromised domains.
% \DK{what ground?}
% \DK{may need a line graph over time.}
% \DK{deeper analysis required}\KL{added trend result}

\rtbox{
\textbf{Takeaway:} 
%Our TLD analysis reveals that 
Phishing domains often exploit new gTLDs due to their lower registration costs. Notably, when \cc{Freenom} discontinued offering free domain registrations, the usage of \cc{.cn} domains increased concurrently. 
%Additionally, 
Certain new gTLDs, such as \cc{.shop}, exhibit distinct trends between maliciously registered domains and compromised domains, highlighting different attack strategies.
%employed by attackers.
% \DK{takeaway box (.cn becomes popular and freenom TLD is not used)}
}

% \begin{figure*}[!t]
%     \centering
%     \includegraphics[width=.98\linewidth]{fig/registration_by_brand_wide_02.pdf}
%     \vspace{-15px}
%     \caption{Days Between Registration and Detection by Top 10 Brand.
%     }
%     \label{fig:registration_by_brand}
%     \vspace{-10px}
% \end{figure*}

% \PP{Result: Trend of Targeted Brand}
% \autoref{tab:impersonated_brand} reveals key patterns in the impersonation of top brands by phishing domains, providing a detailed breakdown of how attackers strategically exploit TLDs and domain registration types. 

% As shown in ~\autoref{fig:top10_brand_by_year}, a clear trend emerges among popular targeted brands. 
% Notably, \cc{USPS} is the second most targeted brand, accounting for 6.0\% of phishing domains. 
% Interestingly, while \cc{USPS}-targeted domains were minimal in 2021 and 2022, there has been a dramatic increase since 2023. 
% This finding aligns with previous reports on phishing domain trends\cite{Phishing18:online}. 
% Conversely, \cc{Microsoft} shows an overall decline in targeting, with a more pronounced decrease observed in compromised domains, as illustrated in ~\autoref{fig:Top10_brand_by_year_comp}. 
% Additionally, \cc{DHL}-targeted domains demonstrate an increasing trend in maliciously registered domains over the years, while showing a decline in compromised domains. 
% % Attackers’ TLD choices further highlight their strategic intent. 
% % \KL{Check with next paragraph (remove this one?)}
% % USPS-targeted domains heavily rely on \cc{.top}, which accounts for 41.27\% of its phishing domains, diverging from the original \cc{.com} TLD used by the legitimate brand. 
% % Similarly, \cc{OZON}, ranked as the fifth most targeted brand, sees 44.39\% of its phishing domains registered under \cc{.tk}, a previously free TLD, instead of its native \cc{.ru}. 
% % This shift to alternative TLDs, particularly those known for low costs or lenient registration policies, reflects attackers’ attempts to reduce expenses while maintaining scalability in their campaigns. 
% % By contrast, compromised domains often align with the brand’s original TLD, leveraging its perceived legitimacy to deceive users and evade detection. 
% % For instance, phishing domains targeting Microsoft frequently use \cc{.com}, making detection more challenging due to their resemblance to genuine brand domains.


\PP{Using Different TLD than Original Brand Domain}
\label{sec:using_diff_tld}
Phishing domains do not always use the same TLD as their original domains. 
For instance, 
%while \cc{Facebook}’s legitimate domain uses \cc{.com}, 
phishing attackers often register \cc{Facebook}-targeted domains using alternative TLDs such as \cc{.top}, rather than \cc{.com} that used by \cc{Facebook}. 
Similarly, \cc{USPS}, the second popular targeted brand in our analysis, is frequently targeted using \cc{.top} domains instead of the brand’s original \cc{.com}. 
Another example is \cc{OZON}, ranked as the 5th most targeted brand, with 49.42\% of its phishing domains registered under the \cc{.tk} instead of its original \cc{.ru}. 
Both \cc{.top} and \cc{.tk} are significantly cheaper than \cc{.com} for registration, with \cc{.tk} previously offered for free by Freenom until January 2023. 
Interestingly, both targeted brands \cc{USPS} and \cc{OZON} have the quickest detected time by blocklists as shown in~\autoref{fig:registration_by_brand}.
We will discuss detection time across different brands in~\autoref{sec:registration_detection}.

Another noteworthy observation from ~\autoref{tab:impersonated_brand} is that 44.39\% of \cc{OZON}-targeted domains are registered under \cc{.tk}, which is significantly more popular than any other brand. Additionally, \cc{OZON}-targeted domains exhibit the smallest number of unique TLDs (34) among the top 10 brands.
%, with only 34 unique TLDs. 
Furthermore, as shown in ~\autoref{fig:top10_brand_by_year}, \cc{OZON} demonstrates a decline in phishing activity over time. 
These findings suggest that attackers targeting \cc{OZON} often prefer low-cost TLDs, such as those offered by Freenom, to minimize costs and maximize the scalability of their phishing campaigns.
% \KH{Makes claims about attacker motivation based on TLD choices without supporting evidence of actual cost impact or attacker behavior analysis.}
% \DK{experiment -> conclusion, is weak}\KL{Added}



% The patterns observed also highlight distinct attacker strategies tailored to specific brands. 
% For USPS, the overwhelming use of \cc{.top} indicates a clear preference for registering new domains under cost-effective or minimally regulated TLDs. 
% For brands like Microsoft, attackers exploit both compromised domains and new registrations, balancing the use of \cc{.com} for compromised domains and alternative TLDs for maliciously registered ones. 
% Meanwhile, the reliance on \cc{.tk} for OZON-targeted domains demonstrates how attackers adapt to exploit lenient TLD policies, with a sharp decline in \cc{.tk} usage expected following Freenom’s discontinuation of free registrations in 2023.

% These findings underscore the varying tactics attackers use when targeting different brands, driven by the characteristics of the brand, the TLD landscape, and the cost-effectiveness of registration. 
% For brands like USPS, which show a high share of malicious registrations, detection efforts should focus on monitoring non-standard TLDs such as \cc{.top}. 
% Conversely, for brands like Microsoft, where compromised domains represent a substantial share, monitoring activity within original TLDs such as \cc{.com} is critical. 
% Additionally, the heavy reliance on low-cost TLDs like \cc{.tk} and \cc{.xyz} across many brands highlights the need for stricter policies and oversight for leniently monitored TLDs to disrupt attackers’ operations. 
% This analysis underscores the importance of adapting detection and enforcement strategies to the specific behaviors and patterns associated with individual brands and their targeted domains.

\rtbox{
\textbf{Takeaway:} 
Phishing attackers prefer low-cost TLDs like \cc{.top} and \cc{.tk} to target brands such as \cc{USPS} and \cc{OZON}, with \cc{OZON} relying on \cc{.tk} for 44.39\% of its phishing domains. These brands also show the fastest detection times by blocklists, highlighting the importance of monitoring cost-effective TLDs to combat phishing campaigns.
% Our result shows that domains targeting certain brands (\ie, \cc{USPS} and \cc{OZON}) use lower-cost TLD (\ie, \cc{top} and \cc{tk}) to register their domains instead using their original benign TLD (\ie, \cc{.com}, \cc{.ru}).
% \KL{highlight if it is a new contribution (surprising), compare with previous work}
% \DK{here.}
}

\PP{Maliciously-registered Vs. Compromised}
The comparison between maliciously registered and compromised domains reveals notable differences in their TLD preferences. 
Among a total of 218,267 \cc{.com} domains, 142,479 (65.3\%) were maliciously registered, while 75,788 (34.7\%) were compromised, indicating that attackers leveraging \cc{.com} domains often register them intentionally for malicious purposes. 
Conversely, new gTLDs such as \cc{.top} and \cc{.xyz} also show a strong preference for malicious registrations, with 54,014 (63.8\%) and 23,085 (61.2\%) domains, respectively, highlighting attackers’ exploitation of low-cost TLDs for scalability. 
In contrast, Freenom TLDs like \cc{.tk} saw relatively balanced usage between maliciously registered and compromised domains before policy changes restricted their availability. 
These patterns suggest that maliciously registered domains favor low-cost or lenient TLDs, while compromised domains may be distributed across a broader range of TLDs, reflecting their opportunistic use of existing infrastructures. 
This distinction underscores the importance of targeted monitoring and stricter enforcement in TLDs that are disproportionately used for malicious registrations.

We analyze the targeted brands between maliciously registered domains and compromised domains.
\cc{Facebook} is the most used targeted brand, with 66,700 phishing domains, of which 58.20\% are maliciously registered. 
\cc{USPS} and \cc{Microsoft} follow, with 41,691 and 26,717 domains, respectively. 
\cc{USPS} exhibits an exceptionally high proportion of maliciously registered domains (90.03\%), indicating that attackers targeting this brand prefer creating new domains rather than compromising existing ones. 
Microsoft demonstrates a more balanced split, with 51.21\% malicious registrations and 46.55\% compromised domains, suggesting a dual approach in leveraging both new and existing infrastructures.


\rtbox{
\textbf{Takeaway:} 
New gTLDs (\eg, \cc{.top}, \cc{.xyz}) are more prevalent in maliciously registered domains, while compromised domains favor legacy gTLDs (\eg, \cc{.net}, \cc{.info}). Freenom TLDs like \cc{.tk} and \cc{.ml} have declined, while \cc{.cn} has increased in 2024.
}



% \begin{table}[t]
% \caption{TLD by Year (Top 10).\KL{add comparison between maliciously vs compromised}}
% \label{tab:tld_by_year}
% \resizebox{0.98\linewidth}{!}{ 
% \begin{NiceTabular}{l r r r r r r c c}
% \toprule
% \multicolumn{1}{c}{\textbf{TLD}}  & \multicolumn{1}{c}{\textbf{Total}} & \multicolumn{1}{c}{\textbf{2021}} & \multicolumn{1}{c}{\textbf{2022}} & \multicolumn{1}{c}{\textbf{2023}} & \multicolumn{1}{c}{\textbf{2024}} & \multicolumn{1}{c}{\textbf{Price${^\ast}$}} & \multicolumn{1}{c}{\textbf{Types${^\dagger}$}} & \multicolumn{1}{c}{\textbf{Freenom}} \\
% \midrule
% .com  & 211,494 & 31,121 & 84,705 & 63,995 & 31,673 & \$6 & gTLD & No \\
% .top  & 75,272 & 5,742 & 23,406 & 33,241 & 12,883 & \$1 & New-gTLD & No \\
% .xyz  & 35,901 & 6,531 & 18,835 & 8,136 & 2,399 & \$1 & New-gTLD & No \\
% .shop  & 26,150 & 1,392 & 4,323 & 8,457 & 11,978 & \$1 & New-gTLD & No \\
% .tk  & 22,453 & 992 & 17,932 & 3,425 & 104 & \$7 & New-gTLD & Yes \\
% .cn  & 20,460 & 2,111 & 10,034 & 3,641 & 4,674 & \$5 & ccTLD & No \\
% .ml  & 14,153 & 977 & 12,256 & 916 & 4 & \$12 & gTLD & Yes \\
% .online  & 13,951 & 935 & 6,734 & 4,829 & 1,453 & \$1 & New-gTLD & No \\
% .net  & 12,607 & 2,132 & 5,343 & 3,646 & 1,486 & \$10 & gTLD & No \\
% .info & 9,584 & 1,264 & 3,832 & 3,036 & 1,452 & \$2 & gTLD & No \\
% \bottomrule
% \multicolumn{9}{l}{${^\ast}$: Cost to register a domain in each TLD~\cite{CompareP38:online}. ${^\dagger}$: gTLD vs ccTLD}\\
% \multicolumn{9}{l}{Note that years 2021 and 2024 are not 12 months.}
% % \multicolumn{9}{l}{${^\dagger}$: gTLD vs ccTLD} \\
% \end{NiceTabular}
% }
% \end{table}

% \begin{figure*}[!t]
% \centering
%     \begin{subfigure}{0.32\textwidth}
%         \includegraphics[width=\linewidth]{fig/Top10_brand_by_year_all_02.pdf}
%         \vspace{-20px}
%         \caption{Top 10 Brand by Year All.}
%         \label{fig:Top10_brand_by_year_all}
%     \end{subfigure}
%     \begin{subfigure}{0.32\textwidth}
%         \includegraphics[width=\linewidth]{fig/Top10_brand_by_year_mal_02.pdf}
%         \vspace{-20px}
%         \caption{Top 10 Brand by Year (Malicious).}
%         \label{fig:Top10_brand_by_year_mal}
%     \end{subfigure}
%     \begin{subfigure}{0.32\textwidth}
%         \includegraphics[width=\linewidth]{fig/Top10_brand_by_year_comp_02.pdf}
%         \vspace{-20px}
%         \caption{Top 10 Brand by Year (Comp.).}
%         \label{fig:Top10_brand_by_year_comp}
%     \end{subfigure}
%     \vspace{-10px}
%     \caption{Top 10 Brand by Year. USPS increases dramatically from 2022 to 2024, specifically in maliciously registered domains. On the other hand, Microsoft decreases in all domains, DHL increases in maliciously registered domains but decreases in the compromised domains.}
%     \label{fig:top10_brand_by_year}
% \end{figure*}

% \begin{table*}[t]
% \caption{Top 10 Targeted Brands. 
% Popular brands (\eg, USPS, OZON, Instagram) predominantly utilize \textcolor{magenta}{\textbf{.top}}, \textcolor{magenta}{\textbf{.tk}}, \textcolor{magenta}{\textbf{.ml}} than the origin of its brand (\eg, \textcolor{dkgreen}{\textbf{.com}}, \textcolor{dkgreen}{\textbf{.ru}}). 
% }\vspace{-10px}
% \label{tab:impersonated_brand}
% \resizebox{0.98\linewidth}{!}{ 
% \begin{NiceTabular}{l c r r r l r r l r r}
% \toprule
% \multicolumn{1}{c}{\multirow{2.25}{*}{\textbf{Brand}}} & \multicolumn{1}{c}{\multirow{2.25}{*}{\textbf{Country}}} & \multicolumn{1}{c}{\multirow{2.25}{*}{\textbf{Total}}} & 
% % \multicolumn{1}{c}{\multirow{2}{*}{\textbf{Malicious\\ domains}}} 
% \multicolumn{1}{c}{\multirow{2.25}{*}{\makecell[l]{{\textbf{Malicious}}\\{\textbf{Domains}}}}}
% & 
% % \multicolumn{1}{c}{\multirow{2}{*}{\textbf{Compromised}}}
% \multicolumn{1}{c}{\multirow{2}{*}{\makecell[l]{{\textbf{Compromised}}\\{\textbf{Domains}}}}}
% & \multicolumn{3}{c}{\textbf{\makecell[l]{Malicious Domains}}} & \multicolumn{3}{c}{\textbf{Compromised Domains}}\\
% % \cmidrule(l){6-11}
% \cmidrule(lr){6-8}\cmidrule(lr){9-11}
% &
% & 
% & 
% &
% & \multicolumn{1}{c}{\textbf{TLD${^\ast}$}} & \multicolumn{1}{c}{\textbf{TLD Count}} & \multicolumn{1}{c}{\textbf{Unique${^\S}$}} & \multicolumn{1}{c}{\textbf{TLD${^\ast}$}} & \multicolumn{1}{c}{\textbf{TLD Count}} & \multicolumn{1}{c}{\textbf{Unique${^\S}$}}\\
% \midrule
% Facebook & US & 66,700 & 38,817 (58.20\%) & 27,227 (40.82\%) & .com & 11,485 (29.59\%) & 439 & .com & 10,764 (39.53\%) & 324 \\
% USPS & US & 41,691 & 37,533 (90.03\%) & 4,109 \xspace\xspace(9.86\%) & \cellcolor{bubblegum}\textbf{.top} & 15,489 (41.27\%) & 259 & \cellcolor{bubblegum}\textbf{.top} & 1,835 (44.66\%) & 153 \\
% Microsoft & US & 26,717 & 13,681 (51.21\%) & 12,438 (46.55\%) & .com & 5,759 (42.09\%) & 449 & .com & 6,358 (51.12\%) & 371 \\
% DHL & GER${^\dagger}$ & 23,539 & 15,784 (67.05\%) & 7,277 (30.91\%) & .com & 5,741 (36.37\%) & 451 & .com & 3,612 (49.64\%) & 322 \\
% OZON & RUS${^\dagger}$ & 18,513 & 10,248 (55.36\%) & 8,465 (45.72\%) & \cellcolor{bubblegum}\textbf{.tk} & 4,549 (44.39\%) & 34 & \cellcolor{bubblegum}\textbf{.tk} & 4,600 (54.34\%) & 17 \\
% Whatsapp & US & 11,521 & 8,264 (71.73\%) & 3,163 (27.45\%) & .com & 2,363 (28.59\%) & 162 & .com & 1,198 (37.88\%) & 104 \\
% Apple & US & 11,253 & 8,942 (79.46\%) & 2,056 (18.27\%) & .com & 3,385 (37.86\%) & 234 & .com & 918 (44.65\%) & 138 \\
% Instagram & US & 11,181 & 7,337 (65.62\%) & 3,681 (32.92\%) & \cellcolor{bubblegum}\textbf{.ml} & 1,482 (20.20\%) & 212 & .com & 982 (26.68\%) & 165 \\
% Naver & KOR${^\dagger}$ & 11,030 & 7,207 (65.34\%) & 3,725 (33.77\%) & .com & 2,506 (34.77\%) & 269 & .com & 1,549 (41.58\%) & 195 \\
% Amazon & US & 9,473 & 7,390 (78.01\%) & 2,000 (21.11\%) & .com & 2,086 (28.23\%) & 192 & .com & 996 (49.80\%) & 110 \\
% \midrule
% \multicolumn{11}{l}{${^\ast}$: Most common TLD in brands. ${^\S}$: \# of unique TLD in brand. ${^\dagger}$: GER: Germany, RUS: Russia, KOR: Korea.}
% % \multicolumn{11}{l}{${^\ddagger}$: Using other TLD (\eg, \textcolor{magenta}{\textbf{.top}}, \textcolor{magenta}{\textbf{.tk}}, \textcolor{magenta}{\textbf{.ml}}) than the origin of its brand (\eg, \textcolor{dkgreen}{\textbf{.com}}, \textcolor{dkgreen}{\textbf{.ru}})} 
% \end{NiceTabular}
% }
% \end{table*}

% \subsection{Impersonated Brand}
% Impersonating trusted brands is a cornerstone of phishing attacks, as it enhances the credibility of malicious domains and increases the likelihood of victim engagement. 
% Attackers often create websites that mimic legitimate, well-known websites to deceive users into accessing their phishing websites.
% We refer to these domains as impersonated brand domains.
% \DK{what's the def of impersonated brand domain?} \KL{Added}

% \PP{Targeted Brand}
% For this analysis, we utilize target brand information from the APWG dataset.
% In our analysis, we identify a diverse range of targeted brands, with \cc{Facebook} standing out as the most impersonated brand, followed by \cc{USPS} as shown in~\autoref{tab:impersonated_brand}. 
% These two brands alone account for a significant portion (\DK{number and \%}) of phishing domains, reflecting their widespread recognition and trust among users. 
% This variety highlights how attackers prioritize brands that are universally trusted or frequently used for online transactions, where the likelihood of phishing success is higher. 

% \autoref{tab:impersonated_brand} reveals key patterns in the impersonation of top brands by phishing domains, providing a detailed breakdown of how attackers strategically exploit TLDs and domain registration types. 
% \cc{Facebook} is the most impersonated brand, with 66,700 phishing domains, of which 58.20\% are maliciously registered. 
% \cc{USPS} and \cc{Microsoft} follow, with 41,691 and 26,717 domains, respectively. 
% Notably, \cc{USPS} exhibits an exceptionally high proportion of maliciously registered domains (90.03\%), indicating that attackers targeting this brand prefer creating new domains rather than compromising existing ones. 
% In comparison, Microsoft demonstrates a more balanced split, with 51.21\% malicious registrations and 46.55\% compromised domains, suggesting a dual approach in leveraging both new and existing infrastructures.
% Attackers’ TLD choices further highlight their strategic intent. 
% USPS-targeted domains heavily rely on \cc{.top}, which accounts for 41.27\% of its phishing domains, diverging from the original \cc{.com} TLD used by the legitimate brand. 
% Similarly, \cc{OZON}, ranked as the fifth most impersonated brand, sees 44.39\% of its phishing domains registered under \cc{.tk}, a previously free TLD, instead of its native \cc{.ru}. 
% This shift to alternative TLDs, particularly those known for low costs or lenient registration policies, reflects attackers’ attempts to reduce expenses while maintaining scalability in their campaigns. 
% By contrast, compromised domains often align with the brand’s original TLD, leveraging its perceived legitimacy to deceive users and evade detection. 
% For instance, phishing domains targeting Microsoft frequently use \cc{.com}, making detection more challenging due to their resemblance to genuine brand domains.

% This approach allows them to reduce costs, evade detection, and target a broad spectrum of victims across industries and regions. 
% These patterns demonstrate the need for targeted countermeasures to address the specific risks posed by impersonation of highly trusted brands.

% Impersonating trusted brands is a cornerstone of phishing attacks, as it enhances the credibility of malicious domains and increases the likelihood of victim engagement. 
% In our analysis, we identify a diverse range of targeted brands, with global giants such as Facebook, Microsoft, and USPS being among the most frequently impersonated. 
% Attackers strategically select TLDs and naming patterns to mimic these brands, leveraging both well-known TLDs like .com to enhance legitimacy and alternative TLDs like .xyz or .tk for cost efficiency. 
% This section delves into the TLD choices, domain naming conventions, and detection timelines associated with impersonated brands, providing insights into how attackers tailor their strategies to exploit specific targets effectively.

% As shown in~\autoref{tab:impersonated_brand}, Facebook is the most targeted brand followed by USPS.
% \KL{Need more detail}

% \begin{figure*}[!t]
%     \centering
%     \includegraphics[width=1\linewidth]{fig/registration_by_brand_wide_02.pdf}
%     \vspace{-20px}
%     \caption{Days Between Registration and Detection by Top 10 Brand.
%     }
%     \label{fig:registration_by_brand}
% \end{figure*}

% \PP{Using Different TLD than Original Brand Domain}
% Attackers strategically craft phishing domains to mimic these brands, often leveraging alternative TLDs, such as \cc{.top} or \cc{.tk}, rather than the brands' original TLDs like \cc{.com}. 

% Impersonated brands do not always use the same TLD as their original domains. For instance, while \cc{Facebook}’s legitimate domain uses \cc{.com}, phishing attackers often register \cc{Facebook}-targeted domains using alternative TLDs such as \cc{.top}. 
% Similarly, \cc{USPS}, the second most impersonated brand in our analysis, is frequently targeted using \cc{.top} domains instead of the brand’s original \cc{.com}. 
% Another example is \cc{OZON}, ranked as the 5th most impersonated brand, with 49.42\% of its phishing domains registered under the \cc{.tk} TLD instead of its original \cc{.ru}. 
% Both \cc{.top} and \cc{.tk} are significantly cheaper than \cc{.com} for registration, with \cc{.tk} previously offered for free by Freenom until January 2023. 
% These findings suggest that attackers often choose low-cost TLDs to minimize expenses while maximizing the scalability of their phishing campaigns.
% \DK{experiment -> conclusion, is weak}



% The patterns observed also highlight distinct attacker strategies tailored to specific brands. 
% For USPS, the overwhelming use of \cc{.top} indicates a clear preference for registering new domains under cost-effective or minimally regulated TLDs. 
% For brands like Microsoft, attackers exploit both compromised domains and new registrations, balancing the use of \cc{.com} for compromised domains and alternative TLDs for maliciously registered ones. 
% Meanwhile, the reliance on \cc{.tk} for OZON-targeted domains demonstrates how attackers adapt to exploit lenient TLD policies, with a sharp decline in \cc{.tk} usage expected following Freenom’s discontinuation of free registrations in 2023.

% These findings underscore the varying tactics attackers use when targeting different brands, driven by the characteristics of the brand, the TLD landscape, and the cost-effectiveness of registration. 
% For brands like USPS, which show a high share of malicious registrations, detection efforts should focus on monitoring non-standard TLDs such as \cc{.top}. 
% Conversely, for brands like Microsoft, where compromised domains represent a substantial share, monitoring activity within original TLDs such as \cc{.com} is critical. 
% Additionally, the heavy reliance on low-cost TLDs like \cc{.tk} and \cc{.xyz} across many brands highlights the need for stricter policies and oversight for leniently monitored TLDs to disrupt attackers’ operations. 
% This analysis underscores the importance of adapting detection and enforcement strategies to the specific behaviors and patterns associated with individual brands and their targeted domains.

% \PP{Detection Time between Targeted Brands}
% As shown in~\autoref{fig:registration_by_brand}, USPS and OZON have the fastest average detection times, at 59 and 42.9 days, respectively. 
% Notably, both brands are targeted using non-original TLDs, which are often cheaper to register. 
% Previous research~\cite{affinito2022domain} indicates that some registrars, such as Freenom, provide APIs for the immediate takedown of phishing domains upon detecting signs of abuse. 
% This suggests that attackers' choice of cost-effective TLDs may have inadvertently backfired, as these domains are detected more quickly than others.
% % USPS and OZON are detected the fastest, with average detection times of 59 and 42.9 days, respectively. 
% % Interestingly both brands are using non-original TLD. 
% % Previous research~\cite{affinito2022domain} highlights that some registrars, such as Freenom, offer APIs to facilitate the immediate takedown of phishing domains upon detecting signs of abuse.
% In contrast, Microsoft experiences the longest detection delays, with an average of 719 days, reflecting either more sophisticated evasion tactics or less stringent monitoring for its targeted domains. 
% Although Facebook is the most frequently impersonated brand, it does not benefit from rapid detection either, with an average detection time of 275 days. 
% These findings highlight significant disparities in detection efficiency across brands and TLDs, emphasizing the impact of attackers' TLD choices on detection timelines.
% % This shows that attackers may have chosen to use a cheaper TLD to register a domain, however that backfired on their domain since those domains gets detected sooner than others.
% % In contrast, Microsoft experiences the longest detection delay, averaging 719 days. 
% % Although Facebook is the most frequently impersonated brand, it does not benefit from quick detection, with an average detection time of 275 days.



% % We utilize the Levenshtein distance to find similar domain names.
% % From previous work~\cite{maroofi2020comar}, having a Levenshtein distance less than 1 can be considered a similar domain.
% % We find that 7.9\% (54,787/689,492) of domains have Levenshtein distance of 1 or less.
% % From there, we also look at the registration timestamp and registrars.
% % We find that 20.1\% (11,016/54,787) domains are registered within a same time (same time between more than two domains) and also registered through the same registrars.

% \rtbox{
% \textbf{Takeaway:} 
% Phishing domains from our dataset often use TLDs different from the original brand’s TLD, such as USPS being targeted primarily with \cc{.top} instead of \cc{.com}, and OZON with \cc{.tk} instead of \cc{.ru}. 
% Detection times also vary significantly by targeted brand; USPS and OZON are detected the fastest, while Microsoft experiences the longest detection delays.
% }

% \begin{table}[t]
% \caption{Impersonated Brands (Top 10).\KL{Will be changed (comparison between mal vs comp}}
% \label{tab:impersonated_brand}
% \resizebox{0.98\linewidth}{!}{ 
% \begin{NiceTabular}{r l l r l r r}
% \toprule
% \multicolumn{1}{c}{\textbf{\#}} & \multicolumn{1}{c}{\textbf{Brands}} & \multicolumn{1}{c}{\textbf{Country}} & \multicolumn{1}{c}{\textbf{\# of Domain}} & \multicolumn{1}{c}{\textbf{TLD${^\ast}$}} & \multicolumn{1}{c}{\textbf{TLD Count}} & \multicolumn{1}{c}{\textbf{TLD${^\S}$}}\\
% \midrule
% 1 & Facebook & US  & 60,295 (9.41\%) & com & 22,016 (36.51\%) & 386 \\
% 2 & USPS & US  & 34,779 (5.43\%) & \textcolor{red}{top}${^\dagger}$ & 12,604 (36.24\%) & 230 \\
% 3 & Microsoft & US  & 24,198 (3.78\%) & com & 12,360 (51.08\%) & 409 \\
% 4 & DHL & GER${^\ddagger}$  & 20,597 (3.21\%) & com & 9,374 (45.51\%) & 409 \\
% 5 & OZON & RUS${^\ddagger}$  & 18,514 (2.89\%) & \textcolor{red}{tk}${^\ddagger}$ & 9,149 (49.42\%) & 34 \\
% 6 & Instagram & US  & 10,921 (1.70\%) & com & 2,437 (22.31\%) & 197 \\
% 7 & Naver & KOR${^\ddagger}$  & 10,796 (1.68\%) & com & 4,097 (37.95\%) & 230 \\
% 8 & Apple & US  & 10,565 (1.65\%) & com & 4,436 (41.99\%) & 231 \\
% 9 & WhatsApp & US  & 10,062 (1.57\%) & com & 3,264 (32.44\%) & 155 \\
% 10 & Amazon & US  & 9,159 (1.43\%) & com & 3,086 (33.69\%) & 172 \\
% \bottomrule
% \multicolumn{7}{l}{${^\ast}$: Most common TLD in brands. ${^\S}$: \# of unique TLD in brand.} \\
% \multicolumn{7}{l}{${^\dagger}$: Using other TLD (\eg, \textcolor{red}{.top}, \textcolor{red}{.tk}) than the origin of its brand (\eg, .com, .ru)} \\
% \multicolumn{7}{l}{${^\ddagger}$: GER: Germany, RUS: Russia, KOR: Korea.} 
% \end{NiceTabular}
% }
% \end{table}


\begin{figure*}[!t]
\centering
    \begin{subfigure}{0.49\textwidth}
        \includegraphics[width=\linewidth]{fig/registration_detection_avg_02.pdf}
        \vspace{-20px}
        \caption{Average Delays (days) Between Registration and Detection.}
        \label{fig:registration_detection_avg}
    \end{subfigure}
    \begin{subfigure}{0.49\textwidth}
        \includegraphics[width=\linewidth]{fig/registration_detection_median_02.pdf}
        \vspace{-20px}
        \caption{Median Delays (days) Between Registration and Detection.}
        \label{fig:registration_detection_median}
    \end{subfigure}
    \vspace{-5px}
    \caption{Delays (days) Between Registration and Detection. Vertical bars show average (or median) days between registration time and detection time of the top 5 most targeted brands.}
    \label{fig:registration_timediff}
    % \vspace{-10px}
\end{figure*}



\subsection{DNS Records}
% \KL{Most used CNAME, NS, MX, TXT}
% \KL{Compare between All vs Mal}
% \KL{Change of A record over time}
% \KL{TTL less than 60 seconds? --> fast-flux dns}
%To understand the characteristics of maliciously registered domains, we look at DNS records of maliciously registered domains. 
We characterize the DNS records of maliciously registered domains collected by our DNS crawler.

% \KH{Is there any comparison between malicious vs. benign domain DNS patterns??}\KL{Just in malicioius}
\looseness=-1

\PP{DNS Records}
We study the values of
commonly used DNS record types
%frequently used values for each DNS record type
(\eg, \cc{A}, \cc{AAAA}, \cc{CNAME}, \cc{NS}, \cc{MX}, and \cc{TXT}). 
Phishing attackers often configure DNS records to evade detection, frequently altering them using techniques such as fast-flux DNS. 
Our analysis reveals that 21.4\% of domains exhibit record changes, with an average frequency of 79.4 days and a median of 125.2 days.
% \KH{States statistics but doesn't connect them to evasion techniques or compare against normal domain behavior patterns.}\KH{DNS record changes percentage appears in multiple places without clear connection to previous findings (64.3\% (here), then 63\% (contributions), then 86.0\% (TTL section)).}

%To gain deeper insights into DNS record changes, we analyze the DNS records of the collected phishing domains. 
We study the types of DNS records configured in phishing domains.
\cc{NS} records were the most common, with a total of 51,459, followed by \cc{A} records (13,218), \cc{SOA} records (8,960), and \cc{TXT} records (5,573).
Focusing specifically on maliciously registered domains, we specifically examined those that exhibited DNS record changes. 
Our analysis shows that only 4.6\% of these domains (117 out of 2,550) demonstrated record changes over time. 
This suggests that modifying DNS records is not a commonly used tactic among maliciously registered phishing domains.
\looseness=-1

To understand the scenarios behind DNS record changes, we manually reviewed domains that exhibited such changes over time. 
One common case involved NS record changes, where domains shifted from one DNS provider to another (\eg, from Cloudflare to Google). 
%Our analysis suggests that these 
Such changes are often motivated by the desire to leverage specific services offered by different DNS providers. 
For instance, attackers may switch to providers like Cloudflare to utilize features, such as free SSL certificates, which are available for a limited duration~\cite{Cloudfla20:online}.

Our analysis reveals that phishing domains show a strong preference for hosting on Amazon Web Services (AWS) infrastructure. 
Specifically, we extract all IP addresses associated with \cc{A} records and utilize the \cc{Summarize IP} feature provided by IPinfo~\cite{IPSummar22:online} to gain insights into their hosting characteristics. 
Among the Autonomous System Numbers (ASNs) analyzed, AS16509 (Amazon.com, Inc.), a primary ASN for AWS, hosts 81.2\% of the phishing domains, while an additional 15.4\% are hosted on AS14618 (Amazon.com, Inc.), another AWS-associated ASN. 
Combined, these two ASNs account for 96.6\% of all analyzed phishing domains, indicating a significant reliance on AWS services. 
This preference may be attributed to AWS’s scalability, cost-effectiveness, and global reach, which make it an attractive option for attackers to host phishing domains. 
In comparison, other hosting providers, such as Google LLC (1.5\%), JSC Selectel (0.2\%), and DigitalOcean, LLC (0.1\%), host far fewer phishing domains. 
% These findings highlight the disproportionate use of AWS infrastructure by phishing operators, likely due to its flexibility and ease of use.
% We analyze \cc{A} records to examine the IP address characteristics of phishing domains. 
% Specifically, we extracted all IP addresses associated with \cc{A} records and utilized the \cc{Summarize} \cc{IP} feature provided by IPinfo~\cite{IPSummar22:online} for further insights. 
% Our analysis revealed that 3 out of 13,218 \cc{A} records were configured with bogon IP addresses, which are in either reserved or unassigned IP spaces ~\cite{BogonIPA97:online}.
% These instances could indicate misconfigurations or other unknown factors in the association of IP addresses with domains.
% Top ASN: 
% AS16509 Amazon.com, Inc. 81.2%
% AS14618 Amazon.com, Inc. 15.4%
% AS15169 Google LLC 1.5%
% AS49505 JSC Selectel 0.2% 
% AS14061 DigitalOcean, LLC 0.1%
% Top countries:
% US: 94\%
% Brazil 1.7\%
% India 1.5\%
% Ireland 0.6\%
% Russia 0.3\%
% \cc{AS16509} Amazon.com, Inc.
% \RM{another other results? e.g., AS type? country?}
% \begin{figure*}[!t]
%     \centering
%     \includegraphics[width=1\linewidth]{fig/registration_detection_wide.pdf}
%     \vspace{-20px}
%     \caption{Average Delays (days) Between Registration and Detection. Vertical bars show an average time between registration time and detection time of the top 5 most targeted brands. \DK{Put the median values.} \DK{we can have another figure for median values.} \KL{Add another graph with median value}}
%     \vspace{-10px}
%     \label{fig:registration_timediff}
% \end{figure*}

% \begin{figure*}[!t]
% \centering
%     \begin{subfigure}{0.48\textwidth}
%         \includegraphics[width=\linewidth]{fig/registration_detection_avg_02.pdf}
%         \caption{Average Delays (days) Between Registration and Detection.}
%         \label{fig:registration_detection_avg}
%     \end{subfigure}
%     \begin{subfigure}{0.48\textwidth}
%         \includegraphics[width=\linewidth]{fig/registration_detection_median_02.pdf}
%         \caption{Median Delays (days) Between Registration and Detection.}
%         \label{fig:registration_detection_median}
%     \end{subfigure}
%     \caption{Delays (days) Between Registration and Detection. Vertical bars show an average time between registration time and detection time of the top 5 most targeted brands.}
%     \label{fig:registration_timediff}
% \end{figure*}

\PP{Vantage Point of DNS Server}
%\KL{check}
Phishing attackers can configure location-aware DNS responses.
%to vary based on the victim's location. 
This allows attackers to deliver localized phishing content (\eg, Spanish-language phishing pages for victims in South America) or to evade detection by serving benign pages when accessed from certain locations commonly used by detection systems.

Our preliminary analysis shows that some phishing domains adapt their content to different languages based on the location of the user accessing them. 
However, we do not find any evidence that these phishing domains alter their DNS records based on the vantage point of the queried DNS servers. 
Instead, further investigation reveals that these domains implement language customization through client-side code rather than DNS configuration.

\PP{TTL in DNS Records}
In DNS records, the time-to-live (TTL) specifies how long DNS settings are cached before they are automatically refreshed. Typical TTL values are 12 or 24 hours, with recommended minimum and maximum values of 1 hour (3600 seconds) and 24 hours (86400 seconds), respectively~\cite{DNSTTLbe81:online}. 
%Our analysis reveals that 
% 86.0\% of phishing domains change their TTL values. \RM{change from what to what?}
%, while 14.0\% retain the same value. Additionally, 
2.1\% of the domains use TTL values less than 60 seconds, and 25.8\% use values shorter than 1 hour (3600 seconds) from our dataset. 
Only 2.9\% of the domains set TTLs longer than 12 hours, and among those, 31 domains set values between 12 and 24 hours. 
The median TTL value across domains is 3,994 seconds, while the average is significantly higher at 60,827 seconds. The use of short-lived TTLs can facilitate fast-flux DNS, a technique that frequently changes IP addresses to evade detection~\cite{bilge2011exposure,galloway2024practical} and often employ by
attackers ~\cite{dagon2008corrupted}.
%This behavior, commonly associated with malicious activity, has also been observed in previous research~\cite{dagon2008corrupted}, further highlighting its role in phishing campaigns.
\looseness=-1
% In DNS record, time-to-live (TTL) sets a duration of DNS settings are cached before they are refreshed automatically.
% The typical TTL value is 12 hours or 24 hours~\cite{DNSTTLbe81:online}.
% Minimum recommended values are 1 hour (3600) and maximum is 24 hours (86400)
% 86.0\% of domains change TTL values while 14.0\% keeps the same value.
% 2.1\% use TTL less than 60 seconds. 
% 25.8\% use TTL less than 1 hour (3600).
% 2.9\% set TTL over 12 hours.
% Median is 3994.
% Average is 60827.
% 31 domains set between 12 hours and 24 hours.

% This short-lived TTL can enable fast-flux DNS which changes IP address frequently.
% ~\cite{bilge2011exposure,galloway2024practical}
% Using short-lived TTL has been observed in previous work~\cite{dagon2008corrupted} 



% We uncover significant modifications in DNS record configurations, with a clear trend towards shorter time-to-live (TTL) durations. Our analysis reveals that 75.84\% of phishing websites alter their DNS settings between discovery and takedown. These values span from 0 to 604,800 seconds, with an average of 1,559.87 seconds and a median of 295 seconds.
% Notably, the majority of these configurations are set to brief intervals: 81.44\% at discovery and 81.89\% at takedown are below 1800 seconds. Even more striking, 71.90\% fall under 600 seconds, and 51.10\% are less than 300 seconds. This tendency towards shorter durations intensifies over time, with the mean value decreasing from 1,705.63 seconds at discovery to 1,383.54 seconds at takedown. In extreme cases, some are set to 0 seconds, effectively disabling caching.
% These short-lived DNS configurations can enable fast-flux networks, a technique observed in sophisticated phishing operations~\cite{bilge2011exposure,galloway2024practical}. Such changes, particularly using brief and decreasing durations, align with known strategies for evading detection and complicating efforts to track and block malicious infrastructure~\cite{dagon2008corrupted}.


\rtbox{
\textbf{Takeaway:} 
Our analysis finds that 21.4\% of domains change their DNS records frequently.
2.1\% of phishing domains configure their DNS TTL values to less than 60 seconds, a configuration commonly associated with fast-flux DNS techniques.
% Our results reveal that \hl{XX} domains exhibit changes in their DNS \hl{XX} domains that are configured to provide different DNS records depending on the location of the request.
}




% For phishing attackers to monetize as much as possible, their phishing websites should stay available as long as possible.
% We want to see how phishing attackers register their domains and see how long it takes to get detected by blocklists (\eg, APWG). 

% \subsection{Trend of Phishing Domain}

% \begin{figure}[!t]
%     \centering
%     \includegraphics[width=1\linewidth]{fig/tld_by_year.pdf}
%     \caption{TLD by Year.\KL{need a better color and sizing}}
%     \label{fig:tld_by_year}
% \end{figure}

% We first measure how phishing attackers deploy phishing domains and which TLDs are widely used.
% \KL{Add phishing domain over time graph}
% \KL{Add top TLD}


% % \begin{table}[t]
% % \caption{TLD by Year (Top 10).}
% % \label{tab:dataset}
% % \resizebox{0.98\linewidth}{!}{ 
% % \begin{NiceTabular}{r r r r r r r r r r r}
% % \toprule
% % \multicolumn{1}{c}{\textbf{Year}} & \multicolumn{1}{c}{\textbf{.com}} & \multicolumn{1}{c}{\textbf{.top}} & \multicolumn{1}{c}{\textbf{.xyz}} & \multicolumn{1}{c}{\textbf{.shop}} & \multicolumn{1}{c}{\textbf{.tk}} & \multicolumn{1}{c}{\textbf{.cn}} & \multicolumn{1}{c}{\textbf{.ml}} & \multicolumn{1}{c}{\textbf{.online}} & \multicolumn{1}{c}{\textbf{.net}} & \multicolumn{1}{c}{\textbf{.info}} \\
% % \midrule
% % Total & 211,494 & 75,272 & 35,901 & 26,150 & 22,453 & 20,460 & 14,153 & 13,951 & 12,607 & 9,584 \\
% % \midrule
% % 2021 & 31,121 & 5,742 & 6,531 & 1,392 & 992 & 2,111 & 977 & 935 & 2,132 & 1,264 \\
% % 2022 & 84,705 & 23,406 & 18,835 & 4,323 & 17,932 & 10,034 & 12,256 & 6,734 & 5,343 & 3,832 \\
% % 2023 & 63,995 & 33,241 & 8,136 & 8,457 & 3,425 & 3,641 & 916 & 4,829 & 3,646 & 3,036 \\
% % 2024 & 31,673 & 12,883 & 2,399 & 11,978 & 104 & 4,674 & 4 & 1,453 & 1,486 & 1,452 \\
% % \bottomrule
% % \end{NiceTabular}
% % }
% % \end{table}

% \begin{table}[t]
% \caption{TLD by Year (Top 10).}
% \label{tab:dataset}
% \resizebox{0.98\linewidth}{!}{ 
% \begin{NiceTabular}{r r r r r r r r c}
% \toprule
% \multicolumn{1}{c}{\textbf{Year}}  & \multicolumn{1}{c}{\textbf{Total}} & \multicolumn{1}{c}{\textbf{2021}} & \multicolumn{1}{c}{\textbf{2022}} & \multicolumn{1}{c}{\textbf{2023}} & \multicolumn{1}{c}{\textbf{2024}} & \multicolumn{1}{c}{\textbf{Price${^\ast}$}} & \multicolumn{1}{c}{\textbf{Types${^\dagger}$}} & \multicolumn{1}{c}{\textbf{Freenom}} \\
% \midrule
% \multicolumn{1}{c}{\textbf{.com}}  & 211,494 & 31,121 & 84,705 & 63,995 & 31,673 & 6 & gTLD & No \\
% \multicolumn{1}{c}{\textbf{.top}}  & 75,272 & 5,742 & 23,406 & 33,241 & 12,883 & 1 & gTLD & No \\
% \multicolumn{1}{c}{\textbf{.xyz}}  & 35,901 & 6,531 & 18,835 & 8,136 & 2,399 & 1 & gTLD & No \\
% \multicolumn{1}{c}{\textbf{.shop}}  & 26,150 & 1,392 & 4,323 & 8,457 & 11,978 & 1 & gTLD & No \\
% \multicolumn{1}{c}{\textbf{.tk}}  & 22,453 & 992 & 17,932 & 3,425 & 104 & 7 & gTLD & Yes \\
% \multicolumn{1}{c}{\textbf{.cn}}  & 20,460 & 2,111 & 10,034 & 3,641 & 4,674 & 5 & ccTLD & No \\
% \multicolumn{1}{c}{\textbf{.ml}}  & 14,153 & 977 & 12,256 & 916 & 4 & 12 & gTLD & Yes \\
% \multicolumn{1}{c}{\textbf{.online}}  & 13,951 & 935 & 6,734 & 4,829 & 1,453 & 1 & gTLD & No \\
% \multicolumn{1}{c}{\textbf{.net}}  & 12,607 & 2,132 & 5,343 & 3,646 & 1,486 & 10 & gTLD & No \\
% \multicolumn{1}{c}{\textbf{.info}} & 9,584 & 1,264 & 3,832 & 3,036 & 1,452 & 2 & gTLD & No \\
% \bottomrule
% \multicolumn{9}{l}{${^\ast}$: Cost to register a domain in each TLD~\cite{CompareP38:online}. ${^\dagger}$: gTLD vs ccTLD}
% % \multicolumn{9}{l}{${^\dagger}$: gTLD vs ccTLD} \\
% \end{NiceTabular}
% }
% \end{table}

% \subsection{Time Taken between Domain Registration to Detection}

% see how long it takes for phishing domains to detect after registration
% Registration information stats (time, trend, registrar)
% Average: 8 days 20 hours 11 minutes 8.14 seconds
% Max: 35 days 1 hour 6 minutes

% see how long it takes for phishing TLDs to detect after registration
% Registration information stats (time, trend, registrar)
% Average: 5 days 21 hours 21 minutes 37.38 seconds
% Max: 35 days 1 hour 6 minutes



% \subsection{Takendown Domains}
% \KL{Add Deregistration result}

