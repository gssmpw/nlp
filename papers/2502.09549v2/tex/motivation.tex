\section{Problem Statement}
\label{sec:motivation}


% \MJ{I think there's quite a bit of overlap/replication from the intro. I wonder if a dedicated problem statement section is altogether needed given that the problem statement, delta/novelty over related work, and research questions were all already presented to readers in the introduction.}

% Goal: How can we take a step further to detect phishing at the DNS level?
% , with attackers constantly evolving their tactics to evade detection and maximize victim reach. 
Phishing remains a major security threat, with traditional blocklist-based defenses (\eg, Google Safe Browsing---Google Chrome default anti-phishing system) suffering from significant detection delays. These systems often take hours or days to update after new phishing domains are registered, creating a critical window of vulnerability during which attackers can successfully target victims~\cite{lin2021phishpedia,liu2022inferring,liu2023knowledge,liu2024less,oest2019phishfarm,oest2020phishtime}.

Addressing phishing at the DNS level---when domains are first registered---is crucial, as domains are the primary gateway to phishing websites. However, while previous studies~\cite{maroofi2020comar,moura2024characterizing,hao2013understanding,oest2019phishfarm,oest2020phishtime,oest2020sunrise}  have focused on URL patterns, visual content, and blocklist data, there is limited understanding of how attackers exploit DNS registration strategies for phishing attacks.
To this end, our work aims to bridge this knowledge gap by focusing on maliciously registered domains for phishing attacks and their abuse of DNS systems. 
Particularly, we seek to answer our research questions through our analysis using our dataset of phishing domains: 
\textbf{RQ1:} \textit{What are the characteristics of maliciously registered domains, and how can we find maliciously registered domains?} and 
\textbf{RQ2:} \textit{What is the lifecycle of a maliciously registered domain?}.
% \textbf{RQ1:} \textit{How can we effectively identify maliciously registered domains?},
% \textbf{RQ2:} \textit{What are the characteristics of maliciously registered domains?},
% \textbf{RQ3:} \textit{What is the lifecycle of a maliciously registered domain?}.
% By providing insights into these tactics, we seek to improve phishing detection mechanisms, particularly for domains that are maliciously registered, and employ DNS-based evasion strategies.

% Phishing continues to be a significant security threat.
% The predominant defense against phishing attacks, blocklist-based systems (\eg, Google Safe Browsing---Google Chrome default anti-phishing system), suffer from substantial operational delays. These systems frequently require updating hours or days after attackers register new phishing domains and launch phishing attacks~\cite{lin2021phishpedia,liu2022inferring,liu2023knowledge,liu2024less,oest2019phishfarm,oest2020phishtime}. 
% This latency creates a critical vulnerability window during which phishing sites remain active, leading to significant financial losses for victims.
% \looseness=-1



% It is essential to address the fundamental problem at the root level---DNS level, particularly phishing domains, before phishing attacks are launched (\ie, when phishing domains are registered).
% Since domains serve as the essential bridge connecting users to phishing websites, they represent an optimal intervention point for preventing attacks before malicious actors can target potential victims.
% Unfortunately, the characteristics of maliciously registered domains for phishing attacks are little explored.
% Prior work~\cite{maroofi2020comar,moura2024characterizing,hao2013understanding,oest2019phishfarm,oest2020phishtime,oest2020sunrise} mainly focused on analyzing URL patterns, visual content, and historical blocklist data, often overlooking the dynamic DNS strategies that phishing attackers increasingly exploit.


% ~\cite{maroofi2020comar,moura2024characterizing}.



% The current primary defense system, blocklist-based mechanism (\eg, Google Safe Browsing~\cite{SafeBrow22:online}), can introduce a significant delay in blocking phishing websites; blocklists often taking hours or
% even days to update after a phishing domain is registered
% and launched\DK{cite}.
% This latency creates a critical vulnerability window during which phishing sites remain active, leading to significant financial losses for victims.



% The current reliance on blocklist mechanisms (\eg, Google Safe Browsing~\cite{SafeBrow22:online}) for phishing detection, while effective in blocking users from accessing phishing websites, suffers from inherent limitations. 
% These limitations primarily arise from detection delays, with blocklists often taking hours or even days to update after a phishing domain is registered and launched. 


% \DK{need to rewrite the below.} \KL{Updated, too long?}
% Despite extensive research on anti-phishing defenses, a substantial gap remains in understanding the behavior and characteristics of maliciously registered domains~\cite{maroofi2020comar,moura2024characterizing}. Current detection methods primarily focus on analyzing URL patterns, visual content, and historical blocklist data, often overlooking the dynamic DNS strategies that phishing attackers increasingly exploit. These strategies include leveraging specific DNS configurations, such as frequent record changes and short TTLs, to evade detection and delay mitigation efforts. Additionally, attackers frequently utilize low-cost TLDs (\ie, New-gTLD) or registrars with lenient and diverse registration policies (\ie, ccTLD), enabling them to register and sustain large volumes of phishing domains with minimal scrutiny.


% Despite extensive research on anti-phishing defenses, a substantial gap remains in understanding the maliciously registered domains.
% Current detection methods largely focus on analyzing URL patterns, and visual contents and relying on historical blocklist data, which does not account for the dynamic DNS behaviors that phishing attackers exploit. 
% Attackers increasingly leverage DNS configurations to evade detection, making blocklists less effective in addressing new phishing domains promptly. 
% Furthermore, phishing domains often use low-cost TLDs or registrars with lenient registration policies, making it easier to register and maintain large volumes of phishing domains without immediate detection.
% Our work aim to shed light on maliciously registered domains so that we can provide insight to help improve phishing detection when a domain is maliciously registered and abusing DNS.