\section{Background}
\label{sec:background}
We provide a brief overview of domain registration and DNS records, with an emphasis on phishing attacks.

% \begin{figure}[!t]
%     \centering
%     \includegraphics[width=1\linewidth]{fig/overview.pdf}
%     \caption{Overview of Our DNS Analysis on Phishing Domains.\KL{Change section as our current version of section names. Add number of filtered domain}\KH{How about change Section X to RQ X?}}
%     \label{fig:overview}
% \end{figure}

\subsection{Domain Registration and DNS Record}

\PP{Domain Registration}
Domain registration is the foundational process through which a unique domain name is acquired and associated with an individual or organization. 
This process involves selecting a domain name and choosing a top-level domain (TLD), such as \cc{.com}, \cc{.org}, or country-specific TLDs, such as \cc{.us} or \cc{.cn}. 
Once a domain is registered through a registrar, critical DNS records---such as A records, which link the domain to an IP address, and NS records, which designate authoritative name servers---are established to facilitate the Web services. 
The registry maintains the TLD's zone file, which includes delegation details for domains under that TLD. 
These zone files are updated in real-time or periodically by the registry as domain registrations and configurations change. 
Separately, organizations like ICANN collect published snapshots of these zone files at regular intervals (\eg, every 24 hours for gTLDs~\cite{HelpCent91:online}), though the exact frequency depends on the TLD administrator’s policies. 
It’s important to note that the frequency of published zone file snapshots is distinct from the registry’s internal updates to the zone file.

% Then, the registrar submits delegation information to the registry, who manages the corresponding TLD zone files. 
% These zone files are typically updated and collected at regular intervals, often every 12 hours, although the exact collection frequency may vary depending on the TLD administrator's policies.
% \MJ{I think you're conflating TLD zone updates with the update interval of published zone file snapshots here} 
% For example, ICANN collects gTLD zone files every 12 hours~\cite{HelpCent91:online}.

\PP{WHOIS}
Registration data is typically accessed through WHOIS or the Registration Data Access Protocol (RDAP). 
WHOIS has been the standard for retrieving domain registration information since the 1970s. 
However, due to its inconsistencies and limitations, RDAP was introduced in 2015 as its successor. RDAP improves upon WHOIS by offering structured, machine-readable registration data along with advanced features such as differentiated access, internationalization, and extensibility.
By examining the domain registration choices of phishing sites, including their TLD preferences, registrar selection, and DNS configuration, researchers can uncover patterns that may inform more effective, proactive detection methods against these evolving threats.
\looseness=-1

\PP{Top-level Domain (TLD)}
As described in prior work~\cite{AtLargeT9:online}, gTLDs can be categorized into legacy gTLDs and new gTLDs. New gTLDs refer to TLDs introduced as part of ICANN's expansion program in 2012.
Initially, there were only 8 gTLDs, and another 8 in 2004.
% (\cc{.com}, \cc{.edu}, \cc{.gov}, \cc{.mil}, \cc{.org}, \cc{.net}, \cc{.int}, \cc{.arpa}) 
% before the year 2000, followed by the addition of 7 more in 2000 (\cc{.aero}, \cc{.biz}, \cc{.coop}, \cc{.info}, \cc{.museum}, \cc{.name}, \cc{.pro}) 
% 
% (\cc{.asia}, \cc{.cat}, \cc{.jobs}, \cc{.mobi}, \cc{.tel}, \cc{.travel}, \cc{.xxx}). 
In 2012, ICANN launched the new gTLD program, which aimed to provide greater flexibility for registrants to create unique and innovative website names. This initiative also alleviated the overcrowding in the legacy gTLD market, offering more options for domain registration. Since the program's introduction, over a thousand new gTLDs have been delegated to the root zone, significantly expanding the domain name landscape.
\looseness=-1

\PP{DNS Record}
DNS records are fundamental components of the Domain Name System (DNS), serving as mappings that enable domain names to link to specific internet resources, such as IP addresses, email servers, and authoritative name servers. Each DNS record type provides unique information and functionality essential for domain operation. For instance, \cc{A} records (Address records) link a domain to an IPv4 address, directing users to the correct server when they access a website. \cc{NS} records (Name Server records) specify which servers are authoritative for a domain, manage DNS queries, and ensure accurate routing. 
% MX records (Mail Exchange records) designate mail servers responsible for receiving emails on behalf of a domain, a key aspect for domains involved in email-based phishing. 

DNS records may differ depending on the geographic or network location, known as the vantage points, from which the DNS query is made. This variation occurs because DNS configurations can be adapted to present different responses based on the requester’s location,
% a technique often used for legitimate purposes, 
such as content delivery optimization or load balancing. 




% However, phishing attackers also exploit this capability to evade detection by selectively serving malicious content. By querying DNS records from multiple global vantage points, researchers can detect discrepancies in DNS responses that may signal evasive behavior.
% \DK{I am not sure whether phishing attackers exploit it. } \KL{removed}

% \PP{Challenges in Traditional Detection Methods}




\subsection{Phishing Attack and Tactic}
Phishing attacks are a type of advanced social engineering where cybercriminals deceive victims into divulging sensitive information. 
Phishing attackers craft fake websites that closely resemble legitimate ones (\eg, Facebook or PayPal), deceiving victims into entering their credentials. 
% \DK{phishing attacks starts with registering domains and deploying websites}
% As phishing techniques continuously evolve, a relentless cat-and-mouse struggle ensues between attackers, who invent new ways to bypass detection, and security systems that depend on databases of previously identified phishing sites to stay ahead.

\PP{Phishing Tactics for Domain Registration}
The choice of the registrar and TLD can significantly impact a domain's visibility, cost, and accessibility, with certain TLDs (\eg, \cc{.tk} or \cc{.xyz}) often being cheaper or subject to less stringent registration requirements. 
Phishing attackers frequently take advantage of this aspect of domain registration, choosing low-cost or lenient TLDs to host their malicious sites in large numbers while minimizing expenses~\cite{moura2024characterizing}.
Additionally, some registrars (\eg,  \cc{Alibaba} \cc{Cloud}~\cite{AlibabaC38:online}) have minimal verification protocols, making it easier for attackers to quickly register multiple domains in bulk under anonymous or fabricated identities~\cite{NewUserD82:online}.
This practice allows attackers to operate on a large scale, using each domain temporarily until it is flagged or blocked by detection systems, then transitioning to newly registered domains.
\looseness=-1




\PP{Phishing Tactics for Domain Name}
When conducting phishing attacks, attackers employ various domain registration strategies to deceive users. 
They commonly use typosquatting, registering domains with subtle misspellings, such as \cc{paypaI}.\cc{com} (using a capital \cc{I} instead of \cc{L}) or missing letters such as \cc{goole}.\cc{com}.
Another tactic involves creating domain variations by adding words or modifying the structure, resulting in domains like \cc{paypal}-\cc{secure}-\cc{login}.\cc{com} or \cc{login}-\cc{paypal}.\cc{net}.
Attackers also abuse different top-level domains (TLDs), using alternatives like \cc{.co} or country-specific codes instead of the legitimate \cc{.com}.
\vspace{-10px}
% A more sophisticated approach is the homograph attack, where they register domains using similar-looking Unicode characters or mixed character sets—for example, using a Greek omicron in 'gοοgle.com' or Cyrillic characters in 'paypal.com'.


\begin{figure*}[!t]
    \centering
    \includegraphics[width=.98\linewidth]{fig/overview_v2_04.pdf}
    \vspace{-10px}
    \caption{Overview of Our DNS Analysis on Phishing Domains.}
    \label{fig:overview}
    % \vspace{-5px}
\end{figure*}
