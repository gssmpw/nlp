\section{Introduction}
\label{sec:intro}

Phishing attacks continue to pose one of the most pervasive cybersecurity threats, with attackers deploying increasingly sophisticated impersonation tactics. 
The attackers create convincing replicas of legitimate websites (\eg, \cc{facebook}.\cc{com} or \cc{USPS}.\cc{com}), to deceive users into divulging their login credentials and sensitive information.
Such attacks have substantial consequences, leading to financial losses for victims~\cite{IC3_Report_2024}, reputational harm for impersonated organizations~\cite{Reputational_damages}, and compromised business infrastructures~\cite{moura2024characterizing}.

% Phishing attacks remain a pervasive cyber threat, with a record number of 597,789 unique phishing in April 2023\DK{what specific number}\KL{Added highest number from report} of incidents reported in the second quarter of 2023~\cite{docsapwg72:online}.
% % ~\cite{docsapwg95:online}. 
% Phishing attackers generate deceptive websites (\eg, facebook.com or USPS.com) to lure victims into revealing their credentials.

% The significant rise in phishing attacks underscores the ongoing challenges posed by increasingly sophisticated phishing websites, phishing kits, and evasion techniques~\cite{add}. 


The blocklisting mechanisms (\eg, Google Safe Browsing~\cite{SafeBrow22:online}) currently serve as the primary defense against phishing attacks.
Google Safe Browsing is integrated into Google Chrome browsers and by default enabled for end-users.
These systems aim to protect users by preventing access to known (\ie, blocklisted) phishing websites. 
However, their reactive (\ie, passive) nature introduces critical security gaps in phishing protection.
The fundamental limitation of blocklists lies in their update latency---the time gap between when attackers register domains and deploy a new phishing site and when security crawlers detect, verify, and add it to the blocklist.
This delay creates a vulnerability window during which new phishing sites remain accessible to potential victims, allowing attackers to freely operate their campaigns.
Notably, a previous work~\cite{oest2020sunrise} indicated that 75\% of victims may encounter the malicious site before blocklist updates take effect.

% To effectively combat phishing attacks, it is essential to address the fundamental problem at the root---DNS (Domain Name System) level, particularly phishing domains.
To effectively combat phishing attacks, it is essential to address the fundamental problem at the root, particularly phishing domains.
Domains play a pivotal role in connecting users to websites, including malicious ones.
This critical position makes domains an ideal intervention point for detecting and preventing phishing attacks before they can reach potential victims.
% \KL{Add importance of analyzing Maliciously registered domain}
% Unfortunately, prior work
% \DK{prior work summary and limitation.}
% 
Particularly, phishing attackers can choose between two strategies for utilizing domain names: 1) registering a new domain specifically for malicious purposes; or 2) compromising an existing, legitimate website with an already established domain. 
Maliciously registered domains present a unique opportunity for mitigation at the domain level, as these domains are intentionally created to facilitate malicious activities. 

% \KL{summarize COMAR, ccTLD papers, start with specifically,... what previous papers looked, what has been missing }
% \KL{this area is not studied before even though it is important area...}
Prior studies~\cite{maroofi2020comar,moura2024characterizing,hao2013understanding,oest2019phishfarm,oest2020phishtime,oest2020sunrise} have explored various aspects of phishing websites, such as ccTLD, URL patterns and visual content.
Specifically, Moura et al.~\cite{moura2024characterizing} analyzed phishing domains mimicking target brand webpages but focused solely on three European ccTLDs: \cc{.nl}, \cc{.ie}, and \cc{.be}.
While Maroofi et al.~\cite{maroofi2020comar} introduced the methods to define maliciously registered domains,
the characteristics of maliciously registered domains for phishing attacks have been unexplored.
% the scope for expanding their approach remains limited 
% \DK{what limited?}\KL{updated, application? implication?}
Despite the importance of understanding the dynamic behaviors and lifecycle of maliciously registered domains, these aspects remain largely unexamined to date.\looseness=-1
% 
% Understanding how domains are registered and utilized by attackers is critical for mitigating phishing attacks. %Phishing attacks continue to be a significant cybersecurity threat, with attackers employing sophisticated impersonation tactics to deceive users.
% While prior studies~\cite{maroofi2020comar,hao2013understanding} have explored this criterion, the prevalence of maliciously registered domains continues to grow compared to compromised domains~\cite{Phishing18:online}. 
% Effectively combating phishing attackers and mitigating the registration of malicious domains requires a systematic understanding of how phishing domains are registered and the underlying strategies attackers employ.
%
% registering a malicious domain or compromising an existing website with an already established domain.
% A maliciously registered domain can be mitigated in domain level since such domains are registered for a malicious purpose.
% Previous works~\cite{add} have looked into this criteria however malicious domains still grows when compare to compromised domains~\cite{Phishing18:online}.
% To combat phishing attackers and mitigate registering malicious domains, we need a systematic understanding of how phishing attackers are registering domains maliciously.
%
% \KL{need to be more specific}

To address this gap, our study undertakes a systematic, longitudinal analysis of phishing attacks using phishing domains, with a focus on those that are maliciously registered domains. 
By examining these phishing domains from registration to detection and eventual deregistration, we aim to better understand the phishing attack ecosystem at the domain level. 
% Specifically, we identify maliciously registered phishing domains and analyze their distinctive characteristics, such as the abuse of specific TLDs, targeted brands, and configurations of DNS records. 
% Additionally, we investigate the lifespan of these maliciously registered domains, tracking their trajectory from registration to detection, and beyond, until deregistration.
% 
To further understand maliciously registered domains, we raise the two following research questions.
% \KL{remove rq1 and keep only rq2 and rq3}
% \textbf{RQ1:} \textit{How can we effectively identify maliciously registered domains?},
% , and what insights can be drawn from their characteristics and behaviors?
\textbf{RQ1:} \textit{What are the characteristics of maliciously registered domains and how can we find maliciously registered domains?}
\textbf{RQ2:} \textit{What is the lifecycle of a maliciously registered domain?}
% Additionally, we analyze DNS records for evasion tactics, such as frequent DNS record changes and location-based response manipulation, which attackers use to delay blocklisting and prolong attack effectiveness.
% design \textit{proactive} defensive measures at the root---DNS (Domain Name System) level.
% A comprehensive measurement study is essential to develop effective DNS-level defenses.
% In this paper, we address 

% The prevalence defense against such severe phishing attacks is blocklists (\eg, Google Safe Browsing).
% the blocklisting mechanism is block end-users from accessing phishing websites in the blocklists.
% However, the passive blocklisting mechanism suffers from inherent limitations due to update delays; these delays can leave phishing sites accessible to potential victims for hours or even days after they are registered and active. 
% Previous work~\cite{oest2020sunrise} indicates that as much as 75\% of victims may access phishing sites before blocklists are updated, especially in the case of targeted attacks like spear phishing. 
% This delay creates a critical vulnerability window that attackers exploit, reducing the efficacy of blocklists in safeguarding users from new and evolving phishing sites.

% This passive defense system has inherent limitation, which can lead to vulnerable windows for victims. 


% To launch a phishing attack, attackers create a fraudulent website and register it with a domain name. 
% The domain name often serves as an indicator of the website's identity. 
% To deceive victims, attackers frequently incorporate well-known brand names into the domain name (\eg, Facebook or USPS), making the phishing site appear legitimate and trustworthy.
% Phishing attackers can choose between two strategies for utilizing domain names: registering a malicious domain or compromising an existing website with an already established domain. 
% To improve detection and reduce delays, identifying maliciously registered domains is crucial.

% To counter phishing, the industry primarily relies on blocklists, which are generally effective in identifying known phishing sites~\cite{lin2021phishpedia,liu2022inferring,liu2023knowledge,li2024knowphish,liu2024less}. 
% However, blocklisting suffers from inherent limitations due to update delays; these delays can leave phishing sites accessible to potential victims for hours or even days after they are registered and active. 
% Previous work~\cite{oest2020sunrise} indicates that as much as 75\% of victims may access phishing sites before blocklists are updated, especially in the case of targeted attacks like spear phishing. 
% This delay creates a critical vulnerability window that attackers exploit, reducing the efficacy of blocklists in safeguarding users from new and evolving phishing sites.

% To mitigate detection delays, previous works have explored several enhancements to blocklisting and phishing detection methods, including machine learning-based analysis~\cite{lin2021phishpedia,liu2022inferring,liu2023knowledge,li2024knowphish,liu2024less}, reference-based detection~\cite{liu2023knowledge,li2024knowphish,liu2024less} models, and image-based detection~\cite{lin2021phishpedia,liu2022inferring}.
% Although these advancements have shown promise, they largely overlook the role of DNS characteristics in phishing detection. 
% Attackers increasingly leverage DNS-based evasion techniques~\cite{moura2024characterizing}, such as using impersonated brand names in a domain, rapid DNS record updates, location-specific responses, and the use of low-cost or lenient TLDs, which current blocklisting enhancements often fail to address effectively. 
% Consequently, attackers continue to exploit these DNS behaviors to bypass detection, exposing a critical gap in existing defenses.

% Addressing this limitation is essential, as DNS-based evasion tactics enable attackers to extend the active lifespan of phishing domains. 
% By focusing on phishing detection at the DNS level, we aim to reduce detection delays and close the critical vulnerability window that attackers exploit. 
% Analyzing the DNS characteristics of phishing domains offers insights that not only improve detection speed but also strengthen anti-phishing defenses.
 

% \hl{Our analysis reveals critical insights into the tactics and longevity of phishing domains. 
% Our study offers critical insights into the ecosystem of maliciously registered phishing domains, highlighting their unique characteristics and lifecycle. 
Our analysis shows that 66.1\% of all names in our phishing domains dataset are specifically registered for malicious purposes.  
To better understand these malicious domains, we examine their characteristics, focusing on TLD usage and targeted brands. 
We observe that new gTLDs (\eg, \cc{.top}, \cc{.shop}) are widely utilized due to their low cost (as little as \$1 per domain). 
Following the cessation of Freenom, the use of the \cc{.cn} TLD increased significantly. 
Notably, the \cc{USPS} brand experienced a sharp rise in domain registrations, frequently under cost-effective TLDs. The latter holds true for \cc{Ozon} as well.
Our observations align with prior research~\cite{Phishing18:online}.

To gain deeper insights into phishing domains with malicious registration activity, we analyze their DNS records, dynamic behavior, and lifespan, spanning from registration to detection and eventual deregistration. Our analysis reveals that phishing domains often exhibit dynamic DNS behavior, frequently updating their DNS records with short TTLs, indicative of fast-flux DNS techniques. Regarding lifespan, maliciously registered domains are detected slightly faster than compromised domains with a median detection time of 16.3 days for malicious domains compared to 86 days for compromised domains.
% \MJ{16.3 to 86 is considered ``marginal``??} \KL{removed marginal} 
On average, deregistration occurs approximately 11.5 days after detection. However, detection delays vary significantly across blocklists, with some domains listed in \cc{Phishing.Database} showing an average detection delay of up to 388.5 days.

% Our findings aim to provide actionable insights for identifying maliciously registered domains at the early stages of registration, potentially improving proactive defenses against phishing attacks.
% Our study provides critical insights into the ecosystem of maliciously registered phishing domains, emphasizing their distinct characteristics and lifecycle.
% Our analysis reveals that 66.1\% of the domains collected in our dataset are maliciously registered domains. 
% To further understand maliciously registered domain, we analyze characteristics of maliciously registered domains.
% From phishing domains we collected, examining the usage of TLDs and targeted brands, we find that new gTLDs (\ie, \cc{.top}, \cc{.shop}) are widely utilized due to their low cost (\$1 to register a domain). 
% Following the cessation of Freenom, there was a noticeable increase in the use of the \cc{.cn} TLD. 
% The \cc{USPS} brand showed a significant upward trend in domain registration, and brands like \cc{USPS} and \cc{Ozon} frequently used cheaper TLDs to register their domains. 
% This result aligns with the previous report~\cite{Phishing18:online}.
% Additionally, many of these domains exhibit dynamic behavior, frequently altering their DNS records with short TTLs, indicative of potential fast-flux DNS techniques. 
% In terms of lifespan, maliciously registered domains are detected slightly faster than compromised ones, but the difference is not substantial. 
% On average, domains are deregistered approximately 11.5 days after detection. 
% Notably, certain domains experience considerable detection delays depending on the blocklist used, with \cc{Phishing}.\cc{Database} showing an average detection delay of up to 388.5 days.
% With our findings we aim to provide insight to find maliciously registered domain in the early stage of domain registration.


% }\KH{Two consecutive sentences starting with nearly identical phrasing.} 
% We reveal that 66.1\% of phishing domains are maliciously registered from our collected dataset.
% , with attackers heavily exploiting low-cost or less-regulated TLDs such as .xyz and .top, which account for 45\% of all observed malicious registrations. Additionally, we find that 32\% of these domains target high-profile brands like \cc{Facebook}, \cc{USPS}, and the number of usage increases over the years leveraging their credibility to deceive victims. 
% Through DNS record analysis, we observe frequent abuse of specific configurations, such as TTL values under 3600 seconds, in 25.8\% of malicious domains, aiding attackers in dynamic infrastructure changes. 
% Furthermore, our longitudinal analysis highlights that the median lifespan of a maliciously registered domain is 11.7 days. 
% These findings underscore the need for proactive measures, such as stricter domain registration policies, enhanced DNS monitoring, and collaborative efforts to disrupt phishing domains early in their lifecycle.

% Our contributions are as follows: 
% (1) a longitudinal analysis of maliciously registered phishing domains that use brand impersonation and squatting techniques, 
% (2) a detailed examination of the lifespan of phishing domains from registration to detection, with emphasis on variations across impersonated brands, 
% (3) an analysis of DNS record changes and location-based responses as evasion tactics, and 
% (4) actionable insights for developing DNS-based detection indicators that could reduce blocklisting delays and enhance anti-phishing defenses.
% Phishing continues to be a major threat, with a record number of incidents reported in the first quarter of 2024 ~\cite{}. 
% The surge in phishing attacks underscores the persistent challenges posed by increasingly sophisticated phishing websites, kits, and evasion techniques like cloaking~\cite{}. 
% These attacks not only lead to significant financial losses for victims but also damage the reputations of impersonated organizations and jeopardize business infrastructures~\cite{}.

% To counter phishing, the industry relies on server-side blocklists, which are widely used and generally accurate in identifying phishing sites with minimal false positives. 
% However, their effectiveness is constrained by update delays, allowing 75\% of victims to reach phishing sites before they are blocked~\cite{}. 
% Furthermore, blocklists are less effective against targeted attacks such as spear phishing~\cite{}. 
% The lag in updating blocklists creates a vulnerability window, highlighting the urgent need for more prompt solutions.

% While blocklists are widely studied in the literature on anti-phishing defenses~\cite{}, less research has focused on time difference between domain registration and blocklisted time.

% To overcome the current issue of blocklisting, we want to systematically measure the current state of phishing domains with their DNS characteristics.
% To further understand phishing domains, we have three research questions:
% \textbf{RQ1:} How can we effectively identify maliciously registered domains?
% % , and what insights can be drawn from their characteristics and behaviors?
% \textbf{RQ2:} What are the characteristics of maliciously registered domains?
% \textbf{RQ3:} What is the lifecycle of a maliciously registered domain?
% \textbf{RQ2:} What are the lifespan characteristics of maliciously-registered phishing domains from registration to detection, and how do these lifespans vary across different phishing tactics, such as impersonating specific brands?
% \textbf{RQ3:} How do phishing attackers configure DNS records strategically to evade detection and prolong the effectiveness of their attacks?
% RQ1: How can we find maliciously-registered domains? In order to detect phishing attack in DNS level, we need to differentiate maliciously-registered domain and compromised domain because compromised domain is not a phishing domain until an attacker takeover the benign domain.
% RQ2: To further understand a behavior of phishing domain, maliciously-registered domains, how is the lifespan of the phishing domain?  
% \textbf{RQ1. What are the detection timelines for phishing domains after registration, and how do these vary across different TLDs?}
% This question explores the time it takes for phishing domains to be flagged by detection mechanisms post-registration. We aim to examine how detection delays differ based on top-level domains (gTLDs and ccTLDs) and identify trends in domain lifespan across various TLDs. Additionally, we investigate the interval between domain registration and initial detection to better understand the persistence of phishing domains.\\
% % - How long do phishing domains live after registration?
% % - How long does it take to detect after the first visitor appears?
% % - Different lifecycle of phishing domains by TLD?
% % - Registration - Detection - Deregistration
% \textbf{RQ2. How do the characteristics of phishing domains’ DNS configurations, including TLD choices, registration methods, and hosting environments, influence their detection and lifespan?}
% Phishing domains can be either maliciously registered when the attacker start the phishing attack or compromise existing domains.
% Here, we want to focus on finding a maliciously registered domains since those are the ones that we can detect right after the registration of the domain.
% We focus on the DNS attributes of phishing domains, such as TLD preferences, the use of self-hosted versus web-hosted domains, and name server (NS) record configurations. We also explore whether bulk-registered domains or those using specific TLDs and registrars have longer lifespans. This question seeks to uncover how economic and structural choices (\eg, lower-cost TLDs) contribute to the longevity of phishing domains.
% % \textbf{RQ2. Which brand do phishing domains target and which targeted brand takes a long time to be detected? (merge with RQ1?)}
% % - For example, if Facebook lives longest (30 days), then what kind of characteristics do phishing domains have?
% % --> What kind of TLD do phishing domains use? If phishing domains use a different TLD than the original TLD, why? Because it is cheaper to register?
% % --> Do phishing domains use a self-hosted domain or a web-hosting domain?
% % --> What kind of NS record do phishing domains use?
% % \textbf{RQ3. What is the registration pattern of domains that take long time to be detected?}
% % - Do bulk registration allow long-lived domains?
% % - Maliciously registered domains live long? or domains using web-hosting service live long?
% % --> Does a specific TLD live long? 
% % --> Cheaper TLD live long? or short?
% % --> Do specific registrar-registered domains live long?
% \textbf{RQ3: What tactics do phishing attackers use to evade detection, and how do DNS record changes affect the detection timeline?}
% To understand evasion strategies, this question investigates whether phishing attackers modify DNS records (\eg, changing NS records) to avoid detection. We examine whether frequent DNS record changes correlate with longer detection times and whether attackers shift DNS configurations, such as redirecting to cheaper or less restrictive servers, to extend the domain’s lifespan.
% % - DNS record change to avoid detection? Do domains take longer to be detected if phishing domains change DNS records more frequently?
% % - Change record (e.g., NS) to the cheaper server? Or due to the policy of NS restricting something?

% \MJ{I would first enum the contributions (directly after formulating the RQs) and then mention some of the most pressing results. I also feel it could be shorter/more punchy (there's some repetition with percentages, the USPS/OZON findings, etc.}
Our contributions are as follows:\looseness=-1
\begin{itemize}[leftmargin=*, topsep=0pt, itemsep=0em]
% \KL{from our dataset, compared to other dataset, similar to other(previous) report}
\item Building on previous methods, we enhance the approach to identify maliciously registered domains. Our analysis reveals that 66.1\% of the domains in our dataset are maliciously registered, with the remainder being compromised domains.
\looseness=-1
% Other than maliciously registered domains defined by COMAR~\cite{maroofi2020comar} method, we also find that 28.2\% domains are creating phishing domains with randomly generated characters. 

\item From our analysis of maliciously registered domains in our dataset, we identify three key characteristics:
1) New gTLDs: Domains frequently use low-cost new gTLDs (\eg, \cc{.top}, \cc{.xyz}, and \cc{.online}), with \cc{.cn} usage rising after Freenom ceased free registrations~\cite{SuedbyMe25:online}, aligning with previous reports~\cite{Phishing18:online}.
% Use of new gTLDs: new gTLDs (\eg, \cc{.top}, \cc{.xyz}, \cc{.shop}, and \cc{.online}) are frequently used in maliciously registered domains. 
% Additionally, after Freenom ceased offering free registrations~\cite{SuedbyMe25:online}, the use of \cc{.cn} in malicious domains increased significantly which aligns with previous reports~\cite{Phishing18:online}.
2) TLD Variation in Brand Targeting: Phishing domains targeting brands (\eg, \cc{USPS}, \cc{OZON}) often use alternative TLDs (\eg, \cc{.top}, \cc{.tk}) instead of the brand’s original TLDs (\eg, \cc{.com}, \cc{.ru}).
% 
% TLD Variation in Brand Targeting: Phishing domains impersonating popular brands (\eg, USPS and OZON) commonly use alternative TLDs (\eg, \cc{.top} and \cc{.tk}) instead of the original brand’s TLD (\eg, \cc{.com} and \cc{.ru}).
3) DNS Fast Flux: 64.3\% of domains show frequent DNS updates, with 25.8\% using TTLs below 3600 seconds to possibly enable DNS fast flux.
% 
% Rapid DNS Record Changes: 64.3\% of domains exhibit frequent DNS record changes, with 25.8\% of domains set TTL value less than 3600 (1 hour) which can be used for a tactic associated with DNS fast flux.
% \item We find that by different impersonated brands, detection time varies, for example, USPS and Ozon are detected within XX days while Microsoft is detected within XX days. Also, we find various characteristics of impersonating brands, for example, attackers targeting Ozon uses Freenom TLD (\eg, .tk) the most. 

\item We find that maliciously registered domains are detected by blocklists (\eg, APWG) faster than compromised domains, with a median detection time of 16.3 days for malicious domains compared to 86 days for compromised domains. 
Additionally, detection times vary significantly across blocklists, with USPS and Ozon being the quickest with 1.4 and 1.3 days respectively. 
Even after detection by blocklists (\eg, APWG), phishing domains remain accessible for an average of 11.5 days, prolonging the risk to potential victims.
% \item We find that maliciously registered domains stay alive longer even after it is detected (23 days compared to 11.5 days). From those of maliciously registered domains, an average lifetime is 286 days. The average detection time of maliciously-registered domains are not much shorter than compromised domains by XX days. \KL{All detection time - maliciously-registered detection time}
\item We present a comprehensive longitudinal analysis of phishing domains (39 months). We publicly share our collected phishing dataset (\ie, phishing domains) to facilitate future phishing research upon acceptance.
% \KL{direction to share data, or share domain}
\end{itemize}


% RQ1: What are the detection timelines for phishing domains post-registration, and how do these vary across different TLDs?

% - How long, on average, does it take for phishing domains to be detected after their registration date?

% - How does the detection time differ between domains registered in different TLDs (e.g., gTLDs vs. ccTLDs)? → Table

% - Are there specific TLDs associated with longer detection times, and what factors contribute to this? 
% Are there any TLD that is still remain in a phishing domain after removal? → Figure (trend graph of TLD with removal TLD pointed)

% RQ2: How does the frequency and pattern of DNS record updates in phishing domains correlate with detection times?

% - Is there a difference in detection times between domains that frequently update their DNS records and those that do not? (compare update time in DNS record) → Figure (graph showing time diff (detection - registration))

% - What types of DNS records (e.g., A, NS, MX, TXT) are most frequently updated in phishing domains, and how do these changes influence detection times? (i.e., if the record changes frequently, how is the detection time?) → Table (Top 10 most updated domains | detection time)

% - Are certain DNS configurations (e.g., use of DNSSEC) more effective in evading detection, and how does this vary across different TLDs? → Table (Top 10 longest detection time | TLD | records (A, NS, MX, DNSSEC))

% RQ3: What impact does phishing domain registration (\ie, bulk registration) have on the detection and lifespan of phishing domains?

% - What percentage of phishing domains are registered in bulk, and how does their detection timeline compare to individually registered domains? → Table (Top 10 bulk domains | average of detection time | Top 10 detection time domains) 

% - How do the registrars and subdomain providers involved in bulk registrations influence the detection time and overall effectiveness of phishing domains? → Table (Top 10 registrars | bulk domains), Table (Top 10 registrars | subdomain providers (i.e., blogspot, weebly)

% - Are there specific patterns in WHOIS/RDAP data (e.g., registration anonymity, registrar history) for bulk-registered phishing domains that correlate with longer evasion times? → Table (Top 10 registrars | detection time)
