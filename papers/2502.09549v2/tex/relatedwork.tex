\section{Related Work}
The number of reports showed the trend of phishing domains and examined phishing websites. However, the characteristics of DNS settings of phishing domains are not well studied.

% \PP{DNS Measure}
% Previous works focus on measured DNS behavior on benign domains~\cite{}.
% Our work focuses on understanding the DNS behavior of phishing domains
% We also focus on patterns of DNS records.

% \PP{DNS in Phishing}
% Previous works:
% Focus on phishing detection with emphasis on evasion techniques.

% Our work:
% Analyze the DNS level on phishing domains to see how phishing attackers register and get detected by blocklists.

% \PP{Phishing Attack Measure}
% Previous researches study on how phishing websites and detection mechanism performs~\cite{}.
% Especially ~\cite{} 

\PP{Coverage of TLDs in Phishing Domains}
% Previous work~\cite{moura2024characterizing} analyzed impersonated domains but limited their focus to three specific ccTLDs, restricting the generalizability of their findings. 
Previous work~\cite{moura2024characterizing} has a narrow focus on three specific ccTLDs, while our work broadens the analysis to include all TLDs (gTLDs and ccTLDs), offering a more comprehensive understanding of phishing trends across a diverse range of domain spaces. 
While the prior study~\cite{moura2024characterizing} focused on identifying impersonated domains, our work delves into how long these phishing domains remain active post-registration, comparing this lifespan to detection delays on blocklists (\eg, APWG). 
Furthermore, our work enhances the analysis by examining the characteristics of specific targeted brands in phishing domains. 
Our work also goes beyond registration trends by exploring the influence of economic factors, such as whether cheaper TLDs contribute to the trend of phishing domains. 
% This economic and strategic perspective on registration patterns adds a valuable dimension to understanding how attackers maintain phishing sites.

\PP{Lifecycle and Classification of Phishing Domains}
The previous study~\cite{maroofi2020comar} introduced a classification method that distinguishes between compromised and maliciously registered domains using 32 extracted features. 
However, their analysis is restricted to identifying malicious domains, leaving a gap in understanding domain behaviors beyond real-world phishing datasets. 
Our work builds upon COMAR~\cite{maroofi2020comar} by applying its method more broadly to understand maliciously registered domains with real-world phishing datasets. 
% Previous research~\cite{hao2013understanding} observed that spammers often use bulk registrations within the \cc{.com} TLD, but this analysis was limited to five months with \cc{.com} TLD. 
Similarly, while prior work~\cite{hao2013understanding} examined bulk registrations, it was limited to five months of \cc{.com} TLD data.
% Our work expands on prior studies by analyzing \emph{multiple} gTLDs and ccTLDs, examining how phishing domains exploit various TLDs over time. 
% By investigating maliciously registered domains, their TLD usage, and targeted brands, we offer deeper insights into the phishing ecosystem and show the difference in characteristics between maliciously registered and compromised domains. 
Our work builds upon these studies by analyzing \emph{multiple} gTLDs and ccTLDs, providing comprehensive insights into how maliciously registered domains behave differently from compromised domains across various TLDs and targeted brands.
\looseness=-1

\PP{Registration Patterns and Longevity of Phishing Domains}
Previous research~\cite{oest2020sunrise} reveals that 75\% of victims access phishing webpages before they are detected. 
Building on this, our work examines the lifespan of phishing domains, compares the lifespans across different targeted brands, and analyzes various types of maliciously registered domains to uncover patterns and strategies used by attackers.

\PP{DNS Behavior and Phishing Detection Delays}
Previous work~\cite{hao2011monitoring} found that 55\% of malicious domains are first detected in spam campaigns over a day after registration. Building on this, our study provides a more comprehensive analysis of phishing domains' lifecycles, spanning from registration to detection and deregistration. By examining the delay between domain registration and blocklist detection, we offer a clearer understanding of detection timelines for both maliciously registered and compromised domains, highlighting gaps in current detection mechanisms.

%%%%%%%%%%%%%%%%%%%%%%%%%%%%%%%%%%%%%%%%%%%%%%%%%%%%%%%%


% 1. Characterizing and Mitigating Phishing Attacks at ccTLD Scale
% - Impersonated domains in ccTLD
% Limitation: 
% - Only looked at 3 ccTLD Impersonated Brands are focused on 3 ccTLD
% Our strength:
% - We cover all TLDs (gTLD, ccTLD)

% 2. COMAR: Classification of Compromised versus Maliciously Registered Domains
% - 1.Domain lifecycle analysis
% - 2.Developed a new approach (COMAR) to find Compromised vs Malicious domains → Extracted 32 features to determine domains
% Limitation:
% - Only focused on finding malicious domains
% Our strength:
% - We use COMAR’s method to find malicious domain

% 3. Understanding the Domain Registration Behavior of Spammers
% - 1.Bulk registration
% - 2.Registration and appeared in .com zone
% Limitation:
% - Only looked at .com TLD over 5 month
% - Focused on domain registration trend (bulk registration is one of findings, didn’t provide any statistical analysis)
% Our strength:
% - We cover all TLDs (gTLD, ccTLD)

% 4. Monitoring the Initial DNS Behavior of Malicious Domains
% - 55\% of malicious domains appeared in spam campaigns more than a day after registration
% - Some IP blocks are used widely in spam attack (96.45.0.0/16 and 216.162.0.0/16)
% - Phishing detection time takes over 1 day from domain registration time
% Our strength:
% - Investigating how phishing domains set DNS records
% - Comparing self-hosted vs. web-provider-hosted domains
% - Add registration data to calculate between domain registration time and blocklist detection time

% 5. DarkDNS: Revisiting the Value of Rapid Zone Update
% - 1\% of domains not detected in CZDS, mostly phishing domains
% Our strength:
% - Using DZDB and OpenIntel data to analyze DNS records

% New finding:
% - DNS record changes over time by domains