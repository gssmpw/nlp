% TEMPLATE for Usenix papers, specifically to meet requirements of
%  USENIX '05
% originally a template for producing IEEE-format articles using LaTeX.
%   written by Matthew Ward, CS Department, Worcester Polytechnic Institute.
% adapted by David Beazley for his excellent SWIG paper in Proceedings,
%   Tcl 96
% turned into a smartass generic template by De Clarke, with thanks to
%   both the above pioneers
% use at your own risk.  Complaints to /dev/null.
% make it two column with no page numbering, default is 10 point

% Munged by Fred Douglis <douglis@research.att.com> 10/97 to separate
% the .sty file from the LaTeX source template, so that people can
% more easily include the .sty file into an existing document.  Also
% changed to more closely follow the style guidelines as represented
% by the Word sample file. 

% Note that since 2010, USENIX does not require endnotes. If you want
% foot of page notes, don't include the endnotes package in the 
% usepackage command, below.

\documentclass[letterpaper,twocolumn,10pt]{article}
\usepackage{usenix}
\usepackage{cite}
\usepackage{todonotes}
\usepackage{acronym}
\usepackage{xspace}
% to be able to draw some self-contained figs
\usepackage{tikz}
\usepackage{amsmath}
\usepackage{graphicx} % Required for inserting images
\usepackage{pdflscape}
 \usepackage{multirow}
\usepackage{pifont}
\usepackage{xcolor}
\usepackage{nicematrix}
\usepackage{tabularx}
\usepackage{makecell}
\usepackage{slashbox, pict2e}
\usepackage{caption}
\usepackage{enumitem}
\usepackage{color, colortbl}
\usepackage[most]{tcolorbox}
% \usepackage{array}
\usepackage[verbose]{placeins}
\usepackage{balance}
% \usepackage{refcheck}
\usepackage{soul}
\usepackage{etoolbox}
%\usepackage{ulem}
%\usepackage[available]{usenixbadges}

%%%%%%%%%%%---SETME-----%%%%%%%%%%%%%
%replace @@ with the submission number submission site.
\newcommand{\thiswork}{INF$^2$\xspace}
%%%%%%%%%%%%%%%%%%%%%%%%%%%%%%%%%%%%


%\newcommand{\rev}[1]{{\color{olivegreen}#1}}
\newcommand{\rev}[1]{{#1}}


\newcommand{\JL}[1]{{\color{cyan}[\textbf{\sc JLee}: \textit{#1}]}}
\newcommand{\JW}[1]{{\color{orange}[\textbf{\sc JJung}: \textit{#1}]}}
\newcommand{\JY}[1]{{\color{blue(ncs)}[\textbf{\sc JSong}: \textit{#1}]}}
\newcommand{\HS}[1]{{\color{magenta}[\textbf{\sc HJang}: \textit{#1}]}}
\newcommand{\CS}[1]{{\color{navy}[\textbf{\sc CShin}: \textit{#1}]}}
\newcommand{\SN}[1]{{\color{olive}[\textbf{\sc SNoh}: \textit{#1}]}}

%\def\final{}   % uncomment this for the submission version
\ifdefined\final
\renewcommand{\JL}[1]{}
\renewcommand{\JW}[1]{}
\renewcommand{\JY}[1]{}
\renewcommand{\HS}[1]{}
\renewcommand{\CS}[1]{}
\renewcommand{\SN}[1]{}
\fi

%%% Notion for baseline approaches %%% 
\newcommand{\baseline}{offloading-based batched inference\xspace}
\newcommand{\Baseline}{Offloading-based batched inference\xspace}


\newcommand{\ans}{attention-near storage\xspace}
\newcommand{\Ans}{Attention-near storage\xspace}
\newcommand{\ANS}{Attention-Near Storage\xspace}

\newcommand{\wb}{delayed KV cache writeback\xspace}
\newcommand{\Wb}{Delayed KV cache writeback\xspace}
\newcommand{\WB}{Delayed KV Cache Writeback\xspace}

\newcommand{\xcache}{X-cache\xspace}
\newcommand{\XCACHE}{X-Cache\xspace}


%%% Notions for our methods %%%
\newcommand{\schemea}{\textbf{Expanding supported maximum sequence length with optimized performance}\xspace}
\newcommand{\Schemea}{\textbf{Expanding supported maximum sequence length with optimized performance}\xspace}

\newcommand{\schemeb}{\textbf{Optimizing the storage device performance}\xspace}
\newcommand{\Schemeb}{\textbf{Optimizing the storage device performance}\xspace}

\newcommand{\schemec}{\textbf{Orthogonally supporting Compression Techniques}\xspace}
\newcommand{\Schemec}{\textbf{Orthogonally supporting Compression Techniques}\xspace}



% Circular numbers
\usepackage{tikz}
\newcommand*\circled[1]{\tikz[baseline=(char.base)]{
            \node[shape=circle,draw,inner sep=0.4pt] (char) {#1};}}

\newcommand*\bcircled[1]{\tikz[baseline=(char.base)]{
            \node[shape=circle,draw,inner sep=0.4pt, fill=black, text=white] (char) {#1};}}

\begin{document}



%don't want date printed
\date{}

%make title bold and 14 pt font (Latex default is non-bold, 16 pt)
% \title{\Large \bf SoK: A Cognitive Framework and Attack Ontology for Deception in Mixed Reality
% }

\title{\Large \bf SoK: Come Together -- Unifying Security, Information Theory, and Cognition for a Mixed Reality Deception Attack Ontology \& Analysis Framework}
% SoK: Fusing Mixed Reality Security, Information Theory, and Cognition: A Framework for Deception Attack Analysis

%\author{Anonymized}
% \author{{\rm Ali Teymourian}, 
% %Division of Computer Science \& Engineering\\
% %Louisiana State University 
% %\and 
% {\rm Andrew M. Webb}, 
% Division of Computer Science \& Engineering\\
% Louisiana State University 
% \and
% {\rm Taha Gharaibeh}\\ 
% Baggil(i) Truth (BiT) Lab\\
% Division of Computer Science \& Engineering\\
% Louisiana State University 
% \and
% {\rm Arushi Ghildiyal}\\
% Division of Computer Science \& Engineering\\
% Louisiana State University 
% \and
% {\rm Ibrahim Baggili}\\
% Baggil(i) Truth (BiT) Lab\\
% Division of Computer Science \& Engineering\\
% Louisiana State University 
% }
\author{{\rm Ali Teymourian}$^1$, {\rm Andrew M. Webb}$^1$, {\rm Taha Gharaibeh}$^{1,2}$, {\rm Arushi Ghildiyal}$^1$, {\rm Ibrahim Baggili}$^{1,2}$\\
$^1$Division of Computer Science \& Engineering\\
$^2$Baggil(i) Truth (BiT) Lab, Center for Computation and Technology\\ %, Division of Computer Science \& Engineering, Louisiana State University\\
Louisiana State University\\
\{ateymo1, andrewwebb, tghara1, aghild1, ibaggili\}@lsu.edu
}

% \\ Center for Computation and Technology (CCT)\\
% Division of Computer Science\\
% Baggil(i) Truth (BiT) Lab\\

% \and
% {\rm Second Name}\\
% Second Institution
% }
\maketitle
\thispagestyle{empty}
% \pagestyle{plain}
\pagestyle{empty}

% Use the following at camera-ready time to suppress page numbers.
% Comment it out when you first submit the paper for review.
%\thispagestyle{empty}

\acrodef{MR}[MR]{Mixed Reality}
\newcommand{\MR}{\ac{MR}\xspace}

\acrodef{VR}[VR]{Virtual Reality}
\newcommand{\VR}{\ac{VR}\xspace}

\acrodef{VE}[VE]{Virtual Environment}
\newcommand{\VE}{\ac{VE}\xspace}

\acrodef{DAF}[DAF]{Deception Analysis Framework}
\newcommand{\DAF}{\ac{DAF}\xspace}

\acrodef{AR}[AR]{Augmented Reality}
\newcommand{\AR}{\ac{AR}\xspace}

\acrodef{AV}[AV]{Augmented Virtuality}
\newcommand{\AV}{\ac{AV}\xspace}

\acrodef{PMDA}[PMDA]{Perceptual Manipulation Deception Attacks}
\acrodefplural{PMDA}{Perceptual Manipulation Deception Attacks}
\newcommand{\PMDA}{\ac{PMDA}\xspace}

\acrodef{PMA}[PMA]{Perceptual Manipulation Attacks}
\acrodefplural{PMA}{Perceptual Manipulation Attacks}
\newcommand{\PMA}{\ac{PMA}\xspace}

\acrodef{SNR}[SNR]{Signal-to-Noise Ratio}
\newcommand{\SNR}{\ac{SNR}\xspace}

\acrodef{UI}[UI]{User Interface}
\newcommand{\UI}{\ac{UI}\xspace}

\acrodef{HMD}[HMD]{Head-Mounted Display}
\newcommand{\HMD}{\ac{HMD}\xspace}

\acrodef{HUD}[HUD]{Heads-Up Display}
\newcommand{\HUD}{\ac{HUD}\xspace}

\acrodef{APA}[APA]{American Psychological Association}
\newcommand{\APA}{\ac{APA}\xspace}

\acrodef{WMC}[WMC]{Working Memory Capacity}
\newcommand{\WMC}{\ac{WMC}\xspace}

\acrodef{CSI}[CSI]{Channel State Information}
\newcommand{\CSI}{\ac{CSI}\xspace}

\acrodef{RDW}[RDW]{Redirected Walking}
\newcommand{\RDW}{\ac{RDW}\xspace}

\acrodef{KF}[KF]{Key Finding}
\acrodef{RG}[RG]{Research Gap}

\acrodef{DARPA}[DARPA]{Defense Advanced Research Projects Agency}
\newcommand{\DARPA}{\ac{DARPA}\xspace}

\acrodef{CVSS}[CVSS]{Common Vulnerability Scoring System}
\newcommand{\CVSS}{\ac{CVSS}\xspace}

\begin{abstract}
We present a primary attack ontology and analysis framework for deception attacks in Mixed Reality (MR). This is achieved through multidisciplinary Systematization of Knowledge (SoK), integrating concepts from MR security, information theory, and cognition. While MR grows in popularity, it presents many cybersecurity challenges, particularly concerning deception attacks and their effects on humans. In this paper, we use the Borden-Kopp model of deception to develop a comprehensive ontology of MR deception attacks. Further, we derive two models to assess impact of MR deception attacks on information communication and decision-making. The first, an information-theoretic model, mathematically formalizes the effects of attacks on information communication. The second, a decision-making model, details the effects of attacks on interlaced cognitive processes. Using our ontology and models, we establish the MR Deception Analysis Framework (DAF) to assess the effects of MR deception attacks on information channels, perception, and attention. Our SoK uncovers five key findings for research and practice and identifies five research gaps to guide future work.
\end{abstract}

\section{Introduction}

Despite the remarkable capabilities of large language models (LLMs)~\cite{DBLP:conf/emnlp/QinZ0CYY23,DBLP:journals/corr/abs-2307-09288}, they often inevitably exhibit hallucinations due to incorrect or outdated knowledge embedded in their parameters~\cite{DBLP:journals/corr/abs-2309-01219, DBLP:journals/corr/abs-2302-12813, DBLP:journals/csur/JiLFYSXIBMF23}.
Given the significant time and expense required to retrain LLMs, there has been growing interest in \emph{model editing} (a.k.a., \emph{knowledge editing})~\cite{DBLP:conf/iclr/SinitsinPPPB20, DBLP:journals/corr/abs-2012-00363, DBLP:conf/acl/DaiDHSCW22, DBLP:conf/icml/MitchellLBMF22, DBLP:conf/nips/MengBAB22, DBLP:conf/iclr/MengSABB23, DBLP:conf/emnlp/YaoWT0LDC023, DBLP:conf/emnlp/ZhongWMPC23, DBLP:conf/icml/MaL0G24, DBLP:journals/corr/abs-2401-04700}, 
which aims to update the knowledge of LLMs cost-effectively.
Some existing methods of model editing achieve this by modifying model parameters, which can be generally divided into two categories~\cite{DBLP:journals/corr/abs-2308-07269, DBLP:conf/emnlp/YaoWT0LDC023}.
Specifically, one type is based on \emph{Meta-Learning}~\cite{DBLP:conf/emnlp/CaoAT21, DBLP:conf/acl/DaiDHSCW22}, while the other is based on \emph{Locate-then-Edit}~\cite{DBLP:conf/acl/DaiDHSCW22, DBLP:conf/nips/MengBAB22, DBLP:conf/iclr/MengSABB23}. This paper primarily focuses on the latter.

\begin{figure}[t]
  \centering
  \includegraphics[width=0.48\textwidth]{figures/demonstration.pdf}
  \vspace{-4mm}
  \caption{(a) Comparison of regular model editing and EAC. EAC compresses the editing information into the dimensions where the editing anchors are located. Here, we utilize the gradients generated during training and the magnitude of the updated knowledge vector to identify anchors. (b) Comparison of general downstream task performance before editing, after regular editing, and after constrained editing by EAC.}
  \vspace{-3mm}
  \label{demo}
\end{figure}

\emph{Sequential} model editing~\cite{DBLP:conf/emnlp/YaoWT0LDC023} can expedite the continual learning of LLMs where a series of consecutive edits are conducted.
This is very important in real-world scenarios because new knowledge continually appears, requiring the model to retain previous knowledge while conducting new edits. 
Some studies have experimentally revealed that in sequential editing, existing methods lead to a decrease in the general abilities of the model across downstream tasks~\cite{DBLP:journals/corr/abs-2401-04700, DBLP:conf/acl/GuptaRA24, DBLP:conf/acl/Yang0MLYC24, DBLP:conf/acl/HuC00024}. 
Besides, \citet{ma2024perturbation} have performed a theoretical analysis to elucidate the bottleneck of the general abilities during sequential editing.
However, previous work has not introduced an effective method that maintains editing performance while preserving general abilities in sequential editing.
This impacts model scalability and presents major challenges for continuous learning in LLMs.

In this paper, a statistical analysis is first conducted to help understand how the model is affected during sequential editing using two popular editing methods, including ROME~\cite{DBLP:conf/nips/MengBAB22} and MEMIT~\cite{DBLP:conf/iclr/MengSABB23}.
Matrix norms, particularly the L1 norm, have been shown to be effective indicators of matrix properties such as sparsity, stability, and conditioning, as evidenced by several theoretical works~\cite{kahan2013tutorial}. In our analysis of matrix norms, we observe significant deviations in the parameter matrix after sequential editing.
Besides, the semantic differences between the facts before and after editing are also visualized, and we find that the differences become larger as the deviation of the parameter matrix after editing increases.
Therefore, we assume that each edit during sequential editing not only updates the editing fact as expected but also unintentionally introduces non-trivial noise that can cause the edited model to deviate from its original semantics space.
Furthermore, the accumulation of non-trivial noise can amplify the negative impact on the general abilities of LLMs.

Inspired by these findings, a framework termed \textbf{E}diting \textbf{A}nchor \textbf{C}ompression (EAC) is proposed to constrain the deviation of the parameter matrix during sequential editing by reducing the norm of the update matrix at each step. 
As shown in Figure~\ref{demo}, EAC first selects a subset of dimension with a high product of gradient and magnitude values, namely editing anchors, that are considered crucial for encoding the new relation through a weighted gradient saliency map.
Retraining is then performed on the dimensions where these important editing anchors are located, effectively compressing the editing information.
By compressing information only in certain dimensions and leaving other dimensions unmodified, the deviation of the parameter matrix after editing is constrained. 
To further regulate changes in the L1 norm of the edited matrix to constrain the deviation, we incorporate a scored elastic net ~\cite{zou2005regularization} into the retraining process, optimizing the previously selected editing anchors.

To validate the effectiveness of the proposed EAC, experiments of applying EAC to \textbf{two popular editing methods} including ROME and MEMIT are conducted.
In addition, \textbf{three LLMs of varying sizes} including GPT2-XL~\cite{radford2019language}, LLaMA-3 (8B)~\cite{llama3} and LLaMA-2 (13B)~\cite{DBLP:journals/corr/abs-2307-09288} and \textbf{four representative tasks} including 
natural language inference~\cite{DBLP:conf/mlcw/DaganGM05}, 
summarization~\cite{gliwa-etal-2019-samsum},
open-domain question-answering~\cite{DBLP:journals/tacl/KwiatkowskiPRCP19},  
and sentiment analysis~\cite{DBLP:conf/emnlp/SocherPWCMNP13} are selected to extensively demonstrate the impact of model editing on the general abilities of LLMs. 
Experimental results demonstrate that in sequential editing, EAC can effectively preserve over 70\% of the general abilities of the model across downstream tasks and better retain the edited knowledge.

In summary, our contributions to this paper are three-fold:
(1) This paper statistically elucidates how deviations in the parameter matrix after editing are responsible for the decreased general abilities of the model across downstream tasks after sequential editing.
(2) A framework termed EAC is proposed, which ultimately aims to constrain the deviation of the parameter matrix after editing by compressing the editing information into editing anchors. 
(3) It is discovered that on models like GPT2-XL and LLaMA-3 (8B), EAC significantly preserves over 70\% of the general abilities across downstream tasks and retains the edited knowledge better.
\section{Background}
\label{sec:background}


\subsection{Code Review Automation}
Code review is a widely adopted practice among software developers where a reviewer examines changes submitted in a pull request \cite{hong2022commentfinder, ben2024improving, siow2020core}. If the pull request is not approved, the reviewer must describe the issues or improvements required, providing constructive feedback and identifying potential issues. This step involves review commment generation, which play a key role in the review process by generating review comments for a given code difference. These comments can be descriptive, offering detailed explanations of the issues, or actionable, suggesting specific solutions to address the problems identified \cite{ben2024improving}.


Various approaches have been explored to automate the code review comments process  \cite{tufano2023automating, tufano2024code, yang2024survey}. 
Early efforts centered on knowledge-based systems, which are designed to detect common issues in code. Although these traditional tools provide some support to programmers, they often fall short in addressing complex scenarios encountered during code reviews \cite{dehaerne2022code}. More recently, with advancements in deep learning, researchers have shifted their focus toward using large-language models to enhance the effectiveness of code issue detection and code review comment generation.

\subsection{Knowledge-based Code Review Comments Automation}

Knowledge-based systems (KBS) are software applications designed to emulate human expertise in specific domains by using a collection of rules, logic, and expert knowledge. KBS often consist of facts, rules, an explanation facility, and knowledge acquisition. In the context of software development, these systems are used to analyze the source code, identifying issues such as coding standard violations, bugs, and inefficiencies~\cite{singh2017evaluating, delaitre2015evaluating, ayewah2008using, habchi2018adopting}. By applying a vast set of predefined rules and best practices, they provide automated feedback and recommendations to developers. Tools such as FindBugs \cite{findBugs}, PMD \cite{pmd}, Checkstyle \cite{checkstyle}, and SonarQube \cite{sonarqube} are prominent examples of knowledge-based systems in code analysis, often referred to as static analyzers. These tools have been utilized since the early 1960s, initially to optimize compiler operations, and have since expanded to include debugging tools and software development frameworks \cite{stefanovic2020static, beller2016analyzing}.



\subsection{LLMs-based Code Review Comments Automation}
As the field of machine learning in software engineering evolves, researchers are increasingly leveraging machine learning (ML) and deep learning (DL) techniques to automate the generation of review comments \cite{li2022automating, tufano2022using, balachandran2013reducing, siow2020core, li2022auger, hong2022commentfinder}. Large language models (LLMs) are large-scale Transformer models, which are distinguished by their large number of parameters and extensive pre-training on diverse datasets.  Recently, LLMs have made substantial progress and have been applied across a broad spectrum of domains. Within the software engineering field, LLMs can be categorized into two main types: unified language models and code-specific models, each serving distinct purposes \cite{lu2023llama}.

Code-specific LLMs, such as CodeGen \cite{nijkamp2022codegen}, StarCoder \cite{li2023starcoder} and CodeLlama \cite{roziere2023code} are optimized to excel in code comprehension, code generation, and other programming-related tasks. These specialized models are increasingly utilized in code review activities to detect potential issues, suggest improvements, and automate review comments \cite{yang2024survey, lu2023llama}. 




\subsection{Retrieval-Augmented Generation}
Retrieval-Augmented Generation (RAG) is a general paradigm that enhances LLMs outputs by including relevant information retrieved from external databases into the input prompt \cite{gao2023retrieval}. Traditional LLMs generate responses based solely on the extensive data used in pre-training, which can result in limitations, especially when it comes to domain-specific, time-sensitive, or highly specialized information. RAG addresses these limitations by dynamically retrieving pertinent external knowledge, expanding the model's informational scope and allowing it to generate responses that are more accurate, up-to-date, and contextually relevant \cite{arslan2024business}. 

To build an effective end-to-end RAG pipeline, the system must first establish a comprehensive knowledge base. It requires a retrieval model that captures the semantic meaning of presented data, ensuring relevant information is retrieved. Finally, a capable LLM integrates this retrieved knowledge to generate accurate and coherent results \cite{ibtasham2024towards}.




\subsection{LLM as a Judge Mechanism}

LLM evaluators, often referred to as LLM-as-a-Judge, have gained significant attention due to their ability to align closely with human evaluators' judgments \cite{zhu2023judgelm, shi2024judging}. Their adaptability and scalability make them highly suitable for handling an increasing volume of evaluative tasks. 

Recent studies have shown that certain LLMs, such as Llama-3 70B and GPT-4 Turbo, exhibit strong alignment with human evaluators, making them promising candidates for automated judgment tasks \cite{thakur2024judging}

To enable such evaluations, a proper benchmarking system should be set up with specific components: \emph{prompt design}, which clearly instructs the LLM to evaluate based on a given metric, such as accuracy, relevance, or coherence; \emph{response presentation}, guiding the LLM to present its verdicts in a structured format; and \emph{scoring}, enabling the LLM to assign a score according to a predefined scale \cite{ibtasham2024towards}. Additionally, this evaluation system can be enriched with the ability to explain reasoning behind verdicts, which is a significant advantage of LLM-based evaluation \cite{zheng2023judging}. The LLM can outline the criteria it used to reach its judgment, offering deeper insights into its decision-making process.





\begin{figure*}[ht]
    \centering
    \includegraphics[width=0.95\textwidth]{quantitative_1_2.jpg}
    \vspace{-5mm}
    \caption{Quantitative comparisons for Scenario 1 and Scenario 2. The simulation setup is illustrated on the left. We compare the prediction accuracy using ADE, and the efficiency and safety in resulting robot plans using Detour and Minimum Distance.}
    % AToM achieves the best accuracy and results in improving efficiency while maintaining safety.}
    \vspace{-5mm}
    \label{fig:quantitative_12}
\end{figure*}

\section{Methodology}
\textbf{Problem Definition.} 
Let the state and control of an agent be $\boldsymbol{x} \in \mathds{R}^{n}$ and $\boldsymbol{u} \in \mathds{R}^{m}$, where $n, m$ are state and control dimensions. 
We use subscripts $H, R, HR$ to represent human, robot, and their joint system, and superscript $k$ to represent timesteps.
The dynamics can be described by $\boldsymbol{x}^{k+1} = \Dynamics(\boldsymbol{x}^{k}, \boldsymbol{u}^{k})$.
The trajectories from timestep $i$ to timestep $j$ can be represented by $\boldsymbol{X}^{i:j} = [\boldsymbol{x}^{i}, \dots, \boldsymbol{x}^{j}]$.
In a long-term interaction setting, we assume that the goal positions and environmental obstacles are known as $\boldsymbol{g}_{HR}$ and $\mathcal{O}$, respectively.
At each timestep $t$, given the observed historical trajectories $\boldsymbol{X}_{HR}^{0:t}$ and known information $\boldsymbol{g}_{HR}$ and $\mathcal{O}$, our goal is to predict future human trajectories $\hat{\boldsymbol{X}}_{H}^{t+1:t+T_{f}}$ with prediction horizon $T_{f}$, which the robot can use for collision-free predictive planning. 
For simplicity, we will omit time superscripts when no confusion is aroused. Our method is planner-agnostic and the robot planning algorithm is outside the scope of this work. 
% We omit the time superscripts for predicted human trajectories $\hat{\boldsymbol{X}}_{H}$ in the rest of the paper for simplicity. Our method is planner-agnostic and the robot planning algorithm is outside the scope of this work. 

% Let the state of the human and the robot be $\boldsymbol{x}_{H} \in \mathds{R}^{n_{H}}$ and $\boldsymbol{x}_{R} \in \mathds{R}^{n_{R}}$, where $n_{H}, n_{R}$ are state dimensions. 
% Let their joint state be $\boldsymbol{x}_{HR} \in \mathds{R}^{n_{H} + n_{R}}$.
% Let the control of the human and the robot be $\boldsymbol{u}_{H} \in \mathds{R}^{m_{H}}$ and $\boldsymbol{u}_{R} \in \mathds{R}^{m_{R}}$, where $m_{H}, m_{R}$ are control dimensions. 
% The dynamics can be described by $\dot{\boldsymbol{x}}_{H} = \Dynamics_{H}(\boldsymbol{x}_{H}, \boldsymbol{u}_{H})$ and $\dot{\boldsymbol{x}}_{R} = \Dynamics_{R}(\boldsymbol{x}_{R}, \boldsymbol{u}_{R})$. 
% In a long-term interaction setting, we assume the goal positions are known for both agents as $g_{H}$ and $g_{R}$. 
% The human and robot share the knowledge of all agents' historical trajectories up to the current step $\boldsymbol{Z}_{HR} = [\boldsymbol{x}_{HR}^{0}, \dots, \boldsymbol{x}_{HR}^{t}]$, as well as the environmental obstacles $\mathcal{O}$. 
% At each step $t$, our goal is to predict future human trajectories $\hat{\boldsymbol{X}}_{H} = [\hat{\boldsymbol{x}}_{H}^{t+1}, \dots, \hat{\boldsymbol{x}}_{H}^{t+T_{pred}}]$ with prediction horizon $T_{pred}$, which the robot can use for collision-free predictive planning. Our method is planner-agnostic and the robot planning algorithm is outside the scope of this work. 

\textbf{Overview.} 
We first present the original nested ToM formulation in robot predictive planning. We then propose our reformulated adaptive ToM (AToM) which detaches the human prediction model from the recursive structure. The human model consists of a game-theoretic solver and a parameter update mechanism, which captures dynamic human behaviours in long-term interactions.

\subsection{Original ToM in Robot Planning}
Following prior works mentioned in Sec. \ref{sec:ToM}, we formulate the original ToM in robot predictive planning task and identify the potential issues with this formulation.

The robot plans with a finite horizon $T_{p}$, where at each timestep t the robot optimises its controls $\boldsymbol{U}_{R} = [\boldsymbol{u}_{R}^{t+1}, \dots, \boldsymbol{u}_{R}^{t+T_{p}}]$ to minimize the cumulative cost:
\begin{equation}
\label{eq:robot_planning}
\begin{aligned}
    \min &\; \Cost_{R}(\boldsymbol{x}_{R}^{t+1:t+T_{p}}, \boldsymbol{x}_{R}^{t}, \hat{\boldsymbol{X}}_{H}, \boldsymbol{g}_{R}, \mathcal{O}), \\
    \text{s.t.} \quad &\boldsymbol{x}_{R}^{k+1} = \Dynamics_{R}(\boldsymbol{x}_{R}^{k}, \boldsymbol{u}_{R}^{k}), \\
                      &\Constraint_{R}(\boldsymbol{x}_{R}^{k}, \boldsymbol{u}_{R}^{k}, \boldsymbol{x}_{H}^{k}, \mathcal{O}) \leq 0, \\
                      &k = t+1, \dots, t+T_{p},
\end{aligned}
\end{equation}
where $\Cost_{R}$ is the cost function that captures the robot's performance goals, 
$\Constraint_{R}$ contains constraints such as collision-avoidance, 
and predicted human trajectories $\hat{\boldsymbol{X}}_{H}$ can be obtained from human prediction models such as the Social Force or prediction neural networks, as described in Sec. \ref{seq:traj_prediction}.

% Human predictions $\hat{X}_{H}$ can be obtained from historical observations and other additional information such as human goal and environmental obstacles: ({\color{blue} can introduce the parameter here and remove equation \eqref{eq:humanpred2}})
% \begin{equation}
% \label{eq:human_pred}
%     \tilde{X}_{H} = f_{pred} (X_{HR}, g_{H}, \mathcal{O}),
% \end{equation}
% where $f_{pred}$ is the prediction method such as the Social Force model or neural networks. 

To obtain $\hat{\boldsymbol{X}}_{H}$ using a ToM model, predicted human actions need to be solved from predicted robot trajectories:
\begin{equation}
\label{eq:original_ToM}
\begin{aligned}
    \boldsymbol{\hat{U}}_{H} = \argmin_{\boldsymbol{u}_H^{t+1:t+T_{f}}} &\Cost_{H}(\boldsymbol{x}_{H}^{t+1:t+T_{f}}, \boldsymbol{x}_{HR}^{t}, \hat{\boldsymbol{X}}_{R}, \boldsymbol{g}_{H}, \mathcal{O}), \\
    \text{s.t.} \quad &\boldsymbol{x}_{H}^{k+1} = \Dynamics_{H}(\boldsymbol{x}_{H}^{k}, \boldsymbol{u}_{H}^{k}), \\
                      &\boldsymbol{\hat{x}}_{R}^{k+1} = \Dynamics_{R}(\boldsymbol{\hat{x}}_{R}^{k}, \boldsymbol{u}_{R}^{k}), \\
                      &\Constraint_{H}(\boldsymbol{x}_{H}^{k}, \boldsymbol{u}_{H}^{k}, \boldsymbol{x}_{R}^{k}, \mathcal{O}) \leq 0, \\
                      &k = t+1, \dots, t+T_{f}. \\
\end{aligned}
\end{equation}
% In the original ToM formulation, future human actions are optimised similarly to robot predictive planning:
% \begin{equation}
% \label{eq:original_ToM}
% \begin{aligned}
%     \min_{u_H^{t+1:t+T_{pred}}} &\sum_{i=t+1}^{t+T_{pred}} J_{H}(x_{H}^{i}, g_{H}, \tilde{X}_{R}, \mathcal{O}), \\
%     \text{s.t.} \quad &x_{H}^{i+1} = x_{H}^{i} + f_{H}(x_{H}^{i}, u_{H}^{i}), \\
%                       &x_{R}^{i+1} = x_{R}^{i} + f_{R}(x_{R}^{i}, u_{R}^{i}), \\
%                       &C(x_{H}^{i}, u_{H}^{i}, x_{R}^{i}) \leq 0, \\
%                       &i = t+1, \dots, t+T_{pred},
% \end{aligned}
% \end{equation}
Note that the predicted robot trajectories are obtained from the robot plans $\boldsymbol{U}_{R}$ solved from Problem \eqref{eq:robot_planning}. 
If we substitute Eq. \eqref{eq:original_ToM} and human dynamics $\Dynamics_{H}$ back to the cost in Problem \eqref{eq:robot_planning}, it results in a recursive optimisation problem for the robot. 
Solving such optimisation is tractable for simple settings as discussed in Sec. \ref{sec:ToM}, but complex and computationally expensive for robot planning with continuous and infinite action space. 
Therefore, we propose a reformulation of the original ToM to detach the human prediction model from the nested optimisation problem.

% \begin{figure}
%     \centering
%     \includegraphics[width=\linewidth]{quantitative_3.jpg}
%     \vspace{-8mm}
%     \caption{Quantitative comparison between AToM and SF in Scenario 3. We compare the number of steps the robot takes to reach the goal which reflects the efficiency. We mark the rounds where collisions happens using red crosses.}
%     % AToM enables the robot to pass through the doorway safely and efficiently in all rounds of experiments.}
%     \vspace{-5mm}
%     \label{fig:quantitative_3}
% \end{figure}

\subsection{Reformulation and Adaptive ToM (AToM)}
Instead of using the true robot plan as human-predicted robot actions, we argue that the human maintains an internal model of the navigation problem and they predict the optimal actions for all agents based on a dynamic belief about each agent's behavioural pattern. 
We formulate the human internal model as a multi-player general-sum differential game with finite horizon $T_{f}$. 
In a navigation scenario, the strategy for player $i$ is a control sequence $\boldsymbol{U}_{i} = [\boldsymbol{u}_{i}^{t+1}, \dots, \boldsymbol{u}_{i}^{t+T_{f}}]$. 
Each player seeks to minimize its cost while respecting constraints and dynamics:
\begin{equation}
\label{eq:game_cost}
\begin{aligned}
    \min_{U} \quad &\sum_{k=t+1}^{t+T_{f}} 
    (\boldsymbol{x}_{i}^{k} - \boldsymbol{g})^\top Q (\boldsymbol{x}_{i}^{k} - \boldsymbol{g}) 
    + \boldsymbol{u}_{i}^{k}{}^\top R \boldsymbol{u}_{i}^{k} \\
    +  &w_{s}(\sum_{n \neq i} \max (0, \|\boldsymbol{x}^{k}_{i} - \boldsymbol{x}^{k}_{n}\|_{2} - d_{s}))) \\
    + &w_{o}(\max (0, D(\boldsymbol{x}^{k}_{i}, \mathcal{O}) - d_{o}))\\
    \text{s.t.} \quad &\boldsymbol{x}_{min} \leq \boldsymbol{x}_{i}^{k} \leq \boldsymbol{x}_{max}, \\
                &\boldsymbol{u}_{min} \leq \boldsymbol{u}_{i}^{k} \leq \boldsymbol{u}_{max}, \\
                &\boldsymbol{x}_{i}^{k+1} = \Dynamics_{i}(\boldsymbol{x}_{i}^{k}, \boldsymbol{u}_{i}^{k}), \\
                &k = t+1, \dots, t+T_{f}, \\
\end{aligned}
\end{equation}
% \begin{equation}
% \label{eq:game_cost}
% \begin{aligned}
%     \min_{U} \quad & \Cost_{goal}(\boldsymbol{X}, \boldsymbol{g}, \theta_{1}) + \Cost_{speed}(\boldsymbol{X}, \theta_{2}) \\
%                    &+ \Cost_{obs}(\boldsymbol{X}, \mathcal{O}, \theta_{3}) + \Cost_{social}(\boldsymbol{X}_{joint}, \theta_{4}), \\
%     \text{s.t.} \quad &\Constraint(\boldsymbol{X}_{joint}, \boldsymbol{U}, \mathcal{O}, \theta_{5}) \leq 0, \\
%                 &\boldsymbol{x}^{k+1} = \Dynamics(\boldsymbol{x}^{k}, \boldsymbol{u}^{k}), \\
%                 &k = t+1, \dots, t+T_{f}, \\
% \end{aligned}
% \end{equation}
% where the 4 cost components are $\Cost_{goal}$ for goal reaching, $\Cost_{speed}$ for speed regulation, $\Cost_{obs}$ for obstacle avoidance, and $\Cost_{social}$ for social distancing. 
% $\boldsymbol{\theta} = [\theta_{1}, \theta_{2}, \theta_{3}, \theta_{4}, \theta_{5}]^\top$ are the parameters from the costs and constraints 
% such as the preferred speed in $\Cost_{speed}$, and preferred distance from neighbours in $\Cost_{social}$. 
where $Q \geq 0$ and $R > 0$ are the weights for the state and control, $w_{s}$ and $w_{o}$ are the weights for social and obstacle avoidance, $D$ calculates the closest distance to obstacle $\mathcal{O}$, 
$d_{s}$ and $d_{o}$ are the preferred social and obstacle distances, $\boldsymbol{x}_{min}$, $\boldsymbol{x}_{max}$, $\boldsymbol{u}_{min}$, and $\boldsymbol{u}_{max}$ are the state and control limits.
We represent these behavioural parameters for all players using $\boldsymbol{\theta}$.
% $\boldsymbol{\theta} = [w_{s}, d, w_{o}, d_{o}, \boldsymbol{x}_{min}, \boldsymbol{x}_{max}, \boldsymbol{u}_{min}, \boldsymbol{u}_{max}]^\top$ are the parameters from the costs and constraints such as the weights, preferred distance from other agents and obstacles, and limits for the states and controls.
% These parameters model the behaviours of each player in a game-theoretic setting.

\begin{figure}
    \centering
    \includegraphics[width=\linewidth]{quantitative_3.jpg}
    \vspace{-8mm}
    \caption{Quantitative comparison between AToM and SF in Scenario 3. We compare the number of steps the robot takes to reach the goal which reflects the efficiency. SF leads to collisions in 7 rounds, which we highlighted using red crosses.}
    % We mark the rounds where collisions happen using red crosses. AToM enables the robot to pass through the doorway efficiently and safely in all rounds of experiments.}
    \vspace{-5mm}
    \label{fig:quantitative_3}
\end{figure}


We then find a Nash equilibrium solution $\tilde{\boldsymbol{U}}_{1}, \dots, \tilde{\boldsymbol{U}}_{n}$ for $n$ players with each player's cost defined in Problem \eqref{eq:game_cost}.
At a Nash equilibrium point, no player $i$ can further decrease its cost by unilaterally changing its strategy $\tilde{\boldsymbol{U}}_{i}$ \cite{facchinei2010generalized}. 
Solving this generalized Nash equilibrium problem can be done using existing dynamic game solvers. In this work, we use ILQSolver \cite{fridovich2020efficient} which iteratively solves linear-quadratic approximations of the original differential game in multi-player settings. 
We consider it an algorithmic module $\GameSolver$ parameterized by $\boldsymbol{\theta}$ that takes as input the current joint state of the system, the known goal positions and environmental obstacles, and returns an open-loop joint strategy that satisfies global Nash equilibrium:
\begin{equation}
\label{eq:our_ToM}
    \tilde{\boldsymbol{U}}_{H}, \tilde{\boldsymbol{U}}_{R} = \GameSolver(\boldsymbol{x}_{HR}^{t}, \boldsymbol{g}_{HR}, \mathcal{O}, \hat{\boldsymbol{\theta}}),
\end{equation}
where $\tilde{\boldsymbol{U}}_{H}$ is the predicted human action and equivalent to $\boldsymbol{\hat{U}}_{H}$, 
$\tilde{\boldsymbol{U}}_{R}$ is the human-predicted robot action, 
and $\hat{\boldsymbol{\theta}}$ is the human-predicted behavioural parameters. 
With this reformulation, we can now easily obtain $\hat{\boldsymbol{X}}_{H}$ from Eq. \eqref{eq:our_ToM} and human dynamics $\Dynamics_{H}$, and therefore solve Problem \eqref{eq:robot_planning} without any recursive structure.
% With this reformulation, we can now substitute Eq. \eqref{eq:our_ToM} and human dynamics back to the robot predictive planning problem in Eq. \eqref{eq:robot_planning} without any recursive structure.

Existing game-theoretic robot planners use similar formulations to solve for the optimal robot plans using known cost functions for each agent \cite{tian2022safety, le2021lucidgames, hu2023emergent}. 
A fundamental difference in this work is that our human internal model $\GameSolver$ is parameterized by $\hat{\boldsymbol{\theta}}$, which represents a dynamic human belief instead of the true agent parameters.

After obtaining predicted human trajectories $\hat{\boldsymbol{X}}_{H}$, the downstream robot planner can now perform predictive planning and execute the actual robot plans. 
As we do not assume any leader-follower structure, the human simultaneously performs her/his actions. 
Up to this point, we have obtained four sets of trajectories: 
1) Predicted Human $\hat{\boldsymbol{X}}_{H}$, 
2) Human-predicted Robot $\hat{\boldsymbol{X}}_{R}$, which can be obtained from $\tilde{\boldsymbol{U}}_{R}$
3) Observed Human $\boldsymbol{X}_{H}$, 
4) Observed Robot $\boldsymbol{X}_{R}$. 
These predicted-observed pairs can be used to correct the estimated behavioural parameters $\hat{\boldsymbol{\theta}}$ in the human internal model.

We use Unscented Kalman Filter (UKF) \cite{wan2000unscented} as the update mechanism. We allow $\hat{\boldsymbol{\theta}}$ to evolve using a random walk as the process model $\ProcessModel$. The measurement states are the agents' trajectories and the measurement model $\MeasurementModel$ is therefore the game solver $\GameSolver$ combined with agent dynamics.
\begin{equation}
\begin{aligned}
    \boldsymbol{\theta}^{t+1} &= \ProcessModel(\boldsymbol{\theta}^{t}, \delta^{t}) = \boldsymbol{\theta}^{t} + \delta^{t}, \quad \delta^t \sim \mathcal{N}(0, Q_t), \\
    \boldsymbol{x}^{t+1} &= \MeasurementModel(\boldsymbol{x}^{t}, \boldsymbol{\theta}^{t}) \\
            &= \Dynamics(\boldsymbol{x}^{t}, \GameSolver(\boldsymbol{x}^{t}, \boldsymbol{\theta}^{t})) + \epsilon^{t}, \quad \epsilon^t \sim \mathcal{N}(0, R),
\end{aligned}
\end{equation}
where $Q_{t}$ and $R$ are the covariance for the process model and measurement model.
In this way, our adaptive ToM human model can be adjusted dynamically to improve its prediction accuracy, and to reflect how humans update their beliefs on others. The complete procedure is detailed in Algorithm \ref{alg:our_method}, where $\Sigma$ is the covariance of $\theta$ estimation.
At the observation step, the robot executes the planned action from the predictive planner and the true human motion is observed.

% \vspace{-1mm}
\begin{algorithm}
\caption{Predict-Observe-Update Procedures with AToM}
\label{alg:our_method}
\begin{algorithmic}[1]
\State \textbf{Inputs:} $\boldsymbol{x}_{HR}^{t}, \hat{\boldsymbol{\theta}}^{t}, \Sigma^{t}, \boldsymbol{g}_{HR}, \mathcal{O}$
\State \textbf{for} $k = t, t+1, \dots$ \textbf{do}
    \State \quad $\tilde{\boldsymbol{U}}_{HR} \gets \GameSolver(\boldsymbol{x}_{HR}^{k}, \boldsymbol{g}_{HR}, \mathcal{O}, \hat{\boldsymbol{\theta}}^{k})$ 
    \State \quad $\hat{\boldsymbol{X}}_{HR} \gets \Dynamics_{HR}(\boldsymbol{x}_{HR}^{k}, \tilde{\boldsymbol{U}}_{HR})$\Comment{Predict}
    \State \quad $\boldsymbol{x}_{R}^{k+1} \gets \text{RobotPlanner}(\boldsymbol{x}_{R}^{k}, \hat{\boldsymbol{X}}_{H}, \boldsymbol{g}_{R}, \mathcal{O})$
    % \State \quad $\boldsymbol{x}_{H}^{k+1} \gets \text{Human}$ \Comment{Observation}
    \State \quad observe human state $\boldsymbol{x}_{H}^{k+1}$ \Comment{Observe}
    \State \quad $\hat{\boldsymbol{\theta}}^{k+1}, \Sigma^{k+1} \gets $
    \Statex \quad $\text{UKF}(\boldsymbol{x}_{HR}^{k+1}, \hat{\boldsymbol{x}}_{HR}^{k+1}, \hat{\boldsymbol{\theta}}^{k}, \Sigma^{k}, \ProcessModel, \MeasurementModel)$ \Comment{Update}
\State \textbf{end for}
\end{algorithmic}
\end{algorithm}
% \vspace{-2mm}

% \begin{equation}
% \begin{aligned}
%     \theta^{t+1} &= \ProcessModel(\theta^{t}, \delta^{t}) \\
%                  &= \theta^{t} + \delta^{t}, \quad \delta^t \sim \mathcal{N}(0, Q_t), \\
%     X^{t+1} &= \MeasurementModel(X^{t}, \theta^{t}) \\
%             &= X^{t} + F(X^{t}, GameSolver_{\theta^{t}}(X^{t})).
% \end{aligned}
% \end{equation}

\section{\MR Attacks and Surveys}
\label{sec:relwork}
%While there are significant advances in the field of \MR security, 
Our literature review categorizes unique aspects of \MR security into three distinct areas: User Manipulation and Deception, Privacy and Data Security, and Frameworks and Surveys.

\subsection{User Manipulation and Deception}
%\MR and \VR technologies have the potential to be excellent environments for manipulating users' perceptions. 

Prior work explored techniques to manipulate facets of users' perceptions and decision-making in \MR. 
Casey et al. \cite{Casey_2021} introduced new proof-of-concept attacks that pose a threat to user safety in a \VE. Their work categorized and defined the following attack types: chaperone, disorientation, human joystick, and overlay. The human joystick successfully manipulated users to move to specific physical locations without their awareness. The chaperone attacks manipulated the \VE boundaries, while the disorientation attack elicited a sense of dizziness and confusion from an immersed \VR user. Lastly, in an overlay attack, an adversary overlaid objects such as images and videos onto a user’s \VR view.
Chandio et al.~\cite{chandio2024stealthy} introduced stealthy and practical multi-modal attacks on \MR tracking, showing that \MR systems relying on sensor fusion algorithms for tracking can be compromised through perceptual manipulation by attacking multiple sensing streams simultaneously.

Nilsson et al. \cite{nilsson201815} provided an overview of \RDW techniques in \VR that use subtle manipulations of gains and overt redirection techniques to manipulate user's perception of space and movement. Brinkman \cite{Brinkman} describes attacks that subtly influence user choices without their awareness as decisional interference.
%Through a series of experiments involving card sorting tasks, the study shows that even when \AR systems are intended to deliberately mislead users, the majority fail to notice the deception. 
De Haas \& Lee \cite{deHaas2022audio} provide a comprehensive analysis of the manipulative potential of audio effects design in \AR which systematically categorizes deceptive audio cues into various categories, each of which uniquely influences user perception and behavior.
Wang et al. \cite{wang2023dark} further investigate how these deceptive design techniques, known as dark patterns, can manipulate users in \AR environments and compromise their information and safety.
% Moved Trivers work from the memory section
Building on psychological aspects of manipulation, Trivers \cite{Trivers2011Deceit} describes how deception is a natural part of life, not just for humans but all living beings.
%describes the relationships between deceit and self-deception, emphasizing their cognitive cost and potential benefits in various interactions. 
This analysis provides a foundational understanding of the psychological dynamics at play, illustrating how \MR systems can exploit the natural tendencies of humans to manipulate and be manipulated, influencing user perception and decision-making.

%``\acp{PMA} are a class of potential attacks aiming to manipulate a human’s multi-sensory perceptions of the physical world to influence users' decision-making and even lead to physical harm through the presented \MR output stimuli'' \cite{Tseng_2022, cheng2023exploring}. 
\acp{PMA} attempt to exploit a user's sensory perceptions to influence their decision-making, which can lead to physical harm \cite{Tseng_2022, cheng2023exploring}.
Ali et al.~\cite{Ali_Mahmood_Qadri_2018} investigated visual deception by creating illusions of 3D views using projections onto 2D surfaces.
%as an example of a \PMA, which is investigated  in the work done by 
Cheng et al.~\cite{cheng2023exploring} derived a framework for comprehending and addressing \acp{PMA} within the context of \MR. 
%They addressed three specific scenarios: 
They demonstrated that \acp{PMA} can manipulate user perceptions to affect reaction times. 
%They emphasized the potential of visual and auditory \acp{PMA} to divert users' attention and impair their ability to focus on tasks. Additionally, 
They investigated the effects of \acp{PMA} on situational awareness, revealing how \MR content can divert users' attention away from essential real-world stimuli, undermining their concentration and attentiveness. 
Ledoux et al.~\cite{ledoux2013using} found that visual cues in \VR can evoke food cravings, showing how sensory manipulations influence user perception. % that can be extended to \acp{PMA} in \MR. 
Tseng et al.~\cite{Tseng_2022} investigated the risks of perceptual manipulations in \VR, focusing on the negative impacts that these manipulations may have on users.

\gap{Most recent research on user manipulation and deception has focused on \VR systems, leaving \AR systems underrepresented. Future research should prioritize \AR security.}

% \subsection{Method}
% The related works can be classified into two primary categories: studies that concentrate on security attacks within the domains of Virtual Reality and Augmented Reality, as well as broader Mixed Realities, and studies that concentrate on Perceptual Manipulation Attacks and establish the terminology and attributes of them.

\subsection{Privacy and Security}
% \MR, \VR, and \AR technologies offer immersive experiences but also raise deception, privacy, and data security concerns. Ling et al. \cite{inproceedings1} expose \VR system vulnerabilities to side-channel attacks, revealing how they can be exploited for deception. Yarramreddy et al. \cite{Yarramreddy} explore the forensic implications of deceptive practices in immersive \VR systems, showing how forensically relevant data can be reconstructed from network traffic and system logs. Casey et al. \cite{Casey} contribute to the detection of deceptive activities with an open-source \VR memory forensics plugin for the Volatility Framework. Nair et al. \cite{nair2023exploring} uncover serious privacy risks in \VR environments, detailing various adversaries capable of leveraging deceptive techniques to extract personal data covertly. They caution against adversarially designed \VR games that exploit user trust to gather sensitive information.
%\MR and \VR not only offer immersive experiences that can be used for deception and manipulation but also 
\addition{MR and VR headsets pose significant challenges for privacy and security.
These headsets collect, use, and present personal information, making them vulnerable to information leaks via side-channel attacks.
Further, attackers can use deception attacks to disrupt information channels and cognitive processes causing users to take actions that may expose additional personal information.}


\addition{%Prior work has explored vulnerabilities that expose users' personal information.
Slocum et al.~\cite{slocum2023going} introduced TyPose, which uses machine learning techniques to classify motion signals from \MR headsets by analyzing subtle head movements made by users when interacting with virtual keyboards. 
Al Arafat et al.~\cite{al2021vr} presented the VR-Spy system, which utilizes the channel state information of Wi-Fi signals to detect and recognize keystrokes based on fine-grained hand movements. Su et al.~\cite{su2024remote} present a method for remotely extracting motion data from network packets and correlating them with keystroke entries to obtain user-typed data such as passwords or private conversations.} 
Ling et al. \cite{ling2019know} highlighted the vulnerability of \VR systems to novel side-channel attacks. 
They showed how these attacks exploit computer vision and motion sensor data to infer keystrokes in a \VE. 
\addition{Knowing what information a user is typing or specific personal details could help attackers develop more believable deception attacks.}

Vondráček et al.~\cite{vondravcek2023rise} introduced the Man-in-the-Room attack in \VR, where an attacker gains unauthorized access to a private \VR room and observes all interactions. 
\addition{Through observation, attackers can develop more targeted deception attacks.} 
Nair et al. \cite{nair2023exploring} outlined significant privacy risks in \VR environments, proposing a threat model with four adversaries: Hardware, Client, Server, and User. These adversaries have access to different aspects of the \VR information flow. These risks can covertly reveal personal data, and adversarially designed \VR games may manipulate users into disclosing sensitive information.

%They implemented \VR-specific worms and botnets, showing the malware propagation within \VE.  
\addition{Prior work has also explored the digital forensics of VR headsets.}
Yarramreddy et al.~\cite{Yarramreddy} presented an exploration of the forensics of immersive \VR systems, which demonstrates the feasibility of reconstructing forensically relevant data from both network traffic and the systems themselves. 
Casey et al.~\cite{Casey} introduced the first open-source \VR memory forensics plugin for the Volatility Framework. 
\addition{Using forensic techniques could allow an attacker to uncover personal information about a user's behavior or interest, which could be leveraged for deception attacks.}

% \removal{One significant privacy challenge for VR is the use of side-channel attacks to infer user input. Slocum et al.~\cite{slocum2023going} introduced TyPose, which uses machine learning techniques to classify motion signals from MR headsets by analyzing subtle head movements made by users when interacting with virtual keyboards. In contrast, Al Arafat et al.~\cite{al2021vr} presented the VR-Spy system, which utilizes the channel state information of Wi-Fi signals to detect and recognize keystrokes based on fine-grained hand movements. Su et al.~\cite{su2024remote} propose a method for remotely extracting motion data from network packets and correlating them with keystroke entries to obtain user-typed data such as passwords or private conversations.} 

% Nair et al. in their recent study \cite{nair2023exploring}, highlight serious privacy risks in \VR environments and mention that the \VR privacy threat model includes four potential adversaries, each associated with different aspects of the \VR information flow, which include the Hardware Adversary, with access to raw sensor data; the Client Adversary, representing the \VR application developer with full API access; the Server Adversary, controlling the external server for multiplayer functionalities; and the User Adversary, a second end-user in the same multiplayer application. They suggest that these privacy risks can covertly infer personal data attributes, and adversarially designed \VR games can trick users into revealing personal information.

% Roesner et al. \cite{10.1145/2580723.2580730} provide a comprehensive analysis of security and privacy in augmented reality \AR, revealing unique vulnerabilities in \AR browsers. Another study \cite{10.1145/2736277.2741657} conducts the first system-level assessment of security features in \AR browsers. Lebeck et al. \cite{10.1145/2873587.2873595} introduce Arya, an \AR platform regulating application output to mitigate risks from malicious or faulty applications. This focus is complemented by additional research \cite{inproceedings3} exploring input privacy risks and malicious \AR output.

\addition{
%In addition to privacy concerns, 
Security issues can expose \MR users to physical harm and potential deception attacks.}
Odeleye et al.~\cite{odeleye2021detecting} showed attacks targeting GPU and network vulnerabilities in \VR systems to manipulate frame rates and cause \VR sickness. 
Roesner et al.~\cite{Roesner2011Security} conducted a comprehensive examination of security and privacy concerns in \AR, unveiling new vulnerabilities unique to \AR applications. 
\addition{For example, they suggest displaying the provenance of AR elements so that users know the source of augmentations. Without this, users are susceptible to deception attacks that inject false information.}
McPherson et al.~\cite{10.1145/2736277.2741657} conducted the first system-level assessment of security and privacy features in \AR browsers. 
Lebeck et al.~\cite{10.1145/2873587.2873595} introduced Arya, an \AR platform aimed at regulating application output to mitigate risks from malicious or faulty applications. 
This focus on output security is complemented by research delving into input privacy risks and the largely unexplored area of malicious \AR output \cite{inproceedings3}.
Cheng et al. \cite{cheng2024user} introduced several proof-of-concept attacks targeting \UI security vulnerabilities in \AR systems. %\addition{By compromising UI security, attackers can undermine user trust by directly affecting user interactions.}
Slocum et al. \cite{Slocum2024Shared} investigate the security vulnerabilities in multi-user \AR applications, focusing on the shared state that maintains a consistent virtual environment across users.

% The paper by Roesner et al. \cite{10.1145/2580723.2580730} provides a comprehensive overview of security and privacy issues in augmented reality \AR, identifying new categories of vulnerabilities unique to \AR browsers and another study \cite{10.1145/2736277.2741657} offers the first system-level evaluation of the security and privacy properties of \AR browsers. Addressing the output side of \AR security, the work by Lebeck et al. \cite{10.1145/2873587.2873595} introduces Arya, an \AR platform designed to control application output according to specified policies, thereby mitigating risks from malicious or buggy applications. This focus on output security is further elaborated by research \cite{inproceedings3} that explores both input privacy risks and the largely unexplored domain of malicious \AR output.

\begin{figure*}[t!]
    \centering
    \includegraphics[width=\textwidth]{figures/mindmap.pdf}
    \caption{Mind Map of \MR Deception Attacks Ontology. Channel attacks on the left. Processing attacks on the right.} % Goals are labeled for Degradation, Denial, Subversion, and Corruption Attacks.}
    \label{fig:attacks-mindmap}
    \vspace{-1ex}
\end{figure*}


\subsection{Frameworks and Surveys} %Security Attacks within \MR}

%Regarding security threats in the context of \MR, 
%Rokhsaritalemi et al.~\cite{article1} categorized \MR systems into five layers: \MR concepts, system architecture, design integration issues, and interfaces.
Garrido et al.~\cite{garrido2023sok} systematized knowledge on \VR privacy threats and countermeasures, %, and sensitive personal information within telemetry data.
%They successfully outline privacy opportunities for practitioners. 
focusing on two types of attacks: profiling and identification.
Profiling attacks collect sensitive personal data to create user profiles.
Identification attacks uniquely pinpoint a user within a \VR environment. %, often by collecting and combining several data points.
Happa et al.~\cite{10.3389/fict.2019.00005} developed an abstraction-based reasoning framework to reveal possible attacks in collaborative \MR applications.
%conducted a comprehensive examination in their study,
De Guzman et al. \cite{De_Guzman_2019} provided a survey of various protection mechanisms proposed for \MR. 
Adams et al.~\cite{219386} conducted interviews with \MR users and developers to survey \MR privacy policies and their perceptions. %focused on \MR security and privacy perceptions by 
Stephenson et al. \cite{stephenson2022sok} systematized knowledge on \AR/\VR authentication mechanisms, evaluating research proposals and practical deployments.

\gap{There is a notable lack of frameworks that address diverse aspects of \MR security, including technical exploits, user experience, detection, and defense.}

% oLD
% \gap{There is a lack of integrated frameworks that address different aspects of \MR security.}
\section{\MR Deception Attacks Ontology}
\label{sec:decattacks}

We derive a \MR deception attack ontology (Figure~\ref{fig:attacks-mindmap}) from our review of the literature, our expert knowledge, and the Borden-Kopp model~\cite{borden1999information,kopp2000information,brumley:2012}.  
\addition{Their model focuses on how deceptions alter a victim's decision-making by manipulating information channels to inject false information or hide true information.
Additionally, they identify how false information can be used to induce biases that influence how information is processed and interpreted. 
As MR headsets directly transmit information to users, their information-theoretic model provides an appropriate and robust framework for describing, categorizing, and analyzing MR deception attacks.
}
%Figure~\ref{fig:attacks-mindmap} shows a mind map of the ontology. 
The Borden-Kopp model divides deception attacks into \textit{channel attacks} and \textit{processing attacks}. 

%The model's emphasis on both manipulating information channels and altering information processing provides a robust framework for understanding deception within MR systems.

%The Borden-Kopp model specifies four forms of manipulation that can mislead perceptions of news media: \textit{Degradation}, \textit{Denial}, \textit{Subversion}, and \textit{Corruption}. They classify Degradation and Denial as \textit{Channel Attacks} and Subversion and Corruption as \textit{Processing Attacks}. 
%Table~\ref{tab:attacks} categorizes technical attacks from the literature review using our ontology.

\subsection{Channel Attacks}
Channel attacks primarily target information communication channels. 
These attacks exploit the physical or logical paths that data takes as it moves between different components of a system or between different entities. 
The Borden-Kopp model identifies three types of channel attacks: \textit{overt degradation}, \textit{covert degradation}, and \textit{denial}.
%Degradation and Denial models are two forms of these attacks.

\subsubsection{Overt Degradation} 
     % In overt degradation, an attacker introduces recognizable noise, disrupting the signal and causing uncertainty, confusion, or false perceptions.
     % For example, adding a repetitive, annoying sound makes it difficult for a user to process task-related information and may evoke frustration that leads to uncertainty.
     %to introduce uncertainty or to mislead perception in the belief of a user by adding noise or background messages to the \MR environment. 
     %Degradation is either overt (active) or covert (passive). 
     With overt degradation, attackers of \MR systems create confusion by introducing substantial visual, auditory, or tactile noise to prevent victims from accurately perceiving or engaging with virtual objects, the physical world, or associated tasks.
     Due to the blatant nature of overt degradation, victims become aware that they are under attack.
     %the attacker generates significant noise to prevent information processing, but additionally revealing that an attack is taking place. 
     %the objective is to introduce doubt, confusion, or uncertainty. 
     The presence of virtual noise can be disorienting in the context of \MR, as users heavily depend on the seamless integration of real and virtual information in order to maintain focus on a task. 
     \addition{Further, it can disrupt immersive experiences, preventing users from becoming fully engaged in a task.}
     We identify the following forms of overt degradation attacks:
     %This deception tactic can be exploited through various methods which encompass:
     
     \begin{itemize}
     \itemsep0em
     \item 
     \emph{Sensory Overload}: Inundate the user's sensory receptors with excessive amounts of stimuli, leading to disorientation or distraction \cite{Roesner2011Security, odeleye2021detecting}. 
     %According to the \APA Dictionary of Psychology~\cite{10dbfdda-1535-34bd-8b0c-3f342ddcf745}, disorientation is ``a state of confusion or loss of awareness of one's surroundings.'' 
     Disorientation can cause a user to feel lost or confused within a \VE, making them more susceptible to manipulation. 
     %Distraction is ``any stimulus or event that diverts attention away from the task at hand.'' 
     Distraction diverts the user's attention, potentially preventing them from detecting or responding to an attack.
     
     \item \emph{Momentary Misdirection}: Redirect the user's attention using virtual content within a \MR systems. Misdirection distracts the user from their task. For example, an attacker can insert flashing virtual elements that draw the user's visual attention away from seeing important information or activities in the physical world.
     
     \item  \emph{Signal Replacement}: Alter or replace sensory input within \MR systems. This can lead to a user perceiving a different reality from what actually exists, potentially causing confusion, disorientation, or exploitation \cite{Tseng_2022}.
     
     \item  \emph{Quality Erosion}: Reduce the quality of the signal from the \MR headset. This can be achieved through actions such as decreasing the resolution of visual elements, introducing distortions to audio, or reducing the vibration intensity of haptic feedback.
     \end{itemize} 





\subsubsection{Covert Degradation}
      Covert degradation attacks subtly suppress or diminish the clarity of information presented by \MR headsets.
      Attackers can blend deceptions seamlessly with the \MR environment, thereby making it harder for the user to discern.
      %utilize a concealed attack method, leveraging the virtual background elements or ambient noise intrinsic to the \MR setting. The objective is to . 
      \addition{By leveraging immersive MR experiences, deception attacks can mask false information as users' attention and interactions are focused elsewhere.}
      We identify the following forms of covert degradation attacks:
      
     \begin{itemize}
     \itemsep0em
     \item \emph{Physical Movement Manipulation}: Relocate a user without their awareness or consent by discreetly shifting the center of a \VE while they focus on a task%, with users naturally adjusting as they focus on their tasks
     ~\cite{Casey_2021}.
     
     % user's movements or interactions within the virtual environment are taken over by an attacker without authorization. They have the ability to cause confusion and discomfort by altering the user's movements, interactions, or sensory experiences.
     \item \emph{Boundary Manipulation}: Altering boundaries within the \VE, which can lead to unexpected collisions with objects or distortions in spatial perception~\cite{schmidt2019blended}.
     
     \item \emph{Dimension Manipulation}: Modifying the proportions, scale, or spatial relationships of virtual objects~\cite{bozgeyikli2021evaluating}.
     \end{itemize} 

\subsubsection{Denial}

Denial attacks seek to increase uncertainty by obstructing the user's access to information. This is achieved by shutting down virtual overlays, prohibiting interaction with virtual objects, or disrupting the seamless blend of real and virtual elements. % that \MR relies upon effectively denying the user access to their own information. 
This is often an overt method of deception, as users may be cognizant of their deprived or diminished accesses \cite{kopp2003shannon}. A user may find themselves subject to a Denial attack if they lose ingress to existing networks, communication channels, and various other system features. 
%Three types of denial attacks are:
We identify the following forms of denial attacks:
    \begin{itemize}
    \itemsep0em 
    
    \item \emph{Shutdown}: Deliberately terminate or disable a \MR communication channel or service. % of the \MR system. % with the intention of disrupting its availability.
    
    \item \emph{Overlay}: Layer content over a communication channel to disrupt normal operations of the channel \cite{Lee2021AdCube, Roesner2011Security, Tseng_2022}. %. The purpose of this overlay is to 
     
    \item \emph{Removal}: Selectively remove or block information \cite{odeleye2021detecting}.
    \end{itemize} 

\begin{table*}[htbp]
\caption{Evaluation results of different models on the adversarial transferability. }
 \resizebox*{\linewidth}{!}{
\begin{tabular}{c|l|llllllll}
\toprule
 Datasets & Attacks & $R^{A}_{18}\mathop{-}\limits^{s}R^{V}_{18}$ & $R^{A}_{18}\mathop{-}\limits^{c}R^{V}_{18}$ & $R^{A}_{18}\mathop{-}\limits^{s}R^{V}_{34}$ & $R^{A}_{18}\mathop{-}\limits^{c}R^{V}_{34}$ & $R^{A}_{34}\mathop{-}\limits^{s}R^{V}_{18}$ & $R^{A}_{34}\mathop{-}\limits^{c}R^{V}_{18}$ & $R^{A}_{34}\mathop{-}\limits^{s}R^{V}_{34}$ & $R^{A}_{34}\mathop{-}\limits^{c}R^{V}_{34}$\\
 \hline
  \multirow{6}{*}{Music}
     & FGSM &  ~~~~~65.2& ~~~~~69.9 & ~~~~~72.4 &  ~~~~~75.6 &  ~~~~~67.0&  ~~~~~70.2 &  ~~~~~78.3& ~~~~~79.2\\
     & I-FGSM&  ~~~~~34.3& ~~~~~38.1 & ~~~~~40.3 &  ~~~~~43.7 &  ~~~~~35.7&  ~~~~~39.6&  ~~~~~45.2& ~~~~~45.7\\
     & MI-FGSM&  ~~~~~38.7& ~~~~~38.2 & ~~~~~40.8 &  ~~~~~44.1 &  ~~~~~39.1 &  ~~~~~40.3& ~~~~~47.2&~~~~~48.5\\
     & TIA&  ~~~~~76.5& ~~~~~77.1 & ~~~~~78.9 &  ~~~~~80.2&  ~~~~~77.6 &  ~~~~~78.0& ~~~~~84.5&~~~~~86.1\\
     & MAA&  ~~~~~45.9& ~~~~~46.6 & ~~~~~47.1 &  ~~~~~47.3 &  ~~~~~46.1 &  ~~~~~47.0& ~~~~~48.5&~~~~~49.2\\
     & TIA-MAA& ~~~~~82.1& ~~~~~84.3 & ~~~~~85.0 &  ~~~~~85.0 &  ~~~~~84.2 &  ~~~~~84.7 & ~~~~~87.9&~~~~~89.2\\
     \hdashline
     \multirow{6}{*}{K-S}
     & FGSM &  ~~~~~52.2 &  ~~~~~56.7 & ~~~~~62.9 &  ~~~~~64.5 &~~~~~54.1 &  ~~~~~55.8 &~~~~~68.0 &  ~~~~~70.3 \\
     & I-FGSM&   ~~~~~27.6 & ~~~~~29.5&~~~~~33.5 &  ~~~~~32.6 &~~~~~28.5 &  ~~~~~29.1 &~~~~~32.0 &  ~~~~~33.4 \\
     & MI-FGSM&    ~~~~~28.3 & ~~~~~30.2&~~~~~33.6 &  ~~~~~34.1 &~~~~~29.5 &  ~~~~~29.9 &~~~~~33.6 &  ~~~~~35.2 \\
     & TIA& ~~~~~67.5 & ~~~~~68.1&~~~~~70.2 &  ~~~~~70.3 &~~~~~68.5 &  ~~~~~68.2 &~~~~~74.5 &  ~~~~~75.3 \\
     & MAA&  ~~~~~34.4 & ~~~~~35.1 &~~~~~38.6 &  ~~~~~39.0 &~~~~~36.5 &  ~~~~~37.1 &~~~~~40.2 &  ~~~~~41.9 \\
     & TIA-MAA& ~~~~~70.8 & ~~~~~72.5&~~~~~73.7 &  ~~~~~74.2 &~~~~~73.3 &  ~~~~~73.9 &~~~~~78.3 &  ~~~~~79.0 \\
     \bottomrule
\end{tabular}}
\label{tab:attack}
\end{table*}

\subsection{Processing Attacks}
Processing attacks target vulnerabilities in how humans cognitively process information, aiming to deceive humans by altering their perceptions, interpretations, and understandings of information. 
The Borden-Kopp model identifies two types of processing attacks: Corruption and Subversion.
%Corruption and Subversion are two categories of processing attacks that exploit this method to deceive a user.

    \subsubsection{Corruption} 
Corruption attacks deliberately manipulate the \MR system by counterfeiting existing virtual elements and information.
These manipulations result in inconspicuous data and actions that are difficult to discern from standard data and actions within the \MR system.
Their primary objective is to create false belief in a user, often causing compromised decision-making, incorrect conclusions, or virtual misdirection. 
\addition{Due to the immersiveness of MR, users may be more susceptible to corruption attacks as their engagement keeps them preoccupied, preventing critical analysis of false information.}
We identify the following corruption attacks:
     
     \begin{itemize}
     \itemsep0.5em 
     \item \emph{Spoofing}: Create or modify data in a way that deceives the recipient or system into believing that the data is authentic or unaltered. %Spoofing can take many forms. 
     Two forms of spoofed data are:
     %Two common types are software telemetry and hardware telemetry ~\cite{zhang2023s, al2021vr}: 

         \begin{itemize}[leftmargin=3mm]
         \itemsep0em
         \item \emph{Software Telemetry}: Alter or fabricate telemetry data from software. Attackers create or manipulate telemetry messages that convey a normally functioning application. Further, attackers may spoof telemetry messages at the system level, affecting multiple applications or impacting critical systems \cite{chandio2024stealthy}. %These messages provide information about the status of the entire system. By altering them, an attacker can hide the presence of failures or other critical system issues.
         
         \item \emph{Hardware Telemetry}: Alter or fabricate telemetry data from hardware sensors. Attackers can generate false sensor readings. Alternatively, attackers can manipulate input data from \MR headsets or peripherals, such as controllers, enacting undesired actions or preventing users from performing desired tasks \cite{tu2018injected, chandio2024stealthy}. %This can lead to input spoofing attacks, where the attacker sends fake input signals to the system, potentially causing unexpected or malicious behavior.
         \end{itemize}
     
     \item \emph{False-Flag Operations}: Disguise the source of an attack in order to blame another party. %Generate confusion and redirect focus away from an attacker's actions.
     \end{itemize} 

     \subsubsection{Subversion} 
     Subversion attacks covertly manipulate a system or its information streams, resulting in falsified and fabricated interpretations by the user. Subversion often employs covert tactics, such as corruption attacks, which weaken trust or disrupt normal operations. 
     \addition{We suspect that the immersiveness of MR can aid false interpretations as users unknowingly engage with deceptive information through repeated interactions, which can correspondingly build trust in deceptive elements.}
     We identify the following subversion attacks:
     \begin{itemize}
     \itemsep0em

     \item \emph{Bias Attacks}: Deliberate manipulation of data or decision-making processes to systematically introduce bias or prejudice toward a specific concept or outcome.
     %\item \emph{A-Priori Attacks}: The attacker manipulates perceptions and decisions by taking advantage of existing beliefs, prejudices, or preconceptions.
     \item \emph{Disinformation}: Spread false information to deceive and cause harm~\cite{guess2020misinformation}.
     \item  \emph{Lure}: Entice users to engage with (harmful) content.
     \item \emph{Propaganda}: Manipulate perceptions, influence narratives, and garner support for a specific cause or element.
     \item \emph{Gaslighting}: Erode trust and confidence, making it difficult for victims to distinguish truth from deception.
     \end{itemize}

%By applying the Borden-Kopp model of deception to \MR systems, we have successfully categorized deception attacks into five different types: Overt Degradation, Covert Degradation, Denial, Corruption, and Subversion. 
%Based on an extensive literature review, this categorization provides valuable insights for researchers and practitioners. 
%Our systematic approach serves to clarify the \MR systems' potential vulnerabilities, thereby directly addressing the RQ1}.

\subsection{Connecting Technical Attacks to Ontology}

\MR deception attacks in our ontology typically rely on technical attacks to facilitate access to \MR systems.
Table~\ref{tab:attacks} characterizes the modalities and deception attacks supported by each technical attack identified in our literature review.
For each technical attack, we identify deception attacks directly mentioned by the authors (\scaledDing{108}) and deception attacks where the technical attack could be deployed but was not specifically mentioned by the authors (\scaledDing{109}).
We found more Channel Attacks (23) mentioned than Processing Attacks (8). 
This is not surprising considering that technical attacks typically target system-level functions which have more impact on the communication channels of \MR headsets than user's cognitive processes.
Still, we see seven attacks that mention Corruption or Subversion, and another eleven that we consider capable of supporting Processing Attacks.

\gap{State-of-the-art \MR technical attacks predominately enable Channel Attacks. More research is needed on technical attacks that facilitate Processing Attacks and how these attacks affect \MR users. }

% We need investigations of technical attacks that support processing attacks.}

We identify the sensory modalities affected by an attack and the technical modalities it targets. 
Sensory modalities include visual, auditory, and tactile (e.g., vibrotactile feedback from controllers).
Technical modalities include hardware, software, network, data, and side-channel~\cite{attkan2022cyber}.
%While we found attacks for all modalities, it is clear that visual and software modalities are the primary targets of existing technical attacks.
%The least targeted modalities are tactile and hardware.

\finding{Technical attacks primarily target the visual and software modalities. \MR headsets include displays and processors, making visual and software modalities convenient targets. These attacks particularly focus on overlaying content or replacing signals as opposed to overloading, eroding, or removing signals. The least targeted modalities are tactile and hardware.} 
\section{Information Theory and Deception Attacks}

\label{sec:information-theoretic}

\addition{While our ontology categorizes \MR deception attacks, it does not explore the effects of these attacks.} To address \textbf{RQ2}, we use Kopp~et al.'s framework \cite{kopp:2018}, which connects Borden-Kopp's deception model \cite{brumley:2012} and Shannon's communication model \cite{shannon1948mathematical}, to derive an information-theoretic model of \MR deception attacks. %We establish an information-theoretic model that provides a structured method for understanding deception attacks within \MR systems.
%, shown in Figure~\ref{fig:model}. 
%The Shannon model of communication was modified to illustrate the process of deceptive attacks in the context of mixed reality. Significantly, we incorporate principles derived from the. 
%The Borden-Kopp model offers a comprehensive framework for understanding the manipulation of information with the intention of deceiving recipients. 
%According to Borden-Kopp, the ultimate objective of a deceptive attack is to manipulate the recipient's perception which could encompass the act of providing users with inaccurate information regarding the spatial positions of virtual objects, manipulating auditory signals, or even introducing illusory tactile sensations.
%Our information-theoretic model of deception attacks in \MR is . 
% \removal{Unlike other models of deception that focus on interpersonal communication~\cite{buller:1996,mcwhirter:2016,levine:2022} or emotions~\cite{gaspar:2013,gaspar:2022,kang:2022}, the Borden-Kopp model emphasizes how deceptions in news media can target both communication channels and cognitive processes.
% Other models lack clear connections between a deception and how it can be enabled through digital media.
% Correspondingly,} 
Shannon's communication model describes how information is transferred from a source to a destination as a message.
The message is sent as a signal through a transmitter to a receiver.
During transmission, the message is affected by noise, which combines with the signal.
In our model, the transmitter is an \MR headset, which acquires information from a source (e.g., an application, sensor, web service), and transmits that information in visual, auditory, or tactile forms to a user (destination) via displays, speakers, and controller vibrations (Figure~\ref{fig:mr-communication-model}).
\MR deception attacks affect the capacity of information transmission by introducing noise to degrade messages, denying access to information, or inserting fake information into messages.

%Human sensory organs (e.g., eyes, ears, and skin) receive signals from the \MR headset, which get perceived, interpreted, and processed by the user as part of decision-making processes (Section~\ref{sec:decision-making}).
%An attacker can attempt to deceive the user by inserting fake information in the signal, denying access to information, or by producing noise.
%which encompasses a wide range of data including visual displays and auditory information. 

% Drawing from Borden-Kopp's model, we recognize that deceptive signals in MR can be introduced in various ways. For instance, in the event of a processing attack, the attacker source sends a false system message to the VR headset to replace the authentic one, resulting in the transmission of a deceptive or subversive signal.
% Alternatively, in the context of a channel attack, in the Degradation Attack, the attacker may employ a noise source. This noise serves the purpose of either generating substantial interference to hinder the player's ability to accurately perceive incoming data or producing a message that closely resembles the ambient background noise, rendering it indistinguishable from the environment. In the Denial Attack, the attacker simply blocks the output signal from the transmitter preventing the user from collecting information.


\begin{figure}[ht!]
    \centering
    \includegraphics[width=\columnwidth]{figures/shannon.pdf}
    \caption{MR Deception Information-Theoretic Model. Messages are transmitted by a \MR headset to a user. Deceptive messages are injected into transmissions. Noise from the attacker or environment affect channel capacity.}
    \label{fig:mr-communication-model}
    \vspace{-1ex}
\end{figure}
%\vspace{-3ex}

\subsection{Channel Capacity}
According to Shannon's channel capacity theorem \cite{shannon1948mathematical}, the capacity of a channel to transmit information depends upon several factors, including the magnitude of the signal used to encode symbols, the level of interfering noise present in the channel, and the bandwidth of the channel.
\begin{equation} \label{eq:channelcapcity}
    C = W \log_2\left(1 + \frac{S}{N_A + N_E}\right)
\end{equation}
Channel capacity \({C}\) represents the maximum amount of information that can be effectively transmitted from a source to a destination in bits per second (Equation~\ref{eq:channelcapcity}).  Bandwidth \({W}\) refers to the information transfer rate of the communication channel in hertz. As $W$ decreases, channel capacity correspondingly decreases through a linear relationship.

For \MR headsets, information is transmitted through a headset to a human user. 
Thus, channel capacity determines how much visual, auditory, and tactile information can be transmitted. 
%that delivers the virtual content to the headset. 
Signal \({S}\) is the virtual content transmitted from the headset through displays, speakers, and vibrotactile motors. 
%The stronger the signal, the more realistic and immersive the \MR experience becomes. 
Noise \({N}\) is categorized into two types: \({N_A}\) which represents noise from an attacker source, and \({N_E}\), which represents noise from the real-world environment as well as noise that comes from the system itself, such as rendering stutters or audio glitches. 
\({N_A}\) refers to potential external interference or malicious disruptions.
\({N_E}\) encompasses both ambient disturbances from the surrounding environment and internal system issues that can affect the \MR experience. 
Both these sources of noise have a negative effect the channel capacity. % and correspondingly the overall quality of the \MR experience. 

\subsection{Channel Attacks}
Channel attacks target channel capacity through reducing bandwidth, manipulating the signal, or introducing noise.
Denial attacks involve an adversary's intention to significantly reduce access to the signal by primarily manipulating bandwidth. The channel capacity \({C}\) tends to zero as the bandwidth \({W}\) tends to zero. %, which means that the available bandwidth for transmitting virtual content is severely reduced. 
%As shown in Figure~\ref{fig:attacks-mindmap} 
%Shutting down the headset, overlaying malicious data, or removing some of the virtual content, are different types of Denial Attacks, and all effectively reduce the bandwidth.
By shutting down the device, the attacker completely blocks the signal output, and bandwidth ($W$) reduces to zero. % which means no information will flow through the information channel.
Attackers can occlude task-specific information with other content, effectively reducing bandwidth and interfering with task performance. 
%access to  interference and reduces the effective capacity of the channel for transmitting meaningful information.
A Removal attack selectively removes information from the signal, reducing bandwidth as less information is transmitted per a second. % and the channel's effective capacity for conveying useful information.

%In Overt Degradation attacks, the adversary can introduce substantial levels of noise into the channel, decreasing the \SNR. As the \SNR tends towards zero,  \({log_2\left(1 + \frac{S}{N_A + N_E}\right)}\) approaches zero. Consequently, this causes the channel capacity \({C}\) to decrease and eventually reach zero. 
In Overt Degradation attacks, the adversary can introduce substantial levels of noise into the channel, decreasing the \SNR. As the \SNR tends towards zero, channel capacity \({C}\) decreases and eventually reaches zero.
In this case, the user is bombarded with excessive noise, making it impossible to distinguish between the intended content and the attacker's noise. 
%According to Figure~\ref{fig:attacks-mindmap} 
An example of this attack is sensory overload, where an attacker overwhelms the user by emitting excessive sensory stimuli through the \MR headset, resulting in disorientation and discomfort.

In Covert Degradation attacks, an adversary can reduce the signal strength, which results in a decrease in the \SNR. As the signal tends toward zero, \SNR also tends toward zero, decreasing \({C}\) towards zero as well. In \MR headsets, these attacks can involve subtle manipulation of sensory cues presented to a user. Subtle boundary manipulation and subtle dimension manipulation are examples of these attacks. Through subtle manipulation of boundaries or the sizes of virtual objects, the attacker can deceive the user into thinking they are not moving~\cite{Casey_2021} or make it harder to interact with virtual objects.
%might move the boundaries of the \MR environment. % or employ misdirection. The goal of these two attacks is to reduce the signal strength.

\subsection{Processing Attacks}
Processing attacks manipulate cognition through deceptive methods that mimic the \MR system.
We use Vitanyi's model~\cite{Vitanyi} to formalize how deceptive information and messages created by an attacker, \({X}\), differ from actual information and messages created by an \MR system, \({Y}\):

\begin{equation}
D(X, Y) = \frac{K(XY) - \min(K(X), K(Y))}{\max(K(X), K(Y))}
\end{equation}

\begin{equation}
M(X, Y) = 1 - D(X, Y)
\end{equation}
where \({D}\) represents the measure of difference, \({M}\) represents the measure of similarity or mimicry, and \({K}\) is the editing function applied to \({X}\) and \({Y}\).

Corruption attacks involve altering data during transmission. Vitanyi's difference measure \({D(X, Y)}\) quantifies the degree of alteration between the original message \({X}\) and the corrupted message \({Y}\).  
%A greater value of \({D(X, Y)}\) signifies a substantial level of corruption. 
In \MR, corruption attacks might involve unauthorized changes to visual information, such as application and system messages, as well as sensory information, including camera, geolocation, and battery status (Figure~\ref{fig:attacks-mindmap}).
Subversion attacks, on the other hand, involve  manipulating how users interpret information within an \MR system. These attacks require repeated corruption or covert degradation attacks to reduce user's trust and understanding.
Thus, $M$ must remain close to $1$ as the user has a greater chance of detecting deceptions through repeated exposure.
%A decreased value of \({M(X, Y)}\) signifies a heightened level of subversion, indicating a substantial departure from the anticipated output as a result of malicious manipulation.

\gap{While Vitanyi's model formalizes mimicry, we lack models that effectively describe how processing attacks impact human behavior. Specialized domains, such as formal methods in human-computer interaction, could offer valuable insights.}

% \gap{While Vitanyi's model formalizes mimicry, we lack models that adequately describe how processing attacks affect human behavior. Specialized domains, such as formal methods in human-computer interaction, may provide valuable insights. }
%\section{Deception Attacks and Decision-Making}
\section{Decision-Making and Deception Attacks}
\label{sec:decision-making}

\addition{Beyond effects on information channels, we seek to model how \MR deception attacks impact human cognition.} 
% \removal{In this section, we} 
To address \textbf{RQ3}, which concerns the interactions between decision-making and \MR deception attacks, \addition{we} conduct a thorough review of the cognition literature and develop a comprehensive decision-making model that outlines the stages of decision-making susceptible to these attacks.
Figure~\ref{fig:process-model} shows our \MR Decision-Making Model. The model provides an overview of how sensory input is cognitively processed by a user to make decisions and where the different types of attacks affect decision making.
%may harm the initial phases of human cognitive processes, subsequently leading to a chain reaction that impacts perception, decision-making, and responses. 
Our decision-making model comprises of seven components: Sensory Inputs, Attention, Perception, Memory, Decision-Making, Decision Execution, and Responses.

\begin{figure*}[ht!]
\centering
  \includegraphics[width=0.95\textwidth]{figures/decision_making_model.pdf}
  \caption{\MR Deception Decision-Making Model. External stimuli (left) are input to cognitive processes (right). Stimuli are first processed by perception. %, where stimuli are selected, organized, and interpreted. 
  Selective attention manages perception on relevant stimuli. Organized stimuli are stored in working memory. Interpreted stimuli are passed to decision-making, where executive attention manages decisions and their execution.} % Channel attacks target sensory inputs, while processing attacks target cognitive processes.}
 % are captured Channel attacks first have a direct impact on the human sensory system, and processing attacks on the other side have an impact on human cognition. The sensory input is processed through different types of attention. Perception comes next and involves three stages: selection, organization, and interpretation of sensory stimuli. Decision-making is the subsequent stage where choices are made based on perceived information, followed by decision execution, where these decisions are acted upon. The final stage is the response, which can be physiological, behavioral, or cognitive.}
  \label{fig:process-model}
 \vspace{-1ex}
\end{figure*}


\begin{itemize}
\itemsep0em
    \item \emph{Sensory Inputs}: Visual, Auditory, Smell, Taste, and Touch are the five different types of sensory inputs that can be affected by deception attacks.
    \item \emph{Attention}: Initial stage where sensory information is gathered. Provides a gateway to perception.
    \item \emph{Perception}: Sensory information gathered is processed and understood.
    \item \emph{Memory}: Processed information is stored in working or long-term memory for future use and retrieval.
    \item \emph{Decision-Making}: Determining a particular course of action predicated on perception.
    \item \emph{Decision Execution}: Decisions are executed.
    \item \emph{Responses}: Physiological, behavioral, or cognitive responses of executed decisions. %Handled by motor processor.
\end{itemize}

\subsection{Perception}
Perception refers to the cognitive process through which one comprehends sensory stimuli~\cite{qiong2017brief}.
Wang et al.~\cite{wang2007cognitive} define perception as ``a set of internal sensational cognitive processes of the brain at the subconscious cognitive function layer that detects, relates, interprets, and searches internal cognitive information in the mind.'' 
Perception is either active or passive.
Active perception involves the intentional direction of attention towards environmental stimuli to extract information \cite{gibson2014ecological}. 
In contrast, passive perception occurs without deliberate effort; sensory information is received as presented \cite{Rock1983-ROCTLO}.

% Gibson \cite{gibson2002theory} uses the term perception for any experience of the environment surrounding the body and the term proprioception or internal perception for any experience of the body itself. One example of perception is the ability of vision to identify an object's motion in relation to a motionless environment. Similarly, one example of proprioception is one's movement in relation to a fixed environment, whether active or passive can be detected by vision.

Perception involves three stages~\cite{qiong2017brief}:
\begin{itemize}
\itemsep0em 
    \item \emph{Selection:} Filter and select environmental stimuli from meaningful experiences.
    \item \emph{Organization:} Structure and categorize the selected information, creating coherent and stable perceptions through grouping by proximity and similarity.
    \item \emph{Interpretation:} Assign meaning to organized stimuli, with individuals' cultural or experiential backgrounds leading to different understandings of the same stimuli.
\end{itemize}

In each stage of perception, \MR deception attacks can target specific vulnerabilities. 
%Dionisio et al. \cite{Dionisio2001Differentiation} highlight the cognitive demands associated with generating deceptive responses, emphasizing the increased cognitive load during these processes. 
%Their findings suggest that deception in \MR not only challenges the perception stages but also significantly taxes the cognitive resources required for maintaining accurate perception.
During selection, attacks can cause sensory overload or misdirect focus on irrelevant stimuli. In the organization step, attacks could involve boundary or dimension manipulation, affecting how stimulus are structured and grouped due to changes in proximity or scale. Propaganda or spoofing attacks can target interpretation, affecting the meaning assigned to stimuli that may seem wrong but is coming from a trusted source (e.g., the system or a collaborator). These potential attacks highlight the importance of the accuracy and reliability of perception in \MR systems.

In addition to the conscious components of perception, subliminal inputs play an important role in how individuals interact with and understand \MR environments.
Cetnarski et al.\ \cite{Cetnarski2014Subliminal} show that subliminal stimuli—information presented below the threshold of conscious awareness—can significantly influence decision-making processes in \MR. This underscores the need to understand these subtle interactions that occur at the subconscious level of perception.
% Gestalt Principles of Perception 
% Perception is the main factor in defining individuals' interpretation of their environment. The accuracy and reliability of perception become critical in \MR systems, where the boundary between the real and virtual world often becomes indistinct.
%Deception attacks can exploit this by altering sensory inputs or manipulating the processing of these inputs, leading to users' misunderstanding. In \MR systems, perception is tightly connected to users' trust. When sensory inputs are reliable and the virtual environment aligns with reality, trust in the system improves. When deception attacks alter perception, mistrust in the system increases, and users may question their interaction with the system. Risk perception, also, is heavily related to trust. As users obtain an increased awareness of the reliability of their perception in an \MR environment, they might perceive increased risks in their actions. Factors like the gravity of events, media coverage, and individual attributes such as age, gender, and each person's past experiences influence how risks are perceived and subsequently, how trust in the system is established or damaged. 
%As mentioned in the previous section, there are two types of attacks on \MR systems, Channel Attacks and Processing Attacks. In our general decision-making model, these two types of attacks are interrelated with specific components of decision-making defined based on where they hold a higher impact in the cognitive decision-making process.


% \subsection{Sensory Input}

% Sensory inputs are frequently the primary focus of Channel Attacks, as their primary objective lies in heightening ambiguity and provoking misleading perceptions; This is achieved through the manipulation and alteration of users' conduct throughout their interactions with the system. For instance, Denial Attacks, have the effect of limiting the range of perceptible information, thereby forcing users to make decisions based on incomplete data. This limited input might also shift users' attention to look for the reasons behind the absence or unavailability of such information. Such situations can be exploited by attackers to divert attention from more critical areas of the system.
% Overt Degradation attacks are apparent and noticeable, making users immediately aware of the compromise in sensory input quality. Sensory overload is a vivid example of these attacks. By overwhelming users with excessive stimuli, these attacks have the potential to affect rational thinking and limit decision-making ability, which can lead to immediate distrust and a reevaluation of the system's reliability.
% Covert degradation attacks, in contrast, subtly degrade the quality of sensory inputs over time, making them harder to detect. The danger with covert attacks lies in their subtle and deceptive characteristics, which may result in users failing to immediately recognize them, leading them to make decisions based on compromised data without realizing it.
%the Human Joystick Attack has the potential to significantly impact users' tactile perception, while the Overlay attack could instead disrupt users' visual experience. This impact has a significant effect on users' decision-making abilities and alters their decision responses by interfering with and influencing their sensory inputs.

%\vspace{-1ex}
\subsection{Attention} 

James~\cite{James1890} described attention as the cognitive process by which the mind selectively concentrates on a singular element from a variety of possible stimuli or thoughts, emphasizing its essential function in creating our conscious perception. 
The seminal work of Posner \cite{posner1980orienting} 
%further delineated attention into more specific types, reflecting its multifaceted nature. Posner
introduced a framework for understanding the neural bases of attention and its various components and extending James's initial descriptions into a more nuanced understanding of the brain's attentional mechanisms. Building upon these early foundations, attention classification has expanded to include four types: %Selective, Divided, Sustained, and Executive.

% From these early foundations, the classification of attention has evolved to include several key types: Selective, Divided, Sustained, and Executive.

\begin{itemize}
    \itemsep0em
    \item \emph{Selective:} Focusing on relevant information while suppressing irrelevant information \cite{stevens2012role, murphy2016twenty}.
    
    \item \emph{Divided:} Capacity to allocate cognitive resources to multiple stimuli simultaneously, enabling individuals to engage in concurrent activities~\cite{spelke1976skills}. 
    %In other words, it refers to situations in which two or more channels are attended simultaneously. 
    Attended stimuli are from the same sensory modality or different ones~\cite{Herbranson2017}.

    \item \emph{Sustained:} Readiness to perceive and respond to stimuli over prolonged periods, often without conscious awareness of this vigilance~\cite{mackworth1948breakdown}.
    % Sustained attention, as outlined in the seminal work by Mackworth \cite{mackworth1948breakdown},refers to the psychological readiness to perceive and respond to stimuli over prolonged periods without necessarily being conscious of this vigilance.
    % Degradation attacks may limit sustained attention by making it more challenging for the user to maintain their focus over time, particularly when the quality of sensory inputs fluctuates or declines, resulting in increased cognitive load.
    \item \emph{Executive:} Regulates cognitive and emotional responses through management of other cognitive processes~\cite{posner1990attention}. Aids orchestration of thought and emotion in alignment with goals and the dynamic demands of the environment.
    % Subversion attacks could challenge executive attention by forcing users to constantly adapt to unexpected or counterintuitive system responses, requiring continuous updating of working memory.
\end{itemize}

Channel attacks primarily target Selective and Sustained attention. 
They manipulate the sensory channels through which users receive information, affecting their ability to focus on relevant stimuli or maintain attention over time.
Selective attention is exploited by degrading the sensory inputs, making it harder for users to distinguish between relevant and irrelevant stimuli. As mentioned in Section \ref{sec:decattacks}, this happens in overt and covert degradation attacks. These attacks may also limit sustained attention by making it more challenging for the user to maintain their focus over time, particularly when the quality of sensory inputs fluctuates or declines, resulting in increased cognitive load. 
Denial attacks block access to certain stimuli or information channels, disrupting selective attention. 
%This may potentially cause cognitive overload to select the relevant stimuli.
%A classic framework for understanding this phenomenon is provided by Broadbent's Filter Model of selective attention, which describes how individuals filter information based on physical characteristics before processing its meaning \cite{broadbent2013perception}.

Processing attacks primarily affect Divided and Executive attention by overloading the cognitive processing capabilities or by requiring constant adjustments to unexpected system behaviors.
Corruption attacks %involve the deliberate manipulation or alteration of sensory data to mislead users.
can directly impact the users' selective and executive attention by altering the information presented within \MR environment and also exploiting perceptual biases.
Subversion attacks could challenge executive attention by forcing users to constantly adapt to unexpected system responses, requiring continuous updating of working memory. They also can target divided attention by interrupting the flow of tasks or actions within an \MR environment, which compels users to divide their attention between correcting system errors and accomplishing their original goals.

\finding{Perception and attention are the primary targets for \MR deception attacks. Channel attacks target selection mechanisms by degrading or denying stimuli. Processing attacks target interpretation and execution by corrupting beliefs or subverting interpretations.}
\vspace{-1ex}

%Users' Attention can be manipulated or misconstrued by the employment of processing attacks; These attacks aim to effectively manipulate user perception, leading to false renditions and inferences. Incorrect or skewed perception then leads to misguided decision-making, which then impacts decision execution and the various types of responses.
%Corruption and Subversion attacks are specifically aimed at undermining the fundamental mechanisms that control attention within \MR systems.
%Corruption attacks have the potential to introduce anomalies or unwanted modifications in sensory data, drawing users' attention to these altered elements. This change in attention can potentially result in users failing to perceive or acknowledge other essential data, leading to a distorted understanding of their environment. Similarly, subversion attacks, by making the system behave in unintended ways, can create distractions or unexpected events that capture users' attention.
%The manipulation of attention through these attacks has particularly negative impacts on the \MR system. When attention is diverted or overwhelmed, the quality and accuracy of perception are reduced. Users might interpret the elements of the environment incorrectly. This compromised perception, affects decision-making, as decisions are based on this inaccurate understanding. As a result of this, decision execution might not be right or even dangerous, especially whenever immediate responses are needed.


\definecolor{lowcolor}{HTML}{bcabcd} % Light red
\definecolor{midcolor}{HTML}{8a6ca8} % Light yellow
\definecolor{highcolor}{HTML}{582D83}
\definecolor{none}{HTML}{ffffff}


% \begin{table}[htbp]
%     \centering
%     \begin{tabular}{|l|c|}
%         \hline
%         \rowcolor{gray!30}
%         \textbf{Category} & \textbf{Value} \\
%         \hline
%         Low & \cellcolor{green!30} Low \\
%         Medium & \cellcolor{yellow!30} Medium \\
%         High & \cellcolor{red!30} High \\
%         \hline
%     \end{tabular}
%     \caption{Visualization of Low, Medium, and High Values}
%     \label{tab:visualization}
% \end{table}


\newcommand{\Low}{%
\begin{tikzpicture}
\fill[fill=lowcolor] (0,0) rectangle (0.33,0.2);
\fill[fill=none] (0.33,0) rectangle (0.66,0.2);
\fill[fill=none] (0.66,0) rectangle (1,0.2);
\end{tikzpicture}%
}

\newcommand{\Medium}{%
\begin{tikzpicture}
\fill[fill=midcolor] (0,0) rectangle (0.33,0.2);
\fill[fill=midcolor] (0.33,0) rectangle (0.66,0.2);
\fill[fill=none] (0.66,0) rectangle (1,0.2);
\end{tikzpicture}%
}

\newcommand{\High}{%
\begin{tikzpicture}
\fill[fill=highcolor] (0,0) rectangle (1,0.2); % Use custom color
%\fill[fill=highcolor] (0.33,0) rectangle (0.66,0.2);
%\fill[fill=highcolor] (0.66,0) rectangle (1,0.2);
\end{tikzpicture}%
}

\newcommand{\LowMedium}{%
\begin{tikzpicture}
\fill[fill=lowcolor] (0,0) rectangle (0.33,0.2);
\fill[fill=midcolor] (0.33,0) rectangle (0.66,0.2);
\fill[fill=none] (0.66,0) rectangle (1,0.2);
\end{tikzpicture}%
}

\newcommand{\MediumHigh}{%
\begin{tikzpicture}
\fill[fill=none] (0,0) rectangle (0.33,0.2);
\fill[fill=midcolor] (0.33,0) rectangle (0.66,0.2);
\fill[fill=highcolor] (0.66,0) rectangle (1,0.2);
\end{tikzpicture}%
}

\newcommand{\LowHigh}{%
\begin{tikzpicture}
%\fill[left color=lowcolor, right color=highcolor] (0,0) rectangle (1,0.2);
\fill[fill=lowcolor] (0,0) rectangle (0.33,0.2);
\fill[fill=midcolor] (0.33,0) rectangle (0.66,0.2);
\fill[fill=highcolor] (0.66,0) rectangle (1,0.2);
\end{tikzpicture}%
}



\begin{table*}[ht!]
\caption{\MR attacks from our ontology are assessed according to the Information-Theoretic Model and Decision-Making Model.}
%mapped to the type of attack and the Information-Theoretic Model. Then, these attacks are assessed according to the degree to which they affect perception, attention, and memory. }





% Taking into account our \MR Deception Information-Theory and Decision-Making models, each attack is evaluated based on how it affects information-theoretic characteristics to alter communication channels and cognitive processes to determine the extent of its interference with various stages of human cognition.

\small
\centering
\renewcommand{\arraystretch}{1.15}
\setlength{\tabcolsep}{1pt}

\begin{tabular}{ccr|cccc|cccccccc|c|}
\cline{4-16}
  & &  & 
  \multicolumn{4}{c|}{Information-} & 
  \multicolumn{9}{c|}{} \\   % First row for spacing

  
  & &  & 
  \multicolumn{4}{c|}{\multirow{-1}{*}{Theoretic Model}} & 
  \multicolumn{9}{c|}{\multirow{-2}{*}{Decision-Making Model}} \\ \cline{4-16} % Second row, "Information Theoretic Model"

  
  & &  & 
  \multicolumn{3}{c|}{$C$} & & 
  \multicolumn{4}{c|}{Perception} & 
  \multicolumn{4}{c|}{Attention} & 
  \multirow{2}{*}{Mem.} \\ \cline{4-6} \cline{8-15} % Third row, details

  
  &
   &
  \multicolumn{1}{r|}{\multirow{-2}{*}{ \backslashbox{Attacks}{Models}}} &
  \multicolumn{1}{c|}{\hspace*{1mm}$W$\hspace*{1mm}} &
  \multicolumn{1}{c|}{\hspace*{1mm}$S$\hspace*{1mm}} &
  \multicolumn{1}{c|}{\hspace*{1mm}$N$\hspace*{1mm}} &
  \multirow{-2}{*}{$M$} &
  \multicolumn{1}{c|}{\hspace*{0.5mm}A/P\hspace*{0.5mm}} &
  \multicolumn{1}{c|}{$Sel$} &
  \multicolumn{1}{c|}{$Org$} &
  \multicolumn{1}{c|}{$Int$} &
  \multicolumn{1}{c|}{$Foc$} &
  \multicolumn{1}{c|}{$Div$} &
  \multicolumn{1}{c|}{$Sus$} &
  \multicolumn{1}{c|}{$Exe$} &
  \multicolumn{1}{c|}{} \\ \hlinewd{1.5pt}

  \multicolumn{1}{|c|}{} & \multicolumn{1}{c|}{}
   &
  Sensory Overload~ &
  \multicolumn{1}{c|}{} &
  \multicolumn{1}{c|}{} &
  \multicolumn{1}{c|}{\ding{51}} &
   &
  \multicolumn{1}{c|}{A} &
  \multicolumn{1}{c|}{\High} &
  \multicolumn{1}{c|}{\High} &
  \multicolumn{1}{c|}{\High} &
  \multicolumn{1}{c|}{\High} &
  \multicolumn{1}{c|}{\High} &
  \multicolumn{1}{c|}{\High} &
  \multicolumn{1}{c|}{\High} &
  \multicolumn{1}{c|}{W} \\ 
   \multicolumn{1}{|c|}{} & \multicolumn{1}{c|}{}
   &
  Momentary Misdirection~ & 
  \multicolumn{1}{c|}{} &
  \multicolumn{1}{c|}{} &
  \multicolumn{1}{c|}{\ding{51}} &
   &
  \multicolumn{1}{c|}{A} &
  \multicolumn{1}{c|}{\LowHigh} &
  \multicolumn{1}{c|}{\LowHigh} &
  \multicolumn{1}{c|}{\LowHigh} &
  \multicolumn{1}{c|}{\High} &
  \multicolumn{1}{c|}{\LowHigh} &
  \multicolumn{1}{c|}{\LowHigh} &
  \multicolumn{1}{c|}{\LowHigh} &
  \multicolumn{1}{c|}{W} \\ 
  \multicolumn{1}{|c|}{} &
  \multicolumn{1}{c|}{\multirow{-3}{*}{Overt} }&
  Signal Replacement~ &
  \multicolumn{1}{c|}{} &
  \multicolumn{1}{c|}{\ding{51}} &
  \multicolumn{1}{c|}{} &
   &
  \multicolumn{1}{c|}{A} &
  \multicolumn{1}{c|}{\High} &
  \multicolumn{1}{c|}{\High} &
  \multicolumn{1}{c|}{\High} &
  \multicolumn{1}{c|}{\High} &
  \multicolumn{1}{c|}{\High} &
  \multicolumn{1}{c|}{\High} &
  \multicolumn{1}{c|}{\High} &
  \multicolumn{1}{c|}{W} \\  
 \multicolumn{1}{|c|}{} & 
  \multicolumn{1}{c|}{\multirow{-3}{*}{Degradation}} & 
  Quality Erosion~ &
  \multicolumn{1}{c|}{\ding{51}} &
  \multicolumn{1}{c|}{} &
  \multicolumn{1}{c|}{\ding{51}} &
   &
  \multicolumn{1}{c|}{A} &
  \multicolumn{1}{c|}{\LowHigh} &
  \multicolumn{1}{c|}{\LowHigh} &
  \multicolumn{1}{c|}{\LowHigh} &
  \multicolumn{1}{c|}{\LowHigh} &
  \multicolumn{1}{c|}{\LowHigh} &
  \multicolumn{1}{c|}{\LowHigh} &
  \multicolumn{1}{c|}{\LowHigh} &
  \multicolumn{1}{c|}{W} \\ \cline{2-16} 
   \multicolumn{1}{|c|}{} & \multicolumn{1}{c|}{} &
  Subtle Boundary Manipulation~  &
  \multicolumn{1}{c|}{} &
  \multicolumn{1}{c|}{\ding{51}} &
  \multicolumn{1}{c|}{} &
   &
  \multicolumn{1}{c|}{P} &
  \multicolumn{1}{c|}{\Low} &
  \multicolumn{1}{c|}{\Low} &
  \multicolumn{1}{c|}{\LowHigh} &
  \multicolumn{1}{c|}{\Low} &
  \multicolumn{1}{c|}{\Low} &
  \multicolumn{1}{c|}{\Low} &
  \multicolumn{1}{c|}{\LowMedium} &
  \multicolumn{1}{c|}{W} \\ 
 \multicolumn{1}{|c|}{} &
  \multicolumn{1}{c|}{\multirow{-2}{*}{Covert}} &
  Subtle Dimension Manipulation~  &
  \multicolumn{1}{c|}{} &
  \multicolumn{1}{c|}{\ding{51}} &
  \multicolumn{1}{c|}{} &
   &
  \multicolumn{1}{c|}{P} &
  \multicolumn{1}{c|}{\Low} &
  \multicolumn{1}{c|}{\LowMedium} &
  \multicolumn{1}{c|}{\LowHigh} &
  \multicolumn{1}{c|}{\Low} &
  \multicolumn{1}{c|}{\Low} &
  \multicolumn{1}{c|}{\Low} &
  \multicolumn{1}{c|}{\LowMedium} &
  \multicolumn{1}{c|}{W} \\
 \multicolumn{1}{|c|}{} &
  \multicolumn{1}{c|}{\multirow{-2}{*}{Degradation}} &
  Physical Movement Manipulation~ &
  \multicolumn{1}{c|}{} &
  \multicolumn{1}{c|}{\ding{51}} &
  \multicolumn{1}{c|}{} &
   &
  \multicolumn{1}{c|}{P} &
  \multicolumn{1}{c|}{\Low} &
  \multicolumn{1}{c|}{\Low} &
  \multicolumn{1}{c|}{\LowHigh} &
  \multicolumn{1}{c|}{\Low} &
  \multicolumn{1}{c|}{\Low} &
  \multicolumn{1}{c|}{\Low} &
  \multicolumn{1}{c|}{\LowHigh} &
  \multicolumn{1}{c|}{W} \\ \cline{2-16} 
 \multicolumn{1}{|c|}{} & \multicolumn{1}{c|}{}
   &
  Shutdown~ &
  \multicolumn{1}{c|}{\ding{51}} &
  \multicolumn{1}{c|}{} &
  \multicolumn{1}{c|}{} &
   &
  \multicolumn{1}{c|}{A} &
  \multicolumn{1}{c|}{\High} &
  \multicolumn{1}{c|}{\High} &
  \multicolumn{1}{c|}{\High} &
  \multicolumn{1}{c|}{\High} &
  \multicolumn{1}{c|}{\High} &
  \multicolumn{1}{c|}{\High} &
  \multicolumn{1}{c|}{\High} &
  \multicolumn{1}{c|}{W} \\ 
  \multicolumn{1}{|c|}{} & \multicolumn{1}{c|}{}
   &
  Overlay~ &
  \multicolumn{1}{c|}{\ding{51}} &
  \multicolumn{1}{c|}{} &
  \multicolumn{1}{c|}{\ding{51}} &
   &
  \multicolumn{1}{c|}{A} &
  \multicolumn{1}{c|}{\LowHigh} &
  \multicolumn{1}{c|}{\LowHigh} &
  \multicolumn{1}{c|}{\LowHigh} &
  \multicolumn{1}{c|}{\LowHigh} &
  \multicolumn{1}{c|}{\LowHigh} &
  \multicolumn{1}{c|}{\LowHigh} &
  \multicolumn{1}{c|}{\LowHigh} &
  \multicolumn{1}{c|}{W} \\  
\multicolumn{1}{|c|}{\multirow{-10}{*}{\rotatebox[origin=c]{90}{Channel Attacks}}} &
  \multicolumn{1}{c|}{\multirow{-3}{*}{Denial}} & 
  Removal~ &
  \multicolumn{1}{c|}{\ding{51}} &
  \multicolumn{1}{c|}{\ding{51}} &
  \multicolumn{1}{c|}{} &
   &
  \multicolumn{1}{c|}{A} &
  \multicolumn{1}{c|}{\LowHigh} &
  \multicolumn{1}{c|}{\LowHigh} &
  \multicolumn{1}{c|}{\LowHigh} &
  \multicolumn{1}{c|}{\LowHigh} &
  \multicolumn{1}{c|}{\LowHigh} &
  \multicolumn{1}{c|}{\LowHigh} &
  \multicolumn{1}{c|}{\LowHigh} &
  \multicolumn{1}{c|}{W} \\ \cline{1-16}
 \multicolumn{1}{|c|}{} &
  \multicolumn{1}{c|}{} &
  Spoofing~ &
  \multicolumn{1}{c|}{} &
  \multicolumn{1}{c|}{} &
  \multicolumn{1}{c|}{} &
  \ding{51} &
  \multicolumn{1}{c|}{P} &
  \multicolumn{1}{c|}{\Low} &
  \multicolumn{1}{c|}{\Low} &
  \multicolumn{1}{c|}{\LowHigh} &
  \multicolumn{1}{c|}{\LowHigh} &
  \multicolumn{1}{c|}{\LowHigh} &
  \multicolumn{1}{c|}{\LowHigh} &
  \multicolumn{1}{c|}{\LowHigh} &
  \multicolumn{1}{c|}{W} \\ 
 \multicolumn{1}{|c|}{} &
  
  \multicolumn{1}{c|}{\multirow{-2}{*}{Corruption}} &
  False-Flag Operations~ &
  \multicolumn{1}{c|}{} &
  \multicolumn{1}{c|}{} &
  \multicolumn{1}{c|}{} &
  \ding{51} &
  \multicolumn{1}{c|}{P} &
\multicolumn{1}{c|}{\Low} &
  \multicolumn{1}{c|}{\Low} &
  \multicolumn{1}{c|}{\LowHigh} &
  \multicolumn{1}{c|}{\Low} &
  \multicolumn{1}{c|}{\Low} &
  \multicolumn{1}{c|}{\Low} &
  \multicolumn{1}{c|}{\Low} &
  \multicolumn{1}{c|}{W/L} \\ \cline{2-16} 
  \multicolumn{1}{|c|}{} &
  \multicolumn{1}{c|}{} &
  Bias Attacks~ &
  \multicolumn{1}{c|}{} &
  \multicolumn{1}{c|}{} &
  \multicolumn{1}{c|}{} &
   &
  \multicolumn{1}{c|}{P} &
  \multicolumn{1}{c|}{\Low} &
  \multicolumn{1}{c|}{\Low} &
  \multicolumn{1}{c|}{\LowHigh} &
  \multicolumn{1}{c|}{\Low} &
  \multicolumn{1}{c|}{\Low} &
  \multicolumn{1}{c|}{\Low} &
  \multicolumn{1}{c|}{\Low} &
  \multicolumn{1}{c|}{W/L} \\  
  \multicolumn{1}{|c|}{} &
  \multicolumn{1}{c|}{} &
  Lure~ &
  \multicolumn{1}{c|}{} &
  \multicolumn{1}{c|}{} &
  \multicolumn{1}{c|}{} &
   &
  \multicolumn{1}{c|}{P} &
  \multicolumn{1}{c|}{\LowHigh} &
  \multicolumn{1}{c|}{\LowHigh} &
  \multicolumn{1}{c|}{\LowHigh} &
  \multicolumn{1}{c|}{\LowHigh} &
  \multicolumn{1}{c|}{\LowHigh} &
  \multicolumn{1}{c|}{\LowHigh} &
  \multicolumn{1}{c|}{\LowHigh} &
  \multicolumn{1}{c|}{W/L} \\  
  \multicolumn{1}{|c|}{} &
  \multicolumn{1}{c|}{} &
  Disinformation~ &
  \multicolumn{1}{c|}{} &
  \multicolumn{1}{c|}{} &
  \multicolumn{1}{c|}{} &
   &
  \multicolumn{1}{c|}{P} &
  \multicolumn{1}{c|}{\Low} &
  \multicolumn{1}{c|}{\Low} &
  \multicolumn{1}{c|}{\LowHigh} &
  \multicolumn{1}{c|}{\Low} &
  \multicolumn{1}{c|}{\Low} &
  \multicolumn{1}{c|}{\Low} &
  \multicolumn{1}{c|}{\Low} &
  \multicolumn{1}{c|}{W/L} \\ 
  \multicolumn{1}{|c|}{} &
  \multicolumn{1}{c|}{} &
  Propaganda~ &
  \multicolumn{1}{c|}{} &
  \multicolumn{1}{c|}{} &
  \multicolumn{1}{c|}{} &
   &
  \multicolumn{1}{c|}{P} &
  \multicolumn{1}{c|}{\Low} &
  \multicolumn{1}{c|}{\Low} &
  \multicolumn{1}{c|}{\LowHigh} &
  \multicolumn{1}{c|}{\Low} &
  \multicolumn{1}{c|}{\Low} &
  \multicolumn{1}{c|}{\Low} &
  \multicolumn{1}{c|}{\Low} &
  \multicolumn{1}{c|}{W/L} \\ 
\multicolumn{1}{|c|}{\multirow{-7}{*}{\rotatebox[origin=c]{90}{Processing Attacks}}} &
  \multicolumn{1}{c|}{\multirow{-5}{*}{Subversion}} &
  Gaslighting~ &
  \multicolumn{1}{c|}{} &
  \multicolumn{1}{c|}{} &
  \multicolumn{1}{c|}{} &
   &
  \multicolumn{1}{c|}{P} &
  \multicolumn{1}{c|}{\Low} &
  \multicolumn{1}{c|}{\Low} &
  \multicolumn{1}{c|}{\LowHigh} &
  \multicolumn{1}{c|}{\Low} &
  \multicolumn{1}{c|}{\Low} &
  \multicolumn{1}{c|}{\Low} &
  \multicolumn{1}{c|}{\LowHigh} &
  \multicolumn{1}{c|}{W/L} \\ \hline
\end{tabular}
\\ \vspace{0.5em}
Low = 
\begin{tikzpicture}
\fill[fill=lowcolor] (0,0) rectangle (0.33,0.2); 
\end{tikzpicture}, 
Low-Medium = \begin{tikzpicture}
\fill[fill=lowcolor] (0,0) rectangle (0.33,0.2);
\fill[fill=midcolor] (0.33,0) rectangle (0.66,0.2);
\end{tikzpicture}, Low-High = \LowHigh, High = \High \protect\\
  \textbf{Information-Theoretic Model}: $C =$ Channel Capacity, $W =$ Bandwidth, $S =$ Signal, $N =$ Noise, $M =$ Mimicry \protect\\ \textbf{Perception}: $A/P =$ Active/Passive, $Sel =$ Selection, $Org =$ Organization, $Int =$ Interpretation \protect\\ \textbf{Attention}: $Foc =$ Selective, $Div =$ Divided, $Sus =$ Sustained, $Exe =$ Executive; \textbf{Memory}: $W =$ Working, $L =$ Long-Term  
\label{tab:framework}
\vspace{-1ex}
\end{table*}


\subsection{Memory}
Working memory and long-term memory are central components of our decision-making model. 
Baddeley~\cite{baddeley1974working,alan1992working,baddeley2007working} derived a multicomponent model of working memory consisting of the visuospatial sketchpad, phonological loop, central executive, and episodic buffer.
The visuospatial sketchpad stores visual and spatial information while the phonological loop stores auditory and verbal information.
The central executive directs attention towards \minoraddition{stored} information \minoraddition{in either one}.
% \minorremoval{stored in either the visuospatial sketchpad or the phonological loop }
The episodic buffer provides temporary storage of information needed by the central executive with connections to the other three components and long-term memory.
Long-term memory represents a permanent store that receives selected inputs from both the sensory register and working memory~\cite{ATKINSON196889}.


%Deception in \MR systems challenges users' perceptions of virtual and real-world content, 
%increasing cognitive load \cite{Dionisio2001Differentiation} and 
%affecting memory recall and decision-making \cite{Sporer2016Deception}.%, which is critical for recalling specific experiences. 
% Working memory and attention enable goal-driven processing by increasing access to relevant information \cite{awh2006interactions}. 
% Working memory influences the deployment of attention to new information and the manipulation and updating of existing representations.
% Attention affects the encoding and maintenance of information in working memory during multiple stages of processing, from early sensory to post-perceptual stages. 


\MR deception attacks affect memory and correspondingly attention.
Downing \cite{downing2000interactions} showed that the content of visuospatial sketchpad can guide selective attention toward matching visual stimuli. 
Through spoofing attacks, adversaries can produce deceptive stimuli that match expected stimuli, leveraging working memory to direct the user's selective attention.
Santangelo and Macaluso~\cite{santangelo2013contribution} identified the critical role of working memory in managing divided attention
%, underscoring the substantial cognitive demands placed on individuals 
when monitoring multiple objects simultaneously. 
Working memory load directly affects the efficiency of the central executive, with increased load impairing attention to multiple stimuli. 
Thus, sensory overload attacks can overwhelm working memory by visualizing too many objects for working memory to maintain.
Unsworth \& Robinson \cite{unsworth2020working} suggested 
%that individual differences in \WMC significantly impact sustained attention. They indicate 
that individuals with lower \WMC may struggle more with maintaining consistent attention, leading to performance degradation in tasks requiring prolonged focus. 
Therefore, the impact of \MR deception attacks that target \WMC, such as sensory overload, will vary from person to person.
%Engle \cite{engle2002working} suggested that \WMC is about controlling attention, rather than merely the quantity of information that can be stored. 
%This capacity enables individuals to maintain information in an easily accessible state while resisting interference, which is crucial for performing various cognitive tasks, including decision-making.

%Deception, characterized by the intentional manipulation of perception, has a profound impact on cognitive implications. 
%Based on a study by Sporer~\cite{Sporer2004The}, the challenge lies in differentiating between true and deceptive memory cues, which is intricately linked to reality monitoring. Trivers~\cite{Trivers2011Deceit} further elaborates on the intricate relationship between deceit and self-deception, emphasizing their cognitive cost and potential benefits in various interactions.

%Deception increases cognitive load \cite{Dionisio2001Differentiation}, affecting memory recall and decision-making \cite{Sporer2016Deception}. 
%Dionisio et al.~\cite{Dionisio2001Differentiation} found that generating deceptive responses requires greater cognitive processing, as indicated by increased pupillary responses. 
%Sporer~\cite{Sporer2016Deception} suggests the need for a broader understanding of cognitive processes in deception and a more comprehensive framework. 
%Cranford et al.'s~\cite{Cranford2021Towards} concept of cyber deception further exemplifies this need, especially in \MR environments where traditional physical cues are absent or altered.
% Deception activates specific brain regions, differentiating between true and false memories.
% Abe et al. \cite{abe2018neural} found that deception and false memory recruit different brain regions, such as the left temporoparietal cortex for true memory and the right anterior hippocampus for false memory. They identified the neural correlates of true memory, false memory, and deception and found that the left temporoparietal regions were more active for true memory, whereas the right anterior hippocampus was more active for false memory.
% Lee et al. \cite{Lee2009Are} demonstrate that intentionally faked responses and unintentional errors activate different areas of the brain, including the left ventrolateral prefrontal region (BA 47), the right posterior cingulate region (BA 23), and the left precuneus. 

% Cetnarski et al. \cite{Cetnarski2014Subliminal} show that subliminal stimuli in \MR can significantly influence decision-making, underscoring the need to understand these subtle interactions.





% \subsection{Sensemaking}

% Both Channel and Processing attacks impact sensemaking, which relies on processing information presented through \MR headsets.
% Channel attacks prevent the effective collecting and organizing of information in sensemaking tasks.
% Denial attacks deny access to certain information, preventing \MR users from adequately making sense of a situation due to lack of information.
% Overt Degradation attacks disrupt an \MR user's ability to engage in sensemaking by overloading senses with too much information to process when sensemaking.
% Covert Degradation attacks seek to mislead sensemaking by subtly reducing the quality of information hindering how \MR users make sense of a situation.
% For example, an attacker may lower the brightness of holographic callouts that label engine parts with abnormal readings causing a mechanic to miss information that may be valuable to making sense of the problem. 
% Processing attacks involve altering or fabricating the information presented through \MR and can profoundly affect the sensemaking process by presenting users with distorted or completely false information. 
% In Corruption attacks, the attacker might alter the data shown in an \MR environment, which can lead users to incorrect interpretations and misguided actions, since they base their decisions on falsified information.
% Subversion attacks are about covertly manipulating the system to misrepresent data, which might involve altering the way virtual objects behave or interact in the \MR environment, leading to confusion and misjudgment.


%Corruption attacks ...
%Subversion attacks ...
 

%\todo[inline]{explain how different types of attacks affect cognitive components}

%\todo[inline]{how is perception affected? discuss trust and risk perception factors}
\section{\MR Deception Analysis Framework (DAF)}
\label{sec:cog-framework}
The culminating, ensemble knowledge that connects our ontology, information-theoretic model, and decision-making model is the \MR \acf{DAF}---an assessment tool for identifying and discussing the multifaceted impact of \MR deception attacks on user cognition (\textbf{RQ4}).

% \subsection{Our Framework}

\DAF classifies attacks according to their operational mechanisms, which can be overt or covert, involving Degradation, Denial, Corruption, or Subversion, and the cognitive processes they aim to disrupt. 
We focus on identifying where attacks manipulate \MR communication channels by altering bandwidth ($W$), signal ($S$), noise ($N$), or by employing mimicry ($M$). Additionally, we explore the cognitive effects of each attack, examining the extent to which they can affect perception, attention, and memory.
%whether they affect active or passive perception ($A/P$)
For perception and attention, we further breakdown analysis into stages of perception---Selection ($Sel$), Organization ($Org$), and Interpretation ($Int$)---and types of attention---Selective ($Sel$), Divided ($Div$), Sustained ($Sus$), and Executive ($Exe$).

Table~\ref{tab:framework} presents our general analysis of the different categories of attacks identified in our ontology.
Overt Degradation and Denial attacks strongly affect both perception and attention.
Covert Degradation, Corruption, and Subversion attacks primarily target the Interpretation stage of perception.
These attacks typically require remaining hidden from the user.
Thus, any effects on attention or early stages of perception are likely too revealing.
%and the extent to which they impact various stages of perception, including selection ($Sel$), organization ($Org$), and interpretation ($Int$). In addition, we evaluate the extent to which attention including selective ($Sel$), divided ($Div$), sustained ($Sus$), and executive ($Exe$) is affected. Furthermore, we investigate memory to see if the attacks have any impact on long-term or working memory.

\finding{The interpretation stage of perception is a primary target of \MR deception attacks. Deceptions seek to cultivate false beliefs, formed initially by interpretations of perceived stimuli.}

% The purpose of mapping these attacks to our models is to help \MR system developers and security experts form systematic understanding of the potential vulnerabilities to users. 
% Table \ref{tab:framework} summarizes the detailed application of our framework to each identified deception attack within \MR environments and provides a comprehensive analysis of each attack, categorized according to our models of information-theoretic and decision-making.
% Here, we analyze the implications of these attacks in more detail, using the specified criteria from Table \ref{tab:framework} to evaluate their influence on the attentional and perceptual processes crucial for effective decision-making in \MR.

%In our extrapolated decision-making model depicted in Figure \ref{fig:process-model}, both perception and attention emerge as critical components, vulnerable to exploitation by attackers in a given \MR attack. Here, we explore the threats of the deceptive attacks on both the perceptual and attentional processes. The three major processes involved in one's perception that may be exploited during a deceptive \MR attack include:

For assessing the degree to which attacks affect stages of perception, we derived the following questions. Answers are either Low, Medium, High, or a combination of the three. \addition{De Meyer et al.~\cite{de2019delphi} state that a three-point scale provides a practical balance between simplicity and reliability. It minimizes measurement error and ensures clarity in response, which can be important for consensus building in Delphi procedures.} % Thus, it is well suited for assessing the stages of perception with sufficient precision in this context.}

\begin{itemize}
\itemsep0em
    \item \emph{Selection:} To what degree does the attack make it difficult to attend to or ignore task-related sensory stimuli from the physical or virtual environments during a decision-making task? 
    \item \emph{Organization:} To what degree does the attack make it difficult to group task-related sensory stimuli, such as by proximity or similarity, for a decision-making task?
    \item \emph{Interpretation:} To what degree does the attack make it difficult to accurately assign meaning to organized, task-related stimuli and correctly interpret patterns and relationships within virtual and physical environments when deriving understanding, making decisions, and taking action in a decision-making task?
\end{itemize}    

For assessing the degree to which attacks affect types of attention, we derived the following questions. Answers are either Low, Medium, High, or a combination of the three.
\begin{itemize}
\itemsep0em
    \item \emph{Selective:} To what degree does the attack make it difficult to focus attention on relevant physical and virtual objects for a decision-making task in MR?
    \item \emph{Divided:} To what degree does the attack make it difficult to switch between concurrent tasks rapidly while maintaining situational awareness in both the virtual and physical environments?
    \item \emph{Sustained:} To what degree does the attack make it difficult to continuously scan and interpret information presented in the mixed reality environment, making timely decisions and adjustments?
    \item \emph{Executive:} To what degree does the attack make it difficult to manage attentional resources effectively to interact with virtual elements while remaining aware of and responsive to the physical environment while performing a decision-making task?    
\end{itemize}

\DAF provides a systematic approach to evaluate threats posed by \minoraddition{\MR} deception attacks. We posit that such analysis is pivotal for developing more resilient \MR systems and training programs that can mitigate the impacts of deceptive threats. 
% \removal{\textbf{Appendix provides a scenario and instructions for how MR system designers and security researchers can use DAF in the DREAD and CVSS threat modeling frameworks to manage defensive efforts.}}

\finding{\DAF is a tool for defining experimental research on \MR deception attacks. We posit that it can be used to explore future attacks and may be extended for deception analysis beyond \MR research.}

\gap{We need empirical findings to validate and precisely model the impact of \MR deception attacks on cognitive processes and information channels.}

% This framework classifies attacks according to their operational mechanisms, which can be overt, covert, involving degradation, denial, corruption, or subversion, and the cognitive processes they aim to disrupt. The purpose of mapping these attacks to the aspects of human cognition they exploit is to offer a systematic understanding of the potential vulnerabilities that exist in MR environments. The methodology entails analyzing channel characteristics, such as bandwidth ($W$), signal ($S$), noise ($N$), and mimicry ($M$), in relation to MR deception. In addition, this study evaluates the impact of these attacks on the perception mechanisms, including active/passive ($A/P$) perception, selection ($Sel$), organization ($Org$), and interpretation ($Int$). It also examines the effects on attention and memory.


% \subsection{Case Study: Applying the \MR Cognitive Framework to a Combat Medic Scenario}

% \input{sections/scenario-table}
% %need for empirical findings to test theoretical models

% %our framework informs how we should design evaluation apparatusese.g., control signal to noise ratio in degradation attacks, eye tracking for attention measures

% %how can we measure the relationship between information capacity and deception?
% As an example for how our framework can be used to assess the cognitive impacts of particular deception attacks on a \MR system, we present the following scenario.
% A combat medic uses a \MR system designed to enhance their decision-making skills and medical response capabilities on the battlefield. The \MR system overlays vitals information for wounded soldiers that the medic is treating.
% %the real battlefield setting, providing an immersive experience that combines virtual patient data with the physical surroundings to simulate emergency medical scenarios.\\
% The enemy gains access to the \MR system, unknown to the medic. 
% They exploit system vulnerabilities and initiate deception attacks to disrupt the medics decision-making. 
% The attacker initiates a selective quality erosion attack by reducing the visual resolution and subtly reducing the clarity of vitals readings, making it difficult for medics to diagnose patients. % and affects their diagnoses and treatment decisions.
% The attacker then selectively removes heart rate information from the medic's view. %This disappearance of vital signs from the \MR overlay creates a critical gap in the information channel and again affects medics' decision-making processes. 
% Alternatively, the attacker could manipulate the heart rate plot to falsely indicate a soldier is going into cardiac arrest causing the medic to react with potentially harmful treatments.  
% %This manipulation is intended to mislead and undermine medics' confidence in their ability to interpret patient data and make them question their own decisions.

% We employ our framework to analyze the impacts of the three mentioned attacks in our scenario to determine how each attack affects the \MR system's communication channel and medic's cognition (Table~\ref{tab:scenario}).
% The initial attack, characterized as a quality erosion attack, notably constrains the bandwidth ($W$) within the \MR system, restricting the volume of information that can be effectively transmitted via the MR headset. Concurrently, the diminished visual resolution injects noise ($N$) into the system, disrupting the clarity of transmitted data.
% In terms of the decision-making model, this attack notably influences active perception, exerting a pronounced effect on the selection stage due to the heightened effort required to discern essential details from compromised visuals. The interpretation stage is similarly impacted at a high level, as the medics are tasked with making sense of ambiguous data. The organization stage experiences a medium impact; the reduced quality of visuals hampers systematically arranging visual information. This degradation challenges selective and executive attention, demanding increased focus to isolate pertinent information and necessitating decisions based on visuals of questionable reliability. Additionally, both divided and sustained attention are moderately affected, underscoring the increased cognitive load placed on medics as they interact with the altered \MR environment.

% The second attack, deliberate removal of critical data, in this case, heart rate information, from the \MR environment, narrows the bandwidth ($W$) and also reduces the signal ($S$) strength.
% This attack heavily affects active perception by entirely removing a vital piece of data from the medics' view and so impacting the selection stage at a high level as medics are forced to seek alternative sources. The interpretation stage also suffers a high impact due to the challenges of making informed decisions with incomplete data. Although, the organization stage sees a low to medium impact, since less data is available to organize.
% %The overall effect on perception is profound, as the structuring and subsequent interpretation of medical scenarios become significantly limited.
% The removal attack places a high demand on both selective and executive attention. The medic must use the remaining accessible data to infer missing information in order to make critical decisions. Divided attention is also affected highly because they have to compensate for missing data potentially affecting their ability to maintain awareness of other simultaneous patient needs. Sustained attention has a medium impact, challenging medics to remain engaged and vigilant despite the gaps in data continuity. %Sustained attention also sees a medium impact, challenging medics to remain engaged and vigilant despite the gaps in data continuity.

% The third attack involves manipulating heart rate data displayed in the \MR system to falsely show significant fluctuations in patient vitals. The distortion involves mimicry ($M$) by presenting altered data as authentic, subtly manipulating users. This attack also heavily affects the interpretation stage of perception, where medics face high difficulty in accurately assessing patient conditions due to deceptive alterations. It also has a high impact on all types of attention. By presenting altered vital signs as accurate, this attack demands intense scrutiny from medics, significantly challenging their selective, divided, sustained, and executive attention. Medics must verify the authenticity of the data, manage multiple sources of information, maintain focus over extended periods, and make critical decisions based on potentially misleading data.






\addition{
\section{Discussion}
\label{sec:discussion}
\DAF provides a systematic method to classify and analyze \MR deception attacks. %, addressing their effects on information channels and cognitive processes. 
While we focus on \MR headsets, \DAF is applicable to other forms of \MR and even other areas of human-computer interaction (HCI).
Kopp et al.'s information-theoretic framework ~\cite{kopp:2018} applied the Borden-Kopp model of deception to news media.
We have broadened its use to \MR deception attacks.
Future work should extend the scope to other areas of HCI that involve information processing and decision-making.
Our information-theoretic model and decision-making model are not tied to specific technologies or attacks, but rather provide generalizable models for studying the effects of deception in computing.
To enhance \DAF, future work should validate it empirically, expand its applicability to diverse contexts, incorporate individual cognitive factors, and refine models for processing attacks.

%By focusing on how \MR deception attacks impact information channels and decision-making processes, 
Researchers and practitioners can use \DAF to assess the security threat of \MR deception attacks.
For example, we can assign values of 1 to 3 for Low to High ratings, respectively.
Then, we can sum the values to identify which attacks pose the highest threat to perception and attention.
Further, \DAF can help develop deception detection and prevention approaches. % by using information theory to model \MR communication channels.
For example, we can compare differences between rendered frames to see how the signal is changing.
%For example, we can diff displayed frames with previous ones or an expected frame to identify changes in visual information presented to a \MR user.
%These diffs can reveal changes in channel capacity as information is either hidden or injected possibly along with noise.
High volatility in changes may indicate overt degradation attacks, particularly if we can identify noise based on differences between expected and actual frames.
More subtle changes that are spatial located in unexpected areas may indicate covert degradation attacks.
Using eye-tracking sensors on these headsets, we can derive models of attention that can help identify when different types of attention are being employed or disrupted.

% For example, we could use display-capture or eye-tracking data to detect a momentary misdirection attack.
% A misdirection attack 
%It offers foundational understanding for both technical and psychological dimensions of deception, with significant implications for future \MR research. 
%Our framework is adaptable for diverse \MR platforms and can guide empirical research into attack impacts and countermeasures. 

}

\addition{
%\subsection{Limitations}
%\label{sec:limitations}
\textbf{Limitations:} This SoK synthesizes existing knowledge towards developing a field of study around \MR deception. % by establishing a generalizable framework.
%It is essential to acknowledge the limitations of this research, which highlight potential areas for further exploration.
%One significant limitation is the lack of empirical validation. 
It is theoretical in nature and would benefit from further empirical validation.
%While it is rooted in empirical evidence from prior work, it lacks empirical validation.
%DAF’s models and  require real-world testing to confirm their accuracy and practical effectiveness. 
Controlled experiments involving \MR deception attacks are essential for refining the framework and assessing its relevance to diverse scenarios. 
Furthermore, \DAF does not fully account for cognitive diversity among users. 
Individual differences in cognitive capacity, attention, and susceptibility to deception are critical factors that could influence the effectiveness of both attacks and countermeasures. 
%Incorporating these factors would improve the framework’s precision and personalization.
% As \MR technology advances, the sophistication of attacks will also increase, highlighting the importance of further study in this area.
}
\section{Conclusion}
This paper presented \toolkit, a do-it-yourself toolkit that empowers novice roboticists with basic electronics and programming skills to rapidly prototype interactions for functional lo-fi exoskeletons targeted at the arms. 
\toolkit~features modular hardware components that allow to easily reconfigure its active degrees of freedom, adjust component's dimensions to accommodate various body sizes, and safety mechanisms. We conceptually identified relevant high-level augmentation strategies and provide them as functional abstractions that simplify the programming of interactive behaviors. These functions are readily accessible and customizable through a command-line interface, GUI, Processing library, and Arduino firmware. 
Through application cases and two usage studies, we demonstrated \toolkit's potential to ease the development of human-exoskeleton interactions and support creative exploration and rapid iteration in early-stage interaction design. We hope that this work will inspire HCI researchers to explore the emerging field of human-exoskeleton interaction and unlock its potential for innovative applications.
 
\begin{acks}
    We thank all participants of our usage studies and express our particular gratitude to Ata Otaran for his feedback. We also thank the reviewers for their valuable comments.
\end{acks}

\section*{Acknowledgments}
\label{sec:ack}

This work was supported by the \DARPA, grant number HR00112320030.
We appreciate the shepherd and anonymous reviewers for their insightful and valuable feedback.
\newpage

\section* {\centering  Data Availability}

The implementation of \system is open-sourced. It is 
currently available at \cite{TRAPSAnonOpenSource}, with detailed 
instructions on how to reproduce \system and its evaluation. 
If/when this paper is accepted, we will make it publicly 
available.\\ \\

\nocite{*}
{\footnotesize \bibliographystyle{plain}
\bibliography{main}}

%\clearpage
%\appendix
%\subsection{Lloyd-Max Algorithm}
\label{subsec:Lloyd-Max}
For a given quantization bitwidth $B$ and an operand $\bm{X}$, the Lloyd-Max algorithm finds $2^B$ quantization levels $\{\hat{x}_i\}_{i=1}^{2^B}$ such that quantizing $\bm{X}$ by rounding each scalar in $\bm{X}$ to the nearest quantization level minimizes the quantization MSE. 

The algorithm starts with an initial guess of quantization levels and then iteratively computes quantization thresholds $\{\tau_i\}_{i=1}^{2^B-1}$ and updates quantization levels $\{\hat{x}_i\}_{i=1}^{2^B}$. Specifically, at iteration $n$, thresholds are set to the midpoints of the previous iteration's levels:
\begin{align*}
    \tau_i^{(n)}=\frac{\hat{x}_i^{(n-1)}+\hat{x}_{i+1}^{(n-1)}}2 \text{ for } i=1\ldots 2^B-1
\end{align*}
Subsequently, the quantization levels are re-computed as conditional means of the data regions defined by the new thresholds:
\begin{align*}
    \hat{x}_i^{(n)}=\mathbb{E}\left[ \bm{X} \big| \bm{X}\in [\tau_{i-1}^{(n)},\tau_i^{(n)}] \right] \text{ for } i=1\ldots 2^B
\end{align*}
where to satisfy boundary conditions we have $\tau_0=-\infty$ and $\tau_{2^B}=\infty$. The algorithm iterates the above steps until convergence.

Figure \ref{fig:lm_quant} compares the quantization levels of a $7$-bit floating point (E3M3) quantizer (left) to a $7$-bit Lloyd-Max quantizer (right) when quantizing a layer of weights from the GPT3-126M model at a per-tensor granularity. As shown, the Lloyd-Max quantizer achieves substantially lower quantization MSE. Further, Table \ref{tab:FP7_vs_LM7} shows the superior perplexity achieved by Lloyd-Max quantizers for bitwidths of $7$, $6$ and $5$. The difference between the quantizers is clear at 5 bits, where per-tensor FP quantization incurs a drastic and unacceptable increase in perplexity, while Lloyd-Max quantization incurs a much smaller increase. Nevertheless, we note that even the optimal Lloyd-Max quantizer incurs a notable ($\sim 1.5$) increase in perplexity due to the coarse granularity of quantization. 

\begin{figure}[h]
  \centering
  \includegraphics[width=0.7\linewidth]{sections/figures/LM7_FP7.pdf}
  \caption{\small Quantization levels and the corresponding quantization MSE of Floating Point (left) vs Lloyd-Max (right) Quantizers for a layer of weights in the GPT3-126M model.}
  \label{fig:lm_quant}
\end{figure}

\begin{table}[h]\scriptsize
\begin{center}
\caption{\label{tab:FP7_vs_LM7} \small Comparing perplexity (lower is better) achieved by floating point quantizers and Lloyd-Max quantizers on a GPT3-126M model for the Wikitext-103 dataset.}
\begin{tabular}{c|cc|c}
\hline
 \multirow{2}{*}{\textbf{Bitwidth}} & \multicolumn{2}{|c|}{\textbf{Floating-Point Quantizer}} & \textbf{Lloyd-Max Quantizer} \\
 & Best Format & Wikitext-103 Perplexity & Wikitext-103 Perplexity \\
\hline
7 & E3M3 & 18.32 & 18.27 \\
6 & E3M2 & 19.07 & 18.51 \\
5 & E4M0 & 43.89 & 19.71 \\
\hline
\end{tabular}
\end{center}
\end{table}

\subsection{Proof of Local Optimality of LO-BCQ}
\label{subsec:lobcq_opt_proof}
For a given block $\bm{b}_j$, the quantization MSE during LO-BCQ can be empirically evaluated as $\frac{1}{L_b}\lVert \bm{b}_j- \bm{\hat{b}}_j\rVert^2_2$ where $\bm{\hat{b}}_j$ is computed from equation (\ref{eq:clustered_quantization_definition}) as $C_{f(\bm{b}_j)}(\bm{b}_j)$. Further, for a given block cluster $\mathcal{B}_i$, we compute the quantization MSE as $\frac{1}{|\mathcal{B}_{i}|}\sum_{\bm{b} \in \mathcal{B}_{i}} \frac{1}{L_b}\lVert \bm{b}- C_i^{(n)}(\bm{b})\rVert^2_2$. Therefore, at the end of iteration $n$, we evaluate the overall quantization MSE $J^{(n)}$ for a given operand $\bm{X}$ composed of $N_c$ block clusters as:
\begin{align*}
    \label{eq:mse_iter_n}
    J^{(n)} = \frac{1}{N_c} \sum_{i=1}^{N_c} \frac{1}{|\mathcal{B}_{i}^{(n)}|}\sum_{\bm{v} \in \mathcal{B}_{i}^{(n)}} \frac{1}{L_b}\lVert \bm{b}- B_i^{(n)}(\bm{b})\rVert^2_2
\end{align*}

At the end of iteration $n$, the codebooks are updated from $\mathcal{C}^{(n-1)}$ to $\mathcal{C}^{(n)}$. However, the mapping of a given vector $\bm{b}_j$ to quantizers $\mathcal{C}^{(n)}$ remains as  $f^{(n)}(\bm{b}_j)$. At the next iteration, during the vector clustering step, $f^{(n+1)}(\bm{b}_j)$ finds new mapping of $\bm{b}_j$ to updated codebooks $\mathcal{C}^{(n)}$ such that the quantization MSE over the candidate codebooks is minimized. Therefore, we obtain the following result for $\bm{b}_j$:
\begin{align*}
\frac{1}{L_b}\lVert \bm{b}_j - C_{f^{(n+1)}(\bm{b}_j)}^{(n)}(\bm{b}_j)\rVert^2_2 \le \frac{1}{L_b}\lVert \bm{b}_j - C_{f^{(n)}(\bm{b}_j)}^{(n)}(\bm{b}_j)\rVert^2_2
\end{align*}

That is, quantizing $\bm{b}_j$ at the end of the block clustering step of iteration $n+1$ results in lower quantization MSE compared to quantizing at the end of iteration $n$. Since this is true for all $\bm{b} \in \bm{X}$, we assert the following:
\begin{equation}
\begin{split}
\label{eq:mse_ineq_1}
    \tilde{J}^{(n+1)} &= \frac{1}{N_c} \sum_{i=1}^{N_c} \frac{1}{|\mathcal{B}_{i}^{(n+1)}|}\sum_{\bm{b} \in \mathcal{B}_{i}^{(n+1)}} \frac{1}{L_b}\lVert \bm{b} - C_i^{(n)}(b)\rVert^2_2 \le J^{(n)}
\end{split}
\end{equation}
where $\tilde{J}^{(n+1)}$ is the the quantization MSE after the vector clustering step at iteration $n+1$.

Next, during the codebook update step (\ref{eq:quantizers_update}) at iteration $n+1$, the per-cluster codebooks $\mathcal{C}^{(n)}$ are updated to $\mathcal{C}^{(n+1)}$ by invoking the Lloyd-Max algorithm \citep{Lloyd}. We know that for any given value distribution, the Lloyd-Max algorithm minimizes the quantization MSE. Therefore, for a given vector cluster $\mathcal{B}_i$ we obtain the following result:

\begin{equation}
    \frac{1}{|\mathcal{B}_{i}^{(n+1)}|}\sum_{\bm{b} \in \mathcal{B}_{i}^{(n+1)}} \frac{1}{L_b}\lVert \bm{b}- C_i^{(n+1)}(\bm{b})\rVert^2_2 \le \frac{1}{|\mathcal{B}_{i}^{(n+1)}|}\sum_{\bm{b} \in \mathcal{B}_{i}^{(n+1)}} \frac{1}{L_b}\lVert \bm{b}- C_i^{(n)}(\bm{b})\rVert^2_2
\end{equation}

The above equation states that quantizing the given block cluster $\mathcal{B}_i$ after updating the associated codebook from $C_i^{(n)}$ to $C_i^{(n+1)}$ results in lower quantization MSE. Since this is true for all the block clusters, we derive the following result: 
\begin{equation}
\begin{split}
\label{eq:mse_ineq_2}
     J^{(n+1)} &= \frac{1}{N_c} \sum_{i=1}^{N_c} \frac{1}{|\mathcal{B}_{i}^{(n+1)}|}\sum_{\bm{b} \in \mathcal{B}_{i}^{(n+1)}} \frac{1}{L_b}\lVert \bm{b}- C_i^{(n+1)}(\bm{b})\rVert^2_2  \le \tilde{J}^{(n+1)}   
\end{split}
\end{equation}

Following (\ref{eq:mse_ineq_1}) and (\ref{eq:mse_ineq_2}), we find that the quantization MSE is non-increasing for each iteration, that is, $J^{(1)} \ge J^{(2)} \ge J^{(3)} \ge \ldots \ge J^{(M)}$ where $M$ is the maximum number of iterations. 
%Therefore, we can say that if the algorithm converges, then it must be that it has converged to a local minimum. 
\hfill $\blacksquare$


\begin{figure}
    \begin{center}
    \includegraphics[width=0.5\textwidth]{sections//figures/mse_vs_iter.pdf}
    \end{center}
    \caption{\small NMSE vs iterations during LO-BCQ compared to other block quantization proposals}
    \label{fig:nmse_vs_iter}
\end{figure}

Figure \ref{fig:nmse_vs_iter} shows the empirical convergence of LO-BCQ across several block lengths and number of codebooks. Also, the MSE achieved by LO-BCQ is compared to baselines such as MXFP and VSQ. As shown, LO-BCQ converges to a lower MSE than the baselines. Further, we achieve better convergence for larger number of codebooks ($N_c$) and for a smaller block length ($L_b$), both of which increase the bitwidth of BCQ (see Eq \ref{eq:bitwidth_bcq}).


\subsection{Additional Accuracy Results}
%Table \ref{tab:lobcq_config} lists the various LOBCQ configurations and their corresponding bitwidths.
\begin{table}
\setlength{\tabcolsep}{4.75pt}
\begin{center}
\caption{\label{tab:lobcq_config} Various LO-BCQ configurations and their bitwidths.}
\begin{tabular}{|c||c|c|c|c||c|c||c|} 
\hline
 & \multicolumn{4}{|c||}{$L_b=8$} & \multicolumn{2}{|c||}{$L_b=4$} & $L_b=2$ \\
 \hline
 \backslashbox{$L_A$\kern-1em}{\kern-1em$N_c$} & 2 & 4 & 8 & 16 & 2 & 4 & 2 \\
 \hline
 64 & 4.25 & 4.375 & 4.5 & 4.625 & 4.375 & 4.625 & 4.625\\
 \hline
 32 & 4.375 & 4.5 & 4.625& 4.75 & 4.5 & 4.75 & 4.75 \\
 \hline
 16 & 4.625 & 4.75& 4.875 & 5 & 4.75 & 5 & 5 \\
 \hline
\end{tabular}
\end{center}
\end{table}

%\subsection{Perplexity achieved by various LO-BCQ configurations on Wikitext-103 dataset}

\begin{table} \centering
\begin{tabular}{|c||c|c|c|c||c|c||c|} 
\hline
 $L_b \rightarrow$& \multicolumn{4}{c||}{8} & \multicolumn{2}{c||}{4} & 2\\
 \hline
 \backslashbox{$L_A$\kern-1em}{\kern-1em$N_c$} & 2 & 4 & 8 & 16 & 2 & 4 & 2  \\
 %$N_c \rightarrow$ & 2 & 4 & 8 & 16 & 2 & 4 & 2 \\
 \hline
 \hline
 \multicolumn{8}{c}{GPT3-1.3B (FP32 PPL = 9.98)} \\ 
 \hline
 \hline
 64 & 10.40 & 10.23 & 10.17 & 10.15 &  10.28 & 10.18 & 10.19 \\
 \hline
 32 & 10.25 & 10.20 & 10.15 & 10.12 &  10.23 & 10.17 & 10.17 \\
 \hline
 16 & 10.22 & 10.16 & 10.10 & 10.09 &  10.21 & 10.14 & 10.16 \\
 \hline
  \hline
 \multicolumn{8}{c}{GPT3-8B (FP32 PPL = 7.38)} \\ 
 \hline
 \hline
 64 & 7.61 & 7.52 & 7.48 &  7.47 &  7.55 &  7.49 & 7.50 \\
 \hline
 32 & 7.52 & 7.50 & 7.46 &  7.45 &  7.52 &  7.48 & 7.48  \\
 \hline
 16 & 7.51 & 7.48 & 7.44 &  7.44 &  7.51 &  7.49 & 7.47  \\
 \hline
\end{tabular}
\caption{\label{tab:ppl_gpt3_abalation} Wikitext-103 perplexity across GPT3-1.3B and 8B models.}
\end{table}

\begin{table} \centering
\begin{tabular}{|c||c|c|c|c||} 
\hline
 $L_b \rightarrow$& \multicolumn{4}{c||}{8}\\
 \hline
 \backslashbox{$L_A$\kern-1em}{\kern-1em$N_c$} & 2 & 4 & 8 & 16 \\
 %$N_c \rightarrow$ & 2 & 4 & 8 & 16 & 2 & 4 & 2 \\
 \hline
 \hline
 \multicolumn{5}{|c|}{Llama2-7B (FP32 PPL = 5.06)} \\ 
 \hline
 \hline
 64 & 5.31 & 5.26 & 5.19 & 5.18  \\
 \hline
 32 & 5.23 & 5.25 & 5.18 & 5.15  \\
 \hline
 16 & 5.23 & 5.19 & 5.16 & 5.14  \\
 \hline
 \multicolumn{5}{|c|}{Nemotron4-15B (FP32 PPL = 5.87)} \\ 
 \hline
 \hline
 64  & 6.3 & 6.20 & 6.13 & 6.08  \\
 \hline
 32  & 6.24 & 6.12 & 6.07 & 6.03  \\
 \hline
 16  & 6.12 & 6.14 & 6.04 & 6.02  \\
 \hline
 \multicolumn{5}{|c|}{Nemotron4-340B (FP32 PPL = 3.48)} \\ 
 \hline
 \hline
 64 & 3.67 & 3.62 & 3.60 & 3.59 \\
 \hline
 32 & 3.63 & 3.61 & 3.59 & 3.56 \\
 \hline
 16 & 3.61 & 3.58 & 3.57 & 3.55 \\
 \hline
\end{tabular}
\caption{\label{tab:ppl_llama7B_nemo15B} Wikitext-103 perplexity compared to FP32 baseline in Llama2-7B and Nemotron4-15B, 340B models}
\end{table}

%\subsection{Perplexity achieved by various LO-BCQ configurations on MMLU dataset}


\begin{table} \centering
\begin{tabular}{|c||c|c|c|c||c|c|c|c|} 
\hline
 $L_b \rightarrow$& \multicolumn{4}{c||}{8} & \multicolumn{4}{c||}{8}\\
 \hline
 \backslashbox{$L_A$\kern-1em}{\kern-1em$N_c$} & 2 & 4 & 8 & 16 & 2 & 4 & 8 & 16  \\
 %$N_c \rightarrow$ & 2 & 4 & 8 & 16 & 2 & 4 & 2 \\
 \hline
 \hline
 \multicolumn{5}{|c|}{Llama2-7B (FP32 Accuracy = 45.8\%)} & \multicolumn{4}{|c|}{Llama2-70B (FP32 Accuracy = 69.12\%)} \\ 
 \hline
 \hline
 64 & 43.9 & 43.4 & 43.9 & 44.9 & 68.07 & 68.27 & 68.17 & 68.75 \\
 \hline
 32 & 44.5 & 43.8 & 44.9 & 44.5 & 68.37 & 68.51 & 68.35 & 68.27  \\
 \hline
 16 & 43.9 & 42.7 & 44.9 & 45 & 68.12 & 68.77 & 68.31 & 68.59  \\
 \hline
 \hline
 \multicolumn{5}{|c|}{GPT3-22B (FP32 Accuracy = 38.75\%)} & \multicolumn{4}{|c|}{Nemotron4-15B (FP32 Accuracy = 64.3\%)} \\ 
 \hline
 \hline
 64 & 36.71 & 38.85 & 38.13 & 38.92 & 63.17 & 62.36 & 63.72 & 64.09 \\
 \hline
 32 & 37.95 & 38.69 & 39.45 & 38.34 & 64.05 & 62.30 & 63.8 & 64.33  \\
 \hline
 16 & 38.88 & 38.80 & 38.31 & 38.92 & 63.22 & 63.51 & 63.93 & 64.43  \\
 \hline
\end{tabular}
\caption{\label{tab:mmlu_abalation} Accuracy on MMLU dataset across GPT3-22B, Llama2-7B, 70B and Nemotron4-15B models.}
\end{table}


%\subsection{Perplexity achieved by various LO-BCQ configurations on LM evaluation harness}

\begin{table} \centering
\begin{tabular}{|c||c|c|c|c||c|c|c|c|} 
\hline
 $L_b \rightarrow$& \multicolumn{4}{c||}{8} & \multicolumn{4}{c||}{8}\\
 \hline
 \backslashbox{$L_A$\kern-1em}{\kern-1em$N_c$} & 2 & 4 & 8 & 16 & 2 & 4 & 8 & 16  \\
 %$N_c \rightarrow$ & 2 & 4 & 8 & 16 & 2 & 4 & 2 \\
 \hline
 \hline
 \multicolumn{5}{|c|}{Race (FP32 Accuracy = 37.51\%)} & \multicolumn{4}{|c|}{Boolq (FP32 Accuracy = 64.62\%)} \\ 
 \hline
 \hline
 64 & 36.94 & 37.13 & 36.27 & 37.13 & 63.73 & 62.26 & 63.49 & 63.36 \\
 \hline
 32 & 37.03 & 36.36 & 36.08 & 37.03 & 62.54 & 63.51 & 63.49 & 63.55  \\
 \hline
 16 & 37.03 & 37.03 & 36.46 & 37.03 & 61.1 & 63.79 & 63.58 & 63.33  \\
 \hline
 \hline
 \multicolumn{5}{|c|}{Winogrande (FP32 Accuracy = 58.01\%)} & \multicolumn{4}{|c|}{Piqa (FP32 Accuracy = 74.21\%)} \\ 
 \hline
 \hline
 64 & 58.17 & 57.22 & 57.85 & 58.33 & 73.01 & 73.07 & 73.07 & 72.80 \\
 \hline
 32 & 59.12 & 58.09 & 57.85 & 58.41 & 73.01 & 73.94 & 72.74 & 73.18  \\
 \hline
 16 & 57.93 & 58.88 & 57.93 & 58.56 & 73.94 & 72.80 & 73.01 & 73.94  \\
 \hline
\end{tabular}
\caption{\label{tab:mmlu_abalation} Accuracy on LM evaluation harness tasks on GPT3-1.3B model.}
\end{table}

\begin{table} \centering
\begin{tabular}{|c||c|c|c|c||c|c|c|c|} 
\hline
 $L_b \rightarrow$& \multicolumn{4}{c||}{8} & \multicolumn{4}{c||}{8}\\
 \hline
 \backslashbox{$L_A$\kern-1em}{\kern-1em$N_c$} & 2 & 4 & 8 & 16 & 2 & 4 & 8 & 16  \\
 %$N_c \rightarrow$ & 2 & 4 & 8 & 16 & 2 & 4 & 2 \\
 \hline
 \hline
 \multicolumn{5}{|c|}{Race (FP32 Accuracy = 41.34\%)} & \multicolumn{4}{|c|}{Boolq (FP32 Accuracy = 68.32\%)} \\ 
 \hline
 \hline
 64 & 40.48 & 40.10 & 39.43 & 39.90 & 69.20 & 68.41 & 69.45 & 68.56 \\
 \hline
 32 & 39.52 & 39.52 & 40.77 & 39.62 & 68.32 & 67.43 & 68.17 & 69.30  \\
 \hline
 16 & 39.81 & 39.71 & 39.90 & 40.38 & 68.10 & 66.33 & 69.51 & 69.42  \\
 \hline
 \hline
 \multicolumn{5}{|c|}{Winogrande (FP32 Accuracy = 67.88\%)} & \multicolumn{4}{|c|}{Piqa (FP32 Accuracy = 78.78\%)} \\ 
 \hline
 \hline
 64 & 66.85 & 66.61 & 67.72 & 67.88 & 77.31 & 77.42 & 77.75 & 77.64 \\
 \hline
 32 & 67.25 & 67.72 & 67.72 & 67.00 & 77.31 & 77.04 & 77.80 & 77.37  \\
 \hline
 16 & 68.11 & 68.90 & 67.88 & 67.48 & 77.37 & 78.13 & 78.13 & 77.69  \\
 \hline
\end{tabular}
\caption{\label{tab:mmlu_abalation} Accuracy on LM evaluation harness tasks on GPT3-8B model.}
\end{table}

\begin{table} \centering
\begin{tabular}{|c||c|c|c|c||c|c|c|c|} 
\hline
 $L_b \rightarrow$& \multicolumn{4}{c||}{8} & \multicolumn{4}{c||}{8}\\
 \hline
 \backslashbox{$L_A$\kern-1em}{\kern-1em$N_c$} & 2 & 4 & 8 & 16 & 2 & 4 & 8 & 16  \\
 %$N_c \rightarrow$ & 2 & 4 & 8 & 16 & 2 & 4 & 2 \\
 \hline
 \hline
 \multicolumn{5}{|c|}{Race (FP32 Accuracy = 40.67\%)} & \multicolumn{4}{|c|}{Boolq (FP32 Accuracy = 76.54\%)} \\ 
 \hline
 \hline
 64 & 40.48 & 40.10 & 39.43 & 39.90 & 75.41 & 75.11 & 77.09 & 75.66 \\
 \hline
 32 & 39.52 & 39.52 & 40.77 & 39.62 & 76.02 & 76.02 & 75.96 & 75.35  \\
 \hline
 16 & 39.81 & 39.71 & 39.90 & 40.38 & 75.05 & 73.82 & 75.72 & 76.09  \\
 \hline
 \hline
 \multicolumn{5}{|c|}{Winogrande (FP32 Accuracy = 70.64\%)} & \multicolumn{4}{|c|}{Piqa (FP32 Accuracy = 79.16\%)} \\ 
 \hline
 \hline
 64 & 69.14 & 70.17 & 70.17 & 70.56 & 78.24 & 79.00 & 78.62 & 78.73 \\
 \hline
 32 & 70.96 & 69.69 & 71.27 & 69.30 & 78.56 & 79.49 & 79.16 & 78.89  \\
 \hline
 16 & 71.03 & 69.53 & 69.69 & 70.40 & 78.13 & 79.16 & 79.00 & 79.00  \\
 \hline
\end{tabular}
\caption{\label{tab:mmlu_abalation} Accuracy on LM evaluation harness tasks on GPT3-22B model.}
\end{table}

\begin{table} \centering
\begin{tabular}{|c||c|c|c|c||c|c|c|c|} 
\hline
 $L_b \rightarrow$& \multicolumn{4}{c||}{8} & \multicolumn{4}{c||}{8}\\
 \hline
 \backslashbox{$L_A$\kern-1em}{\kern-1em$N_c$} & 2 & 4 & 8 & 16 & 2 & 4 & 8 & 16  \\
 %$N_c \rightarrow$ & 2 & 4 & 8 & 16 & 2 & 4 & 2 \\
 \hline
 \hline
 \multicolumn{5}{|c|}{Race (FP32 Accuracy = 44.4\%)} & \multicolumn{4}{|c|}{Boolq (FP32 Accuracy = 79.29\%)} \\ 
 \hline
 \hline
 64 & 42.49 & 42.51 & 42.58 & 43.45 & 77.58 & 77.37 & 77.43 & 78.1 \\
 \hline
 32 & 43.35 & 42.49 & 43.64 & 43.73 & 77.86 & 75.32 & 77.28 & 77.86  \\
 \hline
 16 & 44.21 & 44.21 & 43.64 & 42.97 & 78.65 & 77 & 76.94 & 77.98  \\
 \hline
 \hline
 \multicolumn{5}{|c|}{Winogrande (FP32 Accuracy = 69.38\%)} & \multicolumn{4}{|c|}{Piqa (FP32 Accuracy = 78.07\%)} \\ 
 \hline
 \hline
 64 & 68.9 & 68.43 & 69.77 & 68.19 & 77.09 & 76.82 & 77.09 & 77.86 \\
 \hline
 32 & 69.38 & 68.51 & 68.82 & 68.90 & 78.07 & 76.71 & 78.07 & 77.86  \\
 \hline
 16 & 69.53 & 67.09 & 69.38 & 68.90 & 77.37 & 77.8 & 77.91 & 77.69  \\
 \hline
\end{tabular}
\caption{\label{tab:mmlu_abalation} Accuracy on LM evaluation harness tasks on Llama2-7B model.}
\end{table}

\begin{table} \centering
\begin{tabular}{|c||c|c|c|c||c|c|c|c|} 
\hline
 $L_b \rightarrow$& \multicolumn{4}{c||}{8} & \multicolumn{4}{c||}{8}\\
 \hline
 \backslashbox{$L_A$\kern-1em}{\kern-1em$N_c$} & 2 & 4 & 8 & 16 & 2 & 4 & 8 & 16  \\
 %$N_c \rightarrow$ & 2 & 4 & 8 & 16 & 2 & 4 & 2 \\
 \hline
 \hline
 \multicolumn{5}{|c|}{Race (FP32 Accuracy = 48.8\%)} & \multicolumn{4}{|c|}{Boolq (FP32 Accuracy = 85.23\%)} \\ 
 \hline
 \hline
 64 & 49.00 & 49.00 & 49.28 & 48.71 & 82.82 & 84.28 & 84.03 & 84.25 \\
 \hline
 32 & 49.57 & 48.52 & 48.33 & 49.28 & 83.85 & 84.46 & 84.31 & 84.93  \\
 \hline
 16 & 49.85 & 49.09 & 49.28 & 48.99 & 85.11 & 84.46 & 84.61 & 83.94  \\
 \hline
 \hline
 \multicolumn{5}{|c|}{Winogrande (FP32 Accuracy = 79.95\%)} & \multicolumn{4}{|c|}{Piqa (FP32 Accuracy = 81.56\%)} \\ 
 \hline
 \hline
 64 & 78.77 & 78.45 & 78.37 & 79.16 & 81.45 & 80.69 & 81.45 & 81.5 \\
 \hline
 32 & 78.45 & 79.01 & 78.69 & 80.66 & 81.56 & 80.58 & 81.18 & 81.34  \\
 \hline
 16 & 79.95 & 79.56 & 79.79 & 79.72 & 81.28 & 81.66 & 81.28 & 80.96  \\
 \hline
\end{tabular}
\caption{\label{tab:mmlu_abalation} Accuracy on LM evaluation harness tasks on Llama2-70B model.}
\end{table}

%\section{MSE Studies}
%\textcolor{red}{TODO}


\subsection{Number Formats and Quantization Method}
\label{subsec:numFormats_quantMethod}
\subsubsection{Integer Format}
An $n$-bit signed integer (INT) is typically represented with a 2s-complement format \citep{yao2022zeroquant,xiao2023smoothquant,dai2021vsq}, where the most significant bit denotes the sign.

\subsubsection{Floating Point Format}
An $n$-bit signed floating point (FP) number $x$ comprises of a 1-bit sign ($x_{\mathrm{sign}}$), $B_m$-bit mantissa ($x_{\mathrm{mant}}$) and $B_e$-bit exponent ($x_{\mathrm{exp}}$) such that $B_m+B_e=n-1$. The associated constant exponent bias ($E_{\mathrm{bias}}$) is computed as $(2^{{B_e}-1}-1)$. We denote this format as $E_{B_e}M_{B_m}$.  

\subsubsection{Quantization Scheme}
\label{subsec:quant_method}
A quantization scheme dictates how a given unquantized tensor is converted to its quantized representation. We consider FP formats for the purpose of illustration. Given an unquantized tensor $\bm{X}$ and an FP format $E_{B_e}M_{B_m}$, we first, we compute the quantization scale factor $s_X$ that maps the maximum absolute value of $\bm{X}$ to the maximum quantization level of the $E_{B_e}M_{B_m}$ format as follows:
\begin{align}
\label{eq:sf}
    s_X = \frac{\mathrm{max}(|\bm{X}|)}{\mathrm{max}(E_{B_e}M_{B_m})}
\end{align}
In the above equation, $|\cdot|$ denotes the absolute value function.

Next, we scale $\bm{X}$ by $s_X$ and quantize it to $\hat{\bm{X}}$ by rounding it to the nearest quantization level of $E_{B_e}M_{B_m}$ as:

\begin{align}
\label{eq:tensor_quant}
    \hat{\bm{X}} = \text{round-to-nearest}\left(\frac{\bm{X}}{s_X}, E_{B_e}M_{B_m}\right)
\end{align}

We perform dynamic max-scaled quantization \citep{wu2020integer}, where the scale factor $s$ for activations is dynamically computed during runtime.

\subsection{Vector Scaled Quantization}
\begin{wrapfigure}{r}{0.35\linewidth}
  \centering
  \includegraphics[width=\linewidth]{sections/figures/vsquant.jpg}
  \caption{\small Vectorwise decomposition for per-vector scaled quantization (VSQ \citep{dai2021vsq}).}
  \label{fig:vsquant}
\end{wrapfigure}
During VSQ \citep{dai2021vsq}, the operand tensors are decomposed into 1D vectors in a hardware friendly manner as shown in Figure \ref{fig:vsquant}. Since the decomposed tensors are used as operands in matrix multiplications during inference, it is beneficial to perform this decomposition along the reduction dimension of the multiplication. The vectorwise quantization is performed similar to tensorwise quantization described in Equations \ref{eq:sf} and \ref{eq:tensor_quant}, where a scale factor $s_v$ is required for each vector $\bm{v}$ that maps the maximum absolute value of that vector to the maximum quantization level. While smaller vector lengths can lead to larger accuracy gains, the associated memory and computational overheads due to the per-vector scale factors increases. To alleviate these overheads, VSQ \citep{dai2021vsq} proposed a second level quantization of the per-vector scale factors to unsigned integers, while MX \citep{rouhani2023shared} quantizes them to integer powers of 2 (denoted as $2^{INT}$).

\subsubsection{MX Format}
The MX format proposed in \citep{rouhani2023microscaling} introduces the concept of sub-block shifting. For every two scalar elements of $b$-bits each, there is a shared exponent bit. The value of this exponent bit is determined through an empirical analysis that targets minimizing quantization MSE. We note that the FP format $E_{1}M_{b}$ is strictly better than MX from an accuracy perspective since it allocates a dedicated exponent bit to each scalar as opposed to sharing it across two scalars. Therefore, we conservatively bound the accuracy of a $b+2$-bit signed MX format with that of a $E_{1}M_{b}$ format in our comparisons. For instance, we use E1M2 format as a proxy for MX4.

\begin{figure}
    \centering
    \includegraphics[width=1\linewidth]{sections//figures/BlockFormats.pdf}
    \caption{\small Comparing LO-BCQ to MX format.}
    \label{fig:block_formats}
\end{figure}

Figure \ref{fig:block_formats} compares our $4$-bit LO-BCQ block format to MX \citep{rouhani2023microscaling}. As shown, both LO-BCQ and MX decompose a given operand tensor into block arrays and each block array into blocks. Similar to MX, we find that per-block quantization ($L_b < L_A$) leads to better accuracy due to increased flexibility. While MX achieves this through per-block $1$-bit micro-scales, we associate a dedicated codebook to each block through a per-block codebook selector. Further, MX quantizes the per-block array scale-factor to E8M0 format without per-tensor scaling. In contrast during LO-BCQ, we find that per-tensor scaling combined with quantization of per-block array scale-factor to E4M3 format results in superior inference accuracy across models. 



%\theendnotes

\end{document}







