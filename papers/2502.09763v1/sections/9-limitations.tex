\addition{
\section{Discussion}
\label{sec:discussion}
\DAF provides a systematic method to classify and analyze \MR deception attacks. %, addressing their effects on information channels and cognitive processes. 
While we focus on \MR headsets, \DAF is applicable to other forms of \MR and even other areas of human-computer interaction (HCI).
Kopp et al.'s information-theoretic framework ~\cite{kopp:2018} applied the Borden-Kopp model of deception to news media.
We have broadened its use to \MR deception attacks.
Future work should extend the scope to other areas of HCI that involve information processing and decision-making.
Our information-theoretic model and decision-making model are not tied to specific technologies or attacks, but rather provide generalizable models for studying the effects of deception in computing.
To enhance \DAF, future work should validate it empirically, expand its applicability to diverse contexts, incorporate individual cognitive factors, and refine models for processing attacks.

%By focusing on how \MR deception attacks impact information channels and decision-making processes, 
Researchers and practitioners can use \DAF to assess the security threat of \MR deception attacks.
For example, we can assign values of 1 to 3 for Low to High ratings, respectively.
Then, we can sum the values to identify which attacks pose the highest threat to perception and attention.
Further, \DAF can help develop deception detection and prevention approaches. % by using information theory to model \MR communication channels.
For example, we can compare differences between rendered frames to see how the signal is changing.
%For example, we can diff displayed frames with previous ones or an expected frame to identify changes in visual information presented to a \MR user.
%These diffs can reveal changes in channel capacity as information is either hidden or injected possibly along with noise.
High volatility in changes may indicate overt degradation attacks, particularly if we can identify noise based on differences between expected and actual frames.
More subtle changes that are spatial located in unexpected areas may indicate covert degradation attacks.
Using eye-tracking sensors on these headsets, we can derive models of attention that can help identify when different types of attention are being employed or disrupted.

% For example, we could use display-capture or eye-tracking data to detect a momentary misdirection attack.
% A misdirection attack 
%It offers foundational understanding for both technical and psychological dimensions of deception, with significant implications for future \MR research. 
%Our framework is adaptable for diverse \MR platforms and can guide empirical research into attack impacts and countermeasures. 

}

\addition{
%\subsection{Limitations}
%\label{sec:limitations}
\textbf{Limitations:} This SoK synthesizes existing knowledge towards developing a field of study around \MR deception. % by establishing a generalizable framework.
%It is essential to acknowledge the limitations of this research, which highlight potential areas for further exploration.
%One significant limitation is the lack of empirical validation. 
It is theoretical in nature and would benefit from further empirical validation.
%While it is rooted in empirical evidence from prior work, it lacks empirical validation.
%DAF’s models and  require real-world testing to confirm their accuracy and practical effectiveness. 
Controlled experiments involving \MR deception attacks are essential for refining the framework and assessing its relevance to diverse scenarios. 
Furthermore, \DAF does not fully account for cognitive diversity among users. 
Individual differences in cognitive capacity, attention, and susceptibility to deception are critical factors that could influence the effectiveness of both attacks and countermeasures. 
%Incorporating these factors would improve the framework’s precision and personalization.
% As \MR technology advances, the sophistication of attacks will also increase, highlighting the importance of further study in this area.
}