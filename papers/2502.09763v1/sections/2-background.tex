\section{Background}
\label{sec:background}

%In order to gain a comprehensive understanding of the potential vulnerabilities and deception attacks related to \MR, it is essential to understand the 
We ground this work in foundational research on deception, information processing, decision-making, and \MR. 
%This section explores the broader domains, comprehensively analyzing the fundamental works that have influenced our comprehension of these processes.

\subsection{Deception}
%Gaining an in-depth understanding of the mechanics of deception is essential to studying deception attacks, particularly in the context of interactive digital environments. 
%Prior research has focused on recognizing the complexities of deception and its various manifestations in human behavior. 

% Mitchell~\cite{mitchell1986deception} defines ``deception'' as the following: %a phenomenon fulfilling the following criteria:

% \begin{itemize}
% \itemsep0em
%     \item [(i)]an organism $R$ registers (or believes) something $Y$ from some organism $S$, where $S$ can be described as benefiting when (or desiring that)
%     \item[(ii)] $R$ acts appropriately towards $Y$, because
%     \item[(iii)] $Y$ means $X$; and
%     \item[(iv)] it is untrue that $X$ is the case.
% \end{itemize}

%Deception entails an intentional act ($Y$) to cultivate a belief ($X$) in the recipient ($R$) that the deceiver ($S$) considers false~\cite{mitchell1986deception,zuckerman:1981}.
Deception entails intentional acts to cultivate a belief in a recipient that the deceiver considers false~\cite{mitchell1986deception,zuckerman:1981}.
%\cite{hyman:1989}
In order to induce false beliefs, communication is required~\cite{mitchell:1996}.
This communication may be verbal or nonverbal.
Deception can be modeled as information processing where a  %(Section~\ref{sec:information-processing})
sender presents ``truthful or false information (a signal) to an opponent (the receiver) in order to gain an advantage over the opponent''~\cite{cranford:2021}.
Separate cognitive processes exist for sender (deceiver) and receiver~\cite{jenkins:2016}. 
%The signaler considers context of the receiver and anticipates outcomes when producing a deceptive signal.
%The receiver anticipates outcomes and evaluates actions based upon the received signal.
Accounts of deception must consider how ``information sharing is dominated by unstructured communication involving natural language and a diverse collection of nonverbal cues''~\cite{jenkins:2016}.
\MR is primarily a visual medium where deceptions will often rely on nonverbal stimuli.
%Tversky and Kahneman~\cite{tversky1974judgment} showed how people often rely on similarity heuristics rather than statistical evidence to base their judgments, revealing common errors in reasoning. 
%Additionally, societal norms influence deceptive behavior. 
%DePaulo et al.~\cite{depaulo1996lying} found that individuals often deceive in social contexts to conform or present themselves positively.

%\cite{gombos:2006}
%A psychological perspective of deception is vital for understanding how manipulative signals in \MR environments impact human cognitive processes, including decision-making. 
%Ekman and Friesen~\cite{ekman1969repertoire} found that facial movements, such as micro-expressions, are linked with deception, providing insights for developing detection techniques.

\addition{%\subsubsection{Models of Deception}
%Prior work has derived models of deception.
Models of deception center around interpersonal communication~\cite{buller:1996,levine2014truth,gaspar:2013,gaspar:2022,kang:2022} or information transmission~\cite{borden1999information, kopp2000information, mcwhirter:2016, mccornack1992information}.
The Interpersonal Deception Theory (IDT)~\cite{buller:1996} examines deception as an interactive, reciprocal relationship where  both senders and receivers adapt their strategies in real-time. 
IDT integrates cognitive and emotional dimensions, such as arousal and suspicion, which influence deceptive behaviors and detection mechanisms during interpersonal exchanges.
Levine’s Truth-Default Theory \cite{levine2014truth} identifies cognitive biases underlying deception detection and shows that humans generally operate under a presumption of honesty.
This facilitates efficient communication but leaves individuals vulnerable to deceit.
The Emotion Deception Model~\cite{gaspar:2013,gaspar:2022} considers how both current emotions and anticipated emotions influence decisions to use deceptions during negotiations. 
McCornack \cite{mccornack1992information} models deception as manipulations of information, emphasizing how individuals exploit conversational norms to mislead others while maintaining an appearance of cooperative communication.
Borden~\cite{borden1999information} and Kopp~\cite{kopp2000information} separately proposed models of deception that are grounded in information theory. 
%Their models would later be combined and identified as the Borden-Kopp model. 
The Borden-Kopp model categorizes four deception strategies for manipulating a victim's perception: Degradation (conceal information), Denial (increase uncertainty), Corruption (create false belief), and Subversion (alter information processing).
%, focusing on how uncertainty and ambiguity in information are manipulated to mislead adversaries systematically.
% emphasizes how deceptions in news media can target both communication channels and cognitive processes.
%Other models lack clear connections between a deception and how it can be enabled through digital media. 
We use these strategies as the basis for the foundational organization of an \MR deception attack ontology and analysis framework.}

\subsection{Information Processing}
\label{sec:information-processing}
%Our framework for describing and studying \MR deception attacks is heavily rooted in information processing theory.
Information processing theory emerged as a way of understanding human cognition, particularly problem-solving and decision-making, alongside advancements in computing during the 1950s and 1960s~\cite{simon1979information}. 
%Simon introduced the information processing approach to cognition, which is still a dominant approach today.
In this theory, computational models describe how humans acquire, process, and store information to make decisions and take actions. % based on that information. 
The information processing model operates in a serial manner.
First, information is input through sensory receptors in the body. 
Then, information is sequentially stored in working (short-term) memory and mentally processed in decision-making.
Finally, responses are output as human actions.
In order to avoid sensory overload, attention mechanisms filter what information is stored and processed.
We use information processing theory to derive our \MR Deception Decision-Making Model (Section~\ref{sec:decision-making}), which connects sensory input transmitted from \MR headsets to attention, memory, and other cognitive processes.
%He identified four stages of information processing: input, processing, storage, and output. 

% The Human Processor Model describes human-computer interaction as a set of processors, memories, and their interconnections~\cite{card1986model}. 
% The perceptual processor stores information from sensory inputs in working memory. 
% Then, the cognitive processor uses associative links in long-term memory to alter working memory and make decisions to act. 
% Finally, the motor processor enacts the decisions of the cognitive processor through physical responses. 
% Similar to information processing theory, information is input, stored, and processed, leading to responses. 
% In Sections~\ref{sec:decision-making} and \ref{sec:cog-framework}, we present models that show how deceptive input is stored and processed affecting the perceptual and cognitive processors.

%Our \MR Deception Decision-Making Model (Section~\ref{sec:decision-making}) and \MR Deception Cognitive Framework (Section~\ref{sec:cog-framework}) show how deceptive input is stored and processed affecting the perceptual and cognitive processors.
%This model provides mathematical formulas for assessing human performance based on the rate at which information is transmitted, received, and processed.
%drawing upon information process models of cognition describes the relationship between human and machine as analogous to information processing~\cite{card1986model}.
%A computer transmits information through visual displays, speakers, and other devices to the human user, and the user responds in kind through input devices, such as mice and keyboards, as a signal to the machine.

% Shannon formalized information communication as a mathematical model to describe the processes of information transmission and reception~\cite{shannon1948mathematical}. 
% Within Shannon's model is a representation of noise that disrupts the signal or alters transmitted messages, a potential mechanism for attackers to exploit in \MR environments.
% In particular, we draw upon Shannon's equation for channel capacity to represent how much information is transmitted and received within \MR headsets given bandwidth and the \SNR.
% We apply Kopp et al.~\cite{kopp:2018} information-theoretic model of deception in the news, which draws upon Shannon's model, to \MR deception attacks.




%making it relevant to analyze communication challenges, particularly in virtual environments. 



\subsection{Decision-Making}

Decision-making is a complex cognitive process that is susceptible to deception~\cite{dunbar:2014}. %, influenced by internal and external factors. 
%Ernst and Paulus~\cite{ernst2005neurobiology} identified 
It consists of three stages~\cite{ernst2005neurobiology}. % of decision-making. 
%(1) deriving a preference; (2) selecting and performing an action; and (3) evaluating the action's outcome.
First, sensory input is processed to make assessments and predictions on possible outcomes.
Second, cognitive processes select an action based on the perception of inputs and predictions of outcomes.
Third, action responses are assessed to evaluate the outcome.
%Kahneman and Tversky explored how decisions are made under uncertainty, providing valuable insights for vulnerabilities from deceptive stimuli in \MR. 
Individuals often do not evaluate risks based on mathematical probabilities~\cite{prospect-theory}. 
Instead, psychological factors, such as the certainty effect, play crucial roles. 
With the certainty effect, humans give more weight to outcomes that are seen as certain compared to those that are merely probable.
This insight is valuable when anticipating \MR user responses to deceptive stimuli, where the perception of risk and reward can be manipulated.
%Pertaub et al.~\cite{Pertaub} studied the influence of virtual audiences on a speaker's anxiety during public speaking engagements, highlighting the potential of \MR as a tool for investigating social decision-making processes. 
Niforatos et al.~\cite{niforatos} point out the complexities of ethical decision-making within \MR, emphasizing the impact that immersive technologies have on human cognitive evaluations.

% Sensemaking---the process of searching for and understanding information to answer task-specific questions~\cite{russell1993cost}---is a prerequisite for decision-making that helps formulate options through forming meaningful representations of collected information.
% We find it particularly relevant to \MR environments where physical environments are augmented with digital information.
% \MR headsets are continuously presenting new information to users that they must interpret and then make decisions about what action to take.
% Models of sensemaking typically involve iterative cycles of searching, processing, and evaluating information  \cite{zhang2014towards}.
% Outcomes of these cycles help inform decision-making processes.


% This understanding can be pivotal when considering how users might respond to certain stimuli or deceptive cues in \MR environments, where the perception of risk and reward might be manipulated or framed in specific ways. 

% Both incidental and integral emotions play key roles in decision-making processes~\cite{schwarz2000emotion}.
% %At the time a decision is made, a person's affective state influences their judgment.
% Incidental emotions, which relate to a person's mood, carry over from one situation to another affecting decisions that are unrelated to ones that elicited the emotions~\cite{lerner2015emotion}.
% For example, a person in a happy mood is less likely to consider negative outcomes and will make decisions to take higher risk actions.
% Integral emotions, which arise when a choice is at hand, influence and bias decision-making~\cite{lerner2015emotion}. 
% For example, someone with social interaction anxiety at a party might avoid conversations by excessively dwelling on unlikely negative consequences.

% For example, a person experiencing anxiety related to social interactions at party may choose to avoid conversations due to overly focusing on unlikely harmful outcomes.
%~\cite{bechara2000emotion}

%Regarding decision-making in a \VE, 


 




% A psychological perspective of deception is essential to understanding how manipulative signals within a \MR environment affect cognitive processes associated with human behavior, such as decision-making.
% Ekman and Friesen's research~\cite{ekman1969repertoire} on micro-expressions and the application of lie detection techniques provides valuable insights into transitory facial movements. These insights have played a crucial role in the development of lie-detection techniques and the non-verbal signs associated with human deception. %The work by
% Tversky and Kahneman %on heuristics and biases
% ~\cite{tversky1974judgment} 
% %has had a major impact on our understanding of human judgment and decision-making. Their research 
% illuminated how individuals often rely on the representativeness heuristic, making judgments based on similarity rather than statistical evidence. Furthermore, their findings on misconceptions of chance and regression shed light on common errors in probabilistic reasoning. These insights highlight the complexities of human cognition and the potential vulnerabilities to biases, especially when confronted with uncertainty. In addition to personal biases, deceptive behaviors can be influenced by societal norms and pressures. DePaulo et al.~\cite{depaulo1996lying} found that individuals frequently engage in deceptive behavior within social contexts, primarily motivated by the desire to conform to social expectations or to create a positive self-presentation.

% START OLD STUFF
% \subsection{Risk Factors in Human Cognition}

% Acknowledging passive risk factors inherent in human cognition is crucial. Fiske and Taylor~\cite{fiske1991social} highlight how personal experiences, societal influences, and environmental factors shape perception and decision making. Siegrist~\cite{Siegrist2019} asserts that trust significantly influences risk perception, particularly in knowledge-deficient contexts. The conceptualization and measurement of trust profoundly affect its association with risk perception, revealing a complex relationship with cognitive processes. Fischhoff et al.~\cite{Fischhoff_1978} indicate that uncontrollable risks are often perceived as more severe.

% Other studies, including one by Siegrist and Árvai~\cite{Siegrist2020} stress risk perception's impact on behavior, categorizing it into hazard nature, perceiver traits, and heuristic use in risk judgments. Visschers and Siegrist~\cite{Visschers2018} argue that risk perception is influenced by various factors, including risk characteristics, individual differences, context, societal factors, information quality, and cognitive processing. These factors, rooted in individual and societal contexts, shape information interpretation and response.

% Slovic et al.~\cite{slovic-rating-the-risks} discuss the tendency to overestimate risks of dramatic or catastrophic events due to the availability heuristic—a bias where easily imaginable events are judged as more likely. Such biases and preconceptions can pose vulnerabilities, especially in deceit-prone virtual environments.


% It is vital to acknowledge the passive risk factors that are inherent in human cognition. Fiske and Taylor~\cite{fiske1991social} highlight the ways in which personal experiences, societal influences, and environmental factors can influence perception and decision-making.
% Siegrist~\cite{Siegrist2019} claims that trust plays a pivotal role in how individuals perceive and assess risks, especially in contexts where knowledge might be lacking. The way trust is conceptualized and measured has a profound impact on the observed associations with risk perception, indicating the intricate relationship between trust and individual cognitive processes. Fischhoff et al.~\cite{Fischhoff_1978} in their study on risk characteristics, indicate that risks perceived as uncontrollable are often rated as more severe.

% Numerous studies, including those by Siegrist and Árvai~\cite{Siegrist2020}, emphasize the importance of risk perceptions in shaping people's behavior. This research further classifies risk perception into three dominant perspectives: the nature of hazards, the traits of the risk perceivers, and the application of heuristics in making risk judgments. %Another significant aspect is how being close to danger, especially in areas of high natural hazards, can influence risk perception and the willingness to undertake mitigative measures.
% Visschers and Siegrist~\cite{Visschers2018} present that the perception of risk is shaped by various factors, which include characteristics of the risk, individual differences, the context or setting, societal factors, the type and quality of information provided, and the cognitive processing of the risk.These factors, stemming from individual predispositions and societal contexts, can shape how individuals interpret and respond to information.


% Slovic et al.~\cite{slovic-rating-the-risks} discussed the tendency for events that are dramatic or catastrophic to be overestimated in terms of risk. This overestimation is attributed to the cognitive bias known as the availability heuristic. This cognitive bias causes individuals to judge an event as more likely if instances of it are easily imaginable or recallable.
% As a result, these biases and preconceptions can potentially cause vulnerabilities, particularly in virtual environments that are susceptible to deception.

% END OLD STUFF

% These factors, which are caused by both individual predispositions and larger societal contexts, have the potential to influence how individuals interpret and react to information.
\begin{figure*}[t!]
    \centering
    \includegraphics[width=.9\textwidth]{figures/methodology.pdf}
    \caption{Our five-stage methodology beginning with literature review (top). Outcomes of the literature review informed intermediary stages. % where new knowledge was generated.  
    Knowledge from these stages culminates in the \MR \acf{DAF}.}
    %each develop new knowledge, which includes creating our Deception Attack Mindmap and formulating information-theoretic and decision-making models driven by an iterative literature review. We then use this knowledge to come up with our own MR Deception Cognitive Framework as our ensemble knowledge.}
    \label{fig:method}
    \vspace{-2ex}
\end{figure*}



\subsection{Mixed Reality}
%\MR has received significant attention in the field of technological progress, introducing an era of connected real and virtual environments.
Milgram and Kishino~\cite {milgram} defined \MR as a continuum of blended visual representations residing between the entirely real and the fully virtual. 
Within their continuum are two forms of \MR: \AR and \AV.
%Based on the definition by Milgram and Colquhoun~\cite{milgram&colquhoun},
In \AR, virtual elements overlay physical reality. % transforming how users interact with the real world.
%, providing additional information. %, which helps them in completing their intended tasks. 
%According to Fuchs \& Ackerman~\cite{Fuchs} and Azuma~\cite{Azuma}, 
%One notable example of \AR is the utilization of a \HMD~\cite{Azuma,Fuchs}, such as the Apple Vision Pro or Meta Quest 3.
\VR headsets, such as the Apple Vision Pro or Meta Quest 3, now support \AR experiences through video pass-through where virtual information is overlaid onto camera feeds of the real world. In contrast, 
\AV integrates real-world elements into \VR. % and integrates real-world elements into it.
For example, many \VR headsets display boundaries of physical spaces as users approach to avoid collisions.
In this work, we focus on \AR and \AV systems that facilitate complex information processing scenarios. 
%in which virtual content is overlaid onto physical spaces. %, affecting how users interact with and perceive the real world. 
%require complex information processing as users must effectively perceive and attend to both physical and digital elements. 
This complexity raises questions about how users interact with information in \MR and the potential cognitive risks or vulnerabilities in decision-making. %As \MR technologies mature, it becomes important to understand their nuances in order to study deception attacks with the potential for significant physical harm. %that target these systems.
To the best of our knowledge, this is the first work exploring the impact of \MR deception attacks on information communication and decision-making.

% on \acp{HMD}. %~\cite{Azuma,Fuchs}. 
%These devices block a user's vision of the physical environment, creating a \MR space by overlaying virtual information onto camera feeds of the physical environment.
%These devices allow users to perceive a transparent view of the real world, enhanced by the addition of computer-generated graphics.
%Commercial products, such as the HoloLens 2~\cite{Microsoft}, represent the application of \AR by utilizing headsets to provide immersive experiences that seamlessly integrate the digital and physical domains.


% The popularity of \MR, particularly \AR, is evident in its widespread use in modern applications. Pokémon Go~\cite{pokemon2016pokemon} is a viral example of an \AR mobile game that overlays virtual elements onto the real world as seen through a mobile device's screen. These applications highlight \AR's adaptability and potential to completely change the way that people interact with the physical world. 


%\todo[]{create a diagram (visual-auditory-tactile)}


%\todo[inline]{forms of mixed reality}

%\todo[inline]{what is the scope of MR considered in this paper}


