\section*{Ethics Considerations}
\label{sec:ethics}

Conducting research on deception requires significant ethical considerations.
At the forefront is mitigating risks to human participants and end users.
When deceiving participants, it is necessary to ensure that benefits of the research far outweigh any potential risks to participants.
Typically, research institutions have an IRB to enforce participant protections from unnecessary harm during human-subjects research.
For \MR research, harm can take many forms including physical, cognitive, technological, and social.
As \MR headsets affect how users perceive the physical world, deception attacks pose significant physical risk.
Precautions must be taken to mitigate risks by screening out participants that may have adverse reactions to perceptual manipulations.
Further, researchers should provide safe environments where participants cannot harm themselves by colliding with objects or falling down.
Researchers should also consider how deceptive information may impact participants trust and understanding of \MR systems.
Studies require effective debriefing that helps the participant understand how they were deceived, what elements were deceptive, and how to evaluate potential deceptions.
While this SoK synthesizes knowledge from diverse domains, it does not directly involve human-subjects research or development of interactive systems.
However, we do provide a framework for exploring cognitive and technological harm of deception attacks in \MR.

\addition{
\section*{Open Science}
The primary artifact for this SoK is a comprehensive list of articles analyzed during development of the ontology and corresponding \MR Deception Analysis Framework. 
%This list includes citation details, categorization, and any metadata used for analysis.
%This SoK generated a dataset of scholarly articles as part of our literature review and development of \DAF. 
This list includes articles cited in this work as well as others that are not cited. A link to this list can be found at: \url{https://doi.org/10.5281/zenodo.14732979}. No other research artifacts, besides diagrams and tables presented in this paper, resulted from this research.
}
% All submissions are hence required to have an ethics considerations section in the main body of the paper, or in the extra page offered for "ethics considerations and compliance with the open science policy" (see the Paper Format section above), or both. In some cases, the ethics discussion may be short; in other cases, the ethics consideration may be long. Regardless of length, from reading the main body of the paper and the extra "ethics considerations and compliance with the open science policy" page, it should be clear to reviewers that the authors made sound and responsible ethical decisions.

% \url{https://www.usenix.org/conference/usenixsecurity25/ethics-guidelines}