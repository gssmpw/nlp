\section{Introduction}

%By seamlessly blending digital content with the real world, 
\MR is reshaping how we perceive and interact with our physical surroundings.
In 2023, the global \MR market surged to \$4.6 billion, fueled by leading tech giants Meta, Apple, and Microsoft~\cite{mr-market}. 
\MR headsets overlay virtual information onto the real world to assist human users, such as visualizing navigational aids on sidewalks to guide pedestrians.  
Malicious actors can exploit \MR headsets to manipulate user perceptions and cause significant physical or social harm. 
For example, attackers can guide pedestrians into traffic by obstructing their view of oncoming vehicles. 
%Hence, it is essential to understand, describe, investigate, and evaluate the security vulnerabilities inherent in \MR technologies. 

Deception attacks pose a fundamental security threat for technologies that alter human perception of the real world.
Deceptions introduce false beliefs or interpretations in a target~\cite{hyman1989psychology}. 
Illusions, central to deception, lead to false perceptions of sensory input~\cite{jastrow1900psychology}, achieved through deceit, where truthful information is hidden or false information is shown~\cite{adar2013benevolent}. 
Using \MR, attackers can affect information communication and decision-making, such as by introducing illusions (e.g., fake pedestrian crossings) or hiding essential information (e.g., navigation arrows).
Protecting \MR users is vital, yet we lack theoretical framing to describe and analyze \MR deception attacks and their effects on human cognition. %, crucial for safeguarding users. 

This paper systematizes knowledge from disparate domains, introducing a framework for evaluating %threat models for 
\MR deception attacks. We address the following research questions:

\begin{itemize}[labelsep=0.1cm,leftmargin=*,labelindent=1cm]
\itemsep0em 

    \item[\textbf{RQ1:}] How does existing literature categorize \MR deception attacks? %informed by existing literature to enhance the ongoing discourse in the field?
    %\item[\textbf{RQ2:}] How do we model the ways that \MR deception attacks affect information communication?
    \item[\textbf{RQ2:}] How do we model the effects of \MR deception attacks on information communication?
    \item [\textbf{RQ3:}] How are the cognitive processes associated with decision-making affected by \MR deception attacks?
    \item [\textbf{RQ4:}] How can we systematically analyze \MR deception attacks and their effects? % to support research, practice, and discourse?
\end{itemize}


%To address these RQs, 
Our multi-stage methodology synthesizes knowledge from \MR security, information theory, and cognition to derive our \MR \DAF. 
%First, we performed a multidisciplinary literature review. 
First, we derived an \MR deception attack ontology from the literature. % and used it to explore existing technical attacks. 
Then, we integrated our ontology, an information-theoretic model of communication, and a cognitive decision-making model to derive our framework. 
Our work contributes the following:

\begin{itemize}
\itemsep0em
    \item the \textbf{first in-depth investigation of deception attacks in \MR environments}; 
    \item a \textbf{deception analysis framework for assessing the effects of \MR deception attacks} on information channels and decision-making;
    \item an \textbf{ontology of \MR deception attacks};
    \item an \textbf{information-theoretic model of \MR deception attacks} that formalizes effects on communication;
    %the capacity of \MR environments to communicate information;
    \item a \textbf{decision-making model of \MR deception attacks} that connects cognitive processes, attacks, and effects;
    \item a \textbf{literature review of deception attacks} in \MR;
    \item an \textbf{assessment of state-of-the-art \MR technical attacks} use or potential use in deception attacks.
\end{itemize}

This paper is structured as follows. 
Section \ref{sec:background} grounds our work in foundational research. 
Section \ref{sec:methodology} outlines our methodology. 
Section \ref{sec:relwork} presents a literature review of \MR deception attacks. % and surveys. 
Section \ref{sec:decattacks} describes an ontology of existing attacks. % based upon our review. 
Sections \ref{sec:information-theoretic} and \ref{sec:decision-making} develop information-theoretic and decision-making models to assess how \MR deception attacks affect communication and cognition, respectively. 
%Employing these models, 
Section \ref{sec:cog-framework} introduces our \MR Deception Analysis Framework. % for assessing the impact of \MR deception attacks.
\addition{Section \ref{sec:discussion} discusses implications and limitations of this work.}
Section~\ref{sec:conclusion} summarizes our contributions and suggests future work.
%Section~\ref{sec:ethics} discusses ethical considerations.
% \removal{Lastly, Appendix A illustrates a way to use DAF for threat modeling.
% }
% The rest of the paper is structured as follows. We begin with an introduction to fundamental theories of information processing, decision-making, psychology of human deception, cognitive risk factors, and a comprehensive overview of \MR (Section \ref{sec:background}). 
% Next, we discuss our methodology (Section~\ref{sec:methodology}) for developing our framework. We present a literature survey of prior work studying perceptual manipulation attacks (Section \ref{sec:relwork}). 
% Then, we define an ontology of deception attacks using a mind map to convey the different types of attacks (Section~\ref{sec:decattacks}). 
% Connecting theory and our ontology, we formulate an information-theoretic model of deception attacks (Section \ref{sec:information-theoretic}) and a decision-making model that outlines relevant cognitive processes and associated risks (Section~\ref{sec:decision-making}).  
% Next, we discuss the implications and limitations of our framework (Section \ref{sec:discussion}).
% Finally, we summarize contributions and suggest directions for ongoing \MR security investigations (Section~\ref{sec:conclusion}).
