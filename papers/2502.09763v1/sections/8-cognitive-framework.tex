\section{\MR Deception Analysis Framework (DAF)}
\label{sec:cog-framework}
The culminating, ensemble knowledge that connects our ontology, information-theoretic model, and decision-making model is the \MR \acf{DAF}---an assessment tool for identifying and discussing the multifaceted impact of \MR deception attacks on user cognition (\textbf{RQ4}).

% \subsection{Our Framework}

\DAF classifies attacks according to their operational mechanisms, which can be overt or covert, involving Degradation, Denial, Corruption, or Subversion, and the cognitive processes they aim to disrupt. 
We focus on identifying where attacks manipulate \MR communication channels by altering bandwidth ($W$), signal ($S$), noise ($N$), or by employing mimicry ($M$). Additionally, we explore the cognitive effects of each attack, examining the extent to which they can affect perception, attention, and memory.
%whether they affect active or passive perception ($A/P$)
For perception and attention, we further breakdown analysis into stages of perception---Selection ($Sel$), Organization ($Org$), and Interpretation ($Int$)---and types of attention---Selective ($Sel$), Divided ($Div$), Sustained ($Sus$), and Executive ($Exe$).

Table~\ref{tab:framework} presents our general analysis of the different categories of attacks identified in our ontology.
Overt Degradation and Denial attacks strongly affect both perception and attention.
Covert Degradation, Corruption, and Subversion attacks primarily target the Interpretation stage of perception.
These attacks typically require remaining hidden from the user.
Thus, any effects on attention or early stages of perception are likely too revealing.
%and the extent to which they impact various stages of perception, including selection ($Sel$), organization ($Org$), and interpretation ($Int$). In addition, we evaluate the extent to which attention including selective ($Sel$), divided ($Div$), sustained ($Sus$), and executive ($Exe$) is affected. Furthermore, we investigate memory to see if the attacks have any impact on long-term or working memory.

\finding{The interpretation stage of perception is a primary target of \MR deception attacks. Deceptions seek to cultivate false beliefs, formed initially by interpretations of perceived stimuli.}

% The purpose of mapping these attacks to our models is to help \MR system developers and security experts form systematic understanding of the potential vulnerabilities to users. 
% Table \ref{tab:framework} summarizes the detailed application of our framework to each identified deception attack within \MR environments and provides a comprehensive analysis of each attack, categorized according to our models of information-theoretic and decision-making.
% Here, we analyze the implications of these attacks in more detail, using the specified criteria from Table \ref{tab:framework} to evaluate their influence on the attentional and perceptual processes crucial for effective decision-making in \MR.

%In our extrapolated decision-making model depicted in Figure \ref{fig:process-model}, both perception and attention emerge as critical components, vulnerable to exploitation by attackers in a given \MR attack. Here, we explore the threats of the deceptive attacks on both the perceptual and attentional processes. The three major processes involved in one's perception that may be exploited during a deceptive \MR attack include:

For assessing the degree to which attacks affect stages of perception, we derived the following questions. Answers are either Low, Medium, High, or a combination of the three. \addition{De Meyer et al.~\cite{de2019delphi} state that a three-point scale provides a practical balance between simplicity and reliability. It minimizes measurement error and ensures clarity in response, which can be important for consensus building in Delphi procedures.} % Thus, it is well suited for assessing the stages of perception with sufficient precision in this context.}

\begin{itemize}
\itemsep0em
    \item \emph{Selection:} To what degree does the attack make it difficult to attend to or ignore task-related sensory stimuli from the physical or virtual environments during a decision-making task? 
    \item \emph{Organization:} To what degree does the attack make it difficult to group task-related sensory stimuli, such as by proximity or similarity, for a decision-making task?
    \item \emph{Interpretation:} To what degree does the attack make it difficult to accurately assign meaning to organized, task-related stimuli and correctly interpret patterns and relationships within virtual and physical environments when deriving understanding, making decisions, and taking action in a decision-making task?
\end{itemize}    

For assessing the degree to which attacks affect types of attention, we derived the following questions. Answers are either Low, Medium, High, or a combination of the three.
\begin{itemize}
\itemsep0em
    \item \emph{Selective:} To what degree does the attack make it difficult to focus attention on relevant physical and virtual objects for a decision-making task in MR?
    \item \emph{Divided:} To what degree does the attack make it difficult to switch between concurrent tasks rapidly while maintaining situational awareness in both the virtual and physical environments?
    \item \emph{Sustained:} To what degree does the attack make it difficult to continuously scan and interpret information presented in the mixed reality environment, making timely decisions and adjustments?
    \item \emph{Executive:} To what degree does the attack make it difficult to manage attentional resources effectively to interact with virtual elements while remaining aware of and responsive to the physical environment while performing a decision-making task?    
\end{itemize}

\DAF provides a systematic approach to evaluate threats posed by \minoraddition{\MR} deception attacks. We posit that such analysis is pivotal for developing more resilient \MR systems and training programs that can mitigate the impacts of deceptive threats. 
% \removal{\textbf{Appendix provides a scenario and instructions for how MR system designers and security researchers can use DAF in the DREAD and CVSS threat modeling frameworks to manage defensive efforts.}}

\finding{\DAF is a tool for defining experimental research on \MR deception attacks. We posit that it can be used to explore future attacks and may be extended for deception analysis beyond \MR research.}

\gap{We need empirical findings to validate and precisely model the impact of \MR deception attacks on cognitive processes and information channels.}

% This framework classifies attacks according to their operational mechanisms, which can be overt, covert, involving degradation, denial, corruption, or subversion, and the cognitive processes they aim to disrupt. The purpose of mapping these attacks to the aspects of human cognition they exploit is to offer a systematic understanding of the potential vulnerabilities that exist in MR environments. The methodology entails analyzing channel characteristics, such as bandwidth ($W$), signal ($S$), noise ($N$), and mimicry ($M$), in relation to MR deception. In addition, this study evaluates the impact of these attacks on the perception mechanisms, including active/passive ($A/P$) perception, selection ($Sel$), organization ($Org$), and interpretation ($Int$). It also examines the effects on attention and memory.


% \subsection{Case Study: Applying the \MR Cognitive Framework to a Combat Medic Scenario}

% \input{sections/scenario-table}
% %need for empirical findings to test theoretical models

% %our framework informs how we should design evaluation apparatusese.g., control signal to noise ratio in degradation attacks, eye tracking for attention measures

% %how can we measure the relationship between information capacity and deception?
% As an example for how our framework can be used to assess the cognitive impacts of particular deception attacks on a \MR system, we present the following scenario.
% A combat medic uses a \MR system designed to enhance their decision-making skills and medical response capabilities on the battlefield. The \MR system overlays vitals information for wounded soldiers that the medic is treating.
% %the real battlefield setting, providing an immersive experience that combines virtual patient data with the physical surroundings to simulate emergency medical scenarios.\\
% The enemy gains access to the \MR system, unknown to the medic. 
% They exploit system vulnerabilities and initiate deception attacks to disrupt the medics decision-making. 
% The attacker initiates a selective quality erosion attack by reducing the visual resolution and subtly reducing the clarity of vitals readings, making it difficult for medics to diagnose patients. % and affects their diagnoses and treatment decisions.
% The attacker then selectively removes heart rate information from the medic's view. %This disappearance of vital signs from the \MR overlay creates a critical gap in the information channel and again affects medics' decision-making processes. 
% Alternatively, the attacker could manipulate the heart rate plot to falsely indicate a soldier is going into cardiac arrest causing the medic to react with potentially harmful treatments.  
% %This manipulation is intended to mislead and undermine medics' confidence in their ability to interpret patient data and make them question their own decisions.

% We employ our framework to analyze the impacts of the three mentioned attacks in our scenario to determine how each attack affects the \MR system's communication channel and medic's cognition (Table~\ref{tab:scenario}).
% The initial attack, characterized as a quality erosion attack, notably constrains the bandwidth ($W$) within the \MR system, restricting the volume of information that can be effectively transmitted via the MR headset. Concurrently, the diminished visual resolution injects noise ($N$) into the system, disrupting the clarity of transmitted data.
% In terms of the decision-making model, this attack notably influences active perception, exerting a pronounced effect on the selection stage due to the heightened effort required to discern essential details from compromised visuals. The interpretation stage is similarly impacted at a high level, as the medics are tasked with making sense of ambiguous data. The organization stage experiences a medium impact; the reduced quality of visuals hampers systematically arranging visual information. This degradation challenges selective and executive attention, demanding increased focus to isolate pertinent information and necessitating decisions based on visuals of questionable reliability. Additionally, both divided and sustained attention are moderately affected, underscoring the increased cognitive load placed on medics as they interact with the altered \MR environment.

% The second attack, deliberate removal of critical data, in this case, heart rate information, from the \MR environment, narrows the bandwidth ($W$) and also reduces the signal ($S$) strength.
% This attack heavily affects active perception by entirely removing a vital piece of data from the medics' view and so impacting the selection stage at a high level as medics are forced to seek alternative sources. The interpretation stage also suffers a high impact due to the challenges of making informed decisions with incomplete data. Although, the organization stage sees a low to medium impact, since less data is available to organize.
% %The overall effect on perception is profound, as the structuring and subsequent interpretation of medical scenarios become significantly limited.
% The removal attack places a high demand on both selective and executive attention. The medic must use the remaining accessible data to infer missing information in order to make critical decisions. Divided attention is also affected highly because they have to compensate for missing data potentially affecting their ability to maintain awareness of other simultaneous patient needs. Sustained attention has a medium impact, challenging medics to remain engaged and vigilant despite the gaps in data continuity. %Sustained attention also sees a medium impact, challenging medics to remain engaged and vigilant despite the gaps in data continuity.

% The third attack involves manipulating heart rate data displayed in the \MR system to falsely show significant fluctuations in patient vitals. The distortion involves mimicry ($M$) by presenting altered data as authentic, subtly manipulating users. This attack also heavily affects the interpretation stage of perception, where medics face high difficulty in accurately assessing patient conditions due to deceptive alterations. It also has a high impact on all types of attention. By presenting altered vital signs as accurate, this attack demands intense scrutiny from medics, significantly challenging their selective, divided, sustained, and executive attention. Medics must verify the authenticity of the data, manage multiple sources of information, maintain focus over extended periods, and make critical decisions based on potentially misleading data.





