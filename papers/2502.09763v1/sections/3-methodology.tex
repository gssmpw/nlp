\section{Methodology}
\label{sec:methodology}
%In order to understand and model deception attacks in \MR environments, it is necessary to employ a systematic and comprehensive methodology.


We employed a systematic methodology that included an extensive literature review and development of theoretical models to describe the \MR deception attacks.
The literature review identified relevant theories, models, attacks, and empirical outcomes, which informed each step in our process (Figure~\ref{fig:method}). 
Using our review and expert knowledge, we derived an ontology of \MR deception attacks (Figure~\ref{fig:attacks-mindmap}). 
We connected technical attacks from the literature to our ontology (Table~\ref{tab:attacks}).
Subsequently, we integrated an information-theoretic perspective to examine how deception attacks impact information communication in \MR (Figure~\ref{fig:mr-communication-model}). Next, we developed a decision-making model that describes how cognitive processes handle sensory stimuli from \MR headsets (Figure~\ref{fig:process-model}). Finally, we combined our ontology and two models to derive a framework for assessing the cognitive effects of \MR deception attacks on decision-making (Table~\ref{tab:framework}). %Each step of our methodology is essential in building an in-depth understanding and framework of the subject.

%steps: initial literature review, mind mapping, model development

%mini-Delphi: face-to-face meeting by experts to evaluate and refine the mind map and develop a model

%\begin{itemize}
%\subsection{Literature Survey}
%\item 
\emph{\textbf{Literature Review}}\addition{~(Section~\ref{sec:relwork})}: 
We conducted a systematic literature review covering a wide range of topics, including deception, privacy, perceptual manipulation, cognition, and decision-making.
%defined by explicit criteria for the inclusion and exclusion, ensuring comprehensive coverage of relevant topics. 
We used Google Scholar, ACM Digital Library, IEEE Xplore, MIT Press, and APA PsycArticles. % to collect relevant articles. 
Search terms included mixes of ``\AR/\VR Security'', ``\MR Deception'', ``Perceptual Manipulation'', and ``Decision-Making''. 
We collected articles from reputable journals and conference proceedings, including  USENIX Security, S\&P, ISMAR, IEEE VR, and the Journal of Experimental Psychology. 
We limited articles to those published in the past five years to ensure relevance to current \MR technologies. 
However, we additionally took into account important historical works that had significant impact. % on the field. 
We focused on articles %that clearly addressed our research focus and only considered those 
with well-defined research questions, comprehensive analysis, and innovative contributions.

The filtering process began by retrieving over 200 articles from search engines and databases. Two researchers screened titles and abstracts for relevance to ensure a consensus-based approach. The criteria for relevance included: alignment with \textbf{\MR security}, \textbf{information theory}, \textbf{cognition}, and \textbf{deception psychology}; presence of well-defined research questions; and contributions to the field's innovation and depth of analysis. 
%We assessed the research questions and contributions. 
We reviewed full texts to confirm suitability based on the depth of analysis, innovation, and relevance to our research objectives. 
Articles were excluded if they lacked depth of analysis, innovation, or relevance to the key themes. % of MR security, information theory, deception psychology, and cognition.
%Articles that met all inclusion criteria were selected. %After the full-text review, we included articles that met all inclusion criteria and contributed significantly to understanding MR security, privacy-related attacks, and the psychological impacts of virtual deception and manipulation. 
This resulted in a final selection of ($n=80$) articles across different domains, which are categorized in Table \ref{tab:PaperCategories}.

% ... sources focused on \MR topics and different privacy-related attacks, as well as ... papers regarding psychological aspects and human cognition.

% \section*{References}
\begin{thebibliography}{00}

\bibitem{EnvironmentalPollution} Liu W, Xu Y, Fan D, et al. Alleviating corporate environmental pollution threats toward public health and safety: the role of smart city and artificial intelligence[J]. Safety Science, 2021, 143: 105433.

\bibitem{PervasiveandMobileComputing} Nagy A M, Simon V. Survey on traffic prediction in smart cities[J]. Pervasive and Mobile Computing, 2018, 50: 148-163.

\bibitem{ExpertSystems} Pramanik M I, Lau R Y K, Demirkan H, et al. Smart health: Big data enabled health paradigm within smart cities[J]. Expert Systems with Applications, 2017, 87: 370-383.

\bibitem{ITJ19} Kong X, Liu X, Jedari B, et al. Mobile crowdsourcing in smart cities: Technologies, applications, and future challenges[J]. IEEE Internet of Things Journal, 2019, 6(5): 8095-8113.

\bibitem{BigData} Zhai S, Li R, Yang Y. Crowdsensing big data: sensing, data selection, and understanding[C]//Journal of Physics: Conference Series. IOP Publishing, 2021, 1848(1): 012045.

\bibitem{TMC19} Liu C H, Dai Z, Zhao Y, et al. Distributed and energy-efficient mobile crowdsensing with charging stations by deep reinforcement learning[J]. IEEE Transactions on Mobile Computing, 2019, 20(1): 130-146.

% \bibitem{INFO23} Sun J, Jin H, Ding R, et al. Multi-Objective Order Dispatch for Urban Crowd Sensing with For-Hire Vehicles[C]//IEEE INFOCOM 2023-IEEE Conference on Computer Communications. IEEE, 2023: 1-10.

\bibitem{INFO21} Fan G, Zhao Y, Guo Z, et al. Towards fine-grained spatio-temporal coverage for vehicular urban sensing systems[C]//IEEE INFOCOM 2021-IEEE Conference on Computer Communications. IEEE, 2021: 1-10.


\bibitem{INFO23Privacy}Sun H, Xiao M, Xu Y, et al. Privacy-preserving Stable Crowdsensing Data Trading for Unknown Market[C]//IEEE INFOCOM 2023-IEEE Conference on Computer Communications. IEEE, 2023: 1-10.


\bibitem{INFO22Privacy} Wang H, Wang E, Yang Y, et al. Privacy-preserving online task assignment in spatial crowdsourcing: A graph-based approach[C]//IEEE INFOCOM 2022-IEEE Conference on Computer Communications. IEEE, 2022: 570-579.

\bibitem{TITS2020Privacy} Zhang J, Yang F, Ma Z, et al. A decentralized location privacy-preserving spatial crowdsourcing for internet of vehicles[J]. IEEE Transactions on Intelligent Transportation Systems, 2020, 22(4): 2299-2313.

\bibitem{vehiclecrowdsensingdeepapproach} Zhu X, Luo Y, Liu A, et al. A deep learning-based mobile crowdsensing scheme by predicting vehicle mobility[J]. IEEE Transactions on Intelligent Transportation Systems, 2020, 22(7): 4648-4659.

\bibitem{ICDE23} Ye Y, Liu C H, Dai Z, et al. Exploring both individuality and cooperation for air-ground spatial crowdsourcing by multi-agent deep reinforcement learning[C]//2023 IEEE 39th International Conference on Data Engineering (ICDE). IEEE, 2023: 205-217.

\bibitem{KDD17} Zhang L, Hu T, Min Y, et al. A taxi order dispatch model based on combinatorial optimization[C]//Proceedings of the 23rd ACM SIGKDD international conference on knowledge discovery and data mining. 2017: 2151-2159.

\bibitem{WWW19} Yuen C F, Singh A P, Goyal S, et al. Beyond shortest paths: Route recommendations for ride-sharing[C]//The World Wide Web Conference. 2019: 2258-2269.

\bibitem{Ubiquitous18} Xie X, Zhang F, Zhang D. PrivateHunt: Multi-source data-driven dispatching in for-hire vehicle systems[J]. Proceedings of the ACM on Interactive, Mobile, Wearable and Ubiquitous Technologies, 2018, 2(1): 1-26.

\bibitem{vehicle_dispatch_modle_based} Zhang R, Rossi F, Pavone M. Model predictive control of autonomous mobility-on-demand systems[C]//2016 IEEE international conference on robotics and automation (ICRA). IEEE, 2016: 1382-1389.

\bibitem{introduce_drl} Arulkumaran K, Deisenroth M P, Brundage M, et al. Deep reinforcement learning: A brief survey[J]. IEEE Signal Processing Magazine, 2017, 34(6): 26-38.

\bibitem{TITS22} Haydari A, Yılmaz Y. Deep reinforcement learning for intelligent transportation systems: A survey[J]. IEEE Transactions on Intelligent Transportation Systems, 2020, 23(1): 11-32.

\bibitem{INFO20} Liu C H, Dai Z, Yang H, et al. Multi-task-oriented vehicular crowdsensing: A deep learning approach[C]//IEEE INFOCOM 2020-IEEE Conference on Computer Communications. IEEE, 2020: 1123-1132.

\bibitem{INFO21VEHICLES} Ding R, Yang Z, Wei Y, et al. Multi-agent reinforcement learning for urban crowd sensing with for-hire vehicles[C]//IEEE INFOCOM 2021-IEEE Conference on Computer Communications. IEEE, 2021: 1-10.

\bibitem{INFO18} Oda T, Joe-Wong C. MOVI: A model-free approach to dynamic fleet management[C]//IEEE INFOCOM 2018-IEEE Conference on Computer Communications. IEEE, 2018: 2708-2716.

\bibitem{TITS20} Liu Z, Li J, Wu K. Context-aware taxi dispatching at city-scale using deep reinforcement learning[J]. IEEE Transactions on Intelligent Transportation Systems, 2020, 23(3): 1996-2009.

\bibitem{KDD18} Lin K, Zhao R, Xu Z, et al. Efficient large-scale fleet management via multi-agent deep reinforcement learning[C]//Proceedings of the 24th ACM SIGKDD international conference on Knowledge Discovery and Data Mining. 2018: 1774-1783.

\bibitem{KDD22} Sun J, Jin H, Yang Z, et al. Optimizing long-term efficiency and fairness in ride-hailing via joint order dispatching and driver repositioning[C]//Proceedings of the 28th ACM SIGKDD Conference on Knowledge Discovery and Data Mining. 2022: 3950-3960.

\bibitem{introduce_AoI1} Zhang S, Zhang H, Han Z, et al. Age of information in a cellular internet of UAVs: Sensing and communication trade-off design[J]. IEEE Transactions on Wireless Communications, 2020, 19(10): 6578-6592.

\bibitem{introduce_AoI2} Chen H, Gu Y, Liew S C. Age-of-information dependent random access for massive IoT networks[C]//IEEE INFOCOM 2020-IEEE Conference on Computer Communications Workshops (INFOCOM WKSHPS). IEEE, 2020: 930-935.

\bibitem{the_reason_why_data_freshness_is_important} Abd-Elmagid M A, Pappas N, Dhillon H S. On the role of age of information in the Internet of Things[J]. IEEE Communications Magazine, 2019, 57(12): 72-77.

\bibitem{a_example_for_the_reason_why_data_freshness_is_important} Cheng Y, Wang X, Zhou P, et al. Freshness-Aware Incentive Mechanism for Mobile Crowdsensing with Budget Constraint[J]. IEEE Transactions on Services Computing, 2023.

\bibitem{why_use_MARL}Zhang K, Yang Z, Başar T. Multi-agent reinforcement learning: A selective overview of theories and algorithms[J]. Handbook of reinforcement learning and control, 2021: 321-384.

\bibitem{R-GCN} Schlichtkrull M, Kipf T N, Bloem P, et al. Modeling relational data with graph convolutional networks[C]//The Semantic Web: 15th International Conference, ESWC 2018, Heraklion, Crete, Greece, June 3–7, 2018, Proceedings 15. Springer International Publishing, 2018: 593-607.

\bibitem{liyihong} Li Y, Zeng T, Zhang X, et al. TapFinger: Task Placement and Fine-Grained Resource Allocation for Edge Machine Learning[C]//IEEE INFOCOM. 2023.

\bibitem{neurips22} Yu C, Velu A, Vinitsky E, et al. The surprising effectiveness of ppo in cooperative multi-agent games[J]. Advances in Neural Information Processing Systems, 2022, 35: 24611-24624.

\bibitem{PPO} Schulman J, Wolski F, Dhariwal P, et al. Proximal policy optimization algorithms[J]. arXiv preprint arXiv:1707.06347, 2017.


\bibitem{newyork_taxi} https://www1.nyc.gov/site/tlc/about/tlc-trip-record-data.page.

\bibitem{assume_cost_one_time_slot_to_reach_dispatched_destination} Jin J, Zhou M, Zhang W, et al. Coride: joint order dispatching and fleet management for multi-scale ride-hailing platforms[C]//Proceedings of the 28th ACM international conference on information and knowledge management. 2019: 1983-1992.

\bibitem{UCB} Kuleshov V, Precup D. Algorithms for multi-armed bandit problems[J]. arXiv preprint arXiv:1402.6028, 2014.

\bibitem{espeholt2018impala} Espeholt L, Soyer H, Munos R, et al. Impala: Scalable distributed deep-rl with importance weighted actor-learner architectures[C]//International conference on machine learning. PMLR, 2018: 1407-1416.

\bibitem{li2019privacy} Li M, Zhu L, Lin X. Privacy-preserving traffic monitoring with false report filtering via fog-assisted vehicular crowdsensing[J]. IEEE Transactions on Services Computing, 2019, 14(6): 1902-1913.
% \bibitem{StreetView} Anguelov D, Dulong C, Filip D, et al. Google street view: Capturing the world at street level[J]. Computer, 2010, 43(6): 32-38.
\bibitem{zhang2021data}Zhang X, Wang J, Zhang H, et al. Data-driven transportation network company vehicle scheduling with users’ location differential privacy preservation[J]. IEEE Transactions on Mobile Computing, 2021, 22(2): 813-823.
\bibitem{liu2024multi}Liu L, Huang Z, Xu J. Multi-Agent Deep Reinforcement Learning Based Scheduling Approach for Mobile Charging in Internet of Electric Vehicles[J]. IEEE Transactions on Mobile Computing, 2024.
% \bibitem{INFO22} Xiang C, Li Y, Zhou Y, et al. A comparative approach to resurrecting the market of mod vehicular crowdsensing[C]//IEEE INFOCOM 2022-IEEE Conference on Computer Communications. IEEE, 2022: 1479-1488.

% \bibitem{ICDE23UAV} Wang Y, Wu J, Hua X, et al. Air-ground spatial crowdsourcing with uav carriers by geometric graph convolutional multi-agent deep reinforcement learning[C]//2023 IEEE 39th International Conference on Data Engineering (ICDE). IEEE, 2023: 1790-1802.


\end{thebibliography}
\vspace{12pt}
\color{red}


% To compile a comprehensive list of relevant articles, we employed a systematic methodology, reviewing, categorizing, and evaluating prior literature. 
% Our team of researchers identified various articles defining a range of paramount definitions, concepts, and theories used within this review. 
% These topics include but are not limited to: \MR, virtual deception, information processing, and human cognition. 
% The amassed data subsequently underwent a secondary analysis wherein we meticulously categorized and interrelated the found information, establishing the groundwork for deriving an ontology and mind map. 
%Thus, establishing the groundwork for the development of a comprehensive research framework.

% Thus establishing the foundation for the development of a comprehensive diagram or mind map.

% ...while also setting a precedent for the mind map model
 
%To culminate a list of pertinent references, a researcher employed a systematic and methodical strategy.

%\subsection{\MR Deception Attacks Ontology}
%\item 
\emph{\textbf{\MR Deception Attacks Ontology}}\addition{~(Section~\ref{sec:decattacks})}: After our literature review, two researchers iteratively outlined an encyclopedic map of identified deception attacks. The iterative process was enhanced using the mini-Delphi method \cite{dalkey1963experimental, pan1996mini} in which two subject matter experts reviewed and further refined the ontology across each iteration. The outcome was a mind map of deception attacks in \ac{MR} (Figure~\ref{fig:attacks-mindmap}).
Then, we characterized how technical attacks identified in our literature review fit within the newly developed ontology~(Table~\ref{tab:attacks}).
%the mind map to the existing Modeling Deception Attacks of Mixed-Reality Environments mind map.

%\subsection{Information-Theoretic Model of Deception Attacks in \MR}
%\item 
\emph{\textbf{Information Theory and Deception Attacks}}\addition{~(Section~\ref{sec:information-theoretic})}: We derive an information-theoretic model to describe how \MR deception attacks affect information communication. We employ Borden-Kopp's deception model \cite{kopp:2018} that uses Shannon's information theory \cite{shannon1948mathematical} to formulate how information is transmitted from a source (e.g., \MR application) to a user via a \MR headset. Additionally, we utilize Vitanyi's model of mimicry \cite{Vitanyi} to mathematically assess differences between source-generated messages and those created by an attacker. %, which helps with the identification and measurement of deception.


%\subsection{\MR Decision-Making Model}
 \emph{\textbf{Decision-Making and Deception Attacks}}\addition{~(Section~\ref{sec:decision-making})}: We used our \MR Deception Ontology and our review of deception psychology and cognition literature to develop an \MR Decision-Making Model. This model connects cognitive processes of perception, attention, memory, and decision-making to types of deception attacks. This process involved a mini-Delphi approach in which renditions of the model were iteratively revised using expert knowledge and prior literature. 
 %, highlighting the intents and effects of the defined deception attacks. %\acp{PMA}. 

 %\subsection{\MR Threat-Modeling Cognitive Framework}
 \emph{\textbf{\MR \acf{DAF}}}\addition{~(Section~\ref{sec:cog-framework})}: We utilize our two models to develop a framework for analyzing the effects of \MR deception attacks on information communication and human cognition. This framework allows for qualitatively evaluating the impact of MR deception attacks on cognitive processes associated with perception, attention, memory. 
 %Such evaluations are useful for assessing damage in threat modeling approaches (Appendix~\ref{app:using-daf}).
 %Table~\ref{tab:framework} provides a general account of how categories of MR deception attacks from our ontology affect information transmission and cognition. %This table is a crucial element of our framework, facilitating an in-depth understanding of the utilization of information-theoretic principles to detect and mitigate the effects of deception attacks in MR environments.

\finding{Our multidisciplinary methodology shows how to connect disparate knowledge into an ensemble framework. As computing becomes more ubiquitous, security challenges require broader perspectives and analysis, particularly in terms of human cognition. We are not aware of other work that connects literature and theories from cybersecurity, \MR, and cognitive sciences into a cohesive framework.}

 % This framework classifies attacks according to their operational mechanisms, which can be overt, covert, involving degradation, denial, corruption, or subversion, and the cognitive processes they aim to disrupt. The purpose of mapping these attacks to the aspects of human cognition they exploit is to offer a systematic understanding of the potential vulnerabilities that exist in MR environments. The methodology entails analyzing channel characteristics, such as bandwidth ($W$), signal ($S$), noise ($N$), and mimicry ($M$), in relation to MR deception. In addition, this study evaluates the impact of these attacks on the perception mechanisms, including active/passive ($A/P$) perception, selection ($Sel$), organization ($Org$), and interpretation ($Int$). It also examines the effects on attention and memory.

 

%\end{itemize}


%...using the Information-theoretic processing model 
