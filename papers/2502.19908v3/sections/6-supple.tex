
\clearpage
% \setcounter{page}{1}
\maketitlesupplementary
\appendix


\section{Training Procedure}


\cref{alg:training} outlines the training process for the CarPlanner framework.
% The procedure involves two primary steps: 1)~training the non-reactive transition model, and 2)~training the mode selector and the trajectory generator.
Notably, during the training of the trajectory generator, we have the flexibility to employ either RL or IL, but, in this work, we do not combine RL and IL simultaneously, opting instead to explore their distinct characteristics separately.
The definitions of the loss functions are given in the following.

\noindent\textbf{Loss of non-reactive transition model.}
The non-reactive transition model $\beta$ is trained to simulate agent trajectories based on the initial state $\boldsymbol{s}_0$. For each data sample $(\boldsymbol{s}_0, s^{1:N,\text{gt}}_{1:T}) \in \mathcal{D}$, the model predicts trajectories $s^{1:N}_{1:T} = \beta(\boldsymbol{s}_0)$, and the training objective minimizes the L1 loss:
\begin{equation}
L_{\text{tm}} = \frac{1}{T} \sum_{t=1}^T \sum_{n=1}^N \left\| s_t^n - s_t^{n,\text{gt}} \right\|_1.
\end{equation}

\noindent\textbf{Mode selector loss.}
This contains two parts: cross-entropy and side-task loss.
The cross-entropy loss is defined as:
\begin{equation}
    \text{CrossEntropyLoss}(\boldsymbol{\sigma}, c^*) = -\sum_{i=1}^{N_{\text{mode}}} \mathbb{I}(c_i = c^*) \log \sigma_i,
\end{equation}
where $\sigma_i$ is the assigned score for mode $c_i$, $N_{\text{mode}}$ is the number of candidate modes, and $\mathbb{I}$ is the indicator function.
The side-task loss is defined as:
\begin{equation}
    \text{SideTaskLoss}(\bar{s}^{0}_{1:T}, s^{0,\text{gt}}_{1:T}) = \frac{1}{T} \sum_{t=1}^T \left\| \bar{s}_t^0 - s_t^{0,\text{gt}} \right\|_1,
\end{equation}
where $\bar{s}_t^0$ is the output ego future trajectory.


\noindent\textbf{Generator loss with RL.}
The PPO~\cite{schulman2017proximal} loss consists of three parts: policy, value, and entropy loss.
The policy loss is defined as:
\begin{equation}
\begin{split}
    &\text{PolicyLoss}(a_{0:T-1}, d_{0:T-1, \text{new}}, d_{0:T-1}, A_{0:T-1}) \\
    = &- \frac{1}{T} \sum_{t=0}^{T-1} \min \left( r_t A_t, \text{clip}(r_t, 1-\epsilon, 1+\epsilon) A_t \right),
\end{split}
\end{equation}
\section{The general case: Proof of \texorpdfstring{\Cref{thm:main-decomp}}{Theorem 1.6}}\label{sec:algo}

First, we show that data structure of \Cref{l:max_min_query} can be used to compute distances witnessed by shortest paths that pass through a constant-size separator.

\begin{lemma}\label{l:single_adhesion}
Fix a constant $k \in \mathbb{N}$. There exists an algorithm which as the input receives an edge-weighted graph $G$ on $n$ vertices and $m$ edges together with a partition of its vertices into three sets $A, B, C$ such that $|B| \leq k$ and there are no edges between $A$ and $C$, and as the output computes $\max_{c \in C} \dist(a, c)$ for every $a \in A$. The running time is $\Oh(m \log n + n \log^{k - 1} n)$.
\end{lemma}

\begin{proof}
Let $B = \{b_1, \ldots, b_k\}$. For any $a \in A, c \in C$, we have $\dist(a, c) = \min_{i \in [k]} \dist(a, b_i) + \dist(c, b_i)$. First, we run Dijkstra's algorithm from every vertex in $B$ to find $\dist(v, b_i)$ for every $v \in V(G)$ and $i \in [k]$. Next, we use \Cref{l:max_min_query} to construct a data structure $\mathbb{D}$ for the point set $\{(\dist(c, b_1), \dots, \dist(c, b_k))\colon c\in C\}\subseteq \mathbb{R}^k$. Now, the value $\max_{c \in C} \dist(a, c)$ for any given $a$ is equal to the answer of $\mathbb{D}$ to the query with argument $(\dist(a, b_1), \dots, \dist(a, b_k))$.
\end{proof}

After computing the distances over a constant-size separator, we will use the following observation to simplify one of the sides of the separation.

\begin{lemma}\label{l:inserting_paths}
Let $G$ be a edge-weighted connected graph and let $A, B, C$ be a partition of its vertices such that there are no edges between $A$ and $C$. For every pair of vertices $u, v \in B$, let $P_{u, v}$ be any shortest path from $u$ to $v$ with all internal vertices in $C$ (assuming such a path exists).

Let $G'$ denote a graph obtained from $G[A \cup B]$ by adding an edge from $u$ to $v$ of weight equal to the length of $P_{u, v}$, for all $u, v \in B$ for which $P_{u, v}$ exists. Then,  $$\dist_G(s, t) = \dist_{G'}(s, t)\qquad\textrm{for all }s,t\in A\cup B.$$
\end{lemma}
\begin{proof}
Let $G''$ be the graph obtained by adding new edges of $G'$ to $G$.
Fix any $s, t \in A \cup B$ and let $P$ denote the shortest path from $s$ to $t$ in $G''$ which minimizes the number of vertices from $C$ visited. Naturally, the weight of $P$ is equal $\dist_G(s, t)$. Assume that such path visits at least one vertex of $C$. Then, the path $P$ is of the form $s \xrightarrow{P_1} x \xrightarrow{P_2} y \xrightarrow{P_3} t$, where $x, y \in B$ and all the internal vertices of $P_2$ are in $C$. By the construction of $G'$, $P_2$ can be replaced with a direct edge from $x$ to $y$ of the same weight. We obtain a same weight path with a smaller number of vertices of $C$ visited, which is a contradiction. Therefore, $P$ is entirely contained in $A \cup B$, hence it exists in $G'$. This shows that $\dist_G(s, t) = \dist_{G'}(s, t)$.
\end{proof}


The next lemma encapsulates the main algorithmic content of the proof of \Cref{thm:main-decomp}. The algorithm will split the tree decomposition provided on input into smaller parts for which the eccentricities are easier to calculate. We use the following lemma to handle a single such part.
\begin{lemma}\label{l:star}
Fix constants $k, g \in \mathbb{N}, 0 < \delta < \frac{1}{54}$. Assume we are given $n \in \mathbb{N}$, an edge-weighted graph $G$ on at most $n$ vertices with a weight function $w \colon E(G) \to \mathbb{N}$, a vertex subset $A$ and a collection of non-empty vertex subsets $V_0, V_1, \dots, V_\ell$ satisfying the following conditions:
\begin{itemize}[nosep]
	\item The sum of weights of all the edges in $G$ is bounded by $\Oh(n)$.
	\item $V(G) \setminus A = V_0 \cup V_1 \cup \dots \cup V_\ell$.
	\item $|A| \leq k$.
	\item For every $i \in [\ell]$, $G[V_i \setminus V_0]$ is connected, $N_G(V_i \setminus V_0) = V_i \cap V_0$, $|V_i| = \Oh(n^\delta)$, and $|V_0 \cap V_i| \leq 4$.
	\item For all $i, j \in [\ell], i \neq j$, $V_i \setminus V_0$ and $V_j \setminus V_0$ are disjoint and non-adjacent in $G$.
	\item Every edge $uv \in E(G)$ with $u, v \not\in A$ is contained in $G[V_i]$ for some $i\in \{0,1,\ldots,\ell\}$.
	\item The graph obtained by taking $G[V_0]$ and adding a clique on $V_0 \cap V_i$ for every $i \in [\ell]$ has Euler genus bounded by $g$.
\end{itemize}
Then, we can compute the eccentricity of every vertex of $G$ in time $\Oh \left( n^{1 + \frac{150 + 54 \delta}{151}} \log^k n \right)$.
\end{lemma}

\begin{proof}
Fix $\delta' = \frac{1 + 97 \delta}{151}$; we have $\delta' - \delta = \frac{1 - 54\delta}{151} > 0$.
Let $E_i$ denote the set of edges with one endpoint in $V_i$ and the other endpoint in $V_i \setminus V_0$. For $i \in [\ell]$, we shall say that $V_i$ is {\em{heavy}} if the sum of weights of $E_i$ is larger than $n^{\delta'}$. Since the sets $E_i$ are pairwise disjoint and the total sum of weights of all the edges is bounded by $\Oh(n)$, the number of heavy subsets is bounded by $\Oh(n^{1 - \delta'})$. Without loss of generality, we may assume that $V_{\ell' + 1}, \dots, V_\ell$ are heavy and $V_1, \dots, V_{\ell'}$ are not, for some $\ell'\in \{0,\ldots,\ell\}$.


For any source vertex $s$, we can calculate distances from $s$ to every vertex of $G$  using breadth first search in time $\Oh(\sum_{e \in E(G)} w(e)) = \Oh(n)$.
In particular, for every $\ell' < i \leq \ell$, we can compute the distances from every vertex of $V_i$ to every vertex of $G$ in total time $\Oh(n^{2 - \delta' + \delta})$, because $$|V_{\ell'+1}\cup \ldots\cup V_{\ell}|\leq n^{1-\delta'}\cdot \Oh(n^\delta)=\Oh(n^{1-\delta'+
\delta}).$$
Additionally, we calculate distances $\dist_G(a, v)$ for every $a \in A, v \in V(G)$ in time $O(n)$.

For every $i \in [\ell]$ and $u,v \in V_0 \cap V_i$, there exists a shortest path $P_{i,u,v}$ from $u$ to $v$ with all internal vertices belonging to $V_i - V_0$ due to the assumption that $G[V_i - V_0]$ is connected and $N_G(V_i - V_0) = V_i \cap V_0$. Therefore, the distance from $u$ to $v$ is bounded by the sum of weights of edges in $E_i$. In particular, for $i \in [\ell']$, $\dist_G(u, v) \leq n^{\delta'}$.

We define $\widetilde{G}$ to be the graph obtained by taking $G[A \cup V_0 \cup \dots \cup V_{\ell'}]$ and applying the following operation for every $i \in \{\ell' + 1, \dots, \ell\}$:
for each pair of vertices $u, v \in A \cup (V_0 \cap V_i)$, add an edge in $\widetilde{G}$ between $u$ and $v$ with weight equal to the total weight of $P_{i,u,v}$. For a fixed $i, u$, we can find $P_{i, u, v}$ for all $v$ using breadth first search in time $\Oh(n)$. Taking a sum over all $i, u$, we get that $\tilde{G}$ can be computed in total time $\Oh(n^{2 - \delta'})$.


\begin{claim}\label{cl:wG}
The sum of the edge weights in $\widetilde{G}$ is $\Oh(n)$. Moreover, for all $u, v \in V(\widetilde{G})$, we have $\dist_{\widetilde{G}}(u, v) = \dist_{G}(u, v)$.
\end{claim}

\begin{proof}
Consider $i \in \{\ell' + 1, \dots, \ell\}$ and any $u, v \in A \cup (V_0 \cap V_i)$ for which we added an edge. Its weight is bounded by the sum of weights of edges in $E_i$. Therefore, the total weight of all edges added is at most
$$
\sum_{i \in \{\ell' + 1, \dots, \ell\}} \left( |A \cup (V_0 \cap V_i)|^2 \sum_{e \in E_i} w(e) \right) \leq (4 + k)^2 \sum_{e \in E(G)} w(e) = \Oh(n).
$$
This proves the first part of the claim.

For the second part of the claim, consider any $i \in \{\ell' + 1, \dots, \ell \}$ and observe that by our assumptions, $A \cup (V_0 \cap V_i)$ separates $(V_0 \cup \dots \cup V_{\ell'} \cup V_{i + 1} \cup \dots \cup V_\ell) \setminus V_i$ from $V_i \setminus V_0$. Hence it suffices to repeatedly apply \Cref{l:inserting_paths}.
\end{proof}

For every $u \in V(\widetilde{G})$, we have $\ecc_G(u) = \max(\ecc_{\widetilde{G}}(v), \max_{v \in V(G) \setminus V(\widetilde{G})} \dist_G(u, v))$. Note, that we already know all the distances $\dist_G(u, v)$ for $v \in V(G) \setminus V(\widetilde{G})$. Similarly, we can already compute $\ecc_G(u)$ for every $u \in V(G) \setminus V(\widetilde{G})$. Therefore, it remains to compute $\ecc_{\widetilde{G}}(v)$ for each $v \in V(\widetilde{G})$. Our goal is to show that this can be done efficiently using \Cref{l:main_ecc}.

Now, let $G'$ be the graph obtained from $\tilde{G}$ by replacing every edge $e$ non-indicent to $A$ with $w(e)\geq 2$ with a path of length $w(e)$ consisting of unit-weight edges. This operation again preserves the distances. Since the sum of edge weights in $\tilde{G}$ is of $\Oh(n)$, the total number of vertices in $G'$ is of $\Oh(n)$. For $0 \leq i \leq \ell'$, we write $V'_i$ to denote the set $V_i$ together with all the vertices added as a part of a path between two endpoints in $V_i$.
As $V_i$ is not heavy for each $i\in [\ell']$, we have
$$
|V'_i \setminus V'_0| \leq |V_i| + \sum_{e \in E_i} w(e) = \Oh(n^{\delta'})\qquad \textrm{for all }i\in [\ell'].
$$

Let $G_0$ denote the graph $G'[V'_0]$ and let $G_0^*$ denote the graph $G'- A$ with $V'_i - V'_0$ contracted to a single vertex $v_i^*$, for each $i \in [\ell']$; note that, all edges of $G_0$ and $G_0^*$ have unit weight.

\begin{claim}
	The graph $G_0^*$ is does not contain $K_{t}$ as a minor, where $t = \Oh(\sqrt{g})$.
\end{claim}

\begin{proof}
Let $\bar{G}_0$ denote the graph obtained by taking $G_0$ and adding a clique on $V_0 \cap V_i$ for every $i \in [\ell']$.
By lemma assumptions and the fact that subdividing edges does not increase the Euler genus, $\bar{G}_0$ has Euler genus at most $g$. In particular, $\bar{G}_0$ is $K_{t'}$-minor-free for some $t' = \Oh(\sqrt{g})$, because the Euler genus of $K_{t'}$ is $\Omega({t'}^2)$.

Similarly, let $\bar{G}_0^*$ be the graph obtained by taking $G_0^*$ and adding a clique on each $V_0 \cap V_i$.
Note, that $\bar{G}_0^* - \{v_1^*, \dots, v_{\ell'}^*\}$ is precisely $\bar{G}_0$. Let $t = \max(t', 6)$.
Recall that a minor model of a clique $K_t$ consists of $t$ pairwise vertex-disjoint connected subgraphs, called
branch sets, such that there is at least one edge between each pair of the branch sets.
Consider a minor model $\varphi$ of $K_{t}$ in $\bar{G}^*_0$.
Note that $\varphi$ cannot contain any singleton branch set of the form $\{v^*_i\}$, for the degree of $v^*_i$ in $\bar{G}^*_0$ is at most $4 < t - 1$. Furthermore, since $N_{\bar{G}^*_0}(v^*_i) = V_0 \cap V_i$, any branch set containing $v^*_i$ and at least one other vertex contains some $u \in V_0 \cap V_i$, and $N_{\bar{G}^*_0}(v^*_i)\subseteq N_{\bar{G}^*_0}(u)$, hence removing $v^*_i$ from this branch set preserves the model. Therefore, we can assume without loss of generality that all branch sets of $\varphi$ are disjoint from $\{v^*_1, \dots, v^*_{\ell'}\}$, hence $\varphi$ is a minor model of $K_{t}$ in $\bar{G}_0$. This is a contradiction, as $t \geq t'$ and $\bar{G}_0$ is $K_{t'}$-minor-free. Therefore, $\bar{G}_0^*$ is $K_t$-minor-free, hence $G_0^*$ also.
\end{proof}

Let $\rho' = \frac{2 - 108 \delta}{151} > 0$. The graph $G^*_0$ is a unit-weight graph and is $K_{t}$-minor-free.
Hence, by applying \Cref{t:r_division} to $G^*_0$ (with $\varepsilon = \rho'/2$)
we obtain an $n^{\rho'}$-division $\mathcal{R}_0$ in time $\Oh(n^{1 + \rho'})$.
We extend it to $G' - A$ by mapping every contracted vertex $v^*_i$ to $N_{G' - A}[V'_i - V'_0] = (V'_i - V'_0) \cup (V_0 \cap V_i)$. Formally, we put $V''_i \coloneqq N_{G' - A}[V'_i - V'_0]$ and 
$$
\mathcal{R} \coloneqq \left\{ (R_0 \cap V'_0) \cup \bigcup_{i \colon v^*_i \in R_0} V''_i \colon R_0 \in \mathcal{R}_0 \right\}.
$$

Now, we argue that $\mathcal{R}$ is a reasonable division of $G' - A$. Clearly, all sets $R \in \mathcal{R}$ are connected in $G' - A$. Pick any $R \in \mathcal{R}$ and let $R_0$ be its corresponding set in $\mathcal{R}_0$.
Every vertex $v^*_i$ is mapped to a set of size $\Oh(n^{\delta'})$, therefore
$$|R| \leq |R_0| \cdot \Oh(n^{\delta'}) = \Oh(n^{\rho' + \delta'}).$$

By our construction, for every $i \in [\ell']$, $R$ is either disjoint from $V'_i - V'_0$ or contains whole $N_{G' - A}[V'_i - V'_0]$. This means that no vertex belonging to any $V'_i - V'_0$ can be in $\partial R$, hence $\partial R \subseteq V'_0$.

Pick any $u \in \partial R \cap R_0$. Assume that $u \not\in \partial R_0$. Then every vertex of $N_{G_0^*}(u)$ must be in $R_0$, hence $N_{G - A'}(u) \subseteq R$, which is a contradiction. This means that $\partial R \cap R_0 \subseteq \partial R_0$.

Pick any $u \in \partial R - R_0$. Then, $u \in V_0 \cap V_i$ for some $i \in [\ell']$ such that $v_i^* \in R_0$. Moreover, $v_i^* \in \partial R_0$ and is adjacent to $u$ in $G_0^*$. The number of such $u$ is bounded by $4 |\partial R_0 \cap \{ v_1^*, \dots, v_{\ell'}^* \}|$.

Putting two cases together, we obtain:
$$
\sum_{R \in \mathcal{R}} |\partial R| = \sum_{R \in \mathcal{R}} \left(|\partial R \cap R_0| + |\partial R - R_0|\right) \leq \sum_{R_0 \in \mathcal{R}_0} \left(|\partial R_0| + 4 |\partial R_0 \cap \{ v_1^*, \dots, v_{\ell'}^* \}|\right) = \Oh(n^{1 - \frac{1}{2}\rho'}).
$$

It remains to show the following claim.

\begin{claim}
Pick any $R \in \mathcal{R}, s_R \in R$. The number of different distance profiles on $R$ relative to $s_R$ in $G' - A$ is of $\Oh(n^{48\rho' + 54\delta'})$.
\end{claim}
\begin{proof}
We look at every vertex $v \in V(G') \setminus A$ and consider three cases: $v \in R$, $v \in V'_0$, and $v \in V'_i \setminus (V'_0 \cup R)$ for some $i \in [\ell']$. By our construction, $R \cap V'_0$ is non-empty, hence w.l.o.g. we can assume that $s_R \in V'_0$ as whether two vertices have the same profile on $R$ is independent of the choice of the pivot vertex.

In the first case, there are at most $|R| = \Oh(n^{\rho' + \delta'})$ such vertices, hence they realise at most that many profiles.

In the second case, we want to observe that profile of any vertex $v \in V'_0$ on $R$ depends only on its profile on $R \cap V'_0$ (relative to $s_R$). Pick any $t \in R - V'_0$. Then $t \in V'_i - V'_0$ for some $i \in [\ell']$, $V_i \cap V_0 \subseteq R \cap V'_0$, and every path from $v$ to $t$ intersects $V_i \cap V_0$. In particular, distances from $v$ to vertices of $V_i \cap V_0$ determine its distance to $t$, which proves the observation.

Let $\tilde{G}_0$ denote the graph obtained by taking $G'[V'_0]$ and for every $i \in [\ell'], u, v \in V_0 \cap V_i$ adding a disjoint path from $u$ to $v$ of length $\dist(u, v)$. Let $P_i$ denote the vertex set of paths added between $V_0 \cap V_i$. For every $t \in V'_0$ we have $\dist_{G' - A}(v, t) = \dist_{\tilde{G}_0}(v, t)$, so it suffices to bound the number of profiles on $R \cap V'_0$ in $\tilde{G}_0$. By our assumptions, $\tilde{G}_0$ has Euler genus bounded by $g$ and all $P_i$ are of size $\Oh(n^{\delta'})$.

Let $R_0$ be the set of $\mathcal{R}_0$ corresponding to $R$. Let $\tilde{R}_0$ denote the set $(R \cap V'_0) \cup \bigcup_{i : v^*_i \in R_0} P_i$. Such set is connected in $\tilde{G}_0$. Moreover, similarly to $R$, its size is $\Oh(n^{\rho' + \delta'})$. Applying \Cref{thm:distprofiles}, we get that the number of distance profiles on $\tilde{R}_0$ in $\tilde{G}_0$ is $\Oh(n^{12(\rho' + \delta')})$, which also bounds the number of profiles on $R$ in $G' - A$ realised by $V'_0$.

For the third case, assume $v \in V'_i \setminus (V'_0 \cup R)$ for some $i\in [\ell']$. Every path from $v$ to any vertex of $R$ in $G' - A$ intersects $V_i \cap V_0$. Let $v_1, \dots v_p$ be the vertices of $V_i \cap V_0$, where $p \leq 4$. The profile of $v$ on $R$ is then determined by the following:
\begin{itemize}[nosep]
 \item[(a)] the profile of each $v_j$ on $R$,
 \item[(b)] $\dist_{G' - A}(v, v_j) - \dist_{G' - A}(v, v_1)$ for each $2 \leq j \leq p$, and
 \item[(c)] $\dist_{G' - A}(s_R, v_j) - \dist_{G' - A}(s_R, v_1)$ for each $2 \leq j \leq p$ where $s_R$ is some pivot vertex of $R$.
\end{itemize}
By the previous case, the number of distance profiles of each $v_j$ is $\Oh(n^{12(\rho' + \delta')})$. The distances between $v$ and $v_j$ are bounded by $|V'_i|$, hence each quantity described in (b) can take $\Oh(n^{\delta'})$ different possible values. Similarly, since $v_1$ and $v_j$ are connected via $V'_i$, $|\dist_{G' - A}(s_R, v_j) - \dist_{G' - A}(s_R, v_1)| \leq \Oh(n^{\delta'})$. The number of different possible profiles of such $v$ is therefore bounded by $\Oh(n^{48(\rho' + \delta') + 6\delta'}) = \Oh(n^{48\rho' + 54\delta'})$. This finishes the proof of the claim.
\end{proof}

Now we can apply \Cref{l:main_ecc} to graph $G'$ with apex set $A$, $X = V(\widetilde{G})$, and the following constants: $$\rho = \rho' + \delta',\qquad \gamma = 1 - \frac{1}{2}\rho',\quad \textrm{and}\quad \alpha = 48\rho' + 54 \delta'.$$ This allows us to calculate all $V(\widetilde{G})$-eccentricities in $G'$ in time
$$
\Oh \left( \left(
	n^{ 2 - \frac{1}{2} \rho' } +
	n^{ 1 + 48\rho' + 54 \delta' }
\right) \log^k n \right) =
\Oh \left( n^{1 + \frac{150 + 54 \delta}{151}} \log^k n \right).
$$
Since for each $v\in V(\widetilde{G})$ we have $\ecc_{\widetilde{G}}(v) = \max_{u \in V(\widetilde{G})} \dist_{\widetilde{G}}(v, u) = \max_{u \in V(\widetilde{G})} \dist_{G'}(v, u)$, this means that we have successfully computed all the eccentricities in $\widetilde{G}$; and as we argued, this is enough to compute all the eccentricities in $G$ as well.

Finally, the total running time of the algorithm is
$$
\Oh \left( n^{1 + \frac{150 + 54 \delta}{151}} \log^k n + n^{2 - \delta' + \delta} \right) =
\Oh \left( n^{1 + \frac{150 + 54 \delta}{151}} \log^k n \right).
$$\qedhere
\end{proof}


\begin{lemma}\label{l:star2}
Fix constants $k, g \in \mathbb{N}, 0 < \delta < \frac{1}{54}$. Assume we are given $n \in \mathbb{N}$, an edge-weighted graph $G$ on at most $n$ vertices with a weight function $w \colon E(G) \to \mathbb{N}$, a vertex subset $A$ and a collection of non-empty vertex subsets $V_0, V_1, \dots, V_\ell$ satisfying the same conditions as in \Cref{l:star} with the following differences:
\begin{itemize}
	\item we don't require $G[V_i - V_0]$ to be connected and $V_i - V_0$ to be adjacent to whole $V_i \cap V_0$;
	\item instead of $|V_0 \cap V_i| \leq 4$, we require $|V_0 \cap V_i| \leq k$.
\end{itemize}
Then, we can compute the eccentricity of every vertex of $G$ in time $\Oh \left( n^{1 + \frac{150 + 54 \delta}{151}} \log^{k + 5g} n \right)$.
\end{lemma}

\begin{proof}
We will reduce our input to one which will satisfy the conditions of \Cref{l:star}. We start by addressing the adhesions $V_0 \cap V_i$ containing too many vertices.

Let $G_0$ denote the graph $G[V_0]$ with cliques placed at $V_0 \cap V_i$ for every $i \in [\ell]$.
For every $i \in [\ell]$ we repeat the following procedure: while $|V_0 \cap V_i| > 4$,
remove arbitrary $5$ vertices from $V_0 \cap V_i$. Since $|V_0 \cap V_i| \leq k$ for each $i\in [\ell]$,
this procedure can be implemented in total time $\Oh(n)$. As a result, at the end we have $|V_0 \cap V_i| \leq 4$ for all $i \in [\ell]$. Let $M$ be the set of all the removed vertices. By our assumptions, $G_0$ has Euler genus bounded by $g$, hence it cannot contain $g + 1$ pairwise disjoint copies of $K_5$
(as the Euler genus of a graph is the sum of the Euler genera of its 2-connected components~\cite{StahlB77} and $K_5$ is not planar). Each removed quintiple of vertices induces a $K_5$ in $G_0$, hence we have $|M| \leq 5g$. We set $A' = A \cup M$ and may thus assume that $V_i$ is disjoint from $A'$ for all $0 \leq i \leq \ell$.

Now, fix $i \in [\ell]$. Let $C^i_1, \dots, C^i_{r_i}$ denote the connected components of $V_i - V_0$ in $G - A'$. We define $W^i_j := N_{G - A'}[C^i_j]$ for every $j \in [r_i]$. Clearly, all $W^i_j$ induce a connected subgraph of $G$ and satisfy $N_{G - A'}(W^i_j - V_0) = W^i_j \cap V_0$. We put $V'_0 := V_0$ and enumerate
$$
\{V'_1, V'_2, \dots V'_{\ell'}\} := \{ W^i_j \colon i \in [\ell], j \in [r_i] \}.
$$
It is easy to verify that the sets $A'$ and $V'_0, V'_1, \dots, V'_{\ell'}$ satisfy the conditions of \Cref{l:star}. We apply said lemma to calculate the eccentricity of every vertex of $G$ in the desired time.
\end{proof}



The next statement is a reformulation of \Cref{thm:main-decomp}.

\begin{theorem}
Fix constants $k, g \in \mathbb{N}$. Assume we are given a graph $G$ on $n$ vertices together with its tree decomposition $(T, \beta)$ and a set of private apices $A_t \subseteq \beta(t)$ for each node $t\in V(T)$ such that the following conditions hold:
\begin{itemize}[nosep]
 \item For every node $t \in V(T)$, we have $|A_t| \leq k$.
 \item For every edge $st \in E(T)$,  we have $|\beta(v) \cap \beta(u)|\leq k$.
 \item For every node $t \in V(T)$, graph obtained by taking $G[\beta(t)] - A_t$ and turning  $(\beta(t) \cap \beta(s))\setminus A_t$ into a clique for every edge $st \in E(T)$ has Euler genus bounded by $g$.
\end{itemize}
Then, we can compute the eccentricity of every vertex of $G$ in time $\Oh \left( n^{1 + \frac{355}{356}} \log^{k + 5g} n \right)$.
\end{theorem}

\begin{proof}
We may assume that $|V(T)|\leq n$, for every tree decomposition with no two bags comparable by inclusion has this property; and adjacent comparable bags can be merged by contracting the edge between them.

For a node $t\in V(T)$, by the {\em{weight}} of $t$ we mean the size of the corresponding bag, that is, $|\beta(t)|$. For any subset of nodes $S \subseteq V(T)$, we define $\beta(S) \coloneqq \bigcup_{t \in S} \beta(t)$ By the {\em{weight}} of $S$, we mean the total weight of the elements of $S$, that is, $\sum_{t\in S} |\beta(t)|$. 

\begin{claim}\label{cl:weight-T}
The weight of $V(T)$ is of $\Oh(n)$.
\end{claim}

\begin{proof}
The sets $\beta'(t) := \beta(t) - \bigcup_{s \in N_T(t)} \beta(s)$ are pairwise disjoint. We have
$$
\sum_{t \in V(T)} |\beta(t)| =
\sum_{t \in V(T)} |\beta'(t)| + 2 \cdot \sum_{st \in E(T)} |\beta(s) \cap \beta(t)| \leq
|V(T)| + 2k|E(T)| = \Oh(n).
$$
\end{proof}

Since every bag induces a graph of bounded Euler genus, the number of edges contained in a bag is linear in its size. In particular, this implies that the total number of edges of $G$ is also bounded by $\Oh(n)$.

We set $$\delta \coloneqq \frac{1}{356}\qquad\textrm{and}\qquad \Delta \coloneqq \frac{355}{356}.$$ Root the tree $T$ in an arbitrarily chosen node; this naturally imposes an ancestor-descendant relation in $T$ (for convenience, every node is considered its own ancestor and descendant).

We start by partitioning $T$ into connected subtrees using the following procedure.
We proceed bottom-up over $T$, processing nodes in any order so that a node is processed after all its strict descendants have been processed. Along the way, we mark some nodes and split the edges of $T$ into heavy and light. Let $t \in V(T)$ be the currently processed non-root node of $T$ and let $e \in E(T)$ be the edge connecting $t$ with its parent. If the total weight of all the unmarked nodes that are descendants of $t$ is at least $n^\delta$ (recall that this includes $t$ itself as well), then we declare $e$ heavy and mark all the descendants of $t$ that were unmarked so far. Otherwise, the edge $e$ is declared light and the procedure proceeds to further nodes of $T$.

Observe that
removing all heavy edges splits $T$ into connected subtrees, say $T'_1, \cdots T'_m$. All of the subtrees, except for possibly the subtree containing the root node, are of weight at least $n^\delta$. In particular, the number of subtrees $m$, and therefore the number of heavy edges, is  bounded by $\Oh(n^{1 - \delta})$. Moreover, in every subtree $T'_i$, removing the node closest to the root splits $T'_i$ into smaller components, each of weight less than $n^\delta$.

Fix a heavy edge $e$ and let $T^e_1$ and $T^e_2$ be the two subtrees into which $T$ splits after removing~$e$. Let $X^e_i = \beta(T^e_i)$ for $i \in \{1, 2\}$. Put $A_e = X^e_1 \setminus X^e_2$, $C_e = X^e_2 \setminus X^e_1$, and $B_e = X^e_1 \cap X^e_2$. By the properties of tree decompositions, such choice of $A_e, B_e, C_e$ satisfies the conditions of \Cref{l:single_adhesion}, hence in time $\Oh(n \log^{k - 1} n)$ we can compute $\max_{v \in X^e_2} \dist_G(u,v)$ for every $u \in X^e_1$, and $\max_{u \in X^e_1} \dist_G(u,v)$ for every $v \in X^e_2$. Computing this for every heavy edge $e$ takes total time $\Oh(n^{2 - \delta} \log^{k - 1} n)$.

Fix any subtree $T'=T'_j$. Let $e_1 = t^{e_1}_1t^{e_1}_2, e_2 = t^{e_2}_1 t^{e_2}_2, \dots, e_\ell = t^{e_\ell}_1 t^{e_\ell}_2$ denote the heavy edges incident to $T'$, where $t^{e_i}_1 \in V(T')$ and $V(T') \subseteq V(T_1^{e_i})$ for every $i \in [\ell]$.
For a vertex $v \in \beta(T')$, let
$$d_0(v) = \max_{u \in \beta(T')} \dist_G(v, u)\qquad\textrm{and}\qquad d_i(v) = \max_{u \in X_2^{e_i}}\dist_G(v,u),\quad\textrm{for } i \in [\ell].$$ We have $\ecc(v) = \max \{ d_i(v)\colon i\in \{0,1,\ldots,\ell\}\}$.The values of $d_i(v)$ are already calculated for all $i\in [\ell]$, hence it remains to compute $d_0(v)$.

For every $i \in [\ell]$ and every pair of vertices $u, v \in \beta(t^{e_i}_1) \cap \beta(t^{e_i}_2)$ we find a shortest path between $u$ and $v$ with all internal vertices inside $X^{e_i}_2$ (or determine that it doesn't exist). For a fixed $u, v$ this can be done in time $\Oh(n)$. Since in total we perform this step at most $2k^2$ times per heavy edge, it takes $\Oh(n^{2 - \delta})$ time in total. Let $P_{i, u, v}$ denote such path, assuming it exists.

Let $G'$ denote the graph obtained from $G[\beta(T')]$ by taking every $i, u, v$ for which $P_{i, u, v}$ exists and adding an edge between $u$ and $v$ of weight equal to the total weight of $P_{i, u, v}$.
The weight of every edge inserted in $\beta(t^{e_i}_1) \cap \beta(t^{e_i}_2)$ is bounded by $|X^{e_i}_2|+1$. The total weight of all edges inserted is therefore at most
$$
\sum_{i \in [\ell]} |\beta(t^{e_i}_1) \cap \beta(t^{e_i}_2)|^2 \cdot (|X^{e_i}_2|+1) \leq
k^2 \sum_{i \in [\ell]} (|X^{e_i}_2|+1) = \Oh(n),
$$
where the last equality follows from the fact that all the trees $T^{e_i}_2$ are pairwise disjoint.
By \Cref{l:inserting_paths}, we have $\dist_{G'}(u, v) = \dist_G(u, v)$ for each $u, v \in \beta(T')$. Hence, computing $d_0(v)$ for every $v \in \beta(T')$ is equivalent to computing the eccentricity of every vertex in $G'$.

If the size of $\beta(T')$ is smaller than $n^\Delta$, we compute the eccentricities naively in time $\Oh(|\beta(T')|^2)$, 
noting that $G'$ has $\Oh(|\beta(T')|)$ edges (thanks to Claim~\ref{cl:weight-T} and bounded genus assumption 
of the last bullet of the theorem statement). Otherwise, we argue that we can use the algorithm in \Cref{l:star} as follows.

Let $t$ be the node of $T'$ closest to the root. Let $s_1, \dots, s_p$ be the children of $t$ in $T$ and let $T''_i$ denote the connected component of $T' - \{t\}$ containing $s_i$. Set $V_0 = \beta(t)$ and $V_i = \beta(T''_i)$ for $i \in [p]$.

It is now easy to verify that $G'$ and sets $A, \{V_i\colon 0\leq i\leq p\}$ selected this way satisfy the assumptions of \Cref{l:star2}. This allows us to use it to compute the eccentricities in $G'$ in time
$$
\Oh \left( n^{1 + \frac{150 + 54\delta}{151}} \log^{k + 5g} n \right) =
\Oh \left( n^{1 + \frac{354}{356}} \log^{k + 5g} n \right).
$$
As we argued, from these eccentricities, we may easily compute all the eccentricities in $G$.

Now, let us analyse the total running time of the whole algorithm. We invoke \Cref{l:star} $\Oh(n^{1 - \Delta})$ times, since we apply it only to subtrees $T'_i$ of size at least $n^\Delta$. The total running time of those applications is hence
$$
\Oh \left( n^{2 + \frac{354}{356} - \Delta} \log^{k + 5g} n \right) =
\Oh \left( n^{1 + \frac{355}{356}} \log^{k + 5g} n \right).
$$
We compute the eccentricities naively for subtrees smaller than $n^\Delta$, hence the total running time of this computation is
$$
\sum_{i \in [m] \colon |\beta(T'_i)| \leq n^\Delta} |\beta(T'_i)|^2 \leq
n^\Delta \cdot \sum_{i \in m} |\beta(T'_i)| = \Oh(n^{1 + \Delta})=\Oh\left(n^{1+\frac{355}{356}}\right).
$$
The rest of computation can be done in $\Oh(n^{2 - \delta} \log^k n)$. Therefore, the whole algorithm runs in time $\Oh \left( n^{1 + \frac{355}{356}} \log^{k + 5g} n \right)$.
\end{proof}

where the ratio $r_t$ is given by $r_t = \frac{\text{Prob}(a_t, d_{t,\text{new}})}{\text{Prob}(a_t, d_{t})}$, $d_{t,\text{new}}$ and $d_t$ are the policy distributions (mean and standard deviation of Gaussian distribution) at time step $t$ induced by $\pi$ and $\pi_{\text{old}}$ respectively, the function $\text{Prob}(a, d)$ calculates the probability of a given action $a$ under a distribution $d$, and $A_t$ is the advantage estimated using GAE~\cite{schulman2015high}.
The value and entropy loss are defined as:
\begin{equation}
    \text{ValueLoss}(V_{0:T-1, \text{new}}, \hat{R}_{0:T-1}) = \frac{1}{T} \sum_{t=0}^{T-1} \left\| V_{t,\text{new}} - \hat{R}_t \right\|_2^2,
\end{equation}
\begin{equation}
    \text{Entropy}(d_{0:T-1, \text{new}}) = \frac{1}{T} \sum_{t=0}^{T-1} \mathcal{H}(d_{t,\text{new}}),
\end{equation}
\noindent where $V_{t,\text{new}}$ and $\hat{R}_t$ are the predicted and actual returns, and $\mathcal{H}$ represents the entropy of the policy distribution $d$. 


\noindent\textbf{Generator loss with IL.}
In IL, the generator minimizes the trajectory error between the ego-planned trajectory $s^0_{1:T}$ and the ground-truth trajectory $s^{0,\text{gt}}_{1:T}$. The loss is defined as:
\begin{equation}
    L_{\text{generator}} = \frac{1}{T} \sum_{t=1}^T \left\| s_t^0 - s_t^{0,\text{gt}} \right\|_1.
\end{equation}





\section{Implementation Details}


The hyperparameters of model architecture, PPO-related parameters, and loss weights are summarized in \cref{table:param}. The magnitudes of value, policy, and entropy loss are $10^3$, $10^0$, and $10^{-3}$, respectively. The trajectory generator generates trajectories with a time horizon of 8 seconds at 1-second intervals, corresponding to time horizon $T = 8$. During testing, these trajectories are interpolated to 0.1-second intervals.
The weight of scores generated by the rule and mode selectors is set to a ratio of $1:0.3$.
In cases where no ego candidate trajectory satisfies the safety criteria evaluated by the rule selector, an emergency stop is triggered.
For the Test14-Random benchmark, a replanning frequency of 10Hz is employed, adhering to the official nuPlan simulation configuration. In contrast, for the Reduced-Val14 benchmark, a replanning frequency of 1Hz is used to ensure a fair comparison with Gen-Drive~\cite{huang2024gen}.



\begin{table}[t]
\centering
% \captionsetup{font=large}
\centering
    \begin{tabular}{l c}
    \toprule
    {Parameter} & Value \\
    \midrule
    Feature dimension $D$ & 256 \\
    Static point dimension $D_m$ & 9 \\
    Agent pose dimension $D_a$ & 10 \\
    Activation & ReLU \\
    Number of layers & $3$ \\
    Number of attention heads & 8 \\
    % Hidden size & 512 \\
    % Dimension of key and value & 32 \\
    Dropout & 0.1 \\
    \midrule
    discount factor $\gamma$ & 0.1 \\
    GAE parameter $\lambda$ & 0.9 \\
    Clip range $\epsilon$ & 0.2 \\
    Update interval $I$ & 8 \\
    \midrule
    Weight of selector loss & 1 \\
    Weight of value loss & 3 \\
    Weight of policy loss & 100 \\
    Weight of entropy loss & 0.001 \\
    Weight of IL loss & 1 \\
    \bottomrule
    \end{tabular}
    \caption{Hyperparameters of model architecture, PPO-related parameters, and loss weights.}
    \label{table:param}
\end{table}%







\begin{table}[t]
    % \vspace{6pt}
    % \centering
    % % \vspace{-0.2cm}
    \setlength{\tabcolsep}{2pt} % Adjust the space between columns
    % \setlength{\aboverulesep}{0.02pt}
    % \setlength{\belowrulesep}{0.02pt}
    % \small
    \fontsize{8pt}{10pt}\selectfont % Set font size and line spacing
    \resizebox{0.48\textwidth}{!}{\begin{tabular}{ c |  c | c c c }
    \toprule
    \makecell{Planner}   &  CLS-NR ($\uparrow$) &  \makecell{Efficiency\\(samples/sec, $\uparrow$)}  & Num. Samples & Train Time    \\
    \midrule
        {ScenarioNet~\cite{li2023scenarionet}}        & 55.60  & 25.72    & 7,798,472   & 3d12h11m38s  \\
        CarPlanner-IL     & \underline{93.41}  & \underline{1,181.46} & 70,487,200  & 16h34m12s    \\
        CarPlanner     & \textbf{94.07}  & \textbf{1,632.25} & 70,487,200  & 11h59m44s    \\
        \bottomrule
    \end{tabular}}
    % \vspace{-0.35cm}
    \caption{Comparison of training efficiency with model-free settings. Experimental results are based on the Test14-Random non-reactive benchmark.}
    \label{table:efficiency}
    % \vspace{-0.4cm}
\end{table}




\begin{table}[t]
    % \vspace{6pt}
    \centering
    % \vspace{-0.2cm}
    \setlength{\tabcolsep}{2pt} % Adjust the space between columns
    \setlength{\aboverulesep}{0.3pt}
    \setlength{\belowrulesep}{0.3pt}
    % \small
    \fontsize{8pt}{12pt}\selectfont % Set font size and line spacing
    \resizebox{0.49\textwidth}{!}{\begin{tabular}{c | c c |  c c c c c  }
    \toprule
    & \multicolumn{2}{c|}{Design Choices} & \multicolumn{5}{c}{Closed-loop metrics ($\uparrow$)}  \\
    \midrule
    Model Type & \makecell{Random\\Sample}   & \makecell{Guide\\Reward}   &  CLS-NR &  S-CR  & S-Area  & S-PR & S-Comfort   \\
    \midrule
    \multirow{2}{*}{\makecell{Vanilla}} 
        & {\cmark}   & Progress       &  67.56 $\pm$ 0.38 & 90.97 $\pm$ 0.78 & 94.64 $\pm$ 1.72 & 72.17 $\pm$ 0.21 & 64.21 $\pm$ 1.29    \\
        & {\cmark}& DE            &  86.89 $\pm$ 0.28 & \underline{97.34 $\pm$ 0.37} & 96.36 $\pm$ 0.18 & 89.90 $\pm$ 0.11 & \textbf{94.03 $\pm$ 0.65}    \\
        \midrule
    \multirow{2}{*}{\makecell{Consistent}}
        & {\xmark}    & {FE}       &  \underline{88.14} & 96.86  & \underline{98.43}  & \underline{91.39}  & 73.73    \\
        & {\xmark}    & {DE}       &  \textbf{94.07} & \textbf{99.22}  & \textbf{99.22}  & \textbf{95.06}  & 91.09    \\
    \bottomrule
    \end{tabular}}
    % \vspace{-0.3cm}
    \caption{Comparison of vanilla and consistent auto-regressive frameworks with different guide reward design. Experimental results are based on the Test14-Random non-reactive benchmark.}
    \label{table:supple-abla-rl-reward}
    % \vspace{-0.3cm}
\end{table}




\begin{table}[t]
    % \vspace{6pt}
    \centering
    % % \vspace{-0.2cm}
    \setlength{\tabcolsep}{2pt} % Adjust the space between columns
    \setlength{\aboverulesep}{0.02pt}
    \setlength{\belowrulesep}{0.02pt}
    % % \small
    \fontsize{5pt}{6pt}\selectfont % Set font size and line spacing
    \resizebox{0.48\textwidth}{!}{\begin{tabular}{ c c |  c | c c }
    \toprule
    % \multicolumn{2}{c|}{Design Choices} & \multicolumn{1}{c|}{Closed-loop metrics ($\uparrow$)}  & \multicolumn{2}{c}{Consistent Ratio ($\uparrow$)}  \\
    % \midrule
    \makecell{Model} & \makecell{Loss}    &  CLS-NR ($\uparrow$)  & \makecell{Consistent Ratio\\Lat ($\uparrow$)}   & \makecell{Consistent Ratio\\Lon ($\uparrow$)}    \\
    \midrule
        Vanilla & RL          &  86.89 ± 0.28 & 20.00 ± 0.10 & 8.33 ± 0.00 \\
        PLUTO~\cite{cheng2024pluto}  & IL           &  91.92  &   62.45 &  41.80  \\
        Consistent & IL       &  \underline{93.41} & \underline{68.26} & \underline{43.01} \\
        Consistent & RL       &  \textbf{94.07} & \textbf{79.58} & \textbf{43.03} \\
        \bottomrule
    \end{tabular}}
    % \vspace{-0.35cm}
    \caption{Comparison for consistency. Experimental results are based on the Test14-Random non-reactive benchmark.}
    \label{table:consistency}
    % \vspace{-0.8cm}
\end{table}



\begin{table}[t]
    % \vspace{6pt}
    \centering
    % \vspace{-0.2cm}
    \setlength{\tabcolsep}{3pt} % Adjust the space between columns
    \setlength{\aboverulesep}{0.3pt}
    \setlength{\belowrulesep}{0.3pt}
    % \small
    \fontsize{5pt}{6pt}\selectfont % Set font size and line spacing
    \resizebox{0.48\textwidth}{!}{\begin{tabular}{c |  c c c c c }
    \toprule
    & \multicolumn{5}{c}{Closed-loop metrics ($\uparrow$)}   \\
    \midrule
    \makecell{Transition Model}   &  CLS-NR &  S-CR  & S-Area  & S-PR & S-Comfort   \\
    \midrule
        Reactive &  91.03 & 96.92  & \textbf{99.23}  & 91.28  & 90.00  \\
        Non-reactive  &  \textbf{94.07} & \textbf{99.22}  & 99.22  & \textbf{95.06}  & \textbf{91.09}  \\
    \bottomrule
    \end{tabular}}
    % \vspace{-0.3cm}
    \caption{Comparison of the usage of reactive and non-reactive transition models. Experimental results are based on the Test14-Random non-reactive benchmark.}
    \label{table:supple-abla-rl-tm}
    % \vspace{-0.3cm}
\end{table}





\section{Ablation Study on RL Training}



\begin{figure}[t]
    \begin{center}
    \includegraphics[width=0.46\textwidth]{image/heatmap_horizon.pdf}
    %%% trim={<left> <lower> <right> <upper>}
    \end{center}
    \vspace{-0.5cm}
    \caption{Performance of different training time horizons under different testing time horizons. The value in each cell is the CLS-NR metric on the Test14-Random non-reactive benchmark.
    }
    % \vspace{-0.6cm}
    \label{figure:supple-time-horizon}
\end{figure}



\begin{figure*}[ht]
    \begin{center}
    \includegraphics[width=0.98\textwidth]{image/vis2.pdf}
    %%% trim={<left> <lower> <right> <upper>}
    \end{center}
    \vspace{-0.7cm}
    \caption{Qualitative comparison of using reactive and non-reactive transition model in non-reactive environments. The scenario is annotated as \texttt{waiting\_for\_pedestrian\_to\_cross}. In each frame shot, ego vehicle is marked as {\color{LimeGreen} green}. Traffic agents are marked as {\color{DeepSkyBlue} sky blue}. Lineplot with {\color{Blue} blue} is the ego planned trajectory.
    }
    % \vspace{-0.5cm}
    \label{figure:vis2}
\end{figure*}




In this part, we examine the training efficiency of CarPlanner, performance of vanilla and consistent auto-regressive frameworks, the use of reactive and non-reactive model in RL training, and the impact of varying the time horizon.


\noindent\textbf{Training efficiency.}
We compare the efficiency of our model-based framework with that of ScenarioNet~\cite{li2023scenarionet}, which is an open-source platform for model-free RL training in real-world datasets~\cite{ettinger2021large, caesar2021nuplan}.
As shown in \cref{table:efficiency}, CarPlanner achieves a remarkable improvement in sampling efficiency, outperforming ScenarioNet by two orders of magnitude. Furthermore, CarPlanner not only excels in efficiency but also achieves SOTA performance, surpassing ScenarioNet by a wide margin.
% This indicates that our framework achieves greater training efficiency.

\noindent\textbf{Vanilla vs. consistent auto-regressive framework.}
The results are shown in \cref{table:supple-abla-rl-reward,table:consistency}.
The consistent auto-regressive framework generates multi-modal trajectories by conditioning on mode representations. In contrast, the vanilla framework relies on random sampling from the action Gaussian distribution to produce multi-modal trajectories. To ensure comparability in the number of modes generated by both frameworks, we sample 60 trajectories in parallel for the vanilla framework.
Given that random sampling introduces variability, we average the results across 3 random seeds.
For the consistent framework, we use displacement error (DE) and final error (FE) as guide functions to assist the policy in generating mode-aligned trajectories.
For the vanilla framework, DE is compared against a progress reward, which encourages longitudinal movement along the route while discouraging excessive lateral deviations that move the vehicle too far from any possible route.
The consistent ratio computes the ratio of generated trajectories that fall in their corresponding modes in longitudinal and lateral directions separately.

Overall, our proposed consistent framework outperforms the vanilla framework in terms of closed-loop performance, highlighting the benefits of incorporating consistency. Furthermore, RL provides more consistant trajectories than the vanilla framework and IL-based methods. Additionally, we find that DE serves as an effective guide function for policy training, further enhancing closed-loop performance.

\noindent\textbf{Reactive vs. non-reactive transition model.}
We compare the performance of the CarPlanner framework when trained with reactive and non-reactive transition models. The reactive transition model shares a similar architecture with the auto-regressive planner for the ego vehicle, utilizing relative pose encoding~\cite{zhang2024real} as the backbone network to extract features of traffic agents and predict their subsequent poses. The training loss and hyperparameters are consistent with those used for the non-reactive transition model.
As shown in \cref{table:supple-abla-rl-tm}, except for the S-Area metric, using non-reactive transition model outperforms the reactive transition model in our current implementation.
The primary difference lies in the assumptions about traffic agents: the reactive transition model assumes that the ego vehicle can negotiate with traffic agents and share the same priority, whereas in the non-reactive model, traffic agents do not respond to the ego vehicle, effectively assigning them higher priority.
A representative example is presented in \cref{figure:vis2}.
When trained with the reactive transition model, the planner assumes pedestrians will yield to the vehicle, leading it to attempt to move forward.
However, at $t_{\text{sim}} = 12s$, the planner collides with pedestrians, triggering an emergency brake, which negatively impacts safety, progress, and comfort metrics.
Although the performance of using reactive transition model is not satisfied currently, it is a more realistic assumption and we will further investigate this in future work.


\noindent\textbf{Time horizon.} We evaluate the CarPlanner framework by training it with different time horizons, including 1, 3, 5, and 8 seconds, and testing the planners in each time horizon.
The results in \cref{figure:supple-time-horizon} confirm that increasing the time horizon has a positive effect on the performance for both training and testing.
A special case is when the training time horizon is set to 1, all tested time horizons exhibit poor performance, highlighting the importance of multi-step learning in RL.
Additionally, the observation that increasing the training time horizon enhances closed-loop performance suggests the potential for further improvements by extending the time horizon beyond 8 seconds.
However, due to current limitations in data preparation, which is designed for horizons up to 8 seconds, expanding the time horizon would not provide map information or ground-truth trajectories, hindering further analysis. Consequently, we leave this exploration for future work.


\section{Comparison with Differentiable Loss}



\begin{table*}[t]
    % \vspace{6pt}
    \centering
    % \vspace{-0.2cm}
    \setlength{\tabcolsep}{3pt} % Adjust the space between columns
    \setlength{\aboverulesep}{0.3pt}
    \setlength{\belowrulesep}{0.3pt}
    % \small
    \fontsize{5pt}{6pt}\selectfont % Set font size and line spacing
    \resizebox{0.98\textwidth}{!}{\begin{tabular}{c | c c c |  c c c c c | c c }
    \toprule
    & \multicolumn{3}{c|}{Supervision Signals} & \multicolumn{5}{c|}{Closed-loop metrics ($\uparrow$)}  &  \multicolumn{2}{c}{Open-loop metrics ($\downarrow$)}  \\
    \midrule
    \makecell{Loss\\Type} & DE  & \makecell{Col}   & \makecell{Area}   &  CLS-NR &  S-CR  & S-Area  & S-PR & S-Comfort   & \makecell{Col\\Mean [Min, Max]} & \makecell{Area\\Mean [Min, Max]}  \\
    \midrule
    \multirow{4}{*}{\makecell{IL}} 
        & {\cmark} & {\xmark}   & {\xmark}      &  93.41 & 98.85  & \underline{98.85}  & 93.87  & \textbf{96.15}   &  0.17 [\textbf{0.00}, 0.47]  &  0.09 [\textbf{0.00}, 0.40]  \\
        & {\cmark} & {\cmark}   & {\xmark}      &  \underline{93.67} & \textbf{99.23}  & \underline{98.85}  & \underline{94.63}  & 94.23    &  0.16 [\textbf{0.00}, \underline{0.43}]  &  \underline{0.07} [\textbf{0.00}, \underline{0.27}]  \\ 
        & {\cmark} & {\xmark}   & {\cmark}      &  93.12 & 98.46  & 98.84  & 92.88  & 94.21    &  \underline{0.15} [\textbf{0.00}, 0.44]  &  0.08  [\textbf{0.00}, 0.30]  \\ 
        & {\cmark} & {\cmark}   & {\cmark}      &  93.32 & 98.46  & 98.46  & 94.05  & \underline{95.77}   &  \underline{0.15} [\textbf{0.00}, \underline{0.43}]  &  0.09 [\textbf{0.00}, 0.39]  \\ 
    \midrule
    \multirow{2}{*}{\makecell{RL}} 
        & {\cmark} & {\xmark}    & {\xmark}     &  90.44 & 97.49  & 96.91  & 93.33  & 90.73   &  0.17  [\textbf{0.00}, 0.49]  &  0.14 [\textbf{0.00}, 0.51]  \\
        & {\cmark} & {\cmark}    & {\cmark}     &  \textbf{94.07} & \underline{99.22}  & \textbf{99.22}  & \textbf{95.06}  & 91.09   &  \textbf{0.12}  [\textbf{0.00}, \textbf{0.39}]  &  \textbf{0.05} [\textbf{0.00}, \textbf{0.22}]  \\
        \bottomrule
    \end{tabular}}
    % \vspace{-0.3cm}
    \caption{Comparison with different loss types and supervision signals. Closed-loop results are based on the Test14-Random non-reactive benchmark. Open-loop results are on validation set.}
    \label{table:supple-abla-diff-loss}
    % \vspace{-0.3cm}
\end{table*}





\begin{table*}[t]
    % \vspace{6pt}
    \centering
    % \vspace{-0.2cm}
    \setlength{\tabcolsep}{3pt} % Adjust the space between columns
    \setlength{\aboverulesep}{0.3pt}
    \setlength{\belowrulesep}{0.3pt}
    % \small
    \fontsize{5pt}{6pt}\selectfont % Set font size and line spacing
    \resizebox{0.98\textwidth}{!}{\begin{tabular}{c | c c c c c c |  c c c c c}
    \toprule
     & \multicolumn{6}{c|}{Design Choices} & \multicolumn{5}{c}{Closed-loop metrics ($\uparrow$)}   \\
    \midrule
    \makecell{Loss\\Type} & \makecell{Model\\Type} & \makecell{Mode\\Type} & \makecell{Mode\\Dropout}   &  \makecell{Scorer\\Side Task}  &  \makecell{Ego-history\\Dropout} & \makecell{Backbone\\Sharing}   &  CLS-NR &  S-CR  & S-Area  & S-PR & S-Comfort \\
    \midrule
    \multirow{3}{*}{\makecell{IL}}
        & Vanilla & -  & {-}    & {-}   & {\cmark}   & {-}  &  86.48 & 97.09  & 97.29  & 88.05  & 94.19   \\
        & Consistent & Lon  & {\xmark}    & {\cmark}   & {\cmark}   & {\cmark}   &  88.79 & 96.67  & 96.08  & 89.63  & \underline{94.90} \\
        & Consistent & Lon-Lat & {\cmark}    & {\cmark}  & {\cmark}   & {\cmark}     &  \underline{93.41}             & \underline{98.85}  & \underline{98.85}                & \underline{93.87}                & \textbf{96.15}         \\ 
    \midrule
    \multirow{3}{*}{\makecell{RL}}
        & Vanilla & -  & {-}    & {-}   & {\xmark}   & {-}  &  85.56 & 97.27  & 95.70  & 89.17  & 93.36 \\
        & Consistent & Lon  & {\xmark}    & {\cmark}   & {\xmark}   & {\xmark}  &  90.57 & 97.30  & 97.68  & 92.20  & 94.59 \\
        & Consistent & Lon-Lat & {\cmark}    & {\cmark}    & {\xmark}   & {\xmark}   &  \textbf{94.07}    & \textbf{99.22}     & \textbf{99.22}    & \textbf{95.06}    & 91.09        \\ 
        \bottomrule
    \end{tabular}}
    % \vspace{-0.3cm}
    \caption{Effect of different mode representations. Experimental results are based on the Test14-Random non-reactive benchmark.}
    \label{table:supple-abla-mode}
    % \vspace{-0.3cm}
\end{table*}

 





\begin{figure}[t]
    \begin{center}
    \includegraphics[width=0.4\textwidth]{image/diff_loss.pdf}
    %%% trim={<left> <lower> <right> <upper>}
    \end{center}
    \vspace{-0.3cm}
    \caption{The computational graph of differentiable loss (a) and RL (b) framework for optimizing same metrics such as displacement errors, collision avoidance, and adherence to drivable area.}
    % \vspace{-0.5cm}
    \label{figure:diff-loss}
\end{figure}




In typical IL setting, the supervision signal provided to the trajectory generator is the displacement error (DE) between the ego-planned trajectory and the ground-truth trajectory. Several works~\cite{suo2021trafficsim, huang2023gameformer, cheng2024pluto} propose to convert non-differentiable metrics, such as avoiding collision (Col) and adherence to drivable area (Area), into differentiable loss functions that can directly backpropagate to the generator. In contrast, CarPlanner leverages an RL framework, which introduces surrogate objectives to indirectly optimize these non-differentiable metrics.

In this part, we compare these two approaches which provide rich supervision signals to the trajectory generator. The results are summarized in \cref{table:supple-abla-diff-loss}.
In IL training, the Col and Area metrics are converted into differentiable loss functions, whereas in RL training, Col and Area are treated as reward functions, contributing to the quality reward as described in the main paper.
It is important to note that the implementations for differentiable loss functions and reward functions are identical, except that gradient flow is enabled for differentiable loss functions.
The open-loop metrics compute the Col and Area values across all candidate multi-modal trajectories, with the Mean, Min, and Max referring to the mean, minimum, and maximum values of the Col and Area metrics within the candidate trajectory set.

Our findings suggest that incorporating Col loss benefits the open-loop Col metric and improves the closed-loop S-CR metrics, thereby enhancing closed-loop performance. However, incorporating Area loss results in better open-loop Area metrics but deteriorates closed-loop performance. Compared to differentiable loss functions, RL with Col and Area as quality rewards yields the trajectory set with the highest overall quality, as evidenced by smaller Mean and Max metrics in open-loop metrics.
This improvement can be attributed to RL's ability to optimize the reward-to-go using surrogate objectives that account for future rewards, while differentiable loss functions are limited to timewise-aligned optimization in our current implementation.
This distinction is illustrated in \cref{figure:diff-loss}: in (a), the loss at time step $t$ is directly computed from $s^0_t$, meaning that during backward propagation, the loss at time step $t$ cannot influence the optimization of prior time steps.
In (b), however, the non-differentiable reward is aggregated into a return (reward-to-go), which serves as a reference for computing the loss at time step $t$. Through this process, the reward at time step $t$ can influence the trajectory at earlier time steps $t'$ ($t' < t$). In the future, we aim to combine the advantages of differentiable loss which can provide low-variance gradients, and RL which can provide long-term foresight, by model-based RL optimization techniques~\cite{claveramodel, hansen2022temporal}.






\section{Effect of Mode Representation}

In this part, we examine the impact of mode representations on performance. The results are presented in \cref{table:supple-abla-mode}.
For both the vanilla and consistent frameworks, we disable the use of random sampling to focus solely on mode-aligned trajectories. As a result, the vanilla framework can only generate single-modal trajectories, leading to the lowest performance.
In the consistent framework, we explore two types of mode representations: Lon and Lon-Lat. The Lon representation assigns modes based on longitudinal movements along the route, whereas the Lon-Lat representation decomposes modes by both longitudinal and lateral movements.
Aligned with the main paper, we use ego-history dropout and backbone sharing only for IL training. For the Lon representation, we close mode dropout since it does not rely on any map or agent representation in initial state.
The results indicate that introducing consistency provides greater benefits to RL training, with the Lon-Lat representation proving to be more effective than the Lon representation. This suggests that decomposing mode representations into both longitudinal and lateral components enhances the model's ability by providing more explicit mode information.



% \section{Qualitative Results}

% Qualitative behaviors of our proposed CarPlanner can be found in \texttt{video.mp4}.
