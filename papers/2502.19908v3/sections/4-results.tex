


\section{Experiments}

\subsection{Experimental Setup}

\noindent\textbf{Dataset and simulator.} We use nuPlan~\cite{caesar2021nuplan}, a large-scale closed-loop platform for studying trajectory planning in autonomous driving, to evaluate the efficacy of our method. The nuPlan dataset contains driving log data over 1,500 hours collected by human expert drivers across 4 diverse cities. It includes complex, diverse scenarios such as lane follow and change, left and right turn, traversing intersections and bus stops, roundabouts, interaction with pedestrians, etc. As a closed-loop platform, nuPlan provides a simulator that uses scenarios from the dataset as initialization. During the simulation, traffic agents are taken over by log-replay (non-reactive) or an IDM~\cite{treiber2000idm} policy (reactive). The ego vehicle is taken over by user-provided planners. The simulator lasts for 15 seconds and runs at 10 Hz. At each timestamp, the simulator queries the planner to plan a trajectory, which is tracked by an LQR controller to generate control commands to drive the ego vehicle.


\noindent\textbf{Benchmarks and metrics.} We use two benchmarks: Test14-Random and Reduced-Val14 for comparing with other methods and analyzing the design choices within our method. The Test14-Random provided by PlanTF~\cite{cheng2024rethinking} contains 261 scenarios. The Reduced-Val14 provided by PDM~\cite{Dauner2023CORL} contains 318 scenarios.

We use the closed-loop score (CLS) provided by the official nuPlan devkit\footnote{\url{https://github.com/motional/nuplan-devkit}} to assess the performance of all methods.
The CLS score comprehends different aspects such as safety (S-CR, S-TTC), drivable area compliance (S-Area), progress (S-PR), comfort, etc.
Based on the different behavior types of traffic agents, CLS is detailed into CLS-NR (non-reactive) and CLS-R (reactive). 
% To further analyze various components of our methods, we also use open-loop metrics, such as loss of trajectory generator and selector.

\section{Related Work}

\paragraph{Temporal Point Processes (TPPs).} 
TPPs \citep{Hawkes1971Spectra, daley2007introduction} have emerged as the standard method to model asynchronous time series data. 
% It is a mathematical framework for modeling sequences of events  and autoregressively generate events one after another. 
Over the last decade, a large number of neural temporal point processes have been proposed to capture complex dynamics of stochastic processes in time by using neural networks.
\citet{duTPP, MeiEisner} proposed to use models based on Recurrent Neural Networks (RNNs) to model the sequence of events. Then, more advanced models \citep{Mehrasa2019Variational, DiffusionTPP} were proposed to better model uncertainty when predicting the future. Recently, several neural TPP models incorporate Transformers in order to improve performance by using attention to better model long-term dependencies. These include the Self-attentive Hawkes process (SAHP) \citep{Zhang2020Self}, Transformer Hawkes process (THP) \citep{Zuo2020Transformer}, and Attentive Neural Hawkes Process (Att-NHP) \citep{Mei2022Transformer}.


\paragraph{Transformers for Time Series.} 
Transformers \citep{Vaswani2017Attention} have become popular to model regularly-sampled time series because of their ability to capture long-range dependencies and to extract semantic correlations among the elements of a long sequence. Informer \citep{Zhou2021Informer} introduced a novel self-attention architecture to reduce the quadratic complexity of the original self-attention. Autoformer \citep{Wu2021Autoformer} used a novel decomposition architecture with an auto-correlation mechanism to identify more reliable temporal patterns.  Crossformer \citep{Zhang2023Crossformer} proposed a novel architecture to model both  the cross-time and cross-dimension dependencies multivariate time series forecasting. PatchTST \citep{Yuqi2023PatchTST} tokenizes the time series in patches, and proposes a channel-independent patch time series Transformer to improve the long-term forecasting accuracy.

Due to space limitations, we only review some popular models and invite the reader to \citep{Wen2023Transformers, Zeng2023Transformers} for a more complete literature reviews of Transformer models for regularly-sampled time series.
Most of the time series Transformer models are designed for specific tasks, and cannot be easily extended to asynchronous time series data or other tasks like anomaly detection or imputation.

% Most of these Transformer models are designed for the forecasting task, and cannot be easily extended to other tasks like anomaly detection or imputation.
% Similarly, these models are designed for regularly-sampled time series and cannot be easily extended to asynchronous time series due to the different nature of the data.


\paragraph{Foundation Models (FMs) for Time Series.}
FMs \citep{Bommasani2021Opportunities} are a family of deep models that are pretrained on vast amounts of data, and have caused a paradigm shift due to their unprecedented capabilities for zero-shot and few-shot generalization.
FMs have revolutionized natural language processing \citep{Brown2020Language, BigScience2023Bloom, Wu2024Empirical, Dubey2024Llama3} and computer vision \citep{Radford2021Learning, Kirillov2023Segment}.
% Recently, several methods have been developed to build FMs for time series and can be organized in three main categories based on the pretraining data: text, time series, and image.
The availability of large-scale time series datasets has opened the door to pretrain a large model on time series data.
ForecastPFN \citep{Dooley2024ForecastPFN} proposed the first zero-shot forecasting method trained purely on synthetic data.
Lag-Llama \citep{Rasul2023LagLlama} introduced a univariate probabilistic forecasting model that was pretrained on a large corpus of diverse time series data.
TimeFM \citep{Das2024TimesFM} pretrained a decoder style attention model with input patching, using a large time series corpus comprising both real-world and synthetic
datasets. Chronos \citep{ansari2024chronos} introduced a framework for pretraining on tokenized time series data, achieving state-of-the-art zero-shot forecasting performance and simplifying forecasting workflows.
MOIRAI \citep{Woo2024Unified} is an enhanced Transformer architecture pretrained in the Large-scale Open Time Series Archive, that achieves competitive performance as a zero-shot forecaster.
% \subparagraph{Text-based FMs.} LLMs pretrained on large amounts of text data have emerged as a promising direction to model time series data.
% GPT4TS \citep{Zhou2023One}, LLM4TS \citep{Chang2024Llm4ts}, and TEMPO \citep{Cao2024Tempo} fine-tuned a pretrained GPT2 \citep{Radford2019Language} on some time series downstream tasks to capture intrinsic dynamic properties.
% TimeLLM \citep{Jin2023TimeLLM} proposed a reprogramming framework to repurpose LLMs for general time series forecasting with the backbone language models kept intact.
% PromptCast \citep{Xue2023Promptcast} introduced  a new prompt-based forecasting paradigm, where the numerical input and output are transformed into prompts and the forecasting task is framed in a sentence-to-sentence manner.
% LLMTime \citep{Gruver2023Large} showed that LLMs can zero-shot extrapolate time series if the numerical values of the time series are well represented.
\paragraph{{LLMs for Time Series}} LLMs pretrained on large amounts of text data have emerged as a promising direction to model time series data.
GPT4TS \citep{Zhou2023One}, LLM4TS \citep{Chang2024Llm4ts}, and TEMPO \citep{Cao2024Tempo} fine-tuned a pretrained GPT2 \citep{Radford2019Language} on some time series downstream tasks to capture intrinsic dynamic properties.
TimeLLM \citep{Jin2023TimeLLM} proposed a reprogramming framework to repurpose LLMs for general time series forecasting with the backbone language models kept intact.
PromptCast \citep{Xue2023Promptcast} introduced  a new prompt-based forecasting paradigm, where the numerical input and output are transformed into prompts and the forecasting task is framed in a sentence-to-sentence manner.
LLMTime \citep{Gruver2023Large} showed that LLMs can zero-shot extrapolate time series if the numerical values of the time series are well represented. LLM Processes \citep{requeima2024llm} explores various prompt configurations for using LLMs for time series forecasting condiitoned on a textual context. We refer the reader to \citep{zhang2024large} for a more detailed survey on the topic.


% \subparagraph{Time series-based FMs.} The availability of large-scale time series datasets has opened the door to pretrain a large model on time series data.
% ForecastPFN \citep{Dooley2024ForecastPFN} proposed the first zero-shot forecasting method trained purely on synthetic data.
% Lag-Llama \citep{Rasul2023LagLlama} introduced a univariate probabilistic forecasting model that was pretrained on a large corpus of diverse time series data.
% TimeFM \citep{Das2024TimesFM} pretrained a decoder style attention model with input patching, using a large time series corpus comprising both real-world and synthetic
% datasets. 
% MOIRAI \citep{Woo2024Unified} is an enhanced Transformer architecture pretrained in the Large-scale Open Time Series Archive, that achieves competitive performance as a zero-shot forecaster.
% \subparagraph{Image-based FMs.} Several works started to explore the use of FMs pretrained on images because of the better intrinsic similarities between images and time series such as trend, stationarity, seasonality/periodicity, and sudden change. 
% \citep{Zhou2023One} tried to fine-tune a BEiT \citep{Bao2021BEiT} trained on images for time series forecasting, but it falls short of the leading text-based and time series-based FMs.
% Recently, VisionTS \citep{Chen2024Visionts} proposes to use a vision Transformer pretrained on ImageNet to reduce the cross-domain gap or in-domain heterogeneity between time series and text.

\paragraph{Vision Models for Time Series.} Several works started to explore the use of FMs pretrained on images because of the better intrinsic similarities between images and time series such as trend, stationarity, seasonality/periodicity, and sudden change. 
\citet{Zhou2023One} tried to fine-tune a BEiT \citep{Bao2021BEiT} trained on images for time series forecasting, but it falls short of the leading text-based and time series-based FMs.
Recently, VisionTS \citep{Chen2024Visionts} proposes to use a vision Transformer pretrained on ImageNet to reduce the cross-domain gap or in-domain heterogeneity between time series and text.





\paragraph{Parameter Efficient Fine Tuning (PEFT).} 
PEFT \citep{peft} is a paradigm to adapt pretrained LLMs to various domains without fine-tuning all of a model’s parameters, which can be costly and require large amounts of training data. % The reader may refer to \cite{PEFTreview} for a comprehensive survey.
% Common PEFT methods include Low Rank Adapters (LoRA) and prompt-based methods. 
LoRA \citep{LoRA} methods freeze the pretrained model weights and injects trainable rank decomposition matrices into each layer of the Transformer architecture, greatly reducing the number of trainable parameters for downstream tasks. 
QLoRA \citep{QLoRA} advances finetuning by significantly reducing memory usage while preserving task performance.

\paragraph{Soft Prompt Tuning.}
Soft prompts have emerged as a compute efficient method for adapting a pretrained LLMs to new domains without altering their core architectures.
\citet{Brown2020Language} were among the first to demonstrate the power of prompting for task adaption of pretrained language models, but automatically finding suitable sets of
text prompts remains an open challenge.
\citet{Li2021Prefix, qin-eisner-2021-learning} proposed the prefix tuning technique that preprends a few task specific soft tokens to the input and hidden states of each Transformer layer. During training, the parameters of soft prompts are updated by gradient descent while the model parameters keep frozen.
\citet{Liu2021Ptuning} showed the prefix tuning technique could be effectively applied to
natural language understanding with different scales of models.
\citet{Lester2021Power} simplified the prefix tuning technique such that it only adds soft prompts to the input layer and is now considered the standard soft prompt-tuning.
% where they optimize sequences of continuous-valued embeddings prepended to the real embeddings of the input tokens.


\noindent\textbf{Implementation details.} We follow PDM~\cite{Dauner2023CORL} to construct our training and validation splits. The size of the training set is 176,218 where all available scenario types are used, with a number of 4,000 scenarios per type. The size of the validation set is 1,118 where 100 scenarios with 14 types are selected. We train all models with 50 epochs in 2 NVIDIA 3090 GPUs. The batch size is 64 per GPU. We use AdamW optimizer with an initial learning rate of 1e-4 and reduce the learning rate when the validation loss stops decreasing with a patience of 0 and decrease factor of 0.3. For RL training, we set the discount $\gamma = 0.1$ and the GAE parameter $\lambda = 0.9$. The weights of value, policy, and entropy loss are set to 3, 100, and 0.001, respectively. The number of longitudinal modes is set to 12 and a maximum number of lateral modes are set to 5.


\subsection{Comparison with SOTAs}


\begin{table*}[t]
  % \vspace{6pt}
  \centering
  % \vspace{-0.2cm}
  \setlength{\tabcolsep}{4pt} % Adjust the space between columns
  \setlength{\aboverulesep}{0.3pt}
  \setlength{\belowrulesep}{0.3pt}
  % \small
  \fontsize{5pt}{6pt}\selectfont % Set font size and line spacing
  \resizebox{0.98\textwidth}{!}{\begin{tabular}{ c c c c |  c c c c c | c c }
    \toprule
  \multicolumn{4}{c|}{Design Choices} & \multicolumn{5}{c|}{Closed-loop metrics ($\uparrow$)}  &  \multicolumn{2}{c}{Open-loop losses ($\downarrow$)}  \\
  \midrule
  \makecell{Reward\\DE} &  \makecell{Reward\\Quality} & \makecell{Coord\\Trans} & \makecell{KNN} &  CLS-NR &  S-CR & S-Area & S-PR & S-Comfort  & \makecell{Loss\\Selector} & \makecell{Loss\\Generator}  \\
  \midrule
  % \multirow{2}{*}{\makecell{RL}}
      {\xmark}   & {\cmark}   & {\cmark}    & {\cmark}  &  31.79                & 95.74              & 98.45               & 33.10              & 48.84               &  1.03              &  \textbf{30.3}  \\ 
      {\cmark}   & {\xmark}   & {\cmark}    & {\cmark}  &  90.44                & 97.49              & 96.91               & 93.33              & 90.73               &  \textbf{0.99}     &  \underline{1221.6}  \\ 
      {\cmark}   & {\cmark}   & {\xmark}    & {\cmark}  &  90.78                & 96.92              & \underline{98.46}   & 91.37              & \textbf{94.23}      &  \underline{1.00}  &  2130.7  \\ 
      {\cmark}   & {\cmark}   & {\cmark}    & {\xmark}  &  \underline{92.73}    & \underline{98.07}  & \underline{98.46}   & \underline{94.69}  & \underline{93.44}   &  1.03              &  2083.6  \\ 
      \rowcolor{gray!30} {\cmark}   & {\cmark}   & {\cmark}    & {\cmark}  &  \textbf{94.07}       & \textbf{99.22}     & \textbf{99.22}      & \textbf{95.06}     & 91.09               &  1.03              &  1624.5  \\ 
      \bottomrule
  \end{tabular}}
  \vspace{-0.3cm}
  \caption{Ablation studies on the design choices in RL training. Results are in Test14-random non-reactive benchmark.}
  \label{table:abla-rl}
  \vspace{-0.6cm}
\end{table*}



\noindent\textbf{SOTAs.} We categorize the methods into Rule, IL, and RL based on the type of trajectory generator.
(1) PDM~\cite{Dauner2023CORL} wins the nuPlan challenge 2023, its IL-based and rule-based variants are denoted as PDM-Open and PDM-Closed, respectively. PDM-Closed follows the generation-selection framework where IDM is used to generate multiple candidate trajectories and rule-based selector considering safety, progress, and comfort is used to select the best trajectory.
(2) PLUTO~\cite{cheng2024pluto} also obeys the generation-selection framework and uses contrastive IL to incorporate various data augmentation techniques and trains the generator.
(3) Gen-Drive~\cite{huang2024gen} is a concurrent work that follows a pretrain-finetune pipeline where IL is used to pretrain a diffusion-based planner and RL is used to finetune the denoising process based on a reward model trained by AI preference.

\noindent\textbf{Results.} We compare our method with SOTAs in Test14-Random and Reduced-Val14 benchmark as shown in \cref{table:main-results1} and \cref{table:main-results2}.
Overall, our CarPlanner demonstrates superior performance, particularly in non-reactive environments.

In the non-reactive setting, our method achieves the highest scores across all metrics, with an improvement of 4.02 and 2.15 compared to PDM-Closed and PLUTO, establishing the potential of RL and the superior performance of our proposed framework.
Moreover, CarPlanner reveals substantial improvement in the progress metric S-PR compared to PDM-Closed in \cref{table:main-results2} and comparable collision metric S-CR, indicating the ability of our method to improving driving efficiency while maintaining safe driving.
Importantly, we do not apply any techniques commonly used in IL such as data augmentation~\cite{cheng2024rethinking,cheng2024pluto} and ego-history masking~\cite{Guo-RSS-23}, underscoring the intrinsic capability of our approach to solving the closed-loop task.

In the reactive setting, while our method performs well, it falls slightly short of PDM-Closed. This discrepancy arises because our model was trained exclusively in non-reactive settings and has not interacted with the IDM policy used by reactive settings; as a result, our model is less robust to disturbances generated by reactive agents during testing.


\subsection{Ablation Studies}



\begin{table*}[t]
    % \vspace{6pt}
    \centering
    % \vspace{-0.2cm}
    \setlength{\tabcolsep}{3pt} % Adjust the space between columns
    \setlength{\aboverulesep}{0.3pt}
    \setlength{\belowrulesep}{0.3pt}
    % \small
    \fontsize{5pt}{6pt}\selectfont % Set font size and line spacing
    \resizebox{0.98\textwidth}{!}{\begin{tabular}{c | c c c c |  c c c c c | c c }
    \toprule
    & \multicolumn{4}{c|}{Design Choices} & \multicolumn{5}{c|}{Closed-loop metrics ($\uparrow$)}  &  \multicolumn{2}{c}{Open-loop losses ($\downarrow$)}  \\
    \midrule
    \makecell{Loss\\Type} & \makecell{Mode\\Dropout}   &  \makecell{Selector\\Side Task}  &  \makecell{Ego-history\\Dropout} & \makecell{Backbone\\Sharing}   &  CLS-NR &  S-CR  & S-Area  & S-PR & S-Comfort   & \makecell{Loss\\Selector} & \makecell{Loss\\Generator}  \\
    \midrule
    \multirow{5}{*}{\makecell{IL}} 
        & {\xmark}   & {\xmark}   & {\xmark}    & {\xmark}    &  90.82             & 97.29              & 98.45                & 92.15                & 94.57                 &  \underline{1.04}    &  \textbf{147.5}    \\ 
        & {\cmark}   & {\xmark}   & {\xmark}    & {\xmark}    &  91.21             & 96.54              & 98.46                & 91.44                & \textbf{96.92}        &  1.07                &  \underline{153.0}    \\ 
        & {\cmark}    & {\cmark}  & {\xmark}   & {\xmark}     &  91.51             & 96.91              & 98.46                & \textbf{95.30}       & \underline{96.91}     &  \underline{1.04}    &  162.3    \\ 
        & {\cmark}    & {\cmark}  & {\cmark}   & {\xmark}     &  92.72             & 98.06              & 98.84                & 94.88                & 95.35                 &  \underline{1.04}    &  167.5    \\ 
        \rowcolor{gray!30} & {\cmark}    & {\cmark}  & {\cmark}   & {\cmark}     &  93.41             & \underline{98.85}  & 98.85                & 93.87                & 96.15                 &  \underline{1.04}    &  174.3    \\ 
    \midrule
    % \multirow{5}{*}{\makecell{RL}}
        & {\xmark}    & {\xmark}    & {\xmark}   & {\xmark}   &  91.67             & 98.84              & 98.84                & 91.69                & 90.73                 &  \underline{1.04}    &  1812.6  \\ 
        & {\cmark}    & {\xmark}    & {\xmark}   & {\xmark}   &  \underline{93.46} & 98.07              & \textbf{99.61}       & 94.26                & 92.28                 &  1.09                &  2254.6  \\ 
        \rowcolor{gray!30}
        & {\cmark}    & {\cmark}    & {\xmark}   & {\xmark}   &  \textbf{94.07}    & \textbf{99.22}     & \underline{99.22}    & \underline{95.06}    & 91.09                 &  \textbf{1.03}       &  1624.5  \\ 
        \multirow{-3}{*}{\makecell{RL}}
        & {\cmark}    & {\cmark}    & {\cmark}   & {\xmark}   &  89.51             & 97.27              & 98.44                & 90.93                & 83.20                 &  1.05                &  5424.3  \\ 
        & {\cmark}    & {\cmark}    & {\cmark}   & {\cmark}   &  88.66             & 95.54              & 98.84                & 92.82                & 86.05                 &  1.21                &  1928.1  \\ 
        \bottomrule              
    \end{tabular}}
  \vspace{-0.3cm}
    \caption{Effect of different components on IL and RL loss using our CarPlanner. Results are in Test14-random non-reactive benchmark.}
    \label{table:abla-il-rl}
    \vspace{-0.3cm}
\end{table*}


We investigate the effects of different design choices in RL training. The results are shown in \cref{table:abla-rl}.

\noindent\textbf{Influence of reward items.}
% The results demonstrate that the DE and quality rewards are complementary.
When using the quality reward only, the planner tends to generate static trajectories and achieves a low progress metric. This occurs because the ego vehicle begins in a safe, drivable state, but moving forward is at risk of collisions or leaving the drivable area.
% On the other hand, compared to using DE reward only, incorporating the quality reward significantly improves closed-loop metrics.
On the other hand, when the quality reward is incorporated alongside the DE reward, it leads to significant improvements in closed-loop metrics compared to using the DE reward alone.
For instance, the S-CR metric rises from 97.49 to 99.22, and the S-Area metric rises from 96.91 to 99.22. These improvements indicate that the quality reward encourages safe and comfortable behaviors.

\noindent\textbf{Effectiveness of IVM.}
The results show that the coordinate transformation and KNN techniques in IVM notably improve closed-loop metrics and generator loss.
For instance, with the coordinate transformation technique, the overall closed-loop score increases from 90.78 to 94.07, and S-PR rises from 91.37 to 95.06.
These improvements are attributed to the enhanced accuracy of value estimation in RL, leading to generalized driving in closed-loop.


\begin{figure*}[ht]
    \begin{center}
    \includegraphics[width=0.98\textwidth]{image/vis.pdf}
    %%% trim={<left> <lower> <right> <upper>}
    \end{center}
    \vspace{-0.7cm}
    \caption{Qualitative comparison of PDM-Closed and our method in non-reactive environments. The scenario is annotated as \texttt{waiting\_for\_pedestrian\_to\_cross}. In each frame shot, ego vehicle is marked as {\color{LimeGreen} green}. Traffic agents are marked as {\color{DeepSkyBlue} sky blue}. Lineplot with {\color{Blue} blue} is the ego planned trajectory.
    }
    \vspace{-0.5cm}
    \label{figure:vis}
\end{figure*}


\subsection{Extention to IL}
\label{section:abla-il}

In addition to designing for RL training, we also extend the CarPlanner to incorporate IL. We conduct rigorous analysis to compare the effects of various design choices in IL and RL training, as summarized in \cref{table:abla-il-rl}. Our findings indicate that while mode dropout and selector side task contribute to both IL and RL training, ego-history dropout and backbone sharing, often effective in IL, are less suitable for RL.

\noindent\textbf{Ego-history dropout.} Previous works~\cite{Ogale-RSS-19, Guo-RSS-23, cheng2024rethinking, cheng2024pluto} suggest that planners trained via IL may rely too heavily on past poses and neglect environmental state information. To counter this, we combine techniques from ChauffeurNet~\cite{Ogale-RSS-19} and PlanTF~\cite{cheng2024rethinking} into an ego-history dropout module, randomly masking ego history poses and current velocity to alleviate the causal confusion issue.

Our experiments confirm that ego-history dropout benifits IL training, as it improves performance across closed-loop metrics like S-CR and S-Area. However, in RL training, we observe a negative impact on advantage estimation due to ego-history dropout, which significantly affects the value part of generator loss, leading to closed-loop performance degradation.
This suggests that RL training naturally addresses the causal confusion problem inherent in IL by uncovering causal relationships that align with the reward signal, which explicitly encodes task-oriented preferences. This capability highlights the potential of RL to push the boundaries of learning-based planning.

\noindent\textbf{Backbone sharing.} This choice, often used in IL-based multi-modal planners, promotes feature sharing across tasks to improve generalization. While backbone sharing helps IL by balancing losses across trajectory generator and selector, we find it adversely affects RL training. Specifically, backbone sharing leads to higher losses for both the trajectory generator and selector in RL, indicating that gradients from each task interfere. The divergent objectives in RL for trajectory generation and selection tasks seem to conflict, reducing overall policy performance. Consequently, we avoid backbone sharing in our RL framework to maintain task-specific gradient flow and improve policy quality.






\subsection{Qualitative Results}

We provide qualitative results as shown in \cref{figure:vis}. In this scenario, ego vehicle is required to execute a right turn while navigating around pedestrians. In this case, Our method shows a smooth, efficient performance.
From $t_{\text{sim}} = 0s$ to $t_{\text{sim}} = 9s$, all methods wait for the pedestrians to cross the road.
At $t_{\text{sim}} = 10s$, an unexpected pedestrian goes back and prepares to re-cross the road. PDM-Closed is unaware of this situation and takes an emergency stop, but it still intersects with this pedestrian. In contrast, our IL variant displays an awareness of the pedestrian's movements and consequently conducts a braking maneuver. However, it still remains close to the pedestrian. Our RL method avoids this hazard by starting up early up to $t_{\text{sim}} = 9s$ and achieves the highest progress and safety metrics.




