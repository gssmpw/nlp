\vspace{-0.5ex}

\section{Introduction}\label{sec:intro}

\blue{
Flexible Electronics (FE) have emerged as a promising technology with the potential to revolutionize a wide range of applications, including wearable devices, sensors, and next-generation medical technologies~\cite{Gao:FlexibleWearableSensing, Heng:AM2022:FlexHumanMachInterfaces,gao:2016:flexsensor}. 
FE, based on materials like Indium Gallium Zinc Oxide (IGZO), offers advantages such as lightweight, low cost \purple{manufacturing}, and adaptable form factors, making it suitable for applications where traditional silicon-based electronics may fall short~\cite{Baruah:FabricationFE2023,Jeong:igzoperformance}. 
However, the integration of FE into practical, high-performance systems requires overcoming several inherent challenges, primarily the need for low power, small footprint, and efficient near-sensor or on-sensor processing~\cite{Arokia:feAdvantages2012,Lozano:aspdac25:BinCoDesign}.

One of the key limitations of FE is the larger area required for complex circuits compared to traditional silicon-based systems \purple{due to larger feature sizes}, which makes adopting digital processing techniques bulky and costly for certain applications. 
\purple{As FE applications demand greater functionality, digital hardware becomes increasingly challenging in terms of both cost and performance}~\cite{Baruah:FabricationFE2023,Arokia:feAdvantages2012,Ozer:Nature:2020,Sarpeshkar:AnalogvsDigital1998}.}

The ability to approximate functions efficiently is crucial for many FE applications, \purple{such as monitoring biological signals in wearables (e.g., calculating average heart rate or temperature) or detecting environmental parameters in compact IoT sensors (e.g., approximating pollution levels or humidity)~\cite{Gao:FlexibleWearableSensing,Arokia:feAdvantages2012,Haghi:wearableinhealthmonitoring2021}.
Function approximation enables systems to process signals in real-time while using minimal power and hardware. 
However, traditional digital function approximation methods can be too power-hungry and hardware-demanding for small-scale, low-cost FE applications.}
Kolmogorov-Arnold Networks (KANs) offer a powerful framework for function approximation, providing an efficient and systematic way to model complex relationships in data through the use of spline functions~\cite{Liu:KAN2024}. 
\purple{KANs offer the flexibility to approximate any given function, making them ideal for use in a variety of applications~\cite{Sidhart:fuctionapproxKAN2024}.
}

The key challenge, however, lies in efficiently realizing these approximations in hardware, \blue{particularly given the constraints of FE.} While digital implementations of KANs are straightforward, their associated power consumption and hardware costs often present significant barriers, and this has been studied as a co-optimization problem~\cite{VanDuy:KANCodesign:2024,Huang:KANAcceletaror:2024}.
\blue{Analog realizations of KANs must address issues such as maintaining precision, minimizing power consumption, and dealing with parasitic elements inherent to circuit designs.}

This work proposes a novel solution by using analog circuits to implement KANs for function approximation in FE.
\purple{The key feature of our approach is that it is systematic and generic, enabling analog approximations of any function through a standardized design process that brings structure and precision to analog function approximation.
As a result, our approach overcomes the limitations of traditional methods and provides a reliable way to implement analog function approximation in different applications.}
We present the development of Analog Building Blocks (ABBs) that perform fundamental mathematical operations—such as addition, \purple{subtraction}, multiplication, and squaring—which are essential for creating splines.
By using these ABBs, we construct a hardware-efficient KAN implementation that reduces area by 125$\times$ and power consumption by 10.59\% compared to a digital spline with 8-bit precision.
Despite introducing an approximation error of up to 7.58\%, our approach provides a promising pathway for more efficient and scalable function approximation in FE. 

\textbf{In summary, the main contributions of this work are:}

\begin{enumerate} 
    \item Creation of ABBs in FE \purple{to} perform \blue{basic} mathematical operations.
    \item \purple{A systematic and generic approach for creating splines and KANs, enabling analog approximation of any given function.}
    \item Comparison of the digital and analog cost of each spline in hardware, \purple{including the physical implementation of the analog spline}.
    \item An analysis of the hardware error and its impact on the overall network \blue{performance}.
\end{enumerate}

The structure of this paper is as follows: ~\autoref{sec:background}, discusses the background.~\autoref{sec:methodology} presents the methodology we followed from the ABB to the approximation of the splines. ~\autoref{sec:Results} analyzes the cost of our implementation and compares it with the digital cost of it, we present as well the error of the approximation and limitations of the work.
Finally,~\autoref{sec:conclusion} concludes the work with future research directions.