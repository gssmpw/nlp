\section{Conclusion}\label{sec:conclusion}

Flexible Electronics (FE), while flexible and lightweight, impose challenges such as limited conductivity and larger footprints. 
Analog approximations offer a promising solution by reducing both power and area.
This paper explores the potential of analog Kolmogorov-Arnold Networks (KANs) for function approximation in FE, \purple{presenting a systematic and generic design process that enables approximations for any given function.}
By developing Analog Building Blocks (ABBs) for basic operations such as addition, multiplication, and squaring, we created splines that serve as the foundation of the KAN.
Our results show significant hardware improvements, including a 125$\times$ reduction in area and a 10.59\% decrease in power consumption compared to a digital 8-bit spline implementation. 
Although the analog design introduces an approximation error of up to 7.58\%, it does not substantially degrade the overall performance of the system.
This analog approach offers considerable advantages, particularly in applications where power efficiency and area optimization are prioritized over precision.
