\section{Background}\label{sec:background}

\subsection{Flexible Electronics}

FE encompass electronic devices designed to maintain functionality while bending, stretching, or conforming to various surfaces~\cite{FlexibleElectronics}. 
They are typically built on flexible substrates, such as polyimide, plastics, or thin metal foils, enabling them to adapt to a wide range of applications, from wearable health monitors to compact, portable sensors used in consumer and medical electronics~\cite{Heng:AM2022:FlexHumanMachInterfaces,Gao:FlexibleWearableSensing,gao:2016:flexsensor}. 

The manufacturing process for FE \blue{involve} modifications of traditional semiconductor fabrication techniques: 
Instead of rigid substrates, FE are constructed on pliable materials, and certain complex steps—like ion implantation and high-temperature annealing—are often eliminated~\cite{Baruah:FabricationFE2023}.
\blue{For example}, PragmatIC Semiconductor has optimized a streamlined version of silicon lithography specifically for flexible devices and by removing these costly steps, the manufacturing process becomes faster, more cost-effective, and environmentally sustainable due to reduced resource demands~\cite{FlexICs}. 
Furthermore, PragmatIC’s FlexIC technology integrates Thin-Film Transistors (TFTs) on polyimide substrates, achieving critical dimensions of 600 nm. 
This process utilizes a high-k dielectric material that supports standard IC input voltages, enabling compatibility with existing systems while prioritizing ultra-thin and low-energy designs~\cite{FlexICs}.

One of the materials used in FE is Indium Gallium Zinc Oxide (IGZO), an amorphous oxide semiconductor with high electron mobility, transparency, and compatibility with low-temperature processes~\cite{Zhu:IGZO2021,Pan:IGZOTFT2024}. 
These characteristics make IGZO suitable for flexible applications, though its limitations, including its restriction to N-type transistors, impose design constraints~\cite{Jeong:igzoperformance}. 
As such, FE is unlikely to replace silicon-based technologies but to complement them, enhancing hybrid systems where flexibility, low power, and form factor are prioritized over the ultra-precision found in silicon-based circuits~\cite{Lozano:aspdac25:BinCoDesign}.

\blue{FE are well-suited for use cases that require real-time, localized processing in constrained environments, such as monitoring biological signals in wearable health devices or detecting environmental parameters in compact IoT sensors~\cite{Gao:FlexibleWearableSensing,Baruah:FabricationFE2023,Arokia:feAdvantages2012}. 
To meet the specific demands of these applications, FE devices often need efficient function approximation capabilities to perform tasks like signal smoothing, feature extraction, and basic mathematical transformations. 
For instance, wearables may need to compute average heart rate or temperature values, while environmental sensors might use threshold-based approximations for pollution levels or humidity~\cite{Haghi:wearableinhealthmonitoring2021}.

To address the need for efficient, low-power function approximation in FE, we propose an analog approach using KANs, which can effectively model complex, continuous functions in a hardware-efficient manner. 
KANs offer a promising solution for \purple{systematically} implementing these functions in FE, as they allow for real-time analog computation while minimizing the area and power consumption of the system.}

In this work, we use PragmatIC’s FlexLogIC process for rapid, FE production. 
This platform allows for a fast turnaround, typically achieving custom designs in under six weeks~\cite{FlexICs}.
The FlexIC platform provides efficient, pragmatic solutions for FE, especially in applications that benefit from a low-cost, high-volume production model. 

\subsection{Kolmogorov-Arnold Networks}
KANs are inspired \purple{by} Kolmogoro\purple{v}-Arnold representation theorem \cite{theorem}. In~\cite{Liu:KAN2024}, a neural network model is proposed where learnable activation functions are used, unlike \purple{Multilayer Perceptron} (MLP) which has \purple{a} fixed activation function. This replaces the linear weight matrix in MLP with a learnable function dependent on a single variable.
KAN approximates a multivariate continuous function as a composition of continuous functions of a single variable added together. 
To implement these univariate functions, KAN utilizes \purple{Bézier} spline with learnable coefficients~\cite{Liu:KAN2024}. The KAN layer uses the following equation: 
\begin{equation}
\phi(x) = w_b b(x) + w_s \, \text{spline}(x)
\label{KAN}
\end{equation}
where $\phi(x)$ is the sum of basis function b(x) and the spline function. The spline in eq.~\ref{KAN} is a linear combination of \purple{Bézier} splines such that: 
\begin{equation}
\text{spline}(x) = \sum_{i} c_i B_i(x)
\label{eq:spline_sum}
\end{equation}
While the basis function and multiplication can be implemented by multiply and accumulation using a crossbar array \cite{crossbar}, the implementation of spline in analog domain is the focus. The \purple{Bézier} splines have a set of discreet  control points by which the smooth continuous curve is defined. The spline can have n-control points which define the order of the spline. In this work we implement a second-order Bézier \purple{s}pline \cite{secondorderspline} given by: 
\begin{equation}
\mathbf{B}(x) = P_0(1 - x)^2 + 2P_1(1 - x)x + P_2x^2 
\label{eq:bezier_second_order}
\end{equation}
where $P_{0}$, $P_{1}$, $P_{2}$ are the control points of the second order \purple{Bézier s}pline and, \purple{$x$} the input. 
\purple{Choosing a second-order \purple{Bézier} spline reflects a balance between hardware cost and approximation accuracy: while higher-order splines could improve accuracy, they would require more hardware resources, increasing area and power consumption. 
In our case, the second-order spline minimizes hardware and simplifies~\autoref{eq:bezier_second_order} to:}
\begin{equation}
\mathbf{B}(x) = P_0 + (P_1 - P_0)2x + (P_0 - 2P_1 + P_2)x^2
\label{eq:quadratic_formula}
\end{equation}
Eq.~\ref{eq:quadratic_formula} becomes the basis for the implementation of second order \purple{Bézier} splines in the analog domain \purple{using} FE.

\subsection{Related work}

Due to the recent publication of~\cite{Liu:KAN2024}, at the time of this work, only two studies have explored the hardware implementation of KAN.
The first study~\cite{Huang:KANAcceletaror:2024} proposes a mixed analog-digital circuit approach, utilizing Look-up Tables (LUT) and Analog Computing-In-Memory (ACIM). However, LUTs face scalability challenges, as their size grows exponentially with the number of classes, making them memory-intensive.
The second study~\cite{VanDuy:KANCodesign:2024} explores a fully digital circuit implementation\orange{, which also relies on LUTs.}
Both studies incorporate a co-design methodology with software optimization. 
\orange{In this work we do not replicate the designs presented in\cite{Huang:KANAcceletaror:2024} or in\cite{VanDuy:KANCodesign:2024}, as FE require simpler and more resource-efficient designs due to power and area constrains. Instead, to establish a baseline for comparison, we opted for a digital implementation of~\autoref{eq:quadratic_formula} in FE, incorporating the constraints of this technology while excluding the co-optimization strategies used in prior studies.}


Our work presents the first research of a fully analog-domain implementation, leaving the exploration of co-optimization with software for future research. 
\orange{Furthermore, we are the first to explore this design using non-silicon-based technology, which reduces power consumption and requires the creation of simpler and more efficient circuit designs.}