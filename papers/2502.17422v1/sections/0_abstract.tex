\begin{abstract}
Multimodal Large Language Models (MLLMs) have experienced rapid progress in visual recognition tasks in recent years. Given their potential integration into many critical applications, it is important to understand the limitations of their visual perception. In this work, we study whether MLLMs can perceive small visual details as effectively as large ones when answering questions about images. We observe that their performance is very sensitive to the size of the visual subject of the question, and further show that this effect is in fact causal by conducting an intervention study. Next, we study the attention patterns of MLLMs when answering visual questions, and intriguingly find that they consistently know where to look, even when they provide the wrong answer. Based on these findings, we then propose training-free visual intervention methods that leverage the internal knowledge of any MLLM itself, in the form of attention and gradient maps, to enhance its perception of small visual details. We evaluate our proposed methods on two widely-used MLLMs and seven visual question answering benchmarks and show that they can significantly improve MLLMs' accuracy \emph{without requiring any training}. Our results elucidate the risk of applying MLLMs to visual recognition tasks concerning small details and indicate that visual intervention using the model's internal state is a promising direction to mitigate this risk.\footnote{Our code is available at~\url{https://github.com/saccharomycetes/mllms_know}.}
\end{abstract}