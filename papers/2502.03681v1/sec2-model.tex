% !TEX root=root.tex
% ##############################################################################################
% MODEL
% ##############################################################################################
\section{MODEL}\label{sec:Model}
To model the effects of angular motion around a fixed axis on the estimation, we consider a pendulum as illustrated in Fig.~\ref{IMG:Setup:Pendulum}.
The pendulum generalizes many types of rotational motions, which are not centered around the rotational axis, e.g., loops in aerobatics, joint motions in exoskeletons, or curve driving in vehicles \cite{Nazarahari2021, Gajamohan2012}, such that our results can be easily transferred to different setups.
The \ac{IMU} is placed at a fixed distance $l$ to the point $O$ and rotates with respect to the roll angle $\varphi$ while pitch $\theta$ and yaw $\psi$ are held at zero.
Hence, any acceleration in $\varphi$ causes a linear acceleration that can be measured by the accelerometer of the \ac{IMU}.

We align the rotation of the pendulum with the roll angle to investigate whether an acceleration in $\varphi$ influences the pitch and yaw estimation.
Note that the same analysis can be performed around an arbitrary axis, except for the yaw component, which cannot be estimated using accelerometers.
Reconstruction of the yaw angle requires additional sensors, e.g., a magnetometer, which also provides redundancy for estimating the pitch angle~\cite{Madgwick2011}. 
An investigation including magnetometers is beyond the scope of this paper and is left for future research.
The \ac{IMU} is able to measure the linear acceleration $a^K$ including gravity and the angular velocity $\omega^K$, both expressed in the body fixed coordinate system $K$, which is rotated with respect to the inertial frame $I$. 
This results in the measurement equations
\begin{equation}
	a^K = \begin{pmatrix}
		0 \\
		\sin(\varphi) - \frac{l}{g}\ddot{\varphi} \\
		\cos(\varphi) - \frac{l}{g}\dot{\varphi}^2
	\end{pmatrix} 
	,\qquad 
	\omega^K = \begin{pmatrix}
		\dot{\varphi}\\ 
		0 \\
		0
	\end{pmatrix}.
\end{equation}
These acceleration measurements are already expressed in units of standard gravity [$g$] and we assume that $\varphi$ is twice differentiable to ensure the existence of $\ddot{\varphi}$.
Since the acceleration depends only on the ratio of $l$ and $g$, we normalize $g$ to 1 in the following. 
Further, a common assumption in literature is that the \ac{IMU} is at rest, i.e., $\dot{\varphi}=\ddot{\varphi}=0$ such that only gravity affects $a^K$~\cite{Madgwick2011, Mahony2008, Ludwig2018a}.
Hence, it is possible to compute $\varphi$ by applying a simple trigonometric function to $a^K$.
In the remainder of the paper, we analyze the effects when the \ac{IMU} is not at rest.
\begin{figure}[tb]
	\centering
	\begin{tikzpicture}[scale=1]
	
	\def\varphiang{-30}
	\def\height{4}
	\def\rArc{0.5*\height}
	
	\def\groundlength{1}
	\def\groundheight_plot{0.12}
	
	\def\supportLength{0.4}
	
	\coordinate (origin) at (0,0);
	\coordinate (originKOS) at ($(origin)$); %-(0.5*\groundlength,0)$);
	\coordinate (IMU) at ($(origin) + ({\height*sin(\varphiang)},{\height*cos(\varphiang)})$);
	
	% Pendulum
	\draw[thick] (origin) -- (IMU) node[midway,left] {$l$};
	
	\draw[dashed](origin) --++ ($({0},{\height*cos(\varphiang)})$);
	
	\draw[-Stealth] ($(origin) + ({0},{\rArc})$) arc(90:90-\varphiang:\rArc) node[midway, above]{$\varphi$};
	
	% IMU
	\node[draw,fill = white, rotate = -\varphiang] at (IMU) {IMU};
	
	% Coordinate System
	\draw[-Stealth,thick] (originKOS) --++ (1,0) node[below]{$e_y^I$};
	\draw[-Stealth,thick] (originKOS) --++ (0,1) node[right]{$e_z^I$};
	
	\draw[-Stealth,thick] (originKOS) --++ ($({cos(\varphiang)},{-sin(\varphiang)})$) node[above]{$e_y^K$};
	\draw[-Stealth,thick] (originKOS) --++ ($({sin(\varphiang)},{cos(\varphiang)})$) node[left]{$e_z^K$};
	
	\fill (origin) circle(1.5pt) node[yshift = -2, xshift = -2, above left]{$O$};
	
	% Ground
	
	
	\coordinate (groundStart) at ($(origin)-(0,{\supportLength*sin(60)})-({0.5*\groundlength},0)$);	
	
	\fill[pattern=north east lines] (groundStart) rectangle ++(\groundlength,-\groundheight_plot);	
	\draw[thick] (groundStart) -- ++(\groundlength,0);	
	
	% Support
	\draw[thick] (origin) --++ ($({\supportLength*cos(60)}, {-\supportLength*sin(60)})$) --++ ($(-\supportLength,0)$) -- (origin);
	
	% Gravity
	\draw[thick,-stealth] ($(origin)+({0.5*\groundlength},{\height*cos(\varphiang)})$) --++(0,-1) node[midway, left]{$\mathrm{g}$};
	
\end{tikzpicture}

	\caption{Pendulum with \acs{IMU} including gravity g.}
	\label{IMG:Setup:Pendulum}
\end{figure}
