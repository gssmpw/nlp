% !TEX root=root.tex
% ##############################################################################################
% Filter Analysis
% ##############################################################################################
\section{FILTER ANALYSIS} \label{sec:Analysis}
Next, we use quaternion-based formulations of the dynamic filters. 
Using~\cite{Berner2007}, it is possible to transform quaternions back and forth between roll, pitch and yaw angles. In particular, we obtain
\begin{subequations}
	\begin{align}
		\varphi &= \mathrm{atan2}\left(q_1 q_2 + q_3 q_4,\tfrac{1}{2}-q_2 ^2 - q_3 ^2\right), \label{eq:analysis:phi}\\
		\theta &= \mathrm{arcsin}\left(2(q_1 q_3 - q_2 q_4)\right), \\
		\psi &= \mathrm{atan2}\left(q_1 q_4 + q_2 q_3 ,\tfrac{1}{2}-q_3 ^2 - q_4 ^2\right)
	\end{align}	
\end{subequations}
with $-\pi<\psi\leq\pi$, where 
\begin{subequations}
	\begin{align}
		q_1 &= \cos(\tfrac{\psi}{2})\cos(\tfrac{\theta}{2})\cos(\tfrac{\varphi}{2})+\sin(\tfrac{\psi}{2})\sin(\tfrac{\theta}{2})\sin(\tfrac{\varphi}{2}), \\
		q_2 &= \cos(\tfrac{\psi}{2})\cos(\tfrac{\theta}{2})\sin(\tfrac{\varphi}{2})-\sin(\tfrac{\psi}{2})\sin(\tfrac{\theta}{2})\cos(\tfrac{\varphi}{2}), \\
		q_3 &= \cos(\tfrac{\psi}{2})\sin(\tfrac{\theta}{2})\cos(\tfrac{\varphi}{2})+\sin(\tfrac{\psi}{2})\cos(\tfrac{\theta}{2})\sin(\tfrac{\varphi}{2}), \\
		q_4 &= \sin(\tfrac{\psi}{2})\cos(\tfrac{\theta}{2})\cos(\tfrac{\varphi}{2})-\cos(\tfrac{\psi}{2})\sin(\tfrac{\theta}{2})\sin(\tfrac{\varphi}{2}).
	\end{align}
\end{subequations}
A common preprocessing step before applying the measurements to any filter is to first normalize the acceleration measurements~\cite{Mahony2008,Madgwick2011,Euston2008} as
\begin{equation}
	\bar{a}^K = \frac{a^K}{\|a^K \|}.
\end{equation}

% ##############################################################################################
\subsection{One-dimensional case}\label{sec:Analysis:OneDim}
Before moving to quaternion based formulations for rotation in three dimensions, we consider the one-dimensional case. 
Later, we show that for quaternion formulations, accelerations in the roll angle do not influence the estimate of the yaw and pitch.
In particular, we estimate the angle $\varphi$ solely based on $\bar{a}^K$.
If the \ac{IMU} is at rest, i.e., $\dot{\varphi}=\ddot{\varphi}=0$, it is possible to recover $\varphi$ exactly as
\begin{equation}
	\varphi = \mathrm{atan2}(\bar{a}^K_2, \bar{a}^K_3).
\end{equation}
To investigate the effects of $\dot{\varphi}$ and $\ddot{\varphi}$, we linearize around an operating point $\varphi=\varphi_{\mathrm{op}}$ with $\dot{\varphi}=\ddot{\varphi}=0$ and apply the Laplace transform.
This leads to the acausal dependence
\begin{equation}\label{eq:Ana:tfAtan}
	\Delta \hat{\varphi}(s) = (1-l\cos(\varphi_{\mathrm{op}})s^2) \Delta \varphi(s)	
\end{equation}
in the frequency domain with zeros $z_{1,2}=\pm\frac{1}{\sqrt{l\cos(\varphi_{\mathrm{op}})}}$ for the deviation from the operating point.
Fig.~\ref{IMG:Ana:ZerosAtan} shows the zeros for different operating points.
For $\varphi_{\mathrm{op}} = 0$, we obtain two zeros at $\pm \frac{1}{\sqrt{l}}$. 
This corresponds to one real unstable and one real asymptotically stable zero, leading to deviations from the desired constant transfer function $G_{\Delta \varphi \to \Delta \hat{\varphi}}(s) \equiv 1$.
For low frequencies, the zeros do not affect the estimate.
At high frequencies, however, the zeros result in an amplification of the magnitude.
Due to the mirrored zeros, the phase changes of the zeros cancel each other out, such that there is no phase change for all frequencies.
As $\varphi_{\mathrm{op}}\to\frac{\pi}{2}$, the mirrored zeros remain, but move farther away from the origin, resulting in a transfer function closer to $1$ for the intermediate frequencies.
For $\varphi_{\mathrm{op}}=\frac{\pi}{2}$ the term $l\cos(\varphi_{\mathrm{op}})$ vanishes.
Hence, the transfer function becomes exactly $G_{\Delta \varphi \to \Delta \hat{\varphi}}(s) \equiv1$ and there is no effect of $\ddot{\varphi}$.

For the operating point $\varphi_{\mathrm{op}}\in (\frac{\pi}{2},\pi]$, there is a qualitative change of the locations of the zeros due to the $z$-axis of the \ac{IMU} coordinate system $K$ no longer pointing upwards but downwards.
More precisely, the zeros change from being real numbers to being purely imaginary and tend towards $\pm i \frac{1}{\sqrt{l}}$ as $\varphi_{\mathrm{op}}\to\pi$.
As with the upper position $\varphi_\mathrm{op}=0$ , the magnitude of high frequency signals is getting amplified due to $\ddot{\varphi}$.
However, there is a phase change of $180^\circ$ for frequencies larger than $\SI[parse-numbers = false]{\frac{1}{\sqrt{l}}}{\radian\per\second}$, i.e., the sign of the roll angle $\varphi$ and its estimate $\hat{\varphi}$ differ in sign and magnitude.
Further, sinusoids with a frequency $\SI[parse-numbers = false]{\frac{1}{\sqrt{l}}}{\radian\per\second}$ are getting canceled exactly by the zero, leading to $\hat{\varphi}=0$.
%
Note that the zeros scale proportionally to $\frac{1}{\sqrt{l}}$.
Hence, reducing $l$ also shifts the zeros to higher frequencies, leading to a better estimate in the intermediate frequencies.
This behavior is expected, since reducing the distance between the $\ac{IMU}$ and the rotation axis also decreases the effect of $\ddot{\varphi}$.

After establishing the most basic estimation procedure, we extend this to the simultaneous estimation of all angles and incorporating the additional measurements of the gyroscope.
To this end, we consider two common filters, the Mahony filter~\cite{Mahony2008} and the Madgwick filter~\cite{Madgwick2011}, and discuss how the corresponding tuning parameters can be used to improve the rejection of angular accelerations.

\begin{figure}[tb]	
	\centering	
	\includegraphics[scale = 0.5]{img/AtanZeros.eps} 
	\caption{Location of the zeros for different operating points $\varphi_{\mathrm{op}}\in [ 0,\pi ]$ using~\eqref{eq:Ana:tfAtan}.}
	\label{IMG:Ana:ZerosAtan}
\end{figure}	

% ##############################################################################################
\subsection{Mahony filter}
The Mahony filter is a particular filter structure based on proportional $k_p$ and integral $k_i$ gains, such that the estimated quaternion $\hat{q} $ tracks the quaternion that rotates the gravitational vector into $\bar{a}^K$.
The resulting filter is described by the differential equation~\cite{Mahony2008}
\begin{equation}\label{eq:Mahony-ode}
	\begin{pmatrix}
		\dot{\hat{q}}  \\
		\dot{\zeta}
	\end{pmatrix} 
	= 
	\begin{pmatrix}
		\frac{1}{2} \hat{q} \otimes \begin{pmatrix}0 \\ \omega^K + \zeta\end{pmatrix} \\
		0_{3\times 1}
	\end{pmatrix}+
	\begin{pmatrix}
		\frac{1}{2} \hat{q} \otimes \begin{pmatrix}0 \\ k_p e\end{pmatrix} \\
		k_i e
	\end{pmatrix}
\end{equation}
with
\begin{equation}
	e = \bar{a}^K \times  \left(\hat{q} ^\star \otimes \begin{pmatrix}0_{3\times1} \\ 1 \end{pmatrix} \otimes \hat{q} \right)_{2:4}.
\end{equation}
Note that this filter can also be interpreted as a disturbance observer~\cite{Andrievsky2020}.
The first summand of~\eqref{eq:Mahony-ode} represents the nominal prediction for a constant disturbance in $\omega^K$ and the second summand characterizes the correction due to the observed measurement error $e$.

To analyze the estimator, we linearize the filter as before around $\varphi=\varphi_{\mathrm{op}}$ with $\dot{\varphi}=\ddot\varphi=0$ to obtain a linear differential equation with the linearized roll, pitch, and yaw angle of the 
form
\begin{equation}\label{eq:Ana:ODEMahony}
	\begin{aligned}
		\Delta\dot{\hat{x}} 	&= A\Delta \hat{x} + B \Delta \bar{a}^K, \\
		\Delta y 		&= C\Delta \hat{x}.		
	\end{aligned}
\end{equation}
with $\Delta\hat{x}$ as the stacked linearized estimates $\Delta\hat{q}$ and $\Delta\hat{\zeta}$. Note that other estimation schemes, such as Luenberger observers and, as a special case, the Kalman filter, exhibit the same linear form after linearization, for which $k_p$ and $k_i$ are a particular choice of the observer gains.
In the following, we keep $k_p$ and $k_i$ for easier interpretation.
System~\eqref{eq:Ana:ODEMahony} can also be represented via its corresponding transfer function. 
In particular, after inserting the linearized $\Delta \bar{a}^K$, we get the frequency-domain representation%
\small
\begin{equation*}
	\hspace*{-0.015\linewidth}
	\begin{pmatrix}
		\Delta \hat{\varphi}(s) \\
		\Delta \hat{\theta}(s) \\
		\Delta \hat{\psi}(s)
	\end{pmatrix} 
	\!=\! 
	\begin{pmatrix}
		\frac{k_i + k_ps + (1-k_il\cos(\varphi_{\mathrm{op}}))s^2 - k_pl\cos(\varphi_{\mathrm{op}})s^3}{k_i + k_ps + s^2} \\ 
		0 \\
		0 
	\end{pmatrix} \Delta \varphi(s).
\end{equation*}
\normalsize
We observe that the second and third entries are zero and, hence, there are no coupling effects between the roll, pitch, and yaw angles due to a motion in $\varphi$.
Therefore, we restrict our analysis in the following to the transfer function from $\Delta\varphi$ to $\Delta\hat{\varphi}$.
As expected, the poles depend on $k_p$ and $k_i$, i.e., the gains must be suitably chosen to ensure asymptotic stability.
However, the zeros depend on $l$ and the chosen operating point, i.e., they cannot be influenced via the filter gains.
Hence, we investigate the effects on the zeros in the Mahony filter setup.
Interestingly, for $\varphi_{\mathrm{op}}=\frac{\pi}{2}$ the nominator and the denominator cancel each other out, resulting in the transfer function $G_{\Delta \varphi \to \Delta \hat{\varphi}}(s) \equiv 1$, which coincides with the results discovered for the one-dimensional case.
In particular, $\ddot{\varphi}$ has no effect on the estimate around $\varphi_\mathrm{op}=\frac{\pi}{2}$.
Further, we set $k_i=0$, i.e., we include no integrator in the filter.
This simplification is deployed since integrators are used to compensate for low frequency effects such as a constant bias in the gyroscope.
Thus, we can analyze the effects of only one design parameter $k_p$.

Exploiting the prior discussion, the zeros of the transfer function $G_{\Delta \varphi \to \Delta \hat{\varphi}}(s)$ are determined by
\begin{equation}
	k_p + s  - k_p l \cos(\varphi_{\mathrm{op}}) s^2 = 0
\end{equation}
with $k_p>0$ to ensure asymptotic stability.
The corresponding zeros can be calculated analytically as
\begin{equation}
	z_{1,2} = \frac{-1\pm\sqrt{1+4k_p^2l\cos(\varphi_{\mathrm{op}})}}{2k_p l \cos(\varphi_{\mathrm{op}})}.
\end{equation}
As before, we consider separately two ranges for the operating point.
If $\varphi_{\mathrm{op}}\in[0,\frac{\pi}{2})$, all zeros are on the real axis without imaginary part.
For $k_p$ close to zero, one zero approaches $0$ and is canceled by the corresponding pole.
The other zero approaches infinity and does not affect the relevant frequencies, leading to a transfer function very close to $1$.
This is expected, since by setting the gains to zero the filter only integrates the angular velocity, resulting in the true angle.
However, the filter is not asymptotically stable, such that constant biases in the gyroscope will also be integrated without any corrections.
As $k_p$ approaches infinity, the filter tends aggressively towards the estimate based on the acceleration leading to the zeros $\pm\frac{1}{\sqrt{l\cos(\varphi_{\mathrm{op}})}}$.
Interestingly, these are identical to the zeros of the one-dimensional case in Section~\ref{sec:Analysis:OneDim}.
Note that computing $\arccos(\bar{a}^K_3)$ also returns $\varphi$, if the \ac{IMU} is at rest, while not directly depending on $\ddot{\varphi}$.
Hence, the fact that the zeros align is a particular property of the $\mathrm{atan2}$.

For $\varphi_{\mathrm{op}}\in[0,\frac{\pi}{2})$, there is no qualitative change.
In particular, we get one zero with a negative real part and one with a positive real part, but both are shifted further to the right.
Hence, the asymptotically stable zero is moved closer to the origin and already affects the estimation for lower velocities.
However, the unstable zero is farther away from the zero and hence, affects the estimation only for higher velocities.
In contrast, operating points $\varphi_{\mathrm{op}}\in(\frac{\pi}{2},\pi]$ change the qualitative behavior. 
The zeros are no longer on the imaginary axis, but are shifted to the left half plane.
Further, for $k_p \leq \frac{1}{\sqrt{2l\cos(\varphi_{\mathrm{op}})}}$, the zeros remain real and for $k_p > \frac{1}{\sqrt{2l\cos(\varphi_{\mathrm{op}})}}$ they get an imaginary part approaching $\pm i \frac{-1}{\sqrt{l\cos(\varphi_{\mathrm{op}})}}$ as in the one-dimensional case.
A visualization on how the zeros can be influenced using $k_p$ for $\varphi_{\mathrm{op}} = 0$ and $\varphi_{\mathrm{op}}=\pi$ is given in Fig.~\ref{IMG:Ana:ZerosMahony}.
In both cases, $k_p$ must be chosen as a trade-off between fast poles for fast convergence and rejecting the effects of $\ddot{\varphi}$ by moving the zeros further away from the imaginary axis.

\begin{figure}[tb]
	\centering
	\includegraphics[scale=0.5]{img/MahonyZeros.eps}
	\caption{Location of the zeros 	for $\varphi_{\mathrm{op}}=0$ and $\varphi_{\mathrm{op}}=\pi$ for different $k_p$.}
	\label{IMG:Ana:ZerosMahony}
\end{figure}

% ##############################################################################################
\subsection{Madgwick filter}
In contrast to the Mahony filter, which is designed as a PI-controller, the Madgwick filter can be interpreted as an optimization problem to align the gravitational reference direction with the measured acceleration.
More precisely, the Madgwick filter uses a gradient descent algorithm with feedforward compensation for $\omega^K$ to minimize the cost function 
\begin{equation}
	f(\hat{q}, \bar{a}^K) = \left\|\hat{q} ^\star \otimes \begin{pmatrix}0_{3\times1} \\ 1 \end{pmatrix} \otimes \hat{q} - \begin{pmatrix} 0 \\ \bar{a}^K\end{pmatrix}\right\|^2
\end{equation}
such that each iteration yields a new quaternion estimate~\cite{Madgwick2011,Ludwig2018a}
\begin{equation}\label{eq:ODE-Madgwick}
	\dot{\hat{q}} = \frac{1}{2} \hat{q} \otimes \begin{pmatrix}0 \\ \omega^K \end{pmatrix} -
	\beta \frac{\nabla f(\hat{q}, \bar{a}^K)}{\| \nabla f(\hat{q}, \bar{a}^K)\|}
\end{equation}
with step size $\beta$.
In contrast to the Mahony filter, the differential equation~\eqref{eq:ODE-Madgwick} cannot be linearized directly due to the normalization of $\nabla f$ leading to a discontinuity at $\nabla f = 0$. 
This non-differentiability is similar to sliding mode observers with sliding surface $\nabla f=0$.
For the following analysis, we first assume $\beta$ to be sufficiently large to ensure that the filter is tracking $\nabla f=0$ exactly.
To analyze the behavior, we consider the sliding surface and linearize it around an operating point. 
Without loss of generality, we can fix the yaw $\psi=0$ and the pitch $\theta=0$, which leads directly to $q_{\mathrm{op},3}=q_{\mathrm{op},4}=0$ as operating points. 
Together with $\bar{a}^K_1=0$, this yields the linearized representation
\begin{equation*}
	\nabla f \approx 
	\begin{pmatrix}
		4 & 0 &     0 &     0\\
		0 & 4 &     0 &     0\\
		0 & 0 & \star & \star\\
		0 & 0 & \star & \star
	\end{pmatrix}\Delta \hat{q} +
	\begin{pmatrix}
		-2 q_{\mathrm{op},2} & -2q_{\mathrm{op},1} \\
		-2 q_{\mathrm{op},1} & 2q_{\mathrm{op},2} \\
		                   0 & 0 \\
		                   0 & 0
	\end{pmatrix}
	\Delta \bar{a}^K_{2:3},
\end{equation*}
where we represent irrelevant entries by $\star$.
In order to investigate the effects of the \ac{IMU}, we set all initial conditions to zero and deduce $\Delta\hat{q}_3 = \Delta\hat{q}_4 = 0$. 
Now, we can solve for $\Delta q_1$ and $\Delta q_2$ on the sliding surface $\nabla f = 0$.
Further, when linearizing the transformation from quaternion back to roll, pitch, and yaw around the operating point, it follows from $\frac{\partial \theta}{\partial q_1}=\frac{\partial \theta}{\partial q_2} = \frac{\partial \psi}{\partial q_1} = \frac{\partial \psi}{\partial q_2} = 0$ that the pitch and yaw are independent on the acceleration measurements. 
Hence, as with the Mahony filter, we deduce that violating the rest condition for one angle does not affect the estimation of the other angles.
This can finally be inserted into the linearized equation~\eqref{eq:analysis:phi} for $\varphi$ such that the angle can be estimated by
\begin{equation}
	\Delta \hat{\varphi}(s) = (1-l\cos(\varphi_{\mathrm{op}})s^2) \Delta \varphi(s).
\end{equation}
This estimate is equivalent to the one in~\eqref{eq:Ana:tfAtan} for the one-dimensional case. 
Now, we can apply the previous analysis to this sliding surface.
As with the Mahony filter, we can use the measurement of the angular velocity to reduce the effects of $\ddot{\varphi}$.
For the simplest case $\beta=0$, the filter integrates the gyroscope measurements without any error correction. 
Due to the non-differentiable dynamics, a detailed analysis is not as readily available as for linear systems and the effects of frequency and amplitude cannot be decoupled.
For example, the difference between sliding surface and true angle increases for 1) larger $\ddot{\varphi}$, e.g., faster oscillations, and 2) by scaling $\varphi$ with a constant factor.
Due to the normalization, the filter can only move to the sliding surface with a speed of at most $\beta$.
If the sliding surface changes faster, the filter cannot track it accurately.
However, a small $\beta$ generally leads to a filter which relies more on the integrated gyroscope values, and a larger $\beta$ yields an estimate that is closer to the sliding surface and may also compensate for the bias in the gyroscope measurements.
Hence, the step size $\beta$ acts as a tuning parameter to trade off fast convergence and rejection of the effects of $\ddot{\varphi}$.


% ##############################################################################################
\subsection{Discussion}
We emphasize again that the difference between $\varphi$ and $\hat{\varphi}$ is not caused by an independent, external disturbance.
For the linear case and, thus, also for the nonlinear case close to an operating point, such a signal cannot change the stability properties in a feedback system. 
In particular, $\ddot{\varphi}$ cannot be interpreted as an independent, external signal, since $\ddot{\varphi}$ and $\varphi$ depend on each other.
As shown before, this introduces additional zeros to the transfer function of the estimator, which changes the ideal transfer function of the estimator away from $1$ for higher frequencies.
This validates and explains previous observations in, e.g.,~\cite{Nazarahari2021} that the filters commonly have to be tuned for the corresponding application.
More precisely, our analysis uncovers that the behavior can drastically change depending on the operating point.

The zeros of feedback systems can have a significant effect on the closed loop.
In particular, the unstable zero around the upper equilibrium of the pendulum leads to non-minimum phase systems. 
Thus, the controller design becomes challenging and the achievable performance is limited when using the estimate in the controller design~\cite{Qiu1993,Freudenberg1985,Cheng1980,Misra1989}.
Asymptotically stable zeros, on the other hand, also degrade the estimate, but they introduce additional phase into the system, which can be used to increase the phase margin and, hence, the robustness.
An example of this are lead compensators that employ asymptotically stable zeros due to their phase increase.
Note that canceling undesirable zeros is only possible for zeros in the left-half plane by adding a corresponding pole.
While canceling unstable zeros leads to a loss of internal stability and should be avoided, canceling asymptotically stable zeros should also be done carefully.
Further, adding a slow pole for cancellation can help with the effects of $\ddot{\varphi}$, but may decrease the overall estimator performance.
We emphasize that the location of zeros cannot be modified by feedback.

Our analysis suggests several possibilities to address these issues.
First, when designing the system, the \ac{IMU} should be placed close the the rotational axis.
This reduces $l$ and moves the zeros to a higher frequency range, such that for intermediate frequencies the estimate remains accurate.
Secondly, the filter parameters can be chosen to rely less on the accelerometer and more on the gyroscope. 
For the Mahony and the Madgwick filters, this can be done by reducing the corresponding gains $k_p$ and $\beta$, but at the cost of reduced bandwidth.
Moreover, a model can be used to account for $\ddot{\varphi}$, e.g.,~\cite{Gajamohan2012} uses multiple \acp{IMU} with known kinematics to eliminate $\ddot{\varphi}$ from the measurements.
Further, the numerical differentiation of the gyroscope measurement can be used to calculate $\ddot{\varphi}$, and if $l$ is known, it is possible to compensate for the angular acceleration.
However, this requires a sufficiently accurate gyroscope.
Lastly, when designing a controller, the filter dynamics must be taken into account to avoid loss of stability. 
Our analysis clearly shows that an independent design of estimator and controller, as allowed by the separation principle for linear systems, is not applicable for the considered setup.
