% !TEX root=root.tex
% ##############################################################################################
% Introduction
% ##############################################################################################

\section{INTRODUCTION}
Determining the rotation of a rigid body is a common problem in engineering and finds application in robotics, unmanned vehicles, human motion tracking, and quadcopters~\cite{Ludwig2018a, Nazarahari2021}.
One possibility to estimate the orientation is to use an \ac{IMU} consisting of an accelerometer and a gyroscope.
If available, an additional magnetometer can also be employed.
By designing a sensor fusion algorithm such as the Mahony filter~\cite{Mahony2008}, the Madgwick filter~\cite{Madgwick2011}, or a Kalman filter~\cite{Ludwig2018a}, these measurements can be combined to obtain an accurate estimate of the true orientation.
Since the mere integration of the angular velocities measured by the gyroscope is inaccurate due to drift, the accelerometer can be used to improve the estimate of the true orientation.
This approach is based on the assumption, that the \ac{IMU} is at rest, i.e., the only acceleration affecting the sensor is gravity.
Since this acceleration is known in magnitude and direction, it is possible to determine the angular displacement of the sensor with respect to a reference frame.

A known challenge is that if additional external accelerations affect the system, the filter performance deteriorates due to the \ac{IMU} not being at rest.
There already exist a vast number of ways to address this issue, such as optimizing over the filter parameters~\cite{Ludwig2018}, compensating for this acceleration through a model of the application~\cite{Ahmed2017, Briales2021}, or using  an adaptive scheme as in~\cite{Wei2025, Candan2021, Makni2016, Park2020}.
Model-based schemes have the disadvantage of requiring a model.
Hence, off-the-shelf filters cannot be used.
Meanwhile, adaptive schemes treat the acceleration disturbance as an external signal independent of the quantities to be estimated.
All of these works consider the estimator as an independent element.
When the filter is used in a closed loop together with a controller, an independent external disturbance cannot destabilize the system, at least for linear systems and, by extension, nonlinear systems close to an operating point.
However, if the accelerometer is not placed at the rotational axis, any angular acceleration also results in a linear acceleration, which depends on the to be estimated angles themselves.
This leads to a different dynamical system with different properties, which may cause instabilities.

One such example with angular accelerations is the Cubli~\cite{Gajamohan2012}.
Due to physical limitations, it is impossible to place the \ac{IMU} exactly on the rotational axis, leading to linear accelerations due to rotational motion.
Interestingly, in this example, this acceleration can be compensated for by employing multiple \acp{IMU}~\cite{Gajamohan2012, Trimpe2010}.
However, there is no analysis of the effects of why this compensation is indeed necessary, besides the need to compensate for all accelerations except the gravitational one.

In this work, we investigate the effects of such an additional acceleration caused by rotational motion around a lever arm.
As our main contribution, we show that the linear accelerations caused by angular accelerations lead to a qualitative change of the estimation algorithm by adding additional zeros to the transfer function of the filter.
In addition, we show how this change can negatively affect feedback systems if not accounted for.
Moreover, we investigate how the filter parameters can be used to mitigate the undesired behavior and what trade-off has to be made in suppressing the effects of the angular acceleration.
In particular, our analysis offers insights and tuning guidelines for the Mahony and Madgwick filters, which are widely used in practice.
Then, we verify the presented investigations on a real system, where we are able to recover and demonstrate the discussed behavior.
We expect those insights to be transferable to other practical applications, allowing for an improved estimation using \acp{IMU}.

The paper is structured as follows.
In Section~\ref{sec:Model}, we present a model to represent angular accelerations and provide the corresponding measurement equations of the \ac{IMU}.
Section~\ref{sec:Analysis} analyzes two common filter methods, when they are applied to the presented model to estimate the orientation.
Finally, we validate our findings on an autonomous \ac{e-scooter} in Section~\ref{sec:Experiment} \cite{Soloperto2021}, before concluding the paper in Section~\ref{sec:Outlook}.


\emph{Notation:}
We denote the $n\times m$ zero matrix as $0_{n\times m}$, where we omit the indices if the dimensions are clear from context.
Moreover, we use $q\in\mathbb{R}^4$ to denote a unit quaternion and $\hat{q}\in\mathbb{R}^4$ for its estimate. 
In addition, $\otimes$ denotes quaternion multiplication and $q^\star$ the conjugate of a quaternion $q$, cf.\ \cite{Berner2007}.
Further, $a \times b$ is the vector product of $a\in\mathbb{R}^3$ and $b\in\mathbb{R}^3$.
We write $a_{i:j}$ for the $i$th to $j$th component of the vector $a\in\mathbb{R}^n$, where $1\leq i<j\leq n$. 
Finally, $\alpha = \mathrm{atan2}(y,x)$ is the angle $\alpha$ between the positive $x$-axis and the connection from the origin to the point $(x,y)$ in the Cartesian plane, where $-\pi<\alpha\leq \pi$.