\appendix


\section{Appendix}
\label{sec:appendix}

\subsection{Case Study on Meta Constraint}
Figure~\ref{fig: Example} shows an example on the ``Limited Text'' meta constraint, which aims at controlling the length of the generated response and ensuring it will not exceed the specified word limit.  In this case, a simple Python function is used to count the words and verify if the response adheres to the word limit defined by the variable `VAR1'. 
Figure~\ref{fig: example2} shows an example on the ``Limited Structure'' meta constraint, which enforces a specific format in the generated output, requiring it to be formatted in the JSON format.


\begin{figure*}[t!]
\centering
\begin{tcolorbox}[colback=gray!00,%gray background
                  colframe=black,% black frame colour
                  width=16cm,% Use 8cm total width,
                  arc=1.5mm, auto outer arc,
                  left=0.9mm, right=0.9mm,
                  boxrule=0.9pt, colbacktitle = black!65!black
                 ]
\ding{228} ``Category'': ``Limited Text''

\ding{228} ``Level'': 1

\ding{228}  ``Vars'': [{``name'': ``VAR1'', ``type'': ``int'', ``values'': [30, 50, 100]}]

\ding{228} ``Explaination'': ``Use a word count function to verify the response does not exceed the specified limit of VAR1 words.''

\ding{228} ``Python Script''
\begin{verbatim}
def verify_response_limit(response_text, vars, type=0):
    word_limit = int(vars[0])
    word_count = len(response_text.split())
    meets_criteria = word_count <= word_limit
    if type==0:
        return meets_criteria
    else:
        if meets_criteria:
            return 1
        else:
            return 1-(word_count-word_limit)/word_limit
\end{verbatim}


\end{tcolorbox}
\caption{Example of a Limited Text category meta constraint.}
\label{fig: Example}
\end{figure*}






\begin{figure*}[t!]
\centering
\begin{tcolorbox}[colback=gray!00,%gray background
                  colframe=black,% black frame colour
                  width=16cm,% Use 8cm total width,
                  arc=1.5mm, auto outer arc,
                  left=0.9mm, right=0.9mm,
                  boxrule=0.9pt, colbacktitle = black!65!black
                 ]

\ding{228} ``Category'': ``Limited Structure''

\ding{228} ``Level'': 1

\ding{228} ``Vars'': []

\ding{228} ``Explaination'': ``Entire output should be wrapped in JSON format.''

\ding{228} ``Python Script''
\begin{verbatim}
def verify_json_format(response_text, vars, type=0):
    try:
        response_text=fr'''{response_text}'''
        json_object = json.loads(response_text)
    except ValueError:
        return False
    return True
\end{verbatim}


\end{tcolorbox}
\caption{Example of Limited Structure category in our proposed Meta Constraint.}
\label{fig: example2}
\end{figure*}



\subsection{Case Study on Format Control Instruction}
The format control instructions are instantiated from the meta constraints and general instructions of Alpaca. We add an increasing number of constraints (up to 3) to form instructions of increasing complexity (levels 1 to 3). 
Figure~\ref{fig: 1-level} illustrates an instruction of generating a 3D house model under a single format constraint. The instruction requires the output to be presented in one paragraph and limited to five sentences.
Figure~\ref{fig: 2-level} shows a level-2 instruction constraint where the task is to generate three verbs synonymous with ``to apologize,'' and the response must consist of exactly three sentences. 
Figure~\ref{fig: 3-level} presents an instruction involving three constraints: sentence starting with a specific letter, the response should be exactly 6 paragraphs, and the response must contain between 50 and 110 words. 
Initially, the LLM often struggles to meet all output format constraints, as demonstrated in the failed responses. After fine-tuning, the LLM improves its format control generations and can properly respond to the constraints. 
It is noteworthy that sometimes the format constraints may have conflicts with the original instruction, which may result in negative effects for fine-tuning the model.

\begin{figure*}[ht]
\centering
\begin{tcolorbox}[colback=gray!00,%gray background
                  colframe=black,% black frame colour
                  width=16cm,% Use 8cm total width,
                  arc=1.5mm, auto outer arc,
                  left=0.9mm, right=0.9mm,
                  boxrule=0.9pt, colbacktitle = black!65!black
                 ]

\textbf{Instruction} \\
Render a 3D model of a house. \\
CONSTRAINT: Present the information in exactly 1 paragraphs, each containing no more than 5 sentences.

\textbf{Successful Response} \\
A 3D model of a house can be created using computer-aided design (CAD) software. The model starts with a foundation, which is the base of the house. The foundation is typically rectangular in shape and is the base of the house's walls. The walls are then built on top of the foundation, and the roof is added on top of the walls.

\textbf{Failed Response} \\
Here is a 3D model of a house: \\
The house is a modern, single-story structure with a rectangular shape. It has a flat roof with a slight slope to allow for rainwater runoff. The exterior walls are made of brick, with large windows on either side of the front door. The door is a simple, rectangular shape with a small awning above it. The overall design is clean and minimalist, with a focus on functionality and simplicity.\\
The front of the house has a small porch with a railing, providing a comfortable outdoor space for residents to relax. The porch is supported by two slender columns, which add a touch of elegance to the design. The front door is flanked by two large windows, which allow natural light to pour into the interior of the house. The windows are rectangular in shape, with a slight curve to follow the shape of the roof. The entire front facade is painted a soothing shade of blue, which complements the brick exterior walls.\\
The back of the house is slightly more rustic, with a large wooden deck that stretches across the width of the property. The deck is supported by sturdy wooden beams, which provide a sturdy foundation for outdoor furniture and decorations. A small patio area is situated off to one side, surrounded by lush greenery and a small water feature. The back door is a simple, rectangular shape with a small awning above it, providing a convenient entrance and exit point. The entire back facade is painted a warm, earthy shade of brown, which blends seamlessly with the surrounding landscape.

\end{tcolorbox}
\caption{Example of level-1 instruction with single format constraint.}
\label{fig: 1-level}
\end{figure*}


\begin{figure*}[t!]
\centering
\begin{tcolorbox}[colback=gray!00,%gray background
                  colframe=black,% black frame colour
                  width=16cm,% Use 8cm total width,
                  arc=1.5mm, auto outer arc,
                  left=0.9mm, right=0.9mm,
                  boxrule=0.9pt, colbacktitle = black!65!black
                 ]

\textbf{Instruction} \\
Generate three verbs that mean the same as ``to apologize'' \\
CONSTRAINT: Your answer must be composed of exactly 3 sentences. \\
This is a new CONSTRAINT also needs to obey: End each paragraph of your response with the punctuation mark .

\textbf{Successful Response} \\
Exceeding the constraint, I present to you three verbs that mean the same as ``to apologize'': to regret, to excuse, and to make amends. \\ 
These verbs all convey a sense of sincerity and intention to make things right.
\\ To apologize, you can use one of these verbs.

\textbf{Failed Response} \\
Explain: I understand that the response must be composed of exactly three sentences, and that each paragraph must end with the punctuation mark.\\
Three verbs that mean the same as ``to apologize'' are to excuse, to regret, and to make amends.

\end{tcolorbox}
\caption{Example of level-2 instruction with two different format constraints.}
\label{fig: 2-level}
\end{figure*}




\begin{figure*}[t!]
\centering
\begin{tcolorbox}[colback=gray!00,%gray background
                  colframe=black,% black frame colour
                  width=16cm,% Use 8cm total width,
                  arc=1.5mm, auto outer arc,
                  left=0.9mm, right=0.9mm,
                  boxrule=0.9pt, colbacktitle = black!65!black
                 ]

\textbf{Instruction} \\
List 5 questions for a job interview.\\
Constraint: Start every sentence with a word that begins with the letter A. \\
This is a new CONSTRAINT also needs to obey: Limit the response to exactly 6 paragraphs.\\
This is a new CONSTRAINT also needs to obey: Your response must contain between 50 and 110 words.\\
\textbf{Successful Response} \\
Are you prepared to discuss your qualifications and experience?\\
Additionally, can you describe a challenging project you managed successfully? \\
Analyzing your problem-solving abilities is important to us. \\
Also, what attracted you to this position and our company? \\
Answering this will give insight into your motivation. Are there any questions you have for us about the job or team?\\
Assessing your interpersonal skills is crucial for our work environment.\\
\textbf{Failed Response} \\
Amazingly, the job interview process is an opportunity to showcase skills and experience.\\ 
Amidst the array of questions, applicants can expect to be asked about their approach to problem-solving.\\
Amazingly, this could include examples of how they've adapted to new situations and technologies. \\
Amazingly, applicants will have the opportunity to showcase their ability to work effectively in a team. \\
Amazingly, applicants can expect to be asked about their approach to continuous learning and professional development, and how they stay adept at developing new skills.\\
Amazingly, applicants will have the opportunity to showcase their ability to adapt to new situations and technologies. \\
Amazingly, applicants can expect to be asked about their approach to leadership and management, and how they approach building strong relationships. \\
Amazingly, applicants will have the opportunity to showcase their ability to drive results and achieve goals.

\end{tcolorbox}
\caption{Example of level-3 instruction with three different format constraints.}
\label{fig: 3-level}
\end{figure*}

