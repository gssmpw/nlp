\section{Related Work}
\subsection{SIGCHI research on public sector data}


Public sector agencies in North America have a long history of collecting data on individuals, but as governments turn to data-centric practices, greater value is ascribed to public sector data with the expectation that it can be transformed into actionable insights and guide evidence-based decision-making ____. In line with this movement, the SIGCHI community has actively studied public sector sociotechnical systems, examining values and stakeholder perspectives around public technologies ____, engaging in participatory design work that uplift and empower vulnerable communities ____, and pushing the boundaries of HCI methodologies to deeply engage with the public sector ____. Voida et al. ____ and Møller et al. ____ highlight digital technologies offer affordances to increase governmental efficiencies, expand access to services, and greater opportunities to disseminate information. And yet, conflicting logics can emerge where important questions on whose values should drive the design of public technologies, how, and if those values can be operationalized ____. 

Recently, SIGCHI research has turned their focus on studying algorithmic or AI decision-support systems in the public domain as public agencies increasingly pull data from various public data sources - including data from homelessness management information systems (HMIS), social assistance programs, health services, criminal records, child welfare services, and more - to extract client information, identify common patterns, predict client outcomes, and allocate services to clients ____. While these data-driven systems were created in the hopes that big data can help pre-emptively target resources or interventions consistently for those in need ____, a plethora of SIGCHI work has found these tools often reductively conceptualize client risk ____, and result in biased outcomes for vulnerable population groups ____. Veale et al. ____ argue many of these issues arise because the tools are being developed in isolation from their specific context. In turn, Sambasivan et al. ____ and Suresh and Guttag ____ cite the importance of data work, highlighting that poor data quality and weak incentives to ensure data excellence (i.e., the practice of diligent data documentation, sustained partnerships with data domain experts) can cause algorithmic or AI harms. In light of growing awareness of harms that may arise from such public technologies, HCI scholars have emphasized the importance of engaging in participatory and collaborative work with stakeholders and the need to deeply understand domain-specific data practices when developing tools using public sector data ____. 

\subsection{Data and care work}

Accompanying the greater role of data in public agencies, SIGCHI literature has long interrogated the construct of data highlighting its situatedness ____. Notably, Vertesi and Dourish ____ showed how organizational culture can influence data collection practices where \textit{"work with data enacts social relationships"}. Similarly, Bopp et al. ____ investigating data practices in mission-driven organizations have highlighted how conflicting stakeholder interests can influence what metrics an organization collects and can inadvertently erode organizational autonomy, subverting naïve expectations that data can always improve organizational practices. Because most public agencies are structures that support citizen welfare ____, SIGCHI research on public data work has been closely tied with relational and democratic notions of care ____. Studies in this area have found public sector workers are primarily motivated to provide client care but must also juggle conflicting stakeholder interests and technological constraints, which result in contextualized data practices ____. For example, Tran et al. ____ showed non-profit workers who prioritized care for clients engaged in fragmented data practices through an assemblage of homebrewed ICT tools. Additionally, Boone et al. ____ showed how a food assistance program's lean data practices were shaped by negotiating between different factors: the need to provide financial records for grant fund applications and audits while protecting clients' immigration status and providing low barrier access to services. Studies have also shown that caring practices in public sociotechnical systems are often relational, where trust and rapport are critical factors for data production ____. For example, Nielsen et al. ____ revealed that caseworker's attentiveness and rapport with clients can nudge clients to share personal information, which caseworkers would then translate into relevant and credible data. Homelessness systems are designed to care for unhoused clients, and as such, following current literature on public data, our study was interested in understanding if and how care is enacted through its data practices. 

 

\subsection{HCI research on homelessness}

In recent years, to alleviate the growing demand for shelter and housing services, communities in North America have begun implementing coordinated data-driven approaches using standardized client assessments to assess a client's risk of homelessness and prioritize scarce services to them ____ (see Section \ref{sec:researchcontext} for more information). This approach takes on a system-level approach to supporting clients where client information is tracked on a real-time basis and shared among different service providers to minimize repeated information collection and encourage collaborative client support ____. In response, recent HCI work has turned their efforts to examine data practices within homelessness systems by interviewing stakeholders ____. For example, Slota et al. ____ showed that although homeless systems' data infrastructures enable data sharing between service providers, workers find client information that is being shared is not always sufficient or usable. Karusala et al. ____ also found client data did not necessarily guide how clients are matched to services; instead, a caseworker's assessment of a client's vulnerability would be formalized \text{into} data. For example, if there was a mismatch between a client's risk assessment score and a worker's perception of their vulnerability, workers would sometimes reassess clients to justify matching a client to a particular service. Tracey and Garcia ____ also revealed data documentation requirements can detract workers from providing direct care to clients. Additionally, given the increasing pervasiveness of AI decision-support tools and risk assessment algorithms in homelessness service delivery decisions, the HCI community has critically examined these tools ____. For example, Slota et al. ____ problematized using commonly used risk assessments such as the Vulnerability Index-Service Prioritization Decision Assistance Tool (VI-SPDAT) tool as a prioritization tool in homelessness systems because it relies on self-reported client data but very vulnerable clients cannot always advocate for themselves. Moreover, Moon and Guha ____ and Showkat et al. ____ found such tools are often built using messy and biased public datasets that ignore human values and unduly focus on profiling the client's risk level without fully accounting for how the resource-constrained system can exacerbate client vulnerabilities. 


Prior work to date on coordinated data-driven homelessness systems has conducted interview studies ____ and comic boarding workshops ____ to understand stakeholders' \textit{reported} tensions and values that emerge in these sociotechnical systems. Our study sought to build on these works further by conducting interviews with- and observations of- frontline staff working at different critical points of a homelessness support system to gain a systems-level understanding ____ of both reported and observed worker data practices. Our study's research questions arose after several meetings with the City of Toronto that faces high demand for shelter and housing assistance. City staff expressed the desire to holistically understand how workers engage in similar or different data practices \textit{across their system} to improve their overall data practices. Through our in-depth ethnographic study, we discovered workers engaged in differential data practices depending on the service provider’s role within the overall homelessness system as they navigated uncertainty, resource constraints, and mandated data procedures, all while prioritizing care to clients.