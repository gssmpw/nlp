\section{Related Works}
\label{sec:related}

Although the Hyperledger Fabric (HLF) platform's performance has been widely analyzed in earlier research, this paper aims to address some unexplored aspects.

\blue{Melo et al.~\cite{Melo2022} presented models to assess resource utilization on the HLF platform using Continuous Time Markov Chains (CTMC) and Stochastic Petri Nets (SPNs). Their findings highlighted the utility of these models in planning HLF applications, mainly through sensitivity analysis for detecting infrastructure bottlenecks. However, their research overlooked vital performance metrics such as throughput, latency, and discard rate, which this paper aims to evaluate comprehensively.}

\blue{Jiang et al.~\cite{Jiang2020} introduced a hierarchical model for transaction processes in Hyperledger Fabric v1.4, which considered the constraints of endorsement and block timeouts. This model enabled the calculation of performance metrics such as platform throughput, transaction discard probability, and mean response time.}

\blue{Wu et al.~\cite{wu_acm_ease2022} utilized queuing theory and a two-dimensional continuous-time Markov process to analyze HLF 2.0, focusing on performance metrics including throughput and resource utilization. This approach provided valuable insights but did not incorporate a formal bottleneck detection method.}

\blue{Ke and Park~\cite{Ke2023} created queuing models to evaluate HLF performance, particularly analyzing queue length and mean response time under different service rates for endorsing and validation. Their work highlighted the impact of service rates on system performance but also lacked a formal bottleneck detection framework.}

\blue{Yuan et al.~\cite{yuan2020performance} employed Generalized Stochastic Petri Nets (GSPN) to investigate the effects of arrival rates, block size, and timeout settings on HLF throughput and latency. Their study provided a detailed analysis of these parameters but did not address formal bottleneck detection.}

\blue{Sukhwani et al.~\cite{Sukhwani2018} used Stochastic Reward Networks (SRN) to evaluate performance metrics and estimate average queue sizes throughout the HLF transaction process. Although this research involved experimental validation, it did not include a formal methodology for bottleneck detection, and its focus on earlier HLF versions may limit applicability to current real-world conditions.}

This paper builds on these previous efforts by introducing a robust methodology for bottleneck detection within the HLF 2.5+ version. Our approach integrates formal model validation with percentage difference sensitivity analysis to pinpoint potential bottlenecks in throughput, mean response time, and the impact of queue size parameters on resource utilization.

\begin{figure*}[htpb]
    \centering    
    \begin{subfigure}{\textwidth}
        \centering
        \includegraphics[width=0.75\linewidth]{img/overview_icc.pdf}
        \caption{Transaction Flow}
        \label{fig_overview}
    \end{subfigure}%
    \hfill
    \begin{subfigure}{\textwidth}
        \centering
        \includegraphics[width=0.65\linewidth]{img/modelo_icc.pdf}
        \caption{Performance Model}
        \label{fig_spn_model}
    \end{subfigure}
    \vspace{-10pt}
    \caption{Hyperledger Fabric: Architecture and Stochastic Model}
    \label{fig_combined_figs}
\end{figure*}

%-----------------------