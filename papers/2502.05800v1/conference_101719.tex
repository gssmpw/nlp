\pdfoutput=1
\documentclass[conference]{IEEEtran}
\IEEEoverridecommandlockouts
% The preceding line is only needed to identify funding in the first footnote. If that is unneeded, please comment it out.
\makeatletter
\newcommand*\titleheader[1]{\gdef\@titleheader{#1}}
\AtBeginDocument{%                                                               
  \let\st@red@title\@title
  \def\@title{%                                                                 
    \bgroup\normalfont\normalsize\centering\@titleheader\par\egroup
    \vskip1ex\st@red@title}
}
\makeatother

\usepackage{cite}
\usepackage{amsmath,amssymb,amsfonts}
\usepackage{algorithmic}
\usepackage{array}
\usepackage{graphicx}
\usepackage{textcomp}
% \usepackage{xcolor}
\usepackage[table]{xcolor}
\usepackage{multirow}
\usepackage{subcaption}

\def\BibTeX{{\rm B\kern-.05em{\sc i\kern-.025em b}\kern-.08em
    T\kern-.1667em\lower.7ex\hbox{E}\kern-.125emX}}
\usepackage{listings}

\definecolor{codegreen}{rgb}{0,0.6,0}
\definecolor{codegray}{rgb}{0.5,0.5,0.5}
\definecolor{codepurple}{rgb}{0.58,0,0.82}
\definecolor{backcolour}{rgb}{0.95,0.95,0.92}

\lstdefinestyle{mystyle}{
    backgroundcolor=\color{backcolour},   
    commentstyle=\color{codegreen},
    % keywordstyle=\color{magenta},
    keywordstyle = {\color{magenta}},
    keywordstyle = [2]{\color{lime}},
    keywordstyle = [3]{\color{yellow}},
    keywordstyle = [4]{\color{teal}},
    numberstyle=\tiny\color{codegray},
    stringstyle=\color{codepurple},
    basicstyle=\ttfamily\footnotesize,
    breakatwhitespace=false,         
    breaklines=true,                 
    captionpos=b,                    
    keepspaces=true,                 
    numbers=left,                    
    numbersep=5pt,                  
    showspaces=false,                
    showstringspaces=false,
    showtabs=false,                  
    tabsize=2
}

\lstset{style=mystyle}

\title{MicroViT: A Vision Transformer with Low Complexity Self Attention for Edge Device \\
%\thanks{Identify applicable funding agency here. If none, delete this.}
}
\author{\IEEEauthorblockN{Novendra Setyawan\textsuperscript{1,3}, Chi-Chia Sun\textsuperscript{*,2}, Mao-Hsiu Hsu\textsuperscript{1}, Wen-Kai Kuo\textsuperscript{1}, Jun-Wei Hsieh\textsuperscript{4}}
\IEEEauthorblockA{\textit{\textsuperscript{1}Department of Electro-Optics Engineering, National Formosa University, Taiwan}\\
\textit{\textsuperscript{2}Department of Electrical Engineering, National Taipei University, Taiwan} \\
\textit{\textsuperscript{3}Department of Electrical Engineering, University of Muhammadiyah Malang, Indonesia} \\
\textit{\textsuperscript{4}College of Artificial Intelligence and Green Energy, National Yang Ming Chiao Tung University, Taiwan} \\
\textit{chichiasun@gm.ntpu.edu.tw\textsuperscript{*}}
} 
}

\titleheader{\vspace{-20pt}To appear at the 2025 IEEE International Symposium on Circuits and Systems (ISCAS), May 25-28 2025, London, UK.}

\begin{document}

\maketitle

\begin{abstract}

The Vision Transformer (ViT) has demonstrated state-of-the-art performance in various computer vision tasks, but its high computational demands make it impractical for edge devices with limited resources. This paper presents MicroViT, a lightweight Vision Transformer architecture optimized for edge devices by significantly reducing computational complexity while maintaining high accuracy. The core of MicroViT is the Efficient Single Head Attention (ESHA) mechanism, which utilizes group convolution to reduce feature redundancy and processes only a fraction of the channels, thus lowering the burden of the self-attention mechanism. MicroViT is designed using a multi-stage MetaFormer architecture, stacking multiple MicroViT encoders to enhance efficiency and performance. Comprehensive experiments on the ImageNet-1K and COCO datasets demonstrate that MicroViT achieves competitive accuracy while significantly improving  $3.6 \times$ faster inference speed and reducing energy consumption with 40\% higher efficiency than the MobileViT series, making it suitable for deployment in resource-constrained environments such as mobile and edge devices.

\end{abstract}

\begin{IEEEkeywords}
Classification, Self Attention, Vision Transformer, Edge Device.
\end{IEEEkeywords}

\section{Introduction}\label{sec:intro}

In computational finance, Monte Carlo simulations are used extensively to estimate the expected value of financial payoffs based on the solution of stochastic differential equations (SDEs) which model the evolution of stock prices, interest rates, exchange rates and other quantities \cite{glasserman04}.  Monte Carlo methods are very general and flexible, but for high accuracy it requires generating a large number of costly SDE path approximations, which has motivated research into a number of variance reduction or, equivalently, cost reduction techniques. One such method is
Multilevel Monte Carlo (MLMC), which was proposed in \cite{GILES2008} and was adapted for various applications that are summarised in \cite{Giles_overview17} and successfully combined with other methods such as quasi-Monte Carlo methods. The main idea of MLMC is to approximate the payoff using different time stepping resolutions when numerically solving the underlying SDE and to generate an optimal number of samples on each level, such that the overall computational cost is minimised subject to the desired bound on the variance. %, such that the total computational cost is minimised. 
The computational savings come from the fact that most samples are computed on the coarser levels and hence are less expensive while only a few samples from the finest levels are required \cite{GILES2008}.


Among the directions in which the computational cost 
of MLMC methods could further be reduced, an important avenue is the use of lower precision calculations, especially for the first Monte Carlo levels where the targeted accuracy is relatively low. 
 An overview of the research on mixed precision for the standard Monte Carlo (MC) framework is provided in \cite{ChowMixedPrecisionStandardMC} but only a few references study the potential of low precision computation in the MLMC framework \cite{Rounding_error_oliver}. To the best of our knowledge, the only MLMC framework with customised precision in the literature is \cite{brugger2014mixed}, but they use a uniform precision for all operations on each Monte Carlo level instead of optimising 
 the precision of each intermediary variable to reduce as much as possible the cost of path generation.
 
An important motivation for an MLMC framework with variable precision would be performing the low precision computations on reconfigurable hardware devices such as Field Programmable Gate Arrays (FPGAs). FPGAs contain customizable logic blocks and connectors that make it easy to adapt the digital circuit architecture for a specific application, leading to a highly parallel and optimised implementation. Therefore they are successfully exploited in applications that require high speed and have high computational workload, such as signal processing \cite{woods2008fpga}, and real time applications like high frequency trading \cite{HFT1,HFT2}. That is why a number of previous works in hardware architecture design implemented the MLMC algorithm to price financial options using FPGAs as accelerators, which resulted in improved speed and power efficiency compared to full CPU architectures \cite{Schryver2013AMM}. The paper \cite{lindsey2016domain} also proposed 
a Domain Specific Language to automate the configuration of FPGAs for this specific application. However, only \cite{brugger2014mixed} proposed a heuristic to reduce the precision in calculations.

In addition, all aforementioned works considered that the random number generation (RNG) is performed in single or double precision. Yet in most cases an important portion of the workload in the overall MLMC simulation comes from the RNG and in \cite{brugger2014mixed} this limited the total computational savings.
To reduce the cost of MLMC simulations in particular those based on the Geometric Brownian Motion (GBM), \cite{approximateICDF_Oliver, NestedOliver} have proposed to use approximate random numbers that are generated by applying an approximation of the inverse CDF to uniform random numbers. In \cite{NestedOliver}, the authors proposed a way to integrate these lower precision random variables into a \textit{nested} MLMC framework and completed a numerical analysis to bound the resulting error at each MC level by a product of the time step and the error in the random number approximation. The same authors show in \cite{approximateICDF_Oliver} that using approximate random variables reduces the cost of path generation by a factor 7.


In this paper we propose a nested MLMC framework that combines the use of approximate random normal variables and lower precision calculations to reduce the computational cost of MLMC even further than \cite{brugger2014mixed,NestedOliver}. We illustrate the efficiency of our framework in Matlab, after making several assumptions on the cost of operations and size of the errors that we carefully justify. We focus on the case of GBM and use the approximate RNG methods presented in \cite{approximateICDF_Oliver} as well as a new slightly modified method that combines CDF inversion and the central limit theorem. To choose the precision of the variables in the low precision path generation, we introduce a novel method to optimise the bit-widths. This optimisation is performed before the main path generation loop is executed and is based on a linear model of the payoff error  
due to rounding when computing in low precision. The error model relies on algorithmic differentiation in a similar manner to \cite{unifying-bwoptim,bitwidth-AD,ADAPT}. The bit-width optimisation procedure can be performed off-line, so this stage can be excluded from the on-line time complexity of our framework. The user specified desired accuracy is then enforced by calculating on-line the number of samples that need to be generated.

In terms of hardware design, we suggest implementing the low precision path generation on FPGAs and the full-precision ones on a CPU or GPU. 
The FPGA offers enough flexibility to define a separate bit-width for every variable in the low precision path generation, and can be reconfigured periodically to update the bit-widths when the market parameters have changed considerably. 


The paper is organized as follows : \Cref{sec:MLMC} introduces MLMC and nested MLMC to make clear the estimator that is implemented in our framework. Then in \Cref{sec:RNG} we detail the methods that could be used to obtain approximate random normally distributed numbers very cheaply for the low precision path generation. In \Cref{sec:error_model} and \Cref{sec:costModel} we propose an error model and a cost model (resp.) that we then use to formulate the optimisation problem that is solved to obtain the optimal bit-widths of fixed point variables in \Cref{sec:optimisation}. Finally we summarise our results and future directions in \Cref{sec:conclusion}.



\section{Related Work}

\subsection{First-order logic for natural entailment}

Since the start of the RTE challenge \citep{rte}, multiple works have attempted using FOL representations to solve natural language entailment. These methods first obtain the syntactic/semantic parse tree and apply a rule-based transformation to get the FOL representation \citep{bos-markert-2005-recognising, bos-nli}. However, it was repeatedly shown that these FOL representations are not empirically effective in solving natural language entailment. For instance, \citet{bos-nli} reported that FOL representations translated from the discourse representation structure (DRS) yield only 1.9\% recall in detecting the entailment in the single-premise RTE benchmark \citep{rte}.

Independently from these works, multi-premise logical entailment benchmarks \citep{tafjord-etal-2021-proofwriter, logicnli, folio} were developed to evaluate the reasoning ability of generative models. These benchmarks adopt the classic 3-way entailment label classification format (\textit{entailment, contradiction, neutral}) of single-premise RTE tasks, in which both the NL sentences and their gold FOL representations point to the same entailment label. 

Recent works have applied LLMs to obtain FOL representations for these multi-premise logical entailment tasks \citep{logiclm, linc, divide-and-translate}, fueled by the code generation ability of LLMs. While they achieve significant performance in synthetic, controlled logical reasoning benchmarks, whether they can generalize to natural entailment has remained unanswered. Furthermore, \citet{linc} observed that LLMs are highly susceptible to \textit{arbitrariness}, as they fail to produce coherent predicate names or numbers of arguments even when generating FOL representations of premises and hypotheses in a single inference.

\subsection{Executable semantic representations}

Apart from FOL, a stream of research focuses on the \textit{executability} of semantic representations. From this perspective, semantic representations are \textit{program codes} that can be executed to solve downstream tasks, such as query intent analysis \citep{spider, dligach-etal-2022-exploring} and question answering \citep{semparse-qa}. The performance of the semantic parser is directly assessed by the accuracy of execution results for the downstream tasks, rather than the similarity between the prediction and the reference parse.

To improve the execution accuracy that is often non-differentiable, reinforcement learning (RL) and its variants have been applied to train neural semantic parsers \citep{cheng-etal-2019-learning, cheng-lapata-2018-weakly}. Using only the input sentence and the desired execution result, these methods learn to maximize the probability of the representations that lead to the correct execution result. However, these approaches are not directly applicable to EPF, as EPF requires taking account of \textit{interactions between premises and hypotheses} during execution (\textit{i.e.} theorem proving) while these methods assume that sentences are isolated.




\section{Methodology}
\paragraph{Preliminaries.}
We primarily focus on the homologous model merging, in which $\boldsymbol{\theta}_i$ all come from the same base model $\boldsymbol{\theta}_{\rm{base}}$. Given $K$ tasks $\{T_1,T_2,\cdots,T_K\}$ and $K$ corresponding fine-tuned models with parameters $\{\boldsymbol{\theta}_1,\boldsymbol{\theta}_2,\cdots,\boldsymbol{\theta}_K\}$, model merging aims to combine $K$ fine-tuned models into one single model simultaneously performing on $\{T_1,T_2,\cdots,T_K\}$ without post-training~\cite{method_p1_1,method_p1_2}.
Task vector~\cite{ilharco2023editing,yang2024adamerging} is a key element in merging method which could enhances the base model‘s ability or enable the model to handle other tasks. Specifically, for task $T_i$, the task vector $\boldsymbol\tau_i\in \mathbb{R}^D$ is defined as the vector obtained by subtracting the SFT weights $\boldsymbol{\theta}_i$ from the base model weight
$\boldsymbol{\theta}_{\rm{base}}$, \emph{i.e.}, $\boldsymbol\tau_i=\boldsymbol{\theta}_i-\boldsymbol{\theta}_{\rm{base}}$. The merged model could be denoted as $\boldsymbol{\theta}_m=\boldsymbol{\theta}_{\rm{base}}+\sum_i \lambda_i\boldsymbol{\tau}_i$, which $\lambda_i$ is the scaling factor measuring the importance of task vector. For clarification, we also denote the neuron set in $\boldsymbol{\theta}_i$ as $\mathcal{N}_i$, the neuron set in $\boldsymbol{\tau}_i$ as $\mathcal{T}_i$.



\begin{algorithm}[!ht]
    \caption{LED-Merging}
    \label{alg1}
    \begin{algorithmic}[1]
        \REQUIRE  base model $\boldsymbol{\theta}_{\rm{base}}$, SFT models $\{\boldsymbol{\theta}_{i}\mid i\in [K]\}$, mask ratios \{$r_{i} \mid i\in [K]\}$, scaling factors $\{\lambda_i\mid i\in[K]\}$, location datasets $\{\mathcal{X}_{i}\mid i\in[K]\}$
        \ENSURE merged parameter $\boldsymbol{\theta}_{m}$
        \STATE $\mathcal{M}\leftarrow\phi$
        \STATE $\boldsymbol{\theta}_{m}\leftarrow \boldsymbol{\theta}_{\rm{base}}$
        \FOR{$i\in [K]$}
        \STATE $I(\boldsymbol{\theta}_i)=\mathbb{E}_{x\sim \mathcal{X}_i}|\boldsymbol{\theta}_{i}\odot \nabla_{\boldsymbol{\theta}_i}\mathcal{L}(x)|$
        \STATE $I(\boldsymbol{\theta}_{\rm{base}})=\mathbb{E}_{x\sim \mathcal{X}_i}|\boldsymbol{\theta}_{\rm{base}}\odot \nabla_{\boldsymbol{\theta}_{\rm{base}}}\mathcal{L}(x)|$
        
        \STATE calculate $\mathcal{T}^{r_i}_{i}$ following Equation \ref{vote}
        \STATE  $\mathcal{M}\leftarrow \mathcal{M}\cup\{\mathcal{T}^{r_i}_i\}$
       
        
   
        
        
        \ENDFOR  
        \FOR{$i\in [K]$}
        
        \STATE calculate $\text{Disjoint}(\mathcal{T}_i^{r_i})$ use Equation~\ref{disjoint_safety}
        \STATE $\boldsymbol{m}_i \leftarrow \boldsymbol{0}$
        \FOR{$d\in \mathcal{T}_i^{r_i}$}
        \STATE $\boldsymbol{m}_{i,d}=1$
        \ENDFOR
        \STATE $\boldsymbol{\theta}_{m}\leftarrow \boldsymbol{\theta}_{m}+\lambda_i \boldsymbol{\tau}_i\odot \boldsymbol{m}_{i}$
        \ENDFOR
    \end{algorithmic}
\end{algorithm}
    %\vspace{-5pt}
\begin{figure*}[h!]
    \centering
    \includegraphics[width=\linewidth]{figs/pipeline_v2.pdf}
    \vspace{-40mm}
    \caption{Overview of our two-stage training pipeline {\ours}.}
    \label{fig:pipeline}
\end{figure*}


\paragraph{LED-Merging: Location, Election, and Disjoint Merging}
To address the neuron misidentification and interference issues in existing model merging methods, we propose LED-Merging (Location, Election, and Disjoint Merging). Specifically, previous studies \cite{modelstock, ilharco2023editing, tiesmerging} fail to accurately identify safety-related neurons in task vectors with a single magnitude score, namely \textit{neuron misidentification}. Meanwhile, there exists an interference between safety-related and utility-related task vector neurons during the merging process, namely \textit{neuron interference}. To address neuron misidentification, we first locate important neurons both in the base and fine-tuned models and then elect neurons from the task vector considering these two scores together. Subsequently, to mitigate the interference, we introduce a disjoint step, isolating these important neurons so that they influence different base neurons. The whole process is illustrated in Figure~\ref{fig:method}. 




In the location and election step, we consider the importance score from base and fine-tuned models simultaneously to locate task-specific neurons. In this way, it is more accurate than relying on the magnitude score alone because task-specific neurons with high importance score in the fine-tuned model may not necessarily score high in the base model, and vice versa.

{\textbf{Location}}.  We first calculate importance scores for each neuron in a base/fine-tuned model. Given a location dataset $\mathcal{X}_i=\{(x,y)_k\}$, where $x$ is the question and $y$ is the answer, we calculate the importance scores for the weight $\boldsymbol{\theta}_i\in\mathbb{R}^D$ in any  layer as follows~\cite{snip,spareseGPT,sun2024a}:
\begin{equation}
    I(\boldsymbol{\theta}_i)=\mathbb{E}_{x\sim \mathcal{X}_i}[\boldsymbol{\theta}_i\odot \nabla _{\boldsymbol{\theta}_i}\mathcal{L}(x)],
    \label{location}
\end{equation}
which $\mathcal{L}(x)=-\log p(y\mid x)$ is the conditional negative log-likelihood loss. We choose the SNIP score~\cite{snip} because it balances computational efficiency and performance~\cite{cq}. Please refer to Sec.~\ref{sec:ablation} for the comparison between different location methods. After computing importance scores, we choose top-$r_i$ neurons as the important neuron subset $\mathcal{N}_{i}^{r_i}$ from $I(\boldsymbol{\theta}_i)$.
 
 % After computing locating scores, we select the neurons scoring both high in base and fine-tuned models as important neurons in task vectors. Then in the disjoint step,  with preventing  polysemantic neurons  from receiving gradient updates towards different directions,
 % we use set difference to isolate the safety   and utility-related neurons  and construct corresponding masks for merging process,

{\textbf{Election}}. A natural question is how to select important neurons in the task vector $\boldsymbol{\tau}_i$ based on $I(\boldsymbol{\theta}_{\rm{base}})$ and $I(\boldsymbol{\theta}_{i})$. The important neurons in the base model may be different from neurons in the fine-tuned model. Therefore, we introduce the following election strategy to select neurons with high scores in both base and fine-tuned models:
\begin{equation}
    \mathcal{T}_i^{r_i}=\mathcal{N}_i^{r_i}\cap \mathcal{N}_{\rm{base}}^{r_i}.
    \label{vote}
\end{equation}
\emph{Remark}. We compare different choosing methods, including scoring low or high in base or fine-tuned model in Section~\ref{sec:ablation} and find that Equation \ref{vote} achieves the best performance.





{\textbf{Disjoint}}. As important neurons from different task vectors may conflict with each other at the same position, we use the set difference to disjoint the neurons from others to prevent interference:
\begin{equation}
    \text{Disjoint}(\mathcal{T}^{r_i}_{i})=\mathcal{T}^{r_i}_{i}-\mathop{\cup}\limits_{{J}\subsetneqq [K],|J|\geq 2}\mathop{\cap}\limits_{j\in {J}}\mathcal{T}^{r_j}_{j}.
    \label{disjoint_safety}
\end{equation}

Next, we construct a mask $\boldsymbol{m}_i\in\mathbb{R}^D$ to implement disjoint in the merging process. Specifically, this mask $\boldsymbol{m}_i$ is used to select neurons from $\mathcal{T}_i$. The mask ratio is $r_i$, where $r\in(0,1]$. The mask $\boldsymbol{m}_i$ can be derived from:
\begin{equation}
    \boldsymbol{m}_{i,d}=\begin{aligned} &\left\{ \begin{array}{ll} 1, & \text{if } d\in \text{Disjoint}(\mathcal{T}_{i}^{r_i}), \\ 0, & \text{otherwise}. \end{array} \right. \end{aligned}
    \label{mask_safety}
\end{equation}


% \subsection{Merging Models with Masks}
{\textbf{Merging}}. The final
merged task vector $\boldsymbol{\tau}_m$ is as follows:
\begin{equation}
    \boldsymbol{\tau}_m= \sum_i \lambda_i\boldsymbol{\tau}_{i}\odot\boldsymbol{m}_i.
    \label{merged_task_vector}
\end{equation}
We summarize the workflow in Algorithm \ref{alg1}.



\section{Result}
For the evaluation of MicroViT, the ImageNet-1K dataset \cite{russakovsky2015imagenet} comprising 1.28 million training images and 50,000 validation images over 1,000 categories was employed. Following the DeiT training method \cite{touvron2021training}, models were trained for 300 epochs at a 224×224 resolution with an initial learning rate of 0.004, utilizing various data augmentations. The AdamW optimizer \cite{loshchilov2017decoupled} was used with a batch size of 512 across three A6000 GPUs.

We assessed model throughput in various computation environments, including a GPU (RTX-3090), a CPU (Intel i5-13500), and specifically the Jetson Orin Nano edge device. For throughput, the GPU and CPU had a batch size of 256, whereas the edge device used a batch size of 64 with ONNX Runtime. To enhance performance during inference, BN layers were fused with adjacent layers when possible. On the Jetson Orin Nano, we also examined power and energy usage during latency tests with 1000 images at a consistent resolution.

We further evaluate MicroViT on object detection on the COCO dataset \cite{lin2014microsoft} utilizing RetinaNet \cite{ross2017focal} and conduct training for 12 epochs (1$\times$ schedule), adhering to the configuration used by \cite{liu2023efficientvit} in mmdetection \cite{mmdetection}. In the object detection experiments, we employ AdamW \cite{loshchilov2017decoupled} with a batch size of 16, a learning rate of $1\times10^{-3}$, and a weight decay rate of 0.025. 


\begin{table}[!ht]
\centering
\caption{Comparison of All MicroViT Variant with SOTA on ImageNet-1K Dataset. Res, Par and FLPs denotes as input resolution, parameters and Floating Operation. GPU and CPU denotes a inference throughput (img/s) in device respectively.}
\begin{tabular}{ m{2.6cm}|c|c|c|c|>{\centering}m{0.5cm}|c }
\hline
Model   & Res  & Par & FLPs &  GPU  & CPU &   Top-1      \\ \hline
% Fasternet-T1\cite{liu2022convnet}              & 224 & 7.6  & 0.85 & \textbf{7037}  & 149.3  &   & 76.2      \\
MobileNetV2-1.0\cite{sandler2018mobilenetv2}& 224 & 3.5 & 0.314 & 4527 & 82 & 72.0   \\
MobileViT-XXS\cite{mehta2021mobilevit}  & 256 & 1.3 & 0.261 & 3218 & 99  & 69.0   \\
MobileViTV2-0.5\cite{mehta2022separable}  & 256 & 1.4 & 0.480 & 3885 & 68  & 70.2   \\
EdgeNeXt-XXS\cite{maaz2022edgenext}  & 256 & 1.3 & 0.261 & 3975 & 245  & 71.2   \\
Fasternet-T0\cite{chen2023run}  & 224 & 3.9 & 0.340 & 11775 & 311 & 71.9   \\
SHViT-S1\cite{yun2024shvit} & 224  & 6.3 & 0.241 & 15280 & 475 & 72.8 \\
\rowcolor{gray!30}
\textbf{MicroViT-S1}                    & 224  & 6.4 & 0.231 & 17466 & 552 & 72.6 \\ \hline 
EFormerV2-S0\cite{li2023rethinking}& 224  & 3.6 & 0.407 & 1191 & 91 & 73.7 \\ 
EdgeNeXt-XS\cite{maaz2022edgenext}   & 256 & 2.3 & 0.536 & 2935 & 139  & 75.0   \\
EfficientViT-M4\cite{liu2023efficientvit} & 224 & 8.8 & 0.303 & 10093 & 379 & 74.3   \\
MobileViT-XS\cite{mehta2021mobilevit}   & 256 & 2.3 & 0.935 & 1740 & 43 & 74.8   \\
MobileNetV3-L\cite{howard2019searching}& 224 & 3.5 & 0.314 & 4527 & 82 & 75.2  \\
SHViT-S2\cite{yun2024shvit}            & 224  & 11.5 & 0.366 & 12007 & 367 & 75.2 \\
\rowcolor{gray!30}
\textbf{MicroViT-S2}                    & 224 & 10.0 & 0.345 & 14154 & 435 & 74.6 \\ \hline
FastViT-T8\cite{vasu2023fastvit}   & 256 & 4.0 & 0.687 & 3719 & 83 & 76.2   \\
Fasternet-T1\cite{chen2023run}  & 224 & 7.6 & 0.851 & 7151 & 130 & 76.2   \\
EfficientViT-M5\cite{liu2023efficientvit}& 224 & 12.5 & 0.526 & 6807 & 233 & 77.1   \\
SHViT-S3\cite{yun2024shvit}            & 224  & 14.1 & 0.601 & 8180 & 224 & 77.8 \\
% \hline 
\rowcolor{gray!30}
\textbf{MicroViT-S3}                   & 224 & 16.7 & 0.580 & 9288 & 232 & 77.1 \\ \hline 
    \end{tabular}
    \label{tab:imgnet-result}
\end{table}
\subsection{ImageNet-1K Classification Result}
Table \ref{tab:imgnet-result} presents a comparison of various MicroViT variants with state-of-the-art (SOTA) models on the ImageNet-1K dataset. The evaluation focuses on models' computational efficiency and accuracy, highlighting the trade-offs between resource consumption and performance. 

MicroViT-S1 demonstrated superior performance compared to traditional CNN models, surpassing MobileNetV2-1.0\cite{sandler2018mobilenetv2} and Fasternet-T0\cite{mehta2022separable}, with a $3.6 \times$ faster in GPU and $6.7 \times$ in CPU throughput, while maintaining an accuracy advantage of 0.8 over MobileNetV2-1.0. Additionally, MicroViT-S2 outperformed mobile transformers like EfficientFormerV2-S0\cite{li2023rethinking} and EfficientViT-M4\cite{liu2023efficientvit}, achieving $0.3\%$ better accuracy with similar efficiency metrics. Across the MicroViT models, CPU throughput is robust, notably with MicroViT-S1 achieving 552 img/s, which is $8 \times$ faster than several EfficientViT variants, illustrating MicroViT's adaptability to both high-end GPU and CPU settings.


Table \ref{tab:edge-result} presents the performance of MicroViT variants against various SOTA models on the Edge device using ONNX. MicroViT-S1's throughput reaches 773 img/s, efficiently managing large image volumes on the Jetson Orin Nano. This surpasses several SOTA models like MobileViT-XS\cite{mehta2021mobilevit} and EfficientFormer-V2-S0\cite{li2023rethinking}, making MicroViT-S1 optimal for rapid image processing applications. Furthermore, it has a 9.1 ms latency, outperforming MobileNetV2-1.0\cite{sandler2018mobilenetv2} and EdgeNeXt-XS\cite{maaz2022edgenext}, supporting real-time use. It consumes 2147 Joules, achieving high energy efficiency with $\eta=3.7$. Likewise, MicroViT-S2 and MicroViT-S3 balance throughput and energy use, maintaining accuracy, thus ideal for resource-constrained edge devices with superior power efficiency over other lightweight vision transformers.
\begin{table}[!ht]
\centering
\caption{Comparison of All MicroViT Variant and SOTA on ImageNet-1K Dataset with NVIDIA Jetson Orin Nano Edge Device using ONNX format.}
\begin{tabular}{ m{2.6cm}|>{\centering}m{0.5cm}|>{\centering}m{0.6cm}|>{\centering}m{0.9cm}|>{\centering}m{0.8cm}|c }
\hline
\multirow{2}{*}{Model} & Thg & Lat & Avg Pow & Energy  & $\eta$ \\ 
         & img/s & (ms) & (W) & (Joule) & \% / J \\ \hline
MobileNetV2-1.0\cite{sandler2018mobilenetv2} & 234 & 6.7 & 3549 & 23.9 & 3.01 \\
MobileViT-XXS\cite{mehta2021mobilevit}  & 184 & 9.6 & 3428 & 32.4 & 2.13   \\
MobileViTV2-0.5\cite{mehta2022separable}  & 208 & 10.9 & 2887 & 31.6 & 2.22 \\
EdgeNeXt-XXS\cite{maaz2022edgenext}     & 257 & 8.1 & 2805 & 22.6 &  3.15  \\
Fasternet-T0\cite{chen2023run}   & 675 & 8.4 & 2419 & 20.4 &  3.52 \\
SHViT-S1 \cite{yun2024shvit}       & 813  & 12.6 & 2069 & 26.0 & 2.80  \\
\rowcolor{gray!30}
\textbf{MicroViT-S1}                & 773  & 9.1 & 2147 & 19.6 & 3.7 \\ \hline 
EFormerV2-S0\cite{li2023rethinking}  & 257  & 10.9 & 2847 & 31.1 & 2.37   \\ 
EdgeNeXt-XS\cite{maaz2022edgenext}      & 168 & 11.1 & 3031 & 33.6 &  2.32   \\
EfficientViT-M4\cite{liu2023efficientvit} & 587 & 18.0 & 2019 & 36.3 &  2.05   \\
MobileViT-XS\cite{mehta2021mobilevit}   & 96.6 & 14.3 & 3730 & 53.3 & 1.40    \\
MobileNetV3-L\cite{howard2019searching} & 310 & 8.3 & 2743 & 22.6 &  3.33 \\
SHViT-S2 \cite{yun2024shvit}            & 598  & 12.7 & 2306 & 29.4 & 2.56  \\
\rowcolor{gray!30}
\textbf{MicroViT-S2}                      & 567 & 9.9 & 2481 & 24.3 & 3.07 \\ \hline
FastViT-T8\cite{vasu2023fastvit}          & 176 & 8.6 & 3622 & 30.8 &  2.47  \\
Fasternet-T1\cite{chen2023run}     & 421 & 8.8 & 2860 & 25.3 &  3.01  \\
EfficientViT-M5\cite{liu2023efficientvit} & 409 & 21.0 & 2180 & 45.9 & 1.68    \\
SHViT-S3\cite{yun2024shvit}               & 425  & 14.6 & 2500 & 36.6 & 2.13  \\
% \hline 
\rowcolor{gray!30}
\textbf{MicroViT-S3}                    & 398 & 10.9 & 2609 & 28.6 & 2.69  \\ \hline 
    \end{tabular}
    \label{tab:edge-result}
\end{table}

\subsection{Object Detection Result}
We compare MicroViT-3 with efficient models \cite{sandler2018mobilenetv2, howard2019searching, liu2023efficientvit} on the COCO \cite{lin2014microsoft} object detection task, and present the results in Table \ref{tab:obj} Specifically, MicroViT-3 surpasses MobileNetV2 \cite{sandler2018mobilenetv2} by
7.7\% AP with comparable Flops. Compared to the EfficientViT-M4, our MicroViT-3 uses 46.8\% fewer Flops while achieving 3.3\% higher AP, demonstrating its capacity and generalization ability in different vision tasks.

\begin{table}[!ht]
\centering
\caption{Object detection on COCO val2017 with RetinaNet. $AP^b$ denote bounding box average precision. The FLOPs (G) are measured at resolution 1280 $\times$ 800.
}
\begin{tabular}{ m{2.4cm}|>{\centering}m{0.8cm}|>{\centering}m{0.8cm}|>{\centering}m{0.8cm}|>{\centering}m{0.8cm}|c}
\hline
Backbone & $AP^{b}$  & $AP^{b}_{50}$ & $AP^{b}_{75}$ & Par & FLPs  \\ \hline
MobileNetV2\cite{sandler2018mobilenetv2} & 28.3& 46.7& 29.3& 3.4& 300 \\
MobileNetV3\cite{howard2019searching} & 29.9& 49.3& 30.8& 5.4 & 217   \\ 
EfficientViT-M4\cite{liu2023efficientvit}&32.7 &52.2 &34.1 & 8.8 & 299  \\
\rowcolor{gray!30}
\textbf{MicroViT-S3} & 36.0 & 56.6 & 38.2 & 26.7 & 159    \\ \hline

\end{tabular}
\label{tab:obj}
\end{table}

\subsection{Ablation Study}

Table \ref{tab:abl-result} assesses the effects of three ablations on the MicroViT-S2 model as the baseline. The first ablation, low resolution SA with SR=2, slightly increases the parameter count from 11.0M to 11.1M, with GPU throughput remaining nearly unchanged from 12009 to 12029 images per second. However, it increases latency from 9.9 ms to 10.7 ms and decreases Top-1 accuracy to 72.5\%, resulting in a minor drop in efficiency ($\eta$) from 3.0 to 2.8. This indicates a small trade-off between accuracy and computational cost due to the architectural modification.

The Next ablation, without group convolutions, increases the model's complexity significantly to 20.5M parameters, with higher GFLOPs. This results in lower throughput and efficiency ($\eta=1.8$) but achieves the highest accuracy. However, this variant consumes more energy, making it ideal for use cases where accuracy is prioritized over efficiency.

The ablation study shows that even though spatial reduction can increase throughput, the latency is increased, resulting in an efficiency drop. Group convolution successfully increases the efficiency in ESHA compared to other attention models, such as vanilla attention in MobileViT, maintaining lower complexity and energy usage.

\begin{table}[!ht]
\centering
\caption{Ablation study of MicroViT  on ImageNet-1K Dataset. The Baseline model is MicroViT-S2.}
\begin{tabular}{ m{1.6cm}|c|c|c }
\hline
\textbf{Ablation} & Baseline & Low Res Attn & W/O Group  \\ \hline
\textbf{Param}      & 11.0  & 11.1  & 20.5   \\  
\textbf{GFLOPs}     & 0.343 & 0.344 & 0.4977 \\ 
\textbf{GPU}        & 14154 & 14179 & 13511  \\
\textbf{Throughput} & 567   & 570   & 503    \\
\textbf{Latency}    & 9.9   & 10.7  & 12.5   \\
\textbf{Avg Pow}    & 2481  & 2512  & 3288   \\
\textbf{Energy}     & 24.29  & 25.8  & 41.2   \\
\textbf{Top-1}      & 72.7  & 72.5  & 73.2   \\
\textbf{Efficiency ($\eta$)}& 3.07  & 2.8 & 1.8 \\
\hline


    \end{tabular}
    \label{tab:abl-result}
\end{table}


\section{Conclusion}

%In this paper, w
We propose a new PEFT method called DiffoRA, which enables efficient and adaptive LLM fine-tuning based on LoRA. 
Instead of adjusting every interior rank, 
%of the decomposition matrices 
%of all modules, 
we argue that adopting LoRA module-wisely is sufficient. 
To achieve this, we construct a DAM to select the modules that are most suitable and essential to fine-tune. We theoretically analyze how the DAM impacts the convergence rate and generalization capability.
%of the pre-trained model. 
Furthermore, we adopt continuous relaxation and discretization to establish DAM.
%for each task. 
To alleviate the issue of discretization discrepancy, we utilize the weight-sharing strategy for optimization. 
%We fully implement our method and t
The experimental results demonstrate that our DiffoRA works consistently better than the baselines across all benchmarks. 

\bibliography{biblio}
\bibliographystyle{ieeetr}
\end{document}
