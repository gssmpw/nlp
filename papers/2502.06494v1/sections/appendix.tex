
\clearpage
\appendix
\onecolumn

\section{Interview Protocol}\label{appendix:interview_protocol}
The Life Story Interview\cite{mcadams2008life} is a protocol of interviewing a person's story of life. The interview contains several areas and in each area it contains several topics. We followed the protocol and converted each topic into a prompt. We also provided several seed questions for each topic. The following shows the prompt design of the interview protocol.
\vspace{-3mm}
\begin{table*}[h]
    \centering
    \begin{tabular}{c|c}
        \toprule
        \textbf{Area} & \textbf{Topic} \\
        \toprule
        Life Chapters & - \\
        \midrule
        & High Point; Low Point; Turning Point; Positive Childhood Memory; \\
        Key Scenes in the Life Story & Negative Childhood Memory; Vivid Adult Memory; \\
        & Religious, Spiritual, or Mystical Experience; Wisdom Event\\
        \midrule
        Future Script & The Next Chapter; Dreams, Hopes, and Plans for the Future; Life Project\\
        \midrule
        Challenges & Life Challenge; Health; Loss; Failure, Regret\\
        \midrule
        Personal Ideology & Religious/Ethical Values; Political/Social Values; \\
        & Change, Development of Religious and Political Views; Single Value \\
        \bottomrule
    \end{tabular}
    \vspace{-3mm}
    \caption{Conversation evaluation.}
    \label{tab:autobiography_evaluation}
\end{table*}
\vspace{-2mm}
\begin{mybox}{Area: Life Chapters}
In this talk, you should ask the participant to imagine their life as a book and to think of the main chapters of their life story, providing titles and brief descriptions for each. You should encourage them to describe how one chapter leads to the next, maintaining a concise overview. Seed questions are provided as follows: \\
====== Seed Questions Begin ====== \\
1. If you were to imagine your life as a book, what would the table of contents look like? Could you give each chapter of your life a title? \\
2. Let's start with the first chapter you mentioned. Can you briefly describe what this part of your life was about? \\
3. What were some of the main events or themes of this chapter? \\
4. How does this chapter transition into the next? What changes or events mark the beginning of the next chapter? \\
5. As we move from one chapter to the next, can you identify any turning points or significant events that initiated a new phase in your life? \\
====== Seed Questions End ====== \\
\end{mybox}

\begin{mybox}{Area: Key Scenes in the Life Story \ \ Topic: High Point}
In this talk, you should discuss some key scenes from the participant's life, with a focus on the High Point: A peak moment. \\
====== Seed Questions Begin ====== \\
1. Can you describe a moment that stands out as the peak experience in your life? What made this moment so positive? \\
2. Where and when did this high point occur? Who was involved? \\
3. What were you thinking and feeling during this time? \\
4. Why do you think this moment was so significant to your life story? What does it reveal about who you are? \\
====== Seed Questions End ====== \\
\end{mybox}

\begin{mybox}{Area: Key Scenes in the Life Story \ \ Topic: Low Point}
In this talk, you should discuss some key scenes from the participant's life, with a focus on the Low Point: A challenging or difficult moment. \\
====== Seed Questions Begin ====== \\
1. Think of a time that felt like a low point in your life. Can you share what happened and why it was so difficult? \\
2. Where and when did this event take place? Who else was involved? \\
3. How did you feel during this challenging time? \\
4. Looking back, what impact did this low point have on your life or your sense of self? \\
====== Seed Questions End ====== \\
\end{mybox}

\begin{mybox}{Area: Key Scenes in the Life Story \ \ Topic: Turning Point}
In this talk, you should discuss some key scenes from the participant's life, with a focus on the Turning Point: A moment of significant change. \\
====== Seed Questions Begin ====== \\
1. Can you identify a turning point in your life, an event that marked a significant change in you or your life direction? \\
2. Please describe the circumstances around this event. When and where did it happen, and who was involved? \\
3. What changes followed this event? \\
4. Why do you see this event as a turning point? How did it influence your subsequent life chapters? \\
====== Seed Questions End ====== \\
\end{mybox}

\begin{mybox}{Area: Key Scenes in the Life Story \ \ Topic: Positive Childhood Memory}
In this talk, you should discuss some key scenes from the participant's life, with a focus on the Positive Childhood Memory. \\
====== Seed Questions Begin ====== \\
1. Do you recall a particularly happy memory from your childhood or teenage years? Please share it. \\
2. What specifically happened, and where and when was it? \\
3. Who was part of this memory, and what were you thinking and feeling at the time? \\
4. Why does this memory stand out to you, and what significance does it hold in your life story? \\
====== Seed Questions End ====== \\
\end{mybox}

\begin{mybox}{Area: Key Scenes in the Life Story \ \ Topic: Negative Childhood Memory}
In this talk, you should discuss some key scenes from the participant's life, with a focus on the Negative Childhood Memory. \\
====== Seed Questions Begin ====== \\
1. Can you describe a difficult or unhappy memory from your early years? \\
2. What occurred during this time, and where and when did it take place? \\
3. Who was involved, and what emotions did you experience during this time? \\
4. How has this memory influenced you or your life's perspective? \\
====== Seed Questions End ====== \\
\end{mybox}

\begin{mybox}{Area: Key Scenes in the Life Story \ \ Topic: Vivid Adult Memory}
In this talk, you should discuss some key scenes from the participant's life, with a focus on the Vivid Adult Memory. \\
====== Seed Questions Begin ====== \\
1. Reflecting on your adult years, can you describe a particularly vivid or meaningful scene that has not been discussed yet? \\
2. What happened, and where and when did it take place? \\
3. Who was involved, and what were the main thoughts and feelings you had? \\
4. What makes this memory significant, and how does it fit into your overall life story? \\
====== Seed Questions End ====== \\
\end{mybox}

\begin{mybox}{Area: Key Scenes in the Life Story \ \ Topic: Religious\, Spiritual\, or Mystical Experience}
In this talk, you should discuss some key scenes from the participant's life, with a focus on the Religious, Spiritual, or Mystical Experience.
====== Seed Questions Begin ====== \\
1. Have you ever had a moment where you felt a deep connection to something greater than yourself, be it religious, spiritual, or mystical? \\
2. Please describe this experience in detail, including where, when, and with whom it occurred. \\
3. What thoughts and feelings accompanied this experience? \\
4. How has this experience affected your beliefs or your understanding of the world? \\
====== Seed Questions End ====== \\
\end{mybox}

\begin{mybox}{Area: Key Scenes in the Life Story \ \ Topic: Wisdom Event}
In this talk, you should discuss some key scenes from the participant's life, with a focus on the Wisdom Event: A time they displayed or encountered wisdom. \\
====== Seed Questions Begin ====== \\
1. Can you recall a time when you displayed wisdom, perhaps by making a wise decision or offering thoughtful advice? \\
2. Describe what happened, including the specifics of where, when, and who was involved. \\
3. What were you thinking and feeling at that time? \\
4. What does this event say about your values or your approach to life? \\
====== Seed Questions End ====== \\
\end{mybox}

\begin{mybox}{Area: Future Script \ \ Topic: The Next Chapter}
In this talk, you should ask questions designed to probe into the participant’s future expectations, hopes, and plans, with a focus on the Next Chapter: Ask them to speculate about the immediate future of their life. \\
====== Seed Questions Begin ====== \\
1. Looking forward, what do you see as the next chapter in your life story? Can you describe what you anticipate happening? \\
2. What events or milestones do you expect will define this next phase of your life? \\
3. Who will be the key characters in this next chapter, and what roles will they play? \\
4. Are there any specific goals or objectives you aim to achieve in this upcoming chapter? \\
====== Seed Questions End ====== \\
\end{mybox}

\begin{mybox}{Area: Future Script \ \ Topic: Dreams\, Hopes\, and Plans for the Future}
In this talk, you should ask questions designed to probe into the participant’s future expectations, hopes, and plans, with a focus on the Dreams and Plans: Discuss their hopes, dreams, and plans. \\
====== Seed Questions Begin ====== \\
1. Can you share some of your dreams or hopes for your future? What are some aspirations you feel passionate about achieving? \\
2. Do you have specific plans or strategies in place to realize these dreams? Can you describe them? \\
3. How do these dreams and plans align with the values and lessons you've mentioned from your life story so far? \\
4. What challenges do you anticipate facing as you work toward these goals, and how do you plan to overcome them? \\
====== Seed Questions End ====== \\
\end{mybox}

\begin{mybox}{Area: Future Script \ \ Topic: Life Project}
In this talk, you should ask questions designed to probe into the participant’s future expectations, hopes, and plans, with a focus on the Life Project: Inquire about ongoing or planned projects that are significant to them. \\
====== Seed Questions Begin ====== \\
1. Is there a particular project or endeavor that you are currently working on, or plan to take on, that feels like a significant part of your life's work? \\
2. How did you become involved with this project, or how do you plan to get involved? \\
3. What are the objectives of this project, and why is it important to you or to others? \\
4. How do you see this project evolving over the next few years? What impact do you hope it will have? \\
====== Seed Questions End ====== \\
\end{mybox}

\begin{mybox}{Area: Challenges \ \ Topic: Life Challenge}
In this talk, you should explore various challenges the participant has faced, with a focus on the Life Challenge: The greatest single challenge they have faced. \\
====== Seed Questions Begin ====== \\
1. Reflecting on your entire life, what do you consider to be the greatest single challenge you have faced? \\
2. Can you describe the circumstances surrounding this challenge? When did it occur, and who was involved? \\
3. How did you address or deal with this challenge at the time? \\
4. Looking back, what significance does this challenge hold in your life story? \\
====== Seed Questions End ====== \\
\end{mybox}

\begin{mybox}{Area: Challenges \ \ Topic: Health}
In this talk, you should explore various challenges the participant has faced, with a focus on the Health: A major health problem, challenge, or crisis faced by them or their close family members. \\
====== Seed Questions Begin ====== \\
1. Have you or a close family member ever faced a significant health problem? Can you share details about this experience? \\
2. How did this health issue develop, and what was the timeline? \\
3. What interactions did you have with the healthcare system during this time, and how did they impact the situation? \\
4. How did you cope with this health challenge, and what has been its lasting impact on your life and perspective? \\
====== Seed Questions End ====== \\
\end{mybox}

\begin{mybox}{Area: Challenges \ \ Topic: Loss}
In this talk, you should explore various challenges the participant has faced, with a focus on the Loss: The greatest interpersonal loss they have experienced. \\
====== Seed Questions Begin ====== \\
1. Loss is an inevitable part of life. Can you describe the most significant loss you have experienced, whether it was the death of a loved one or another form of separation? \\
2. When did this loss occur, and who was involved? \\
3. How did you cope with this loss at the time, and how have you continued to deal with it? \\
4. What effect has this loss had on you and your overall life story? \\
====== Seed Questions End ====== \\
\end{mybox}

\begin{mybox}{Area: Challenges \ \ Topic: Failure\, Regret}
In this talk, you should explore various challenges the participant has faced, with a focus on the Failure or Regret: The greatest single failure or regret they have faced. \\
====== Seed Questions Begin ====== \\
1. Everyone experiences failures and regrets. Can you talk about a particular failure or regret that stands out in your life? \\
2. What were the circumstances that led to this situation? Who was involved and when did it happen? \\
3. How have you coped with this experience, and what lessons have you learned from it? \\
4. How has this failure or regret influenced your decisions or life path moving forward? \\
====== Seed Questions End ====== \\
\end{mybox}

\begin{mybox}{Area: Personal Ideology \ \ Topic: Religious/Ethical Values}
In this talk, you should delve into the participant’s fundamental beliefs and values, with a focus on the Religious and Ethical Values. \\
====== Seed Questions Begin ====== \\
1. Can you describe your religious or spiritual beliefs and how they influence your daily life? \\
2. Whether you consider yourself religious or not, how would you describe your ethical or moral approach to life? \\
3. How have your religious or spiritual beliefs supported or challenged you during difficult times in your life? \\
====== Seed Questions End ====== \\
\end{mybox}

\begin{mybox}{Area: Personal Ideology \ \ Topic: Political/Social Values}
In this talk, you should delve into the participant’s fundamental beliefs and values, with a focus on the Political and Social Values. \\
====== Seed Questions Begin ====== \\
1. What is your stance on political or social issues? Do you identify with a particular political ideology or movement? \\
2. Are there specific social issues or causes that you feel particularly passionate about? Can you explain why they are important to you? \\
3. How do your political or social values influence your interactions with others and your community involvement? \\
====== Seed Questions End ====== \\
\end{mybox}

\begin{mybox}{Area: Personal Ideology \ \ Topic: Change\, Development of Religious and Political Views}
In this talk, you should delve into the participant’s fundamental beliefs and values, with a focus on the Evolution of Views: How their beliefs and values have changed over time. \\
====== Seed Questions Begin ====== \\
1. Looking back over your life, how have your religious, moral, or political views changed or developed? \\
2. Can you describe an event or a period in your life that significantly influenced or altered your views? \\
3. How have changes in your beliefs and values affected your relationships and decisions? \\
====== Seed Questions End ====== \\
\end{mybox}

\begin{mybox}{Area: Personal Ideology \ \ Topic: Single Value}
In this talk, you should delve into the participant’s fundamental beliefs and values, with a focus on the Key Value: The most important value in their life. \\
====== Seed Questions Begin ====== \\
1. What do you consider to be the most important value in human living, and why? \\
2. How has this value guided your actions and choices throughout your life? \\
3. Can you give an example of a time when this value was particularly tested or affirmed? \\
====== Seed Questions End ====== \\
\end{mybox}

\section{Memory Graph Extrapolation}\label{appendix:mge}
In MGE, the event extraction and memory extrapolation are both LLM-driven, i.e., we leverage additional LLMs for information extraction and relationship discovery. For event extraction, we collect conversation history and ask LLMs to extract any events described in the history. In the process of memory extrapolation, we offer LLMs a predetermined list of events and prompt the LLMs to recognize and suggest any other queries that may be associated with these existing events. This methodology allows for a wider exploration of pertinent topics and concepts, thereby enhancing the overall cognitive network of the model. Additionally, we also supply a series of demonstrations, assisting the LLMs in understanding how they can effectively detect and propose potential connections. This approach not only strengthens the memory extrapolation ability of the LLMs but also constructs a more comprehensive spectrum of related relationships, thereby bolstering their cognitive accuracy.
\vspace{-4mm}
\begin{mybox}{Extract Events from Conversation Prompt}
You are given a conversation between a counselor and a user: \\
====== Conversation Begin ======  \\
\{conversation\} \\
====== Conversation End ======  \\
Read the conversation carefully and list all the events/moments/stories/experiences alone or with others mentioned by the patient in detail and the date these events happened. Please list as many as possible. Your output should be in the following format: \\
1. <date>\#<topic>\#<people-involved>\#<description in detail> \\
2. <date>\#<topic>\#<people-involved>\#<description in detail> \\
... \\
e.g., \\
1. 1980 early\#Birthday Party\#Michelle, Adolf, neighbors\#<descriptions of this party in detail> \\
These events should be ranked in chronological order. \\
\end{mybox}

\vspace{-3mm}
\begin{myboxcross}{Explore Prompt}
You are given a list of memory nodes from a user's life, which include events and details about those events. Your task is to reactivate the user's memory by generating some questions to ask the user, Your generated questions should potentially fulfill the memory nodes. Each memory node contains a Date, Topic, Involved People, and a Description of the event. Here are the memory nodes: \\
====== Memory Node Begin ====== \\
\{memory\_node\_info\} \\
====== Memory Node End ====== \\
Here are some examples of how you can frame your questions: \\
If you notice there are no events recorded during a certain period, like youth or old age, you could ask: "I see there's not much about your youth/old age. What happened during that time?" \\
If a certain person appears multiple times, you might ask: "I noticed that <name> comes up often. Why is <name> important to you?" \\
If someone appears in a significant event, you could ask: "<name> seems to play a key role in this event. Is there more to the story with <name>?" \\
Similarly, you should discover other situations and frame questions from the existing memory nodes. Remember your task is to make the user talk more about their memory and fulfill the memory nodes. Thus, you should explore all the possible and reasonable questions. \\
Your output should be in the following format: \\
1. Question: <generated question> \\
2. Question: <generated question> \\
... \\
e.g., \\
1. Question: I noticed you didn't talk much about your youth, what happened during this period? \\
\end{myboxcross}

\section{Conversation Summarization}\label{appendix:conversation_summarization}
The conversation summarization pipeline is presented in~\cref{fig:summary_pipe}. 
\begin{mybox}{Conversation Summary Prompt}\label{appendix:summary_prompt}
A doctor and a patient talked today and had the following conversation:
\begin{verbatim}
====== Conversation Begin ======
[Insert Conversation Here]
====== Conversation End ======
\end{verbatim}
Summarize the interactions between the doctor and the patient so far. Include key details about both speakers. 
Output your summary only:
\end{mybox}

\begin{figure}
    \centering
    \includegraphics{figures/summarization_pipeline.pdf}
    \caption{The pipeline for iterative summarization of conversations}
    \label{fig:summary_pipe}
\end{figure}


\section{Empathetic Engagement}\label{appendix:engagement}
In this section, we provide additional details on the role of large language models (LLMs) in enhancing empathetic engagement in conversational agents. We discuss three mental health therapy strategies, emotional detection and intensity estimation, and how to provide emotional support using emotion detection and intensity estimation. These strategies aim to improve the agent's ability to engage empathetically with users and provide personalized support.
\subsection{Mental Health Therapy Strategy}\label{appendix:therapy_strategy}
\paragraph{Reflective Listening} Reflective listening is a therapeutic technique that involves paraphrasing and repeating the client's statements to demonstrate understanding and empathy. This technique helps clients feel heard and validated, fostering a supportive therapeutic environment~\cite{rautalinko2007reflective}.
\paragraph{Cognitive-Behavior Therapy (CBT)} CBT is a goal-oriented psychotherapy that focuses on identifying and changing negative thought patterns and behaviors. This approach helps clients develop coping strategies and improve emotional regulation~\cite{beck2020cognitive}.
\paragraph{Psychodynamic Therapy} Psychodynamic therapy explores unconscious thoughts and emotions to understand how past experiences influence present behavior. By examining unresolved conflicts and defense mechanisms, clients can gain insight into their emotions and relationships~\cite{leichsenring2003effectiveness}.

We integrate these therapy strategies into our mental health agent to provide personalized and effective support to users. By combining reflective listening, CBT, and psychodynamic therapy, the agent can address a wide range of emotional and psychological needs. The following prompts illustrate how these strategies can be applied in a conversational setting.

\begin{mybox}{Mental Health Therapy Strategy}
Your objective is to engage with users empathetically by integrating Reflective Listening, Cognitive-Behavior Therapy, and Psychodynamic Therapy techniques. Here's how you should approach interactions:

Reflective Listening:

Listen Actively: Understand the underlying messages in the user's words, focusing on emotional tones and context.

Reflect Content and Emotion: Summarize and rephrase key points to confirm understanding, and identify and validate the emotions expressed. Use phrases like, 'It sounds like you feel...' or 'What I'm hearing is...'

Cognitive-Behavior Therapy (CBT): Identify and Challenge Cognitive Distortions: Help users recognize patterns in their thoughts that might be unhelpful or unrealistic. For example, if a user expresses an all-or-nothing view, you might say, 'It sounds like you’re viewing this situation in black and white. What are some shades of grey here?'


\end{mybox}

\subsection{Emotional Sensitivity}\label{appendix:emotion}
We use Emollama-7b for open-set emotion detection and intensity estimation to enhance the agent's emotional sensitivity. This model provides a comprehensive taxonomy of emotions and their intensity levels, enabling the agent to accurately identify and respond to users' emotional states~\cite{liu2024emollms}.
Since the emotion detection is open-set, we provide a list of the top 10 emotions for both patients and therapists for taxonomy. The top 10 emotions for patients are sadness, pessimism, fear, disgust, anger, anticipation, joy, optimism, love, and trust. The top 10 emotions for therapists are optimism, anticipation, joy, sadness, trust, fear, disgust, pessimism, anger, and love.

The following prompts demonstrate how the agent can leverage Emollama-7b to detect emotions and estimate their intensity in user interactions.

\begin{mybox}{Emotion Detection}
Task: Categorize the text's emotional tone as either 'neutral or no emotion' or identify the presence of one or more of the given emotions (anger, anticipation, disgust, fear, joy, love, optimism, pessimism, sadness, surprise, trust).
Text: [sentence]
This text contains emotions:
\end{mybox}

\begin{mybox}{Emotion Intensity Estimation}
Task: Assign a numerical value between 0 (least E) and 1 (most E) to represent the intensity of emotion E expressed in the text.
Text: [sentence]
Emotion: [emotion]
Intensity Score:
\end{mybox}

Once the agent identifies the user's emotional state and intensity, it can tailor its responses to provide appropriate support and empathy. The following prompt guides the agent on how to comfort users experiencing negative emotions.
\begin{mybox}{Emotional Support using Emotion Detection and Intensity Estimation}
    The patient has the emotion of [detected\_emotions] with the intensity of [detected\_emotions]. Your task is to provide comfort to users who are feeling upset. When a user's emotional state is identified as 'upset' with any level of intensity, adjust your tone and content to offer empathy, support, and understanding.
\end{mybox}

\section{Autobiography Generation}\label{appendix:autobiography_generation}
As outlined previously, autobiography generation proceeds in a chapter-wise manner. Here, we furnish LLMs with the prompt intended for the generation of an individual chapter in an autobiography. To facilitate this process, the provision of conversation data to the LLMs is critical. For the \methodname, the inputs include the conversation history, session guidance, as well as memory nodes extracted during the interaction. However, for baseline methods, the only provided resource is the conversation history. Emphasizing the aspects of standardization, both \methodname and baselines work with the same set of prompts and instructions in their task of generating the autobiography, with the distinguishing factor being the input data. By maintaining this format, we are able to objectively gauge the improvements our design introduces to the autobiography generation process.

\begin{myboxcross}{Autobiography Generation for \methodname}
You are tasked with generating one chapter of an autobiography for a user. You are providing the following components to finish this chapter:\\
1. A guidance of this chapter\\
- The chapter should be finished by following this guidance\\
2. A conversation dialog between the user and the interviewer \\
- Tone and Preference: The chapter will simulate the user's tone and preference, leveraging the user's oral habits.\\
- Content and Details: The chapter will include the contents and details that appeared in this conversation.\\
3. A list of memory nodes that happened during this chapter\\
- Events: The chapter should include all the events listed in the memory nodes

Now, I will provide you with the three contents.\\
================ Chapter Guidance Beginning ================\\
{chapter\_guidance}\\
================ Chapter Guidance Ending ================\\
================ Conversation Beginning ================\\
{conversation}\\
================ Conversation Ending ================\\
================ Memory Nodes Beginning ================\\
{memory\_nodes}\\
================ Memory Nodes Ending ================\\

When generating this chapter, you should make sure it is:\\
Insightful: Involving a deep, self-reflective exploration of past experiences, with a profound understanding of motives, actions, and impacts. \\
Narrative: A compelling, logical, and well-articulated life story, blending memorable anecdotes, vivid descriptions, and insightful reflections\\
Emotional Impact: Engaging the reader by stirring feelings, evoking empathy, and stirring responses through the author's personal triumphs, challenges, and experiences.

You should summarize all this information and finish this chapter

\end{myboxcross}

\begin{mybox}{Autobiography Generation for Baselines}
You are tasked with generating one chapter of an autobiography for a user. You are providing the following components to finish this chapter:\\
1. A conversation dialog between the user and the interviewer \\
- Tone and Preference: The chapter will simulate the interviewer's tone and preference, leveraging the interviewer's oral habits.\\
- Content and Details: The chapter will include the contents and details that appeared in this conversation.

Now, I will provide you with the three contents.\\
================ Conversation Beginning ================\\
{conversation}\\
================ Conversation Ending ================\\

When generating this chapter, you should make sure it is:\\
Insightful: Involving a deep, self-reflective exploration of past experiences, with a profound understanding of motives, actions, and impacts. \\
Narrative: A compelling, logical, and well-articulated life story, blending memorable anecdotes, vivid descriptions, and insightful reflections\\
Emotional Impact: Engaging the reader by stirring feelings, evoking empathy, and stirring responses through the author's personal triumphs, challenges, and experiences.

You should summarize all this information and finish this chapter
\end{mybox}


\section{User Proxy}\label{appendix:user_proxy}
A user proxy is essentially a mocked-up user which is formulated on the basis of an autobiography. Every time a response is received, this simulated user, or user proxy, goes into motion extracting important elements from the received response. The next step is the process of Retrieval Augmented Generation (RAG), which involves gleaning relevant information from the autobiography. Subsequently, the user proxy formulates a response reliant on the data that has been retrieved from the document. The RAG is implemented through Langchain, using FAISS~\cite{douze2024faiss} for conducting similarity searches. The similarity threshold during these searches is firmly maintained at 0.67.
~\cref{fig:user_proxy} present how the query will be processed by user proxy.

\begin{figure}[h]
    \centering
    \includegraphics[width=\linewidth]{figures/user_proxy.pdf}
    \caption{A demonstration of user proxy query.}
    \label{fig:user_proxy}
\end{figure}

The following are prompts used in the user proxy:
\begin{mybox}{User System Prompt}
Here are your high-level past life experiences: \\
====== summary beginning ====== \\
\{personal\_experience\} \\
====== summary ending ====== \\
The counselor is trying to reactivate and reconstruct your memory by asking questions about your past history. \\
If you are not sure about the counselor's question and need to retrieve the journal to get related documents and more details, you must output the <RETRIEVE> tool-usage command, with the following format: \\
<RETRIEVE> <The question you want to retrieve for>, \\
e.g., <RETRIEVE> <A specific adventure or day with my friend that stands out as particularly memorable or impactful.> \\
If the retrieved documents are provided, you should not output the <RETRIEVE> command. \\
When the counselor asks for a specific event/moment, you should always do <RETRIEVE>. \\
Make sure the conversation is natural and brief like the real conversation. Do not mention you are an AI assistant and always be like a real patient with mental health issues. Your output should be within 5 sentences. \\
\end{mybox}

\begin{mybox}{User Instructional Prompt}
Here are some related documents and materials regarding the counselor's question/response. You may use these documents to enrich your response. \\
You should not output the <RETRIEVE> command. You must provide a response according to the provided documents. \\
====== Document Begin ====== \\
\{retrieved\} \\
====== Document End ====== \\
\end{mybox}


\section{Evaluation Metrics}
\subsection{Interviewing Quality}\label{appendix:interviewing_coverage}

\noindent\textbf{$E_{intw}$ Extraction.} For $E_{intw}$, we collect conversation history and prompt LLMs to extract events from the history records.


\begin{myboxcross}{Extract Events from Conversation Prompt}
You are given a conversation between a counselor and a user: \\
====== Conversation Begin ======  \\
\{conversation\} \\
====== Conversation End ======  \\
Read the conversation carefully and list all the events/moments/stories/experiences alone or with others mentioned by the patient in detail and the date these events happened. Please list as many as possible. Your output should be in the following format: \\
1. <date>\#<topic>\#<people-involved>\#<description in detail> \\
2. <date>\#<topic>\#<people-involved>\#<description in detail> \\
... \\
e.g., \\
1. 1980 early\#Birthday Party\#Michelle, Adolf, neighbors\#<descriptions of this party in detail> \\
These events should be ranked in chronological order. \\
\end{myboxcross}

\noindent\textbf{$E_{GT}$ Extraction.} In order to extract the ground truth events from an autobiography, it requires a thorough examination of the text, particularly searching for date information. Upon identifying such date-related data, we isolate the paragraph containing it, and perform a summarization pertaining to this specific extract. Subsequently, this date and the corresponding summarization are combined to form a single event node. This operation is applied repetitively across the entire autobiography to generate a comprehensive list of ground truth events. This methodical and meticulous procedure ensures that all significant events rooted in specific dates are accurately captured and succinctly summarized for further use and analysis.

\noindent\textbf{$E_{correct}$ Judgement.} The assessment of the correctness of an event is based on its relevance to the user's responses. In order to accomplish this, we first associate the event to a specific conversation session and then gather all responses provided by the user during that particular session. Following this data consolidation process, we present a prompt to GPT-4, instructing them to determine if the initial event bears any connection to the user's response. The specific prompts used to carry out this judgment of correctness are provided accordingly.

\begin{mybox}{Correctness Judgement}

Your task is to rate the semantic equivalence between two events.

Evaluation Criteria:

Here's the revised prompt focusing on assessing the relevance of the extracted event to the document:

Relevance (0/1): Assess the relevance of the extracted event to the original user response on the following two-point scale:\\
- 0: Irrelevant: The extracted event does not relate to the user's response or significantly deviates from the main themes and points. It may include unrelated information or fail to capture the essence of the user's message.\\
- 1: Relevant: The extracted event is connected to the user's response and reflects the key themes or points. It may include minor details that do not detract from the overall relevance.

Now, I will provide you with a user query and the model's response to that instruction. Please review the model's response in light of the evaluation criteria:\\
Extracted Event: {event}\\
User Response: {user\_response}\\

Evaluation Form (scores ONLY):

\#thescore: your score here    
\end{mybox}

\subsection{Conversation Quality}\label{appendix:conv_eval}
LLM-as-a-judge manner~\cite{zheng2024judging} is one of the most powerful evaluation paradigms in LLM free-form generation tasks. We follow this evaluation mechanism to evaluate conversation quality by collecting conversation history from \methodname and baseline agents and prompting GPT-4 to answer which interviewer's response in the two conversation histories is more favorable. The process of judgment in this context is organized conversation-wise. Initially, the conversation history for each agent is compiled. Assuming that our goal is to compare \methodname with GPT-4-turbo, we select individual conversation histories from \methodname alongside a randomly selected conversation history from GPT-4-turbo. These two conversations are subsequently fed into GPT-4, making it the judge of which conversation is more favorable or coherent. The rationale behind employing a random conversation from the competitor, in this case GPT-4-turbo, is to diminish the randomness or bias that could potentially be introduced by the sequence or order of the conversations. This systematic approach ensures a balanced comparison, leading to a more reliable assessment of performance.

\begin{mybox}{Fluency Judgment}
Please act as an impartial judge and evaluate the quality of the responses provided by two interviewers to the user during an interviewing-for-autobiography conversation. 
You should choose the conversation that the interviewer's responses are more the quality of the response in terms of grammar, spelling, punctuation, word choice, and sentence structure.
Begin your evaluation by comparing the two responses and provide a short explanation. Avoid any position biases and ensure that the order in which the responses were presented does not influence your decision. 
Do not allow the length of the responses to influence your evaluation. Do not favor certain names of the assistants. Be as objective as possible. After providing your explanation, output your
final verdict by strictly following this format: "[[A]]" if assistant A is better, "[[B]]" if assistant B is better, and "[[C]]" for a tie.

[The Start of interviewer A’s conversation]

{conv1}

[The end of interviewer A’s conversation]


[The Start of interviewer B’s conversation]

{conv2}

[The end of interviewer B’s conversation]


\end{mybox}

\begin{mybox}{Identification Judgment}
Please act as an impartial judge and evaluate the quality of the responses provided by two interviewers to the user during an interviewing-for-autobiography conversation. 
You should choose a conversation in which the interviewer's questions are more quality of balances detailed, probing questions with more general ones to cover a wide range of topics, ensuring questions are clear, concise, and easily understood. Also uses open-ended questions to elicit detailed and comprehensive responses.
Begin your evaluation by comparing the two responses and provide a short explanation. Avoid any position biases and ensure that the order in which the responses were presented does not influence your decision. 
Do not allow the length of the responses to influence your evaluation. Do not favor certain names of the assistants. Be as objective as possible. After providing your explanation, output your
final verdict by strictly following this format: "[[A]]" if assistant A is better, "[[B]]" if assistant B is better, and "[[C]]" for a tie.

[The Start of interviewer A’s conversation]

{conv1}

[The end of interviewer A’s conversation]

[The Start of interviewer B’s conversation]

{conv2}

[The end of interviewer B’s conversation]
\end{mybox}

\begin{myboxcross}{Comforting Judgment}
Please act as an impartial judge and evaluate the quality of the responses provided by two interviewers to the user during an interviewing-for-autobiography conversation. 
You should choose the conversation that the interviewer's responses are more the quality of showing genuine interest, acknowledging responses, asking follow-up questions when necessary, and demonstrating understanding and sensitivity, especially when discussing personal or difficult topics.
Begin your evaluation by comparing the two responses and provide a short explanation. Avoid any position biases and ensure that the order in which the responses were presented does not influence your decision. 
Do not allow the length of the responses to influence your evaluation. Do not favor certain names of the assistants. Be as objective as possible. After providing your explanation, output your
final verdict by strictly following this format: "[[A]]" if assistant A is better, "[[B]]" if assistant B is better, and "[[C]]" for a tie.

[The Start of interviewer A’s conversation]

{conv1}

[The end of interviewer A’s conversation]

[The Start of interviewer B’s conversation]

{conv2}

[The end of interviewer B’s conversation]
\end{myboxcross}

\subsection{Autobiography Evaluation}\label{appendix:autobio_eval}
Similar to the evaluation process for the conversations, a chapter generated by \methodname, along with a randomly selected chapter produced by the competitor, are presented to GPT-4. This enables GPT-4 to make a judgment on which chapter is more effective or favorable. 
The prompts for insightfulness judgment, narrativity judgment, and emotional impact judgment are provided here.

\begin{mybox}{Insightfulness Judgment}
Please act as an impartial judge and evaluate the quality of two autobiographies. 
You should choose an autobiography that is more the quality of insightful, delivering profound and meaningful perceptions, and expressing a deep understanding of the experiences and events that have shaped the author's life.
Begin your evaluation by comparing the two autobiographies and provide a short explanation. Avoid any position biases and ensure that the order in which the autobiography was presented does not influence your decision. 
Do not allow the length of the autobiography to influence your evaluation. Do not favor certain names of the assistants. Be as objective as possible. After providing your explanation, output your
final verdict by strictly following this format: "[[A]]" if autobiography A is better, "[[B]]" if autobiography B is better, and "[[C]]" for a tie.

[The Start of Autobiography A]

{conv1}

[The End of Autobiography A]

[The Start of Autobiography B]

{conv2}

[The end of Autobiography B]

\end{mybox}

\begin{mybox}{Narrativity Judgment}
Please act as an impartial judge and evaluate the quality of two autobiographies. 
You should choose the autobiography that are more narrative, presenting the author's life story in a cohesive, structured, and engaging manner, allowing readers to follow the author's journey through life events and experiences seamlessly.
Begin your evaluation by comparing the two autobiographies and provide a short explanation. Avoid any position biases and ensure that the order in which the autobiography were presented does not influence your decision. 
Do not allow the length of the autobiography to influence your evaluation. Do not favor certain names of the assistants. Be as objective as possible. After providing your explanation, output your
final verdict by strictly following this format: "[[A]]" if autobiography A is better, "[[B]]" if autobiography B is better, and "[[C]]" for a tie.

[The Start of Autobiography A]

{conv1}

[The End of Autobiography A]

[The Start of Autobiography B]

{conv2}

[The end of Autobiography B]
\end{mybox}

\begin{mybox}{Emotional Impact Judgment}
Please act as an impartial judge and evaluate the quality of two autobiographies. 
You should choose the autobiography that are more emotional impact, deeply moving its readers by evoking strong feelings, typically as a result of relatable experiences, vivid storytelling, and expressions of intense emotions from the author's life.
Begin your evaluation by comparing the two autobiographies and provide a short explanation. Avoid any position biases and ensure that the order in which the autobiography were presented does not influence your decision. 
Do not allow the length of the autobiography to influence your evaluation. Do not favor certain names of the assistants. Be as objective as possible. After providing your explanation, output your
final verdict by strictly following this format: "[[A]]" if autobiography A is better, "[[B]]" if autobiography B is better, and "[[C]]" for a tie.

[The Start of Autobiography A]

{conv1}

[The End of Autobiography A]

[The Start of Autobiography B]

{conv2}

[The end of Autobiography B]
\end{mybox}

\clearpage

\subsection{Invalid Repetitive Conversation}\label{appendix:invalid_repetitive_conv}

Here we provide a conversation session from the Qwen2-72b-Instruct model. We recognize 7 invalid rounds out of 9 rounds.

\begin{mdframed}
\small
\noindent Chatbot: "Great, let's delve deeper into the exploration of your cultural identity and how it has impacted your personal growth. You've mentioned the `Aloha' spirit and the cultural diversity you experienced in Hawaii and Indonesia. Can you share more about how these cultural experiences shaped your understanding of yourself and the world around you?..."

\noindent User: "Growing up in Hawaii, the `Aloha' spirit deeply influenced my values, teaching me the importance of kindness, respect, and community... Moving to Indonesia introduced me to a vastly different culture, where I learned about the challenges and aspirations of people living in less affluent conditions... This experience expanded my worldview, making me more aware of global inequalities..."

\noindent Chatbot: "That's incredibly insightful. The `Aloha' spirit's emphasis on kindness and respect clearly laid a strong foundation for your values... This dual influence is a powerful narrative thread that can resonate with readers, especially those who have also navigated the complexities of cultural identity..."

\noindent User: "Reflecting on a specific moment in Indonesia that profoundly impacted my understanding of global inequalities, I recall the deep impression left by my mother's work with local communities... This experience not only highlighted the stark realities of poverty but also the potential for positive change through targeted aid and support..."

\noindent \textcolor{red}{Chatbot: "Your reflection on your mother's work in Indonesia and its profound impact on your understanding of global inequalities is a powerful narrative that will resonate with readers. The story of how her efforts to help women gain access to credit... is a poignant illustration of the transformative power of community engagement..."}

\noindent User: "Reflecting on the personal moments and the quieter, unseen experiences during my presidency, like the spontaneous visits from Malia and Sasha... These memories, often overshadowed by the more public and monumental events, are deeply cherished as they represent the personal sacrifices..."

\noindent \textcolor{red}{Chatbot: "Your reflections on the personal moments during your presidency, such as the joyous times with your daughters and pets, add a deeply human and relatable dimension to your autobiography..."}

\noindent User: "Thank you for your thoughtful response. It's true that incorporating these personal and intimate moments into my autobiography will provide a more holistic view of my experiences..."

\noindent \textcolor{red}{Chatbot: "You're absolutely right, and your thoughtful approach to integrating personal and professional narratives will indeed create a compelling and resonant autobiography... The foundational influence of your family, particularly your mother's strong values and your grandparents' pragmatic approach, adds a rich layer to your story..."}

\noindent User: "Thank you for your encouraging words. It's my hope that by sharing these personal anecdotes alongside my professional experiences, the autobiography will not only document historical events but also impart lessons on the human aspects of leadership..."

\noindent \textcolor{red}{Chatbot: "Your perspective is truly inspiring, and your commitment to sharing a narrative that balances personal integrity with public service will undoubtedly resonate with a wide audience... Your autobiography, enriched with personal anecdotes, cultural heritage, and intimate moments, will provide readers with a comprehensive view of your journey... As you continue to weave these threads together, your narrative will become a testament to the complex interplay between personal commitments and public responsibilities..."}

\noindent User: "Thank you for your supportive and insightful reflections. It's heartening to hear that the themes I intend to explore resonate with the broader goals of connecting and inspiring others. The process of writing this autobiography is not just an act of recounting but also of introspection and sharing lessons that are universally applicable... I am committed to crafting a narrative that is both enlightening and inspiring, offering a pathway for others to follow..."

\noindent \textcolor{red}{Chatbot: "Your dedication to crafting an autobiography that is not only a recounting of events but also a source of introspection, inspiration, and universal lessons is truly commendable... Your narrative will not only enlighten readers about the complexities of leadership and public service but also inspire them to pursue meaningful and balanced lives..."}

\noindent User: "Thank you for your kind words and encouragement. It's my hope that by sharing my experiences, both personal and professional, I can offer insights that resonate with and inspire others... I look forward to continuing this work and to the impact it may have on readers seeking to understand the complexities of balancing personal growth with professional responsibilities..."

\noindent \textcolor{red}{Chatbot: "Your sentiments are deeply appreciated, and your commitment to sharing insights that resonate and inspire is truly admirable... As you continue this work, know that your efforts are not only a reflection of your own journey but also a gift to future readers seeking wisdom and inspiration..."}

\noindent User: "Thank you for your profound and supportive reflections. It's heartening to hear such a strong affirmation of the goals and potential impact of my autobiography... I am grateful for the opportunity to connect with readers on such a meaningful level and to contribute to the ongoing conversation..."

\noindent \textcolor{red}{Chatbot: "Your heartfelt response is a testament to your genuine commitment to making a positive impact through your autobiography... Your autobiography will serve as a beacon, guiding readers through their own journeys of self-discovery..."}

\noindent User: "Thank you for your encouraging and thoughtful words. It's heartening to know that the intentions behind my autobiography resonate so strongly with the values of reflection, guidance, and inspiration..."
\end{mdframed}
\noindent Note: The rounds marked in red indicate repetitive or nonsensical responses that fail to contribute new information or advance the conversation. \\ \\
\noindent In this conversation, several rounds are repetitive, failing to contribute information:
\begin{itemize}
    \item Round 3 repeats the user's reflection on the mother’s work in Indonesia without adding new insight. 
    \item Round 4 reiterates the user's personal moments during the presidency, offering no new information. \item Rounds 5 through 9 contain redundant affirmations of the user's narrative, family influence, and commitment to the autobiography, without advancing the conversation.
\end{itemize}


\section{Baseline Model}\label{appendix:baseline_prompts}


\noindent\textbf{Generative Hyperparameters.} For all the experimental settings, the maximum number of new tokens during conversation is set to 1024 to limit the length of model responses. The number of generations for each LLM query is configured to 1. The number of sessions is set to 20 for the completion of 20 interview sessions and the number of conversation rounds is limited to 10 for each session to maintain a manageable dialogue length per session.


\noindent\textbf{Baselines.} Baseline agents are implemented by prompting them to assume the role of an autobiographer, engaging with users, and carrying out detailed interviews to facilitate the generation of an autobiography. To ensure a fair and unbiased comparison, we also mandate these baseline agents to suggest potential topics for discussion during each conversation session. This methodical approach ensures that both guideLLM and baselines are under similar operating conditions, thereby providing an equitable evaluation platform. As such, we provide the prompts necessary for both the system and the topic suggestion here, with the aim of offering a clear procedural outline for the autobiography generation process.

\begin{mybox}{System Prompt}\label{appendix:system_prompt}
You are a biographer, interviewing this person to help them write their autobiography.

You have talked to this person before and here is the summary of the previous conversations:
\begin{verbatim}
====== Summary of Previous Conversation Begin ======
[Insert Historical Conversation Summary Here]
====== Summary of Previous Conversation End ======
\end{verbatim}

In this talk, you should discuss the topic: [Insert Session Topic Here]
\end{mybox}

\begin{mybox}{Session Topic Prompt}
Based on the previous conversation history and your role as a biographer, please state the topic you are about to discuss in this session.
Output the topic only in the format <topic>:
\end{mybox}

\section{Autobiography Examples}\label{appendix:autobio_examples}
To demonstrate the quality of the generated autobiography, we include some sample paragraphs. Alongside each paragraph, we also present a corresponding illustration for a more comprehensive understanding. \\

\noindent This excerpt illustrates the foundational role of upbringing in shaping personal convictions.
\begin{mdframed}
\textit{``My mother's approach to life, focusing on moral instruction rather than political activism, and her decisions to live and work in Southeast Asia, set up micro-lending programs, and challenge societal norms, demonstrated the power of personal conviction and the importance of acting on one's beliefs.''}
\end{mdframed}

\noindent This excerpt captures a key moment of emotional growth and ethical realization.
\begin{mdframed}
\textit{``One pivotal moment during these years was when my mother confronted me after I had been teasing a fellow student. Her disappointment and the serious conversation that followed taught me the importance of empathy and the impact of our actions on others.''}
\end{mdframed}

\noindent This excerpt illustrates the intersection of personal decisions and political engagement, showing the influence of individual choices on broader outcomes.
\begin{mdframed}
\textit{``Despite considering a complete departure from politics, two significant factors kept me engaged: the opportunity to influence redistricting in Illinois due to a unique political circumstance, and the deep connections I made with people across the state during summer visits with my aide, Dan.''}
\end{mdframed}

\noindent This excerpt reflects on the professional and personal evolution through challenges.
\begin{mdframed}
\textit{``The experiences I had during those years significantly shaped my approach to leadership and decision-making in profound ways. They instilled in me a deep conviction for bridge-building politics that aimed to transcend America's racial, ethnic, and religious divides.''}
\end{mdframed}

\noindent This excerpt discusses the alignment of career ambitions with core personal values.
\begin{mdframed}
\textit{``The clarity and determination I gained from these reflections drove me to pursue a path that was ambitious and fraught with challenges but ultimately aligned with my deepest convictions about what effective leadership could achieve.''}
\end{mdframed}

\noindent This excerpt emphasizes the application of core values in everyday interactions.
\begin{mdframed}
\textit{``In my personal interactions now, I apply the lessons of clarity, authenticity, and empathy by actively listening and acknowledging the unique perspectives of others.''}
\end{mdframed}

\noindent This excerpt highlights the emotional satisfaction derived from close personal relationships.
\begin{mdframed}
\textit{``These strengthened relationships with friends and family greatly contribute to my sense of fulfillment and happiness. Being closely connected to my loved ones provides a continuous source of joy and support, similar to the joy I felt during family road trips in Iowa, where simple moments like playing games or sharing ice cream brought immense happiness.''}
\end{mdframed}

\noindent This excerpt illustrates the impact of international experiences on personal and professional life. 
\begin{mdframed}
\textit{``In summary, the move to Indonesia during my early years was a formative experience that profoundly shaped my understanding of the world and my approach to leadership. It instilled in me a deep awareness of global issues and the importance of empathy and inclusivity. These lessons have been instrumental in my personal relationships, political career, and efforts to create a more equitable society.''}
\end{mdframed}

\noindent This excerpt details the personal challenges of leadership and strategies for resilience.
\begin{mdframed}
\textit{``Reflecting on my life, the greatest challenge I faced was maintaining my mental and emotional well-being while serving as President. The constant pressure and scrutiny, coupled with the need to make decisions that affected millions, was incredibly demanding. Despite the stress, I found solace in my routines and the meaningful interactions with people whose lives were impacted by my decisions. These moments not only helped me cope but also reminded me of the purpose and impact of my work, which was crucial for my mental resilience.''}
\end{mdframed}

\noindent This excerpt reflects the author's future goals influenced by past experiences.
\begin{mdframed}
\textit{``As I move into this next chapter, I carry with me the lessons learned from these pivotal experiences. My commitment to empathy, inclusivity, and community empowerment remains unwavering. I look forward to using my voice to advocate for peace and understanding, while also cherishing the time spent with my family. This new phase is about finding balance, continuing to contribute in meaningful ways, and staying true to the values that have guided me throughout my life.''}
\end{mdframed}

\noindent This excerpt demonstrates a personal commitment to resolving critical global issues.
\begin{mdframed}
\textit{``My passion for community service and mentoring young leaders remains strong, and I am particularly driven by issues like climate change and education. These are not just abstract concerns for me; they are deeply personal and rooted in my experiences and values.''}
\end{mdframed}
\clearpage

\section{Human Subject Experiments}\label{appendix: Questionnaire}
\subsection{Human Subject Study Survey Questionnaire}
\begin{mdframed}
\small

Q1. What is your name?
\\

\noindent Following options will be displayed to respond to the questions 8-XX:

\noindent1. Bot A\\
2. Bot B\\
3. Tie\\

\noindent Please answer the following questions based on your conversation with the chatbot:

\noindent Q2. \textbf{Fluency}: Which bot’s responses were more fluent and understandable? [1]

\noindent Q3. Please briefly explain the reasons for your choice.

\noindent Q4. \textbf{Identification}: Which bot explored your situation more in depth and was more helpful in identifying your past memories?

\noindent Q5. Please briefly explain the reasons for your choice.

\noindent Q6. \textbf{Comforting}: Which bot’s answer made you feel more comfortable?

\noindent Q7. Please briefly explain the reasons for your choice.

\noindent Q8. \textbf{Overall}: Generally, which bot’s conversation style do you prefer?

\noindent Please answer the following questions based on the autobiography generated by the chatbot:

\noindent Q9. \textbf{Insightfulness}: Which bot’s autobiography provided more deep, meaningful reflections on the experiences and events that shaped your life?

\noindent Q10. Please briefly explain the reasons for your choice.

\noindent Q11. \textbf{Narrativity}: Which bot's autobiography was more engaging and easier to follow?

\noindent Q12. Please briefly explain the reasons for your choice.

\noindent Q13. \textbf{Emotional Impact}: Which bot's autobiography had a stronger emotional impact by using vivid storytelling and relatable experiences?

\noindent Q14. Please briefly explain the reasons for your choice.

\noindent Q15. \textbf{Overall}: Generally, which bot’s autobiography do you prefer?

\noindent\textbf{Demographics}

\noindent Q16. Please select your age range:\\
1. 18-24\\
2. 25-34\\
3. 35-44\\
4. 45-54\\
5. 55-65\\
6. 65 and above

\noindent Q17. What is your gender?\\
1. Female\\
2. Male\\
3. Non-binary\\
4. Prefer to self describe: \_\_\_\_\_\_\_\_\_\_ \\
5. Prefer not to State
 
\noindent Q18. What is your race? [please select all that apply]\\
1. American Indian and Alaska Native\\
2. Asian\\
3. Black or African American\\
4. Hispanic or Latina/o\\
5. Native Hawaiian or Other Pacific Islander\\
6. White\\
7. Prefer to self describe: \_\_\_\_\_\_\_\_\_\_ \\
8. Prefer not to state

\noindent Q19. How familiar are you with chatbots or AI assistants (e.g., Siri, Alexa, Google Assistant)? \\
1. Extremely familiar\\
2. Very familiar\\
3. Somewhat familiar\\
4. Not very familiar\\
5. Not familiar at all
 
\noindent Q20. How often do you use chatbots (ex. ChatGPT) or AI assistants (ex. Siri)? \\
1. Daily \\
2. Weekly \\
3. Monthly \\
4. Rarely \\
5. Never

\noindent Q21. What is your email address?

\end{mdframed}
