\section{Parameters and Floating Operation Estimation}
\label{sec:parameters-and-floating-operation-estimation}

Here we outline the number of parameters (\Cref{ssec:model-parameters})
and the number of FLOPs per token (\Cref{ssec:flops-per-token}) for our experimental settings.
The symbol notation is provided in \Cref{tab:param-flop-notation}.
For our scaling laws, we find, as in \citet{DBLP:journals/corr/abs-2001-08361}
using that the number of \emph{non-embedding-parameters} provides the cleanest fit and extrapolation behavior.

Our expressions for approximate compute (FLOPs per token) differ from prior work in that we are interested in \emph{small models that are capable}.
This means we are unable to ignore the context-dependent term that arises from the quadratic computational complexity of the attention mechanism.
As our architectures are \emph{fixed aspect ratio}, there is a modified approximation we can use.
This expression is discussed in \Cref{ssec:alternative-approximation-for-flops-per-token-as-a-function-of-n}

For ease of reference, we provide a comparison of the expressions we use to commonly used existing expressions \citep{DBLP:journals/corr/abs-2001-08361,DBLP:journals/corr/abs-2203-15556,DBLP:conf/sc/NarayananSCLPKV21},
and provide comments for significant differences.

\begin{table}[h]
        \vspace{-0.1cm}
	\caption{The notation we use for parameter and \flops estimation.}
	\label{tab:param-flop-notation}
	\centering
	\rowcolors{2}{AppleChartGrey2}{white}
	\small
	\begin{tabular}{lc}
		\toprule
		Component                                   & Notation          \\
		\midrule
		Sequence length/context size                & $\nctx$           \\
		Vocabulary size                             & $\nvocab$         \\
		Number of blocks/layers                     & $\nlayers$        \\
		Number of query heads                       & $\nheads$         \\
		Number of key/value heads                   & $\nkvheads$       \\
		Model/embedding dimension                   & $\dmodel$         \\
		Head dimension                              & $\dhead$          \\
		Feed-forward dimension                      & $\dffn$           \\
		Number of feed-forward linears              & $\nffn$           \\
		Group size in \gls{gqa} $\nheads/\nkvheads$ & $g_{\text{size}}$ \\
		Model aspect ratio $\dmodel/\nlayers$       & $\rmodel$         \\
		Feed-forward ratio $\dffn/\dmodel$          & $\rffn$           \\
		\bottomrule
	\end{tabular}
        \vspace{-0.15cm}
\end{table}


\FloatBarrier
\subsection{Alternative approximation for FLOPs per token as a function of \texorpdfstring{$N$}{N}}
\label{ssec:alternative-approximation-for-flops-per-token-as-a-function-of-n}

From \Cref{tab:param-count-simple} and \Cref{eq:non-embedding-parameters}
and
\Cref{tab:flops-count-simple}
we can read our approximate values for non-embedding parameters and total compute (dropping contributions from normalization layers) as\footnote{It was shown in \citet{DBLP:journals/corr/abs-2406-19146} that ignoring the embedding parameters and \flops can lead to systematic estimation bias for small models, and is one of the primary drivers between different exponents reported in \citet{DBLP:journals/corr/abs-2001-08361} and \citet{DBLP:journals/corr/abs-2203-15556}.
We find that the the \emph{non-embedding parameters} gives a tighter scaling behavior.
However, in the \emph{fixed-aspect-ratio} setting, we are able to use both the
\emph{non-embedding parameters} in the scaling law \emph{and} the \emph{approximate} total compute simultaneously, removing estimation bias.
Indeed, in the supervised setting, our coefficients $a$ and $b$ are consistent with those from \citet{DBLP:journals/corr/abs-2203-15556} (see \Cref{tab:scaling-law-parameter-estimates}).}
\begin{align}
	N                & =\nlayers\dmodel^2\left(2+ \frac2\gsize + \nffn\rffn\right) \\
	C_\text{Forward} & =
	2\nlayers\dmodel^2\left(2+ \frac2\gsize + \nffn\rffn\right)
	+2\nlayers\nctx\dmodel                                                         \\
	                 & =
	2N
	+2\nlayers\nctx\dmodel
    +2\nvocab\dmodel.
	\label{eq:c-forward-as-n}
\end{align}
Typically the term $2\nlayers\nctx\dmodel$ would be dropped, and the embedding parameters included into the total parameters \citep{DBLP:journals/corr/abs-2203-15556}
or discarded \citep{DBLP:journals/corr/abs-2001-08361}
yielding the expression $C_\text{Forward}$ and the familiar expression $C=6ND$ \citep{DBLP:journals/corr/abs-2001-08361,DBLP:journals/corr/abs-2203-15556}.
For our investigation we are interested in small, capable models, which may have a large context,
and so both of these terms cannot be ignored in general at the peril of making a systematic error in the region of configuration space we are most interested in.
Fortunately, we will see that our choice of \emph{fixed aspect ratio} $\rmodel=\dmodel/\nlayers$ architectures allows us a simple to use, more precise estimate.
The trick will be to use this fixed aspect ratio to come up with an approximation for $\nlayers$ and $\dmodel$ as a function of $N$ and $\rmodel$.
With these approximated, the term $2\nlayers\nctx\dmodel$ can be represented as a function of $N$.
First define\footnote{In our setting (\Cref{sec:model-architecture})
	$\omega$ takes values
	\begin{align}
		\omega
		 & =2+ \frac2\gsize + \nffn\rffn
		=2+ \frac21 + 3\times\frac83=12.
	\end{align}}
\begin{equation}
	\omega \equiv	2+ \frac2\gsize + \nffn\rffn
\end{equation}
so that
\begin{align}
	N
	 & =\nlayers\dmodel^2\omega.
\end{align}
Then we can substitute in $\rmodel\equiv\dmodel/\nlayers$ so that
\begin{align}
	N
	 & =\nlayers\dmodel^2\omega
	=\nlayers^3\rmodel^2\omega,
\end{align}
and solve for $\nlayers$ and $\dmodel$
\begin{align}
	\nlayers &= \left(\frac{N}{\rmodel^2\omega}\right)^{1/3},
    &
    \dmodel &= \left(\frac{N\rmodel}{\omega}\right)^{1/3},
\end{align}
The $C_\text{Forward}$ term can then be represented as a function of $N$.
The context-dependent term becomes
\begin{align}
	2\nctx\nlayers\dmodel
	 & =
	2\nctx\nlayers^2\rmodel
	=
	2\left(\frac{N}{\rmodel^2\omega}\right)^{2/3}\rmodel\nctx
	\equiv
    2\nctx\sigma_1 N^{2/3}
\end{align}
where
\begin{equation}
	\sigma_1
	=\left(\frac{1}{\rmodel^2\omega}\right)^{2/3}\rmodel
	=\left(\frac{1}{\rmodel\omega^2}\right)^{1/3}.
\end{equation}
The vocabulary projection term becomes
\begin{equation}
    2\nvocab\dmodel
    =2\nvocab\left(\frac{N\rmodel}{\omega}\right)^{1/3}
    =2\nvocab\left(\frac{\rmodel}{\omega}\right)^{1/3}N^{1/3}
    \equiv 2\nvocab\sigma_2N^{1/3},
\end{equation}
where
\begin{equation}
	\sigma_2
	=\left(\frac{\rmodel}{\omega}\right)^{1/3}.
\end{equation}
In total
\begin{equation}
	C_\text{Forward}
	=
	2N
    + 2\nctx\sigma_1 N^{2/3}
    + 2\nvocab\sigma_2 N^{1/3}
    =
    2N\left(1+\sigma_1 \frac{\nctx}{N^{1/3}} + \sigma_2 \frac{\nvocab}{N^{2/3}}\right),
    \label{eq:forward-flops-sigma}
\end{equation}
where $\sigma_1$ and $\sigma_2$ are independent of model and context size.
In the large $N$ limit, or the small $\nctx$ small $\nvocab$ limit this becomes the familiar $C_\text{Forward} = 2N$.
The backward FLOPS per token is taken as twice the forward FLOPs \citep{DBLP:journals/corr/abs-2403-14606}
\begin{equation}
	C_\text{Backward} = 2\,C_\text{Forward}.
\end{equation}

Given the simplicity of the compute expression as a function of $N$,
the better tightness of fit in the scaling law,
the improved intuition that the model size more directly corresponds to \emph{work being done by the model}, 
and the predictability of hyperparameters at larger scales,
we recommend the scaling law community consider adopting fixed aspect ratio models.








\FloatBarrier
\subsection{Model parameters}
\label{ssec:model-parameters}

In \Cref{tab:param-count} we present our parameter counting compared to commonly used existing expressions \citep{DBLP:journals/corr/abs-2001-08361,DBLP:journals/corr/abs-2203-15556,DBLP:conf/sc/NarayananSCLPKV21}.
We present a convenient substitution in \Cref{tab:param-count-simple} which can be easier to work with analytically.
Our total expressions match the architecture we are using, which includes only gains for the normalization layers, whereas while \cite{DBLP:conf/sc/NarayananSCLPKV21} has both weights and biases.
We account for potential use of \citep{DBLP:conf/emnlp/AinslieLJZLS23} as well as
use of gated linear attention mechanisms which are becoming prevalent in modern architectures \citep{DBLP:journals/corr/abs-2002-05202}
including the one used in this work (\Cref{sec:model-architecture}).

\begin{table}[h]
	\caption{Parameter counts for embedding projector, a single transformer layer, final normalization and output layer. \emph{Ours} indicates the expressions we use in the paper for the total number of parameters (note that the quantity $N$ that appears in our scaling laws is the number of \emph{non-embedding parameters}, but still includes parameters associated with normalization layers). \emph{Approx.} indicates taking the within-section total and dropping all terms that are not at least quadratic in one of $\dmodel,\nvocab$, and will be used for estimating the FLOPs per token from a given model size (\Cref{ssec:alternative-approximation-for-flops-per-token-as-a-function-of-n}), and does not differ significantly from the number of non-embedding parameters.}
	\label{tab:param-count}
	\centering
	\resizebox{1.0\textwidth}{!}{
		\begin{tabular}{lcccc}
			\toprule
			Parameters   & \cite{DBLP:journals/corr/abs-2001-08361} & \cite{DBLP:journals/corr/abs-2203-15556} & \cite{DBLP:conf/sc/NarayananSCLPKV21}                & Ours (Total)                                           \\ \midrule
			Embedding    & $(\nvocab+\nctx)\dmodel$                 & $(\nvocab+\nctx) \dmodel$                & $(\nvocab+\nctx) \dmodel$                            & $\nvocab \dmodel$                                      \\ \midrule
			\multicolumn{5}{l}{\emph{Attention (one transformer layer)}}                                                                                                                                                       \\ \midrule
			PreNorm      & ---                                      & ---                                      & $2 \dmodel$                                          & $\dmodel$                                              \\
			QKNorm       & ---                                      & ---                                      & ---                                                  & $2\dhead$                                              \\
			QKV          & $3 \nheads \dmodel \dhead$               & $3 \nheads \dmodel \dhead$               & $3 \nheads (\dmodel + 1) \dhead$                     & $ (\nheads+ 2\nkvheads) \dmodel \dhead$                \\
			Project      & $\nheads \dhead \dmodel$                 & $\nheads \dhead \dmodel$                 & $(\nheads \dhead + 1)\dmodel$                        & $\nheads \dhead \dmodel$                               \\
			Total        & $4\nheads \dhead \dmodel$                & $4\nheads \dhead \dmodel$                & $4\nheads \dhead \dmodel + 3(\nheads\dhead+\dmodel)$ & $2(\nheads +\nkvheads) \dhead \dmodel+2\dhead+\dmodel$ \\
			Approx.      & $4\nheads \dhead \dmodel$                & $4\nheads \dhead \dmodel$                & $4\nheads \dhead \dmodel + 3(\nheads\dhead+\dmodel)$ & $2(\nheads +\nkvheads) \dhead \dmodel$                 \\
			\midrule
			\multicolumn{5}{l}{\emph{Feed-forward (one transformer layer)}}                                                                                                                                                    \\ \midrule
			PreNorm      & ---                                      & ---                                      & $2 \dmodel$                                          & $\dmodel$                                              \\
			MLP          & $2 \dmodel \dffn$                        & $2 \dmodel \dffn$                        & $2\dmodel \dffn + \dffn + \dmodel$                   & $\nffn \dmodel \dffn$                                  \\
			Total        & $2 \dmodel \dffn$                        & $2 \dmodel \dffn$                        & $2\dmodel \dffn + \dffn + 3\dmodel$                  & $\nffn \dmodel \dffn+\dmodel$                          \\
			Approx.      & $2 \dmodel \dffn$                        & $2 \dmodel \dffn$                        & $2\dmodel \dffn + \dffn + 3\dmodel$                  & $\nffn \dmodel \dffn$                                  \\

			\midrule
			OutputNorm   & ---                                      & ---                                      & ---                                                  & $\dmodel$                                              \\
			Final logits & ---                                      & ---                                      & ---                                                  & ---                                                    \\
			\bottomrule
		\end{tabular}
	}
\end{table}

\begin{table}[h]
	\caption{Parameter counts displayed in \Cref{tab:param-count} using simplified notation $\nheads\dhead=\dmodel$, $\dffn=\rffn \dmodel$, and $\nheads=\gsize \nkvheads$.}
	\label{tab:param-count-simple}
	\centering
	\resizebox{1.0\textwidth}{!}{
		\begin{tabular}{lcccc}
			\toprule
			Parameters   & \cite{DBLP:journals/corr/abs-2001-08361} & \cite{DBLP:journals/corr/abs-2203-15556} & \cite{DBLP:conf/sc/NarayananSCLPKV21} & Ours (Total)                              \\ \midrule
			Embedding    & $(\nvocab+\nctx)\dmodel$                 & $(\nvocab+\nctx) \dmodel$                & $(\nvocab+\nctx) \dmodel$             & $\nvocab \dmodel$                         \\ \midrule
			\multicolumn{5}{l}{\emph{Attention (one transformer layer)}}                                                                                                                           \\ \midrule
			PreNorm      & ---                                      & ---                                      & $2 \dmodel$                           & $\dmodel$                                 \\
			QKNorm       & ---                                      & ---                                      & ---                                   & $2\dhead$                                 \\
			QKV          & $3 \dmodel^2$                            & $3 \dmodel^2$                            & $3 (\dmodel^2 + \dmodel)$             & $ (1+ 2/\gsize) \dmodel^2$                \\
			Project      & $\dmodel^2$                              & $ \dmodel^2$                             & $\dmodel^2 +\dmodel$                  & $\dmodel^2$                               \\
			Total        & $4\dmodel^2$                             & $4 \dmodel^2$                            & $4 \dmodel^2+6\dmodel$                & $2(1+ 1/\gsize)\dmodel^2+2\dhead+\dmodel$ \\
			Approx.      & $4\dmodel^2$                             & $4 \dmodel^2$                            & $4 \dmodel^2+6\dmodel$                & $2(1+ 1/\gsize)\dmodel^2$                 \\
			\midrule
			\multicolumn{5}{l}{\emph{Feed-forward (one transformer layer)}}                                                                                                                        \\ \midrule
			PreNorm      & ---                                      & ---                                      & $2 \dmodel$                           & $\dmodel$                                 \\
			MLP          & $2 \rffn \dmodel^2$                      & $2 \rffn \dmodel^2$                      & $2\rffn \dmodel^2 + (1+\rffn)\dmodel$ & $\nffn \rffn \dmodel^2$                   \\
			Total        & $2 \rffn \dmodel^2$                      & $2 \rffn \dmodel^2$                      & $2\rffn \dmodel^2 + (3+\rffn)\dmodel$ & $\nffn \rffn\dmodel^2+\dmodel$            \\
			Approx.      & $2 \rffn \dmodel^2$                      & $2 \rffn \dmodel^2$                      & $2\rffn \dmodel^2 + (3+\rffn)\dmodel$ & $\nffn \rffn\dmodel^2$                    \\

			\midrule
			OutputNorm   & ---                                      & ---                                      & ---                                   & $\dmodel$                                 \\
			Final logits & ---                                      & ---                                      & ---                                   & ---                                       \\
			\bottomrule
		\end{tabular}
	}
\end{table}


This results in an approximation for the number of non-embedding parameters, dropping subleading terms
\begin{equation}
    N \approx \nlayers\dmodel^2\left(2+ \frac2\gsize + \nffn\rffn\right)
    \label{eq:non-embedding-parameters}
\end{equation}
which can be used to estimate forward \flops per token from the model size (\Cref{ssec:alternative-approximation-for-flops-per-token-as-a-function-of-n}).

\FloatBarrier
\subsection{FLOPs per token}
\label{ssec:flops-per-token}

In \Cref{tab:flops-count} we present our counting of the total number of \flops per token performed per token during a forward pass compared to commonly used existing expressions \citep{DBLP:journals/corr/abs-2001-08361,DBLP:journals/corr/abs-2203-15556,DBLP:conf/sc/NarayananSCLPKV21}.
We present a convenient substitution in \Cref{tab:flops-count-simple} which can be easier to work with analytically.

Beyond the potential accounting for gated linear layers and grouped query attention, the
most important discrepancy across methods is how the attention mechanism is handled.
As was also noted in \citet{DBLP:journals/corr/abs-2406-19146},
the expression used in \citet{DBLP:journals/corr/abs-2001-08361}
is consistent with efficiently computing a \emph{causal} attention mechanism \citep{DBLP:conf/nips/DaoFERR22,DBLP:conf/iclr/Dao24}
whereas \citet{DBLP:journals/corr/abs-2203-15556,DBLP:conf/sc/NarayananSCLPKV21}
are consistent with counting attention \flops for a bidirectional (non-causal) attention mechanism,
where the masked component of the attention matrix (zero by construction) is still being computed.
We adopt the efficient expression of assuming a causal computation as this more closely reflects best practice.







\begin{table}[h]
	\caption{Forward \flops per for token for embedding projector, a single transformer layer, final normalization and output layer. \emph{Ours} indicates the expressions we use in the paper for the total (note that the quantity $C_\text{Forward}$ that appears in compute constraints is the number of \emph{non-embedding floating operations}. \emph{Approx.} indicates taking the within-section total and dropping all terms that are not at least quadratic in one of $\dmodel,\nvocab$, and will be used for estimating the FLOPs per token from a given model size (\Cref{ssec:alternative-approximation-for-flops-per-token-as-a-function-of-n}).}
	\label{tab:flops-count}
	\centering
	\resizebox{1.0\textwidth}{!}{
		\begin{tabular}{lcccc}
			\toprule
			\flops                           & \cite{DBLP:journals/corr/abs-2001-08361}                                    & \cite{DBLP:journals/corr/abs-2203-15556}       & \cite{DBLP:conf/sc/NarayananSCLPKV21} & Ours   (Total)                          \\
			\midrule
			Embedding                        & $4\dmodel$                                                                  & $2\nvocab \dmodel$                             & ---                                   & $2\dmodel$                              \\ \midrule
			\multicolumn{5}{l}{\emph{Attention (one transformer layer)}}                                                                                                                                                                                      \\ \midrule
			PreNorm                          & ---                                                                         & ---                                            & ---                                   & ---                                     \\
			QKNorm                           & ---                                                                         & ---                                            & ---                                   & ---                                     \\
			QKV                              & $3 \nheads 2\dmodel \dhead$                                                 & $3 \nheads2 \dmodel \dhead$                    & $3 \nheads 2 \dmodel \dhead$          & $ (\nheads+ 2\nkvheads)2\dmodel \dhead$ \\
			Logits                           & $2 \nheads \nctx \dhead$                                                    & $2 \nheads \nctx \dhead$                       & $2 \nheads \nctx \dhead$              & $ \nheads \nctx \dhead$                 \\
			Softmax                          & ---                                                                         & $3 \nheads \nctx$                              & ---                                   & $2.5 \nheads \nctx$                     \\
			Values                           & ---                                                                         & $2 \nheads \nctx \dhead$                       & $2 \nheads \nctx \dhead$              & $ \nheads \nctx \dhead$                 \\
			Project                          & $\nheads 2\dhead \dmodel$                                                   & $\nheads 2\dhead \dmodel$                      & $\nheads 2\dhead \dmodel$             & $\nheads 2\dhead \dmodel$               \\
			Total                            & $2\nheads\dhead(4\dmodel+\nctx )$                                           & $4\nheads\dhead(2\dmodel+\nctx)+3\nheads\nctx$ &
			$4\nheads\dhead(2\dmodel+\nctx)$ & $4\nheads\dhead(\dmodel+\nctx / 2)+4\nkvheads\dmodel\dhead+2.5\nheads\nctx$                                                                                                                                    \\
			Approx.                          & $2\nheads\dhead(4\dmodel+\nctx)$                                            & $4\nheads\dhead(2\dmodel+\nctx)+3\nheads\nctx$ &
			$4\nheads\dhead(2\dmodel+\nctx)$ & $4\nheads\dhead(\dmodel+\nctx/ 2)+4\nkvheads\dmodel\dhead$
			\\
			\midrule
			\multicolumn{5}{l}{\emph{Feed-forward (one transformer layer)}}                                                                                                                                                                                   \\ \midrule
			PreNorm                          & ---                                                                         & ---                                            & ---                                   & ---                                     \\
			MLP                              & $4 \dmodel \dffn$                                                           & $4 \dmodel \dffn$                              & $4\dmodel \dffn$                      & $2\nffn\dmodel \dffn$                   \\ \midrule
			OutputNorm                       & ---                                                                         & ---                                            & ---                                   & ---                                     \\
			Final logits                     & $2 \nvocab \dmodel$                                                         & $2 \nvocab \dmodel$                            & $2 \nvocab \dmodel$                   & $2 \nvocab \dmodel$                     \\
			\bottomrule
		\end{tabular}
	}
\end{table}

\begin{table}[h]
	\caption{Forward \flops counts per token from \Cref{tab:flops-count} simplified using $\nheads\dhead=\dmodel$, $\dffn=\rho \dmodel$, and $\nheads=\gsize \nkvheads$.}
	\label{tab:flops-count-simple}
	\centering
	\resizebox{1.0\textwidth}{!}{
		\begin{tabular}{lcccc}
			\toprule
			\flops                     & \cite{DBLP:journals/corr/abs-2001-08361}                  & \cite{DBLP:journals/corr/abs-2203-15556} & \cite{DBLP:conf/sc/NarayananSCLPKV21} & Ours (Total)               \\
			\midrule
			Embedding                  & $4\dmodel$                                                & $2\nvocab \dmodel$                       & ---                                   & $2\dmodel$                 \\ \midrule
			\multicolumn{5}{l}{\emph{Attention (one transformer layer)}}                                                                                                                                           \\ \midrule
			PreNorm                    & ---                                                       & ---                                      & ---                                   & ---                        \\
			QKNorm                     & ---                                                       & ---                                      & ---                                   & ---                        \\
			QKV                        & $6 \dmodel^2$                                             & $6\dmodel^2$                             & $6\dmodel^2$                          & $ 2(1+ 2/\gsize)\dmodel^2$ \\
			Logits                     & $2 \dmodel \nctx$                                         & $2 \dmodel \nctx$                        & $2 \dmodel \nctx$                     & $\dmodel \nctx$            \\
			Softmax                    & ---                                                       & $3 \nheads \nctx$                        & ---                                   & $2.5\nheads \nctx$         \\
			Values                     & ---                                                       & $2 \dmodel \nctx$                        & $2 \dmodel \nctx$                     & $\dmodel \nctx$            \\
			Project                    & $2 \dmodel^2$                                             & $2\dmodel^2$                             & $ 2 \dmodel^2$                        & $ 2 \dmodel^2$             \\
			Total                      & $8\dmodel^2+2\nctx\dmodel$                                & $8\dmodel^2+4\nctx\dmodel+3\nheads\nctx$ &
			$8\dmodel^2+4\nctx\dmodel$ & $(4 + 4/\gsize)\dmodel^2 + 2\nctx\dmodel+2.5\nheads\nctx$                                                                                                                 \\
			Approx.                    & $8\dmodel^2+2\nctx\dmodel$                                & $8\dmodel^2+4\nctx\dmodel+3\nheads\nctx$ &
			$8\dmodel^2+4\nctx\dmodel$ & $(4 + 4/\gsize)\dmodel^2 + 2\nctx\dmodel$
			\\
			\midrule
			\multicolumn{5}{l}{\emph{Feed-forward (one transformer layer)}}                                                                                                                                        \\ \midrule
			PreNorm                    & ---                                                       & ---                                      & ---                                   & ---                        \\
			MLP                        & $4 \rffn\dmodel^2$                                        & $4 \rffn\dmodel^2$                       & $4 \rffn\dmodel^2$                    & $2\nffn \rffn\dmodel^2$    \\ \midrule
			OutputNorm                 & ---                                                       & ---                                      & ---                                   & ---                        \\
			Final logits               & $2 \nvocab \dmodel$                                       & $2 \nvocab \dmodel$                      & $2 \nvocab \dmodel$                   & $2 \nvocab \dmodel$        \\
			\bottomrule
		\end{tabular}
	}
\end{table}


This results in an approximation for the number of non-embedding floating operations per token, dropping subleading terms
\begin{equation}
    C_{\text{Forward}}\approx 2\nlayers\dmodel^2\left(2+ \frac2\gsize + \nffn\rffn\right)
	+2\nlayers\nctx\dmodel
    +2\nvocab\dmodel
    \label{eq:non-embedding-flops}
\end{equation}
which can be used to estimate forward \flops per token from the model size (\Cref{ssec:alternative-approximation-for-flops-per-token-as-a-function-of-n}).
