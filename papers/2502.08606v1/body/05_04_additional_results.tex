\clearpage
\section{Additional Results}
\label{sec:additional-results}

In this section, we provide an extensive list of studies, including downstream evaluations of distillation. We cover the models used as teachers, examine the \gls{kld} between teacher and student in fixed token-to-size ratios, and present supplementary materials to \Cref{ssec:experimental-setup}. Additionally, we investigate the limiting behavior of our scaling law, weak-to-strong generalization, and conduct a model calibration study to assess fidelity. These analyses offer a comprehensive view of the factors influencing distillation performance and the behavior of our proposed scaling laws.

\subsection{Downstream evaluations}
\label{ssec:downstream-evaluations}

In all settings, we optimize for and predict model cross-entropy on the validaiton set. 
To confirm that the validation cross-entropy $L_S$ is a good proxy for the downstream evaluation that we ultimately care about, we show how each downstream result is affected by the teacher and student loss.
\Cref{fig:downstream-evals-all} shows a set of English downstream evaluation tasks. ARC Easy \citep{DBLP:journals/corr/abs-2102-03315}, ARC Challenge \citep{DBLP:journals/corr/abs-2102-03315}, HellaSwag \citep{DBLP:conf/acl/ZellersHBFC19}, Piqa \citep{DBLP:conf/aaai/BiskZLGC20}, Sciq \citep{DBLP:conf/aclnut/WelblLG17}, WinoGrande \citep{DBLP:journals/cacm/SakaguchiBBC21} and Lambada OpenAI \citep{DBLP:conf/acl/PapernoKLPBPBBF16} are zero-shot tasks. TriviaQA \citep{DBLP:conf/acl/JoshiCWZ17} and WebQS \citep{DBLP:conf/emnlp/BerantCFL13} are one-shot tasks. TriviaQA evaluation is on the larger and more challenging \emph{Web} split. 
CoreEn is the average of both the zero-shot and one-shot tasks.

Finally, we have included GSM8K \citep{DBLP:journals/corr/abs-2110-14168} and MMLU \citep{DBLP:conf/iclr/HendrycksBBZMSS21,DBLP:conf/iclr/HendrycksBBC0SS21}. GSM8K is used in an 8-shot chain of thought setting, following \llama \citep{DBLP:journals/corr/abs-2302-13971,DBLP:journals/corr/abs-2307-09288,DBLP:journals/corr/abs-2407-21783}. MMLU is used in a five-shot setting. 
These perform near-random for most of the models, and only show a slightly upwards trend when decreasing student/teacher loss. 
This is due to the use of the C4 dataset in training, and we note that we do not aim for competitive downstream evaluation results.

All models are evaluated using an internal version of the open-source \texttt{lm-evaluation-harness} \citep{eval-harness}.

\begin{figure}[h]
	\centering
	\includegraphics[width=0.99\textwidth]{plots/downstream_evals_all.pdf}
	\caption{\textbf{All student downstream evaluations.} For a discussion of the individual metrics and datasets, see \Cref{ssec:downstream-evaluations}.
	}
	\label{fig:downstream-evals-all}
\end{figure}

\FloatBarrier
\clearpage
\subsection{Teachers used in distillation}
\label{ssec:teachers-used-in-distillation}

In \Cref{fig:supervised-models} we show the cross-entropies of the models used as teachers in \Cref{ssec:distillation-scaling-law-experiments},
and for fitting the supervised scaling law:
i) eleven of fixed-$M$ ratio models following the Chinchilla rule of thumb $D/N=M^*\approx20$ \citep{DBLP:journals/corr/abs-2203-15556},
ii) six models on $D=512B$ tokens (\Cref{fig:supervised-fixed-long}), and
iii) four IsoFLOP profiles (\Cref{fig:sub-supervised-isoflops}).
Together this produces 74 runs corresponding to tuples of $(N,D,L)$. 


\begin{figure}[h]
	\centering
        \vspace{-0.7cm}
        \subfloat[Fixed-$M$ and 512B Teachers.]{
		\includegraphics[width=0.305\textwidth]{plots/fixedm_long_teachers.pdf}
		\label{fig:supervised-fixed-long}
	}
        \hfill
	\subfloat[Supervised IsoFLOPs.]{
		\includegraphics[width=0.32\textwidth]{plots/isoflop_teachers.pdf}
		\label{fig:sub-supervised-isoflops}
	}
	\hfill
	\subfloat[Supervised IsoFLOP minima.]{
		\includegraphics[width=0.27\textwidth]{plots/isoflop_teachers_parabola.pdf}
		\label{fig:supervised-isoflop-minima}
	}
        \vspace{-0.1cm}
	\caption{\textbf{Supervised IsoFLOPs.}
        \textbf{(a)} The cross-entropy of supervised models trained with either a Chinchilla optimal $M=D/N\approx 20$ or on 512B tokens.
	\textbf{(b)} The cross-entropy supervised models trained with four ISOFLOP profiles $C\in\{3\times10^{19},10^{20},3\times10^{20},10^{21}\}$.
	\textbf{(c)} The optimal supervised parameters $N^*(C)=\argmin_{N} L(C)$ for each IsoFLOP profile, and the loss $L^*(C)$ achieved by that model.}
        \vspace{-0.1cm}
	\label{fig:supervised-models}
\end{figure}
Coefficient estimation (\Cref{ssec:supervised-scaling-law-coefficient-estimation})
yields the scaling coefficients shown in
\Cref{tab:scaling-law-parameter-estimates},
and a scaling law which has $\lesssim1\%$ relative prediction error, including when extrapolated from weaker to stronger models (see \Cref{fig:supervised-scaling-law}).

\FloatBarrier

\subsection{Fixed-\texorpdfstring{$M$}{M} teacher/fixed-\texorpdfstring{$M$}{M} students and the capacity gap}
\label{ssec:fixed-m-teacher-fixed-m-students}

\begin{figure}[h]
	\centering
        \vspace{-0.15cm}
		\includegraphics[width=0.67\textwidth]{plots/fixedm_teacher_fixedm_student_appendix.pdf}
        \vspace{-0.15cm}
	\caption{\textbf{Fixed $\bm M$ Teacher/Fixed $\bm M$ Student.} Students of three sizes trained with different $M_S=D_S/N_S=20$ ratios are distilled from teachers with $M_T=D_T/N_T\approx 20$.
	This is a more complete version of \Cref{fig:isoflop-teacher-fixedm-students}.}
        \vspace{-0.15cm}
	\label{fig:fixedm-teacher-fixedm-students-appendix}
\end{figure}

In \Cref{fig:fixedm-teacher-fixedm-students-appendix}, the \emph{capacity gap} in knowledge distillation can be seen.
Improving a teacher's performance does not always improve a student's, and even reduces the performance after a certain point.
The \gls{kld} between teacher and student is an increasing function of teacher size in all cases, which means as the teacher improves its own performance, the student finds the teacher more challenging to model, which eventually prevents the student from taking advantage of teacher gains.
See \Cref{sssec:198m-students-trained-on-20n-tokens} for an investigation using calibration to understand where this mismatch occurs.

\FloatBarrier
\subsection{Full distillation scaling law IsoFLOP profiles}
\label{ssec:distillation-isoflop-profiles}
In \Cref{fig:fixedm-teacher-isoflop-students-app} we provide the full six fixed $M$ Teacher/IsoFLOP Student profiles,
only two of which were shown in \Cref{fig:fixedm-teacher-isoflop-students}.
These experiments enable the reliable determination of $\alpha^\prime,\beta^\prime,\gamma^\prime,A^\prime$ and $B^\prime$.
In \Cref{fig:isoflop-teacher-fixedm-students-app}
we provide the full four IsoFLOP teacher/ fixed $M$ student,
only two of which were shown in \Cref{fig:isoflop-teacher-fixedm-students}.
These experiments enable the reliable determination of $c_0,c_1,f_1$ and $d_1$.

\paragraph{Strong-to-weak generalization occurs.} For the weaker teachers ($N_T\leq 2.72B$),
The horizontal dashed line in each pane shows the cross-entropy achieved by the teacher (\Cref{ssec:teachers-used-in-distillation}).
we see that for students larger than the teacher ($N_S>N_T$) and for sufficiently large compute budgets, 
\emph{the student is able to outperform the teacher}
(see \Cref{ssec:weak-to-strong-generalization} for a detailed one-dimensional slice).

\paragraph{A stronger teacher signal is needed in order for stronger students to outperfom the supervised baseline.}
The horizontal dashed line in each pane shows the cross-entropy achieved by the student if trained using supervised learning (\Cref{ssec:teachers-used-in-distillation}).
We see that weaker students benefit more from distillation, as e.g. the 198M student has all observed data below this dashed line, meaning all distillations outperform the supervised baseline.
However, for the 1.82B student, only $10^{21}$ FLOP teachers produce distilled students that outperform the supervised baseline.

\begin{figure}[h]
	\centering
    \vspace{-0.1cm}
	\subfloat[Fixed $\bm M$ Teacher/Student IsoFLOP profiles.]{
		\includegraphics[width=0.47\textwidth]{plots/fixedm_teacher_isoflop_students_app.pdf}
		\label{fig:fixedm-teacher-isoflop-students-app}
	}
	\subfloat[IsoFLOP Teacher/Fixed $\bm M$ Student profiles.]{
		\includegraphics[width=0.47\textwidth]{plots/isoflop_teacher_fixedm_students_app.pdf}
		\label{fig:isoflop-teacher-fixedm-students-app}
	}
    \vspace{-0.1cm}
	\caption{\textbf{Supervised IsoFLOPs.}
	\textbf{(a)} Teachers of six sizes with $M_T=D_T/N_T\approx 20$
    are distilled into Students with four IsoFLOP profiles, and a small number with $C_S=3\times 10^{21}$.
    The horizontal grey and vertical black dashed lines indicate teacher cross entropy $L_T$ and size $N_T$ respectively.
	\textbf{(b)} Students of four sizes
    trained with a $M=D_S/N_S=20$ are distilled from teachers with four IsoFLOP profiles.
    Horizontal (vertical) dashed lines indicate student supervised cross entropy $\widetilde{L}_S$ (student size $N_S$).}
    \vspace{-0.1cm}
	\label{fig:distillation-isoflops}
\end{figure}







\FloatBarrier


\subsection{Distillation scaling law IsoFLOP optima}
\label{ssec:distillation-scaling-law-isoflop-optima}
The optimal loss values of each IsoFLOP in \Cref{fig:fixedm-teacher-isoflop-students-app} are shown in \Cref{fig:isoflop-optima}.

\begin{figure}[h]
	\centering
	\subfloat[Fixed $M$-Ratio Teacher/Student ISOFlop optima.]{
		\includegraphics[width=0.3\textwidth]{plots/fixedm_teacher_isoflop_students_optima.pdf}
		\label{fig:fixedm-teacher-student-isoflop-optima}
	}
	\subfloat[Fixed $M$-Ratio Student/Teacher ISOFlop optima.]{
		\includegraphics[width=0.3\textwidth]{plots/isoflop_teacher_fixedm_students_optima.pdf}
		\label{fig:fixedm-student-teacher-isoflop-optima}
	}
	\caption{\textbf{ISOFlop optima.}
		\textbf{a)} The optimal student parameters $N_S^*=\argmin_{N_S} \Ls(N_S)$ that give the lowest student validation loss for each teacher-student combination shown in \Cref{fig:fixedm-teacher-isoflop-students-app}. The dashed lines correspond to the validation loss of the optimal supervised models trained with the four corresponding compute budget.
		\textbf{b)} The optimal teacher parameters $N_T^*=\argmin_{N_T} \Ls(T_S)$ that give the lowest student validation loss for each teacher-student combination shown in \Cref{fig:isoflop-teacher-fixedm-students}. The black dashed line correspond to the validation loss of a $M=D/N=20$ supervised model of the indicated student size.
		In both figures, the shaded region corresponds to where \emph{weak to strong generalization} may occur, as $N_S>N_T$ (see \Cref{ssec:weak-to-strong-generalization}).}
	\label{fig:isoflop-optima}
\end{figure}

\FloatBarrier

\subsection{Distillation with infinite data}
\label{ssec:distillation-with-infinite-data}

From the supervised scaling law (\Cref{eq:supervised-scaling-law})
a model with $N$ parameters
has a cross-entropy lower bound
\begin{equation}
    L(N)\equiv L(N,D=\infty)=E+(A N^{-\alpha})^\gamma
    \label{eq:supervised-lower-bound}
\end{equation}
which represents the best solution to the training objective
subject to constraints from that model's hypothesis space \citep{DBLP:journals/corr/abs-2203-15556}
and is achieved when the number of training tokens is large ($D\rightarrow\infty$).
As the hypothesis space of a model is independent of the procedure used to find the solutions,
we anticipate that the student with $N_S$ parameters has a cross-entropy lower bound
that is the same as the supervised one \Cref{eq:supervised-lower-bound}.
However, it not immediately clear if this is true in practice, since
\begin{align}
    L_S(N_S)
    &\equiv L_S(N_S,D_S=\infty,L_T=L_T^*)\\
    &=L_T^*+\frac{(A^\prime N_S^{-\alpha^\prime})^{\gamma^\prime}}{(L_T^*)^{c_0}}\left(1+\left(\frac{L_T^*d_1^{-1}}{L(N_S)}\right)^{1/{f_1}}\right)^{-c_1f_1},
    \label{eq:distillation-lower-bound}
\end{align}
where $L_T^*=\argmin_L(N_S,D_S=\infty,L_T)$ is the teacher cross-entropy that minimizes \Cref{eq:distillation-scaling-law}.
Upon checking numerically, we do find that \Cref{eq:supervised-lower-bound}
is consistent with \Cref{eq:distillation-lower-bound}
for a range of models $N,N_S\in[100M,100B]$
(\Cref{fig:scaling-law-d-infinity}).
We stress that unlike our three motivations for the equation properties (\Cref{ssec:distillation-scaling-law-functional-form}), this infinite data limit was imposed added by hand, and is only true for certain values scaling coefficients.
This lower bound consistency is evidence that that our distillation scaling law has desired behavior far outside of observed models, at least along the data and teacher axes.
We also note that only the optimal teacher for each student size produces a student cross-entropy lower bound that is consistent with the supervised one.
Any other choice produces higher student cross-entropies, either because the teacher is too weak, or due to the capacity gap.
\begin{figure}[h]
	\centering
		\includegraphics[width=0.27\textwidth]{plots/student_loss_d_infinity.pdf}
	\caption{\textbf{Scaling behavior in the infinite data regime.}
    For the \emph{optimal} choice of teacher, the loss achieved by all student sizes under distillation is consistent with the loss achievable by supervised learning. This is \emph{not} true for \emph{any} choice of teacher, \emph{only} the optimal one, which can be determined through numerical optimization of the provided distillation scaling laws (see \Cref{sec:distillation-scaling-law-applications}).}
	\label{fig:scaling-law-d-infinity}
\end{figure}

\FloatBarrier


\FloatBarrier
\subsection{Weak-to-strong generalization}
\label{ssec:weak-to-strong-generalization}

In \Cref{fig:distillation-fixedm-teacher-varydata-student}
we see that weak-to-strong generalization \citep{DBLP:conf/icml/BurnsIKBGACEJLS24,DBLP:journals/corr/abs-2410-18837} occurs \emph{only in the finite distillation data regime},
and when the number of tokens is sufficiently large, the student cross-entropy increases again, eventually matching the teacher cross-entropy.
This can be understood in the following way: i) when the student is larger than the teacher, the student contains in its hypothesis space the function represented by the teacher, ii)
when the student is shown the teacher outputs on enough of the data manifold, it
eventually matches what the teacher does on the whole data manifold.
We note this doesn't explain how and why the student outperforms its teacher, and only constrains its asymptotic (low and high distillation data) behaviors.
\begin{figure}[h]
	\centering
    \includegraphics[width=0.45\textwidth]{plots/fixedm_teacher_varydata_student.pdf}
	\caption{\textbf{Fixed $\bf M$-Ratio Teacher varying student data.} We look at \emph{strong to weak} generalization (left) and \emph{weak to strong} (right) distillation, varying distillation tokens $D_S\in[8B,512B]$.
	}
	\label{fig:distillation-fixedm-teacher-varydata-student}
\end{figure}

\FloatBarrier
\newcommand{\calwidth}{0.48\textwidth}

\subsection{Model calibration}
\label{ssec:model-calibration}

Calibration in \gls{lm}s refers to the alignment between the model’s confidence in its predictions and the actual correctness of those predictions. 
Well-calibrated models provide confidence scores that accurately reflect their probability of correctness, enabling more decision-making. 
\gls{ece} is a common metric to quantify miscalibration, and measures the difference between \emph{predicted confidence} and \emph{actual accuracy} across multiple confidence intervals
\begin{align}\label{eq:ece}
	\mathrm{ECE} =
    \sum_{m=1}^{M} \frac{|\gB_m|}{N_{\mathrm{Samples}}} \left| \mathrm{Accuracy}(\gB_m) - \mathrm{Confidence}(\gB_m) \right|,
\end{align}
where $M$ is the number of bins,
$\gB_m$ is the set of samples whose confidence scores fall into the $m$-th bin, 
$|\gB_m|$ denotes the number of samples in bin $\gB_m$, 
$N_{\mathrm{Samples}}=\sum_{m=1}^M|\gB_m|$ is the total number of samples, 
$\mathrm{Accuracy}(\gB_m)$ and 
$\mathrm{Confidence}(\gB_m)$ are the empirical accuracy and average confidence of the model being evaluated in bin $m$ respectively.
Lower \gls{ece} indicates better model calibration.




To measure \gls{ece}, we use $M=21$ bins uniformly partitioned across the output probability space. Accuracy and confidence are computed in the standard manner: the predicted label is determined via the argmax over the output probabilities for each prediction, and the confidence is defined as the maximum probability assigned to the predicted label. Accuracy is then measured as the proportion of instances where the predicted label matches the ground truth. Notably, this approach focuses solely on the maximum probability prediction, disregarding the calibration of lower-probability predictions. To assess calibration across the entire output distribution rather than just the top prediction, alternative metrics could be considered.

\subsubsection{Teachers}
\label{sssec:teachers}

In \Cref{fig:calibration-teacher}, we see that the \gls{ece} value across different sizes of teachers. For all models, the \gls{ece} ranges between $0.4\%$ and $0.6\%$, suggesting that the models’ confidence estimates closely align with their actual accuracies. We also observe that for each plot, the blue points, i.e.~, the teacher's actual accuracy for predictions falling into specific confidence intervals, closely follow the diagonal, which shows that the models are well-calibrated. There is a small deviation at low and high confidence values denoted by the orange points.

\begin{figure}[h]
	\centering
	\includegraphics[width=0.7\textwidth]{plots/calibration/c4_valid_teacher_20n_top0.0.pdf}
	\caption{\textbf{Teacher calibration.} The calibration of teachers of seven different sizes. The $x$-axis shows the teacher probability assigned to the most confident class, and the $y$-axis is the empirical accuracy of predictions within each confidence bin. Blue points represent the teacher accuracy for predictions falling into specific confidence intervals. Orange points represent the proportion of samples in each confidence bin (helpful for understanding sample distribution across confidence levels). The dashed line represents perfect calibration, where confidence matches empirical accuracy. The \gls{ece} (\Cref{eq:ece}) for each teacher is shown as the title of each plot.}
	\label{fig:calibration-teacher}
\end{figure}

\FloatBarrier
\subsubsection{198M students trained on 20N tokens}
\label{sssec:198m-students-trained-on-20n-tokens}

In this section we consider students trained on the teacher distribution, as in our main study.
We also study students trained on the teacher top-1 distribution, as described in \Cref{ssec:top-k-top-p-sensitivity},
as the qualitative difference in behavior can be informative for student design.

Evaluating the calibration of a student can be done in a number of ways:
\begin{enumerate}
    \item We can compare student outputs relative ground-truth data, as in \Cref{sssec:teachers} for the teachers.
    \item We can compare student outputs with the outputs of its teacher.
\end{enumerate}

\paragraph{Calibration against ground-truth.}
First, let's consider comparison against ground truth data.
In \Cref{fig:calibration-student-data-20n} we show  student calibration with respect to the dataset labels
for both \emph{teacher distribution} distillation and \emph{teacher top-1} distillation.
\begin{enumerate}
  \item \emph{Distilled on the full teacher distribution.} In \Cref{fig:calibration-student-data-20n-dist}, we observe that the student is well-calibrated against ground truth data. Similar to the teacher's calibration plot in \Cref{fig:calibration-teacher}, we see a small discrepancy at very low and very high confidence values, and the \gls{ece} value is low.
  \item \emph{Distilled on teacher top-1.} In \Cref{fig:calibration-student-data-20n-top1}, we see that \emph{a student trained only on its teacher's top-$1$ prediction, 
  is not calibrated against ground truth data.} 
  The blue points below the dashed line indicate an overconfident student, i.e.~, its predicted confidence is higher than the actual accuracy in that confidence range. 
  This is because training the student on top-$1$ assigns the student to the most plausible outcome rather than all the plausible outcomes with correct frequencies. 
  Confidence proportions are low for all bins that are not the most confident bin, and \gls{ece} is high, although decreases with increasing teacher size $N_T$.
\end{enumerate}
\Cref{fig:calibration-student-data-20n} shows that training the student on the teacher's distribution results in a calibrated student, whereas training on the teacher top-$1$
does not.
Indeed, optimizing against the teacher's top-$1$ is not a proper scoring metric, and that teacher top-$1$ is \emph{not} an \emph{unbiased} estimator for the data, while the teacher distribution is.
\begin{figure}[h]
  \centering
  \subfloat[Distillation target: teacher distribution.]{
      \includegraphics[width=\calwidth]{plots/calibration/c4_valid_student_20n_top0.0.pdf}
      \label{fig:calibration-student-data-20n-dist}
  }
  \hfill
  \subfloat[Distillation target: teacher top-1.]{
      \includegraphics[width=\calwidth]{plots/calibration/c4_valid_student_20n_top1.0.pdf}
      \label{fig:calibration-student-data-20n-top1}
  }
  \caption{\textbf{Student calibration (data).} Calibration of the student with respect to the actual data labels, trained with different teacher sizes ($N_T$), on \textbf{(a)} the teacher distribution and \textbf{(b)} the teacher's top-$1$. For axis definitions and the figure legend, refer to \Cref{fig:calibration-teacher}. Blue points below the dashed line indicate student overconfidence.}
  \label{fig:calibration-student-data-20n}
\end{figure}

\paragraph{Calibration against teacher top-1.} 
Next we investigate the first student calibration against the teacher.
In \Cref{fig:calibration-student-ttop1-20n} we show student calibration with respect to the teacher's top-$1$ label.
That is, the next-token label used for accuracy computation, and extract the students confidence
is the most probable next-token according to the teacher, instead of the label from data.
Here no next token labels are used at all.
These teacher top-1 labels are also used for the \gls{ece} calculation, which is still computed using \Cref{eq:ece}.
\begin{enumerate}
  \item \emph{Distilled on the full teacher distribution.} We see in \Cref{fig:calibration-student-ttop1-20n-dist} that when distilled from the full teacher distribution, the student is \emph{not} calibrated against the teacher top-1. The blue points are above the dashed line, which means that the empirical accuracy is higher than the model’s predicted confidence, i.e. with respect to the teacher top-1, the student is \emph{underconfident}.
  This can be understood by noting that the top-1 objective is an easier objective than modeling the full vocabulary at each step.
  \item \emph{Distilled on teacher top-1.} In \Cref{fig:calibration-student-ttop1-20n-top1} we observe that a student is distilled from its teacher's top-$1$ \emph{is calibrated with respect to teacher's top-1}.
\end{enumerate}
      \begin{figure}[h]
          \centering
          \subfloat[Distillation target: teacher distribution.]{
              \includegraphics[width=\calwidth]{plots/calibration/c4_valid_student_teacher_v1_20n_top0.0.pdf}
              \label{fig:calibration-student-ttop1-20n-dist}
          }
          \hfill
          \subfloat[Distillation target: teacher top-1.]{
              \includegraphics[width=\calwidth]{plots/calibration/c4_valid_student_teacher_v1_20n_top1.0.pdf}
              \label{fig:calibration-student-ttop1-20n-top1}
          }
          \caption{\textbf{Student calibration (teacher top-1).} Calibration of the student with respect to the teacher's top $1$, trained with different teacher sizes ($N_T$), on \textbf{(a)} the teacher distribution and \textbf{(b)} the teacher's top-1. For axis definitions and the figure legend, refer to \Cref{fig:calibration-teacher}. Blue points above the dashed line indicate the student is \emph{underconfident}.}
          \label{fig:calibration-student-ttop1-20n}
      \end{figure}
\Cref{fig:calibration-student-ttop1-20n} shows that training the student on teacher top-1 results in calibration against teacher top-1,
whereas a model trained on data, or distilled on the full teacher distribution is not calibrated against teacher top-1.
As above, this can be understood as now
teacher's top-$1$ is now a proper scoring metric, and teacher top-$1$ is an unbiased estimator for itself.

\paragraph{Calibration against teacher distribution.} 
Here we develop a modified calibration measure that will help us understand if the student matches the teacher in a distributional sense.
As we have two distributions to compare, we can ask, for a given teacher confidence, what is the expected student confidence.
This leads to $\mathrm{ECE}_{\mathrm{Dist}}$, a distributional form of \gls{ece}:
\begin{align}
\label{eq:ece-dist}
    \mathrm{ECE}_{\mathrm{Dist}}(A,B) =
    \sum_{m=1}^{M} \frac{|\gB_m|}{N_{\mathrm{Samples}}} \left| \mathrm{Confidence}(\gB_m; A) - \mathrm{Confidence}(\gB_m; B) \right|,
\end{align}
and is similar in spirit to divergence measures like \gls{kld}.
$\gB_m$, $|\gB_m|$, and $N_{\mathrm{Samples}}$ are defined as before,
and $\text{Confidence}_S(\gB_m;A|B)$
is the average confidence of model $A$ or $B$ in bin $m$ respectively. 
The bins $\gG_m$ are always witin the bins of confidence of model $B$.
In the current evaluation, we take $A$ as the teacher and $B$ as the student,
and we are measuring the average confidence of the teacher is measured within a student's confidence bin.
\begin{enumerate}
  \item \emph{Distilled on the full teacher distribution.} In \Cref{fig:calibration-student-tdist-20n-dist}, we see that when the student is confident, it matches the teacher confidence.
  However, as the teacher model grows in size, 
  when the student is less confident,
  it it systematically underestimates its confidence.
  This suggests that the student has not effectively learned low-probability outcomes, or that these outcomes are particularly challenging for the student to replicate. 
  The underconfidence in these regions may be a result of the distillation process not providing sufficient learning signal for these difficult cases, or the inherent difficulty of capturing the uncertainty associated with low-confidence predictions.
  This observation of confidence mismatch helps indicate which parts of the distribution the student finds challenging to model, giving rise to the increasing \gls{kld} and capacity gap observed in \Cref{fig:fixedm-teacher-fixedm-students} and \Cref{ssec:fixed-m-teacher-fixed-m-students}.
  \item \emph{Distilled on teacher top-1.} In \Cref{fig:calibration-student-tdist-20n-top1}, for small teachers, we observe student overconfidence.
  As the teacher increases in size, the student's overconfidence in low-confidence bins transitions to underconfidence. 
  At the same time, the student's overconfidence in high-confidence bins improves, leading to an overall reduction in distributional \gls{ece}. 
  This pattern of overconfidence in the student is similar to what we saw in \Cref{fig:calibration-student-data-20n-top1}, but the change in behavior at low-confidence bins as the teacher’s size varies is different. 
  This shift in the student's calibration behavior, especially in low-confidence bins, aligns with findings from \Cref{fig:calibration-student-tdist-20n-dist} and may highlight the difficulty the small student faces in learning rare events.
\end{enumerate}

\begin{figure}[h]
	\centering
    \vspace{-0.1cm}
	\subfloat[Train target: teacher distribution.]{
		\includegraphics[width=\calwidth]{plots/calibration/c4_valid_student_teacher_v2_20n_top0.0.pdf}
		\label{fig:calibration-student-tdist-20n-dist}
	}
	\hfill
	\subfloat[Train target: teacher top 1.]{
		\includegraphics[width=\calwidth]{plots/calibration/c4_valid_student_teacher_v2_20n_top1.0.pdf}
		\label{fig:calibration-student-tdist-20n-top1}
	}
    \vspace{-0.1cm}
	\caption{\textbf{Student calibration (teacher distribution).} Calibration of the student with respect to the teacher's distribution, trained with different teacher sizes ($N_T$), on \textbf{(a)} the teacher distribution and \textbf{(b)} the teacher's top-1. For \gls{ece} calculation on the full distribution, see \Cref{eq:ece-dist}. For axis definitions and the figure legend, refer to \Cref{fig:calibration-teacher}. Blue points below the dashed line indicate student overconfidence, while points above the dashed line indicate underconfidence.}
	\label{fig:calibration-student-tdist-20n}
    \vspace{-0.1cm}
\end{figure}

We can also inspect the student confidences within a bin of teacher confidences, and compute the distributional \gls{ece} (\Cref{eq:ece-dist}), swapping the roles of teacher and student (see \Cref{fig:calibration-teacher-tdist-20n}).
\begin{enumerate}
  \item \emph{Distilled on the full teacher distribution.} In \Cref{fig:calibration-student-tdist-20n-dist} we complete the picture from \Cref{fig:calibration-student-tdist-20n-dist} and see that the part of the distribution the student struggles to model is actually the place where teacher is most confident.
  \item \emph{Distilled on teacher top-1.} In \Cref{fig:calibration-student-tdist-20n-top1} we see that the student is systematically overconfident for all values of teaacher confidence, except for the largest teachers, where the student is underconfident when those teachers are most confident.
\end{enumerate}

\begin{figure}[h]
	\centering
	\subfloat[Train target: teacher distribution.]{
		\includegraphics[width=\calwidth]{plots/calibration/c4_valid_teacher_student_v2_20n_top0.0.pdf}
		\label{fig:calibration-teacher-tdist-20n-dist}
	}
	\hfill
	\subfloat[Train target: teacher top 1.]{
		\includegraphics[width=\calwidth]{plots/calibration/c4_valid_teacher_student_v2_20n_top1.0.pdf}
		\label{fig:calibration-teacher-tdist-20n-top1}
	}
	\caption{\textbf{Student calibration (under teacher confidence bins).} Calibration of the student with respect to the teacher's confidence bins, trained with different teacher sizes ($N_T$), on \textbf{(a)} the teacher distribution and \textbf{(b)} the teacher's top-1. For \gls{ece} calculation on the full distribution, see \Cref{eq:ece-dist}. For axis definitions and the figure legend, refer to \Cref{fig:calibration-teacher}. Blue points below the dashed line indicate the teacher is less confident than the student.}
	\label{fig:calibration-teacher-tdist-20n}
\end{figure}


\FloatBarrier
\subsubsection{198M Students trained on 128B tokens}
\label{sssec:198m-students-trained-on-128b-tokens}
In this section, we study the effect of increasing the number distillation tokens 
in \Cref{sssec:198m-students-trained-on-20n-tokens} from $D_S\approx 20N_S$ to $D_S\approx 512B$.
Here, we reserve discussion for the observed differences compared to \Cref{sssec:198m-students-trained-on-20n-tokens}.

\begin{figure}[h]
  \centering
  \vspace{-0.1cm}
  \subfloat[Train target: teacher distribution.]{
      \includegraphics[width=\calwidth]{plots/calibration/c4_valid_student_128b_top0.0.pdf}
      \label{fig:calibration-student-data-128b-dist}
  }
  \hfill
  \subfloat[Train target: teacher Top 1.]{
      \includegraphics[width=\calwidth]{plots/calibration/c4_valid_student_128b_top1.0.pdf}
      \label{fig:calibration-student-data-128b-top1}
  }
  \vspace{-0.1cm}
  \caption{\textbf{Student calibration (data).} Calibration of the student with respect to the actual data labels with increased training tokens. Compare to \Cref{fig:calibration-student-data-20n} for the effect of tokens and refer to \Cref{fig:calibration-teacher} for legend and axis explanations.}
  \label{fig:calibration-student-data-128b}
  \vspace{-0.1cm}
\end{figure}

\paragraph{Calibration against ground-truth.} 
As the number of distillation tokens increases, we observe a consistent decrease in the \gls{ece} when the student is trained on the teacher's distribution, as shown by the comparison between \Cref{fig:calibration-student-data-128b-dist} and \Cref{fig:calibration-student-data-20n-dist} across different teacher sizes. 
However, when the student is trained on the teacher's top-$1$ predictions, increasing the number of tokens \emph{negatively} impacts \gls{ece}, as evidenced by the comparison between \Cref{fig:calibration-student-data-128b-top1} and \Cref{fig:calibration-student-data-20n-top1}. 
This suggests that the teacher's top-$1$ predictions are not a reliable, unbiased estimator of the actual data, and increasing the number of training tokens only exacerbates this issue. See \Cref{ssec:top-k-top-p-sensitivity} for further discussion.

\paragraph{Calibration against teacher top-1.} 
Increasing the number of distillation tokens leads to worse calibration between the student and the teacher's top-$1$ predictions when the student is trained on the full distribution. 
This change primarily occurs in the low-confidence bins, and results in a higher \gls{ece} (compare \Cref{fig:calibration-student-ttop1-128b-dist} and \Cref{fig:calibration-student-ttop1-20n-dist}). 
However, when comparing the \gls{ece}s for the student trained on the teacher's top-$1$ predictions (\Cref{fig:calibration-student-ttop1-20n-top1,fig:calibration-student-ttop1-128b-top1}), there is an improvement across all teacher sizes. 
When the student is trained and evaluated using the same metric, increasing the training tokens helps improve calibration, demonstrating consistency between the learning objective and the evaluation metric.

\begin{figure}[h]
  \centering
  \subfloat[Train target: teacher distribution.]{
      \includegraphics[width=\calwidth]{plots/calibration/c4_valid_student_teacher_v1_128b_top0.0.pdf}
      \label{fig:calibration-student-ttop1-128b-dist}
  }
  \hfill
  \subfloat[Train target: teacher top 1.]{
      \includegraphics[width=\calwidth]{plots/calibration/c4_valid_student_teacher_v1_128b_top1.0.pdf}
      \label{fig:calibration-student-ttop1-128b-top1}
  }
  \caption{\textbf{Student calibration (teacher top 1).} Calibration of the student with respect to the teacher's top $1$ when the training tokens have increased. Compare to \Cref{fig:calibration-student-ttop1-20n} for the effect of tokens and refer to \Cref{fig:calibration-teacher} for legend and axis explanations.}
  \label{fig:calibration-student-ttop1-128b}
\end{figure}

\paragraph{Calibration against teacher distribution.} A comparison between \Cref{fig:calibration-student-tdist-128b-dist} and \Cref{fig:calibration-student-tdist-20n-dist} shows that when the student is trained on the teacher's full distribution and evaluated against the full distribution using \Cref{eq:ece-dist}, increasing the number of training tokens consistently improves calibration across all teacher sizes. 
However, when the student is trained on the teacher's top-$1$ predictions, a quick comparison between \Cref{fig:calibration-student-tdist-128b-top1} and \Cref{fig:calibration-student-tdist-20n-top1} reveals worse calibration uniformly across all confidence bins.

\begin{figure}[h]
  \centering
  \subfloat[Train target: teacher distribution.]{
      \includegraphics[width=\calwidth]{plots/calibration/c4_valid_student_teacher_v2_128b_top0.0.pdf}
      \label{fig:calibration-student-tdist-128b-dist}
  }
  \hfill
  \subfloat[Train target: teacher Top-1.]{
      \includegraphics[width=\calwidth]{plots/calibration/c4_valid_student_teacher_v2_128b_top1.0.pdf}
      \label{fig:calibration-student-tdist-128b-top1}
  }
  \caption{\textbf{Student calibration (teacher distribution).} Calibration of the student with respect to the teacher's distribution as the number of training tokens increases. Compare to \Cref{fig:calibration-student-tdist-20n} for the effect of tokens and refer to \Cref{fig:calibration-teacher} for legend and axis explanations.}
  \label{fig:calibration-student-tdist-128b}
\end{figure}

Similarly, when comparing within teacher confidence bins (\Cref{fig:calibration-teacher-tdist-128b})
increasing the number of distillation tokens from 20N to 128B primarily amplifies the observed phenomena at lower distillation token budgets,
and improving calibration in cases where there is a proper scoring metric present (\Cref{fig:calibration-teacher-tdist-128b-dist}).

\begin{figure}[h]
	\centering
	\subfloat[Train target: teacher distribution.]{
		\includegraphics[width=\calwidth]{plots/calibration/c4_valid_teacher_student_v2_128b_top0.0.pdf}
		\label{fig:calibration-teacher-tdist-128b-dist}
	}
	\hfill
	\subfloat[Train target: teacher top 1.]{
		\includegraphics[width=\calwidth]{plots/calibration/c4_valid_teacher_student_v2_128b_top1.0.pdf}
		\label{fig:calibration-teacher-tdist-128b-top1}
	}
	\caption{\textbf{Student calibration (teacher distribution).} Calibration of the student with respect to the teacher' confidence bins distribution as the number of training tokens increases. Compare to \Cref{fig:calibration-teacher-tdist-20n} for the effect of tokens.}
	\label{fig:calibration-teacher-tdist-128b}
\end{figure}

In general, increasing the number of training tokens has a positive effect when the training metric is an unbiased estimator of the actual data or the measured calibration quantities (see \Cref{fig:calibration-student-data-128b-dist,fig:calibration-student-ttop1-128b-top1,fig:calibration-student-tdist-128b-dist}) and reduces the \gls{ece}, while it has a negative impact when there is a mismatch between the learned and measured quantities (see \Cref{fig:calibration-student-data-128b-top1,fig:calibration-student-ttop1-128b-dist,fig:calibration-student-tdist-128b-top1}).

