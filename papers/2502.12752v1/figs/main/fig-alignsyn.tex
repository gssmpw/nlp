\def\imgWidth{0.195\textwidth} %
\def\scc{(-1.9,-1.4)}
\def\rebigone{(-0.43, 0.5)} %
\def\refour{(1.45,-0.66)} %

\def\ssizz{1.3cm} %
\def\ssmag{3}

\begin{figure*}[t]
    \centering
    \tikzstyle{img} = [rectangle, minimum width=\imgWidth, draw=black]
\centering
    \begin{subfigure}[b]{\linewidth}
    \centering
    \includegraphics[width=\linewidth]{imgs/main/img-alignsyn-pipeline.jpg}
    \caption{Comparison between naive and our training pair generation}
    \label{fig:main-alignsyn-pipeline}
    \end{subfigure}
    \begin{subfigure}[b]{\linewidth}
    \centering
    \begin{subfigure}{\imgWidth}
        \begin{tikzpicture}[spy using outlines={green,magnification=\ssmag,size=\ssizz},inner sep=0]
            \node [align=center, img] {\includegraphics[width=\textwidth]{imgs/main/align_syn/input.png}};
    	\end{tikzpicture}
     \caption*{Input View}
    \end{subfigure}
    \begin{subfigure}{\imgWidth}
		\begin{tikzpicture}[spy using outlines={green,magnification=\ssmag,size=\ssizz},inner sep=0]
            \node [align=center, img] {\includegraphics[width=\textwidth]{imgs/main/align_syn/splat.png}};
            \spy on \refour in node [left] at \rebigone;
    	\end{tikzpicture}
     \caption*{Splatted View}
    \end{subfigure}
    \begin{subfigure}{\imgWidth}
		\begin{tikzpicture}[spy using outlines={green,magnification=\ssmag,size=\ssizz},inner sep=0]
            \node [align=center, img] {\includegraphics[width=\textwidth]{imgs/main/align_syn/naive.png}};
            \spy on \refour in node [left] at \rebigone;
    	\end{tikzpicture}
     \caption*{Naive Training}
    \end{subfigure}
    \begin{subfigure}{\imgWidth}
		\begin{tikzpicture}[spy using outlines={green,magnification=\ssmag,size=\ssizz},inner sep=0]
            \node [align=center, img] {\includegraphics[width=\textwidth]{imgs/main/align_syn/TPA.png}};
            \spy on \refour in node [left] at \rebigone;
    	\end{tikzpicture}
     \caption*{Training w/ TPA}
    \end{subfigure}
    \begin{subfigure}{\imgWidth}
        \begin{tikzpicture}[spy using outlines={green,magnification=\ssmag,size=\ssizz},inner sep=0]
            \node [align=center, img] {\includegraphics[width=\textwidth]{imgs/main/align_syn/ours.png}};
            \spy on \refour in node [left] at \rebigone;
    	\end{tikzpicture}
      \caption*{Training w/ TPA and SES}
      \end{subfigure}
    \caption{Results of the proposed training pair alignment (TPA) and splatting error simulation (SES)}
    \label{fig:main-alignsyn-result}
    \end{subfigure}
    \caption{\textbf{Aligned Synthesis.} Naive training pair generation often produces pairs that are locally unaligned in geometry and texture (green box in (a)), which results in unaligned novel view synthesis (naive training in (b)). By masking the target view, the proposed training pair alignment (TPA) enables aligned synthesis, but the generated contents tend to be misled by the splatting errors in the splatted view (training with TPA in (b)). Combining TPA with splatting error simulation (SES) in training, our \method\ learns to handle splatting errors and generates geometry- and texture-aligned novel views.}
    \label{fig:main-alignsyn}
\end{figure*}


