\def\imgWidth{0.238\textwidth} %
\def\scc{(-1.9,-1.4)}
\def\rebigone{(2.1, -0.68)} %

\def\reone{(0.35,-1)} %
\def\retwo{(-0.1,0.05)} %

\def\ssizz{1cm} %
\def\ssmag{3}

\begin{figure*}[!t] 
\centering
\tikzstyle{img} = [rectangle, minimum width=\imgWidth, draw=black]
\centering
\begin{subfigure}{\textwidth}
    \centering
          \begin{subfigure}{\imgWidth}
        \begin{tikzpicture}[spy using outlines={green,magnification=\ssmag,size=\ssizz},inner sep=0]
            \node [align=center, img] {\includegraphics[width=\textwidth]{imgs/supp/ablation/010.png}};
            \spy on \reone in node [left] at \rebigone;
    	\end{tikzpicture}
     \caption*{Splatted View}
    \end{subfigure}
    \begin{subfigure}{\imgWidth}
        \begin{tikzpicture}[spy using outlines={green,magnification=\ssmag,size=\ssizz},inner sep=0]
            \node [align=center, img] {\includegraphics[width=\textwidth]{imgs/supp/ablation/baseline.png}};
            \spy on \reone in node [left] at \rebigone;
    	\end{tikzpicture}
     \caption*{Baseline (\#1)}
    \end{subfigure}
    \begin{subfigure}{\imgWidth}
		\begin{tikzpicture}[spy using outlines={green,magnification=\ssmag,size=\ssizz},inner sep=0]
            \node [align=center, img] {\includegraphics[width=\textwidth]{imgs/supp/ablation/alignedsyn.png}};
            \spy on \reone in node [left] at \rebigone;
    	\end{tikzpicture}
     \caption*{w/ Aligned Synthesis (\#3)}
    \end{subfigure}
    \begin{subfigure}{\imgWidth}
        \begin{tikzpicture}[spy using outlines={green,magnification=\ssmag,size=\ssizz},inner sep=0]
            \node [align=center, img] {\includegraphics[width=\textwidth]{imgs/supp/ablation/ours.png}};
            \spy on \reone in node [left] at \rebigone;
    	\end{tikzpicture}
      \caption*{w/ All (\#5)}
\end{subfigure}
\begin{subfigure}{\textwidth}
    \centering
    \begin{subfigure}{\imgWidth}
        \begin{tikzpicture}[spy using outlines={green,magnification=\ssmag,size=\ssizz},inner sep=0]
            \node [align=center, img] {\includegraphics[width=\textwidth]{imgs/supp/ablation/inv_mask_010.png}};
    	\end{tikzpicture}
     \caption*{Unkown Region}
    \end{subfigure}
    \begin{subfigure}{\imgWidth}
		\begin{tikzpicture}[spy using outlines={green,magnification=\ssmag,size=\ssizz},inner sep=0]
            \node [align=center, img] {\includegraphics[width=\textwidth]{imgs/supp/ablation/diff_baseline.png}};
    	\end{tikzpicture}
     \caption*{Diff. Map (Baseline)}
    \end{subfigure}
    \begin{subfigure}{\imgWidth}
        \begin{tikzpicture}[spy using outlines={green,magnification=\ssmag,size=\ssizz},inner sep=0]
            \node [align=center, img] {\includegraphics[width=\textwidth]{imgs/supp/ablation/diff_alignedsyn.png}};
    	\end{tikzpicture}
     \caption*{Diff. Map (w/ Aligned Synthesis)}
    \end{subfigure}
    \begin{subfigure}{\imgWidth}
        \begin{tikzpicture}[spy using outlines={green,magnification=\ssmag,size=\ssizz},inner sep=0]
            \node [align=center, img] {\includegraphics[width=\textwidth]{imgs/supp/ablation/diff_ours.png}};
    	\end{tikzpicture}
      \caption*{Diff. Map (w/ All)}
    \end{subfigure}
\end{subfigure}
\end{subfigure}
    \caption{\textbf{Visual comparisons of ablation study.} The difference map shows the absolute difference between the splatted view and the corresponding novel view. The model ID is consistent with Tab. 2 in the main paper. The baseline model generates a misaligned novel view with the conditioned splatted view, showing significant differences across the image. With the proposed aligned synthesis strategy, model \#3 better follows the conditioning but still suffers from texture hallucination (green box). Combining the aligned synthesis and the texture bridge, our model (\#5) synthesizes geometry-aligned novel views while recovering high-fidelity texture. }
 \label{fig:supp-ablation-vis}
\end{figure*}

