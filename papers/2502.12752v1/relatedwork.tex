\section{Related Work}
\subsection{Feed-Forward Novel View Synthesis}
A significant number of attempts are devoted to synthesizing novel views from single/sparse observations in a feed-forward manner. Due to the limited input information, early works usually adopt depth estimation methods to model scene geometry and then utilize inpainting approaches for content synthesis~\cite{wiles2020synsin,rombach2021geometry,rockwell2021pixelsynth}. To generate realistic novel views, several techniques are developed, including GAN-based inpainting~\cite{wiles2020synsin}, VQ-VAE outpainter~\cite{rockwell2021pixelsynth}, and implicit 3D transformer~\cite{rombach2021geometry}. Recently, novel scene representations are proposed to achieve high-quality view synthesis. For instance, pixelNeRF combines convolutional networks with NeRF representation to render novel views from two images~\cite{yu2021pixelnerf}. Meanwhile, layer-based representations, \eg, multi-plane images (MPI)~\cite{li2021mine,tucker2020svmpi,adampi,khan2023tiled} and layered depth images (LDI)~\cite{shih20203dphoto,jiang2023diffuse3d}, are exploited for efficient rendering.
However, since previous feed-forward NVS methods are mainly designed for specific domains, they often suffer from performance drops in complex scenes due to the limited model capability.

\subsection{Diffusion-Based Novel View Synthesis} 
Diffusion models have demonstrated exceptional performance in generating realistic images and videos~\cite{rombach2022stablediffusion,xing2025dynamicrafter}, reflecting a profound understanding of the 3D world.
To utilize the diffusion prior for novel view synthesis, previous attempts develop conditional diffusion frameworks, \eg, 3D feature-conditioned diffusion~\cite{chan2023gennvs} and viewpoint-conditioned diffusion~\cite{liu2023zero123}, to generate novel views for simple inputs like 3D objects~\cite{zheng2024free3d}. Considering complex real-world scenes, multi-view diffusion models are often employed to synthesize high-quality novel views, which are then used to generate 3D scenes (\eg, 3D Gaussians) for novel view rendering~\cite{liu2024reconx,wu2024reconfusion}. Based on this, ZeroNVS combines diverse training datasets to acquire zero-shot NVS performance~\cite{sargent2024zeronvs}, and Cat3D designs an efficient parallel sampling strategy for fast generation of 3D-consistent images~\cite{gao2024cat3d}. In addition, GenWarp exploits the diffusion prior to achieve semantic-preserving warping~\cite{seo2024genwarp}. Recent works also explore the potential of video diffusion models for novel view synthesis. For instance, ViewCrafter constructs a point-conditioned video diffusion model to iteratively complete the point cloud for consistent view rendering~\cite{yu2024viewcrafter}, and StereoCrafter proposes a tiled processing strategy to generate stereoscopic videos with video diffusion models~\cite{zhao2024stereocrafter}. While diffusion-based NVS approaches excel at synthesizing realistic novel views, the generative nature of diffusion models often introduces hallucinated content (\eg, Fig.~\ref{fig:main-teaser}), leading to inconsistent texture across different viewpoints.


\subsection{Splatting-Based Novel View Synthesis}
Splatting-based NVS approaches are typically trained in a regression manner with pixel-level or feature-level constraints~\cite{zhang2018lpips}. As a result, they often preserve better textures compared to diffusion-based methods. Previous study employs depth-based warping to achieve real-time novel view rendering~\cite{cao2022fwd}. With the rapid advancement of 3DGS techniques~\cite{kerbl3Dgs}, a considerable amount of attention has been drawn to feed-forward Gaussian splatting methods. The pioneer work pixelSplat estimates Gaussian parameters from neural networks and dense probability distributions, achieving efficient novel view synthesis with a pair of images~\cite{charatan2024pixelsplat}. Following this, several techniques are developed for improved performance and efficiency, including cost volume encoding~\cite{chen2025mvsplat} and depth-aware transformer~\cite{zhang2024transplat}. Recent method DepthSplat integrates monocular features from depth models and achieves better geometry in the estimated 3D Gaussians~\cite{xu2024depthsplat}. Instead of utilizing multi-view cues, another line of work focuses on predicting Gaussian parameters from a single image. Splatter Image obtains 3D Gaussian parameters from pure image features~\cite{szymanowicz2024splatterimage}, and Flash3D employs zero-shot depth models for generalizable single-view NVS~\cite{szymanowicz2024flash3d}. 
However, due to the challenges of estimating accurate geometry from limited observations, existing splatting-based methods often suffer from splatting errors, resulting in novel views with distorted geometry (\eg, Fig.~\ref{fig:main-teaser}). By contrast, our \method\ leverages the geometric priors of diffusion models to correct splatting errors, achieving geometry-consistent and high-fidelity novel view synthesis.


\def\imgWidth{0.32\linewidth} %
\def\scc{(-1.9,-1.4)}
\def\rebigone{(-0.54, 0.5)} %
\def\refour{(0.08,0.23)} %

\def\ssizz{0.8cm} %
\def\ssmag{3}


\begin{figure}[t]
\centering
\tikzstyle{img} = [rectangle, minimum width=\imgWidth, draw=black]
    \centering
    \begin{subfigure}{\imgWidth}
        \begin{tikzpicture}[spy using outlines={green,magnification=\ssmag,size=\ssizz},inner sep=0]
            \node [align=center, img] {\includegraphics[width=\textwidth]{imgs/main/misalignment/018.png}};
            \spy on \refour in node [left] at \rebigone;
    	\end{tikzpicture}
     \caption*{Splatted View}
    \end{subfigure}
    \begin{subfigure}{\imgWidth}
		\begin{tikzpicture}[spy using outlines={green,magnification=\ssmag,size=\ssizz},inner sep=0]
            \node [align=center, img] {\includegraphics[width=\textwidth]{imgs/main/misalignment/viewcrafter.png}};
            \spy on \refour in node [left] at \rebigone;
    	\end{tikzpicture}
     \caption*{ViewCrafter}
    \end{subfigure}
    \begin{subfigure}{\imgWidth}
        \begin{tikzpicture}[spy using outlines={green,magnification=\ssmag,size=\ssizz},inner sep=0]
            \node [align=center, img] {\includegraphics[width=\textwidth]{imgs/main/misalignment/ours.png}};
            \spy on \refour in node [left] at \rebigone;
    	\end{tikzpicture}
      \caption*{Ours}
      \end{subfigure}
      \\ %
      \begin{subfigure}{\imgWidth}
        \begin{tikzpicture}[spy using outlines={green,magnification=\ssmag,size=\ssizz},inner sep=0]
            \node [align=center, img] {\includegraphics[width=\textwidth]{imgs/main/misalignment/inv_mask.png}};
    	\end{tikzpicture}
     \caption*{Unknown Region}
    \end{subfigure}
    \begin{subfigure}{\imgWidth}
		\begin{tikzpicture}[spy using outlines={green,magnification=\ssmag,size=\ssizz},inner sep=0]
            \node [align=center, img] {\includegraphics[width=\textwidth]{imgs/main/misalignment/diff_viewcrafter.png}};
    	\end{tikzpicture}
     \caption*{Diff. Map (ViewCrafter)}
    \end{subfigure}
    \begin{subfigure}{\imgWidth}
        \begin{tikzpicture}[spy using outlines={green,magnification=\ssmag,size=\ssizz},inner sep=0]
            \node [align=center, img] {\includegraphics[width=\textwidth]{imgs/main/misalignment/diff_ours.png}};
    	\end{tikzpicture}
      \caption*{Diff. Map (Ours)}
      \end{subfigure}
    \caption{\textbf{Misalignment.} The difference map shows the absolute difference between the splatted view and the generated view. Diffusion-based methods, \eg, ViewCrafter, often generate misaligned contents in novel views, resulting in significant differences across the image. In contrast, our \method\ is faithful to inputs and shows differences mainly around the unknown region.}
    \label{fig:main-diff}
\end{figure}



\def\imgWidth{0.195\textwidth} %
\def\scc{(-1.9,-1.4)}
\def\rebigone{(-0.43, 0.5)} %
\def\refour{(1.45,-0.66)} %

\def\ssizz{1.3cm} %
\def\ssmag{3}

\begin{figure*}[t]
    \centering
    \tikzstyle{img} = [rectangle, minimum width=\imgWidth, draw=black]
\centering
    \begin{subfigure}[b]{\linewidth}
    \centering
    \includegraphics[width=\linewidth]{imgs/main/img-alignsyn-pipeline.jpg}
    \caption{Comparison between naive and our training pair generation}
    \label{fig:main-alignsyn-pipeline}
    \end{subfigure}
    \begin{subfigure}[b]{\linewidth}
    \centering
    \begin{subfigure}{\imgWidth}
        \begin{tikzpicture}[spy using outlines={green,magnification=\ssmag,size=\ssizz},inner sep=0]
            \node [align=center, img] {\includegraphics[width=\textwidth]{imgs/main/align_syn/input.png}};
    	\end{tikzpicture}
     \caption*{Input View}
    \end{subfigure}
    \begin{subfigure}{\imgWidth}
		\begin{tikzpicture}[spy using outlines={green,magnification=\ssmag,size=\ssizz},inner sep=0]
            \node [align=center, img] {\includegraphics[width=\textwidth]{imgs/main/align_syn/splat.png}};
            \spy on \refour in node [left] at \rebigone;
    	\end{tikzpicture}
     \caption*{Splatted View}
    \end{subfigure}
    \begin{subfigure}{\imgWidth}
		\begin{tikzpicture}[spy using outlines={green,magnification=\ssmag,size=\ssizz},inner sep=0]
            \node [align=center, img] {\includegraphics[width=\textwidth]{imgs/main/align_syn/naive.png}};
            \spy on \refour in node [left] at \rebigone;
    	\end{tikzpicture}
     \caption*{Naive Training}
    \end{subfigure}
    \begin{subfigure}{\imgWidth}
		\begin{tikzpicture}[spy using outlines={green,magnification=\ssmag,size=\ssizz},inner sep=0]
            \node [align=center, img] {\includegraphics[width=\textwidth]{imgs/main/align_syn/TPA.png}};
            \spy on \refour in node [left] at \rebigone;
    	\end{tikzpicture}
     \caption*{Training w/ TPA}
    \end{subfigure}
    \begin{subfigure}{\imgWidth}
        \begin{tikzpicture}[spy using outlines={green,magnification=\ssmag,size=\ssizz},inner sep=0]
            \node [align=center, img] {\includegraphics[width=\textwidth]{imgs/main/align_syn/ours.png}};
            \spy on \refour in node [left] at \rebigone;
    	\end{tikzpicture}
      \caption*{Training w/ TPA and SES}
      \end{subfigure}
    \caption{Results of the proposed training pair alignment (TPA) and splatting error simulation (SES)}
    \label{fig:main-alignsyn-result}
    \end{subfigure}
    \caption{\textbf{Aligned Synthesis.} Naive training pair generation often produces pairs that are locally unaligned in geometry and texture (green box in (a)), which results in unaligned novel view synthesis (naive training in (b)). By masking the target view, the proposed training pair alignment (TPA) enables aligned synthesis, but the generated contents tend to be misled by the splatting errors in the splatted view (training with TPA in (b)). Combining TPA with splatting error simulation (SES) in training, our \method\ learns to handle splatting errors and generates geometry- and texture-aligned novel views.}
    \label{fig:main-alignsyn}
\end{figure*}


