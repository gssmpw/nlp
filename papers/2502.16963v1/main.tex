\documentclass[10pt,conference]{IEEEtran}
\usepackage{cite}
\usepackage{amsmath,amssymb,amsfonts}
% \usepackage{algorithmic}
\usepackage[linesnumbered,ruled,vlined]{algorithm2e}
\usepackage{graphicx}
\usepackage{textcomp}
\usepackage{tikz}
\usepackage{xcolor}
\usepackage[hyphens]{url}
\usepackage{fancyhdr}
\usepackage{hyperref}
\usepackage{marvosym}


% Ensure letter paper
\pdfpagewidth=8.5in
\pdfpageheight=11in

\newcommand*\circled[1]{\tikz[baseline=(char.base)]{
            \node[shape=circle,fill, inner sep=0pt, minimum width=0.3cm] (char) {\textcolor{white}{#1}};}}

\newcommand{\hpcayear}{2025}

%%%%%%%%%%%%%%%%%%%%%%%%%%%%%%%%%%%%%%%%
%%%%%%%%%%%%%% -- UPDATE -- %%%%%%%%%%%%%%%
\newcommand{\hpcasubmissionnumber}{306}
\title{Make LLM Inference Affordable to Everyone: Augmenting GPU Memory with NDP-DIMM}
%%%%%%%%%%%%%%%%%%%%%%%%%%%%%%%%%%%%%%%%


%%%%%%%%%%%%%%%%%%%%%%%%%%%%%%%%%%%%%%%%
%%%%%%%% -- ONLY FOR CAMERA READY -- %%%%%%%%
\def\hpcacameraready{} % Uncomment to build camera-ready version
\newcommand{\hpcapubid}{0000--0000/00\$00.00}
\newcommand\hpcaauthors{Lian Liu\textsuperscript{$1, 2, 3, \dagger$}, Shixin Zhao\textsuperscript{$1, 2, \dagger$}, Bing Li\textsuperscript{4}, Haimeng Ren\textsuperscript{1,5}, Zhaohui Xu\textsuperscript{1,5}, \\ Mengdi Wang\textsuperscript{1,2}, Xiaowei Li\textsuperscript{1,2,3}, Yinhe Han\textsuperscript{1,2}, and Ying Wang\textsuperscript{1,2, \Letter}}
\newcommand\hpcaaffiliation{Institute of Computing Technology, Chinese Academic of Sciences\textsuperscript{$1$}, \\ University of Chinese Academy of Sciences \textsuperscript{$2$}, Zhongguancun Laboratory\textsuperscript{$3$}, \\
Institute of Microelectronics, Chinese Academy of Sciences\textsuperscript{$4$}, \\
School of Information Science and Technology, ShanghaiTech University\textsuperscript{$5$}}
\newcommand\hpcaemail{ \{liulian211, zhaoshixin18\}@mails.ucas.ac.cn \quad libing2024@ime.ac.cn \quad \{renhm2022, xuzhh12022\}@shanghaitech.edu.cn \\ \{wangmengdi, lxw, yinhes, \textcolor{blue}{wangying2009}\}@ict.ac.cn}

%%%%% -- ARTEFACT EVALUATION RESULTS -- %%%%%%
% Uncomment the following based on the badges that were awarded to this paper1
%\def\aeopen{}           % The artifact is publically available
%\def\aereviewed{}     % The artefact has been reviewed
%\def\aereproduced{} % The results have been reproduced
%%%%%%%%%%%%%%%%%%%%%%%%%%%%%%%%%%%%%%%%
\def\method{\text MixMin~}
\def\methodnospace{\text MixMin}
\def\genmethod{$\mathbb{R}$\text Min~}
\def\genmethodnospace{ $\mathbb{R}$\text Min}

%%%%%%%%%%%%%%%%%%%%%%%%%%%%%%%%%%%%%
%%%%%%%%%% -- DO NOT MODIFY -- %%%%%%%%%%
%%%%%%%%%%%%%%%%%%%%%%%%%%%%%%%%%%%%%

\author{
  \ifdefined\hpcacameraready
    \IEEEauthorblockN{\hpcaauthors{}}
      \IEEEauthorblockA{
        \hpcaaffiliation{} \\
        \hpcaemail{}
      }
  \else
    \IEEEauthorblockN{\normalsize{HPCA AE \hpcayear{} Submission \textbf{\#\hpcasubmissionnumber{}}} \\
      \IEEEauthorblockA{
        Confidential Draft \\
        Do NOT Distribute!!
      }
    }
  \fi 
}

% Heading and footer for title page
\fancypagestyle{camerareadyfirstpage}{%
  \fancyhead{}
  \renewcommand{\headrulewidth}{0pt}
  \fancyhead[C]{
    \ifdefined\aeopen
    \parbox[][12mm][t]{13.5cm}{\hpcayear{} IEEE International Symposium on High-Performance Computer Architecture (HPCA)}    
    \else
      \ifdefined\aereviewed
      \parbox[][12mm][t]{13.5cm}{\hpcayear{} IEEE International Symposium on High-Performance Computer Architecture (HPCA)}
      \else
      \ifdefined\aereproduced
      \parbox[][12mm][t]{13.5cm}{\hpcayear{} IEEE International Symposium on High-Performance Computer Architecture (HPCA)}
      \else
      \parbox[][0mm][t]{13.5cm}{\hpcayear{} IEEE International Symposium on High-Performance Computer Architecture (HPCA)}
    \fi 
    \fi 
    \fi 
    \ifdefined\aeopen 
      \includegraphics[width=12mm,height=12mm]{ae-badges/open-research-objects.pdf}
    \fi 
    \ifdefined\aereviewed
      \includegraphics[width=12mm,height=12mm]{ae-badges/research-objects-reviewed.pdf}
    \fi 
    \ifdefined\aereproduced
      \includegraphics[width=12mm,height=12mm]{ae-badges/results-reproduced.pdf}
    \fi
  }
  %\fancyfoot[L]{\hpcapubid{} \copyright \hpcayear{} IEEE}
  \fancyfoot[C]{}
}
% Heading and footer for remaining pages
\fancyhead{}
\renewcommand{\headrulewidth}{0pt}
%\fancyhead[C]{\hpcayear{} IEEE International Symposium on
% High-Performance Computer Architecture (HPCA)}


\begin{abstract}
The billion-scale Large Language Models (LLMs) necessitate deployment on expensive server-grade GPUs with large-storage HBMs and abundant computation capability. As LLM-assisted services become popular, achieving cost-effective LLM inference on budget-friendly hardware becomes the current trend. This has sparked extensive research into relocating LLM parameters from expensive GPUs to external host memory. However, the restricted bandwidth between the host and GPU memory limits the inference performance of existing solutions.

This work introduces Hermes, a budget-friendly system that leverages the near-data processing units (NDP) within commodity DRAM DIMMs to enhance the performance of a single consumer-grade GPU, achieving efficient LLM inference. We recognize that the inherent activation sparsity in LLMs naturally divides weight parameters into two categories, termed ``hot" and ``cold" neurons, respectively. Hot neurons, which consist of only approximately 20\% of all weight parameters, account for 80\% of the total computational load. In contrast, cold neurons make up the other 80\% of parameters but are responsible for just 20\% of the computational workload. Leveraging this observation, we propose a heterogeneous computing strategy: mapping hot neurons to a single computation-efficient GPU without large-capacity HBMs, while offloading cold neurons to NDP-DIMMs, which offer large memory size but limited computation capabilities. In addition, the dynamic nature of activation sparsity necessitates a real-time partition of hot and cold neurons and adaptive remapping of cold neurons across multiple NDP-DIMM modules. To tackle these issues, we introduce a lightweight predictor that ensures optimal real-time neuron partition and adjustment between GPU and NDP-DIMMs. Furthermore, we utilize a window-based online scheduling mechanism to maintain load balance among multiple NDP-DIMM modules. In summary, Hermes facilitates the deployment of LLaMA2-70B on consumer-grade hardware at a rate of 13.75 tokens/s and realizes an average 75.24$\times$ speedup over the state-of-the-art offloading-based inference system on popular LLMs.

\end{abstract}

\maketitle % should come after the abstract
\thispagestyle{empty}
\pagestyle{empty}

% \pagestyle{plain} % should come right after \maketitle
\def\thefootnote{$\dagger$}\footnotetext{Both authors contributed equally to this research}\def\thefootnote{\arabic{footnote}}
\def\thefootnote{\Letter}\footnotetext{Corresponding author}\def\thefootnote{\arabic{footnote}}


\section{Introduction}

Deep Reinforcement Learning (DRL) has emerged as a transformative paradigm for solving complex sequential decision-making problems. By enabling autonomous agents to interact with an environment, receive feedback in the form of rewards, and iteratively refine their policies, DRL has demonstrated remarkable success across a diverse range of domains including games (\eg Atari~\citep{mnih2013playing,kaiser2020model}, Go~\citep{silver2018general,silver2017mastering}, and StarCraft II~\citep{vinyals2019grandmaster,vinyals2017starcraft}), robotics~\citep{kalashnikov2018scalable}, communication networks~\citep{feriani2021single}, and finance~\citep{liu2024dynamic}. These successes underscore DRL's capability to surpass traditional rule-based systems, particularly in high-dimensional and dynamically evolving environments.

Despite these advances, a fundamental challenge remains: DRL agents typically rely on deep neural networks, which operate as black-box models, obscuring the rationale behind their decision-making processes. This opacity poses significant barriers to adoption in safety-critical and high-stakes applications, where interpretability is crucial for trust, compliance, and debugging. The lack of transparency in DRL can lead to unreliable decision-making, rendering it unsuitable for domains where explainability is a prerequisite, such as healthcare, autonomous driving, and financial risk assessment.

To address these concerns, the field of Explainable Deep Reinforcement Learning (XRL) has emerged, aiming to develop techniques that enhance the interpretability of DRL policies. XRL seeks to provide insights into an agent’s decision-making process, enabling researchers, practitioners, and end-users to understand, validate, and refine learned policies. By facilitating greater transparency, XRL contributes to the development of safer, more robust, and ethically aligned AI systems.

Furthermore, the increasing integration of Reinforcement Learning (RL) with Large Language Models (LLMs) has placed RL at the forefront of natural language processing (NLP) advancements. Methods such as Reinforcement Learning from Human Feedback (RLHF)~\citep{bai2022training,ouyang2022training} have become essential for aligning LLM outputs with human preferences and ethical guidelines. By treating language generation as a sequential decision-making process, RL-based fine-tuning enables LLMs to optimize for attributes such as factual accuracy, coherence, and user satisfaction, surpassing conventional supervised learning techniques. However, the application of RL in LLM alignment further amplifies the explainability challenge, as the complex interactions between RL updates and neural representations remain poorly understood.

This survey provides a systematic review of explainability methods in DRL, with a particular focus on their integration with LLMs and human-in-the-loop systems. We first introduce fundamental RL concepts and highlight key advances in DRL. We then categorize and analyze existing explanation techniques, encompassing feature-level, state-level, dataset-level, and model-level approaches. Additionally, we discuss methods for evaluating XRL techniques, considering both qualitative and quantitative assessment criteria. Finally, we explore real-world applications of XRL, including policy refinement, adversarial attack mitigation, and emerging challenges in ensuring interpretability in modern AI systems. Through this survey, we aim to provide a comprehensive perspective on the current state of XRL and outline future research directions to advance the development of interpretable and trustworthy DRL models.
\section{Background}
\label{sec:background}

\noindent
In this section, we first overview the principles governing transformer architecture. Next, we present a concise overview of DP-SFGs, which we employ to map OTA circuits into transformer-friendly sequential data. Finally, we describe a precomputed LUT-based width estimator to translate DP-SFG parameters to transistor widths.
\vspace{-1mm}
\subsection{The transformer architecture}

\noindent
The transformer~\cite{vaswani_17} is viewed as one of the most promising deep learning architectures for sequential data prediction in NLP.  It relies on an attention mechanism that reveals interdependencies among sequence elements, even in long sequences. The architecture takes a series of inputs \((x_1, x_2, x_3, \ldots, x_n\)) and generates corresponding outputs \((y_1, y_2, y_3, \ldots, y_n\)).

\begin{figure}[b]
\vspace{-5mm}
\centering
\includegraphics[width=0.5\textwidth, bb=0 0 370 190]{fig/TransformermODEL.pdf}
\vspace{-5mm}
\caption{Architecture of a transformer.}
\label{fig:simpleTrans}
% \vspace{-2mm}
\end{figure}

The simplified architecture shown in Fig.~\ref{fig:simpleTrans} consists of $N$ identical stacked encoder blocks, followed by $N$ identical stacked decoder blocks. The encoder and decoder is fed by an input embedding block, which converts a discrete input sequence to a continuous representation for neural processing. Additionally, a positional encoding block encodes the relative or absolute positional details of each element in the sequence using sine-cosine encoding functions at different frequencies. This allows the model to comprehend the position of each element in the sequence, thus understanding its context. Each encoder block comprises a multi-head self-attention block and a position-wise feed-forward network (FFN); each decoder block, which has a similar structure to the encoder, consists of an additional multi-head cross-attention block, stacked between the multi-head self-attention and feed-forward blocks. The attention block tracks the correlation between elements in the sequence and builds a contextual representation of interdependencies using a scaled dot-product between the query ($Q$), key ($K$), and value ($V$) vectors:
\begin{equation}
\text{{Attention}}(Q, K, V) = \text{softmax}\left(\frac{QK^T}{\sqrt{d_k}}\right)V,
\end{equation}
where $d_k$ is the dimension of the query and key vectors. The FFN consists of two fully connected networks with an activation function and dropout after each network to avoid overfitting. The model features residual connections across the attention blocks and FFN to mitigate vanishing gradients and facilitate information flow.

\subsection{Driving-point signal flow graphs}

\noindent
The input data sequence to the transformer must encode information that relates the parameters of a circuit to its performance metrics.  Our method for representing circuit performance is based on the signal flow graph (SFG).  The classical SFG proposed by Mason~\cite{Mason53} provides a graph representation of linear time-invariant (LTI) systems, and maps on well to the analysis of linear analog circuits such as amplifiers. In our work, we employ the driving-point signal flow graph (DP-SFG)~\cite{ochoa_98,schmid_18}. The vertices of this graph are the set of excitations (voltage and current sources) in the circuit and internal states (e.g., voltages) in the circuit.  
% An edge is drawn between vertices that have an electrical relationship, and the weight on each edge is the gain of the edge;
An edge connects vertices with an electrical relationship, and the edge weight is the gain; 
for example, if a vertex $z$ has two incoming edges from vertices $x$ and $y$, with gains $a$ and $b$, respectively, then $z = ax + by$, using the principle of superposition in LTI systems.  To effectively use superposition to assess the impact of each node on every other node, the DP-SFG introduces auxiliary voltages at internal nodes of the circuit that are not connected to excitations. These auxiliary sources are structured to not to alter any currents or voltages in the original circuit, and simplifies the SFG formulation for circuit analysis.
% enable easy formulation of the SFG to analyze circuit behavior. 

\begin{figure}[t]
% \vspace{-6mm}
\centering
\includegraphics[width=0.9\linewidth, bb=0 0 320 140]{fig/DPSFG.pdf}
\vspace{-0.25cm}
\caption{~(a) Schematic and (b) DP-SFG for an active inductor.}
\label{fig:DP-SFG_ex}
\vspace{-5mm}
\end{figure}

Fig.~\ref{fig:DP-SFG_ex}(a) shows a circuit of an active inductor, which is an inductor-less circuit that replicates the behavior of an inductor over a certain range of frequencies. Fig.~\ref{fig:DP-SFG_ex}(b) shows the equivalent DP-SFG. In Section~\ref{sec:dp-sfg}, we provide a detailed explanation that shows how a circuit may be mapped to its equivalent DP-SFG. 


\ignore{
\subsection{Lookup table for MOSFET sizing}
\label{sec:LUT}

\noindent
As seen in Fig.~\ref{fig:DP-SFG_ex}, the edge weights in a DP-SFG include circuit parameters such as the transistor transconductance, $g_m$, and various capacitances in the circuit.  The circuit may be optimized to find values of these parameters that meet specifications, but ultimately these must be translated into physical transistor parameters such as the transistor width.   In older technologies, the square-law model for MOS transistors could be used to perform a translation between DP-SFG parameters and transistor widths, but square-law behavior is inadequate for capturing the complexities of modern MOS transistor models.
In this work, we use a precomputed lookup table (LUT) that rapidly performs the mapping to device sizes while incorporating the complexities of advanced MOS models.

\begin{figure}[htbp]
\vspace{-0.4cm}
\centering
\includegraphics[height=4cm]{fig/lut_fig_1.pdf}
\vspace{-0.55cm}
\caption{LUT generation using three DOFs, $V_{gs}$, $V_{ds}$ and $L$.}
\label{fig:lutgen}
\vspace{-0.1cm}
\end{figure}

The LUT is indexed by the $V_{gs}$, $V_{ds}$, and length $L$ of the transistor, and provides four outputs: the drain current ($I_d$), transconductance ($g_m$), source-drain conductance ($g_{ds}$), and drain-source capacitance ($C_{ds}$).
The entries of the LUT are computed by performing a nested DC sweep simulation across the three input indices for the MOSFET with a specific reference width, $W_{ref}$, as shown in Fig.~\ref{fig:lutgen}, and for each input combination, the four outputs are recorded.
\blueHL{Empirically, we see that the impact of $V_{sb}$ is small enough that it can be neglected, and therefore we set $V_{sb} = 0$ in the sweeps used to create the LUT.}

Our methodology uses this LUT, together with the $g_m/I_d$ methodology~\cite{silviera_96}, to translate circuit parameters predicted by the transformer to transistor widths. The cornerstone of this methodology relies on the inherent width independence of the ratio $g_m/I_d$ to estimate the unknown device width: this makes it feasible to use an LUT characterized for a reference width $W_{ref}$. 
We will elaborate on this procedure further in Section~\ref{sec:precomputedLUTs}, and show how the LUT, together with the $g_m/I_d$ method, can effectively estimate the device widths corresponding to the transformer outputs.
% \redHL{\sout{required to achieve equivalent DC operating characteristics within the circuit. Section III D \redHL{Do not hardcode section numbers!!} provides an in-depth explanation of the implementation details of this methodology.}}
}
\section{Motivations \& Challenges}

\subsection{Why NDP-DIMM Enhanced GPU?}
% 这里再跟 powerinfer 比,以及说明 CPU 的扩展的问题。

% 1. offloading 不行 -> 哪怕有activation sparsity, 也不行;但是我们进一步分析 activation sparsity 的特点,说明 hot/cold 的划分特点; 根据特点,选择硬件。 (low cost ?)

% 首先要说明为什么选择 NDP-DIMM
Offloading is essential for LLM inference on low-budget systems with a single consumer-grade GPU. However, as noted in Section \ref{sec:background-offloading}, even utilizing activation sparsity to reduce weight parameter access, the PCIe bandwidth remains the bottleneck. Thus, costly data transfers between extended memory and GPU must be minimized. However, simply offloading the corresponding computation of cold neurons on the host CPU~\cite{llama.cpp, song2023powerinfer} can only achieve a limited performance improvement, \update{as the host CPU can only access DRAM with limited improved bandwidth than PCIe (e.g., 89.6 GB/s vs. 64 GB/s)}. To this end, we choose to employ multiple NDP-DIMMs as the extended memory, as they offer comparable bandwidth and larger storage capacity than a single consumer-grade GPU. Need to mention that as a budget-friendly host memory solution, we do not consider high-performance but expensive HBM-PIM and AiM~\cite{cong2017aim, park2024attacc} in this study. Given the limited computation capability, only utilizing the processing units in NDP-DIMMs cannot boost the inference efficiency~\cite{wu2024pim}. Consequently, we are motivated to use NDP-DIMMs to enhance GPU for efficient LLM inference.

% neurons 划分来协助相应的计算
Our observation indicates that the activation sparsity within LLMs effectively partitions weight parameters into two distinct regions, which are ideally suited to consumer-grade GPU and NDP-DIMMs, respectively. Specifically, activation sparsity in LLMs follows a power-law distribution~\cite{xue2024powerinfer, song2023powerinfer}. About 20\% of neurons (\textit{hot neurons}) account for 80\% of computations, while 80\% (\textit{cold neurons}) handle only 20\%. Hot neurons, with $16 \times$ higher computation intensity, fit GPU memory, while cold neurons suit NDP-DIMMs. During inference, GPU can provide high computation capability for hot neurons and NDP-DIMMs enable the cold neurons computation in memory.

\subsection{Necessity of Hot/Cold Neuron Partition} \label{sec:similar}

% 1. 首先需要说明为什么要进行 hot/cold neurons partition, 然后说明对应的 challenges,以及目前的预测的缺陷
% 根据我们的评估,在LLaMA-70B 中,有约 52% 的 hot neurons 在推理过程中会发生变化,导致the predetermined hot/cold neurons partition 相比于 oracle 的划分产生 1.63x 的performance 下降。 
Hot/cold neuron partition impacts the computational load on GPU/NDP-DIMMs, affecting the inference performance of the heterogeneous system. Due to the input-specific nature of activation sparsity, solely relying on the offline partition is insufficient. Our evaluation on LLaMA2-70B reveals significant dynamics in when the neuron will be activated (hereafter, neuron activity patterns) during inference. Approximately 52\% of the initialized hot neurons exhibit varied activity during inference. This variability in neuron behavior results in suboptimal performance with a fixed hot/cold partition, causing a $1.63\times$ degradation compared to an oracle (the theoretically optimal partition) scheme. Thus, we must dynamically predict and adjust the hot/cold neuron partition.

However, typical MLP-based predictors~\cite{liu2023deja, song2024prosparse, zhang2024relu, mirzadeh2023relu} for activation sparsity in LLMs are costly. For example, predicting the activated neurons in LLaMA-7B needs per-layer MLP-based predictors, requiring an extra 2GB storage and inducing 10\%-25\% inference runtime. Fortunately, the inherent locality of activation sparsity leads us to design a lightweight and accurate predictor for efficient online partition adjustments. To be specific, we found that activation sparsity in LLM inference shows considerable token-wise similarity and layer-wise correlation, worth exploiting.

\begin{figure}[t]
    \centering
    \includegraphics[width=0.98\linewidth]{Fig/prediction-motivation.pdf}
    \vspace{-0.3cm}
   \caption{Distribution patterns for activation sparsity. (a) \update{The adjacent tokens enjoy high similarity on activated neurons for various models and datasets.} (b) The activated neurons between consecutive layers are highly correlated.}
    \label{fig:similarity}
\vspace{-0.3cm}
\end{figure}

\subsubsection{Token-wise Similarity}
% As mentioned in the previous section, the dynamically sparse neurons exhibit a power-law distribution, resulting in high locality among activated neurons. This indicates that we can divide all neurons into hot neurons and cold neurons. However, this division is not fixed and requires dynamic adjustment. 
We analyzed the similarity between tokens to explore the distribution characteristics of activation sparsity. \update{As shown in \fig \ref{fig:similarity}a, we evaluate the token-wise similarity for LLaMA-13B and Falcon-40B with multiple widely adopted datasets, including COPA~\cite{roemmele2011choice}, Wikitext2~\cite{merity2016pointer} and PIQA~\cite{bisk2020piqa}. As one can notice, the adjacent tokens have a higher distribution similarity than distant tokens. Specifically, the similarity between adjacent tokens exceeds 90\% (95\% for Falcon-40B), but drops to 70\% once the tokens' distance exceeds 10.} This indicates that in context, adjacent tokens often express similar meanings, leading to high similarity in their activity distribution. Additionally, we observe that when the distance between tokens exceeds 25, the distribution similarity almost no longer decreases, indicating that beyond a certain window size, the semantic correlation becomes weak and has less impact on the overall distribution.

\subsubsection{Layer-wise Correlation} 
We further observed that the distribution of activated neurons in two consecutive layers is highly correlated. As shown in \fig \ref{fig:similarity}b, when the 6th neuron in layer-30 of LLaMA-13B is activated, the probability of neurons 0 and 5 being activated in layer-31 exceeds 90\%. This suggests that we can use the results of the preceding layer to predict the distribution of activated neurons in the current layer.

Overall, the token-wise similarity and layer-wise correlation motivate us to design a lightweight online predictor based on historical activation information. According to the prediction results, we can online adjust the hot/cold neurons partition to effectively exploit the processing advantages of the consumer-grade GPU and NDP-DIMMs, respectively.

% to predict whether to compute neurons of the current token in a specific layer. 
% Compared to costly MLP predictors, this historical activation-based strategy introduces negligible overhead.

\subsection{Load Imbalance across Multiple NDP-DIMMs}
% When initially distributing cold neurons across multiple DIMMs, we analyze past computational patterns to estimate the potential workload of these neurons. This allows us to distribute their corresponding weights across DIMMs in a manner that aims to evenly distribute the expected computational load among them. However, due to the inherently stochastic nature of cold neuron computations, relying on a static weight distribution strategy can result in imbalanced workload distribution across DIMMs.
% imbalance的来源+数据imbanlance有多大 -> 会很影响性能,所以我们要去做load balance的事情,DIMM间也需要有个简易的数据通路
Due to the storage limitation of a single DIMM, multiple DIMMs are required to store all the neurons (weight parameters) in LLM. Specifically, one DIMM only stores portions of the neurons and the corresponding processing unit can only directly assess neurons in the DIMM with high internal bandwidth. However, due to the input-specific nature of activation sparsity, the computational load on each NDP-DIMM can be diverse. For example, when fixing the cold neuron distribution on multiple DIMMs for LLaMA-13B, the most overloaded NDP-DIMM will have 1.2-2.5$\times$ more computational load than others.
% Furthermore, given the limited computational capacity of individual DIMMs, an overloaded DIMM can potentially bottleneck the entire system.

%是不是也可以在这里说,使用center buffer的DIMM,通过在buffer中添加很少的单元就可以实现DIMM间高效的通信而无需CPU参与,且每个DIMM中的PU都可以获得对应DIMM上的所有数据,所以我们选择了center-buffer 的NDP-DIMM
Therefore, we need an online scheduling strategy to remap the cold neuron across DIMMs to achieve load balance. Meanwhile, an efficient data transmission pathway among DIMMs is essential to help adjust the neuron placement. By optimizing neuron computation scheduling, we can minimize data transfers across NDP-DIMMs while ensuring balanced computational loads across DIMMs. This ensures that all parts of the system can maximize their performance.

In summary, the NDP-DIMM enhanced GPU approach effectively addresses the substantial data transfer overhead in offloading processes, providing a promising solution to improve LLM inference efficiency by leveraging the activation sparsity patterns inherent in LLMs.

\begin{figure*}
    \centering
    \includegraphics[width=0.8\linewidth]{Fig/overview.pdf}
    \vspace{-0.3cm}
    \caption{Overview of our proposed Hermes System. (a) Hermes augments GPU memory with NDP-DIMMs, and utilizes a scheduler to control the inference workflow. (b) Multiple NDP-DIMMs are connected to support LLM inference and inter-DIMM communication. }
    \label{fig:system-overview}
\vspace{-0.3cm}
\end{figure*}

\section{\name~System}\label{sec:hermes-system}
\subsection{System \& Architecture Overview}

\subsubsection{Architecture}

\fig\ref{fig:system-overview}a illustrates the overview of \name. \name~augments consumer-grade GPU with NDP-DIMMs to achieve low-budget, high-performance inference system for LLMs. 

% We elaborate on the architecture detail of each component as follows.

\textbf{Consumer-grade GPU:} For LLM inference, we only use one accessible and budget-friendly consumer-grade GPU. Despite limited graphic memory, it has ample computing units, like tensor cores, for high-performance parallel processing. It also features high-speed GDDR memory with superior bandwidth. For instance, the NVIDIA RTX 4090 with 24GB GDDR6 provides 82.6 TFLOPS, 1321 Tensor TOPS, and 936 GB/s bandwidth, making it suitable for LLM inference. Hermes uses a single GPU to efficiently execute hot neurons.
%[726] 不确定下面这句还要不要 : liu 暂时觉得可以不要,后面会主要说workflow,
% Given the abundant computational power of the GPU and the high-speed bandwidth but limited storage space of GDDR memory, during inference, we load only the hot neurons into GDDR and utilize the GPU's tensor cores to efficiently compute the hot neurons in parallel.

\textbf{NDP-DIMM:}
Given that cold neurons are randomly activated, all data stored on each DIMM should be accessible to its own NDP units. \update{Meanwhile, DIMM is required to support the normal data access from GPU for hot neuron transmission.} Therefore, as illustrated in \fig \ref{fig:system-overview}b, we have chosen the center buffer-based NDP-DIMM design\cite{alian2018application,cong2017aim,ke2020recnmp,kwon2019tensordimm}, which allows the processing unit to access all data in its own DIMM. 
\update{The center buffer-based NDP DIMM design also complies with the normal memory access as the newly added units will not influence the memory access function supported by the local memory controller \cite{alian2018application, ke2020recnmp}.}
Here, we detail the microarchitecture of our NDP-DIMM design. To facilitate typical operations in LLMs and potential inter-DIMM data moving, each NDP-DIMM is equipped with GEMV units, activation units, and DIMM-links~\cite{zhou2023dimm}.

\textit{GEMV Unit}: 
The GEMV unit reads data from the DRAM cell and the center buffer, performing the GEMV computation associated with cold neurons. 
\update{To support batched inference and fully utilize the bandwidth achievable within the DIMM center buffer, each GEMV unit contains 256 multipliers.}
Each multiplier is responsible for 128-bit multiplication \update{in a typical bit-serial manner~\cite{devaux2019true}}, a reduction tree-based accumulator, and a 256 KB buffer. During computation, each multiplier computes eight FP16 values simultaneously, and the accumulator is responsible for the addition of partial sums with data dependencies. The buffer stores the intermediate result generated by LLM layers. 

% \todo{add details on why GEMV design, partly done}
% Q: do we need to add some data to support our design, like the analysis of NDP-DIMM accessible data amount in one cycle and the batched inference requirement}

\textit{Activation Unit}: The activation unit is designed to support the necessary non-linear functions, such as softmax and ReLU operation for LLM inference. 
This unit is composed of 256 FP16 exponentiation units, 256 FP16 addition units, and 256 FP16 multiplication units, in addition to a comparator tree, an adder tree, and a divider. 

% The setup of the activation units is consistent with the throughput needs of GEMV units. Moreover, the activation unit repurposes the comparator tree to support both softmax and other activation functions.
% By incorporating an activation unit in the central buffer, we can reduce unnecessary data transfers. For example, softmax requires the `score' output, and having the activation unit and score computation unit both in DIMM allows direct data access from the same center buffer, avoiding additional data transfer to the GPU.

\textit{DIMM-link}: 
Due to the input-specific nature of the activated neurons, it is necessary to adjust the neuron mapping in multiple NDP-DIMMs to further ensure the load balance of computation in the DIMMs. Therefore, we adopt DIMM-link~\cite{zhou2023dimm} to achieve inter-DIMM communication with a bandwidth of 25 GB/s. Each DIMM-link employs bidirectional external data links between DIMMs, facilitating efficient point-to-point data transfers. The DIMM-link controller and bridge enable high-speed neuron redistribution between DIMMs.
\update{Compared to relying on the host for inter-DIMM data movement, using DIMM links provides over a 62$\times$ speedup for data transfer with negligible hardware overhead.
For example, when running OPT-66B, the introduction of DIMM-link effectively reduces the migration overhead for cold neurons from 5.3\% of total time to below 0.2\% .}
% For example, DIMM0 and DIMM1 store weights for N channels, but if no computations are needed for DIMM0 and x channels need computation in DIMM1, we can exchange x/2 channels' weights between DIMM0 and DIMM1 to balance the load, thereby reducing computation latency.

\textbf{Scheduler}:
During LLM inference, the scheduler in the host CPU redistributes neuron computation tasks to the GPU and NDP-DIMMs. The scheduler primarily comprises two components: a lightweight predictor and a neuron mapper, \update{which are both implemented by software}. In addition, the scheduler includes a monitor that gathers runtime information to assist the predictor and an instruction queue that triggers instructions for the GPU and NDP-DIMMs. With the help of the monitor, the lightweight predictor leverages token-wise similarity and layer-wise correlation patterns to accurately predict neuron activity. Based on the prediction results, the neuron mapper assigns hot and cold neurons to DIMMs and GPU memory, respectively, and it also dynamically adjusts the neurons' placement to ensure efficient inference on both the GPU and NDP-DIMMs. The subsequent sections will provide detailed descriptions of these two components. 

\textbf{\update{Programming Interface}}:
\update{We use a standard programming model, PIM-SYCL \cite{kim2023samsung}, to compile the heterogeneous platform. Unified memory programming \cite{zhao2024pim, nvidia_unified_memory} allows data to be transferred implicitly between heterogeneous memory devices, enabling cooperative processing on GPU and NDP-DIMMs. Additionally, \name~provides a set of extra NDP commands, such as MAC and softmax, to support various operators in LLMs. Taking GEMV computation as an example, the NDP-DIMM computations can be invoked through the memory command interface by sending a series of MAC commands. On the GPU side, the corresponding computations are triggered through APIs like cudaLaunchKernel. }
 % \update{For instance, the instruction queue issues the GEMV operator on GPU using APIs like cudaLaunchKernel. Conversely, an additional GEMV instruction, which contains the MAC command with the data address, is defined to support the GEMV operator on the NDP-DIMM side. The instruction queue can invoke the corresponding computation through the memory command interface~\cite{cong2017aim, park2024attacc, heo2024neupims}.}


% \update{Once the scheduler determines the computation tasks assigned to the GPU and NDP-DIMM, it invokes the corresponding instructions for each device. For instance, in the case of a GEMV computation, the scheduler will issue a CUDA instruction when the GPU is required. Conversely, when leveraging the NDP-DIMM, it sends the NDP computation instruction, which contains the MAC command with the data address, through the memory command interface.}
% 感觉上面这个programming interface感觉确实可要可不要?如果说的话,还得说我们在DIMM里面的controller里添加了对于计算指令解析的通路。

% 流水线,两段,上面总结,下面详细说MLP(参考ATTACC
\begin{figure}[t]
    \centering
    \includegraphics[width=\linewidth]{Fig/workflow.pdf}
    \vspace{-0.3cm}
    \caption{Workflow of Hermes system (a) The whole workflow of LLMs inference on the \name~system. (b) Illustration of computation process for FC layers with activation sparsity. The block with a number in the Mem. means one neuron's weight.}
    \label{fig:workflow}
\vspace{-0.3cm}
\end{figure}

\subsubsection{Workflow}
% During this stage, the KV matrices are transferred to the NDP-DIMMs as soon as they are created.

The workflow for LLM inference within the \name~system is depicted in \fig \ref{fig:workflow}a. \update{Given the significant computational demands, the entire prompting stage is processed on the GPU, adhering to a traditional offloading strategy~\cite{sheng2023flexgen}.} During this stage, the host scheduler records neuron activity for future scheduling optimization. Upon completing the prompting stage, only the selected hot neurons are loaded back into GPU memory. The offline partition of hot and cold neurons will be further detailed in Section \ref{sec:offline}.
In the token generation stage, for each transformer layer, the QKV generation is collaboratively completed by GPU and NDP-DIMMs. The output of QKV generation will be collected in the NDP-DIMMs for further attention computation. 
The memory bandwidth-intensive nature of attention computation~\cite{yu2022orca, park2024attacc} makes it ideal for execution on NDP-DIMMs, which benefit from the abundant internal bandwidth. Additionally, transferring attention computation to NDP-DIMMs helps save the limited GPU memory by eliminating the need for storing KV cache.
Since the projection layer cannot utilize the activation sparsity, it is handled solely by the computation-efficient GPU. During the projection computation, as the DIMMs are entirely idle, the host takes advantage of this period to dynamically reconfigure the hot/cold partitions and redistribute neurons across DIMMs based on the prediction results, which will be detailed in the \sec \ref{sec:partition-design} and \ref{sec:cold-neuron-mapping}. Then, similar to the QKV generation, MLP is offloaded to both GPU and NDP-DIMMs. Finally, the output of each transformer layer is reduced in the NDP-DIMMs. 

% Once the result of attention is obtained, the projection step requires substantial computational resources, since it lacks the feature of sparsity of activation. To avoid extra synchronization overhead from concurrent execution on both GPU and NDP-DIMM, the projection computation is handled solely by the GPU.


% The Par denotes the predetermined or offline profiled parameters. The Var denotes the variables that need to be solved.

\fig \ref{fig:workflow}b illustrates the computation process for FC layers (for both QKV generation and MLP block) with activation sparsity. Specifically, it includes three steps.
After completing the related prediction, the host CPU determines the computation allocation for both the GPU and NDP-DIMMs based on the location of the activated neurons.
\update{Once the neuron mapping is determined, the host CPU invokes APIs for both the GPU and NDP-DIMMs to load data and perform computation. For example, the host CPU uses ``cudaLaunchKernel" to launch GPU kernels for GEMM and GEMV operations. To ensure correctness, the host CPU inserts barriers for the GPU and NDP-DIMMs to synchronize their computations. Once the DIMMs and GPU complete their computations, a merge kernel is invoked on the NDP-DIMMs side to gather the results from both sources.} This method is advantageous for two reasons. Firstly, as the GPU generally finishes computation tasks more quickly owing to its superior computation capability, the latency in transferring data from the GPU to DIMMs can be hidden by the DIMMs' computation, thus not penalizing the overall system runtime. Secondly, with the attention computation occurring on NDP-DIMMs, merging the QKV generation outcomes on the NDP-DIMMs side minimizes the additional data transfer overhead.
% In the next sections, the important components of Hermes will be introduced in detail.

\begin{table}
\vspace{-0.3cm}
\caption{Terminology for the offline partition solver. } 
\label{tab:terminology}
\scriptsize{
\centering
\begin{tabular}{c|p{7.2cm}}
\hline
% \textbf{Symbol} & \textbf{Description} \\[1ex]
% \hline
% \Xhline{3\arrayrulewidth}
\rowcolor{black!10}
\multicolumn{2}{c}{\textbf{\textit{Parameters - predetermined or offline profiled}} }\\
\hline
$\mathbb{L}$ & All layers \\
$\mathbb{N}$ & All neurons \\
$\mathbb{D}$ & All NDP-DIMMs \\ 
$f_{i}$ & Activation frequency of neuron $i$ \\
$N_{l}$ & Neuron in layer $l$ \\
$M_{i}$ & The memory space required by neuron $i$ \\
$T_{sync}$ & The time required for one synchronization \\
$T_{l}^{j}$ & The time for computing one neuron in layer $l$ on processing $j$ \\
$S_{j}$ & The storage size for processing unit $j$ \\
\hline
\rowcolor{black!10}
\multicolumn{2}{c}{\textbf{\textit{Binary Variables - needed to be solved}} } \\
\hline
\multirow{2}{*}{$x_{il}^{j}$} & Whether neuron $i$ in layer $l$ is placed on processing unit $j$ \\
& $x_{il}^{j}=1$ means the neuron $i$ in layer $l$ is placed on processing unit $j$\\
\hline 
\end{tabular}
}
\vspace{-0.3cm}
\end{table}

\subsection{Offline Neuron Mapping} \label{sec:offline}
% Since both NDP-DIMMs and GPU cooperate to process the same operator, 
Since NDP-DIMMs and GPU are responsible for the computational load of the neurons stored in them, the predetermined mapping for each neuron's location greatly influences the inference efficiency. However, due to the huge neuron mapping space (e.g., more than $2^{1000}$ for LLaMA-7B), solely relying on online mapping solutions is impractical and will contribute to considerable performance degradation. Therefore, in the belief that ``hot" and ``cold" neurons are partly attributed to the pretrained LLM's nature~\cite{song2023powerinfer, song2024prosparse, zheng2024learn, song2024turbo}, we utilize the offline profiled information to deduce the initial offline neuron mapping. It alleviates the adjustment cost of subsequent online partition and scheduling during inference. Please note that the optimal initial mapping denotes the mapping that can be found during the offline stage, which will be adjusted during runtime.

% To fully unleash the computational capabilities of both NDP-DIMMs and GPU, we need to effectively partition the hot and cold neurons. Therefore, we first propose an offline neuron partition scheme to optimize data layout. 

% \subsubsection{Offline Parameter Placement}


% 感觉这里要不要加一句说,我们的将该问题抽象为ILP的最终目标是求解是系统时延最小的方案,然后再说求解的过程考虑了XXX,最后说具体的建模如下
To determine the optimal location for each neuron \update{that minimizes the inference latency using our heterogeneous system}, we formalize the mapping issue as an integer linear programming problem (ILP). In particular, we analyze several factors, including each neuron's activated frequency, computational overhead, memory usage, and synchronization delays, to model the inference performance of the \name~system. \update{To gather these factors accurately, we test the model on popular datasets such as C4~\cite{raffel2020exploring} and Pile~\cite{gao2020pile} with 128 samples}, and also employ an execution monitor in the host CPU to record during inference. The notation for solving the optimal offline neuron placement problem is summarized in Table \ref{tab:terminology}.

\textbf{Objective function.}
The objective of the optimal neuron mapping is to minimize the total inference latency, as shown in Equation \ref{eq1}. Since the execution of each layer involves both GPU and NDP-DIMMs, the total execution time is determined by the longer duration of the GPU and NDP-DIMMs execution times. For NDP-DIMMs, the single-layer execution time is the longest execution time among all DIMM modules, as shown in Equation \ref{eq2}. For the GPU, the single-layer execution time includes both computation time and extra synchronization overhead, while the synchronization overhead includes that of fetching input activation data from the DIMM and sending the computation results back to the DIMM to trigger the merge kernel. Hence, as illustrated in Equation \ref{eq3}, the total GPU execution time also includes twice the single-direction synchronization overhead. 

% 整体的求解目标 for layer l 
{
\setlength\abovedisplayskip{0pt}
\setlength\belowdisplayskip{0pt}
\begin{equation}
    \min \textstyle \sum_{l} max (T_{GPU-l}, T_{DIMM-l}), \quad \forall l \in \mathbb{L} \label{eq1}
\end{equation}
\begin{equation}
     T_{DIMM-l} = max (T_{dimm-jl}), \quad \forall j \in \mathbb{D} \label{eq2}
\end{equation}
\begin{equation}
    T_{GPU-l} = T_{compute-GPU-l} + 2\cdot T_{sync} \label{eq3}
\end{equation}
}


The computation times for a single layer on both the GPU and NDP-DIMMs depend on the number of activated neurons located in each device. Let $T_{l}^{GPU}$ represent the time required to compute a single neuron on the GPU. Consequently, the computation time for a single layer on the GPU is the product of the number of activated neurons in the GPU memory and the time taken to compute each neuron, as illustrated in Equation \ref{eq4}. Similarly, the single-layer computation time for each NDP-DIMM is demonstrated in Equation \ref{eq5}.% Similarly, the NDP-DIMM's single-layer computation time is the product of the activation counts of all neurons stored in a single DIMM module and the computation time per neuron, as shown in \eq \ref{eq5}.


% GPU & NDP-DIMM 计算时间
{
\setlength\abovedisplayskip{0pt}
\setlength\belowdisplayskip{5pt}
\begin{align}
    T_{compute-GPU-l} = T_{l}^{GPU} \cdot \textstyle \sum_{i} f_{i}\cdot x_{il}^{GPU}, \quad \forall i \in \mathbb{N} \label{eq4}\\ 
    T_{dimm-jl} = T_{l}^{DIMM} \cdot \textstyle \sum_{i} f_{i} \cdot x_{il}^{dimm-j}, \quad \forall i \in \mathbb{N} \label{eq5}
\end{align}
}


\textbf{Constraints.}
The offline optimal neuron placement issue must adhere to the conditions listed in \eq \ref{eq6} and \ref{eq7}, which limit the memory space occupied by neurons not to exceed the available memory size of each DIMM and GPU.

% GPU & DIMM 存储约束
{
\setlength\abovedisplayskip{0pt}
\setlength\belowdisplayskip{5pt}
\begin{align}
   \textstyle \sum_{l} M_{i} \cdot x_{il}^{GPU} \le S_{GPU}, \quad \forall l \in \mathbb{L} \label{eq6}\\ 
   \textstyle \sum_{l} M_{i} \cdot x_{il}^{dimm-j} \le S_{dimm-j}, \quad \forall l \in \mathbb{L} \label{eq7}
\end{align}
}

Consequently, we employ the open-sourced optimization solver, PulP~\cite{pulp-solver}, to determine the optimal offline neuron mapping.
% Given that ILP problems are inherently NP-complete, directly solving them for an LLM with hundreds of billions of parameters presents a considerable computational challenge. 
Based on our assessment, it takes about 110 seconds to solve for the optimal neuron mapping, making it appropriate for a single offline compilation process. Before LLM inference, we initially transfer relevant hot neurons to GPU memory based on the mapping outcomes and further adjust the mapping during runtime to improve efficiency.
% However, the activation sparsity causes the neuron activation distribution to be non-static, 


% 准备把 C 和 D 两个小节的逻辑都换一下,重点应该是 hot/cold partition 以及 partition 之后的对应操作导致的影响,predictor 对 inference 的影响要弱化,不要去过多强调这个,是一个附加的。后续重点去考虑那部分,而非当前的predictor 的内容带来的实质性影响。 然后 C 的话,重点就是 hot/cold neurons 的预测,跟 D 的关系就没有那么大
\subsection{Online Adjustment for Hot/Cold Neuron Partition}\label{sec:partition-design}  

Although the optimal offline neuron mapping provides an effective hot/cold partition, the input-specific nature of activation sparsity makes the hot/cold neuron partition change dynamically in practice. Our evaluation indicates that about 52\% of the initialized hot neurons exhibit varied activity during inference. Therefore, it is necessary to adjust the hot/cold neuron partition online to improve inference efficiency before neuron computation, which requires an in-advance prediction of the neuron partition. In this section, we leverage the distribution patterns of activation sparsity to create a novel lightweight predictor to guide the online adjustment of the hot/cold neuron partition.
\begin{figure}[t]
    \centering
    \includegraphics[width=\linewidth]{Fig/predictor.pdf}
    \vspace{-0.3cm}
    \caption{The predictor design in Hermes. (a) We are motivated to utilize the temporal locality of token generation for prediction. (b) The layer-wise correlation effectively predicts activated neurons.}
    \label{fig:predictor}
\vspace{-0.3cm}
\end{figure}

\subsubsection{Predictor Design}\label{sec:predictor-design}
% 一方面,workflow中需要提前决定the computation loads for both GPU and NDP-DIMMs to utilize the activation sparsity; 另一方面,提前将 hot neurons 映射到 GPU 上有助于充分发挥 GPU 的算力,缓解NDP-DIMMs的负载。
Accurately forecasting activated neurons and the hot/cold neuron partition is crucial for improving inference performance. On one hand, to effectively harness activation sparsity, the \name~workflow necessitates predetermining the computation loads for both the GPU and NDP-DIMMs. On the other hand, assigning hot neurons to the GPU before computation can fully utilize the GPU’s computation capability and ease the burden on NDP-DIMMs. Nevertheless, existing MLP-based predictors~\cite{song2023powerinfer, song2024prosparse, liu2023deja} incur considerable storage and computation overhead, reducing inference efficiency. To address it, we introduce a lightweight predictor that exploits token-wise similarity and layer-wise correlation (discussed in Section \ref{sec:similar}) for accurate predictions.

\textbf{Token-wise Prediction. } The token-wise similarity suggests that the distribution of activated neurons is similar among adjacent tokens. Given that tokens are generated one by one during the token generation stage, token-wise similarity can be considered as a temporal locality of activated neurons. Inspired by well-known branch prediction strategies~\cite{smith1981study, yeh1991two, mcfarling1993combining} that also benefit from temporal locality, we propose a novel prediction strategy. As shown in \fig \ref{fig:predictor}a, we establish a neuron state table where each neuron has a 4-bit state, ranging from 0 to 15, used to predict whether the neuron will be activated. After the prefill stage, we initialize each neuron's state based on the activated frequency in the whole prefill stage. Specifically, we divide the distribution of the activated frequency into 16 stages and initialize each state accordingly. For example, if a neuron's activated frequency exceeds 90\%, its state is initialized as `15', whereas if the ratio is below 2\%, the state is set as `0'.

We update each neuron's state based on the actual activated neurons during each token generation step using a finite state machine. If a neuron is not activated, its state decreases by 1; if it is activated, its state increases by \( s \), which is set to 4 in this paper. The left part of \fig \ref{fig:predictor}a shows that, when neuron 6 is activated, the state is updated from $7$ to $11$, while the state of neuron 5 is updated from 10 to 9 as it is not activated.  

\begin{figure}
    \centering
    \includegraphics[width=0.9\linewidth]{Fig/mapper.pdf}
    \vspace{-0.3cm}
    \caption{Neuron mapper design. (a) The mapper utilizes the information in the neuron state table to adjust the hot/cold neuron partition. (b) Cold neurons are remapped based on the neuron activity within a window.}
    \label{fig:mapper}
\vspace{-0.3cm}
\end{figure}

\textbf{Layer-wise Prediction. } 
Token-wise similarity alone cannot address fluctuations in neuron activity between tokens~\cite{zhang2024relu, zheng2024learn}. Therefore, we further employ layer-wise correlation to improve prediction accuracy. Insights from Section \ref{sec:similar} suggest that if neurons with high correlation in the preceding layer are activated, the activated probability for the current neuron is significantly increased. Consequently, we create a neuron correlation table to boost layer-wise prediction. As depicted in Figure \ref{fig:predictor}b, we initially offline sampled the top 2 correlated neurons from the previous layer and documented their relationships in the neuron correlation table.
% During token generation, to determine whether it needs activation, we record the activation number of its top 2 correlated neurons. If the number exceeds a threshold \( T2 \), the neuron is predicted to be activated; otherwise, it is not.

% if a neuron has a state $s_1$ from the neuron state table and a number of $s_2$ highly-correlated neurons from the neuron correlation table are activated, its activation prediction follows: $s_1 + \lambda\cdot s_2> T $.

Finally, we combine the token-wise and layer-wise prediction strategies to achieve accurate prediction for activated neurons during token generation. Specifically, we use $s_1$ to denote the state in the neuron state table for one neuron, and use $s_2$ to indicate the activated number of the highly correlated neurons for one neuron. To predict the activation state for such a neuron, we examine the inequation: $s_1 + \lambda\cdot s_2> T $. In this paper, we set $\lambda$ as 6, and the threshold $T$ as 15. As Figure \ref{fig:predictor} shows, following the prediction criterion, we finally activate neurons 3, 6 and 9 for subsequent computation. During context switches, token similarity may vanish, but layer-wise correlation is still available for effective prediction. Conversely, even if correlated neurons are not activated, observing neighboring tokens' activation states still helps achieve accurate prediction. Experimental result shows that the accuracy of our proposed predictor achieves 98\% using less than 1MB of memory. \update{For instance, LLaMA-7B occupies 32 layers, with each one having 4K neurons for the self-attention block and 10.5K for the MLP block. In our implementation, only 4-bit data is used to record the corresponding state of each neuron. Consequently, it only costs 232 KB for the neuron state table of LLaMA-7B.} We integrate the proposed predictor into the host CPU and store the table values in the last level cache for fast prediction.


\subsubsection{Online Adjustment guided by Predictor}
Given their ample memory capacity, instead of mapping only cold neurons, we store all the weight parameters on DIMMs. Thus, we only need to reload the actual hot neurons onto GPU memory to achieve online adjustment. The neuron state in our proposed predictor effectively represents the activity of each neuron. Specifically, as shown in the \fig \ref{fig:mapper}a, once the neuron state exceeds a certain threshold $T_h$, it can be viewed as the hot neuron. In this paper, we set the threshold $T_h = 10$. Accordingly, neurons 3, 6, and 9 are identified as hot neurons. We then use the neuron mapper to locate the corresponding hot neuron. As the hot neuron 6 is originally located on the DIMMs, an instruction is issued to copy the corresponding hot neuron to the GPU memory during the projection computation. Meanwhile, the neuron with the lowest state value (neuron 5) stored in GPU memory will be swapped out. Note that, since all neurons are stored in DIMMs, we only need to overwrite the location of the neuron to be swapped out in the GPU memory to achieve neuron swapping. In general, online neuron adjustment between GPU and NDP-DIMMs significantly improves the inference efficiency without inducing additional data transfer overhead.


\subsection{Online Remapping for Cold Neurons}\label{sec:cold-neuron-mapping}
Due to our implementation of a center buffer-based NDP-DIMM architecture, the total computation delay correlates with the count of activated neurons in each DIMM module. As shown in Equation \ref{eq2}, the total execution duration is constrained by the slowest-performing NDP-DIMM module. Hence, determining the optimal cold neuron assignment to ensure a balanced load across multiple NDP-DIMMs is crucial.
Despite using DIMM-link for inter-DIMM communication, the limited bandwidth (25GB/s) cannot afford over-frequent data exchanges between DIMMs. Therefore, we need to achieve a load balance across multiple NDP-DIMMs while minimizing the remapping of cold neurons.

\begin{algorithm}[t]
\scriptsize
    \caption{Window-based online scheduling}\label{alg:balance}

\SetKwInOut{Input}{Input}\SetKwInOut{Output}{Output}

\Input{neuron mapping $C_{j,i}$; Activity for neuron $i$ within a window $A_{i}$; Number of NDP-DIMM modules $J$;}

\emph{{\textcolor{magenta}{// Compute the number of activated neurons for NDP-DIMM $i$.}}}
$Z_{j} = \sum_{i} C_{j,i} \cdot A_{i}$ \\ 
Sort $Z$ with the descending order \\

\For{{\textcolor{blue}{int}} id = 0; $id$ $<$ J/2; id++}{
  \While{$Z_{id} \le Z_{J-id}$}{
    Find the most activated neurons $h$ in NDP-DIMM $id$\\ 
    \emph{{\textcolor{magenta}{// Remapping the most activated neurons from $id$ to $J-id$}}}
    $C_{id, h} = 0$; $C_{J-id, h} = 1$
  }
}
\end{algorithm}

The similarity between tokens inspires us to develop a novel window-based online scheduling method for remapping cold neurons. In particular, we group every five consecutive tokens into a window. Based on our observations, due to the token-wise similarity, once the optimal mapping for cold neurons is identified, the runtime variance among different NDP-DIMMs within a window is under 5\%, indicating a balanced assignment. Nevertheless, when surpassing the window size, the performance disparity among different NDP-DIMMs varies from 1.2$\times$ to 2.5$\times$. Consequently, we can leverage the neuron activity within a window to guide the remapping of cold neurons. As shown in Algorithm \ref{alg:balance}, we initially gather the activated times for each neuron $i$ within a window and calculate the total activated neurons in NDP-DIMM $j$ based on the current neuron mapping $C_{j,i}$. $C_{j,i}$ is a binary matrix that denotes if neuron $i$ is mapped on NDP-DIMM $j$. We then sort the total activated neurons for NDP-DIMMs within the window and adjust neuron mappings between DIMM pairs accordingly. Specifically, the NDP-DIMM with the largest number of activated neurons is paired with the one that has the fewest activated neurons. Finally, the most activated neurons in the NDP-DIMM pair are remapped to achieve balance. As depicted in \fig \ref{fig:mapper}b, we record the activated neurons within a window into the neuron activity table, and calculate the activity for each NDP-DIMM based on the mapping results. As the count of activated neurons in DIMM-1 exceeds that of DIMM-2, neuron 5 from DIMM-1 is remapped to DIMM-2 for load balance between the two NDP-DIMMs. This strategy offers two advantages: first, the fixed inter-DIMM communication traffic is directed to different bridges to prevent congestion; second, the greedy remapping approach can quickly achieve balance with minimal data transfer.

\section{Evaluation}
\label{sec:evaluation}

\stelevaltable
\avevaltable

\paragraph{STEL-or-Content (SoC) Benchmark}

In order to evaluate our style embeddings, we construct a multilingual version of the SoC benchmark \citep{styleemb}.\footnote{The English SoC benchmark covered formality, complexity, number usage, contraction usage, and emoji usage.} SoC measures the ability of a model to embed sentences with the same style closer in the embedding space than sentences with the same content. We construct our \textbf{multilingual SoC benchmark} by sampling 100 pairs of parallel {\tt pos}-{\tt neg} examples for each language from four ground-truth datasets covering four style features and 22 languages: simplicity \citep{ryan-etal-2023-revisiting}, formality \citep{briakou-etal-2021-ola}, toxicity \citep{dementieva2024overview}, and positivity \citep{mukherjee2024multilingualtextstyletransfer}.\footnote{Combined, these datasets cover the following languages: Amharic, Arabic, Bengali, German, English, Spanish, French, Hindi, Italian, Japanese, Magahi, Malayalam, Marathi, Odia, Punjabi, Portuguese (Brazil), Russian, Slovenian, Telugu, Ukrainian, Urdu, and Chinese.} Each instance in our multilingual SoC benchmark consists of a triplet ($a$, {\tt pos}, {\tt neg}) constructed as explained in Section \ref{sec:styledistance}. However, following \citet{styleemb}, the distractor text in our SoC benchmark is always a paraphrase of {\tt pos}.
% \textcolor{gray}{For each instance of our multilingual SoC benchmark, we take two pairs of parallel examples to get (1) an anchor sentence, (2) a sentence with the same style but different content than the anchor, and (3) a sentence with the same content but different style than the anchor.}
A model tested on this benchmark is expected to embed $a$ and {\tt pos} closer than $a$ and {\tt neg}. We rate a model by computing the percent of instances it achieves this goal for across all instances. We form test instances for each $f \in F$ in a language corresponding to all possible triplets, resulting in 4,950  instances for each language-style combination.

We also construct a \textbf{cross-lingual SoC benchmark} that addresses embeddings' ability to capture style similarity \textit{across languages}. This can be useful, for example, to evaluate style preservation in translations. We construct the benchmark with the XFormal dataset \citep{briakou-etal-2021-ola}, which includes parallel data in French, Italian and Portuguese. %We use a similar formulation as described above to create each instance. However, rather than taking both pairs from the same language, we take the sentence pair (which is (2) and (3) described above) from a different language as the anchor pair. 
We again create triplets as described above, but instead of using {\tt pos} and {\tt neg} texts from the same language as the anchor ($a$), we sample them from a different language than $a$. We end up with 19,800 instances for each style in each language. Appendix \ref{sec:appendix:stelfig} contains illustrative examples from each benchmark.



% Next, we sample every combination of two paraphrase pairs from the 100 pairs for each style feature and language to form $_{100} C _{2} = 4,950$ STEL-or-Content instances. For each instance, we select one of the paraphrase pairs to be the anchor pair and the other to be the sentence pair. Following \citet{stel}, we replace one of the texts in the sentence pair with one of the texts from the anchor pair, resulting in (1) an anchor sentence, (2) a sentence with the same style but different content than the anchor, and (3) a sentence with the same content but different style than the anchor. The model is asked to embed (1) and (2) closer together than (1) and (3), and we compute the average across all instances.

% We elect not to use the STEL benchmark proposed by \citet{stel} because we believe STEL isn't as strong of a test on the content-independence of style embeddings as STEL-or-Content, and \citet{patel2024styledistancestrongercontentindependentstyle} find that base models trained for semantic embeddings perform well on STEL but not STEL-or-Content.

% We also construct crosslingual benchmarks that evaluate the ability of embeddings to capture style similarity \textit{across languages}. This can be useful, for example, to evaluate style preservation in translations. We address formality in this benchmark and use the XFormal dataset \citep{briakou-etal-2021-ola}, which includes data parallel in content across French, Italian and Portuguese. Given 100 paraphrase pairs in language A and 100 paraphrase pairs in language B, we create a STEL-or-Content instance out of every combination of pairs, resulting in $100 \cdot 99 = 9,900$ instances because we ignore the instances in which both pairs have the same content. We then use a similar process as described previously to assign the anchor and sentence pairs for every instance. Finally, we repeat this process with the style feature of the anchor sentence swapped (i.e. using the informal sentences rather than the formal sentences) to end up with $9,900 \cdot 2 = 19,800$ instances. Appendix \ref{sec:appendix:stelfig} contains examples from each benchmark.
% We will make our STEL-or-Content evaluation datasets publicly available as a resource.

\paragraph{SoC Evaluation Results}

The results obtained by \textsc{mStyleDistance} on the multilingual and cross-lingual SoC benchmarks are presented in Table \ref{table:steleval}. As no general multilingual style embeddings are currently available, we compare with a base multilingual encoder model \texttt{xlm-roberta-base} \citep{Conneau2019UnsupervisedCR} as well as a number of English-trained style embedding models applied in zero-shot fashion to multilingual text: \citet{styleemb}, \textsc{StyleDistance} embeddings \citep{patel2024styledistancestrongercontentindependentstyle}, and \texttt{LISA} \citep{lisa}. 
%\citep{lisa} as baseline models to compare against. 
\textsc{mStyleDistance} embeddings outperform these models on multilingual and cross-lingual SoC tasks confirming its suitability for multilingual applications. The other models perform slightly better than the untrained \texttt{xlm-roberta-base} but still worse than \textsc{mStyleDistance}. 

\paragraph{Ablation Experiments}

\ablationsimpletable

Following \citep{patel2024styledistancestrongercontentindependentstyle}, we run several ablation experiments to evaluate how well our model generalizes to unseen style features and languages. In the \textbf{In-Domain} condition, all style features are included in the training data for every language. To test generalization to unseen style features, in the \textbf{Out of Domain} condition, any style feature directly equivalent to those features tested in the Multilingual and Cross-lingual SoC  benchmarks are excluded from the training data. \textbf{Out of Distribution} further removes any style features indirectly similar or related to those tested in the benchmarks. Finally, \textbf{No Language Overlap} removes the languages present in the benchmark from the training data, in order to test generalization to new languages. Our results are given in Table \ref{table:simplifiedablation} where we measure how much of the performance increase on SoC benchmarks over the base model is retained, despite ablating training data. The results indicate that our method generalizes reasonably well to both ``out of domain'' and ``out of distribution'' style features, and very well to languages not in the training data. Further details on features and languages ablated and full results are provided in Appendices  \ref{sec:appendix:ablationdetails} and \ref{sec:appendix:ablationfull}.

\paragraph{Downstream Task}

Following \citet{patel2024styledistancestrongercontentindependentstyle}, we also evaluate our \textsc{mStyleDistance} embeddings in the authorship verification (AV) task, where the goal is to determine if two documents were written by the same author using stylistic features \citep{authorshipverification}. We use the datasets released by PAN\footnote{\url{https://pan.webis.de}} between 2013 and 2015 in Greek, Spanish, and Dutch. Our results are given in Table \ref{table:aveval_table}. \textsc{mStyleDistance} vectors outperform existing English style embedding models on Spanish and Greek, while Dutch shows similar performance to English \textsc{StyleDistance}. We hypothesize that the linguistic proximity (West Germanic roots) of the two languages helps \textsc{StyleDistance} to generalize to Dutch.






\section{Related Works}\label{sec:related-works}

\subsection{LLM Inference with PIM}\label{sec:related-works-PIM}

% It discusses the placement of computation units within HBM and identifies that GEMV units in the near bank of HBM are the optimal PIM placement for LLMs.

% Given that LLM inference is primarily memory bandwidth-bound, using PIM to accelerate LLM inference is a natural choice. AttAcc!~\cite{park2024attacc} utilizes a hybrid architecture of HBM-PIM and xPU (GPU/TPU), offloading the attention computation to HBM-PIM. It explores pipeline parallelism at the attention head level to further improve the inference efficiency. NeuPIMs~\cite{heo2024neupims} and IANUS~\cite{seo2024ianus} address the compatibility issue between PIM functionality and regular memory access by adopting dual buffers and incorporating additional control units, respectively. They optimize the design of HBM-PIM to support both processing and memory access simultaneously, utilizing PIM and xPU collaboration for LLM inference acceleration. These approaches adopt a fixed offloading strategy, delegating specific operators such as attention to PIM. SpecPIM~\cite{li2024specpim}, on the other hand, targets speculative LLM models with a multi-device architecture, where each device includes a xPU and multiple HBM-PIM chips. SpecPIM formalizes the mapping and scheduling of various models during speculative inference as a design space exploration problem, achieving flexible resource allocation based on actual model requirements. However, these works are all designed for server-grade devices (such as H100) and rely on expensive HBM-PIM for LLM inference acceleration, making them unsuitable for local deployment with limited budget. 

Given that LLM inference is primarily memory bandwidth-bound, accelerating it with processing in memory (PIM) is a natural choice~\cite{li2020hitm,zhai2023star,zhu2023processing}. AttAcc!~\cite{park2024attacc} utilizes a hybrid architecture of HBM-PIM and xPU (GPU/TPU), offloading the attention computation to HBM-PIM. NeuPIMs~\cite{heo2024neupims} and IANUS~\cite{seo2024ianus} address the compatibility issue between PIM functionality and regular memory access by adopting dual buffers and incorporating additional control units, respectively. They optimize the design of HBM-PIM to support both processing and memory access simultaneously, utilizing PIM and xPU collaboration for LLM inference acceleration. SpecPIM~\cite{li2024specpim}, on the other hand, targets speculative LLM models with a multi-device architecture, where each device includes an xPU and multiple HBM-PIM chips. However, these works are all designed for server-grade devices (such as H100) and rely on expensive HBM-PIM for LLM inference acceleration, making them unsuitable for local deployment with a limited budget. 

\subsection{LLM Acceleration with Activation Sparsity}\label{sec:related-works-sparsity}

% The promising activation sparsity in deep learning models motivates researches~\cite{zheng2023pit, cui2023optimizing} to further improve their inference efficiency, especially for LLMs. Deja Vu~\cite{liu2023deja} utilizes the activation sparsity to reduce the memory access on the unified memory of multiple server-grade GPUs. However, it it still requires storing all parameter data in GPU memory, failing to reduce GPU storage overhead, and thus is not suitable for local deployment scenarios. Powerinfer~\cite{song2023powerinfer} introduces a CPU-GPU hybrid system to achieve activation sparsity-based LLM inference. It stores hot neurons in GPU memory and uses GPU tensor cores for the corresponding computations while storing cold neurons in CPU memory and utilizing the CPU as a computing unit. However, the CPU-side memory bandwidth is significantly lower than that in the GPU, making CPU-side computation a bottleneck. Additionally, Powerinfer primarily targets single-batch computation, resulting in suboptimal utilization of GPU computation capability. Overall, existing systems do not fully exploit the advantages of activation sparsity. It is necessary to utilize processing units with high bandwidth and large storage to fully leverage activation sparsity. 

The promising activation sparsity in deep learning models motivates researchers~\cite{zheng2023pit, cui2023optimizing,liu2024drift} to further improve their inference efficiency, especially for LLMs. Deja Vu~\cite{liu2023deja} utilizes the activation sparsity to reduce the memory access on the unified memory of multiple server-grade GPUs. However, it still requires storing all parameter data in GPU memory, failing to reduce GPU storage overhead. Powerinfer~\cite{song2023powerinfer} introduces a CPU-GPU hybrid system to achieve activation sparsity-based LLM inference. It stores hot neurons in GPU memory and uses GPU tensor cores for the corresponding computations while offloading cold neurons in CPU memory and utilizing the CPU as a computing unit. However, the CPU-side memory bandwidth is significantly lower than that in the GPU, making CPU-side computation a bottleneck. Overall, existing systems do not fully exploit the advantages of activation sparsity.
% \section{Discussion and Future Work}

% %Onscad is insample data cause these LLMS have seen openscad

% The decision space of language design is enormous, so we had to make some decisions about what to explore in the language design of AIDL. In particular, we did not build a new constraint system from scratch and instead developed ours based on an open-source constraint solver. This limited the types of primitives we allow, e.g. ellipses are not currently supported. \jz{Additionally, rectangles in AIDL are constrained to be axis-aligned by default because we found that in most use cases, a rectangle being rotated by the solver was unintuitive, and we included a parameter in the language allows rectangles to be marked as rotatable. While this feature was included in the prompts to the LLM, it was never used by the model. We hope to explore prompt-engineering techniques to rectify this issue in the future. Similarly, we hope to reduce the frequency of solver errors by providing better prompts for explaining the available constraints.} \adriana{Add two other limitations to this paragraph: that we typically noticed that things are axis alignment, say why we use this as default and in the future could try to get the gpt to not use default more often. Mention that we still have Solver failures that could be addressed by better engineering in future. }

% In testing our front-end, we observed that repeated instances of feedback tends to reduce the complexity of models as the LLM would frequently address the errors by removing the offending entity. This leads to unnecessarily removed details. More extensive prompt-engineering could be employed in future work to encourage the LLM to more frequently modify, rather than remove, to fix these errors. \adriana{no idea what this paragraph is trying to say}


% \adriana{This seems  like a future work paragraph so maybe start by saying that in the future you could do other front end or fine tune a model with aidl, we just tested the few shot.  } \jz{In the future, we hope to improve our front-end generation pipeline by finetuning a pretrained LLM on example AIDL programs.} In addition, multi-modal vision-langauge model development has exploded in recent months. Visual modalities are an obvious fit for CAD modeling -- in fact, most procedural CAD models are produced in visual editors -- but we decided not to explore visual inputs yet based on reports ([PH] cite OPENAIs own GPT4V paper) that current vision-language models suffter from the same spatial reasoning issues as purely textual models do (identifying relative positions like above, left of, etc.). This also informed our decision to omit spline curves which are difficult to describe in natural language. This deficit is being addressed by the development of new spatial reasoning datasets ([PH] cite visual math reasoning paper), so allowing visual user input as well as visual feedback in future work with the next generation of models seems promising.

% The decision space of language design is enormous, so it was impossible to explore it all here. We had to make some decisions about what to explore, guided by experience, conjecture, technical limitations, and anecdotal experience. Since we primarily explore the interaction between language design and language models in order to overcome the shortcomings in the latter, we did not wish to focus effort on building new constraint systems. This led us to use an open-source constraint solver to build our solver off of. This limited the types of primitives we allow; in particular, most commercial geometric solvers also support ellipses.

% In testing our generation frontend, we observed that repeated instances of feedback tended to reduce the complexity of models as the LLM would frequently address the errors by removing the offending entity. This is a fine strategy for over-constrained systems, but can unnecessarily remove detail when done in response to a syntax or validation error. More extensive prompt-engineering could be employed to encourage the LLM to more frequently modify, rather than remove, to fix these errors


% In recent months, multi-modal vision-language model development has exploded. Visual modalities are an obvious fit for CAD modeling -- in fact, most procedural CAD models are produced in visual editors -- but we decided not to explore visual input yet based on reports (cite OpenAIs own GPT4V paper) that current vision-language models suffer from the same spatial reasoning issues as purely textual models do (identifying relative positions like above, left of, etc.). This also informed our decision to omit spline curves; they are not easily described in natural language. This deficit is being addressed by the development of new spatial reasoning datasets (cite visual math reasoning paper), so allowing visual user input as well as visual feedback in future work with the next generation of models seems promising.



\section{Conclusion}

AIDL is an experiment in a new way of building graphics systems for language models; what if, instead of tuning a model for a graphics system, we build a graphics system tailored for language models? By taking this approach, we are able to draw on the rich literature of programming languages, crafting a language that supports language-based dependency reasoning through semantically meaningful references, separation of concerns with a modular, hierarchical structure, and that compliments the shortcomings of LLMs with a solver assistance. Taking this neurosymbolic, procedural approach allows our system to tap into the general knowledge of LLMs as well as being more applicable to CAD by promoting precision, accuracy, and editability. Framing AI CAD generation as a language design problem is a complementary approach to model training and prompt engineering, and we are excited to see how advance in these fields will synergize with AIDL and its successors, especially as the capabilities of multi-modal vision-language models improve. AI-driven, procedural design coming to CAD, and AIDL provides a template for that future.

% Using procedural generation instead of direct geometric generation enables greater editability, accuracy, and precision
% Using a general language model allows for generalizability beyond existing CAD datasets and control via common language.
% Approaches code gen in LLMs through language design rather than training the model or constructing complexing querying algorithms. This could be a complimentary approach
% Embedding as a DSL in a popular language allows us to leverage the LLMs syntactic knowledge while exploiting our domain knowledge in the language design
% LLM-CAD languages should hierarchical, semantic, support constraints and dependencies




%In this paper, we proposed AIDL, a language designed specifically for LLM-driven CAD design. The AIDL language simultaneously supports 1) references to constructed geometry (\dgone{}), 2) geometric constraints between components (\dgtwo{}), 3) naturally named operators (\dgthree{}), and 4) first-class hierarchical design (\dgfour{}), while none of the existing languages supports all the above. These novel designs in AIDL allow users to tap into LLMs' knowledge about objects and their compositionalities and generate complex geometry in a hierarchical and constrained fashion. Specifically, the solver for AIDL supports iterative editing by the LLM by providing intermediate feedback, and remedies the LLM's weakness of providing explicit positions for geometries.

%\adriana{This seems  like a future work paragraph so maybe start by saying that in the future you could do other front end or fine tune a model with aidl, we just tested the few shot.  }
%\paragraph{Future work} In recent months, multi-modal vision-language model development has exploded. Visual modalities are an obvious fit for CAD modeling -- in fact, most procedural CAD models are produced in visual editors -- but we decided not to explore visual input yet based on reports (cite OpenAIs own GPT4V paper) that current vision-language models suffer from the same spatial reasoning issues as purely textual models do (identifying relative positions like above, left of, etc.). This also informed our decision to omit spline curves; they are not easily described in natural language. This deficit is being addressed by the development of new spatial reasoning datasets (cite visual math reasoning paper), so allowing visual user input as well as visual feedback in future work with the next generation of models seems promising. 

\section{Acknowledgments}
We sincerely thank the anonymous reviewers for their insightful suggestions. This work was partially supported by the National Key R\&D Program of China (Grant No. 2023YFB4404400) and the National Natural Science Foundation of China (Grant No. 62222411, 62204164). Ying Wang is the corresponding author (wangying2009@ict.ac.cn).

\bibliographystyle{plain}
\bibliography{references}

\end{document}

