\documentclass[10pt,conference]{IEEEtran}
\usepackage{cite}
\usepackage{amsmath,amssymb,amsfonts}
% \usepackage{algorithmic}
\usepackage[linesnumbered,ruled,vlined]{algorithm2e}
\usepackage{graphicx}
\usepackage{textcomp}
\usepackage{tikz}
\usepackage{xcolor}
\usepackage[hyphens]{url}
\usepackage{fancyhdr}
\usepackage{hyperref}
\usepackage{marvosym}


% Ensure letter paper
\pdfpagewidth=8.5in
\pdfpageheight=11in

\newcommand*\circled[1]{\tikz[baseline=(char.base)]{
            \node[shape=circle,fill, inner sep=0pt, minimum width=0.3cm] (char) {\textcolor{white}{#1}};}}

\newcommand{\hpcayear}{2025}

%%%%%%%%%%%%%%%%%%%%%%%%%%%%%%%%%%%%%%%%
%%%%%%%%%%%%%% -- UPDATE -- %%%%%%%%%%%%%%%
\newcommand{\hpcasubmissionnumber}{306}
\title{Make LLM Inference Affordable to Everyone: Augmenting GPU Memory with NDP-DIMM}
%%%%%%%%%%%%%%%%%%%%%%%%%%%%%%%%%%%%%%%%


%%%%%%%%%%%%%%%%%%%%%%%%%%%%%%%%%%%%%%%%
%%%%%%%% -- ONLY FOR CAMERA READY -- %%%%%%%%
\def\hpcacameraready{} % Uncomment to build camera-ready version
\newcommand{\hpcapubid}{0000--0000/00\$00.00}
\newcommand\hpcaauthors{Lian Liu\textsuperscript{$1, 2, 3, \dagger$}, Shixin Zhao\textsuperscript{$1, 2, \dagger$}, Bing Li\textsuperscript{4}, Haimeng Ren\textsuperscript{1,5}, Zhaohui Xu\textsuperscript{1,5}, \\ Mengdi Wang\textsuperscript{1,2}, Xiaowei Li\textsuperscript{1,2,3}, Yinhe Han\textsuperscript{1,2}, and Ying Wang\textsuperscript{1,2, \Letter}}
\newcommand\hpcaaffiliation{Institute of Computing Technology, Chinese Academic of Sciences\textsuperscript{$1$}, \\ University of Chinese Academy of Sciences \textsuperscript{$2$}, Zhongguancun Laboratory\textsuperscript{$3$}, \\
Institute of Microelectronics, Chinese Academy of Sciences\textsuperscript{$4$}, \\
School of Information Science and Technology, ShanghaiTech University\textsuperscript{$5$}}
\newcommand\hpcaemail{ \{liulian211, zhaoshixin18\}@mails.ucas.ac.cn \quad libing2024@ime.ac.cn \quad \{renhm2022, xuzhh12022\}@shanghaitech.edu.cn \\ \{wangmengdi, lxw, yinhes, \textcolor{blue}{wangying2009}\}@ict.ac.cn}

%%%%% -- ARTEFACT EVALUATION RESULTS -- %%%%%%
% Uncomment the following based on the badges that were awarded to this paper1
%\def\aeopen{}           % The artifact is publically available
%\def\aereviewed{}     % The artefact has been reviewed
%\def\aereproduced{} % The results have been reproduced
%%%%%%%%%%%%%%%%%%%%%%%%%%%%%%%%%%%%%%%%
%%%%%%%%%%%%%%%%%%%%%%%%%%%%%%%%%%%%%%%%%%%%%%%%%%%%%%%
%%%%%%%%%%%%%%%    theorems %%%%%%%%%%%%%%%%%%%%%%%%%%%
%%%%%%%%%%%%%%%%%%%%%%%%%%%%%%%%%%%%%%%%%%%%%%%%%%%%%%%
% \usepackage{mdframed}
\usepackage{kantlipsum}

%%%%%%%%%%%%%%%%%%%%%%%%%%%%%%%%%%%%%%%%%%%%%%%%%%%%%%%
%%%%%%%%%%%%%%%    theorems %%%%%%%%%%%%%%%%%%%%%%%%%%%
%%%%%%%%%%%%%%%%%%%%%%%%%%%%%%%%%%%%%%%%%%%%%%%%%%%%%%%
\theoremstyle{plain}
\newtheorem{theorem}{Theorem}[section]
\newtheorem{proposition}[theorem]{Proposition}
\newtheorem{lemma}[theorem]{Lemma}
\newtheorem{example}[theorem]{Example}
\newtheorem{corollary}[theorem]{Corollary}
\theoremstyle{definition}
\newtheorem{definition}[theorem]{Definition}
\newtheorem{assumption}[theorem]{Assumption}
\theoremstyle{remark}
\newtheorem{remark}[theorem]{Remark}


% \titleformat{\subsection}[runin]% runin puts it in the same paragraph
%        {\normalfont\bfseries}% formatting commands to apply to the whole heading
%        {\thesubsection}% the label and number
%        {0.5em}% space between label/number and subsection title
%        {}% formatting commands applied just to subsection title
%        [.]% punctuation or other commands following subsection title


%%%%%%%%%%%%%%%%%%%%%%%%%%%%%%%%%%%%%%%%%%%%%%%%%%%%%%%
%%%%%%%%%%%%%%%  mathematical notations%%%%%%%%%%%%%%%%
% \usepackage[english]{babel}
% \usepackage[utf8]{inputenc}
% \usepackage[T1]{fontenc}

%% Figures, tables and lists
\usepackage[dvipsnames]{xcolor}
\usepackage{paralist}
\usepackage{graphicx}
\usepackage{subcaption}
\usepackage{longtable} 
\usepackage{multirow}
\usepackage{listings}
\usepackage{makecell}
\usepackage{array}
\usepackage{float}
\usepackage{dsfont}
\usepackage{rotating}
\usepackage{booktabs}
\usepackage{enumerate}
\usepackage{tikz}
\usepackage{pgf}
\usepackage{enumitem}
\usepackage{lipsum} % for generating filler text
\usepackage{titlesec}

%% Math
% \usepackage{amssymb, amsthm,bbm}
\usepackage{mathtools}
\usepackage{mathrsfs}
%% References and author info 
\mathtoolsset{showonlyrefs}
\usepackage{natbib}
\usepackage{authblk}
\usepackage{todonotes}
\usepackage{xr-hyper}


%%%%%%%%%%%%%%%%%%%%%%%%%%%%%%%%%%%%%%%%%%%%%%%%%%%%%%%
\newcommand{\R}{\mathbb R}
\newcommand{\EE}{\mathbb{E}}

\DeclareMathOperator{\Tr}{Tr}
\DeclareMathOperator*{\argmin}{argmin}
\DeclareMathOperator*{\argmax}{argmax}

\newcommand{\bs}[1]{\ensuremath{\boldsymbol{#1}}}
\newcommand{\mc}{\mathcal}
\newcommand{\opt}{^\star}


\newcommand{\diff}{\textnormal{d}}


\def \iid {\stackrel{\textnormal{i.i.d.}}{\sim}}
\def \iidtext {\textnormal{i.i.d.}}





%%%%%%%%%%%%%%%%%%%%%%%%%%%%%%%%%%%%%%%%%%%%%%%%%%%%%%%
%%%%%%%%%%%%%%%%%%%%% colors     %%%%%%%%%%%%%%%%%%%%%%
%%%%%%%%%%%%%%%%%%%%%%%%%%%%%%%%%%%%%%%%%%%%%%%%%%%%%%%
\definecolor{myblue}{rgb}{.8, .8, 1}
\definecolor{mathblue}{rgb}{0.2472, 0.24, 0.6} % mathematica's Color[1, 1--3]
\definecolor{mathred}{rgb}{0.6, 0.24, 0.442893}
\definecolor{mathyellow}{rgb}{0.6, 0.547014, 0.24}


% May add more in future.






%%%%%%%%%%%%%%%%%%%%%%%%%%%%%%%%%%%%%
%%%%%%%%%% -- DO NOT MODIFY -- %%%%%%%%%%
%%%%%%%%%%%%%%%%%%%%%%%%%%%%%%%%%%%%%

\author{
  \ifdefined\hpcacameraready
    \IEEEauthorblockN{\hpcaauthors{}}
      \IEEEauthorblockA{
        \hpcaaffiliation{} \\
        \hpcaemail{}
      }
  \else
    \IEEEauthorblockN{\normalsize{HPCA AE \hpcayear{} Submission \textbf{\#\hpcasubmissionnumber{}}} \\
      \IEEEauthorblockA{
        Confidential Draft \\
        Do NOT Distribute!!
      }
    }
  \fi 
}

% Heading and footer for title page
\fancypagestyle{camerareadyfirstpage}{%
  \fancyhead{}
  \renewcommand{\headrulewidth}{0pt}
  \fancyhead[C]{
    \ifdefined\aeopen
    \parbox[][12mm][t]{13.5cm}{\hpcayear{} IEEE International Symposium on High-Performance Computer Architecture (HPCA)}    
    \else
      \ifdefined\aereviewed
      \parbox[][12mm][t]{13.5cm}{\hpcayear{} IEEE International Symposium on High-Performance Computer Architecture (HPCA)}
      \else
      \ifdefined\aereproduced
      \parbox[][12mm][t]{13.5cm}{\hpcayear{} IEEE International Symposium on High-Performance Computer Architecture (HPCA)}
      \else
      \parbox[][0mm][t]{13.5cm}{\hpcayear{} IEEE International Symposium on High-Performance Computer Architecture (HPCA)}
    \fi 
    \fi 
    \fi 
    \ifdefined\aeopen 
      \includegraphics[width=12mm,height=12mm]{ae-badges/open-research-objects.pdf}
    \fi 
    \ifdefined\aereviewed
      \includegraphics[width=12mm,height=12mm]{ae-badges/research-objects-reviewed.pdf}
    \fi 
    \ifdefined\aereproduced
      \includegraphics[width=12mm,height=12mm]{ae-badges/results-reproduced.pdf}
    \fi
  }
  %\fancyfoot[L]{\hpcapubid{} \copyright \hpcayear{} IEEE}
  \fancyfoot[C]{}
}
% Heading and footer for remaining pages
\fancyhead{}
\renewcommand{\headrulewidth}{0pt}
%\fancyhead[C]{\hpcayear{} IEEE International Symposium on
% High-Performance Computer Architecture (HPCA)}


\begin{abstract}
The billion-scale Large Language Models (LLMs) necessitate deployment on expensive server-grade GPUs with large-storage HBMs and abundant computation capability. As LLM-assisted services become popular, achieving cost-effective LLM inference on budget-friendly hardware becomes the current trend. This has sparked extensive research into relocating LLM parameters from expensive GPUs to external host memory. However, the restricted bandwidth between the host and GPU memory limits the inference performance of existing solutions.

This work introduces Hermes, a budget-friendly system that leverages the near-data processing units (NDP) within commodity DRAM DIMMs to enhance the performance of a single consumer-grade GPU, achieving efficient LLM inference. We recognize that the inherent activation sparsity in LLMs naturally divides weight parameters into two categories, termed ``hot" and ``cold" neurons, respectively. Hot neurons, which consist of only approximately 20\% of all weight parameters, account for 80\% of the total computational load. In contrast, cold neurons make up the other 80\% of parameters but are responsible for just 20\% of the computational workload. Leveraging this observation, we propose a heterogeneous computing strategy: mapping hot neurons to a single computation-efficient GPU without large-capacity HBMs, while offloading cold neurons to NDP-DIMMs, which offer large memory size but limited computation capabilities. In addition, the dynamic nature of activation sparsity necessitates a real-time partition of hot and cold neurons and adaptive remapping of cold neurons across multiple NDP-DIMM modules. To tackle these issues, we introduce a lightweight predictor that ensures optimal real-time neuron partition and adjustment between GPU and NDP-DIMMs. Furthermore, we utilize a window-based online scheduling mechanism to maintain load balance among multiple NDP-DIMM modules. In summary, Hermes facilitates the deployment of LLaMA2-70B on consumer-grade hardware at a rate of 13.75 tokens/s and realizes an average 75.24$\times$ speedup over the state-of-the-art offloading-based inference system on popular LLMs.

\end{abstract}

\maketitle % should come after the abstract
\thispagestyle{empty}
\pagestyle{empty}

% \pagestyle{plain} % should come right after \maketitle
\def\thefootnote{$\dagger$}\footnotetext{Both authors contributed equally to this research}\def\thefootnote{\arabic{footnote}}
\def\thefootnote{\Letter}\footnotetext{Corresponding author}\def\thefootnote{\arabic{footnote}}


\section{Introduction}

Chain-of-Thought (CoT) prompting~\cite{Nye:2021, cot, Kojima:2022cotzero} has emerged as a cornerstone strategy for enhancing Large Language Models (LLMs) in complex reasoning tasks. By eliciting step-by-step inference, CoT enables LLMs to decompose intricate problems into manageable subtasks, thereby improving their problem-solving performance~\cite{Yao:2023tot, Wang:2023self-consistency, Zhou:2023least, Shinn:2023Reflexion}. Recent advancements, such as OpenAI's o1~\cite{o1} and DeepSeek-R1~\cite{deepseekr1}, further demonstrate that scaling up CoT lengths from hundreds to thousands of reasoning steps could continuously improve LLM reasoning. These breakthroughs have underscored CoT’s potential to advance LLM capabilities, expanding the boundaries of AI-driven problem-solving.

\begin{figure}[t]
\centering
    \includegraphics[width=0.95\columnwidth]{fig/intro.pdf}
    \caption{In contrast to vanilla CoT that generates all reasoning tokens sequentially, \method enables LLMs to \textit{skip} tokens with less semantic importance (\textit{e.g.,} \includegraphics[width=7pt]{fig/token.pdf}~) and learn shortcuts between critical reasoning tokens, facilitating controllable CoT compression.}
    \label{fig:intro}
\end{figure}

Despite its effectiveness, the increased length of CoT sequences introduces substantial computational overhead. Due to the autoregressive nature of LLM decoding, longer CoT outputs lead to proportional increases in both inference latency and memory footprints of key-value cache. Additionally, the quadratic computational cost of attention layers further exacerbates this burden. These issues become particularly pronounced when CoT sequences extend into thousands of reasoning steps, resulting in significant computational costs and prolonged response times. While prior research has explored methods for selectively skipping reasoning steps~\cite{Ding:2024cotshortcut, liu2024skipstep}, recent findings~\cite{jin:2024cotlength, Merrill:2024cotlength} suggest that such reductions may conflict with test-time scaling~\cite{o1-blog, snell2025scaling}, ultimately impairing LLM reasoning performance. Therefore, striking an optimal balance between CoT efficiency and reasoning accuracy remains a critical open challenge.

In this work, we delve into CoT efficiency and seek the answer to an important question: \textit{``Does every token in the CoT output contribute equally to deriving the answer?''} We empirically analyze the semantic importance of tokens within CoT outputs and reveal that their contributions to the reasoning performance vary, as depicted in Figure 2. Building on this insight, we introduce \method, a simple yet effective approach that enables LLMs to \textit{skip} less important tokens within CoT sequences and learn shortcuts between critical reasoning tokens, thereby allowing for controllable CoT compression with adjustable ratios. Specifically, as shown in Figure~\ref{fig:intro}, \method constructs compressed CoT training data with various compression ratios, by pruning unimportance tokens from original LLM CoT trajectories. Then, it conducts a general supervised fine-tuning process on target LLMs with this training data, facilitating LLMs to automatically trim redundant tokens during reasoning.

We conduct extensive experiments across various models, including LLaMA-3.1-8B-Instruct and the Qwen2.5-Instruct series, using two widely recognized math reasoning benchmarks: GSM8K and MATH-500. The results validate the effectiveness of \method in compressing CoT outputs while maintaining robust reasoning performance. Notably, Qwen2.5-14B-Instruct exhibits almost \textbf{NO} performance drop (less than $0.4\%$) with a $\bm{40\%}$ reduction in token usage on GSM8K. On the challenging MATH-500 dataset, LLaMA-3.1-8B-Instruct effectively reduces CoT token usage by $\bm{30}\%$ with a performance decline of less than $4\%$, resulting in a $\bm{1.4}\times$ inference speedup. Further analysis underscores the coherence of \method in specified compression ratios and its potential scalability with stronger compression techniques.

\method is distinguished by its low training cost. For Qwen2.5-14B-Instruct, \method fine-tunes only 0.2\% of the model's parameters using LoRA. The size of the compressed CoT training data is no larger than that of the original training set, with 7,473 examples in GSM8K and 7,500 in MATH. The training is completed in approximately 2 hours for the 7B model and 2.5 hours for the 14B model on two 3090 GPUs. These characteristics make \method an efficient and reproducible approach, suitable for use in efficient and cost-effective LLM deployment.

To sum up, our key contributions are:
\begin{enumerate}
    \item To the best of our knowledge, this work is the \textit{first} to investigate the potential of enhancing CoT efficiency through \textit{token skipping}, inspired by the varying semantic importance of tokens in CoT trajectories of LLMs.
    \item We introduce \method, a simple yet effective approach that enables LLMs to skip redundant tokens within CoTs and learn shortcuts between critical tokens, facilitating CoT compression with adjustable ratios.
    \item Our experiments validate the effectiveness of \method. When applied to Qwen2.5-14B-Instruct, \method reduces reasoning tokens by $40\%$ (from 313 to 181) on GSM8K, with less than a $0.4\%$ performance drop.
\end{enumerate}

\subsection{Gene Expression Classification with ML models}
Gene expression classification \cite{do2024enhancing,do2023ensemble,huynh2019novel} lies at the forefront of biomedical research, offering profound insights into the molecular mechanisms underlying various diseases. ML models have become indispensable in this domain, as they can uncover complex patterns within vast and high-dimensional gene expression datasets. However, these datasets often contain a plethora of features, many of which are redundant or irrelevant, potentially obscuring the most critical biological signals and leading to overfitting. Consequently, feature selection becomes imperative—it refines the dataset by isolating the most informative genes, thereby enhancing model accuracy, interpretability, and computational efficiency. By focusing solely on the pivotal biomarkers, this research is able to achieve more reliable predictive outcomes. In this paper, we investigate and evaluate the classification with various ML techniques. Namely, we experiment our selected features with ML algorithms, i.e., SVM \cite{vapnik1995support}, Random Forest \cite{breiman2001random}, XGB \cite{chen2015xgboost}, Gradient Boosting \cite{friedman2002stochastic}.

\begin{definition}[Classification]
Let \( D = (X, y) \) be a dataset where \( X \subseteq \mathbb{R}^n \) is the feature space and \( y \in \mathcal{Y} = \{1,2,\dots,k\} \) represents the class labels. A classifier is a function
\[
f: X \to \mathcal{Y},
\]
that assigns a predicted label \( \hat{y} = f(x) \) to each input \( x \in X \). The function \( f \) is learned from the labeled examples
\[
D = \{(x_i, y_i) \mid x_i \in X,\; y_i \in \mathcal{Y},\; i = 1, \dots, N\},
\]
by minimizing a loss function \( \ell: \mathcal{Y} \times \mathcal{Y} \to \mathbb{R}_{\ge 0} \) that quantifies the error between the predicted and true labels. Once trained, \( f \) is used to classify new, unseen inputs.
\end{definition}

% \begin{definition}[Classification Using Machine Learning]
% Let \( D_{\text{selected}} = (X_{\text{selected}}, y) \) be the dataset with features \( X_{\text{selected}} \subseteq X^* \) as determined by LIME. A classifier is a function 
% \[
% f: X_{\text{selected}} \to \mathcal{Y},
% \]
% that assigns a predicted label \( \hat{y} = f(x) \) to each input \( x \in X_{\text{selected}} \). The classifier is trained on the labeled examples
% \[
% D_{\text{selected}} = \{(x_i, y_i) \mid x_i \in X_{\text{selected}},\; y_i \in \mathcal{Y},\; i = 1, \dots, N\},
% \]
% by minimizing a loss function \( \ell: \mathcal{Y} \times \mathcal{Y} \to \mathbb{R}_{\ge 0} \) that measures the discrepancy between the predicted and true labels. The trained classifier is then used to predict the classes of new, unseen instances.
% \end{definition}


Feature selection is crucial before classification begins. Our study focuses on two techniques: Boruta and LIME. 
% Boruta is chosen for its robustness in identifying all relevant features in high-dimensional datasets, ensuring no important predictor is missed. LIME is used for its ability to provide interpretable, local explanations of model predictions, which is essential for evaluating feature importance. 
We now introduce Boruta and LIME in the following sections.

\subsection{Leveraging Boruta for Robust Feature Extraction}
Boruta \cite{kursa2010boruta,zhou2023diabetes} is a powerful wrapper-based feature selection algorithm designed to identify all truly relevant variables in a dataset. By comparing the importance of actual features with that of randomly generated ``shadow'' features, Boruta systematically filters out irrelevant variables while preserving essential predictors. This rigorous selection process is particularly valuable in high-dimensional applications, such as gene expression classification, where capturing meaningful signals is crucial. For clarity, we formally define Boruta as follows:
\begin{definition}[Boruta Feature Selection]
Let \( D = (X, y) \) be a dataset with features \( X = \{x_1, x_2, \dots, x_p\} \) and target \( y \). The Boruta algorithm identifies all relevant features in \( X \) as follows:
\begin{enumerate}
    \item \textbf{Shadow Feature Generation:} For each \( x_i \in X \), create a shadow feature \( x_i^{\text{shadow}} \) by randomly permuting its values, forming the set \( X^{\text{shadow}} \).
    \item \textbf{Importance Estimation:} Train a classifier (e.g., Random Forest) on the combined set \( X \cup X^{\text{shadow}} \) and compute the importance score \( I(z) \) for each \( z \).
    \item \textbf{Feature Comparison:} For each \( x_i \), define
    \[
    I^{\text{shadow}}_{\max} = \max_{z \in X^{\text{shadow}}} I(z).
    \]
    Then classify \( x_i \) as \emph{relevant} if \( I(x_i) \) is significantly greater than \( I^{\text{shadow}}_{\max} \), \emph{irrelevant} if significantly lower, or \emph{tentative} otherwise.
    \item \textbf{Iteration:} Remove irrelevant and tentative features and repeat until all features are decisively classified.
\end{enumerate}
The final selected subset \( X^* \subseteq X \) comprises all features deemed relevant.
\end{definition}

After applying the Boruta algorithm, we retain only the relevant features (confirmed) and excluded the tentative and irrelevant features (rejected). To further enhance the selection of features in \(X^*\), we employed the AI explanation technique outlined in the following section.

% \begin{definition}[Boruta Feature Selection]
% Given a dataset \( D = (X, y) \) with original features \( X = \{ x_1, x_2, \dots, x_p \} \), Boruta augments \( X \) by creating shadow features \( X^{\text{shadow}} = \{ x_1^{\text{shadow}}, \dots, x_p^{\text{shadow}} \} \) via random permutation. A model \( M \) (e.g., Random Forest) is then trained on \( X \cup X^{\text{shadow}} \) to compute importance scores \( I(z) \) for every feature \( z \). For each \( x_i \in X \), if \( I(x_i) \) is significantly greater than the maximum shadow importance \( I^{\text{shadow}}_{\max} = \max_{z \in X^{\text{shadow}}} I(z) \), then \( x_i \) is marked as relevant; otherwise, it is rejected or considered tentative. Iterating this process yields the final set of selected features \( X^* \subseteq X \).
% \end{definition}

\subsection{XAI for Feature Selection}
Explainable AI (XAI) \cite{dwivedi2023explainable,zacharias2022designing} represents a forefront of AI research, aiming to elucidate the decision-making processes of complex models. In the context of gene expression classification, where feature selection is pivotal to model performance and interpretability, our study leverages LIME—Local Interpretable Model-Agnostic Explanations—to demystify and select critical features. LIME approximates the behavior of a sophisticated, black-box model with a simpler, locally interpretable surrogate, thereby pinpointing the most influential predictors in the vicinity of a given instance. This approach enhances the transparency of the model's predictions and facilitates a more informed and rigorous feature selection process, ultimately contributing to both improved accuracy and trustworthiness of the classification system.  Now, we provide a formal definition of LIME as follows:

% \begin{definition}[LIME-based Feature Selection]
% Let \( D = (X, y) \) be a dataset and \( f: X \to \mathcal{Y} \) a trained black-box classifier, where \( X \subseteq \mathbb{R}^p \) and \( \mathcal{Y} = \{1,2,\dots,k\} \). For a given instance \( x \in X \), LIME constructs an interpretable surrogate model \( g \) from a simple model class \( G \) (typically linear), expressed as
% \[
% g(z) = w_0 + \sum_{j=1}^{p} w_j z_j.
% \]
% The surrogate \( g \) is fitted by minimizing the weighted loss
% \[
% \min_{g \in G} \sum_{z \in Z_x} \pi_x(z) \left( f(z) - g(z) \right)^2 + \Omega(g),
% \]
% where \( Z_x \) is a set of perturbed samples around \( x \), \( \pi_x(z) \) is a proximity measure between \( z \) and \( x \), and \( \Omega(g) \) is a regularization term enforcing simplicity. The absolute coefficients \( |w_j| \) quantify the local importance of each feature, thus guiding feature selection.
% \end{definition}
\begin{definition}[LIME-based Feature Selection]
Let \( D^* = (X^*, y) \) be the dataset resulting from Boruta, where \( X^* \subseteq \mathbb{R}^{p^*} \) is the set of relevant features. Given a trained black-box classifier \( f: X^* \to \mathcal{Y} \) and an instance \( x \in X^* \), LIME constructs an interpretable surrogate model \( g \in G \) (typically linear), expressed as
\[
g(z) = w_0 + \sum_{j=1}^{p^*} w_j z_j,
\]
by solving the optimization problem
\[
\min_{g \in G} \sum_{z \in Z_x} \pi_x(z) \left( f(z) - g(z) \right)^2 + \Omega(g),
\]
where \( Z_x \) is a set of perturbed samples in the neighborhood of \( x \), \( \pi_x(z) \) is a proximity measure, and \( \Omega(g) \) enforces simplicity. The absolute coefficients \( |w_j| \) indicate the local importance of each feature, enabling a further refined selection \( X_{\text{selected}} \subseteq X^* \) for classification.
\end{definition}


To clarify, our choice of LIME for feature selection arises from the critical question of determining the optimal number of features for the model. In this context, assessing the local importance of each vector proves to be the most effective strategy, leading us to introduce the BOLIMES algorithm. The following section will provide a comprehensive explanation of the BOLIMES algorithm and its application.

%--------------------





\section{Motivations \& Challenges}

\subsection{Why NDP-DIMM Enhanced GPU?}
% 这里再跟 powerinfer 比,以及说明 CPU 的扩展的问题。

% 1. offloading 不行 -> 哪怕有activation sparsity, 也不行;但是我们进一步分析 activation sparsity 的特点,说明 hot/cold 的划分特点; 根据特点,选择硬件。 (low cost ?)

% 首先要说明为什么选择 NDP-DIMM
Offloading is essential for LLM inference on low-budget systems with a single consumer-grade GPU. However, as noted in Section \ref{sec:background-offloading}, even utilizing activation sparsity to reduce weight parameter access, the PCIe bandwidth remains the bottleneck. Thus, costly data transfers between extended memory and GPU must be minimized. However, simply offloading the corresponding computation of cold neurons on the host CPU~\cite{llama.cpp, song2023powerinfer} can only achieve a limited performance improvement, \update{as the host CPU can only access DRAM with limited improved bandwidth than PCIe (e.g., 89.6 GB/s vs. 64 GB/s)}. To this end, we choose to employ multiple NDP-DIMMs as the extended memory, as they offer comparable bandwidth and larger storage capacity than a single consumer-grade GPU. Need to mention that as a budget-friendly host memory solution, we do not consider high-performance but expensive HBM-PIM and AiM~\cite{cong2017aim, park2024attacc} in this study. Given the limited computation capability, only utilizing the processing units in NDP-DIMMs cannot boost the inference efficiency~\cite{wu2024pim}. Consequently, we are motivated to use NDP-DIMMs to enhance GPU for efficient LLM inference.

% neurons 划分来协助相应的计算
Our observation indicates that the activation sparsity within LLMs effectively partitions weight parameters into two distinct regions, which are ideally suited to consumer-grade GPU and NDP-DIMMs, respectively. Specifically, activation sparsity in LLMs follows a power-law distribution~\cite{xue2024powerinfer, song2023powerinfer}. About 20\% of neurons (\textit{hot neurons}) account for 80\% of computations, while 80\% (\textit{cold neurons}) handle only 20\%. Hot neurons, with $16 \times$ higher computation intensity, fit GPU memory, while cold neurons suit NDP-DIMMs. During inference, GPU can provide high computation capability for hot neurons and NDP-DIMMs enable the cold neurons computation in memory.

\subsection{Necessity of Hot/Cold Neuron Partition} \label{sec:similar}

% 1. 首先需要说明为什么要进行 hot/cold neurons partition, 然后说明对应的 challenges,以及目前的预测的缺陷
% 根据我们的评估,在LLaMA-70B 中,有约 52% 的 hot neurons 在推理过程中会发生变化,导致the predetermined hot/cold neurons partition 相比于 oracle 的划分产生 1.63x 的performance 下降。 
Hot/cold neuron partition impacts the computational load on GPU/NDP-DIMMs, affecting the inference performance of the heterogeneous system. Due to the input-specific nature of activation sparsity, solely relying on the offline partition is insufficient. Our evaluation on LLaMA2-70B reveals significant dynamics in when the neuron will be activated (hereafter, neuron activity patterns) during inference. Approximately 52\% of the initialized hot neurons exhibit varied activity during inference. This variability in neuron behavior results in suboptimal performance with a fixed hot/cold partition, causing a $1.63\times$ degradation compared to an oracle (the theoretically optimal partition) scheme. Thus, we must dynamically predict and adjust the hot/cold neuron partition.

However, typical MLP-based predictors~\cite{liu2023deja, song2024prosparse, zhang2024relu, mirzadeh2023relu} for activation sparsity in LLMs are costly. For example, predicting the activated neurons in LLaMA-7B needs per-layer MLP-based predictors, requiring an extra 2GB storage and inducing 10\%-25\% inference runtime. Fortunately, the inherent locality of activation sparsity leads us to design a lightweight and accurate predictor for efficient online partition adjustments. To be specific, we found that activation sparsity in LLM inference shows considerable token-wise similarity and layer-wise correlation, worth exploiting.

\begin{figure}[t]
    \centering
    \includegraphics[width=0.98\linewidth]{Fig/prediction-motivation.pdf}
    \vspace{-0.3cm}
   \caption{Distribution patterns for activation sparsity. (a) \update{The adjacent tokens enjoy high similarity on activated neurons for various models and datasets.} (b) The activated neurons between consecutive layers are highly correlated.}
    \label{fig:similarity}
\vspace{-0.3cm}
\end{figure}

\subsubsection{Token-wise Similarity}
% As mentioned in the previous section, the dynamically sparse neurons exhibit a power-law distribution, resulting in high locality among activated neurons. This indicates that we can divide all neurons into hot neurons and cold neurons. However, this division is not fixed and requires dynamic adjustment. 
We analyzed the similarity between tokens to explore the distribution characteristics of activation sparsity. \update{As shown in \fig \ref{fig:similarity}a, we evaluate the token-wise similarity for LLaMA-13B and Falcon-40B with multiple widely adopted datasets, including COPA~\cite{roemmele2011choice}, Wikitext2~\cite{merity2016pointer} and PIQA~\cite{bisk2020piqa}. As one can notice, the adjacent tokens have a higher distribution similarity than distant tokens. Specifically, the similarity between adjacent tokens exceeds 90\% (95\% for Falcon-40B), but drops to 70\% once the tokens' distance exceeds 10.} This indicates that in context, adjacent tokens often express similar meanings, leading to high similarity in their activity distribution. Additionally, we observe that when the distance between tokens exceeds 25, the distribution similarity almost no longer decreases, indicating that beyond a certain window size, the semantic correlation becomes weak and has less impact on the overall distribution.

\subsubsection{Layer-wise Correlation} 
We further observed that the distribution of activated neurons in two consecutive layers is highly correlated. As shown in \fig \ref{fig:similarity}b, when the 6th neuron in layer-30 of LLaMA-13B is activated, the probability of neurons 0 and 5 being activated in layer-31 exceeds 90\%. This suggests that we can use the results of the preceding layer to predict the distribution of activated neurons in the current layer.

Overall, the token-wise similarity and layer-wise correlation motivate us to design a lightweight online predictor based on historical activation information. According to the prediction results, we can online adjust the hot/cold neurons partition to effectively exploit the processing advantages of the consumer-grade GPU and NDP-DIMMs, respectively.

% to predict whether to compute neurons of the current token in a specific layer. 
% Compared to costly MLP predictors, this historical activation-based strategy introduces negligible overhead.

\subsection{Load Imbalance across Multiple NDP-DIMMs}
% When initially distributing cold neurons across multiple DIMMs, we analyze past computational patterns to estimate the potential workload of these neurons. This allows us to distribute their corresponding weights across DIMMs in a manner that aims to evenly distribute the expected computational load among them. However, due to the inherently stochastic nature of cold neuron computations, relying on a static weight distribution strategy can result in imbalanced workload distribution across DIMMs.
% imbalance的来源+数据imbanlance有多大 -> 会很影响性能,所以我们要去做load balance的事情,DIMM间也需要有个简易的数据通路
Due to the storage limitation of a single DIMM, multiple DIMMs are required to store all the neurons (weight parameters) in LLM. Specifically, one DIMM only stores portions of the neurons and the corresponding processing unit can only directly assess neurons in the DIMM with high internal bandwidth. However, due to the input-specific nature of activation sparsity, the computational load on each NDP-DIMM can be diverse. For example, when fixing the cold neuron distribution on multiple DIMMs for LLaMA-13B, the most overloaded NDP-DIMM will have 1.2-2.5$\times$ more computational load than others.
% Furthermore, given the limited computational capacity of individual DIMMs, an overloaded DIMM can potentially bottleneck the entire system.

%是不是也可以在这里说,使用center buffer的DIMM,通过在buffer中添加很少的单元就可以实现DIMM间高效的通信而无需CPU参与,且每个DIMM中的PU都可以获得对应DIMM上的所有数据,所以我们选择了center-buffer 的NDP-DIMM
Therefore, we need an online scheduling strategy to remap the cold neuron across DIMMs to achieve load balance. Meanwhile, an efficient data transmission pathway among DIMMs is essential to help adjust the neuron placement. By optimizing neuron computation scheduling, we can minimize data transfers across NDP-DIMMs while ensuring balanced computational loads across DIMMs. This ensures that all parts of the system can maximize their performance.

In summary, the NDP-DIMM enhanced GPU approach effectively addresses the substantial data transfer overhead in offloading processes, providing a promising solution to improve LLM inference efficiency by leveraging the activation sparsity patterns inherent in LLMs.

\begin{figure*}
    \centering
    \includegraphics[width=0.8\linewidth]{Fig/overview.pdf}
    \vspace{-0.3cm}
    \caption{Overview of our proposed Hermes System. (a) Hermes augments GPU memory with NDP-DIMMs, and utilizes a scheduler to control the inference workflow. (b) Multiple NDP-DIMMs are connected to support LLM inference and inter-DIMM communication. }
    \label{fig:system-overview}
\vspace{-0.3cm}
\end{figure*}

\section{\name~System}\label{sec:hermes-system}
\subsection{System \& Architecture Overview}

\subsubsection{Architecture}

\fig\ref{fig:system-overview}a illustrates the overview of \name. \name~augments consumer-grade GPU with NDP-DIMMs to achieve low-budget, high-performance inference system for LLMs. 

% We elaborate on the architecture detail of each component as follows.

\textbf{Consumer-grade GPU:} For LLM inference, we only use one accessible and budget-friendly consumer-grade GPU. Despite limited graphic memory, it has ample computing units, like tensor cores, for high-performance parallel processing. It also features high-speed GDDR memory with superior bandwidth. For instance, the NVIDIA RTX 4090 with 24GB GDDR6 provides 82.6 TFLOPS, 1321 Tensor TOPS, and 936 GB/s bandwidth, making it suitable for LLM inference. Hermes uses a single GPU to efficiently execute hot neurons.
%[726] 不确定下面这句还要不要 : liu 暂时觉得可以不要,后面会主要说workflow,
% Given the abundant computational power of the GPU and the high-speed bandwidth but limited storage space of GDDR memory, during inference, we load only the hot neurons into GDDR and utilize the GPU's tensor cores to efficiently compute the hot neurons in parallel.

\textbf{NDP-DIMM:}
Given that cold neurons are randomly activated, all data stored on each DIMM should be accessible to its own NDP units. \update{Meanwhile, DIMM is required to support the normal data access from GPU for hot neuron transmission.} Therefore, as illustrated in \fig \ref{fig:system-overview}b, we have chosen the center buffer-based NDP-DIMM design\cite{alian2018application,cong2017aim,ke2020recnmp,kwon2019tensordimm}, which allows the processing unit to access all data in its own DIMM. 
\update{The center buffer-based NDP DIMM design also complies with the normal memory access as the newly added units will not influence the memory access function supported by the local memory controller \cite{alian2018application, ke2020recnmp}.}
Here, we detail the microarchitecture of our NDP-DIMM design. To facilitate typical operations in LLMs and potential inter-DIMM data moving, each NDP-DIMM is equipped with GEMV units, activation units, and DIMM-links~\cite{zhou2023dimm}.

\textit{GEMV Unit}: 
The GEMV unit reads data from the DRAM cell and the center buffer, performing the GEMV computation associated with cold neurons. 
\update{To support batched inference and fully utilize the bandwidth achievable within the DIMM center buffer, each GEMV unit contains 256 multipliers.}
Each multiplier is responsible for 128-bit multiplication \update{in a typical bit-serial manner~\cite{devaux2019true}}, a reduction tree-based accumulator, and a 256 KB buffer. During computation, each multiplier computes eight FP16 values simultaneously, and the accumulator is responsible for the addition of partial sums with data dependencies. The buffer stores the intermediate result generated by LLM layers. 

% \todo{add details on why GEMV design, partly done}
% Q: do we need to add some data to support our design, like the analysis of NDP-DIMM accessible data amount in one cycle and the batched inference requirement}

\textit{Activation Unit}: The activation unit is designed to support the necessary non-linear functions, such as softmax and ReLU operation for LLM inference. 
This unit is composed of 256 FP16 exponentiation units, 256 FP16 addition units, and 256 FP16 multiplication units, in addition to a comparator tree, an adder tree, and a divider. 

% The setup of the activation units is consistent with the throughput needs of GEMV units. Moreover, the activation unit repurposes the comparator tree to support both softmax and other activation functions.
% By incorporating an activation unit in the central buffer, we can reduce unnecessary data transfers. For example, softmax requires the `score' output, and having the activation unit and score computation unit both in DIMM allows direct data access from the same center buffer, avoiding additional data transfer to the GPU.

\textit{DIMM-link}: 
Due to the input-specific nature of the activated neurons, it is necessary to adjust the neuron mapping in multiple NDP-DIMMs to further ensure the load balance of computation in the DIMMs. Therefore, we adopt DIMM-link~\cite{zhou2023dimm} to achieve inter-DIMM communication with a bandwidth of 25 GB/s. Each DIMM-link employs bidirectional external data links between DIMMs, facilitating efficient point-to-point data transfers. The DIMM-link controller and bridge enable high-speed neuron redistribution between DIMMs.
\update{Compared to relying on the host for inter-DIMM data movement, using DIMM links provides over a 62$\times$ speedup for data transfer with negligible hardware overhead.
For example, when running OPT-66B, the introduction of DIMM-link effectively reduces the migration overhead for cold neurons from 5.3\% of total time to below 0.2\% .}
% For example, DIMM0 and DIMM1 store weights for N channels, but if no computations are needed for DIMM0 and x channels need computation in DIMM1, we can exchange x/2 channels' weights between DIMM0 and DIMM1 to balance the load, thereby reducing computation latency.

\textbf{Scheduler}:
During LLM inference, the scheduler in the host CPU redistributes neuron computation tasks to the GPU and NDP-DIMMs. The scheduler primarily comprises two components: a lightweight predictor and a neuron mapper, \update{which are both implemented by software}. In addition, the scheduler includes a monitor that gathers runtime information to assist the predictor and an instruction queue that triggers instructions for the GPU and NDP-DIMMs. With the help of the monitor, the lightweight predictor leverages token-wise similarity and layer-wise correlation patterns to accurately predict neuron activity. Based on the prediction results, the neuron mapper assigns hot and cold neurons to DIMMs and GPU memory, respectively, and it also dynamically adjusts the neurons' placement to ensure efficient inference on both the GPU and NDP-DIMMs. The subsequent sections will provide detailed descriptions of these two components. 

\textbf{\update{Programming Interface}}:
\update{We use a standard programming model, PIM-SYCL \cite{kim2023samsung}, to compile the heterogeneous platform. Unified memory programming \cite{zhao2024pim, nvidia_unified_memory} allows data to be transferred implicitly between heterogeneous memory devices, enabling cooperative processing on GPU and NDP-DIMMs. Additionally, \name~provides a set of extra NDP commands, such as MAC and softmax, to support various operators in LLMs. Taking GEMV computation as an example, the NDP-DIMM computations can be invoked through the memory command interface by sending a series of MAC commands. On the GPU side, the corresponding computations are triggered through APIs like cudaLaunchKernel. }
 % \update{For instance, the instruction queue issues the GEMV operator on GPU using APIs like cudaLaunchKernel. Conversely, an additional GEMV instruction, which contains the MAC command with the data address, is defined to support the GEMV operator on the NDP-DIMM side. The instruction queue can invoke the corresponding computation through the memory command interface~\cite{cong2017aim, park2024attacc, heo2024neupims}.}


% \update{Once the scheduler determines the computation tasks assigned to the GPU and NDP-DIMM, it invokes the corresponding instructions for each device. For instance, in the case of a GEMV computation, the scheduler will issue a CUDA instruction when the GPU is required. Conversely, when leveraging the NDP-DIMM, it sends the NDP computation instruction, which contains the MAC command with the data address, through the memory command interface.}
% 感觉上面这个programming interface感觉确实可要可不要?如果说的话,还得说我们在DIMM里面的controller里添加了对于计算指令解析的通路。

% 流水线,两段,上面总结,下面详细说MLP(参考ATTACC
\begin{figure}[t]
    \centering
    \includegraphics[width=\linewidth]{Fig/workflow.pdf}
    \vspace{-0.3cm}
    \caption{Workflow of Hermes system (a) The whole workflow of LLMs inference on the \name~system. (b) Illustration of computation process for FC layers with activation sparsity. The block with a number in the Mem. means one neuron's weight.}
    \label{fig:workflow}
\vspace{-0.3cm}
\end{figure}

\subsubsection{Workflow}
% During this stage, the KV matrices are transferred to the NDP-DIMMs as soon as they are created.

The workflow for LLM inference within the \name~system is depicted in \fig \ref{fig:workflow}a. \update{Given the significant computational demands, the entire prompting stage is processed on the GPU, adhering to a traditional offloading strategy~\cite{sheng2023flexgen}.} During this stage, the host scheduler records neuron activity for future scheduling optimization. Upon completing the prompting stage, only the selected hot neurons are loaded back into GPU memory. The offline partition of hot and cold neurons will be further detailed in Section \ref{sec:offline}.
In the token generation stage, for each transformer layer, the QKV generation is collaboratively completed by GPU and NDP-DIMMs. The output of QKV generation will be collected in the NDP-DIMMs for further attention computation. 
The memory bandwidth-intensive nature of attention computation~\cite{yu2022orca, park2024attacc} makes it ideal for execution on NDP-DIMMs, which benefit from the abundant internal bandwidth. Additionally, transferring attention computation to NDP-DIMMs helps save the limited GPU memory by eliminating the need for storing KV cache.
Since the projection layer cannot utilize the activation sparsity, it is handled solely by the computation-efficient GPU. During the projection computation, as the DIMMs are entirely idle, the host takes advantage of this period to dynamically reconfigure the hot/cold partitions and redistribute neurons across DIMMs based on the prediction results, which will be detailed in the \sec \ref{sec:partition-design} and \ref{sec:cold-neuron-mapping}. Then, similar to the QKV generation, MLP is offloaded to both GPU and NDP-DIMMs. Finally, the output of each transformer layer is reduced in the NDP-DIMMs. 

% Once the result of attention is obtained, the projection step requires substantial computational resources, since it lacks the feature of sparsity of activation. To avoid extra synchronization overhead from concurrent execution on both GPU and NDP-DIMM, the projection computation is handled solely by the GPU.


% The Par denotes the predetermined or offline profiled parameters. The Var denotes the variables that need to be solved.

\fig \ref{fig:workflow}b illustrates the computation process for FC layers (for both QKV generation and MLP block) with activation sparsity. Specifically, it includes three steps.
After completing the related prediction, the host CPU determines the computation allocation for both the GPU and NDP-DIMMs based on the location of the activated neurons.
\update{Once the neuron mapping is determined, the host CPU invokes APIs for both the GPU and NDP-DIMMs to load data and perform computation. For example, the host CPU uses ``cudaLaunchKernel" to launch GPU kernels for GEMM and GEMV operations. To ensure correctness, the host CPU inserts barriers for the GPU and NDP-DIMMs to synchronize their computations. Once the DIMMs and GPU complete their computations, a merge kernel is invoked on the NDP-DIMMs side to gather the results from both sources.} This method is advantageous for two reasons. Firstly, as the GPU generally finishes computation tasks more quickly owing to its superior computation capability, the latency in transferring data from the GPU to DIMMs can be hidden by the DIMMs' computation, thus not penalizing the overall system runtime. Secondly, with the attention computation occurring on NDP-DIMMs, merging the QKV generation outcomes on the NDP-DIMMs side minimizes the additional data transfer overhead.
% In the next sections, the important components of Hermes will be introduced in detail.

\begin{table}
\vspace{-0.3cm}
\caption{Terminology for the offline partition solver. } 
\label{tab:terminology}
\scriptsize{
\centering
\begin{tabular}{c|p{7.2cm}}
\hline
% \textbf{Symbol} & \textbf{Description} \\[1ex]
% \hline
% \Xhline{3\arrayrulewidth}
\rowcolor{black!10}
\multicolumn{2}{c}{\textbf{\textit{Parameters - predetermined or offline profiled}} }\\
\hline
$\mathbb{L}$ & All layers \\
$\mathbb{N}$ & All neurons \\
$\mathbb{D}$ & All NDP-DIMMs \\ 
$f_{i}$ & Activation frequency of neuron $i$ \\
$N_{l}$ & Neuron in layer $l$ \\
$M_{i}$ & The memory space required by neuron $i$ \\
$T_{sync}$ & The time required for one synchronization \\
$T_{l}^{j}$ & The time for computing one neuron in layer $l$ on processing $j$ \\
$S_{j}$ & The storage size for processing unit $j$ \\
\hline
\rowcolor{black!10}
\multicolumn{2}{c}{\textbf{\textit{Binary Variables - needed to be solved}} } \\
\hline
\multirow{2}{*}{$x_{il}^{j}$} & Whether neuron $i$ in layer $l$ is placed on processing unit $j$ \\
& $x_{il}^{j}=1$ means the neuron $i$ in layer $l$ is placed on processing unit $j$\\
\hline 
\end{tabular}
}
\vspace{-0.3cm}
\end{table}

\subsection{Offline Neuron Mapping} \label{sec:offline}
% Since both NDP-DIMMs and GPU cooperate to process the same operator, 
Since NDP-DIMMs and GPU are responsible for the computational load of the neurons stored in them, the predetermined mapping for each neuron's location greatly influences the inference efficiency. However, due to the huge neuron mapping space (e.g., more than $2^{1000}$ for LLaMA-7B), solely relying on online mapping solutions is impractical and will contribute to considerable performance degradation. Therefore, in the belief that ``hot" and ``cold" neurons are partly attributed to the pretrained LLM's nature~\cite{song2023powerinfer, song2024prosparse, zheng2024learn, song2024turbo}, we utilize the offline profiled information to deduce the initial offline neuron mapping. It alleviates the adjustment cost of subsequent online partition and scheduling during inference. Please note that the optimal initial mapping denotes the mapping that can be found during the offline stage, which will be adjusted during runtime.

% To fully unleash the computational capabilities of both NDP-DIMMs and GPU, we need to effectively partition the hot and cold neurons. Therefore, we first propose an offline neuron partition scheme to optimize data layout. 

% \subsubsection{Offline Parameter Placement}


% 感觉这里要不要加一句说,我们的将该问题抽象为ILP的最终目标是求解是系统时延最小的方案,然后再说求解的过程考虑了XXX,最后说具体的建模如下
To determine the optimal location for each neuron \update{that minimizes the inference latency using our heterogeneous system}, we formalize the mapping issue as an integer linear programming problem (ILP). In particular, we analyze several factors, including each neuron's activated frequency, computational overhead, memory usage, and synchronization delays, to model the inference performance of the \name~system. \update{To gather these factors accurately, we test the model on popular datasets such as C4~\cite{raffel2020exploring} and Pile~\cite{gao2020pile} with 128 samples}, and also employ an execution monitor in the host CPU to record during inference. The notation for solving the optimal offline neuron placement problem is summarized in Table \ref{tab:terminology}.

\textbf{Objective function.}
The objective of the optimal neuron mapping is to minimize the total inference latency, as shown in Equation \ref{eq1}. Since the execution of each layer involves both GPU and NDP-DIMMs, the total execution time is determined by the longer duration of the GPU and NDP-DIMMs execution times. For NDP-DIMMs, the single-layer execution time is the longest execution time among all DIMM modules, as shown in Equation \ref{eq2}. For the GPU, the single-layer execution time includes both computation time and extra synchronization overhead, while the synchronization overhead includes that of fetching input activation data from the DIMM and sending the computation results back to the DIMM to trigger the merge kernel. Hence, as illustrated in Equation \ref{eq3}, the total GPU execution time also includes twice the single-direction synchronization overhead. 

% 整体的求解目标 for layer l 
{
\setlength\abovedisplayskip{0pt}
\setlength\belowdisplayskip{0pt}
\begin{equation}
    \min \textstyle \sum_{l} max (T_{GPU-l}, T_{DIMM-l}), \quad \forall l \in \mathbb{L} \label{eq1}
\end{equation}
\begin{equation}
     T_{DIMM-l} = max (T_{dimm-jl}), \quad \forall j \in \mathbb{D} \label{eq2}
\end{equation}
\begin{equation}
    T_{GPU-l} = T_{compute-GPU-l} + 2\cdot T_{sync} \label{eq3}
\end{equation}
}


The computation times for a single layer on both the GPU and NDP-DIMMs depend on the number of activated neurons located in each device. Let $T_{l}^{GPU}$ represent the time required to compute a single neuron on the GPU. Consequently, the computation time for a single layer on the GPU is the product of the number of activated neurons in the GPU memory and the time taken to compute each neuron, as illustrated in Equation \ref{eq4}. Similarly, the single-layer computation time for each NDP-DIMM is demonstrated in Equation \ref{eq5}.% Similarly, the NDP-DIMM's single-layer computation time is the product of the activation counts of all neurons stored in a single DIMM module and the computation time per neuron, as shown in \eq \ref{eq5}.


% GPU & NDP-DIMM 计算时间
{
\setlength\abovedisplayskip{0pt}
\setlength\belowdisplayskip{5pt}
\begin{align}
    T_{compute-GPU-l} = T_{l}^{GPU} \cdot \textstyle \sum_{i} f_{i}\cdot x_{il}^{GPU}, \quad \forall i \in \mathbb{N} \label{eq4}\\ 
    T_{dimm-jl} = T_{l}^{DIMM} \cdot \textstyle \sum_{i} f_{i} \cdot x_{il}^{dimm-j}, \quad \forall i \in \mathbb{N} \label{eq5}
\end{align}
}


\textbf{Constraints.}
The offline optimal neuron placement issue must adhere to the conditions listed in \eq \ref{eq6} and \ref{eq7}, which limit the memory space occupied by neurons not to exceed the available memory size of each DIMM and GPU.

% GPU & DIMM 存储约束
{
\setlength\abovedisplayskip{0pt}
\setlength\belowdisplayskip{5pt}
\begin{align}
   \textstyle \sum_{l} M_{i} \cdot x_{il}^{GPU} \le S_{GPU}, \quad \forall l \in \mathbb{L} \label{eq6}\\ 
   \textstyle \sum_{l} M_{i} \cdot x_{il}^{dimm-j} \le S_{dimm-j}, \quad \forall l \in \mathbb{L} \label{eq7}
\end{align}
}

Consequently, we employ the open-sourced optimization solver, PulP~\cite{pulp-solver}, to determine the optimal offline neuron mapping.
% Given that ILP problems are inherently NP-complete, directly solving them for an LLM with hundreds of billions of parameters presents a considerable computational challenge. 
Based on our assessment, it takes about 110 seconds to solve for the optimal neuron mapping, making it appropriate for a single offline compilation process. Before LLM inference, we initially transfer relevant hot neurons to GPU memory based on the mapping outcomes and further adjust the mapping during runtime to improve efficiency.
% However, the activation sparsity causes the neuron activation distribution to be non-static, 


% 准备把 C 和 D 两个小节的逻辑都换一下,重点应该是 hot/cold partition 以及 partition 之后的对应操作导致的影响,predictor 对 inference 的影响要弱化,不要去过多强调这个,是一个附加的。后续重点去考虑那部分,而非当前的predictor 的内容带来的实质性影响。 然后 C 的话,重点就是 hot/cold neurons 的预测,跟 D 的关系就没有那么大
\subsection{Online Adjustment for Hot/Cold Neuron Partition}\label{sec:partition-design}  

Although the optimal offline neuron mapping provides an effective hot/cold partition, the input-specific nature of activation sparsity makes the hot/cold neuron partition change dynamically in practice. Our evaluation indicates that about 52\% of the initialized hot neurons exhibit varied activity during inference. Therefore, it is necessary to adjust the hot/cold neuron partition online to improve inference efficiency before neuron computation, which requires an in-advance prediction of the neuron partition. In this section, we leverage the distribution patterns of activation sparsity to create a novel lightweight predictor to guide the online adjustment of the hot/cold neuron partition.
\begin{figure}[t]
    \centering
    \includegraphics[width=\linewidth]{Fig/predictor.pdf}
    \vspace{-0.3cm}
    \caption{The predictor design in Hermes. (a) We are motivated to utilize the temporal locality of token generation for prediction. (b) The layer-wise correlation effectively predicts activated neurons.}
    \label{fig:predictor}
\vspace{-0.3cm}
\end{figure}

\subsubsection{Predictor Design}\label{sec:predictor-design}
% 一方面,workflow中需要提前决定the computation loads for both GPU and NDP-DIMMs to utilize the activation sparsity; 另一方面,提前将 hot neurons 映射到 GPU 上有助于充分发挥 GPU 的算力,缓解NDP-DIMMs的负载。
Accurately forecasting activated neurons and the hot/cold neuron partition is crucial for improving inference performance. On one hand, to effectively harness activation sparsity, the \name~workflow necessitates predetermining the computation loads for both the GPU and NDP-DIMMs. On the other hand, assigning hot neurons to the GPU before computation can fully utilize the GPU’s computation capability and ease the burden on NDP-DIMMs. Nevertheless, existing MLP-based predictors~\cite{song2023powerinfer, song2024prosparse, liu2023deja} incur considerable storage and computation overhead, reducing inference efficiency. To address it, we introduce a lightweight predictor that exploits token-wise similarity and layer-wise correlation (discussed in Section \ref{sec:similar}) for accurate predictions.

\textbf{Token-wise Prediction. } The token-wise similarity suggests that the distribution of activated neurons is similar among adjacent tokens. Given that tokens are generated one by one during the token generation stage, token-wise similarity can be considered as a temporal locality of activated neurons. Inspired by well-known branch prediction strategies~\cite{smith1981study, yeh1991two, mcfarling1993combining} that also benefit from temporal locality, we propose a novel prediction strategy. As shown in \fig \ref{fig:predictor}a, we establish a neuron state table where each neuron has a 4-bit state, ranging from 0 to 15, used to predict whether the neuron will be activated. After the prefill stage, we initialize each neuron's state based on the activated frequency in the whole prefill stage. Specifically, we divide the distribution of the activated frequency into 16 stages and initialize each state accordingly. For example, if a neuron's activated frequency exceeds 90\%, its state is initialized as `15', whereas if the ratio is below 2\%, the state is set as `0'.

We update each neuron's state based on the actual activated neurons during each token generation step using a finite state machine. If a neuron is not activated, its state decreases by 1; if it is activated, its state increases by \( s \), which is set to 4 in this paper. The left part of \fig \ref{fig:predictor}a shows that, when neuron 6 is activated, the state is updated from $7$ to $11$, while the state of neuron 5 is updated from 10 to 9 as it is not activated.  

\begin{figure}
    \centering
    \includegraphics[width=0.9\linewidth]{Fig/mapper.pdf}
    \vspace{-0.3cm}
    \caption{Neuron mapper design. (a) The mapper utilizes the information in the neuron state table to adjust the hot/cold neuron partition. (b) Cold neurons are remapped based on the neuron activity within a window.}
    \label{fig:mapper}
\vspace{-0.3cm}
\end{figure}

\textbf{Layer-wise Prediction. } 
Token-wise similarity alone cannot address fluctuations in neuron activity between tokens~\cite{zhang2024relu, zheng2024learn}. Therefore, we further employ layer-wise correlation to improve prediction accuracy. Insights from Section \ref{sec:similar} suggest that if neurons with high correlation in the preceding layer are activated, the activated probability for the current neuron is significantly increased. Consequently, we create a neuron correlation table to boost layer-wise prediction. As depicted in Figure \ref{fig:predictor}b, we initially offline sampled the top 2 correlated neurons from the previous layer and documented their relationships in the neuron correlation table.
% During token generation, to determine whether it needs activation, we record the activation number of its top 2 correlated neurons. If the number exceeds a threshold \( T2 \), the neuron is predicted to be activated; otherwise, it is not.

% if a neuron has a state $s_1$ from the neuron state table and a number of $s_2$ highly-correlated neurons from the neuron correlation table are activated, its activation prediction follows: $s_1 + \lambda\cdot s_2> T $.

Finally, we combine the token-wise and layer-wise prediction strategies to achieve accurate prediction for activated neurons during token generation. Specifically, we use $s_1$ to denote the state in the neuron state table for one neuron, and use $s_2$ to indicate the activated number of the highly correlated neurons for one neuron. To predict the activation state for such a neuron, we examine the inequation: $s_1 + \lambda\cdot s_2> T $. In this paper, we set $\lambda$ as 6, and the threshold $T$ as 15. As Figure \ref{fig:predictor} shows, following the prediction criterion, we finally activate neurons 3, 6 and 9 for subsequent computation. During context switches, token similarity may vanish, but layer-wise correlation is still available for effective prediction. Conversely, even if correlated neurons are not activated, observing neighboring tokens' activation states still helps achieve accurate prediction. Experimental result shows that the accuracy of our proposed predictor achieves 98\% using less than 1MB of memory. \update{For instance, LLaMA-7B occupies 32 layers, with each one having 4K neurons for the self-attention block and 10.5K for the MLP block. In our implementation, only 4-bit data is used to record the corresponding state of each neuron. Consequently, it only costs 232 KB for the neuron state table of LLaMA-7B.} We integrate the proposed predictor into the host CPU and store the table values in the last level cache for fast prediction.


\subsubsection{Online Adjustment guided by Predictor}
Given their ample memory capacity, instead of mapping only cold neurons, we store all the weight parameters on DIMMs. Thus, we only need to reload the actual hot neurons onto GPU memory to achieve online adjustment. The neuron state in our proposed predictor effectively represents the activity of each neuron. Specifically, as shown in the \fig \ref{fig:mapper}a, once the neuron state exceeds a certain threshold $T_h$, it can be viewed as the hot neuron. In this paper, we set the threshold $T_h = 10$. Accordingly, neurons 3, 6, and 9 are identified as hot neurons. We then use the neuron mapper to locate the corresponding hot neuron. As the hot neuron 6 is originally located on the DIMMs, an instruction is issued to copy the corresponding hot neuron to the GPU memory during the projection computation. Meanwhile, the neuron with the lowest state value (neuron 5) stored in GPU memory will be swapped out. Note that, since all neurons are stored in DIMMs, we only need to overwrite the location of the neuron to be swapped out in the GPU memory to achieve neuron swapping. In general, online neuron adjustment between GPU and NDP-DIMMs significantly improves the inference efficiency without inducing additional data transfer overhead.


\subsection{Online Remapping for Cold Neurons}\label{sec:cold-neuron-mapping}
Due to our implementation of a center buffer-based NDP-DIMM architecture, the total computation delay correlates with the count of activated neurons in each DIMM module. As shown in Equation \ref{eq2}, the total execution duration is constrained by the slowest-performing NDP-DIMM module. Hence, determining the optimal cold neuron assignment to ensure a balanced load across multiple NDP-DIMMs is crucial.
Despite using DIMM-link for inter-DIMM communication, the limited bandwidth (25GB/s) cannot afford over-frequent data exchanges between DIMMs. Therefore, we need to achieve a load balance across multiple NDP-DIMMs while minimizing the remapping of cold neurons.

\begin{algorithm}[t]
\scriptsize
    \caption{Window-based online scheduling}\label{alg:balance}

\SetKwInOut{Input}{Input}\SetKwInOut{Output}{Output}

\Input{neuron mapping $C_{j,i}$; Activity for neuron $i$ within a window $A_{i}$; Number of NDP-DIMM modules $J$;}

\emph{{\textcolor{magenta}{// Compute the number of activated neurons for NDP-DIMM $i$.}}}
$Z_{j} = \sum_{i} C_{j,i} \cdot A_{i}$ \\ 
Sort $Z$ with the descending order \\

\For{{\textcolor{blue}{int}} id = 0; $id$ $<$ J/2; id++}{
  \While{$Z_{id} \le Z_{J-id}$}{
    Find the most activated neurons $h$ in NDP-DIMM $id$\\ 
    \emph{{\textcolor{magenta}{// Remapping the most activated neurons from $id$ to $J-id$}}}
    $C_{id, h} = 0$; $C_{J-id, h} = 1$
  }
}
\end{algorithm}

The similarity between tokens inspires us to develop a novel window-based online scheduling method for remapping cold neurons. In particular, we group every five consecutive tokens into a window. Based on our observations, due to the token-wise similarity, once the optimal mapping for cold neurons is identified, the runtime variance among different NDP-DIMMs within a window is under 5\%, indicating a balanced assignment. Nevertheless, when surpassing the window size, the performance disparity among different NDP-DIMMs varies from 1.2$\times$ to 2.5$\times$. Consequently, we can leverage the neuron activity within a window to guide the remapping of cold neurons. As shown in Algorithm \ref{alg:balance}, we initially gather the activated times for each neuron $i$ within a window and calculate the total activated neurons in NDP-DIMM $j$ based on the current neuron mapping $C_{j,i}$. $C_{j,i}$ is a binary matrix that denotes if neuron $i$ is mapped on NDP-DIMM $j$. We then sort the total activated neurons for NDP-DIMMs within the window and adjust neuron mappings between DIMM pairs accordingly. Specifically, the NDP-DIMM with the largest number of activated neurons is paired with the one that has the fewest activated neurons. Finally, the most activated neurons in the NDP-DIMM pair are remapped to achieve balance. As depicted in \fig \ref{fig:mapper}b, we record the activated neurons within a window into the neuron activity table, and calculate the activity for each NDP-DIMM based on the mapping results. As the count of activated neurons in DIMM-1 exceeds that of DIMM-2, neuron 5 from DIMM-1 is remapped to DIMM-2 for load balance between the two NDP-DIMMs. This strategy offers two advantages: first, the fixed inter-DIMM communication traffic is directed to different bridges to prevent congestion; second, the greedy remapping approach can quickly achieve balance with minimal data transfer.

\section{Evaluation}

\begin{table}
    \centering
    \caption{Configuration details of NDP-DIMM.}
    \vspace{-0.3cm}
    \resizebox{\linewidth}{!}{
    \begin{tabular}{c|c|c}
    \hline
    \multicolumn{3}{c}{\textbf{NDP core}} \\
    \hline
    \multicolumn{3}{c}{Configuration: 256 multipliers, reduction tree-based accumulator, Buffer size: 256KB}\\ 
    \hline
    One NDP core per DIMM & Frequency: @ 1 GHz  & area overhead: $1.23mm^2$ per core\\
    \hline
    \multicolumn{3}{c}{\textbf{DIMM Parameters}} \\
    \hline
     \multicolumn{3}{c}{DDR4-3200, 32GB/DIMM$\times$8, \update{2 DIMMs/channel}}\\
     \multicolumn{3}{c}{4 rank/DIMM, 2 bank groups/rank, 4 bank/BG}\\
     \hline
    \multicolumn{3}{c}{\textbf{DIMM Timing}} \\
    \hline
     \multicolumn{3}{c}{tRC=76, tRCD=24, tCL=24, tRP=24, tBL=4}\\
     \multicolumn{3}{c}{tCCD S=4, tCCD L=8,tRRD S=4, tRRD L=6, tFAW=26}\\
    \hline
    \multicolumn{3}{c}{\textbf{DIMM-Link Parameters}} \\
     \hline
    \multicolumn{3}{c}{25Gb/s/Lane, 1.17 pJ/b, 8 $\times$ Lanes (25GB/s per Link)} \\
    \hline
    \end{tabular}
    }
    \label{tab:dimmcfg}
\vspace{-0.3cm}
\end{table}

\subsection{Experimental Setup}\label{sec:experimental-setup}

\subsubsection{\name~System}
The proposed \name~system integrates a single NVIDIA RTX 4090 GPU with 24GB of graphic memory \update{and 330 tensor TOPS (FP16)} to process hot neurons. Additionally, we provide 8 NDP-DIMMs, each including 32GB DDR4 memory as the extension of GPU memory. We use PCIe 4.0 to support data interaction between NDP-DIMMs and GPU memory with a bandwidth of 64GB/s. The kernel performance of the NVIDIA RTX 4090 is measured using NVIDIA Nsight Compute~\cite{nsight}. Furthermore, we develop an in-house simulator by modifying Ramulator 2.0~\cite{luo2023ramulator, ramulator2.0} to evaluate the performance efficiency of NDP-DIMM devices. For the NDP core, we implemented it in RTL and synthesized it using the Synopsys Design Compiler~\cite{synopsys.org} with the TSMC 7nm technology. \tab \ref{tab:dimmcfg} shows the configuration details of adopted NDP-DIMMs.

\subsubsection{Baseline Systems}
We selected several offloading-based inference systems, such as Huggingface Accelerate~\cite{jain2022hugging, huggingface-accelerate}, FlexGen~\cite{sheng2023flexgen}, and Deja Vu~\cite{liu2023deja}, as the baselines. FlexGen and Deja Vu are restricted to OPT models. Moreover, Deja Vu, initially optimized for LLM activation sparsity within high-performance distributed systems, has been adapted to support offloading-based serving systems. In contrast to \name, these methods depend solely on the basic host memory to expand capacity without offering additional computational resources. \update{We also provided a system (Hermes-host) that offloads cold neurons to the host CPU while handling hot neurons on GPU, demonstrating the necessity of NDP-DIMMs. Hermes-host follows the configuration in~\cite{song2023powerinfer}, which equips an Intel i9-13900K processor as the host CPU (providing a maximum bandwidth of 89.6 GB/s), and also uses a single NVIDIA RTX 4090 as the GPU for hot neurons.} Additionally, to highlight the significance of activation sparsity in boosting \name~system efficiency, we also compare \name~against a straightforward NDP-DIMM extended system (referred to as Hermes-base) that does not leverage activation sparsity in LLMs.

\begin{figure}
    \centering
    \includegraphics[width=.98\linewidth]{Fig/end1_rebuttal.pdf}
    \vspace{-0.3cm}
    \caption{\update{Performance comparison with existing offloading-based systems.}}
    \label{fig:offloading-performance}
\vspace{-0.3cm}
\end{figure}

\begin{figure}[t]
    \centering
    \includegraphics[width=0.98\linewidth]{Fig/end2_rebuttal.pdf}
    \vspace{-0.3cm}
    \caption{\update{The effectiveness of activation sparsity and NDP design on \name.}}
    \label{fig:base-hermes-performance}
\vspace{-0.3cm}
\end{figure}

\subsubsection{Workloads}
We chose OPT-13B, OPT-30B, OPT-66B~\cite{zhang2022opt}, LLaMA2-13B, LLaMA2-70B~\cite{touvron2023llama2}, and Falcon-40B~\cite{almazrouei2023falcon} as target models. For the OPT series models, we utilized their native ReLU activations to achieve activation sparsity. For the LLaMA2 and Falcon models, we use the open-source models\footnote{The modified LLMs can be found at \href{https://huggingface.co/SparseLLM}{https://huggingface.co/SparseLLM}, including both LLaMA2 and Falcon models} that substituted their original activation functions with ReLU~\cite{mirzadeh2023relu, zhang2024relu}. Furthermore, we added additional ReLU functions before generating QKV to achieve activation sparsity in self-attention blocks. Evaluation results show that these alterations result in negligible accuracy loss (under 1\%). \update{Furthermore, we adopt ChatGPT prompts~\cite{gpt-prompts} and Alpaca~\cite{alpaca} as the datasets to evaluate the end-to-end performance, following configurations in \cite{xue2024powerinfer, song2023powerinfer}.}
% \todo{The dataset used for evaluation with description of the distribution of the prompt and generated output lengths}

\begin{figure*}
    \centering
    \includegraphics[width=\linewidth]{Fig/batching_rebuttal.pdf}
    \vspace{-0.3cm}
    \caption{\update{End-to-end performance on different batch sizes (ranging from 1 to 16). N.P. denotes the model is not supported by the current inference system.}}
    \label{fig:batching-inference}
\vspace{-0.3cm}
\end{figure*}

\subsubsection{Evaluation Metric}
Given our focus on local deployment scenarios, we primarily optimized LLM inference with small batch sizes. We concentrated on the average number of tokens generated per second (tokens/s) to evaluate model inference efficiency. Hereafter, the number above each bar in each figure indicates the end-to-end generation speed (tokens/s). In our experiments, we used batch sizes between 1 and 16, and kept the lengths of both input and output sequences fixed at 128.


% Moreover, 

% Moreover, we concentrated on token-to-token latency (ms/token) and the average number of tokens generated per second (tokens/s) to evaluate model inference efficiency.

\subsection{\name~Performance}\label{sec:end-to-end}

\subsubsection{End-to-End Performance}
We begin by evaluating the end-to-end inference performance of \name~and baseline systems at a batch size of 1, which is commonly used for local deployments~\cite{cai2023medusa}. Noting that FlexGen and Deja Vu are limited to support OPT family models, we first compare \name~against existing offload-based inference systems on OPT models. \update{Additionally, we evaluate the Hermes-host and Hermes-base systems' performance across various LLMs to illustrate the necessity of NDP-DIMMs design and activation sparsity in Hermes, respectively.}
% 这里需要包含两个部分,分别是 end-to-end performance 和 token-to-token latency

\textbf{Comparison with Offloading-based Systems. } \fig \ref{fig:offloading-performance} presents the end-to-end performances on OPT family models. Compared with the Accelerate and FlexGen systems, \name~can achieve an average $578.42 \times$ and $247.25 \times$ speedup, respectively. \name~is capable of achieving a rate of $20.37$ tokens/s for OPT-66B, which substantially surpasses current inference systems. In contrast, Deja Vu only attains an average speedup of $2.12 \times$ over FlexGen due to the necessity of loading cold neurons. The frequent data transfer on PCIe compromises the performance improvement of activation sparsity, while the expensive MLP-based predictor used in Deja Vu further diminishes its benefits. Compared to OPT-13B, \name~achieves greater performance gains on OPT-66B. This is because 80\% of the parameters in OPT-13B can be stored in GPU memory, whereas only 15\% of parameters in OPT-66B can be stored in GPU memory. This further exacerbates the data transfer overhead between host memory and GPU memory. 
% Fortunately, \name~can effectively utilize NDP-DIMMs to process the offloaded parameters without introducing significant data movement.

\textbf{Necessity of Activation Sparsity. } We further compare \name~with the Hermes-base system, which only adopts a na\"ive NDP-DIMM extended system without utilizing activation sparsity, as shown in \fig \ref{fig:base-hermes-performance}. 
The Hermes-based system processes the FC layers on the GPU when their parameters are available, switches to NDP-DIMMs when their parameters are stored in those modules, and offloads all attention computations to NDP-DIMMs.
% The Hermes-based system activates nearby processing units based on the location of parameters to compute FC operators and offloads all attention computations to NDP-DIMMs. 
This approach leverages the high internal bandwidth of NDP-DIMMs and
reduces data transfer between DIMMs and GPU memory. In comparison to Huggingface Accelerate, the Hermes-base system can achieve $53.89 \times$ speedup on average, as it greatly reduces the data transfer on PCIe. By effectively leveraging activation sparsity in LLMs, Hermes outperforms the Hermes-base system with average speedups of $5.17 \times$, specifically for large models such as Falcon-40B and
LLaMA2-70B. This is due to when running large models, most layers are offloaded on the computation-limited NDP-DIMMs for the Hermes-base system. 

\update{\textbf{NDP-DIMMs instead of host CPU. } 
Experimental results in Figure \ref{fig:offloading-performance}, \ref{fig:base-hermes-performance} demonstrate the necessity of NDP-DIMMs. Hermes achieves $4.79\times$ - $7.75\times$ speedup when compared to Hermes-host. Specifically, the Hermes-host system also utilizes the hot/cold neuron partition, but computes the cold neurons on the host CPU. This approach effectively alleviates the burdensome data loading on PCIe for existing offloading-based systems. In comparison to Huggingface Accelerate and FlexGen, the Hermes-host system can achieve $62.00 \times$ and $44.96\times$ speedup on average, respectively. However, the memory bandwidth on the CPU side is significantly lower than that of NDP-DIMMs, making the Hermes-host system still far less efficient than our proposed Hermes system. }



\subsubsection{Batching Inference}

\begin{figure*}
    \centering
    \includegraphics[width=\linewidth]{Fig/breakdown.pdf}
    \vspace{-0.3cm}
    \caption{Evaluating the performance breakdown on Deja Vu, \name, and \name-base (H-base) on various LLMs with different batch sizes. }
    \label{fig:performance-breakdown}
\vspace{-0.3cm}
\end{figure*}

We also evaluate the end-to-end performance of \name~with different batch sizes. As shown in the \fig \ref{fig:batching-inference}, \name~demonstrates consistent performance improvement with the batch sizes varying from 1 to 16. Hermes attains average speedups of $148.98\times$ and $75.24\times$ for various batch sizes when compared to FlexGen and Deja Vu, respectively, offering promising support for larger batch sizes. \update{Furthermore, \name~achieves an average $7.17 \times$ speedup over \name-host for various batch sizes. As the batch size increases, the performance gap between Hermes-host and Hermes becomes more pronounced. This occurs as the consumer-grade GPU with sufficient computation capability is minimally impacted by larger batch sizes, whereas the dynamic loading overhead of cold neurons is closely tied to bandwidth. Consequently, as batch sizes grow, the limited memory bandwidth on the CPU side increasingly affects overall system performance.} The performance gap between \name~and the Hermes-base system is the smallest when the batch size is 2. This is because for \name-base, the computation capability of the NDP core can still effectively handle the corresponding computational load, and larger batches can effectively amortize the DRAM cell access overhead as weight parameters are reused by the two batches. At other batch sizes, \name~demonstrates a significant performance advantage over Hermes-base. First, at a batch size of 1, Hermes can utilize activation sparsity to significantly reduce the number of neurons that need to be activated, thereby lowering data access overhead. Second, as the batch size increases, Hermes is not constrained by the computation capability of NDP-DIMMs due to the presence of activation sparsity. 

\subsection{Ablation Studies}\label{sec:ablation-study}
% 要包括这样几种: offline modeling and mapping; online placement; load balance optimization 


\begin{figure}
    \centering
    \includegraphics[width=\linewidth]{Fig/ab_rebuttal.pdf}
    \vspace{-0.3cm}
    \caption{\update{Ablation study on proposed offline and online scheduling strategies.}}
    \label{fig:ablation-study}
\vspace{-0.3cm}
\end{figure}



To evaluate the scheduling strategies proposed in Section \ref{sec:hermes-system}, we compare the normalized inference latency on MLP block for different LLMs with various scheduling settings. Specifically, \name-random denotes utilizing a random offline mapper to achieve neuron placement, \name-partition denotes that it only considers the optimal offline neuron placement, \name-adjustment denotes the system that further uses online adjustment for hot/cold neuron partition, and \name~is the one that integrates all the scheduling strategies proposed in Section \ref{sec:hermes-system}. \update{Furthermore, we also explore when only adopting token-wise prediction or layer-wise prediction to guide the online adjustment of hot/cold partition, denoted as Hermes-token-adjustment and Hermes-layer-adjustment, respectively.} 

\textbf{Load Balancing with Multi-level Optimization. } \fig \ref{fig:ablation-study} shows the contributions of each component in \name~. Utilizing the offline mapper can effectively identify the frequent hot neurons, reducing the computation cost of NDP-DIMMs. As a result, \name-partition can achieve $1.63 \times$ speedup than \name-random. However, the input-specific nature of activation sparsity challenges the offline partition approach. Therefore, further adopting online adjustment for hot/cold partition (\name-adjustment) achieves $1.33 \times$ performance gains over \name-partition. Despite this, the overall execution efficiency is still constrained by the NDP-DIMMs, which possess limited computation capability. Thus, the performance of the resource-constrained NDP-DIMMs can be improved by tackling the load imbalance issues in several NDP-DIMMs. The introduced online remapping method successfully addresses this problem. As a consequence,
the fully optimized Hermes system demonstrates a $1.29 \times$ boost in performance when compared with \name-adjustment.
% 这里就是分析每一部分的优势

\update{\textbf{Benefits of Token-wise and Layer-wise Prediction.}
Compared to \name-partition which only considers the optimal offline neuron placement, \name-token-adjustment and \name-layer-adjustment can achieve $1.08\times$ and $1.11\times$ speedup, respectively, demonstrating the benefits of online adjustment. However, token-wise prediction cannot address fluctuations in neuron activity, making it inaccurate for frequent changes in hot/cold neurons. Simultaneously, layer-wise prediction only relies on the static sampled neuron correlation table to guide the online adjustment, inefficient for constant changes of online adjustment. As a result, using token-wise or layer-wise prediction only cannot effectively unleash the benefits of prediction-based online adjustment.  
}

\subsection{Performance Breakdown}\label{sec:breakdown}

% 这里要分析几个点,首先是 deja vu 中的load weight 的影响,communication 的开销,以及predictor 的开销;然后是 prefill 的开销占比;然后是计算的开销

\fig \ref{fig:performance-breakdown} illustrates the performance breakdown of Deja Vu, \name-base, and \name~on various LLMs. It provides detailed insights into the efficiency sources of \name.

Figure \ref{fig:performance-breakdown}a shows that while Deja Vu benefits from activation sparsity, it still requires loading cold neurons when activated, resulting in communication costs—especially PCIe data transfer—comprising about 89\% of the execution time. On the right side of Figure \ref{fig:performance-breakdown}a, we disregard the effect of communication on performance. The MLP-based predictor in Deja Vu consumes roughly 18.1\% of computation time, further reducing the gains from activation sparsity. Our lightweight predictor, in contrast, contributes less than 0.1\% to runtime overhead. Even with communication costs lowered through reusable neurons at large batch sizes, Deja Vu's performance remains inferior to Hermes.

Figure \ref{fig:performance-breakdown}b compares \name-base and \name. Without activation sparsity, \name-base incurs higher computation costs, especially as batch sizes increase, due to intensive computation on NDP-DIMMs. For example, running LLaMA2-70B offloads over 80\% of computation to NDP-DIMMs, leading to a substantial portion of the execution time being occupied by FC computation. In Hermes, token generation takes 66.40\% of execution time at batch size 1. After optimizing token generation, the prompting stage becomes the bottleneck, accounting for about 33.01\% of the overhead, limiting further inference efficiency improvements.

% \fig \ref{fig:performance-breakdown} presents the performance breakdown of Deja Vu, \name-base and \name~on OPT-30B, OPT-66B, Falcon-40B and LLaMA2-70B. 
% It effectively describes in detail the sources of efficiency of \name. 

% As \fig \ref{fig:performance-breakdown}a shows, despite benefiting from activation sparsity, Deja Vu still needs to load cold neurons when they are activated. Consequently, the communication cost, particularly data transfer on PCIe, makes up about 89\% of the total execution time. On the right side of Figure \ref{fig:performance-breakdown}a, we disregard the effect of communication on performance. The MLP-based predictor used in Deja Vu consumes approximately 18.1\% of the overall computation time, diminishing the gains from activation sparsity. In contrast, our proposed lightweight predictor contributes to less than 0.1\% of the total runtime overhead. The proportion of communication in Deja Vu reduces due to reusable neurons when running at large batch sizes. Nonetheless, the overall performance of Deja Vu is still significantly inferior to Hermes.

% \fig \ref{fig:performance-breakdown}b provides a comparative analysis of the performance breakdown between \name-base and \name. The absence of activation sparsity in the Hermes-base results in considerably higher computation costs compared to \name, especially as the batch size increases. This is primarily due to the intensive computation on NDP-DIMMs, which significantly impacts overall execution efficiency. For instance, when running LLaMA2-70B, over 80\% of the computation is offloaded to NDP-DIMMs, leading to a substantial portion of the execution time being occupied by FC computation. In contrast, in Hermes, the token generation time occupies 66.40\% of the total execution time when at batch size is 1. With the token generation stage fully optimized, the prompting stage becomes the bottleneck, accounting for approximately 33.01\% of the overhead, thus limiting further improvements in inference efficiency.

\begin{figure}
    \centering
    \includegraphics[width=\linewidth]{Fig/dimm_rebuttal.pdf}
    \vspace{-0.3cm}
    \caption{\update{Throughput of four typical LLMs with different numbers of NDP-DIMMs. N.P. denotes the model is not supported by current system.}}
    \label{fig:dimm}
\vspace{-0.3cm}
\end{figure}

\update{\subsection{Sensitivity Studies}}

\subsubsection{\update{Sensitivity analysis of the number of DIMMs}}

\update{\fig~\ref{fig:dimm} illustrates the improvement in LLM throughput as the number of NDP-DIMMs increases. We evaluated four distinct LLM models using a single batch to understand the impact of varying numbers of NDP-DIMMs, while mitigating the effect of limited computation capability. An increase in NDP-DIMMs enhances both memory size and internal bandwidth. Larger memory capacity facilitates the deployment of more extensive models; for instance, deploying Falcon-40B on Hermes necessitates a minimum of four NDP-DIMMs. Additionally, higher internal bandwidth significantly enhances end-to-end performance, addressing the bandwidth limitations that bottleneck current offloading-based systems. However, once sufficient bandwidth is achieved, further increases in the number of NDP-DIMMs do not proportionally boost throughput. For example, LLaMA2-70B exhibits similar throughput with both 8 and 16 NDP-DIMMs. Once the NDP-DIMMs surpass the GPU in performance, additional NDP-DIMMs do not yield further performance gains.}

% \update{
% \fig~\ref{fig:dimm} shows that LLM throughput improves as the number of NDP-DIMMs increases. We evaluated four different LLM models with a single batch to assess the impact on different numbers of NDP-DIMMs, avoiding the effect of limited computation capacity. More NDP-DIMMs provide larger memory size as well as higher internal bandwidth. Abundant memory size allows the deployment of larger models. For example, deploying Falcon-40B on Hermes needs at least 4 NDP-DIMMs. Furthermore, higher internal bandwidth can effectively boost the end-to-end performance, as the limited bandwidth is the bottleneck of existing offloading-based system. However, when sufficient bandwidth is provided, the end-to-end throughput will not be further improved proportionally with the increasing number of NDP-DIMMs. For instance, LLaMA2-70B shows similar throughput with 8 and 16 NDP-DIMMs. Once the NDP-DIMMs outperform the GPU, adding more NDP-DIMMs no longer impacts performance.
% }

\begin{figure}
    \centering
    \includegraphics[width=\linewidth]{Fig/GPU_rebuttal.pdf}
    \vspace{-0.3cm}
    \caption{\update{Throughput of OPT-13B and OPT-30B with various GPUs, including RTX 4090, RTX 3090 and Tesla T4.}}
    \label{fig:gpu}
\vspace{-0.3cm}
\end{figure}

\subsubsection{\update{Sensitivity analysis of various GPUs}}

\update{\fig~\ref{fig:gpu} illustrates the significant impact of different GPUs on the end-to-end throughput of LLM execution. We have included two additional consumer-grade GPUs, Tesla T4 and RTX 3090, in our evaluation. Specifically, Tesla T4 offers 16GB of graphic memory, 320GB/s memory bandwidth, and 65 tensor TOPS (FP16), whereas RTX 3090 provides almost the same graphic memory and bandwidth as RTX 4090, but with 142 tensor TOPS (FP16). Overall, \name~with RTX 4090 achieves an average throughput improvement of $2.02\times$ and $1.34\times$ compared to \name~with Tesla T4 and RTX 3090, respectively. The data loading cost for RTX 3090 is nearly identical to that of RTX 4090. However, RTX 3090 spends more time on prefill and hot neuron computations due to its weaker computation capability. Tesla T4, with its smaller graphic memory and lower memory bandwidth compared to RTX 3090, is inefficient for data loading. Consequently, the choice of GPU device is crucial for optimizing \name~performance.}


\subsubsection{\update{Design Space Exploration for NDP-DIMMs}}
\begin{figure}
    \centering
    \includegraphics[width=\linewidth]{Fig/gemv_rebuttal.pdf}
    \vspace{-0.3cm}
    \caption{\update{Design Space Exploration for NDP-DIMMs with different number of multipliers in each GEMV unit.}}
    \label{fig:gemv}
\vspace{-0.3cm}
\end{figure}

\update{
\fig~\ref{fig:gemv} highlights the impact of increasing the number of multipliers within a GEMV unit per DIMM on LLM inference performance, especially with larger batch sizes. We varied the number of multipliers within a GEMV unit from 32 to 512, thereby enhancing computation capability by 16$\times$. For OPT-13B with a batch size of 1, performance stabilizes once 64 multipliers are reached, as further computation capability yields minimal gains. In contrast, with a batch size of 16, performance continuously improves with additional multipliers, achieving up to a $3.86\times$ speedup. This difference arises because memory bandwidth limits performance for smaller batch sizes due to lower arithmetic intensity, while computation capability becomes the bottleneck with larger batch sizes. To optimize the balance between hardware overhead and performance across various batch sizes, we selected 256 multipliers within the GEMV unit per DIMM.
}

\subsection{Comparison with High-Performance System}\label{sec:comparison-high-performance}

\begin{figure}[t]
    \centering
    \includegraphics[width=\linewidth]{Fig/comparison.pdf}
    \vspace{-0.3cm}
    \caption{Comparison with TensorRT-LLM on LLaMA2-70B.}
    \label{fig:trt-llm-comparison}
\vspace{-0.3cm}
\end{figure}

% 这里可以考虑一下系统的开销和对应的结果
This section discusses the performance gap between our budget-friendly LLM inference system Hermes and state-of-the-art high-performance serving system TensorRT-LLM~\cite{tensorrt-llm}. We kept the input and output sequence lengths set at 128. To handle LLaMA2-70B with a batch size of 16, TensorRT-LLM requires five NVIDIA A100-40GB-SXM4 GPUs. In contrast, 
\name~operates with only one NVIDIA RTX-4090 GPU and affordable NDP-DIMMs. Figure \ref{fig:trt-llm-comparison} displays the performance comparison between TensorRT-LLM and Hermes. For a batch size of 1, Hermes achieves 79.1\% inference efficiency of TensorRT-LLM. Even at a batch size of 16, Hermes retains 24.4\% inference efficiency of TensorRT-LLM. Despite this, Hermes is far more economical than TensorRT-LLM, which is equipped with 5 NVIDIA A100-40GB-SMX4 GPUs. Specifically, Hermes only costs approximately \$2,500, whereas TensorRT-LLM requires \$50000 to support LLaMA2-70B. Hermes provides efficient and low-budget LLM inference for local deployments. 

% Despite this, \name~is far more economical, costing approximately \$2,500 compared to the \$50,000 needed to build a high-performance system with 5 NVIDIA A100-40GB-SMX4 GPUs, making it highly efficient for local deployments with smaller batch sizes.
\section{Related Works}\label{sec:related-works}

\subsection{LLM Inference with PIM}\label{sec:related-works-PIM}

% It discusses the placement of computation units within HBM and identifies that GEMV units in the near bank of HBM are the optimal PIM placement for LLMs.

% Given that LLM inference is primarily memory bandwidth-bound, using PIM to accelerate LLM inference is a natural choice. AttAcc!~\cite{park2024attacc} utilizes a hybrid architecture of HBM-PIM and xPU (GPU/TPU), offloading the attention computation to HBM-PIM. It explores pipeline parallelism at the attention head level to further improve the inference efficiency. NeuPIMs~\cite{heo2024neupims} and IANUS~\cite{seo2024ianus} address the compatibility issue between PIM functionality and regular memory access by adopting dual buffers and incorporating additional control units, respectively. They optimize the design of HBM-PIM to support both processing and memory access simultaneously, utilizing PIM and xPU collaboration for LLM inference acceleration. These approaches adopt a fixed offloading strategy, delegating specific operators such as attention to PIM. SpecPIM~\cite{li2024specpim}, on the other hand, targets speculative LLM models with a multi-device architecture, where each device includes a xPU and multiple HBM-PIM chips. SpecPIM formalizes the mapping and scheduling of various models during speculative inference as a design space exploration problem, achieving flexible resource allocation based on actual model requirements. However, these works are all designed for server-grade devices (such as H100) and rely on expensive HBM-PIM for LLM inference acceleration, making them unsuitable for local deployment with limited budget. 

Given that LLM inference is primarily memory bandwidth-bound, accelerating it with processing in memory (PIM) is a natural choice~\cite{li2020hitm,zhai2023star,zhu2023processing}. AttAcc!~\cite{park2024attacc} utilizes a hybrid architecture of HBM-PIM and xPU (GPU/TPU), offloading the attention computation to HBM-PIM. NeuPIMs~\cite{heo2024neupims} and IANUS~\cite{seo2024ianus} address the compatibility issue between PIM functionality and regular memory access by adopting dual buffers and incorporating additional control units, respectively. They optimize the design of HBM-PIM to support both processing and memory access simultaneously, utilizing PIM and xPU collaboration for LLM inference acceleration. SpecPIM~\cite{li2024specpim}, on the other hand, targets speculative LLM models with a multi-device architecture, where each device includes an xPU and multiple HBM-PIM chips. However, these works are all designed for server-grade devices (such as H100) and rely on expensive HBM-PIM for LLM inference acceleration, making them unsuitable for local deployment with a limited budget. 

\subsection{LLM Acceleration with Activation Sparsity}\label{sec:related-works-sparsity}

% The promising activation sparsity in deep learning models motivates researches~\cite{zheng2023pit, cui2023optimizing} to further improve their inference efficiency, especially for LLMs. Deja Vu~\cite{liu2023deja} utilizes the activation sparsity to reduce the memory access on the unified memory of multiple server-grade GPUs. However, it it still requires storing all parameter data in GPU memory, failing to reduce GPU storage overhead, and thus is not suitable for local deployment scenarios. Powerinfer~\cite{song2023powerinfer} introduces a CPU-GPU hybrid system to achieve activation sparsity-based LLM inference. It stores hot neurons in GPU memory and uses GPU tensor cores for the corresponding computations while storing cold neurons in CPU memory and utilizing the CPU as a computing unit. However, the CPU-side memory bandwidth is significantly lower than that in the GPU, making CPU-side computation a bottleneck. Additionally, Powerinfer primarily targets single-batch computation, resulting in suboptimal utilization of GPU computation capability. Overall, existing systems do not fully exploit the advantages of activation sparsity. It is necessary to utilize processing units with high bandwidth and large storage to fully leverage activation sparsity. 

The promising activation sparsity in deep learning models motivates researchers~\cite{zheng2023pit, cui2023optimizing,liu2024drift} to further improve their inference efficiency, especially for LLMs. Deja Vu~\cite{liu2023deja} utilizes the activation sparsity to reduce the memory access on the unified memory of multiple server-grade GPUs. However, it still requires storing all parameter data in GPU memory, failing to reduce GPU storage overhead. Powerinfer~\cite{song2023powerinfer} introduces a CPU-GPU hybrid system to achieve activation sparsity-based LLM inference. It stores hot neurons in GPU memory and uses GPU tensor cores for the corresponding computations while offloading cold neurons in CPU memory and utilizing the CPU as a computing unit. However, the CPU-side memory bandwidth is significantly lower than that in the GPU, making CPU-side computation a bottleneck. Overall, existing systems do not fully exploit the advantages of activation sparsity.
\section{Conclusion}
We operationalized the theory of instrumental interaction for generative AI, with an in-depth unpacking of the principles of reification of user intent, reflection, and grounding. We argue that leveraging this re-appropriated and refined theory can drive the creation of a \textit{new generation of expressive AI-Instruments} that afford better expression of intent, make it easier to discover what is possible, and provide powerful degrees of freedom for steering the generation towards the best possible results. Those new tools and instruments can truly leverage the polymorphic and non-deterministic behavior of generative AI models, unleashing new and empowering forms of expressive HCI+AI experiences. 

Beyond our focus on AI-Instruments, theories play an important role in the advancement of our wider research field~\cite{rogers_hci_2012, halverson_activity_2002}. Rogers argues that there is a need for theories as lenses bringing critical design characteristics into focus, and which can function as a generative source: providing "\textit{design dimensions and constructs to inform the design and selection of interactive representations}"~\cite{rogers_new_2004}. We hope that our work on operationalizing the theory of instrumental interaction for AI can inspire other new -- and re-appropriated -- theories to advance HCI+AI. 










\section{Acknowledgments}
We sincerely thank the anonymous reviewers for their insightful suggestions. This work was partially supported by the National Key R\&D Program of China (Grant No. 2023YFB4404400) and the National Natural Science Foundation of China (Grant No. 62222411, 62204164). Ying Wang is the corresponding author (wangying2009@ict.ac.cn).

\bibliographystyle{plain}
\bibliography{references}

\end{document}

