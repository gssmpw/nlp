\begin{figure*}[h]
    \centering
    \begin{subfigure}[b]{\linewidth}
        \includegraphics[width=\linewidth]{figures/fig_approach.pdf}
    \end{subfigure}
    \caption{
        \textbf{Approach.}
        Recordings from 35 participants were obtained using electro-encephalography (EEG) and magneto-encephalography (MEG). Sentences were displayed word-by-word on a screen. Following the final word, a visual cue prompted them to begin typing this sentence, without visual feedback. Our Brain2Qwerty model includes three core stages to decode text from brain activity: (1) a convolutional module, input with 500\,ms windows of M/EEG signals, (2) a transformer module trained at the sentence level, and (3) a pretrained language model to correct the outputs of the transformer. Performance is assessed using a Character Error Rate (CER) at the sentence level. An analysis of \emph{how} the brain performs typing is described in a companion paper \citep{lucy2025}.
    }
    \label{fig:approach}
\end{figure*}