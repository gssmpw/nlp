\section{Related Works}
Generative model-based image super-resolution methods can be primarily categorized into three types: Flow-based methods, GAN-based methods, and Diffusion-based methods. The following sections introduce each of these approaches. We will also present work on super-resolution tasks combined with fractal.
\subsection{Flow-based Methods}
A flow-based approach encodes the original image into latent space using a function $f$, and then samples from the latent space to recover the image via the inverse function $f^{-1}$. Lugmayr et al. proposed a flow-based super-resolution model capable of learning the conditional probability distribution of a given low-resolution image, utilizing only one loss function: negative log-likelihood. Notably, SRFlow____ outperforms many current GAN-based methods in face super-resolution scenarios. While flow-based methods align well with mathematical intuitions, performing reversible processing is more challenging for neural networks.
\subsection{GAN-based Methods}
After the introduction of GANs____, Ledig et al.____ designed SRGAN, the first application of GANs for image super-resolution tasks, introducing a perceptual loss function to enhance image generation quality. Building on SRGAN, Kim et al.____ introduced RRDB and utilized features before activation for perceptual loss calculation, resulting in ESRGAN____, which aimed for improved super-resolution outcomes. GAN-based super-resolution methods combine various loss functions, enabling the model to generate higher quality super-resolution images. However, the adversarial training approach of GANs often leads to pattern collapse during training____.
\subsection{Diffusion-based Methods}
DDPM____ employs a stepwise denoising process capable of producing clear images. Saharia et al. proposed SR3,____ which combines low-resolution images with noise-laden feature maps in a denoiser, achieving strong performance across various super-resolution tasks. Li et al.____ introduced SRDiff, which incorporates residual prediction throughout the framework. In this approach, the original image is encoded through an encoder for processing conditional images, leading to improved super-resolution results. Shang et al.____ proposed ResDiff, which utilizes a simple CNN to recover the low-frequency components of the image, while DDPM predicts the residuals between the real image and the CNN-predicted image. Additionally, high-frequency information is introduced into the denoising network using Discrete Wavelet Transform (DWT) information. The ResShift proposed by Yue et al.____ realizes the conversion between high-resolution images and low-resolution images by shifting the residuals between them, thus greatly improving the conversion efficiency. The PASD network proposed by Yang et al.____ utilizes a high-level information extraction module to provide semantic signals for realistic image super-resolution and personalized stylization. The DiffBIR proposed by Lin et al.____ balances the realism a priori inherent in the diffusion model and the fidelity requirements needed for the image restoration task. The strong generative power of the diffusion model makes the image super-resolution effect better.
% Ultimately, ResDiff outperforms previous diffusion-based super-resolution methods, delivering higher generation quality.

% 分形与超分辨率结合的相关文献,先删除
% \subsection{Super-resolution with Fractal}
% Fractal interpolation and convolutional neural network can be utilized together for image super resolution____. This work uses fractal interpolation technique instead of bicubic interpolation for upscaling LR images. Fractal interpolation provides better approximation by preserving the edge and texture features in the image. ____ utilize a special orthogonal fractal coding method to construct higher resolution images from low resolution images. ____ proposed a SR method for noisy single images by applying the local fractal dimension to the local characterization of images. ____ proposed a single image super-resolution and enhancement algorithm using local fractal analysis. The gradient of the high resolution image is estimated from the gradient of the low resolution image based on the scale invariance feature of the fractal dimension. The regularization term based on the scale invariance of fractal dimension and length can effectively recover the details of high resolution images.