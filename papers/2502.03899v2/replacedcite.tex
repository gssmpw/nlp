\section{Related Work}
\label{sec:soa}
\subsection{Network Slicing Standards}
\label{subsec:soa-standards}
The \gls{3gpp} has done a great deal of standardization work on network slicing and \gls{qos} in 5G. \gls{ts} 23.501 ____ defines the overall 5G system architecture, including a framework providing \gls{qos} to network flows and an identification mechanism for managing network slices. 

\gls{ts} 28.530____ defines the general concepts and definitions for network slicing, as well as the phases that compose the lifecyle of a network slice instance. \gls{ts} 23.502____ describes the procedures to define the \gls{qos} policies in the 5G network functions and to connect mobile terminals to slices. \gls{tr} 28.801____ defines network slice management functions. Management operations and procedures for provisioning network slices are defined in \gls{ts} 28.533____.

The \protect\gls{gsma}____ has defined generic network slices templates (GST) that contain a set of attributes that can characterize a type of network slice or service. A \gls{nest} is obtained by assigning specific values to the fields in the GST. 3GPP network slice management functions build a network slice based on a provided \gls{nest}.

The above standards define control and management plane mechanisms required for the administration and orchestration of network slices. However, they do not specify how network slices can be implemented in the data plane. A topic neither addressed by the above standards is how to define an end-to-end network slice across the \gls{ran}, \gls{tn}, and core networks. 

The \gls{ietf} has made progress in this area by defining: (1) an orchestration and management framework that enables the inter-operation between 5G network slices in non-\gls{3gpp} \glspl{tn} that connect with network slices defined within a \gls{3gpp} domain____; and (2) how to identify a \gls{3gpp} network slice for associating it to a \gls{tn} slice and how the \glspl{5qi}, which define levels of \gls{qos} for the traffic flows, are translated to the \gls{tn} classes in the non-\gls{3gpp} domain________. By means of this identification, the \gls{tn} is capable of treating these traffic flows in the \gls{tn} with the expected level of \gls{qos}. This makes it possible to create end-to-end slices across all the network segments. 


Network slice management, coordination and signaling between \gls{3gpp} and \gls{ietf} control elements are also necessary and partially defined in ____, but they are out of the scope of this paper.

\subsection{Network Slicing Literature}
\label{subsec:soa-literature}
There is extensive literature covering the application of network slicing techniques across the different segments of the network. Wang et al.____ followed an approach based on queuing disciplines implemented over a software-defined network based on P4. They proposed a hierarchical queuing structure that has two levels. The first one is composed of eight round robin queues that ensure a proper load balance between the processed traffic. In the second level, each round robin queue is connected to four priority queues. In this way, it is possible to limit the delay of different traffic classes in each round robin queue. The problem with this approach is that it is not aligned with the model proposed by \gls{ietf} for \glspl{tn}, for instance, because it does not consider a policer controlling the traffic entering the network.

Other works____,____ proposed using traffic shaping mechanisms based on \gls{htb} for providing each network slice with a bandwidth guarantee and allowing a network slice to consume unused bandwidth by other slices. The proposed traffic shaping mechanism provides traffic isolation between network slices that are sharing the network infrastructure. Raussi \textit{et al.}____ also proposed the use of traffic shaping mechanisms based on \glspl{htb} to improve the reliability of communications in smart grids scenarios with wired connections. Lin et al.____ implemented \gls{qos} framework over a P4-based network composed of four functional blocks: a classifier, a marker, a policer and a packet scheduler based on priority queues. In this work, all traffic marked as \textit{``green"} from all of the slices is enqueued into the same priority queue, so, it is not possible to control the delay of the \textit{``green"} traffic of each slice. Additionally, traffic marked as \textit{``yellow"} is enqueued in a lower priority queue, so it does not meet the latency requirements. \rev{Chen \textit{et al.} proposal____  uses two priority queues, one for the \textit{``green"} traffic coming from all the slices, and a lower priority one for \textit{``yellow"} traffic from all the slices and for non-sliced or best-effort traffic (this is a similar arrangement to____). The lower priority queue is further divided into four queues using a \gls{drr} to separate different types of flows, which allows to control the sharing of the available bandwidth beyond the one guaranteed to the network slices. This proposal meets the bandwidth slice requirements and has a mechanism to control the sharing of the additional bandwidth available, but because all the slices share the same queue, bandwidth sharing among them is not controlled and delay constraints are not addressed. In general, proposals that use \textit{``yellow"} traffic to enable bandwidth sharing have the drawback that losses and delay for such traffic cannot be controlled.}

\rev{Huin \textit{et al.}____ and Martin \textit{et al.}____ have followed a different approach, where network resources are exclusively dedicated to each slice. While this approach provides a high level of isolation between slices, it reduces network utilization and efficiency by preventing the sharing of unused bandwidth from one slice with others, potentially leading to under-utilization of resources.}

None of the above works addressed the problem of control the delay of different types of traffic flows. On the other hand, \rev{Chang \textit{et al.}\rev{____} proposed a network slicing technique at the data link layer (L2) that is able to meet the latency and bandwidth requirements of different network slices by using a queuing system based on priority queues and Active Queue Management (AQM). The proposed model is implemented over a network of P4-programmable data plane switches, an approach that is still not aligned with \gls{ietf} priorities. Besides, a worst-case scenario with bursty traffic, as proposed in this paper, has also not been analyzed either.}

Baba et al. ____ analyzed the impact of micro-bursts on the \gls{qos} of the 5G network, comparing the case of using a priority queuing scheduler versus a \gls{wfq} scheduler. However, they did not propose a solution to limit this impact.

Regarding vehicular networks, several works____ focused on the management of resources on the radio interface to support network slicing. The work in____, in addition to studying slicing techniques in the radio interface, also explored how to apply network slicing in the core of the network, and proposed the use of priority queues to achieve a latency in mission critical traffic lower than in best effort traffic. For the specific case of \gls{tod}, the work in____ identified the need to define a network slice capable of meeting the strict \gls{qos} requirements of the service. 

\rev{None of the existing works in the state of the art align with the slicing model defined by the \gls{ietf} for 5G \glspl{tn}. Additionally, they fail to account for multiple flows with different \gls{qos} requirements per slice or to address how to limit delays while guaranteeing bandwidth requirements in worst-case scenarios under bursty traffic conditions. The solution proposed in this paper makes it possible to satisfy both the bandwidth and the latency requirements of \glspl{tn} slices and incorporate traffic burst control for worst-case scenarios.}