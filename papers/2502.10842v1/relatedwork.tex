\section{Related Works}
\label{sec:relatedworks}
Hern${\acute{\text{a}}}$ndez et al. \citep{hernandez2008multiview} introduced the classical MVPS setup. Yet, Nehab et al. \citep{nehab2005efficiently} pioneered the integration of PS images with active range scanning techniques, highlighting the complementary nature of shape from shading and active range modalities in 3D data acquisition. However, the widely used MVPS setup involves a turntable experimental setup, where PS images of the subject positioned on the turntable are captured from a staged viewpoint. It is essential to highlight that within this setup, both the camera and the lighting sources maintain a stationary position, with only the turntable undergoing rotation. This rotation facilitates the acquisition of the subject's viewpoints from different angles with each turn. Specifically, each turntable rotation facilitates capturing and storing MVS and PS images for each lighting source (see Fig.\ref{fig:mvps_hardware_comparison}(a)). 

\vspace{-0.3cm}
\revisedtwo{
\subsection{Traditional MVPS Methods}
}
Earlier MVPS approaches often relied on a specific Bidirectional Reflectance Distribution Function (BRDF) model, leading to unreliable outcomes for real-world objects whose reflectance properties deviate from the presupposed BRDF model. To this end, \citep{park2016robust} introduced a method based on piece-wise planar mesh parameterization, designed to enhance the object's fine surface details reconstruction through displacement texture maps. Yet, their work overlooked surface reflectance modeling. Contrarily, other methodologies, such as \citep{ren2011pocket, dong2010manifold}, do engage in BRDF modeling. Still, their usage is limited to nearly flat surfaces and presupposes a prior knowledge of the surface normals.

\vspace{-0.2cm}
\revisedtwo{
\subsection{Deep Learning based MVPS Methods}
}
In recent years, deep learning methodologies have been suggested as better alternatives to traditional MVPS techniques, albeit employing classical MVPS hardware setups. To this end, \citep{zhang2022iron} proposed a neural inverse rendering approach to reconstruct objects' shape and material attributes. However, this method is predicated on the assumption of a co-located camera and lighting source, hence incompatible with standard MVPS setups. On the contrary, \citep{kaya2021neural}  proposed a method based on neural radiance fields. This approach endeavors to estimate the object's surface normal and integrates them within a volume rendering formula to enhance object representation learning. Despite its conceptual benefit, this technique struggles to accurately capture objects' high-quality geometric details.
Conversely, an uncertainty-based MVPS methodology has been introduced \citep{kaya2022uncertainty}, demonstrating superior performance in 3D object reconstruction. Recently, \citep{kaya2023multi} extended \citep{kaya2022uncertainty} to make deep MVPS work for diverse material types. Whereas, \citep{zhao2023mvpsnet} worked on speeding up the deep-MVSP model inference time while maintaining the 3D acquisition accuracy. Not long ago, \citep{lichy2021shape} proposed estimating the geometry and reflectance of objects using a camera, flashlight, and a tripod setup. Whereas \citep{Cheng_2023_CVPR} uses a smartphone’s built-in flashlight and combines darkroom photometric methods with neural light fields \citep{wang2021neus} to recover an object's 3D geometry. 

\begin{figure*}[t]
    \centering
\includegraphics[width=1.0\textwidth]{Camera_Trajectory_Acquisition_2.pdf}
    \caption{\textbf{Our mobile robotic test time setup.} (a) Our mobile robot moves around the object at test time, performing 3D data acquisition. (b) The robot's ground truth and recovered camera pose trajectory are shown in red and green, respectively. (c) side view of the recovered 3D data compared to ground truth shown in millimeters (mm) along chosen geodesic (shown with a red line on BUDDHA image).}
    \label{fig:img_acquisition_camera_traj}
\end{figure*}

\smallskip
\smallskip
\noindent
Alternatively, to all the approaches to MVPS mentioned above, this paper introduces a novel mobile robotic hardware setup and supporting algorithm based on the MVPS principle for performing an object's 3D acquisition in an incremental setting. Our approach helps recover fine, detailed 3D object geometry under limited lighting conditions. It further provides flexibility in the movement of the MVPS system via a controlled robotic setup for far or near capture, depending on the object size and its surface details, which is not practically possible with previous MVPS experimental setup.