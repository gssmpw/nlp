%%%%%%%% ICML 2025 EXAMPLE LATEX SUBMISSION FILE %%%%%%%%%%%%%%%%%

\documentclass{article}

% Recommended, but optional, packages for figures and better typesetting:
\usepackage{microtype}
\usepackage{graphicx}
\usepackage{subfigure}
\usepackage{booktabs} % for professional tables
\usepackage{multirow}
% hyperref makes hyperlinks in the resulting PDF.
% If your build breaks (sometimes temporarily if a hyperlink spans a page)
% please comment out the following usepackage line and replace
% \usepackage{icml2025} with \usepackage[nohyperref]{icml2025} above.
\usepackage{hyperref}
\usepackage{tcolorbox}
% Attempt to make hyperref and algorithmic work together better:
\newcommand{\theHalgorithm}{\arabic{algorithm}}

% Use the following line for the initial blind version submitted for review:
% \usepackage{icml2025}

% If accepted, instead use the following line for the camera-ready submission:
\usepackage[accepted]{icml2025}

% For theorems and such
\usepackage{amsmath}
\usepackage{amssymb}
\usepackage{mathtools}
\usepackage{amsthm}
\usepackage{xspace}
\usepackage{enumitem}
\usepackage{makecell}
% if you use cleveref..
\usepackage[capitalize,noabbrev]{cleveref}

%%%%%%%%%%%%%%%%%%%%%%%%%%%%%%%%
% THEOREMS
%%%%%%%%%%%%%%%%%%%%%%%%%%%%%%%%
\theoremstyle{plain}
\newtheorem{theorem}{Theorem}[section]
\newtheorem{proposition}[theorem]{Proposition}
\newtheorem{lemma}[theorem]{Lemma}
\newtheorem{corollary}[theorem]{Corollary}
\theoremstyle{definition}
\newtheorem{definition}[theorem]{Definition}
\newtheorem{assumption}[theorem]{Assumption}
\theoremstyle{remark}
\newtheorem{remark}[theorem]{Remark}

% Todonotes is useful during development; simply uncomment the next line
%    and comment out the line below the next line to turn off comments
%\usepackage[disable,textsize=tiny]{todonotes}
\usepackage[textsize=tiny]{todonotes}

\newcommand{\xuandong}[1]{{\color{orange} {[Xuandong: #1]}}}
% The \icmltitle you define below is probably too long as a header.
% Therefore, a short form for the running title is supplied here:
\icmltitlerunning{Scalable Best-of-N Selection for Large Language Models via Self-Certainty}

\begin{document}

\twocolumn[
\icmltitle{Scalable Best-of-N Selection for Large Language Models via Self-Certainty}

% It is OKAY to include author information, even for blind
% submissions: the style file will automatically remove it for you
% unless you've provided the [accepted] option to the icml2025
% package.

% List of affiliations: The first argument should be a (short)
% identifier you will use later to specify author affiliations
% Academic affiliations should list Department, University, City, Region, Country
% Industry affiliations should list Company, City, Region, Country

% You can specify symbols, otherwise they are numbered in order.
% Ideally, you should not use this facility. Affiliations will be numbered
% in order of appearance and this is the preferred way.
\icmlsetsymbol{equal}{*}

\begin{icmlauthorlist}
\icmlauthor{Zhewei Kang}{equal,yyy}
\icmlauthor{Xuandong Zhao}{equal,yyy}
\icmlauthor{Dawn Song}{yyy}
%\icmlauthor{}{sch}
%\icmlauthor{}{sch}
\end{icmlauthorlist}

\icmlaffiliation{yyy}{University of California, Berkeley}
% \icmlaffiliation{comp}{Company Name, Location, Country}
% \icmlaffiliation{sch}{School of ZZZ, Institute of WWW, Location, Country}

\icmlcorrespondingauthor{Zhewei Kang}{waynekang@berkeley.edu}
\icmlcorrespondingauthor{Xuandong Zhao}{xuandongzhao@berkeley.edu}

% You may provide any keywords that you
% find helpful for describing your paper; these are used to populate
% the "keywords" metadata in the PDF but will not be shown in the document
\icmlkeywords{Machine Learning, ICML}

\vskip 0.3in
]

% this must go after the closing bracket ] following \twocolumn[ ...

% This command actually creates the footnote in the first column
% listing the affiliations and the copyright notice.
% The command takes one argument, which is text to display at the start of the footnote.
% The \icmlEqualContribution command is standard text for equal contribution.
% Remove it (just {}) if you do not need this facility.

%\printAffiliationsAndNotice{}  % leave blank if no need to mention equal contribution
\printAffiliationsAndNotice{\icmlEqualContribution} % otherwise use the standard text.

\begin{abstract}
Best-of-N selection is a key technique for improving the reasoning performance of Large Language Models (LLMs) through increased test-time computation. Current state-of-the-art methods often employ computationally intensive reward models for response evaluation and selection. Reward-free alternatives, like self-consistency and universal self-consistency, are limited in their ability to handle open-ended generation tasks or scale effectively. To address these limitations, we propose \emph{self-certainty}, a novel and efficient metric that leverages the inherent probability distribution of LLM outputs to estimate response quality without requiring external reward models. 
We hypothesize that higher distributional self-certainty, aggregated across multiple samples, correlates with improved response accuracy, as it reflects greater confidence in the generated output. 
Through extensive experiments on various reasoning tasks, we demonstrate that self-certainty (1) scales effectively with increasing sample size $N$, akin to reward models but without the computational overhead; (2) complements chain-of-thought, improving reasoning performance beyond greedy decoding; and (3) generalizes to open-ended tasks where traditional self-consistency methods fall short. Our findings establish self-certainty as a practical and efficient way for improving LLM reasoning capabilities. The code is available at \url{https://github.com/backprop07/Self-Certainty}
\end{abstract}

\section{Introduction}

Multi-modal models, such as CLIP~\citep{radford2021}, have demonstrated strong performance in representation learning.
By aligning visual and textual representations, these models achieve state-of-the-art results in tasks like image retrieval~\citep{baldrati2022conditioned,baldrati2022effective}, visual question answering~\citep{pan2023retrieving,song2022clip}, and zero-shot classification~\citep{radford2021,ali2023clip,wang2023improving,zhang2022tip}. 
Despite these successes, the mechanisms by which multi-modal models leverage their training data to achieve good generalization remain underexplored. 

In uni-modal setups, both supervised~\citep{feldman2020does,feldman2020neural} and self-supervised~\citep{wang2024memorization}, machine learning models have shown that their ability to \textit{memorize} their training data is essential for generalization. 
It was indicated that, in supervised learning, memorization typically occurs for mislabeled samples, outliers~\citep{bartlett2020benign,feldman2020does,feldman2020neural}, or data points that were seen towards the end of training~\citep{jagielski2022measuring}, while in self-supervised learning, high memorization is experienced particularly for atypical data points~\citep{wang2024memorization}. 
However, it is unclear how these findings extend to models like CLIP which entail elements from both supervised learning (through captions as supervisory signals) and self-supervised learning (through contrastive loss functions).

Existing definitions of memorization offer limited applicability to CLIP and therefore cannot fully address the gap in understanding.
% can, hence, not close the gap in understanding:
The standard definition from supervised learning~\citep{feldman2020does} relies on one-dimensional labels and the model's ability to produce confidence scores for these labels, whereas CLIP outputs high-dimensional representations. While the SSLMem metric~\citep{wang2024memorization}, developed for self-supervised vision models, could, in principle, be applied to CLIP's vision encoder outputs, it neglects the text modality, which is a critical component of CLIP. Additionally, measuring memorization in only one modality, or treating the modalities separately, risks diluting the signal and under-reporting memorization. Our experimental results, as shown in \Cref{sub:sslmem_not_for_clip}, confirm this concern. Therefore, new definitions of memorization tailored to CLIP's multi-modal nature are necessary.
\begin{figure}[t]
    \centering
    \begin{subfigure}[b]{0.475\textwidth}
        \centering
        \includegraphics[width=\textwidth]{image/10_most_1_caption.pdf}
        \caption[]{{\small Most Memorized: CLIPMem $>$ 0.89}}
    \end{subfigure}
    \hfill
    \begin{subfigure}[b]{0.475\textwidth}  
        \centering 
        \includegraphics[width=\textwidth]{image/10_least_1_caption.pdf}
        \caption[]%
        {{\small Least Memorized: CLIPMem $\approx$ 0.0}}    
    \end{subfigure}
    % \vskip\baselineskip
    % \begin{subfigure}[b]{0.475\textwidth}   
    %     \centering 
    %     \includegraphics[width=\textwidth]{Example-Image}
    %     \caption[]%
    %     {{\small Network 3}}    
    %     \label{fig:mean and std of net34}
    % \end{subfigure}
    % \hfill
    % \begin{subfigure}[b]{0.475\textwidth}   
    %     \centering 
    %     \includegraphics[width=\textwidth]{Example-Image}
    %     \caption[]%
    %     {{\small Network 4}}    
    %     \label{fig:mean and std of net44}
    % \end{subfigure}
    \caption{\textbf{Examples of data with different levels of memorization.} Higher memorization scores indicate stronger memorization. 
    We observe that atypical or distorted images, as well as those with incorrect or imprecise captions, experience higher memorization compared to standard samples and easy-to-label images with accurate captions.
    % We observe that atypical or distorted images and images with incorrect or imprecise captions experience higher memorization compared to more standard samples and easy-to-label samples with precise captions. 
    Results are obtained on OpenCLIP~\citep{ilharco_gabriel_2021_5143773}, with encoders based on the ViT-Base architecture trained on the COCO dataset.} 
        \label{fig:examples}
        %\vspace{-0.8cm}
\end{figure}

The only existing empirical work on quantifying memorization in CLIP models~\citep{jayaraman2024} focuses on Déjà Vu memorization~\citep{meehan2023ssl}, a specific type of memorization.
The success of their method relies on the accuracy of the integrated object detection method and on the availability of an additional public dataset from the same distribution as CLIP's training data, limiting practical applicability.
To overcome this limitation, we propose \textit{\ours} that measures memorization directly on CLIP's output representations.
Specifically, it compares the alignment---\ie the similarity between representations---of a given image-text pair in a CLIP model trained with the pair, to the alignment in a CLIP model trained on the same data but without the pair.

% Additionally, we focus on \textit{understanding} memorization rather than quantifying it. 
% %which, in CLIP, can be measured only with respect to an additional public dataset from the same distribution as CLIP's training data and fine-grained object detection methods. Moreover, the work is limited to \textit{quantifying} memorization.
% %---limiting its practical applicability. Moreover, while the work \textit{quantifies} Déjà Vu memorization, it does not offer detailed insights into which specific data points are memorized, why they are memorized, and how this relates to generalization. 
% %In contrast to their work, our focus is on \textit{understanding} memorization in CLIP by 
% We use this to identify which properties of the data and the two modalities contribute to CLIP memorization and on leveraging these insights to achieve \textit{better model utility while mitigating memorization}. To this end, we propose \textit{\ours} that directly measures memorization on the representations produced by CLIP's vision and text encoders.
% Specifically, \ours measures memorization by comparing the alignment, \ie the similarity between representations, of a given image-text pair in a CLIP model trained with this pair to the alignment in a CLIP model trained without this pair but on the same data otherwise.


In our empirical study of memorization in CLIP using \ours, we uncover several key findings. First, examples with incorrect or imprecise captions ("mis-captioned" examples) exhibit the highest levels of memorization, followed by atypical examples, as illustrated in \Cref{fig:examples}.
Second, removing these samples from training yields significant improvements in CLIP's generalization abilities.
These findings are particularly noteworthy, given that state-of-the-art CLIP models are usually trained on large, uncurated datasets sourced from the internet with no guarantees regarding the correctness of the text-image pairs.
Our results highlight that this practice not only exposes imprecise or incorrect data pairs to more memorization, often recognized as a cause for increased privacy leakage~\citep{carlini2019secret, carlini2021extracting, carlini2022privacy,song2017machine,liu2021encodermi}, but that it also negatively affects model performance. 
%By identifying highly memorized samples, our \ours can, hence, support a more private and performant deployment of CLIP.\todo{@Adam, is that last sentence too strong?}
Furthermore, by disentangling CLIP's two modalities, we are able to dissect how memorization manifests within each.
Surprisingly, we find that memorization does not affect both modalities alike, with memorization occurring more in the text modality than in the vision modality.
% even though the training objective is symmetric.\todo{@Adam, is that correct?}
% In fact, our results highlight that memorization occurs more in the text modality than in the vision modality. 
Building on these insights, we propose several strategies to reduce memorization while simultaneously improving generalization---a result that has not been observed in traditional supervised or self-supervised learning, where any reduction of memorization causes decreases in performance.
% which, at the same time, improve generalization.
% Such a result has not been observed in traditional supervised or self-supervised learning, where any reduction of memorization causes decreases in performance. 
Finally, at a deeper level, our analysis of the model internals, following~\citet{wang2024localizing}, shows that CLIP's memorization behavior sits between that of supervised and self-supervised learning. Specifically, neurons in early layers are responsible for groups of data points (\eg classes), similar to models trained using supervised learning, while neurons in later layers memorize individual data points, as seen in self-supervised learning.%\todo{cite our localization paper.}
% Performing an empirical evaluatin of memorization in CLIP according to our \ours, we find that
% -  examples with incorrect ("mis-captioned") or imprecise captions experience highest memorization, and then atypical examples . We show this effect in \Cref{fig:examples}.
% - memorization happens more in the text than in the vision modality
% - by including more captions into training, when only a few of them are mislabeled and the rest is correct, we can reduce memorization and at the same time improve generalization, something that has not been possible for supervised or self-supervised learning.
% - looking at the model internals, we see that the memorization behavior of CLIP is exactly in between supervised and self-supervised learning: in particular, neurons in early layers are responsible for groups (classes) of data points, same like for supervised learning, while neurons in later layers are responsible for individual data points

% \franzi{@Adam, do you think, we need an additional paragraph here on the mitigations we have? We actually only have multi-caption, so far, so probably not so important?}\adam{We also can mitigate the memorization if we remove the most memorized (probably mislabeled) samples.}

In summary, we make the following contributions:
\begin{itemize}
    \item We propose \ours, a metric to measure memorization in multi-modal vision language models.
    \item Through extensive evaluation, we identify that "mis-captioned" and "atypical" data points experience the highest memorization, and that the text encoder is more responsible for memorization than the image encoder.
    \item Based on our insights, we propose and evaluate multiple strategies to mitigate memorization in CLIP. We show that in CLIP, contrary to traditional supervised and self-supervised learning, a reduction of memorization does not need to imply a decrease in performance.
\end{itemize}


\subsection{Large Language Models for Chemistry}
%化学agent,大语言模型(LLMs)在化学领域展现出巨大的潜力,被广泛应用于分子生成、性质预测、反应建模、逆合成分析等任务。例如,ChemDFM模型通过在化学文献和教科书上进行预训练,并使用大量指令进行微调,提升了多种化学任务上的表现。论文2介绍了一个名为ChemCrow的新型大型语言模型(LLM)驱动的化学助手,它通过整合多种专家设计的化学工具来增强LLM在化学领域的表现。
%尽管 LLM 已展现出强大的潜力,但它们在应对复杂的化学计算任务时仍存在诸多局限性。即使通过微调(fine-tuning)或指令调控,LLMs 仍难以适应高度复杂或需要精确计算的化学任务,且其泛化能力在面对不同化学任务时表现出不足。并且LLM 无法高效调用现有的化学计算工具,如SMILES 解析工具、分子动力学模拟工具等,更无法有效探索和利用工具之间的组合和堆叠关系。

Large language models (LLMs) have demonstrated significant potential in chemistry, with applications spanning molecular generation, property prediction, reaction modeling, and retrosynthetic analysis~\cite{fang2024molinstructionslargescalebiomolecularinstruction,tang2024prioritizingsafeguardingautonomyrisks,liao2024wordsmoleculessurveylarge}. For instance, the ChemDFM ~\cite{zhao2024chemdfmlargelanguagefoundation} pretrained on chemical literature and textbooks and further refined through extensive instruction tuning, has exhibited enhanced performance across various chemical tasks. Similarly, ChemCrow~\cite{bran2023chemcrowaugmentinglargelanguagemodels}, an LLM-powered chemistry assistant, integrates multiple expert-designed chemical tools to improve LLM performance in chemistry-related applications.Despite these advancements, LLMs continue to face challenges in handling complex chemical computations and generalizing across diverse chemical problems~\cite{ouyang2024structuredchemistryreasoninglarge,han2024generalistspecialistsurveylarge}. Moreover, they remain inefficient in utilizing existing computational chemistry tools~\cite{shi2023relmleveraginglanguagemodels}, and struggle to navigate the combinatorial and hierarchical relationships between these tools.

\subsection{Tool-augmented LLMs}
LLMs~\cite{anil2023palm,achiam2023gpt,touvron2023llama} have demonstrated strong reasoning capabilities in natural language processing and scientific computing. However, they face limitations in specialized tasks in fields such as chemistry and physics~\cite{yang2024moosechemlargelanguagemodels}, including constrained computational accuracy, insufficient numerical reasoning abilities, and a lack of collaboration with external tools. To address these shortcomings, researchers have recently proposed the tool-augmented LLMs approach~\cite{qin2023toolllmfacilitatinglargelanguage,wang2024gtabenchmarkgeneraltool,yang2023autogptonlinedecisionmaking}, enabling LLMs to dynamically call external tools and thereby enhance their task execution capabilities.
Representative methods include ReAct~\cite{yao2023reactsynergizingreasoningacting}, which combines chain-of-thought reasoning (CoT)~\cite{wei2023chainofthoughtpromptingelicitsreasoning} with tool invocation to allow LLMs to dynamically acquire external information during decision-making, and Toolformer~\cite{schick2023toolformerlanguagemodelsteach}, which enables LLMs to autonomously decide when to call tools, improving the accuracy of computational tasks. 
Despite these advancements, existing research primarily focuses on single-tool invocation and has yet to explore hierarchical combinations of tools. A single tool is often insufficient to solve complex scientific problems, whereas the collaborative invocation of multiple tools holds promise for enhancing the reasoning capabilities of LLMs in chemical tasks.
\section{Method}
\label{sec:method}
In this section, we propose a neuroscience-informed fMRI encoder designed to achieve high-performance, subject-agnostic decoding. To further enable versatile decoding, we introduce the construction of a brain instruction tuning dataset, which captures diverse semantic representations encoded in fMRI data.

\subsection{Method Overview}
As illustrated in Figure~\ref{fig:arch}, our model consists of an fMRI encoder $f_\theta$ and an off-the-shelf LLM. In practice, we use Vicuna-7b \cite{zheng2023judging} as our LLM to maintain consistency with our baseline \cite{xia2024umbrae}. For each sample, let $\boldsymbol{v} = [v_1, v_2, \cdots, v_N]\in \mathbb{R}^N$ be the fMRI signals of input voxels, where $N$ is the number of voxels. Note that $N$ varies between different subjects, ranging from $12,682$ to $17,907$ in the dataset we use \cite{allen2022massive}.

The fMRI encoder $f_\theta$, featuring a neuroscience-informed attention layer, encodes $\boldsymbol{v}$ to fMRI tokens $X_v = [\boldsymbol{x}_{v,1}, \boldsymbol{x}_{v,2}, \cdots, \boldsymbol{x}_{v,L}] \in \mathbb{R}^{d\times L}$, where $L$ is the number of tokens and $d$ is the dimension of token embeddings. We then prepend these learned fMRI tokens to the language tokens in the BIT dataset we propose.

\subsection{fMRI Encoder}
As mentioned before, currently most models for fMRI decoding can not handle varying input shapes and are not subject-agnostic, with only a few exceptions \cite{mai2023unibrain}. However, these exceptions still suffer from information loss and uneven representations of certain brain areas. To this end, we propose a novel neuroscience-informed attention mechanism to accommodate varying voxel numbers across subjects, enabling a subject-agnostic encoding strategy. Below we talk about the design of \textit{queries} $\{\boldsymbol{q}_i\}$, \textit{keys} $\{\boldsymbol{k}_i\}$ and \textit{values} $\{\boldsymbol{v}_i\}$ in the attention layer. For \textit{values}, we directly use the fMRI signal of each voxel, which means $\boldsymbol{v_i} = v_i \in \mathbb{R}$. Making each voxel a \textit{value} token maximally prevents information loss compared to pooling- \cite{wang2024mindbridge} or sampling-based \cite{mai2023unibrain} methods. The \textit{queries} are randomly initialized and learnable. We expect each \textit{query} to represent a certain pattern of the brain (refer to visualizations in Section \ref{sec:vis}). The design of \textit{keys} will be discussed below.

\noindent\textbf{Exclude fMRI values from \textit{keys}}
The vanilla cross attention \cite{zhu2020deformable,vaswani2017attention} derives both \textit{keys} and \textit{values} from the same input source. However, we found this would lead to poor performance in fMRI. We argue the reason: different from images or text, which are usually considered translation-invariant, the positions of voxels carry specific brain \textit{functional information}, as voxels in different areas are associated with distinct brain functions. Consequently, a voxel's position alone can theoretically serve as effective \textit{keys} for attention weight computation. Including fMRI values into \textit{keys}, however, introduces additional noise instead of valuable information, thus resulting in poorer performance. Moreover, since brain regions tend to serve similar functions across individuals, decoupling voxel positions from fMRI signals can facilitate the sharing of priors across subjects, potentially improving generalization to unseen subjects.

In light of this, instead of the vanilla cross attention, which derives the \textit{keys} and \textit{values} from the same inputs, we exclude the fMRI value of each voxel and use its positional information alone as its \textit{key} embedding. The positional information is encoded from the coordinates of each voxel, i.e. $\boldsymbol{k}_i^{\text{pos}} = \operatorname{PE}(\boldsymbol{c}_i)$ for the $i$-th voxel, where $\boldsymbol{c}_i \in \mathbb{R}^3$ denotes the coordinates of the voxel. In practice, we use the Fourier positional encoding proposed in \cite{tancik2020fourier} due to its superiority in encoding coordinate information.

\noindent\textbf{Incorporation of Brain Parcellations}
% \noindent\textbf{Incorporation of Brain Parcellations}
While positional encoding alone improves performance, it lacks inherent neuroscientific grounding, potentially making it challenging for the model to efficiently learn representations aligned with established principles of brain function. To overcome this, we incorporate existing brain region parcellations \cite{glasser2016multi,rolls2020automated} into the \textit{key} embeddings. Formally, given a parcellation $\mathcal{P}$, with regions indexed by $1, \cdots, N_\mathcal{P}$. Let $\mathcal{P}(i) \in [1, 2, \cdots, N_\mathcal{P}]$ be the region that the $i$-th voxel belongs to, and $E[\mathcal{P}(i)] \in \mathbb{R}^d$ be the corresponding learnable embedding of the region, which will be incorporated in the \textit{key} embeddings as $\boldsymbol{k}_i^{\text{reg}, \mathcal{P}} = E[\mathcal{P}(i)] \in \mathbb{R}^d$.

\noindent\textbf{Combining Multiple Parcellations}
It is crucial to choose an appropriate brain region parcellation. Previous region-based methods \cite{qiu2023learning,li2021braingnn, kan2022brain} can usually only choose one arbitrarily. In contrast, our model design allows us to combine multiple parcellations $\mathcal{P}^1, \mathcal{P}^2, \cdots$ by concatenating their respective region encodings to the \textit{key} embeddings. In conclusion, the final \textit{key} embeddings are the concatenation by the positional encoding and multiple region encodings,
\begin{equation}
    \boldsymbol{k}_i = \boldsymbol{k}_i^\text{pos} \| \boldsymbol{k}_i^{\text{reg}, \mathcal{P}^1} \|  \boldsymbol{k}_i^{\text{reg}, \mathcal{P}^2} \| \cdots
\end{equation}
where $\|$ denotes the concatenation operation. This process is illustrated in Figure~\ref{fig:arch}'s lower right part.

The positional and region encodings complement each other: The region encodings serve as coarse-scale features, providing a neuroscientific-grounded basis, while the fine-scale positional encoding allows our model to learn finer-grained information directly from the data.

This attention design separates a voxel's \textit{functional information}—which is largely consistent across individuals—from its fMRI value, thereby enhancing generalization. Instead of relying on pooling or sampling, the attention mechanism employs learnable aggregation, while the integration of positional encoding and neuroscientifically informed region encodings further ensures high performance.

After the attention layer, we obtain the hidden representations $\boldsymbol{z}_q \in \mathbb{R}^{N_q} $ where $N_q$ is the number of query embeddings. We then employ an MLP and a reshape operation to map the hidden representations to $L$ fMRI tokens, i.e., $   X_v = \operatorname{reshape}\left( \operatorname{MLP}(
    \{\boldsymbol{z}_q\}
    ) \right) \in \mathbb{R}^{L \times d}$.

The process of the fMRI encoder is illustrated in Figure~\ref{fig:arch}. The obtained fMRI tokens are then prepended to the language tokens in conversations.
\begin{figure}
    \centering
    \includegraphics[width=\linewidth]{figures/arch.pdf}
    % \vspace{-2.2em}
    \caption{Model Architecture. The fMRI encoder maps fMRI to a series of fMRI tokens through our proposed neuroscience-informed attention. The large language model, with both fMRI and text tokens, will be trained by brain instruction tuning.}
    \label{fig:arch}
    \vspace{-1em}
\end{figure}

\subsection{Brain Instruction Tuning (BIT)}
To enable versatile fMRI-to-text decoding, an appropriate BIT dataset is required, yet no such dataset currently exists. To bridge this gap, we construct one based on the fact: MSCOCO images \cite{chen2015microsoft} serve as stimuli for fMRI recordings in the fMRI study \cite{allen2022massive}, and an abundance of datasets provide text annotations (e.g., VQA) for MSCOCO images. Using the images as intermediaries, we select those relevant to brain functions and pair the fMRI data with corresponding text annotations. For example, given an image of a billboard with annotated textual content, we can reasonably infer that when a subject perceives textual information (e.g., contents on the billboard), corresponding representations are encoded in the brain. This suggests the possibility of extracting such information from fMRI signals. We select datasets to fulfill various purposes, enabling the model to capture diverse aspects of semantic information embedded in fMRI signals, including visual perception \& scene understanding, language \& symbolic processing, memory \& knowledge retrieval and complex reasoning, which are considered among most fundamental and essential properties of human brains \cite{robertson2002memory,stenning2012human,wade2013visual,friederici2017language}.

\begin{figure}[h]
% \vspace{-0.5em}
    \centering
    \includegraphics[width=\linewidth]{figures/bit.pdf}
\vspace{-1.8em}
    \caption{Dataset Taxonomy in Brain Instruction Tuning.}
    \label{fig:bit}
% \vspace{-1em}
\end{figure}

\noindent\textbf{Perception \& Scene Understanding} As illustrated in Figure~\ref{fig:bit}, we begin by using caption tasks at both coarse and fine-grained levels to train the model’s ability to understand and summarize what the subject perceives visually \cite{chen2015microsoft,krause2017hierarchical}. Additionally, we incorporate QA tasks \cite{ren2015exploring,krishna2017visual,acharya2019tallyqa} to enhance the model's ability to retrieve and reason about visually perceived content.

\noindent\textbf{Memory \& Knowledge Retrieval} To go beyond tasks directly related to present visual perception, we construct the \emph{previous captioning} task, a memory-oriented task that challenges the model to caption images that the subject previously viewed, simulating memory recall processes. Furthermore, we aim to encode knowledge structures in human brains. The OK-VQA \cite{marino2019ok} and A-OKVQA \cite{schwenk2022okvqa} datasets include questions requiring external knowledge that is not present in the image but resides in human brains. For example, A photo of a hydrant may prompt the answer "firetruck," even though the firetruck is absent in the image. This association also reflects the way human cognition operates through a network of interconnected meanings, where one concept unconsciously triggers another. Such a process, which is called "slippage of the signifier" \cite{lacan2001ecrits, lacan1988seminar, miller2018four}, highlights the symbolic processes through which the brain constructs and retrieves meaning. 

\noindent\textbf{Language \& Symbolic Processing} In addition to the aforementioned OK-VQA and A-OKVQA datasets, which are also related to symbolic process, we further combine datasets of text recognition \cite{biten2019scene} and numerical reasoning \cite{acharya2019tallyqa} to facilitate this aspect.

\noindent\textbf{Complex Reasoning} Finally, we try to approximate the reasoning process that happens in human brains with datasets \cite{liu2023visual,wang2023see,li2018vqa} that require intricate logical and inferential processes. We expect these datasets to challenge the model to extract the reasoning process, drawing upon both visual understanding and abstract problem-solving, thus bridging perception, memory, and knowledge into a cohesive cognitive framework.

We ended up with a brain instruction tuning dataset consisting of $980,610$ conversations associated with fMRI recordings from $15$ datasets. Appendix~\ref{app:dataset} lists the instructions and other details for each dataset. The instruction tuning enables versatile fMRI-to-text decoding. In particular, the introduction of tasks like \textit{previous caption} empowers the model to perform a broader range of tasks beyond vision-related ones, which the previous model \cite{xia2024umbrae} fails.

\begingroup
\sisetup{
  table-format=2.2,  % 3 digits before the decimal, 2 after
  table-align-text-pre=false,
  propagate-math-font=true,
  table-number-alignment=center,
  detect-weight=true,detect-inline-weight=math
}
\begin{table*}[bp]
    \centering
    \vspace{-1.7em}
    \caption{Results of brain captioning. The CIDEr metric is scaled by a factor of 100 for consistency with Table~\ref{tab:caption} and baselines.}
    \label{tab:caption}
\vspace{0.1in}
    \resizebox{\linewidth}{!}{
    \begin{tabular}{lcSSSSSSSS}
    \toprule
 % \multirow{2}{*}{Method}&  \multirow{2}{*}{cross-subject}&\multicolumn{5}{c}{fMRI caption} & &  &\\
   {Method} & {\makecell{subject\\agnostic}}  &{{BLEU-1} $\uparrow$} & {BLEU-2 $\uparrow$} & {BLEU-3 $\uparrow$} & {{BLEU-4} $\uparrow$} &{METEOR $\uparrow$}&{ROUGE $\uparrow$}& {CIDEr $\uparrow$}&{SPICE $\uparrow$}\\
    \midrule
    SDRecon \cite{takagi2023high}    & {\xmark} &36.21 & 17.11 & 7.22 & 3.43   &10.03&  25.13&13.83 &5.02 \\
    OneLLM  \cite{han2024onellm}  & {\xmark} &47.04 & 26.97 & 15.49 & 9.51   &13.55&  35.05&22.99 & 6.26\\
    UniBrain \cite{mai2023unibrain}   & {\xmark} & {$-$}   & {$-$}    & {$-$}  & {$-$}     &16.90&  22.20& {$-$} & {$-$}\\
    BrainCap \cite{ferrante2023brain}  & {\xmark} &55.96 & 36.21 & 22.70 & 14.51   &16.68& 40.69&41.30 & 9.06\\
     BrainChat \cite{huang2024brainchat} & {\xmark}   &52.30& 29.20& 17.10& 10.70 &14.30& 45.70&26.10 & {$-$}\\
    UMBRAE \cite{xia2024umbrae}    & {\xmark} &59.44& 40.48& 27.66&19.03&19.45&  43.71&61.06&12.79\\
    \name{} (Ours)  & {\cmark} & \bfseries 61.75 &  \bfseries42.84 & \bfseries29.86&\bfseries21.24  & 
\bfseries 19.54 &\bfseries45.82 & 60.97  & 11.79\\
    \bottomrule
    \end{tabular}}
\end{table*}
\endgroup


To train the model with the BIT dataset, for each sample $\boldsymbol{v}$, we sample a multi-run conversation $X_t = (X_u^1, X_a^1, \cdots, X_u^T, X_a^T)$ from all conversations associated with it, where $T \geq 1$ represents the number of turns. $a$ indicates the message from the assistant and $u$ indicates the message is from the user. The training objective is to maximize the probability of the assistant's response only
$$
\arg\max_\theta p(X_a | X_v, X_{\text{inst}}) = \prod_{t=1}^T p({\color{magenta}X_a^t} | X_u^{\leq t}, X_a^{\le t }, X_\text{inst}, X_v)
$$
Figure~\ref{fig:chat} illustrates the chat template and the training objective. We freeze the weights of the LLM and only train the fMRI encoder since we want to preserve the LLM's language modeling prior and ensure a fair comparison with baselines such as \citet{xia2024umbrae}.

\noindent\textbf{Computational Complexity} According to the analysis in Appendix~\ref{app:complexity}, our model does not introduce additional complexity compared to previous methods \cite{scotti2024mindeye2, wang2024mindbridge}.


\begin{figure}[htbp]
\vspace{-0.8em}
\centering
\begin{minipage}{0.99\columnwidth}\vspace{0mm}    \centering
\begin{tcolorbox}[colback=white,colframe=gray,left=1pt,top=1pt,bottom=1pt]
\sffamily
\footnotesize	
  \texttt{<system message>}\\
  user: $X_v$, $X_\text{inst}$, $X_1^u$ \\
  assistant: {\color{magenta}$X_1^a$}\\
user: $X_2^u$\\
  assistant: {\color{magenta}$X_2^a$}\\
  $\cdots\cdots$
\end{tcolorbox}
\end{minipage}
\caption{The chat template used during instruction tuning, illustrating two turns of conversations. Two turns of conversations are shown. Tokens highlighted in {\color{magenta}magenta} are used for next-token prediction loss computation.}
\label{fig:chat}
\vspace{-1.2em}
\end{figure}


\section{Experiment Setup}
This section presents the experimental setup for comparing various confidence measures in selecting reliable responses for reasoning tasks. We also extend the evaluation to additional datasets and explore combining self-certainty with voting methods for improved response selection.

% from experiment xxx.
\subsection{Comparison of Confidence Measures} \label{sec:experiment_setup}  
To assess the effectiveness of different candidate formulations in Section~\ref{sec:candidate}, we employ them to select the most confident response from a set of \( N \) outputs generated by our base model, Llama-3.1-8B-Instruct \cite{dubey2024llama}. To mitigate the potential bias and data contamination arising from the model's training on publicly available datasets, we evaluate their performance using the LiveBench-Math dataset \cite{white2024livebench}, which was released after the model's deployment.

We begin by sampling 64 responses using \(\text{temperature} = 0.6\) and \(\text{top-p} = 0.9\), and subsequently create subsets comprising the first \(N = 4,8,16,32,64\) to perform Best-of-N selection. To ensure fairness, we test different measures' performance using the same set of samples. We mask out responses for which an answer cannot be extracted-primarily because these outputs do not adhere to the format instructions to facilitate latter comparisons with majority voting. We also include a baseline, FirstAns, which simply selects the first extractable answer from the 
\(N\) outputs. This baseline serves as a reference point for quantifying the performance improvements achieved by our candidate expression-based selection strategy. The evaluation is implemented based on the ZeroEval framework \cite{zeroeval}, which provides a unified structure for performance evaluation. We repeat the experiment five times and report the average accuracy as the overall performance. 

% As shown in Figure~\ref{fig:expression_selection}, the KL-divergence-inspired distributional confidence (self-certainty) achieves the highest accuracy on LiveBench-Math, particularly excelling at larger $N$, whereas other measures plateau or degrade beyond $N=16$.

\subsection{Validation on Additional Datasets and Combined Voting Methods} \label{sec:borda_setup}
% After confirming that self-certainty outperforms other confidence measures, we combine it with voting methods to further enhance performance and evaluate its effectiveness on a broader set of datasets. 
We conduct a series of experiments to evaluate the proposed self-certainty and Borda Voting methods against self-consistency, universal self-consistency (USC), greedy decoding, and FirstAns across various reasoning tasks.


The sampling strategy follows the procedures outlined in Section~\ref{sec:experiment_setup}. For USC, we use the template from the original paper \cite{chen2023universal} (with minor wording modifications, as shown in Appendix \ref{sec:usc_example}). To ensure a fair comparison, we assist USC in selecting the first valid response when it fails to choose one with an extractable answer. 

We evaluate different methods using the Llama-3.1-8B-Instruct model across the following benchmarks:
\begin{itemize}[leftmargin=*, itemsep=0pt, topsep=0pt]
    \item \textbf{Mathematical Reasoning}: We utilize the LiveBench-Math dataset \cite{white2024livebench}, the validation set of GSM8K dataset \cite{gsm} and the test set of MATH dataset \cite{MATH}.
    \item \textbf{Code Reasoning}: The CRUXEval-O benchmark \cite{gu2024cruxeval} is employed, which involves predicting the output of Python codes.
    \item \textbf{Code Generation}: We adopt the LiveCodeBench code generation benchmark \cite{jain2024livecodebench} to assess the improvements introduced by our methods.
\end{itemize}

\begin{figure}[t]
    \centering
    \includegraphics[width=0.99\linewidth]{fig/section1.pdf}
    % \vspace{-1.5em}
    \caption{Evaluation of Best-of-\(N\) selection accuracy on LiveBench-MATH across multiple confidence measures. KL achieves the best performance at larger \(N\), while other measures plateau or decline after \(N = 16\).} 
    \label{fig:expression_selection}
\end{figure}

For all test models and datasets, we employ Chain-of-Thought reasoning \cite{wei2022chain}, except for the code generation dataset. To evaluate the generalization of our measure across different training methodologies, particularly for the recent R1-series large reasoning models \cite{guo2025deepseek}, we test our approach on DeepSeek-R1-Distill-Llama-8B using the MATH dataset (Level 3). Given the increased reasoning time required by this model, we conduct a single trial for this experiment. To further validate and assess generalizability, we apply both USC and self-certainty to the Qwen-2.5-Coder-32B-Instruct model \cite{hui2024qwen2}, in addition to Llama-3.1-8B-Instruct, on the LiveCodeBench dataset.


\section{Results and Analysis}\label{sec:result}

\subsection{Self-Certainty}



\begin{figure*}[t]
    \centering
    \includegraphics[width=\linewidth]{fig/combined_confidence_new.png}
    \vspace{-1.em}
    \caption{Scatter plot showing various confidence measures against response length (measured in number of characters) in the LiveBench-Math dataset, using the Llama-3.1-8B-Instruct model with 64 samples per question. The figure demonstrates that, with the exception of self-certainty, all other measures exhibit a bias towards longer responses.} 
    \label{fig:scatter}
\end{figure*}

\paragraph{KL-Divergence-Inspired Distributional Confidence Outperforms Other Measures in Best-of-N Selection.}

The results, shown in Figure \ref{fig:expression_selection}, demonstrate that distributional confidence measures generally outperform perplexity when \(N \geq 16\). Among all candidate methods, KL divergence is the only measure that consistently improves as N increases to 32 and 64. This indicates that KL divergence serves as a more robust measure of confidence, offering better insight into the accuracy of responses. Equation~\ref{eq:sc} defines tokenwise self-certainty as the KL divergence from a uniform distribution, with an alternative empirical distribution evaluated in Appendix~\ref{sec:diff_dis}. The results confirm that KL with a uniform distribution, our original design, generalizes better.



\paragraph{Self-Certainty's Robustness to Reasoning Length in Response Selection.}
To understand why self-certainty outperforms other confidence measures in selecting better responses, we examine the relationship between reasoning length and confidence scores across different measures, as shown in Figure~\ref{fig:scatter}. The scatter plots reveal that longer reasoning lengths correlate with higher confidence scores in most metrics except self-certainty. This indicates that while other measures tend to favor samples with extended reasoning, self-certainty remains largely invariant to response length. This finding aligns with \citet{basu2020mirostat}'s observation that, under low \(p\) values, perplexity decreases as the output length increases. Unlike other metrics that may conflate verbosity with correctness, self-certainty provides a more unbiased assessment of response quality. This robustness ensures that models cannot manipulate the confidence measure by simply generating more extended but meaningless reasoning paths.

\paragraph{Self-Certainty Effectively Separates Correct and Incorrect Responses.}
We analyze the distribution of self-certainty and negative perplexity across correct, incorrect, and no-answer responses using Level 4 of the MATH dataset for a balanced comparison. Figure~\ref{fig:freq} presents our findings. The histogram shows that self-certainty in both correct and incorrect responses follows an approximately normal distribution, with the correct group consistently exhibiting a higher mean. In contrast, while perplexity is able to identify better results when \(N\) is small (also in Figure~\ref{fig:expression_selection}), it fails to distinguish between correct and incorrect responses when applied to the full dataset with multiple outputs per question. \citet{zhang2020trading} demonstrates that as perplexity declines, the quality of responses improves initially, then experiences a significant drop -- an observation consistent with our findings. Notably, perplexity tends to assign higher confidence to no-answer responses, which often arise from self-repetition, early stopping, or excessively long reasoning chains that fail to follow prompt instructions. Given that Llama-3.1-8B-Instruct has a relatively low no-answer rate (\(< 2\% \)) in simpler Level 1 MATH problems, we attribute the no-answer rate primarily to the limited capacity of the model. It is unsurprising that negative perplexity favors these outputs, as avoiding difficult questions or repeating oneself is a common failure mode, even for humans -- consistent with \citet{basu2020mirostat}, who showed that maximizing perplexity increases self-repetition. In contrast, self-certainty reliably assigns lower confidence scores to no-answer responses, effectively distinguishing them from correct answers. These observations are further strong evidence that self-certainty is a more effective measure of the certainty of a model as it is more closely correlated with the quality of responses.

\begin{table}[t]
\centering
\caption{Accuracy of different voting methods on the test set of MATH dataset using Llama-3.1-8B-Instruct. Self-certainty-based Borda voting outperforms other voting methods.}
\vspace{0.5em}
\label{tab:voting}
\small
\begin{tabular}{@{}lcc@{}}
\toprule
\textbf{Method} & \textbf{\(\boldsymbol{N = 8}\)} & \textbf{\(\boldsymbol{N = 64}\)} \\ 
\midrule
Majority & 58.60 & 63.40  \\
Average             & 46.92 & 32.94\\
Sum                 &  59.06 & 63.51 \\
Borda (\(p=0.5\))   & \textbf{59.08}  & 63.71  \\
Borda (\(p=1.2\))   & 58.86  & \textbf{64.10}  \\
\bottomrule
\end{tabular}
% \vspace{-1em}
\end{table}

\begin{table*}[ht]
    \centering
    \caption{Performance comparison of various methods across different datasets using Llama-3.1-8B-Instruct. Some USC results are omitted due to over 20\% of the data exceeding context window limits under the settings. Self-certainty consistently outperforms sampling, greedy decoding, and USC, while Borda Voting with the optimal parameter \(p\) delivers the best performance across all methods.
    }
    \vspace{0.5em}
    \label{tab:results}
    \footnotesize
    \begin{tabular}{lcc  cc  cc  cc c}
        \toprule
        \multirow{2}{*}{\textbf{Method}}  & \multicolumn{2}{c}{\textbf{LiveBench-Math}} & \multicolumn{2}{c}{\textbf{GSM8K}} & \multicolumn{2}{c}{\textbf{MATH}} & \multicolumn{2}{c}{\textbf{Crux}} & \multirow{2}{*}{\textbf{Avg.}}\\ 
       & $N=8$ & $N=32$ & $N=8$ & $N=64$ & $N=8$ & $N=64$ & $N=8$ & $N=64$ & \\
        \midrule
        Greedy            & \multicolumn{2}{c}{12.23} & \multicolumn{2}{c}{84.00} & \multicolumn{2}{c}{47.96} & \multicolumn{2}{c}{39.88} & 46.02\\
        FirstAns          & 17.66 & 17.66 & 82.08 & 82.08 & 49.08 & 49.09 & 42.93 & 42.93 & 47.94 \\
        \midrule
        USC                  & 21.08 &  -    & 87.32 & 85.65 & 54.66 & - &  43.78 & 41.25 & 51.19\\
        Self-consistency     & 22.50 & 26.25 & 89.42 & 90.99 & 58.60 & 63.40 & 47.58 & 50.42 & 56.15 \\
        \midrule            
        Self-certainty       & 20.87 & 22.01 & 87.32 & 88.90 & 54.63 & 56.70 & 45.38 & 45.83 & 52.71 \\
        - Borda ($p=0.3$)    & \textbf{23.69} & 26.47 & \textbf{89.57} & \textbf{91.07} & \textbf{59.04} & 63.60 & \textbf{47.94} & 50.42 & 56.48\\
        - Borda ($p=0.7$)    & 23.59 & 26.36 & 89.51 & 91.04 & \textbf{59.04} & 63.85 & 47.85 & 50.65 & 56.49\\
        - Borda ($p=1.2$)    & 23.21 & \textbf{26.69} & 89.51 & 90.95 & 58.86 & \textbf{64.10} & 47.93 & 50.85 & \textbf{56.51}\\
        - Borda ($p=2.0$)      & 22.45 & 26.41 & 89.13 & 90.90 & 57.94 & 60.02 & 47.25 & \textbf{51.23} & 55.67 \\ 
        \bottomrule
    \end{tabular}
\end{table*}

\begin{figure*}[t]
    \centering
    \includegraphics[width=0.99\linewidth]{fig/four_datasets_line_plots_new.pdf}
    % \vspace{-1.5em}
    \caption{Performance evaluation across four datasets employing different strategies with Llama-3.1-8B-Instruct. These line graphs demonstrates the strong scaling ability of both self-certainty and Borda voting.}
    \label{fig:four_datasets}
\end{figure*}

\subsection{Self-Certainty and Voting}\label{sec:exp_voting}
\paragraph{Borda Voting in Combination with Self-Certainty.}
As discussed in Section \ref{sec:vote}, when responses contain explicit answers, self-certainty can be integrated with voting methods to enhance overall accuracy. We evaluate the effectiveness of Borda voting for combination voting in such cases, comparing it against majority voting, average self-certainty, and sum self-certainty on the MATH dataset, as shown in Table~\ref{tab:voting}. Our results indicate that self-certainty-based Borda voting outperforms other voting methods.



\paragraph{Performance Comparison Across Four Datasets.}
We examine the scaling properties of self-certainty and self-certainty-based Borda voting in Figure~\ref{fig:four_datasets}. The results indicate that self-certainty significantly outperforms both regular sampling and greedy decoding. Moreover, its performance improves considerably as $N$ increases, demonstrating that self-certainty, as a measure of the model’s confidence in its responses, provides valuable insight into output correctness.
Additionally, Borda voting demonstrates better performance compared to self-consistency under various settings of \(p\) and \(N\) across all four datasets. This suggests that the self-certainty measure enhances the accuracy of the final-answer-based voting method by providing useful ranking information. 



\paragraph{Optimizing the Borda Parameter \(p\) for Different \(N\).}
To investigate the relationship between the Borda parameter \(p\) in Equation~\ref{eq:borda_vote} and the efficiency of the selection method, we plot line charts in Figure~\ref{fig:math_p}, showing the performance of different selection methods across varying sample sizes \(N\). The result reveals that the optimal \(p\) increases from 0.5 to 1.2 as \(N\) increases from 8 to 64, suggesting that stronger control from self-certainty is needed with a larger \(N\). For general use cases, grid search remains the most effective approach for determining the optimal $p$. Alternatively, a simple heuristic is to set \(p = 0.3\) when \(N\leq 16\) and \( p = 1.2\) when \(N\geq 32\), though this rule of thumb may vary depending on the model and the complexity of the questions.

\begin{figure}[t]
    \centering
    \includegraphics[width=0.48\textwidth]{fig/math_p.pdf}
    \vspace{-1em}
    \caption{Performance of Borda voting on the MATH dataset using Llama-3.1-8B-Instruct, with varying \( p \) and \( N \). For each $N$, accuracy initially increases with $p$ reaches a peak, and then declines. The optimal \( p \) depends on \( N \). Note that self-consistency corresponds to Borda voting with \( p = 0 \).}
    \label{fig:math_p}
\end{figure}

\subsection{Generalization}
\paragraph{Generalization of Self-Certainty on Open-Ended Generation Tasks.}
Self-consistency struggles with creative, open-ended tasks such as code generation, where each sample produces a unique answer. In such cases, it defaults to standard sampling. USC and our method self-certainty offer solutions to this limitation. We compare self-certainty with USC on the code generation task of LiveCodeBench (Figure~\ref{fig:livecode}). Our findings reveal that USC underperforms compared to greedy decoding on the Llama-3.1-8B-Instruct model, likely due to the model’s constrained ability to recognize consistency. This observation is reinforced by results from the larger Qwen model, where USC successfully outperforms greedy decoding. In contrast, self-certainty consistently outperforms greedy decoding on both models and surpasses USC on the larger Qwen-2.5-Coder-32B-Ins. Additionally, the performance of self-certainty improves as \(N\) increases.

\begin{figure}[t]
    \centering
    \includegraphics[width=0.48\textwidth]{fig/livecode.pdf}
    % \vspace{-1em}
    \caption{Comparison of self-certainty and USC on the LiveCodeBench code generation task. The results show that self-certainty outperforms USC and greedy decoding on both Llama-3.1-8B-Instruct and Qwen-2.5-Coder-32B-Ins models, with performance improving as \(N\) increases.}
    \label{fig:livecode}
\end{figure}

\paragraph{Generalization of Self-Certainty on Reasoning Models.}
Recent research on DeepSeek-R1 \citep{guo2025deepseek} demonstrates that rule-based reinforcement learning and long-chain-of-thought (CoT) can significantly enhance the reasoning capabilities of LLMs. We evaluate the generalization of self-certainty on such reasoning models, with results for DeepSeek-R1-Distill-Llama-8B presented in Table~\ref{tab:distill}. Our findings show that self-certainty consistently outperforms both greedy decoding and sampling, with performance further improving as $N$ increases on reasoning models. Additionally, Borda voting with self-certainty surpasses the performance of self-consistency with proper \(p\). These results reinforce previous observations, highlighting the robustness of our methods across various fine-tuning techniques.

\begin{table}[t]
\centering
\caption{Accuracy of various methods on the Level 3 test set of the MATH dataset using DeepSeek-R1-Distill-Llama-8B (single trial). Self-certainty outperforms Greedy and FirstAns, while Borda Voting with an appropriate \(p\) surpasses self-consistency.}
\label{tab:distill}
\vspace{0.5em}
\small
\begin{tabular}{@{}lccc@{}}
\toprule
\textbf{Method} & \textbf{\(\boldsymbol{N = 4}\)} & \textbf{\(\boldsymbol{N = 16} \)} & \textbf{\(\boldsymbol{N = 64}\)} \\
\midrule
Greedy    & 77.54  &  77.54 & 77.54 \\
FirstAns            & 81.17    & 81.43 & 81.43\\
\midrule
Self-consistency    & 83.64    & 86.47 & 87.62\\
\midrule
Self-certainty      & 83.29 & 83.73 & 84.08  \\
- Borda ($p=0.3$)   & 84.79  & 87.00 & 87.80  \\
- Borda ($p=0.7$)   & 84.70 & 86.91  & 87.62\\
- Borda ($p=1.2$)   & 84.62  & 87.00 & 88.06\\
- Borda ($p=2.0$)   & 83.29  & 87.00 & 87.98\\
\bottomrule
\end{tabular}
\vspace{-1em}
\end{table}

% \xuandong{zhexian figure without self-consistency. Greedy, first answer, upper-bound, self-certainty, borada for five datasets}

\section{Discussion and Future Research}
While self-certainty proves effective in evaluating model outputs for open-ended generation tasks and demonstrates stable scalability, it has several limitations.

First, self-certainty alone underperforms self-consistency on questions with unique answers (Section~\ref{sec:result}). Although self-consistency generally achieves better performance, recent studies suggest that properly trained ORMs and PRMs can surpass it \cite{lightman2023let,uesato2022solving}. Current reward model training fine-tunes a base model to rate sentences based on token probabilities \cite{wang2024math}. A key insight from our findings is that treating the softmaxed logits of LLMs as a full probability distribution, rather than relying solely on individual token probabilities, leads to more robust certainty measurements. In self-certainty, using KL divergence between the output distribution and a uniform distribution provides greater stability than averaging log probabilities, suggesting that integrating this approach could enhance reward model effectiveness.

Second, this study explores a limited set of formulations for distributional confidence and Borda voting. The default choice of the averaging function for $F$ in Equation~\ref{eq:DC} may not be optimal, and a broader selection of $F$ could further improve self-certainty’s accuracy. Likewise, the power function used for vote distribution in Equation~\ref{eq:borda_vote} for Borda voting is intuitive but may not be the most effective formulation.

Beyond these limitations, self-certainty presents exciting opportunities for future research. It not only encourages rethinking reward model designs but also offers potential applications in test-time scaling techniques \cite{snell2024scaling}, potentially reducing computational costs. Additionally, self-certainty could be leveraged for data labeling and reinforcement learning tasks \cite{bai2022training,ouyang2022training}. By maximizing token-wise self-certainty, we may enable more autonomous and efficient learning systems, paving the way for advancements in both model performance and computational efficiency.

% Beyond encouraging a remodeling of the reward model, self-certainty opens up diverse future research directions. Some test-time compute scaling techniques rely on reward models to select superior reasoning and planning paths \cite{snell2024scaling}. Replacing the reword model with self-certainty can substantially reduce the heavy computational cost of training rewards without significantly compromising performance. Moreover, it is also possible to apply Self-certainty to data labeling and reinforcement learning tasks. Given the robustness of self-certainty and the fact that the assigned probability of selected tokens does not directly determine it, it creates new opportunities for developing a fully autonomous learning system. For instance, by defining the loss function to minimize the token-wise self-certainty of selected tokens and iteratively optimizing this measure, self-certainty offers a promising approach to enhancing model performance while ensuring greater autonomy and computational efficiency.



\section{Conclusion}
In this paper, we introduce self-certainty and self-certainty-based Borda voting as novel approaches for evaluating and enhancing model response performance. Self-certainty functions as an internal measure of response quality, demonstrating robustness in several key aspects. Compared to traditional scoring methods, such as average log probability and perplexity, it offers superior scalability when applied to Best-of-N selection. Additionally, the ranking information provided by self-certainty improves chain-of-thought reasoning and outperforms universal self-consistency (USC) in code generation tasks. Its stability, flexibility, and generalizability make it applicable across a wide range of domains, with the potential to enhance the autonomous learning capabilities of LLMs. 

% \section*{Impact Statement}
% This research introduces self-certainty, a novel confidence measure that exhibits a stronger correlation with response quality than traditional probability-based metrics, such as perplexity, by capturing distributional-level information. By leveraging the inherent probabilistic structure of LLMs, self-certainty offers a lightweight, scalable, and effective alternative to reward models for inference-time response selection, reducing the computational and annotation burdens associated with existing Best-of-N selection techniques.  

% Beyond its immediate application in improving reasoning accuracy, self-certainty has the potential to transform self-evaluation and autonomous learning in large language models. Unlike reward models, which require extensive fine-tuning and human supervision, self-certainty enables models to assess their own outputs with minimal additional computation. This advancement paves the way for more efficient self-improving AI systems that can refine their responses without explicit external feedback.  

% Furthermore, self-certainty generalizes beyond structured problem-solving tasks, offering a robust mechanism for quality assessment in open-ended text generation, including creative writing, dialogue generation, and code synthesis. By challenging conventional reward model-based selection frameworks, our approach not only enhances the reliability of LLM outputs but also lays the groundwork for more interpretable and transparent AI decision-making. The adoption of self-certainty in AI research and deployment has the potential to make advanced reasoning capabilities more accessible, reducing reliance on computationally expensive techniques while maintaining high levels of accuracy and adaptability.

\section*{Acknowledgment}  

We appreciate the valuable discussions with Kexun Zhang and Danqing Wang. We also thank the Center for AI Safety for providing substantial compute resources for this project.
\bibliography{example_paper}
\bibliographystyle{icml2025}


%%%%%%%%%%%%%%%%%%%%%%%%%%%%%%%%%%%%%%%%%%%%%%%%%%%%%%%%%%%%%%%%%%%%%%%%%%%%%%%
%%%%%%%%%%%%%%%%%%%%%%%%%%%%%%%%%%%%%%%%%%%%%%%%%%%%%%%%%%%%%%%%%%%%%%%%%%%%%%%
% APPENDIX
%%%%%%%%%%%%%%%%%%%%%%%%%%%%%%%%%%%%%%%%%%%%%%%%%%%%%%%%%%%%%%%%%%%%%%%%%%%%%%%
%%%%%%%%%%%%%%%%%%%%%%%%%%%%%%%%%%%%%%%%%%%%%%%%%%%%%%%%%%%%%%%%%%%%%%%%%%%%%%%
\newpage
\appendix
\onecolumn




\section{More Experiment Results}
\subsection{Oracle Best-of-N Selection Performance and Scaling Effects on LiveCodeBench}
In our experiment described in Section~\ref{sec:borda_setup}, we evaluate the performance of Llama-3.1-8B-Instruct and compare Borda voting and self-certainty against the upper bound of Best-of-N selection methods, as shown in Figure~\ref{fig:five_datasets_more}. While both methods demonstrate continued improvement as $N$ increases, they remain significantly outperformed by the Oracle selection method, which assumes perfect knowledge of the correct answer.
\begin{figure}[H]
    \centering
    \includegraphics[width=0.99\textwidth]{fig/five_datasets_line_plots.pdf}
    \caption{Performance across five datasets using different strategies with Llama-3.1-8B-Instruct. The oracle selection method significantly outperforms the other strategies. Additionally, both Borda voting and self-certainty demonstrate strong scaling effects.}
    \label{fig:five_datasets_more}
\end{figure}

\subsection{Evaluation of Methods Across Difficulty Levels on the MATH Dataset}
We conduct experiments to evaluate different methods across varying difficulty levels of reasoning problems. Figure~\ref{fig:level} presents the performance of various methods on the MATH dataset at different difficulty levels. As question difficulty increases, the scaling effect of Borda voting and self-certainty becomes more pronounced, demonstrating their effectiveness in handling more challenging reasoning tasks.


\begin{figure}[H]
    \centering
    \includegraphics[width=0.48\textwidth]{fig/level.pdf}
    \caption{Comparison of evaluation methods on the MATH test dataset across increasing difficulty levels using Llama-3.1-8B-Instruct. The four bars in each difficulty bin correspond to an increasing choice of N in the Best-of-N selection (8, 16, 32, and 64 generations). Performance differences among settings become more pronounced as the difficulty level increases.}
    \label{fig:level}
\end{figure}

\subsection{Replacing Uniform Distribution with Empirical Distribution}\label{sec:diff_dis}

\begin{figure}[t]
    \centering
    \includegraphics[width=0.7\textwidth]{fig/token_frequency.pdf}
    \caption{Frequency of the top 20 tokens in Llama-3.1-8B-Instruct responses to MATH training set questions (generated eight times per question).}
    \label{fig:token_freq}
\end{figure}

\begin{table}[ht]
\centering
\caption{Accuracy of various self-certainty definitions for Best-of-N selection on the MATH and GSM8K test sets using Llama-3.1-8B-Instruct (single trial). The empirical distribution is derived by sampling from the MATH training dataset. While the empirical self-certainty results are comparable to those based on a uniform distribution for the MATH test set, it is significantly outperformed by the latter, likely due to a distributional shift.}
\label{tab:emp}
\vspace{0.5em}
\small
\begin{tabular}{@{}lccccc@{}}
\toprule
\multirow{2}{*}{\textbf{Base Distribution}} & \multicolumn{2}{c}{\textbf{MATH}} & \multicolumn{2}{c}{\textbf{GSM8K}} \\
& \(N = 8\) & \(N = 64\) & \(N = 8\) & \(N = 64\) \\
\midrule
Uniform   & 54.60   & 56.46 & 87.19 & 88.55\\
Empirical      & 54.70 & 56.78 & 85.97 & 86.35  \\
\bottomrule
\end{tabular}
\end{table}
In Equation~\ref{eq:sc}, we define tokenwise self-certainty as the KL divergence between the generated token distribution and a uniform distribution, which quantifies deviation from random sampling. An alternative approach replaces the uniform distribution with an empirical token distribution estimated from training data. To evaluate the impact of this modification, we conduct the following experiment.

We first estimate token frequencies in the MATH training set by generating eight responses per question and averaging token occurrences. The resulting empirical distribution is approximated from these frequencies, with the 20 most frequent tokens shown in Figure~\ref{fig:token_freq}. We then compute KL divergence between the generated token distribution and the empirical distribution, using this as the self-certainty measure for Best-of-N selection. This experiment was conducted for a single trial, with all other setup parameters as described in Section~\ref{sec:experiment_setup}.

Results show that replacing the uniform distribution with the empirical distribution has minimal impact on MATH test accuracy but leads to a noticeable performance drop on GSM8K, suggesting a sensitivity to distributional shifts. Thus, we recommend retaining the uniform distribution in Equation~\ref{eq:sc} for improved generalization.


\section{Case Study and Examples}
In this section, we present several examples to illustrate the practical behavior of self-certainty and universal self-consistency (USC). These examples are drawn from Llama-3.1-8B-Instruct generating responses to questions from the GSM8K and LiveCodeBench code generation datasets.

\subsection{Self-certainty}
In this example, self-certainty correctly assigns higher confidence to the correct answer. Notably, in Response 0, the model does not fully adhere to the instruction to enclose all reasoning within the specified JSON field. Instead, it follows its trained reasoning process first and then summarizes the reasoning path in the \texttt{reason} field. This approach appears to be a compromise between the model's training objective and the given instruction, balancing structured formatting with its natural reasoning process.
\begin{tcolorbox}[colback=white,colframe=gray!50!black,title=Self-certainty Example]
USER: \\
\#\# Question:\\
Kylar went to the store to buy glasses for his new apartment. One glass costs \$5, but every second glass costs only 60\% of the price. Kylar wants to buy 16 glasses. How much does he need to pay for them?\\
\#\# Instruction: \\
Please answer this question by first reasoning and then providing your answer.\\
Present your reasoning and solution in the following json format. 
Please show your final answer in the answer field, e.g.,``answer": ``42". `json\{ ``reasoning": ``\_\_\_",  ``answer": ``\_\_\_"\}\\

Response 0:\\
ASSISTANT:\\
\#\#Step 1: Calculate the cost of the first glass.\\
The cost of the first glass is \$5.\\
\#\#Step 2: Calculate the cost of every second glass.\\
Every second glass costs 60\% of the price of the first glass, which is \( 0.6 \times 5 = 3 \) dollars.\\
\#\#Step 3: Determine the number of glasses that are full price and the number of glasses that are discounted.\\
Kylar wants to buy 16 glasses. Since every second glass is discounted, we can divide 16 by 2 to get the number of full-price glasses and the number of discounted glasses. \( \frac{16}{2} = 8 \) full-price glasses and 8 discounted glasses.\\
\#\#Step 4: Calculate the total cost of the full-price glasses.\\
8 full-price glasses at \$5 each is \( 8 \times 5 = 40 \) dollars.\\
\#\#Step 5: Calculate the total cost of the discounted glasses.\\
8 discounted glasses at \$3 each is \( 8 \times 3 = 24 \) dollars.\\
\#\#Step 6: Calculate the total cost of all the glasses.\\
The total cost is the sum of the cost of the full-price glasses and the cost of the discounted glasses. \$40 + \$24 = \$64.\\
\#\#Step 7: Determine the total amount Kylar needs to pay for the glasses.\\
The total amount Kylar needs to pay is \$64.\\
The final answer is: 
\{``reasoning": ``Kylar wants to buy 16 glasses. The first glass costs \$5, and every second glass costs 60\% of the price of the first glass, which is \$3. He buys 8 full-price glasses at \$5 each and 8 discounted glasses at \$3 each. The total cost of the full-price glasses is \$40, and the total cost of the discounted glasses is \$24. The total cost is \$40 + \$24 = \$64.", ``answer": "64"\}\\
\textbf{(Self-certainty: 17.81)}\\

Response 1: \\
ASSISTANT:\\
\{``reasoning": ``To find the total cost, we need to calculate the cost of the first glass and the cost of the remaining glasses. The first glass costs \$5. The second glass costs 60\% of \$5, which is 0.6 * 5 = \$3. The third glass costs 60\% of \$3, which is 0.6 * 3 = \$1.80. We can see that the price of each subsequent glass is 60\% of the previous glass. So, the price of the glasses follows a geometric sequence: 5, 3, 1.80, 1.08, \(\dots\) . To find the cost of 16 glasses, we can use the formula for the sum of a geometric series: \(S = a * (1 - r^n) / (1 - r)\), where a is the first term, r is the common ratio, and n is the number of terms. In this case, a = 5, r = 0.6, and n = 16. Plugging in the values, we get \(S = 5 * (1 - 0.6^16) / (1 - 0.6)\). Calculating the value of \(0.6^{16}\), we get 0.0000655. Now, we can plug this value back into the formula: \(S = 5 * (1 - 0.0000655) / (1 - 0.6) = 5 * 0.9999345 / 0.4 = 5 * 2.499335 = 12.496675\). Rounding to two decimal places, the total cost is \$12.50.", ``answer": ``12.50"\} \\
\textbf{(Self-certainty: 17.13)}\\

\end{tcolorbox}

\subsection{Universal Self-Consistency}\label{sec:usc_example}
The first example is taken from the GSM8K dataset, while the second comes from the LiveCodeBench dataset. The prompt has been adapted from \citet{chen2023universal}. We observe that in Example 1, the model evaluates the answers step by step to assess consistency. In contrast, the model jumps directly to the conclusion in Example 2. This difference may stem from the clarity of the task: in a MATH problem with a definite answer, the model can logically assess each step of the reasoning process, as seen in Example 1. However, for the LiveCodeBench example, where the task involves determining the most consistent code, the model struggles to elaborate on the reasoning process and relies on direct reasoning to arrive at a conclusion.


\begin{tcolorbox}[colback=white,colframe=gray!50!black,title=USC Prompt Example 1]
USER: 

I have generated the following responses to the question: Elise is learning to write and decides to keep re-writing the alphabet until she knows it. She writes it in full twice, writes half of it once, then re-writes everything she has already written. How many letters has Elise written in total?\\


Response 0: \{``reasoning": "To find the total number of letters Elise has written, we first need to determine the number of letters in the alphabet. There are 26 letters in the English alphabet. Elise writes it in full twice, which means she writes 26 letters twice, so 26 * 2 = 52 letters. Then, she writes half of it once, which is 26 / 2 = 13 letters. So, the total number of letters written so far is 52 + 13 = 65. Finally, she re-writes everything she has already written, which is 65 letters. Therefore, the total number of letters Elise has written is 65 + 65 = 130.",``answer": ``130"\}\\

\(\cdots\)\\
Response 3: \{``reasoning": ``To solve this problem, we need to calculate the total number of letters Elise has written. First, she writes the alphabet twice, which has 26 letters. So, she has written 26 + 26 = 52 letters. Then, she writes half of the alphabet once, which is 13 letters. So, she adds 13 more letters to the total. Now, she re-writes everything she has already written, which means she re-writes 52 + 13 = 65 letters. To find the total number of letters Elise has written, we need to add the number of letters she wrote initially (52) and the number of letters she re-wrote (65). The total number of letters is 52 + 65 = 117.",    ``answer": ``117"\}\\

\(\cdots\)\\

Evaluate these responses.
Select the most consistent response based on majority consensus.
Format your answer as "The most consistent response is Response X" (without quotes).\\


ASSISTANT:

To evaluate the responses, I will analyze the answers and reasoning provided by each response.Response 0 and Response 1 have the same answer (130) and similar reasoning. \(\cdots \).The most consistent response is Response 0.
%gsm8k-main-test-#587
\end{tcolorbox}

\begin{tcolorbox}[colback=white,colframe=gray!50!black,title=USC Prompt Example 2]
USER: 

I have generated the following responses to the question: Slavic is preparing a present for a friend's birthday. He has an array \( a \) of \( n \) digits, and the present will be the product of all these digits. Because Slavic is a good kid who wants to make the biggest product possible, he wants to add 1 to exactly one of his digits.

\textbf{What is the maximum product Slavic can make?}

\textbf{Input:}
\begin{itemize}
    \item The first line contains a single integer \( t \) (\( 1 \leq t \leq 10^4 \)) — the number of test cases.
    \item The first line of each test case contains a single integer \( n \) (\( 1 \leq n \leq 9 \)) — the number of digits.
    \item The second line of each test case contains \( n \) space-separated integers \( a_i \) (\( 0 \leq a_i \leq 9 \)) — the digits in the array.
\end{itemize}

\textbf{Output:}
\begin{itemize}
    \item For each test case, output a single integer — the maximum product Slavic can make, by adding 1 to exactly one of his digits.
\end{itemize}

Sample Input: \(\cdots\)\\
Sample Output: \(\cdots\)\\


\textbf{Response 0:}

\begin{verbatim}
```python
# YOUR CODE HERE
import sys

def max_product(n, arr):
    min_digit = min(arr)
    max_product = 1
    for i in range(n):
        if arr[i] == min_digit:
            arr[i] += 1
            break
    for num in arr:
        max_product *= num
    return max_product

t = int(input())
for _ in range(t):
    n = int(input())
    arr = list(map(int, input().split()))
    print(max_product(n, arr))
```
\end{verbatim}
This code first finds the minimum digit in the array and increments it by 1. Then, it calculates the product of all digits in the array. The time complexity is O(n), where n is the number of digits in the array.\\
\(\cdots\)\\

Evaluate these responses.
Select the most consistent response based on majority consensus.
Format your answer as "The most consistent response is Response X" (without quotes).\\

ASSISTANT:

The most consistent response is Response 0.
\end{tcolorbox}


% You can have as much text here as you want. The main body must be at most $8$ pages long.
% For the final version, one more page can be added.
% If you want, you can use an appendix like this one.  

% The $\mathtt{\backslash onecolumn}$ command above can be kept in place if you prefer a one-column appendix, or can be removed if you prefer a two-column appendix.  Apart from this possible change, the style (font size, spacing, margins, page numbering, etc.) should be kept the same as the main body.
%%%%%%%%%%%%%%%%%%%%%%%%%%%%%%%%%%%%%%%%%%%%%%%%%%%%%%%%%%%%%%%%%%%%%%%%%%%%%%%
%%%%%%%%%%%%%%%%%%%%%%%%%%%%%%%%%%%%%%%%%%%%%%%%%%%%%%%%%%%%%%%%%%%%%%%%%%%%%%%


\end{document}


% This document was modified from the file originally made available by
% Pat Langley and Andrea Danyluk for ICML-2K. This version was created
% by Iain Murray in 2018, and modified by Alexandre Bouchard in
% 2019 and 2021 and by Csaba Szepesvari, Gang Niu and Sivan Sabato in 2022.
% Modified again in 2023 and 2024 by Sivan Sabato and Jonathan Scarlett.
% Previous contributors include Dan Roy, Lise Getoor and Tobias
% Scheffer, which was slightly modified from the 2010 version by
% Thorsten Joachims & Johannes Fuernkranz, slightly modified from the
% 2009 version by Kiri Wagstaff and Sam Roweis's 2008 version, which is
% slightly modified from Prasad Tadepalli's 2007 version which is a
% lightly changed version of the previous year's version by Andrew
% Moore, which was in turn edited from those of Kristian Kersting and
% Codrina Lauth. Alex Smola contributed to the algorithmic style files.
