%%%%%%%% ICML 2025 EXAMPLE LATEX SUBMISSION FILE %%%%%%%%%%%%%%%%%

\documentclass{article}

% Recommended, but optional, packages for figures and better typesetting:
\usepackage{microtype}
\usepackage{graphicx}
\usepackage{subfigure}
\usepackage{booktabs} % for professional tables
\usepackage{multirow}
% hyperref makes hyperlinks in the resulting PDF.
% If your build breaks (sometimes temporarily if a hyperlink spans a page)
% please comment out the following usepackage line and replace
% \usepackage{icml2025} with \usepackage[nohyperref]{icml2025} above.
\usepackage{hyperref}
\usepackage{tcolorbox}
% Attempt to make hyperref and algorithmic work together better:
\newcommand{\theHalgorithm}{\arabic{algorithm}}

% Use the following line for the initial blind version submitted for review:
% \usepackage{icml2025}

% If accepted, instead use the following line for the camera-ready submission:
\usepackage[accepted]{icml2025}

% For theorems and such
\usepackage{amsmath}
\usepackage{amssymb}
\usepackage{mathtools}
\usepackage{amsthm}
\usepackage{xspace}
\usepackage{enumitem}
\usepackage{makecell}
% if you use cleveref..
\usepackage[capitalize,noabbrev]{cleveref}

%%%%%%%%%%%%%%%%%%%%%%%%%%%%%%%%
% THEOREMS
%%%%%%%%%%%%%%%%%%%%%%%%%%%%%%%%
\theoremstyle{plain}
\newtheorem{theorem}{Theorem}[section]
\newtheorem{proposition}[theorem]{Proposition}
\newtheorem{lemma}[theorem]{Lemma}
\newtheorem{corollary}[theorem]{Corollary}
\theoremstyle{definition}
\newtheorem{definition}[theorem]{Definition}
\newtheorem{assumption}[theorem]{Assumption}
\theoremstyle{remark}
\newtheorem{remark}[theorem]{Remark}

% Todonotes is useful during development; simply uncomment the next line
%    and comment out the line below the next line to turn off comments
%\usepackage[disable,textsize=tiny]{todonotes}
\usepackage[textsize=tiny]{todonotes}

\newcommand{\xuandong}[1]{{\color{orange} {[Xuandong: #1]}}}
% The \icmltitle you define below is probably too long as a header.
% Therefore, a short form for the running title is supplied here:
\icmltitlerunning{Scalable Best-of-N Selection for Large Language Models via Self-Certainty}

\begin{document}

\twocolumn[
\icmltitle{Scalable Best-of-N Selection for Large Language Models via Self-Certainty}

% It is OKAY to include author information, even for blind
% submissions: the style file will automatically remove it for you
% unless you've provided the [accepted] option to the icml2025
% package.

% List of affiliations: The first argument should be a (short)
% identifier you will use later to specify author affiliations
% Academic affiliations should list Department, University, City, Region, Country
% Industry affiliations should list Company, City, Region, Country

% You can specify symbols, otherwise they are numbered in order.
% Ideally, you should not use this facility. Affiliations will be numbered
% in order of appearance and this is the preferred way.
\icmlsetsymbol{equal}{*}

\begin{icmlauthorlist}
\icmlauthor{Zhewei Kang}{equal,yyy}
\icmlauthor{Xuandong Zhao}{equal,yyy}
\icmlauthor{Dawn Song}{yyy}
%\icmlauthor{}{sch}
%\icmlauthor{}{sch}
\end{icmlauthorlist}

\icmlaffiliation{yyy}{University of California, Berkeley}
% \icmlaffiliation{comp}{Company Name, Location, Country}
% \icmlaffiliation{sch}{School of ZZZ, Institute of WWW, Location, Country}

\icmlcorrespondingauthor{Zhewei Kang}{waynekang@berkeley.edu}
\icmlcorrespondingauthor{Xuandong Zhao}{xuandongzhao@berkeley.edu}

% You may provide any keywords that you
% find helpful for describing your paper; these are used to populate
% the "keywords" metadata in the PDF but will not be shown in the document
\icmlkeywords{Machine Learning, ICML}

\vskip 0.3in
]

% this must go after the closing bracket ] following \twocolumn[ ...

% This command actually creates the footnote in the first column
% listing the affiliations and the copyright notice.
% The command takes one argument, which is text to display at the start of the footnote.
% The \icmlEqualContribution command is standard text for equal contribution.
% Remove it (just {}) if you do not need this facility.

%\printAffiliationsAndNotice{}  % leave blank if no need to mention equal contribution
\printAffiliationsAndNotice{\icmlEqualContribution} % otherwise use the standard text.

\begin{abstract}
Best-of-N selection is a key technique for improving the reasoning performance of Large Language Models (LLMs) through increased test-time computation. Current state-of-the-art methods often employ computationally intensive reward models for response evaluation and selection. Reward-free alternatives, like self-consistency and universal self-consistency, are limited in their ability to handle open-ended generation tasks or scale effectively. To address these limitations, we propose \emph{self-certainty}, a novel and efficient metric that leverages the inherent probability distribution of LLM outputs to estimate response quality without requiring external reward models. 
We hypothesize that higher distributional self-certainty, aggregated across multiple samples, correlates with improved response accuracy, as it reflects greater confidence in the generated output. 
Through extensive experiments on various reasoning tasks, we demonstrate that self-certainty (1) scales effectively with increasing sample size $N$, akin to reward models but without the computational overhead; (2) complements chain-of-thought, improving reasoning performance beyond greedy decoding; and (3) generalizes to open-ended tasks where traditional self-consistency methods fall short. Our findings establish self-certainty as a practical and efficient way for improving LLM reasoning capabilities. The code is available at \url{https://github.com/backprop07/Self-Certainty}
\end{abstract}

Gaussian Processes (GPs) \citep{kolmogorov1940wienersche,rasmussen2006gaussian} are an important class of stochastic processes used in machine learning and statistics, with use cases including spatial data analysis \citep{liu2021missing}, time series forecasting \citep{girard2002gaussian}, bioinformatics \citep{luo2023diseasegps} and Bayesian optimization \citep{frazier2018tutorial}. GPs offer a non-parametric framework for modeling distributions over functions, enabling both flexibility and uncertainty quantification. These capabilities, combined with the ability to incorporate prior knowledge and specify relationships by choice of kernel function, make Gaussian Processes effective for both regression and classification.

However, GPs have substantial computational and memory bottlenecks. Both training and inference require computing the action of the inverse kernel Gram matrix, while training requires computing its log-determinant: both are $O(n^3)$ operations with sample size $n$. Further, storing the full Gram matrix requires $O(n^2)$ memory. These bottlenecks require scalable approximations for larger datasets.

Structured Kernel Interpolation (SKI) \cite{wilson2015kernel} helps scale Gaussian Processes (GPs) to large datasets by approximating the kernel matrix using interpolation on a set of inducing points. For stationary kernels, this requires $O(n+m \log m)$ computational complexity. The core idea is to express the original kernel as a combination of interpolation functions and a kernel matrix defined on a set of inducing points. However, despite its effectiveness, popularity (over $600$ citations, a large number for a GP paper) and high quality software availability (\cite{gardner2018gpytorch} has 3.5k stars on github), it currently lacks theoretical analysis. A key initial question is, given a fixed error bound for the SKI Gram matrix and use of cubic convolutional interpolation, how many inducing points are required to achieve that error bound? Given the required value of $m$ as a function of $n$, for what error tolerance is $O(n+m\log m)$ still linear? Following this, what do these errors imply for hyperparameter estimation and posterior inference?

\begin{table*}[h]
\centering
\begin{tabular}{|l|l|}
\hline
\textbf{Quantity} & \textbf{Bound} \\
\hline
SKI kernel error & $O(\frac{c^{2d}}{m^{3/d}})$ \\
\hline
SKI Gram matrix error & $O(\frac{nc^{2d}}{m^{3/d}})$ \\
\hline
SKI cross-kernel matrix error & $O(\frac{\max(n,T)c^{2d}}{m^{3/d}})$ \\
\hline
SKI score function error & $O(\frac{\sqrt{p}n^{2}c^{4d}}{m^{3/d}})$ \\
\hline
SKI posterior mean error & $O(c^{2d}\frac{\max(T,n)+\sqrt{Tn}n}{m^{3/d}})$ \\
\hline
SKI posterior covariance error & $O(\frac{Tn^{2}mc^{4d}+\sqrt{Tn}mc^{4d}\max(T,n)}{m^{3/d}})$ \\
\hline
\end{tabular}
\caption{Summary of Theoretical Results when using SKI with convolutional cubic interpolation. This shows the rate at which the error of using SKI (vs the exact kernel) grows as a function of important variables. Here $n$ and $T$ are the train/test sample sizes, $d$ is the dimensionality, $m$ the number of inducing points, $p$ is the number of hyperparameters and $c>0$ is a constant. Most importantly, the Gram matrix error grows linearly with the sample size, exponentially with the dimension while decaying at an $m^{3/d}$ rate in the inducing points.}
\label{table:theoretical_results}
\end{table*}

In this paper, we begin to bridge the gap between practice and a theoretical understanding of SKI. We have three primary contributions: 1) The first error analysis for the SKI kernel and relevant quantities, including the SKI gram matrix's spectral norm error. Based on this we provide \textit{a practical guide to select the number of inducing points}: they should grow as $n^{d/3}$ to control error. 2) SKI hyperparameter estimation analysis. 3) SKI inference analysis: the error of the GP posterior means and variances at test points. We find two interesting results: 1) we identify two dimensionality regimes relating SKI Gram matrix error to computational complexity. For $d\leq 3$, for \textit{any} fixed spectral norm error, we can achieve it in linear time using SKI with a sufficient sample size. For $d>3$, the error must \textit{increase} with the sample size to maintain our guarantee of linear time. 2) For a $\mu$-smooth log-likelihood, gradient ascent on the SKI log-likelihood will approach a neighborhood of a stationary point of the true log-likelihood at a $O\left(\frac{1}{K}\right)$ rate, with the neighborhood size determined by the SKI score function's error, which aside from the response variables grows \textit{linearly} with the sample size when increasing inducing points as we suggested. To obtain this, we leverage a recent result \cite{stonyakin2023stopping} from the inexact gradient descent \cite{daspremont2008smooth,devolder2014first} literature.

% By our downstream analysis, this implies that for small dimensionality and kernels exhibiting desirable regularity conditions, we can have arbitrarily small error in the SKI Gram matrix, in our estimated parameters and in posterior inference all in linear time as long as the sample size is sufficiently large.

In section \ref{sec:related} we describe related work. In section \ref{sec:ski-background} we give a brief background on SKI. In section \ref{sec:important-quantities} we bound the error of important quantities: specifically the SKI kernel, Gram matrix and cross-kernel matrix errors. In section \ref{sec:gp-applications} we use these to analyze the error of the SKI MLE and posteriors. We conclude in section \ref{sec:discussion} by summarizing our results and discussing limitations and future work.
\subsection{Retrieval-Augmented Generation}
Retrieval-Augmented Generation (RAG) has been widely applied to enhance the performance of Large Language Models (LLMs) by retrieving relevant information from external sources, addressing the limitation of LLMs' restricted context windows, improving factual accuracy, and mitigating hallucinations~\cite{fan2024survey, gao2023retrieval}. Most RAG systems primarily process text data by first splitting it into chunks~\cite{finardi2024chronicles}. When a query is received, RAG retrieves relevant chunks either through lexical search~\cite{ram2023context} or by computing semantic similarity~\cite{karpukhin2020dense}, embeddings both the query and text chunks into a shared vector space. Advanced techniques, such as pre-retrieval processing~\cite{ma2023query, zheng2023take} and post-retrieval processing~\cite{dong2024don, xu2023recomp}, as well as fine-tuning strategies~\cite{li2023structure}, have further enhanced RAG’s effectiveness across various domains, including QA)~\cite{yan2024corrective}, dialogue generation~\cite{izacard2023atlas}, and text summarization~\cite{jiang2023active}.

Several studies have evaluated the effectiveness of RAG systems across various tasks~\cite{yu2024evaluation, chen2024benchmarking, es2023ragas}, such as multi-hop question answering~\cite{tang2024multihop}, biomedical question answering~\cite{xiong2024benchmarking}, and text generation~\cite{liu2023recall}. However, no existing study has simultaneously and systematically evaluated and compared RAG and GraphRAG on these general text-based tasks.
% \yu{However, no existing study has simultaneously and systematically evaluated and compared RAG and GraphRAG on these general tasks. Add simultaneously to highlight our novelty is not evaluate any of them but at the same time and so we can compare.}

\begin{figure*}[!htb]
    \centering
   \includegraphics[width=\linewidth]{figures/RAGGraphRAG1.pdf}
    \caption{The illustration of RAG, KG-based GraphRAGs and Community-based GraphRAGs.}
    \label{fig:framework}
    \vspace{-0.2in}
\end{figure*}

\subsection{Graph Retrieval-Augmented Generation}
While RAG primarily processes text data, many real-world scenarios involve graph-structured data, such as knowledge graphs (KGs), social graphs, and molecular graphs~\cite{xia2021graph, ma2021deep}. GraphRAG~\cite{han2024retrieval, peng2024graph} aims to retrieve information from various types of graph-structured data. The inherent structure of graphs enhances retrieval by capturing relationships between connected nodes. For example, hyperlinks between documents can improve retrieval effectiveness in question answering tasks\cite{li2022dynamic}.  Currently, most GraphRAG studies focus on retrieving information from existing KGs for downstream tasks such as KG-based QA~\cite{tian2024graph, yasunaga2021qa} and Fact-Checking~\cite{kim2023factkg}. 
% The retrieved information can be subgraphs~\cite{he2024g} relevant to the query or reasoning paths~\cite{luo2023reasoning} that facilitate inference. 

Despite leveraging the existing graphs, recent studies have explored incorporating graph construction into GraphRAG to enhance text-based tasks. For example, \citet{dong2024don} construct document graphs using Abstract Meaning Representation (AMR) to improve document ranking. \citet{edge2024local} construct graphs from documents using LLMs, where nodes represent entities and edges capture relationships between them. Based on these graphs, they generate hierarchical communities and corresponding community summaries or reports. Their approach focuses on the global query summarization task, retrieving information from both the constructed graphs and their hierarchical communities. Additionally, \citet{han2025reasoning} propose an iterative graph construction approach using LLMs to improve reasoning tasks.
% \yu{Here I feel we spend so much context on introducing graph-based GraphRAG. However, the motivation of this work (also in the introduction part) is more around the issue of GraphRAG applied on text data. So it might be better to  emphasize more on that aspect}

These studies highlight the potential of GraphRAG in processing text-based tasks by constructing graphs from textual data.
However, their focus is limited to specific tasks and evaluation settings. It remains unclear how GraphRAG performs on general text-based tasks compared to RAG. More importantly, when and how should GraphRAG be applied to such tasks for optimal effectiveness? Our work aims to bridge this gap by systematically evaluating GraphRAG and comparing it with RAG on general text-based tasks.


% However, most existing studies focus on specific tasks and datasets, and there is no systematic evaluation—like those conducted for RAG—to assess the effectiveness of GraphRAG on text-based tasks using widely adopted datasets and evaluation metrics.\yu{I think here the thing is if we spend so much context on graph-based GraphRAG, then our evaluation should also be in that regard. However, our systematically evaluation is on GraphRAG for traditional text data. It might be better to focus our narratives more on that perspective.} Our work aims to fill this gap by systematically evaluating GraphRAG and comparing it with RAG.
\begin{figure}[t!]
    \centering
    \includegraphics[width=0.45\textwidth]{images/method_visual-crop.pdf}
    \caption{Process of Visual Attention Evaluation.}
    \label{img1}
\end{figure}

\subsection{API Prompting}
API Prompting~(sketched in Figure~\ref{api}) is a Visual Prompting method that highlights important parts in an image using a Visual Attention Heatmap derived from a Vision-Language Model~\citep{api}. The Attribution Map, representing the contribution of image tokens to model outputs, is extracted from a VLM (referred to as Heatmap VLM or H-VLM), convolved, resized to match the image size, and then overlaid on the original image.

Following the study by \citet{api}, Vision-Transformer-based CLIP and LLaVA are employed as Heatmap VLMs. The methods for extracting Visual Attention Attribution Maps from each model are described below.

\paragraph{CLIP Attribution Map}
CLIP computes similarity between text and image representations, and the Attribution Map \( \Psi \) is obtained by decomposing the similarity function \( \text{sim}(\hat{I}, \hat{T}) \). %The image representation \( \hat{I} \) is expressed as:
% \begin{equation}
%    \begin{aligned}
%        \hat{I} &= \mathcal{L}([Z^{0}_{\text{cls}}])
%         + \sum_{\ell=1}^{L} \mathcal{L}([\text{MSA}^{\ell}(Z^{\ell-1})]_{\text{cls}}) \\
%        &\quad + \sum_{\ell=1}^{L} \mathcal{L}([\text{MLP}^{\ell}(\hat{Z}^{\ell})]_{\text{cls}}).
%    \end{aligned}
% \end{equation}

Since later MSA layers greatly impact image representation~\citep{clipdec}, the similarity function is approximated as:
\begin{equation}
   \begin{aligned}
\text{sim}(\hat{I}, \hat{T}) \approx \text{sim}\left(\sum_
{\ell=L'}^{L} \mathcal{L}(\text{MSA}^{\ell}([Z^{\ell-1}]))_{\text{cls}}, \hat{T}\right).
\end{aligned}
\end{equation}
To filter out irrelevant regions, a complementary Attribution Map \( \Psi^{\text{comp}} \) is introduced:
\begin{equation}
   \begin{aligned}
   \Psi_{i,j}^{\text{comp}} &\triangleq 1 - \text{sim}(\mathcal{L}(Z_{t}^{L}), \hat{T}), \\
&\quad \text{where} \quad t = 1 + j + P \cdot (i - 1).
\end{aligned}
\end{equation}
Combining both maps, the final CLIP Attribution Map is defined as:
\begin{equation}
   \begin{aligned}
   \Psi = \Psi^{\text{cls}} + \Psi^{\text{comp}} - \Psi^{\text{comp}} \odot \Psi^{\text{cls}}.
\end{aligned}
\end{equation}

\paragraph{LLaVA Attribution Map}
LLaVA can provide an Attribution Map \( \Psi \) using Multi-Head Self-Attention (MSA) weights between output text tokens and image tokens. The Attribution Map is computed by averaging over all output tokens and attention heads:
\begin{equation}
   \begin{aligned}
\Psi_{i,j} &\triangleq \frac{1}{MH} 
  \sum_{m=1}^{M} 
  \sum_{h=1}^{H} 
  A_{m,t}^{(\bar{L},h)}, \\
&\quad \text{where} \quad 
t = j + P \cdot (i - 1).
\end{aligned}
\end{equation}
Here, \(M\) is the number of output tokens, \(H\) is the number of attention heads, \(P\) is the number of patches per image side, and \(A^{(\bar{L},h)}\) represents cross-attention between output text and image tokens at layer \(\bar{L}\) and attention head \(h\).

\subsection{Background Role Examination}
To assess the necessity of background information for object recognition, ground truth segmentation data is used as a Heatmap during API Prompting and the accuracy of output is evaluated (hereafter referred to as API - Seg.). Binary segmentation masks, overlaid in gray are input into VLMs to evaluate their impact on output accuracy. If POPE response accuracy remains unchanged, background information is deemed unnecessary.

\subsection{Minimum Cutoff}
Minimum cutoff redefines the minimum value in segmentation or Visual Attention Heatmap based on a threshold. Since a threshold of 0.5 showed improvement in Table~\ref{table2}, values below 0.5 in the cutoff are replaced with 0.5, refining segmentation granularity.

\begin{table*}[t]
\centering
\begin{tabular}{lllrrrrr}
\toprule
\textbf{Dataset} & \textbf{Model} & \textbf{Prompting} & \textbf{Acc.} & \textbf{Prec.} & \textbf{Rec.} & \textbf{TNR} & \textbf{F1} %& \textbf{Yes (\%)} 
\\ \cmidrule(lr){1-8}
\multirow{7}{*}{\textbf{MSCOCO}} & \multirow{7}{*}{LLaVA} & w/o prpt. & 86.23 & 84.21  & 89.19 & 83.27 & 86.63 \\ %& 52.96 \\
& & API~(CLIP) & \ensuremath{\blacktriangle}86.52 & \ensuremath{\blacktriangle}84.78 & \ensuremath{\triangledown}89.02 & \ensuremath{\blacktriangle}84.02 & \ensuremath{\blacktriangle}86.85 \\ %& \ensuremath{\blacktriangle}52.50  \\
& & API~(CLIP) w Cutoff & \ensuremath{\blacktriangle}88.59 & \ensuremath{\blacktriangle}85.84 & \ensuremath{\blacktriangle}92.43 & \ensuremath{\blacktriangle}84.75 & \ensuremath{\blacktriangle}89.01 \\
 & & API~(LLaVA) & \ensuremath{\triangledown}86.11 & \ensuremath{\blacktriangle}84.72 & \ensuremath{\triangledown}88.12 & \ensuremath{\blacktriangle}84.10 & \ensuremath{\triangledown}86.39 \\ %&  \ensuremath{\blacktriangle}52.01 \\
 & & API~(LLaVA) w Cutoff & \ensuremath{\blacktriangle}87.98 & \ensuremath{\blacktriangle}85.10 & \ensuremath{\blacktriangle}92.09 & \ensuremath{\blacktriangle}83.88 & \ensuremath{\blacktriangle}88.46 \\
  & & API - Seg. & - & - & \ensuremath{\triangledown}71.78 & - &  - \\ %& - \\
  & & API - Seg. w Cutoff & - & - & \ensuremath{\blacktriangle}89.24 & - &  - \\ %& - \\
 \bottomrule
\end{tabular}
\caption{POPE results on MSCOCO datasets with API Prompting.}
\label{table1}
\end{table*}

\begin{table*}[h!]
\centering
\begin{tabular}{llrrrr}
\toprule
%\toprule
\multirow{2}{*}{\textbf{H-VLM}} & \multirow{2}{*}{\textbf{Output}} & \multicolumn{4}{c}{\textbf{Visual Attention Alignment}} \\ %\cline{2-9} 
& & \textbf{Prec.} & \textbf{Rec.}  & \textbf{IoU}    & \textbf{MSE}   \\ 
%\cmidrule(lr){1-9}
\cmidrule(lr){1-6}
%Overall   & 0.1270   & 0.8318 & 0.1092 & 0.2920  & 0.1010   & 0.6523 & 0.0767 & 0.1337 \\ \hline
%\multirow{3}{*}{CLIP} & - & 12.70 & 83.18 & 10.92 & 29.20 \\
\multirow{2}{*}{CLIP}& Correct~(87\%)  &\ensuremath{\blacktriangle}13.52   & \ensuremath{\blacktriangle}83.94 & \ensuremath{\blacktriangle}11.68 & \ensuremath{\blacktriangle}28.46  \\
& Incorrect~(13\%) & 6.09   & 77.02 & 4.78 & 35.22 \\
%\multirow{3}{*}{LLaVA} & - & 10.10 & 65.23 & 7.67 & 13.37 \\
\multirow{2}{*}{LLaVA}& Correct~(86\%)  & \ensuremath{\blacktriangle}10.62  & 64.62 & \ensuremath{\blacktriangle}8.10 & 13.60 \\ 
& Incorrect~(14\%) & 6.17   & \ensuremath{\blacktriangle}69.78 & 4.44 & \ensuremath{\blacktriangle}11.60 \\
\bottomrule
%\bottomrule
\end{tabular}
\caption{Alignment of CLIP/LLaVA Visual Attention.}
\label{table3}
\end{table*}

\subsection{Evaluation of Visual Attention Alignment}
To assess how well visual attention focuses on target objects, Precision, Recall, Intersection over Union~(IoU), and Mean Squared Error~(MSE) are computed between the object segmentation data and the Visual Attention Heatmap as depicted in Figure~\ref{img1}. 
Visual Attention Heatmaps are converted into binary arrays using thresholds set to the average value of each heatmap.


% \begin{figure}[t]
%      \centering
%     \includegraphics[width=0.47\textwidth]{images/cutoff-crop.pdf}
%      \caption{Example of Cutoff Segmentation Annotation}    \label{clip}
%  \end{figure}
\section{Experiment Setup}
This section presents the experimental setup for comparing various confidence measures in selecting reliable responses for reasoning tasks. We also extend the evaluation to additional datasets and explore combining self-certainty with voting methods for improved response selection.

% from experiment xxx.
\subsection{Comparison of Confidence Measures} \label{sec:experiment_setup}  
To assess the effectiveness of different candidate formulations in Section~\ref{sec:candidate}, we employ them to select the most confident response from a set of \( N \) outputs generated by our base model, Llama-3.1-8B-Instruct \cite{dubey2024llama}. To mitigate the potential bias and data contamination arising from the model's training on publicly available datasets, we evaluate their performance using the LiveBench-Math dataset \cite{white2024livebench}, which was released after the model's deployment.

We begin by sampling 64 responses using \(\text{temperature} = 0.6\) and \(\text{top-p} = 0.9\), and subsequently create subsets comprising the first \(N = 4,8,16,32,64\) to perform Best-of-N selection. To ensure fairness, we test different measures' performance using the same set of samples. We mask out responses for which an answer cannot be extracted-primarily because these outputs do not adhere to the format instructions to facilitate latter comparisons with majority voting. We also include a baseline, FirstAns, which simply selects the first extractable answer from the 
\(N\) outputs. This baseline serves as a reference point for quantifying the performance improvements achieved by our candidate expression-based selection strategy. The evaluation is implemented based on the ZeroEval framework \cite{zeroeval}, which provides a unified structure for performance evaluation. We repeat the experiment five times and report the average accuracy as the overall performance. 

% As shown in Figure~\ref{fig:expression_selection}, the KL-divergence-inspired distributional confidence (self-certainty) achieves the highest accuracy on LiveBench-Math, particularly excelling at larger $N$, whereas other measures plateau or degrade beyond $N=16$.

\subsection{Validation on Additional Datasets and Combined Voting Methods} \label{sec:borda_setup}
% After confirming that self-certainty outperforms other confidence measures, we combine it with voting methods to further enhance performance and evaluate its effectiveness on a broader set of datasets. 
We conduct a series of experiments to evaluate the proposed self-certainty and Borda Voting methods against self-consistency, universal self-consistency (USC), greedy decoding, and FirstAns across various reasoning tasks.


The sampling strategy follows the procedures outlined in Section~\ref{sec:experiment_setup}. For USC, we use the template from the original paper \cite{chen2023universal} (with minor wording modifications, as shown in Appendix \ref{sec:usc_example}). To ensure a fair comparison, we assist USC in selecting the first valid response when it fails to choose one with an extractable answer. 

We evaluate different methods using the Llama-3.1-8B-Instruct model across the following benchmarks:
\begin{itemize}[leftmargin=*, itemsep=0pt, topsep=0pt]
    \item \textbf{Mathematical Reasoning}: We utilize the LiveBench-Math dataset \cite{white2024livebench}, the validation set of GSM8K dataset \cite{gsm} and the test set of MATH dataset \cite{MATH}.
    \item \textbf{Code Reasoning}: The CRUXEval-O benchmark \cite{gu2024cruxeval} is employed, which involves predicting the output of Python codes.
    \item \textbf{Code Generation}: We adopt the LiveCodeBench code generation benchmark \cite{jain2024livecodebench} to assess the improvements introduced by our methods.
\end{itemize}

\begin{figure}[t]
    \centering
    \includegraphics[width=0.99\linewidth]{fig/section1.pdf}
    % \vspace{-1.5em}
    \caption{Evaluation of Best-of-\(N\) selection accuracy on LiveBench-MATH across multiple confidence measures. KL achieves the best performance at larger \(N\), while other measures plateau or decline after \(N = 16\).} 
    \label{fig:expression_selection}
\end{figure}

For all test models and datasets, we employ Chain-of-Thought reasoning \cite{wei2022chain}, except for the code generation dataset. To evaluate the generalization of our measure across different training methodologies, particularly for the recent R1-series large reasoning models \cite{guo2025deepseek}, we test our approach on DeepSeek-R1-Distill-Llama-8B using the MATH dataset (Level 3). Given the increased reasoning time required by this model, we conduct a single trial for this experiment. To further validate and assess generalizability, we apply both USC and self-certainty to the Qwen-2.5-Coder-32B-Instruct model \cite{hui2024qwen2}, in addition to Llama-3.1-8B-Instruct, on the LiveCodeBench dataset.


\section{Results and Analysis}\label{sec:result}

\subsection{Self-Certainty}



\begin{figure*}[t]
    \centering
    \includegraphics[width=\linewidth]{fig/combined_confidence_new.png}
    \vspace{-1.em}
    \caption{Scatter plot showing various confidence measures against response length (measured in number of characters) in the LiveBench-Math dataset, using the Llama-3.1-8B-Instruct model with 64 samples per question. The figure demonstrates that, with the exception of self-certainty, all other measures exhibit a bias towards longer responses.} 
    \label{fig:scatter}
\end{figure*}

\paragraph{KL-Divergence-Inspired Distributional Confidence Outperforms Other Measures in Best-of-N Selection.}

The results, shown in Figure \ref{fig:expression_selection}, demonstrate that distributional confidence measures generally outperform perplexity when \(N \geq 16\). Among all candidate methods, KL divergence is the only measure that consistently improves as N increases to 32 and 64. This indicates that KL divergence serves as a more robust measure of confidence, offering better insight into the accuracy of responses. Equation~\ref{eq:sc} defines tokenwise self-certainty as the KL divergence from a uniform distribution, with an alternative empirical distribution evaluated in Appendix~\ref{sec:diff_dis}. The results confirm that KL with a uniform distribution, our original design, generalizes better.



\paragraph{Self-Certainty's Robustness to Reasoning Length in Response Selection.}
To understand why self-certainty outperforms other confidence measures in selecting better responses, we examine the relationship between reasoning length and confidence scores across different measures, as shown in Figure~\ref{fig:scatter}. The scatter plots reveal that longer reasoning lengths correlate with higher confidence scores in most metrics except self-certainty. This indicates that while other measures tend to favor samples with extended reasoning, self-certainty remains largely invariant to response length. This finding aligns with \citet{basu2020mirostat}'s observation that, under low \(p\) values, perplexity decreases as the output length increases. Unlike other metrics that may conflate verbosity with correctness, self-certainty provides a more unbiased assessment of response quality. This robustness ensures that models cannot manipulate the confidence measure by simply generating more extended but meaningless reasoning paths.

\paragraph{Self-Certainty Effectively Separates Correct and Incorrect Responses.}
We analyze the distribution of self-certainty and negative perplexity across correct, incorrect, and no-answer responses using Level 4 of the MATH dataset for a balanced comparison. Figure~\ref{fig:freq} presents our findings. The histogram shows that self-certainty in both correct and incorrect responses follows an approximately normal distribution, with the correct group consistently exhibiting a higher mean. In contrast, while perplexity is able to identify better results when \(N\) is small (also in Figure~\ref{fig:expression_selection}), it fails to distinguish between correct and incorrect responses when applied to the full dataset with multiple outputs per question. \citet{zhang2020trading} demonstrates that as perplexity declines, the quality of responses improves initially, then experiences a significant drop -- an observation consistent with our findings. Notably, perplexity tends to assign higher confidence to no-answer responses, which often arise from self-repetition, early stopping, or excessively long reasoning chains that fail to follow prompt instructions. Given that Llama-3.1-8B-Instruct has a relatively low no-answer rate (\(< 2\% \)) in simpler Level 1 MATH problems, we attribute the no-answer rate primarily to the limited capacity of the model. It is unsurprising that negative perplexity favors these outputs, as avoiding difficult questions or repeating oneself is a common failure mode, even for humans -- consistent with \citet{basu2020mirostat}, who showed that maximizing perplexity increases self-repetition. In contrast, self-certainty reliably assigns lower confidence scores to no-answer responses, effectively distinguishing them from correct answers. These observations are further strong evidence that self-certainty is a more effective measure of the certainty of a model as it is more closely correlated with the quality of responses.

\begin{table}[t]
\centering
\caption{Accuracy of different voting methods on the test set of MATH dataset using Llama-3.1-8B-Instruct. Self-certainty-based Borda voting outperforms other voting methods.}
\vspace{0.5em}
\label{tab:voting}
\small
\begin{tabular}{@{}lcc@{}}
\toprule
\textbf{Method} & \textbf{\(\boldsymbol{N = 8}\)} & \textbf{\(\boldsymbol{N = 64}\)} \\ 
\midrule
Majority & 58.60 & 63.40  \\
Average             & 46.92 & 32.94\\
Sum                 &  59.06 & 63.51 \\
Borda (\(p=0.5\))   & \textbf{59.08}  & 63.71  \\
Borda (\(p=1.2\))   & 58.86  & \textbf{64.10}  \\
\bottomrule
\end{tabular}
% \vspace{-1em}
\end{table}

\begin{table*}[ht]
    \centering
    \caption{Performance comparison of various methods across different datasets using Llama-3.1-8B-Instruct. Some USC results are omitted due to over 20\% of the data exceeding context window limits under the settings. Self-certainty consistently outperforms sampling, greedy decoding, and USC, while Borda Voting with the optimal parameter \(p\) delivers the best performance across all methods.
    }
    \vspace{0.5em}
    \label{tab:results}
    \footnotesize
    \begin{tabular}{lcc  cc  cc  cc c}
        \toprule
        \multirow{2}{*}{\textbf{Method}}  & \multicolumn{2}{c}{\textbf{LiveBench-Math}} & \multicolumn{2}{c}{\textbf{GSM8K}} & \multicolumn{2}{c}{\textbf{MATH}} & \multicolumn{2}{c}{\textbf{Crux}} & \multirow{2}{*}{\textbf{Avg.}}\\ 
       & $N=8$ & $N=32$ & $N=8$ & $N=64$ & $N=8$ & $N=64$ & $N=8$ & $N=64$ & \\
        \midrule
        Greedy            & \multicolumn{2}{c}{12.23} & \multicolumn{2}{c}{84.00} & \multicolumn{2}{c}{47.96} & \multicolumn{2}{c}{39.88} & 46.02\\
        FirstAns          & 17.66 & 17.66 & 82.08 & 82.08 & 49.08 & 49.09 & 42.93 & 42.93 & 47.94 \\
        \midrule
        USC                  & 21.08 &  -    & 87.32 & 85.65 & 54.66 & - &  43.78 & 41.25 & 51.19\\
        Self-consistency     & 22.50 & 26.25 & 89.42 & 90.99 & 58.60 & 63.40 & 47.58 & 50.42 & 56.15 \\
        \midrule            
        Self-certainty       & 20.87 & 22.01 & 87.32 & 88.90 & 54.63 & 56.70 & 45.38 & 45.83 & 52.71 \\
        - Borda ($p=0.3$)    & \textbf{23.69} & 26.47 & \textbf{89.57} & \textbf{91.07} & \textbf{59.04} & 63.60 & \textbf{47.94} & 50.42 & 56.48\\
        - Borda ($p=0.7$)    & 23.59 & 26.36 & 89.51 & 91.04 & \textbf{59.04} & 63.85 & 47.85 & 50.65 & 56.49\\
        - Borda ($p=1.2$)    & 23.21 & \textbf{26.69} & 89.51 & 90.95 & 58.86 & \textbf{64.10} & 47.93 & 50.85 & \textbf{56.51}\\
        - Borda ($p=2.0$)      & 22.45 & 26.41 & 89.13 & 90.90 & 57.94 & 60.02 & 47.25 & \textbf{51.23} & 55.67 \\ 
        \bottomrule
    \end{tabular}
\end{table*}

\begin{figure*}[t]
    \centering
    \includegraphics[width=0.99\linewidth]{fig/four_datasets_line_plots_new.pdf}
    % \vspace{-1.5em}
    \caption{Performance evaluation across four datasets employing different strategies with Llama-3.1-8B-Instruct. These line graphs demonstrates the strong scaling ability of both self-certainty and Borda voting.}
    \label{fig:four_datasets}
\end{figure*}

\subsection{Self-Certainty and Voting}\label{sec:exp_voting}
\paragraph{Borda Voting in Combination with Self-Certainty.}
As discussed in Section \ref{sec:vote}, when responses contain explicit answers, self-certainty can be integrated with voting methods to enhance overall accuracy. We evaluate the effectiveness of Borda voting for combination voting in such cases, comparing it against majority voting, average self-certainty, and sum self-certainty on the MATH dataset, as shown in Table~\ref{tab:voting}. Our results indicate that self-certainty-based Borda voting outperforms other voting methods.



\paragraph{Performance Comparison Across Four Datasets.}
We examine the scaling properties of self-certainty and self-certainty-based Borda voting in Figure~\ref{fig:four_datasets}. The results indicate that self-certainty significantly outperforms both regular sampling and greedy decoding. Moreover, its performance improves considerably as $N$ increases, demonstrating that self-certainty, as a measure of the model’s confidence in its responses, provides valuable insight into output correctness.
Additionally, Borda voting demonstrates better performance compared to self-consistency under various settings of \(p\) and \(N\) across all four datasets. This suggests that the self-certainty measure enhances the accuracy of the final-answer-based voting method by providing useful ranking information. 



\paragraph{Optimizing the Borda Parameter \(p\) for Different \(N\).}
To investigate the relationship between the Borda parameter \(p\) in Equation~\ref{eq:borda_vote} and the efficiency of the selection method, we plot line charts in Figure~\ref{fig:math_p}, showing the performance of different selection methods across varying sample sizes \(N\). The result reveals that the optimal \(p\) increases from 0.5 to 1.2 as \(N\) increases from 8 to 64, suggesting that stronger control from self-certainty is needed with a larger \(N\). For general use cases, grid search remains the most effective approach for determining the optimal $p$. Alternatively, a simple heuristic is to set \(p = 0.3\) when \(N\leq 16\) and \( p = 1.2\) when \(N\geq 32\), though this rule of thumb may vary depending on the model and the complexity of the questions.

\begin{figure}[t]
    \centering
    \includegraphics[width=0.48\textwidth]{fig/math_p.pdf}
    \vspace{-1em}
    \caption{Performance of Borda voting on the MATH dataset using Llama-3.1-8B-Instruct, with varying \( p \) and \( N \). For each $N$, accuracy initially increases with $p$ reaches a peak, and then declines. The optimal \( p \) depends on \( N \). Note that self-consistency corresponds to Borda voting with \( p = 0 \).}
    \label{fig:math_p}
\end{figure}

\subsection{Generalization}
\paragraph{Generalization of Self-Certainty on Open-Ended Generation Tasks.}
Self-consistency struggles with creative, open-ended tasks such as code generation, where each sample produces a unique answer. In such cases, it defaults to standard sampling. USC and our method self-certainty offer solutions to this limitation. We compare self-certainty with USC on the code generation task of LiveCodeBench (Figure~\ref{fig:livecode}). Our findings reveal that USC underperforms compared to greedy decoding on the Llama-3.1-8B-Instruct model, likely due to the model’s constrained ability to recognize consistency. This observation is reinforced by results from the larger Qwen model, where USC successfully outperforms greedy decoding. In contrast, self-certainty consistently outperforms greedy decoding on both models and surpasses USC on the larger Qwen-2.5-Coder-32B-Ins. Additionally, the performance of self-certainty improves as \(N\) increases.

\begin{figure}[t]
    \centering
    \includegraphics[width=0.48\textwidth]{fig/livecode.pdf}
    % \vspace{-1em}
    \caption{Comparison of self-certainty and USC on the LiveCodeBench code generation task. The results show that self-certainty outperforms USC and greedy decoding on both Llama-3.1-8B-Instruct and Qwen-2.5-Coder-32B-Ins models, with performance improving as \(N\) increases.}
    \label{fig:livecode}
\end{figure}

\paragraph{Generalization of Self-Certainty on Reasoning Models.}
Recent research on DeepSeek-R1 \citep{guo2025deepseek} demonstrates that rule-based reinforcement learning and long-chain-of-thought (CoT) can significantly enhance the reasoning capabilities of LLMs. We evaluate the generalization of self-certainty on such reasoning models, with results for DeepSeek-R1-Distill-Llama-8B presented in Table~\ref{tab:distill}. Our findings show that self-certainty consistently outperforms both greedy decoding and sampling, with performance further improving as $N$ increases on reasoning models. Additionally, Borda voting with self-certainty surpasses the performance of self-consistency with proper \(p\). These results reinforce previous observations, highlighting the robustness of our methods across various fine-tuning techniques.

\begin{table}[t]
\centering
\caption{Accuracy of various methods on the Level 3 test set of the MATH dataset using DeepSeek-R1-Distill-Llama-8B (single trial). Self-certainty outperforms Greedy and FirstAns, while Borda Voting with an appropriate \(p\) surpasses self-consistency.}
\label{tab:distill}
\vspace{0.5em}
\small
\begin{tabular}{@{}lccc@{}}
\toprule
\textbf{Method} & \textbf{\(\boldsymbol{N = 4}\)} & \textbf{\(\boldsymbol{N = 16} \)} & \textbf{\(\boldsymbol{N = 64}\)} \\
\midrule
Greedy    & 77.54  &  77.54 & 77.54 \\
FirstAns            & 81.17    & 81.43 & 81.43\\
\midrule
Self-consistency    & 83.64    & 86.47 & 87.62\\
\midrule
Self-certainty      & 83.29 & 83.73 & 84.08  \\
- Borda ($p=0.3$)   & 84.79  & 87.00 & 87.80  \\
- Borda ($p=0.7$)   & 84.70 & 86.91  & 87.62\\
- Borda ($p=1.2$)   & 84.62  & 87.00 & 88.06\\
- Borda ($p=2.0$)   & 83.29  & 87.00 & 87.98\\
\bottomrule
\end{tabular}
\vspace{-1em}
\end{table}

% \xuandong{zhexian figure without self-consistency. Greedy, first answer, upper-bound, self-certainty, borada for five datasets}

\section{Discussion and Future Research}
While self-certainty proves effective in evaluating model outputs for open-ended generation tasks and demonstrates stable scalability, it has several limitations.

First, self-certainty alone underperforms self-consistency on questions with unique answers (Section~\ref{sec:result}). Although self-consistency generally achieves better performance, recent studies suggest that properly trained ORMs and PRMs can surpass it \cite{lightman2023let,uesato2022solving}. Current reward model training fine-tunes a base model to rate sentences based on token probabilities \cite{wang2024math}. A key insight from our findings is that treating the softmaxed logits of LLMs as a full probability distribution, rather than relying solely on individual token probabilities, leads to more robust certainty measurements. In self-certainty, using KL divergence between the output distribution and a uniform distribution provides greater stability than averaging log probabilities, suggesting that integrating this approach could enhance reward model effectiveness.

Second, this study explores a limited set of formulations for distributional confidence and Borda voting. The default choice of the averaging function for $F$ in Equation~\ref{eq:DC} may not be optimal, and a broader selection of $F$ could further improve self-certainty’s accuracy. Likewise, the power function used for vote distribution in Equation~\ref{eq:borda_vote} for Borda voting is intuitive but may not be the most effective formulation.

Beyond these limitations, self-certainty presents exciting opportunities for future research. It not only encourages rethinking reward model designs but also offers potential applications in test-time scaling techniques \cite{snell2024scaling}, potentially reducing computational costs. Additionally, self-certainty could be leveraged for data labeling and reinforcement learning tasks \cite{bai2022training,ouyang2022training}. By maximizing token-wise self-certainty, we may enable more autonomous and efficient learning systems, paving the way for advancements in both model performance and computational efficiency.

% Beyond encouraging a remodeling of the reward model, self-certainty opens up diverse future research directions. Some test-time compute scaling techniques rely on reward models to select superior reasoning and planning paths \cite{snell2024scaling}. Replacing the reword model with self-certainty can substantially reduce the heavy computational cost of training rewards without significantly compromising performance. Moreover, it is also possible to apply Self-certainty to data labeling and reinforcement learning tasks. Given the robustness of self-certainty and the fact that the assigned probability of selected tokens does not directly determine it, it creates new opportunities for developing a fully autonomous learning system. For instance, by defining the loss function to minimize the token-wise self-certainty of selected tokens and iteratively optimizing this measure, self-certainty offers a promising approach to enhancing model performance while ensuring greater autonomy and computational efficiency.



\section{Conclusion}
In this paper, we introduce self-certainty and self-certainty-based Borda voting as novel approaches for evaluating and enhancing model response performance. Self-certainty functions as an internal measure of response quality, demonstrating robustness in several key aspects. Compared to traditional scoring methods, such as average log probability and perplexity, it offers superior scalability when applied to Best-of-N selection. Additionally, the ranking information provided by self-certainty improves chain-of-thought reasoning and outperforms universal self-consistency (USC) in code generation tasks. Its stability, flexibility, and generalizability make it applicable across a wide range of domains, with the potential to enhance the autonomous learning capabilities of LLMs. 

% \section*{Impact Statement}
% This research introduces self-certainty, a novel confidence measure that exhibits a stronger correlation with response quality than traditional probability-based metrics, such as perplexity, by capturing distributional-level information. By leveraging the inherent probabilistic structure of LLMs, self-certainty offers a lightweight, scalable, and effective alternative to reward models for inference-time response selection, reducing the computational and annotation burdens associated with existing Best-of-N selection techniques.  

% Beyond its immediate application in improving reasoning accuracy, self-certainty has the potential to transform self-evaluation and autonomous learning in large language models. Unlike reward models, which require extensive fine-tuning and human supervision, self-certainty enables models to assess their own outputs with minimal additional computation. This advancement paves the way for more efficient self-improving AI systems that can refine their responses without explicit external feedback.  

% Furthermore, self-certainty generalizes beyond structured problem-solving tasks, offering a robust mechanism for quality assessment in open-ended text generation, including creative writing, dialogue generation, and code synthesis. By challenging conventional reward model-based selection frameworks, our approach not only enhances the reliability of LLM outputs but also lays the groundwork for more interpretable and transparent AI decision-making. The adoption of self-certainty in AI research and deployment has the potential to make advanced reasoning capabilities more accessible, reducing reliance on computationally expensive techniques while maintaining high levels of accuracy and adaptability.

\section*{Acknowledgment}  

We appreciate the valuable discussions with Kexun Zhang and Danqing Wang. We also thank the Center for AI Safety for providing substantial compute resources for this project.
\bibliography{example_paper}
\bibliographystyle{icml2025}


%%%%%%%%%%%%%%%%%%%%%%%%%%%%%%%%%%%%%%%%%%%%%%%%%%%%%%%%%%%%%%%%%%%%%%%%%%%%%%%
%%%%%%%%%%%%%%%%%%%%%%%%%%%%%%%%%%%%%%%%%%%%%%%%%%%%%%%%%%%%%%%%%%%%%%%%%%%%%%%
% APPENDIX
%%%%%%%%%%%%%%%%%%%%%%%%%%%%%%%%%%%%%%%%%%%%%%%%%%%%%%%%%%%%%%%%%%%%%%%%%%%%%%%
%%%%%%%%%%%%%%%%%%%%%%%%%%%%%%%%%%%%%%%%%%%%%%%%%%%%%%%%%%%%%%%%%%%%%%%%%%%%%%%
\newpage
\appendix
\onecolumn




\section{More Experiment Results}
\subsection{Oracle Best-of-N Selection Performance and Scaling Effects on LiveCodeBench}
In our experiment described in Section~\ref{sec:borda_setup}, we evaluate the performance of Llama-3.1-8B-Instruct and compare Borda voting and self-certainty against the upper bound of Best-of-N selection methods, as shown in Figure~\ref{fig:five_datasets_more}. While both methods demonstrate continued improvement as $N$ increases, they remain significantly outperformed by the Oracle selection method, which assumes perfect knowledge of the correct answer.
\begin{figure}[H]
    \centering
    \includegraphics[width=0.99\textwidth]{fig/five_datasets_line_plots.pdf}
    \caption{Performance across five datasets using different strategies with Llama-3.1-8B-Instruct. The oracle selection method significantly outperforms the other strategies. Additionally, both Borda voting and self-certainty demonstrate strong scaling effects.}
    \label{fig:five_datasets_more}
\end{figure}

\subsection{Evaluation of Methods Across Difficulty Levels on the MATH Dataset}
We conduct experiments to evaluate different methods across varying difficulty levels of reasoning problems. Figure~\ref{fig:level} presents the performance of various methods on the MATH dataset at different difficulty levels. As question difficulty increases, the scaling effect of Borda voting and self-certainty becomes more pronounced, demonstrating their effectiveness in handling more challenging reasoning tasks.


\begin{figure}[H]
    \centering
    \includegraphics[width=0.48\textwidth]{fig/level.pdf}
    \caption{Comparison of evaluation methods on the MATH test dataset across increasing difficulty levels using Llama-3.1-8B-Instruct. The four bars in each difficulty bin correspond to an increasing choice of N in the Best-of-N selection (8, 16, 32, and 64 generations). Performance differences among settings become more pronounced as the difficulty level increases.}
    \label{fig:level}
\end{figure}

\subsection{Replacing Uniform Distribution with Empirical Distribution}\label{sec:diff_dis}

\begin{figure}[t]
    \centering
    \includegraphics[width=0.7\textwidth]{fig/token_frequency.pdf}
    \caption{Frequency of the top 20 tokens in Llama-3.1-8B-Instruct responses to MATH training set questions (generated eight times per question).}
    \label{fig:token_freq}
\end{figure}

\begin{table}[ht]
\centering
\caption{Accuracy of various self-certainty definitions for Best-of-N selection on the MATH and GSM8K test sets using Llama-3.1-8B-Instruct (single trial). The empirical distribution is derived by sampling from the MATH training dataset. While the empirical self-certainty results are comparable to those based on a uniform distribution for the MATH test set, it is significantly outperformed by the latter, likely due to a distributional shift.}
\label{tab:emp}
\vspace{0.5em}
\small
\begin{tabular}{@{}lccccc@{}}
\toprule
\multirow{2}{*}{\textbf{Base Distribution}} & \multicolumn{2}{c}{\textbf{MATH}} & \multicolumn{2}{c}{\textbf{GSM8K}} \\
& \(N = 8\) & \(N = 64\) & \(N = 8\) & \(N = 64\) \\
\midrule
Uniform   & 54.60   & 56.46 & 87.19 & 88.55\\
Empirical      & 54.70 & 56.78 & 85.97 & 86.35  \\
\bottomrule
\end{tabular}
\end{table}
In Equation~\ref{eq:sc}, we define tokenwise self-certainty as the KL divergence between the generated token distribution and a uniform distribution, which quantifies deviation from random sampling. An alternative approach replaces the uniform distribution with an empirical token distribution estimated from training data. To evaluate the impact of this modification, we conduct the following experiment.

We first estimate token frequencies in the MATH training set by generating eight responses per question and averaging token occurrences. The resulting empirical distribution is approximated from these frequencies, with the 20 most frequent tokens shown in Figure~\ref{fig:token_freq}. We then compute KL divergence between the generated token distribution and the empirical distribution, using this as the self-certainty measure for Best-of-N selection. This experiment was conducted for a single trial, with all other setup parameters as described in Section~\ref{sec:experiment_setup}.

Results show that replacing the uniform distribution with the empirical distribution has minimal impact on MATH test accuracy but leads to a noticeable performance drop on GSM8K, suggesting a sensitivity to distributional shifts. Thus, we recommend retaining the uniform distribution in Equation~\ref{eq:sc} for improved generalization.


\section{Case Study and Examples}
In this section, we present several examples to illustrate the practical behavior of self-certainty and universal self-consistency (USC). These examples are drawn from Llama-3.1-8B-Instruct generating responses to questions from the GSM8K and LiveCodeBench code generation datasets.

\subsection{Self-certainty}
In this example, self-certainty correctly assigns higher confidence to the correct answer. Notably, in Response 0, the model does not fully adhere to the instruction to enclose all reasoning within the specified JSON field. Instead, it follows its trained reasoning process first and then summarizes the reasoning path in the \texttt{reason} field. This approach appears to be a compromise between the model's training objective and the given instruction, balancing structured formatting with its natural reasoning process.
\begin{tcolorbox}[colback=white,colframe=gray!50!black,title=Self-certainty Example]
USER: \\
\#\# Question:\\
Kylar went to the store to buy glasses for his new apartment. One glass costs \$5, but every second glass costs only 60\% of the price. Kylar wants to buy 16 glasses. How much does he need to pay for them?\\
\#\# Instruction: \\
Please answer this question by first reasoning and then providing your answer.\\
Present your reasoning and solution in the following json format. 
Please show your final answer in the answer field, e.g.,``answer": ``42". `json\{ ``reasoning": ``\_\_\_",  ``answer": ``\_\_\_"\}\\

Response 0:\\
ASSISTANT:\\
\#\#Step 1: Calculate the cost of the first glass.\\
The cost of the first glass is \$5.\\
\#\#Step 2: Calculate the cost of every second glass.\\
Every second glass costs 60\% of the price of the first glass, which is \( 0.6 \times 5 = 3 \) dollars.\\
\#\#Step 3: Determine the number of glasses that are full price and the number of glasses that are discounted.\\
Kylar wants to buy 16 glasses. Since every second glass is discounted, we can divide 16 by 2 to get the number of full-price glasses and the number of discounted glasses. \( \frac{16}{2} = 8 \) full-price glasses and 8 discounted glasses.\\
\#\#Step 4: Calculate the total cost of the full-price glasses.\\
8 full-price glasses at \$5 each is \( 8 \times 5 = 40 \) dollars.\\
\#\#Step 5: Calculate the total cost of the discounted glasses.\\
8 discounted glasses at \$3 each is \( 8 \times 3 = 24 \) dollars.\\
\#\#Step 6: Calculate the total cost of all the glasses.\\
The total cost is the sum of the cost of the full-price glasses and the cost of the discounted glasses. \$40 + \$24 = \$64.\\
\#\#Step 7: Determine the total amount Kylar needs to pay for the glasses.\\
The total amount Kylar needs to pay is \$64.\\
The final answer is: 
\{``reasoning": ``Kylar wants to buy 16 glasses. The first glass costs \$5, and every second glass costs 60\% of the price of the first glass, which is \$3. He buys 8 full-price glasses at \$5 each and 8 discounted glasses at \$3 each. The total cost of the full-price glasses is \$40, and the total cost of the discounted glasses is \$24. The total cost is \$40 + \$24 = \$64.", ``answer": "64"\}\\
\textbf{(Self-certainty: 17.81)}\\

Response 1: \\
ASSISTANT:\\
\{``reasoning": ``To find the total cost, we need to calculate the cost of the first glass and the cost of the remaining glasses. The first glass costs \$5. The second glass costs 60\% of \$5, which is 0.6 * 5 = \$3. The third glass costs 60\% of \$3, which is 0.6 * 3 = \$1.80. We can see that the price of each subsequent glass is 60\% of the previous glass. So, the price of the glasses follows a geometric sequence: 5, 3, 1.80, 1.08, \(\dots\) . To find the cost of 16 glasses, we can use the formula for the sum of a geometric series: \(S = a * (1 - r^n) / (1 - r)\), where a is the first term, r is the common ratio, and n is the number of terms. In this case, a = 5, r = 0.6, and n = 16. Plugging in the values, we get \(S = 5 * (1 - 0.6^16) / (1 - 0.6)\). Calculating the value of \(0.6^{16}\), we get 0.0000655. Now, we can plug this value back into the formula: \(S = 5 * (1 - 0.0000655) / (1 - 0.6) = 5 * 0.9999345 / 0.4 = 5 * 2.499335 = 12.496675\). Rounding to two decimal places, the total cost is \$12.50.", ``answer": ``12.50"\} \\
\textbf{(Self-certainty: 17.13)}\\

\end{tcolorbox}

\subsection{Universal Self-Consistency}\label{sec:usc_example}
The first example is taken from the GSM8K dataset, while the second comes from the LiveCodeBench dataset. The prompt has been adapted from \citet{chen2023universal}. We observe that in Example 1, the model evaluates the answers step by step to assess consistency. In contrast, the model jumps directly to the conclusion in Example 2. This difference may stem from the clarity of the task: in a MATH problem with a definite answer, the model can logically assess each step of the reasoning process, as seen in Example 1. However, for the LiveCodeBench example, where the task involves determining the most consistent code, the model struggles to elaborate on the reasoning process and relies on direct reasoning to arrive at a conclusion.


\begin{tcolorbox}[colback=white,colframe=gray!50!black,title=USC Prompt Example 1]
USER: 

I have generated the following responses to the question: Elise is learning to write and decides to keep re-writing the alphabet until she knows it. She writes it in full twice, writes half of it once, then re-writes everything she has already written. How many letters has Elise written in total?\\


Response 0: \{``reasoning": "To find the total number of letters Elise has written, we first need to determine the number of letters in the alphabet. There are 26 letters in the English alphabet. Elise writes it in full twice, which means she writes 26 letters twice, so 26 * 2 = 52 letters. Then, she writes half of it once, which is 26 / 2 = 13 letters. So, the total number of letters written so far is 52 + 13 = 65. Finally, she re-writes everything she has already written, which is 65 letters. Therefore, the total number of letters Elise has written is 65 + 65 = 130.",``answer": ``130"\}\\

\(\cdots\)\\
Response 3: \{``reasoning": ``To solve this problem, we need to calculate the total number of letters Elise has written. First, she writes the alphabet twice, which has 26 letters. So, she has written 26 + 26 = 52 letters. Then, she writes half of the alphabet once, which is 13 letters. So, she adds 13 more letters to the total. Now, she re-writes everything she has already written, which means she re-writes 52 + 13 = 65 letters. To find the total number of letters Elise has written, we need to add the number of letters she wrote initially (52) and the number of letters she re-wrote (65). The total number of letters is 52 + 65 = 117.",    ``answer": ``117"\}\\

\(\cdots\)\\

Evaluate these responses.
Select the most consistent response based on majority consensus.
Format your answer as "The most consistent response is Response X" (without quotes).\\


ASSISTANT:

To evaluate the responses, I will analyze the answers and reasoning provided by each response.Response 0 and Response 1 have the same answer (130) and similar reasoning. \(\cdots \).The most consistent response is Response 0.
%gsm8k-main-test-#587
\end{tcolorbox}

\begin{tcolorbox}[colback=white,colframe=gray!50!black,title=USC Prompt Example 2]
USER: 

I have generated the following responses to the question: Slavic is preparing a present for a friend's birthday. He has an array \( a \) of \( n \) digits, and the present will be the product of all these digits. Because Slavic is a good kid who wants to make the biggest product possible, he wants to add 1 to exactly one of his digits.

\textbf{What is the maximum product Slavic can make?}

\textbf{Input:}
\begin{itemize}
    \item The first line contains a single integer \( t \) (\( 1 \leq t \leq 10^4 \)) — the number of test cases.
    \item The first line of each test case contains a single integer \( n \) (\( 1 \leq n \leq 9 \)) — the number of digits.
    \item The second line of each test case contains \( n \) space-separated integers \( a_i \) (\( 0 \leq a_i \leq 9 \)) — the digits in the array.
\end{itemize}

\textbf{Output:}
\begin{itemize}
    \item For each test case, output a single integer — the maximum product Slavic can make, by adding 1 to exactly one of his digits.
\end{itemize}

Sample Input: \(\cdots\)\\
Sample Output: \(\cdots\)\\


\textbf{Response 0:}

\begin{verbatim}
```python
# YOUR CODE HERE
import sys

def max_product(n, arr):
    min_digit = min(arr)
    max_product = 1
    for i in range(n):
        if arr[i] == min_digit:
            arr[i] += 1
            break
    for num in arr:
        max_product *= num
    return max_product

t = int(input())
for _ in range(t):
    n = int(input())
    arr = list(map(int, input().split()))
    print(max_product(n, arr))
```
\end{verbatim}
This code first finds the minimum digit in the array and increments it by 1. Then, it calculates the product of all digits in the array. The time complexity is O(n), where n is the number of digits in the array.\\
\(\cdots\)\\

Evaluate these responses.
Select the most consistent response based on majority consensus.
Format your answer as "The most consistent response is Response X" (without quotes).\\

ASSISTANT:

The most consistent response is Response 0.
\end{tcolorbox}


% You can have as much text here as you want. The main body must be at most $8$ pages long.
% For the final version, one more page can be added.
% If you want, you can use an appendix like this one.  

% The $\mathtt{\backslash onecolumn}$ command above can be kept in place if you prefer a one-column appendix, or can be removed if you prefer a two-column appendix.  Apart from this possible change, the style (font size, spacing, margins, page numbering, etc.) should be kept the same as the main body.
%%%%%%%%%%%%%%%%%%%%%%%%%%%%%%%%%%%%%%%%%%%%%%%%%%%%%%%%%%%%%%%%%%%%%%%%%%%%%%%
%%%%%%%%%%%%%%%%%%%%%%%%%%%%%%%%%%%%%%%%%%%%%%%%%%%%%%%%%%%%%%%%%%%%%%%%%%%%%%%


\end{document}


% This document was modified from the file originally made available by
% Pat Langley and Andrea Danyluk for ICML-2K. This version was created
% by Iain Murray in 2018, and modified by Alexandre Bouchard in
% 2019 and 2021 and by Csaba Szepesvari, Gang Niu and Sivan Sabato in 2022.
% Modified again in 2023 and 2024 by Sivan Sabato and Jonathan Scarlett.
% Previous contributors include Dan Roy, Lise Getoor and Tobias
% Scheffer, which was slightly modified from the 2010 version by
% Thorsten Joachims & Johannes Fuernkranz, slightly modified from the
% 2009 version by Kiri Wagstaff and Sam Roweis's 2008 version, which is
% slightly modified from Prasad Tadepalli's 2007 version which is a
% lightly changed version of the previous year's version by Andrew
% Moore, which was in turn edited from those of Kristian Kersting and
% Codrina Lauth. Alex Smola contributed to the algorithmic style files.
