\section{Related Work}
\subsection{3D Gaussian representation}

Recently, Gaussian Splatting-based representations offer real-time rendering with high training efficiency and garner considerable research interests\cite{tang2023dreamgaussian, diolatzis2024nd-gaussian, mallick2024taming, zhou2024feature3dgs}. Motivated by the success of 3D Gaussian Splatting, numerous studies have extended it to real-time dynamic scene reconstruction and rendering.
%
Incremental translation methods \cite{dyn-3dgs, 3dgstream} tackle this challenge by initializing each frame based on the preceding one, leveraging motion constraints to enforce temporal coherence. Other approaches extend Gaussian representations to 4D space-time \cite{4d-gaussian} or model global scene deformations with neural networks or polynomial \cite{4dgs, deformable_gaussian, spacetime-gaussian}, enabling efficient reconstructions. Despite these advancements, such methods face limitations in accurately capturing complex or discontinuous scene dynamics. Moreover, they often rely on fixed color and opacity settings to ensure consistency.

\textcolor{edit}{
Recent studies \cite{shen2024gamba, yi2024mvgamba, ziwen2024long, zhang2025gslrm} have explored integrating Mamba \cite{mamba, mamba2} or Transformer architectures with 3D Gaussian representations, focusing on reconstructing 3D Gaussians from single or multi-view images. However, these approaches are limited in their ability to model spatiotemporal dynamics, as the lack of memory mechanisms. Our method leverages sequences of 3D Gaussians to represent the temporal evolution of 3D radar echo data and employs a Memory-Augmented GauMamba model to effectively integrate information from preceding frames. 
}

\subsection{Spatio-temporal prediction}

Spatio-temporal prediction is crucial in meteorological forecasting, requiring models to capture both spatial patterns and temporal dynamics. U-Net architectures using 2D or 3D CNNs have been applied to tasks like precipitation nowcasting, Arctic Sea ice prediction, and ENSO forecasting \cite{sevir, sea-ice, enos}, though they struggle with temporal dependencies. To improve this, methods such as ConvLSTM \cite{convlstm}, ConvGRU \cite{convgru}, and PredRNN \cite{predrnn, predrnnv2} integrate memory mechanisms to better handle spatio-temporal correlations. E3D-LSTM \cite{e3d} combines 3D CNNs with LSTM for long-term forecasting, while PhyDNet \cite{phydnet} embeds physical constraints into models. SimVP \cite{simvp} simplifies prediction using convolutional encoders and decoders, while transformer-based models \cite{fourcastnet, rainformer, earthformer} capture long-range dependencies. However, deterministic models often struggle with prediction blur and fail to capture the stochastic nature of weather systems. To address this, diffusion-based models \cite{prediff, diffcast} have been introduced to estimate spatio-temporal uncertainty.