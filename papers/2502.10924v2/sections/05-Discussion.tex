
\section{Discussion}

\ly{In this section, we situate ``AI Afterlife'' in digital legacy, and delve into design implications for maintaining identity consistency and balancing intrusiveness and support in future ``AI Afterlife'' based digital legacy practices.}

\subsection{Situating ``AI Afterlife'' in Digital Legacy}

Our work focuses on ``AI Afterlife'', an emerging form of digital legacy that leverages the generative, interactive, and dynamic capabilities of generative AI. Unlike traditional static digital legacy, this new form offers transformative possibilities in two key areas: (1) maintaining a sense of continuity after death by emphasizing identity consistency (in section \ref{identity consistency}), and (2) enabling interaction with the living, particularly fostering a healthy ``continuing bond'' with loved ones by balancing intrusiveness and support (in section \ref{balance intrusiveness and support}). In alignment with the concept of ``digital immortality'' defined by Bell \textit{et al.} \cite{bell2001digital}, this new form of digital legacy achieves a degree of ``two-way immortality''. 

In addressing this emerging form of digital legacy, we first identified the diverse and complex attitudes people hold toward it, shaped by personal, familial, technological, and social factors (in section \ref{attitudes factors}). Notably, ``AI Afterlife'' challenges traditional definitions of life and death, while the interplay between these perspectives significantly shapes participants' attitudes toward this new form of digital legacy. Moreover, ``AI Afterlife'' presents opportunities to positively impact the living by supporting the grieving process, providing enduring value, and preserving family heritage across generations. However, these benefits are also contingent upon family dynamics, technological capabilities, and broader social contexts.

Besides, we examined how people perceive the differences between ``AI Afterlife'' based digital legacy and traditional digital legacy (in section \ref{comparision ai and traditional}). Three key distinctions emerged: data distribution, content authenticity, and interaction forms. This comparison highlights the tension between the two approaches. On one side are the benefits of generative AI, such as integrated information, realistic simulations, and immersive interactivity. On the other side are the advantages of static digital legacy, including accessibility, emotional connections to authentic experiences, and controllable interaction. These differences reveal their unique strengths and limitations. They suggest the need for context-dependent decisions - choosing the approach that best suits users' needs or combining both to maximize their complementary benefits.

Additionally, ``AI Afterlife'' sheds light on the design aspects throughout the life cycle (in section \ref{life cycle section}) and interaction process (in section \ref{sec: interaction design}) of digital legacy.
Aligning with the framework used for traditional digital legacy \cite{doyle2023digital}, our work further unpacks key aspects of this new form across the encoding, accessing, and dispossession processes, providing comprehensive considerations for the entire life cycle of future digital legacy practices. The unique characteristics of generative AI also introduce distinct interaction dynamics in replicating, simulating, and responding to both the self and external environments.

Furthermore, ``AI Afterlife'' introduces distinct concerns regarding digital legacy, particularly in terms of technology, mental health, security, and socioeconomics, shaped by the unique features of generative AI, as well as personal and contextual issues (in section \ref{concerns}). Therefore, careful consideration is necessary to address these potential issues and ensure that this emerging form of digital legacy remains reliable and has a positive impact on both the deceased and the living.

In summary, ``AI Afterlife'' represents a pivotal evolution in the fields of digital legacy. By exploring the complex attitudes, differences, life cycles, interaction design, and challenges, this study highlights the transformative potential of generative AI in shaping digital legacy. Building on this understanding, we propose more targeted design implications for future practice, which will be illustrated in detail in section \ref{identity consistency} and section \ref{balance intrusiveness and support}.


\subsection{Design Implications for Maintaining Identity Consistency} \label{identity consistency}

Extending life and values beyond death is a crucial factor shaping positive attitudes toward ``AI Afterlife'' as a form of digital legacy (see Section \ref{attitudes factors}). This post-mortem continuity is intrinsically linked to the concept of ``identity consistency'', which refers to the preservation of core traits, such as appearance, memory, thoughts, and beliefs, despite changes over time and across contexts \cite{wyness2019childhood, marcia1966development, taylor1989sources}. By ensuring the stability of these core traits, individuals can continue to be recognized as the ``same person'' in evolving circumstances \cite{hume2000treatise, locke1847essay, wyness2019childhood}. As ``AI Afterlife'' serves as an emerging form of digital legacy of individuals, maintaining the consistent identity of the deceased within these agents becomes a fundamental design consideration. Insights from results highlight three dimensions of identity - appearance, knowledge, thinking, and emphasize the need to manage evolution while preserving identity consistency.

\textit{Appearance.} 
While technology enables the realistic simulation of human physical traits (in section \ref{content}), how these traits are presented significantly affects identity consistency. Participants preferred to display their most meaningful or ideal appearances while allowing audiences the freedom to interpret or choose based on personal memories (in section \ref{age and appearance}). This flexibility underscores identity consistency as a dynamic process, balancing the stability of core traits with individual expression in changing contexts. 
Therefore, we propose to \textit{support customizable and adaptive representations}, allowing users to display meaningful life-stage appearances while enabling audience-driven selection based on personal memories or conversation topics.

\textit{Knowledge.}
Findings also reveal tensions between identity consistency and incorporation of AI knowledge (in section \ref{ai knowledge}). Proponents believed that extra knowledge could enhance interaction diversity, while opponents worried that it might undermine identity consistency, especially when such knowledge conflicts with their values \cite{hume2000treatise}. In this case, ensuring that any new knowledge aligns with and does not compromise core traits is essential. 
Therefore, we propose to \textit{include careful boundary management when integrating AI knowledge}, suggesting the implementation of pre-defined personal knowledge domains (e.g., education and professional experiences) and value-alignment protocols to prevent identity conflicts.

\textit{Thinking.}
Participants emphasized the importance of agents reflecting both external traits and internal attributes such as thoughts, beliefs, and values \cite{locke1847essay} (in section \ref{simulation level}). This perspective highlights the necessity of designing generative AI-based digital legacy that faithfully represents the personality and inner worlds. Addressing challenges in data collection and organization for capturing inner traits (in section \ref{content}) is a critical step in achieving this goal. 
Therefore, we emphasize the \textit{necessities of innovative data collection methods for inner trait collection}, such as gamified interfaces and context-aware prompts, to capture personal values and thought patterns from their daily life seamlessly and effectively \cite{rabbi2017sara, wilson2005context, dergousoff2015mobile}. 

\textit{Evolution.}
Participants held different views on whether ``AI Afterlife'' should evolve over time (in section \ref{evolution}). While some supported adaptation to new contexts, others raised concerns about preserving core traits to avoid identity confusion \cite{marcia1966development, taylor1989sources}. Evolution can result from both environmental factors and deliberate human intervention in the updating process (in section \ref{update}), requiring careful management to balance adaptability with consistency. 
Therefore, we propose that \textit{evolution should be managed through user-defined update permissions and version control mechanisms}, ensuring adaptability while maintaining core identity markers. 

In summary, designing for continuity in ``AI Afterlife'' involves addressing the complexities of identity across multiple dimensions - appearance, knowledge, thinking, and evolution. These considerations contribute to the broader HCI discourse on digital legacy, providing an ethical and philosophical framework for future afterlife design that prioritizes the preservation of identity while navigating the dynamic nature of generative AI. 


\subsection{Design Implications for Balancing Intrusiveness and Support} \label{balance intrusiveness and support}

\ly{
Alleviating grief and creating values for the living are critical factors shaping positive attitudes toward ``AI Afterlife'' as a form of digital legacy (in section \ref{attitudes factors}). While coexistence with this kind of AI-generated agents can provide emotional and practical support for the living, it also raises significant concerns about intrusiveness. Intrusiveness, defined as the degree to which technology disrupts users' attention and emotional well-being \cite{case2015calm}, has been extensively studied in HCI and design research \cite{love2004dealing, kondratova2024cultural, benlian2020mitigating, conti2010malicious, pohl2019charting}. Designing to balance support and intrusiveness involves minimizing disruption, preserving user autonomy, and ensuring emotional well-being, aligning with the goal of creating comforting rather than burdensome digital experiences. Insights from our results highlight three dimensions of intrusiveness - bidirectional interaction, proactivity, and physical space, and emphasize the need for balancing these aspects to ensure meaningful coexistence. 
}

\textit{Bidirectional Interaction Intrusiveness.}
Interactive elements in ``AI Afterlife'' introduce potential for emotional and attentional strain. Bidirectional interactions can be immersive and alleviate loneliness by simulating continued presence, but they also risk prolonging grief through excessive engagement (in section \ref{Interactive and Static}). 
Therefore, we emphasize that designs should \textit{support user control through rule-based mechanisms} that allow individuals to manage interaction intensity and frequency. \textit{Automatic disengagement features should also be incorporated} to prevent undue burden while maintaining meaningful support, such as pausing or ending interactions based on detected emotional distress \cite{slovak2023designing, benke2020chatbot}.

\textit{Proactivity Intrusiveness.}
Proactive behavior of this kind of AI-generated agents, such as sensing and responding autonomously to emotional cues, can lead to discomfort if users feel overwhelmed or that their autonomy is being undermined (in section \ref{reactivity}). 
Therefore, we propose that these \textit{proactive behaviors need customizable settings} that enable users to specify the timing, frequency, context, and type of interactions, ensuring support is delivered only when desired and avoiding unsolicited disruptions to daily life.

\textit{Physical Space Intrusiveness.}
The embodiment of this kind of AI-generated agents, especially in physical forms, poses unique challenges. Participants expressed a preference for virtual agents over physical embodiments due to the emotional and ethical implications (in section \ref{embodiment}). Physical forms can act as persistent reminders of loss, intruding upon emotional spaces and potentially becoming burdensome over time. It is important to respect emotional boundaries and avoid excessive physical presence is essential to ensuring these agents support rather than disrupt.
Therefore, we emphasize that \textit{designs should prioritize virtual forms as the default solution} due to their flexibility and reduced intrusiveness. \textit{Physical embodiments should be limited to specific scenarios} where users express a clear need for tangible interactions. Even in such cases, designs should ensure that \textit{physical forms are easy to manage} which can be adapted or removed to avoid long-term emotional strain.

In summary, the intrusiveness of ``AI Afterlife'' requires careful attention to emotional, ethical, and practical dimensions. By emphasizing autonomy, adaptability, and emotional sensitivity, designers can create AI-generated agent-based digital legacies that coexist harmoniously with the living, providing meaningful support without becoming intrusive.  