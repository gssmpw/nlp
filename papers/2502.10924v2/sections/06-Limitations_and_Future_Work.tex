\section{Limitations and Future Work}
While this study provides valuable insights into ``AI Afterlife'' as digital legacy, it is important to acknowledge the limitations stemming from the cultural and social backgrounds of the participants. The research was conducted primarily with individuals from a Chinese cultural context, which inherently shapes their views on life, death, and the afterlife.

In many East Asian cultures, there is a strong emphasis on familial bonds \cite{freedman1961family, qi2016family}, filial piety \cite{hwang1999filial, whyte1997fate}, and the continuity of ancestry \cite{song2015ancestry}. Although participants exhibited some variation, particularly in the context of ongoing social transformation, these cultural values significantly influence how individuals perceive death and the role of AI-generated legacy. For instance, the cultural importance placed on remembrance and honoring ancestors may foster a favorable attitude toward ``AI Afterlife'' as a way to preserve familial connections and legacy. Conversely, the belief in the finality of death and the natural cycle of life may lead to reservations about extensive engagement with AI-generated agents of the deceased. Additionally, in cultures with more individualistic orientations, the emphasis on self-expression may shape attitudes and expectations toward AI-generated agents in unique ways \cite{hunter2008beyond, kim2007express, kim2003choice, inglehart2004individualism}, which contrast with the collectivist tendencies observed in this study, such as prioritizing support for family and alleviating grief.

Given these cultural nuances, the findings of this study may not be fully generalizable to other cultural contexts where beliefs about death and the afterlife differ significantly. However, some key findings, such as the expected interaction designs and concerns, can be applied more broadly. This paper serves as a starting point for understanding human attitudes toward this emerging form of digital legacy. Future research should include a more diverse participant pool from various cultural backgrounds to better explore how different cultural and social values shape perceptions of ``AI Afterlife'' as digital legacy, ensuring it meets the diverse needs and values of users globally.