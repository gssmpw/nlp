\section{Related Work}

% In this section, we review multiple strands of literature to situate our work, including digital legacy, and ``AI Afterlife''.

\subsection{Digital Legacy}
% gap: what features AI distinguish from traditional ones

Legacy material refers to items passed down by the deceased to recipients such as bereaved loved ones \cite{doyle2023digital}. These materials include various aspects: the spirit aspect (e.g., values, wishes, identities), tangible items (e.g., objects, heirlooms), and even digital content \cite{doyle2023digital}. Such materials carry meanings that reflect important aspects of the deceased's life, allowing their legacy to extend beyond physical presence \cite{hunter2008beyond, gach2021getting}. They also ensure that bereaved loved ones maintain access to significant parts of the deceased's life \cite{jung2023bereaved, kim2024maintaining, brubaker2024ai}. This access is especially meaningful during emotional, spiritual, and logistical dilemmas surrounding the death of a loved one \cite{doyle2023digital}, thereby preventing potential distress or loss \cite{chen2021happens}. 

As many of the activities we do are now almost exclusively on digital devices, digital legacy, as valued collections of digital information, including social media accounts \cite{ohman2019dead, gach2021getting}, passwords \cite{holt2021personal, ferreira2011password}, files and media \cite{pfister2017will}, is becoming a more and more important part of legacy materials \cite{doyle2023digital, chen2021happens}.
% \ly{It reflects, to some extent, people's aspiration for achieving ``digital immortality'' \cite{bell2001digital,galvao2021discussing}.}
However, preparing and passing down the digital legacy runs into numerous challenges (e.g., vast quantity and selection \cite{bergman2016science, vitale2018hoarding, jones2016curating}, cross devices and platforms \cite{odom2012lost, vertesi2016data}, limited time \cite{hunter2005leaving}, complex social environment \cite{doyle2023digital}, security and privacy \cite{edwards2013protecting, harbinja2019emails, holt2021personal}, lifespan \cite{gulotta2013digital}). These concerns have profound implications for researchers and designers to consider the potential of technologies in supporting digital legacy issues.

To assist in the design and deployment of effective technologies, over the past decades, HCI and CSCW researchers have conducted important research on preserving and passing down digital legacy. Doyle and Brubaker's recent literature review \cite{doyle2023digital} identified four key areas of focus within this field of research: how identity is navigated in the passing of digital legacy (e.g., \cite{gulotta2014legacy, kaye2006have, thomas2014older}), how digital legacy is engaged with (e.g., \cite{gach2020experiences, massimi2011matters, brubaker2014stewarding, brubaker2016legacy}), how digital legacy is put to rest (e.g., \cite{gulotta2016engaging, kirk2010human, jamison2016ps}), and how technology interfaces with offline legacy technologies (e.g., \cite{odom2012technology, uriu2021floral, uriu2021generating}). Much of this work has focused on static digital traces left behind after death \cite{morris2024generative}, such as social media profiles \cite{brubaker2016legacy, brubaker2014stewarding, brubaker2011we, brubaker2019orienting, gach2020experiences, gach2021getting, getty2011said, mori2012design}, personal archives \cite{kaye2006have, she2021living}, and burner accounts \cite{gulotta2016engaging}, which are central to legacy crafting and memorialization \cite{morris2024generative}. 

However, while traditional digital legacy offers valuable resources for reflection, this kind of digital legacy often lacks interactive, expressive, or evolving feature, failing to fully address people's desires to maintain agency over their posthumous future \cite{gulotta2017digital, brubaker2024ai, morris2024generative}. Meanwhile, their inability to evolve over time limits their capacity for deeper engagement and dynamic representation of an individual's identity, leaving gaps in continuity and interaction (e.g, continuing their life and passing on values \cite{gulotta2017digital}, and maintaining potential legal or economic benefits \cite{brubaker2024ai, morris2024generative}). Moreover, it has been shown that the simulation of deceased individuals can act as a buffer space and add resilience to loss in these destabilizing and painful life-changing moments for the bereaved \cite{xygkou2023conversation}, which can not be provided by traditional static digital legacy. 

To better accommodate people's needs in the afterlife, a more expressive, interactive, and evolving technology for creating digital legacy is needed. An emerging form of digital legacy, based on AI-generated agents, with its unique benefits in generative, interactive, and dynamic features suits these needs. Our study adds to this literature by adopting a user-centered method to explore the potential of AI-generated agent-based digital legacy.

\subsection{AI Afterlife}
% gap: who make it
Recent years have witnessed remarkable advancements in generative AI, especially large language models (LLMs) \cite{achiam2023gpt, touvron2023llama, touvron2023llama2, anil2023palm}, with progress made in generative images \cite{qu2023layoutllm}, video \cite{openai2024sora}, audio \cite{ghosal2023text}, and multi-modal models integrating various media types \cite{yu2024spae, girdhar2023imagebind}. These innovations have paved the way for AI-generated agents that can simulate realistic humans \cite{abramson2020creating}, including their voice \cite{Eren_Coqui_TTS_2021}, appearance \cite{musetalk, zhang2023sadtalker}, and even their personalities \cite{la2024open}, memories \cite{pataranutaporn2023living}, and so on. In particular, \ly{AI-generated agents and related technologies} are expected to gain fidelity and popularity as model capabilities improve and computational costs decrease \cite{hu2021lora,zheng2024response}, resulting in potential transformations in various fields (e.g., working \cite{brachman2024knowledge, ibmdigital}, education \cite{kazemitabaar2024codeaid,extance2023chatgpt}, healthcare \cite{yang2024talk2care}, entertainment \cite{isaza2024prompt}). % , clusmann2023future

As generative AI technology quickly advances in creating powerful and realistic agents of specific people, there is growing interest in creating AI-generated agents for deceased individuals by the bereaved loved ones, in order to continue bonds \cite{morris2024generative}. For instance, there have been concrete practices \cite{guardian2024chinese, bbc2024the, cnn2024when, projectdecember}, and research \cite{abramson2020creating, morris2024generative, brubaker2024ai} aimed at providing opportunities for the living ones to engage in conversations, seek comfort, or guidance from these agents of the deceased individuals. 
% definition
\ly{These AI-generated agents, simulating specific individuals for post-mortem purposes with various potential forms (e.g., chatbots \cite{bbc2024the, xygkou2023conversation}, avatars in mixed reality \cite{meet2020}, and robots \cite{berightback2013}), are called ``AI Afterlives'' in this paper (similar to ``generative ghosts'' in \cite{morris2024generative})}. They are anticipated to become common within our lifetimes for people to interact with loved ones and the broader world after death \cite{morris2024generative}. 

Meanwhile, more and more people are interested in joining the creation and governance of their own \ly{AI-generated agent-based digital legacies} while they are still alive \cite{hereafter2022, rememory}, with hopes of extending their influence and leaving a lasting impression \cite{gulotta2017digital, brubaker2024ai, morris2024generative}. Yet, there is little exploration from the perspectives of these people. In particular, it is unclear how people perceive it, what their expectations are for the design and usage, and what concerns they might have in real practices. Seeking answers to these questions is crucial for researchers, designers, and developers to better address potential issues in integrating generative AI into digital legacy, and avoid legal, social, and ethical challenges \cite{guardian2024george, weidinger2023sociotechnical}.

To better unpack the potential of using AI-generated agents to empower digital legacy, perspectives from users, especially the individuals represented by these agents, need more attention. Our study adds to this literature by investigating the perceptions, expectations, and concerns of AI-generated agent-based digital legacy.