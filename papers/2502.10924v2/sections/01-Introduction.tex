\section{Introduction}

As generative AI technology rapidly advances in simulating specific individuals \cite{achiam2023gpt, anil2023palm, abramson2020creating, Eren_Coqui_TTS_2021, zhang2023sadtalker, la2024open, pataranutaporn2023living}, there is growing interest in using digital information to create AI-generated agents as digital legacy \cite{morris2024generative, brubaker2024ai}. 
% define
We define this kind of agent, simulating individuals for post-mortem purposes, broadly as \textbf{\textit{``AI Afterlife''}}, which is expected to become commonplace within our lifetimes \cite{morris2024generative}. 
\label{definition of AI Afterlives}
``AI afterlives'', similar to other collections of digital legacies of the deceased, such as social media accounts \cite{ohman2019dead, gach2021getting}, passwords \cite{holt2021personal,ferreira2011password}, and files and media \cite{pfister2017will}, have the potential to capture meaningful aspects of a person’s life \cite{hunter2008beyond, gach2021getting}, ensure that loved ones can maintain access to significant parts of the deceased's life \cite{jung2023bereaved, kim2024maintaining, brubaker2024ai}, and help prevent potential distress or loss \cite{chen2021happens}.
\ly{This emerging form of digital legacy also has its unique benefit given the generative, interactive, and dynamic features of generative AI. They allow people to engage with the world even after death \cite{hunter2008beyond, gach2021getting} and to interact and maintain a healthy ``continuing bond'' with their loved ones \cite{jung2023bereaved, kim2024maintaining, brubaker2024ai}, especially during destabilizing and painful life changes \cite{gulotta2017digital, brubaker2024ai, morris2024generative, gulotta2014legacy, xygkou2023conversation}. }

Over the past few decades, Human-Computer Interaction (HCI) and Computer-Supported Cooperative Work (CSCW) communities have studied the concept of digital legacy, uncovering broader societal concerns about preserving and passing down digital legacy. Research in this area generally \ly{focuses} on the static digital legacy left by the deceased and can be categorized into four themes \cite{doyle2023digital}: digital legacy as reflections of identity (e.g., \cite{gulotta2014legacy, kaye2006have, thomas2014older}), people's engagement with digital legacy (e.g., \cite{gach2020experiences, massimi2011matters, brubaker2014stewarding, brubaker2016legacy}), practices for and implications of laying digital legacy to rest (e.g., \cite{gulotta2016engaging, kirk2010human, jamison2016ps}), and integrating technology into traditional legacy practices (e.g., \cite{odom2012technology, uriu2021floral, uriu2021generating}).

Despite that many of the existing findings on traditional digital legacy may be generalized to ``AI afterlife'', this new form of digital legacy has its unique challenges. \ly{First, the fact that it can continue to generate new information, react to people and the environment, and even evolve over time is likely to induce new answers to research questions under the aforementioned four themes \cite{doyle2023digital, brubaker2024ai, morris2024generative}.} In particular, the absence of common norms around digital legacy \cite{pfister2017will} poses a challenge that AI is likely to exacerbate \cite{morris2024generative}.
Second, while there are concrete practices \cite{guardian2024chinese, bbc2024the, cnn2024when, projectdecember, meet2020} for and research efforts \cite{abramson2020creating, morris2024generative, brubaker2024ai, AugmentedMIT} on creating AI-generated agents of deceased individuals by those who have lost a loved one \cite{guardian2024chinese, bbc2024the, cnn2024when}, more and more people are interested in joining the creation and governance of their own \ly{AI-generated agent-based digital legacies} while they are still alive \cite{hereafter2022, rememory}. 
There, however, has been little exploration from the perspectives of these individuals \cite{xygkou2023conversation}. 
Specifically, it is unclear how people perceive \ly{this form of digital legacy} when designing their own ``AI afterlives'', what their expectations are for the design and usage, and what concerns they might have in real practices. As digital legacy and generative AI become increasingly prevalent in both practice and research, seeking answers to these questions is crucial for researchers, designers, and developers to better address potential issues in integrating generative AI into existing digital legacy practices. Without such user-centered research, the development of AI-generated agent-based digital legacy may face legal, social, and ethical challenges \cite{guardian2024george, weidinger2023sociotechnical}.

This paper aims to address these gaps by providing an in-depth and empirical understanding of AI-generated agents as digital legacies, particularly from the perspectives of individuals being represented by these agents. We explore the nuanced perceptions, expectations, and concerns that individuals might have regarding their own ``AI Afterlives'' through semi-structured interviews with 18 participants from diverse demographic backgrounds.

Through a thorough thematic analysis, we identified several critical themes, including key factors influencing individuals' attitudes - encompassing personal, familial, technological, and societal aspects, as well as perceived differences between AI-generated agent-based and traditional digital legacy. Our findings also outline the life cycle of AI-generated agent-based digital legacy, detailing stages from encoding to accessing and dispossession. Additionally, \ly{we explore the desired interaction design for and the concerns participants have regarding these agents}. \ly{Our findings suggest the importance of maintaining identity consistency and balancing intrusiveness and support in ``AI Afterlife'' as digital legacy. }

Our study contributes to the HCI community by providing
(1) an empirical study of how individuals represented by AI-generated agents perceive and express concerns about ``AI Afterlife'' as digital legacy, 
(2) a user-centered speculative design to the life cycle and interaction of AI-generated agents in the context of digital legacy, and
(3) an in-depth discussion around ``AI Afterlife'' as digital legacy, alongside a set of design implications.