\section{Method} 
In this section, we outline the methods used in the design and execution of our qualitative study. 
Throughout the study, we apply the definition of ``AI Afterlife'' (mentioned in \autoref{definition of AI Afterlives}) consistently, while encouraging participants to imagine hypothetical post-mortem scenarios and technical forms of this emerging form of digital legacy.

\subsection{Participants: Inclusion Criteria and Recruitment}

\ly{Digital legacy is used for post-mortem purposes} \cite{doyle2023digital}. Therefore, we sought participants who met one or more of the following criteria: (1) type 1 - those who have had a near-death experience (e.g., due to illness, accidents, or aging), (2) type 2 - those who have experienced bereavement, and (3) type 3 - those who have deeply contemplated life and death.

We recruited participants primarily through personal social networks, social media groups, and snowball sampling \cite{goodman1961snowball}. The recruitment process began with the personal networks of two authors 
% (e.g., friends, schoolmates, and colleagues) 
and outreach through relevant social media groups. Participants were then encouraged to recommend family members or friends who met our criteria, following the snowball sampling method. To ensure diversity, we aimed to recruit participants with varied demographic characteristics, including gender, age, health status, education, location, occupation, AI literacy, and related experiences.

\ly{After obtaining the IRB approval, we recruited participants for our qualitative study until we reached saturation, ensuring that no new themes or insights emerged from additional interviews \cite{guest2006many}.} We ultimately interviewed 18 participants, whose demographic details are presented in \autoref{tab:demographicinfo}. Among them, 11 were female and 7 were male, with ages evenly distributed between 20 and 69. They resided in 11 different provinces in China and represented a variety of occupational backgrounds. Their self-reported health statuses, assessed AI literacy levels, and experiences related to our topic were also diverse, further ensuring that we captured perspectives from individuals with different backgrounds.
\ly{To clarify, none of our participants had prior experience creating their own AI-generated agents. Instead, they contributed hypothetical perspectives on the potential applications and implications of this emerging form of digital legacy. }

\begin{table*}[htbp]
  \caption{Demographic information of participants. Rel. exp. means participants' related experience with death. Corp. Manager means the occupation - Corporate Manager. Type 1 means those who have had a near-death experience (e.g., due to illness, accidents, or aging); type 2 means those who have experienced bereavement; type 3 means those who have deeply contemplated life and death. }
  \label{tab:demographicinfo}
  \begin{tabular}{lllllllll}
    \toprule
    ID & Gender & Age & Location & Health & Education & Occupation & AI Literacy & Rel. exp. \\
    \midrule
    P1 & Male & 20-29 & Hongkong & Fair & Master's & Student & Advanced & type1, 3 \\
    P2 & Female & 40-49 & Hunan & Moderate & High School & Business Owner & Beginner & type1, 2 \\
    P3 & Female & 50-59 & Chongqin & Fair & Bachelor's & Teacher & Beginner & type2, 3 \\
    P4 & Female & 40-49 & Shandong & Healthy & Associate's & Bank Staff & Novice & type3 \\
    P5 & Female & 50-59 & Henan & Moderate & Master's & Lecturer & Intermediate & type1-3 \\
    P6 & Female & 20-29 & Beijing & Healthy & Master's & Student & Intermediate & type2, 3 \\
    P7 & Male & 20-29 & Shandong & Healthy & Bachelor's & Student & Advanced & type3 \\
    P8 & Male & 50-59 & Henan & Fair & Bachelor's & Corp. Manager & Advanced & type2, 3 \\
    P9 & Female & 40-49 & Beijing & Fair & Doctorate & Professor & Intermediate & type2, 3 \\
    P10 & Male & 30-39 & Liaoning & Healthy & Master's & Doctor & Novice & type2, 3 \\
    P11 & Female & 30-39 & Tianjin & Healthy & Master's & UX Researcher & Advanced & type2, 3 \\
    P12 & Male & 30-39 & Beijing & Healthy & Doctorate & Professor & Advanced & type2, 3 \\
    P13 & Female & 60-69 & Guangdong & Moderate & Bachelor's & Corp. Manager & Beginner & type1 \\
    P14 & Female & 50-59 & Fujian & Healthy & Associate's & Designer & Novice & type1-3 \\
    P15 & Female & 60-69 & Guangdong & Poor & High School & Plant Worker & Novice & type1-3 \\
    P16 & Male & 60-69 & Chongqing & Poor & Middle School & Plant Worker & Beginner & type1-3 \\
    P17 & Male & 70-79 & Guangdong & Moderate & Middle School & Trader & Intermediate & type1-3 \\
    P18 & Female & 70-79 & Beijing & Fair & Bachelor's & Corp. Manager & Beginner & type1-3 \\
  \bottomrule
\end{tabular}
\end{table*}

\subsection{Data Collection: Semi-structured Interviews}
We conducted semi-structured interviews to gain a deep understanding of participants' perceptions, expectations, and concerns regarding the use of ``AI Afterlife'' as digital legacy. 
% beginning
\ly{At the beginning of each interview, we provided a brief introduction to legacy, digital legacy, AI, generative AI, and existing practices of using this new technology for post-mortem purposes. This introduction aimed to give participants a clear understanding of the social and technical background of our study. We then presented the definition of ``AI Afterlife'' in our study (mentioned in \autoref{definition of AI Afterlives}), emphasizing the flexibility in imagining digital forms to clarify our research focus and scope. In particular, we highlighted that the purpose of ``AI Afterlife''} is limited to creating their digital legacies in post-mortem contexts, that is, the expected simulated entities are the representations of participants themselves, not their deceased loved ones.

% interview question content and features
The interview questions were designed based on the digital legacy literature \cite{doyle2023digital, morris2024generative} and organized into three main parts. The first part focused on participants' daily practices with traditional digital information (e.g., digital photos and social media accounts), their attitudes towards using AI-generated agents as their digital legacies, and their perceptions of the relationships and differences between traditional digital legacy and AI-generated legacy. The second part explored participants' design expectations for AI-generated agents as digital legacies, covering aspects such as the life cycle of the agents and interaction design. The third part addressed participants' concerns about ``AI Afterlife'' as digital legacy, including potential ethical, legal, and social issues. 
The questions were designed to be inclusive, accommodating interviewees with varying levels of experience. When interesting points or relevant experiences were mentioned, we followed up with additional questions to gather more details and concrete examples.

The interviews were conducted remotely in Mandarin between June and August 2024 by two authors via Tencent Meeting, WeChat calls, or phone calls. \ly{Each interview lasted about 60 minutes with 15 USD compensation, allowing for an in-depth exploration of participant perspectives.} All interviews were recorded and transcribed verbatim in Chinese. 

\subsection{Data Analysis: Thematic Analysis}

We applied thematic analysis \cite{braun2012thematic} to analyze all the interview data using an inductive approach \cite{patton1990qualitative}. This method, grounded in the data itself, allowed us to explore emerging themes organically, ensuring flexibility and alignment with our research objectives. Two of the authors were involved in the data analysis process. First, we used a speech-to-text tool to transcribe the recorded interviews, which were then manually reviewed and corrected by the two authors.

%  add inter-rater reliability
We began coding and analysis concurrently with data collection, 
adopting an iterative process to identify and refine emerging themes. Manual coding was performed using Google Documents and Sheets. During the open coding phase, both authors independently reviewed the data and generated initial codes related to our research questions. We held weekly meetings to discuss and reconcile these codes, establishing a consensus to apply consistent codes for the same passages. This process enhanced coding consistency among researchers while maintaining alignment with the study objectives \cite{creswell2017research}.
% add saturation details
\ly{
We continuously assessed thematic saturation throughout the interview process, conducting interviews in batches of three participants. In total, we conducted six batches. Thematic saturation was determined to be achieved when no new themes emerged, as observed in the most recent batch of interviews \cite{guest2006many}. }
After finalizing the initial code list, we refocused our analysis at a broader thematic level, synthesizing related codes into overarching themes. Through several rounds of discussion and refinement, we developed a thematic map that encapsulated five primary themes, \ly{as shown in \autoref{fig:thematic map} in \autoref{sec: thematic map}, which were detailed in \autoref{result} with representative quotes translated from Chinese to English.}

\subsection{Ethical Considerations}

We obtained ethical approval from the Ethics Committee of the authors' institution to conduct all procedures involving human subjects. Throughout the research process, we took careful steps to protect participants' rights and privacy. We also sought feedback from peers both within and outside our faculty to validate the ethical aspects of our research and to refine our interview procedures and questions, minimizing the risk of discomfort for participants. Before conducting the interviews, we fully informed participants of our intentions and provided background information, assuring them that the data would only be used for research. \ly{We obtained their informed consent before proceeding and clearly stated that they could withdraw at any time if they felt uncomfortable. All collected data were anonymized, with participants denoted as PX and no identifiable links to their identities.}