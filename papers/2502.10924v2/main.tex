%%
%% This is file `sample-manuscript.tex',
%% generated with the docstrip utility.
%%
%% The original source files were:
%%
%% samples.dtx  (with options: `manuscript')
%% 
%% IMPORTANT NOTICE:
%% 
%% For the copyright see the source file.
%% 
%% Any modified versions of this file must be renamed
%% with new filenames distinct from sample-manuscript.tex.
%% 
%% For distribution of the original source see the terms
%% for copying and modification in the file samples.dtx.
%% 
%% This generated file may be distributed as long as the
%% original source files, as listed above, are part of the
%% same distribution. (The sources need not necessarily be
%% in the same archive or directory.)
%%
%% The first command in your LaTeX source must be the \documentclass command.
%%%% Small single column format, used for CIE, CSUR, DTRAP, JACM, JDIQ, JEA, JERIC, JETC, PACMCGIT, TAAS, TACCESS, TACO, TALG, TALLIP (formerly TALIP), TCPS, TDSCI, TEAC, TECS, TELO, THRI, TIIS, TIOT, TISSEC, TIST, TKDD, TMIS, TOCE, TOCHI, TOCL, TOCS, TOCT, TODAES, TODS, TOIS, TOIT, TOMACS, TOMM (formerly TOMCCAP), TOMPECS, TOMS, TOPC, TOPLAS, TOPS, TOS, TOSEM, TOSN, TQC, TRETS, TSAS, TSC, TSLP, TWEB.
%\documentclass[acmsmall]{acmart}

%%%% Large single column format, used for IMWUT, JOCCH, PACMPL, POMACS, TAP, PACMHCI
% \documentclass[acmlarge,screen]{acmart}

%%%% Large double column format, used for TOG
% \documentclass[acmtog, authorversion]{acmart}

%%%% Generic manuscript mode, required for submission
%%%% and peer review
%\documentclass[manuscript,screen,review]{acmart}

%\documentclass[manuscript]{acmart}
% I use it for manuscript
% \documentclass[manuscript,review]{acmart}
% \documentclass[sigconf]{acmart}

% \documentclass[manuscript, review, anonymous]{acmart}
% I use it for TAPS submussion
\documentclass[sigconf]{acmart}

\usepackage{color}

% added 2021/5/22
\usepackage{graphicx}
\usepackage{subfigure} 
\usepackage{float}

% \usepackage{threeparttable} 
\usepackage{booktabs}
\usepackage{multirow}
\usepackage{color, colortbl}
\definecolor{LightCyan}{rgb}{0.88,0.88,0.88}
\definecolor{darkgreen}{rgb}{0.0, 0.5, 0.0}

\usepackage[toc,page]{appendix} 
% \usepackage{chngpage}
\usepackage{tabularx}
\usepackage{url}
% \usepackage[inkscapelatex=false]{svg}
\usepackage{enumitem}
\setlist[itemize]{leftmargin=*}


\newcommand{\mxj}[1]{{\color{red} #1}}
\newcommand{\xm}[1]{{\color{blue} #1}}
% \newcommand{\ly}[1]{{\color{darkgreen} #1}}
% \newcommand{\ly}[1]{{\color{blue} #1}}
\newcommand{\ly}[1]{{\color{black} #1}}
\newcommand{\ms}[1]{{\color{teal} #1}}
% \newcommand{\lycr}[1]{\textcolor{blue}{#1}}
% \newcommand{\lycr}[1]{\textcolor{black}{#1}}

%\usepackage[T1]{fontenc}
%\usepackage[skip=1ex]{caption}
%\usepackage{booktabs,tabularx}
%\usepackage{siunitx}
%\newcolumntype{C}{>{\centering\arraybackslash}X}
%\newcommand{\mcx}[2]{\multicolumn{#1}{>{\hsize=\dimexpr#1\hsize
%                                        + #1\tabcolsep + #1\tabcolsep\relax}C}{#2}}
%\newcommand{\mcone}[1]{\multicolumn{1}{C}{#1}}




% \usepackage{floatrow}



\usepackage[ruled]{algorithm2e}
%\usepackage{algorithm}
%\usepackage{algorithmic}

%%
%% \BibTeX command to typeset BibTeX logo in the docs
\AtBeginDocument{%
  \providecommand\BibTeX{{%
    \normalfont B\kern-0.5em{\scshape i\kern-0.25em b}\kern-0.8em\TeX}}}

\copyrightyear{2025}
\acmYear{2025}
\setcopyright{cc}
\setcctype{by-nc-nd}
\acmConference[CHI '25]{CHI Conference on Human Factors in Computing Systems}{April 26-May 1, 2025}{Yokohama, Japan}
\acmBooktitle{CHI Conference on Human Factors in Computing Systems (CHI '25), April 26-May 1, 2025, Yokohama, Japan}\acmDOI{10.1145/3706598.3713933}
\acmISBN{979-8-4007-1394-1/2025/04}

\begin{document}


\title{``AI Afterlife'' as Digital Legacy: Perceptions, Expectations, and Concerns}


\author{Ying Lei}
\affiliation{
  \institution{Simon Fraser University}
  \city{Vancouver}
  \country{Canada}
}
\email{ying_lei@sfu.ca}

\author{Shuai Ma}
\authornote{Corresponding author.}
\orcid{0000-0002-7658-292X}
\affiliation{
  \institution{Aalto University}
  \city{Espoo}
  \country{Finland}
}
\email{shuai.ma@aalto.fi}

\author{Yuling Sun}
\affiliation{
  \institution{Fudan University}
  \city{Shanghai}
  \country{China}
}
\email{yulingsun.lv@gmail.com}


\author{Xiaojuan Ma}
\affiliation{
  \institution{The Hong Kong University of Science and Technology}
  \city{Hong Kong}
  \country{China}
}
\email{mxj@cse.ust.hk}




\renewcommand{\shortauthors}{Ying Lei, et al.}





%%
%% The abstract is a short summary of the work to be presented in the
%% article.
\begin{abstract}
% background + motivation
The rise of generative AI technology has sparked interest in using digital information to create AI-generated agents as digital legacy. These agents, often referred to as ``AI Afterlives'', present unique challenges compared to traditional digital legacy. Yet, there is limited human-centered research on ``AI Afterlife'' as digital legacy, especially from the perspectives of the individuals being represented by these agents.
This paper presents a qualitative study examining users' perceptions, expectations, and concerns regarding AI-generated agents as digital legacy. 
We identify factors shaping people's attitudes, their perceived differences compared with the traditional digital legacy, and concerns they might have in real practices. We also examine the design aspects throughout the life cycle and interaction process. 
Based on these findings, we situate ``AI Afterlife'' in digital legacy, and delve into design implications for maintaining identity consistency and balancing intrusiveness and support in ``AI Afterlife'' as digital legacy.
\end{abstract} 

\begin{CCSXML}
<ccs2012>
    <concept>
        <concept_id>10003120.10003121.10011748</concept_id>
        <concept_desc>Human-centered computing~Empirical studies in HCI</concept_desc>
        <concept_significance>500</concept_significance>
    </concept>
 </ccs2012>
\end{CCSXML}

\ccsdesc[500]{Human-centered computing~Empirical studies in HCI}

% \ccsdesc[500]{Human-centered computing~Interactive systems and tools}


%%
%% Keywords. The author(s) should pick words that accurately describe
%% the work being presented. Separate the keywords with commas.
\keywords{Generative AI, Agent, Afterlife, Digital Legacy, Perception, Expectation, Concern, Design}


\maketitle



People engage in activities in online forums to exchange ideas and express diverse opinions. Such online activities can evolve and escalate into binary-style debates, pitting one person against another~\cite{sridhar_joint_2015}. Previous research has shown the potential benefits of debating in online forums such as enhancing deliberative democracy~\cite{habermas_theory_1984, semaan_designing_2015, baughan_someone_2021} and debaters' critical thinking skills~\cite{walton_dialogue_1989, tanprasert_debate_2024}. For example, people who hold conflicting stances can help each other rethink from a different perspective. However, research has also shown that such debates could result in people attacking each other using aggressive words, leading to depressive emotions~\cite{shuv-ami_new_2022}. Hatred could spread among various groups debating different topics~\cite{iandoli_impact_2021, nasim_investigating_2023, vasconcellos_analyzing_2023, qin_dismantling_2024}, such as politics, sports, and gender.

In recent years, people have integrated Generative AI (GenAI) into various writing tasks, such as summarizing~\cite{august_know_2024}, editing~\cite{li_value_2024}, creative writing~\cite{chakrabarty_help_2022, li_value_2024, yang_ai_2022, yuan_wordcraft_2022}, as well as constructing arguments~\cite{jakesch_co-writing_2023, li_value_2024} and assisting with online discussions~\cite{lin_case_2024}. This raises new concerns in online debates. For example, an internally synthesized algorithm of Large Language Models (LLMs) could produce hallucinations~\cite{fischer_generative_2023, razi_not_2024}, which may act as a catalyst for the spread of misinformation in online forums~\cite{fischer_generative_2023}. In addition, GenAI could introduce biased information to forum members~\cite{razi_not_2024}, which may intensify pre-existing debates. Moreover, integrating GenAI into various writing scenarios may also result in weak insights~\cite{hadan_great_2024}, raising concerns about the impact of GenAI on the ecology of online forum debates.

Given these concerns, this study aims to explore how people use GenAI to engage in debates in online forums. The integration of GenAI is not only reshaping everyday writing practices but also has the potential to redefine the online argument-making paradigm. Previous research has demonstrated the potential of co-writing with GenAI, focusing primarily on its influence on individual writing tasks~\cite{august_know_2024, chakrabarty_help_2022, jakesch_co-writing_2023, li_value_2024, yang_ai_2022}. However, the use of GenAI in the context of online debates, which combine elements of both confrontation and collaboration among remote members, remains underexplored. To explore it, we created an online forum for participants to engage in debates with the assistance of ChatGPT (GPT-4o) (\autoref{fig1}). This study enables us to closely observe how people make arguments and analyze their process data of using GenAI. We will examine three research questions to understand how the use of GenAI shapes debates in online forums: 

\begin{itemize}
\item{\textbf{RQ1}: How do people who participate in a debate on online forums collaborate with GenAI in making arguments?}

\item{\textbf{RQ2}: What patterns of arguments emerge when collaborating with GenAI to participate in a debate on online forums?}

\item{\textbf{RQ3}: How does the use of GenAI for making arguments change when a new member joins an existing debate in online forums?}
\end{itemize}

Given the universality and accessibility of debate topics, we chose one that is widely recognized and able to spark intense debates: soccer, which is regarded as the world's most popular sport~\cite{stolen_physiology_2005}. Building on this topic, we selected "Messi vs. Ronaldo: Who is better?" as the case for our study because it has been an enduring and heated debate among soccer fans. We created a small online forum as the platform for AI-mediated debates, particularly focusing on the debates among members and their interactions with ChatGPT. This approach enables more detailed observation and analysis of the entire process while fostering a nuanced understanding. The study consists of two parts: a one-on-one turn-based debate and a three-person free debate. In the first part, two participants, one supporting Messi and the other Ronaldo, took turns sharing their points of view to challenge each other through forum posts, mirroring the polarized debates that are omnipresent online. In the second part, a new participant joined the ongoing debate, and three participants were allowed to post freely without turn-based restriction, reflecting the spontaneous and unstructured nature of debates on social media. After the two-part study, semi-structured interviews were conducted to explore the participants' experiences. The researchers then applied content analysis and thematic analysis, triangulating the data from forum posts, ChatGPT records, and interview transcripts.

We found that participants prompted ChatGPT for aggressive responses, trying to tailor ChatGPT to fit the debate scenario. While ChatGPT provided participants with statistics and examples, it also led to the creation of similar posts. Furthermore, participants' posts contained logical fallacies such as hasty generalizations, straw man arguments, and ad hominem attacks. Participants reduced the use of ChatGPT to foster better human-human communication when a new member joined an ongoing debate midway. This work highlights the importance of examining how polarized forum members collaborated with GenAI to engage in online debates, aiming to inspire broader implications for socially oriented applications of GenAI.
\section{Related work}
\label{sec:related_work}
Anomaly detection, also known as outlier detection or novelty detection, is an important problem that has been studied within diverse research areas and application domains~\citep{chandola2009anomaly,chalapathy2019deep}. The problem of traffic AD bares similarities with the disciplines of robot AD and AD for surveillance cameras. In this section, we briefly review the related research and introduce common techniques in ensemble deep learning.

Recent research efforts have made noteworthy progress in developing learning-based AD algorithms for robots and mechanical systems. \cite{malhotra2016lstm} introduces an LSTM-based encoder-decoder scheme for multi-sensor AD (EncDec-AD) that learns to reconstruct normal data and uses reconstruction error to detect anomalies. \cite{park2018multimodal} proposes an LSTM-based variational autoencoder (VAE) that fuses sensory signals and reconstructs their expected distribution. The detector then reports an anomaly when a reconstruction-based anomaly score is higher than a state-based threshold. \cite{feng2022unsupervised} attacks multimodal AD with missing sources at any modality. A group of autoencoders (AEs) first restore missing sources to construct complete modalities, and then a skip-connected AE reconstructs the complete signal. Although similar in ideas, these approaches were proposed for low-dimensional signals (e.g., accelerations and pressures) and have not shown effective on high-dimensional data (e.g., images).

AD for robot navigation often involves complex perception signals from cameras and LiDARs in order to understand the environment. \cite{ji2020multi} proposes a supervised VAE (SVAE) model, which utilizes the representational power of VAE for supervised learning tasks, to identify anomalous patterns in 2D LiDAR point clouds during robot navigation. The predictive model proposed in LaND~\citep{kahn2021land} takes as input an image and a sequence of future control actions to predict probabilities of collision for each time step within the prediction horizon. \cite{schreiber2023attentional} further enhances the robot perception capability with the fusion of RGB images and LiDAR point clouds using an attention-based recurrent neural network, achieving improved AD performance on field robots. Different from these supervised-learning-based methods, \cite{wellhausen2020safe} uses normalizing flow models to learn distributions of normal samples of multimodal images, in order to realize safe robot navigation in novel environments. However, driving scenarios have additional complexities than field environments. While road environments are more structured than field environments, additional hazards arise from the presence of and interactions between dynamic road participants, which pose extra challenges on AD algorithms.

Another widely explored research area that is relevant to our work is AD for surveillance cameras, which mainly focuses on detecting the start and end time of anomalous events within a video. Under the category of frame-level methods, \cite{hasan2016learning} proposes a convolutional autoencoder to detect anomalous events by reconstructing stacked images. \cite{chong2017abnormal} and~\cite{luo2017remembering} extend such an idea by learning spatial features and the temporal evolution of the spatial features separately using convolution layers and ConvLSTM layers~\citep{shi2015convolutional}, respectively. Instead of reconstructing frames, \cite{liu2018future} trains a fully convolutional network to predict future frames based on past observations and uses the Peak Signal to Noise Ratio of the predicted frame as the anomaly score. \cite{gong2019memorizing} develops an autoencoder with a memory module, called memory-augmented autoencoder, to limit the generalization capability of the network on reconstructing anomalies. To focus more on small anomalous regions, patch-level methods generate the anomaly score of a frame as the max pooling of patch errors in the image rather than the averaged pixel error used in frame-level methods~\citep{wang2023memory}. In addition, object-level approaches have also been explored, which often focus on modeling normal object motions either through extracted features (e.g., human skeletons)~\citep{morais2019learning} or raw pixel values within bounding boxes~\citep{liu2021hybrid}. Although these methods have achieved promising results on surveillance cameras, the performance is often compromised in egocentric driving scenarios due to moving cameras and complex scenes~\citep{yao2022dota}.

In the domain of traffic AD in first-person videos, pioneering works borrow ideas from surveillance camera applications and detect abnormality by reconstructing motion features at frame level~\citep{yuan2016anomaly}. However, to overcome the issues introduced by rapid motions of cameras and thus backgrounds, object-centric methods are becoming increasingly popular. One of the most representative works by~\cite{yao2019unsupervised} proposes a recurrent encoder-decoder framework to predict future trajectories of an object in the image plane based on the object's past trajectories, spatiotemporal features, and ego motions. The accuracy and consistency of the predictions are then used to generate anomaly scores. One critical problem of such a method is the inevitable miss detection in the absence of traffic participants. As a result, ensemble methods emerge recently to combine the strengths of frame-level and object-centric methods. For example, ~\cite{yao2022dota} fuses the object location prediction model with the frame prediction model to achieve all-scenario detection capability and ~\cite{fang2022traffic} monitors the temporal consistency of frames, object locations, and spatial relation structures of scenes for AD. In this work, we derive each module and the corresponding learning objective in the ensemble based on a comprehensive analysis on anomaly patterns in egocentric driving videos. In particular, an interaction module is introduced to monitor anomalous interactions between road participants. The scores from each module are then fed as observations of a Kalman filter, from which the final anomaly score is obtained.

In this paper, we introduce an ensemble of detectors to capture different classes of anomalies.
We take inspiration from recent advances in ensemble deep learning, which aims to improve the generalization performance of a learning system by combing several individual deep learning models~\citep{ganaie2022ensemble} and has been applied to different application domains, such as speech recognition~\citep{li2017semi},  image classification~\citep{wang2020particle}, forecasting~\citep{singla2022ensemble}, and fault diagnosis~\citep{wen2022new}. Out of different classes of ensemble deep learning approaches, the most similar work to Xen is the heterogeneous ensemble (HEE), in which the components are trained on the same dataset but use different algorithms/architectures~\citep{li2018heterogeneous,tabik2020mnist}. However, each component in an HEE is usually trained with an identical learning objective, while each expert in Xen is assigned with different learning tasks. In terms of result fusion strategies, unweighted model averaging is one of the most popular approaches in the literature~\citep{ganaie2022ensemble}, which simply averages the outcomes of the base learners to get the final prediction of the ensemble model. By contrast, we exploit a Kalman filter to further combat the noise in scores from different components in a time-series task, and the unweighted model averaging can be viewed as a special case of such a method at a point along the time axis.
\section{Methodology
\draftStatus{few fixes remain, revision highlights are in}
}

\boldify{at a high level, what did we do and who did it?}

In the third quarter of 2023, we crawled and scraped data via the \tosdr{} streaming API and BeautifulSoup~\cite{beautifulSoup}.
This data set contains numerous attributes, such as \textit{Description}, \textit{Case}, \textit{DocType}, \textit{Title}, \textit{Status}, \textit{Comments}, and \textit{Authors}.
\revised{%
While the scrape provides labels for fragments of privacy policies, this study will focus on just the \emph{Case} attribute: text that the \tosdr{} team has developed for a privacy concept, about a sentence in length.
This team has developed 245 such cases, which have more detail in a longer paragraph describing the case and offer a ``summary'' of the main ideas in the document.
} % end revised
However, our survey will only focus on 243 of these cases because there were two cases that the \tosdr{} team had not confirmed are actually present in any documents since they rejected all submitted identifications.

In order for people to learn about their ability to understand privacy-related concepts, as well as their impression of how negative or positive that concept was to the affected parties, we conducted a crowd worker survey%
\footnote{The survey is still available at \REDACT{\url{https://severityandunderstandability.ist.psu.edu/}}.}.
To that end, upon obtaining IRB approval, we recruited 519 individuals from Prolific, with inclusion/exclusion criteria that they must speak English fluently and reside in the USA.
The survey contained one attention check question, which would cause rejected work.
\revised{%
Survey responses took FIXME minutes and we paid \$FIXME for accepted work, for an average pay rate of \$FIXME/hr.
} % end revised
Out of the 519 participants, we rejected work from 19 of them and could not recover data from one more, leaving 499 participants.
\revised{%
We chose this sample size based on budget constraints, with a focus on understanding groups of people, meaning we had no intent to compute comparative statistics between cases.
The average time taken by the participants to fill this survey was about 5 minutes and the payment was \$3.} % end revised



\boldify{What core questions did we ask?}

The main task of the survey was to evaluate five random cases.
For each case, we presented the case label and then asked several questions about it.
We first asked about understandability: \textit{``On a scale of 1 to 10, how well do you understand this case? (1 means `Not understanding at all' and 10 means `Understanding this case completely')''}.
When asking about severity, we split the question into two pieces.
The first piece asked: \textit{``Which party does this case tend to favor?''} and participants responded by checking one of three mutually exclusive boxes: \texttt{User}, \texttt{Service Provider}, or \texttt{Neither}.
The second piece then asked: \textit{``How severe is the favoritism that party gains from this case?
(1 means `Not severe at all' and 10 means `Very severe')''}.
\revised{%
We chose this two-step question format to help clarify the participants' thinking by letting them subdivide the problem, akin to determining a vector's direction and then its magnitude.
} % end revised
For the last case, participants saw an additional question that asked them to rewrite the case based on both the case label and its description (\textit{Rewrite the last case (Case \#) in your own words.}).
The minimum requirement for text entry was 21 characters, which helps ensure that participants make a meaningful response.

\boldify{How did the survey flow and what are its constituent parts?}

After obtaining informed consent, the first page collects demographic information from the participant.
The second page gives instructions for the task alongside a single dummy case to clarify the task.
In particular, they answered our questions by checking boxes, repositioning sliders from 0 into the range 1--10, and rewriting one case.
Participants were not able to advance to the next page if any sliders remained at 0 since we consider default values a non-response.
Last comes the page containing the main task (i.e., the five random cases and the aforementioned questions for each case).
The main task page contains one attention check question, which was \textit{``Set the slider to position 7 (This is a special question not related to any case.)''}.

%%%%%%%%%%%%%%%%%%%%%%%%%%%%%%%%%%%%%%%%%%%%%%%%%%%%%%%%%%%%%%%%%%%%%%%%%
\subsection{Demographic and Attitude Information}

\boldify{Now we describe what demographic information we collect and where we got the questions}

In order to learn about what human traits might moderate responses, we asked about fundamental information such as age, occupation, student status, and gender\footnote{Though we provided multiple gender options in accordance with best practices~\cite{scheuerman2020gender}, nearly all participants identified as male/female, with a few preferring not to state.
} % end footnote

We also posed 11 questions to discover their level of awareness of privacy policies, listed later in Table~\ref{table: demoStats}.
We based several questions on language and categories from the US Census~\cite{UScensus}.
Furthermore, we adapted 11 questions adapted from GenderMag~\cite{burnett2016gendermag}, which specifically identifies five facets of users' problem-solving style: motivation, information processing style, learning style, computer self-efficacy, and risk aversion.

%%%%%%%%%%%%%%%%%%%%%%%%%%%%%%%%%%%%%%%%%%%%%%%%%%%%%%%%%%%%%%%%%%%%%%%%%
\subsection{Analysis}
\label{secAnalysis}

\boldify{how did we prepare the data?}
In order to collect the information on understandability, we asked participants their perceived understanding of each case presented to them, collecting the response on a Likert scale [1, 10].
We then calculated descriptive statistics to assess the understandability with respect to participants, as well as the cases.
After collecting the data, we applied a simple transformation to combine the two questions about severity.
We interpret the response to the second question (which party the case favors) as giving the \emph{sign}, while the Likert response on [1, 10] range gives the \textit{magnitude}.
Therefore, if a participant stated a case favored the \texttt{User}, we would use their severity response directly as a positive value.
However, if they said a case favored the \texttt{Service Provider}, we would negate their severity response.
Finally, if they said a favored \texttt{Neither}, then we would multiply their severity response by 0.
Ultimately, this creates a severity range from [-10, 10], with 10 indicating the case fully favors the User, and -10 fully in favor of the Service Provider.
However, at times we considered the absolute value to assess the statistics with respect to individual cases, each of which are marked clearly in the results.
We analyzed severity similar to understandability, with respect to participants and cases.
We also correlated understandability and severity to assess their impact on each other.

\boldify{And then what kind of analysis did we do?}

Much of our analysis relies on descriptive statistics, including mean, median, standard deviation, count of extremities (values of 1 and 10), and a direct comparison between the scores of understandability and severity and their corresponding means.
Since we used different lenses (per participant, per case, etc.), we extracted data from the raw data in different ways to support each perspective, resulting in different sets of data specific to the analysis variable.
\revised{%
Our last quantitative analysis was to perform hypothesis testing about the influence of demographic traits on participant responses.
Specifically, we checked normality with the Kolmogorov-Smirnov test and equivariance with Levene's test.
Fortunately, none of the distributions violated assumptions and we were able to proceed with parametric statistics, such as two-sample t-test and one-way ANOVA.
} % end revised


%Before conducting the analysis, we tested the assumptions necessary for parametric statistics to ensure the validity of the two-sample t-test. Levene’s test for equality of variances confirmed that the variance in understandability scores between the two groups was equal, while the Kolmogorov-Smirnov (KS) test for normality indicated that the distribution of scores met the assumption of normality. These results justified the use of a two-sample t-test to compare the means of the two groups.
% JED: commented because

\section{Result} \label{result}
In alignment with our research questions, we present the key findings in six dimensions, consisting of predefined themes based on the existing literature and emerging themes identified through thematic analysis: (1) factors influencing attitudes, (2) comparing AI-generated legacy and traditional digital legacy, (3) life cycle, (4) interaction, and (5) concerns. 

% 1,145
\subsection{Factors Influencing Attitudes: Person, Family, Technology and Society}\label{attitudes factors}

The attitudes of our participants towards the ``AI Afterlife'' as digital legacy are diverse and complex, being shaped by personal, familial, technological, and social factors, as shown in \autoref{tab:factors influencing attitudes}.
Generally speaking, personal views on life and death, along with concerns about the grief and loss of loved ones, significantly influence their perspectives. Furthermore, the desire to create lasting values and preserve family heritage can increase interest in ``AI Afterlife'', although this is also affected by family needs, technological capabilities, and broader social circumstances.

\begin{table*}[htbp]
  \caption{Summary of themes and main points in factors influencing attitudes on ``AI Afterlife'' as digital legacy.}
  \label{tab:factors influencing attitudes}
  \begin{tabular}{p{0.2\textwidth}p{0.75\textwidth}}
    \toprule
    \textbf{Theme} & \textbf{Main Point} \\
    \midrule
    \multirow{2}{=}{Views on Life and Death} 
            & Serve as a means of extending life and values (P1-3,11-14,17) \\ %\cline{2-2}
             & \cellcolor{gray!15} Blur the line between life and death (P4-5,7-10,15) \\
    \hline
     \multirow{3}{=}{Grief and Regret Endured by Family} 
            & Alleviate the grief of their loved ones (P2-3,6,12,14) \\ %\cline{2-2}
             & \cellcolor{gray!15} Prepare their loved ones for facing similar regrets (P2-4,11,15-16) \\ %\cline{2-2}
             & Attitudes varied with specific circumstances (P2,4,9) \\
    \hline
    \multirow{5}{=}{Sustained Values for the Living in Afterlife} 
            & \cellcolor{gray!15} Pass on practical knowledge and offer assistance (P1,3,13,14) \\ %\cline{2-2}
            & Provide daily companionship and emotional support (P6-7,10,13) \\ %\cline{2-2}
            & \cellcolor{gray!15} Lack relevance over time due to outdated knowledge (P8,15,17) \\ %\cline{2-2}
            & Limited emotional bandwidth to engage with them amidst busy lives (P5,8) \\
            & \cellcolor{gray!15} Impose mental burden and social pressure, disrupting daily life (P4, 8, 10, 12, 17-18) \\ %\cline{2-2}
    \hline
     \multirow{3}{=}{Bridging Family Heritage Across Generations} 
             & Maintain and share cherished family memories (P5,14,16,18) \\ %\cline{2-2}
             & \cellcolor{gray!15} Hesitant due to uncertainties about family dynamics (P7-8,15-16,18) \\ %\cline{2-2}
             & Influenced by evolving social trends (P15-18) \\ %\cline{2-2}
    \hline
    \multirow{3}{=}{Technical Capacity, Acceptance, and Needs} 
             & \cellcolor{gray!15} Simulation quality of human emotion and thought (P1,8,17) \\ %\cline{2-2}
             & Traditional attitudes toward new technologies (P11,14,16-18) \\ %\cline{2-2}
             & \cellcolor{gray!15} Match the features of a specific technology with personal needs (P1-2,7-8,14,17) \\
    \bottomrule
    % Add more rows as needed
  \end{tabular}
\end{table*}

\subsubsection{Views on Life and Death.} \label{Views on Life and Death}

\ly{Participants' perceptions of life and death impact their acceptance of ``AI Afterlife'' as digital legacy. }
% positive
\ly{Those with positive opinions emphasized the spiritual aspect of life over the physical aspect of death, and viewed ``AI Afterlife'' as a means to extend their life and values (P1-3,11-14,17),} with P3 and P12 both expressing, \textit{``As long as I am remembered, my life continues.''} 
\ly{From their perspective, ``AI Afterlife'' provides opportunities to enhance the remembrance of both their image and spirit through more interactive and engaging connections with the living. }
P17 regarded experiences as the most valuable assets, and wished, \textit{``The emergence of digital human allows my experiences to be re-told and my life to leave traces in the lives of my children.''}

% reserved
\ly{In contrast, participants with more reserved attitudes adhered to a strict definition of the boundary of death, focusing on its physical aspect and highlighting the natural cycle of life (P4-5, 7-10, 15).}
Just like the metaphor P15 made, \textit{``My life is like the burning of a candle, and my death is like the extinction of it.''}
\ly{However, ``AI Afterlife'', which allows people to engage with the world after death, might blur the line between life and death, challenging the traditional understanding of death as a natural endpoint.}
\ly{For example, P5 noted it might diminish the value of actual living,} \textit{``If my value continues to be highlighted through AI-generated agents after my death, what is the meaning of my biological life?''} 
\ly{Similarly, P10 expressed his concern on the potential consequence of hindering natural grieving and lessening the value of cherishing loved ones in life,} \textit{``The pain caused by the death of loved ones is part of the life experience... the existence of my agents may make my children not cherish me when I am alive...''} 

\subsubsection{Grief and Regret Endured by Family.} \label{Grief and Regret Endured by Family}

\ly{The grief and regret imagined by our participants after death emerged as a key factor influencing their attitudes toward ``AI Afterlife'' as digital legacy. }
% positive
\ly{Consistent with \cite{xygkou2023conversation} where the bereaved used chatbots of the deceased for seeking support, some desired to leverage their interactive and high-fidelity agents to alleviate the grief of their loved ones (P2-3, 6, 12, 14).}
\ly{In particular, those who experienced regrets, such as accidents, separation, or misunderstandings, during their bereavement wished that ``AI Afterlife'' could prepare their loved ones for facing similar regrets (P2-4, 11, 15-16).}
For instance, P2 regretted not preserving information about her father, who died in an accident when she was young, due to limited technology. She said, \textit{``My memories of him have faded... If I could use AI to preserve my digital being, future generations would have fewer regrets when they think of me.''}

% adaptive
\ly{Further, participants with more nuanced observations of grief noted that their attitudes varied with specific circumstances, such as the cause of death (P2), age at death (P4), or duration of bereavement (P9)}
They found ``AI Afterlife'' based digital legacy more necessary in cases of sudden or early death, due to the potential for greater family regret and loss, but saw it as less needed after a long, peaceful life.
P4 took her family for example, stating, \textit{``My parents are quite old and content with their lives, so there's little regret... But if I die suddenly, my husband and children would definitely need my agent to cope with loneliness.''} 
However, although \cite{xygkou2023conversation} highlights the potential of it to ease sharp grief, P9 cautioned, \textit{``My family may not accept my agent right after my death. In the future, occasionally recalling me through my agent might help them better cope with grief.''}

% \ly{In summary, participants experiencing sudden loss or unresolved regrets desired for ``AI Afterlife'' based digital legacy alleviate future grief, while those living longer and more peacefully with less regret felt less need.}

\subsubsection{Sustained Values for the Living in Afterlife.}

Some participants with positive attitudes envisioned ``AI Afterlife'' based digital legacy as a means to provide sustained value for their family, extending beyond grief support. Practical uses included passing on knowledge, such as cooking techniques (P1), work lessons (P13), and offering assistance, such as resolving disputes (P14) or caring for grandchildren (P3). The potential daily companionship and emotional support were also highlighted, allowing families to feel their presence and continue receiving love (P6-7, 10, 13).

\ly{However, these values were seen as limited by several factors. Some participants questioned the everyday relevance of such agents, noting that their utility might diminish over time due to outdated knowledge (P8, 15, 17) or the family's limited emotional capacity to engage with them amid busy lives (P5, 8).}
\ly{Some also worried that maintaining regular interactions with these agents might impose a mental burden and even social pressure, disrupting daily life rather than providing comfort (P4, 8, 10, 12, 17-18). }
For example, P8 explained, \textit{``As an agent represents whom family misses, they might feel guilty and conflicted if they don't use it for a long time.''} 
Further, when ``AI Afterlife'' becomes mainstream, P8 added, \textit{``Others might judge one's level of respect and nostalgia based on how often they interact with the agent''} . 

% In summary, while participants saw ``AI Afterlife'' as a way to pass on knowledge and emotional support, concerns about outdated relevance, limited emotional bandwidth, and unintended mental burden and social pressures highlight the need for designs that balance long-term value with minimal disruption to daily life.

\subsubsection{Bridging Family Heritage Across Generations.}

\ly{Preserving and passing on family heritage emerged as a key motivation for creating ``AI Afterlife'' based digital legacy. Participants highlighted that AI-generated agents could play an essential role in connecting generations by maintaining and sharing cherished family memories that might otherwise fade over time (P5, 14, 16, 18).} As P16 explained, \textit{``After a person dies, they are gradually forgotten by future generations; however, AI-generated agents can preserve precious family memories, providing an interactive and evolving way to pass down family history for generations.''}

\ly{While this vision was encouraging, participants expressed hesitation due to concerns about its acceptance by younger generations, influenced by family dynamics (e.g., wishes, relationships) and social trends. For example, some were unsure whether their children would value such legacies or consider them overly sentimental (P7-8, 15-16, 18). In particular, many older participants noted a decline in younger generations' interest in spiritual family connections, attributing this shift to changing societal values (P15-18).} P17 highlighted challenges in evolving cultural contexts, \textit{``Traditional practices like funeral rituals and visiting graves during Qingming Festival\footnote{During Qingming, Chinese families visit the tombs of their ancestors to clean the gravesites and make ritual offerings to their ancestors.} are fading, as many young people prioritize advocated modern simplicity over spiritual remembrance.''}

\subsubsection{Technical Capacity, Acceptance, and Needs.}

Technical issues, such as capacity, personal openness to new technology, and individual needs, influenced participants' attitudes toward ``AI Afterlife'' based digital legacy. 
\ly{Above all, simulation quality was deemed critical, especially when compared to interactions with emotionally thoughtful real people.} P7 emphasized, \textit{``If AI can fully simulate my emotions and intelligence, I support it providing emotional comfort to my family; however, if the simulation ability is only half-baked, it could be counterproductive and hurt their feelings.''} Others similarly noted that inadequate imitation might fail to deliver positive experiences (P1, 8, 17).

\ly{Traditional attitudes toward advanced technologies also influenced participants' acceptance. Some were highly open to embracing innovative technologies (P11, 14, 16-18), even envisioning various possibilities for how ``AI Afterlife'' might shape their lives. (P11, 17-18).} For instance, P11 shared, \textit{``People make choices throughout lives. When growing old, they often find that lives are full of regrets. Therefore, I want my agent to revisit my past life moments, exploring the outcomes of different choices - what would happen if I chose to go to graduate school? or if I chose another girl as my partner?''}

\ly{Moreover, participants' attitudes were shaped by how well technology features aligned with their personal needs for digital legacy.} Some felt that remembrance through spirits, artifacts, or audio and video files was sufficient (P7-8, 17). Others believed that an AI-generated agent could better organize and preserve scattered information (P1-2, 14). As P2 explained, \textit{``It is meaningful to systematize fragmented information for preserving myself and for the nostalgia of the living.''}

\subsection{Comparing ``AI Afterlife'' with Traditional Digital Legacy}\label{comparision ai and traditional}

AI-generated and traditional digital legacy exhibit three key differences: (1) data distribution - integrated versus scattered, (2) content authenticity - generative versus authentic, and (3) interaction form - interactive versus static.
\ly{These differences underscore their unique strengths and limitations. Participants consistently emphasized that neither approach can fully replace the other due to their distinct expressive tendencies (P1, 7, 9, 12, 16). 
% For example, P7 noted, \textit{``Generative AI cannot replace real images.''} 
They advocated for a context-dependent approach, allowing users to select the form that best meets their needs or combine both approaches to maximize their complementary benefits.}

\subsubsection{Integrated and Scattered.}

AI-generated digital legacy presents an integrated approach by consolidating vast amounts of dispersed, easy-to-lose digital information, often incorporating elements of traditional digital legacy as part of AI model training. This integration enhances accessibility and usability (P2, 14), enabling users to derive more value from their digital archives, and potentially motivate the collection and preservation of traditional digital legacy.

\ly{However, limitations arise due to the exclusion of sensitive or private information}, such as passwords, which are typically shared directly and remain outside the scope of AI models (P7, 13). 
\ly{Additionally, the significant costs associated with data collection, model training, and maintaining AI-based services can pose challenges to accessibility, particularly for users with limited resources, thereby exacerbating technical inequities (P6, 16).}
As P6 observed, \textit{``For ordinary people, traditional digital legacy is more cost-effective''}.
%
\ly{This comparison highlights the tension between the potential benefits of integrated features of generative AI and the practical limitations related to privacy concerns and accessibility.}

\subsubsection{Generative and Authentic.}

\ly{Generative AI advancements enable agents to simulate human-like interactions with increasing realism, offering participants a sense of ongoing presence and control over their posthumous connections (P2-3, 10-11, 16).
For instance, P3 highlights the opportunities for continuity and emotional support,} \textit{``Through my AI agent, I would never truly leave the world, as I could continue to actively interact with and accompany my family.''} 

\ly{However, participants emphasized that these simulations, no matter how sophisticated, lack authenticity because they do not originate from genuine experiences or personal histories unique to the deceased  (P2, 4, 6, 14, 18). This gap in authenticity makes traditional digital legacy irreplaceable in preserving deeply personal and emotional connections. }
As P4 noted, \textit{``Traditional digital legacy like old photos that capture genuine details and history, serve as irreplaceable carriers of spiritual sustenance and cherished memories.''}
%
\ly{This comparison highlights the tension between the potential of realistic generative AI simulations and the irreplaceable value of traditional legacies in preserving authentic emotional connections tied to real-life experiences.
}


\subsubsection{Interactive and Static.} \label{Interactive and Static}

AI-generated digital legacy introduces interactive features that transform memories into dynamic and responsive experiences, fostering companionship and emotional support (P4, 8, 10, 13). 
P4 remarked, \textit{``This interactivity enables meaningful responses to thoughts and emotions, easing loneliness and offering comfort during grief.''} 
\ly{These interactions mimic real conversations, offering users a unique sense of connection.}

\ly{However, participants raised concerns about the potential risks of prolonged engagement, which could deepen grief instead of alleviating it. }
P8 highlighted, \textit{``Engaging with an AI agent involves back-and-forth exchanges like real conversations, entering and exiting such interactions is gradual. If abruptly interrupted, it can feel awkward and disjointed.''} 
In contrast, traditional digital legacies are static but offer the living the freedom to pause or revisit memories at their own pace, unburdened by the pressures of interaction. As P8 and P18 suggested, these static artifacts allow people to engage with memories selectively, integrating them seamlessly into daily life without emotional overwhelm.
\ly{This comparison emphasizes the need to balance the immersive interactivity of generative AI with the emotional neutrality and flexibility of static digital legacies.}

\subsection{Life Cycle of ``AI Afterlife'': Encode, Access, and Dispossess} \label{life cycle section}
As proposed in \cite{doyle2023digital}, the life cycle of digital legacy consists of three stages: encode, access, and dispossess. ``Encode'' refers to converting legacy content into a digital format, ``access'' involves retrieving that content, and ``dispossess'' pertains to its transfer or deletion. The themes identified in our study align with these stages, allowing us to categorize our findings accordingly. This alignment is illustrated in \autoref{fig:life cycle} and summarized in \autoref{tab:summary of themes and main points in life cycle}.

\begin{figure*}[htbp]
	\centering 
	\includegraphics[width=\linewidth]{figures/lifecircle.pdf}
	\caption{Life cycle of ``AI Afterlife'' as digital legacy, including encoding the legacy content into a digital form, accessing that content, and the dispossession of that legacy to another individual or through deletion.}
	\label{fig:life cycle}
        \Description{}
\end{figure*}

\begin{table*}[htbp]
  \caption{Summary of themes and main points in the life cycle of ``AI Afterlife'' as digital legacy.}
  \label{tab:summary of themes and main points in life cycle}
  \begin{tabular}{p{0.1\textwidth}p{0.85\textwidth}}
    \toprule
    \textbf{Theme} & \textbf{Main Point} \\
    \midrule
    \textbf{Encode}&\\
    \hline
    \multirow{3}{=}{Author} 
            & Manage the creation themselves for security, simulation accuracy, preferences (P1,3,6-7,10-14,16-18) \\ %\cline{2-2}
             &\cellcolor{gray!15}Allow family to retain the creation right in case age-related declines or unforeseen accidents (P2-4,7,14)  \\ %\cline{2-2}
             & Leave the decision to family, allowing flexibility in creation (P8,12,18) \\
    \hline
    \multirow{4}{=}{Time} 
            & \cellcolor{gray!15}Create in daily life to get enough data preparation time and better control over agents (P2,4,7,10-11,17) \\
            & Create in later life or near death for necessity (P1,15) \\ %\cline{2-2}
             & \cellcolor{gray!15}Leave it posthumously due to concern about family's acceptance and needs (P8,12,14,18) \\ %\cline{2-2}
             & Postpone the process out of prioritizing family needs during life (P4,6) \\
    \hline
    \multirow{5}{=}{Content}
            & \cellcolor{gray!15}Extensive existing digital information (P1-18) \\ %\cline{2-2}
            & Document personal details such as physical attributes and habitual behaviors (P1, 3, 6, 10) \\ %\cline{2-2}
            & \cellcolor{gray!15}Incorporate data from close family (P1-2, 4, 6, 8, 14) and social connections (P7, 10, 13, 16, 18)  \\ %\cline{2-2}
            & Insufficient data for simulating deeper human attributes, expect for P17 \\ %\cline{2-2}
            & \cellcolor{gray!15}Challenging for those willing to record and organize deeper information (P3-4,6-7,11,13,17) \\
    \hline
    \multirow{5}{=}{Selection criteria}
            & Sincerity is essential to ensure authenticity (P1,7-8,18) \\
            & \cellcolor{gray!15}Daily trivia is viewed as irrelevant for future use (P13,17) \\ %\cline{2-2}
            & Disgraceful experiences are commonly excluded (P13-14,16) \\ %\cline{2-2}
            & \cellcolor{gray!15}Painful memories and imperfect personality traits are filtered out to avoid unpleasantness (P4,6,13) \\ %\cline{2-2}
            & Confidential information, such as bank passwords, was universally excluded (P1-18) \\ %\cline{2-2}
    \hline
    \multirow{4}{=}{Update}
            & \cellcolor{gray!15}Incorporate new data or address inconsistencies arising from real-life interactions (P1-18) \\ %\cline{2-2}
            & Keep the update right private or limited to their closest family (P10-11,14,16) \\ %\cline{2-2}
            & \cellcolor{gray!15}Extended the update right to a wider circle of relationships (P7,11-13,16,18) \\ %\cline{2-2}
            & Updates often undergo review by an assigned manager with family-centered priorities (P7,13) \\
    \hline
    \textbf{Access}&\\
    \hline
     \multirow{3}{=}{Audience} 
            & \cellcolor{gray!15}Open-minded about who has access, prioritizing the wishes of others (P1,3,8,15,18). \\ %\cline{2-2}
             & Impose restrictions by closeness (P10,12), communication value (P2,4,6,13,16), and risks (P6,9,13-14) \\ %\cline{2-2}
             & \cellcolor{gray!15}Different views in who can set access permissions - self (P1-18), maintainer (P7,13,16), AI (P1,7) \\
    \hline
    \multirow{3}{=}{Platform} 
            & Virtual spaces, like social media, memorial websites, mobile apps, XR apps (P1-18) \\ %\cline{2-2}
             & \cellcolor{gray!15}Robots in the domestic setting, controlled by an assigned manager (P6,9,16) \\ %\cline{2-2}
             & Combined with traditional memorial artifacts, like gravestones (P1) and plaques in ancestral halls (P6) \\
    \hline
    \textbf{Dispossess}&\\
    \hline
    \multirow{3}{=}{Preservation or deletion} 
             & \cellcolor{gray!15}Based on the wishes of the living (P2,6-10,16-17) \\ %\cline{2-2}
             & Based on the depth of communal connections and affection (P3-4,14,17-18) \\ %\cline{2-2}
             & \cellcolor{gray!15}Preserve permanently with concerns about reputation, capabilities, and maintenance cost (P1-2,9)\\
    \hline
    \multirow{3}{=}{Transfer} 
            & Specifying transfer details in the will (P10-11)  \\ %\cline{2-2}
            & \cellcolor{gray!15} Rights transferred, such as author (P6-7, 10-12), update (P10-11, 14), and audience (P1-2, 7) \\
            & Closest family members are the only recipient choice (P1-18) \\ %\cline{2-2}
    \bottomrule
    % Add more rows as needed
  \end{tabular}
\end{table*}

\subsubsection{Encode}

In the context of ``AI Afterlife'' as digital legacy, ``encoding'' refers to the process of gathering and using digital information to train generative AI models that simulate realistic human beings. Key aspects of the encoding process include author, time, content, selection criteria, and updates.

\textbf{Author.} \label{author}
% control
\ly{Many participants preferred to manage the creation process rather than leaving the whole task to their families.} They believed self-creation ensured data security (P1, 3, 6), maintained simulation accuracy (P6-7, 10, 12), and aligned with personal preferences (P13, 14, 16, 18). P6 emphasized the importance of controlling data to prevent privacy leaks, while P7 noted, \textit{``Without my involvement, the simulation might be biased and lack authenticity.''}
% reserve right
\ly{However, in anticipation of potential challenges such as age-related decline (e.g., mobility issues, decreasing digital skills) (P3-4, 14) or unforeseen accidents (P2, 7), they considered allowing their trusted family to retain the creation right, ensuring that the creation would align with their intentions even in the absence of their active involvement.}
P2 mentioned, \textit{``It doesn't matter who created my agent because it's impossible to predict whether tomorrow or an accident will come first.''} She believed her role was to keep her digital information organized for future flexibility.
% totally transfer
\ly{Additionally, some who were less motivated to create agents for themselves and primarily for family purposes, preferred to leave the decision to their family, allowing them to create the agents as they wished. Uncertainty about whether their family would accept the agent further reinforced this choice (P8, 12, 18).} P12 explained, \textit{``It's not something obligatory... If my family wants it, they can create it themselves. If they don't feel connected to me, making it for them would be meaningless.''}

\textbf{Time.}
% daily life
\ly{Many participants preferred involving the creation process as part of daily life (P2, 4, 7, 10-11, 17), rather than waiting until later life or near death (P1, 15).} 
This aligns with findings under the ``Author'', where earlier involvement allowed for better control over their agents. 
Besides, P7 noted that prolonged data collection led to more realistic simulations. Similarly, P11 emphasized, \textit{``It is necessary to collect data 10 years in advance... It is impossible to collect data one or two months before death.''} 
% after death
\ly{In contrast, some considered delaying the creation after death, leaving the creation decision entirely to their family due to the concerns about their acceptance and needs (P8, 12, 14, 18), or prioritizing family needs over self agent creation during life (P4, 6).}
For example, P6 expected to postpone creating her agent until completing her parents' agents, while P4 prioritized spending time with family and expressed concern, \textit{``My agent might disrupt my family's normal life after my death... It should be created only after my family has recovered from grief.''}

\textbf{Content.} \label{content}
% physical features
Most participants could provide extensive data stored on digital devices, such as media (photos, audio, videos), files (writing samples), and social media accounts (chat history, updates). Many were also willing to document personal details such as physical attributes (e.g., height, weight, skin texture) and habitual behaviors (P1, 3, 6, 10). Despite challenges in data recording and preservation, such as insufficient attention (P8, 16), data loss during migration, and deletion due to limited storage (P1, 5, 15), the available data is generally sufficient for simulating users' physical features and facilitating basic conversations, owing to advancements in generative AI. Most participants were also open to incorporating data from close family (P1-2, 4, 6, 8, 14) and social connections (P7, 10, 13, 16, 18) to enhance simulation performance.
% thought
However, simulating deeper human attributes, such as personal experiences and inner thoughts, presents greater challenges. Only P17, who has maintained a daily diary since his twenties, could provide such data directly. While others expressed willingness to record details of their experiences and thoughts (P3-4, 6-7, 11, 13, 17), barriers such as memory gaps and the time required for organization were significant. 
% \ly{This contrast highlights the disparity between easily accessible, physical data and more abstract, introspective data, the latter of which is more difficult to collect and organize.}
For example, P18 was open to providing existing data but hesitant to organize her life story, stating, \textit{``I don't like writing down my thoughts or experiences. Daily communication already covers them. Creating my AI agent doesn't motivate me to spend time on this.''}

\textbf{Selection criteria. } \label{selection criteria}
\ly{Although the importance of sincerity to ensure their agents remain authentic is emphasized (P1, 7-8, 18), most participants considered several factors when selecting digital information for AI-generated agents, including daily trivia, disgraceful experiences, painful memories, imperfect traits, and confidential data.} Many deemed daily trivia irrelevant for future use (P13, 17), with P17 stating, \textit{``People interacting with my agent would likely be more interested in my thoughts than in what I did at a specific time.''} \ly{Disgraceful experiences, such as sharp remarks, conflicts of interest, and harm caused to others, were expected to be excluded to avoid potential ethical, economic, or legal issues (P13-14, 16).} Painful memories (P4, 13) and imperfect traits (P6) were also filtered out to minimize unpleasant interactions. P6 explained, \textit{``I want my agent to get along better with my family because I have a bad temper... I don't want to bring back the unpleasant memories of our quarrels.''} Confidential information, such as bank passwords, was universally excluded, as P7 stated, \textit{``They are just secrets that have no meaning in representing my true self.''} 

\textbf{Update.} \label{update}
\ly{Most participants emphasized the necessity of updating their agents after the initial version was created, either to incorporate new data or address inconsistencies arising from real-life interactions to enhance the agents' comprehensiveness and authenticity.}
% only self/closest family
\ly{Opinions on update permissions were diverse, reflecting a balance between individual control and relational dynamics. }
Some participants preferred to limit this right to themselves or their closest family members (P10-11, 14, 16). For example, P14 stated, \textit{``I don't want others to update my agent, as everyone has different ideas and may not understand me; otherwise, it might not align with my intentions.''} P16 extended this right to his son, who knew him well.
% others
In contrast, other participants extended the update right to a wider circle of relationships (P7,11-13,16,18), with P7 noting, \textit{``A person is defined by his social connections.''} 
These updates were often subject to review by a trusted manager (P7,13). 
\ly{During the review process, family-centered priorities often took precedence over strict authenticity to ensure the agent served the emotional and practical needs of the living.}
For instance, P13 explained, \textit{``The purpose of creating my agent is to accompany my son, so the updates must be acceptable to him. If he finds them distasteful, they will not be approved.''}

\subsubsection{Access}
In the context of ``AI Afterlife'' as digital legacy, ``access'' refers to who can access and how to access AI-generated agent-based digital legacy. We identify two key aspects - audience and platform in this process. 

\textbf{Audience.} \label{audience}
\ly{Participants indicated that access to their agents is typically granted in descending order of priority: close family, common connections, and strangers. There are several factors influencing access permissions significantly.} 
% open minded
\ly{Some were open-minded about who could access their agents, prioritizing the wishes of others (P1, 3, 8, 15, 18).} For instance, P8 remarked, \textit{``I'm flexible about who has access rights. Only those with a deep relationship with me will remember and miss me. If the relationship is ordinary, they likely wouldn't interact with the agent of a deceased individual.''}
% with constrains
\ly{In contrast, those prioritizing practical values and risk prevention set restrictions based on relationship closeness (P10, 12), communication value (P2, 4, 6, 13, 16), and concerns about reputation, security, and legal issues (P6, 9, 13-14).} P6 restricted access for strangers, explaining, \textit{``Few strangers would be interested in my story... there is also a risk of data leakage.''} P9 highlighted legal concerns, stating, \textit{``Currently, the legal rights and obligations of this technology are undefined... \ly{For example, if it infringes upon others' rights in chat, determining responsibility can be difficult... From a risk management perspective, access should be restricted.''}}
% who can set the permission
\ly{There were also differing views on who should set access permissions.} All agreed that the individual being represented should initially hold this right. Once the legacy is transferred, those who maintain it should manage it (P7, 13, 16). However, some suggested once AI-generated agents develop their own thoughts, they should have priority (P1, 7). P1 stated, \textit{``I hope my agent can choose who to interact with and for how long, rather than simply following fixed rules... It should behave like an AI agent with its own personality, not just an AI assistant.''}
\ly{It provides a glimpse into a more complex issue of autonomy after AI agents have developed independent consciousness.}

\textbf{Platform.}
% virtual space
\ly{Participants envisioned a diverse set of digital platforms for their agents to be accessed.}
\ly{Most participants supported their agents to be accessed in virtual spaces, through various commonly, easy-to-use online platforms such as social media, memorial websites, and specific mobile or Mixed Reality (XR) applications.} Among them, WeChat \cite{WeChat2024}, one of China's most popular social media, was been regarded as the most suitable platform to embed it due to convenience and familiarity (P1, 3-4, 7, 10, 13-14, 16). P16 explained, \textit{``Since most people communicate through WeChat, it would be very convenient if my family and friends could access my agent directly through it...They could also use familiar chat and audio/video call functions to interact with my agent.''} P14 added, \textit{``Using other platforms complicates things (e.g., requiring new accounts). In contrast, WeChat is great for remembrance, as it retains previous chat history.''} 
Besides, P11 preferred embedding his agents in XR glasses, while others had no particular preference in these virtual spaces.
% tangible: robot
\ly{Another significant consideration was the use of robots to access these agents at home, managed by an assigned person (P6, 9, 16), offering a tangible presence for interactions (more details in section \ref{embodiment})}
% combined with traditional memorial in fixed places
\ly{Some participants also proposed combining AI agents with traditional memorial artifacts, such as gravestones (P1) and plaques in ancestral halls (P6). This approach preserves traditional commemorations while recreating their meaning in new contexts.}
P6 envisioned, \textit{``If ancestral halls had plaques with AI-generated agents of older generations, it could create a sense of large family during gatherings... It could also preserve and pass down family heritage in a more personal and face-to-face way.''}

\subsubsection{Dispossess}

In ``AI Afterlife'' as digital legacy, ``dispossess'' refers to transferring the AI-generated agent-based digital legacy, or deleting them when no longer needed. We identify deletion conditions and transfer preferences.

\textbf{Preservation and deletion.}
% user's wishes & communal connection and affection
\ly{Participants consider whether to preserve or delete the agents based on the living's wishes (P2, 6-10, 16-17) and the depth of their relationships (P3-4, 14, 17-18).} P16 speculated that three generations might be the maximum for transferring his agents, stating, \textit{``I respect users' wishes when deleting it... Once there's no willingness to use it, its meaning fades.''} P3 added, \textit{``Only when people are close will they develop a desire to communicate.''} 
% preserve permanently
\ly{In contrast, some wished to preserve their agents permanently but were concerned about reputation, capabilities, and maintenance costs (P1-2, 9).} P2 worried, \textit{``If I were a person of prestige in my family or society, I might pass down it through generations... but I'm just an ordinary person without the means to maintain it.''} 

\textbf{Transfer.}
\ly{Transferring the agents with essential details between generations is significant, such as the approaches, recipients, and rights.} P10 and P11 emphasized the importance of specifying transfer details in their wills, stating, \textit{``The transfer process should respect my wishes. I would set them in advance, such as who can inherit them and what rights can be transferred.''} 
Most participants preferred to pass on their agents to close family members. In terms of rights transferred, they typically include topics mentioned above in section \ref{life cycle section}, such as author assignment (P6-7, 10-12), update rights (P10-11, 14), and audience permission (P1-2, 7). 


\subsection{Interaction Design: Replicating, Simulating, and Responding}
\label{sec: interaction design}
Exploring the interactive nature of ``AI Afterlife'' as digital legacy, we delved into how it replicates, simulates, and responds to aspects of self and interactions with external environments. We identified six themes as shown in \autoref{tab:summary of themes and main points in interaction}: age and appearance, embodiment, simulation level, expression of extra knowledge, evolution, and reactivity.

\begin{table*}[htbp]
  \caption{Summary of themes identified in participants' anticipated interaction designs for their ``AI Afterlives''.}
  \label{tab:summary of themes and main points in interaction}
  \begin{tabular}{p{0.15\textwidth}p{0.8\textwidth}}
    \toprule
    \textbf{Theme} & \textbf{Main Point} \\
    \midrule
    \multirow{5}{=}{Age and Appearance} 
            & Customized to reflect the most familiar and meaningful stages for the intended audience (P2,3,6) \\ %\cline{2-2}
             & \cellcolor{gray!15}Customized to be status shortly before death to continue their presence (P5,7)  \\ %\cline{2-2}
             & Customized to be healthy and energetic status to avoid evoking unsettling memories (P3,6,16) \\ %\cline{2-2}
             & \cellcolor{gray!15}Customized to be other expected status like youthful, mature, and pleasing (P2,4,8,10,13-14,16) \\ %\cline{2-2}
             & Dynamically adaptive to different users and interaction needs (P1-2,6,11-12,14-18) \\
    \hline
    \multirow{6}{=}{Embodiment} 
            & \cellcolor{gray!15}Virtual forms offer more controllable engagement with memories of loved ones (P3-4,8) \\ %\cline{2-2}
            & Virtual forms are sufficient for remembrance (P6,18) \\ %\cline{2-2}
             & \cellcolor{gray!15}Physical forms associated discomfort (P4,6,9)  \\ %\cline{2-2}
             & Physical forms related ethical concerns, particularly how to treat AI-generated agents (P7,18) \\ %\cline{2-2}
             & \cellcolor{gray!15}Physical forms also have a higher maintenance cost (P6) \\
             & Adaptability of forms based on age, individual needs, and the duration of grief (P6,9,14) \\
    \hline
    \multirow{3}{=}{Simulation Level} 
            & \cellcolor{gray!15}Simulating the surface features is not enough (P1-5,7-18) \\ %\cline{2-2}
             & It is needed to enable AI-generated agents with thinking ability (P1-8,10-18) \\ %\cline{2-2}
             & \cellcolor{gray!15}Advocate for a balance between authenticity and discretion (P1,7-8,18) \\
    \hline
    \multirow{3}{=}{Expression of Extra AI Knowledge} 
            & Create discrepancies from real self in aspects like knowledge, capabilities and values (P1,16-17) \\ %\cline{2-2}
             & \cellcolor{gray!15}Enhance usefulness and interaction quality by adding knowledge and cultural depth (P6,12,18)  \\ %\cline{2-2}
             & Balance between personal characteristics and additional knowledge (P8,11,13-14) \\
    \hline
    \multirow{3}{=}{Evolution} 
            & \cellcolor{gray!15}Stay presence and maintain relevance in communication (P1,3-4,6-8,12-14,16-18) \\ %\cline{2-2}
             & Bring uncontrollable risks, authenticity and identity issues, and philosophical reflection (P9-11) \\ %\cline{2-2}
             & \cellcolor{gray!15}Adaptive based on life stages (P2-4) \\
    \hline
    \multirow{3}{=}{Reactivity} 
            & Autonomously sense and engage with their environment (P1-3,6,11) \\ %\cline{2-2}
             & \cellcolor{gray!15}Potential intrusiveness of active reactivity (P4,9-10,12-14,16-18) \\ %\cline{2-2}
             & Advocating for a balanced approach based on needs, relationship, and preparation (P4,7-8,11) \\
    \bottomrule
    % Add more rows as needed
  \end{tabular}
\end{table*}

\subsubsection{Age and Appearance} \label{age and appearance}
Most participants wanted to customize their agents' age and appearance to reflect familiar and meaningful life stages. Many preferred their agents to represent key moments for their intended audience (P2, 3, 6). For example, P2 said, \textit{``If my agent is for my children, I would choose the time when I took care of them... the most profound memory is when they were with us before college.''} Some also preferred to depict themselves shortly before death to maintain a sense of presence (P5,7), as P5 explained, \textit{``It feels like I haven't left them and can continue growing with them.''} Others wanted a healthy, energetic appearance to avoid negative associations with declining health (P3, 6, 16), or considered youthful, mature, or pleasing looks (P2, 4, 8, 10, 13-14, 16).

In contrast, some participants suggested that their agents should dynamically adapt based on personalized experiences with different users (P1-2, 6, 11-12, 14-18). P12 explained, \textit{``As I interact with different people at various life stages, the memories I share with each occur at different times, so I can't present the same image to everyone.''}


\subsubsection{Embodiment}\label{embodiment}

Most participants preferred virtual forms over physical embodiment for comfortable remembrance, ethical concerns, and affordability. Virtual agents allowed controllable engagement with memories (P3-4, 8). P3 explained, \textit{``With a virtual agent, my family can reach out when they miss me... I don't want a physical presence constantly reminding them of my death.''} Many felt virtual agents were sufficient, with P18 noting, \textit{``The connection between my family and me is primarily spiritual, not physical.''}
Participants also expressed discomfort with physical forms (P4, 6, 9). P4 stated, \textit{``When seeking comfort, there's an awareness that a physical agent isn't truly the loved one, which can induce fear.''} Ethical concerns included managing the physical agents of the deceased (P7, 18). P7 highlighted complexities in \textit{``handling and storing them resembling family members.''} P18 added, \textit{``Physical forms might show disrespect for the body... breakage could cause trauma.''} Economic factors were also noted, with P6 mentioning higher maintenance costs.

\ly{Adaptability based on age, needs, and grief duration was also discussed (P6, 9, 14). P9 noted that her physical agent could better support her husband in old age due to accessibility and lower digital literacy. Similarly, P6 noted, \textit{``Physical forms might support my young children but become burdensome over time.''} P14 added that while physical forms offer temporary comfort, they may not sustain long-term emotional well-being.}


\subsubsection{Simulation Level} \label{simulation level}

Different simulation levels during interaction have been explored, including external physical features, thinking, and privacy. Few participants believed simulating physical features alone is sufficient, except for P6, who suggested, \textit{``Replacing the voice of a voice assistant with my own, such as in navigation apps, would make interactions feel more personal and engaging... ChatGPT's voice always feels cold and lacks variety''}

\ly{Besides, participants agreed that agents should exhibit personal ``thinking'' ability. P8 stated, \textit{My agent should reflect my personal characteristics and family knowledge, generating content based on my style, beliefs, and values. If it only has AI knowledge, interacting with it would be no different from using ChatGPT, which is confusing.''} P9, with a background in philosophy, noted, \textit{``Philosophically, life is defined by thought, so achieving this simulation level may require rethinking boundaries of life and death, along with related social, legal, and societal issues.''}}

Regarding privacy, participants advocated for a balance between authenticity and discretion (P1, 7-8, 18). While some preferred minimal data filtering for a more authentic simulation rather than a ``saint simulation'', they cautioned against including much sensitive information to protect privacy. 

\subsubsection{Expression of Extra AI Knowledge} \label{ai knowledge}

Participants had mixed feelings about incorporating external knowledge embedded in large foundational models into their agents. Concerns focused on its impact on interaction quality and authenticity (P1, 16-17). P1 noted, \textit{``Embedding this knowledge may lead to a misrepresentation and incomplete understanding of values and knowledge...''} P16 shared a similar view, stating additional knowledge might create discrepancies between the agent and the real self, but also noted, \textit{``Not including it may make the agent seem outdated.''} P17 questioned the practicality, saying, \textit{``It might discuss topics with my children that I'm unfamiliar with, making interactions feel odd.''}

On the other hand, integrating AI knowledge could enhance interaction quality (P6, 12, 18). P6 believed it would make the agent more useful and appealing to her family, while P18 felt it \textit{``enriches the agent's knowledge and cultural depth.''} This was particularly useful when \textit{``interaction scenarios require feedback that can't be predetermined''} (P12).

Some participants favored a balance between personal traits and added knowledge (P8, 11, 13-14). P8 suggested blending personal traits with extra information, rather than relying solely on AI knowledge. Similarly, P11 emphasized, \textit{``Extra AI knowledge should align with my perception of knowledge.''} P14 welcomed the integration, saying, \textit{``As long as it enhances interaction quality without compromising my essence.''} P13 also noted the need to avoid negative traits, adding, \textit{``If the AI knowledge is more positive and insightful than mine, it could be beneficial.''} 

\subsubsection{Evolution} \label{evolution}

Participants emphasized the need for a balance between authenticity and adaptability in agent evolution (e.g., age and thinking) over time. Many supported the idea of allowing agents to evolve to stay present and maintain relevance in communication (P1, 3-4, 6-8, 12-14, 16-18) as evolving agents could better adapt to new contexts. \ly{P8 stated, \textit{``My agents should understand my family updates and the evolving world around us, providing relevant interactions.''} P4 echoed, \textit{``It would be ideal for my agent to continue meaningful conversations, as if I were still present.''}}

In contrast, some raised concerns about the risks and philosophical implications, particularly regarding authenticity and identity (P9-P11). P10 preferred static agents, saying, \textit{``It should remain as designed to preserve its essence, avoiding new topics that might seem inauthentic.''} P11 echoed these concerns, noting that evolution could introduce uncontrollable changes. P9 questioned the philosophical implications, stating, \textit{``Evolving agents challenge individual identity consistency... It's important to consider what makes someone the same person despite physical, psychological, or digital changes.''}

Some also emphasized adapting based on life stages (P2-4). P4 noted, \textit{``Further aging for the agents of the elderly is inappropriate.''} While P2 added the younger might benefit from continued growth.

\subsubsection{Reactivity} \label{reactivity}
\ly{
Reactive interaction highlights the balance between autonomy and non-intrusiveness.
Some participants expressed interest in active reactivity, suggesting it as a feature motivated by love for their family (P1-3, 6, 11).} 
For instance, P2 noted it would be comforting if the agent detected family emotions and provided support. P3 added, \textit{``My family wouldn't mind having sensors at home for my agent to respond to their needs in real-time.''}

\ly{Conversely, many preferred passive reactivity (P4, 9-10, 12-14, 16-18), emphasizing the importance of respecting personal space and avoiding disturbing the normal life of their families.} P16 stated, \textit{``It would be unsettling if my agent \ly{(robot form)} was walking around the house without being explicitly called upon...''}

\ly{Confronted with the contradiction of non-intrusiveness and autonomy, some suggested a balanced approach (P4, 7-8, 11), proposing that agents remain largely passive while providing comfort when needed by the family.}
P8 noted that reactivity might depend on the interaction context and the relationships between the deceased and their loved ones. P7 raised ethical concerns, emphasizing the need for family involvement in customizing this function to avoid unintended consequences. 
\ly{Those interested in active reactivity also saw it as an optional choice, with P11 explaining, \textit{``My family should have the right to customize this function.''}}

\subsection{Concerns: Technology, Mental Health, Security, and Socioeconomic Issues}\label{concerns}

Several significant concerns regarding ``AI Afterlife'' as digital legacy have been identified, including technology, mental health, security, and socioeconomic issues, as summarized in \autoref{tab:summary of themes and main points in concerns}.

\begin{table*}[htbp]
  \caption{Summary of themes and main points in concerns.}
  \label{tab:summary of themes and main points in concerns}
  \begin{tabular}{p{0.15\textwidth}p{0.8\textwidth}}
    \toprule
    \textbf{Theme} & \textbf{Main Point} \\
    \midrule
    \multirow{3}{=}{Technology} 
             & Technological limitations of AI in simulating realistic human (P1,7,11-12,17-18)  \\ %\cline{2-2}
             & \cellcolor{gray!15}{Unpredictability and potential loss of control over AI systems (P9-11,13)} \\ %\cline{2-2}
             & Lack of standards and the challenges in regulating the proper use (P7,P12)  \\ 
    \hline
    \multirow{3}{=}{Mental Health}
            & \cellcolor{gray!15}{The blending of real and virtual worlds could evoke feelings of fear and discomfort (P5,7-8,16)} \\ %\cline{2-2}
            & Interactions with the digital being of the deceased could hinder the healing process (P2-5,8,10,12,16,18) \\ %\cline{2-2}
            & \cellcolor{gray!15}{Design technology features that encourage moderation in interactions and move on  (P2,12,16,18)} \\
    \hline
    \multirow{3}{=}{Security}
            & Handling of personal data and privacy in the creation and access stages (P2-4,6,11,13) \\ %\cline{2-2}
            & \cellcolor{gray!15}{AI-generated agents negatively impact reputation (P11, 13-14, 16-17)}  \\ %\cline{2-2}
            & Abused by malicious actors for harmful purposes, like fraud (P10-12,17-18) \\ 
            & \cellcolor{gray!15}{Abused by service providers without proper oversight (P1,2)} \\
    \hline
    \multirow{3}{=}{Socioeconomic Issues}
            & Impact of AI-generated agents on social interactions (P5,8,10) \\ %\cline{2-2}
            & \cellcolor{gray!15}{Impact traditional cultural practices surrounding death and mourning (P6,9,16)} \\ %\cline{2-2}
            & Economic implications of creating and maintaining AI-generated agents (P5,11,15-16) \\
    \bottomrule
    % Add more rows as needed
  \end{tabular}
\end{table*}

\subsubsection{Technology}
Technical concerns include AI's limitations in simulating human traits, the unpredictability of autonomous systems, and the lack of regulatory standards.
%
Many noted that incomplete or biased input data could create flawed or skewed agents (P1, 7, 11-12, 17-18). \ly{For example, P1 explained, \textit{``AI may be influenced by some default values... Besides, although its capabilities are relatively comprehensive, it may lack the depth in my area of expertise... The data I provide is also limited and cannot reflect my complete life.''}}

Unpredictability and potential loss of control were also significant worries (P9-11, 13). Participants feared that AI systems could evolve beyond their original settings, causing unforeseen ethical or social consequences. For instance, P13 mentioned, \textit{``If my agent develops harmful tendencies, it could negatively impact my son's emotions and well-being.''}

Additionally, the lack of regulatory standards raised concerns about proper use and oversight (P7, 12). P12 stated, \textit{``It's challenging to educate and regulate users on responsible usage due to freedom and accessibility of this technology.''}

\subsubsection{Mental Health}
Mental health concerns center on fear and obsession with grief.
Some noted that blending the real and virtual worlds often evoked discomfort (P5, 7-8, 16). P8 noted, \textit{``Long-term use can cause emotional turmoil... the presence of the deceased can trigger uncomfortable psychological reactions.''} The ``uncanny valley'' effect was also highlighted, which caused unsettling feelings (P7, 16), especially for those anxious or superstitious (P16).

Obsession with interacting with agents of the deceased was another concern, as it could hinder healing and moving on from grieving (P2-5, 8, 10, 12, 16, 18). P5 shared, \textit{``It's easy to become immersed in grief if interacting with these agents.''} P2 added, \textit{``This has diverged from my intent of creating my own agent.''} P8 pointed out that these agents provided a more immersive experience than photos or memories, potentially disrupting daily life.

To mitigate these risks, participants suggested features to moderate interactions (P2, 12, 16, 18). P12 recommended limiting realism, like using silent video, to reduce immersion. P16 proposed adding reminders to prevent excessive mourning, with detection features to identify when the living are overly sad. P18 emphasized his concern for his family, stating, \textit{``While leaving an agent for my family may help them remember me, I hope it encourages them to move on.''}

\subsubsection{Security}

Security concerns mainly focus on privacy, reputation, and potential abuse. A key issue was the handling of personal data in creating and accessing these agents (P2-4, 6, 11, 13). P2 emphasized, \textit{``My digital information should remain private, not made public.''} P6 added, \textit{``The interactions with my agents should not be accessible to strangers.''}

Some also feared the potential harm to their reputation (P11, 13-14, 16-17). While some tried to filter out disgraceful experiences during the creation process to prevent exposure (P13-14, 16), concerns remained, as P11 notes, \textit{``If my agent reveals my darker side, it could damage my reputation and negatively impact my family.''}

Additionally, concerns about misuse were prominent, particularly the risk of agents being exploited for fraud (P10-12, 17-18). P10 warned, \textit{``Malicious individuals could exploit my agents for profit.''} P12 highlighted the difficulty of managing misuse, stating, \textit{``Images and voices can be easily extracted from my agents.''} P17 emphasized the need for safeguards, \textit{``I lived my life doing good. It's unacceptable for my agents to engage in wrongdoing.''} \ly{Concerns about abuse also extended to service providers. Some expressed expressed distrust toward private, profit-driven companies, citing issues with oversight and sustainability (P1-2). As P1 explained, \textit{``A well-managed organization might succeed, but without proper oversight, my agent could become an electronic doll. It would be preferable if a non-profit organization, using open-source technology and free from capital influence, were responsible, with adequate government and public supervision...''}}

\subsubsection{Socioeconomic Issues}
Socioeconomic concerns involve social interactions, cultural implications, and economic factors. Participants had mixed feelings about the impact of ``AI Afterlife'' on social interactions (P5, 8, 10). P8 mentioned the guilt and social pressure the family might face if the agent is not used long-term, while P5 warned it could harm real-world interactions. P10 added, \textit{``People are already addicted to the internet, sitting together but focused on their phones. Using AI agents of the deceased may lead to a future where people no longer interact with real people, which is troubling. Living should involve real-world experiences, not just virtual ones.''}

AI-generated agents are also expected to disrupt traditional cultural practices around death and mourning (P6, 9, 16). P9 noted, \textit{``AI agents for the deceased will change how we conduct funerals and memorials.''} P6 added that it could \textit{``allow the deceased to join family reunions for remembrance.''}

Additionally, economic concerns about creating and maintaining these agents were raised (P5, 11, 15-16). P5 said she would consider creating an agent if costs were manageable, while P11 pointed out the high data requirements, \textit{``If data collection costs are too high, the agent may not be feasible.''} P15 expressed concern over ongoing maintenance costs.
\section{Discussion}

\textbf{Batched Inference:}
In \name, the layer allocation is decided by the controller prior to backbone execution, enabling support for batched inference. Classically, adaptive models cannot support batched inference as different samples will proceed through different layers. However, since \name decides the layer allocation at the controller, we can group samples into \emph{sub-batches} based on similar layer allocation. For instance, samples with high depth noise and low image noise will activate a similar set of layers, allowing them to be grouped together for batched execution.

\begin{figure}
    \centering
    \includegraphics[width=1\linewidth]{Figures/LayerDrop_line.png}
    \vspace{-10pt}
    \caption{Effect of LayerDrop on 12-layer unimodal image and depth localization networks. ``No LD'' indicates no use of LayerDrop, ``LD FT'' indicates use only during finetuning, and finally ``LD Both'' employs LayerDrop in both phases}
    % \vspace{-10pt}
        \vspace{-0.2in}
    \label{fig:layerdrop_plot}
        % \vspace{-0.1in}
\end{figure}

\textbf{Fusion with Early Exit:}
While \name and unimodal Early-Exit methods tackle fundamentally different problems, the two techniques can be combined for further computation efficiency. \name's controller always allocates $L$ layers across all the modalities. However, on simple inputs, all $L$ layers may not be necessary, allowing for Early-Exit techniques to be integrated for further performance gains. 

% \textbf{Trade-off:}
% In scenarios with noisy input data, single-modal systems often require increased computational resources to extract reliable features, leading to higher computational costs. 
% In contrast, multimodal systems benefit from the inherent complementary and redundant information across modalities. 
% Our work dynamically adjusts computational resource allocation based on input fidelity and the interplay between modalities. 
% By leveraging reliable modalities and bypassing or downweighting noisy ones, our approach minimizes computational overhead while maintaining robust performance. 
% This adaptive strategy efficiently handles noisy inputs, ensuring both resource savings and feature extraction quality.



\section{Conclusion}

This paper proposes \name, a multimodal network capable of dynamically adjusting the number of active Transformer layers across modalities according to the quality of each sample's input modalities. Through this continuous reallocation, \name can match the accuracy of far larger networks while utilizing a fraction of their operations. Additionally, the dynamic backbones of \name are also well suited for scenarios with adaptive compute, ranging from heterogeneous deployment devices to fluctuating energy availability. We demonstrate the superiority of \name compared to other baselines across both classification and localization tasks. 



\section{Limitations and Future Work}
While this study provides valuable insights into ``AI Afterlife'' as digital legacy, it is important to acknowledge the limitations stemming from the cultural and social backgrounds of the participants. The research was conducted primarily with individuals from a Chinese cultural context, which inherently shapes their views on life, death, and the afterlife.

In many East Asian cultures, there is a strong emphasis on familial bonds \cite{freedman1961family, qi2016family}, filial piety \cite{hwang1999filial, whyte1997fate}, and the continuity of ancestry \cite{song2015ancestry}. Although participants exhibited some variation, particularly in the context of ongoing social transformation, these cultural values significantly influence how individuals perceive death and the role of AI-generated legacy. For instance, the cultural importance placed on remembrance and honoring ancestors may foster a favorable attitude toward ``AI Afterlife'' as a way to preserve familial connections and legacy. Conversely, the belief in the finality of death and the natural cycle of life may lead to reservations about extensive engagement with AI-generated agents of the deceased. Additionally, in cultures with more individualistic orientations, the emphasis on self-expression may shape attitudes and expectations toward AI-generated agents in unique ways \cite{hunter2008beyond, kim2007express, kim2003choice, inglehart2004individualism}, which contrast with the collectivist tendencies observed in this study, such as prioritizing support for family and alleviating grief.

Given these cultural nuances, the findings of this study may not be fully generalizable to other cultural contexts where beliefs about death and the afterlife differ significantly. However, some key findings, such as the expected interaction designs and concerns, can be applied more broadly. This paper serves as a starting point for understanding human attitudes toward this emerging form of digital legacy. Future research should include a more diverse participant pool from various cultural backgrounds to better explore how different cultural and social values shape perceptions of ``AI Afterlife'' as digital legacy, ensuring it meets the diverse needs and values of users globally.
\section{Conclusion}
RPA evaluation lacks consistency due to varying tasks, domains, and agent attributes. Our systematic review of $1,676$ papers reveals that task-specific requirements shape agent attributes, while both task characteristics and agent design influence evaluation metrics. By identifying these interdependencies, we propose guidelines to enhance RPA assessment reliability, contributing to a more structured and systematic evaluation framework.

\begin{acks}
The work described in this paper was partially supported by a grant from the Research Grants Council of the Hong Kong Special Administrative Region, China (Project Reference Number: AoE/E-601/24-N).
\end{acks}

% \balance

\bibliographystyle{ACM-Reference-Format}
\bibliography{main}


\appendix
\section{Thematic map with intermediary codes}
\label{sec: thematic map}
\begin{figure*}[htbp]
	\centering 
	\includegraphics[width=0.83\linewidth]{figures/crop_thematic_map.pdf}
    % angle=-90, 
    \caption{Thematic map with  intermediary codes.}
	\label{fig:thematic map}
        \Description{}
\end{figure*}
% \end{appendices}


\end{document}

%%
%% End of file `sample-manuscript.tex'.



