\begin{abstract}
%Due to the immutability of the code after deployment and the absence of intermediaries, smart contracts are an attractive target for attackers. 
Formal verification plays a crucial role in making smart contracts safer, being able to find bugs or to guarantee their absence, as well as checking whether the business logic is correctly implemented.
For Solidity, %the most used smart contract language,
even though there already exist several mature verification tools, the semantical quirks of the language can make verification quite hard in practice. %, even for simple specifications.
Move, on the other hand, has been designed with security and verification in mind, and it has been accompanied since its early stages by a formal verification tool, the Move Prover.
In this paper, we investigate
through a comparative analysis: 1) how the different designs of the two contract languages impact verification, and 2) what is the state-of-the-art of verification tools for the two languages, and how do they compare on three paradigmatic use cases.
Our investigation is supported by an open dataset of verification tasks performed in Certora and in the Aptos Move Prover. 
\end{abstract}