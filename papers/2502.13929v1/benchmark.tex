\section{Use cases}
\label{sec:benchmark}

We describe here the use cases considered in our analysis.
%, mentioning the main properties used for our comparison.
The full implementation of the use cases in Solidity and Move, and of all the sets of specifications in CVL and MSL, are in \githuburl.
\enriconote{Evidenziare le caratteristiche particolari di ogni singolo contratto e perché lo rendono interessante a livello di verifica}


%\begin{itemize}
%    \item Bank: mappe, special users (quantificatori?)
%    \item Vault: state-based, tempo, adversary, transaction-ordering dependencies
%    \item Price-bet: external call ($\implies$ aleatorietà, "non-determinismo"), transaction-ordering dependencies (con le tx dell'oracolo) 
%\end{itemize}

\subsection{Bank}

The \contracturl{bank} contract stores assets (native crypto-currency) deposited by users, and pays them out when required. The contract state consists of:
\code{credits}, a key-value map that associates each user with the amount of assets available for that user;
\code{owner}, the address that deploys the contract;
\code{opLimit}, a limit set during contract deployment that restricts the maximum amount that can be deposited or withdrawn in a single transaction. This limit applies to all users except the owner.
The contract has the following entry points:
\begin{itemize}
\item \code{deposit}, which allows anyone to deposit assets. When a deposit is made, the corresponding amount is added to the credit of the transaction sender.
\item \code{withdraw}, which allows the sender to receive any desired amount of assets deposited in their account. The contract checks that the user has sufficient credit and then transfers the specified amount to the sender.
\end{itemize}
%
We have identified \nBankProperties relevant properties of the Bank contract. 
%For the purpose of the comparison in~\Cref{sec:comparison}, we mention:  \enriconote{check questi elenchi non credo siano aggiornati}
%\enriconote{Togliere? alla fine il testo lo riportiamo già in Sec 4}
%\begin{itemize}
 %   \item \specurl{bank}{credits-leq-balance}: the assets controlled by the contract are (at least) equal to the sum of all the credits.
    
  %  \item \specurl{bank}{deposit-assets-transfer}:     after a successful \code{deposit(amount)}, exactly amount token units pass from the control of the sender to that of the contract.
    % \bartnote{(implica bank/deposit-assets-credit)}
    
    %\item \specurl{bank}{withdraw-not-revert}: a transaction \code{withdraw(amount)} does not abort if \code{amount} is less or equal to the transaction sender's credit and operation limit.
    
    %\item \specurl{bank}{deposit-additivity}: two consecutive successful deposits of $n_1$ and $n_2$ token units performed by the same sender are equivalent to a single deposit of $n_1+n_2$ token units, if $n_1+n_2$ is less than or equal to the sender’s operation limit
%\end{itemize}


\subsection{Vault}

The \contracturl{vault} contract is a security mechanism to prevent an adversary who has stolen the owner's private key from stealing the owner's tokens.
To create the vault, the owner specifies:
itself as the vault's owner;
a recovery key, which can be used to cancel a withdraw request;
a wait time, which has to elapse between a withdraw request and the actual finalization of the cryptocurrency transfer.
The contract has the following entry points:
\begin{itemize}

\item \code{receive(amount)}, which allows anyone to deposit tokens into the contract;

\item \code{withdraw(receiver, amount)}, which allows the owner to issue a withdraw request, specifying the receiver and the desired amount;

\item \code{finalize()}, which allows the owner to finalize the pending withdraw after the wait time has passed since the request;

\item \code{cancel()}, which allows the owner of the recovery key to cancel the withdraw request during the wait time.
\end{itemize}
%
We have identified \nVaultProperties relevant properties of the Vault contract. 

%\noindent
%Among the \nVaultProperties properties we have identified, for the purpose of our comparison, we mention: 
%\begin{itemize}
%    \item \specurl{vault}{keys-immutable}: 
    
%    \item \specurl{vault}{withdraw-withdraw-revert}: a transaction \code{withdraw()} aborts if performed immediately after another \code{withdraw()}	
    
%    \item \specurl{vault}{owner-immutable}: the vault owner never changes.
    
%\end{itemize}


\subsection{Price Bet}

The \contracturl{price-bet} contract allows a single player to place a bet against the contract owner. The bet is based on a future exchange rate between two tokens.
To create the contract, the owner specifies:
itself as the contract owner;
the initial pot, which is transferred from the owner to the contract;
an oracle, \ie a contract that is queried for the exchange rate between two given tokens;
a deadline, \ie a time limit after which the player loses the bet;
a target exchange rate, which must be reached in order for the player to win the bet.
The contract has the following entry points:
\begin{itemize}
\item \code{join()}, which allows a player to join the bet. This requires the player to deposit an amount of native cryptocurrency equal to the initial pot;
\item \code{win()}, which allows the joined player to withdraw the whole contract balance if the oracle exchange rate is greater than the bet rate. The player can call win() multiple times before the deadline. This action is disabled after the deadline;
\item \code{timeout()}, which can be called by anyone after the deadline, and transfers the whole contract balance to the owner
\end{itemize}
%
We have identified \nPricebetProperties relevant properties of the Price Bet contract. 

%\noindent
%Among the \nPricebetProperties properties we have identified, for the purpose of our comparison, we mention: 
%\begin{itemize}
%    \item \specurl{price-bet}{no-frozen-funds}: after the deadline, any user can perform some transaction that transfers the contract pot to the owner
%\end{itemize}
