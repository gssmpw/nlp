\section{Conclusions}
\label{sec:conclusions}

The empirical analysis of our study validates the folklore knowledge 
that Move is better suited for verification than Solidity.
In particular, Move's resource-orientation facilitates the verification of properties concerning, \eg, resource preservation, ownership, and transferring of assets.
%The only  weakness of Move \wrt Solidity that we have observed,
The only weak spot we have observed in (Aptos) Move
is the lack of a construct to enforce the immutability of contract variables ---
a feature that instead is present in Solidity.
We have noted that, in order to properly specify certain properties and determine their truth, some low-level aspects of the underlying contract layers must be taken into account.
While this could be discouraging for smart contract developers unfamiliar with these low-level details, it can also serve as an incentive to deepen their understanding on these aspects, ultimately leading to more secure smart contract implementations.  

Concerning verification tools,
we have observed that the Certora Prover can express a broader
set of properties than the Move Prover, \eg, 
transition invariants involving multiple transactions,
metamorphic properties,   
and intra-function invariants.
We believe that all the functionalities needed to verify such properties 
could be smoothly added to the Move Prover, as well. 
We have also noted that there are several relevant classes of properties that are out of the scope of both tools (\eg, liveness, liquidity/enabledness, and, more generally, other complex temporal properties concerning the business logic of the contract).
We have observed that some of these properties can be addressed by other tools, although their current maturity level remains below that of the Certora and Move provers.

We have contributed with an open dataset of smart contract implementations and verification tasks performed in the two tools (the first of this kind), that we envision will further encourage research on %the analysis of 
formal verification of Solidity and Move.

\mypar{Limitations}
Although our empirical analysis is based on a set of \nTotProperties verification tasks covering a broad range of properties, we expect that extending our dataset would highlight additional differences between verification in Solidity and Move. Moreover, it could reveal some further kinds of properties that would be desirable to verify on real-world smart contracts but currently fall beyond the scope of existing verification tools.
This could be the case, \eg, of economic properties of DeFi protocols, whose verification currently requires either using weaker analysis techniques than formal verification (\eg, property-based testing~\cite{Milo22fmbc}, statistical model checking~\cite{BartolettiCJLMV22isola}), or 
abstracting from actual contract code~\cite{Tolmach21wtsc,SunLSJ21wtsc,Babel23clockwork,Nielsen23cpp,Pusceddu24fmbc}.
% cmq anche su contratti molto semplici si riescono a studiare proprietà diverse
