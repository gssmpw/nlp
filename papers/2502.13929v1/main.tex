\documentclass[final,a4paper,UKenglish,cleveref,autoref]{oasics-v2021}

\bibliographystyle{plainurl} % the mandatory bibstyle

% titolo provvisorio!
\title{Formal verification in Solidity and Move: insights from a comparative analysis}  
% (Experience report)

%\titlerunning{Dummy short title} %TODO optional, please use if title is longer than one line

\author{Massimo Bartoletti}{University of Cagliari, Italy \and \url{http://blockchain.unica.it} }{bart@unica.it}{https://orcid.org/0000-0003-3796-9774}{Partially supported by project SERICS (PE00000014) and PRIN 2022 DeLiCE (F53D23009130001) under the MUR National Recovery and Resilience Plan funded by the European Union -- NextGenerationEU.}

\author{Silvia Crafa}{University of Padova, Italy}{crafa@math.unipd.it}{https://orcid.org/0000-0003-0993-4734}{Supported by the National Recovery and Resilience Plan (NRRP) Project ``Securing sOftware Platforms - SOP'', CUP H73C22000890001}

\author{Enrico Lipparini}{University of Cagliari, Italy \and \url{https://elipparini.github.io/}}{enrico.lipparini@unica.it}{https://orcid.org/0009-0009-0428-4403}{Supported by project PRIN 2022 DeLiCE (F53D23009130001) under the MUR National Recovery and Resilience Plan funded by the European Union -- NextGenerationEU.}

%TODO mandatory, please use full name; only 1 author per \author macro; first two parameters are mandatory, other parameters can be empty. Please provide at least the name of the affiliation and the country. The full address is optional. Use additional curly braces to indicate the correct name splitting when the last name consists of multiple name parts.

\authorrunning{M. Bartoletti, S. Crafa, E. Lipparini} 

%TODO mandatory. First: Use abbreviated first/middle names. Second (only in severe cases): Use first author plus 'et al.'

\Copyright{M. Bartoletti, S. Crafa, E. Lipparini}
%TODO mandatory, please use full first names. LIPIcs license is "CC-BY";  http://creativecommons.org/licenses/by/3.0/

% \ccsdesc[500]{Software and its engineering~Formal methods}
\ccsdesc[500]{Software and its engineering~Formal software verification}

%TODO mandatory: Please choose ACM 2012 classifications from https://dl.acm.org/ccs/ccs_flat.cfm 

\keywords{Smart contracts, Solidity, Move, Verification, Blockchain}

\category{} %optional, e.g. invited paper

\relatedversion{} %optional, e.g. full version hosted on arXiv, HAL, or other respository/website
%\relatedversiondetails[linktext={opt. text shown instead of the URL}, cite=DBLP:books/mk/GrayR93]{Classification (e.g. Full Version, Extended Version, Previous Version}{URL to related version} %linktext and cite are optional

% \supplement{\url{https://github.com/fsainas/contracts-verification-benchmark}}%optional, e.g. related research data, source code, ... hosted on a repository like zenodo, figshare, GitHub, ...
%\supplementdetails[linktext={opt. text shown instead of the URL}, cite=DBLP:books/mk/GrayR93, subcategory={Description, Subcategory}, swhid={Software Heritage Identifier}]{General Classification (e.g. Software, Dataset, Model, ...)}{URL to related version} %linktext, cite, and subcategory are optional

%\funding{(Optional) general funding statement \dots}%optional, to capture a funding statement, which applies to all authors. Please enter author specific funding statements as fifth argument of the \author macro.

% \acknowledgements{I want to thank \dots}%optional

\nolinenumbers %uncomment to disable line numbering

% \EventEditors{Bruno Bernardo and Diego Marmsoler}
% \EventNoEds{2}
% \EventLongTitle{5th International Workshop on Formal Methods for Blockchains (FMBC 2024)}
% \EventShortTitle{FMBC 2024}
% \EventAcronym{FMBC}
% \EventYear{2024}
% \EventDate{April 7, 2024}
% \EventLocation{Luxembourg City, Luxembourg}
% \EventLogo{}
% \SeriesVolume{118}
% \ArticleNo{6}

%%%%%%%%%%%%%%%%%%%%%%%%%%%%%%%%%%%%%%%%%%%%%%%%%%%%%%%
%%%%%%%%%%%%%%%    theorems %%%%%%%%%%%%%%%%%%%%%%%%%%%
%%%%%%%%%%%%%%%%%%%%%%%%%%%%%%%%%%%%%%%%%%%%%%%%%%%%%%%
% \usepackage{mdframed}
\usepackage{kantlipsum}

%%%%%%%%%%%%%%%%%%%%%%%%%%%%%%%%%%%%%%%%%%%%%%%%%%%%%%%
%%%%%%%%%%%%%%%    theorems %%%%%%%%%%%%%%%%%%%%%%%%%%%
%%%%%%%%%%%%%%%%%%%%%%%%%%%%%%%%%%%%%%%%%%%%%%%%%%%%%%%
\theoremstyle{plain}
\newtheorem{theorem}{Theorem}[section]
\newtheorem{proposition}[theorem]{Proposition}
\newtheorem{lemma}[theorem]{Lemma}
\newtheorem{example}[theorem]{Example}
\newtheorem{corollary}[theorem]{Corollary}
\theoremstyle{definition}
\newtheorem{definition}[theorem]{Definition}
\newtheorem{assumption}[theorem]{Assumption}
\theoremstyle{remark}
\newtheorem{remark}[theorem]{Remark}


% \titleformat{\subsection}[runin]% runin puts it in the same paragraph
%        {\normalfont\bfseries}% formatting commands to apply to the whole heading
%        {\thesubsection}% the label and number
%        {0.5em}% space between label/number and subsection title
%        {}% formatting commands applied just to subsection title
%        [.]% punctuation or other commands following subsection title


%%%%%%%%%%%%%%%%%%%%%%%%%%%%%%%%%%%%%%%%%%%%%%%%%%%%%%%
%%%%%%%%%%%%%%%  mathematical notations%%%%%%%%%%%%%%%%
% \usepackage[english]{babel}
% \usepackage[utf8]{inputenc}
% \usepackage[T1]{fontenc}

%% Figures, tables and lists
\usepackage[dvipsnames]{xcolor}
\usepackage{paralist}
\usepackage{graphicx}
\usepackage{subcaption}
\usepackage{longtable} 
\usepackage{multirow}
\usepackage{listings}
\usepackage{makecell}
\usepackage{array}
\usepackage{float}
\usepackage{dsfont}
\usepackage{rotating}
\usepackage{booktabs}
\usepackage{enumerate}
\usepackage{tikz}
\usepackage{pgf}
\usepackage{enumitem}
\usepackage{lipsum} % for generating filler text
\usepackage{titlesec}

%% Math
% \usepackage{amssymb, amsthm,bbm}
\usepackage{mathtools}
\usepackage{mathrsfs}
%% References and author info 
\mathtoolsset{showonlyrefs}
\usepackage{natbib}
\usepackage{authblk}
\usepackage{todonotes}
\usepackage{xr-hyper}


%%%%%%%%%%%%%%%%%%%%%%%%%%%%%%%%%%%%%%%%%%%%%%%%%%%%%%%
\newcommand{\R}{\mathbb R}
\newcommand{\EE}{\mathbb{E}}

\DeclareMathOperator{\Tr}{Tr}
\DeclareMathOperator*{\argmin}{argmin}
\DeclareMathOperator*{\argmax}{argmax}

\newcommand{\bs}[1]{\ensuremath{\boldsymbol{#1}}}
\newcommand{\mc}{\mathcal}
\newcommand{\opt}{^\star}


\newcommand{\diff}{\textnormal{d}}


\def \iid {\stackrel{\textnormal{i.i.d.}}{\sim}}
\def \iidtext {\textnormal{i.i.d.}}





%%%%%%%%%%%%%%%%%%%%%%%%%%%%%%%%%%%%%%%%%%%%%%%%%%%%%%%
%%%%%%%%%%%%%%%%%%%%% colors     %%%%%%%%%%%%%%%%%%%%%%
%%%%%%%%%%%%%%%%%%%%%%%%%%%%%%%%%%%%%%%%%%%%%%%%%%%%%%%
\definecolor{myblue}{rgb}{.8, .8, 1}
\definecolor{mathblue}{rgb}{0.2472, 0.24, 0.6} % mathematica's Color[1, 1--3]
\definecolor{mathred}{rgb}{0.6, 0.24, 0.442893}
\definecolor{mathyellow}{rgb}{0.6, 0.547014, 0.24}


% May add more in future.






% fix listings UTF-8
\usepackage[utf8]{inputenc}
\usepackage{color}
\usepackage[usenames,dvipsnames]{xcolor}

\definecolor{keywordcolor}{rgb}{0.7, 0.1, 0.1}   % red
\definecolor{tacticcolor}{rgb}{0.0, 0.1, 0.6}    % blue
\definecolor{commentcolor}{rgb}{0.4, 0.4, 0.4}   % grey
\definecolor{symbolcolor}{rgb}{0.0, 0.1, 0.6}    % blue
\definecolor{sortcolor}{rgb}{0.1, 0.5, 0.1}      % green
\definecolor{attributecolor}{rgb}{0.7, 0.1, 0.1} % red
\definecolor{white}{rgb}{1.0,1.0,1.0}

\usepackage{xspace}
\usepackage{nicefrac}
\usepackage{caption}
\captionsetup[table]{name=Listing}
\usepackage{makecell}

\usepackage{tikz}

% colors for links
\hypersetup{
    colorlinks=true,  
    urlcolor=blue,
}

\usepackage[final,nomargin,inline,index]{fixme} % Simplified management of FIXME's
\fxusetheme{color}
\FXRegisterAuthor{bart}{anbart}{\color{magenta} {\underline{bart}}}
\FXRegisterAuthor{enrico}{anenrico}{\color{red} {\underline{enrico}}}
\FXRegisterAuthor{silvia}{ansilvia}{\color{blue} {\underline{silvia}}}

\crefname{table}{Listing}{Listings}  

% \renewcommand{\ttdefault}{lmtt}  % to use bold and italic within code

\definecolor{LightGrey}{rgb}{0.975,0.975,0.975}
% \definecolor{keyword}{HTML}{7F0055}

% Packages initialization
%
% Packages initialization

\lstset{
	nolol=true,
	breaklines=true,
	xleftmargin=0pt,
	xrightmargin=0pt,
	framexleftmargin=0pt,
	framextopmargin=0pt,
        % frame=ltbr, 
	framerule=0pt,        
	% framexrightmargin=10pt,
	framexbottommargin=2pt, 
	% frame=ltbr, 
        % framerule=0pt,
	showstringspaces=false,
	%basicstyle=%
        %\ttfamily\linespread{1.15}\scriptsize%
        %\lst@ifdisplaystyle\footnotesize\fi,
        % basicstyle=\fontseries{m}\scriptsize\ttfamily,
	backgroundcolor=\color{LightGrey},
        numberstyle=\fontsize{4}{4}\color{MidnightBlue}\ttfamily,
        stringstyle=\color{RoyalBlue}
}

\lstdefinelanguage{solidity}{
, basicstyle=\ttfamily\linespread{1.15}\scriptsize\lst@ifdisplaystyle\scriptsize\fi
, commentstyle=\color{Gray}
, morecomment=[l]{//}
, morecomment=[s]{/*}{*/}
, escapechar=\$
% commands => NavyBlue
, classoffset=0,
, keywordstyle=\color{NavyBlue}\bfseries
, morekeywords={assert,require,if,then,else,for,break,call,delegatecall,transfer,send,case, catch,continue,do,while,emit, new, return, revert, selfdestruct, try, with, throw, switch, suicide}
% declarations context => BlueGreen
, classoffset=1
, keywordstyle=\color{YellowGreen}\bfseries
, morekeywords={external, implements, import, interface, internal, library, payable, pragma, private, protected, public, pure, returns, super, using, view}
% types => blue
, classoffset=2
, keywordstyle=\color{blue}
, morekeywords={function, constructor, contract, constant, struct, address, bool, byte, bytes, bytes1, bytes2, bytes3, bytes4, bytes5, bytes6, bytes7, bytes8, bytes9, bytes10, bytes11, bytes12, bytes13, bytes14, bytes15, bytes16, bytes17, bytes18, bytes19, bytes20, bytes21, bytes22, bytes23, bytes24, bytes25, bytes26, bytes27, bytes28, bytes29, bytes30, bytes31, bytes32, enum, int, int8, int16, int24, int32, int40, int48, int56, int64, int72, int80, int88, int96, int104, int112, int120, int128, int136, int144, int152, int160, int168, int176, int184, int192, int200, int208, int216, int224, int232, int240, int248, int256, mapping, string, uint, uint8, uint16, uint24, uint32, uint40, uint48, uint56, uint64, uint72, uint80, uint88, uint96, uint104, uint112, uint120, uint128, uint136, uint144, uint152, uint160, uint168, uint176, uint184, uint192, uint200, uint208, uint216, uint224, uint232, uint240, uint248, uint256, var, void, ether, finney, szabo, wei, days, hours, minutes, seconds, weeks, years}
% constants and fields => Plum
, classoffset=3
, keywordstyle=\color{Plum}\bfseries
, morekeywords={balance, block, blockhash, instanceof, coinbase, difficulty, gaslimit, number, timestamp, msg, data, gas, sender, value, sig, value, now, tx, gasprice, origin}
}

\lstdefinelanguage{move}{%
, basicstyle=\ttfamily\linespread{1.15}\scriptsize\lst@ifdisplaystyle\scriptsize\fi
% , sensitive
, commentstyle=\color{Gray},
, morecomment=[l]{//}%
, morecomment=[s]{/*}{*/}%
, moredelim=[s][{\itshape\color[rgb]{0,0,0.75}}]{\#[}{]}%
% , morestring=[b]{"}%
% , alsodigit={}%
% , alsoother={}%
% , alsoletter={!}%
, escapechar=\$
%
% commands => NavyBlue
, classoffset=0,
, keywordstyle=\color{NavyBlue}\bfseries
, morekeywords={assert!, move_to, move_from, borrow_global, borrow_mut, borrow_global_mut, exists, break, continue, else, for, if, in, loop, match, return, while}  
% declarations context => BlueGreen
, classoffset=1
, keywordstyle=\color{YellowGreen}\bfseries
, morekeywords={as, const, let, move, mut, ref, static}  % in the context of variables
% in the context of declarations
, morekeywords={module, public, external, acquires}  % in the context of modularisation
%
, keywordstyle=\color{blue}\bfseries
, classoffset=2
, morekeywords={spec, enum, fun, impl, Self, self, struct, trait, type, union, use, where}
, morekeywords={signer, address, bool, char, f32, f64, i8, i16, i32, i64, isize, str, u8, u16, u32, u64, unit, usize, i128, u128}  % primitive types
%
% constants and fields => Plum
, classoffset=3
, keywordstyle=\color{Plum}\bfseries
, morekeywords={address_of, Err, false, None, Ok, Some, true},
, classoffset=4
, keywordstyle=\color{red}\bfseries
, morekeywords={spec,ensures,post,global,spec_get}
}

\lstdefinelanguage{cvl}{
, basicstyle=\ttfamily\linespread{1.15}\scriptsize\lst@ifdisplaystyle\scriptsize\fi
, commentstyle=\color{Gray}
, morecomment=[l]{//}
, morecomment=[s]{/*}{*/}
, classoffset=0
, escapechar=\$
, morekeywords={anonymous, assembly, balance, break, call, callcode, case, catch, class, constant, continue, constructor, contract, debugger, default, delegatecall, delete, do, else, emit, event, experimental, export, external, false, finally, for, function, gas, if, implements, import, in, indexed, instanceof, interface, internal, is, length, library, log0, log1, log2, log3, log4, memory, modifier, new, payable, pragma, private, protected, public, pure, push, return, returns, revert,  selfdestruct, send, solidity, storage, struct, suicide, super, switch, then, this, throw, transfer, true, try, typeof, using, value, view, while, with, addmod, ecrecover, keccak256, mulmod, ripemd160, sha256, sha3}
, keywordstyle=\color{NavyBlue}\bfseries
, classoffset=1
, morekeywords={address, bool, byte, bytes, bytes1, bytes2, bytes3, bytes4, bytes5, bytes6, bytes7, bytes8, bytes9, bytes10, bytes11, bytes12, bytes13, bytes14, bytes15, bytes16, bytes17, bytes18, bytes19, bytes20, bytes21, bytes22, bytes23, bytes24, bytes25, bytes26, bytes27, bytes28, bytes29, bytes30, bytes31, bytes32, enum, int, int8, int16, int24, int32, int40, int48, int56, int64, int72, int80, int88, int96, int104, int112, int120, int128, int136, int144, int152, int160, int168, int176, int184, int192, int200, int208, int216, int224, int232, int240, int248, int256, mapping, string, uint, uint8, uint16, uint24, uint32, uint40, uint48, uint56, uint64, uint72, uint80, uint88, uint96, uint104, uint112, uint120, uint128, uint136, uint144, uint152, uint160, uint168, uint176, uint184, uint192, uint200, uint208, uint216, uint224, uint232, uint240, uint248, uint256, var, void, ether, finney, szabo, wei, days, hours, minutes, seconds, weeks, years}
, keywordstyle=\color{blue}
, classoffset=2
, morekeywords={block, blockhash, coinbase, difficulty, gaslimit, number, timestamp, msg, data, gas, sender, sig, value, now, tx, gasprice, origin}
, keywordstyle=\color{Plum}\bfseries
, classoffset=3
, morekeywords={invariant,rule,assert,satisfy,require}
, keywordstyle=\color{red}\bfseries,
}

\lstset{
	language=solidity,
}

% correct the font size of lstinline 
\makeatletter
\lstdefinestyle{mystyle}{
  basicstyle=%
    \ttfamily
    \lst@ifdisplaystyle\footnotesize\fi
}
\makeatother
\lstset{style=mystyle}

%\newcommand{\urlSheets}{\url{https://tinyurl.com/3ndkhj57}}

\newcommand{\githubaddress}{https://github.com/blockchain-unica/solidity-vs-move-verification}
\newcommand{\githuburl}{\url{\githubaddress}}
\newcommand{\sheeturl}{https://docs.google.com/spreadsheets/d/1oy43kISC2iM63XfksfonRBzjXPkpLoDglz-m24nazvc/edit?usp=sharing}

\newcommand{\specurl}[2]{\href{\githubaddress/blob/main/contracts/#1/README.md\#{#2}}{\textcolor{blue}{\texttt{#1}/\texttt{#2}}}}

\newcommand{\contracturl}[1]{\href{\githubaddress/blob/main/contracts/#1/README.md}{\textcolor{blue}{\texttt{#1}}}}

% TODO: fix command
% \newcommand{\solidityurl}[1]{\href{\githubaddress/blob/main/contracts/#1/certora/#1.sol}{\texttt{#1}}}


% Statistics on the dataset
\newcounter{nBankCounter}
\setcounter{nBankCounter}{28}

\newcounter{nVaultCounter}
\setcounter{nVaultCounter}{21}

\newcounter{nPricebetCounter}
\setcounter{nPricebetCounter}{17}

\newcounter{nTotCounter}
\setcounter{nTotCounter}{\numexpr\value{nBankCounter}+\value{nVaultCounter}+\value{nPricebetCounter}\relax}

\newcommand{\nBankProperties}{\arabic{nBankCounter}\xspace}
\newcommand{\nVaultProperties}{\arabic{nVaultCounter}\xspace}
\newcommand{\nPricebetProperties}{\arabic{nPricebetCounter}\xspace}
\newcommand{\nTotProperties}{\arabic{nTotCounter}\xspace}

\begin{document}

\maketitle

\begin{abstract}
Retrieval-Augmented Generation (RAG) is often used with Large Language Models (LLMs) to infuse domain knowledge or user-specific information. In RAG, given a user query, a retriever extracts chunks of relevant text from a knowledge base. These chunks are sent to an LLM as part of the input prompt. Typically, any given chunk is repeatedly retrieved across user questions. However, currently, for every question, attention-layers in LLMs fully compute the key values (KVs) repeatedly for the input chunks, as state-of-the-art methods cannot reuse KV-caches when chunks appear at arbitrary locations with arbitrary contexts. Naive reuse leads to output quality degradation.  This leads to potentially redundant computations on expensive GPUs and increases latency. In this work, we propose \sys, a system for managing and reusing precomputed KVs corresponding to the text chunks (we call \textit{chunk-caches}) in RAG-based systems. We present how to identify \hl{\textit{chunk-caches} that are reusable}, how to efficiently perform a small fraction of recomputation to \textit{fix} the cache to maintain output quality, and how to efficiently store and evict \textit{chunk-caches} in the hardware for maximizing reuse while masking any overheads. With real production workloads as well as synthetic datasets, we show that \sys reduces redundant computation by \textbf{51\%} over SOTA prefix-caching and \textbf{75\%} over full recomputation.
\hl{Additionally, with continuous batching on a real production workload, we get a \textbf{1.6$\times$} speedup in throughput and a \textbf{2$\times$} reduction in end-to-end response latency over prefix-caching while maintaining quality, for both the \llama-3-8B and \llama-3-70B models. 
}
\end{abstract}






%\enriconote{Nel titolo togliere "insights from"? O "a comparative analysis-based study"?}

\section{Introduction}
\label{sec:intro}

\begin{figure*}[tb]
    \centering
    \includegraphics[width=0.848\linewidth]{figs/circuitnn.pdf} 
    \caption{Illustration of differentiable CircuitNN. CircuitNN is designed based on differentiable NAND gates. After DAS is guided by PI and PO pairs of the truth table, CircuitNN can get the precise circuit architecture logic equivalent to the truth table.}
    \label{fig:circuitnn}
\end{figure*}

% 1. Describe the importance of logic synthesis
% 2. Existing Problems
% (a) Neural Architecture Search: Unstable, Predefined Setting, etc.
% (b) Circuit Generation: Probabilistic Model, Logic Equivalence

With the rapid advancement of technology, the scale of integrated circuits (ICs) has expanded exponentially. 
This expansion has introduced significant challenges in chip manufacturing, particularly concerning power and area metrics.
A primary objective in IC design is achieving the same circuit function with fewer transistors, thereby reducing power usage and area occupancy.

Logic synthesis~\cite{hachtel2005logicsynth}, a critical step in electronic design automation (EDA), transforms behavioral-level circuit designs into optimized gate-level circuits, ultimately yielding the final IC layout. 
The primary goal of logic synthesis is to identify the physical implementation with the fewest gates for a given circuit function. 
This task constitutes a challenging NP-hard combinatorial optimization problem. 
Current logic synthesis tools~\cite{brayton2010abc, wolf2013yosys} rely on human-designed heuristics, often leading to sub-optimal outcomes.

Differentiable architecture search (DAS) techniques~\cite{liu2018darts, chu2020darts} offer novel perspectives on addressing challenges in this problem.
Circuit functions can be represented through truth tables, which map binary inputs to their corresponding outputs. 
Truth tables provide a precise representation of input-output relationships, ensuring the design of functionally equivalent circuits.
Inspired by this, researchers~\cite{deepmind2024ai4sys, wang2024tnet} have begun exploring the application of DAS to synthesize circuits directly from truth tables.
Specifically, \citet{deepmind2024ai4sys} proposed CircuitNN, a framework that learns differentiable connection structures with logic gates, enabling the automatic generation of logic circuits from truth tables.
This approach significantly reduces the complexity of traditional circuit generation. 
Building on this, \citet{wang2024tnet} introduced T-Net, a triangle-shaped variant of CircuitNN, incorporating regularization techniques to enhance the efficiency of DAS.

Despite these advancements, several challenges remain. 
The computational complexity of DAS grows quadratically with the number of gates, posing scalability issues.
Although triangle-shaped architecture~\cite{wang2024tnet} partially mitigates this problem, redundancy persists. 
%Additionally, DAS is susceptible to converging to local optima, limiting the ability to search architectures that satisfy the given truth tables~\cite{liu2018darts}. 
%Furthermore, hyperparameters (network depth and layer width) require extensive searches, introducing complexity and prolonging the synthesis process. 
Additionally, DAS is susceptible to converging to local optima~\cite{liu2018darts} and hyperparameters (network depth and layer width) require extensive searches. 
The challenges arise from the vast search space in DAS. 
% Even with predefined settings for CircuitNN, finding a configuration that meets the truth table requires extensive trial and error during the DAS process. 
Intuitively, limiting the search space through predefined parameters (network depth, gates per layer, and connection probabilities) can significantly reduce the complexity.

Recent advances~\cite{openai2023gpt4, abramson2024alphafold3, esser2024sd3, li2024mar} in conditional generative models have demonstrated remarkable performance across language, vision, and graph generation tasks. 
Motivated by these developments, we propose a novel approach to circuit generation that generates preliminary circuit structures to guide DAS in generating refined circuits matching specified truth tables. 
Firstly, we introduce CircuitVQ, a tokenizer with a discrete codebook for circuit tokenization. 
Built upon our Circuit AutoEncoder framework~\cite{hou2022graphmae,li2023maskgae,wu2025mgvga}, CircuitVQ is trained through a circuit reconstruction task. 
Specifically, the CircuitVQ encoder encodes input circuits into discrete tokens using a learnable codebook, while the decoder reconstructs the circuit adjacency matrix based on these tokens.
Subsequently, the CircuitVQ encoder serves as a circuit tokenizer for CircuitAR pretraining, which employs a masked autoregressive modeling paradigm~\cite{chang2022maskgit, li2023mage}. 
In this process, the discrete codes function as supervision signals. 
After training, CircuitAR can generate discrete tokens progressively, which can be decoded into initial circuit structures by the decoder of the CircuitVQ. 
These prior insights can guide DAS in producing refined circuits that match the target truth tables precisely.

Our key contributions can be summarized as follows:
\begin{itemize}
\item We introduce CircuitVQ, a circuit tokenizer that facilitates graph autoregressive modeling for circuit generation, based on our Circuit AutoEncoder framework;
\item Develop CircuitAR, a model trained using masked autoregressive modeling, which generates initial circuit structures conditioned on given truth tables;
\item Propose a refinement framework that integrates differentiable architecture search to produce functionally equivalent circuits guided by target truth tables;
\item Comprehensive experiments demonstrating the scalability and capability emergence of our CircuitAR and the superior performance of the proposed circuit generation approach.
\end{itemize}

% Motivation
% (a) Diffusion (Vision, Graph), Autoregressive (Language, Vision)
% (b) Circuit Generation for Predefined Setting
% (c) Neural Architecture Search for Strict Logic Equivalence

% Contribution
% (a) Circuit Tokenizer (new transformer arch, training strategy)
% (b) CircuitAR (train and gen strategies, post-ar strategy)
% (c) Extensive Evaluation including BitD (Bit Distance) for Scalability

%\enriconote{da qualche parte specificare che con Move ci riferiremo specificamente a Aptos Move}
\section{Basic Background: Supervised Learning and the PAC Model}
\label{sec:background}

At this point almost everyone has heard of machine learning (ML). Anyone likely to stumble upon this article will have also heard of its most influential special case, supervised learning, and those theoretically inclined will also be familiar with the PAC model. Nonetheless, I will set the stage by  recapping the basics.

\subsection{Basics of Supervised Learning}%Let's set the stage in any case

\emph{Supervised Learning} is the task of ``coming up'' with a function $f: \X \to \Y$ to ``explain'' or ``fit'' a sequence of input/output examples   $(x_1,y_1), \ldots, (x_n,y_n)$, with $x_i \in \X$ and $y_i \in \Y$.  Here $\X$ is a \emph{data domain} consisting of \emph{datapoints} $x \in \X$, $\Y$ is a \emph{label set} consisting of \emph{labels} $y \in \Y$, and the sequence $(x_1,y_1),\ldots,(x_n,y_n)$ is the \emph{training data} consisting of \emph{labeled examples (a.k.a. samples)}~$(x_i,y_i)$.  I~will refer to the chosen function $f$ as a \emph{predictor}, and to $n$ as the \emph{sample size}. A \emph{learning algorithm} takes as input training data, and outputs (some representation of) a predictor $f \in \Y^\X$.\footnote{Note that this describes the usual \emph{batch}, a.k.a.~\emph{offline}, setting of supervised learning. I do not discuss other paradigms such as online or active learning in this article.} 



Success in supervised learning is defined as \emph{generalization} to  future examples: For a typical \emph{test example}  $(x_{\tst},y_{\tst})$, the predicted label $y'_{\tst}=f(x_{\tst})$ should ``equal'' $y_{\tst}$, perhaps approximately. We usually assume the test example is drawn from the same  ``source'' as the training data  --- commonly, i.i.d.~from the same distribution. The quality of the prediction is quantified by $\ell(y'_{\tst},y_{\tst})$, where $\ell:~\Y~\times~\Y \to \RR_{\geq 0}$ is a \emph{loss function} chosen as part of the problem definition. Common loss functions include the 0-1 loss $\ell_{0-1}(y',y) = [y' \neq y]$ for \emph{classification} problems,\footnote{The notation $[P]$ denotes $1$ when predicate $P$ is true, and denotes $0$ when $P$ is false.} as well as the absolute loss $|y'-y|$ or squared loss $(y'-y)^2$ for \emph{regression problems} featuring $\Y  \sse \RR$.

Nontrivial generalization properties are typically only possible if one assumes something about the data.\footnote{The need for such an assumption is formalized by the  \emph{no free lunch theorems} of supervised learning \cite{wolpert_connection_1992,wolpert_lack_1996,schaffer_conservation_1994}.} The Bayesian approach to  machine learning, common in many applications, assumes some parametric form for the distribution generating the data, and postulates a prior on the parameters. This is not the approach I will take in this article. Instead, I will focus on the frequentist --- and some would say ``worst-case'' or ``adversarial'' ---  approach that is common in the computational learning theory community, embodied by the PAC model. Here we assume that the (training and test) data can be explained, perhaps approximately, by a function in some ``simple enough to learn'' class of functions $\H \sse \Y^\X$, often called the \emph{hypotheses}. Equivalently, we  seek a predictor which explains the unseen data roughly  as well as the best hypothesis $h^* \in \H$, whether or not we assume that $h^*$ itself provides a perfect explanation.



 \paragraph{Common Algorithmic Templates.} Perhaps the best known general-purpose supervised learning algorithm is \emph{empirical risk minimization (ERM)}, which chooses as its predictor a hypothesis $f \in \H$ minimizing $\frac{1}{n} \sum_{i=1}^n \ell(f(x_i),y_i)$ --- a quantity called the \emph{training error}, \emph{empirical error}, or \emph{empirical risk} of $f$. %\footnote{When multiple hypotheses minimize the empirical risk, we assume ERM breaks ties arbitrarily.}
A common template for generalizing ERM involves adding a \emph{regularization term} $\psi(f)$ to the  objective function, typically chosen to measure some notion of ``hypothesis complexity.'' An algorithm instantiating this template is known as a \emph{structural risk minimizer (SRM)}, and chooses as its predictor the hypothesis $f \in \H$ minimizing the \emph{structural risk} $\frac{1}{n} \sum_{i=1}^n \ell(f(x_i),y_i) + \psi(f)$. Other well-known algorithms, such as gradient descent and its variations,  can frequently be interpreted as approximate implementations of ERM or SRM.


\paragraph{Proper vs Improper Learning.} A learning algorithm is said to be \emph{proper} if its predictor $f$ is always chosen from the hypothesis class, i.e., $f \in \H$, otherwise it is said to be \emph{improper}. ERM  is an example of a proper learning algorithm, as are SRM algorithms of the form described above.  In the \emph{proper regime} of learning, algorithms are required to be proper. This article will be concerned with the more flexible \emph{improper regime} (a.k.a \emph{representation-independent learning}), where no such constraint is placed on the learner. In other words, all we care about is predictive power at test time, rather than any insights derived from the functional form or representation of the predictor~itself.


\subsection{The PAC Model}
A standard mathematical setup for evaluation of supervised learning algorithms, at least in the theoretical computer science community, is Valiant's \emph{Probably Approximately Correct (PAC) model} of learning (see e.g.~\cite{kearns_introduction_1994,mohri_foundations_2018}). Here, we assume there is an unknown distribution $\D$ on $\X \times \Y$ from which training and test data are  drawn.  Specifically, the labeled datapoints of the training set  $(x_1,y_1), \ldots, (x_n,y_n)$, as well as the test data  $(x_\tst,y_\tst)$, are i.i.d.~from $\D$. Often it is assumed that $\D$ lies in some class of distributions of interest. The \emph{true expected loss}, or simply \emph{loss}, of a predictor $f: \X \to \Y$ is the expected loss it incurs on draws from $\D$, written $L_\D(f) = \Ex_{(x,y) \sim \D} \ell(f(x),y)$.


There are two main ``settings'' in PAC learning. The  \emph{realizable setting} only requires that the data be perfectly explained by some hypothesis in $\H$. More generally, the \emph{agnostic setting} makes no assumption relating the data to the hypotheses, but shifts the goalposts as necessary to allow nontrivial guarantees: the expected loss at test time is evaluated only ``relative'' to that of the best hypothesis $h^* \in \H$. There are other settings which make more nuanced assumptions, such as $\D$ being of a particular parametric form or its support living in some (unknown) lower-dimensional space, etc. I will mostly discuss the realizable and agnostic settings in this article, those being the simplest and most studied from a theoretical perspective. %TODO:We will briefly discuss other settings in Section ??

The PAC model demands high probability guarantees of learners, in the worst case over distributions of interest. Consider first the realizable setting, where $\D$ is such that $\min_{h \in \H} L_{\D}(h) = 0$. A PAC learner has \emph{error} $\epsilon=\epsilon(n)$ and \emph{confidence} $\delta=\delta(n)$ if, when training data consists of $n$ i.i.d~samples from a realizable distribution $\D$, it produces a predictor $f$  satisfying $L_\D(f) \leq \epsilon$ with probability at least $1-\delta$. In the agnostic setting, where $\D$ can be arbitrary, we require $L_\D(f) - \min_{h \in \H} L_\D(h) \leq \epsilon$ with probability $1-\delta$.

In both the realizable and agnostic settings, we look for PAC learners with small $\epsilon$ and $\delta$ as a function of the sample size $n$. An equivalent perspective looks at the sample complexity $m(\epsilon,\delta)$, which is the minimum sample size which guarantees error  at most $\epsilon$ with probability at least $1-\delta$. We say a problem is \emph{PAC learnable} if its PAC sample complexity is finite whenever $\epsilon,\delta > 0$.

For most PAC learning problems, learnability and sample complexity are characterized in terms of a  ``dimension'' of the hypothesis class. Most prominently this is the \emph{VC dimension} for binary classification, the \emph{fat shattering dimension} for agnostic regression, and the \emph{DS dimension} for multiclass classification (see \cite{anthony_neural_1999,daniely_optimal_2014,brukhim_characterization_2022}). Treatment of these is beyond the scope of this article. The unfamiliar reader need not worry, however,  as dimensions will feature only tangentially in our~discussion.




%\paragraph{Learning settings: Realizable, Agnostic, etc.} In learning theory, evaluating a supervised learning algorithm requires specifying a data model and an objective. We will leave the details of the data model flexible for now, to allow for both the PAC model and the adversarial transductive model. Nonetheless we will describe two variations, which we call ``settings'', which cut across different models. The  \emph{realizable setting}  requires only that the data be perfectly explained by some hypothesis $h \in \H$ --- i.e., there exists a hypothesis which is guaranteed to suffer a loss of $0$ on training and test data. The performance of the learning algorithm is its expected loss at test time for some ``worst case'' realizable instance. More generally, the \emph{agnostic setting} makes no assumption relating the data to the hypotheses, but shifts the goalposts as necessary to allow nontrivial guarantees: the expected loss at test time is evaluated only ``relative'' to that of the best hypothesis $h^* \in \H$, again for some ``worst case'' instance. There are other settings which make more nuanced assumptions about the data, such as it is drawn from a distribution of a particular parametric form, or that it lives in some (unknown) lower-dimensional space, etc. We will mostly discuss the realizable and agnostic settings, those being the simplest and most studied from a theoretical perspective.




%%% Local Variables:
%%% mode: latex
%%% TeX-master: "learning_matching"
%%% End:


\section{\label{sec:method}Methodology}

Each SIEM system uses its own RDL to define threat detection rules, and each RDL has its own schema.
For example, the Splunk SIEM uses the SPL to define its threat detection rules.
The task of understanding threat detection rules and recommending relevant MITRE ATT\&CK techniques (or sub-techniques) requires complex reasoning skills.
In the case of LLMs, this can be achieved with a technique called prompt chaining in which each task is divided into multiple sub-tasks in order to understand the complex reasoning behind the task.
Therefore, we employ a multi-phase architecture based on prompt chaining that leverages the power of LLMs to take a SIEM rule defined in any RDL and map it to relevant MITRE ATT\&CK techniques using the power of LLMs.
Our approach is based on the following intuitions:
\begin{itemize}[nosep,leftmargin=*]
    \item \textit{LLMs' implicit knowledge}: LLMs possess deep understanding of diverse RDLs. This enables them to interpret any rule, regardless of the RDL it is defined in, and convert it into comprehensible natural language text.
    \item \textit{LLMs' similarity comparison capability}: LLMs are adept at analyzing and comparing textual descriptions. 
    They can intelligently assess the similarity between two textual inputs to establish a meaningful connection.
\end{itemize}

\methodName has two main phases: (1) the rule to text translation phase, and (2) the MITRE ATT\&CK techniques recommendation phase.
These two phases in the pipeline include six key steps to determine relevant TTPs, as illustrated in Figure~\ref{fig:r2t}.

Although LLMs excel at translating SIEM rules into natural language, they often lack critical domain-specific contextual information related to IoCs in the rules.
To overcome this limitation, the \textit{rule to text translation} phase includes three steps: IoC extraction, contextual information retrieval, and natural language translation.

The workflow begins with the extraction of IoCs from the rules (for example, processes, log source, event codes, and file names) that the rule searches for in the logs (step (1)).In the next sstep a web search agent performs the task of obtaining additional contextual information about the IoCs discovered ((step 2)).
By incorporating this additional domain-specific information, the pipeline enhances the language translation, resulting in a more accurate and meaningful interpretation of SIEM rules.
The rule itself and the IoCs' contextual additional information from the previous stage are then used to translate the rule from RDL to natural language (step (3)).

The \textit{MITRE ATT\&CK techniques} recommendation phase of the pipeline includes the following three steps.
The rule in processed in data source identification step in which the probable origin of the data is identified (step (4)).
The description of the rule is then used to determine probable MITRE ATT\&CK techniques based on the implicit knowledge of the LLM (step (5)).
Finally, using chain-of-thought~\cite{wei2022chain} prompting, the most relevant techniques are extracted from the list of probable techniques (step (6)).
Each step of our method is further described in detail below.


% [bb=0 0 1440 900,width=1.43\linewidth,height=0.9\textwidth]
\begin{figure*}[htbp]
   \includegraphics[width=\textwidth]{Images/stages.jpg}
    
   \caption{An illustration of the different steps in \methodName.}
   \label{fig:stages}
\end{figure*} 

\subsection{IoC Extraction}
The context associated with a SIEM detection rule is crucial for its accurate interpretation and effective application. 
Obtaining this contextual understanding requires comprehensive analysis of the embedded IoCs in the SIEM rule.
In the first step, \methodName systematically identifies and extracts all IoCs, identifying the types of IoCs and their corresponding values that form the foundational elements of the detection rules. 
Leveraging the LLM's inherent understanding of rule structures and IoCs, we employ a zero-shot prompting approach for this task. 
Zero-shot prompting enables the direct extraction of IoCs from the rules without requiring extensive pre-training on specific datasets.

\noindent The result of this stage is a dictionary structure, where:
\begin{itemize}[nosep,leftmargin=*]
    \item Keys represent types of IoC, such as processes, files, IP addresses, and log sources.
    \item Values are lists containing specific IoC details, such as process names, file names, IP addresses, and log source identifiers.
\end{itemize}

In the example depicted in Figure~\ref{fig:stages}(a), the pipeline processes a rule for which relevant MITRE ATT\&CK techniques need to be recommended. 
The IoC extractor LLM produces a dictionary structure as output, organizing the IoCs in a structured format to support subsequent stages in the analysis pipeline. 



\subsection{Contextual Information Retrieval}
In this step, an LLM agent is employed to retrieve relevant information pertaining to the IoCs extracted from the rule.
A REACT agent~\cite{react} was used in this case to generate both reasoning traces and task-specific actions in an interleaved manner.
REACT agents interact with external tools to retrieve additional information that leads to more factual and reliable responses.
The LLM agent conducts a systematic search across web resources to gather additional contextual information for each IoC value present in the rule. 
This step addresses LLMS' lack of up-to-date knowledge or specialized domain expertise (which is critical to understanding the role and significance of the IoCs in the rule), without the need for retraining or fine-tuning.
Figure~\ref{fig:stages}(b) presents an example in which the rule includes the process name \texttt{soaphound.exe} as an IoC.
As can be seen, the web search results indicate that \texttt{soaphound.exe} is being used for active directory (AD) enumeration, which is important for the understanding of the attack. 

\subsection{Natural Language Translation}

The translation of detection rules into natural language textual descriptions fulfills three key objectives:
\begin{enumerate}[nosep,leftmargin=*]
    \item \textbf{Ensures that \methodName is format-agnostic}: It converts rules defined in various RDL formats into a generic, unstructured text format, ensuring compatibility with different SIEM systems, regardless of the specific rule format.
    \item \textbf{Provides contextual explanation}: It includes all relevant contextual information to produce a concise and comprehensible explanation of the rule.
    \item \textbf{Enhances the comprehension for LLMs}: It enables LLMs to more effectively compare the translated rule with descriptions in the MITRE ATT\&CK framework by providing a unified textual representation.
\end{enumerate}
To achieve these objectives, a zero-shot prompting technique is employed. 
The input to the LLM comprises two components:
\begin{itemize}
    \item \textbf{Syntactical information}: The rule itself, providing the structural and operational details.
    \item \textbf{Contextual information}: Details of the IoCs extracted from the rule, providing semantic insights into the rule's intent and function.
\end{itemize}
The LLM utilizes these inputs to generate a natural language textual description of the rule. 
This transformation not only ensures a more interpretable representation but also facilitates further steps of analysis and comparison, particularly in aligning the rule with MITRE ATT\&CK techniques and sub-techniques.



\subsection{Data Source or Mitigation Identification}
Identifying the most relevant data component or mitigation associated with the rule description in this step is critical for filtering out irrelevant MITRE ATT\&CK techniques (or sub-techniques) in subsequent steps of the pipeline.
In the MITRE ATT\&CK framework, data sources represent various categories of information that can be gathered from sensors or logs. 
These data sources include \textit{data components}, which are specific attributes or properties within a data source that are directly relevant to detecting a particular technique or sub-technique~. 
For example, in the context of the rule described in Figure~\ref{fig:stages}(a), the term \texttt{Endpoint.Processes} indicates that the activity is happening on an endpoint. 
Presence of the terms such as, \texttt{soaphound.exe}, \texttt{--buildcache}, \texttt{--certdump} and etc. indicate that the rule searches for command line execution of an executable named \texttt{soaphound.exe} with specific parameters. 
Therefore, the appropriate data source in this example is \textit{Command}, with the corresponding data component being \textit{Command Execution}.
Additionally, \textit{mitigations} are defined as categories of technologies or strategies that can prevent or reduce the impact of specific techniques or sub-techniques. 
The MITRE ATT\&CK framework explicitly establishes relationships between data components, mitigations, and techniques (or sub-techniques), enabling a systematic approach for identifying relevant elements.

To identify the most relevant data component or mitigation associated with a given rule description, we utilize agentic retrieval augmented generation (RAG), which incorporates an AI Agent-based implementation of the RAG framework.
Data from the MITRE ATT\&CK framework, specifically related to data components and mitigations, is stored in a vector database (e.g., ChromaDB). 
The process begins with the rule description from the previous stage, which serves as the input to the AI Agent. 
The LLM-powered agent automatically generates a search query tailored to retrieve relevant information from the RAG database.

For each query, the system retrieves the five most similar documents from the database, each containing contextual information about data components or mitigations. 
These documents are then utilized by the LLM agent to contextualize the rule description. 
By comparing the content of these retrieved documents with the rule description, the LLM agent determines and outputs the most relevant data component or mitigation along with a chain-of-thought as to why the data component or mitigation is related to the rule.


\subsection{Probable Technique Recommendation}

In this step, an LM agent is utilized to propose probable MITRE ATT\&CK techniques (and sub-techniques) that may be relevant to the description of the provided rule.
We used a REACT agent in this step as well to utilize both implicit and explicit knowledge during reasoning.
For explicit knowledge, the agent searches the MITRE ATT\&CK framework to obtain the list of probable techniques (and sub-techniques).
The natural language description of the rule from the previous step serves as input to the LLM agent.
The output of this stage consists of a list of JSON objects, each containing the MITRE technique ID, technique name, and technique description as seen in Figure~\ref{fig:stages}(c).

Throughout our experiments, we observed that as the number of recommendations ($k$) increases, both the framework's average recall and precision initially improve, however beyond a certain threshold of $k$, the %average 
precision begins to decline.
Based on these observations(please refer Table~\ref{tab:results3}), we selected a $k$-value of 11 to ensure a high recall.



\subsection{Relevant Technique Extraction}
In this step, \methodName refines the set of probable MITRE ATT\&CK techniques identified in the previous stage by eliminating irrelevant entries.
This step in the pipeline serves two primary purposes: (1) to enhance precision while maintaining recall achieved in previous step, and (2) to provide a clear rationale for the selection of the labels, ensuring transparency and interpretability of the mapping process.
This refinement process is grounded in the assumption that LLMs are effective for text similarity matching tasks.

The process comprises two key steps:
\begin{itemize}
    \item \textit{Rule-technique comparison}: The description of each technique in the set of probable techniques is compared with the rule description. 
    A chain-of-thought technique is then applied to elucidate the reasoning behind the association of each technique with the rule.
    \item \textit{Confidence calculation}: The generated chain-of-thought rationale for each technique (or sub-technique) is compared with the rule description to compute a relevance (or confidence) score, as done in prior work~\cite{freitas2024ai}.
    % \item \textbf{Reasoning}: \new{Add here the reasoning that it provides - explaining in NLP why it was selected...}
\end{itemize}

Techniques with higher confidence scores are deemed more relevant to the rule. 
Conversely, techniques with scores falling below a predefined threshold are excluded.
The techniques retained after this filtering step represent the most relevant techniques corresponding to the given rule's description. 


The chain-of-thought (CoT) rationale generated during the comparison of each rule to its probable technique is also provided as an output in this step.
This rationale offers a detailed natural language explanation, articulating why a particular technique is relevant to the given rule. 
Such explanations are highly valuable for security analysts, as they provide clear and transparent reasoning behind the mapping, enabling analysts to better understand and validate the association between the rule and the technique.
Other classification models proposed in previous works within this domain also suffer from the limitation of being black-box models, which lack the ability to provide clear reasoning or explanations. 
Unlike \methodName, these models fail to generate transparent, CoT rationales that explain why a particular rule is mapped to a specific technique, making them less interpretable and less useful for security analysts.
\newcolumntype{g}{>{\columncolor{green!10}}c}
\setlength\tabcolsep{7pt}
\begin{table}[htbp]
\centering
\huge
\newcolumntype{b}{>{\columncolor{blue!10}}c}
\renewcommand{\arraystretch}{1.6}
\resizebox{0.5\textwidth}{!}{

\begin{tabular}{lccccc}

\toprule
\multicolumn{1}{c}{\multirow{2}{*}{Method}} & \multicolumn{4}{c}{Data Quality} &  \\ \cline{2-6} 
\multicolumn{1}{c}{}                       & Nums.       & Cons.    & Avg Attr.      & Synt.    \\ \midrule
IFeval~\cite{zhou2023instruction} & 541  & H & 1.54 & \ding{51} \\
FollowBench~\cite{jiang2023followbench} & 820 & H/S & 3.0 & \ding{51}  \\
CFBench~\cite{zhang2024cfbench} & 1000 & H/S & 4.24 & \ding{55} \\
InFoBench~\cite{qin2024infobench} & 500 & H/S & 4.5 & \ding{55} \\
\our (FineWeb Split) & 6159 & H/S & \textbf{45.9} & \ding{55} \\
\our (Multi-source Split) & 1600 & H/S & \textbf{29.9} & \ding{55} \\
\bottomrule
\end{tabular}%
}
\caption{
  Detailed comparison of relevant works. Ours
represents our dataset construction approach. \textquotesingle Nums.\textquotesingle, \textquotesingle Cons.\textquotesingle, \textquotesingle Avg Attr.\textquotesingle,
and \textquotesingle Synt.\textquotesingle\  denote the number of samples, constraint types, average number of attributes, and whether the data is synthesized.
}


  \label{tab:comparison}
\end{table}

\section{Conclusion Remarks}
This work proposes a RBG graph model for disease spreading via hubs. We study the joint effect of the agent density, hub density, and connection function. The existence of a critical hub density depends only on the boundedness of the support of the connection function, which relates to curbing the traveling distance of individuals. When it comes to dispersion, both the degree distribution and the percolation threshold suggest that increasing dispersion helps spread the disease. The percolation properties of RBG graphs relate to unipartite graphs with modified connection functions. 
An interesting question in this direction is if and when the properties of the RBG graphs can be well represented by unipartite graphs with some modified connection functions. Our conjecture is that for independent connections between different pairs of agents, such representation is unlikely due to the oblivion of the local dependence (present in the RBG models). 
 Another direction is to consider hybrid models where agents may get infected either through common hubs or direct interactions between agents. The former infection mechanism is more centralized than the latter. 
%%
%% Bibliography
%%
\bibliography{main}

\newpage
\appendix
Frontier language models demonstrate a remarkable mismatch between their problem-solving capabilities and poor out-of-box verification capabilities.
These limitations have largely been attributed to the inability of current language models to self-diagnose hallucinations or enforce rigour \citep{zhang_how_2023,orgad_llms_2024,snyder_early_2024,kamoi_evaluating_2024, tyen_llms_2024, DBLP:conf/iclr/0009CMZYSZ24}.
However, our findings that models can be directed to accurately perform verifications at scale suggest that these out-of-box limitations can be addressed with standard methods like instruction tuning.
We compiled a set of challenging reasoning problems and candidate solutions to provide a benchmark for these deficits.

Each entry in this benchmark consists of a question, a correct candidate response, and an incorrect candidate response, and is manually curated from the residuals of our sampling-based search experiments (Section~\ref{section:pipeline}).
An example entry from this benchmark can be found below (see Appendix~\ref{app:examplebenchmark} for more).

\vspace{0.4cm}
\begin{tcolorbox}[title=Question from LiveBench Reasoning (Web-of-Lies Puzzle), breakable]
In this question, assume each person either always tells the truth or always lies. The person at the campground thinks their friend is lying. Mateo is at the aquarium. The person at the restaurant says the person at the hotel lies. Farid is at the movie theater. The person at the movie theater says the person at the campground lies. Ryan is at the shopping mall. The person at the cafe says the person at the campground lies. The person at the observatory says the person at the museum lies. The person at the museum says the person at the restaurant tells the truth. The person at the campground says the person at the airport tells the truth. The person at the campground tells the truth. Luna is at the restaurant. The person at the movie theater saw a firetruck. Isabella is at the observatory. The person at the amusement park says the person at the museum tells the truth. The person at the restaurant says the person at the shopping mall tells the truth. Liam is at the beach. Ayaan is at the airport. The person at the restaurant says the person at the beach tells the truth. Maya is at the amusement park. The person at the restaurant says the person at the aquarium lies. The person at the restaurant says the person at the movie theater lies. The person at the library says the person at the campground tells the truth. Elowen is at the hotel. Priya is at the library. The person at the gym says the person at the restaurant tells the truth. Kehinde is at the cafe. Anika is at the museum. Quan is at the campground. Soren is at the gym. Does the person at the museum tell the truth? Does the person at the restaurant tell the truth? Does the person at the movie theater tell the truth? Think step by step, and then put your answer in **bold** as a list of three words, yes or no (for example, **yes, no, yes**). If you don't know, guess.
\end{tcolorbox}
\definecolor{mydarkgreen}{rgb}{0.0, 0.5, 0.0} %


\begin{paracol}{2}

\switchcolumn[0]
\begin{tcolorbox}[breakable, colframe=mydarkgreen!70, colback=mydarkgreen!10, title=Correct response]
Here's how we can break down this logic puzzle:

1. **Quan is at the campground and tells the truth.** This is a given fact.

...

4. **The person at the restaurant (Luna) says the person at the hotel (Elowen) lies.**  We don't know yet if Luna is lying or telling the truth.

...

10. **Since Luna tells the truth about Farid, Luna *always* tells the truth.**  This unlocks several other pieces of information.

...

13. **The person at the cafe (Kehinde) says the person at the campground lies.** Since Quan tells the truth, Kehinde is lying.

Therefore:
* **Anika (museum):** Tells the truth.
* **Luna (restaurant):** Tells the truth.
* **Farid (movie theater):** Lies.

**yes, yes, no**
\end{tcolorbox}

\switchcolumn
\begin{tcolorbox}[breakable, colframe=red!70, colback=red!10, title=Incorrect Response Due to Hallucination]
Here's how we can solve this logic puzzle:

1. **Quan is at the campground and tells the truth.** This is given.

2. **The person at the airport (Ayaan) tells the truth.** Quan (campground) says the person at the airport tells the truth, and Quan tells the truth.

...

4. **Luna (restaurant) lies.**  The person at the restaurant says the person at the hotel (Elowen) lies. \textbf{\color{red} Later, the person at the restaurant says the person at the campground (Quan) lies.} Since Quan tells the truth, Luna (restaurant) must be lying.

...

Therefore:
* **Anika (museum): lies**
* **Luna (restaurant): lies**
* **Farid (movie theater): lies**

**Answer: no, no, no** 
\end{tcolorbox}
\end{paracol}

\noindent\begin{minipage}{\textwidth}
\captionof{figure}{Example of an entry in our verification benchmark. The question is sourced from the LiveBench Reasoning benchmark, and the two responses are generated by Gemini v1.5 Pro. The green response has the correct final answer; the red response has the wrong final answer due to hallucinating a non-existent clause.}
\label{fig:example}
\vspace{0.4cm}
\end{minipage}

\noindent
On each entry, our benchmark studies verification accuracy on two tasks:
\begin{enumerate}
    \item \textbf{Scoring task.} When given only the question and one of the responses, is the model able to discern the correctness of the response?
    \item \textbf{Comparison task.} When provided the whole tuple with the correctness labels of the responses masked and a guarantee that at least one response is correct, is the model able to discern which response is correct and which is incorrect?
\end{enumerate}

\noindent
The scoring task is also evaluated over a separate set of (question, response) pairs where the response reaches the correct final answer by coincidence but contains fatal errors and should be labeled by a reasonable verifier as being incorrect; an example can be found in Appendix~\ref{app:examplebenchmark}.
In the scoring task, models are provided only with the task description; in the comparison task, models are provided only with the task description and a suggestion to identify disagreements between responses in its reasoning.

Table~\ref{tab:benchmark} lists the baseline performances of current commercial model offerings on this benchmark.
Gemini v1.5 Pro is omitted from the benchmark as the entries in the benchmark are curated from the residuals of Gemini v1.5 Pro.
The prompts used in Table~\ref{tab:benchmark} are provided in Appendix~\ref{app:benchmarkprompts}.

As we previously observed, and has been noted in prior works \citep{tyen_llms_2024, kamoi_evaluating_2024}, verification errors are typically due to low recall.
Even the easier comparison task, models perform only marginally better---and often worse---than random chance.
In many cases, Consistency@5 performs worse than one-shot inference because Consistency simply averages out noise from an output distribution, meaning that a model biased towards producing an incorrect answer will do so with higher probability under Consistency.
Addressing these deficits in verification capabilities---which we see as low-hanging fruit for post-training---would enable not only better sampling-based search, but also other downstream applications of verification including reinforcement learning \citep[e.g.][]{o1-preview,deepseekai2025deepseekr1incentivizingreasoningcapability}, data flywheeling \citep[e.g.,][]{welleck_generating_2022}, and end-user experience (see Section~\ref{sec:related} for further discussion).


\begin{table}[htbp]
\centering
\begin{tabular}{llcccccc}
\toprule
\textbf{Model} & \textbf{Metric} & \multicolumn{3}{c}{\textbf{Scoring Accuracy}} & \multicolumn{1}{c}{\textbf{Comparison Accuracy}} \\
\cmidrule(lr){3-5} \cmidrule(lr){6-6}
 &  & \textbf{Correct} & \textbf{Wrong} & \textbf{Flawed} &  \\
\midrule
\multirow{2}{*}{GPT-4o} & Pass@1    & 76.5\%  & 31.0\% & 22.2\% & 43.2\%\\
 & Consistency@5 & 77.4\% & 30.0\% & 11.1\% & 35.4\% \\
\midrule
\multirow{2}{*}{Claude 3.5 Sonnet} & Pass@1 & 89.6\% & 22.5\% & 33.3\% & 56.1\% \\
 & Consistency@5 & 90.3\% & 17.5\% & 33.3\% & 61.2\% \\
\midrule
\multirow{2}{*}{o1-preview} & Pass@1 & 100\% & 68.8\% & 80.0\% & 84.5\% \\
 & Consistency@5 & 100\% & 79.4\% & 88.8\% & 92\% \\
\midrule
\multirow{2}{*}{Gemini 2.0 Flash} & Pass@1 & 73.5\% & 44.5\% & 60\% & 58\%  \\
 & Consistency@5 & 77.4\% & 42.5\% & 66.6\% & 58.7\% \\
\midrule
\multirow{2}{*}{Gemini 2.0 Thinking Flash} & Pass@1 & 75.4\% & 56.5\% & 53.3\%  & 80\%  \\
 & Consistency@5 & 77.4\%  & 55\% & 55.5\%  & 89.1\% \\
\midrule
\multicolumn{2}{c}{Random guessing}  & 80\% & 20\% & 20\% & 50\% \\
\bottomrule
\end{tabular}
\caption{Accuracy rates of commercial language models on our verification benchmark. For the task of response scoring (Scoring Accuracy), accuracy rates are broken down for entries that require identifying a correct response as being correct (Correct), entries that require identifying a wrong response as being wrong (Wrong), and entries that require identifying a wrong response that coincidentally reaches the correct answer as being wrong (Flawed).
GPT-4o and Claude 3.5 Sonnet only perform marginally better than random guessing across all tasks. o1-Preview performs better, but still fails to identify 20-30\% of wrong responses.
}
\label{tab:benchmark}
\end{table}

\section{Reward functions: properties and examples}\label{subsec:properties}

Recall that miner strategies take as input the amount of reward available for extending each existing block at time $t$, as specified by the reward function $R$, and make decisions about where to mine, what to include, and what to broadcast accordingly. This section defines a set of natural properties that reward functions might have. In \Cref{sec:examples}, we apply these properties to transaction fees and LVR, two of the primary MEV sources observed empirically to date.
While we define these properties in the context of the NCG in this paper, we believe their applicability extends far beyond Proof-of-Work and selfish mining. Our framework can be used to characterize rewards and their implications for the incentives of consensus participants across blockchain protocols.

Recall that in the NCG, given a set of miner strategies, three independent random variables $\vec{t},\vec{m},r$ are drawn and are used to compute a set of views $V_t^m$ for all miners $m$ and all times $t$.
Let $\mathcal{V}_t^m$ be the support $V_t^m$, meaning the set of views achievable at time $t$ for \emph{some} realization of $\vec{t},\vec{m},r$. Initially, $\mathcal{V}_0^m=\{V_0\}$ for all $m$, where $V_0:=\{B_0\}$ is the view containing a single genesis block $B_0$ such that $\tstamp(B_0)=0$. Miner strategies in the NCG take the realization of a reward function as input. That is, at time $t$, miner $m$ sees the reward $R^m(t,V_t^m,B,r,B')$ for extending each block $B\in V_t^m$ by a valid block $B'\in\mathcal{B}^m(t,V_t^m,B,r)$.

A miner-independent reward function yields the same value for the block regardless of who created it. This 
corresponds to a setting where all miners have access to the same set of rewards (e.g., the common value setting), and thus, we drop the superscript $m$. In practice, some reward sources may be heterogeneous between block producers (e.g., from private order flow or from differing abilities to extract MEV \cite{bahrani2024centralization}). All reward functions considered in this paper will be miner-independent, but the properties can be readily generalized by tracking the subset of miners with access to each reward source. See \Cref{sec:conclusion} for a discussion of extending this work.

\begin{definition}[Miner-Independent Rewards] \label{def:minerindependent}
    A reward function $R$ is \emph{miner-independent} if for all times $t$, all miners have the same set of valid views, the same set of valid blocks extending each block in those views, and equal rewards from any such valid block.\footnote{\label{foot:bijection}Technically, since blocks include information about their creator, it would be more accurate to say that there is a bijection between the set of valid views/blocks for any pair of miners. We overlook this formality to simplify notation.} Formally, $R$ is miner-independent if for all $t$, and all $m,m'\in M$, 
    \begin{itemize}
        \item $\mathcal{V}_t^m=\mathcal{V}_t^{m'}$,
        \item for all $V\in\mathcal{V}^m_t$, all blocks $B$ in $V$, and all $r$, we have $\mathcal{B}^m(t,V,B,r)= \mathcal{B}^{m'}(t,V,B,r)$,
        \item for all $V\in \mathcal{V}^m_t$, all $r$, all parent blocks $B$ in $V$, and all valid blocks $B' \in \mathcal{B}^m(t,V,B,r)$, we have $R^m(t,V,B,r,B')= R^{m'}(t,V,B,r,B')$.
    \end{itemize}
\end{definition}

We can also characterize reward functions that grow according to the same distribution without depending on the chain's history. The following property limits the dependence of $R$ on the view. Intuitively, it says that the only relevant information in the view that affects the amount of reward in a block is the \emph{timestamp of its parent}.

\begin{definition}[View-Independent Rewards]\label{def:viewindependent}
    A reward function $R$ is \emph{view-independent} if for all times $t'<t$, any two views $V_1,V_2\in\mathcal{V}_{t'}$ such that $\tstamp(B_1)=\tstamp(B_2)=t'$ for some blocks $B_1\in V_1,B_2\in V_2$, we have:
    \begin{itemize}
        \item for all $r$, the set of valid blocks extending $B_1$ at $t$ in $V_1$ is the same as the set of valid blocks extending $B_2$ at $t$ in $V_2$, $\mathcal{B}(t,V_1,B_1,r)=\mathcal{B}(t,V_2,B_2,r)$,\footnote{Recall that when we invoke a view and randomness together as inputs to a function, we implicitly assume that the randomness could give rise to the view.}
        and
        \item for every valid block $B'\in\mathcal{B}(t,V_1,B_1,r)$, we have
        \[
            \Pr_{r,\vec{t},\vec{m}\vert V_1}[R(t,V_1,B_1, r, B')=x]=\Pr_{r,\vec{t},\vec{m}\vert V_2}[R(t,V_2,B_2, r, B')=x]
        \]
        for all $x$.
    \end{itemize}
\end{definition}

Note that fixing a view $V_1$ (resp. $V_2$) can update the distribution of the $r,\vec{t},\vec{m}$. We use the subscript $r,\vec{t},\vec{m}\vert V_i$ to refer to the posterior distribution of these random variables conditioned on $V_1,V_2$. Block rewards are view-independent if and only if there is no halving (since halving occurs at fixed block heights).
As another (non-)example, \Cref{rem:patient-not-vi} demonstrates how transaction fees that are not fully claimed by block $B$ (e.g., from finite block sizes) are not view-independent.  

View-independence already limits the dependence of $R$ on the view to the timestamp of the parent block. We next define a subset of view-independent rewards where the dependence on view is limited to the length of \emph{elapsed time since the parent block} (and is the same regardless of the exact parent block timestamp).

\begin{definition}[Static Rewards]\label{def:static}
    A reward function $R$ is \emph{static} if for all $\Delta>0$, all times $t_1,t_2$ and views $V_1\in\mathcal{V}_{t_1}$ and $V_2\in\mathcal{V}_{t_2}$ such that $\tstamp(B_1)=t_1-\Delta$ and $\tstamp(B_2)=t_2-\Delta$, we have:
    \begin{itemize}
        \item for all $r$, the set of valid blocks extending $B_1$ at $t_1$ in $V_{1}$ is the same as the set of valid blocks extending $B_2$ at $t_2$ in $V_{2}$, $\mathcal{B}(t_1,V_{1},B_1,r)=\mathcal{B}(t_2,V_{2},B_2,r)$, and
        \item for all valid blocks $B'\in\mathcal{B}(t_1,V_1,B_1,r)$, we have
        \[
            \Pr_{r,\vec{t},\vec{m}\vert V_1}[R(t_1,V_1,B_1, r, B')=x]=\Pr_{r,\vec{t},\vec{m}\vert V_2}[R(t_2,V_2,B_2,r,B')=x]
        \]
        for all $x$.
    \end{itemize}
\end{definition}

\Cref{rem:patientisstatic} highlights that transaction fees are static using the \citet{carlsten2016instability} model with constant arrival rate and infinite block sizes. Conversely, \Cref{rem:non-local-lvr} illustrates how LVR is not static because it depends on the CEX price of an asset (which impacts the step size of the Geometric Brownian Motion). \Cref{rem:local-lvr-static} demonstrates that within the same price neighborhood, a type of LVR (which we call ``resetting'') \textit{is} static.

\begin{definition}[Maximum Rewards \& Maximizing Blocks]\label{def:max-block}
    Given a reward function $R$, we define the maximizing block function $B_{\text{opt}}$ as 
    \[
    	B_{\text{opt}}(t,V,B,r):=\underset{B'\in\mathcal{B}(t,V,B,r)}{\arg\max} R(t,V,B,r,B').
    \]
    We further define the maximum reward function $R_{\text{opt}}$ as 
    \[
    	R_{\text{opt}}(t,V,B,r):=R(t,V,B,r,B')
    \] for some $B'\in B_{\text{opt}}(t,V,B,r)$.
\end{definition}

Observe that if a reward function $R$ is static, then $\Ropt(t,V,B,r,B')$ can be rewritten as a two-variable function of just $r$ and the time $\Delta$ between $\tstamp(B)$ and $t$. 

Lastly, we define \emph{persistent rewards}, which arrive at some time and can be claimed at most once. Upon arrival, they remain indefinitely claimable by any block whose ancestors have not already claimed them. Let $\Rc(B)$ denote the amount of reward attributed to the block creator if the block becomes canonical and $\chain(B)$ the set of blocks on the ancestral path of $B$ (including $B$).

\begin{definition}[Persistent Rewards]\label{def:persistent}
A reward function $R$ is \emph{persistent} if for all realizations of $\vec{t},\vec{m},r$, at any time $t$, for all blocks $B$ in the resulting view $V$, we have:
\begin{itemize}
    \item for all $B'\in\mathcal{B}(t,V,B,r)$, 
    \begin{align}\label{eq:persistence}
        R(t,V,B,r,B')\leq R_{\text{opt}}(t,V_0,B_0,r)-\sum_{B''\in \chain(B)} \Rc(B''),
    \end{align}
    \item there exists some $B'\in\mathcal{B}(t,V,B,r)$ for which the above holds with equality.
\end{itemize}
\end{definition}

We sometimes call a non-persistent reward function \emph{ephemeral}. \Cref{rem:patient-persistent} highlights that transaction fees are persistent if the users creating the transactions are patient (willing to wait for inclusion and not cancel pending transactions). On the other hand, fees from transactions submitted by impatient users (as in \Cref{rem:impatient}) are not persistent since the canceled transactions are no longer claimable by future blocks. 

Persistent rewards are not affected by orphan or uncle blocks, but they \emph{may} be view-dependent since they are affected by the claimed rewards on the ancestral path of a block. The following lemma states that persistent rewards functions are view-independent if all blocks in all valid views claim the maximum available rewards. The proof can be found in \Cref{pr:lemm1}.

\begin{lemma}[Persistent \& Maximizing Blocks $\implies$ View-Independent]\label{lem:all-claim}
    Let $R$ be persistent. Then $R$ is view-independent if for all $t$, all $V\in\mathcal{V}_t$, all parent-child blocks $B,B'$ in $V$, and all $r$, we have $\Rc(B')=\Bopt(t,V,B,r)$.
\end{lemma}

The following lemma shows that static and persistent reward functions accrue linearly over time since the parent block, with a constant slope and intercept across blocks (but may be random depending on $r$). If a reward function is persistent and static, it can be simulated by drawing the randomness of $r$ to set the slope $a$ and the intercept $b$ of the maximum available reward function $\Ropt$. Then, for any block $B$ in any view, the reward for extending $B$ at time $\tstamp(B)+\Delta$ equals $a\cdot \Delta + b$. This is the model of transaction fee accrual in \citet{carlsten2016instability} and MEV accrual in \citet{schwarz2023time}. The proof can be found in \Cref{pr:lemm2}

\begin{lemma}[Static \& Persistent $\implies$ Linear]\label{lem:linear}
    Let $R$ be static and persistent. If $\Ropt(t,V,B,r)$ is differentiable with respect to $t$, then it is of the form $a(r)\cdot (t-\tstamp(B)) + b(r)$.
\end{lemma}

Note that the transaction fees defined in \citet{carlsten2016instability} are linear; we use this same reward function as part of our instantiation in \Cref{sec:multiplerewards}.




%\newpage


%\begin{itemize}

%\item Move sembra più adatto per la verifica rispetto a Solidity (fare esempi di quando siamo riusciti a verificare meglio) In Section \ref{sec:comparison:language}  sembra sempre meglio, eccetto che per immutable.

%\item \enriconote{più facile capire la ground truth in Move perché non ci sono le schifezze di Solidity}

%\item in entrambi i casi, esprimere bene alcune proprietà richiede conoscenze di basso livello di come funziona il linguaggio (per Solidity, reentrancy, fallback; per Move, quella roba sulle features dei token, ...). Dare spec richiede conoscenza della piattaforma abbastanza low level, perché a volte la verifica fallisce e non capisci perché

%\item per Solidity ecosistema più ricco di tool per la verifica, anche che vanno oltre le proprietà verificabili da Certora

%\item Certora nel complesso più espressivo, ma sembra possibile migliorare il Move Prover per renderlo paragonabile a Certora riguardo ad espressività (ad esempio, trace properties, metamorphic properties, reference a transition variables)

%\item esistono comunque proprietà interessanti che nessuno dei tool sembra di grado di esprimere (hyperproperties?)
%(It has been observed that properties that abstract the system %, although in general harder to verify in practice, are  more desirable and have a better return-on-investment than low-level properties~\cite{Xu24langsec}.) 

%\item Non abbiamo pretesa di esaustività: è possibile/probabile che ci siano altre differenze rilevanti che non abbiamo incontrato nei nostri esperimenti. Ciò comunque non invalida i nostri finds. 

%\item Future work includes 
%extending our dataset and exploring other features, % \enriconote{mah potrebbe far pensare che abbia esplorato giusto un paio di features}
%as well as investigating  verification in other contract languages.
% \begin{itemize}
%    \item sarebbe bello mettere mutants per vedere quanto i tool sono bravi a fare bug detection
%    \item altri linguaggi?
%    \item gas rent fees?
%\end{itemize} 
%\end{itemize}


\noindent \enriconote{TODOs:  \ \\ \ \\ 
 %* change  github url to solidity-vs-move-verification \ \\ \ \\ 
 * add link to table in each README \ \\ %(e.g. https://github.com/blockchain-unica/certora-vs-move-prover-benchmark/blob/main/contracts/bank/README.md)
 * define N/A and N/D in tables \ \\
 %* link codice appendice da Sez 4 \ \\
 * %cambiare nomi prop local-global intra-inter 
 in MSL a rigore nella inter dovremmo usare disable\_invariants\_in\_body \ \\ 
}

% \section{Taxonomy of Research on SDN Software Security}\label{sec:tx}
To systematically extract insights and understand the current state-of-the-art in SDN software security, our SLR focuses on analyzing specific features of each publication. The primary outcome of this analysis is developing a novel, four-dimensional taxonomy. This taxonomy will structure the body of existing research and directly address the research questions outlined in Section\ref{sec:rqs}.
\subsection{Structure of the Taxonomy}
The proposed taxonomy is a four-dimensional model designed to categorize and analyze the research landscape on SDN software security. The dimensions and their defining features are as follows:
\begin{itemize}
    \item \textbf{Objectives (What):} This dimension identifies the security goals targeted by the research. Objectives include bug detection, fixing, localization, exploitation, mitigation, categorization, and hardening.
    %This dimension classifies the security goals research studies aim to achieve or address. Seven recurring objectives have been identified, including but not limited to bug detection, attack detection/prevention, and performance/scalability optimization.
    \item \textbf{Targets (Where):} This dimension focuses on the specific SDN software components subject to security analysis or investigation. Common targets encompass controllers, data planes, APIs, and SDN applications.
    \item \textbf{Methodology (How):}  This dimension categorizes the diverse research methodologies employed in the reviewed literature. These methodologies can be further subdivided into testing approaches (e.g., static analysis, dynamic testing), testing types (e.g., white box, black box, gray box), and specific analysis techniques (e.g., model checking, fuzzing, symbolic execution).
    \item \textbf{Representations (Which):} This dimension encompasses the various approaches used to represent and structure information related to the testing process. The choice of representation can significantly impact the efficiency, comprehensibility, and effectiveness of test execution.
\end{itemize}
Figure\ref{fig_txn} provides a visual representation of the proposed four-dimensional taxonomy.
\begin{figure}[ht!]
\centering
\begin{adjustbox}{width=\linewidth, center}
\includegraphics{Diagram2.png}
\end{adjustbox}
\caption{Taxonomy on Security of SDN Software.}
\label{fig_txn}
\end{figure}





% \input{rimasugli.tex}

\end{document}