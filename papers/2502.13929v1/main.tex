\documentclass[final,a4paper,UKenglish,cleveref,autoref]{oasics-v2021}

\bibliographystyle{plainurl} % the mandatory bibstyle

% titolo provvisorio!
\title{Formal verification in Solidity and Move: insights from a comparative analysis}  
% (Experience report)

%\titlerunning{Dummy short title} %TODO optional, please use if title is longer than one line

\author{Massimo Bartoletti}{University of Cagliari, Italy \and \url{http://blockchain.unica.it} }{bart@unica.it}{https://orcid.org/0000-0003-3796-9774}{Partially supported by project SERICS (PE00000014) and PRIN 2022 DeLiCE (F53D23009130001) under the MUR National Recovery and Resilience Plan funded by the European Union -- NextGenerationEU.}

\author{Silvia Crafa}{University of Padova, Italy}{crafa@math.unipd.it}{https://orcid.org/0000-0003-0993-4734}{Supported by the National Recovery and Resilience Plan (NRRP) Project ``Securing sOftware Platforms - SOP'', CUP H73C22000890001}

\author{Enrico Lipparini}{University of Cagliari, Italy \and \url{https://elipparini.github.io/}}{enrico.lipparini@unica.it}{https://orcid.org/0009-0009-0428-4403}{Supported by project PRIN 2022 DeLiCE (F53D23009130001) under the MUR National Recovery and Resilience Plan funded by the European Union -- NextGenerationEU.}

%TODO mandatory, please use full name; only 1 author per \author macro; first two parameters are mandatory, other parameters can be empty. Please provide at least the name of the affiliation and the country. The full address is optional. Use additional curly braces to indicate the correct name splitting when the last name consists of multiple name parts.

\authorrunning{M. Bartoletti, S. Crafa, E. Lipparini} 

%TODO mandatory. First: Use abbreviated first/middle names. Second (only in severe cases): Use first author plus 'et al.'

\Copyright{M. Bartoletti, S. Crafa, E. Lipparini}
%TODO mandatory, please use full first names. LIPIcs license is "CC-BY";  http://creativecommons.org/licenses/by/3.0/

% \ccsdesc[500]{Software and its engineering~Formal methods}
\ccsdesc[500]{Software and its engineering~Formal software verification}

%TODO mandatory: Please choose ACM 2012 classifications from https://dl.acm.org/ccs/ccs_flat.cfm 

\keywords{Smart contracts, Solidity, Move, Verification, Blockchain}

\category{} %optional, e.g. invited paper

\relatedversion{} %optional, e.g. full version hosted on arXiv, HAL, or other respository/website
%\relatedversiondetails[linktext={opt. text shown instead of the URL}, cite=DBLP:books/mk/GrayR93]{Classification (e.g. Full Version, Extended Version, Previous Version}{URL to related version} %linktext and cite are optional

% \supplement{\url{https://github.com/fsainas/contracts-verification-benchmark}}%optional, e.g. related research data, source code, ... hosted on a repository like zenodo, figshare, GitHub, ...
%\supplementdetails[linktext={opt. text shown instead of the URL}, cite=DBLP:books/mk/GrayR93, subcategory={Description, Subcategory}, swhid={Software Heritage Identifier}]{General Classification (e.g. Software, Dataset, Model, ...)}{URL to related version} %linktext, cite, and subcategory are optional

%\funding{(Optional) general funding statement \dots}%optional, to capture a funding statement, which applies to all authors. Please enter author specific funding statements as fifth argument of the \author macro.

% \acknowledgements{I want to thank \dots}%optional

\nolinenumbers %uncomment to disable line numbering

% \EventEditors{Bruno Bernardo and Diego Marmsoler}
% \EventNoEds{2}
% \EventLongTitle{5th International Workshop on Formal Methods for Blockchains (FMBC 2024)}
% \EventShortTitle{FMBC 2024}
% \EventAcronym{FMBC}
% \EventYear{2024}
% \EventDate{April 7, 2024}
% \EventLocation{Luxembourg City, Luxembourg}
% \EventLogo{}
% \SeriesVolume{118}
% \ArticleNo{6}

\def\method{\text MixMin~}
\def\methodnospace{\text MixMin}
\def\genmethod{$\mathbb{R}$\text Min~}
\def\genmethodnospace{ $\mathbb{R}$\text Min}

% fix listings UTF-8
\usepackage[utf8]{inputenc}
\usepackage{color}
\usepackage[usenames,dvipsnames]{xcolor}

\definecolor{keywordcolor}{rgb}{0.7, 0.1, 0.1}   % red
\definecolor{tacticcolor}{rgb}{0.0, 0.1, 0.6}    % blue
\definecolor{commentcolor}{rgb}{0.4, 0.4, 0.4}   % grey
\definecolor{symbolcolor}{rgb}{0.0, 0.1, 0.6}    % blue
\definecolor{sortcolor}{rgb}{0.1, 0.5, 0.1}      % green
\definecolor{attributecolor}{rgb}{0.7, 0.1, 0.1} % red
\definecolor{white}{rgb}{1.0,1.0,1.0}

\usepackage{xspace}
\usepackage{nicefrac}
\usepackage{caption}
\captionsetup[table]{name=Listing}
\usepackage{makecell}

\usepackage{tikz}

% colors for links
\hypersetup{
    colorlinks=true,  
    urlcolor=blue,
}

\usepackage[final,nomargin,inline,index]{fixme} % Simplified management of FIXME's
\fxusetheme{color}
\FXRegisterAuthor{bart}{anbart}{\color{magenta} {\underline{bart}}}
\FXRegisterAuthor{enrico}{anenrico}{\color{red} {\underline{enrico}}}
\FXRegisterAuthor{silvia}{ansilvia}{\color{blue} {\underline{silvia}}}

\crefname{table}{Listing}{Listings}  

% \renewcommand{\ttdefault}{lmtt}  % to use bold and italic within code

\definecolor{LightGrey}{rgb}{0.975,0.975,0.975}
% \definecolor{keyword}{HTML}{7F0055}

% Packages initialization
%
% Packages initialization

\lstset{
	nolol=true,
	breaklines=true,
	xleftmargin=0pt,
	xrightmargin=0pt,
	framexleftmargin=0pt,
	framextopmargin=0pt,
        % frame=ltbr, 
	framerule=0pt,        
	% framexrightmargin=10pt,
	framexbottommargin=2pt, 
	% frame=ltbr, 
        % framerule=0pt,
	showstringspaces=false,
	%basicstyle=%
        %\ttfamily\linespread{1.15}\scriptsize%
        %\lst@ifdisplaystyle\footnotesize\fi,
        % basicstyle=\fontseries{m}\scriptsize\ttfamily,
	backgroundcolor=\color{LightGrey},
        numberstyle=\fontsize{4}{4}\color{MidnightBlue}\ttfamily,
        stringstyle=\color{RoyalBlue}
}

\lstdefinelanguage{solidity}{
, basicstyle=\ttfamily\linespread{1.15}\scriptsize\lst@ifdisplaystyle\scriptsize\fi
, commentstyle=\color{Gray}
, morecomment=[l]{//}
, morecomment=[s]{/*}{*/}
, escapechar=\$
% commands => NavyBlue
, classoffset=0,
, keywordstyle=\color{NavyBlue}\bfseries
, morekeywords={assert,require,if,then,else,for,break,call,delegatecall,transfer,send,case, catch,continue,do,while,emit, new, return, revert, selfdestruct, try, with, throw, switch, suicide}
% declarations context => BlueGreen
, classoffset=1
, keywordstyle=\color{YellowGreen}\bfseries
, morekeywords={external, implements, import, interface, internal, library, payable, pragma, private, protected, public, pure, returns, super, using, view}
% types => blue
, classoffset=2
, keywordstyle=\color{blue}
, morekeywords={function, constructor, contract, constant, struct, address, bool, byte, bytes, bytes1, bytes2, bytes3, bytes4, bytes5, bytes6, bytes7, bytes8, bytes9, bytes10, bytes11, bytes12, bytes13, bytes14, bytes15, bytes16, bytes17, bytes18, bytes19, bytes20, bytes21, bytes22, bytes23, bytes24, bytes25, bytes26, bytes27, bytes28, bytes29, bytes30, bytes31, bytes32, enum, int, int8, int16, int24, int32, int40, int48, int56, int64, int72, int80, int88, int96, int104, int112, int120, int128, int136, int144, int152, int160, int168, int176, int184, int192, int200, int208, int216, int224, int232, int240, int248, int256, mapping, string, uint, uint8, uint16, uint24, uint32, uint40, uint48, uint56, uint64, uint72, uint80, uint88, uint96, uint104, uint112, uint120, uint128, uint136, uint144, uint152, uint160, uint168, uint176, uint184, uint192, uint200, uint208, uint216, uint224, uint232, uint240, uint248, uint256, var, void, ether, finney, szabo, wei, days, hours, minutes, seconds, weeks, years}
% constants and fields => Plum
, classoffset=3
, keywordstyle=\color{Plum}\bfseries
, morekeywords={balance, block, blockhash, instanceof, coinbase, difficulty, gaslimit, number, timestamp, msg, data, gas, sender, value, sig, value, now, tx, gasprice, origin}
}

\lstdefinelanguage{move}{%
, basicstyle=\ttfamily\linespread{1.15}\scriptsize\lst@ifdisplaystyle\scriptsize\fi
% , sensitive
, commentstyle=\color{Gray},
, morecomment=[l]{//}%
, morecomment=[s]{/*}{*/}%
, moredelim=[s][{\itshape\color[rgb]{0,0,0.75}}]{\#[}{]}%
% , morestring=[b]{"}%
% , alsodigit={}%
% , alsoother={}%
% , alsoletter={!}%
, escapechar=\$
%
% commands => NavyBlue
, classoffset=0,
, keywordstyle=\color{NavyBlue}\bfseries
, morekeywords={assert!, move_to, move_from, borrow_global, borrow_mut, borrow_global_mut, exists, break, continue, else, for, if, in, loop, match, return, while}  
% declarations context => BlueGreen
, classoffset=1
, keywordstyle=\color{YellowGreen}\bfseries
, morekeywords={as, const, let, move, mut, ref, static}  % in the context of variables
% in the context of declarations
, morekeywords={module, public, external, acquires}  % in the context of modularisation
%
, keywordstyle=\color{blue}\bfseries
, classoffset=2
, morekeywords={spec, enum, fun, impl, Self, self, struct, trait, type, union, use, where}
, morekeywords={signer, address, bool, char, f32, f64, i8, i16, i32, i64, isize, str, u8, u16, u32, u64, unit, usize, i128, u128}  % primitive types
%
% constants and fields => Plum
, classoffset=3
, keywordstyle=\color{Plum}\bfseries
, morekeywords={address_of, Err, false, None, Ok, Some, true},
, classoffset=4
, keywordstyle=\color{red}\bfseries
, morekeywords={spec,ensures,post,global,spec_get}
}

\lstdefinelanguage{cvl}{
, basicstyle=\ttfamily\linespread{1.15}\scriptsize\lst@ifdisplaystyle\scriptsize\fi
, commentstyle=\color{Gray}
, morecomment=[l]{//}
, morecomment=[s]{/*}{*/}
, classoffset=0
, escapechar=\$
, morekeywords={anonymous, assembly, balance, break, call, callcode, case, catch, class, constant, continue, constructor, contract, debugger, default, delegatecall, delete, do, else, emit, event, experimental, export, external, false, finally, for, function, gas, if, implements, import, in, indexed, instanceof, interface, internal, is, length, library, log0, log1, log2, log3, log4, memory, modifier, new, payable, pragma, private, protected, public, pure, push, return, returns, revert,  selfdestruct, send, solidity, storage, struct, suicide, super, switch, then, this, throw, transfer, true, try, typeof, using, value, view, while, with, addmod, ecrecover, keccak256, mulmod, ripemd160, sha256, sha3}
, keywordstyle=\color{NavyBlue}\bfseries
, classoffset=1
, morekeywords={address, bool, byte, bytes, bytes1, bytes2, bytes3, bytes4, bytes5, bytes6, bytes7, bytes8, bytes9, bytes10, bytes11, bytes12, bytes13, bytes14, bytes15, bytes16, bytes17, bytes18, bytes19, bytes20, bytes21, bytes22, bytes23, bytes24, bytes25, bytes26, bytes27, bytes28, bytes29, bytes30, bytes31, bytes32, enum, int, int8, int16, int24, int32, int40, int48, int56, int64, int72, int80, int88, int96, int104, int112, int120, int128, int136, int144, int152, int160, int168, int176, int184, int192, int200, int208, int216, int224, int232, int240, int248, int256, mapping, string, uint, uint8, uint16, uint24, uint32, uint40, uint48, uint56, uint64, uint72, uint80, uint88, uint96, uint104, uint112, uint120, uint128, uint136, uint144, uint152, uint160, uint168, uint176, uint184, uint192, uint200, uint208, uint216, uint224, uint232, uint240, uint248, uint256, var, void, ether, finney, szabo, wei, days, hours, minutes, seconds, weeks, years}
, keywordstyle=\color{blue}
, classoffset=2
, morekeywords={block, blockhash, coinbase, difficulty, gaslimit, number, timestamp, msg, data, gas, sender, sig, value, now, tx, gasprice, origin}
, keywordstyle=\color{Plum}\bfseries
, classoffset=3
, morekeywords={invariant,rule,assert,satisfy,require}
, keywordstyle=\color{red}\bfseries,
}

\lstset{
	language=solidity,
}

% correct the font size of lstinline 
\makeatletter
\lstdefinestyle{mystyle}{
  basicstyle=%
    \ttfamily
    \lst@ifdisplaystyle\footnotesize\fi
}
\makeatother
\lstset{style=mystyle}

%\newcommand{\urlSheets}{\url{https://tinyurl.com/3ndkhj57}}

\newcommand{\githubaddress}{https://github.com/blockchain-unica/solidity-vs-move-verification}
\newcommand{\githuburl}{\url{\githubaddress}}
\newcommand{\sheeturl}{https://docs.google.com/spreadsheets/d/1oy43kISC2iM63XfksfonRBzjXPkpLoDglz-m24nazvc/edit?usp=sharing}

\newcommand{\specurl}[2]{\href{\githubaddress/blob/main/contracts/#1/README.md\#{#2}}{\textcolor{blue}{\texttt{#1}/\texttt{#2}}}}

\newcommand{\contracturl}[1]{\href{\githubaddress/blob/main/contracts/#1/README.md}{\textcolor{blue}{\texttt{#1}}}}

% TODO: fix command
% \newcommand{\solidityurl}[1]{\href{\githubaddress/blob/main/contracts/#1/certora/#1.sol}{\texttt{#1}}}


% Statistics on the dataset
\newcounter{nBankCounter}
\setcounter{nBankCounter}{28}

\newcounter{nVaultCounter}
\setcounter{nVaultCounter}{21}

\newcounter{nPricebetCounter}
\setcounter{nPricebetCounter}{17}

\newcounter{nTotCounter}
\setcounter{nTotCounter}{\numexpr\value{nBankCounter}+\value{nVaultCounter}+\value{nPricebetCounter}\relax}

\newcommand{\nBankProperties}{\arabic{nBankCounter}\xspace}
\newcommand{\nVaultProperties}{\arabic{nVaultCounter}\xspace}
\newcommand{\nPricebetProperties}{\arabic{nPricebetCounter}\xspace}
\newcommand{\nTotProperties}{\arabic{nTotCounter}\xspace}

\begin{document}

\maketitle

\begin{abstract}  
Test time scaling is currently one of the most active research areas that shows promise after training time scaling has reached its limits.
Deep-thinking (DT) models are a class of recurrent models that can perform easy-to-hard generalization by assigning more compute to harder test samples.
However, due to their inability to determine the complexity of a test sample, DT models have to use a large amount of computation for both easy and hard test samples.
Excessive test time computation is wasteful and can cause the ``overthinking'' problem where more test time computation leads to worse results.
In this paper, we introduce a test time training method for determining the optimal amount of computation needed for each sample during test time.
We also propose Conv-LiGRU, a novel recurrent architecture for efficient and robust visual reasoning. 
Extensive experiments demonstrate that Conv-LiGRU is more stable than DT, effectively mitigates the ``overthinking'' phenomenon, and achieves superior accuracy.
\end{abstract}  

%\enriconote{Nel titolo togliere "insights from"? O "a comparative analysis-based study"?}

\section{Introduction}


\begin{figure}[t]
\centering
\includegraphics[width=0.6\columnwidth]{figures/evaluation_desiderata_V5.pdf}
\vspace{-0.5cm}
\caption{\systemName is a platform for conducting realistic evaluations of code LLMs, collecting human preferences of coding models with real users, real tasks, and in realistic environments, aimed at addressing the limitations of existing evaluations.
}
\label{fig:motivation}
\end{figure}

\begin{figure*}[t]
\centering
\includegraphics[width=\textwidth]{figures/system_design_v2.png}
\caption{We introduce \systemName, a VSCode extension to collect human preferences of code directly in a developer's IDE. \systemName enables developers to use code completions from various models. The system comprises a) the interface in the user's IDE which presents paired completions to users (left), b) a sampling strategy that picks model pairs to reduce latency (right, top), and c) a prompting scheme that allows diverse LLMs to perform code completions with high fidelity.
Users can select between the top completion (green box) using \texttt{tab} or the bottom completion (blue box) using \texttt{shift+tab}.}
\label{fig:overview}
\end{figure*}

As model capabilities improve, large language models (LLMs) are increasingly integrated into user environments and workflows.
For example, software developers code with AI in integrated developer environments (IDEs)~\citep{peng2023impact}, doctors rely on notes generated through ambient listening~\citep{oberst2024science}, and lawyers consider case evidence identified by electronic discovery systems~\citep{yang2024beyond}.
Increasing deployment of models in productivity tools demands evaluation that more closely reflects real-world circumstances~\citep{hutchinson2022evaluation, saxon2024benchmarks, kapoor2024ai}.
While newer benchmarks and live platforms incorporate human feedback to capture real-world usage, they almost exclusively focus on evaluating LLMs in chat conversations~\citep{zheng2023judging,dubois2023alpacafarm,chiang2024chatbot, kirk2024the}.
Model evaluation must move beyond chat-based interactions and into specialized user environments.



 

In this work, we focus on evaluating LLM-based coding assistants. 
Despite the popularity of these tools---millions of developers use Github Copilot~\citep{Copilot}---existing
evaluations of the coding capabilities of new models exhibit multiple limitations (Figure~\ref{fig:motivation}, bottom).
Traditional ML benchmarks evaluate LLM capabilities by measuring how well a model can complete static, interview-style coding tasks~\citep{chen2021evaluating,austin2021program,jain2024livecodebench, white2024livebench} and lack \emph{real users}. 
User studies recruit real users to evaluate the effectiveness of LLMs as coding assistants, but are often limited to simple programming tasks as opposed to \emph{real tasks}~\citep{vaithilingam2022expectation,ross2023programmer, mozannar2024realhumaneval}.
Recent efforts to collect human feedback such as Chatbot Arena~\citep{chiang2024chatbot} are still removed from a \emph{realistic environment}, resulting in users and data that deviate from typical software development processes.
We introduce \systemName to address these limitations (Figure~\ref{fig:motivation}, top), and we describe our three main contributions below.


\textbf{We deploy \systemName in-the-wild to collect human preferences on code.} 
\systemName is a Visual Studio Code extension, collecting preferences directly in a developer's IDE within their actual workflow (Figure~\ref{fig:overview}).
\systemName provides developers with code completions, akin to the type of support provided by Github Copilot~\citep{Copilot}. 
Over the past 3 months, \systemName has served over~\completions suggestions from 10 state-of-the-art LLMs, 
gathering \sampleCount~votes from \userCount~users.
To collect user preferences,
\systemName presents a novel interface that shows users paired code completions from two different LLMs, which are determined based on a sampling strategy that aims to 
mitigate latency while preserving coverage across model comparisons.
Additionally, we devise a prompting scheme that allows a diverse set of models to perform code completions with high fidelity.
See Section~\ref{sec:system} and Section~\ref{sec:deployment} for details about system design and deployment respectively.



\textbf{We construct a leaderboard of user preferences and find notable differences from existing static benchmarks and human preference leaderboards.}
In general, we observe that smaller models seem to overperform in static benchmarks compared to our leaderboard, while performance among larger models is mixed (Section~\ref{sec:leaderboard_calculation}).
We attribute these differences to the fact that \systemName is exposed to users and tasks that differ drastically from code evaluations in the past. 
Our data spans 103 programming languages and 24 natural languages as well as a variety of real-world applications and code structures, while static benchmarks tend to focus on a specific programming and natural language and task (e.g. coding competition problems).
Additionally, while all of \systemName interactions contain code contexts and the majority involve infilling tasks, a much smaller fraction of Chatbot Arena's coding tasks contain code context, with infilling tasks appearing even more rarely. 
We analyze our data in depth in Section~\ref{subsec:comparison}.



\textbf{We derive new insights into user preferences of code by analyzing \systemName's diverse and distinct data distribution.}
We compare user preferences across different stratifications of input data (e.g., common versus rare languages) and observe which affect observed preferences most (Section~\ref{sec:analysis}).
For example, while user preferences stay relatively consistent across various programming languages, they differ drastically between different task categories (e.g. frontend/backend versus algorithm design).
We also observe variations in user preference due to different features related to code structure 
(e.g., context length and completion patterns).
We open-source \systemName and release a curated subset of code contexts.
Altogether, our results highlight the necessity of model evaluation in realistic and domain-specific settings.





%\enriconote{da qualche parte specificare che con Move ci riferiremo specificamente a Aptos Move}
\section{Background}\label{sec:backgrnd}

\subsection{Cold Start Latency and Mitigation Techniques}

Traditional FaaS platforms mitigate cold starts through snapshotting, lightweight virtualization, and warm-state management. Snapshot-based methods like \textbf{REAP} and \textbf{Catalyzer} reduce initialization time by preloading or restoring container states but require significant memory and I/O resources, limiting scalability~\cite{dong_catalyzer_2020, ustiugov_benchmarking_2021}. Lightweight virtualization solutions, such as \textbf{Firecracker} microVMs, achieve fast startup times with strong isolation but depend on robust infrastructure, making them less adaptable to fluctuating workloads~\cite{agache_firecracker_2020}. Warm-state management techniques like \textbf{Faa\$T}~\cite{romero_faa_2021} and \textbf{Kraken}~\cite{vivek_kraken_2021} keep frequently invoked containers ready, balancing readiness and cost efficiency under predictable workloads but incurring overhead when demand is erratic~\cite{romero_faa_2021, vivek_kraken_2021}. While these methods perform well in resource-rich cloud environments, their resource intensity challenges applicability in edge settings.

\subsubsection{Edge FaaS Perspective}

In edge environments, cold start mitigation emphasizes lightweight designs, resource sharing, and hybrid task distribution. Lightweight execution environments like unikernels~\cite{edward_sock_2018} and \textbf{Firecracker}~\cite{agache_firecracker_2020}, as used by \textbf{TinyFaaS}~\cite{pfandzelter_tinyfaas_2020}, minimize resource usage and initialization delays but require careful orchestration to avoid resource contention. Function co-location, demonstrated by \textbf{Photons}~\cite{v_dukic_photons_2020}, reduces redundant initializations by sharing runtime resources among related functions, though this complicates isolation in multi-tenant setups~\cite{v_dukic_photons_2020}. Hybrid offloading frameworks like \textbf{GeoFaaS}~\cite{malekabbasi_geofaas_2024} balance edge-cloud workloads by offloading latency-tolerant tasks to the cloud and reserving edge resources for real-time operations, requiring reliable connectivity and efficient task management. These edge-specific strategies address cold starts effectively but introduce challenges in scalability and orchestration.

\subsection{Predictive Scaling and Caching Techniques}

Efficient resource allocation is vital for maintaining low latency and high availability in serverless platforms. Predictive scaling and caching techniques dynamically provision resources and reduce cold start latency by leveraging workload prediction and state retention.
Traditional FaaS platforms use predictive scaling and caching to optimize resources, employing techniques (OFC, FaasCache) to reduce cold starts. However, these methods rely on centralized orchestration and workload predictability, limiting their effectiveness in dynamic, resource-constrained edge environments.



\subsubsection{Edge FaaS Perspective}

Edge FaaS platforms adapt predictive scaling and caching techniques to constrain resources and heterogeneous environments. \textbf{EDGE-Cache}~\cite{kim_delay-aware_2022} uses traffic profiling to selectively retain high-priority functions, reducing memory overhead while maintaining readiness for frequent requests. Hybrid frameworks like \textbf{GeoFaaS}~\cite{malekabbasi_geofaas_2024} implement distributed caching to balance resources between edge and cloud nodes, enabling low-latency processing for critical tasks while offloading less critical workloads. Machine learning methods, such as clustering-based workload predictors~\cite{gao_machine_2020} and GRU-based models~\cite{guo_applying_2018}, enhance resource provisioning in edge systems by efficiently forecasting workload spikes. These innovations effectively address cold start challenges in edge environments, though their dependency on accurate predictions and robust orchestration poses scalability challenges.

\subsection{Decentralized Orchestration, Function Placement, and Scheduling}

Efficient orchestration in serverless platforms involves workload distribution, resource optimization, and performance assurance. While traditional FaaS platforms rely on centralized control, edge environments require decentralized and adaptive strategies to address unique challenges such as resource constraints and heterogeneous hardware.



\subsubsection{Edge FaaS Perspective}

Edge FaaS platforms adopt decentralized and adaptive orchestration frameworks to meet the demands of resource-constrained environments. Systems like \textbf{Wukong} distribute scheduling across edge nodes, enhancing data locality and scalability while reducing network latency. Lightweight frameworks such as \textbf{OpenWhisk Lite}~\cite{kravchenko_kpavelopenwhisk-light_2024} optimize resource allocation by decentralizing scheduling policies, minimizing cold starts and latency in edge setups~\cite{benjamin_wukong_2020}. Hybrid solutions like \textbf{OpenFaaS}~\cite{noauthor_openfaasfaas_2024} and \textbf{EdgeMatrix}~\cite{shen_edgematrix_2023} combine edge-cloud orchestration to balance resource utilization, retaining latency-sensitive functions at the edge while offloading non-critical workloads to the cloud. While these approaches improve flexibility, they face challenges in maintaining coordination and ensuring consistent performance across distributed nodes.


\section{Research Methodology}~\label{sec:Methodology}

In this section, we discuss the process of conducting our systematic review, e.g., our search strategy for data extraction of relevant studies, based on the guidelines of Kitchenham et al.~\cite{kitchenham2022segress} to conduct SLRs and Petersen et al.~\cite{PETERSEN20151} to conduct systematic mapping studies (SMSs) in Software Engineering. In this systematic review, we divide our work into a four-stage procedure, including planning, conducting, building a taxonomy, and reporting the review, illustrated in Fig.~\ref{fig:search}. The four stages are as follows: (1) the \emph{planning} stage involved identifying research questions (RQs) and specifying the detailed research plan for the study; (2) the \emph{conducting} stage involved analyzing and synthesizing the existing primary studies to answer the research questions; (3) the \emph{taxonomy} stage was introduced to optimize the data extraction results and consolidate a taxonomy schema for REDAST methodology; (4) the \emph{reporting} stage involved the reviewing, concluding and reporting the final result of our study.

\begin{figure}[!t]
    \centering
    \includegraphics[width=1\linewidth]{fig/methodology/searching-process.drawio.pdf}
    \caption{Systematic Literature Review Process}
    \label{fig:search}
\end{figure}

\subsection{Research Questions}
In this study, we developed five research questions (RQs) to identify the input and output, analyze technologies, evaluate metrics, identify challenges, and identify potential opportunities. 

\textbf{RQ1. What are the input configurations, formats, and notations used in the requirements in requirements-driven
automated software testing?} In requirements-driven testing, the input is some form of requirements specification -- which can vary significantly. RQ1 maps the input for REDAST and reports on the comparison among different formats for requirements specification.

\textbf{RQ2. What are the frameworks, tools, processing methods, and transformation techniques used in requirements-driven automated software testing studies?} RQ2 explores the technical solutions from requirements to generated artifacts, e.g., rule-based transformation applying natural language processing (NLP) pipelines and deep learning (DL) techniques, where we additionally discuss the potential intermediate representation and additional input for the transformation process.

\textbf{RQ3. What are the test formats and coverage criteria used in the requirements-driven automated software
testing process?} RQ3 focuses on identifying the formulation of generated artifacts (i.e., the final output). We map the adopted test formats and analyze their characteristics in the REDAST process.

\textbf{RQ4. How do existing studies evaluate the generated test artifacts in the requirements-driven automated software testing process?} RQ4 identifies the evaluation datasets, metrics, and case study methodologies in the selected papers. This aims to understand how researchers assess the effectiveness, accuracy, and practical applicability of the generated test artifacts.

\textbf{RQ5. What are the limitations and challenges of existing requirements-driven automated software testing methods in the current era?} RQ5 addresses the limitations and challenges of existing studies while exploring future directions in the current era of technology development. %It particularly highlights the potential benefits of advanced LLMs and examines their capacity to meet the high expectations placed on these cutting-edge language modeling technologies. %\textcolor{blue}{CA: Do we really need to focus on LLMs? TBD.} \textcolor{orange}{FW: About LLMs, I removed the direct emphase in RQ5 but kept the discussion in RQ5 and the solution section. I think that would be more appropriate.}

\subsection{Searching Strategy}

The overview of the search process is exhibited in Fig. \ref{fig:papers}, which includes all the details of our search steps.
\begin{table}[!ht]
\caption{List of Search Terms}
\label{table:search_term}
\begin{tabularx}{\textwidth}{lX}
\hline
\textbf{Terms Group} & \textbf{Terms} \\ \hline
Test Group & test* \\
Requirement Group & requirement* OR use case* OR user stor* OR specification* \\
Software Group & software* OR system* \\
Method Group & generat* OR deriv* OR map* OR creat* OR extract* OR design* OR priorit* OR construct* OR transform* \\ \hline
\end{tabularx}
\end{table}

\begin{figure}
    \centering
    \includegraphics[width=1\linewidth]{fig/methodology/search-papers.drawio.pdf}
    \caption{Study Search Process}
    \label{fig:papers}
\end{figure}

\subsubsection{Search String Formulation}
Our research questions (RQs) guided the identification of the main search terms. We designed our search string with generic keywords to avoid missing out on any related papers, where four groups of search terms are included, namely ``test group'', ``requirement group'', ``software group'', and ``method group''. In order to capture all the expressions of the search terms, we use wildcards to match the appendix of the word, e.g., ``test*'' can capture ``testing'', ``tests'' and so on. The search terms are listed in Table~\ref{table:search_term}, decided after iterative discussion and refinement among all the authors. As a result, we finally formed the search string as follows:


\hangindent=1.5em
 \textbf{ON ABSTRACT} ((``test*'') \textbf{AND} (``requirement*'' \textbf{OR} ``use case*'' \textbf{OR} ``user stor*'' \textbf{OR} ``specifications'') \textbf{AND} (``software*'' \textbf{OR} ``system*'') \textbf{AND} (``generat*'' \textbf{OR} ``deriv*'' \textbf{OR} ``map*'' \textbf{OR} ``creat*'' \textbf{OR} ``extract*'' \textbf{OR} ``design*'' \textbf{OR} ``priorit*'' \textbf{OR} ``construct*'' \textbf{OR} ``transform*''))

The search process was conducted in September 2024, and therefore, the search results reflect studies available up to that date. We conducted the search process on six online databases: IEEE Xplore, ACM Digital Library, Wiley, Scopus, Web of Science, and Science Direct. However, some databases were incompatible with our default search string in the following situations: (1) unsupported for searching within abstract, such as Scopus, and (2) limited search terms, such as ScienceDirect. Here, for (1) situation, we searched within the title, keyword, and abstract, and for (2) situation, we separately executed the search and removed the duplicate papers in the merging process. 

\subsubsection{Automated Searching and Duplicate Removal}
We used advanced search to execute our search string within our selected databases, following our designed selection criteria in Table \ref{table:selection}. The first search returned 27,333 papers. Specifically for the duplicate removal, we used a Python script to remove (1) overlapped search results among multiple databases and (2) conference or workshop papers, also found with the same title and authors in the other journals. After duplicate removal, we obtained 21,652 papers for further filtering.

\begin{table*}[]
\caption{Selection Criteria}
\label{table:selection}
\begin{tabularx}{\textwidth}{lX}
\hline
\textbf{Criterion ID} & \textbf{Criterion Description} \\ \hline
S01          & Papers written in English. \\
S02-1        & Papers in the subjects of "Computer Science" or "Software Engineering". \\
S02-2        & Papers published on software testing-related issues. \\
S03          & Papers published from 1991 to the present. \\ 
S04          & Papers with accessible full text. \\ \hline
\end{tabularx}
\end{table*}

\begin{table*}[]
\small
\caption{Inclusion and Exclusion Criteria}
\label{table:criteria}
\begin{tabularx}{\textwidth}{lX}
\hline
\textbf{ID}  & \textbf{Description} \\ \hline
\multicolumn{2}{l}{\textbf{Inclusion Criteria}} \\ \hline
I01 & Papers about requirements-driven automated system testing or acceptance testing generation, or studies that generate system-testing-related artifacts. \\
I02 & Peer-reviewed studies that have been used in academia with references from literature. \\ \hline
\multicolumn{2}{l}{\textbf{Exclusion Criteria}} \\ \hline
E01 & Studies that only support automated code generation, but not test-artifact generation. \\
E02 & Studies that do not use requirements-related information as an input. \\
E03 & Papers with fewer than 5 pages (1-4 pages). \\
E04 & Non-primary studies (secondary or tertiary studies). \\
E05 & Vision papers and grey literature (unpublished work), books (chapters), posters, discussions, opinions, keynotes, magazine articles, experience, and comparison papers. \\ \hline
\end{tabularx}
\end{table*}

\subsubsection{Filtering Process}

In this step, we filtered a total of 21,652 papers using the inclusion and exclusion criteria outlined in Table \ref{table:criteria}. This process was primarily carried out by the first and second authors. Our criteria are structured at different levels, facilitating a multi-step filtering process. This approach involves applying various criteria in three distinct phases. We employed a cross-verification method involving (1) the first and second authors and (2) the other authors. Initially, the filtering was conducted separately by the first and second authors. After cross-verifying their results, the results were then reviewed and discussed further by the other authors for final decision-making. We widely adopted this verification strategy within the filtering stages. During the filtering process, we managed our paper list using a BibTeX file and categorized the papers with color-coding through BibTeX management software\footnote{\url{https://bibdesk.sourceforge.io/}}, i.e., “red” for irrelevant papers, “yellow” for potentially relevant papers, and “blue” for relevant papers. This color-coding system facilitated the organization and review of papers according to their relevance.

The screening process is shown below,
\begin{itemize}
    \item \textbf{1st-round Filtering} was based on the title and abstract, using the criteria I01 and E01. At this stage, the number of papers was reduced from 21,652 to 9,071.
    \item \textbf{2nd-round Filtering}. We attempted to include requirements-related papers based on E02 on the title and abstract level, which resulted from 9,071 to 4,071 papers. We excluded all the papers that did not focus on requirements-related information as an input or only mentioned the term ``requirements'' but did not refer to the requirements specification.
    \item \textbf{3rd-round Filtering}. We selectively reviewed the content of papers identified as potentially relevant to requirements-driven automated test generation. This process resulted in 162 papers for further analysis.
\end{itemize}
Note that, especially for third-round filtering, we aimed to include as many relevant papers as possible, even borderline cases, according to our criteria. The results were then discussed iteratively among all the authors to reach a consensus.

\subsubsection{Snowballing}

Snowballing is necessary for identifying papers that may have been missed during the automated search. Following the guidelines by Wohlin~\cite{wohlin2014guidelines}, we conducted both forward and backward snowballing. As a result, we identified 24 additional papers through this process.

\subsubsection{Data Extraction}

Based on the formulated research questions (RQs), we designed 38 data extraction questions\footnote{\url{https://drive.google.com/file/d/1yjy-59Juu9L3WHaOPu-XQo-j-HHGTbx_/view?usp=sharing}} and created a Google Form to collect the required information from the relevant papers. The questions included 30 short-answer questions, six checkbox questions, and two selection questions. The data extraction was organized into five sections: (1) basic information: fundamental details such as title, author, venue, etc.; (2) open information: insights on motivation, limitations, challenges, etc.; (3) requirements: requirements format, notation, and related aspects; (4) methodology: details, including immediate representation and technique support; (5) test-related information: test format(s), coverage, and related elements. Similar to the filtering process, the first and second authors conducted the data extraction and then forwarded the results to the other authors to initiate the review meeting.

\subsubsection{Quality Assessment}

During the data extraction process, we encountered papers with insufficient information. To address this, we conducted a quality assessment in parallel to ensure the relevance of the papers to our objectives. This approach, also adopted in previous secondary studies~\cite{shamsujjoha2021developing, naveed2024model}, involved designing a set of assessment questions based on guidelines by Kitchenham et al.~\cite{kitchenham2022segress}. The quality assessment questions in our study are shown below:
\begin{itemize}
    \item \textbf{QA1}. Does this study clearly state \emph{how} requirements drive automated test generation?
    \item \textbf{QA2}. Does this study clearly state the \emph{aim} of REDAST?
    \item \textbf{QA3}. Does this study enable \emph{automation} in test generation?
    \item \textbf{QA4}. Does this study demonstrate the usability of the method from the perspective of methodology explanation, discussion, case examples, and experiments?
\end{itemize}
QA4 originates from an open perspective in the review process, where we focused on evaluation, discussion, and explanation. Our review also examined the study’s overall structure, including the methodology description, case studies, experiments, and analyses. The detailed results of the quality assessment are provided in the Appendix. Following this assessment, the final data extraction was based on 156 papers.

% \begin{table}[]
% \begin{tabular}{ll}
% \hline
% QA ID & QA Questions                                             \\ \hline
% Q01   & Does this study clearly state its aims?                  \\
% Q02   & Does this study clearly describe its methodology?        \\
% Q03   & Does this study involve automated test generation?       \\
% Q04   & Does this study include a promising evaluation?          \\
% Q05   & Does this study demonstrate the usability of the method? \\ \hline
% \end{tabular}%
% \caption{Questions for Quality Assessment}
% \label{table:qa}
% \end{table}

% automated quality assessment

% \textcolor{blue}{CA: Our search strategy focused on identifying requirements types first. We covered several sources, e.g., ~\cite{Pohl:11,wagner2019status} to identify different formats and notations of specifying requirements. However, this came out to be a long list, e.g., free-form NL requirements, semi-formal UML models, free-from textual use case models, UML class diagrams, UML activity diagrams, and so on. In this paper, we attempted to primarily focus on requirements-related aspects and not design-level information. Hence, we generalised our search string to include generic keywords, e.g., requirement*, use case*, and user stor*. We did so to avoid missing out on any papers, bringing too restrictive in our search strategy, and not creating a too-generic search string with all the aforementioned formats to avoid getting results beyond our review's scope.}


%% Use \subsection commands to start a subsection.



%\subsection{Study Selection}

% In this step, we further looked into the content of searched papers using our search strategy and applied our inclusion and exclusion criteria. Our filtering strategy aimed to pinpoint studies focused on requirements-driven system-level testing. Recognizing the presence of irrelevant papers in our search results, we established detailed selection criteria for preliminary inclusion and exclusion, as shown in Table \ref{table: criteria}. Specifically, we further developed the taxonomy schema to exclude two types of studies that did not meet the requirements for system-level testing: (1) studies supporting specification-driven test generation, such as UML-driven test generation, rather than requirements-driven testing, and (2) studies focusing on code-based test generation, such as requirement-driven code generation for unit testing.





% \begin{wraptable}{r}{0.65\textwidth}
% \centering
% \caption{The comparison of resource requirements between Eurus-2-7B-PRIME and Qwen2.5-Math-7B-Instruct.}
% \label{tab:comparision}
% \resizebox{0.65\textwidth}{!}{
% \begin{tabular}{l >{\columncolor[HTML]{D7E8E8}}l l}
% \toprule
% \textbf{Model} & \textbf{Eurus-2-7B-PRIME} & \textbf{Qwen2.5-Math-7B-Instruct} \\ \midrule
% Base Model     & Qwen2.5-Math-7B           & Qwen2.5-Math-7B                  \\
% SFT Data       & 230K (open-source)        & 2.5M (open-source and in-house)  \\
% RM Data        & 0                         & 618K (in-house)                 \\
% RM             & Eurus-2-7B-SFT            & Qwen2.5-Math-RM (72B)           \\
% RL Data        & 150K queries $\times$ 4 samples & 66K queries $\times$ 32 samples \\ \bottomrule
% \end{tabular}
% }
% \end{wraptable}

\begin{wraptable}{r}{0.65\textwidth}
\centering
\caption{The comparison of resource requirements between Eurus-2-7B-PRIME and Qwen2.5-Math-7B-Instruct.}
\label{tab:comparision}
% \resizebox{0.65\textwidth}{!}{
\resizebox{\linewidth}{!}{
\begin{tabular}{l >{\columncolor[HTML]{D7E8E8}}l l}
\toprule
\textbf{Model} & \textbf{Eurus-2-7B-PRIME} & \textbf{Qwen2.5-Math-7B-Instruct} \\ \midrule
Base Model     & Qwen2.5-Math-7B           & Qwen2.5-Math-7B                  \\
SFT Data       & 230K (open-source)        & 2.5M (open-source \& in-house)  \\
RM Data        & 0                         & 618K (in-house)                 \\
RM             & Eurus-2-7B-SFT            & Qwen2.5-Math-RM (72B)           \\
RL Data        & 150K queries $\times$ 4 samples & 66K queries $\times$ 32 samples \\ \bottomrule
\end{tabular}
}
\end{wraptable}





% 字体标橙色
% \begin{wraptable}{r}{0.65\textwidth}  % r表示表格在右侧,0.5\textwidth表示表格宽度为文本宽度的50%
% \centering
% \caption{The comparison of resource requirements between Eurus-2-7B-PRIME and Qwen2.5-Math-7B-Instruct.}
% \label{tab:comparision}
% \resizebox{0.65\textwidth}{!}{
% \begin{tabular}{lll}
% \toprule
% \textbf{Model} & {\color[HTML]{F8A102}\textbf{Eurus-2-7B-PRIME}}                             & \textbf{Qwen2.5-Math-7B-Instruct}            \\ \midrule
% Base Model     & {\color[HTML]{F8A102}Qwen2.5-Math-7B}                                       & Qwen2.5-Math-7B                              \\
% SFT Data       & {\color[HTML]{F8A102}\textbf{230K (open-source)}}                           & 2.5M (open-source and in-house)              \\
% RM Data        & {\color[HTML]{F8A102}\textbf{0}}                                            & 618K (in-house)                              \\
% RM             & {\color[HTML]{F8A102}\textbf{Eurus-2-7B-SFT}}                               & Qwen2.5-Math-RM (72B)                        \\
% RL Data        & {\color[HTML]{F8A102}\textbf{150K queries $\times$ 4 samples}} & 66K queries $\times$ 32 samples \\ \bottomrule
% \end{tabular}

% }
% \end{wraptable}




% \begin{table}[]
% \centering
% \caption{The comparison of resource requirements between Eurus-2-7B-PRIME and Qwen2.5-Math-7B-Instruct.\hanbin{Embed into text}}
% \label{tab:comparision}
% \resizebox{0.8\textwidth}{!}{
% \begin{tabular}{lll}
% \midrule
% \textbf{Model} & \textbf{Eurus-2-7B-PRIME}                             & \textbf{Qwen2.5-Math-7B-Instruct}            \\ \midrule
% Base Model     & Qwen2.5-Math-7B                                       & Qwen2.5-Math-7B                              \\
% SFT Data       & \textbf{230K (open-source)}                           & 2.5M (open-source and in-house)              \\
% RM Data        & \textbf{0}                                            & 618K (in-house)                              \\
% RM             & \textbf{Eurus-2-7B-SFT}                               & Qwen2.5-Math-RM (72B)                        \\
% RL Data        & \textbf{150K queries $\times$ 4 samples} & 66K queries $\times$ 32 samples \\ \midrule
% \end{tabular}
% }
% \end{table}


\vspace{-0.2cm}
\section{Impact: Why Free Scientific Knowledge?}
\vspace{-0.1cm}

Historically, making knowledge widely available has driven transformative progress. Gutenberg’s printing press broke medieval monopolies on information, increasing literacy and contributing to the Renaissance and Scientific Revolution. In today's world, open source projects such as GNU/Linux and Wikipedia show that freely accessible and modifiable knowledge fosters innovation while ensuring creators are credited through copyleft licenses. These examples highlight a key idea: \textit{access to essential knowledge supports overall advancement.} 

This aligns with the arguments made by Prabhakaran et al. \cite{humanrightsbasedapproachresponsible}, who specifically highlight the \textbf{ human right to participate in scientific advancement} as enshrined in the Universal Declaration of Human Rights. They emphasize that this right underscores the importance of \textit{ equal access to the benefits of scientific progress for all}, a principle directly supported by our proposal for Knowledge Units. The UN Special Rapporteur on Cultural Rights further reinforces this, advocating for the expansion of copyright exceptions to broaden access to scientific knowledge as a crucial component of the right to science and culture \cite{scienceright}. 

However, current intellectual property regimes often create ``patently unfair" barriers to this knowledge, preventing innovation and access, especially in areas critical to human rights, as Hale compellingly argues \cite{patentlyunfair}. Finding a solution requires carefully balancing the imperative of open access with the legitimate rights of authors. As Austin and Ginsburg remind us, authors' rights are also human rights, necessitating robust protection \cite{authorhumanrights}. Shareable knowledge entities like Knowledge Units offer a potential mechanism to achieve this delicate balance in the scientific domain, enabling wider dissemination of research findings while respecting authors' fundamental rights.

\vspace{-0.2cm}
\subsection{Impact Across Sectors}

\textbf{Researchers:} Collaboration across different fields becomes easier when knowledge is shared openly. For instance, combining machine learning with biology or applying quantum principles to cryptography can lead to important breakthroughs. Removing copyright restrictions allows researchers to freely use data and methods, speeding up discoveries while respecting original contributions.

\textbf{Practitioners:} Professionals, especially in healthcare, benefit from immediate access to the latest research. Quick access to newer insights on the effectiveness of drugs, and alternative treatments speeds up adoption and awareness, potentially saving lives. Additionally, open knowledge helps developing countries gain access to health innovations.

\textbf{Education:} Education becomes more accessible when teachers use the latest research to create up-to-date curricula without prohibitive costs. Students can access high-quality research materials and use LM assistance to better understand complex topics, enhancing their learning experience and making high-quality education more accessible.

\textbf{Public Trust:} When information is transparent and accessible, the public can better understand and trust decision-making processes. Open access to government policies and industry practices allows people to review and verify information, helping to reduce misinformation. This transparency encourages critical thinking and builds trust in scientific and governmental institutions.

Overall, making scientific knowledge accessible supports global fairness. By viewing knowledge as a common resource rather than a product to be sold, we can speed up innovation, encourage critical thinking, and empower communities to address important challenges.

\vspace{-0.2cm}
\section{Open Problems}
\vspace{-0.1cm}

Moving forward, we identify key research directions to further exploit the potential of converting original texts into shareable knowledge entities such as demonstrated by the conversion into Knowledge Units in this work:


\textbf{1. Enhancing Factual Accuracy and Reliability:}  Refining KUs through cross-referencing with source texts and incorporating community-driven correction mechanisms, similar to Wikipedia, can minimize hallucinations and ensure the long-term accuracy of knowledge-based datasets at scale.

\textbf{2. Developing Applications for Education and Research:}  Using KU-based conversion for datasets to be employed in practical tools, such as search interfaces and learning platforms, can ensure rapid dissemination of any new knowledge into shareable downstream resources, significantly improving the accessibility, spread, and impact of KUs.

\textbf{3. Establishing Standards for Knowledge Interoperability and Reuse:}  Future research should focus on defining standardized formats for entities like KU and knowledge graph layouts \citep{lenat1990cyc}. These standards are essential to unlock seamless interoperability, facilitate reuse across diverse platforms, and foster a vibrant ecosystem of open scientific knowledge. 

\textbf{4. Interconnecting Shareable Knowledge for Scientific Workflow Assistance and Automation:} There might be further potential in constructing a semantic web that interconnects publicly shared knowledge, together with mechanisms that continually update and validate all shareable knowledge units. This can be starting point for a platform that uses all collected knowledge to assist scientific workflows, for instance by feeding such a semantic web into recently developed reasoning models equipped with retrieval augmented generation. Such assistance could assemble knowledge across multiple scientific papers, guiding scientists more efficiently through vast research landscapes. Given further progress in model capabilities, validation, self-repair and evolving new knowledge from already existing vast collection in the semantic web can lead to automation of scientific discovery, assuming that knowledge data in the semantic web can be freely shared.

We open-source our code and encourage collaboration to improve extraction pipelines, enhance Knowledge Unit capabilities, and expand coverage to additional fields.

\vspace{-0.2cm}
\section{Conclusion}
\vspace{-0.1cm}

In this paper, we highlight the potential of systematically separating factual scientific knowledge from protected artistic or stylistic expression. By representing scientific insights as structured facts and relationships, prototypes like Knowledge Units (KUs) offer a pathway to broaden access to scientific knowledge without infringing copyright, aligning with legal principles like German \S 24(1) UrhG and U.S. fair use standards. Extensive testing across a range of domains and models shows evidence that Knowledge Units (KUs) can feasibly retain core information. These findings offer a promising way forward for openly disseminating scientific information while respecting copyright constraints.

\section*{Author Contributions}

Christoph conceived the project and led organization. Christoph and Gollam led all the experiments. Nick and Huu led the legal aspects. Tawsif led the data collection. Ameya and Andreas led the manuscript writing. Ludwig, Sören, Robert, Jenia and Matthias provided feedback. advice and scientific supervision throughout the project. 

\section*{Acknowledgements}

The authors would like to thank (in alphabetical order): Sebastian Dziadzio, Kristof Meding, Tea Mustać, Shantanu Prabhat for insightful feedback and suggestions. Special thanks to Andrej Radonjic for help in scaling up data collection. GR and SA acknowledge financial support by the German Research Foundation (DFG) for the NFDI4DataScience Initiative (project number 460234259). AP and MB acknowledge financial support by the Federal Ministry of Education and Research (BMBF), FKZ: 011524085B and Open Philanthropy Foundation funded by the Good Ventures Foundation. AH acknowledges financial support by the Federal Ministry of Education and Research (BMBF), FKZ: 01IS24079A and the Carl Zeiss Foundation through the project "Certification and Foundations of Safe ML Systems" as well as the support from the International Max Planck Research School for Intelligent Systems (IMPRS-IS). JJ acknowledges funding by the Federal Ministry of Education and Research of Germany (BMBF) under grant no. 01IS22094B (WestAI - AI Service Center West), under grant no. 01IS24085C (OPENHAFM) and under the grant DE002571 (MINERVA), as well as co-funding by EU from EuroHPC Joint Undertaking programm under grant no. 101182737 (MINERVA) and from Digital Europe Programme under grant no. 101195233 (openEuroLLM) 
%%
%% Bibliography
%%
\bibliography{main}

\newpage
\appendix
\begin{table}[t]
    \centering
    \caption{The performance of different pre-trained models on ImageNet and infrared semantic segmentation datasets. The \textit{Scratch} means the performance of randomly initialized models. The \textit{PT Epochs} denotes the pre-training epochs while the \textit{IN1K FT epochs} represents the fine-tuning epochs on ImageNet \citep{imagenet}. $^\dag$ denotes models reproduced using official codes. $^\star$ refers to the effective epochs used in \citet{iBOT}. The top two results are marked in \textbf{bold} and \underline{underlined} format. Supervised and CL methods, MIM methods, and UNIP models are colored in \colorbox{orange!15}{\rule[-0.2ex]{0pt}{1.5ex}orange}, \colorbox{gray!15}{\rule[-0.2ex]{0pt}{1.5ex}gray}, and \colorbox{cyan!15}{\rule[-0.2ex]{0pt}{1.5ex}cyan}, respectively.}
    \label{tab:benchmark}
    \centering
    \scriptsize
    \setlength{\tabcolsep}{1.0mm}{
    \scalebox{1.0}{
    \begin{tabular}{l c c c c  c c c c c c c c}
        \toprule
         \multirow{2}{*}{Methods} & \multirow{2}{*}{\makecell[c]{PT \\ Epochs}} & \multicolumn{2}{c}{IN1K FT} & \multicolumn{4}{c}{Fine-tuning (FT)} & \multicolumn{4}{c}{Linear Probing (LP)} \\
         \cmidrule{3-4} \cmidrule(lr){5-8} \cmidrule(lr){9-12} 
         & & Epochs & Acc & SODA & MFNet-T & SCUT-Seg & Mean & SODA & MFNet-T & SCUT-Seg & Mean \\
         \midrule
         \textcolor{gray}{ViT-Tiny/16} & & &  & & & & & & & & \\
         Scratch & - & - & - & 31.34 & 19.50 & 41.09 & 30.64 & - & - & - & - \\
         \rowcolor{gray!15} MAE$^\dag$ \citep{mae} & 800 & 200 & \underline{71.8} & 52.85 & 35.93 & 51.31 & 46.70 & 23.75 & 15.79 & 27.18 & 22.24 \\
         \rowcolor{orange!15} DeiT \citep{deit} & 300 & - & \textbf{72.2} & 63.14 & 44.60 & 61.36 & 56.37 & 42.29 & 21.78 & 31.96 & 32.01 \\
         \rowcolor{cyan!15} UNIP (MAE-L) & 100 & - & - & \underline{64.83} & \textbf{48.77} & \underline{67.22} & \underline{60.27} & \underline{44.12} & \underline{28.26} & \underline{35.09} & \underline{35.82} \\
         \rowcolor{cyan!15} UNIP (iBOT-L) & 100 & - & - & \textbf{65.54} & \underline{48.45} & \textbf{67.73} & \textbf{60.57} & \textbf{52.95} & \textbf{30.10} & \textbf{40.12} & \textbf{41.06}  \\
         \midrule
         \textcolor{gray}{ViT-Small/16} & & & & & & & & & & & \\
         Scratch & - & - & - & 41.70 & 22.49 & 46.28 & 36.82 & - & - & - & - \\
         \rowcolor{gray!15} MAE$^\dag$ \citep{mae} & 800 & 200 & 80.0 & 63.36 & 42.44 & 60.38 & 55.39 & 38.17 & 21.14 & 34.15 & 31.15 \\
         \rowcolor{gray!15} CrossMAE \citep{crossmae} & 800 & 200 & 80.5 & 63.95 & 43.99 & 63.53 & 57.16 & 39.40 & 23.87 & 34.01 & 32.43 \\
         \rowcolor{orange!15} DeiT \citep{deit} & 300 & - & 79.9 & 68.08 & 45.91 & 66.17 & 60.05 & 44.88 & 28.53 & 38.92 & 37.44 \\
         \rowcolor{orange!15} DeiT III \citep{deit3} & 800 & - & 81.4 & 69.35 & 47.73 & 67.32 & 61.47 & 54.17 & 32.01 & 43.54 & 43.24 \\
         \rowcolor{orange!15} DINO \citep{dino} & 3200$^\star$ & 200 & \underline{82.0} & 68.56 & 47.98 & 68.74 & 61.76 & 56.02 & 32.94 & 45.94 & 44.97 \\
         \rowcolor{orange!15} iBOT \citep{iBOT} & 3200$^\star$ & 200 & \textbf{82.3} & 69.33 & 47.15 & 69.80 & 62.09 & 57.10 & 33.87 & 45.82 & 45.60 \\
         \rowcolor{cyan!15} UNIP (DINO-B) & 100 & - & - & 69.35 & 49.95 & 69.70 & 63.00 & \underline{57.76} & \underline{34.15} & \underline{46.37} & \underline{46.09} \\
         \rowcolor{cyan!15} UNIP (MAE-L) & 100 & - & - & \textbf{70.99} & \underline{51.32} & \underline{70.79} & \underline{64.37} & 55.25 & 33.49 & 43.37 & 44.04 \\
         \rowcolor{cyan!15} UNIP (iBOT-L) & 100 & - & - & \underline{70.75} & \textbf{51.81} & \textbf{71.55} & \textbf{64.70} & \textbf{60.28} & \textbf{37.16} & \textbf{47.68} & \textbf{48.37} \\ 
        \midrule
        \textcolor{gray}{ViT-Base/16} & & & & & & & & & & & \\
        Scratch & - & - & - & 44.25 & 23.72 & 49.44 & 39.14 & - & - & - & - \\
        \rowcolor{gray!15} MAE \citep{mae} & 1600 & 100 & 83.6 & 68.18 & 46.78 & 67.86 & 60.94 & 43.01 & 23.42 & 37.48 & 34.64 \\
        \rowcolor{gray!15} CrossMAE \citep{crossmae} & 800 & 100 & 83.7 & 68.29 & 47.85 & 68.39 & 61.51 & 43.35 & 26.03 & 38.36 & 35.91 \\
        \rowcolor{orange!15} DeiT \citep{deit} & 300 & - & 81.8 & 69.73 & 48.59 & 69.35 & 62.56 & 57.40 & 34.82 & 46.44 & 46.22 \\
        \rowcolor{orange!15} DeiT III \citep{deit3} & 800 & 20 & \underline{83.8} & 71.09 & 49.62 & 70.19 & 63.63 & 59.01 & \underline{35.34} & 48.01 & 47.45 \\
        \rowcolor{orange!15} DINO \citep{dino} & 1600$^\star$ & 100 & 83.6 & 69.79 & 48.54 & 69.82 & 62.72 & 59.33 & 34.86 & 47.23 & 47.14 \\
        \rowcolor{orange!15} iBOT \citep{iBOT} & 1600$^\star$ & 100 & \textbf{84.0} & 71.15 & 48.98 & 71.26 & 63.80 & \underline{60.05} & 34.34 & \underline{49.12} & \underline{47.84} \\
        \rowcolor{cyan!15} UNIP (MAE-L) & 100 & - & - & \underline{71.47} & \textbf{52.55} & \underline{71.82} & \textbf{65.28} & 58.82 & 34.75 & 48.74 & 47.43 \\
        \rowcolor{cyan!15} UNIP (iBOT-L) & 100 & - & - & \textbf{71.75} & \underline{51.46} & \textbf{72.00} & \underline{65.07} & \textbf{63.14} & \textbf{39.08} & \textbf{52.53} & \textbf{51.58} \\
        \midrule
        \textcolor{gray}{ViT-Large/16} & & & & & & & & & & & \\
        Scratch & - & - & - & 44.70 & 23.68 & 49.55 & 39.31 & - & - & - & - \\
        \rowcolor{gray!15} MAE \citep{mae} & 1600 & 50 & \textbf{85.9} & 71.04 & \underline{51.17} & 70.83 & 64.35 & 52.20 & 31.21 & 43.71 & 42.37 \\
        \rowcolor{gray!15} CrossMAE \citep{crossmae} & 800 & 50 & 85.4 & 70.48 & 50.97 & 70.24 & 63.90 & 53.29 & 33.09 & 45.01 & 43.80 \\
        \rowcolor{orange!15} DeiT3 \citep{deit3} & 800 & 20 & \underline{84.9} & \underline{71.67} & 50.78 & \textbf{71.54} & \underline{64.66} & \underline{59.42} & \textbf{37.57} & \textbf{50.27} & \underline{49.09} \\
        \rowcolor{orange!15} iBOT \citep{iBOT} & 1000$^\star$ & 50 & 84.8 & \textbf{71.75} & \textbf{51.66} & \underline{71.49} & \textbf{64.97} & \textbf{61.73} & \underline{36.68} & \underline{50.12} & \textbf{49.51} \\
        \bottomrule
    \end{tabular}}}
    \vspace{-2mm}
\end{table}
\section{Reward functions: properties and examples}\label{subsec:properties}

Recall that miner strategies take as input the amount of reward available for extending each existing block at time $t$, as specified by the reward function $R$, and make decisions about where to mine, what to include, and what to broadcast accordingly. This section defines a set of natural properties that reward functions might have. In \Cref{sec:examples}, we apply these properties to transaction fees and LVR, two of the primary MEV sources observed empirically to date.
While we define these properties in the context of the NCG in this paper, we believe their applicability extends far beyond Proof-of-Work and selfish mining. Our framework can be used to characterize rewards and their implications for the incentives of consensus participants across blockchain protocols.

Recall that in the NCG, given a set of miner strategies, three independent random variables $\vec{t},\vec{m},r$ are drawn and are used to compute a set of views $V_t^m$ for all miners $m$ and all times $t$.
Let $\mathcal{V}_t^m$ be the support $V_t^m$, meaning the set of views achievable at time $t$ for \emph{some} realization of $\vec{t},\vec{m},r$. Initially, $\mathcal{V}_0^m=\{V_0\}$ for all $m$, where $V_0:=\{B_0\}$ is the view containing a single genesis block $B_0$ such that $\tstamp(B_0)=0$. Miner strategies in the NCG take the realization of a reward function as input. That is, at time $t$, miner $m$ sees the reward $R^m(t,V_t^m,B,r,B')$ for extending each block $B\in V_t^m$ by a valid block $B'\in\mathcal{B}^m(t,V_t^m,B,r)$.

A miner-independent reward function yields the same value for the block regardless of who created it. This 
corresponds to a setting where all miners have access to the same set of rewards (e.g., the common value setting), and thus, we drop the superscript $m$. In practice, some reward sources may be heterogeneous between block producers (e.g., from private order flow or from differing abilities to extract MEV \cite{bahrani2024centralization}). All reward functions considered in this paper will be miner-independent, but the properties can be readily generalized by tracking the subset of miners with access to each reward source. See \Cref{sec:conclusion} for a discussion of extending this work.

\begin{definition}[Miner-Independent Rewards] \label{def:minerindependent}
    A reward function $R$ is \emph{miner-independent} if for all times $t$, all miners have the same set of valid views, the same set of valid blocks extending each block in those views, and equal rewards from any such valid block.\footnote{\label{foot:bijection}Technically, since blocks include information about their creator, it would be more accurate to say that there is a bijection between the set of valid views/blocks for any pair of miners. We overlook this formality to simplify notation.} Formally, $R$ is miner-independent if for all $t$, and all $m,m'\in M$, 
    \begin{itemize}
        \item $\mathcal{V}_t^m=\mathcal{V}_t^{m'}$,
        \item for all $V\in\mathcal{V}^m_t$, all blocks $B$ in $V$, and all $r$, we have $\mathcal{B}^m(t,V,B,r)= \mathcal{B}^{m'}(t,V,B,r)$,
        \item for all $V\in \mathcal{V}^m_t$, all $r$, all parent blocks $B$ in $V$, and all valid blocks $B' \in \mathcal{B}^m(t,V,B,r)$, we have $R^m(t,V,B,r,B')= R^{m'}(t,V,B,r,B')$.
    \end{itemize}
\end{definition}

We can also characterize reward functions that grow according to the same distribution without depending on the chain's history. The following property limits the dependence of $R$ on the view. Intuitively, it says that the only relevant information in the view that affects the amount of reward in a block is the \emph{timestamp of its parent}.

\begin{definition}[View-Independent Rewards]\label{def:viewindependent}
    A reward function $R$ is \emph{view-independent} if for all times $t'<t$, any two views $V_1,V_2\in\mathcal{V}_{t'}$ such that $\tstamp(B_1)=\tstamp(B_2)=t'$ for some blocks $B_1\in V_1,B_2\in V_2$, we have:
    \begin{itemize}
        \item for all $r$, the set of valid blocks extending $B_1$ at $t$ in $V_1$ is the same as the set of valid blocks extending $B_2$ at $t$ in $V_2$, $\mathcal{B}(t,V_1,B_1,r)=\mathcal{B}(t,V_2,B_2,r)$,\footnote{Recall that when we invoke a view and randomness together as inputs to a function, we implicitly assume that the randomness could give rise to the view.}
        and
        \item for every valid block $B'\in\mathcal{B}(t,V_1,B_1,r)$, we have
        \[
            \Pr_{r,\vec{t},\vec{m}\vert V_1}[R(t,V_1,B_1, r, B')=x]=\Pr_{r,\vec{t},\vec{m}\vert V_2}[R(t,V_2,B_2, r, B')=x]
        \]
        for all $x$.
    \end{itemize}
\end{definition}

Note that fixing a view $V_1$ (resp. $V_2$) can update the distribution of the $r,\vec{t},\vec{m}$. We use the subscript $r,\vec{t},\vec{m}\vert V_i$ to refer to the posterior distribution of these random variables conditioned on $V_1,V_2$. Block rewards are view-independent if and only if there is no halving (since halving occurs at fixed block heights).
As another (non-)example, \Cref{rem:patient-not-vi} demonstrates how transaction fees that are not fully claimed by block $B$ (e.g., from finite block sizes) are not view-independent.  

View-independence already limits the dependence of $R$ on the view to the timestamp of the parent block. We next define a subset of view-independent rewards where the dependence on view is limited to the length of \emph{elapsed time since the parent block} (and is the same regardless of the exact parent block timestamp).

\begin{definition}[Static Rewards]\label{def:static}
    A reward function $R$ is \emph{static} if for all $\Delta>0$, all times $t_1,t_2$ and views $V_1\in\mathcal{V}_{t_1}$ and $V_2\in\mathcal{V}_{t_2}$ such that $\tstamp(B_1)=t_1-\Delta$ and $\tstamp(B_2)=t_2-\Delta$, we have:
    \begin{itemize}
        \item for all $r$, the set of valid blocks extending $B_1$ at $t_1$ in $V_{1}$ is the same as the set of valid blocks extending $B_2$ at $t_2$ in $V_{2}$, $\mathcal{B}(t_1,V_{1},B_1,r)=\mathcal{B}(t_2,V_{2},B_2,r)$, and
        \item for all valid blocks $B'\in\mathcal{B}(t_1,V_1,B_1,r)$, we have
        \[
            \Pr_{r,\vec{t},\vec{m}\vert V_1}[R(t_1,V_1,B_1, r, B')=x]=\Pr_{r,\vec{t},\vec{m}\vert V_2}[R(t_2,V_2,B_2,r,B')=x]
        \]
        for all $x$.
    \end{itemize}
\end{definition}

\Cref{rem:patientisstatic} highlights that transaction fees are static using the \citet{carlsten2016instability} model with constant arrival rate and infinite block sizes. Conversely, \Cref{rem:non-local-lvr} illustrates how LVR is not static because it depends on the CEX price of an asset (which impacts the step size of the Geometric Brownian Motion). \Cref{rem:local-lvr-static} demonstrates that within the same price neighborhood, a type of LVR (which we call ``resetting'') \textit{is} static.

\begin{definition}[Maximum Rewards \& Maximizing Blocks]\label{def:max-block}
    Given a reward function $R$, we define the maximizing block function $B_{\text{opt}}$ as 
    \[
    	B_{\text{opt}}(t,V,B,r):=\underset{B'\in\mathcal{B}(t,V,B,r)}{\arg\max} R(t,V,B,r,B').
    \]
    We further define the maximum reward function $R_{\text{opt}}$ as 
    \[
    	R_{\text{opt}}(t,V,B,r):=R(t,V,B,r,B')
    \] for some $B'\in B_{\text{opt}}(t,V,B,r)$.
\end{definition}

Observe that if a reward function $R$ is static, then $\Ropt(t,V,B,r,B')$ can be rewritten as a two-variable function of just $r$ and the time $\Delta$ between $\tstamp(B)$ and $t$. 

Lastly, we define \emph{persistent rewards}, which arrive at some time and can be claimed at most once. Upon arrival, they remain indefinitely claimable by any block whose ancestors have not already claimed them. Let $\Rc(B)$ denote the amount of reward attributed to the block creator if the block becomes canonical and $\chain(B)$ the set of blocks on the ancestral path of $B$ (including $B$).

\begin{definition}[Persistent Rewards]\label{def:persistent}
A reward function $R$ is \emph{persistent} if for all realizations of $\vec{t},\vec{m},r$, at any time $t$, for all blocks $B$ in the resulting view $V$, we have:
\begin{itemize}
    \item for all $B'\in\mathcal{B}(t,V,B,r)$, 
    \begin{align}\label{eq:persistence}
        R(t,V,B,r,B')\leq R_{\text{opt}}(t,V_0,B_0,r)-\sum_{B''\in \chain(B)} \Rc(B''),
    \end{align}
    \item there exists some $B'\in\mathcal{B}(t,V,B,r)$ for which the above holds with equality.
\end{itemize}
\end{definition}

We sometimes call a non-persistent reward function \emph{ephemeral}. \Cref{rem:patient-persistent} highlights that transaction fees are persistent if the users creating the transactions are patient (willing to wait for inclusion and not cancel pending transactions). On the other hand, fees from transactions submitted by impatient users (as in \Cref{rem:impatient}) are not persistent since the canceled transactions are no longer claimable by future blocks. 

Persistent rewards are not affected by orphan or uncle blocks, but they \emph{may} be view-dependent since they are affected by the claimed rewards on the ancestral path of a block. The following lemma states that persistent rewards functions are view-independent if all blocks in all valid views claim the maximum available rewards. The proof can be found in \Cref{pr:lemm1}.

\begin{lemma}[Persistent \& Maximizing Blocks $\implies$ View-Independent]\label{lem:all-claim}
    Let $R$ be persistent. Then $R$ is view-independent if for all $t$, all $V\in\mathcal{V}_t$, all parent-child blocks $B,B'$ in $V$, and all $r$, we have $\Rc(B')=\Bopt(t,V,B,r)$.
\end{lemma}

The following lemma shows that static and persistent reward functions accrue linearly over time since the parent block, with a constant slope and intercept across blocks (but may be random depending on $r$). If a reward function is persistent and static, it can be simulated by drawing the randomness of $r$ to set the slope $a$ and the intercept $b$ of the maximum available reward function $\Ropt$. Then, for any block $B$ in any view, the reward for extending $B$ at time $\tstamp(B)+\Delta$ equals $a\cdot \Delta + b$. This is the model of transaction fee accrual in \citet{carlsten2016instability} and MEV accrual in \citet{schwarz2023time}. The proof can be found in \Cref{pr:lemm2}

\begin{lemma}[Static \& Persistent $\implies$ Linear]\label{lem:linear}
    Let $R$ be static and persistent. If $\Ropt(t,V,B,r)$ is differentiable with respect to $t$, then it is of the form $a(r)\cdot (t-\tstamp(B)) + b(r)$.
\end{lemma}

Note that the transaction fees defined in \citet{carlsten2016instability} are linear; we use this same reward function as part of our instantiation in \Cref{sec:multiplerewards}.




%\newpage


%\begin{itemize}

%\item Move sembra più adatto per la verifica rispetto a Solidity (fare esempi di quando siamo riusciti a verificare meglio) In Section \ref{sec:comparison:language}  sembra sempre meglio, eccetto che per immutable.

%\item \enriconote{più facile capire la ground truth in Move perché non ci sono le schifezze di Solidity}

%\item in entrambi i casi, esprimere bene alcune proprietà richiede conoscenze di basso livello di come funziona il linguaggio (per Solidity, reentrancy, fallback; per Move, quella roba sulle features dei token, ...). Dare spec richiede conoscenza della piattaforma abbastanza low level, perché a volte la verifica fallisce e non capisci perché

%\item per Solidity ecosistema più ricco di tool per la verifica, anche che vanno oltre le proprietà verificabili da Certora

%\item Certora nel complesso più espressivo, ma sembra possibile migliorare il Move Prover per renderlo paragonabile a Certora riguardo ad espressività (ad esempio, trace properties, metamorphic properties, reference a transition variables)

%\item esistono comunque proprietà interessanti che nessuno dei tool sembra di grado di esprimere (hyperproperties?)
%(It has been observed that properties that abstract the system %, although in general harder to verify in practice, are  more desirable and have a better return-on-investment than low-level properties~\cite{Xu24langsec}.) 

%\item Non abbiamo pretesa di esaustività: è possibile/probabile che ci siano altre differenze rilevanti che non abbiamo incontrato nei nostri esperimenti. Ciò comunque non invalida i nostri finds. 

%\item Future work includes 
%extending our dataset and exploring other features, % \enriconote{mah potrebbe far pensare che abbia esplorato giusto un paio di features}
%as well as investigating  verification in other contract languages.
% \begin{itemize}
%    \item sarebbe bello mettere mutants per vedere quanto i tool sono bravi a fare bug detection
%    \item altri linguaggi?
%    \item gas rent fees?
%\end{itemize} 
%\end{itemize}


\noindent \enriconote{TODOs:  \ \\ \ \\ 
 %* change  github url to solidity-vs-move-verification \ \\ \ \\ 
 * add link to table in each README \ \\ %(e.g. https://github.com/blockchain-unica/certora-vs-move-prover-benchmark/blob/main/contracts/bank/README.md)
 * define N/A and N/D in tables \ \\
 %* link codice appendice da Sez 4 \ \\
 * %cambiare nomi prop local-global intra-inter 
 in MSL a rigore nella inter dovremmo usare disable\_invariants\_in\_body \ \\ 
}

% % \section{Taxonomy}

% As illustrated by Fig. \ref{}, the typical process of vision models based time series analysis has five components: (1) normalization/scaling; (2) time series to image transformation; (3) image modeling; (4) image to time series recovery; and (5) task processing. In the rest of this paper, we will discuss the typical methods for each of these components. The detailed taxonomy of the methods are summarized in Table \ref{tab.taxonomy}.

%Typical step: normalization/scaling, transformation, vision modeling, task-specific head, inverse transformation (for tasks that output time series, e.g., forecasting, generation, imputation, anomaly detection). Normalization is to fit the arbitrary range of time series values to RGB representation.

\begin{figure*}[!t]
\centering
\includegraphics[width=1.0\textwidth]{fig/fig_3.pdf}
% \vspace{-1em}
\caption{An illustration of different methods for imaging time series with a sample (length=336) from the \textit{Electricity} benchmark dataset \protect\cite{nie2023time}. (a)(c)(d)(e)(f) %are univariate methods.
visualize the same variate. (b) visualizes all 321 variates. Filterbank is omitted due to its %high
similarity to STFT.}\label{fig.tsimage}
\vspace{-0.2cm}
\end{figure*}

\begin{table*}[t]
\centering
\scriptsize
\setlength{\tabcolsep}{2.7pt}{
% \begin{tabular}{llllllllllll}
\begin{tabular}{llcccccccccl}
\toprule[1pt]
\multirow{2}{*}{Method} & \multirow{2}{*}{TS-Type} & \multirow{2}{*}{Imaging} & \multicolumn{5}{c}{Imaged Time Series Modeling} & \multirow{2}{*}{TS-Recover} & \multirow{2}{*}{Task} & \multirow{2}{*}{Domain} & \multirow{2}{*}{Code}\\ \cmidrule{4-8}
 & & & Multi-modal & Model & Pre-trained & Fine-tune & Prompt & & & & \\ \midrule
\cite{silva2013time} & UTS & RP & \xmark & \texttt{K-NN} & \xmark & \xmark & \xmark & \xmark & Classification & General & \xmark\\
\cite{wang2015encoding} & UTS & GAF & \xmark & \texttt{CNN} & \xmark & \cmark$^{\flat}$ & \xmark & \cmark & Classification & General & \xmark\\
\cite{wang2015imaging} & UTS & GAF & \xmark & \texttt{CNN} & \xmark & \cmark$^{\flat}$ & \xmark & \cmark & Multiple & General & \xmark\\
% \multirow{2}{*}{\cite{wang2015imaging}} & \multirow{2}{*}{UTS} & \multirow{2}{*}{GAF} & \multirow{2}{*}{\xmark} & \multirow{2}{*}{\texttt{CNN}} & \multirow{2}{*}{\xmark} & \multirow{2}{*}{\cmark$^{\flat}$} & \multirow{2}{*}{\xmark} & \multirow{2}{*}{\cmark} & Classification & \multirow{2}{*}{General} & \multirow{2}{*}{\xmark}\\
% & & & & & & & & & \& Imputation & & \\
\cite{ma2017learning} & MTS & Heatmap & \xmark & \texttt{CNN} & \xmark & \cmark$^{\flat}$ & \xmark & \cmark & Forecasting & Traffic & \xmark\\
\cite{hatami2018classification} & UTS & RP & \xmark & \texttt{CNN} & \xmark & \cmark$^{\flat}$ & \xmark & \xmark & Classification & General & \xmark\\
\cite{yazdanbakhsh2019multivariate} & MTS & Heatmap & \xmark & \texttt{CNN} & \xmark & \cmark$^{\flat}$ & \xmark & \xmark & Classification & General & \cmark\textsuperscript{\href{https://github.com/SonbolYb/multivariate_timeseries_dilated_conv}{[1]}}\\
MSCRED \cite{zhang2019deep} & MTS & Other ($\S$\ref{sec.othermethod}) & \xmark & \texttt{ConvLSTM} & \xmark & \cmark$^{\flat}$ & \xmark & \xmark & Anomaly & General & \cmark\textsuperscript{\href{https://github.com/7fantasysz/MSCRED}{[2]}}\\
\cite{li2020forecasting} & UTS & RP & \xmark & \texttt{CNN} & \cmark & \cmark & \xmark & \xmark & Forecasting & General & \cmark\textsuperscript{\href{https://github.com/lixixibj/forecasting-with-time-series-imaging}{[3]}}\\
\cite{cohen2020trading} & UTS & LinePlot & \xmark & \texttt{Ensemble} & \xmark & \cmark$^{\flat}$ & \xmark & \xmark & Classification & Finance & \xmark\\
% \cite{du2020image} & UTS & Spectrogram & \xmark & \texttt{CNN} & \xmark & \cmark$^{\flat}$ & \xmark & \xmark & Classification & Finance & \xmark\\
\cite{barra2020deep} & UTS & GAF & \xmark & \texttt{CNN} & \xmark & \cmark$^{\flat}$ & \xmark & \xmark & Classification & Finance & \xmark\\
% \cite{barra2020deep} & UTS & GAF & \xmark & \texttt{VGG-16} & \xmark & \cmark$^{\flat}$ & \xmark & \xmark & Classification & Finance & \xmark\\
% \cite{cao2021image} & UTS & RP & \xmark & \texttt{CNN} & \xmark & \cmark$^{\flat}$ & \xmark & \xmark & Classification & General & \xmark\\
VisualAE \cite{sood2021visual} & UTS & LinePlot & \xmark & \texttt{CNN} & \xmark & \cmark$^{\flat}$ & \xmark & \cmark & Forecasting & Finance & \xmark\\
% VisualAE \cite{sood2021visual} & UTS & LinePlot & \xmark & \texttt{CNN} & \xmark & \cmark$^{\flat}$ & \xmark & \xmark & Img-Generation & Finance & \xmark\\
\cite{zeng2021deep} & MTS & Heatmap & \xmark & \texttt{CNN,LSTM} & \xmark & \cmark$^{\flat}$ & \xmark & \cmark & Forecasting & Finance & \xmark\\
% \cite{zeng2021deep} & MTS & Heatmap & \xmark & \texttt{SRVP} & \xmark & \cmark$^{\flat}$ & \xmark & \cmark & Forecasting & Finance & \xmark\\
AST \cite{gong2021ast} & UTS & Spectrogram & \xmark & \texttt{DeiT} & \cmark & \cmark & \xmark & \xmark & Classification & Audio & \cmark\textsuperscript{\href{https://github.com/YuanGongND/ast}{[4]}}\\
TTS-GAN \cite{li2022tts} & MTS & Heatmap & \xmark & \texttt{ViT} & \xmark & \cmark$^{\flat}$ & \xmark & \cmark & Ts-Generation & Health & \cmark\textsuperscript{\href{https://github.com/imics-lab/tts-gan}{[5]}}\\
SSAST \cite{gong2022ssast} & UTS & Spectrogram & \xmark & \texttt{ViT} & \cmark$^{\natural}$ & \cmark & \xmark & \xmark & Classification & Audio & \cmark\textsuperscript{\href{https://github.com/YuanGongND/ssast}{[6]}}\\
MAE-AST \cite{baade2022mae} & UTS & Spectrogram & \xmark & \texttt{MAE} & \cmark$^{\natural}$ & \cmark & \xmark & \xmark & Classification & Audio & \cmark\textsuperscript{\href{https://github.com/AlanBaade/MAE-AST-Public}{[7]}}\\
AST-SED \cite{li2023ast} & UTS & Spectrogram & \xmark & \texttt{SSAST,GRU} & \cmark & \cmark & \xmark & \xmark & EventDetection & Audio & \xmark\\
\cite{jin2023classification} & UTS & %Multiple
LinePlot & \xmark & \texttt{CNN} & \cmark & \cmark & \xmark & \xmark & Classification & Physics & \xmark\\
ForCNN \cite{semenoglou2023image} & UTS & LinePlot & \xmark & \texttt{CNN} & \xmark & \cmark$^{\flat}$ & \xmark & \xmark & Forecasting & General & \xmark\\
Vit-num-spec \cite{zeng2023pixels} & UTS & Spectrogram & \xmark & \texttt{ViT} & \xmark & \cmark$^{\flat}$ & \xmark & \xmark & Forecasting & Finance & \xmark\\
% \cite{wimmer2023leveraging} & MTS & LinePlot & \xmark & \texttt{CLIP,LSTM} & \cmark & \cmark & \xmark & \xmark & Classification & Finance & \xmark\\
ViTST \cite{li2023time} & MTS & LinePlot & \xmark & \texttt{Swin} & \cmark & \cmark & \xmark & \xmark & Classification & General & \cmark\textsuperscript{\href{https://github.com/Leezekun/ViTST}{[8]}}\\
MV-DTSA \cite{yang2023your} & UTS\textsuperscript{*} & LinePlot & \xmark & \texttt{CNN} & \xmark & \cmark$^{\flat}$ & \xmark & \cmark & Forecasting & General & \cmark\textsuperscript{\href{https://github.com/IkeYang/machine-vision-assisted-deep-time-series-analysis-MV-DTSA-}{[9]}}\\
TimesNet \cite{wu2023timesnet} & MTS & Heatmap & \xmark & \texttt{CNN} & \xmark & \cmark$^{\flat}$ & \xmark & \cmark & Multiple & General & \cmark\textsuperscript{\href{https://github.com/thuml/TimesNet}{[10]}}\\
ITF-TAD \cite{namura2024training} & UTS & Spectrogram & \xmark & \texttt{CNN} & \cmark & \xmark & \xmark & \xmark & Anomaly & General & \xmark\\
\cite{kaewrakmuk2024multi} & UTS & GAF & \xmark & \texttt{CNN} & \cmark & \cmark & \xmark & \xmark & Classification & Sensing & \xmark\\
HCR-AdaAD \cite{lin2024hierarchical} & MTS & RP & \xmark & \texttt{CNN,GNN} & \xmark & \cmark$^{\flat}$ & \xmark & \xmark & Anomaly & General & \xmark\\
FIRTS \cite{costa2024fusion} & UTS & Other ($\S$\ref{sec.othermethod}) & \xmark & \texttt{CNN} & \xmark & \cmark$^{\flat}$ & \xmark & \xmark & Classification & General & \cmark\textsuperscript{\href{https://sites.google.com/view/firts-paper}{[11]}}\\
% \multirow{2}{*}{FIRTS \cite{costa2024fusion}} & \multirow{2}{*}{UTS} & Spectrogram & \multirow{2}{*}{\xmark} & \multirow{2}{*}{\texttt{CNN}} & \multirow{2}{*}{\xmark} & \multirow{2}{*}{\cmark$^{\flat}$} & \multirow{2}{*}{\xmark} & \multirow{2}{*}{\xmark} & \multirow{2}{*}{Classification} & \multirow{2}{*}{General} & \multirow{2}{*}{\cmark\textsuperscript{\href{https://sites.google.com/view/firts-paper}{[2]}}}\\
%  & & \& GAF,RP,MTF & & & & & & & & & \\
% \cite{homenda2024time} & UTS\textsuperscript{*} & Multiple & \xmark & \texttt{CNN} & \xmark & \cmark$^{\flat}$ & \xmark & \xmark & Classification & General & \xmark\\
CAFO \cite{kim2024cafo} & MTS & RP & \xmark & \texttt{CNN,ViT} & \xmark & \cmark$^{\flat}$ & \xmark & \xmark & Explanation & General & \cmark\textsuperscript{\href{https://github.com/eai-lab/CAFO}{[12]}}\\
% \multirow{2}{*}{CAFO \cite{kim2024cafo}} & \multirow{2}{*}{MTS} & \multirow{2}{*}{RP} & \multirow{2}{*}{\xmark} & \texttt{ShuffleNet,ResNet} & \multirow{2}{*}{\cmark} & \multirow{2}{*}{\cmark} & \multirow{2}{*}{\xmark} & \multirow{2}{*}{\xmark} & Classification & \multirow{2}{*}{General} & \multirow{2}{*}{\cmark}\\
%  & & & & \texttt{MLP-Mixer,ViT} & & & & & \& Explanation & & \\
ViTime \cite{yang2024vitime} & UTS\textsuperscript{*} & LinePlot & \xmark & \texttt{ViT} & \cmark$^{\natural}$ & \cmark & \xmark & \cmark & Forecasting & General & \cmark\textsuperscript{\href{https://github.com/IkeYang/ViTime}{[13]}}\\
ImagenTime \cite{naiman2024utilizing} & MTS & Other ($\S$\ref{sec.othermethod}) & \xmark & %\texttt{Diffusion}
\texttt{CNN} & \xmark & \cmark$^{\flat}$ & \xmark & \cmark & Ts-Generation & General & \cmark\textsuperscript{\href{https://github.com/azencot-group/ImagenTime}{[14]}}\\
TimEHR \cite{karami2024timehr} & MTS & Heatmap & \xmark & \texttt{CNN} & \xmark & \cmark$^{\flat}$ & \xmark & \cmark & Ts-Generation & Health & \cmark\textsuperscript{\href{https://github.com/esl-epfl/TimEHR}{[15]}}\\
VisionTS \cite{chen2024visionts} & UTS\textsuperscript{*} & Heatmap & \xmark & \texttt{MAE} & \cmark & \cmark & \xmark & \cmark & Forecasting & General & \cmark\textsuperscript{\href{https://github.com/Keytoyze/VisionTS}{[16]}}\\ \midrule
InsightMiner \cite{zhang2023insight} & UTS & LinePlot & \cmark & \texttt{LLaVA} & \cmark & \cmark & \cmark & \xmark & Txt-Generation & General & \xmark\\
\cite{wimmer2023leveraging} & MTS & LinePlot & \cmark & \texttt{CLIP,LSTM} & \cmark & \cmark & \xmark & \xmark & Classification & Finance & \xmark\\
% \cite{dixit2024vision} & UTS & Spectrogram & \cmark & \texttt{GPT4o,Gemini} & \cmark & \xmark & \cmark & \xmark & Classification & Audio & \xmark\\
\multirow{2}{*}{\cite{dixit2024vision}} & \multirow{2}{*}{UTS} & \multirow{2}{*}{Spectrogram} & \multirow{2}{*}{\cmark} & \texttt{GPT4o,Gemini} & \multirow{2}{*}{\cmark} & \multirow{2}{*}{\xmark} & \multirow{2}{*}{\cmark} & \multirow{2}{*}{\xmark} & \multirow{2}{*}{Classification} & \multirow{2}{*}{Audio} & \multirow{2}{*}{\xmark}\\
 & & & & \& \texttt{Claude3} & & & & & & & \\
\cite{daswani2024plots} & MTS & LinePlot & \cmark & \texttt{GPT4o,Gemini} & \cmark & \xmark & \cmark & \xmark & Multiple & General & \xmark\\
TAMA \cite{zhuang2024see} & UTS & LinePlot & \cmark & \texttt{GPT4o} & \cmark & \xmark & \cmark & \xmark & Anomaly & General & \xmark\\
\cite{prithyani2024feasibility} & MTS & LinePlot & \cmark & \texttt{LLaVA} & \cmark & \cmark & \cmark & \xmark & Classification & General & \cmark\textsuperscript{\href{https://github.com/vinayp17/VLM_TSC}{[17]}}\\
\bottomrule[1pt]
\end{tabular}}
\vspace{-0.25cm}
\caption{Taxonomy of vision models on time series. The top panel includes single-modal models. The bottom panel includes multi-modal models. {\bf TS-Type} denotes type of time series. {\bf TS-Recover} denotes %whether time series recovery ($\S$\ref{sec.processing}) has been performed.
recovering time series from predicted images ($\S$\ref{sec.processing}). \textsuperscript{*}: %the model has been %applied on MTSs by %processing %modeling the individual UTSs of each MTS.
the method has been used to model the individual UTSs of an MTS. $^{\natural}$: a new pre-trained model was proposed in the work. $^{\flat}$: %without using a pre-trained model, fine-tune means training from scratch.
when pre-trained models were unused, ``Fine-tune'' refers to train a task-specific model from scratch. %In the
{\bf Model} column: \texttt{CNN} could be regular CNN, ResNet, VGG-Net, %U-Net,
{\em etc.}}\label{tab.taxonomy}
% The code only include verified official code from the authors.
\vspace{-0.3cm}
\end{table*}

\begin{table*}[t]
\centering
\small
\setlength{\tabcolsep}{2.9pt}{
\begin{tabular}{l|l|l|l}\hline
% \toprule[1pt]
\rowcolor{gray!20}
{\bf Method} & {\bf TS-Type} & {\bf Advantages} & {\bf Limitations}\\ \hline
Line Plot ($\S$\ref{sec.lineplot}) & UTS, MTS & matches human perception of time series & limited to MTSs with a small number of variates\\ \hline
Heatmap ($\S$\ref{sec.heatmap}) & UTS, MTS & straightforward for both UTSs and MTSs & the order of variates may affect their correlation learning\\ \hline
Spectrogram ($\S$\ref{sec.spectrogram}) & UTS & encodes the time-frequency space & limited to UTSs; needs a proper choice of window/wavelet\\ \hline
GAF ($\S$\ref{sec.gaf}) & UTS & encodes the temporal correlations in a UTS & limited to UTSs; $O(T^{2})$ time and space complexity\\ \hline% for long time series\\ \hline
% RP ($\S$\ref{sec.rp}) & UTS & flexibility in image size by tuning $m$ and $\tau$ & limited to UTSs; the pattern has a threshold-dependency\\ \hline
RP ($\S$\ref{sec.rp}) & UTS & flexibility in image size by tuning $m$ and $\tau$ & limited to UTSs; information loss after thresholding\\ \hline
% \bottomrule[1pt]
\end{tabular}}
\vspace{-0.2cm}
\caption{Summary of the five primary methods for transforming time series to images. {\bf TS-Type} denotes type of time series.}\label{tab.tsimage}
\vspace{-0.2cm}
\end{table*}

\section{Time Series To Image Transformation}\label{sec.tsimage}

% This section summarizes 5 major methods for imaging time series ($\S$\ref{sec.lineplot}-$\S$\ref{sec.rp}). We also discuss some other methods ($\S$\ref{sec.othermethod}) and how to model MTS with these methods ($\S$\ref{sec.modelmts}).
This section summarizes the methods for imaging time series ($\S$\ref{sec.lineplot}-$\S$\ref{sec.othermethod}) and their extensions to encode MTSs ($\S$\ref{sec.modelmts}).

% This section summarizes 5 major methods for transforming time series to images, including Line Plot, Heatmap, Spetrogram, GAF and RP, and several minor methods. We discuss their pros and cons and how to deal with MTS.

% This section discusses the advantages and limitations of different methods for time series to image transformation (invertible, efficiency, information preservation, MTS, long-range time series, parametric, etc.).

%\subsection{Methods}

\vspace{-0.08cm}

\subsection{Line Plot}\label{sec.lineplot}

Line Plot is a straightforward way for visualizing UTSs for human analysis ({\em e.g.}, stocks, power consumption, {\em etc.}). As illustrated by Fig. \ref{fig.tsimage}(a), the simplest approach is to draw a 2D image with x-axis representing %the time horizon
time steps and y-axis representing %the magnitude of the normalized time series.
time-wise values, %A line is used to connect all values of the series over time.
with a line connecting all values of the series over time. This image can be %represented by either three-channel pixels or single-channel pixels
either three-channel ({\em i.e.}, RGB) or single-channel as the colors may not %provide additional information
be informative %\cite{cohen2020trading,sood2021visual,jin2023classification,zhang2023insight,zhuang2024see}.
\cite{cohen2020trading,sood2021visual,jin2023classification,zhang2023insight}. ForCNN \cite{semenoglou2023image} even uses a single 8-bit integer to represent each pixel for black-white images. So far, there is no consensus on whether other graphical components, such as legend, grids and tick labels, could provide extra benefits in any task. For example, ViTST \cite{li2023time} finds these components are superfluous in a classification task, while TAMA \cite{zhuang2024see} finds grid-like auxiliary lines help enhance anomaly detection.

In addition to the regular Line Plot, MV-DTSA \cite{yang2023your} and ViTime \cite{yang2024vitime} divide an image into $h\times L$ grids, %where $h$ is the number of rows and $L$ is the number of columns,
and %introduced
define a function to map each time step of a UTS to a grid, producing a grid-like Line Plot. Also, we include methods that use Scatter Plot \cite{daswani2024plots,prithyani2024feasibility} in this category because %the only difference between a Scatter Plot and a Line Plot is whether the time-wise values are connected by lines.
a Scatter Plot resembles a Line Plot but doesn't connect %time-wise values
data points with a line. By comparing them, \cite{prithyani2024feasibility} finds a Line Plot could induce better time series classification.

For MTSs, we defer the discussion on Line Plot to $\S$\ref{sec.modelmts}.

% For MTS, some methods use the channel-independence assumption proposed in \cite{nie2023time} and represent each variate in MTS with an individual Line Plot \cite{yang2023your,yang2024vitime}. ViTST \cite{li2023time} also uses an individual Line Plot per variate, but colors different lines and assembles all plots to form a bigger image. The method in \cite{wimmer2023leveraging} plots %the time series of
% all variates in a single Line Plot and distinguish them by %use different
% types of lines ({\em e.g.}, solid, dashed, dotted, {\em etc.}). %to distinguish them.
% However, these methods only work for a small number of variates. For example, in \cite{wimmer2023leveraging}, there are only 4 variates in its financial MTSs.

%\cite{li2023time} space-costly because of blank pixels. scatter plot.

%Invertible with a numeric prediction head \cite{sood2021visual}. It fits tasks such as forecasting, imputation, etc.

\vspace{-0.08cm}

\subsection{Heatmap}\label{sec.heatmap}

As shown in Fig. \ref{fig.tsimage}(b), Heatmap is a 2D visualization of the magnitude of the values in a matrix using color. %The variation of color represents the intensity of each value. %Therefore,
It has been used to %directly
represent the matrix of an MTS, {\em i.e.}, $\mat{X} \in \mathbb{R}^{d\times T}$, as a one-channel $d\times T$ image \cite{li2022tts,yazdanbakhsh2019multivariate}. Similarly, TimEHR \cite{karami2024timehr} represents an {\em irregular} MTS, where the intervals between time steps are uneven, as a $d\times H$ Heatmap image by grouping the uneven time steps into $H$ even time bins. In \cite{zeng2021deep}, a different method is used for visualizing a 9-variate financial %time series.
MTS. It reshapes the 9 variates at each time step to a $3\times 3$ Heatmap image, and uses the sequence of images to forecast future %image
frames, achieving %time series
%MTS
time series forecasting. In contrast, VisionTS \cite{chen2024visionts} uses Heatmap to visualize UTSs. %instead.
Similar to TimesNet \cite{wu2023timesnet}, it first segments a length-$T$ UTS into $\lfloor T/P\rfloor$ length-$P$ subsequences, where $P$ is a parameter representing a periodicity of the UTS. Then the subsequences are stacked into a $P\times \lfloor T/P\rfloor$ matrix, %and duplicated 3 times to produce a 3-channel
with 3 duplicated channels, to produce a grayscale image %which serves as an
input to %a vision foundation model.
an LVM. To encode MTSs, VisionTS adopts the channel independence assumption \cite{nie2023time} and individually models each variate in an MTS.

\vspace{0.2cm}

\noindent{\bf Remark.} Heatmap can be used to visualize matrices of various forms. It is also used for matrices generated by the subsequent methods ({\em e.g.}, Spectrogram, GAF, RP) in this section. In this paper, the name Heatmap refers specifically to images that use color to visualize the (normalized) values in UTS $\mat{x}$ or MTS $\mat{X}$ without performing other transformations.

%\cite{chen2024visionts,karami2024timehr} bin version of TSH \cite{karami2024timehr}, DE and STFT \cite{naiman2024utilizing} (DE can be used for constructing RP), rearrange variates for video version of TSH \cite{zeng2021deep}.

%\vspace{0.2cm}

\subsection{Spectrogram}\label{sec.spectrogram}

A {\em spectrogram} is a visual representation of the spectrum of frequencies of a signal as it varies with time, which are extensively used for analyzing audio signals \cite{gong2021ast}. Since audio signals are a type of UTS, spectrogram can be considered as a method for imaging a UTS. As shown in Fig. \ref{fig.tsimage}(c), a common format is a 2D heatmap image with x-axis representing time steps and y-axis representing frequency, {\em a.k.a.} a time-frequency space. %The color at each point
Each pixel in the image represents the (logarithmic) amplitude of a specific frequency at a specific time point. Typical methods for %transforming a UTS to
producing a spectrogram include {\bf Short-Time Fourier Transform (STFT)} \cite{griffin1984signal}, {\bf Wavelet Transform} \cite{daubechies1990wavelet}, and {\bf Filterbank} \cite{vetterli1992wavelets}.

\vspace{0.2cm}

\noindent{\bf STFT.} %Discrete Fourier transform (DFT) can be used to represent a UTS signal %$\mat{x}=[x_{1}, ..., x_{T}]$
%$\mat{x}\in\mathbb{R}^{1\times T}$ as a sum of sinusoidal components. The output of the transform is a function of frequency $f(w)$, describing the intensity of each constituent frequency $w$ of the entire UTS. 
Discrete Fourier transform (DFT) can be used to describe the intensity $f(w)$ of each constituent frequency $w$ of a UTS signal $\mat{x}\in\mathbb{R}^{1\times T}$. However, $f(w)$ has no time dependency. It cannot provide dynamic information such as when a specific frequency appear in the UTS. STFT addresses this deficiency by sliding a window function $g(t)$ over the time steps in %the UTS,
$\mat{x}$, and computing the DFT within each window by
\begin{equation}\label{eq.stft}
\small
\begin{aligned}
f(w,\tau) = \sum_{t=1}^{T}x_{t}g(t - \tau)e^{-iwt}
\end{aligned}
\end{equation}
where $w$ is frequency, $\tau$ is the position of the window, $f(w,\tau)$ describes the intensity of frequency $w$ at time step $\tau$.

%With a proper selection of the
By selecting a proper window function $g(\cdot)$ ({\em e.g.}, Gaussian/Hamming/Bartlett window), %({\em e.g.}, Gaussian window, Hamming window, Bartlett window), %{\em etc.}),
a 2D spectrogram ({\em e.g.}, Fig. \ref{fig.tsimage}(c)) can be drawn via a heatmap on the squared values $|f(w,\tau)|^{2}$, with $w$ as the y-axis, and $\tau$ as the x-axis. For example, \cite{dixit2024vision} uses STFT based spectrogram as an input to LMMs %\hh{do you mean LVMs? check}
for time series classification.

%Fourier transform is a powerful data analysis tool that represents any complex signal as a sum of sines and cosines and transforms the signal from the time domain to the frequency domain. However, Fourier transform can only show which frequencies are present in the signal, but not when these frequencies appear. The STFT divides original signal into several parts using a sliding window to fix this problem. STFT involves a sliding window for extracting frequency components within the window.

\vspace{0.2cm}

\noindent{\bf Wavelet Transform.} %Like Fourier transform, %\hh{this paragraph needs a citation}
Continuous Wavelet Transform (CWT) uses the inner product to measure the similarity between a signal function $x(t)$ and an analyzing function. %In STFT (Eq.~\eqref{eq.stft}), the analyzing function is a windowed exponential $g(t - \tau)e^{-iwt}$.
%In CWT,
The analyzing function is a {\em wavelet} $\psi(t)$, where the typical choices include Morse wavelet, Morlet wavelet, %Daubechies wavelet, %Beylkin wavelet, 
{\em etc.} %The
CWT compares $x(t)$ to the shifted and scaled ({\em i.e.}, stretched or shrunk) versions of the wavelet, and output a CWT coefficient by
\begin{equation}\label{eq.cwt}
\small
\begin{aligned}
c(s,\tau) = \int_{-\infty}^{\infty}x(t)\frac{1}{s}\psi^{*}(\frac{t - \tau}{s})dt
\end{aligned}
\end{equation}
where $*$ denotes complex conjugate, $\tau$ is the time step to shift, and $s$ represents the scale. In practice, a discretized version of CWT in Eq.~\eqref{eq.cwt} is implemented for UTS $[x_{1}, ..., x_{T}]$.

It is noteworthy that the scale $s$ controls the frequency encoded in a wavelet -- a larger $s$ leads to a stretched wavelet with a lower frequency, and vice versa. As such, by varying $s$ and $\tau$, a 2D spectrogram ({\em e.g.}, Fig. \ref{fig.tsimage}(d)) can be drawn %, often with a heatmap
on $|c(s,\tau)|$, where $s$ is the y-axis and $\tau$ is the x-axis. Compared to STFT, which uses a fixed window size, Wavelet Transform allows variable wavelet sizes -- a larger size %region
for more precise low frequency information. 
%Usually, $s$ and $\tau$ vary dependently -- a larger $s$ leads to a stretched wavelet that shifts slowly, {\em i.e.}, a smaller $\tau$. This property %of CWT
%yields a spectrogram that balances the resolutions of frequency %$s$
%and time, %$\tau$,
%which is an advantage over the fixed time resolution in STFT.
% Thus, both of the methods in %\cite{du2020image}
% \cite{namura2024training} and \cite{zeng2023pixels} choose CWT (with Morlet wavelet) to generate the spectrogram.
Thus, the methods in \cite{du2020image,namura2024training,zeng2023pixels} choose CWT (with Morlet wavelet) to generate the spectrogram.

%A wavelet is a wave-like oscillation that has zero mean and is localized in both time and frequency space.

\vspace{0.2cm}

\noindent{\bf Filterbank.} This method %is relevant to
resembles STFT and is often used in processing audio signals. Given an audio signal, it firstly goes through a {\em pre-emphasis filter} to boost high frequencies, which helps improve the clarity of the signal. Then, STFT is applied on the signal. %with a sliding window $g(t)$ of size $k$ that shifts in a fixed stride $\tau$. %where the adjacent windows may overlap in $k$ time length.
%Finally, filterbank features are computed by applying multiple ``triangle-shaped'' filters spaced on the Mel-scale to the STFT output $f(w, \tau)$. %where Mel-scale is a method to make the filters more discriminative on lower frequencies, %than higher frequencies,
%imitating the non-linear human ear perception of sound.
Finally, multiple ``triangle-shaped'' filters spaced on a Mel-scale are applied to the STFT power spectrum $|f(w, \tau)|^{2}$ to extract frequency bands. The outcome filterbank features $\hat{f}(w, \tau)$ can be used to yield a spectrogram with $w$ as the y-axis, and $\tau$ as the x-axis.

%Filterbank was introduced in AST \cite{gong2021ast} with %$k$=25ms
Filterbank was adopted in AST \cite{gong2021ast} with 
a 25ms Hamming window that shifts every 10ms for classifying audio signals using Vision Transformer (ViT). It then becomes widely used in the follow-up works such as SSAST \cite{gong2022ssast}, MAE-AST \cite{baade2022mae}, and AST-SED \cite{li2023ast}, as summarized in Table \ref{tab.taxonomy}.



%Use MLP to predict TS directly \cite{zeng2023pixels}.

%\vspace{0.2cm}

% \vspace{0.2cm}

\subsection{Gramian Angular Field (GAF)}\label{sec.gaf}

GAF was introduced for classifying UTSs using CNNs %using %image based CNNs
by \cite{wang2015encoding}. It was then extended %with an extension
to an imputation task in \cite{wang2015imaging}. Similarly, \cite{barra2020deep} applied GAF for financial time series forecasting.

Given a UTS $\mat{x}\in\mathbb{R}^{1\times T}$, %$[x_{1}, ..., x_{T}]$,
the first step %before GAF
is to rescale each $x_{t}$ to a value $\tilde{x}_{t}$ %in the interval of
within $[0, 1]$ (or $[-1, 1]$). %by min-max normalization.
This range enables mapping $\tilde{x}_{t}$ to polar coordinates by $\phi_{t}=\text{arccos}(\tilde{x}_{i})$, with a radius $r=t/N$ encoding the time stamp, where $N$ is a constant factor to regularize the span of the polar coordinates. %system. Then,
Two types of GAF, Gramian Sum Angular Field (GASF) and Gramian Difference Angular Field (GADF) are defined as
\begin{equation}\label{eq.gaf}
\small
\begin{aligned}
&\text{GASF:}~~\text{cos}(\phi_{t} + \phi_{t'})=x_{t}x_{t'} - \sqrt{1 - x_{t}^{2}}\sqrt{1 - x_{t'}^{2}}\\
&\text{GADF:}~~\text{sin}(\phi_{t} - \phi_{t'})=x_{t'}\sqrt{1 - x_{t}^{2}} - x_{t}\sqrt{1 - x_{t'}^{2}}
\end{aligned}
\end{equation}
which exploits the pairwise temporal correlations in the UTS. Thus, the outcome is a $T\times T$ matrix $\mat{G}$ with $\mat{G}_{t,t'}$ specified by either type in Eq.~\eqref{eq.gaf}. A GAF image is a heatmap on $\mat{G}$ with both axes representing time, as illustrated by Fig. \ref{fig.tsimage}(e).

% Invertible.

% \vspace{0.2cm}

\subsection{Recurrence Plot (RP)}\label{sec.rp}

%RP \cite{eckmann1987recurrence} is a method to encode a UTS into an image that aims to capture the periodic patterns in the UTS by using its reconstructed {\em phase space}. The phase space of a UTS $[x_{1}, ..., x_{T}]$ can be reconstructed by {\em time delay embedding}, which is a set of new vectors $\mat{v}_{1}$, ..., $\mat{v}_{l}$ with

RP \cite{eckmann1987recurrence} encodes a UTS into an image that captures its periodic patterns by using its reconstructed {\em phase space}. The phase space of %a UTS %$[x_{1}, ..., x_{T}]$
$\mat{x}\in\mathbb{R}^{1\times T}$ can be reconstructed by {\em time delay embedding} -- a set of new vectors $\mat{v}_{1}$, ..., $\mat{v}_{l}$ with
\begin{equation}\label{eq.de}
\small
\begin{aligned}
\mat{v}_{t}=[x_{t}, x_{t+\tau}, x_{t+2\tau}, ..., x_{t+(m-1)\tau}]\in\mathbb{R}^{m\tau},~~~1\le t \le l
\end{aligned}
\end{equation}
where $\tau$ is the time delay, $m$ is the dimension of the phase space, both %of which
are hyperparameters. Hence, $l=T-(m-1)\tau$. With vectors $\mat{v}_{1}$, ..., $\mat{v}_{l}$, an RP image %is constructed by measuring
measures their pairwise distances, results in an $l\times l$ image whose element
\begin{equation}\label{eq.rp}
\small
\begin{aligned}
\text{RP}_{i,j}=\Theta(\varepsilon - \|\mat{v}_{i} - \mat{v}_{j}\|),~~~1\le i,j\le l
\end{aligned}
\end{equation}
where $\Theta(\cdot)$ is the Heaviside step function, $\varepsilon$ is a threshold, and $\|\cdot\|$ is a norm function such as $\ell_{2}$ norm. Eq.~\eqref{eq.rp} %states RP produces a heatmap image on a binary matrix with $\text{RP}_{i,j}=1$ if $\mat{v}_{i}$ and $\mat{v}_{j}$ are sufficiently similar.
generates a binary matrix with $\text{RP}_{i,j}=1$ if $\mat{v}_{i}$ and $\mat{v}_{j}$ are sufficiently similar, producing a black-white image ({\em e.g.}, Fig. \ref{fig.tsimage}(f)).% ({\em e.g.}, a periodic pattern).

An advantage of RP is its flexibility in image size by tuning $m$ and $\tau$. Thus it has been used for time series classification %\cite{cao2021image},
\cite{silva2013time,hatami2018classification}, forecasting \cite{li2020forecasting}, anomaly detection \cite{lin2024hierarchical} and %feature-wise
explanation \cite{kim2024cafo}. Moreover, the method in \cite{hatami2018classification}, and similarly in HCR-AdaAD \cite{lin2024hierarchical}, omit the thresholding in Eq.~\eqref{eq.rp} and uses $\|\mat{v}_{i} - \mat{v}_{j}\|$ to produce continuously valued images %in a classification task
to avoid information loss.


% \vspace{0.2cm}

\subsection{Other Methods}\label{sec.othermethod}

%There are some less commonly used methods. For example, in
Additionally, %there are some peripheral methods. %In addition to GAF,
\cite{wang2015encoding} introduces Markov Transition Field (MTF) for imaging a UTS. %$\mat{x}\in\mathbb{R}^{1\times T}$. 
%MTF first assigns each $x_{t}$ to one of $Q$ quantile bins, then builds a $Q\times Q$ Markov transition matrix $\mat{M}$ {\em s.t.} $\mat{M}_{i,j}$ represents the frequency %with which
%of the case when a point $x_{t}$ in the $i$-th bin is followed by a point $x_{t'}$ in the $j$-th bin, {\em i.e.}, $t=t'+1$. Matrix $\mat{M}$ serves as the input of a heatmap image.
MTF is a matrix $\mat{M}\in\mathbb{R}^{Q\times Q}$ encoding the transition probabilities over time segments, where $Q$ is the number of segments. %Moreover,
ImagenTime \cite{naiman2024utilizing} stacks the delay embeddings $\mat{v}_{1}$, ..., $\mat{v}_{l}$ in Eq.~\eqref{eq.de} to an $l\times m\tau$ matrix for visualizing UTSs. %It also uses a variant of STFT.
% The method in \cite{homenda2024time} introduces five different 2D images by counting, rearranging, replicating the values in a UTS. 
MSCRED \cite{zhang2019deep} uses heatmaps on the $d\times d$ correlation matrices of MTSs with $d$ variates for anomaly detection. 
Furthermore, some methods use a mixture of imaging methods by stacking different transformations. \cite{wang2015imaging} stacks GASF, GADF, MTF to a 3-channel image. %Similarly,
FIRTS \cite{costa2024fusion} builds a 3-channel image by stacking GASF, MTF and RP. %the GASF, MTF, RP representations of each UTS.
%\cite{jin2023classification} combines Line Plot with Constant-Q Transform (CQT) \cite{brown1991calculation}, a method related to wavelet transform ($\S$\ref{sec.spectrogram}), to generate 2-channel images.
The mixture methods encode a UTS with multiple views and were found more robust than single-view images in these works for %time series
classification tasks.

\subsection{How to Model MTS}\label{sec.modelmts}

In the above methods, Heatmap ($\S$\ref{sec.heatmap}) can be %directly
used to visualize the %2D
variate-time matrices, $\mat{X}$, of MTSs ({\em e.g.}, Fig. \ref{fig.structure}(b)), where correlated variates %are better to
should be spatially close to each other. Line Plot ($\S$\ref{sec.lineplot}) can be used to visualize MTSs by plotting all variates in the same image \cite{wimmer2023leveraging,daswani2024plots} or combining all univariate images to compose a bigger %1-channel
image \cite {li2023time}, but these methods only work for a small number of variates. Spectrogram ($\S$\ref{sec.spectrogram}), GAF ($\S$\ref{sec.gaf}), and RP ($\S$\ref{sec.rp}) were designed specifically for UTSs. For these methods and Line Plot, which are not straightforward %for MTS transformation,
in imaging MTSs, the general approaches %to use them %for MTS
include using channel independence assumption to model each variate individually \cite{nie2023time}, %like VisionTS \cite{chen2024visionts},
or stacking the images of $d$ variates to form a $d$-channel image %as did by
\cite{naiman2024utilizing,kim2024cafo}. %\cite{prithyani2024feasibility,naiman2024utilizing,kim2024cafo}.
However, the latter does not fit some vision models pre-trained on RGB images which requires 3-channel inputs (more discussions are deferred to $\S$\ref{sec.processing}).

\vspace{0.2cm}

\noindent{\bf Remark.} As a summary, Table \ref{tab.tsimage} recaps the salient advantages and limitations of the five primary imaging methods that are introduced in this section.

% \hh{can we have a table (e.g., rows are different imaging methods and columns are a few desirable propoerties) or a short paragraph to discuss/summarize/compare the strenths and weakness of different imaging methods for ts? This might bring some structure/comprehension to this section (as opposed to, e.g., some reviewer might complain that what we do here is a laundry list)}

\section{Imaged Time Series Modeling}\label{sec.model}

With image representations, time series analysis can be readily performed with vision models. This section discusses such solutions from %traditional vision models %($\S$\ref{sec.cnns})
%to the recent large vision models %($\S$\ref{sec.lvms})
%and large multimodal models.% ($\S$\ref{sec.lmms}).
the traditional models to the SOTA models.

\begin{figure*}[!t]
\centering
\includegraphics[width=0.9\textwidth]{fig/fig_2.pdf}
% \vspace{-1em}
\caption{An illustration of different modeling strategies on imaged time series in (a)(b)(c) and task-specific heads in (d).}\label{fig.models}
\vspace{-0.2cm}
\end{figure*}

\subsection{Conventional Vision Models}\label{sec.cnns}

%Similar to
Following traditional %methods on
image classification, \cite{silva2013time} applies a K-NN classifier on the RPs of time series, \cite{cohen2020trading} applies an ensemble of fundamental classifiers such as %linear regression, SVM, Ada Boost, {\em etc.}
SVM and AdaBoost on the Line Plots %images
for time series classification. As an image encoder, %a typical encoder, %of images,
CNNs have been %extensively
widely used for learning image representations. %\cite{he2016deep}.
Different from using 1D CNNs on sequences %UTS or MTS
\cite{bai2018empirical}, %regular
2D or 3D CNNs can be applied on imaged time series as shown in Fig. \ref{fig.models}(a). %to learn time series representations by encoding their image transformations.
For example, %standard
regular CNNs have been used on Spectrograms \cite{du2020image}, tiled CNNs have been used on GAF images \cite{wang2015encoding,wang2015imaging}, dilated CNNs have been used on Heatmap images \cite{yazdanbakhsh2019multivariate}. More frequently, ResNet \cite{he2016deep}, Inception-v1 \cite{szegedy2015going}, and VGG-Net \cite{simonyan2014very} have been used on Line Plots \cite{jin2023classification,semenoglou2023image}, Heatmap images \cite{zeng2021deep}, RP images \cite{li2020forecasting,kim2024cafo}, GAF images \cite{barra2020deep,kaewrakmuk2024multi}, 
% Heatmaps \cite{zeng2021deep}, RPs \cite{li2020forecasting,kim2024cafo}, GAFs \cite{barra2020deep,kaewrakmuk2024multi},
and even a mixture of GAF, MTF and RP images \cite{costa2024fusion}. In particular, for time series generation tasks, %a diffusion model with U-Nets \cite{naiman2024utilizing} and GAN frameworks of CNNs \cite{li2022tts,karami2024timehr} have also been explored.%investigated.
GAN frameworks of CNNs \cite{li2022tts,karami2024timehr} and a diffusion model with U-Nets \cite{naiman2024utilizing} have also been explored.

Due to their small to medium sizes, these models are often trained from scratch using task-specific training data. %per task using the task's training set. %of time series images.
Meanwhile, fine-tuning {\em pre-trained vision models}  %such as those pre-trained on ImageNet, %\cite{deng2009imagenet}, 
have already been found promising in cross-modality knowledge transfer for time series anomaly detection \cite{namura2024training}, forecasting \cite{li2020forecasting} and classification \cite{jin2023classification}.

% \cite{li2020forecasting} uses ImageNet pretrained CNNs.

\subsection{Large Vision Models (LVMs)}\label{sec.lvms}

Vision Transformer (ViT) \cite{dosovitskiy2021image} has %given birth to
inspired the development of %some
modern LVMs %large vision models (LVMs)
such as %DeiT \cite{touvron2021training}, 
Swin \cite{liu2021swin}, BEiT \cite{bao2022beit}, and MAE \cite{he2022masked}. %Given an input image, ViT splits it
As Fig. \ref{fig.models}(b) shows, ViT splits an %input
image into {\em patches} of fixed size, then embeds each patch and augments it with a positional embedding. The %resulting
vectors of patches are processed by a Transformer %encoder
as if they were token embeddings. Compared to CNNs, ViTs are less data-efficient, but have higher capacity. %Consequently,
Thus, %the
{\em pre-trained} ViTs have been explored for modeling %the images of time series.
imaged time series. For example, AST \cite{gong2021ast} fine-tunes DeiT \cite{touvron2021training} on the filterbank spetrogram of audios %signals
for classification tasks and finds %using
ImageNet-pretrained DeiT is remarkably effective in knowledge transfer. The fine-tuning paradigm has also been %similarly
adopted in \cite{zeng2023pixels,li2023time} but with different pre-trained models %initializations
such as Swin by \cite{li2023time}. 
VisionTS \cite{chen2024visionts} %explains
attributes %the superiority of LVMs
LVMs' superiority over LLMs in knowledge transfer %over LLMs %as an outcome of
to the small gap between the pre-trained images and imaged time series. %the patterns learned from the large-scale pre-trained images and the patterns in the images of time series.
It %also
finds that with one-epoch fine-tuning, MAE becomes the SOTA time series forecasters on %many
some benchmark datasets.

Similar to %build
time series foundation models %\cite{das2024decoder,goswami2024moment,ansari2024chronos,shi2024time}, %such as TimesFM \cite{das2024decoder}, MOMENT \cite{goswami2024moment}, Chronos \cite{ansari2024chronos} and Time-MoE \cite{shi2024time},
such as TimesFM \cite{das2024decoder}, %and MOMENT \cite{goswami2024moment}, 
there are some initial efforts in pre-training ViT architectures with imaged time series. Following AST, SSAST \cite{gong2022ssast} introduced a %joint discriminative and generative
%masked spectrogram patch prediction self-supervised learning framework
masked spectrogram patch prediction framework for pre-training ViT on a large dataset -- AudioSet-2M. Then it becomes a backbone of some follow-up works such as AST-SED \cite{li2023ast} for sound event detection. %To be effective for UTSs,
For UTSs, ViTime \cite{yang2024vitime} generates a large set of Line Plots of synthetic UTSs for pre-training ViT, which was found superior over TimesFM in zero-shot forecasting tasks on benchmark datasets.

\subsection{Large Multimodal Models (LMMs)}\label{sec.lmms}

%As Large Multimodal Models (LMMs)
As LMMs %are getting
get growing attentions, some %of the
notable LMMs, such as LLaVA \cite{liu2023visual}, Gemini \cite{team2023gemini}, GPT-4o \cite{achiam2023gpt} and Claude-3 \cite{anthropic2024claude}, have been explored to consolidate the power of LLMs %on time series
and LVMs in time series analysis. 
Since LMMs support multimodal input via prompts, methods in this thread typically prompt LMMs with the textual and imaged representations of time series, %textual representation of time series and their %image transformations, transformed images,
%then instruct LMMs
and instructions on what tasks to perform ({\em e.g.}, Fig. \ref{fig.models}(c)).

InsightMiner \cite{zhang2023insight} is a pioneer work that uses the LLaVA architecture to generate %textual descriptions about
texts describing the trend of each input UTS. It extracts the trend of a UTS by Seasonal-Trend decomposition, encodes the Line Plot of the trend, and concatenates the embedding of the Line Plot with the embeddings of a textual instruction, which includes a sequence of numbers representing the UTS, {\em e.g.}, ``[1.1, 1.7, ..., 0.3]''. The concatenated embeddings are taken by a language model for generating trend descriptions. %It also fine-tunes a few layers with the generated texts to align LLaVA checkpoints with time series domain.
Similarly, \cite{prithyani2024feasibility} adopts the LLaVA architecture, but for MTS classification. An MTS is encoded by %a sequence of
the visual %token
embeddings of the stacked Line Plots of all variates. %meanwhile
%The method also stacks
%The time series of all variate are also stacked in a prompt % of all variates in a prompt
The matrix of the MTS is also verbalized in a prompt 
as the textual modality. %By manipulating token embeddings,
By integrating token embeddings, both %of these %works propose to
methods fine-tune some layers of the LMMs with some synthetic data.

Moreover, zero-shot and in-context learning performance of several commercial LMMs have been evaluated for audio classification \cite{dixit2024vision}, anomaly detection \cite{zhuang2024see}, and some synthetic tasks \cite{daswani2024plots}, where the image %({\em e.g.}, spectrograms, Line Plots)
and textual representations of a query %UTS or MTS
time series are integrated into a prompt. For in-context learning, these methods inject the images of a few example time series and their labels ({\em e.g.}, classes) %({\em e.g.}, classes, normal status)
into an instruction to prompt LMMs for assisting the prediction of the query time series.

\subsection{Task-Specific Heads}\label{sec.task}

%With the image embedding of a time series, the next step is to produce its prediction.
For classification tasks, most of the methods in Table \ref{tab.taxonomy} adopt a fully connected (FC) layer or multilayer perceptron (MLP) to transform an embedding into a probability distribution over all classes. For forecasting tasks, there are two approaches: (1) using a $d_{e}\times W$ MLP/FC layer to directly predict (from the $d_{e}$-dimensional embedding) the time series values in a future time window of size $W$ \cite{li2020forecasting,semenoglou2023image}; (2) predicting the pixel values that represent the future part of the time series and then recovering the time series from the predicted image \cite{yang2023your,chen2024visionts,yang2024vitime} ($\S$\ref{sec.processing} discusses the recovery methods). Imputation and generation tasks resemble forecasting %in the sense of predicting
as they also predict time series values. Thus approach (2) has been used for imputation \cite{wang2015imaging} and generation \cite{naiman2024utilizing,karami2024timehr}. %LMMs have been used for classification, text generation, and anomaly detection. For these tasks,
When using LMMs for classification, text generation, and anomaly detection, most of the methods prompt LMMs to produce the desired outputs in textual answers, circumventing task-specific heads \cite{zhang2023insight,dixit2024vision,zhuang2024see}.

%Forecasting: MLP, FC to predict numerical values using embeddings. Imputation of images (TSH). Classification: MLP, FC using embeddings.

\section{Pre-Processing and Post-Processing}\label{sec.processing}

To be successful in using vision models, some subtle design desiderata %to be considered
include {\bf time series normalization}, {\bf image alignment} and {\bf time series recovery}.

\vspace{0.2cm}

\noindent{\bf Time Series Normalization.} Vision models are usually trained on %images after Gaussian normalization (GN).
standardized images. To be aligned, the images introduced in $\S$\ref{sec.tsimage} should be normalized with a controlled mean and standard deviation, as did by \cite{gong2021ast} on spectrograms. In particular, as Heatmap is built on raw time series values, the commonly used Instance Normalization (IN) \cite{kim2022reversible} can be applied on the time series as suggested by VisionTS \cite{chen2024visionts} since IN share similar merits as Standardization. %although min-max normalization was used by \cite{karami2024timehr,zeng2021deep}.
Using Line Plot requires a proper range of y-axis. In addition to rescaling time series %by min-max or GN
\cite{zhuang2024see}, ViTST \cite{li2023time} introduced several methods to remove extreme values from the plot. GAF requires min-max normalization on its input, as it transforms time series values withtin $[0, 1]$ to polar coordinates ({\em i.e.}, arccos). In contrast, input to RP is usually normalization-free as an $\ell_{2}$ norm is involved in Eq.~\eqref{eq.rp} before thresholding.%for a comparison with a threshold.

\vspace{0.2cm}

\noindent{\bf Image Alignment.} When using pre-trained models, it is imperative to fit the image size to the input requirement of the models. This is especially true for Transformer based models as they use a fixed number of positional embeddings to encode the spacial information of image patches. For 3-channel RGB images such as Line Plot, it is straightforward to meet a pre-defined size by adjusting the resolution when producing the image. For images built upon matrices such as Heatmap, Spectrogram, GAF, RP, the number of channels and matrix size need adjustment. For the channels, one method is to duplicate a matrix to 3 channels \cite{chen2024visionts}, another way is to average the weights of the 3-channel patch embedding layer into a 1-channel layer \cite{gong2021ast}. For the image size, bilinear interpolation is a common method to resize input images \cite{chen2024visionts}. Alternatively, AST \cite{gong2021ast} %use cut and bilinear interpolation on
resizes the positional embeddings instead of the images to fit the model to a desired input size. However, the interpolation in these methods may either alter the time series or the spacial information in positional embeddings.

% single-channel (UTS), RGB channel (UTS), duplicate channels (UTS), multi-channel (MTS).

%Bilinear interpolation.

%Correlated variates are better to be spatially close to each other.

%\subsection{Pre-training}

\vspace{0.2cm}

\noindent{\bf Time Series Recovery.} As stated in $\S$\ref{sec.task}, tasks such as forecasting, imputation and generation requires predicting time series values. For models that predict pixel values of images, post-processing involves recovering time series from the predicted images. Recovery from Line Plots is tricky, it requires locating pixels that %correspond to
represent time series and mapping them back to the original values. This can be done by manipulating a grid-like Line Plot as introduced in \cite{yang2023your,yang2024vitime}, which has a recovery function. In contrast, recovery from Heatmap is straightforward as it directly stores the predicted time series values \cite{zeng2021deep,chen2024visionts}. Spectrogram is underexplored in these tasks and it remains open on how to recover time series from it. The existing work \cite{zeng2023pixels} uses Spectrogram for forecasting only with an MLP head that directly predicts time series. %predicts time series values.
GAF supports accurate recovery by an inverse mapping from polar coordinates to normalized time series \cite{wang2015imaging}. However, RP lost time series information during thresholding (Eq.~\ref{eq.rp}), thus may not fit recovery-demanded tasks without using an {\em ad-hoc} prediction head.


% Line Plot was regarded as matrices with rows and columns for mapping in \cite{sood2021visual}.


%\section{Tasks and Time Series Recovery}

%\subsection{Task-Specific Head}

% \subsection{Time Series Recovery}



% \input{rimasugli.tex}

\end{document}