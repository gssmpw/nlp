% Optional math commands from https://github.com/goodfeli/dlbook_notation.
\usepackage{times}
\usepackage{epsfig}
\usepackage{graphicx}
\usepackage{amsmath}
\usepackage{amssymb}

\usepackage{multirow}
\usepackage{tabularx}
\usepackage{booktabs}
\usepackage{longtable}
\usepackage{subcaption}
\usepackage{colortbl}

% Include other packages here, before hyperref.
\usepackage[accsupp]{axessibility}  % Improves PDF readability for those with disabilities.
\usepackage[font=footnotesize,labelfont=bf]{caption}
\usepackage[belowskip=-5pt,aboveskip=3pt]{caption} % re-arrange tables on last page, select
% Remove spacing after figures and tables.
% other options: \usepackage[font=small,labelfont=bf]{caption}

\usepackage{dsfont}
\usepackage{pifont}
\usepackage{url}
% \usepackage[tight]{subfigure}
\usepackage{xcolor}
\usepackage{amsthm}
%\usepackage[ruled,vlined]{algorithm2e}
\usepackage{adjustbox}
\usepackage{comment}
\usepackage{algorithm} 
%\usepackage{tablefootnote}

\usepackage[utf8]{inputenc} % allow utf-8 input
\usepackage[T1]{fontenc}    % use 8-bit T1 fonts
\usepackage{url}            % simple URL typesetting
%\usepackage{booktabs}       % professional-quality tables
\usepackage{amsfonts}       % blackboard math symbols
\usepackage{nicefrac}       % compact symbols for 1/2, etc.
\usepackage{microtype}      % microtypography
\usepackage{enumitem}
\usepackage{blindtext}% for dummy text only
\usepackage{sidecap}
\usepackage{wrapfig}


\definecolor{citecolor}{RGB}{65,105,225}
% \usepackage[pagebackref=true,breaklinks=true,letterpaper=true,colorlinks,bookmarks=false,citecolor=citecolor]{hyperref}

\newtheorem{mydef}{Definition}
\usepackage{dsfont}

%-----------------for equation definition --------------------
\def\argmin{\operatornamewithlimits{arg\,min}}
\def\argmax{\operatornamewithlimits{arg\,max}}
\def\minimize{\operatornamewithlimits{minimize}}
\def\maximize{\operatornamewithlimits{maximize}}

%----------------for bib fontsize stuff----------------------
% \usepackage[numbers]{natbib}
% \setcitestyle{authoryear,open={((},close={))}}
% \renewcommand{\bibfont}{\scriptsize}
%-----------------for table stuff----------------------------
\newcommand{\tabincell}[2]{\begin{tabular}{@{}#1@{}}#2\end{tabular}}

%-----------------for theorem stuff--------------------------
%\newtheorem{theorem}{Theorem}[section]
\newtheorem{theorem}{Theorem}
\newtheorem{lemma}[theorem]{Lemma}
\newtheorem{proposition}[theorem]{Proposition}
\newtheorem{corollary}[theorem]{Corollary}

%-----------------for color stuff---------------------------
\definecolor{dg}{rgb}{0,0.694,0.298}
\definecolor{purple}{rgb}{0.4,0.176,0.569}
\definecolor{royalblue}{RGB}{65,105,225}
\usepackage{pifont}% http://ctan.org/pkg/pifont
\newcommand{\cmark}{\textcolor{dg}{\ding{52}}}%
\newcommand{\xmark}{\textcolor{red}{\ding{56}}}%
\newcommand{\red}[1]{\textbf{\textcolor{red}{#1}}}

%-----for figure & equation reference----------------------
\newcommand{\figref}[1]{Fig.~\ref{#1}}
\newcommand{\reqref}[1]{Eq.~(\ref{#1})}
\newcommand{\secref}[1]{Sec.~\ref{#1}}
\newcommand{\tableref}[1]{Table~\ref{#1}}

%------------ for basic abs definition --------------------
\makeatletter
\DeclareRobustCommand\onedot{\futurelet\@let@token\@onedot}
\def\@onedot{\ifx\@let@token.\else.\null\fi\xspace}
\def\eg{\emph{e.g}\onedot} \def\Eg{\emph{E.g}\onedot}
\def\ie{\emph{i.e}\onedot} \def\Ie{\emph{I.e}\onedot}
\def\cf{\emph{c.f}\onedot} \def\Cf{\emph{C.f}\onedot}
\def\etc{\emph{etc}\onedot} \def\vs{\emph{vs}\onedot}
\def\wrt{w.r.t\onedot} \def\dof{d.o.f\onedot}
\def\etal{\emph{et al}\onedot}
\makeatother

%------------ for author comments ------------------------
\definecolor{americanrose}{rgb}{1.0, 0.01, 0.24}
\newcommand{\rui}[1]{\textbf{\textcolor{blue}{Rui: #1}}}
\newcommand{\jiang}[1]{\textbf{\textcolor{magenta}{Jiang: #1}}}
\newcommand{\error}[1]{\textbf{\textcolor{americanrose}{error: #1}}}
\newcommand{\qing}[1]{\textbf{\textcolor{purple}{Qing: #1}}}
\newcommand{\zqy}[1]{\textbf{\textcolor{magenta}{Zhao: #1}}}

%------------ for rank highlights -----------------------
\newcommand{\topone}[1]{\textbf{\textcolor{red}{#1}}}
\newcommand{\toptwo}[1]{\textbf{\textcolor{purple}{#1}}}

%------------ for revision highlights ------------------
\newcommand{\revised}{\textcolor[rgb]{0,0,1}}
\newcommand{\response}{\textcolor[rgb]{1,0,0.5}}
\newcommand{\highlight}[1]{\textbf{#1.}}

% % Support for easy cross-referencing
% \usepackage[capitalize]{cleveref}
% \crefname{section}{Sec.}{Secs.}
% \Crefname{section}{Section}{Sections}
% \Crefname{table}{Table}{Tables}
% \crefname{table}{Tab.}{Tabs.}

% if you use cleveref..
% \usepackage[capitalize,noabbrev]{cleveref}

%%%%%%%%%%%%%%%%%%%%%%%%%%%%%%%%
% THEOREMS
%%%%%%%%%%%%%%%%%%%%%%%%%%%%%%%%
\theoremstyle{plain}
\theoremstyle{definition}
\newtheorem{definition}[theorem]{Definition}
\newtheorem{assumption}[theorem]{Assumption}
\theoremstyle{remark}
\newtheorem{remark}[theorem]{Remark}

% Todonotes is useful during development; simply uncomment the next line
%    and comment out the line below the next line to turn off comments
%\usepackage[disable,textsize=tiny]{todonotes}
\usepackage[textsize=tiny]{todonotes}