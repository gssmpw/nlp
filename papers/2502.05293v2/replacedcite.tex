\section{Related Work}
\label{sec:related-work}

%\noindent
\textbf{Parallel runtime systems} 
%OpenMP, Cilk, libfork, tbb, mpi, charm++
such as OpenMP ____, Charm++ ____, and Swift/T ____ use concurrent queues to share data between threads or processes.
Charm++ goes further and supports lock-free queues to demonstrate performance improvements ____.
Regarding locks, researchers have investigated contention management in thread-safe MPI libraries ____ and the use of abort locking ____ to improve performance.
We aim to achieve better performance with finer granularity and task decomposition than these solutions.

Some classical with-locks task-based runtimes (e.g., OmpSs____, PaRSEC____, StarPU____) have made efforts to improve data locality. 
For example, OmpSs and XKaapi____ rely on work-stealing for load balancing.
XKaapi also provides a lower bound on the number of data accesses required by the scheduler____.
Legion____ allows users to specify locality explicitly using data regions, and provides a data mapping strategy to ensure that data are only moved when needed.
Some of StarPU's scheduling strategies focus on data reuse and task stealing to increase the performance of linear algebra applications____.
These solutions are orthogonal to our research: We focus on removing the synchronization cost of barriers and thus cannot use similar techniques that require regular synchronization and updating of current system state.

%\noindent
\textbf{Dynamic Load balancing:}
% Theoretical paper on random NA-WS, hierarchical work stealing, scabale work stealing, adws
Several papers have proposed load
balancing mechanisms____.
Quintin et al. proposed hierarchical work stealing to exploit data
locality to achieve speedup over classical work stealing algorithms____. 
Shiina et al. addressed the issue of data locality by making scheduling deterministic____.
In order to balance load efficiently, these all relied on synchronization mechanisms.
In our work, we explore lock-less techniques to achieve comparable dynamic load balancing mechanisms.

\textbf{Lock-less runtime systems:}
XQueue____ demonstrated the benefits of a lock-less, task-oriented runtime in OpenMP.  
Recent work on XQueue____ introduced work-stealing, but focused on simply redirecting a newly spawned task to another thread. Results indicated a need to develop more sophisticated strategies and also consider NUMA-awareness, as we explore here.
Among other things, in this paper, we integrate XQueue into GNU-OpenMP, introduce a tree barrier, and propose sophisticated NUMA-aware work-stealing strategies, which dynamically migrate tasks from a victim to a thief thread.


% DISCUSSION AND FUTURE WORK