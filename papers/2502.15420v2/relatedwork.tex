\section{Related Work}
\label{sec:related}
%%%%%%%%%%%%%%%%%%%%%%%%%%%%%%%%%%%%%%%%

% %%%%%%%%%%%%%%%%%%%%%%%
% \subsection{MEV}
% \label{ssec:mev-rw}
% %%%%%%%%%%%%%%%%%%%%%%%
\textbf{MEV Auctions.}~\cite{10634354} study various strategic interactions and auction setups of block builders with proposers. They evaluate how access to MEV opportunities and improved relay connectivity impact bidding performance.~\cite{10271857} propose an Ethereum gas auction model using the First Price Sealed-Bid Auction (FPSBA) between different bots and miners.

\noindent\textbf{MEV redistributions.}\cite{chionas2023gets} model the MEV setting as a dynamical system with a fraction of MEV going to the miner as a dynamical variable updated with every time step. The miners and builders are assumed to be one entity, with the rest of the MEV returning to transaction creators. ~\cite{mazorra2023towards} discuss rebates in the context of liquidity providers in constant function market makers and discuss the auction between searcher and builders with the assumption of perfect MEV oracle that can compute the MEV extracted given the state of the blockchain and a new block of transactions.~\cite{chitra2022improving} proposes MEV redistribution as a dynamical system in which lending and staking portfolios of block proposer are chosen as a parameter that determines how much of the MEV extracted in a block is redistributed to staking.

\noindent\textbf{Game Theory and Blockchains} Researchers explored various game theoretic concepts in blockchains. E.g., the authors of \cite{roughgarden, damle2024designing,damle2024no}  use concepts from mechanism design to design transaction fee mechanisms and fairness. \cite{jain2021we,chen2024game} study scalability issues in blockchains through game theory.\cite{siddiqui2020bitcoinf} discusses on achieving fairness for Bitcoin in a transaction fee-only model. \cite{jain2022tiramisu} studies the equilibrium behavior of the miners. In this work, we explore the use of cooperative game theory in matchmaking. \cite{damle2021fasten} proposes a fair and secure distributed voting system that utilizes smart contracts to ensure that votes remain anonymous and are not tampered with during the voting process. \cite{faltings2021orthos} explores an AI framework designed for trustworthy data acquisition to enhance the reliability of data collection processes in multi-agent systems. Furthermore, \cite{srivastava2024decent} proposes a mechanism that promotes decentralization through block reward systems, aiming to enhance fairness and efficiency in blockchain networks.

%%%%%%%%%%%%%%%%%%%%%%%%%%%%%%%%%%%%%%%%