\documentclass[lettersize,journal]{IEEEtran}
%%% REVIEW
\newcommand{\tocite}{{\color{red}CITE} }
\newcommand{\toref}{{\color{red}REF} }

%%% LOGO
\newcommand{\usc}{\raisebox{-1pt}{\includegraphics[height=0.8em]{figures/usc_logo.png}}}
\newcommand{\vuam}{\raisebox{-1pt}{\includegraphics[height=0.8em]{figures/vu_logo.png}}}

%%% SIGNS and SYMBOLS
\newcommand{\grad}{\texttt{grad-CROP}}
\newcommand{\att}{\texttt{att-CROP}}
\newcommand{\seg}{\texttt{seg}}
\newcommand{\clip}{\texttt{clip-CROP}}
\newcommand{\sam}{\texttt{sam-CROP}}
\newcommand{\yolo}{\texttt{yolo-CROP}}
\newcommand{\hc}{\texttt{human-CROP}}
\newcommand{\zsvqa}{\texttt{ZSVQA}}
\newcommand{\vic}{\textbf{ViCrop}}
\newcommand{\xmark}{\text{\ding{55}}}
\newcommand{\cmark}{\text{\ding{51}}}
\newcommand{\success}{\texttt{\color{green} \cmark}}
\newcommand{\failure}{\texttt{\color{red} \xmark}}
\newcommand{\rel}{\texttt{rel-att}}
\newcommand{\gra}{\texttt{grad-att}}
\newcommand{\pgra}{\texttt{pure-grad}}
\newcommand{\relh}{\texttt{rel-att$^h$}}
\newcommand{\grah}{\texttt{grad-att$^h$}}
\newcommand{\pgrah}{\texttt{pure-grad$^h$}}


%%% Text Abb.
\makeatletter
\DeclareRobustCommand\onedot{\futurelet\@let@token\@onedot}
\def\@onedot{\ifx\@let@token.\else.\null\fi\xspace}

\def\aka{\emph{a.k.a}\onedot} \def\Eg{\emph{E.g}\onedot}
\def\eg{\emph{e.g}\onedot} \def\Eg{\emph{E.g}\onedot}
\def\ie{\emph{i.e}\onedot} \def\Ie{\emph{I.e}\onedot}
\def\cf{\emph{c.f}\onedot} \def\Cf{\emph{C.f}\onedot}
\def\etc{\emph{etc}\onedot} \def\vs{\emph{vs}\onedot}
\def\wrt{w.r.t\onedot} \def\dof{d.o.f\onedot}
\def\etal{\emph{et al}\onedot}
\makeatletter



\definecolor{myred}{HTML}{FF8577}
\definecolor{mygreen}{HTML}{0FA958}
\definecolor{myblue}{HTML}{1982C4}
\definecolor{codegreen}{rgb}{0,0.5,0}
\definecolor{codegray}{rgb}{0.5,0.5,0.5}
\definecolor{codepurple}{rgb}{0.07,0,0.53}
\definecolor{codered}{RGB}{189,41,0}
\definecolor{codecomment}{RGB}{153,153,153}
\definecolor{backcolour}{rgb}{0.96,0.96,0.96}
\definecolor{royalblue}{rgb}{0.0, 0.14, 0.4}
\definecolor{egyptianblue}{rgb}{0.06, 0.2, 0.65}
\definecolor{royalazure}{rgb}{0.0, 0.22, 0.66}
\definecolor{portlandorange}{rgb}{1.0, 0.35, 0.21}
\definecolor{sienna}{RGB}{183,105,68}
\definecolor{saddlebrown}{RGB}{139,69,19}
\definecolor{mediumbrown}{RGB}{83,41,11}
\definecolor{darkbrown}{RGB}{58,28,7}
\hypersetup{
    colorlinks=true,
    linkcolor=sienna,
    urlcolor=royalblue,
    citecolor=royalblue,
}

\ifCLASSINFOpdf
            \else
                  \fi

\hyphenation{op-tical net-works semi-conduc-tor}

\begin{document}
 
\title{Securing 5G Bootstrapping: A Two-Layer\\ IBS Authentication Protocol}
 
\author{Yilu~Dong,
        Rouzbeh~Behnia,
        Attila~A.~Yavuz,
        and~Syed~Rafiul~Hussain,~\IEEEmembership{Member,~IEEE}}

\maketitle

\begin{abstract}
The lack of authentication during the initial bootstrapping phase between cellular devices and base stations allows attackers to deploy fake base stations and send malicious messages to the devices. These attacks have been a long-existing problem in cellular networks, enabling adversaries to launch denial-of-service (DoS), information leakage, and location-tracking attacks. While some defense mechanisms are introduced in 5G, (e.g., encrypting user identifiers to mitigate IMSI catchers), the initial communication between devices and base stations remains unauthenticated, leaving a critical security gap. To address this, we propose \scheme{}, a novel and efficient two-layer identity-based signature scheme designed for seamless integration with existing cellular protocols. We implement \scheme{} on an open-source 5G stack and conduct a comprehensive performance evaluation against alternative solutions. Compared to the state-of-the-art \texttt{Schnorr-HIBS}, \scheme{} reduces attack surfaces, enables fine-grained lawful interception, and achieves 2x speed in verification, making it a practical solution for securing 5G base station authentication.
\end{abstract}

\begin{IEEEkeywords}
Cellular Networks, Fake Base Station Attacks, Identity-Based Signatures, Authentication Protocol.  
\end{IEEEkeywords}

\IEEEpeerreviewmaketitle

\hfill  
 
\hfill January 30, 2025

\section{Introduction}
\documentclass[../main.tex]{subfiles}
\graphicspath{{../images/}}
\makeatletter
\def\input@path{{../images/}}
\makeatother
\begin{document}
\section{Introduction}
\begin{figure}
\centering
\begin{tikzpicture}
\node[inner sep=0pt] (ws) at (0, 0) {
\includegraphics[height=.4\textwidth, trim={10cm 0 10cm 0},clip]{world_space.png}};
\node[inner sep=0pt] (cs) at (6,0) {\includegraphics[height=.4\textwidth, trim={10cm 1cm 10cm 4cm},clip]{conf_space.png}};
\end{tikzpicture}
\vspace{-5pt}
\label{fig:pbrm_intro}
\caption{\textbf{Left}: Shows world space obstacles as grey spheres. Robots start and goal configuration is colored red and green, respectively. Configurations along the computed path are colored transparent blue. \textbf{Right:} Mapped world space scenario to configuration space. Obstacle region is the grey mesh. Red spheres are collision-free regions computed by the neural SCDF. The optimized shortest path in the convex corridor is the blue curve.}
\vspace{-25pt}
\end{figure}
Motion planning is the problem of finding a collision-free trajectory that connects a given start and goal configuration. The planning takes place in the configuration space of the robot. For single body robots, like mobile robots or drones, the configuration space and the world space are usually the same. This simplifies the planning, since explicit obstacle representations are available which enables geometrical tools like separating hyperplanes, smallest distance to obstacles etc., to be used when designing motion planning algorithms. For multi-body robots like manipulators, the situation is completely different. The world space obstacles are usually mapped to non-convex regions, and to make the problem even harder, the mapping is usually not known. Forming explicit representations of the obstacle region in the configuration space is usually too expensive or intractable. Despite all of this, sampling based planners are used with great success, which mainly is due to their use of implicit representations of the obstacle region. The basic idea is to construct a graph in the configuration space that covers and connects the collision-free region. From this graph, a path can be extracted that connects a given start and goal configuration. The approach is computationally expensive, since the graph is constructed with the smallest geometrical building block available, points, which represents a collision-check. Furthermore, the extracted paths from the graph are non-smooth and jagged due to the stochastic nature of the approach. This adds an additional post-processing step to the process, where the paths are shortcutted and smoothened, before the path can be used for tracking. Clearly a lot of time is invested to form this graph and produce smooth paths. Thus, if the obstacles start to move, then all of this work is done in no use, since all points that make up this graph need to be re-verified, which is simply too time consuming to be done in real time.
\\\\
In this work, we want to address the existing drawbacks of the sampling based planners. Our main contribution is an improved motion planner where each vertex in the graph covers a collision-free region in the form of a sphere instead of a point and where the edges are formed with neighboring intersecting spheres. This representation has the advantage of instead of returning piecewise linear paths, returning a sequence of overlapping spheres, i.e. a convex corridor, that connects a given start and goal configuration, illustrated in Figure \ref{fig:pbrm_intro}. This convex corridor allows us to use convex optimization to produce smooth trajectories, instead of computationally expensive post-processing methods. The representation further allows us to estimate the coverage of the collision-free space, which gives us awareness and feedback in the offline roadmap construction phase. Finally, our representation is simple to adapt to moving obstacles, simply requery for the new radii and recheck for intersections. 
\\\\
The spherical collision-free regions are formed using a signed distance function (SDF), which is a function that returns the smallest distance from an arbitrary point to the boundary of an obstacle. As the name implies, the distance is signed, thus if the point is inside the obstacle it is negative otherwise positive. If the distance is positive, a sphere with radius equal to the distance is guaranteed to cover a collision-free region. Using an SDF in motion planning is not new, but what is novel about our approach is that we express the distance in the configuration space instead of the world space and by doing so allows us to form these convex collision-free regions. We refer to the resulting SDF as a signed configuration distance function (SCDF). Computing an SCDF analytically is non-trivial, our approach is therefore to parameterize the SCDF with a deep neural network and learn the mapping by supervised learning. Our resulting neural SCDF can compute distances for different parameter values of obstacle shapes and we also show how multiple distances can be combined, thus making our approach flexible.
\section{Related work}
Motion planning algorithms can roughly be divided into three families, grid-based, sampling based and optimization based methods. Grid-based methods (GBM) discretize the planning space from which a graph is then compiled. A standard search method is A$^\star$ \citep{a_star}, which is classified as an \textit{informed} search method, since it employs a heuristic function to speed up the search. A$^\star$ guarantees to return an optimal path at the level of discretization used. GBMs usually discretize the planning space by a regular lattice and this limits the GBMs to problems with low dimensionality due to the curse of dimensionality. Thus, GBMs are usually limited to single-body robots where the degrees of freedom (DOF) are low. To overcome the inherent scaling problem with the GBMs, stochastic methods are usually used for multi-body robots. These methods are termed as sampling-based methods (SBM) and core members within this family are the rapidly-exploring random trees (RRT) \citep{rrt} and the probabilistic roadmap (PRM) \citep{prm}. RRT grows a tree from the start configuration and explores the collision-free region in a rapid way until it is able to connect to the goal region. RRT is usually improved by bi-directional planning \citep{rrt_connect}, i.e. an additional tree is grown from the goal configuration and the trees are tested for connection after any tree has been expanded. RRT is a single-query method, thus it searches for a path from scratch each time it is queried. Contrary to this, PRM is a multi-query method, which solves for multiple queries without starting from scratch. PRM does this by creating a roadmap (graph) that covers the collision-free space as an offline step. The graph is then used to solve for multiple queries. PRMs are used in cases where the environment does not change since the extra offline step is too computationally costly and needs to be re-done if the environment is changed. In our work, we address this inherent issue by using a different roadmap representation. Our vertices in the graph cover a collision-free region in the form of spheres and we form the edges by checking for intersecting spheres. If something in the environment changes, we recompute the spheres radii and recheck the intersections, without relying on collision detection. We use a trained neural network to compute the sphere radius, therefore querying for the radius can be done fast, hence our representation enables the PRM for dynamic environments.
\\\\
In the recent decades, optimization based methods (OBM) \citep{chomp, schulman, itomp, stomp} have been introduced as an alternative to SBM for multi-body robots. Like the SBM, the OBMs scale well to higher dimensional problems and produce smoother motion. It is common to use a SDF in the optimization since it is a smooth function, thus enabling gradient-based methods. However, the standard way of expressing the SDF is in world space. The distance therefore needs to be mapped to the configuration space by the forward kinematics. This mapping makes the optimization problem a non-linear program (NLP), which is computationally expensive to solve. Recently, a different approach has been proposed. In \cite{mp_gcs} motion planning is formulated as a convex optimization problem by using the graph of convex sets framework \citep{gcs}. The underlying idea is to decompose the collision-free space into intersecting convex sets from which a convex optimization problem is formulated. In cases where an explicit representation of the obstacles in the configuration space exists, like for single-body robots, creating collision-free convex regions can be done fast \citep{iris}. For multi-body robots, this is non-trivial. Existing work does this successfully \citep{iris_nlp, iris_c} by an optimization based approach, but the methods are still too time consuming to be used in the presence of moving obstacles. Our approach is instead to use deep learning to learn an SDF expressed in the configuration space. With this, we can query for shortest distances to the collision boundary, which allows us to expand spherical regions which are collision-free. Our approach is fast and therefore enables our suggested roadmap planner to be used in dynamic environments.
\\\\
Recent research has focused on learning collision detection \citep{fk_kernel_distance, diffco, graphdistnet} by predicting the signed distance between the robot links and the surrounding obstacles in the world space. The learned SDF is used in trajectory optimization but since the distance is expressed in the world space, the problem becomes an NLP and therefore takes a long time to solve. We take a novel approach and suggest to instead express the signed distance in the configuration space. This allows us to improve the PRM at the same time as it enables convex optimization for trajectory optimization, which runs faster and is more reliable than NLP solvers. In \cite{cspf} a learned signed distance function in the configuration space is proposed similar to our approach. However, their approach is restricted to point cloud representations, while we propose to represent the obstacles as parameterized geometric shapes, e.g. spheres. Furthermore, we also show how to use our learned SCDF to improve an existing roadmap planner.
\section{Problem formulation}
A robot is located in the world space, $\W \subset \R^3 $. The unique location of the robot is given by its configuration $\q \in \C$, where $\C$ is the configuration space. The set of points covered by the robots bodies at a certain configuration is expressed as $\B(\q) \subset \W$. The robot is surrounded by $\NrObst$ obstacles $\O = \bigcup_{i=1}^{\NrObst} \O_i$, where  $\O_i \subset \W$. The representation of the obstacle in the configuration space is the set $\C\O_i = \{\q \in \C \: |\: \B(\q) \cap \O_i \neq \emptyset \}$. The obstacle space is formed as $\Co = \bigcup_{i=1}^{\NrObst} \C \O_i$. The complement is referred to as the free space, $\Cf = \C \setminus \Co$. The path planning problem is a tuple, ($\Cf$, $\qStart$, $\qGoal$), where we want to connect a query pair, consisting of a start, $\qStart$, and goal configuration, $\qGoal$, with a geometric path, $\q(s): [0, 1] \mapsto \Cf$, such that $\q(0)=\qStart$ and $\q(1)=\qGoal$, or report correctly when such a path does not exist.
\end{document}


\section{Background}
\label{Background}
\section{Basic Background: Supervised Learning and the PAC Model}
\label{sec:background}

At this point almost everyone has heard of machine learning (ML). Anyone likely to stumble upon this article will have also heard of its most influential special case, supervised learning, and those theoretically inclined will also be familiar with the PAC model. Nonetheless, I will set the stage by  recapping the basics.

\subsection{Basics of Supervised Learning}%Let's set the stage in any case

\emph{Supervised Learning} is the task of ``coming up'' with a function $f: \X \to \Y$ to ``explain'' or ``fit'' a sequence of input/output examples   $(x_1,y_1), \ldots, (x_n,y_n)$, with $x_i \in \X$ and $y_i \in \Y$.  Here $\X$ is a \emph{data domain} consisting of \emph{datapoints} $x \in \X$, $\Y$ is a \emph{label set} consisting of \emph{labels} $y \in \Y$, and the sequence $(x_1,y_1),\ldots,(x_n,y_n)$ is the \emph{training data} consisting of \emph{labeled examples (a.k.a. samples)}~$(x_i,y_i)$.  I~will refer to the chosen function $f$ as a \emph{predictor}, and to $n$ as the \emph{sample size}. A \emph{learning algorithm} takes as input training data, and outputs (some representation of) a predictor $f \in \Y^\X$.\footnote{Note that this describes the usual \emph{batch}, a.k.a.~\emph{offline}, setting of supervised learning. I do not discuss other paradigms such as online or active learning in this article.} 



Success in supervised learning is defined as \emph{generalization} to  future examples: For a typical \emph{test example}  $(x_{\tst},y_{\tst})$, the predicted label $y'_{\tst}=f(x_{\tst})$ should ``equal'' $y_{\tst}$, perhaps approximately. We usually assume the test example is drawn from the same  ``source'' as the training data  --- commonly, i.i.d.~from the same distribution. The quality of the prediction is quantified by $\ell(y'_{\tst},y_{\tst})$, where $\ell:~\Y~\times~\Y \to \RR_{\geq 0}$ is a \emph{loss function} chosen as part of the problem definition. Common loss functions include the 0-1 loss $\ell_{0-1}(y',y) = [y' \neq y]$ for \emph{classification} problems,\footnote{The notation $[P]$ denotes $1$ when predicate $P$ is true, and denotes $0$ when $P$ is false.} as well as the absolute loss $|y'-y|$ or squared loss $(y'-y)^2$ for \emph{regression problems} featuring $\Y  \sse \RR$.

Nontrivial generalization properties are typically only possible if one assumes something about the data.\footnote{The need for such an assumption is formalized by the  \emph{no free lunch theorems} of supervised learning \cite{wolpert_connection_1992,wolpert_lack_1996,schaffer_conservation_1994}.} The Bayesian approach to  machine learning, common in many applications, assumes some parametric form for the distribution generating the data, and postulates a prior on the parameters. This is not the approach I will take in this article. Instead, I will focus on the frequentist --- and some would say ``worst-case'' or ``adversarial'' ---  approach that is common in the computational learning theory community, embodied by the PAC model. Here we assume that the (training and test) data can be explained, perhaps approximately, by a function in some ``simple enough to learn'' class of functions $\H \sse \Y^\X$, often called the \emph{hypotheses}. Equivalently, we  seek a predictor which explains the unseen data roughly  as well as the best hypothesis $h^* \in \H$, whether or not we assume that $h^*$ itself provides a perfect explanation.



 \paragraph{Common Algorithmic Templates.} Perhaps the best known general-purpose supervised learning algorithm is \emph{empirical risk minimization (ERM)}, which chooses as its predictor a hypothesis $f \in \H$ minimizing $\frac{1}{n} \sum_{i=1}^n \ell(f(x_i),y_i)$ --- a quantity called the \emph{training error}, \emph{empirical error}, or \emph{empirical risk} of $f$. %\footnote{When multiple hypotheses minimize the empirical risk, we assume ERM breaks ties arbitrarily.}
A common template for generalizing ERM involves adding a \emph{regularization term} $\psi(f)$ to the  objective function, typically chosen to measure some notion of ``hypothesis complexity.'' An algorithm instantiating this template is known as a \emph{structural risk minimizer (SRM)}, and chooses as its predictor the hypothesis $f \in \H$ minimizing the \emph{structural risk} $\frac{1}{n} \sum_{i=1}^n \ell(f(x_i),y_i) + \psi(f)$. Other well-known algorithms, such as gradient descent and its variations,  can frequently be interpreted as approximate implementations of ERM or SRM.


\paragraph{Proper vs Improper Learning.} A learning algorithm is said to be \emph{proper} if its predictor $f$ is always chosen from the hypothesis class, i.e., $f \in \H$, otherwise it is said to be \emph{improper}. ERM  is an example of a proper learning algorithm, as are SRM algorithms of the form described above.  In the \emph{proper regime} of learning, algorithms are required to be proper. This article will be concerned with the more flexible \emph{improper regime} (a.k.a \emph{representation-independent learning}), where no such constraint is placed on the learner. In other words, all we care about is predictive power at test time, rather than any insights derived from the functional form or representation of the predictor~itself.


\subsection{The PAC Model}
A standard mathematical setup for evaluation of supervised learning algorithms, at least in the theoretical computer science community, is Valiant's \emph{Probably Approximately Correct (PAC) model} of learning (see e.g.~\cite{kearns_introduction_1994,mohri_foundations_2018}). Here, we assume there is an unknown distribution $\D$ on $\X \times \Y$ from which training and test data are  drawn.  Specifically, the labeled datapoints of the training set  $(x_1,y_1), \ldots, (x_n,y_n)$, as well as the test data  $(x_\tst,y_\tst)$, are i.i.d.~from $\D$. Often it is assumed that $\D$ lies in some class of distributions of interest. The \emph{true expected loss}, or simply \emph{loss}, of a predictor $f: \X \to \Y$ is the expected loss it incurs on draws from $\D$, written $L_\D(f) = \Ex_{(x,y) \sim \D} \ell(f(x),y)$.


There are two main ``settings'' in PAC learning. The  \emph{realizable setting} only requires that the data be perfectly explained by some hypothesis in $\H$. More generally, the \emph{agnostic setting} makes no assumption relating the data to the hypotheses, but shifts the goalposts as necessary to allow nontrivial guarantees: the expected loss at test time is evaluated only ``relative'' to that of the best hypothesis $h^* \in \H$. There are other settings which make more nuanced assumptions, such as $\D$ being of a particular parametric form or its support living in some (unknown) lower-dimensional space, etc. I will mostly discuss the realizable and agnostic settings in this article, those being the simplest and most studied from a theoretical perspective. %TODO:We will briefly discuss other settings in Section ??

The PAC model demands high probability guarantees of learners, in the worst case over distributions of interest. Consider first the realizable setting, where $\D$ is such that $\min_{h \in \H} L_{\D}(h) = 0$. A PAC learner has \emph{error} $\epsilon=\epsilon(n)$ and \emph{confidence} $\delta=\delta(n)$ if, when training data consists of $n$ i.i.d~samples from a realizable distribution $\D$, it produces a predictor $f$  satisfying $L_\D(f) \leq \epsilon$ with probability at least $1-\delta$. In the agnostic setting, where $\D$ can be arbitrary, we require $L_\D(f) - \min_{h \in \H} L_\D(h) \leq \epsilon$ with probability $1-\delta$.

In both the realizable and agnostic settings, we look for PAC learners with small $\epsilon$ and $\delta$ as a function of the sample size $n$. An equivalent perspective looks at the sample complexity $m(\epsilon,\delta)$, which is the minimum sample size which guarantees error  at most $\epsilon$ with probability at least $1-\delta$. We say a problem is \emph{PAC learnable} if its PAC sample complexity is finite whenever $\epsilon,\delta > 0$.

For most PAC learning problems, learnability and sample complexity are characterized in terms of a  ``dimension'' of the hypothesis class. Most prominently this is the \emph{VC dimension} for binary classification, the \emph{fat shattering dimension} for agnostic regression, and the \emph{DS dimension} for multiclass classification (see \cite{anthony_neural_1999,daniely_optimal_2014,brukhim_characterization_2022}). Treatment of these is beyond the scope of this article. The unfamiliar reader need not worry, however,  as dimensions will feature only tangentially in our~discussion.




%\paragraph{Learning settings: Realizable, Agnostic, etc.} In learning theory, evaluating a supervised learning algorithm requires specifying a data model and an objective. We will leave the details of the data model flexible for now, to allow for both the PAC model and the adversarial transductive model. Nonetheless we will describe two variations, which we call ``settings'', which cut across different models. The  \emph{realizable setting}  requires only that the data be perfectly explained by some hypothesis $h \in \H$ --- i.e., there exists a hypothesis which is guaranteed to suffer a loss of $0$ on training and test data. The performance of the learning algorithm is its expected loss at test time for some ``worst case'' realizable instance. More generally, the \emph{agnostic setting} makes no assumption relating the data to the hypotheses, but shifts the goalposts as necessary to allow nontrivial guarantees: the expected loss at test time is evaluated only ``relative'' to that of the best hypothesis $h^* \in \H$, again for some ``worst case'' instance. There are other settings which make more nuanced assumptions about the data, such as it is drawn from a distribution of a particular parametric form, or that it lives in some (unknown) lower-dimensional space, etc. We will mostly discuss the realizable and agnostic settings, those being the simplest and most studied from a theoretical perspective.




%%% Local Variables:
%%% mode: latex
%%% TeX-master: "learning_matching"
%%% End:


\section{\texorpdfstring{Characterization of Attacks Enabled\\ by Fake Base Station}{Characterization of Attacks Enabled by Fake Base Station}}
\label{False Base Stations}
Fake base stations (FBS) have been demonstrated to be feasible in real-world scenarios using Commercial-Off-The-Shelf (COTS) hardware and open-source cellular software stacks~\cite{strobel2007imsi, paget2010practical, kune2012location, shaik2015practical}. To operate a fake base station, an attacker configures their radio to transmit signals at a higher strength than legitimate base stations, enticing users to connect to it instead. Figure~\ref{FBS_a} illustrates a typical fake base station setup used for conducting off-path attacks, while Figure~\ref{FBS_b} depicts its configuration for executing Man-in-the-Middle (MitM) relay attacks.

\begin{figure}[t]
 \centering
    \begin{subfigure}{1\linewidth}
    \includegraphics[width=\linewidth]{images/FBS_a.pdf}
        \caption{Setup for off-path attacks.}
        \label{FBS_a}
    \end{subfigure}
    
    \begin{subfigure}{1\linewidth}
    \includegraphics[width=\linewidth]{images/FBS_b.pdf}
        \caption{Setup for man-in-the-middle attacks.}
        \label{FBS_b}
    \end{subfigure}
        \caption{Common FBS configurations for carrying out attacks.}
    \end{figure}

To mitigate fake base station attacks, cellular protocol specifications have introduced enhancements across multiple generations, including 3G, 4G LTE, and 5G. A notable advancement in 5G is the introduction of the Subscription Concealed Identifier (SUCI), which protects against IMSI catchers. SUCI is an encrypted identifier used in the UE registration procedure, designed to conceal the Subscription Permanent Identifier (SUPI)---commonly known as the IMSI (International Mobile Subscriber Identity) in earlier generations. By rotating encryption keys, SUCI can change over time, making it significantly harder for IMSI catchers to track users. However, SUCIs are still transmitted over-the-air, which means an IMSI catcher could temporarily track a user if the SUCI value remains unchanged. Moreover, some network operators do not enable SUCI encryption in their systems, further undermining its effectiveness~\cite{nie2022measuring}.

\begin{table*}[ht]
\setlength\tabcolsep{4pt}
\renewcommand{\arraystretch}{0.6}
\fontsize{8}{6}\selectfont
\newcolumntype{P}[1]{>{\centering\arraybackslash}p{#1}}
\centering
\begin{tabular} {| c | P{2.8cm} | P{7.6cm} |}
\hline
\textbf{Attack} & \textbf{Attack Category} & \textbf{Impact} \\ \hline

Send RRC or NAS Reject messages \cite{shaik2015practical, hussain2018lteinspector, hussain20195greasoner, 3GPP:33.809} & DoS; Downgrade; Battery Depletion; & Denial of services; force UE to downgrade to older radio technology; Increase in power consumption for UE \\ \hline

Replay \texttt{RRC\_Resume\_Request}~\cite{3GPP:33.809} & DoS & Denial of services \\ \hline

\texttt{Authentication\_Request} with separation bit 0~\cite{rashid2024state} & DoS & Denial of services \\ \hline

Manipulate Self Organizing Networks (SON)~\cite{shaik2018impact} & DoS; Battery Depletion & Call dropping; Increase in power consumption for UE; Increased handovers and signaling load; Legitimate base station blacklisted \\ \hline

Modify \texttt{UE\_Capability\_Information}~\cite{shaik2019new} & Downgrade & Denial of some services; Lower data rate; Downgrade to 2G/3G \\ \hline

Authentication Relay Attack~\cite{hussain2018lteinspector} & Information Leak; DoS & Complete or selective DoS; Location history poisoning; Network profiling \\ \hline

5G AKA Bypass~\cite{5gbasechecker} & Information Leak & Monitor and manipulate user traffic; Provide Internet access; Phishing \\ \hline

NAS \texttt{Security\_Mode\_Command} replay~\cite{5gbasechecker} & Location Tracking; Fingerprinting & Check if the UE in the range of base station; Fingerprinting UE model \\ \hline

SUCI-Catcher~\cite{chlosta20215g} & Location Tracking; Information Leak &  Obtain user identifier; Track user movement\\ \hline

IMEI-Catcher~\cite{park2022doltest} & Location Tracking; Information Leak &  Obtain user identifier; Track user movement \\ \hline

Lullaby~\cite{hussain20195greasoner} & DoS; Battery Depletion &  Move UE to idle state; Increase in power consumption for UE \\ \hline

NAS counter reset~\cite{hussain20195greasoner} & DoS; Battery Depletion & Force UE reconnect; Increase in power consumption for UE \\ \hline

Authentication Sync Failure Attack~\cite{hussain20195greasoner} & DoS; Battery Depletion & Force UE reconnect; Increase in power consumption for UE \\ \hline

\texttt{Counter\_Check} Fingerprinting~\cite{park2022doltest, 5gbasechecker} & Fingerprinting & Fingerprinting UE model \\ \hline

\texttt{Authentication\_Request} Fingerprinting~\cite{rashid2024state} & Fingerprinting & Fingerprinting UE model \\ \hline

RRC \texttt{Security\_Mode\_Command} Bypass~\cite{kim2019touching} & Information leak & Monitor and manipulate user traffic  \\ \hline
\end{tabular}

\caption{Attacks enabled by fake base stations in 4G \& 5G cellular networks. }
\label{fbs_impact}
\end{table*}

Despite some advancements, these defenses fail to address the root cause of fake base station attacks: the lack of authentication for base stations during the connection bootstrapping process. Consequently, fake base station attacks remain feasible, even in 5G networks. To better understand the causes and implications of these attacks, we analyze fake base station attacks found in recent studies. A summary of these attacks and their impact is provided in Table~\ref{fbs_impact}.

\noindent \textbf{Denial-of-Service.}
Denial-of-Service (DoS) attacks are among the most prevalent tactics used by an FBS (Fake Base Station) attacker. By sending a reject message as defined in the specifications \cite{shaik2015practical, hussain2018lteinspector, hussain20195greasoner, 3GPP:33.809, rashid2024state}, the attacker can prevent the User Equipment (UE) from connecting to the network. Additionally, the attacker can manipulate the order or fields of control-plane messages \cite{3GPP:33.809, shaik2018impact, hussain2018lteinspector, hussain20195greasoner} to achieve a similar effect. As a result, the affected user cannot connect to a legitimate base station, rendering them unable to receive SMS, phone calls, or access the Internet. 

\noindent \textbf{Battery Depletion.}
Most cellular devices rely on battery power, making them vulnerable to energy-draining attacks. An FBS attacker can disrupt the UE's connection to the base station, forcing it into a reconnection loop that rapidly depletes the battery \cite{shaik2015practical, hussain2018lteinspector, hussain20195greasoner, 3GPP:33.809, shaik2018impact}. The frequent cell selection and registration procedures consume significant power, ultimately preventing the user from operating their device.

\noindent \textbf{Downgrade.}
In a downgrade, or bidding-down attack, the attacker forces the UE to connect to an older radio generation (e.g., 2G, 3G, LTE). This can be achieved by leveraging specific reject messages (e.g., \texttt{5GS\_Services\_Not\_Allowed}) \cite{shaik2015practical, hussain2018lteinspector, hussain20195greasoner, 3GPP:33.809} or manipulating fields in control-plane messages \cite{shaik2018impact}. Consequently, the user loses the benefits of modern radio technologies, such as faster speeds and stronger cryptographic protections. Since older generations like 2G and 3G employ weaker cryptography, attackers can exploit these connections to send fake SMS or execute other malicious activities.

\noindent \textbf{Information Leak.}
Certain FBS attacks can lead to sensitive information leakage \cite{hussain2018lteinspector, 5gbasechecker, chlosta20215g, park2022doltest, kim2019touching}. For instance, SUCI-Catcher \cite{chlosta20215g} and IMEI-Catcher \cite{park2022doltest} get the identifier of the user or device and can track the user's location. The Authentication Relay Attack \cite{hussain2018lteinspector}, 5G AKA Bypass \cite{5gbasechecker}, and RRC \texttt{Security\_Mode\_Command} Bypass attack can relay or bypass the authentication procedure between the base station and the UE, which causes the user traffic unencrypted. The attacker can monitor and manipulate the traffic and even provide Internet access to the user \cite{5gbasechecker}. Thus, the attacker can hijack users into phishing websites and perform more complicated attacks. 

\noindent \textbf{Fingerprinting.}
By analyzing UE responses to identical messages, attackers can infer device characteristics such as the baseband vendor or software version \cite{5gbasechecker, park2022doltest, rashid2024state}. With this information, attackers can execute targeted attacks tailored to specific devices or software implementations. 

\noindent \textbf{Location Tracking.}
Many attacks \cite{5gbasechecker, chlosta20215g, park2022doltest} enable location tracking of a specific UE. IMEI-Catcher \cite{park2022doltest} captures the permanent identifier of the device, IMEI (International Mobile Equipment Identity), and SUCI-Catcher \cite{chlosta20215g} captures SUCI, which is an identifier of the user. With an FBS network, the attacker can track user location if the same identifier is used elsewhere. Additionally, the NAS \texttt{Security\_Mode\_Command} replay attack \cite{5gbasechecker, hussain2018lteinspector} can verify the presence of a UE within the attacker's range by replaying previously successful \texttt{Security\_Mode\_Command} messages.

To mitigate these threats, ensuring UE authentication of base stations is crucial. This paper aims to address this gap and propose solutions to enhance security against FBS attacks.

\section{Overview of Our Solution}
\label{Overview}
\begin{figure*}[t]
\begin{center}
\includegraphics[width=.85\linewidth]{fig_overview_v3.pdf}
\end{center}
\caption{
FastAtlas Overview: In each frame, we compute charts spanning fully or partially visible triangles (a), determine texture space bounding boxes for the visible portions of the view-space projections of each chart, and tightly pack these boxes into atlases (b, here $2K \times 2K$). We simultaneously bijectively parameterize and shade the charts into their atlas boxes, obtaining high quality texture space shading (c), and use this shading to render the shaded frames (d).}
\label{fig:overview}
\label{fig:alg_overview}
\end{figure*}

\section{Overview}
\label{sec:overview}
Our work has two core contributions: a real-time, GPU-based algorithm for tight packing of general parameterized charts into compact atlases; and a real-time TSS method that
utilizes this packing.  

\paragraph*{FastAtlas Packing.}
FastAtlas runs entirely on the GPU as a series of compute shaders. It takes the bounding boxes of parameterized charts as input, and packs them into an atlas (Fig~\ref{fig:overview}b, Sec.~\ref{sec:pack}). As such, the only input it requires are the dimensions of the bounding boxes.
Its outputs are deterministic; identical input charts are packed into identical atlases. This is critical for TSS and similar applications, as it ensures that consecutive frames taken from the same camera view have the same shading. Even minute shading differences across such frames can cause sampling jitter, leading to undesirable flicker \cite{baker2012rock}. 
While prior methods such as \cite{mueller2018shading,hladky2019tessellated,hladky2021snakebinning,Neff2022MSA} cap the dimensions of the charts that can be packed as-is for a given atlas size, and scale down all charts that exceed these dimensions, we scale all charts by the same factor, and do so only when strictly necessary to achieve packing success (Figs~\ref{fig:atlas},~\ref{fig:sas_issues}). 

\paragraph*{TSS using FastAtlas.}
Our end-to-end TSS atlas generation method combines the packing method above with a novel approach for computing seamless per-frame charts. 
We define our charts as the connected components of the visible surfaces in each frame (Fig.~\ref{fig:overview}a), and efficiently compute them using a parallel union-find algorithm (Sec.~\ref{sec:visible}). Since the boundaries of these charts coincide with the contours of the rendered surface, they are {\em invisible} to the viewer. This approach 
eliminates the artifacts caused by shading discontinuities along visible seams (Fig.~\ref{fig:seams}). 

\begin{parWithWrapFigure}
\begin{wrapfigure}{l}{.27\columnwidth}%
\includegraphics[width=\linewidth]{fig_inset_view_plane.pdf}%
\end{wrapfigure}
We bijectively parametrize the {\em visible portions} of our charts by projecting them to view space (inset). This maps a constant number of texels to each pixel in the final rendered output, evenly distributing residual undersampling error across all image pixels. While conceptually straightforward, efficiently parameterizing charts containing partially visible triangles using viewspace projection is non-trivial, as the visible portions may no longer be triangular (e.g. green triangle in the inset); applying naive projection to triangles with vertices behind the camera may produce ill-posed results. Clipping triangles before projection is both computationally expensive and significantly complicates downstream operations. We avoid explicit clipping by observing that all that is required for atlas packing is the dimensions of, potentially conservative, bounding boxes of these projected visible portions. We compute such bounding boxes without explicit chart clipping by adapting a conservative screen coverage estimator \shortcite{Blinn:CalculatingScreenCoverage} (Sec.~\ref{sec:box}). We then pack the computed boxes using FastAtlas. 
\end{parWithWrapFigure}

Finally, we shade the visible portion of each chart into its corresponding atlas bounding box (Fig~\ref{fig:overview}c). 
The resulting texture is then used during rasterization as a standard texture map (Fig. ~\ref{fig:overview}d). 
Our framework is compatible with all existing approaches for texture space shading, including forward shading, raytraced illumination, or deferred shading in texture space \cite{baker:2016}. In the examples shown, we use the standard forward shading based rendering pipeline included in the G3D Innovation Engine \cite{G3D17}, a commercial grade renderer.


\section{New Efficient 2-Layer Identity-Based Signature Scheme (\scheme{})}\label{sec:scheme}
\label{Crypto Scheme}
In this section, we provide the general definition and discussion of our newly proposed scheme, the Efficient 2-layer Identity-Based Signature Scheme (\scheme{}).

In PKI-based schemes, the security of cryptographic primitives (e.g., digital signatures) hinges on the authenticity of public keys. In conventional systems, this is achieved by digital certificates or certificate chains. 
For instance, to verify a digital signature, the verifier must confirm the authenticity of the signer's public key by ensuring the validity of the associated certificates. However, the communication and computation overhead introduced by certificates might not be tolerable in some applications (mobile devices operating in low-bandwidth environments). To address this, in identity-based cryptography, the user's public key is derived from their publicly available information (e.g., IP address).  
Existing efficient identity-based signature schemes (e.g., \cite{singla2021look}) are primarily based on the Schnorr signature \cite{Schnorr91}. 
Despite their elegant design, their inherent design and the key generation process (which derives keys from the Schnorr signature) can result in less efficient verification algorithms. 

With the efficiency and security requirements of 5G networks in mind, we present a new Efficient 2-layer Identity-Based Signature (\scheme{}).  \scheme{}, presented in  Algorithm \ref{alg:IBS}, offers highly efficient signing and verification, ensures high resiliency by avoiding a single point of failure, and supports fine-grained lawful interception.  
This is achieved by deriving a new identity-based signature from the highly efficient certificate-based scheme, ARIS \cite{ARIS}. The efficiency of ARIS is due to the ability to convert costly exponentiation operations to a few point additions by utilizing the homomorphic property of the underlying one-way function. This results in computation efficiency in both signing and verification algorithms. 

As depicted in Algorithm \ref{alg:IBS}, after the parameter selection (similar to \cite{ARIS,Tachyon}), the Setup algorithm computes the $t$ public key elements $mpk=\{Z_i\}_{i=1}^t$ and publishes the public parameters. We note that parameters $t$ and $k$ are related to the k-combinatorial problem \cite{Tachyon,ARIS}, i.e., ${t \choose k}\geq 2^\kappa $ (for security parameter $\kappa$)  and play an important role in providing storage and computation overhead trade-off.  For instance, a larger $t$ results in larger keys but more efficient signing and verification, as it allows for a smaller $k$.   During the extract algorithm, by harnessing the scheme in \cite{ARIS}, the PKG computes the user's key pair ($sk_U,{C}_U$) based on the provided identity $U$.   
After computing the user keys and considering their structure, the signing and verification processes can be carried out in a manner similar to that in \cite{Schnorr91}.
 
\newcommand{\algrule}[1][.2pt]{\par\vskip.3\baselineskip\hrule height #1\par\vskip.3\baselineskip}

\begin{algorithm}\caption{$\scheme{}$}\label{alg:IBS}
\small
$(msk,params)\gets\setup(1^\kappa)$
\algrule[0.5pt]
\begin{algorithmic}[1]
 
\item Given $\kappa$, select $p,q$, $msk \Ra \mathbb{Z}_p$ and  $t,k\Ra \mathbb{N}$ where   ${t \choose k}\geq 2^\kappa $
\item  Compute $z_i \gets \PRF_{msk}(i)$  and $Z_i \gets z_iP \mod q$ $\mathbf{for}$  $i= \{1,\dots,t\}$ and set $\mathbf{Z}\gets \{Z_i\}_{i=1}^t$
\item  Output $msk$ and $params=(mpk,p,q,t,k)$, where $mpk=\mathbf{Z}$
 
\end{algorithmic}
\algrule[0.5pt]

 $(\sk_U,{C}_U)\gets\sgnextract(msk,U)$
\algrule[0.5pt]

\begin{algorithmic}[1]

\item Compute $u \gets \PRF_{msk}(U)$ and ${C}_{U}\gets u P \mod q$ 
\item Compute $\{j_1\dots, j_k\}\gets \h_1(U,C_{U})$ where each $|j_i| = |t|$

 \item Compute $x_{U}\gets   \sum_{i=1}^{k} z_{j_i} + u \mod p$  
  \item Output $(\sk_U  = x_{U}$, ${C}_U$) 
\end{algorithmic}
\algrule[0.5pt]

$\sigma_{m,U}\gets\sign(m,\sk_U)$
\algrule[0.5pt]
\begin{algorithmic}[1]
\item Select $r \Ra \mathbb{Z}_p$ and compute $  h\gets \h_2(m,rP \mod q) $
\item Compute $s\gets r - h \times \sk_U$
\item Outputs $\sigma_{m,U} = (s, h)$

\end{algorithmic}

\algrule[0.5pt]
$\{\text{valid},\text{invalid}\}\gets\verify(m,\sigma_{m,U},U,{C}_U,mpk)$
\algrule[0.5pt]
\begin{algorithmic}[1]

\item Compute $\{j_1,\dots,j_{k}\} \gets \h_1(U,C_U)$ 
\item $R'\gets sP+h(\sum_{i=1}^k \mathbf{Z}[j_i] \mod q +C_U)$
\item Output `valid' \textbf{if} $h=\h_2(m,R')$, \textbf{else} output `invalid'

\end{algorithmic}

\end{algorithm}

\subsection{Robust Security and fine-grained lawful interception} \label{sec:lawful}
Identity-based systems solely rely on the PKG to generate the keys for all users, creating a single point of failure. Thus, if the PKG is compromised, the entire system's security is at risk. To address this vulnerability, we leverage the additive property of the key generation algorithms of \scheme{} to propose new key generation methods (Algorithm \ref{alg:IBS_lawful}). This is achieved by dividing the signer's key into two components, $u_1$ and $z_U$. During the new key generation process, the signer selects $u_1$ and computes its commitment $Q_u$.
The PKG then computes the other secret component, i.e., $z_U$, by deriving an ARIS signature (see $\sgnextract(\cdot)$ in Algorithm \ref{alg:IBS_lawful}) on user identity $U$ and $Q_u$ (implicit certification).  After receiving $z_U$, by leveraging the additive property of \scheme{}, the signer computes the \emph{final} secret key as $x_U\gets u_1+z_U \mod p$. In this case, even if the PKG is compromised, the user secret key remains secure since the adversary needs knowledge of $ u_1$ to compute it. 

With this improvement, the new key generation algorithm can also enable a fine-grained lawful interception by allowing the signer to reissue its key by running $(u_1,Q_U)\gets \userkg(params)$  and requesting a new $x_U$ from the PKG. This requires including a sequence number $t$, supplied by the user $U$, in the input of the hash function $\h_1(\cdot)$ during the $\sgnextract(\cdot)$ algorithm. Using the sequence number in key generation prevents the misuse of the old keys and provides an efficient approach for fine-grained control of the key lifespan. 

\begin{algorithm}\caption{Key Generation for robust security and fine-grained lawful interception}\label{alg:IBS_lawful}
\small
$(u_1,Q_U)\gets\userkg(params)$
\algrule[0.5pt]

\begin{algorithmic}[1]
 
\item Compute $u_1 \gets  \ZZ_p$
\item Compute $Q_U \gets u_1 P \mod q $
\item Output $(u_1,Q_U)$
 
\end{algorithmic}
\algrule[0.5pt]

 $(x_U,{C}_U)\gets\sgnextract(msk,Q_U)$
\algrule[0.5pt]

\begin{algorithmic}[1]

\item Compute $u_2 \gets \PRF_{msk}(U)$ and ${B}_{U}\gets u_2 P \mod q$ 
\item Compute $C_U \gets Q_U+B_U \mod q$
\item Compute  $\{j_1\dots, j_k\}\gets \h_1(U,C_{U})$ where each $|j_i| = |t|$

 \item Compute $z_{U}\gets   \sum_{i=1}^{k} z_{j_i} + u_2 \mod p$  
  \item Output $( z_{U}, B_U,{C}_U$) 
\end{algorithmic}
\algrule[0.5pt]

$\sk\gets\cmpkey(u_1,z_U)$
\algrule[0.5pt]

\begin{algorithmic}[1]
 
\item Compute $x_U\gets u_1+z_U \mod p$
\item Output ($x_U,Q_U$)
 
\end{algorithmic}

\end{algorithm}

\subsection{Security Analysis}
\begin{theorem}
    In the random oracle model, if an adversary \A~can break the scheme proposed in Algorithm \ref{alg:IBS}, in the sense of Definition \ref{def:eucma}, then one can construct another algorithm \C~that runs the adversary and \A~as a subroutine and can solve an instance of the ECDL problem in Definition \ref{def:ecdl}.
\end{theorem}
 
\begin{proof}

    Given  $X\gets \EC$ as an instance of the ECDL problem, \C~works as follows to find a solution $z^*\gets \ZZ_p$, such that $z^*P=Z^* \mod q$. 
    
    \noindent\emph{Setup:} \C~keeps two lists ($L_1, L_2$) to keep track of the output of the random oracles $\h_1(\cdot)$ and $\h_2(\cdot)$  and lists $L_\sigma$ and $L_U$ to keep track of the messages submitted to the sign and corrupt oracles, respectively. \C~sets up the following random oracles to handle queries to hash functions. 
    \begin{itemize}
        \item $\alpha_1\gets \h_1$-$\mathtt{sim}(U,C_{U},L_1)$: If the input (i.e., $U,C_{U}$) already exists, it returns the corresponding $\alpha_1$, else, it returns  $\alpha_1 \Ra \{0,1\}^{k|t|}$ and stores $(U,C_{U},\alpha_1)$   in $L_1$. 
        
        \item $\alpha_2\gets \h_2$-$\mathtt{sim}(m, R, L_2)$: If the input $(m,R)$ already exists, it returns the corresponding $\alpha_2$, else, it returns  $\alpha_2 \Ra  \ZZ_p$ and stores $(m, R,\alpha_2)$   in $L_2$. 
            
            \end{itemize}
     Next, \C~selects a target index $j^*\gets \{1,\dots,t\} $ and sets the target $mpk$ element $ Z_{j^*} =  Z^*$. Then it selects $z_i \Ra \ZZ_p$ and compute  $Z_i\gets z_i P \mod q $ where $i\in\{1,\dots,t\}$ and $i\neq j^*$ and output $mpk=(Z_i, \dots, Z_t)$.

    \noindent \emph{Queries:} 
    
    \begin{itemize}
    \item \emph{Hash queries:} Hash queries on $\h_1$, $\h_2$ and $\h_3$ will be handled by   $\h_1$-$\mathtt{sim}(\cdot)$, $\h_2$-$\mathtt{sim}(\cdot)$ and $\h_3$-$\mathtt{sim}(\cdot)$ functions defined above, respectively. 
    \item \emph{$\mathcal{O}_{Corrupt}({U})$ Queries:} Given a user $U$, if $U$ exists in $L_U$, it returns $(x_{U})$. Next, it checks $L_1$; if such $U$ exits with an index corresponding to $j^*$, it aborts. Else, it selects $u\Ra \ZZ_p$, computes $U\gets uP \mod q$. Next, it selects $j_i\Ra\{1,\dots,t\}$ for $i=\{1,\dots,k\}$ and $j_i \neq j^*$, for each $j_i$, recovers the corresponding $z_{j_i}$ and computes and returns the secret key $x_{U}\gets \sum_{i=1}^k z_{j_i}+u \mod p$. To respond to future queries, the output is stored in $L_U$. 
    \item \emph{$\mathcal{O}_{Sign}(m,U)$ Queries:} For signature queries for users $U$ where $\alpha_1 $ does not contain $j^*$, \C~can work similar to the $\sign(\cdot)$ to generate the signature. Otherwise, when $\alpha_1$ contains $j^*$, \C~uses its access to the random oracle and works similarly to Schnorr Signature to generate a valid signature for \A.
    
    \end{itemize}
    \noindent \emph{$\A$'s Forgery:}  Finally, $\A$ will output a forgery message-signature pair $(m^*,\sigma_{U^*})$. $\A$ wins the game if $\text{`valid'}\gets\scheme.\verify(m^*,\sigma_{U^*},U^*,{C}_{U^*},mpk)$ and  $m^*$ was not submitted to $\mathcal{O}_{Sign}(\cdot)$.   

\noindent \emph{Solving the hard problem:} After outputting a valid forgery by $\A$, $\C$ checks if for $U^*$  the target public key element $Z^*$ is embedded $\alpha_1$. Else, it fails. If   $\alpha_1$ indeed contains $j^*$, similar to \cite{Tachyon,ARIS}, \C~utilizes the forking lemma \cite{Bellare-Neven:2006}, to obtain a second forgery $m^*,\sigma'_{U^*}$, where with a very high probability $s^*\neq s'$ and $h^*= h' $. Then, given the results of Lemma 1 in \cite{Bellare-Neven:2006}, to solve the ECDL problem.
    
\end{proof}

\begin{cor}
 The key generation algorithm provided in Algorithm \ref{alg:IBS_lawful} offers robust security by  preventing a single point of failure inherent in identity-based schemes. 
\end{cor}
\begin{proof}
      In the original scheme (i.e., Algorithm \ref{alg:IBS}), the secret component of the user key is computed as $x_U\sum_{i=1}^{k} z_{j_i} + u \mod p$, where both $z_j$'s and $u$ are known and selected by the PKG.  
      Consequently, a compromised PKG can issue private keys on behalf of the user. In the key generation algorithm with robust security in Algorithm \ref{alg:IBS_lawful}, the secret component of user key is computed as $x_U = \sum_{i=1}^{k} z_{j_i} + u_2+u_1$ where $u_1$ is not known to the PKG. This offers the binding property by incorporating the commitment of $u_1$ as the input of the hash function $\h_1(\cdot)$. We note that the correctness of the key supplied by the PKG can be simply  verified by the user by running the verification algorithm in ARIS \cite{ARIS} on $z_U$.
\end{proof}

\section{Instantiation of \scheme{} for 5G Networks}
\label{Detailed Design}
In this section, we discuss the limitations of \texttt{Schnorr-HIBS} and introduce our new scheme. 

The preliminary version of \scheme{}, \texttt{Schnorr-HIBS} has a hierarchal design with 2 PKGs. As shown in Figure \ref{pre_protocol}, the core-PKG generates key pairs for AMFs as the first-level PKG. The AMFs, serve as second-level PKGs, generate key pairs for their corresponding base stations. Then, the base stations sign their \texttt{SIB1} messages using the received private keys and broadcast the message to the UEs. The UEs, with the $PK_{PKG}$ provisioned, need to receive all the public keys and identities from the base station and the AMF to verify the signature. 

\begin{figure*}[t]
 \centering
    \includegraphics[width=0.9\linewidth]{images/Schnorr_HIBS.pdf}
    \caption{Instantiation of \texttt{Schnorr-HIBS}. }
        \label{pre_protocol}
\end{figure*}

\subsection{\texorpdfstring{Limitations of \texttt{Schnorr-HIBS} \cite{singla2021look}}{Limitations of Schnorr-HIBS}}
\label{limitations}

\noindent \textbf{Hierarchal Architecture. }
\texttt{Schnorr-HIBS} proposed a hierarchal architecture signature scheme, which supports multiple layers from the PKG to the verifier. In the instantiation, the core-PKG and the AMF serve as the first- and second-level PKG. However, this hierarchal architecture has several limitations: \ding{182} Using AMFs in the middle creates more complexity for the scheme. The operators need to implement and manage the crypto functionalities in AMF, which introduces more cost. A misconfigured AMF can make the system unavailable and even leak its private keys. If an attacker has control over an AMF, it can sign its own fake base stations and the user cannot detect it from the signature. \ding{183} The core-PKG is centralized, it can be configured with powerful hardware and the signature extraction is very fast. However, the AMF is distributed and can have various types of configurations. It can be difficult to ensure service quality if one AMF is serving a large amount of base stations. \ding{184} Adding additional levels in the scheme will cause more overhead on the verifier side. Because the verifier needs to get the public keys for all layers and verify the signature with more arithmetic operations. In addition, the multi-level design introduces more communication costs. 

\noindent \textbf{Lawful Interception. }
\texttt{Schnorr-HIBS} assumed that the law enforcement department could obtain key pairs for base stations from the AMF and set up their fake base stations like a normal base station. However, it did not provide a fine-grained access control mechanism to limit the usage of the key and revoke the key if needed. In that case, the law enforcement department can set up a fake base station to intercept user traffic at any time and any location with a valid key pair. This may violate the authorized scope, and give the law enforcement department more power to track users than it should have. Furthermore, if the key is leaked, the attacker can set up \textit{authenticated} fake base stations, which are considered valid by the UE. 

\noindent \textbf{End-to-end Implementation. }
\texttt{Schnorr-HIBS} did not provide an end-to-end implementation for the proposed scheme. Thus, it overlooked the challenges of deploying the proposed scheme in the real world. For example, the hierarchal architecture requires new implementation for both core-PKG and AMF, increasing the complexity and the development and maintenance costs. Also, the base station needs to send the signature of \texttt{SIB1} along with the public key and identity of both the base station and the AMF to the UE. The total overhead is 150 bytes, which takes a large amount of the available space in \texttt{SIB1} (372 bytes) \cite{3GPP:38.331} message. A commercial base station may not be able to append this information after the existing configurations. 

\subsection{Design Decisions of \scheme{}}
\label{design_decisions}
 To address these limitations, 
we outline the detailed design of our protocol and the rationale behind the design decisions. 

\noindent \textbf{2-layer Design. } 
We specify a 2-layer architecture for our protocol: the core-PKG generates the keys for base stations and the base station creates the signatures. We use a 2-layer approach instead of a hierarchical approach where a core-PKG generates keys for AMFs and the AMFs generate the signing keys for the base stations for several reasons: 
In our approach, the AMF only needs to forward the \textit{keyext\_request} and the response from core-PKG. 
In our scheme, we are aiming to make the scheme both signer and verifier efficient.

\noindent \textbf{Fine-grained Lawful Interception. } 
We design our protocol with fine-grained lawful interception in mind. Algorithm \ref{alg:IBS_lawful} introduces key generation with a sequence number, ensuring that keys generated with a previous sequence number are implicitly invalidated when a new sequence number is used. This mechanism enables seamless key revocation and can be further enhanced by integrating fine-grained access control policies directly within the core-PKG, ensuring efficient and secure key management. 

\noindent \textbf{Minimize Bytes Sent Over-The-Air.} 
To comply with the current protocol and introduce a minimum overhead while satisfying modern security requirements, we use a Schnorr signature scheme. 
Only 111 bytes (see \ref{Sec:counterparts}) are required to send over-the-air to the UE. The communication overhead is 26\% smaller than the previous scheme.  

\noindent \textbf{Choice of Messages to Sign.} \texttt{System information} messages are broadcast periodically by the base stations to allow UEs to initiate a connection to them. \texttt{System Information} messages are divided into a \texttt{Master Information Block (MIB)} and multiple \texttt{System Information Block (SIB)} messages~\cite{3GPP:38.331}. \texttt{MIB} includes the basic parameters required by the UE to acquire the \texttt{SIB1} message. The \texttt{SIB1} message is the most important \texttt{System Information} message and contains the base station selection parameters, scheduling info for the rest of the SIB messages, whether one or more SIB messages are only provided on-demand, and configuration needed by the UE to perform the system information request.  
Since the MIB and SIB1 messages are two messages required for a UE to connect to a base station, our protocol signs the two messages together and provides the signature in the SIB1 message. After the UE receives the SIB1 message, it is able to authenticate the base stations. 

\noindent \textbf{Construction of Identities.} Our protocol requires assigning IDs to the base stations. We utilize the IDs for the dual purpose of uniquely identifying the base stations as well as for communicating the validity period of their signing keys. 
For $U_{BS}$ we use a concatenation of \texttt{NRCell\_ID}~\cite{3GPP:29.571} and an expiry timestamp. \texttt{NRCell\_ID} is a string of size 36 bits and uniquely identifies a base station for a particular mobile network operator. Each expiry timestamp is 32 bits long. Therefore, $U_{BS}$ can be a maximum of 9 bytes. 

\noindent \textbf{Validity period of the keys.}
\label{validity} Instead of using complex key revocation techniques, we assign different validity periods to each generated keypair after which the keys would need to be refreshed. For the core-PKG, we create the key-pair with a 1 year validity period by default as it needs to be installed inside the UE's USIM, and requires a confidentiality and integrity-protected channel to be updated. The core-PKG needs to be physically secured and protected so that its private key is not leaked. 
For the base stations, we generate a key pair valid for only 10 minutes. base stations are located around the world in physically insecure areas. Therefore, it may be easier for the attacker to compromise them. A validity period of only 10 minutes minimizes the period during which an attacker can launch attacks, even if it obtains a base station's private key. These validity periods are recorded in the expiry timestamps in the $U_{BS}$, as well as in the UE's USIM for the core-PKG. These are the default validity periods and can be changed by the network operators when required. Since our key generation is efficient (1000 keys per 5.5 milliseconds), its impact on core-PKG is negligible. 

\begin{figure*}[t]
 \centering
    \includegraphics[width=0.9\linewidth]{images/protocol.pdf}
    \caption{Our protocol for authenticating 5G cellular base stations.}
        \label{protocol}
\end{figure*}

\subsection{Protocol Description}\label{protocolDescription}
We now detail our authentication protocol steps. We abstract some cryptographic details for readability. 
For instance, we do not explicitly mention the mod operation, but all the operations in $E(\mathbb{F}_p)$ are executed in mod~$p$ and operations in $\mathbb{Z}_q$ are in mod~$q$. 
Figure~\ref{protocol} gives a graphical representation of our protocol for a 5G scenario. 

\subsubsection{Initialization phase for the core-PKG}
\noindent The core-PKG generates the public system parameters and its own public-private key pair during the initialization phase. From $sk_{PKG}$, core-PKG generates $t$ sets of public-private key pairs. $PK_{PKG}$ contains all $t$ public keys generated. This phase is executed at the beginning of the 5GC deployment. The default validity period of the core-PKG's keys is 1 year. The public key of the core-PKG along with its expiry date is installed in the USIM of all UEs during initial registration. The core-PKG's public key installed in the USIM has to be replaced, whenever the core-PKG refreshes its keys. This can be done using the confidentiality and integrity-protected channel created between the AMF and the UE after mutual authentication. The core-PKG uses the \texttt{Setup} step from Algorithm~\ref{alg:IBS} to generate its key pair
\{$sk_{PKG},\ PK_{PKG}$\}.

\begin{gather*}
sk_{PKG} \gets \mathbb{Z}_p \\
z_i \gets \PRF_{sk_{PKG}}(i), Z_i \gets z_i \times P \textbf{ for } i\in \{1,\dots,t\} \\
PK_{PKG} \gets \{Z_i\}_{i=1}^t
\end{gather*}

\subsubsection{Key extraction for the base station}
\noindent Base stations send a key extraction request with their \texttt{NRCell\_ID} to the core-PKG through their serving AMFs. The core-PKG concatenates the received \texttt{NRCell\_ID} with an expiry timestamp for the key being generated as $U_{BS}$. From $U_{BS}$, the core-PKG follows the \texttt{Extact} step from Algorithm~\ref{alg:IBS} to generate $sk_{BS}$ and $PK_{BS}$. The base stations have to periodically refresh their key-pair by sending the key-generation request to the core-PKG when nearing the key-pair expiration. 

\begin{gather*}
U_{BS} \gets \text{NRCell\_ID} || \text{Expiry\_Timestamp} \\
u \gets \mathtt{PRF}_{sk_{PKG}}(U_{BS}), PK_{BS} \gets u \times P \\
\{j_1, ..., j_k\} \gets \mathtt{H}_1(U_{BS}||PK_{BS}) \text{, where each } |j_i|=|t| \\
sk_{BS} \gets \sum^k_{i=1}PK_{PKG}[j_i] + u
\end{gather*}

\subsubsection{Signing phase at the base station} 
The base stations sign the MIB and SIB1 message via \texttt{Sign} step of Algorithm~\ref{alg:IBS} and generate the signature $sig_{SIB1}$. They attach the $sig_{SIB1}$, $PK_{BS}$, and $U_{BS}$ along with the SIB1 message broadcast. Before signing, the base station needs to ensure that their own keys have not expired. \begin{gather*}
rand_s \stackrel{\$}{\leftarrow} \mathbb{Z}_p, R \gets rand_s \times P\\
h \gets \mathtt{H}_2(\text{MIB}||\text{SIB1}||R)\\
s \gets r - h \times sk_{BS}
\end{gather*}
where $\langle s,\ h \rangle$ is the signature.

\subsubsection{Verification phase at the UE} 
The UE uses the $U_{BS}$ and the $PK_{BS}$ sent by the base station attached to the SIB1 message to verify $sig_{SIB1}$. The UE first verifies that the key $PK_{BS}$ is not expired by looking at the expiry timestamps embedded in the $U_{BS}$. If the timestamps have not expired, the UE computes the indices to select the corresponding public keys from $PK_{PKG}$ and then verifies the signature $sig_{SIB1}$.
For verification, the UE uses the public keys of core-PKG and the base station: 
\begin{gather*}
\{j_1,...,j_{k}\} \gets \mathtt{H}_1(U_{BS}||PK_{BS})\\
R' \gets s \times P + h \times (\sum^{k}_{i=1}(PK_{PKG}[j_i] + PK_{BS})\\
\text{If } h = \mathtt{H}_3(\text{MIB}||\text{SIB1}||R') \text{, then valid, else invalid.}
\end{gather*}
\noindent \textbf{Authentication failure action.} In case of authentication failure or the absence of authentication capabilities at the base station, the UE does not connect to the base station and keeps searching for other base stations available in the area. If there are no available base stations that can be authenticated, the UE can connect to an unauthenticated base station or keep looking for a base station that can be authenticated. We propose this to be a UE-specific choice, which can be configured depending on the mobile user's security/connectivity needs. If the UE decides to connect to an unauthenticated base station, it keeps checking the \texttt{System Information} messages to find a base station that can be authenticated.

\subsection{Handling Roaming Scenario}
\label{roaming}
Roaming services enable a UE to connect to base stations operated by a different network operator. Since each operator manages its own core-PKG, the UE must first obtain the public key of the roaming operator. The UE's primary operator can sign the roaming operator’s public key and provision it through non-3GPP access networks, such as Wi-Fi.

\subsection{Protection Against Relay Attacks}
Our authentication protocol protects against fake base stations by allowing UEs to authenticate \texttt{System Information} messages. However, it is vulnerable to relay attacks, where an adversary retransmits these messages with a stronger signal, tricking the UE into connecting to a fake base station. Distance-bounding protocols \cite{rasmussen2010realization, tippenhauer2015uwb, durholz2011formal} could prevent these attacks but would require major changes to cellular protocols. An alternative is to time-bound \texttt{System Information} message validity by estimating the time for an adversary to intercept and retransmit messages, though this doesn't account for base station frequency or coverage area. The 5G base stations can vary in configurations and use different frequencies to cover different ranges \cite{3GPP:38.104}, making the use of a fixed transmission time less practical. 

To protect against relay attacks in all scenarios, we propose time-bounding the \texttt{System Information} message signatures based on the base station’s configuration. The validity period, denoted by $\mathsf{\bigtriangleup t}$, is calculated from the configuration-specific transmission delay ($\mathsf{\bigtriangleup t_{conf}}$) and cryptographic signature delay ($\mathsf{\bigtriangleup t_{sign}}$). $\mathsf{\bigtriangleup t_{conf}}$ is configured by the operators and can be derived from a lookup table stored securely in the base station's memory. $\mathsf{\bigtriangleup t_{sign}}$ varies with the cryptographic scheme used. The base station signs the message with a timestamp $\mathsf{T_{sign}}$ and the validity period. The UE checks the validity by verifying if $\mathsf{T_{current} < T_{sign} + \bigtriangleup t}$ when receiving the message.

\section{Evaluation}
\label{Evaluation}
\section{Evaluation}
We provide three sets of insights into this section, organised as \textit{findings (F*)}. We quantitatively study the effect of the adversarial and counterfactual perturbations on the performance of informal reasoners and autoformalisation methods. Then, we dive deeper into method variants. Finally, 
we analyse the nature of formalisation errors made by the models.

\subsection{Robustness Analysis}
\paragraph{\textbf{\emph{F1: Noise perturbations have a stronger effect on formalisation methods than informal \ac{LLM} reasoners.}}}
Table~\ref{tab:distraction_k4_formalisation} shows that, on average, the accuracy of both direct and \ac{CoT} informal reasoning remains between $73\%$ and $74\%$ in the face of added noise. While the autoformalisation method performs similarly to informal reasoners on the original dataset, its performance decreases between $4\%$ and $11\%$. The accuracy drops especially with logical (L) and tautological (T) distractions, whose logical language formats trick the \ac{LLM} into formalizing the noisy clauses. On the other hand, the linguistically complex and more natural sentences of encyclopedic distractions show a minor effect, suggesting that \acp{LLM} successfully avoids formalizing the more complicated sentences.

\paragraph{\textbf{\emph{F2: All \ac{LLM}-based reasoning methods suffer a drop for counterfactual perturbations.}}} % influence .}}}
Table~\ref{tab:distraction_k4_formalisation} shows that counterfactual statements cause a significant decrease in performance for both the informal reasoners and autoformalisation methods of between $12\%$ and $13\%$ on average. 
Moreover, this observation also holds for all tested models, i.e., none are robust towards counterfactual perturbations across every evaluated dimension. Even the strongest model, GPT 4o-mini, yields a performance of 63-68\%, which is relatively close to the random performance of 50\%. The high impact of counterfactual statements (the single ``not'' inserted) could be due to the inability of \acp{LLM} to overwrite prior knowledge with explicitly stated information or memorization of the answers. We study the error sources further in §\ref{subsec:errors}.  

\noindent \paragraph{\textbf{\emph{F3: Introducing multiple noise sentences has an effect only for logical distractions.}}}
We show the impact of introducing between one and four sentences for the two top-performing autoformalisation models in Figure~\ref{fig:length_distraction}. The figure shows similar trends with and without counterfactual perturbations.
As additional logical distractions are introduced, the model performance consistently decreases. Tautological (T) distractions lead to a decline in accuracy with a single disruptive sentence, yet adding more noise does not worsen the outcome. 
The tautological corpus introduces truth constants for all sentences as a persistent unseen logical construct. Given that this leads only to a decrease for a single occurrence, we can assume that a model can consistently handle the same unseen logical construct. In contrast, the logical corpus increases the chance of adding text, requiring new, previously unseen reasoning constructs for each added sentence. The impact of encyclopedic noise remains negligible, generalising F1 to $k$ sentences. Similarly, counterfactual perturbations remain much more effective for all settings, generalising F2.

\begin{table}[!t]
\small
\setlength{\modelspacing}{2pt}
\setlength{\tabcolsep}{1.7pt} % Default value: 6pt
\setlength{\belowrulesep}{4pt}
\begin{threeparttable}
    \centering
    \begin{tabular}{cc l r rrr @{\quad} rrrr}
\toprule
\multirow{2}{*}{} & \multirow{2}{*}{} & Reasoning & \multirow{2}{*}{O} & \multicolumn{3}{c}{Distraction} & \multicolumn{4}{c}{Counterfactual} \\
 & & Format & & E& L & T & $\text{O}_C$ & $\text{E}_C$& $\text{L}_C$ & $\text{T}_C$\\
\midrule
\multirow{6}{*}{\rotatebox{90}{Gemma-2}} & \multirow{3}{*}{\rotatebox{90}{9b}}
   & Informal (direct) & \textbf{0.78} & \textbf{0.80} & \textbf{0.79} & \textbf{0.77} & 0.58 & 0.52 & 0.50 & 0.59 \\
 & & Informal (CoT) & 0.72 & 0.78 & 0.73 & 0.76 & 0.61 & \textbf{0.57} & \textbf{0.60} & \textbf{0.66} \\
 & & Formal (FOL) & 0.62 & 0.58 & 0.52 & 0.53 & \textbf{0.63} & 0.52 & 0.46 & 0.46 \\[\modelspacing]
\cmidrule{2-11}
 & \multirow{3}{*}{\rotatebox{90}{27b}} 
   & Informal (direct) & 0.71 & 0.69 & \textbf{0.66} & \textbf{0.68} & 0.59 & 0.51 & 0.54 & 0.59 \\
 & & Informal (CoT) & 0.66 & 0.65 & 0.64 & 0.63 & 0.62 & 0.58 & \textbf{0.62} & \textbf{0.64} \\
 & & Formal (FOL) & \textbf{0.74} & \textbf{0.74} & 0.61 & 0.61 & \underline{\textbf{0.72}} & \underline{\textbf{0.67}} & 0.58 & 0.51 \\[\modelspacing]
\midrule
\multirow{6}{*}{\rotatebox{90}{Mistral}} & \multirow{3}{*}{\rotatebox{90}{7B}} 
   & Informal (direct) & 0.77 & \textbf{0.77} & 0.75 & \textbf{0.79} & \textbf{0.63} & \textbf{0.54} & \textbf{0.54} & \textbf{0.66} \\
 & & Informal (CoT) & \textbf{0.79} & 0.75 & \textbf{0.77} & 0.78 & 0.55 & 0.52 & \textbf{0.54} & 0.58 \\
 & & Formal (FOL) & 0.62 & 0.58 & 0.54 & 0.57 & 0.50 & \textbf{0.54} & 0.51 & 0.52 \\[\modelspacing]
\cmidrule{2-11}
 & \multirow{3}{*}{\rotatebox{90}{Small}} 
   & Informal (direct) & \textbf{0.77} & \textbf{0.76} & \textbf{0.76} & \textbf{0.75} & 0.61 & 0.51 & 0.56 & 0.59 \\
 & & Informal (CoT) & 0.72 & 0.72 & 0.72 & 0.71 & \textbf{0.62} & \textbf{0.59} & \textbf{0.62} & \textbf{0.68} \\
 & & Formal (FOL) & 0.68 & 0.59 & 0.53 & 0.64 & 0.54 & 0.55 & 0.49 & 0.51 \\[\modelspacing]
\midrule
\multirow{6}{*}{\rotatebox{90}{Llama-3.1}} & \multirow{3}{*}{\rotatebox{90}{8B}} 
   & Informal (direct) & 0.63 & 0.61 & 0.64 & 0.66 & 0.61 & \textbf{0.62} & 0.59 & 0.61 \\
 & & Informal (CoT) & 0.73 & \textbf{0.73} & \textbf{0.71} & \textbf{0.72} & \textbf{0.62} & 0.59 & \textbf{0.61} & \textbf{0.65} \\
 & & Formal (FOL) & \textbf{0.77} & 0.71 & 0.63 & 0.52 & 0.60 & 0.58 & 0.55 & 0.52 \\[\modelspacing]
\cmidrule{2-11}
 & \multirow{3}{*}{\rotatebox{90}{70B}} 
   & Informal (direct) & 0.77 & 0.74 & 0.74 & 0.73 & 0.62 & 0.53 & 0.56 & 0.64 \\
 & & Informal (CoT) & \textbf{0.78} & \textbf{0.75} & \textbf{0.76} & \textbf{0.76} & 0.64 & 0.61 & \textbf{0.66} & \underline{\textbf{0.73}} \\
 & & Formal (FOL) & 0.74 & 0.73 & 0.71 & 0.71 & \textbf{0.66} & \textbf{0.62} & 0.59 & 0.57 \\[\modelspacing]
 \midrule
\multirow{3}{*}{\rotatebox{90}{GPT}} & \multirow{3}{*}{\rotatebox{90}{4o-mini}} 
   & Informal (direct) & 0.78 & 0.77 & 0.79 & 0.79 & 0.64 & 0.61 & 0.61 & 0.63 \\
 & & Informal (CoT) & 0.80 & 0.80 & \underline{\textbf{0.81}} & \underline{\textbf{0.82}} & \textbf{0.68} & \textbf{0.63} & \underline{\textbf{0.68}} & \textbf{0.64} \\
 & & Formal (FOL) & \underline{\textbf{0.84}} & \underline{\textbf{0.82}} & 0.73 & 0.79 & 0.63 & 0.62 & 0.57 & 0.54 \\[\modelspacing]
 \midrule
\multicolumn{2}{c}{\multirow{3}{*}{\textbf{Avg}}} 
 & Informal (direct) & 0.74 & 0.73 & 0.73 & 0.73 & 0.61 & 0.55 & 0.56 & 0.62 \\
 & & Informal (CoT) & 0.74 & 0.74 & 0.73 & 0.74 & 0.62 & 0.58 & 0.62 & 0.65 \\
  & & Formal (FOL) & 0.72 & 0.68 &	0.61 & 0.62 & 0.61 & 0.59 & 0.54 & 0.52 \\
\bottomrule
\end{tabular}
\caption{Accuracies of informal and autoformalisation-based deductive reasoners. The best overall model per dataset is underlined; the best model version is marked in bold.}
\label{tab:distraction_k4_formalisation}
\end{threeparttable}
\end{table} 

\begin{figure}[!t]
    \centering
    \scriptsize
    \begin{tikzpicture}
        \begin{axis}[name=gpt,
            title={GPT-4o-mini},
            width=0.6\linewidth,
            height=0.6\linewidth,
            xlabel={\# Noise sentences},
            ylabel={Accuracy},
            xmin=-0.1, xmax=4.1,
            ymin=0.5, ymax=0.9,
            xtick={1,2,4},
            ytick={0.55, 0.6, 0.65, 0.75, 0.8, 0.85},
            title style={yshift=-0.6em},
            legend style={at={(1,-0.15)},
	           anchor=north,legend columns=-1},
            x label style={at={(axis description cs:1,-0.05)},anchor=north},
            y label style={at={(axis description cs:-0.15,0.5)},anchor=south},
            ymajorgrids=true,
            grid style=dashed,
        ]
            \addplot[color=blue, mark=square,]
                coordinates {
                (0,0.848076939582825)(1,0.823076903820038)(2,0.826923072338104)(4,0.821153819561005)
                };
            \addplot[color=red, mark=triangle,]
                coordinates {
                (0,0.848076939582825)(1,0.817307710647583)(2,0.801923096179962)(4,0.759615361690521)
                };
            \addplot[color=green, mark=diamond,] 
                coordinates {
                (0,0.848076939582825)(1,0.767307698726654)(2,0.769230782985687)(4,0.803846180438995)
                };
            \addplot[color=blue, mark=square*] 
                coordinates {
                (0,0.627777755260468)(1,0.622222244739533)(2,0.600000023841858)(4,0.633333325386047)
                };
            \addplot[color=red, mark=triangle*,] 
                coordinates {
                (0,0.627777755260468)(1,0.611111104488373)(2,0.611111104488373)(4,0.594444453716278)
                };
            \addplot[color=green, mark=diamond*,] 
                coordinates {
                (0,0.627777755260468)(1,0.572222232818604)(2,0.538888871669769)(4,0.555555582046509)
                };
                \legend{E,L,T,$\text{E}_C$, $\text{L}_C$ , $\text{T}_C$}
        \end{axis}

        \begin{axis}[name=llama, at={($(gpt.east)+(0.1cm,0)$)},anchor=west,
            title={Llama 3.1 70b},
            width=0.6\linewidth,
            height=0.6\linewidth,
            xmin=-0.1,, xmax=4.1,
            ymin=0.5, ymax=0.9,
            xtick={1,2,4},
            ytick={0.55, 0.6, 0.65, 0.75, 0.8, 0.85},
            title style={yshift=-0.6em},
            yticklabel=\empty,
            ymajorgrids=true,
            grid style=dashed,
        ]
            \addplot[color=blue, mark=square,]
                coordinates {
                (0,0.838461518287659)(1,0.817307710647583)(2,0.805769205093384)(4,0.817307710647583)
                };
            \addplot[color=red, mark=triangle,]
                coordinates {
                (0,0.838461518287659)(1,0.819230794906616)(2,0.803846180438995)(4,0.771153867244721)
                };
            \addplot[color=green, mark=diamond,]
                coordinates {
                (0,0.838461518287659)(1,0.803846180438995)(2,0.807692289352417)(4,0.805769205093384)
                };
            \addplot[color=blue, mark=square*]
                coordinates {
                (0,0.627777755260468)(1,0.622222244739533)(2,0.577777802944183)(4,0.594444453716278)
                };
            \addplot[color=red, mark=triangle*,]
                coordinates {
                (0,0.627777755260468)(1,0.583333313465118)(2,0.561111092567444)(4,0.577777802944183)
                };
            \addplot[color=green, mark=diamond*,]
                coordinates {
                (0,0.627777755260468)(1,0.627777755260468)(2,0.566666662693024)(4,0.577777802944183)
                };
        \end{axis}
    \end{tikzpicture}
    \caption{Influence of the number of noisy sentences for FOL.}
    \label{fig:length_distraction}
\end{figure}



\subsection{Impact of Method Design}
\paragraph{\textbf{\emph{F4: \ac{CoT} prompting is most impactful when both noise and counterfactual perturbations are applied.}}}
The accuracies for the individual \acp{LLM} in Table~\ref{tab:distraction_k4_formalisation} show that the impact of \ac{CoT} is negligible for noise-only datasets (first four columns). Meanwhile, the benefit from \ac{CoT} is most pronounced in the datasets that combine noise and counterfactual perturbations.
The better-performing informal prompting strategy for a model remains stable for all types of distractions. Still, the decline in performance due to counterfactuals leads to a less consistent preference for a specific prompting style.

\paragraph{\textbf{\emph{F5: The best-performing grammar differs per model and is unstable across data versions.}}}

The evaluation of different logical forms for formal \ac{LLM}-based reasoning in Table~\ref{tab:distraction_k4_logical_form} shows the preference of some models for specific syntactic formats.
Llama 3.1 70B has a considerable improvement of $12\%$ with TPTP syntax on the original set, while Llama 3.1 8B benefits from the R-FOL syntax. However, all grammars show a declining accuracy trend and increased syntax errors for noise perturbations, where the best grammar loses its advantage over the rest. 
When comparing the grammars on the counterfactual partitions, we observe that TPTP is consistently more robust than the standard first-order logic grammar. Here, GPT 4o-mini shows a reduction from $O$ to $O_C$ of $20\%$ for FOL and only $12\%$ for the TPTP grammar. Since this does not correlate with fewer syntax errors, the formalisation in TPTP prevents semantical errors for counterfactual premises. 
A positive reading of these results, especially the minor differences between FOL and R-FOL, is that autoformalisation \acp{LLM} can adapt to the grammar syntax prescribed in the prompt without further loss in performance.

\begin{table}[!t]
\small
\setlength{\modelspacing}{2pt}
\setlength{\tabcolsep}{1.7pt} % Default value: 6pt
\setlength{\belowrulesep}{4pt}
\begin{threeparttable}
    \centering
    \begin{tabular}{cc l r rrr @{\quad} rrrr}
\toprule
\multirow{2}{*}{} & \multirow{2}{*}{} & Grammar & \multirow{2}{*}{O} & \multicolumn{3}{c}{Distraction} & \multicolumn{4}{c}{Counterfactual} \\
 & & Syntax & & E& L & T & $\text{O}_C$ & $\text{E}_C$& $\text{L}_C$ & $\text{T}_C$\\
\midrule
\multirow{6}{*}{\rotatebox{90}{Llama-3.1}} & \multirow{3}{*}{\rotatebox{90}{8B}} 
   & FOL & 0.77 & \textbf{0.71} & 0.61 & \textbf{0.53} & 0.58 & \textbf{0.55} & 0.52 & \textbf{0.56} \\
 & & R-FOL & \textbf{0.78} & 0.69 & \textbf{0.62} & \textbf{0.53} & 0.58 & \textbf{0.55} & \textbf{0.54} & 0.52 \\
 & & TPTP & 0.73 & 0.67 & 0.55 & 0.51 & \textbf{0.68} & 0.54 & 0.46 & 0.51 \\[\modelspacing]
\cmidrule{2-11}
 & \multirow{3}{*}{\rotatebox{90}{70B}} 
   & FOL & 0.76 & 0.73 & 0.71 & \textbf{0.72} & 0.67 & 0.57 & 0.63 & 0.56 \\
 & & R-FOL & 0.76 & 0.73 & 0.67 & 0.71 & 0.64 & 0.57 & 0.53 & 0.64 \\
 & & TPTP & \underline{\textbf{0.88}} & \underline{\textbf{0.84}} & \underline{\textbf{0.81}} & \textbf{0.72} & \underline{\textbf{0.81}} & \underline{\textbf{0.68}} & \underline{\textbf{0.67}} & \underline{\textbf{0.68}} \\[\modelspacing]
\midrule
\multirow{3}{*}{\rotatebox{90}{GPT}} & \multirow{3}{*}{\rotatebox{90}{4o-mini}} 
   & FOL & \textbf{0.84} & \textbf{0.82} & \textbf{0.72} & \underline{\textbf{0.78}} & 0.64 & \textbf{0.63} & \textbf{0.61} & 0.51 \\
 & & R-FOL & \textbf{0.84} & 0.77 & 0.70 & \underline{\textbf{0.78}} & \textbf{0.72} & 0.56 & 0.54 & \textbf{0.63} \\
 & & TPTP & 0.83 & \textbf{0.82} & 0.71 & 0.71 & 0.69 & \textbf{0.63} & 0.57 & 0.57 \\
\bottomrule
\end{tabular}
\caption{Accuracies of different formalisation grammars for autoformalisation.}
\label{tab:distraction_k4_logical_form}
\end{threeparttable}
\end{table} 

\paragraph{\textbf{\emph{F6: Feedback does not help \acp{LLM} self-correct to mitigate robustness issues.}}}
\autoref{tab:distraction_k4_feedback} shows the results with different error recovery mechanisms. The results indicate that no feedback strategy emerges as a winner in the different datasets. 
All feedback variants reduce syntax errors for noise perturbations, but given the lack of a consistent increase in accuracy, the corrected formalisations are most likely to contain semantic errors still. 
The type of feedback message only has a minor influence on correcting syntax errors, whereas Llama 3.1 70b and GPT 4o-mini correct slightly more syntax errors with specific error messages. This finding aligns with \cite{huang2023large}, who also found that \acp{LLM} cannot consistently self-correct their reasoning after receiving relevant feedback.

\begin{table}[!ht]
\small
\setlength{\modelspacing}{2pt}
\setlength{\tabcolsep}{1.7pt} % Default value: 6pt
\setlength{\belowrulesep}{4pt}
\begin{threeparttable}
    \centering
    \begin{tabular}{cc l r rrr @{\quad} rrrr}
\toprule
\multirow{2}{*}{} & \multirow{2}{*}{} & \multirow{2}{*}{Feedback} & \multirow{2}{*}{O} & \multicolumn{3}{c}{Distraction} & \multicolumn{4}{c}{Counterfactual} \\
 & & & & E& L & T & $\text{O}_C$ & $\text{E}_C$& $\text{L}_C$ & $\text{T}_C$\\
\midrule
\multirow{8}{*}{\rotatebox{90}{Llama-3.1}} & \multirow{4}{*}{\rotatebox{90}{8B}} 
   & No recovery & 0.77 & \textbf{0.72} & 0.62 & 0.53 & 0.59 & 0.58 & 0.56 & \textbf{0.56} \\
 & & Error type & \textbf{0.79} & 0.71 & 0.63 & \textbf{0.56} & \textbf{0.66} & 0.54 & 0.52 & 0.51 \\
 & & Error message & 0.78 & 0.71 & \textbf{0.67} & 0.55 & 0.59 & 0.53 & \underline{\textbf{0.64}} & 0.49 \\
 & & Warning & 0.74 & 0.66 & 0.58 & 0.55 & 0.55 & \textbf{0.60} & 0.49 & 0.49 \\[\modelspacing]
\cmidrule{2-11}
 & \multirow{4}{*}{\rotatebox{90}{70B}} 
   & No recovery & \textbf{0.77} & \textbf{0.72} & \textbf{0.73} & 0.71 & \textbf{0.64} & 0.59 & \textbf{0.61} & 0.56 \\
 & & Error type & 0.72 & 0.70 & 0.72 & \textbf{0.73} & 0.62 & 0.56 & 0.60 & 0.58 \\
 & & Error message & 0.71 & 0.70 & \textbf{0.73} & 0.71 & \textbf{0.64} & 0.59 & 0.54 & \underline{\textbf{0.64}} \\
 & & Warning & 0.69 & \textbf{0.72} & 0.72 & 0.72 & 0.62 & \underline{\textbf{0.65}} & \textbf{0.61} & 0.63 \\[\modelspacing]
\midrule
\multirow{4}{*}{\rotatebox{90}{GPT}} & \multirow{4}{*}{\rotatebox{90}{4o-mini}} 
   & No recovery & \underline{\textbf{0.84}} & \underline{\textbf{0.82}} & 0.73 & 0.79 & 0.64 & \textbf{0.62} & 0.56 & \textbf{0.56} \\
 & & Error type & 0.83 & 0.79 & 0.74 & 0.76 & 0.67 & 0.57 & 0.56 & \textbf{0.56} \\
 & & Error message & \underline{\textbf{0.84}} & 0.78 & \underline{\textbf{0.77}} & \underline{\textbf{0.80}} & 0.62 & 0.59 & 0.56 & \textbf{0.56} \\
 & & Warning & \underline{\textbf{0.84}} & 0.75 & 0.73 & 0.76 & \underline{\textbf{0.70}} & 0.61 & \textbf{0.61} & 0.55 \\
 \bottomrule
\end{tabular}
\caption{Accuracies of error recovery strategies.}
\label{tab:distraction_k4_feedback}
\end{threeparttable}
\end{table} 

\subsection{Error Analysis}
\label{subsec:errors}
\paragraph{\textbf{\emph{F7: Autoformalisation increases syntax errors for noise perturbations.}}}
The low performance for noise perturbations correlates with more syntax errors for all models and distraction categories (cf. execution rates in Table~\ref{tab:appendix_k4_formalisation_exec}). The three worst-performing models (both Mistral models, Gemma-2 9b) generate, at best, for $37\%$  and, at worst, for only $4\%$ of the samples, a valid logical form.
Gemma-2 9b and Llama3.1 8b produce more syntax errors than the larger counterparts, suggesting that larger models are more robust towards noise perturbations. 
The accuracy of syntactically valid samples is higher than the informal reasoning methods for most distractions (Table~\ref{tab:appendix_k4_formalisation_vacc}), motivating informal reasoning as a backup strategy for formal reasoning. The error message feedback reveals two common syntax errors: 1) errors by models with an initial low execution rate exhibit issues with the template structure, including using incorrect keywords or adding conversational phrases;
2) perturbation-related errors, the most common of which is using undefined truth constants as part of tautological distractions. 

\paragraph{\textbf{\emph{F8: Autoformalisation increases semantic errors for counterfactuals.}}}
Unlike the introduced noise, counterfactual perturbations do not lead to more syntax errors. The execution rate in Table~\ref{tab:appendix_k4_formalisation_exec} is stable or improves for counterfactuals. However, we see a drop in accuracy for the counterfactual column $\text{O}_C$ in Table~\ref{tab:distraction_k4_formalisation} and can conclude that the number of logical forms with semantic errors has to increase. This suggests that the introduced negation is not correctly formalised. Looking at the warnings generated by the feedback mechanism, for GPT 4o-mini, $161$ warning messages are generated on the unperturbed data. $54$ of these were fixed with a single iteration. Not considering predicates and individuals as part of the context is the most frequent warning across all models. 

\section{Related Work}
\label{Related Work}
\section{Related Work}
% \subsection{Vision Language Model}
% 시각장애인에서 상황을 설명할 DB가 없으니 만들었다. 그리고 이를 VLM에 튜닝했다.
\subsection{Technical approaches for assisting the visually-impaired}


\subsection{Datasets for visual instruction tuning}


\section{Conclusion}
\label{Conclusion and Future Work}
\section*{Conclusion}
This paper aims to enhance our understanding of the computational complexity of computing various Shapley value variants. We found that for various ML models --- including decision trees, regression tree ensembles, weighted automata, and linear regression --- both local and global interventional and baseline SHAP can be computed in polynomial time under HMM modeled distributions. This extends popular algorithms, such as TreeSHAP, beyond their empirical distributional scope. We also establish strict complexity gaps between the various SHAP variants (baseline, interventional, and conditional) and prove the intractability of computing SHAP for tree ensembles and neural networks in simplified scenarios. Overall, we present SHAP as a versatile framework whose complexity depends on four key factors: \begin{inparaenum}[(i)] \item model type, \item SHAP variant, \item distribution modeling approach, \item and local vs. global explanations\end{inparaenum}. We believe this perspective provides deeper insight into the computational complexity of SHAP, paving the way for future work.




%We believe that our framework provides a more intricate understanding of SHAP computation complexity across different models, distributions, and variants, paving the way for further research.

Our work opens promising directions for future research. First, expanding our computational analysis to other SHAP-related metrics, such as asymmetric SHAP~\citep{frye20} and SAGE~\citep{covert2020understanding}, would be valuable. Additionally, we aim to explore more expressive distribution classes and relaxed assumptions beyond those in Section \ref{sec:tractable} while maintaining tractable SHAP computation. Finally, when exact computation is intractable (Section \ref{sec:intractable}), investigating the approximability of SHAP metrics through approximation and parameterized complexity theory~\citep{downey2012parameterized} is an important direction.

%Our work opens several promising avenues for future research on the computational properties of explainable AI methods, with a particular focus on SHAP. First, it would be interesting to broaden the computational analysis conducted in this work to include other popular SHAP-related metrics in the literature, such as asymmetric SHAP \cite{frye20} and SAGE \cite{covert2020understanding}. Also, in the future, we aim to explore more expressive distribution classes and relaxed distributional assumptions—extending beyond those examined in Section \ref{sec:tractable} —that still yield tractable SHAP computation. Finally, when exact computation proves intractable (Section \ref{sec:intractable}), it is worthwhile to theoretically investigate the question of the approximability of computing the SHAP metrics across various configurations, through the lens of approximation and parametrized complexity theory \cite{arora2009computational}.

%This paper aims to deepen our understanding of the computational complexity involved in obtaining different Shapley value variants. We found that for a variety of ML models, including decision trees, tree ensembles for regression, weighted automata, and linear regression models — computing both local and global interventional and baseline SHAP can be done in polynomial time when distributions are modeled by HMMs. This extends the distributional scope of popular algorithms like TreeSHAP, which is limited to empirical distributions. Additionally, we demonstrate a strict complexity gap between SHAP variants, showing that interventional and baseline SHAP can be strictly easier to compute than conditional SHAP. Despite these positive results, we uncovered intractability for various SHAP variants in neural networks and tree ensembles. Finally, we provided generalized complexity relations across SHAP variants. We believe that our framework offers a deeper understanding of the complexity involved in computing SHAP across various variants, models, distributions, as well as in both local and global computations, laying the groundwork for future research.

\section*{Acknowledgements}
\label{Acknowledgements}
\section*{Acknowledgments}
{\textcopyright}2025 All rights reserved. The research described in this paper was carried out at the Jet Propulsion Laboratory, California Institute of Technology, under a contract with the National Aeronautics and Space Administration (80NM0018D0004).

\appendices

\ifCLASSOPTIONcaptionsoff
  \newpage
\fi

\bibliographystyle{IEEEtran}
\bibliography{bibtext}

\begin{IEEEbiography}
[{\includegraphics[width=1in,height=1.25in,clip,keepaspectratio]{images/yvd5140.jpg}}]
{Yilu Dong}
is a Ph.D. student at Penn State University. He also 
received his M.S. and B.S. degrees from Penn State University. He is interested in communication protocols, software testing, and applied cryptography. His work focuses on the security of 5G systems. Specifically, improving the security and privacy of both UE and core network implementations. 
\end{IEEEbiography}

\begin{IEEEbiography} 
[{\includegraphics[width=1in,height=1.25in,clip,keepaspectratio]{images/rouzbeh_image.jpg}}]
{Rouzbeh Behnia}
is an assistant professor at the School of Information Systems and Management (SISM) at the University of South Florida. He received his Ph.D. in Computer Science from the University of South Florida.
His research focuses on different aspects of cybersecurity and applied cryptography. He is particularly interested in addressing privacy challenges in AI systems, developing post-quantum cryptographic solutions, and enhancing authentication protocols to ensure computation and communication integrity.
\end{IEEEbiography}

\begin{IEEEbiography}[{\includegraphics[width=1in,height=1.25in,clip,keepaspectratio]{images/AAY.jpg}}]{Attila A. Yavuz}
is an Associate Professor in the Department of Computer Science and Engineering and the Director of the Applied Cryptography Research Laboratory at the University of South Florida (USF). He was an Assistant Professor in the School of Electrical Engineering and Computer Science, at Oregon State University (2014-2018) and in the Department of Computer Science and Engineering, USF (2018-June 2021). He was a member of the security and privacy research group at the Robert Bosch Research and Technology Center North America (2011-2014). He received his Ph.D. degree in Computer Science from North Carolina State University (2011). He received his MS degree in Computer Science from Bogazici University (2006) in Istanbul, Turkey. He is broadly interested in the design, analysis, and application of cryptographic tools and protocols to enhance the security of computer systems. Attila Altay Yavuz is a recipient of the NSF CAREER Award, Cisco Research Award (thrice - 2019,2020,2022), and unrestricted research gifts from Robert Bosch (five times). He has authored over 100 products including research articles in top conferences, journals, and patents, some of which world-wide impact with actual deployments.
\end{IEEEbiography}

\begin{IEEEbiography}
[{\includegraphics[width=1in,height=1.25in,clip,keepaspectratio]{images/syed-rafiul-hussain-1.png}}]
{Syed Rafiul Hussain} (Member, IEEE) is an assistant professor in the Department of Computer Science and Engineering, Pennsylvania State University, State College, PA 16802 USA. His research interests include systems and network security, formal methods, program analysis and applied cryptography. He received a Ph.D. in computer science from Purdue University. He is a Member of IEEE and the Association for Computing Machinery.
\end{IEEEbiography}

\end{document}