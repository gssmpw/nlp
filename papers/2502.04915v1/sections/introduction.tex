\IEEEPARstart{5}{G} cellular networks are revolutionizing connectivity, offering faster speeds, greater bandwidth, and enhanced security compared to previous generations. These advancements are driven by innovative physical layer communication technologies and new security policies. Despite these leaps, 5G retains certain mechanisms from earlier generations to ensure seamless backward compatibility.
One such mechanism is the cell selection procedure used during the initial bootstrapping phase. In this phase, a user device selects a suitable base station that enables it to establish a connection with the core network and subsequently with the Internet.
Base stations periodically broadcast information about the network in \texttt{System Information} messages to announce their presence. Cellular devices scan these broadcast messages and connect to an appropriate base station that meets the cell (re-)selection criteria, which include the received signal strength of the broadcast messages, the base station's acceptability to the device, and the type of services offered by that base station.

Despite being critical to establish the root of trust for the communications between devices and networks, \texttt{System Information} broadcast messages in 5G networks are not authenticated. This allows an adversary to spoof~\cite{dabrowski2014imsi,hussain2019insecure} or tamper with~\cite{yang2019hiding} \texttt{System Information} messages by a fake base station emitting signals with a higher strength than that of the nearby legitimate base stations. After luring the cellular devices to connect to it, the fake base station can then launch security and privacy attacks, including DNS-redirection~\cite{rupprecht2019breaking}, denial-of-service (DoS)~\cite{shaik2015practical, hussain2018lteinspector, hussain20195greasoner, 3GPP:33.809, shaik2018impact, rashid2024state}, downgrade~\cite{shaik2015practical, hussain2018lteinspector, hussain20195greasoner, 3GPP:33.809, shaik2019new} attacks, battery depletion attacks~\cite{shaik2015practical, hussain2018lteinspector, hussain20195greasoner, 3GPP:33.809}, information leak attacks~\cite{hussain2018lteinspector, 5gbasechecker, chlosta20215g, park2022doltest, kim2019touching}, location tracking attacks~\cite{5gbasechecker, chlosta20215g, park2022doltest}, and fingerprinting attacks~\cite{5gbasechecker, park2022doltest, rashid2024state}. Although the 5G specifications \cite{3GPP:21.915} have introduced a new cryptographic scheme for preventing the exposure of cellular device's permanent identifier in plaintext, it does not address the root cause of the fake base station problem, which is the absence of authenticating the \texttt{System Information} broadcast messages. A broadcast authentication scheme is essential for a cellular device to verify the legitimacy of the base station it initially connects to. However, such schemes are currently lacking due to deployment challenges and concerns about backward compatibility. This paper aims to address this gap by proposing a practical authentication mechanism that secures the initial connection bootstrapping process between cellular devices and base stations.

Although symmetric-key primitives (e.g., HMAC) can provide efficient authentication, aside from their inherent key distribution and storage hurdles, they fail to provide public verifiability and non-repudiation. 
A recent study by 3GPP~\cite{3GPP:33.809} and other efforts~\cite{lee2009extended, yi1998optimized, zheng1996authentication, gao2021evaluating} have explored using certificate-based Public Key Infrastructure (PKI) or identity-based signature schemes~\cite{boneh2001short, cheng2017sm9, IEEE1363} to authenticate
base stations. However, these techniques are expensive in terms of communication and computation overheads at both the signer and verifier sides. 
Hussain et al.'s scheme~\cite{hussain2019insecure} demonstrates the viability of PKI-based authentication with optimizations like shorter certificates and an offline-online signature generation mechanism. However, it requires costly cryptographic operations and causes long delays in signature verification at the device. These issues are compounded in 5G networks, where higher frequency radio waves and frequent base station handovers introduce significant communication and computational overheads, making it difficult for existing schemes to be adopted in 5G specifications or deployed by service providers. 

In addition to meeting security and performance requirements, the solution must support lawful interception. Lawful interception enables law enforcement agencies to monitor communications for crime investigation and national security purposes while adhering to strict legal frameworks to balance privacy and accountability. For this, law enforcement agencies must authenticate their base stations by obtaining a key from a legitimate PKG. To uphold legal compliance and transparency, law enforcement should deploy temporary base stations only at authorized locations and times, with authentication keys that expire after use. Without a robust revocation scheme, expired keys could be exploited for unauthorized eavesdropping. This risk is particularly severe in identity-based cryptosystems, where user keys are derived from publicly available information.
Existing base station authentication schemes either neglect this scenario~\cite{hussain2019insecure} or lack efficient key management and revocation mechanisms~\cite{singla2021look}, limiting their practicality in real-world deployments.

In summary, a candidate protocol for authenticating initial broadcast messages in 5G networks must satisfy the following requirements: \textbf{[R1]:} The protocol must be efficient for both the signer and the verifier. It must minimize the computation overhead, especially on the verifier side, to preserve battery life for cellular devices without affecting the quality-of-service (QoS) and strict scheduling constraints of broadcast messages. 
\textbf{[R2]:} It must comply with the restriction on the maximum transmission unit of the broadcast radio messages, which further restricts the maximum size of public keys and signatures. In addition, the authentication protocol should limit the communication overhead due to certificates, signatures, and keys as minimal as possible, since additional bytes in broadcast radio packets transmitted over licensed spectrum induce additional costs to cellular service providers. 
\textbf{[R3]:} The protocol must be resilient against relay attacks by an adversary by just re-transmitting \texttt{System Information} messages from a legitimate base station without changes. \textbf{[R4]:} It can handle roaming scenarios, e.g. when the cellular device is outside the coverage area of its service provider and has to use the network of a partner cellular service provider. \textbf{[R5]:} Finally, the protocol can handle lawful interception for law enforcement agencies. Law enforcement agencies need to deploy their fake base stations to locate criminals and intercept their traffic. If an authentication scheme is deployed, law enforcement agencies must be able to authenticate their base stations to user devices.  

Our prior work proposes a broadcast authentication mechanism, \texttt{Schnorr-HIBS}~\cite{singla2021look}, using a hierarchical identity-based variant of Schnorr signature scheme. The proposed protocol introduces a new entity called core Private Key Generator (PKG) or core-PKG in the authentication server function in the 5G core network. Core-PKG generates public-private key-pairs for the Access and Mobility Management Function (AMF), which is the mobility anchor point in the core network and controls multiple base stations.
The AMF, in turn, generates public-private key-pairs for the base stations under its control. A base station uses its private key to sign \texttt{System Information} broadcast messages by following the proposed signature generation scheme to enable cellular devices to efficiently authenticate broadcast messages.
Although \texttt{Schnorr-HIBS} ensures better security and performance than the state-of-the-art, it has the following limitations: \ding{182} higher overhead in verification and communication 
and more attack surfaces due to the hierarchical architecture design, \ding{183} no practical solution for lawful interception, 
\ding{184} no end-to-end implementation and evaluation of the proposed scheme.

To address these challenges, in this paper, we introduce a novel \emph{verifier-efficient 2-layer Identity-Based Signature} (\scheme{}) scheme based on a highly efficient certificate-based method proposed in \cite{ARIS}. By leveraging the key-additive property of the new scheme, we propose a novel method to avoid the single point of failure common in most identity-based cryptosystems while allowing for fine-grained lawful interception.
Compared to \texttt{Schnorr-HIBS} \cite{singla2021look}, \scheme{} provides the following enhancements:  \ding{182} Simpler design and fewer attack surfaces due to the 2-layer design. \ding{183} Key generation for robust security and fine-grained lawful interception with the new key generation methods as discussed in Algorithm \ref{alg:IBS_lawful}. \ding{184} We incorporated the new scheme on an open-source 5G implementation and evaluated the overhead. \ding{185} 2x faster verification from the improved scheme. 
%We also provide several implementation optimizations, including fast elliptic-curves for generating and verifying the signatures, and pre-computing random tokens in an offline phase to further reduce the signing stage computations. %\revision{TODO: Our authentication protocol provides 205 times better verification performance as well as 21 times better signing performance than the cellular authentication scheme proposed by Hussain et al.~\cite{hussain2019insecure}. Our scheme also achieves a communication cost reduction of 46\% over their scheme.}

In summary, this paper makes the following contribution:
\ding{182} A comprehensive characterization of the attacks enabled by fake base stations for cellular networks.
\ding{183} \scheme{}, a verifier-efficient identity-based signature scheme and an authentication protocol, significantly reduces the attack surfaces and minimizes the performance overhead. 
\ding{184} A proof-of-concept implementation of our protocol and the integration with open-source 5G stack. We open-source our implementation \cite{E2IBS-github} with all the alternative schemes used for evaluation to provide a foundation for further research.