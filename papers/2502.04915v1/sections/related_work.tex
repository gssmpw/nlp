To address fake base station attacks, some defense techniques rely on improving fake base station detection and then blacklisting them. One such solution is to use measurement reports sent by UEs to detect inconsistencies between the tampered information being broadcast by the fake base stations and the legitimate base station deployment information like the base station identifier or operation frequencies of the base stations~\cite{3GPP:33.501}. Other techniques rely on machine-learning solutions~\cite{jin2019rogue, van2015detecting, engelstad2016strengthening} or  gathering surrounding network signal statistics from the UEs, legitimate base stations or other newly deployed hardware~\cite{dabrowski2014imsi, steig2016network, alrashede2019imsi, dabrowski2016messenger, li2017fbs}. Such techniques can be easily bypassed and have been shown to enable attacks resulting in degradation of network performance and blacklisting of legitimate base stations~\cite{shaik2018impact}. 

Some other approaches \cite{fan2019rehand, alnashwan2023privacy, 10.1145/3658644.3690331} focus on 5G handover to protect user privacy during the change of base stations. Although they also provide base station authentication in the scheme, due to the extra security properties, the performance can be much worse than our signature scheme. Moreover, these approaches require a significant change in the 5G protocol. That makes the adaptation more difficult.

Other IBS schemes \cite{meshram2021iboost, zhang2023identity, au2013realizing} focused on different applications with different security goals. Due to the specific limitations of the 5G system (e.g., message size limit and performance requirements), it is difficult to apply these schemes.