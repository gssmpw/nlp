In this section, we discuss the limitations of \texttt{Schnorr-HIBS} and introduce our new scheme. 

The preliminary version of \scheme{}, \texttt{Schnorr-HIBS} has a hierarchal design with 2 PKGs. As shown in Figure \ref{pre_protocol}, the core-PKG generates key pairs for AMFs as the first-level PKG. The AMFs, serve as second-level PKGs, generate key pairs for their corresponding base stations. Then, the base stations sign their \texttt{SIB1} messages using the received private keys and broadcast the message to the UEs. The UEs, with the $PK_{PKG}$ provisioned, need to receive all the public keys and identities from the base station and the AMF to verify the signature. 

\begin{figure*}[t]
 \centering
    \includegraphics[width=0.9\linewidth]{images/Schnorr_HIBS.pdf}
    \caption{Instantiation of \texttt{Schnorr-HIBS}. }
        \label{pre_protocol}
\end{figure*}

\subsection{\texorpdfstring{Limitations of \texttt{Schnorr-HIBS} \cite{singla2021look}}{Limitations of Schnorr-HIBS}}
\label{limitations}

\noindent \textbf{Hierarchal Architecture. }
\texttt{Schnorr-HIBS} proposed a hierarchal architecture signature scheme, which supports multiple layers from the PKG to the verifier. In the instantiation, the core-PKG and the AMF serve as the first- and second-level PKG. However, this hierarchal architecture has several limitations: \ding{182} Using AMFs in the middle creates more complexity for the scheme. The operators need to implement and manage the crypto functionalities in AMF, which introduces more cost. A misconfigured AMF can make the system unavailable and even leak its private keys. If an attacker has control over an AMF, it can sign its own fake base stations and the user cannot detect it from the signature. \ding{183} The core-PKG is centralized, it can be configured with powerful hardware and the signature extraction is very fast. However, the AMF is distributed and can have various types of configurations. It can be difficult to ensure service quality if one AMF is serving a large amount of base stations. \ding{184} Adding additional levels in the scheme will cause more overhead on the verifier side. Because the verifier needs to get the public keys for all layers and verify the signature with more arithmetic operations. In addition, the multi-level design introduces more communication costs. 

\noindent \textbf{Lawful Interception. }
\texttt{Schnorr-HIBS} assumed that the law enforcement department could obtain key pairs for base stations from the AMF and set up their fake base stations like a normal base station. However, it did not provide a fine-grained access control mechanism to limit the usage of the key and revoke the key if needed. In that case, the law enforcement department can set up a fake base station to intercept user traffic at any time and any location with a valid key pair. This may violate the authorized scope, and give the law enforcement department more power to track users than it should have. Furthermore, if the key is leaked, the attacker can set up \textit{authenticated} fake base stations, which are considered valid by the UE. 

\noindent \textbf{End-to-end Implementation. }
\texttt{Schnorr-HIBS} did not provide an end-to-end implementation for the proposed scheme. Thus, it overlooked the challenges of deploying the proposed scheme in the real world. For example, the hierarchal architecture requires new implementation for both core-PKG and AMF, increasing the complexity and the development and maintenance costs. Also, the base station needs to send the signature of \texttt{SIB1} along with the public key and identity of both the base station and the AMF to the UE. The total overhead is 150 bytes, which takes a large amount of the available space in \texttt{SIB1} (372 bytes) \cite{3GPP:38.331} message. A commercial base station may not be able to append this information after the existing configurations. 

\subsection{Design Decisions of \scheme{}}
\label{design_decisions}
 To address these limitations, 
we outline the detailed design of our protocol and the rationale behind the design decisions. 

\noindent \textbf{2-layer Design. } 
We specify a 2-layer architecture for our protocol: the core-PKG generates the keys for base stations and the base station creates the signatures. We use a 2-layer approach instead of a hierarchical approach where a core-PKG generates keys for AMFs and the AMFs generate the signing keys for the base stations for several reasons: 
In our approach, the AMF only needs to forward the \textit{keyext\_request} and the response from core-PKG. 
In our scheme, we are aiming to make the scheme both signer and verifier efficient.

\noindent \textbf{Fine-grained Lawful Interception. } 
We design our protocol with fine-grained lawful interception in mind. Algorithm \ref{alg:IBS_lawful} introduces key generation with a sequence number, ensuring that keys generated with a previous sequence number are implicitly invalidated when a new sequence number is used. This mechanism enables seamless key revocation and can be further enhanced by integrating fine-grained access control policies directly within the core-PKG, ensuring efficient and secure key management. 

\noindent \textbf{Minimize Bytes Sent Over-The-Air.} 
To comply with the current protocol and introduce a minimum overhead while satisfying modern security requirements, we use a Schnorr signature scheme. 
Only 111 bytes (see \ref{Sec:counterparts}) are required to send over-the-air to the UE. The communication overhead is 26\% smaller than the previous scheme.  

\noindent \textbf{Choice of Messages to Sign.} \texttt{System information} messages are broadcast periodically by the base stations to allow UEs to initiate a connection to them. \texttt{System Information} messages are divided into a \texttt{Master Information Block (MIB)} and multiple \texttt{System Information Block (SIB)} messages~\cite{3GPP:38.331}. \texttt{MIB} includes the basic parameters required by the UE to acquire the \texttt{SIB1} message. The \texttt{SIB1} message is the most important \texttt{System Information} message and contains the base station selection parameters, scheduling info for the rest of the SIB messages, whether one or more SIB messages are only provided on-demand, and configuration needed by the UE to perform the system information request.  
Since the MIB and SIB1 messages are two messages required for a UE to connect to a base station, our protocol signs the two messages together and provides the signature in the SIB1 message. After the UE receives the SIB1 message, it is able to authenticate the base stations. 

\noindent \textbf{Construction of Identities.} Our protocol requires assigning IDs to the base stations. We utilize the IDs for the dual purpose of uniquely identifying the base stations as well as for communicating the validity period of their signing keys. 
For $U_{BS}$ we use a concatenation of \texttt{NRCell\_ID}~\cite{3GPP:29.571} and an expiry timestamp. \texttt{NRCell\_ID} is a string of size 36 bits and uniquely identifies a base station for a particular mobile network operator. Each expiry timestamp is 32 bits long. Therefore, $U_{BS}$ can be a maximum of 9 bytes. 

\noindent \textbf{Validity period of the keys.}
\label{validity} Instead of using complex key revocation techniques, we assign different validity periods to each generated keypair after which the keys would need to be refreshed. For the core-PKG, we create the key-pair with a 1 year validity period by default as it needs to be installed inside the UE's USIM, and requires a confidentiality and integrity-protected channel to be updated. The core-PKG needs to be physically secured and protected so that its private key is not leaked. 
For the base stations, we generate a key pair valid for only 10 minutes. base stations are located around the world in physically insecure areas. Therefore, it may be easier for the attacker to compromise them. A validity period of only 10 minutes minimizes the period during which an attacker can launch attacks, even if it obtains a base station's private key. These validity periods are recorded in the expiry timestamps in the $U_{BS}$, as well as in the UE's USIM for the core-PKG. These are the default validity periods and can be changed by the network operators when required. Since our key generation is efficient (1000 keys per 5.5 milliseconds), its impact on core-PKG is negligible. 

\begin{figure*}[t]
 \centering
    \includegraphics[width=0.9\linewidth]{images/protocol.pdf}
    \caption{Our protocol for authenticating 5G cellular base stations.}
        \label{protocol}
\end{figure*}

\subsection{Protocol Description}\label{protocolDescription}
We now detail our authentication protocol steps. We abstract some cryptographic details for readability. 
For instance, we do not explicitly mention the mod operation, but all the operations in $E(\mathbb{F}_p)$ are executed in mod~$p$ and operations in $\mathbb{Z}_q$ are in mod~$q$. 
Figure~\ref{protocol} gives a graphical representation of our protocol for a 5G scenario. 

\subsubsection{Initialization phase for the core-PKG}
\noindent The core-PKG generates the public system parameters and its own public-private key pair during the initialization phase. From $sk_{PKG}$, core-PKG generates $t$ sets of public-private key pairs. $PK_{PKG}$ contains all $t$ public keys generated. This phase is executed at the beginning of the 5GC deployment. The default validity period of the core-PKG's keys is 1 year. The public key of the core-PKG along with its expiry date is installed in the USIM of all UEs during initial registration. The core-PKG's public key installed in the USIM has to be replaced, whenever the core-PKG refreshes its keys. This can be done using the confidentiality and integrity-protected channel created between the AMF and the UE after mutual authentication. The core-PKG uses the \texttt{Setup} step from Algorithm~\ref{alg:IBS} to generate its key pair
\{$sk_{PKG},\ PK_{PKG}$\}.

\begin{gather*}
sk_{PKG} \gets \mathbb{Z}_p \\
z_i \gets \PRF_{sk_{PKG}}(i), Z_i \gets z_i \times P \textbf{ for } i\in \{1,\dots,t\} \\
PK_{PKG} \gets \{Z_i\}_{i=1}^t
\end{gather*}

\subsubsection{Key extraction for the base station}
\noindent Base stations send a key extraction request with their \texttt{NRCell\_ID} to the core-PKG through their serving AMFs. The core-PKG concatenates the received \texttt{NRCell\_ID} with an expiry timestamp for the key being generated as $U_{BS}$. From $U_{BS}$, the core-PKG follows the \texttt{Extact} step from Algorithm~\ref{alg:IBS} to generate $sk_{BS}$ and $PK_{BS}$. The base stations have to periodically refresh their key-pair by sending the key-generation request to the core-PKG when nearing the key-pair expiration. 

\begin{gather*}
U_{BS} \gets \text{NRCell\_ID} || \text{Expiry\_Timestamp} \\
u \gets \mathtt{PRF}_{sk_{PKG}}(U_{BS}), PK_{BS} \gets u \times P \\
\{j_1, ..., j_k\} \gets \mathtt{H}_1(U_{BS}||PK_{BS}) \text{, where each } |j_i|=|t| \\
sk_{BS} \gets \sum^k_{i=1}PK_{PKG}[j_i] + u
\end{gather*}

\subsubsection{Signing phase at the base station} 
The base stations sign the MIB and SIB1 message via \texttt{Sign} step of Algorithm~\ref{alg:IBS} and generate the signature $sig_{SIB1}$. They attach the $sig_{SIB1}$, $PK_{BS}$, and $U_{BS}$ along with the SIB1 message broadcast. Before signing, the base station needs to ensure that their own keys have not expired. \begin{gather*}
rand_s \stackrel{\$}{\leftarrow} \mathbb{Z}_p, R \gets rand_s \times P\\
h \gets \mathtt{H}_2(\text{MIB}||\text{SIB1}||R)\\
s \gets r - h \times sk_{BS}
\end{gather*}
where $\langle s,\ h \rangle$ is the signature.

\subsubsection{Verification phase at the UE} 
The UE uses the $U_{BS}$ and the $PK_{BS}$ sent by the base station attached to the SIB1 message to verify $sig_{SIB1}$. The UE first verifies that the key $PK_{BS}$ is not expired by looking at the expiry timestamps embedded in the $U_{BS}$. If the timestamps have not expired, the UE computes the indices to select the corresponding public keys from $PK_{PKG}$ and then verifies the signature $sig_{SIB1}$.
For verification, the UE uses the public keys of core-PKG and the base station: 
\begin{gather*}
\{j_1,...,j_{k}\} \gets \mathtt{H}_1(U_{BS}||PK_{BS})\\
R' \gets s \times P + h \times (\sum^{k}_{i=1}(PK_{PKG}[j_i] + PK_{BS})\\
\text{If } h = \mathtt{H}_3(\text{MIB}||\text{SIB1}||R') \text{, then valid, else invalid.}
\end{gather*}
\noindent \textbf{Authentication failure action.} In case of authentication failure or the absence of authentication capabilities at the base station, the UE does not connect to the base station and keeps searching for other base stations available in the area. If there are no available base stations that can be authenticated, the UE can connect to an unauthenticated base station or keep looking for a base station that can be authenticated. We propose this to be a UE-specific choice, which can be configured depending on the mobile user's security/connectivity needs. If the UE decides to connect to an unauthenticated base station, it keeps checking the \texttt{System Information} messages to find a base station that can be authenticated.

\subsection{Handling Roaming Scenario}
\label{roaming}
Roaming services enable a UE to connect to base stations operated by a different network operator. Since each operator manages its own core-PKG, the UE must first obtain the public key of the roaming operator. The UE's primary operator can sign the roaming operator’s public key and provision it through non-3GPP access networks, such as Wi-Fi.

\subsection{Protection Against Relay Attacks}
Our authentication protocol protects against fake base stations by allowing UEs to authenticate \texttt{System Information} messages. However, it is vulnerable to relay attacks, where an adversary retransmits these messages with a stronger signal, tricking the UE into connecting to a fake base station. Distance-bounding protocols \cite{rasmussen2010realization, tippenhauer2015uwb, durholz2011formal} could prevent these attacks but would require major changes to cellular protocols. An alternative is to time-bound \texttt{System Information} message validity by estimating the time for an adversary to intercept and retransmit messages, though this doesn't account for base station frequency or coverage area. The 5G base stations can vary in configurations and use different frequencies to cover different ranges \cite{3GPP:38.104}, making the use of a fixed transmission time less practical. 

To protect against relay attacks in all scenarios, we propose time-bounding the \texttt{System Information} message signatures based on the base station’s configuration. The validity period, denoted by $\mathsf{\bigtriangleup t}$, is calculated from the configuration-specific transmission delay ($\mathsf{\bigtriangleup t_{conf}}$) and cryptographic signature delay ($\mathsf{\bigtriangleup t_{sign}}$). $\mathsf{\bigtriangleup t_{conf}}$ is configured by the operators and can be derived from a lookup table stored securely in the base station's memory. $\mathsf{\bigtriangleup t_{sign}}$ varies with the cryptographic scheme used. The base station signs the message with a timestamp $\mathsf{T_{sign}}$ and the validity period. The UE checks the validity by verifying if $\mathsf{T_{current} < T_{sign} + \bigtriangleup t}$ when receiving the message.