Here we evaluate \scheme{} with other state-of-the-art authentication schemes and report the experiment results. 
Furthermore, to demonstrate the practicality of our scheme, we incorporate our scheme into the open-source 5G implementation. The experiment results show our scheme has a negligible performance impact on the base stations and the UEs. 
The implementations used in the evaluation are available at \cite{E2IBS-github}. 
\subsection{Testbed Setup}

\noindent \textbf{Hardware and software components.}
We evaluate the efficacy of our scheme on a desktop machine with Intel Core i9-14900k and 64 GB DDR5 RAM, running Ubuntu 22.04. We use the PBC-library~\cite{pbc} for implementing pairing-based schemes and the FourQlib~\cite{fourqlib} for FourQ elliptic curves.

For the 5G testbed, we implemented our scheme on top of the open-source 5G stack implementation, OpenAirInterface~\cite{openAirInterface}. OpenAirInterface (OAI) has 5G Core, base station, and UE implementations. We modified the components to support our scheme. Since significant changes are required for a normal UE to authenticate the base station, we can only use our modified OAI-UE to conduct the evaluation. We plan to open-source our modified cellular stack to help further research in this field. We use two \textit{USRP B210}~\cite{USRP_B210} software-defined-radio (SDR) boards connected to the desktop machine mentioned above. One \textit{USRP B210} serves as the 5G base station and another serves as the 5G UE.

\subsection{Evaluation Results}
\noindent \textbf{Signature Schemes.}
\label{Sec:counterparts}
We consider 6 signature schemes for qualitative and quantitative comparisons with our scheme. Following the recommendations of 3GPP~\cite{3GPP:33.809}, we evaluate identity-based signature schemes BLS~\cite{boneh2001short}
We believe that BLS has been mentioned as an identity-based scheme by mistake. While one can devise an identity-based signature from BLS, such schemes are deemed to be very expensive (see Section 3.3 in~\cite{DBLP:series/ciss/KiltzN09}).  
3GPP also recommends the ECCSI scheme from RFC-6507~\cite{rfc6507}, which is based on the improved schemes proposed by Arazi et al.~\cite{Arazi-SelfCertified,araz2006load}. However, to the best of our knowledge, there is no provable security argument for these schemes, and the earlier versions of such schemes are insecure~\cite{Arazi-Attacked}. We thus omit this scheme from our comparisons. We include ECDSA~\cite{johnson2001elliptic} in our comparisons because of its wide adoption. We also include SCRA-BGLS~\cite{yavuz2017real}, since it is a recent proposal for base station authentication in 5G~\cite{hussain2019insecure}. 
In addition, we evaluate two fast authentication schemes based on the FourQlib, SchnorrQ \cite{SchnorrQ} and ARIS \cite{ARIS}. We also evaluate our preliminary scheme, \texttt{Schnorr-HIBS} \cite{singla2021look}. 
We have implemented the pairing-based schemes with the PBC-library~\cite{pbc} curve d224 which provides 112 bits of symmetric key security. These schemes are even slower for the 128-bit security level. All other schemes provide 128-bit symmetric key security according to NIST recommendations~\cite{keylength}. 
Table~\ref{comparison_schemes} summarizes our comparisons.

\begin{table*}[t]
\centering 
\renewcommand{\arraystretch}{}
			\begin{tabular}{|c||c|c|c|c|c|c|c|c|c|c|}
\hline
\multirow{2}{*}{\textbf{Scheme}} & \multicolumn{2}{c|}{\textbf{Sign}} & \multicolumn{2}{c|}{\textbf{Verify}} & \multirow{2}{*}{\begin{tabular}[c]{@{}c@{}}\textbf{Signature}\\ (B)\end{tabular}} & \multirow{2}{*}{\begin{tabular}[c]{@{}c@{}}\textbf{PK}\\ (B)\end{tabular}} & \multirow{2}{*}{\begin{tabular}[c]{@{}c@{}}\textbf{Crypto E2E }\\ \textbf{Delay} ($\mu$s)\end{tabular}} & \multirow{2}{*}{\textbf{System}} & \multirow{2}{*}{\begin{tabular}[c]{@{}c@{}}\textbf{Scheme} \\ \textbf{Type}\end{tabular}} & \multirow{2}{*}{\begin{tabular}[c]{@{}c@{}}\textbf{Lawful}\\ \textbf{Interception} \end{tabular}} \\ \cline{2-5}
 & \multicolumn{1}{l|}{$\mu$s} & \multicolumn{1}{l|}{Cycles} & \multicolumn{1}{l|}{$\mu$s} & \multicolumn{1}{l|}{Cycles} &  &  &  &  & & \\ \hline

ECDSA-256~\cite{johnson2001elliptic} & 521.14 & 1.662 & 172.78 & 0.550 & 64 & 32 & 693.92 & Flat & CB & No \\ \hline
BLS~\cite{boneh2001short}$\dagger$ & 1658.78 & 5.287 & 4957.02 & 15.799 & 48 & 96 & 6615.80 & Flat & CB & No \\ \hline
SCRA-BGLS~\cite{yavuz2017real} & 79.75 & 0.254 & 50265.27 & 160.208 & 29 & 85 & 50345.02 & Flat & CB & No \\ \hline
SchnorrQ~\cite{SchnorrQ} & 6.29 & 0.020 & 10.36 & 0.033 & 64 & 32 & 16.65 & Flat & CB & No \\ \hline
ARIS~\cite{ARIS} & 8.02 & 0.026 & 6.99 & 0.022 & 64 & t*32 & 15.01 & Flat & CB & No \\ \hline
\texttt{Schnorr-HIBS}~\cite{singla2021look} & 5.93 & 0.019 & 30.23 & 0.096 & 64 & 32 & 36.16 & Hierarchical & IDB & No \\ \hline
\hline
\scheme{} & 6.04 & 0.019 & 15.45 & 0.049 & 64 & 32 & 21.48 & Hierarchical & IDB & Yes \\ \hline
\end{tabular}

\begin{tablenotes}[list=off,flushleft]
\footnotesize{
\item All sizes are in bytes, and all computations are in microseconds. We also represent the number of CPU cycles for computation in millions.
\textbf{Signature} and \textbf{PK}  represent the signature size and public size, respectively. \textbf{Scheme Type} indicates whether the scheme is certificate-based (CB) or identity-based (IDB). \textbf{Crypto E2E Delay} for certificate-based schemes includes the verification of the sender's public key authenticity via certificates provided by a CA. For ARIS, we use t=1024, and for \scheme{}, we use t=1024 and k=18 in our evaluation. To be favorable to certificate-based schemes, we only consider a certificate chain of size 1. We implemented the pairing-based schemes with the PBC-library~\cite{pbc} curve d224, providing 112-bit security. These schemes are even slower for the 128-bit security level. All other schemes provide 128-bit security according to NIST recommendations~\cite{keylength}. 

$\dagger$ BLS is listed as an identity-based scheme by 3GPP~\cite{3GPP:33.809} but it is certificate-based (see Section \ref{Sec:counterparts}). 
}
\end{tablenotes}

\caption{Quantitative and qualitative comparison of the candidate signature schemes for authenticating cellular base stations.}	\label{comparison_schemes} 
\end{table*}

\noindent \textbf{Quantitative comparison.}
We evaluate signing and verification costs, end-to-end cryptographic delay, and total communication overhead of the compared schemes. 

\noindent \textit{\underline{Signing cost:}} Signature generation in \scheme{} only takes 6.04 $\mu$s which is 13x faster than SCRA-BGLS and 86x faster than ECDSA. \scheme{} is only slightly slower than Schnorr-HIBS. Keeping the signing time low is critical for base stations as new SIB1 messages are transmitted with a periodicity of 160 ms. Within this 160 ms, repetitive transmission occurs every 20 ms typically. A low signing overhead should also incentivize the network operators to configure the base stations to sign all broadcast messages providing full integrity protection.

\noindent \textit{\underline{Verification cost:}} Our scheme also has the second fastest verification phase taking just 15.45 $\mu$s. Our scheme outperforms the ECDSA by 11x and \texttt{Schnorr-HIBS} by 2x. Only ARIS and SchnorrQ are faster than \scheme{}. However, in the 5G setup, we need at least a 2-level certificate chain. The UE needs to verify both the signature of gNB's public key and the signature of SIB1, which makes ARIS slower than our proposed scheme. 
The verification phase is performed by UEs, which are usually resource-constrained devices, so keeping a low verification overhead is critical for saving energy, thus extending the battery life. Moreover, as 5G cellular networks are expected to deploy small base stations with much smaller coverage areas, UEs would be forced to switch between base stations at a much faster rate than with conventional base stations. 
The presence of (large numbers of) base stations with small coverage areas would require the UEs to execute the verification phase for the SIB1 message of base stations at a much higher rate, so keeping the verification overhead low is critical for 5G cellular networks. 

\noindent \textit{\underline{End-to-end cryptographic delay:}} We calculate the total cryptographic overhead of the signature schemes instantiated for cellular base station authentication, consisting of the signing and verification cost. The end-to-end delay for certificate-based schemes also includes the verification of the sender's public key authenticity via certificates provided by a certification authority. To be favorable to these schemes, we only consider a certificate chain of size 1. Our scheme provides the lowest end-to-end delay with only 21.48 $\mu$s, making it a very good candidate for authentication, especially in the presence of base stations with small coverage areas. 

\noindent \textit{\underline{Communication overhead:}} SCRA-BGLS has the smallest signature of 29 bytes 
, whereas our scheme \scheme{} and ECDSA-224 have a signature of size 64 bytes. Our scheme has the smallest public key size of 32 bytes, equivalent to ECDSA-256, SchnorrQ, and \texttt{Schnorr-HIBS}. Radio wireless channels are a limited resource, making any cryptographic scheme that adds a lot of communication overhead unlikely to be adopted by network operators. Our scheme provides the smallest communication overhead for authentication. It requires attaching the \scheme{} signature (64 bytes), base station's ID (9 bytes), base station's public key (32 bytes) along with the signing timestamp (4 bytes), and the signature validity period $\mathsf{\bigtriangleup t}$ (2 bytes) for relay attack prevention. This is a total overhead of 111 bytes, which is much lower than the communication overhead for ECDSA-based PKI (277 bytes) and SCRA-BGLS-based PKI (220 bytes). Our scheme can fit in the spare space of SIB1 perfectly. 

\noindent \textbf{Qualitative comparison.}
We compare the schemes based on the type of authentication system and the type of scheme.

\noindent \textit{\underline{System:}} A multi-layer construction is crucial for an identity-based signature scheme suitable for the cellular network architecture (see discussion in Section~\ref{design_decisions}). Both our scheme and \texttt{Schnorr-HIBS} are identity-based and pairing-based. \texttt{Schnorr-HIBS} supports a hierarchal construction while \scheme{} has a fixed two-layer design. On the other hand, signature-based schemes use certificate chains to support multiple signer levels. 

\noindent \textit{\underline{Type of scheme:}} ECDSA, BLS, SCRA-BGLS, SchnorrQ, and ARIS are certificate-based schemes and require a costly public key infrastructure (PKI). On the other hand, \texttt{Schnorr-HIBS} and \scheme{} are identity-based schemes and are more lightweight than certificate-based solutions as they do not require sending huge certificates. However, since the PKI is not available, the user cannot detect a compromised PKG. 

\subsection{Evaluation on 5G Platform}
\noindent \textbf{Implementation Details:}
To measure the performance impact of our scheme, we integrated our scheme into OpenAirInterface. Since the SIB1 message is periodic, we created a separate thread in the base station to pre-generate the next signature. When the current SIB1 message is sent, we calculate the next timestamp and signature for the next SIB message. We also modified the UE part to verify the signature. If the signature is not valid, the UE should abort the registration process. 

\noindent \textbf{Results and Analyzes:}
We measured the SIB1 generation time on the base station, the SIB1 processing time on the UE, and the time difference between the signed timestamp and the time UE verifies the signature. The results are shown in Table \ref{tab:OAI_eval}. During the experiment, the local time of gNB and UE must be synchronized to minimize the errors. To make our result accurate, we run two \textit{USRP B210}s on the same desktop machine. The result shows that our scheme has negligible performance impacts on SIB1 generation and the time difference. The result shows our scheme can generate the SIB with signature in a short time, without adding a heavy load of the base stations. We also observed an 84\% increase in the SIB1 processing time on the UE. However, since the UE does not need to verify the base station very often, the ~5ms time increase can hardly be perceived by the user. 

\begin{table}[t]
    \centering
    \begin{tabular}{|c|c|c|c|}
    \hline
                    &  SIB1 Gen & SIB1 Proc & Time $\Delta$\\
    \hline
    w/o \scheme{}   & 23.99 & 6178.12 & 3.28 \\
    \hline
    w/ \scheme{}    & 24.72 & 11337.08 & 3.40 \\
    \hline
    \end{tabular}
    \caption{Execution time ($\mu$s) in 5G platform}
    \label{tab:OAI_eval}
\end{table}