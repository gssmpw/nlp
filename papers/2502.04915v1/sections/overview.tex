\noindent \textbf{Adversary Model.}
We consider a Dolev-Yao adversary model~\cite{dolev1983security} in which the adversary can drop, modify, inject, or eavesdrop on messages sent by legitimate participants over the public radio channel. 
According to this model, the adversary is capable of setting up fake base stations and emitting unauthenticated broadcast messages with a higher signal strength than the legitimate base stations. 
However, the adversary cannot physically access and tamper the legitimate base stations, cellular devices, or core network components, and cannot access the secret keys or other sensitive information stored in a target cellular device's USIM or base stations.

\noindent \textbf{Scope of our Solution.}
Our solution allows cellular devices to reliably authenticate a base station before establishing a connection
by ensuring the authenticity of the public broadcast messages. We do not consider passive attacks caused by adversaries eavesdropping the traffic between the target device and legitimate base stations over the public radio channel. We also do not consider DoS attacks using a wireless jammer operating at the physical layer. Finally, our solution is envisioned for 5G cellular networks but can be extended to 4G LTE, 3G, and 2G networks with minimal modifications. 

\noindent \textbf{Our Authentication Protocol.}
Our protocol allows a UE to verify the identity of the base station it is connecting to and validate the authenticity of \texttt{System Information} messages being sent by the base station. Our protocol is based on an IBS scheme (details in Section~\ref{Crypto Scheme}) and organized according to a 2-layered system consisting of: \ding{182} the core-PKG (hosted by the 5GC),  and \ding{183} base stations. We provide a high-level overview of our authentication protocol below (see Section~\ref{Detailed Design} for further details). 

The core-PKG, co-located with the Authentication Server Function (AUSF) in the core network, is responsible for generating the public-private key pairs for all the AMFs deployed for a particular network operator. 
At first, core-PKG generates its public-private key pair $[sk_{PKG}, PK_{PKG}]$ during the initial setup phase. The $PK_{PKG}$ is then provisioned in the USIM of all UEs for that particular network operator. 

The base stations controlled by that network operator periodically send a key generation request to their network operator's core-PKG. The base stations send their cell IDs, i.e., \texttt{NRCell\_ID} to the core-PKG and receive a public-private key pair [$sk_{BS}$, $PK_{BS}$] and the base station identity $U_{BS}$. The $U_{BS}$ is a concatenation of \texttt{NRCell\_ID} and the expiration timestamp of the particular key pair.

The base stations use their assigned private key $sk_{BS}$ to sign the \texttt{System Information} broadcast messages using the \scheme{} (Section~\ref{Crypto Scheme}) and generate a signature $sig_{SIB1}$. The base stations include $sig_{SIB1}$, $PK_{BS}$, and $U_{BS}$ in SIB1 message.
The UE uses this information to verify $sig_{SIB1}$. The UE first verifies if the key $PK_{BS}$ is expired by checking the expiry timestamps embedded in the $U_{BS}$. If the timestamp is not expired, the UE verifies the signature of the \texttt{System Information} message. If this verification step is successful, the UE connects to the base station.