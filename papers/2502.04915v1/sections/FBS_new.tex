Fake base stations (FBS) have been demonstrated to be feasible in real-world scenarios using Commercial-Off-The-Shelf (COTS) hardware and open-source cellular software stacks~\cite{strobel2007imsi, paget2010practical, kune2012location, shaik2015practical}. To operate a fake base station, an attacker configures their radio to transmit signals at a higher strength than legitimate base stations, enticing users to connect to it instead. Figure~\ref{FBS_a} illustrates a typical fake base station setup used for conducting off-path attacks, while Figure~\ref{FBS_b} depicts its configuration for executing Man-in-the-Middle (MitM) relay attacks.

\begin{figure}[t]
 \centering
    \begin{subfigure}{1\linewidth}
    \includegraphics[width=\linewidth]{images/FBS_a.pdf}
        \caption{Setup for off-path attacks.}
        \label{FBS_a}
    \end{subfigure}
    
    \begin{subfigure}{1\linewidth}
    \includegraphics[width=\linewidth]{images/FBS_b.pdf}
        \caption{Setup for man-in-the-middle attacks.}
        \label{FBS_b}
    \end{subfigure}
        \caption{Common FBS configurations for carrying out attacks.}
    \end{figure}

To mitigate fake base station attacks, cellular protocol specifications have introduced enhancements across multiple generations, including 3G, 4G LTE, and 5G. A notable advancement in 5G is the introduction of the Subscription Concealed Identifier (SUCI), which protects against IMSI catchers. SUCI is an encrypted identifier used in the UE registration procedure, designed to conceal the Subscription Permanent Identifier (SUPI)---commonly known as the IMSI (International Mobile Subscriber Identity) in earlier generations. By rotating encryption keys, SUCI can change over time, making it significantly harder for IMSI catchers to track users. However, SUCIs are still transmitted over-the-air, which means an IMSI catcher could temporarily track a user if the SUCI value remains unchanged. Moreover, some network operators do not enable SUCI encryption in their systems, further undermining its effectiveness~\cite{nie2022measuring}.

\begin{table*}[ht]
\setlength\tabcolsep{4pt}
\renewcommand{\arraystretch}{0.6}
\fontsize{8}{6}\selectfont
\newcolumntype{P}[1]{>{\centering\arraybackslash}p{#1}}
\centering
\begin{tabular} {| c | P{2.8cm} | P{7.6cm} |}
\hline
\textbf{Attack} & \textbf{Attack Category} & \textbf{Impact} \\ \hline

Send RRC or NAS Reject messages \cite{shaik2015practical, hussain2018lteinspector, hussain20195greasoner, 3GPP:33.809} & DoS; Downgrade; Battery Depletion; & Denial of services; force UE to downgrade to older radio technology; Increase in power consumption for UE \\ \hline

Replay \texttt{RRC\_Resume\_Request}~\cite{3GPP:33.809} & DoS & Denial of services \\ \hline

\texttt{Authentication\_Request} with separation bit 0~\cite{rashid2024state} & DoS & Denial of services \\ \hline

Manipulate Self Organizing Networks (SON)~\cite{shaik2018impact} & DoS; Battery Depletion & Call dropping; Increase in power consumption for UE; Increased handovers and signaling load; Legitimate base station blacklisted \\ \hline

Modify \texttt{UE\_Capability\_Information}~\cite{shaik2019new} & Downgrade & Denial of some services; Lower data rate; Downgrade to 2G/3G \\ \hline

Authentication Relay Attack~\cite{hussain2018lteinspector} & Information Leak; DoS & Complete or selective DoS; Location history poisoning; Network profiling \\ \hline

5G AKA Bypass~\cite{5gbasechecker} & Information Leak & Monitor and manipulate user traffic; Provide Internet access; Phishing \\ \hline

NAS \texttt{Security\_Mode\_Command} replay~\cite{5gbasechecker} & Location Tracking; Fingerprinting & Check if the UE in the range of base station; Fingerprinting UE model \\ \hline

SUCI-Catcher~\cite{chlosta20215g} & Location Tracking; Information Leak &  Obtain user identifier; Track user movement\\ \hline

IMEI-Catcher~\cite{park2022doltest} & Location Tracking; Information Leak &  Obtain user identifier; Track user movement \\ \hline

Lullaby~\cite{hussain20195greasoner} & DoS; Battery Depletion &  Move UE to idle state; Increase in power consumption for UE \\ \hline

NAS counter reset~\cite{hussain20195greasoner} & DoS; Battery Depletion & Force UE reconnect; Increase in power consumption for UE \\ \hline

Authentication Sync Failure Attack~\cite{hussain20195greasoner} & DoS; Battery Depletion & Force UE reconnect; Increase in power consumption for UE \\ \hline

\texttt{Counter\_Check} Fingerprinting~\cite{park2022doltest, 5gbasechecker} & Fingerprinting & Fingerprinting UE model \\ \hline

\texttt{Authentication\_Request} Fingerprinting~\cite{rashid2024state} & Fingerprinting & Fingerprinting UE model \\ \hline

RRC \texttt{Security\_Mode\_Command} Bypass~\cite{kim2019touching} & Information leak & Monitor and manipulate user traffic  \\ \hline
\end{tabular}

\caption{Attacks enabled by fake base stations in 4G \& 5G cellular networks. }
\label{fbs_impact}
\end{table*}

Despite some advancements, these defenses fail to address the root cause of fake base station attacks: the lack of authentication for base stations during the connection bootstrapping process. Consequently, fake base station attacks remain feasible, even in 5G networks. To better understand the causes and implications of these attacks, we analyze fake base station attacks found in recent studies. A summary of these attacks and their impact is provided in Table~\ref{fbs_impact}.

\noindent \textbf{Denial-of-Service.}
Denial-of-Service (DoS) attacks are among the most prevalent tactics used by an FBS (Fake Base Station) attacker. By sending a reject message as defined in the specifications \cite{shaik2015practical, hussain2018lteinspector, hussain20195greasoner, 3GPP:33.809, rashid2024state}, the attacker can prevent the User Equipment (UE) from connecting to the network. Additionally, the attacker can manipulate the order or fields of control-plane messages \cite{3GPP:33.809, shaik2018impact, hussain2018lteinspector, hussain20195greasoner} to achieve a similar effect. As a result, the affected user cannot connect to a legitimate base station, rendering them unable to receive SMS, phone calls, or access the Internet. 

\noindent \textbf{Battery Depletion.}
Most cellular devices rely on battery power, making them vulnerable to energy-draining attacks. An FBS attacker can disrupt the UE's connection to the base station, forcing it into a reconnection loop that rapidly depletes the battery \cite{shaik2015practical, hussain2018lteinspector, hussain20195greasoner, 3GPP:33.809, shaik2018impact}. The frequent cell selection and registration procedures consume significant power, ultimately preventing the user from operating their device.

\noindent \textbf{Downgrade.}
In a downgrade, or bidding-down attack, the attacker forces the UE to connect to an older radio generation (e.g., 2G, 3G, LTE). This can be achieved by leveraging specific reject messages (e.g., \texttt{5GS\_Services\_Not\_Allowed}) \cite{shaik2015practical, hussain2018lteinspector, hussain20195greasoner, 3GPP:33.809} or manipulating fields in control-plane messages \cite{shaik2018impact}. Consequently, the user loses the benefits of modern radio technologies, such as faster speeds and stronger cryptographic protections. Since older generations like 2G and 3G employ weaker cryptography, attackers can exploit these connections to send fake SMS or execute other malicious activities.

\noindent \textbf{Information Leak.}
Certain FBS attacks can lead to sensitive information leakage \cite{hussain2018lteinspector, 5gbasechecker, chlosta20215g, park2022doltest, kim2019touching}. For instance, SUCI-Catcher \cite{chlosta20215g} and IMEI-Catcher \cite{park2022doltest} get the identifier of the user or device and can track the user's location. The Authentication Relay Attack \cite{hussain2018lteinspector}, 5G AKA Bypass \cite{5gbasechecker}, and RRC \texttt{Security\_Mode\_Command} Bypass attack can relay or bypass the authentication procedure between the base station and the UE, which causes the user traffic unencrypted. The attacker can monitor and manipulate the traffic and even provide Internet access to the user \cite{5gbasechecker}. Thus, the attacker can hijack users into phishing websites and perform more complicated attacks. 

\noindent \textbf{Fingerprinting.}
By analyzing UE responses to identical messages, attackers can infer device characteristics such as the baseband vendor or software version \cite{5gbasechecker, park2022doltest, rashid2024state}. With this information, attackers can execute targeted attacks tailored to specific devices or software implementations. 

\noindent \textbf{Location Tracking.}
Many attacks \cite{5gbasechecker, chlosta20215g, park2022doltest} enable location tracking of a specific UE. IMEI-Catcher \cite{park2022doltest} captures the permanent identifier of the device, IMEI (International Mobile Equipment Identity), and SUCI-Catcher \cite{chlosta20215g} captures SUCI, which is an identifier of the user. With an FBS network, the attacker can track user location if the same identifier is used elsewhere. Additionally, the NAS \texttt{Security\_Mode\_Command} replay attack \cite{5gbasechecker, hussain2018lteinspector} can verify the presence of a UE within the attacker's range by replaying previously successful \texttt{Security\_Mode\_Command} messages.

To mitigate these threats, ensuring UE authentication of base stations is crucial. This paper aims to address this gap and propose solutions to enhance security against FBS attacks.