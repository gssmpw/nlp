In this section, we first present some notations that will be used throughout the paper. We then briefly describe the architecture of the 5G cellular networks. We also introduce identity-based signatures (IBS), the basic building block of our authentication protocol.

\noindent \textbf{Notations.}  Given two primes $p$ and $q$, we define a finite field $\Fq$~and a group $\ZZ$.  We use \EC~as an elliptic curve (EC) over $\Fq$, where  $P\in \EC$ is the generator of the points on the curve.  We denote a scalar and a point on a curve with small and capital letters, respectively, e.g.,  $ x\Ra S $ denotes a random uniform selection of  $x$ from a set $S$. $||$ denotes string concatenation.  We define two hash functions  $ \h_1$: $ \{0,1\}^{*}  \rightarrow \{0,1\}^{k|t|}$  and  $ \h_2$: $ \{0,1\}^*  \rightarrow   \{0,1\}^{k'|k|}$, for parameters $t,k,k'\in \mathbb{N}$ defined in Section \ref{sec:scheme}. We view these hash functions as random oracles in our security 
analysis~\cite{RandomOracleModel93}.  We define $\PRF_x(\cdot)\rightarrow \mathbb{Z}_p$ as a Pseudo-Random Function initialized by secret $x$. $|x|$ defines the bit-length of variable $x$, (i.e.,$|x|\gets\log_2(x)$).
 
\subsection{5G Cellular Network Architecture}

A 5G cellular network consists of 3 main components (see Figure~\ref{network_architecture}): User Equipment (UE), Next Generation Radio Access Network (NG-RAN) and the 5G Core Network (5GC). 

\noindent  \textbf{UE} refers to the subscriber device (e.g., cell phone) used to access the cellular network. The UE is provided with a Universal Subscriber Identity Module (USIM) card, provisioned by 
a mobile network operator with the permanent identity of the UE, the Subscription Permanent Identifier (SUPI). 

\noindent \textbf{NG-RAN} consists of base stations that UEs can connect to using radio transmission. The base stations broadcast \texttt{System Information} messages, including a Master Information Block (MIB) and multiple System Information Block (SIB) messages periodically. 
MIB and SIB1 together are referred to as \textit{minimum SI}
to enable further communication between the UE and the base stations. The UE listens for SI messages and connects to the base station with the highest signal strength. 

\begin{figure}[t]
 \centering
        \includegraphics[width=\linewidth]{images/network_architecture.pdf}
        \caption{Cellular Network Architecture.}
        \label{network_architecture}
\end{figure}

\noindent  \textbf{5GC} is the brain of the 5G cellular network and houses several components to provide services to the UEs. An important component is the Access and Mobility Management Function (AMF).
supports UE authentication, mobility management, and paging, handles the NAS layer traffic and security, and checks UE's roaming rights. The AMF authenticates the UE in collaboration with the Authentication Server Function (AUSF) and Unified Data Management (UDM). 

\subsection{Identity-Based Signatures}

Digital signatures, achieved via public key infrastructure (PKI), offer several security properties, including message and sender authentication. 
In conventional PKI-based cryptography, the authenticity of public keys is derived through certificates. 
Communicating and verifying these certificates can incur additional overhead; this is especially important for mobile and battery-powered devices.  
Certificate-free cryptography was designed to address this overhead by deriving public keys directly from the user's identifying information, eliminating the need to communicate and verify these certificates. 
This is achieved by enabling a trusted third party (TTP) to compute the user's private key based on their identifying information (e.g., MAC address). 
We formally define the notion of identity-based digital signature in the following definition. 

\begin{definition}\label{def:ibs}

    An identity-based signature scheme $\IBS=(\setup,\keygen,\sign,\verify)$ is defined as follows. 
    \begin{itemize}
        \item  $(msk,params)\gets\IBS.\setup(1^\kappa)$: Given the security parameter $\kappa$, computes TTP's key pair ($msk,mpk$)  and the system parameters $params$. 
                For conciseness, we exclude $params$ as the input of the following algorithms.
                \item $(sk_U,C_U)\gets\IBS.\keygen(msk,U)$: Given the user identity $U$ and $msk$, the TTP computes a commitment value $C_U$ and the secret key $sk_U$.
                \item $\sigma_{m,U}\gets\IBS.\sign(m,sk_U)$: Given a message $m$ and the user's secret key $sk_U$, the user computes the signature $\sigma_{m,U}$.
                \item $\{\text{valid},\text{invalid}\}\gets\IBS.\verify(m,\sigma_{m,U},U,C_U,mpk)$: Given a message-signature pair $(m,\sigma_{m,U})$, the user's identity $U$, its commitment value $C_U$, and the master public key, this algorithm outputs either `valid' or `invalid.' 
            \end{itemize}
   
\end{definition}
The security of an IBS scheme is defined in the following.
\begin{definition}\label{def:eucma}
    The existential unforgeability under the chosen message attack (EU-CMA) for an identity-based signature scheme $\IBS$ is defined via the following experiment $Expt^{\text{EU-CMA}}_{\IBS}$ between a challenger \B~and the adversary \A. 
    \begin{itemize}
        \item The challenger \B~runs $\IBS.\setup()$ and sends $params$, including $mpk$, to \A.
                \item \A~interacts with the $(sk_{U_i},C_{U_i})\gets\mathcal{O}_{Corrupt}({U_i})$ and $(\sigma_{m,{U_i}})\gets\mathcal{O}_{Sign}(m,{U_i})$ oracles for any ${U_i}\in \{U_1, \dots, U_n\}$. 
            \end{itemize}
       \A~succeeds in the above experiment if it outputs a message-signature pair $(m^*,\sigma_{m^*,U^*})$ after a polynomially bounded number of interactions with the above oracles, given $U^* \notin \{U_1, \dots, U_n\}$ and $\sigma_{m^*,U^*}$ was never outputted from $\mathcal{O}_{Sign}(\cdot)$.  
\end{definition}

\begin{definition}\label{def:ecdl} Given an elliptic curve \EC~over a finite field \Fq, and $P,Q\in\EC$, the Elliptic Curve Discrete Logarithm (ECDL) problem asks to find $a\in\ZZ_p$ where $Q=aP \mod q$
    
\end{definition}