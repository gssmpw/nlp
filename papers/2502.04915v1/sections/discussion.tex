\noindent\textbf{Relay attacks.} Our time-bounding technique for relay attack protection is a best-effort approach. It cannot completely thwart relay attacks, rather it raises the bar for the attackers. A precise defense for relay attack would require a sophisticated approach with major changes in the protocol (e.g., including new messages) and a precise estimation of the timing/latency of message transmission and calculation, environmental interference, and hardware used by base stations and
cellular devices. We leave this for future work.

\noindent\textbf{Emergency services.} According to 3GPP~\cite{3GPP:33.501}, devices without SIM/USIM/eSIM do not perform authentication with the network for emergency calls/SMSs. To authenticate base stations in cases where a SIM was installed in the cellular device but was subsequently removed, the device can use a cached copy of the public key of their network operator's core-PKG in the UICC. We do not support base station authentication if a SIM was never inserted on a cellular device.

\noindent\textbf{UICC vs. UE.} 3GPP recommends either UICC or UE for public key encryption of SUPI~\cite{3GPP:33.501}. We, therefore, envision the implementation of signature verification at UE.

\noindent\textbf{Backward compatibility.} Our solution is backward-compatible since base stations only include the information required for our scheme in the \emph{optional fields} of SIB messages. Legacy devices ignore those fields and authentication.