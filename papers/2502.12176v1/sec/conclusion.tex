\section{Summary}
\label{sec:summay}
This paper provides a broad and quantitative examination of ten challenging problems inherent in FedFMs, encompassing issues of foundational theory, utilization of private data, continual learning, unlearning, Non-IID and graph data, bidirectional knowledge transfer, incentive mechanism design, game mechanism design, model watermarking, and efficiency. 
In this section, we unify the objective functions of these ten challenging problems in FedFMs into the following equation: 
\begin{equation}\label{eq:unifed}
    \begin{split}
        \min&_{w_g=(w_s, \{w_k\}_{k=1}^K), \{a_k\}_{k=1}^K,F,C}F_{k\in[K], t\in [T] }\Big( \\
        &\alpha_1\underbrace{\ell_{u,k}(w_s,w_k,a_k,C,\{D_k^t\}_{t=1}^T, G)}_{\text{Utility loss}}, \\
        &\alpha_2\underbrace{\ell_{e,k}(w_s,w_k,a_k,C,\{D_k^t\}_{t=1}^T,G)}_{\text{Efficiency loss}}\\
               &\alpha_3\underbrace{\ell_{m,k}(w_s,w_k,a_k,C,\{D_k^t\}_{t=1}^T, G)}_{\text{Watermark loss}}, \\
              &\alpha_4\underbrace{\ell_{c,k}(w_s,w_k,a_k,C,\{D_k^t\}_{t=1}^T, G) }_{\text{Contribution loss}}, \\
              & \alpha_5\underbrace{\ell_{p,k}(w_s,w_k,a_k,C,\{D_k^t\}_{t=1}^T, G) \Big)}_{\text{Privacy loss}} \\
            & s.t.\quad  \underbrace{\ell_u(w_g) + \ell_p(w_g)+\ell_e(w_g)>0}_{\text{No free lunch constraint}},
    \end{split}
\end{equation}
where $\ell_{u,k}$, $\ell_{e,k}$, $\ell_{m,k}$, $\ell_{c,k}$, and $\ell_{p,k}$ denote the utility loss, efficiency loss, watermark loss, contribution loss, and privacy loss, respectively. These losses are defined in terms of several parameters: aggregation mechanism $F$, contribution evaluation mechanism $C$, server's model $w_s$, client $k$'s model $w_k$, and action $a_k$. The coefficients $\alpha_1$, $\alpha_2$, $\alpha_3$, $\alpha_4$ and $\alpha_5$ take values in the range $[0, 1]$ and meet $\sum_{i=1}^5\alpha_i=1$. 
Specifically, when $\alpha_1 = 1$ and $\alpha_2=\alpha_3=\alpha_4 = \alpha_5= 0$, Eq.  \eqref{eq:unifed} minimizes the utility loss through the utilization of private data, continual learning data, graph data (i.e., problem 2, 3, 5, 6, 8) When $\alpha_2 = 1$ and $\alpha_1=\alpha_3=\alpha_4 = \alpha_5= 0$, Eq.  \eqref{eq:unifed} optimizes the efficiency loss (i,e., problem 10). When $\alpha_3 = 1$ and $\alpha_1=\alpha_2=\alpha_4 = \alpha_5= 0$, Eq.  \eqref{eq:unifed} optimizes the watermark loss (i.e., problem 9). When $\alpha_4 = 1$ and $\alpha_1=\alpha_2=\alpha_3 = \alpha_5= 0$, Eq.  \eqref{eq:unifed} optimizes the contribution loss (i.e., problem 7). When $\alpha_5 = 1$ and $\alpha_1=\alpha_2=\alpha_3 = \alpha_4= 0$, Eq.  \eqref{eq:unifed} optimizes the privacy loss (i.e., problem 4). Finally, the constraint of Eq.  \eqref{eq:unifed} is based on the no free lunch theory of FedFMs (i.e., problem 1). To the best of our knowledge, \textit{this unified equation is the first to present the key problems of FedFMs from a mathematical perspective}.


In summary,  this paper delves into the ten challenging problems faced by FedFMs, which are outlined in Table \ref{tab:framework}. 
Each of the ten problems in FedFMs can be formulated as an optimization problem with a specific objective function. These objective functions, when unified, enable a holistic analysis of the trade-offs and boundaries across different dimensions. The distributed learning paradigm in FedFMs, which emphasizes mutual learning between large FMs and smaller DMs, exhibits high generalization capabilities. The optimization objective in this paradigm is to achieve a balance across multiple sub-objectives.
By overcoming these challenges, this paper underscores the importance of advancing FedFMs to unlock their full potential.
Our ultimate goal is to foster a distributed model ecosystem that emphasizes mutual learning between FMs in the server and local DMs in the clients, ensuring robust, efficient, and privacy-preserving FedFMs.  This will not only advance the theoretical foundations of FedFMs but also facilitate their widespread adoption in real-world applications, paving the way for revolutionary advancements across various domains, including healthcare, finance, and IoTs.






