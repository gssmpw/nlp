\begin{abstract}
Fine-tuning large language models (LLMs) using diverse datasets is crucial for enhancing their overall performance across various domains.
In practical scenarios, existing methods based on modeling the mixture proportions of data composition often struggle with data whose domain labels are missing, imprecise or non-normalized, while methods based on data selection usually encounter difficulties in balancing multi-domain performance.
To address these challenges, in this paper, we study the role of data diversity in enhancing the overall abilities of LLMs by empirically constructing contrastive data pools and theoretically deriving explanations for both inter- and intra-diversity. 
Building upon the insights gained, we propose a new method that gives the LLM a dual identity: an output model to cognitively probe and select data based on diversity reward, as well as an input model to be tuned with the selected data.
Extensive experiments show that the proposed method notably boosts performance across domain-undetermined data and a series of foundational downstream tasks when applied to various advanced LLMs. We release our code and hope this study can shed light on the understanding of data diversity and advance feedback-driven data-model co-development for LLMs.
\end{abstract}

