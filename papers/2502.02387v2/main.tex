
\documentclass[letterpaper,twocolumn,10pt]{article}
\usepackage{hyperref}
\PassOptionsToPackage{table}{xcolor}
\usepackage{usenix2019}

\usepackage{booktabs,multirow,multicol}
\usepackage{xspace}
%\usepackage{natbib}
\usepackage{graphicx,xcolor}
\usepackage{cite}
\usepackage[nointegrals]{wasysym}
\usepackage{pifont}
\usepackage{threeparttable}
\usepackage{graphicx}
\newcommand{\cmark}{\ding{51}}%
\newcommand{\xmark}{\ding{55}}%
\newcommand{\omark}{\ding{108}}
% \usepackage{bbding}
\usepackage[official]{eurosym}
\usepackage{makecell}
% to be able to draw some self-contained figs
\usepackage{tikz}
%\usepackage{amsmath}
\usepackage{amsthm}
\usepackage{enumitem}
\usepackage{url}
\usepackage{amssymb}
\newtheorem{theorem}{Theorem}
\theoremstyle{definition}
\newtheorem{definition}{Definition}[section]
\newcommand{\blue}[1]{ {\textcolor{blue}{#1}}}
\newcommand{\mypara}[1]{\noindent\textbf{{#1: }}}
\newcommand{\ljk}[1]{\textcolor{brown}{#1}}
\newcommand{\zk}{\text{zk-SNARK}\xspace}
\newcommand{\Zk}{\text{zk-SNARK}\xspace}
\newcommand{\ZK}{\text{zk-SNARK}\xspace}
\newcommand{\delete}[1]{}
\newcommand{\PCP}{\text{PCP}\xspace}
\newcommand{\IOP}{\text{IOP}\xspace}
\newcommand{\IP}{\text{IP}\xspace}
\newcommand{\PIOP}{\text{PIOP}\xspace}
\newcommand{\PCS}{\text{PCS}\xspace}
\newcommand{\KZG}{\text{KZG}\xspace}
\newcommand{\FRI}{\text{FRI}\xspace}
\newcommand{\IPA}{\text{IPA}\xspace}
\newcommand{\MIP}{\text{MIP}\xspace}
\newcommand{\LPCP}{\text{LPCP}\xspace}
\newcommand{\new}[1]{{#1}\xspace}
\newcommand{\news}[1]{{#1}\xspace}
\newcommand{\newc}[1]{{\color{blue}#1}\xspace}
\newcommand{\ZKP}{\text{ZKP}\xspace}
\newcommand{\NIZK}{\text{NIZK}\xspace}
\newcommand{\nlib}{11\xspace}
\newcommand{\nzk}{43\xspace}
\newcommand{\recipe}{\textit{master recipe}\xspace}
\renewcommand{\figureautorefname}{Figure}
\renewcommand{\tableautorefname}{Table}
\renewcommand{\sectionautorefname}{Section}
  \renewcommand{\subsectionautorefname}{Section}
  \renewcommand{\subsubsectionautorefname}{Section}

\newcommand{\lib}[1]{\texttt{#1}\xspace}

  
\providecommand{\definitionautorefname}{Definition}
%\usepackage{bbm}
\usepackage{graphicx}
\usepackage{amsmath,amssymb,amsthm,amsfonts}

\usepackage{paralist}
\usepackage{bm}
\usepackage{xspace}
\usepackage{url}
\usepackage{prettyref}
\usepackage{boxedminipage}
\usepackage{wrapfig}
\usepackage{ifthen}
\usepackage{color}
\usepackage{xspace}

\newcommand{\ii}{{\sc Indicator-Instance}\xspace}
\newcommand{\midd}{{\sf mid}}


\usepackage{amsmath,amsthm,amsfonts,amssymb}
\usepackage{mathtools}
\usepackage{graphicx}


% \usepackage{fullpage}

\usepackage{nicefrac}

\newtheorem{inftheorem}{Informal Theorem}
\newtheorem{claim}{Claim}
\newtheorem*{definition*}{Definition}
\newtheorem{example}{Example}

\DeclareMathOperator*{\argmax}{arg\,max}
\DeclareMathOperator*{\argmin}{arg\,min}
\usepackage{subcaption}

\newtheorem{problem}{Problem}
\usepackage[utf8]{inputenc}
\newcommand{\rank}{\mathsf{rank}}
\newcommand{\tr}{\mathsf{Tr}}
\newcommand{\tv}{\mathsf{TV}}
\newcommand{\opt}{\mathsf{OPT}}
\newcommand{\rr}{\textsc{R}\space}
\newcommand{\alg}{\textsf{Alg}\space}
\newcommand{\sd}{\textsf{sd}_\lambda}
\newcommand{\lblq}{\mathfrak{lq} (X_1)}
\newcommand{\diag}{\textsf{diag}}
\newcommand{\sign}{\textsf{sgn}}
\newcommand{\BC}{\texttt{BC} }
\newcommand{\MM}{\texttt{MM} }
\newcommand{\Nexp}{N_{\mathrm{exp}}}
\newcommand{\Nrep}{N_{\mathrm{replay}}}
\newcommand{\Drep}{D_{\mathrm{replay}}}
\newcommand{\Nsim}{N_{\mathrm{sim}}}
\newcommand{\piBC}{\pi^{\texttt{BC}}}
\newcommand{\piRE}{\pi^{\texttt{RE}}}
\newcommand{\piEMM}{\pi^{\texttt{MM}}}
\newcommand{\mmd}{\texttt{Mimic-MD} }
\newcommand{\RE}{\texttt{RE} }
\newcommand{\dem}{\pi^E}
\newcommand{\Rlint}{\mathcal{R}_{\mathrm{lin,t}}}
\newcommand{\Rlipt}{\mathcal{R}_{\mathrm{lip,t}}}
\newcommand{\Rlin}{\mathcal{R}_{\mathrm{lin}}}
\newcommand{\Rlip}{\mathcal{R}_{\mathrm{lip}}}
\newcommand{\Rmax}{R_{\mathrm{max}}}
\newcommand{\Rall}{\mathcal{R}_{\mathrm{all}}}
\newcommand{\Rdet}{\mathcal{R}_{\mathrm{det}}}
\newcommand{\Fmax}{F_{\mathrm{max}}}
\newcommand{\Nmax}{\mathcal{N}_{\mathrm{max}}}
\newcommand{\piref}{\pi^{\mathrm{ref}}}
\newcommand{\green}{\text{\color{green!75!black} green}\;}
\newcommand{\thetaBC}{\widehat{\theta}^{\textsf{BC}}}
\newcommand{\ent}{\mathcal{E}_{\Theta,n,\delta}}
\newcommand{\eNt}{\mathcal{E}_{\Theta_t,\Nexp,\delta}}
\newcommand{\eNtH}{\mathcal{E}_{\Theta_t,\Nexp,\delta/H}}

\newcommand{\eref}[1]{(\ref{#1})}
\newcommand{\sref}[1]{Sec. \ref{#1}}
\newcommand{\dr}{\widehat{d}_{\mathrm{replay}}}
\newcommand{\figref}[1]{Fig. \ref{#1}}

\usepackage{xcolor}
\definecolor{expert}{HTML}{008000}
\definecolor{error}{HTML}{f96565}
\newcommand{\GKS}[1]{{\textcolor{violet}{\textbf{GKS: #1}}}}
\newcommand{\Q}[1]{{\textcolor{red}{\textbf{Question #1}}}}
\newcommand{\ZSW}[1]{{\textcolor{orange}{\textbf{ZSW: #1}}}}
\newcommand{\JAB}[1]{{\textcolor{teal}{\textbf{JAB: #1}}}}
\newcommand{\jab}[1]{{\textcolor{teal}{\textbf{JAB: #1}}}}
\newcommand{\SAN}[1]{{\textcolor{blue}{\textbf{SC: #1}}}}
\newcommand{\scnote}[1]{\SAN{#1}}
\newcommand{\norm}[1]{\left\lVert #1 \right\rVert}

\usepackage{color-edits}
\addauthor{sw}{blue}

\usepackage{thmtools}
\usepackage{thm-restate}

\usepackage{tikz}
\usetikzlibrary{arrows,calc} 
\newcommand{\tikzAngleOfLine}{\tikz@AngleOfLine}
\def\tikz@AngleOfLine(#1)(#2)#3{%
\pgfmathanglebetweenpoints{%
\pgfpointanchor{#1}{center}}{%
\pgfpointanchor{#2}{center}}
\pgfmathsetmacro{#3}{\pgfmathresult}%
}

\declaretheoremstyle[
    headfont=\normalfont\bfseries, 
    bodyfont = \normalfont\itshape]{mystyle} 
\declaretheorem[name=Theorem,style=mystyle,numberwithin=section]{thm}

% \usepackage{algorithm}
% \usepackage{algorithmic}
\usepackage[linesnumbered,algoruled,boxed,lined,noend]{algorithm2e}

\usepackage{listings}
\usepackage{amsmath}
\usepackage{amsthm}
\usepackage{tikz}
\usepackage{caption}
\usepackage{mdwmath}
\usepackage{multirow}
\usepackage{mdwtab}
\usepackage{eqparbox}
\usepackage{multicol}
\usepackage{amsfonts}
\usepackage{tikz}
\usepackage{multirow,bigstrut,threeparttable}
\usepackage{amsthm}
\usepackage{bbm}
\usepackage{epstopdf}
\usepackage{mdwmath}
\usepackage{mdwtab}
\usepackage{eqparbox}
\usetikzlibrary{topaths,calc}
\usepackage{latexsym}
\usepackage{cite}
\usepackage{amssymb}
\usepackage{bm}
\usepackage{amssymb}
\usepackage{graphicx}
\usepackage{mathrsfs}
\usepackage{epsfig}
\usepackage{psfrag}
\usepackage{setspace}
\usepackage[%dvips,
            CJKbookmarks=true,
            bookmarksnumbered=true,
            bookmarksopen=true,
%						bookmarks=false,
            colorlinks=true,
            citecolor=red,
            linkcolor=blue,
            anchorcolor=red,
            urlcolor=blue
            ]{hyperref}
%\usepackage{algorithm}
\usepackage[linesnumbered,algoruled,boxed,lined]{algorithm2e}
\usepackage{algpseudocode}
\usepackage{stfloats}
\RequirePackage[numbers]{natbib}

\usepackage{comment}
\usepackage{mathtools}
\usepackage{blkarray}
\usepackage{multirow,bigdelim,dcolumn,booktabs}

\usepackage{xparse}
\usepackage{tikz}
\usetikzlibrary{calc}
\usetikzlibrary{decorations.pathreplacing,matrix,positioning}

\usepackage[T1]{fontenc}
\usepackage[utf8]{inputenc}
\usepackage{mathtools}
\usepackage{blkarray, bigstrut}
\usepackage{gauss}

\newenvironment{mygmatrix}{\def\mathstrut{\vphantom{\big(}}\gmatrix}{\endgmatrix}

\newcommand{\tikzmark}[1]{\tikz[overlay,remember picture] \node (#1) {};}

%% Adapted form https://tex.stackexchange.com/questions/206898/braces-for-cases-in-tabular-environment/207704#207704
\newcommand*{\BraceAmplitude}{0.4em}%
\newcommand*{\VerticalOffset}{0.5ex}%  
\newcommand*{\HorizontalOffset}{0.0em}% 
\newcommand*{\blocktextwid}{3.0cm}%
\NewDocumentCommand{\InsertLeftBrace}{%
	O{} % #1 = draw options
	O{\HorizontalOffset,\VerticalOffset} % #2 = optional brace shift options
	O{\blocktextwid} % #3 = optional text width
	m   % #4 = top tikzmark
	m   % #5 = bottom tikzmark
	m   % #6 = node text
}{%
	\begin{tikzpicture}[overlay,remember picture]
	\coordinate (Brace Top)    at ($(#4.north) + (#2)$);
	\coordinate (Brace Bottom) at ($(#5.south) + (#2)$);
	\draw [decoration={brace, amplitude=\BraceAmplitude}, decorate, thick, draw=black, #1]
	(Brace Bottom) -- (Brace Top) 
	node [pos=0.5, anchor=east, align=left, text width=#3, color=black, xshift=\BraceAmplitude] {#6};
	\end{tikzpicture}%
}%
\NewDocumentCommand{\InsertRightBrace}{%
	O{} % #1 = draw options
	O{\HorizontalOffset,\VerticalOffset} % #2 = optional brace shift options
	O{\blocktextwid} % #3 = optional text width
	m   % #4 = top tikzmark
	m   % #5 = bottom tikzmark
	m   % #6 = node text
}{%
	\begin{tikzpicture}[overlay,remember picture]
	\coordinate (Brace Top)    at ($(#4.north) + (#2)$);
	\coordinate (Brace Bottom) at ($(#5.south) + (#2)$);
	\draw [decoration={brace, amplitude=\BraceAmplitude}, decorate, thick, draw=black, #1]
	(Brace Top) -- (Brace Bottom) 
	node [pos=0.5, anchor=west, align=left, text width=#3, color=black, xshift=\BraceAmplitude] {#6};
	\end{tikzpicture}%
}%
\NewDocumentCommand{\InsertTopBrace}{%
	O{} % #1 = draw options
	O{\HorizontalOffset,\VerticalOffset} % #2 = optional brace shift options
	O{\blocktextwid} % #3 = optional text width
	m   % #4 = top tikzmark
	m   % #5 = bottom tikzmark
	m   % #6 = node text
}{%
	\begin{tikzpicture}[overlay,remember picture]
	\coordinate (Brace Top)    at ($(#4.west) + (#2)$);
	\coordinate (Brace Bottom) at ($(#5.east) + (#2)$);
	\draw [decoration={brace, amplitude=\BraceAmplitude}, decorate, thick, draw=black, #1]
	(Brace Top) -- (Brace Bottom) 
	node [pos=0.5, anchor=south, align=left, text width=#3, color=black, xshift=\BraceAmplitude] {#6};
	\end{tikzpicture}%
}%

\usetikzlibrary{patterns}

\definecolor{cof}{RGB}{219,144,71}
\definecolor{pur}{RGB}{186,146,162}
\definecolor{greeo}{RGB}{91,173,69}
\definecolor{greet}{RGB}{52,111,72}

% provide arXiv number if available:
% \arxiv{cs.IT/1502.00326}

% put your definitions there:

%\newtheorem{remark}{Remark} \def\remref#1{Remark~\ref{#1}}
%\newtheorem{conjecture}{Conjecture} \def\remref#1{Remark~\ref{#1}}
%\newtheorem{example}{Example}

%\theorembodyfont{\itshape}
%\newtheorem{theorem}{Theorem}
%\newtheorem{proposition}{Proposition}
%\newtheorem{lemma}{Lemma} \def\lemref#1{Lemma~\ref{#1}}
%\newtheorem{corollary}{Corollary}


%\theorembodyfont{\rmfamily}
%\newtheorem{definition}{Definition}
%\numberwithin{equation}{section}
% \theoremstyle{plain}
% \newtheorem{theorem}{Theorem}
% \newtheorem{Example}{Example}
% \newtheorem{lemma}{Lemma}
% \newtheorem{remark}{Remark}
% \newtheorem{corollary}{Corollary}
% \newtheorem{definition}{Definition}
% \newtheorem{conjecture}{Conjecture}
% \newtheorem{question}{Question}
% \newtheorem*{induction}{Induction Hypothesis}
% \newtheorem*{folklore}{Folklore}
% \newtheorem{assumption}{Assumption}

\def \by {\bar{y}}
\def \bx {\bar{x}}
\def \bh {\bar{h}}
\def \bz {\bar{z}}
\def \cF {\mathcal{F}}
\def \bP {\mathbb{P}}
\def \bE {\mathbb{E}}
\def \bR {\mathbb{R}}
\def \bF {\mathbb{F}}
\def \cG {\mathcal{G}}
\def \cM {\mathcal{M}}
\def \cB {\mathcal{B}}
\def \cN {\mathcal{N}}
\def \var {\mathsf{Var}}
\def\1{\mathbbm{1}}
\def \FF {\mathbb{F}}


\newenvironment{keywords}
{\bgroup\leftskip 20pt\rightskip 20pt \small\noindent{\bfseries
Keywords:} \ignorespaces}%
{\par\egroup\vskip 0.25ex}
\newlength\aftertitskip     \newlength\beforetitskip
\newlength\interauthorskip  \newlength\aftermaketitskip















%%%%%%%%%%%%%%%%%%%%%%%%%%%% by Wu %%%%%%%%%%%%%%%%%%%%%%%%%%%%
\usepackage{xspace}

\newcommand{\Lip}{\mathrm{Lip}}
\newcommand{\stepa}[1]{\overset{\rm (a)}{#1}}
\newcommand{\stepb}[1]{\overset{\rm (b)}{#1}}
\newcommand{\stepc}[1]{\overset{\rm (c)}{#1}}
\newcommand{\stepd}[1]{\overset{\rm (d)}{#1}}
\newcommand{\stepe}[1]{\overset{\rm (e)}{#1}}
\newcommand{\stepf}[1]{\overset{\rm (f)}{#1}}


\newcommand{\floor}[1]{{\left\lfloor {#1} \right \rfloor}}
\newcommand{\ceil}[1]{{\left\lceil {#1} \right \rceil}}

\newcommand{\blambda}{\bar{\lambda}}
\newcommand{\reals}{\mathbb{R}}
\newcommand{\naturals}{\mathbb{N}}
\newcommand{\integers}{\mathbb{Z}}
\newcommand{\Expect}{\mathbb{E}}
\newcommand{\expect}[1]{\mathbb{E}\left[#1\right]}
\newcommand{\Prob}{\mathbb{P}}
\newcommand{\prob}[1]{\mathbb{P}\left[#1\right]}
\newcommand{\pprob}[1]{\mathbb{P}[#1]}
\newcommand{\intd}{{\rm d}}
\newcommand{\TV}{{\sf TV}}
\newcommand{\LC}{{\sf LC}}
\newcommand{\PW}{{\sf PW}}
\newcommand{\htheta}{\hat{\theta}}
\newcommand{\eexp}{{\rm e}}
\newcommand{\expects}[2]{\mathbb{E}_{#2}\left[ #1 \right]}
\newcommand{\diff}{{\rm d}}
\newcommand{\eg}{e.g.\xspace}
\newcommand{\ie}{i.e.\xspace}
\newcommand{\iid}{i.i.d.\xspace}
\newcommand{\fracp}[2]{\frac{\partial #1}{\partial #2}}
\newcommand{\fracpk}[3]{\frac{\partial^{#3} #1}{\partial #2^{#3}}}
\newcommand{\fracd}[2]{\frac{\diff #1}{\diff #2}}
\newcommand{\fracdk}[3]{\frac{\diff^{#3} #1}{\diff #2^{#3}}}
\newcommand{\renyi}{R\'enyi\xspace}
\newcommand{\lpnorm}[1]{\left\|{#1} \right\|_{p}}
\newcommand{\linf}[1]{\left\|{#1} \right\|_{\infty}}
\newcommand{\lnorm}[2]{\left\|{#1} \right\|_{{#2}}}
\newcommand{\Lploc}[1]{L^{#1}_{\rm loc}}
\newcommand{\hellinger}{d_{\rm H}}
\newcommand{\Fnorm}[1]{\lnorm{#1}{\rm F}}
%% parenthesis
\newcommand{\pth}[1]{\left( #1 \right)}
\newcommand{\qth}[1]{\left[ #1 \right]}
\newcommand{\sth}[1]{\left\{ #1 \right\}}
\newcommand{\bpth}[1]{\Bigg( #1 \Bigg)}
\newcommand{\bqth}[1]{\Bigg[ #1 \Bigg]}
\newcommand{\bsth}[1]{\Bigg\{ #1 \Bigg\}}
\newcommand{\xxx}{\textbf{xxx}\xspace}
\newcommand{\toprob}{{\xrightarrow{\Prob}}}
\newcommand{\tolp}[1]{{\xrightarrow{L^{#1}}}}
\newcommand{\toas}{{\xrightarrow{{\rm a.s.}}}}
\newcommand{\toae}{{\xrightarrow{{\rm a.e.}}}}
\newcommand{\todistr}{{\xrightarrow{{\rm D}}}}
\newcommand{\eqdistr}{{\stackrel{\rm D}{=}}}
\newcommand{\iiddistr}{{\stackrel{\text{\iid}}{\sim}}}
%\newcommand{\var}{\mathsf{var}}
\newcommand\indep{\protect\mathpalette{\protect\independenT}{\perp}}
\def\independenT#1#2{\mathrel{\rlap{$#1#2$}\mkern2mu{#1#2}}}
\newcommand{\Bern}{\text{Bern}}
\newcommand{\Poi}{\mathsf{Poi}}
\newcommand{\iprod}[2]{\left \langle #1, #2 \right\rangle}
\newcommand{\Iprod}[2]{\langle #1, #2 \rangle}
\newcommand{\indc}[1]{{\mathbf{1}_{\left\{{#1}\right\}}}}
\newcommand{\Indc}{\mathbf{1}}
\newcommand{\regoff}[1]{\textsf{Reg}_{\mathcal{F}}^{\text{off}} (#1)}
\newcommand{\regon}[1]{\textsf{Reg}_{\mathcal{F}}^{\text{on}} (#1)}

\definecolor{myblue}{rgb}{.8, .8, 1}
\definecolor{mathblue}{rgb}{0.2472, 0.24, 0.6} % mathematica's Color[1, 1--3]
\definecolor{mathred}{rgb}{0.6, 0.24, 0.442893}
\definecolor{mathyellow}{rgb}{0.6, 0.547014, 0.24}


\newcommand{\red}{\color{red}}
\newcommand{\blue}{\color{blue}}
\newcommand{\nb}[1]{{\sf\blue[#1]}}
\newcommand{\nbr}[1]{{\sf\red[#1]}}

\newcommand{\tmu}{{\tilde{\mu}}}
\newcommand{\tf}{{\tilde{f}}}
\newcommand{\tp}{\tilde{p}}
\newcommand{\tilh}{{\tilde{h}}}
\newcommand{\tu}{{\tilde{u}}}
\newcommand{\tx}{{\tilde{x}}}
\newcommand{\ty}{{\tilde{y}}}
\newcommand{\tz}{{\tilde{z}}}
\newcommand{\tA}{{\tilde{A}}}
\newcommand{\tB}{{\tilde{B}}}
\newcommand{\tC}{{\tilde{C}}}
\newcommand{\tD}{{\tilde{D}}}
\newcommand{\tE}{{\tilde{E}}}
\newcommand{\tF}{{\tilde{F}}}
\newcommand{\tG}{{\tilde{G}}}
\newcommand{\tH}{{\tilde{H}}}
\newcommand{\tI}{{\tilde{I}}}
\newcommand{\tJ}{{\tilde{J}}}
\newcommand{\tK}{{\tilde{K}}}
\newcommand{\tL}{{\tilde{L}}}
\newcommand{\tM}{{\tilde{M}}}
\newcommand{\tN}{{\tilde{N}}}
\newcommand{\tO}{{\tilde{O}}}
\newcommand{\tP}{{\tilde{P}}}
\newcommand{\tQ}{{\tilde{Q}}}
\newcommand{\tR}{{\tilde{R}}}
\newcommand{\tS}{{\tilde{S}}}
\newcommand{\tT}{{\tilde{T}}}
\newcommand{\tU}{{\tilde{U}}}
\newcommand{\tV}{{\tilde{V}}}
\newcommand{\tW}{{\tilde{W}}}
\newcommand{\tX}{{\tilde{X}}}
\newcommand{\tY}{{\tilde{Y}}}
\newcommand{\tZ}{{\tilde{Z}}}

\newcommand{\sfa}{{\mathsf{a}}}
\newcommand{\sfb}{{\mathsf{b}}}
\newcommand{\sfc}{{\mathsf{c}}}
\newcommand{\sfd}{{\mathsf{d}}}
\newcommand{\sfe}{{\mathsf{e}}}
\newcommand{\sff}{{\mathsf{f}}}
\newcommand{\sfg}{{\mathsf{g}}}
\newcommand{\sfh}{{\mathsf{h}}}
\newcommand{\sfi}{{\mathsf{i}}}
\newcommand{\sfj}{{\mathsf{j}}}
\newcommand{\sfk}{{\mathsf{k}}}
\newcommand{\sfl}{{\mathsf{l}}}
\newcommand{\sfm}{{\mathsf{m}}}
\newcommand{\sfn}{{\mathsf{n}}}
\newcommand{\sfo}{{\mathsf{o}}}
\newcommand{\sfp}{{\mathsf{p}}}
\newcommand{\sfq}{{\mathsf{q}}}
\newcommand{\sfr}{{\mathsf{r}}}
\newcommand{\sfs}{{\mathsf{s}}}
\newcommand{\sft}{{\mathsf{t}}}
\newcommand{\sfu}{{\mathsf{u}}}
\newcommand{\sfv}{{\mathsf{v}}}
\newcommand{\sfw}{{\mathsf{w}}}
\newcommand{\sfx}{{\mathsf{x}}}
\newcommand{\sfy}{{\mathsf{y}}}
\newcommand{\sfz}{{\mathsf{z}}}
\newcommand{\sfA}{{\mathsf{A}}}
\newcommand{\sfB}{{\mathsf{B}}}
\newcommand{\sfC}{{\mathsf{C}}}
\newcommand{\sfD}{{\mathsf{D}}}
\newcommand{\sfE}{{\mathsf{E}}}
\newcommand{\sfF}{{\mathsf{F}}}
\newcommand{\sfG}{{\mathsf{G}}}
\newcommand{\sfH}{{\mathsf{H}}}
\newcommand{\sfI}{{\mathsf{I}}}
\newcommand{\sfJ}{{\mathsf{J}}}
\newcommand{\sfK}{{\mathsf{K}}}
\newcommand{\sfL}{{\mathsf{L}}}
\newcommand{\sfM}{{\mathsf{M}}}
\newcommand{\sfN}{{\mathsf{N}}}
\newcommand{\sfO}{{\mathsf{O}}}
\newcommand{\sfP}{{\mathsf{P}}}
\newcommand{\sfQ}{{\mathsf{Q}}}
\newcommand{\sfR}{{\mathsf{R}}}
\newcommand{\sfS}{{\mathsf{S}}}
\newcommand{\sfT}{{\mathsf{T}}}
\newcommand{\sfU}{{\mathsf{U}}}
\newcommand{\sfV}{{\mathsf{V}}}
\newcommand{\sfW}{{\mathsf{W}}}
\newcommand{\sfX}{{\mathsf{X}}}
\newcommand{\sfY}{{\mathsf{Y}}}
\newcommand{\sfZ}{{\mathsf{Z}}}


\newcommand{\calA}{{\mathcal{A}}}
\newcommand{\calB}{{\mathcal{B}}}
\newcommand{\calC}{{\mathcal{C}}}
\newcommand{\calD}{{\mathcal{D}}}
\newcommand{\calE}{{\mathcal{E}}}
\newcommand{\calF}{{\mathcal{F}}}
\newcommand{\calG}{{\mathcal{G}}}
\newcommand{\calH}{{\mathcal{H}}}
\newcommand{\calI}{{\mathcal{I}}}
\newcommand{\calJ}{{\mathcal{J}}}
\newcommand{\calK}{{\mathcal{K}}}
\newcommand{\calL}{{\mathcal{L}}}
\newcommand{\calM}{{\mathcal{M}}}
\newcommand{\calN}{{\mathcal{N}}}
\newcommand{\calO}{{\mathcal{O}}}
\newcommand{\calP}{{\mathcal{P}}}
\newcommand{\calQ}{{\mathcal{Q}}}
\newcommand{\calR}{{\mathcal{R}}}
\newcommand{\calS}{{\mathcal{S}}}
\newcommand{\calT}{{\mathcal{T}}}
\newcommand{\calU}{{\mathcal{U}}}
\newcommand{\calV}{{\mathcal{V}}}
\newcommand{\calW}{{\mathcal{W}}}
\newcommand{\calX}{{\mathcal{X}}}
\newcommand{\calY}{{\mathcal{Y}}}
\newcommand{\calZ}{{\mathcal{Z}}}

\newcommand{\bara}{{\bar{a}}}
\newcommand{\barb}{{\bar{b}}}
\newcommand{\barc}{{\bar{c}}}
\newcommand{\bard}{{\bar{d}}}
\newcommand{\bare}{{\bar{e}}}
\newcommand{\barf}{{\bar{f}}}
\newcommand{\barg}{{\bar{g}}}
\newcommand{\barh}{{\bar{h}}}
\newcommand{\bari}{{\bar{i}}}
\newcommand{\barj}{{\bar{j}}}
\newcommand{\bark}{{\bar{k}}}
\newcommand{\barl}{{\bar{l}}}
\newcommand{\barm}{{\bar{m}}}
\newcommand{\barn}{{\bar{n}}}
\newcommand{\baro}{{\bar{o}}}
\newcommand{\barp}{{\bar{p}}}
\newcommand{\barq}{{\bar{q}}}
\newcommand{\barr}{{\bar{r}}}
\newcommand{\bars}{{\bar{s}}}
\newcommand{\bart}{{\bar{t}}}
\newcommand{\baru}{{\bar{u}}}
\newcommand{\barv}{{\bar{v}}}
\newcommand{\barw}{{\bar{w}}}
\newcommand{\barx}{{\bar{x}}}
\newcommand{\bary}{{\bar{y}}}
\newcommand{\barz}{{\bar{z}}}
\newcommand{\barA}{{\bar{A}}}
\newcommand{\barB}{{\bar{B}}}
\newcommand{\barC}{{\bar{C}}}
\newcommand{\barD}{{\bar{D}}}
\newcommand{\barE}{{\bar{E}}}
\newcommand{\barF}{{\bar{F}}}
\newcommand{\barG}{{\bar{G}}}
\newcommand{\barH}{{\bar{H}}}
\newcommand{\barI}{{\bar{I}}}
\newcommand{\barJ}{{\bar{J}}}
\newcommand{\barK}{{\bar{K}}}
\newcommand{\barL}{{\bar{L}}}
\newcommand{\barM}{{\bar{M}}}
\newcommand{\barN}{{\bar{N}}}
\newcommand{\barO}{{\bar{O}}}
\newcommand{\barP}{{\bar{P}}}
\newcommand{\barQ}{{\bar{Q}}}
\newcommand{\barR}{{\bar{R}}}
\newcommand{\barS}{{\bar{S}}}
\newcommand{\barT}{{\bar{T}}}
\newcommand{\barU}{{\bar{U}}}
\newcommand{\barV}{{\bar{V}}}
\newcommand{\barW}{{\bar{W}}}
\newcommand{\barX}{{\bar{X}}}
\newcommand{\barY}{{\bar{Y}}}
\newcommand{\barZ}{{\bar{Z}}}

\newcommand{\hX}{\hat{X}}
\newcommand{\Ent}{\mathsf{Ent}}
\newcommand{\awarm}{{A_{\text{warm}}}}
\newcommand{\thetaLS}{{\widehat{\theta}^{\text{\rm LS}}}}

\newcommand{\jiao}[1]{\langle{#1}\rangle}
\newcommand{\gaht}{\textsc{GoodActionHypTest}\;}
\newcommand{\iaht}{\textsc{InitialActionHypTest}\;}
\newcommand{\true}{\textsf{True}\;}
\newcommand{\false}{\textsf{False}\;}

% \usepackage[capitalize,noabbrev]{cleveref}
% \crefname{lemma}{Lemma}{Lemmas}
% \Crefname{lemma}{Lemma}{Lemmas}
% \crefname{thm}{Theorem}{Theorems}
% \Crefname{thm}{Theorem}{Theorems}
% \Crefname{assumption}{Assumption}{Assumptions}
% \Crefname{inftheorem}{Informal Theorem}{Informal Theorems}
% \crefformat{equation}{(#2#1#3)}

% % if you use cleveref..
% \usepackage[capitalize,noabbrev]{cleveref}
% \crefname{lemma}{Lemma}{Lemmas}
% \crefname{proposition}{Proposition}{Propositions}
% \crefname{remark}{Remark}{Remarks}
% \crefname{corollary}{Corollary}{Corollaries}
% \crefname{definition}{Definition}{Definitions}
% \crefname{conjecture}{Conjecture}{Conjectures}
% \crefname{figure}{Fig.}{Figures}

% \setlength{\belowcaptionskip}{-10pt} 
% \setlength{\footskip}{30pt}
% \setlength{\abovecaptionskip}{5pt plus 3pt minus 2pt} 

\usepackage{color}
\usepackage{tabularx}
\definecolor{darkgreen}{rgb}{0,0.5,0}
\definecolor{darkblue}{rgb}{0,0,0.5}
\definecolor{purple}{rgb}{1,0,1}

\usepackage[most]{tcolorbox}
\usepackage{authblk}

\definecolor{boxcolor}{RGB}{230,240,250}

\newenvironment{takeaway}[1][]
  {
    % \vspace{-0.1em}
 \begin{tcolorbox}
 [%
    enhanced, 
    breakable,
    boxrule=0.5pt,
    arc=4pt,
    left=2pt,
    right=2pt,
    bottom=2pt,
    top=2pt,
    % colback=boxcolor, 
    % colframe=black,
    rounded corners
    % frame hidden,
    % overlay broken = {
    %     \draw[line width=0pt, black, rounded corners]
    %     (frame.north west) rectangle (frame.south east);},
    ]{}
  \textbf{#1.}
  \small \itshape}
  {
\end{tcolorbox}
    % \vspace{-0.1em}
}
\newcommand{\myfoot}[1]{\footnote{#1}}

\pagestyle{empty}
% \usepackage[available]{usenixbadges}
\begin{document}

\date{}


\title{\Large \bf SoK: Understanding {\zk}s: The Gap Between Research and Practice}

\author{
{\rm Junkai Liang$^{1,*}$, Daqi Hu$^{1,*}$, Pengfei Wu$^{2,*}$, Yunbo Yang$^{3}$, Qingni Shen$^{1,\dag}$, Zhonghai Wu$^{1,\dag}$}\\
$^{1}$Peking University, \quad $^{2}$Singapore Management University, \quad $^{3}$East China Normal University\\
{\tt \{ljknjupku, hudaqi0507\}@gmail.com, pfwu@smu.edu.sg, yyb9882@gmail.com, \\ \tt \{qingnishen, wuzh\}@pku.edu.cn}
} % end author


%\author{
% {\rm Anonymous Author(s)}\\
% \and
% {\rm Junkai Liang}\\
% Peking University
% copy the following lines to add more authors
% \and
% {\rm Name}\\
%Name Institution
%} % end author



\maketitle
\renewcommand{\thefootnote}{}
%-------------------------------------------------------------------------------
\begin{abstract}
\footnote{\rm \textbf{*:} The authors contribute equally to this paper.} \footnote{\rm \textbf{$\dag$:} Corresponding author. This work was supported by the National Key R\&D Program of China under Grant No. 2022YFB2703301, School of Computer Science, Peking University and PKU-OCTA Laboratory for Blockchain and Privacy Computing.}
Zero-knowledge succinct non-interactive argument of knowledge (\zk) serves as a powerful technique for proving the correctness of computations and has attracted significant interest from researchers. Numerous concrete schemes and implementations have been proposed in academia and industry. Unfortunately, the inherent complexity of \zk has created gaps between researchers, developers and users, as they focus differently on this technique. For example, researchers are dedicated to constructing new efficient proving systems with stronger security and new properties. At the same time, developers and users care more about the implementation's toolchains, usability and compatibility. This gap has hindered the development of \zk field.

In this work, we provide a comprehensive study of \zk, from theory to practice, pinpointing gaps and limitations. We first present a \recipe that unifies the main steps in converting a program into a \zk. We then classify existing {\zk}s according to their key techniques. Our classification addresses the main difference in practically valuable properties between existing \zk schemes. We survey over 40 {\zk}s since 2013 and provide a reference table listing their categories and properties. Following the steps in \recipe, we then survey 11 general-purpose popular used libraries. We elaborate on these libraries' usability, compatibility, efficiency and limitations. Since installing and executing these zk-SNARK systems is challenging, we also provide a completely virtual environment in which to run the compiler for each of them. We identify that the proving system is the primary focus in cryptography academia. In contrast, the constraint system presents a bottleneck in industry. To bridge this gap, we offer recommendations and advocate for the open-source community to enhance documentation, standardization and compatibility.


\end{abstract}
\renewcommand{\thefootnote}{\arabic{footnote}}

\section{Introduction}

\new{
	{\textit{Imagine you have a friend who is red-green colour-blind and doubts that red and green are actually distinct colours. You want to prove to your friend that the two colours are indeed different. Our question is: How do you do that without revealing the actual colours of the objects you're using?}
	}
}

\new{The above colour-blind verifier~\cite{colorblind} is a classical problem when thinking about zero-knowledge proof (ZKP) with daily life scenarios. The solution is also easy to understand: 
	You prepare a red ball and a green ball for your friend and ask her to choose one as her favorite. Then she conceals both balls, chooses one ball randomly and asks you to tell if it is her favorite. If red and green are indeed different, you can succeed with probability 1, otherwise, you can only succeed with probability 1/2\footnote{\new{In our simplified question you are not motivated to convince your friend that red and green are the same.}}. Your friend can repeat this process to convince herself that the probability of coincidence is negligible.
}

\new{
	A natural formalism of the above thought experiment yields an interactive form of \ZKP, where there are one or many rounds of interactions between the verifier and the prover~\cite{goldreich1998complexity}, a.k.a. the interactive proof (\IP). \IP is a breakthrough in \ZKP field as it has been used to prove the knowledge of solutions in all problems within non-deterministic polynomial time (NP) space (e.g., 3-colour problem and boolean satisfiability problem), which extends the capability of \ZKP from daily scenarios to computational models~\cite{arora1998proof}. IP is powerful but may need multiple rounds of interaction, which increases the communication burden and is unrealistic for some applications like blockchain or confidential machine learning. Non-interactive zero-knowledge (\NIZK) proof focuses on the protocols where the prover just sends one message (i.e., the proof) to the verifier and the verifier can decide to accept it or not. The main purpose of \NIZK is to solve latency issues caused by interactivity. Luckily, \IP and \NIZK can be bridged through generic transforms, e.g., Fiat-Shamir transform~\cite{fiat1986prove} which allows the prover to generate hash values as if they are random messages given by the verifier. Following the theoretical progress, \IP and \NIZK protocols for the 3-colourability problem and 3-satisfiability have been proposed ~\cite{arora1998proof,arora1998probabilistic}. However, these works suffer from large asymptotic costs and are not practical. To better address real-world scenarios, \NIZK is further required to have succinctness, which means the time and memory used by the prover and verifier are bounded. \NIZK with succinctness, a.k.a. \ZK has been the mainstream of the ZKP research with practical applications. The relations of \ZKP, \NIZK and \ZK are shown in} \autoref{fig:zkp}.

\begin{figure}[ht]
	\centering
	\includegraphics[height=40mm, width=60mm]{img/ZKP.pdf}
	\caption{\new{Relations of inclusion for ZKP, NIZK and \ZK.}}
	\label{fig:zkp}
\end{figure}

\news{Evolved from \ZKP and \NIZK}, \ZK provides a mechanism for a distrustful party to prove the knowledge of NP relations, where the generated proof reveals nothing about the private witness. This valuable property makes \zk a powerful cryptographic primitive, enabling the verification of computation correctness without exposing private inputs. In the past several years, a surge of groundbreaking scientific achievements has emerged across \zk applications, including but not limited to financial services like blockchain payments~\cite{sasson2014zerocash,bowe2020zexe,bunz2020zether}, smart contract~\cite{wan2022zk,steffen2022zeestar}, and other academic areas like machine learning~\cite{weng2021mystique,liu2021zkcnn}, multiparty computation~\cite{beaver1991secure,ishai2007zero,boyle2019practical} and post-quantum cryptography~\cite{giacomelli2016zkboo,chase2020picnic}. \new{The \ZK also has a promising market outlook. Till today, there are more than 10 widely used blockchains based on \ZK and it has been estimated that only the transaction fee for generating ZK proofs will reach 10 billion by 2030~\cite{zkmarket}. Besides blockchain services, many companies like Axiom \cite{axiom2024}, FedML \cite{fedml2024}, and Giza \cite{giza2024} are cooperating to build ZK ecosystems for privacy-preserving machine learning and other applications. }

\new{
	Despite \ZK having great generality,  succinctness and the potential for wide usage just like encryption and signature algorithms, there are gaps between research and practice that prevent the development of \ZK. 
	Researchers and practitioners have different focuses on three concepts of \ZK: constraint system, proving system, and compiler. Constraint system represents the problems that we want to prove, such as some specific NP relations like the 3-satisfiability. Proving system represents specific cryptographic techniques that generate proof of the relation. Compilers are practical tools that convert a high-level program we want to prove to the constraint system in a mathematical form.
	
	Researchers mainly focus on designing different proving systems for different constraint systems, aiming to achieve special properties. Till today, there are schemes with very practical properties, such as constant proof size, linear prover, post-quantum security, and transparent setup. However, these properties are not integrated into one single scheme and there are trade-offs. To understand these trade-offs, one needs to have substantial knowledge of \ZK mathematical background which is arduous from a practical perspective, preventing a practitioner from choosing an appropriate scheme for her application. Besides, the most time-consuming and error-prone part for practitioners is using the compilers. As reported in \cite{campanelli2017zero,wen2023practical,ozdemir2023bounded,chaliasos2024sok}, programmers struggle to correctly implement their own \ZK applications and there are hundreds of vulnerabilities due to the misunderstanding of the compiler's language.}

We identify a few gaps between academia and industry perspectives in the \zk field: (1) A user requires expert knowledge to choose a scheme, and
% \delete{We find that some libraries provide a mathematical background for their scheme through formulas to persuade their users. However, it is often impossible for users to do a theoretical walk-through.} 
(2) {The importance of the compiler has been underestimated.} To this end, we are interested in the following research questions:
\begin{tcolorbox}
	[%
	enhanced, 
	breakable,
	boxrule=0.5pt,
	arc=4pt,
	left=2pt,
	right=2pt,
	bottom=2pt,
	top=2pt,
	% colback=boxcolor, 
	% colframe=black,
	rounded corners
	% frame hidden,
	% overlay broken = {
		%     \draw[line width=0pt, black, rounded corners]
		%     (frame.north west) rectangle (frame.south east);},
	]{\mypara{RQ1} How to present a unified \emph{master recipe} outlining the design principles and optimizations behind different {\zk}s?
		
		\mypara{RQ2} Can we provide guidelines on selecting {\zk}s in different real-world scenarios?
		
		\mypara{RQ3} From the master recipe and experiments, by scrutinizing prior works, can we provide novel insights for academic researchers and library designers?}
\end{tcolorbox}
\mypara{Our work} To address these questions, we conduct a systematic review of {\zk}s and their libraries. First, we establish a unified master recipe to outline the design principles of mainstream {\zk}s. This recipe includes key steps: compiling a high-level program into a circuit, passing the circuit to a proving system to generate an \IP, and applying a generic transformation to produce the final \zk. Additionally, we explore the main applications of \ZK, such as confidential blockchain, zero-knowledge machine learning, and cryptographic uses.

\new{Using the master recipe, we classify proving systems and trace their evolution in each category.} This helps non-expert users choose suitable \zk schemes. We then evaluate all \nlib state-of-the-art \zk libraries based on performance and usability. By analyzing performance, we recommend best practices for implementing {\zk}s based on different needs. Additionally, we identify common issues in current libraries and advocate for better documentation and standardization.

\news{We emphasize the goal of this paper and its open-source materials aim at four distinct types of readers: (1) researchers who want to move beyond theory to practice by understanding state-of-the-art libraries; (2) developers who want to implement a component as \ZK toolkit; (3) programmers who want to implement their own \ZK applications; and (4) users who want to understand if a certain \ZK application meets their requirements.} 
We believe that our efforts are necessary and can facilitate the practitioners to utilize \zk achievements. 

\mypara{Summary of Contributions} While we are not the first to review this topic, we
position our work as {the first to systematize the research and practice field over the past decade, which tackles the emerging challenges using state-of-the-art libraries.} 
In summary, we have made five main contributions:

\begin{itemize}[noitemsep, topsep=2pt, partopsep=0pt,leftmargin=0.4cm]
	
	\item We establish a unified master recipe showing how a high-level program is converted into a \zk, from the origin to the end. Within the master recipe, we establish a comprehensive overview in \autoref{sec:overview}, considering different circuits, constraint systems, techniques, \new{and applications} used in the practical {\zk}s. 
	
	\item Under the guideline of the master recipe, we further survey more than 40 {\zk}s and provide a comprehensive comparison table for the proving systems. \new{We discuss how the master recipe and the investigation help mitigate the gaps}.
	
	\item We survey all \nlib \zk libraries 
	and make comparisons based on performance and usability. We recommend the best practice implementations and analyze each library's architecture, toolkits and documentation.
	
	\item We provide our well-designed test code examples in docker containers, which we believe will help the development of \zk open source society and users utilize the achievements of \zk field. \new{All our codes and documents are posted on a permanent repository and available at \url{https://doi.org/10.5281/zenodo.14682405}.}
	
	\item Based on comprehensive analyses, we provide key insights and suggestions from 3 perspectives: library selection and programming for non-experts, future directions for researchers, and suggestions for library designers.
\end{itemize}

\mypara{Related Work}
Prior surveys on ZKP fall into two categories. \emph{First}, surveys on \zk constructions and theoretical applications. For example, Feng and Mcllin~\cite{li2014survey} introduce \zk basics and its use for NP computations. Nitulescu~\cite{nitulescu2020zk} focuses on Quadratic Arithmetic Programs (QAP)-based {\zk}s. Li et al.~\cite{LiWei-Han:379} classify {\zk}s by techniques but focus on niche implementations like constraint systems and layered circuits. Others~\cite{morais2019survey,christ2024sok} discuss range proofs and offer practical advice. These works, however, cover only a small portion of {\zk}s and are largely academic. In contrast, our work bridges theory and practice, offering broader insights.
\emph{Second}, surveys on vulnerabilities in practical \zk implementations. Prior works highlight issues in the circuit layer~\cite{wen2023practical,fan2024snarkprobe,isabel2024scalable}, compilation phase~\cite{ozdemir2023bounded}, and application-specific integrity layer~\cite{cerdeira2020sok,zhou2023sok}. Chaliasos et al.~\cite{chaliasos2024sok} summarize these vulnerabilities comprehensively. Our work differs by providing a comprehensive walk-through for \zk practitioners and
focusing on usability, efficiency, compatibility, and library selection, aiming to reduce errors for practitioners unfamiliar with cryptography while emphasizing software security.


\section{Background}
In this section, we focus on the concept of \zk and introduce the definition in \autoref{sec-notion}, as well as the mainstream techniques in \autoref{sec-tech}. In addition, we summarize all abbreviations and their full names in \autoref{abbr}. With these symbols, we discuss the research development of \zk.

\begin{table}[t]
	\centering
	\begin{tabularx}{\linewidth}{p{1.7cm} X}
		\hline
		\textbf{Abbreviation} & \textbf{Full Form} \\
		\hline
		AIR & Arithmetic Intermediate Representation \\
		CRS & Common Reference String \\ 
		DEIP & Doubly Efficient Interactive Proofs \\ 
		(e)DSL & (embedded) Domain-Specific Language \\ 
		FFT & Fast Fourier Transform \\ 
		FRI & {Fast Reed-Solomon IOP of Proximity} \\ 
		HDL & Hardware Description Language\\
		I(O)P & Interactive (Oracle) Proof \\ 
		IPA & Inner Product Argument \\
		ITP & Information-Theoretic Proof \\
		%LPCP & Linear Probabilistically Checkable Proof \\ 
		MPC & Multi-Party Computation \\ 
		NIZK & Non-Interactive Zero-Knowledge \\ 
		NP & Non-deterministic Polynomial Time \\ 
		PL & Programming Language\\
		(L)PCP & (Linear) Probabilistically Checkable Proof \\ 
		PCS & Polynomial Commitment Scheme \\ 
		PIOP & {Polynomial Interactive Oracle Proof} \\ 
		QAP & Quadratic Arithmetic Program \\ 
		QSP & Quadratic Span Program \\ 
		R1CS & Rank-1 Constraint System \\ 
		{STARK} & {Scalable Transparent ARguments of Knowledge} \\ 
		ZKP & Zero-Knowledge Proof \\ 
		ZKML & Zero-Knowledge Machine Learning\\
		{zk-SNARK} & {Zero-Knowledge Succinct Non-Interactive Argument of Knowledge} \\ 
		zk-VM & Zero-Knowledge Virtual Machine \\ 
		\bottomrule
	\end{tabularx}
	\caption{\new{Abbreviations and Corresponding Full Names}}
	\label{abbr}
\end{table}


\subsection{Notions of IP, NIZK and zk-SNARK}
\label{sec-notion}
\new{Here we introduce the formal notions of \IP~\cite{goldwasser2019knowledge}, \NIZK~\cite{groth2010short} and \zk~\cite{gennaro2013quadratic}, which are popular used in the \ZKP field.
	The similarity between these notions is that, for a fixed NP relation $R$, the prover can convince the verifier that for the public input $x$ they know a witness $w$ such that $(x,w)\in R$. The difference is that \IP allows multiple rounds of communication while \NIZK and \zk are non-interactive. Besides, \zk further has efficiency requirements.}
\new{\begin{definition}[\IP]
		Let $R$ be a binary relation induced by a NP language $L$.
		On common input $x$ and prover's input $w$, we denote the interaction between the prover $P$ and the verifier $V$ as $\langle P(w), V\rangle(x)$. A pair $(P,V)$ is called an IP system for $L$ if there exists a negligible function $\epsilon$ such that the following properties hold:
		\begin{itemize}[noitemsep, topsep=2pt, partopsep=0pt,leftmargin=0.4cm]
			\item \textit{Completeness}: If $(x,w)\in R$, then $\Pr[\langle P(w),V\rangle(x)=1]=1$.
			\item \textit{Soundness}: If $(x,w)\notin R$ and for any malicious prover $P^{*}$, we have $\Pr[\langle P^{*}(w),V\rangle(x)=1]<\epsilon(|x|)$. 
		\end{itemize}
\end{definition}}
\begin{definition}[\text{NIZK}]
	{{A NIZK proof consists of three algorithms} $(\textsf{Setup}, \textsf{Prove}, \textsf{Verify})$ that are defined as follows: 
		\begin{itemize}[noitemsep, topsep=2pt, partopsep=0pt,leftmargin=0.4cm]
			\item $\textsf{Setup}(\textsf{pp}) \rightarrow (\textsf{pk},\textsf{vk})$: On input a public parameter $\textsf{pp}$, it outputs a proving and verification key $\textsf{pk}$ and $\textsf{vk}$.
			\item $\textsf{Prove}(\textsf{pk},x,w,R)\rightarrow \pi$: On input $\textsf{pk}$, an instance and witness pair $(x,w)$, and the relation $R$, it outputs a proof $\pi$.
			\item $\textsf{Verify}(\textsf{vk},x,\pi)\rightarrow \{0,1\}$: On input $\textsf{vk},x$, and $\pi$, it outputs 1 or 0 to show if $\pi$ is accepted or not. 
		\end{itemize}
		Besides, a NIZK proof needs to satisfy the following three properties:}
	\begin{itemize}[noitemsep, topsep=2pt, partopsep=0pt,leftmargin=0.4cm]
		\item \textit{Completeness}: Given $(x,w)\in R$, the honest prover results in the verifier outputting 1.
		
		\item \textit{Soundness}: Given $(x,w)\notin R$, a malicious prover interacting with the verifier can only make it output 1 with negligible probability.
		
		\item \textit{Zero knowledge}: Given $(x,w)\in R$, a simulator can produce a view of an honest prover with a possibly malicious verifier that is computationally indistinguishable from an actual execution transcript of the prover with the verifier. Note that the simulator does not get $w$, while the prover gets $w$, so the proof does not contain information of $w$ from the perspective of the verifier.
	\end{itemize}
\end{definition}

\new{A \text{NIZK} proof is termed a \zk if the proof size and verification time are bounded by the size of the statement to be proven:
	
	\begin{itemize}[noitemsep, topsep=2pt, partopsep=0pt,leftmargin=0.4cm]
		\item The proof size is polylogarithmic in the circuit size.
		\item The verification time is polylogarithmic in the circuit size.
	\end{itemize}
}

There are other notions like Scalable Transparent ARguments of Knowledge (\text{STARK})~\cite{ben2019scalable} and Doubly Efficient Interactive Proofs (DEIP)~\cite{wahby2018doubly}, presenting a similar ZKP system like \zk. These notions actually belong to \zk, and the main difference is that they incorporate new properties. For example, \text{STARK} requires a transparent setup, a construction of \zk in the standard model, and post-quantum security; \text {DEIP} requires quasi-linear complexity on the prover side. In this paper, we use \zk to represent the efficient
\text{NIZK} proofs for simplicity.

\subsection{Cryptographic Techniques}
\label{sec-tech}
In this section, we introduce interactive oracle proof (\text{IOP}), which is a generalization of \IP. We also introduce the polynomial commitment scheme (PCS), which can instantiate the oracles in \text{IOP}. We attach great importance to \IOP and \PCS because they help build the structure of the mainstream proving systems. 
We refer to the references~\cite{gabizon2019plonk} for more information, including their concrete constructions.

\begin{definition}[IOP]
	{Let $x$ be a common input known by verifier and prover, $w$ be a witness string only known by prover, and $r(x)\in\mathbb{N}$ be the round complexity on $x$. An IOP system with $r(x)$ rounds asks that for each round, the prover sends a message (which may depend on witness $w$ and prior messages) to the verifier which is given \textit{oracle access}, and the verifier responds with a message to the prover. After interacting with the prover, the output of the verifier is either \texttt{accept} or \texttt{reject}. }
	
	\par Specifically, given $R$ as a binary relation induced by a NP language $L$  and a soundness error $\epsilon \in [0,1]$, we say that a pair of interactive randomized algorithms $(P, V)$ is an IOP system for $L$ with $\epsilon$ if it satisfies the properties below.
	
	\begin{itemize}[noitemsep, topsep=2pt, partopsep=0pt,leftmargin=0.4cm]
		\item \textit{Completeness}: If $(x,w) \in R$, then $\Pr[V(P(x,w),x)= \texttt{accept}]=1$.
		\item \textit{Soundness}: If $(x,w) \notin R$, then for any proof $\pi$, $\Pr[V(\pi,x)= \texttt{accept}]\leq \epsilon$.
	\end{itemize}
\end{definition}

% \begin{definition}[Interactive Oracle Proof (IOP)]
	
	%     For NP relation $R$, soundness error $\epsilon \in [0,1]$, an interactive oracle proof system for $R$ with soundness error $\epsilon$ is a pair of interactive randomized algorithms $(\textsf{P},\textsf{V})$  satisfying the properties below.
	
	%     \begin{itemize}[noitemsep, topsep=2pt, partopsep=0pt,leftmargin=0.4cm]
		%         \item \mypara{Operation} The input of the verifier is $x$, and the input of the prover is $(x,w)$. 
		%         for some witness string $w$. The number of interactive rounds denoted $r(x)$, is called
		%         the round complexity of the system. During a single round, the prover sends
		%         a message (which may depend on witness $w$ and prior messages) to which the verifier
		%         is given \textit{oracle access}, and the verifier responds with a message to the prover. The output of \textsf{V} after interacting with \textsf{P} is either \texttt{accept} or \texttt{reject}.
		
		%         \item \mypara{Completeness} If $(x,w) \in R$, then $\Pr[\textsf{V}(\textsf{P}(x,w),x)= \texttt{accept}]=1$.
		
		%         \item \mypara{Soundness} If $(x,w) \notin R$, then for any proof $\pi$, $\Pr[\textsf{V}(\pi,x)= \texttt{accept}]\leq \epsilon$.
		%     \end{itemize}
	% \end{definition}

As a special case of \text{IOP}, polynomial IOP (\text{PIOP}) denotes a similar interactive process where a proof produces oracles that evaluate polynomials with a degree lower than a given bound.
To ensure privacy, \text{PIOP} is typically instantiated through a PCS, which we define as below.

\begin{definition}[PCS]
	{The PCS allows a prover to commit to a polynomial $f$ %$f\in R[x]$
		and later prove that the committed polynomial was correctly evaluated at a specified point. A PCS consists of four algorithms: $\textsf{Setup}$, $\textsf{Commit}$, $\textsf{Open}$, and $\textsf{VerifyPoly}$.
		\begin{itemize}[noitemsep, topsep=2pt, partopsep=0pt,leftmargin=0.4cm]
			\item $\textsf{Setup}(1^{\kappa})\rightarrow\textsf{ck}$: On input a security parameter $\kappa$, it outputs a commitment key $\textsf{ck}$.
			
			\item $\textsf{Commit}(\textsf{ck},f)\rightarrow\textsf{com}$: On input $\textsf{ck}$ and a polynomial $f$, it outputs a commitment $\textsf{com}$ to $f$.
			
			\item $\textsf{Open}(\textsf{ck},f,\textsf{com},i)\rightarrow {(f(i),\pi)}$: On input $\textsf{ck},f,\textsf{com}$, and a given point $i$, it outputs the evaluation $f(i)$ and a proof $\pi$.
			
			\item $\textsf{VerifyPoly}(\textsf{ck},\textsf{com},i,f(i),\pi)\rightarrow\{0,1\}$: On input $\textsf{ck},\textsf{com},$ $i,f(i)$, and $\pi$, it outputs 1 if $\pi$ is accepted and 0 otherwise.
	\end{itemize}}
\end{definition}

We emphasize \text{PIOP} with \text{PCS} is the mainstream technique in constructing \zk currently. With different instantiations of a \text{PCS}, one can achieve the required properties needed in a \zk (e.g., short proof size, transparency, and post-quantum security). \new{There are also other techniques like the quadratic arithmetic program (\text{QAP}) used to construct a constant-size probabilistically checkable proof (\PCP) as \zk ~\cite{gennaro2013quadratic}. Here, we give a brief introduction to them.}

\new{
	\begin{definition}[PCP]
		\label{PCP}
		Let $R$ be a binary relation induced by a NP language $L$ and $\epsilon\in (0,1)$ be a probability. We say that $R\in PCP(r,q)$ if there is a probabilistic polynomial-time algorithm $V$ for the verifier satisfying the following properties:
		\begin{itemize}[noitemsep, topsep=2pt, partopsep=0pt,leftmargin=0.4cm]
			\item \textit{Efficiency}: After the proof $\pi$ is generated from the witness $w$, $V$ uses at most $r$ random coins and reads at most $q$ bits of $\pi$ to verify it.
			\item \textit{Completeness}: If $(x,w)\in R$, then $\Pr[V(x,\pi)=1]=1$.
			\item \textit{Soundness}: If $x\notin L$, then for all $\pi$, $\Pr[V(x,\pi)=1]<\epsilon$.
		\end{itemize}
	\end{definition}
}

\new{IP, PCP and IOP are all called Information-Theoretic Proof (ITP) which serves as an abstraction of the final \zk scheme. There are two differences among them. First, \IP and \IOP allow interaction without explicitly generating the proof $\pi$, while PCP is non-interactive. Second, \PCP and \IOP use oracles that the verifier can access freely. The oracles serve as a block box to provide additional computation power for the verifier and simplify the protocol design. To help better understand these concepts, we provide a sudoku puzzle example in \autoref{app: ITP}.
}

\new{
	\begin{definition}[QAP]\label{Def-QAP}
		A QAP $Q$ over a field $\mathbb{F}$ involves three sets of $m+1$ polynomials, $L=\{l_{k}(x)\}$, $R=\{r_{k}(x)\}$, $O=\{o_{k}(x)\}$, for $k=\{0,...,m\}$, and a target polynomial $q(x)$. We say that an assignment $(c_1,\ldots,c_m)$ satisfies $Q$ if $q(x)$ divides $p(x)$ (with the quotient denoted as $t(x)$), where
		\begin{equation}
			\begin{aligned}
				p(x)&=L(x)\cdot R(x)-O(x),
			\end{aligned}  
			\label{equ:qap}
		\end{equation}
		$L(x)=l_{0}(x)+\sum_{k=1}^{m}(c_{k}\cdot l_{k}(x))$, $R(x)=r_{0}(x)+\sum_{k=1}^{m}(c_{k}\cdot r_{k}(x))$, and $O(x)=o_{0}(x)+\sum_{k=1}^m(c_{k}\cdot o_{k}(x))$.
	\end{definition}
}

\new{Especially, a circuit with addition and multiplication gates (arithmetic circuit) can be directly represented by QAP by instantiating the polynomials. With this property, QAP has been widely used and abstracted as a constraint system called R1CS. In this paper, we do not distinguish these two concepts.}


\begin{figure*}[t]
	\centering
	\includegraphics[width=\linewidth]{img/overview.pdf}
	\caption{\textbf{The master recipe.} General steps of converting a high-level program to a \zk.}
	\label{fig:overview}
\end{figure*}

\section{Overview}
\label{sec:overview}

In this section, we introduce the \textit{master recipe} of constructing a \zk and discuss the development within each component in \autoref{fig:overview}.
To construct a \zk for general programs, an original program (written in a specific high-level language) is first converted to a circuit form called compilation. Then different constraint systems are utilized to represent the circuit satisfiability problem in mathematical form, a.k.a. Arithmetic Intermediate Representation (\text{AIR}).
Then we need cryptographic protocols to prove the satisfiability of an AIR.
For instance, giving an R1CS, we need an information theoretical protocol to actually prove it. The techniques to instantiate such protocols mainly determine the properties of the final \zk, such as transparency, post-quantum
security and efficiency. They are also our main classification criteria.
Finally, we take a generic transformation to transform the instantiated information theoretical proof into \zk. 
Despite the variations in tools and implementation details, the majority of research topics in \zk fall into our \textit{master recipe}, and we discuss each component in detail as follows.

\mypara{Compiling High-level Programs} \news{Generally, a compiler in \zk implementation compiles a high-level program into AIR that fits a certain constraint system. Currently, the compilers only compile languages that are specific to ZK. These languages are different from the commonly used, general languages like C and Python. Their behaviors are specific to defining a circuit, and the tools and libraries in commonly used languages cannot be recognized by a ZK compiler.}

\mypara{Constraint Systems} With efficient compilers, the high-level program is compiled into the AIR of the circuit, which contains all cryptographic expressions for the relationship between the program's input and output. Generally, a circuit is an abstraction of high-level computation, and a constraint system is a mathematical NP statement that we want to prove. In most cases, these two are similar, and in this paper, we do not distinguish them. \new{Here, we show a classical example where a circuit-like function is transformed to NP language R1CS. Assume we want to prove the computation of $f(w,a,b)=w\cdot(a+b)+(1-w)(a\cdot b)$. If we denote variable $y$ as the output, we can represent the computation by adding variable constraints: $w\cdot(a+b)=y_{1}$, $(1-w)\cdot a=y_{2}$, $b\cdot y_{2}=y_{3}$, $(y_{1}+y_{3})\cdot 1=y$. Following the QAP definition in \autoref{Def-QAP}, the form of R1CS constraint system is: 
	\begin{equation}
		\begin{aligned}
			(l_{0}(x)+\sum_{k=1}^{m}(c_{k}\cdot l_{k}(x)))\cdot(r_{0}(x)+\sum_{k=1}^{m}(c_{k}\cdot r_{k}(x))) \\        
			=(o_{0}(x)+\sum_{k=1}^m(c_{k}\cdot o_{k}(x))).
		\end{aligned}    
	\end{equation}
	
	\noindent Since we totally have 6 variables $w,a,b,y_{1},y_{2},y_{3}$, $m$ is set as 6. Besides, consider that there are 4 constraints. Polynomials $l_{i}, r_{i}$ and $o_{i}$ are evaluated at 4 points and their values should equal the coefficients of the corresponding variable. For instance, let $w$ denotes $c_{1}$, we have $l_{1}(1)=1$ and $l_{1}(2)=-1$, while other points on $l_{1}$ equal $0$ as $w$ does not exist.
}

Common constraint systems include R1CS~\cite{gennaro2013quadratic}, plonk circuit~\cite{gabizon2019plonk} and their variants  such as layered circuits~\cite{xie2019libra,ben2019scalable} and custom plonk~\cite{chen2023hyperplonk}. These constraint systems differ in algebraic structures for high-level computation, making it troublesome for a non-expert developer to understand them completely. \new{For instance, all wire values in plonk circuit are evaluated in one polynomial, while in R1CS the evaluations only encode the existence and coefficients of the variables.}
In most libraries, the languages that define a circuit are related to underlying constraint systems, and developers are required to understand these systems.

\mypara{Proving Systems} 
\new{Proving systems refer to the protocols between the prover and verifier, proving the correctness of a well-defined circuit. {A specific proving system~\cite{gennaro2013quadratic} for the above R1CS example utilizes the bilinear group. The basic idea is that the prover generates group elements $g^{L(x)},g^{R(x)},g^{O(x)}$ and $g^{t(x)}$, then the verifier checks if 
		\begin{equation}\label{equ:proving}
			e(g^{L(x)},g^{R(x)})=e(g^{t(x)},g^{q(x)})\cdot e(g^{O(x)},g),
		\end{equation}
		where $L(x),R(x),O(x),q(x),t(x)$ are defined in \autoref{Def-QAP}, $e$ is bilinear mapping function, and $g$ is the generator of the group.} The advantage of such a proving system is that the proof only consists of a few group elements.}

The proving system is the core component in a \zk and has been widely studied in research. 
A main consideration in choosing proving systems is the desired properties, such as scalability, transparency, post-quantum security and universal setup. Currently, practical {\zk}s with constant proof size and fast verifier are based on QAP techniques~\cite{gennaro2013quadratic,groth2016size} or pairing \PCS~\cite{gabizon2019plonk,chiesa2020marlin}. Those {\zk}s require a trust setup. To eliminate the trust setup, there are {\zk}s utilizing \PCS based on discrete logarithm problem~\cite{bunz2018bulletproofs,bunz2020transparent,halo2book} or hash function with code theory~\cite{chiesa2020fractal,ben2019aurora}. The above schemes all have a slow prover, which is quasi-linear. To achieve a fast prover with linear time, several works~\cite{chen2023hyperplonk,setty2020spartan,golovnev2023brakedown,xie2022orion} design multilinear IOP and multilinear \PCS. However, these approaches utilize more rounds of communication, which significantly increases the proof size. Due to the complicated categories of {\zk}s, it requires expert knowledge of the underlying construction of {\zk}s to choose an appropriate scheme for a particular application. In \autoref{sec:4} we solve this problem by providing a comprehensive classification of existing proving systems.

% We classify existing proving systems by information theoretic proof (ITP), which captures the high-level idea for the construction of these proving systems.
% In the context of ITP, existing proving systems fall in the category of \PCP and \IP. \PCP protocols have only one round of interaction and can be directly transformed into a non-interactive form. The mainstream technique of constructing a \PCP is through QAPs, and there are lots of constructions~\cite{gennaro2013quadratic,danezis2013pinocchio,ben2013snarks,groth2016size,groth2018updatable,groth2017snarky} where Groth16~\cite{groth2016size} is the state-of-the-art. \IP protocols allow multiple rounds of interaction and must use a generic transform to obtain a \text{NIZK}. \IP is flexible and can be easily constructed with different properties, including transparency, fast prover or verifier, sublinear proof size, post-quantum security, etc., and thus is the mainstream of \zk research. A popular way of constructing a \IP is through \PIOP and \PCS. Recent works leverage the power of \PCS or design new pattern of {\PIOP}s to optimize further the performance of \zk~\cite{}. Additional, \IP can also be constructed by multiparty computation~\cite{giacomelli2016zkboo,chase2017post,gvili2021booligero,bhadauria2020ligero++,katz2018improved} or the sumcheck protocol in layered circuits~\cite{ben2019scalable,wahby2018doubly,xie2019libra,zhang2020transparent,zhang2017vsql}. However, those approaches suffer from efficiency problems and have been substituted by the \text{PIOP} technique. 
% We elaborate our classification in detail in \autoref{sec:4}.

\mypara{Optimizers}Nowadays, \text{PIOP}-based {\zk}s have achieved the optimized asymptotic complexity for general circuits by introducing linear provers, sublinear proof size and sublinear verifiers. However, the efficiency in specific circumstances can still be improved. For example, recursive~\cite{halo2,plonky2,bunz2020recursive} or aggregate proof~\cite{bunz2018bulletproofs,chung2022bulletproofs+} shrinks the proof size where the verifier needs to verify a sequence of computations. Elastic proof~\cite{bootle2022gemini} and parallel proof~\cite{ephraim2020sparks} allow the prover to adjust the memory and time when proving dynamically. Lookup tables~\cite{campanelli2024lookup} specify the range of the witness to shrink the size of the generating circuit.
It is also possible to improve the performance of modern CPU architecture and specific schemes by optimizing elliptic curve operations~\cite{el2022families}.

\new{
	\mypara{Applications}We can use a general purpose \zk in various applications and prove different computations: (1) In the confidential blockchain, \zk can be utilized to prove a transaction is valid (e.g., if the sender has sufficient funds, the transaction is properly signed and the value is in a certain range) without revealing the details of the transaction to the public, which solves the privacy problem in Bitcoin. Existing blockchain applications include zcash~\cite{sasson2014zerocash}, Ethereum~\cite{wood2014ethereum}, zkSync~\cite{zksync}, and Aztec~\cite{aztecprotocol}, etc. (2) In zero-knowledge machine learning (ZKML), \zk can be used to verify the correctness of training process without revealing the underlying data. This allows the prover to train a model in a verifiable way without sharing her local datasets. Existing ZKML applications focus on generating the proof for decision trees~\cite{zhang2020zero}, federated learning~\cite{duan2024verifiable}, and convolutional neural networks~\cite{liu2021zkcnn}, etc. (3) In cryptography, \zk has been employed to build post-quantum signatures~\cite{chase2020picnic}, verifiable differential privacy mechanisms~\cite{biswas2023interactive}, and oblivious transfer~\cite{hazay2010efficient}, etc. 
}

% \begin{takeaway}[Takeaways]
	%     \textbf{\new{Determine the scope of the work --}}
	%     \new{With the master recipe, a practitioner can better determine the scope of their work, position their problems and understand how the pieces work together as a \zk. }
	
	% \end{takeaway}

\begin{takeaway}[Takeaways]
	\textbf{\new{Determine the scope of the open problems --}}
	\new{With the master recipe, a practitioner can better determine the scope of their work, position their problems and understand how the pieces work together as a \zk. For instance: (1) The latest works which reduce prover time include developing more efficient proof systems, improving circuit compilers and leveraging hardware acceleration (optimizer). (2) In \cite{groth2016size}, a theoretical problem is proposed if three elements are the optimized proof size for \zk. The question is positioned in the proving system and interested readers can focus on its progress without being distracted after understanding the functionality of other components.}
	
\end{takeaway}

\section{Classification of Proving Systems}
\label{sec:4}
\begin{table*}
	
	\renewcommand{\arraystretch}{1.2}
	\setlength{\tabcolsep}{3pt}
	\resizebox{\textwidth}{!}{%
		\begin{threeparttable}
			\begin{tabular}{llllllcllllcl}
				\toprule
				\multicolumn{2}{c}{\textbf{Information Theory}} & \multicolumn{2}{c}{\textbf{Methodology}} & \multicolumn{3}{c}{\textbf{Privacy}} & \multicolumn{3}{c}{\textbf{Scability}} & \multicolumn{1}{c}{\textbf{Examples}} & \multicolumn{1}{c}{\textbf{References}} \\ \cmidrule(lr){1-2}\cmidrule(lr){3-4} \cmidrule(lr){5-7} \cmidrule(lr){8-10} \cmidrule(lr){11-11}
				\cmidrule(lr){12-12}
				\makecell[l]{Type} & \makecell[l]{Variants} & \makecell[l]{Constraint\\ System} & Technique& \makecell[l]{Underlying \\ Problem } &   \makecell[l]{Post \\ Quantum } &
				\makecell[l]{Transparent \\ Setup } &
				\makecell[l]{P Time} & \makecell[l]{V Time} &
				\makecell[l]{Proof Size} &
				&  &  \\ \cline{1-12}
				
				{\makecell[l]{PCP}} & \makecell[l]{LPCP} & R1CS & QAP  & \makecell[l]{q-type\\KoE}& \Circle & \xmark & $\mathcal{O}(N\log N)$ & $\mathcal{O}(l)$ & $\mathcal{O}(1)$ & Groth16 & 
				\cite{gennaro2013quadratic,danezis2013pinocchio,groth2016size,groth2018updatable} \\  
				\cline{1-12}
				
				\multirow{7}{*}{\makecell[l]{IP}} & / &  Layered circuits & GKR & hash& \CIRCLE & \cmark &  $\mathcal{O}(N)$ & $\mathcal{O}(d\log N)$ & $\mathcal{O}(d\log N)$ & \makecell[l]{Virgo, Stark} & {\cite{wahby2018doubly,xie2019libra,zhang2020transparent,zhang2021doubly}} \\ 
				\cline{2-12}
				
				& \multirow{4}{*}{\makecell[l]{PIOP}} & \multirow{4}{*}{\makecell[l]{R1CS/ Plonk}} & KZG PCS & pairing& \Circle &\xmark & $\mathcal{O}(N\log N)$ & $\mathcal{O}(l)$ & $\mathcal{O}(1)$ & Plonk, Marlin & \cite{maller2019sonic,chiesa2020marlin,zhang2017vsql}  \\ 
				
				\cline{4-12}
				&  &  &IPA PCS & \makecell[l]{discrete \\log}& \Circle & \cmark & $\mathcal{O}(N)$ & $\mathcal{O}(\log N)$ & $\mathcal{O}(\log N)$ & \makecell[l]{Halo,\\Bulletproof} & \cite{bunz2018bulletproofs,dalek-bulletproofs,chung2022bulletproofs+,eagen2024bulletproofs++} \\
				\cline{4-12}
				& & & FRI PCS& hash & \CIRCLE & \cmark & $\mathcal{O}(N)$ & $\mathcal{O}(\log^{2}N)$  & $\mathcal{O}(\text{polylog} \ N)$ & Aurora, Fractal & \cite{ben2019aurora,chiesa2020fractal,zhang2021doubly}\\
				\cline{2-12}
				& \makecell[l]{Multi-\\PIOP}& R1CS/ Plonk & Multi-PCS & / & \RIGHTcircle & \cmark & $\mathcal{O}(N)$ & $\mathcal{O}(l)$ & $\mathcal{O}(\log N)$& Hyperplonk, Spartan& \cite{chen2023hyperplonk,setty2020spartan,xie2019libra,xie2022orion,golovnev2023brakedown}\\
				\cline{2-12}
				
				& / &  \makecell[l]{Boolean/Arithmetic \\circuits} & MPC & / & \RIGHTcircle & \cmark &  $\mathcal{O}(N)$ & $\mathcal{O}(N)$ & $\mathcal{O}(N)$ & \makecell[l]{Zkboo} & {\cite{giacomelli2016zkboo,chase2017post,katz2018improved}} \\ 
				
				
				\bottomrule
			\end{tabular}%
		\end{threeparttable}
	}
	\caption{\small \textbf{Classification of ZKPs from different perspectives.} Post Quantum: \Circle: not post-quantum secure, \CIRCLE: plausible post-quantum secure, \RIGHTcircle: partial works in the category are post-quantum secure. Scalability: For R1CS, the circuit size $N$ denotes the number of multiplication gates. For plonk circuit, $N$ is the sum of the addition gate and the multiplication gate. For layered circuits, the circuit size $N=dg$, where $d$ and $g$ are the depth and width of the circuit, respectively. In these circuits, $l$ denote the input size.
		The asymptotic complexity in scalability stands for the optimized scheme in the category.}
	\label{tab:class}
\end{table*}


In this section, we discuss proving systems, the core of \zk field. We classify {\zk}s into two categories termed as \text{PCP} and \text{IP} based on the information-theoretic proof. We discuss the techniques used to construct a \zk in each category and summarize the properties essential for both researchers and developers, such as transparency, post-quantum security, universal setup and efficiency. A comprehensive classification table is provided in~\autoref{tab:class}.
% and a comparison table in ~\autoref{tab:info}.

\subsection{PCP-based {\zk}s}
\label{sec:4.1}
\new{Probabilistically checkable proof (PCP, see \autoref{PCP}) allows for the verification of proofs with extremely high probability by checking only a tiny, randomly chosen portion of the proof. This is in stark contrast to traditional proof verification, which requires reading the entire proof.}

Earlier works~\cite{kilian1992note,groth2010short} of PCPs have high asymptotic complexity and do not focus on general computation models. In 2013, Gennaro et al.~\cite{gennaro2013quadratic} proposed the first efficient {\zk} for general circuits utilizing the quadratic span program (a.k.a. QSP, a weak form of QAP) technique. \new{The basic idea of this category is to construct a set of polynomial equations and use pairings to verify these equations. As an example, to check the validity of \autoref{equ:proving}, one needs four group elements $g^{L(x)}$, $g^{R(x)}$, $g^{O(x)}$ and $g^{t(x)}$ ($q(x)$ can be predefined when instantiating QAP). However, more elements are required to make sure these four elements are indeed computed from the linear combinations of the polynomial coefficients. Besides, we also need to ensure that the same coefficients are used in each linear combination, which we call consistency checks. These checks are based the the Knowledge of Exponent (KoE) assumption~\cite{bellare2004knowledge} and the security guarantee for the group operations is q-type assumption, discussed in \cite{gennaro2013quadratic}. 
	% saying that given two group elements $g,g^\alpha$, it is infeasible to find out another two elements $h,h^\alpha$ without knowing an exponent $c$ such that $h=g^c$ and $h^\alpha=(g^\alpha)^c$. 
	Specifically, the consistency check consists of two aspects:
	% where the verifier sends $(g^{l},g^{\alpha l})$ and the prover returns ($g_{1},g_{2}$) to pass the check $g_{1}^{\alpha}=g_{2}$. If $g_{1}$ is not generated by doing exponentiation from $g^{l}$, such check will fail. The consistency checks are as follows: 
	\begin{itemize}[noitemsep, topsep=2pt, partopsep=0pt,leftmargin=0.4cm]
		\item Polynomial consistency check: The prover computes $g^{L(x)}$ and $g^{\alpha L(x)}$, and the verifier checks if $e(g^{L(x)},g^{\alpha})=e(g^{\alpha L(x)},g)$ holds. For all polynomials, the prover also computes group elements for $R(x),O(x),t(x)$ and carries out this check on them.
		\item Variable consistency check: Given random values $\beta_l,\beta_r,\beta_o$ generated by trusted setup, the prover computes $\prod_{i}^{m} (g^{\beta_{l}l_{i}(x)+\beta_{r}r_{i}(x)+\beta_{o}o_{i}(x)})^{c_{i}}$ as part of the proof, denoted as $g^{Z(x)}$. The verifier checks if $e(g^{L(x)},g^{\beta_{l}\gamma})\cdot e(g^{R(x)},g^{\beta_{r}\gamma})\cdot e(g^{O(x)},g^{\beta_{o}\gamma})=e(g^{Z(x)},g^{\gamma})$.
	\end{itemize}
}

\new{To shrink the proof size, Danezis et al. \cite{danezis2013pinocchio} replace $g^{\beta_{l}}$, $g^{\beta_{r}}$ and $g^{\beta_{o}}$ with three basic group elements $g_{l},g_{r},g_{o}$. Such a replacement saves the need for $\gamma$ and eliminates one element from the proof. In 2016, Groth~\cite{groth2016size} integrated the validity check, polynomial and variable consistency checks into one equation using only three pairings. The proof size was further reduced to an optimized three elements.} Following these theoretical advances, practical work has been done on building concrete implementations. Those works focus on designing a compiler for QAP~\cite{danezis2013pinocchio,ben2013snarks,ben2014succinct}. Since Groth16~\cite{groth2016size} is the optimized QAP-based approach in theory, follow-up works further analyze the security properties~\cite{lipmaa2022unified} and apply it to specific applications together with different models, such as multiparty setup~\cite{bowe2017scalable}, universal reference string (URS)~\cite{groth2018updatable} and recursive proof~\cite{ben2017scalable}. 

The proof size in these systems remains constant, and the time for a prover is linear. These attributes are particularly advantageous and have facilitated real-world implementations, such as ZCash~\cite{sasson2014zerocash} and Pinocchio coin~\cite{danezis2013pinocchio}. Nevertheless, a significant limitation of QAP-based systems is the substantial overhead in prover running time and memory consumption, which poses challenges for scaling to large statements. Additionally, each statement necessitates a separate trusted setup.

\subsection{IP-based {\zk}s}
Interactive proof (\text{IP}) is a generalization of \text{PCP} in which the verifier can send random messages to the prover for multiple rounds.
The construction of IP is divided into two steps: (1) construct a proof which models the message sent by the prover as oracles; and (2) instantiate the oracles with well-defined cryptographic techniques. The first part is also known as PIOP where the prover needs to send a commitment of a polynomial. The technique in the second part is PCS which convinces a verifier that evaluations of a polynomial sent by the prover are correct.
IP can eliminate the trust setup, long common reference string (CRS), and slow prover in QAP-based {\zk}s, and it has been a mainstream in the design of state-of-the-art proving systems.

\subsubsection{GKR-based IP for Layered Circuits}
\label{sec:GKR}
Earlier IPs are mainly designed for layered circuits where each gate can only connect to the layer above. Goldwasser-Kalai-Rothblum (GKR) protocol~\cite{goldwasser2015delegating} is designed to prove the satisfiability of such a circuit by a layer-to-layer reduction. 
\new{The basic idea in this category is that for each layer the prover proves that the gate's output is correctly computed from last layer's output. Denote the number of gates in the $i$-th layer as $S_{i}$ and $s_{i}=\log {S_{i}}$, the label of the wire is $a$, the value of wire $a$ in layer $i$ as $V_{i}(a)$, and the wire predict $\textsf{ADD}_{i}(a,b,c)$ and $\textsf{MUL}_{i}(a,b,c)$ (return 1 when $a,b,c$ combine an addition or multiplication gate, respectively). The GKR prover proves for each wire $c$ in each layer $i$, the following equation holds:
	\begin{equation}
		\label{equ:gkr}
		\begin{aligned}
			V_{i+1}(c)=\sum_{a,b\in \{0,1\}^{s_{i}}} (&\textsf{ADD}_{i}(a,b,c)\cdot (V_{i}(a)+V_{i}(b))\\
			+ \ &\textsf{MUL}_{i}(a,b,c)\cdot V_{i}(a)V_{i}(b))
		\end{aligned}
	\end{equation}
}
\new{\noindent The first GKR protocol has cubic complexity prover, which proves \autoref{equ:gkr} by sending commitments of the circuit values $V_{i}(c)$ and their linear combinations. Several follow-up works~\cite{wahby2018doubly,xie2019libra,zhang2020transparent,zhang2021doubly}
	extend the functions $V,\textsf{ADD},\textsf{MUL}$ in \autoref{equ:gkr} to polynomials as if they are defined in a large field and utilize polynomial evaluations to optimize the complexity to quasi-linear.}
The GKR-based approaches are doubly efficient, meaning that they have a quasi-linear prover along with an efficient verifier where the verifier time is linear to the input of the layered circuit. Despite the advancements of the GKR protocol, a significant limitation is that it only works on layered arithmetic circuits. This introduces a significant overhead when padding general circuits to layered circuits using dummy gates. 

\subsubsection{PIOP for General Circuits}
\label{sec:PIOP}
To construct {\zk}s for general circuits such as R1CS and plonkish circuit, a new construction of IP has been proposed. It utilizes a generalized form of IP called PIOP, which models the message sent by the prover as polynomial oracles, which returns polynomial evaluations. To get an IP, the oracles in PIOP must be instantiated with a PCS, which evaluates a polynomial on a specific point with soundness and privacy. We discuss the features of three different constructions of PCS for univariate PIOP and briefly outline the idea of multivariate PIOP.

\mypara{Univariant PIOP} The idea of univariant PIOP is to model the computation in the general circuit as a polynomial and then prove its properties. \new{The prover uses a polynomial $T$ to encode the values in the whole computation trace, such as the inputs and wire values, and a gate polynomial $S$ to encode all the addition and multiplication gates, e.g., $S(a)=0$ if $a$ is an addition gate and $S(a)=1$ represents a multiplication gate. The prover proves the circuit satisfiability by the following equation for any $y$:
	\begin{equation}
		\label{equ:plonk}
		\begin{aligned}
			S(y)[T(y)+T(\omega y)] + (1-S(y))T(y)T(\omega y)=T(\omega^{2}y),
		\end{aligned}
	\end{equation}
	where $\omega$ is a gate offset, $T(y),T(\omega y),T(\omega^{2}y)$ denote the left input, right input and output of gate $y$, respectively. There are various other polynomial relations related to $T$ and $S$ to ensure the circuit is correct such as zero-test, product-test and permutation-test. All the tests are proved by utilizing PCS, where the prover sends the commitment of these polynomials first and then evaluates them on the point given by the verifier with zero knowledge. The soundness and privacy of all the tests are based on underlying PCS which can fall into three categories.}

\noindent \underline{\textit{{PIOP} with pairing.}} 
The polynomial commitment by Kate, Zaverucha and Goldberg (KZG)~\cite{KZG10} has evaluation proofs that consist of only a single bilinear group element, and verifying an evaluation requires only a single
pairing computation. \new{To evaluate $f(u)=v$ on point $u$, the prover constructs $f(x)-v=(x-u)t(x)$ for some polynomial $t(x)$ and computes the proof as $\pi=g^{t(s)}$, where $s$ is a secret value computed in the trust setup. The verification is done through a pairing operation $e(com/g^{v},g)=e(g^{s}/g^{u},\pi)$ ($com$ is the commitment for the polynomial generated in the setup).
	However, this asymptotically optimal performance comes at the cost of a trusted setup that outputs $g^{s}$ and $s$ must be deleted after generation. 
	
	Many efforts have been made to integrate the KZG PCS into {\zk}s. Plonk~\cite{gabizon2019plonk} utilizes the PCS to evaluate \autoref{equ:plonk}, achieving a short proof and fast quasi-linear prover.
	Similar to Plonk's technique, Marlin~\cite{chiesa2020marlin} applies the KZG PCS to instantiate PIOP to prove the satisfiability of R1CS. It achieves better efficiency for certain types of computation that map well to R1CS (addition gates do not contribute to R1CS's complexity). Some other works~\cite{bunz2021proofs,campanelli2021lunar,zhang2024efficient,aranha2022eclipse} add more features to the \zk in this category like updatable setup and accelerators.
}

\noindent\underline{\textit{{PIOP} with inner-product argument (IPA).}} 
To eliminate the trust setup in pairing-based PCS, BulletProof~\cite{bunz2018bulletproofs} instantiates the \text{PIOP} through a new \text{PCS} using IPA-based techniques. \new{The idea of IPA PCS utilizes algebraic tricks. By proving a polynomial $f$ with degree $m$ equals $v$ at point $u$
	(i.e., $f(u)=\sum_{i=0}^{m}c_{i}u^{i}=v$ where $c_{i}$ is the coefficient), the prover folds the polynomial to two parts as $f(u)=f_{L}(u)+u^{m/2}f_{R}(u)$. By first proving the correctness of the folding and then recursively invoking the procedure, the prover is able to get a logarithmic proof and a linear proving and verifying time related to the polynomial degree.}

\new{Following this technique, Hyrax~\cite{wahby2018doubly} represents the coefficients in a matrix achieving $O(\sqrt{m})$ prover complexity as a refinement. Dory~\cite{lee2021dory} improves the verifier time to logarithmic by introducing a linear combination of the polynomial's coefficients. Other works further optimize the performance in this category achieving both logarithmic time in prover and verifier sides~\cite{bunz2020transparent,wang2022flashproofs,lipmaa2020succinct,arun2023dew}. Several works find IPA PCS is suitable for range proofs and have continued to design optimizers such as aggregate proof, recursive proof and updatable proof in blockchain settings~\cite{bowe2019halo,halo2,chung2022bulletproofs+,eagen2024bulletproofs++,attema2020compressed,wang2022flashproofs,daza2020updateable}. As IPA PCS is based on the hardness of the discrete logarithm problem, the resulting schemes are not post-quantum secure. 
}

\noindent\underline{\textit{{PIOP} with code theory.}} To achieve both transparent setup and post-quantum security, Ligero~\cite{ames2017ligero} utilizes the linear code in code theory to construct a PCS. \new{In linear code, an $[n,k,\Delta]$-code has three properties: (1) it can encode an arbitrary message to a codeword; (2) the minimum distance (Hamming) between any two codewords is $\Delta$; and (3) any linear combination of codewords is also a codeword.
	In Ligero, Reed-Solomon code~\cite{wicker1999reed} is used which views the message as a $k-1$ degree polynomial and views the codeword as its evaluations at $n$ fixed points. In PCS, the $m+1$ coefficients of the polynomial are first encoded into $\mathcal{O}(\sqrt{m})$ codewords. Then the prover commits to the codewords using the Merkle tree to enable the existence check of specific codewords. To verify the evaluation $f(u)=v$, the verifier sends a message $(1,u,\ldots,u^{\mathcal{O}(\sqrt{m})})$ requesting the prover to do linear combinations of the codewords using the message as coefficients. The prover checks (1) the result is generated using the codeword committed before (utilizing the Merkle tree); and (2) the result is a codeword in the same class of the encoding codewords.
	As the message is $\mathcal{O}(\sqrt{m})$-length, the prover size and verifier time both have $\mathcal{O}(\sqrt{m})$ complexity. A bottleneck in the prover side is encoding the polynomial requires FFT which has $\mathcal{O}(\sqrt{m})$ complexity.}

\new{Later works generalize the idea of polynomial encoding by dividing the coefficients in the polynomial into multi-dimensions and encoding them into more codewords~\cite{bootle2020linear,bootle2018efficient} to achieve time-space tradeoff. In~\cite{golovnev2023brakedown}, a different code encoding algorithm is used to further accelerate the prover. 
	In Fractal~\cite{chiesa2020fractal} and other subsequent works~\cite{ben2019aurora,szepieniec2022polynomial}, a novel variant called Fast Reed-Solomon IOP of proximity (\text{FRI})~\cite{ben2018fast} is used. FRI treats the polynomial coefficients as a $\mathcal{O}(m)$-sized vector and recursively encodes it by folding it in half each time to achieve logarithmic proof size. By applying all above-mentioned advanced techniques in code theory, existing code PCS can achieve a logarithmic verifier and proof size, a linear prover and post-quantum security}. 

\mypara{Multivariant PIOP} Though efficient PCS can shrink the proof size and reduce the workload of the verifier, the usage of FFT to construct the key polynomial in the univariate PIOP has been a bottleneck on the prover side as it introduces a quasi-linear complexity. To resolve this efficiency issue, several works~\cite{ chen2023hyperplonk,setty2020spartan,xie2019libra,libert2024simulation,xie2022orion,golovnev2023brakedown} aim at multi-variant polynomial evaluation for eliminating FFT. Those works require modifying the PIOP protocol and PCS to a multivariant type and then using the sumcheck protocol for proving. The key polynomial can be constructed using the multilinear extension technique which only needs linear time.

\mypara{MPC-in-the-head} Several works prove the computation by letting the prover simulate the multiparty protocol~\cite{giacomelli2016zkboo,chase2017post,katz2018improved,ghosal2022efficient,baum2020concretely}.
The technique is called "MPC-in-the-head''. Since it incurs great overhead of the proof size and verifier, this kind of {\zk}s has not been widely implemented.

\begin{takeaway}[Takeaways]
	\textbf{\new{Trade-off between efficiency and security--}}\new{Linear PCP achieves constant proof size but at the cost of a trust setup. The {\zk}s in other categories try to mitigate this issue and all incur a sublinear proof size. In PIOP, compared to the usage of KZG PCS and IPA PCS, the code-based PCS incurs a significant constant overhead in proof size and prover time though the asymptotic complexity is similar.}
	
	\par \textbf{\new{Guidelines for choosing an appropriate proving system--}}
	\new{As a summary of this section, \autoref{tab:class} serves as a guideline for practitioners to choose their appropriate proving systems. We address a few important properties: (1) determine whether a trust setup is accepted. If yes, more considerations shall be taken when choosing the trust third party; (2) determine the appropriate scalability. For instance, blockchain applications prefer a fast verifier and small proof size in order to save transaction fee and the schemes in PIOP with pairing PCS category can be a good choice; and (3) determine if post-quantum security is necessary and choose code-based schemes if yes.}
\end{takeaway}

\section{Library Evaluation}
\label{sec:implementation}

\newcommand{\bin}{\textbf{Binary}}
\newcommand{\ct}{\textbf{CT}}
\begin{table*}[ht]
	\centering
	\resizebox{\textwidth}{!}{
		\begin{threeparttable}
			\rowcolors{2}{gray!20}{white}
			\begin{tabular}{l l l l  c c c l c c c c c c}
				\toprule
				Library & Year & Language & Technique& \makecell[l]{Circuit \\Generality} &  Compiler&\makecell[l]{User\\docus} & \makecell[l]{Example \\docus} & \makecell[l]{Example \\code}  & \makecell[l]{Online support} & \makecell[l]{Academic} &Commercial & \makecell[l]{Last \\update} \\
				\midrule
				
				libsnark~\cite{libsnark} & 2014 & C++ & LPCP-QAP & \cmark & eDSL & \CIRCLE & \Circle & \CIRCLE & \LEFTcircle & \xmark& \xmark& 02/2024 \\
				
				bellman~\cite{bellman} & 2017 & Rust & PIOP-IPA & \cmark & \textbackslash & \Circle & \Circle & \Circle & \Circle &\xmark & \cmark & 07/2024\\
				
				libSTARK~\cite{libsTark} & 2018 & C++ & IP-GKR & \xmark & \textbackslash & \CIRCLE & \Circle & \Circle & \Circle &\cmark & \xmark & 12/2018\\
				
				dalek~\cite{dalek-bulletproofs} & 2018 & Rust & PIOP-IPA & \xmark & \textbackslash & \CIRCLE & \CIRCLE & \CIRCLE & \LEFTcircle &\xmark & \xmark & 01/2024\\
				
				libiop~\cite{libiop} & 2019 & C++ & PIOP-FRI & \xmark & \textbackslash & \CIRCLE & \Circle & \Circle & \Circle &\cmark & \xmark & 05/2021\\
				
				snarkjs~\cite{snarkjs} & 2019 & JavaScript & PCP,PIOP & \cmark & DSL & \CIRCLE & \CIRCLE & \CIRCLE & \CIRCLE &\xmark & \cmark & 04/2024\\
				
				Spartan~\cite{spartan} & 2019 & Rust & PIOP & \cmark & eDSL & \CIRCLE & \LEFTcircle & \CIRCLE & \LEFTcircle &\xmark & \xmark & 04/2024\\
				
				gnark~\cite{gnark} & 2022 & Go & PCP,PIOP & \cmark & eDSL & \CIRCLE & \CIRCLE & \CIRCLE & \CIRCLE &\xmark & \cmark & 07/2024\\
				
				arkworks~\cite{arkworks} & 2022 & Rust & PCP,PIOP & \cmark & DSL & \CIRCLE & \Circle & \CIRCLE & \LEFTcircle &\xmark & \xmark & 01/2023\\
				
				halo2~\cite{halo2} & 2022 & Rust & PIOP-IPA & \cmark & eDSL & \CIRCLE & \CIRCLE & \CIRCLE & \CIRCLE &\xmark & \cmark & 02/2024\\
				
				plonky2~\cite{plonky2} & 2023 & Rust & PIOP & \cmark & eDSL & \CIRCLE & \CIRCLE & \CIRCLE & \CIRCLE &\xmark & \cmark & 08/2024\\
				\bottomrule
			\end{tabular}
	\end{threeparttable}}
	\caption{Comparison table of ZKP implementation libraries. In Circuit generality, \cmark: targets general circuit, \xmark: targets specific circuit. 
		In docus, example codes and online support column, \CIRCLE: full support, \LEFTcircle: partial support, \Circle: lack of support.  }
	\label{tab:info}
\end{table*}
We survey 11 general-purpose popular ZK libraries, all of which contain implementations for \zk protocols aforementioned. Our survey follows the steps in~\autoref{fig:overview} where a high-level program is first converted to an intermediate representation, a.k.a. a circuit, specified by a constraint system. Then, the circuit is passed to a proving system, which implements specific \zk techniques to output a proof. We limit our scope to \zk schemes proposed in the last decade with open-source implementations. Note that the industry in this field is rapidly developing, and some popular protocols, such as halo2~\cite{halo2book} and Plonk~\cite{gabizon2019plonk}, do not have peer-reviewed published papers yet. 
\news{We include those libraries as long as they have basic tools for implementing a circuit (e.g., gadget functions or compiler), their proving systems are popular (at least 5 citations in our references), and they are widely used (e.g., in commercial privacy-focused blockchain projects, or open-source project which have more than 200 GitHub stars and forks).}
% We include those libraries as long as they have enough applications, their corresponding published paper has great influence, or they are widely used in practice (many users are shown in GitHub stars). 
In this section, we compare each library from the perspectives of usability and efficiency\footnote{\new{All our codes and documents are available at \url{https://doi.org/10.5281/zenodo.14682405}.}}. 

\subsection{Basic Information}

We first survey basic information about these libraries, including language, technique, circuit generality, compilers and documentation. Our findings are summarized in~\autoref{tab:info}. The language refers to the programming language that implements the library. The techniques fall into four categories, with PIOP-based schemes being the most common. Circuit generality indicates whether a library supports general circuits. In~\autoref{sec:overview}, we classify R1CS and Plonk circuits as general, while layered circuits and range proofs are not. The latter two can be adapted to general circuits but at an efficiency cost.

Compilers refer to tools that convert high-level languages into circuit constraints, which we categorize in \autoref{sec:compiler}. We also identify valuable documentation types: user documentation (installation, usage, and testing) and example documentation (sample code for applications). Some projects offer additional support via GitHub issues or email.

While some libraries target commercial applications like blockchain transactions, others are research-focused. \news{Due to page limits, detailed discussions on basic information, toolkits, and documentation for each library are provided in \autoref{sec:applib}.}

\subsection{Usability Issues}
Note that some of the attributes in~\autoref{tab:info} represent critical challenges in engineering, which we explain below.

%\begin{itemize}[noitemsep, topsep=2pt, partopsep=0pt,leftmargin=0.4cm]
\mypara{Various Languages and Compatibility}Implementations of \zk schemes are limited across programming languages. For example, Plonk~\cite{gabizon2019plonk} is only implemented in Rust, making it challenging to use in applications written in other languages. Developers needing Plonk-based schemes must use Rust, which may not align with their preferences.
\new{Additionally, none of the libraries provide interfaces for compatibility. While components like constraint systems and proving systems can be separated in code, their functions and tools are confined to their respective libraries. For instance, we attempted to use circuits generated in \lib{libsnark} with \lib{libiop}'s proving systems to test Aurora and Fractal, as suggested by \cite{libiop-issue}. However, we faced significant challenges due to incompatible circuit formats, as there are no interface functions or documentation to bridge the gap.}

\mypara{Misuse of Circuits}Current libraries are not all focused on the general circuits. For instance, Bulletproof~\cite{bunz2018bulletproofs} targets range proofs and is not competitive enough compared with other schemes targeting general circuits like R1CS when designing complex applications. However, an appropriate choice requires expert knowledge of constraint systems, which is impractical for programmers. \news{We believe the master recipe in \autoref{sec:overview} and the classification table and explanations in \autoref{sec:4} can help mitigate this problem by enabling a practitioner to choose an appropriate scheme for her application.}

\new{\mypara{Misuse of Curves}The choice and usage of curves in each library are often implicit, leading 
programmers to overlook this critical configuration. However, selecting an inappropriate curve can reduce efficiency or introduce vulnerabilities. For instance, if the computation exceeds the finite field's limits, the system becomes unsafe, yet programmers may remain unaware. A common example is in blockchain range proofs, where programmers must ensure the curve's bit size exceeds the maximum transaction value; otherwise, severe commercial losses can occur. To address this, we documented the curves used in the surveyed libraries and provided guidelines for proper configuration.
}

\mypara{Lack of Compilers}Many libraries lack a compiler to convert high-level code into circuit representations, forcing programmers to manually add constraints. At the circuit level, programmers must handle intricate details like curve operations, loops, and permutations. For example, implementing a hash function like SHA256 requires tens of thousands of constraints, placing a significant burden on the programmer. Additionally, this task demands deep familiarity with both the programming language and the constraint system.

\mypara{Lack of Documentation}Here, we find that in many libraries, example documents are rather limited. For example, arithmetic circuits operate over a finite field whose size must be set in advance, but very few documents tell how to choose the size. The programmer is responsible for avoiding field overflow, which requires preliminary knowledge of complex field operations.
%\end{itemize}

\new{We have taken steps to address or mitigate these issues. For language and compatibility challenges, we created runnable Docker images for our test sample codes, enabling programmers to configure their environments without relying on cross-platform functions. To tackle circuit and curve misuse, we provided comprehensive guidelines in earlier sections and included a detailed discussion of curves in our project. For compiler-related problems, we categorized existing compilers in each library and analyzed their strengths and weaknesses to help programmers understand compiler concepts in the ZK context. Regarding documentation, we developed open-source materials, including a wiki-book documenting all APIs related to our master recipe components and three walk-through tutorials for our sample code in each library.}

\subsection{Compilers}
\label{sec:compiler}
\new{We identify compilers as the bottleneck of \ZK applications for two reasons. Firstly, during the implementation of our test code, most of the codes are for compilers and we have spent most of time debugging compiler-related issues. Secondly, according to~\cite{chaliasos2024sok}, more than 90\% of the vulnerabilities are found at the circuit level due to misunderstanding the compiler's languages. Here we discuss the categorization of existing compilers for practitioners to understand their features and functionality.
}
\subsubsection{Categorization}
\new{Commonly used compilers for zk are categorized into Domain-Specific Languages (DSLs), Embedded Domain-Specific Languages (eDSLs), and Zero-Knowledge Virtual Machines (zk-VMs). The input of DSL is an independent file with syntax tied to circuit constraints, separate from library functions, and its output is a separate file containing circuit information. The input of eDSL combines library functions related to the constraint system, often using \textbf{gadgets} (built-in functions for complex constraints like inner products or loop specifications); gadgets are tools, not compilers, that help build compiler inputs, and the output of eDSL is a data structure for the proving system. The input of zk-VM is opcodes compiled by general-purpose compilers, and its output is circuit information. We discuss the strengths and drawbacks of these compilers as follows. 
}

%\begin{itemize}[noitemsep, topsep=2pt, partopsep=0pt,leftmargin=0.4cm]

\new{\mypara{Domain-specific languages (DSLs)}DSLs are specialized programming languages designed for specific problem domains, offering tailored syntax to efficiently express constraints in arithmetic circuits for \zk. Current DSLs are categorized as hardware description languages (HDLs)~\cite{belles2022circom} or programming languages (PLs)~\cite{chin2021leo,ozdemir2022circ,amin2023lurk,eberhardt2018zokrates}. HDLs describe circuit synthesis directly in wire form, providing elegant syntax but posing challenges for programmers due to their independent wire-based structure and limited data type abstraction, as inputs are represented as signal data structures. In contrast, PLs define constraints in high-level programming languages, supporting various data types and resembling languages like Rust or Python. This makes PLs more accessible to programmers without wire form circuit knowledge, offering the easiest way to define constraints. However, PLs' flexible syntax increases vulnerability risks and introduces efficiency issues. Currently, learning DSLs is challenging due to the lack of standardization, with each DSL having an entirely different syntax.}

\new{\mypara{Embedded Domain-Specific Languages (\text{eDSL}s)}eDSLs for \zk have gained popularity in recent years and are implemented as functions within general-purpose programming languages, making them distinct from traditional compilers in the context of programming languages. In this paper, we generalize the concept of a compiler to include any tool that transforms its input into a circuit definition. eDSLs are designed to describe circuit synthesis, similar to HDLs, but they target wire form circuits while offering greater expressiveness and ease of use by inheriting data structures and programming features from the embedded language. Examples of eDSLs include implementations in Golang~\cite{gnark}, Rust~\cite{halo2,halo2ce,arkworks,plonky2,zksecurity2023noname,bellman}, C\&C++~\cite{libsnark,libiop}, Java~\cite{kosba2018xjsnark}, and TypeScript~\cite{mina2021o1js}. These eDSLs streamline the development of ZK proofs by integrating circuit definition and proof generation into a single file, simplifying code and enabling programmers to leverage existing library functionalities. However, writing code in eDSLs requires developers to explicitly distinguish between in-circuit and out-circuit operations, necessitating expert knowledge of the specific language and library design.}

\new{\mypara{Zero-Knowledge Virtual Machines (\text{zk-VM}s)}zk-VMs target the opcode of the fetch-decode-execute cycle, replicating the computation trace for general programs (typically smart contracts) and generating corresponding ZK proofs. They support various instruction set architectures (ISAs), including Ethereum Virtual Machine~\cite{scroll2023,polygon,era2023}, RISC~\cite{bruestle2023riscZeroZkVM,arun2024jolt}, and custom ISAs~\cite{goldberg2021cairo,zhang2023polynomial,bootle2018arya}. zk-VMs are compatible with existing high-level programming languages and can leverage features of existing compilers, such as gcc. However, despite targeting low-level opcodes, zk-VMs are not fully compatible with top-level applications and often require minor or major program modifications, which can be error-prone and difficult for programmers to manage. Additionally, zk-VMs use a Turing machine computation model instead of circuits, introducing significant overhead. While zk-VMs reduce the burden of writing constraints for programmers, they may suffer from efficiency issues, particularly for large-scale applications.}

\subsubsection{Compatiability}
\new{We assess the compatibility of these compilers according to two properties:}

\new{
	\mypara{Cross-compatibility} This indicates whether the compilation result of a compiler can be utilized by another one. DSL compilers offer moderate cross-compatibility as they separate the constraint system and the proving system. With the standardization of compilation results in the future, the libraries can only focus on providing the proving systems by taking DSL results as inputs. The eDSL compilers have low cross-compatibility as they define the circuit within a programming language which makes it difficult to use their defined circuit in platforms with other languages. Even in the same language, the compilation result may not be compatible because gadget functions are different, as we find in \lib{libiop} \cite{libiop} and \lib{libsnark} \cite{libsnark}. The zk-VMs have low cross-compatibility as they are only designed for some specific high-level programs. 
	
	\mypara{Syntax-compatibility} This indicates whether the input language of a compiler has a similar syntax to another one. Syntax-compatibility is important as it allows a programmer familiar with a language to move to another one without comprehensive studies. Unfortunately, we find even in the same category, the languages of the compiler have a completely different syntax and it will be hard to learn them all. In DSL, HDL is a hardware circuit language while PL is more like a general programming language. In eDSL, the syntax depends on the basic language of the library, ranging from C, C++, Rust, Go and JavaScript. In zk-VM, only opcodes from smart contract languages are well supported and opcodes from other general languages will not pass the compilation.
}

%\end{itemize} 
\begin{takeaway}[Takeaways]
	\textbf{\new{Absence of universal standardization}} -- \new{Current compilers are categorized as DSL, eDSL and zk-VM, and each has pros and cons. We identify two issues related to compatibility.
		Firstly, even in the same category, there are significant differences in the syntax, which makes it difficult to migrate projects and confuse programmers. Secondly, even for the same circuit, the compilation result cannot be used by a proving system in another library, though the compilers are designed separately from the proving system. We thus call for a universal standardization for these compilers including a standard language syntax and the compilation output.} 
\end{takeaway}

% The purpose of code provided by academia and industry is different. The academia focuses on the construction of a ZKP scheme, so their codes are just to verify the correctness of the scheme, and thus, the codes generally lack modularity and have poor readability and usability. Such as the code provided by ~\cite{giacomelli2016zkboo,wahby2018doubly,maller2019sonic} can only be compile and pass the tests provided by the author, with little documentation and no code description. For some other libraries such as ~\cite{xie2019libra,zhang2020transparent}, we tried to compile them according to the steps provided by the author but failed. While the industry focus on the code's modularity and efficiency, they usually have detailed documentation and teaching steps to facilitate user. That is exactly what we recommend coders to do, and it is also a criterion for us to choose libraries. We surveyed xx libraries, and list xx libraries here due to their extensive documentation, detailed usage steps, hot usage, various use cases or their supplements to some academic libraries.

\begin{table*}[t]
	\resizebox{1\textwidth}{!}{%
		\begin{threeparttable}
			
			\begin{tabular}{ll|lllll|lllll|lllll}
				\toprule
				\multirow{2}{*}{Library} & \multirow{2}{*}{Scheme} & \multicolumn{5}{c|}{Cubic expression} &\multicolumn{5}{c|}{Range proof} &\multicolumn{5}{c}{Hash} \\
				
				& &CRS &N &P &V &S &CRS &N &P &V &S &CRS &N & P &V &S \\
				
				\midrule
				\multirow{3}{*}{\textbf{libsnark}} & \small{Groth16} &0.86 &3 &0.008 &0.001 &0.13 &7.56 &39 &0.023 &0.001 &0.13 &4.19k &27.30k &0.92 &0.001 &0.13\\
				&\small{BCTV14} &1.74 &3 &0.013 &0.004 &0.28 &9.63 &39 &0.024 &0.004 &0.28 &6.28k &27.30k &0.97 &0.004 &0.28\\
				&\small{GM17} &2.11 &3 &0.010 &0.002 &0.13 &15.21 &39 &0.035 &0.002 &0.13 &10.30k &27.30k &1.78 &0.002 &0.13\\
				
				\midrule
				\multirow{2}{*}{\textbf{gnark}} & \small{Groth16} &4.65 &3 &0.002 &0.002 &0.56 &17.09 &22 &0.005 &0.003 &0.70 &100.50k &153.00k &0.28 &{0.002} &0.70 \\
				&\small{Plonk} &31.09 &4 &0.010 &0.004 &1.32 &40.11 &90 &0.003 &0.014 &1.44 &78.91k &599.20k &9.55 &0.002 &1.44\\
				
				\midrule
				\multirow{3}{*}{\textbf{snarkjs}} &\small{Groth16} &5.87 &2 &0.78 &0.70 &0.79 &26.22 &33 &0.76 &0.70 &0.79 &33.00k &59.00k &2.19 &0.71 &0.79 \\
				&\small{Plonk} &13.20 &4 &0.83 &0.71 &2.20 &195.14 &100 &0.94 &0.73 &2.20 &100.50k &241.70k &549.95 &0.76 &2.20  \\
				&\small{FFlonk} &19.67 &4 &0.81 &0.72 &2.20 &291.62 &100 &0.97 &0.70 &2.20 &11044.20k &241.70k &556.31 &0.71 &2.20 \\
				
				\midrule
				\multirow{3}{*}{\textbf{libiop}} &\small{Ligero} &\makecell[c]{\textbackslash} &4 &0.04 &0.01 &608.00 &\makecell[c]{\textbackslash} &32 &0.04 &0.02 &608.00 &\makecell[c]{\textbackslash} &\new{27.28k} &\new{2.195} &\new{2.081} &\new{3.01k} \\
				&\small{Aurora} &\makecell[c]{\textbackslash} &4 &0.022 &0.004 &35.40 &\makecell[c]{\textbackslash} &32 &0.026 &0.007 &50.78 &\makecell[c]{\textbackslash} &\new{32.77k} &\new{7.44} &\new{0.41} &\new{125.98}  \\
				&\small{Fractal} &\makecell[c]{\textbackslash} &4 &0.014 &0.007 &54.69 &\makecell[c]{\textbackslash} &32 &0.044 &0.013 &156.25 &\makecell[c]{\textbackslash} &\new{32.77k} &\new{8.83} &\new{0.012} &\new{201.44} \\
				
				\midrule
				\textbf{Spartan} & \small{Spartan} &\makecell[c]{\textbackslash} &4 &0.59 &0.32 &9.67 &\makecell[c]{\textbackslash} &32 &1.07 &0.45 &15.29 &\makecell[c]{\textbackslash} &32.77k &103.20 &2.07 &67.49 \\
				
				\midrule
				\textbf{arkworks} & \small{Groth16} &2.05 &3 &0.036 &0.033 &0.25 &15.48 &33 &0.037 &0.037 &0.25 &22.32k &58.94k &3.40 &0.036 &0.25 \\
				
				%\small{marlin} &? &? &? &? &? &? &? &? &? &? &? &? &? &? &? \\
				%\small{gm17} &? &? &? &? &? &? &? &? &? &? &? &? &? &? &? \\
				\midrule
				\textbf{halo2} & \small{Halo2} &\makecell[c]{\textbackslash} &2 &0.001 &0.001 &11.97 &\makecell[c]{\textbackslash} &33 &0.002 &0.002 &117.04 &\makecell[c]{\textbackslash} &242.65k &4.16 &0.13 &3.97 \\
				
				\midrule
				\textbf{plonky2} & \small{Plonky2} &\makecell[c]{\textbackslash} &2 &15.36 &0.19 &145.33 &\makecell[c]{\textbackslash} &9 &15.42 &0.19 &145.32 &\makecell[c]{\textbackslash} &261.98k &274.96 &0.28 &175.59 \\
				
				\midrule
				\textbf{dalek}& \small{Bulletproofs} 
				&\multicolumn{5}{c|}{\textbackslash}
				&\multicolumn{2}{c}{\textbackslash} &0.008 &0.001 &0.66 &\multicolumn{5}{c}{\textbackslash} \\
				\bottomrule
			\end{tabular}%
		\end{threeparttable}
		% \vspace{-4mm} 
	}
	\caption{{Main results.
			CRS: the size of a common reference string (KB), N: the number of constraints in a circuit, P: running time of generating a proof (s), V: running time of verifying a proof (s), S: the size of a proof (KB). Since some {\zk}s don't need trust setup, they have no CRS and we mark them with `\textbackslash'. Since Dalek-bulletproofs is used to generate range proofs and not for general circuits, we do not evaluate the Cubic expression or Hash on it.}}
	\label{tab:stat}
\end{table*}

\subsection{Experimental Evaluation}
In this section, we benchmark the performance of {\zk}s on three sample programs. \news{These  programs are all} well-designed and popular in real-world applications. \news{All our experiments are conducted on a server equipped with an Intel Xeon Silver 4314 CPU running at 2.40 GHz. The system is powered by 64 GB of RAM and the operating system used is Ubuntu 20.04.6 LTS.} Our results are reported in \autoref{tab:stat} and here we make two comments. 
\par Firstly, we compare the performance results with the asymptotical complexity of each scheme and give interesting findings that optimal theoretical complexity does not always result in better performance. We discuss why this is the case and recommend researchers discuss more suitable applications for their approach.
\par Secondly, we believe the quantitative
results of our sample programs are meaningful as a reference to practical applications for specific proving systems, but we emphasize that the results would not accurately represent the performance abilities of each scheme. The circuits used in each scheme are different in theory or have different implementations in practice. Besides, the security models vary in transparent setup, post-quantum security, universal reference string, etc., and in academic papers, they are only compared to counterparts in the same category. In our evaluation, we aim to show the common characteristics of each scheme and provide an intuitive comparison from an engineering perspective.

\subsubsection{Sample Programs}

We carefully design sample programs to evaluate the efficiency and usability of each library.

\mypara{A Cubic Expression} Our first example is a cubic expression proof that the prover proves that she knows $x$ that satisfies a polynomial $x^{3}+x+5=y$. This example tests the usability of a library and checks whether it is possible to add constraints manually for arbitrary small-size circuits without compilers. It also tests the basic efficiency of implemented schemes on small-sized circuits.

\mypara{Range Proof} Our second example proves that a value $x$ is in a certain range $[0, 2^{32})$. Range proof is a popular application in blockchain because it enables confidential transactions. Some systems like Bulletproof~\cite{bunz2018bulletproofs} are not designed for general circuits but for range proofs. This example compares such schemes with other general-purpose ones.

\mypara{Hash Function} Our third example is SHA256 hash function. The prover proves that she knows a value $x$ such that $y=\text{SHA256}(x)$ and only $y$ is known to the verifier. A SHA2 hash function is inefficient and has more than 30,000 constraints, which is impossible without compilers. For those libraries that only have proving systems, we test random circuits in the same quantity of constraints instead. The hash example tests the efficiency of the proving systems for large constraints. Additionally, it tests different constant systems, e.g., R1CS and Plonk, when representing the same function.



\subsubsection{Experimental Setup}
In this section, we talk about the criteria for choosing schemes for evaluation and evaluation metrics.

\mypara{Inclusion \& Exclusion Criteria} We aim to build a comprehensive benchmark for more \zk schemes both from papers accepted at top Crypto \& security conferences and industry popular projects (due to the long review cycle, several schemes have not yet been published but have various applications). We then made an initial attempt to run each approach, following the instructions in README on their GitHub homepages and applying the frontend and backend programming styles documented in their evaluation settings. We exclude libraries that either (1) are implemented by authors as materials for the paper or (2) fail to compile and with limited documentation or online support. In the end, we evaluate \textbf{twelve} schemes in {9} libraries, with \textbf{three} (Groth16~\cite{groth2016size}, BCTV14~\cite{ben2014succinct}, GM17~\cite{groth2017snarky}) under the category of QAP (\autoref{sec:4.1}), 
\textbf{one} (Ligero~\cite{ames2017ligero}) under GKR interactive proof (\autoref{sec:GKR}),
and \textbf{eight} (Plonk~\cite{gabizon2019plonk}, Aurora~\cite{ben2019aurora}, Spartan~\cite{setty2020spartan}, Bulletproof~\cite{bunz2018bulletproofs}, Halo2~\cite{halo2book}, 
Plonky2~\cite{plonky2},
Fractal~\cite{chiesa2020fractal},
FFlonk~\cite{gabizon2021fflonk}) 
under \text{PIOP} (\autoref{sec:PIOP}). 

\mypara{Evaluation Metrics} As each scheme has different properties and security models, we choose five general criteria, i.e., (1) the size of common reference string, (2) the number of constraints in the circuit, (3) the running time of the prover, (4) the running time of the verifier, and (5) the size of the proof. We assess different schemes with our three basic examples while more complex examples, such as scenarios that need to verify many proofs at once and a sequence of range proofs, are excluded. Some optimizers like recursive~\cite{halo2} or aggregate proof~\cite{eagen2024bulletproofs++} may perform well in these complex scenarios, but testing them is beyond the scope of this work.
\subsubsection{Performance Highlight}

\mypara{Best Practice} For different application scenarios, we recommend the best scheme along with its implementation. Groth16~\cite{groth2016size} is the best practice for applications that need a fast prover, a small proof size, and can tolerate a trust setup. \lib{gnark}~\cite{gnark} implements Groth16 more efficiently in Go, while \lib{snarkjs}~\cite{snarkjs} provides an implementation of Groth16 in Rust with more compatibility (using a DSL compiler).
Plonk~\cite{gabizon2019plonk} is the best practice for applications that need a transparent setup and are not sensitive to the slight increase of the proof size. For the widely used range proof, we recommend \lib{dalek}~\cite{dalek-bulletproofs}, which is designed for range proof, specifically. We also recommend \lib{gnark}~\cite{gnark}, \lib{arkworks}~\cite{arkworks}, \lib{snarkjs}~\cite{snarkjs}, \lib{halo2}~\cite{halo2} for study or research purposes as they have well-formed documents and running a proof in these libraries follows a complete walk-through of our master recipe.

\section{Discussion}
According to our findings, we advocate for documentation, standardization, and designing specific proving systems, which we explain in detail as follows.

\mypara{Documentation} Universally, the biggest obstacle when using \zk libraries is the lack of documentation. The community has dedicated thousands of hours to producing the work presented here, but not enough documentation makes these contributions less accessible. In the context of \zk field, we recommend two kinds of documents. One is the user document, which contains not only the necessary steps to run an example in the library but also the details of gadgets API in eDSL or the language syntax in DSL about how to define a circuit. The lack of documentation about the compiling phase hinders the library from the cryptographic developers. Besides, We find online support valuable when experimenting with these libraries where the issues in Github solved most of our problems. We also find a walk-through of examples provided by the developers of the library is very helpful. We thus advocate for dynamic documentation such as executable codes (as the docker resources we provided) and enough support through mail or Github. 

\mypara{Standardization} We advocate for two types of standardization. One is about the feature in the \zk field. Many of these libraries are designed around a particular feature, e.g., small proof with a trust setup in \lib{libsnark}~\cite{libsnark}, transparent and fast prover in \lib{arkworks}~\cite{arkworks}. The library's documentation about these core features is implicit, and developers need to understand underlying cryptographic techniques to choose an appropriate scheme. The standardization can help developers compare essential features across libraries and also set a more consistent baseline for performance. The other standardization we advocate is for the compiler. The existing libraries use different approaches such as DSL, eDSL and zk-VMs for defining circuits, which makes it difficult to reuse existing tools due to non-standardization of compilers.

\mypara{Specific Proving Systems} During our exploration, we find some libraries are designed for specific tasks, such as \lib{halo2}~\cite{halo2}, \lib{plonky2}~\cite{plonky2} for recursive proof and \lib{dalek} \cite{dalek-bulletproofs} for range proof. It remains an open question if proving systems for specific scenarios will perform better than generic proving systems. Designing such specific proving systems requires cooperation between the theory progress and engineering.  

\section{Conclusion}

In this paper, we systematically summarized research of \zk from theory to practice. We begin by presenting a master recipe for \zk, which outlines the key steps in constructing {\zk}s. We then examined each component in the recipe from both theoretical and engineering perspectives and identified gaps between them. Extensive efforts were made to evaluate different \zk libraries, and based on our findings, we offered recommendations for programmers and developers while providing new insights for future research.

\section*{Acknowledgments}
The authors appreciate \zk engineers Zhiwen Zhang and Yu'ao Zhou for their help and suggestions when preparing our open-source project.


\section{Ethics Statements and Compliance with the Open Science Policy}

\mypara{Ethics Statements} In this paper, all evaluated \zk libraries are open-source and freely available on GitHub or their respective homepages. As such, this research does not involve any ethical concerns, as it does not include activities that could pose harm or risk to individuals or organizations. We hope this work helps bridge the gap between theory and practice, providing valuable insights for researchers and developers working on \zk applications.

\mypara{Open Science Policy} We fully adhere to the principles of the Open Science Policy and are committed to promoting transparency and reproducibility in scientific research. In line with these principles, we ensure that all evaluated \zk libraries are available with their links provided in the references. 
Our artifacts consist of a completely virtual environment (Docker image), a walk-through tutorial for every test code and an API wiki book in which to run the compiler for each system and are available at \url{https://doi.org/10.5281/zenodo.14682405} as per the conference's requirements. 

\bibliographystyle{unsrt}
\bibliography{def,main}

\appendix

\section{Sudoku example for ITP}
\label{app: ITP}
\new{
	\mypara{Scenario} When convincing someone that a Sudoku puzzle has a unique solution, we can use IP, PCP or IOP and compare their difference.
	
	\mypara{IP} The verifier can ask any question she likes to the prover who has the complete solution, such as:
	\begin{itemize}[noitemsep, topsep=2pt, partopsep=0pt,leftmargin=0.4cm]
		\item "What's the number in row 3, column 5?"
		\item "Why can't the number 8 be in the 7th box?"
		\item "Explain how you deduced the number in row 2, column 1?"
	\end{itemize}
	
	\mypara{PCP} The prover writes the complete solution on a very large piece of paper (the PCP proof). The verifier is allowed to randomly choose a few cells to check (random oracle access to the proof):
	\begin{itemize}[noitemsep, topsep=2pt, partopsep=0pt,leftmargin=0.4cm]
		\item "Check the number in row 2, column 8."
		\item "Check the number in row 6, column 3."
		\item "Check the number in row 9, column 9."
	\end{itemize}
	
	\mypara{IOP} The prover also provides oracles like PCP but the verifier has more kinds of interactions. 
	\begin{itemize}[noitemsep, topsep=2pt, partopsep=0pt,leftmargin=0.4cm]
		\item First, the prover writes some hints on several sheets of paper (oracles), such as "the sum of each row and column is 45", "each box contains digits from 1 to 9", or a specific deduction step.
		\item Then, the verifier can ask questions about the hints, such as "show me the arrangement of numbers in row 3", or "show me the numbers in box 5".
		\item Last, the verifier can randomly check parts of the hints provided.
	\end{itemize}
}
\section{Library Surveys}
\label{sec:applib}
We survey each library in detail, including \lib{libsnark}~\cite{libsnark}, 
\lib{bellman}~\cite{bellman}, 
\lib{libSTARK}~\cite{libsTark},
\lib{dalek}~\cite{dalek-bulletproofs},
\lib{libiop}~\cite{libiop}, \lib{snarkjs}~\cite{snarkjs}, \lib{gnark}~\cite{gnark}, 
\lib{arkworks}~\cite{arkworks},
\lib{halo2}~\cite{halo2}, 
\lib{Spartan}~\cite{spartan}, 
and \lib{plonky2}~\cite{plonky2}. We discuss the challenges we encountered when implementing the sample programs and elaborate on limitations noted in the tables on the overall usability of each library. We compare the differences between academic and commercial projects and address recommendations to help the developer improve their projects. We also mention the history and the great contributions those projects made to \zk field. 

%We provide the detailed discussion in our open source materials in \url{https://doi.org/10.5281/zenodo.14682405} for interested readers.



% \begin{takeaway}[Recommended Practices]
	%     Across the course of our replication
	%     study, we have pinpointed several concerns in the main steps to construct an application that is universal in all libraries. 
	%     \begin{itemize}[noitemsep, topsep=2pt, partopsep=0pt,leftmargin=0.4cm]
		%         \item \textbf{Find a compiler.} It often remains unclear how to express the statements to be proved as R1CS or plonkish circuits. If no explicit compiler (i.e., DSL) is given, we recommend using the library's built-in gadgets as an alternative.
		%         \item \textbf{Frontend and backend are not separated.} The frontend functions as arithmetizing the circuit constraints to polynomials and field elements, and the backend selects a proving system. In the frontend, the user needs to choose an appropriate curve to avoid data overflow and to be compatible with backend schemes. However, we have found very few documents addressing this problem thus we recommend double configuration before implementation.
		
		%     \end{itemize}
	% \end{takeaway}



\subsection{libsnark}
\lib{libsnark}~\cite{libsnark} is a C++ project started in 2014 and is the first project that aims to provide comprehensive support for {\zk}s. The schemes~\cite{ben2014succinct,groth2016size,danezis2014square,groth2017snarky,backes2015adsnark,bitansky2013recursive} implemented in \lib{libsnark} are all based on QAP because at that time only QAP-based schemes are efficient. \lib{libsnark} offers a great overlook of QAP-based schemes, as it not only includes the most popular Groth16~\cite{groth2016size} scheme but also provides comparisons with former related works.

\mypara{Toolkits} The core toolkit in \lib{libsnark} involves \texttt{relation} (defining the representation of NP language), \texttt{reduction} (containing functions that convert each relation), \texttt{ppzksnark} (implementing the core proof systems), and \texttt{gadget} (simplifying the procedure of specified constraints). \lib{libsnark} defines various relations like \text{R1CS}, \text{TBCS} (Two-input Boolean Circuit Satisfiability), \text{USCS} (Unitary-Square Constraint System), \text{RAM} (Random Access Model), etc. A notable feature of \lib{libsnark} is that those relations can be converted from one to another using functions in \texttt{reduction}. Nowadays, the relations except \text{R1CS} and \text{RAM} are rarely used due to efficiency issues, and our tests are based on \text{R1CS}. \new{ In the compiler's syntax, \lib{libsnark} recommends defining the circuit using its gadgets interface, which allows us focus on the function level of our constraints and not a circuit level.
	In the \texttt{gadget}, \lib{libsnark} implements four useful data types such as variables, array, linear combination variable, linear combination array and arithmetic and loop operations for these types. All the variables are defined in C++ template and need to be instantiated using a specific curve. During our test, we find the syntax is powerful as we define the SHA2 circuit in less than 40 lines.}

\mypara{Documents and Support} The documentation in \lib{libsnark} for installing and testing is abundant. \lib{libsnark} carefully documents all its dependency libraries and discusses whether or not many platforms are compatible. However, there is not enough documentation about how to generate a proof for a new computation. The \texttt{gadget} only contains examples for the inner product and has no comments about how to use other gadget functions. Besides, \lib{libsnark} mentions that it is capable of providing complex high-level language. The language should first be compiled into an R1CS and then linked with \lib{libsnark}, but we find no documentation of this procedure. There are issues opened about this problem, but this library does not have online support. Overall, we solve this issue by reading the source code of the gadget in \texttt{gadget.hpp} and finding more useful gadgets like comparison and multiplex to implement the sample program.

\mypara{Recommendation} We recommend \lib{libsnark} for research or studying \zk. It has an explicit structure of circuits, providing systems and useful tools like gadgets. The subsequent libraries all have a similar structure. \lib{libsnark} is not suitable for implementing proofs for arbitrary applications because it lacks documents about linking an external R1CS. Additionally, we recommended that \lib{libsnark} includes a naming comparison table for \zk schemes, such as mapping \textit{Groth16} to \textit{r1cs\_gg\_ppzksnark}. Currently, this mapping is not documented but only explained in the comments of \textit{r1cs\_gg\_ppzksnark.hpp}, which may confuse users.

\subsection{bellman}
\lib{bellman} \cite{bellman} is a commercial library established in 2017 which implemented groth16~\cite{groth2016size}. \lib{bellman} provides circuit traits and primitive structures, as well as basic gadget implementations such as booleans and number abstractions. \lib{bellman} is currently in its infancy and can only be used to construct simple constraints with its low-level APIs. The goal of \lib{bellman} in 2017 is to pinpoint the future of zcash~\cite{sasson2014zerocash} by implementing the most efficient \zk scheme at that time. \lib{bellman} distinguishes its design from \lib{libsnark}'s gadgetlib as its all variable allocation, assignment, and constraint enforcement are done over the same code path which enables a concise and lightweight program. 

\mypara{Toolkits} 
\new{In the compiler's syntax}, \lib{bellman} provides basic circuit synthesis such as linear combination and multiplication in the elliptic curve and finite field arithmetic. The APIs in \lib{bellman} are low-level and do not offer features that would facilitate ease of use for most users.

\mypara{Documents and Support}
There are few documents about \lib{bellman} as its design intention is for research or proof-of-concept. The developers provide online support for installing but no more others. Concurrently, it is not a easy work to implement complex circuits in \lib{bellman}, but implementing simple circuits is straightforward. 

\mypara{Recommendation}
We recommend \lib{bellman} as a learning opportunity for constructing zk-SNARKs safely and efficiently. Besides, some application-level projects are recommended building on top of the \lib{bellman}.

\subsection{libSTARK}
\lib{libSTARK} \cite{libsTark} is an academic library established in 2018 and has implemented the STARK scheme in C++. Its features include scalability, transparency, and post-quantity security. \lib{libSTARK} provides two examples for testing, i.e., DNA fingerprint blacklist and tinyRAM programs. We managed to install \lib{libSTARK} and run its examples but found that \lib{libSTARK} has not provided any support of implementing our own programs.

\mypara{Toolkit} We delve into the source code to see the gadgets contained in \lib{libSTARK}. \new{In the compiler's syntax, \lib{libSTARK} supports NAND gate, field multiplication, and for-loops. In some of its test code repositories, gadget functions are provided.
	However, the gadgets in \lib{libSTARK} are high-level circuits such as SHA2 and there are no useful gadgets for a user to define her own circuits as the library is mainly for testing its own codes. We conclude that \lib{libSTARK} does not provide a compiler like eDSL as the support of circuit functions is limited.
}

\mypara{Documentation and Support} There are limited documentation on \lib{libSTARK} for user and developer. \lib{libSTARK} has many functionalities, but it is unclear how to use them. Concurrently, \lib{libSTARK} is not under development.

\mypara{Recommendation} We do not recommend \lib{libSTARK} for general use, both for study or real world application. It is because \lib{libSTARK} has not been updated for years and may have serious issues. However, for expert researchers or C++ programmers who want to implement complex circuits using basic gadgets, better to read \lib{libSTARK} source code carefully.

\subsection{dalek}
\lib{dalek}~\cite{dalek-bulletproofs} is a Rust library started in 2018 that aims to have the fastest range proofs with Bulletproofs~\cite{bunz2018bulletproofs}. \lib{Dalek-bulletproof} accelerates range proofs by choosing \texttt{curve25519-dalek} instead of \texttt{libsecp256k1} to enable parallel execution. The APIs provided by \lib{dalek} library are succinct and implementing a new range proof application is straightforward. \lib{dalek} only focuses on range proofs and inner product proofs and does not have R1CS constraints for general circuits.

\mypara{Toolkits} \lib{dalek} provides support for a single range proof, aggregating range proofs and inner product proofs. The range proof aims to demonstrate that an unsigned variable (32 or 64 bits in the \lib{dalek} library) is in a certain range. Aggregated range proofs allow multiple range proofs to be combined into a single proof, and aggregating optimizers can be used to shrink the proof size. The inner product proof aims to demonstrate that a vector $c$ is the inner product of $a$ and $b$, without revealing $b$. \lib{dalek} does not focus on proving constraints for high-level languages.

\mypara{Documents and Support} \lib{dalek}'s documents comprise two parts. The first is telling how range proof and inner product proof work theoretically, and the second is the user-facing documentation. We find it a good insight to combine theoretical formulas with practice in \lib{dalek}. However, its user-facing documents are not enough, and it has not provided examples for its three toolkits, except for a simple single range proof example. Moreover, the theoretical background contains only formulas without fine-grained reasonings or well-designed blogs.

\mypara{Recommendation} We recommend \lib{dalek} as a good start for understanding and practicing range proofs. Although \lib{dalek} is mainly for research purposes, it is convenient to create lightweight applications.

\subsection{libiop}
\lib{libiop}~\cite{libiop} is a C++ library created in 2019 that implemented the three latest IOP-based schemes at that time. 
A feature of \lib{libiop} is that it packs three different proving systems as alternatives for users, including aurora~\cite{ben2019aurora}, fractal~\cite{chiesa2020fractal} and ligero~\cite{ames2017ligero}.
However, \lib{libiop} is mainly for research purposes and has not provided a complete toolchain for creating an R1CS circuit. \new{We follow the library's API and test the efficiency of randomly generated circuits with roughly the same quantity of constraints as the sample programs.}

\mypara{Toolkits} To support three proving systems, \lib{libiop} uses a namespace \textit{iop} which contains \textit{aurora\_iop}, \textit{fractal\_iop}, and \textit{ligero\_iop} as specific protocols. The user-level APIs of the three schemes are the same, which makes them convenient to use. \new{In the compiler's syntax, \lib{libiop} only supports defining the circuit using \textit{generate\_random\_R1CS}. To define a specific circuit, one may have to manually write the constraints, computes the R1CS parameters for them and pass it to \lib{libiop}'s data structures which are time-consuming and not realistic. In our test, we generate R1CS using the output of \lib{libsnark}'s compiler and pass it to \lib{libiop} to make sure our experimental data in \lib{libiop} is based on real circuits.
}

\mypara{Documents and Support} There are few documents about the APIs of \lib{libiop}, and all test examples do not have comments for explanation. Most examples generate circuits randomly rather than implementing specific ones, which makes it challenging to develop practical applications and products. The developers provide online support, but the issues have not been solved for a long period of time.

\mypara{Recommendation} We recommend \lib{libiop} for curious researchers to study the implementation details of such three schemes.

\subsection{snarkjs}
\lib{snarkjs}~\cite{snarkjs} is a javascript library started in 2019 which implements Groth16~\cite{groth2016size}, Plonk~\cite{gabizon2019plonk}, and FFlonk~\cite{gabizon2021fflonk}. The goal of \lib{snarkjs} is to provide comprehensive ZK toolchains for website and blockchain scenarios. In zero-knowledge concepts, snarkjs provides a compiler called \texttt{circom}, and the syntax is similar to C or javascript. 

\mypara{Toolkits} \texttt{Circom} in \lib{snarkjs} is a powerful HDL compiler. It allows users to independently write their constraints in a file with a high-level language (e.g., DSL) that is similar to C, JavaScript and Verilog. When writing \texttt{circom} codes, the user is not required to use snarkjs APIs, which distinguishes \lib{snarkjs} from all other \zk libraries. \new{In the compiler's syntax, all variables are modeled as wire signals. The function in \texttt{circom} is defined as a subcircuit through the keywords \textit{template} and the output is defined by $component$. Some syntax in \texttt{circom} is counter-intuition at the beginning but one can define more complicated constraints once familiar.
}
When using \texttt{circom}, we successfully implement our sample programs and then compile them to R1CS for testing. Other \zk libraries, such as \lib{arkworks}~\cite{arkworks}, are also gradually starting to support \texttt{circom}.

\mypara{Documents and Support} \lib{snarkjs} provides a detailed document about its DSL and its proving systems. It also offers a \texttt{circom} library named circomlib~\cite{circomlib}, which implements many common cryptographic primitives. It has been updated recently and provides timely online support.

\mypara{Recommendation} We recommend using \lib{snarkjs} for proving large and complex computations, as defining a circuit in \texttt{circom} is as convenient as writing a C program. For research or study purposes, \lib{gnark}~\cite{gnark} and \lib{halo2}~\cite{halo2} are better because \lib{snarkjs} does not provide walk-through tutorials, and the basic examples are also not enough.

\subsection{gnark}
\lib{gnark}~\cite{gnark} is a high-performance, open-source Golang library for creating zero-knowledge proofs, particularly \zk applications originating from 2022. \lib{gnark} implements two schemes, Groth16~\cite{groth2016size} and Plonk~\cite{gabizon2019plonk}. One main feature is that \lib{gnark} provides a high-level language for specifying the proof's logic, and its APIs allow developers to easily create, verify, and deploy zero-knowledge proofs. It also includes a built-in compiler that transforms the high-level language into a low-level representation that can be run on various platforms. \lib{gnark} aims to be user-friendly and has various tutorials, including an executable playground for beginners to learn its programming style.

\mypara{Toolkits} \lib{gnark} provides two relevant classes, frontend and backend. The output of the frontend is a preprocessed circuit. With the circuit, the backend chooses a proving system, assigns a valid witness, and outputs a proof. In the frontend, \lib{gnark} provides a DSL specified in \textit{api} class, which is convenient to add constraints. \new{In its syntax, most of the data structures defined in Go can be operated by the \textit{api} interface and recorded in a circuit. With this flexibility, \lib{gnark} claims it has no need for gadgets, because the functions of circuit are implemented in a Go package like any other piece of code.}
Additionally, \lib{gnark} provides a set of pre-built circuit components, such as SHA256 and elliptic curve arithmetic, that can be used to build more complex circuits. In the backend, \lib{gnark} continues to optimize the performance of the two schemes, and we find efficiency improvement in our tests.

\mypara{Documents and Support} The documentation of \lib{gnark} ranges from user's documents to developer's documents. The documentation not only contains guidelines for installing, running, and writing sample codes but also the paradigm of designing \lib{gnark} and implementation details. However, there are some limitations. Firstly, the principle of selection elliptic curve is implicit. \lib{gnark} provides seven curves but has not documented how to choose an appropriate curve to avoid overflow. Secondly, the DSL syntax in \lib{gnark} is implicit. The only way to specify circuits in the DSL is through the \textit{frontend.API} class, but the usage of its internal functions remains unclear. Luckily, \lib{gnark} has full online support, and we find some solutions from its closed issues.

\mypara{Recommendation} We recommend \lib{gnark} for both study, research purposes and commercial use as it has relatively better support from the developers. Concurrently, \lib{gnark} has only implemented two academic schemes, and we believe there will be more in the future.

\subsection{arkworks}
\lib{arkworks}~\cite{arkworks} is a Rust ecosystem for \zk programming that started in 2022. It implements several latest academic \zk approaches including Groth16~\cite{groth2016size}, Plonk~\cite{gabizon2019plonk}, marlin~\cite{chiesa2020marlin}, gm17~\cite{groth2017snarky}, gemini~\cite{bootle2022gemini}, and Bulletproofs~\cite{bunz2018bulletproofs}. The schemes implemented in \lib{arkworks} span various categories and exhibit diverse properties, such as transparency, small proof size, URS, elastic proofs, and post-quantum security.

\mypara{Toolkits} \lib{arkworks} provides an explicit toolchain for compiling circuits and choosing proof systems. In the compiling phase, there are three predefined configures: finite field, elliptic curve, and polynomial, and we document our choice in our project. \new{In the compiler's syntax, \lib{arkworks} supports variable, array, function as data structures and operations like basic arithmetic, for-loop, while-loop and built-in gadgets like inner product proof and hashing.}
When choosing proof systems, \lib{arkworks} provides several sublibraries separating different categories of \zk approaches. 
It also provides a repository binding to \texttt{circom}'s R1CS, facilitating the generation of Groth16 Proof and Witness generation in Rust.

\mypara{Documentation and Support} There are redundant documents, tutorials, and blogs about the algebra, constraint systems, R1CS gadgets, and proof systems of Groth16~\cite{groth2016size}. Developers also continue to solve the issues in each sub-libraries. However, there is still a lack of tutorials and examples for other \zk schemes, such as marlin~\cite{chiesa2020marlin} and gm17~\cite{groth2017snarky}.

\mypara{Recommendation} We recommend \lib{arkworks} for general use. Both for beginners, researchers, and industries as \lib{arkworks} has the best ecosystem in \zk industry. We also recommend \lib{arkworks} to design more circom-compatible for different proof systems. 

\subsection{halo2}
\lib{halo2}~\cite{halo2} is a Rust library developed by Electric Coin Company (\text{ECC}). \lib{halo2} introduces new features and improvements based on Plonk~\cite{gabizon2019plonk}, including recursive proofs and parallel computation. Due to the long review cycle in submission, \lib{halo2} has not been published yet, but a detailed online book is provided to demonstrate its design~\cite{halo2book}.

\mypara{Toolkits} \lib{halo2} provides many available embedded functions for building a circuit. \new{In the compiler's syntax, }it implements redundant gate-level constraints, including addition, multiplication, array, sum, etc. There are also high-level constraints like the inner product and range check (with lookup tables). For popular zero-knowledge applications like hash and signature, \lib{halo2} provides an integrated API as a part of its tools.

\mypara{Documents and Support} All user-level data types and functions in \lib{halo2} are well documented with many executable examples. \lib{halo2} also provides a step-by-step tutorial walk-through and all preliminaries, including Rust tutorials for beginners. The \lib{halo2} handbook is a developer document providing design details in theory and its corresponding source code. The developers also provide online support through email and GitHub issues.

\mypara{Recommendation} 
\lib{halo2} is concurrently a mature, well-documented, and efficient library used in zcash~\cite{sasson2014zerocash}. We recommend \lib{halo2} for both study and commercial purposes. One drawback of \lib{halo2} is that it contains only one proving system, and for users, there are no alternatives.

\subsection{Spartan}
\lib{Spartan}~\cite{spartan} is a Rust library from Microsoft originating in 2020 that aims to implement transparent {\zk}s with a fast prover. \lib{Spartan} supports generating R1CS randomly for testing and also provides a compiling toolchain from high-level programs of interest. 

\mypara{Toolkits} \lib{Spartan} is claimed to be general, high performance, and with a complete toolchain of compiling high-level programs. \new{In the compiler's syntax, the data structures and operations are similar to \lib{halo2} \cite{halo2} as they both utilize features of Rust. Compared to \lib{halo2}, \lib{Spartan} provides less gadget functions as it mainly focuses on optimizing the efficiency of proving systems for R1CS.}
We test its efficiency by a built-in random circuit generator. However, there are no tutorials about its related language syntax and how to use the compiler. The default example defining a simple circuit is very complex compared to other libraries.

\mypara{Documents and Support} There is little documentation about the programming pattern of \lib{Spartan}. The inputs and outputs of prover transcripts remain unclear, and we find many users struggle to compile R1CS for their own applications. A popular hash application in its documentation, however, is not provided as an example. 

\mypara{Recommendation} \lib{Spartan} achieves high performance but is not capable of supporting commercial use since this library has not received a security review or audit. We recommend users focus on the syntax of \lib{Spartan}'s DSL, which wish to compile a R1CS on their own. 

\subsection{plonky2}
\lib{plonky2}~\cite{plonky2} is a Rust library that started in 2023, where the Polygon team implemented several optimizations for the Plonk scheme. The goal of \lib{plonky2} is to scale zero-knowledge proof in Ethereum layer 2 rollup by providing fast, efficient, and secure proving systems. The main feature of Plonky is that it utilizes a recursive proof technique. Recursive proof is a space-saving technique, and it allows the prover to demonstrate many statements at once without increasing the proof size, which is indeed the requirement in zk-rollup. 

\mypara{Toolkits} \new{In the compiler's syntax, \lib{plonky2} contains the data structures and operations similar to \lib{halo2} \cite{halo2} and \lib{Spartan} \cite{spartan} and we find the same programming pattern when defining circuits using the Rust-based eDSLs.} To utilize the recursive feature, \lib{plonky2} defines an extra class \textit{cyclic\_circuit} for the proof and verification.

\mypara{Documents and Support} \lib{plonky2} is a commercial project for Ethereum developers and is still in development. The documents now only elaborate on its features but without API details. The only way for independent developers to compile \lib{plonky2} is to walk through online examples. However, the recursive proof example is incomplete and cannot be extended to generic circuits.

\mypara{Recommendation} We only recommend \lib{plonky2} for developing Ethereum ZK rollup applications as, at this time, the only way to use Plonky API is by reading its source code.





%\section{Introduction}
\label{sec::intro}

Embodied Question Answering (EQA) \cite{das2018embodied} represents a challenging task at the intersection of natural language processing, computer vision, and robotics, where an embodied agent (e.g., a UAV) must actively explore its environment to answer questions posed in natural language. While most existing research has concentrated on indoor EQA tasks \cite{gao2023room, pena2023visual}, such as exploring and answering questions within confined spaces like homes or offices \cite{liu2024aligning}, relatively little attention has been dedicated to EQA tasks in  open-ended city space. Nevertheless, extending EQA to city space is crucial for numerous real-world applications, including autonomous systems \cite{kalinowska2023embodied}, urban region profiling \cite{yan2024urbanclip}, and city planning \cite{gao2024embodiedcity}. 
% 1. 环境复杂性   
%    - 地标重复性问题(如区分相似建筑)  
%    - 动态干扰因素(交通流、行人)  
% 2. 行动复杂性  
%    - 长程导航路径规划  
%    - 移动控制、角度等  
% 3. 感知复杂性  
%    - 复合空间关系推理("A楼东侧商铺西边的车辆")  
%    - 时序依赖的观察结果整合

EQA tasks in city space (referred to as CityEQA) introduce a unique set of challenges that fundamentally differ from those encountered in indoor environments. Compared to indoor EQA, CityEQA faces three main challenges: 

1) \textbf{Environmental complexity with ambiguous objects}: 
Urban environments are inherently more complex,  featuring a diverse range of objects and structures, many of which are visually similar and difficult to distinguish without detailed semantic information (e.g., buildings, roads, and vehicles). This complexity makes it challenging to construct task instructions and specify the desired information accurately, as shown in Figure \ref{fig:example}. 

2) \textbf{Action complexity in cross-scale space}: 
The vast geographical scale of city space compels agents to adopt larger movement amplitudes to enhance exploration efficiency. However, it might risk overlooking detailed information within the scene. Therefore, agents require cross-scale action adjustment capabilities to effectively balance long-distance path planning with fine-grained movement and angular control.

3) \textbf{Perception complexity with observation dynamics}: 
% Rich semantic information in urban settings leads to varying observations depending on distance and orientation, which can impact the accuracy of answer generation. 
Observations can vary greatly depending on distance, orientation, and perspective. For example, an object may look completely different up close than it does from afar or from different angles. These differences pose challenges for consistency and can affect the accuracy of answer generation, as embodied agents must adapt to the dynamic and complex nature of urban environments.


\begin{table}
\centering
\caption{CityEQA-EC vs existing benchmarks.}
\label{table:dataset}
\renewcommand\arraystretch{1.2}
\resizebox{\linewidth}{!}{
\begin{tabular}{cccccc}
             & Place  & Open Vocab & Active & Platform  & Reference \\ \hline
EQA-v1      & Indoor & \textcolor{red}{\ding{55}}          & \textcolor{green}{\ding{51}}      & House3D      & \cite{das2018embodied}  \\
IQUAD        & Indoor & \textcolor{red}{\ding{55}}          & \textcolor{green}{\ding{51}}      & AI2-THOR     & \cite{gordon2018iqa} \\
MP3D-EQA     & Indoor & \textcolor{red}{\ding{55}}          & \textcolor{green}{\ding{51}}      & Matterport3D & \cite{wijmans2019embodied} \\
MT-EQA       & Indoor & \textcolor{red}{\ding{55}}          & \textcolor{green}{\ding{51}}      & House3D      & \cite{yu2019multi} \\
ScanQA       & Indoor & \textcolor{red}{\ding{55}}          & \textcolor{red}{\ding{55}}      & -            & \cite{azuma2022scanqa} \\
SQA3D        & Indoor & \textcolor{red}{\ding{55}}          & \textcolor{red}{\ding{55}}      & -            & \cite{masqa3d} \\
K-EQA        & Indoor & \textcolor{green}{\ding{51}}          & \textcolor{green}{\ding{51}}      & AI2-THOR     & \cite{tan2023knowledge} \\
OpenEQA      & Indoor & \textcolor{green}{\ding{51}}          & \textcolor{green}{\ding{51}}      & ScanNet/HM3D & \cite{majumdar2024openeqa} \\
 \hline
CityEQA-EC   & City (Outdoor)  & \textcolor{green}{\ding{51}}          & \textcolor{green}{\ding{51}}      & EmbodiedCity & - \\ \hline
\end{tabular}}
\end{table}

\begin{figure*}[!htb]
\centering
    \includegraphics[width=0.78\linewidth]{figures/example.pdf}
% \vspace{-0.2cm}
\caption{The typical workflow of the PMA to address City EQA tasks. There are two cars in this area, thus a valid question must contain landmarks and spatial relationships to specify a car. Given the task, PMA will sequentially complete multiple sub-tasks to find the answer.}
% \vspace{-0.2cm}
\label{fig:example}
\end{figure*}

As an initial step toward CityEQA, we developed \textbf{CityEQA-EC}, a benchmark dataset to evaluate embodied agents' performance on CityEQA tasks. The distinctions between this dataset and other EQA benchmarks are summarized in Table \ref{table:dataset}. CityEQA-EC comprises six task types characterized by open-vocabulary questions. These tasks utilize urban landmarks and spatial relationships to delineate the expected answer, adhering to human conventions while addressing object ambiguity. This design introduces significant complexity, turning CityEQA into long-horizon tasks that require embodied agents to identify and use landmarks, explore urban environments effectively, and refine observation to generate high-quality answers.

To address CityEQA tasks, we introduce the \textbf{Planner-Manager-Actor (PMA)}, a novel baseline agent powered by large models, designed to emulate human-like rationale for solving long-horizon tasks in urban environments, as illustrated in Figure \ref{fig:example}. PMA employs a hierarchical framework to generate actions and derive answers. The Planner module parses tasks and creates plans consisting of three sub-task types: navigation, exploration, and collection. The Manager oversees the execution of these plans while maintaining a global object-centric cognitive map \cite{deng2024opengraph}. This 2D grid-based representation enables precise object identification (retrieval) and efficient management of long-term landmark information. The Actor generates specific actions based on the Manager's instructions through its components: Navigator, Explorer, and Collector. Notably, the Collector integrates a Multi-Modal Large Language Model (MM-LLM) as its Vision-Language-Action (VLA) module to refine observations and generate high-quality answers.
PMA's performance is assessed against four baselines, including humans. 
Results show that humans perform best in CityEQA, while PMA achieves 60.73\% of human accuracy in answering questions, highlighting both the challenge and validity of the proposed benchmarks. 

% The Frontier-Based Exploration (FBE) Agent, widely used in indoor EQA tasks, performs worse than even a blind LLM. This underscores the importance of PMA's hierarchical framework and its use of landmarks and spatial relationships for tackling CityEQA tasks.

In summary, this paper makes the following significant contributions:
\vspace{-8pt}
\begin{itemize}[leftmargin=*]
    \item To the best of our knowledge, we present the first open-ended embodied question answering benchmark for city space, namely CityEQA-EC.
    \vspace{-7pt}
    \item We propose a novel baseline model, PMA, which is capable of solving long-horizon tasks for CityEQA tasks with a human-like rationale.
     \vspace{-7pt}
    \item Experimental results demonstrate that our approach outperforms existing baselines in tackling the CityEQA task. However, the gap with human performance highlights opportunities for future research to improve visual thinking and reasoning in embodied intelligence for city spaces.
\end{itemize}




%\section{Background}
Integrating LLMs into real-world systems demands a robust and interconnected technical stack, driving the creation of a diverse ecosystem of tools and frameworks to support their lifecycle. This section provides an overview of the LLM lifecycle and technical stack, highlighting their complexity and the associated security challenges.

% \subsection{LLM Lifecycle and Tech Stack}
\noindent \textbf{LLM Lifecycle and Tech Stack.} The lifecycle of LLMs involves multiple interconnected stages, each supported by specialized tools and frameworks, forming a complex and comprehensive technical stack. These stages include data collection and preprocessing, model training, optimization, deployment, and post-deployment monitoring. Each stage is crucial for the integration of LLMs into real-world systems, and together they ensure the models are effective and scalable. However, as the stack is highly interconnected, vulnerabilities introduced at any stage—whether during data handling, model training, or deployment—can compromise the overall integrity and performance of the LLM system. \autoref{fig:stack} illustrates the general architecture of the LLM tech stack, which consists of three primary layers:
\begin{figure}[t]
    \centering
    \includegraphics[width=0.95\linewidth]{Figures/stack.pdf}
    \caption{LLM Lifecycle and Tech Stack.}
    \label{fig:stack}
\end{figure}

\noindent \textbf{[A] Data Layer.} The data layer serves as the foundation of the LLM lifecycle, responsible for managing the collection, transformation, storage, and retrieval of large datasets. This layer handles the initial steps of the data pipeline, beginning with transforming raw data into vector representations using embedding models. Tools like SentenceTransformers~\cite{sentence-transformers} are employed to create high-quality embeddings that convert textual data into vector formats suitable for downstream processes. The embedded data is then indexed and stored in systems that facilitate efficient and scalable retrieval, such as vector databases like FAISS~\cite{faiss} and Qdrant~\cite{qdrant}, which facilitates rapid access to relevant data for tasks such as Retrieval-Augmented Generation (RAG) or LLM caching. 
    
\noindent \textbf{[B] Model Layer.} The model layer is essential for the core development, optimization, and deployment of LLMs, providing the necessary tools and frameworks to enhance model performance. Frameworks like Hugging Face’s Transformers~\cite{transformers} facilitate the implementation and fine-tuning of pre-trained models. Supporting techniques such as model quantization and model merging help optimize the model’s size and computational efficiency. LLM operations (LLMOps), such as lunary~\cite{lunary}, are also integrated into this layer, enabling continuous monitoring and refinement of the model's performance throughout its lifecycle, from the initial development phase to deployment.
Once the model is prepared, it is served and utilized through model serving and inference processes. Frameworks such as Triton Inference Server~\cite{triton-inference-server} or Ollama~\cite{ollama} provide the necessary infrastructure to deploy models into production environments, enabling real-time predictions via API endpoints. The inference process then utilizes these models to generate outputs for various tasks, such as text generation or question answering, based on user queries or system requests.
    
\noindent \textbf{[C] Application Layer.} The application layer is responsible for connecting trained LLMs to real-world systems and users, enabling seamless integration and deployment. This layer focuses primarily on orchestration frameworks that automate workflows and manage the interactions between different components. Orchestration tools like LangChain~\cite{langchain} and AutoGPT~\cite{AutoGPT} enable autonomous decision-making and process automation by chaining LLM calls together.
Supporting tools are essential for extending the LLM’s capabilities. For example, LiteLLM~\cite{litellm} acts as an LLM gateway, serving as a proxy that provides a unified interface for calling multiple models in a consistent format. GPTCache~\cite{gptcache} provides caching services to optimize performance and reduce latency, ensuring faster responses during inference. Tools like Haystack~\cite{haystack} support retrieval-augmented generation (RAG), enhancing the LLM's ability to respond to complex queries by retrieving relevant information from external data sources. Additionally, function-calling frameworks like Composio~\cite{composio} can be integrated to enhance agent capabilities, allowing for dynamic interactions with external APIs and systems.
As many LLM systems interact directly with users, front-end frameworks are also a critical part of this layer. Platforms like Anything-LLM~\cite{anythingllm} and LocalAI~\cite{localai} provide interfaces for users to interact with LLMs, enabling easy access to LLM functionalities through user-friendly interfaces. 

% \subsection{LLM Infrastructure Vulnerabilities}
% As the adoption of LLMs continues to grow, the complexity and interconnected nature of their supporting infrastructure have significantly expanded the potential attack surface for adversaries. The LLM ecosystem consists of various components, including third-party libraries, deployment platforms, and orchestration frameworks, all of which introduce unique vulnerabilities. These vulnerabilities, if exploited, can undermine the security of the entire system—affecting aspects such as data integrity, model performance, and user privacy.

% Each layer of the LLM tech stack introduces its own set of challenges. For example, in the Data Layer, vulnerabilities related to data preprocessing, such as improper data sanitization or malicious data injection, can lead to poisoned training data or biased embeddings that degrade model performance or generate biased outputs. Moreover, vulnerabilities in data indexing and retrieval processes, especially in vector databases like FAISS~\cite{faiss} and Qdrant~\cite{qdrant}, can expose sensitive information or enable unauthorized access to stored data~\cite{vecdbvul}, impacting the overall integrity of the LLM system.

% In the Model Layer, vulnerabilities often arise from the tools and techniques used for model development and optimization. Issues such as insecure model merging, unsafe handling of training data, or flaws in quantization processes could lead to compromised models, performance degradation, or adversarial attacks. LLM operations (LLMOps) integrated into this layer—such as tools for monitoring and adjusting model performance—may also introduce risks if improperly configured or managed, enabling potential misuse or unintentional bias during the deployment phase.

% Finally, in the Application Layer, security risks often stem from the orchestration and deployment frameworks that facilitate communication between the LLM and external systems. Vulnerabilities in API endpoints, poor authentication mechanisms, or improper access control can expose the system to attacks such as unauthorized access, denial-of-service attacks, or even the injection of malicious inputs. Frameworks like LangChain~\cite{langchain}, AutoGPT~\cite{AutoGPT}, and LiteLLM~\cite{litellm}, while providing powerful capabilities for automation and model integration, can be vulnerable if not properly secured. Additionally, front-end frameworks like Anything-LLM~\cite{anythingllm} and LocalAI~\cite{localai} that facilitate user interactions also pose risks if they fail to implement adequate input validation and authentication.

% The increasingly complex LLM ecosystem highlights the need for comprehensive security practices at every stage of development and deployment. Vulnerabilities introduced at any layer can cascade and result in substantial security breaches, affecting the performance, reliability, and safety of LLM systems. 
%\section{Overview of {\tool}}
% \yang{TO BE DONE AS DISCUSSED}

% \yang{\textit{(COMMENT).}}
%这块可以补1-2为啥通过历史样本的反馈强化,能够实现更有效信息窃取的core idea。比如在工具集合范围内,工具调用通常具有一定的规律或者特征,调用当前工具的时候通常都是有一定的前置。能够通过历史信息大概推测出调用当前工具的时候,前面大模型和其他工具完成过什么任务,交互过什么信息。比如调用一个订酒店tool的时候,前面大概调用过定飞机/火车(可能有更好的例子)。因此通过历史样本的学习,可以让你的攻击方法具有一定的泛化。大概花个1-2句话,让人先从直觉上get到你方法起效的原理。紧接着再说你要处理历史case,然后做强化。
%然后两个关键阶段,攻击图谱和强化,在这里写的时候,在重点明确一下他们的目标和产出是什么,用哪些样本评估、优化哪些(比如最后一句话),都可以不要。这一段让大家先从高层有一个目标认知。比如,指示图刻画了什么内容,为什么需要处理成KG,KG对于后面强化的意义。强化那部分也类似哈。现在读起来有点没法从high-level把握住你两步的目标是什么。
{Within a toolset, the invocation chains often exhibit certain patterns and regularities when processing different user queries. 
When invoking a specific tool, there are usually certain prerequisites or preconditions. }
For example, a tool for hotel reservation in LLM inference may be invoked simultaneously with the other tool for booking a flight/train ticket in the previous.
{
In such cases, it is generally possible to infer what tasks the upstream tools have completed in previous steps, as well as what information has been exchanged upstream, by learning from historical toolchains.
}
% In LLM tool learning systems, the invoked tools have some sequentially based logical relations in historical cases.
% \yang{
% {\tool} aims to achieve more effective information exfiltration by learning from examples generated by open-source tool learning systems. Based on the tool invoked at the current step, we infer the information keys of the upstream tools and generate more explicit instructions.
% }
% Learning from these cases, adversaries can design more reasonable attack commands for specific user queries and inference tasks.

{
Figure \ref{fig:model_tool_learning} shows the overview of {\tool}.
Guided by the concept, {\tool} first constructs AttackDB with attack cases that provide examples with key information to guide the generation of black-box commands.
After that, {\tool} incorporates AttackDB to train an initial command generation model, then reinforces it guided by the reward combined with attack results and the sentiment score of the generated command. 
}
% After that, {\tool} proposes the dynamic command generator and incorporates AttackDB to generate commands in the black-box scenario, then utilizes RL to continuously optimize this generator based on attack results.
% {In this section, we illustrate the framework of {\tool} in Figure \ref{fig:model_tool_learning}.
% Specifically, {\tool} first forms AttackDB with attack cases that provide examples with key information to guide the generation of black-box commands.
% Second, {\tool} proposes the dynamic command generator and incorporates AttackDB to generate commands in the black-box scenario, then utilizes RL to continuously optimize this generator based on attack results.
% In practice, the optimized model will utilize only the malicious tool's information and key information in AttackDB to generate effective info-theft commands.
% After these two steps, we apply the optimized dynamic command generator to the other black-box LLM tool learning systems to evaluate its practicability. 



% scenario=<vict/att, malicious tool>
% case=<vict, attack cases, malicious tool>
% caseDB=case1,case2,...,casen



\subsection{Attack-Case Database Preparation}
% To prepare the attack cases, we .
% In historical cases, we manually label whether the information theft attack is successfully achieved based on the specific commands.
Given inference examples $[E^A_1,E^A_2,...,E^A_n]$ that are used to generate attack cases, where $E^A_i$ is a white-box example with frontend inference and backend toolchain, we use white-box {Attack Case Generator} and {Attack Case's Guidance Completer} to prepare attack cases and form the AttackDB.


\paragraph{The Definition of Attack Cases.}
The attack case is a five-tuple array, which can be formalized as $\langle\mathcal{T}_{vict}^{A},\mathcal{T}_{att}^{A},\mathcal{C}^{A},\mathcal{R}^{A}\,\mathcal{G}^{A}\rangle$:
(1) $\mathcal{T}_{vict}^{A}$ and $\mathcal{T}_{att}^{A}$ are the victim and malicious tool's details and its relevant information, i.e., \textit{Tool's Name}, \textit{Description}, \textit{Function Code}, and \textit{Relevant Information to Attack}.
(2) $\mathcal{C}^{A}$ is the details of commands $\mathcal{C}$ that are used to steal the information.
% We regulate components that each command needs to contain as $\text{[\textit{ToolRecall}][\textit{Attack}][\textit{NotExpose}]}$,
% where [\textit{ToolRecall}] is the command for calling this tool again; [\textit{Attack}] indicates the task of information theft, and [\textit{NotExpose}] asks the LLM to hide the attack in the frontend.
(3) $\mathcal{R}^{A}$ is the result of whether the attack is successful and has stealthiness.
(4) $\mathcal{G}^{A}$ is the guidance that summarizes the current commands and attack results and finds the key information between the tools that may affect the attack success rate.
As is shown in Figure \ref{fig:example_db}, the key information in $\langle\mathcal{T}_2,\mathcal{T}_3\rangle$ indicates the commonalities between the tool's input value, and using some specific tasks such as "registration" can improve the success and stealthiness of this attack. 
We have illustrated more details in Appendix \ref{sec:gen_kg}.

\begin{figure}[t]
\centering
\includegraphics[width=\columnwidth]{Fig/attack_case_db_example_v3.pdf}
\vspace{-0.6cm}
\caption{Example of attack cases.}
\label{fig:example_db}
\vspace{-0.6cm}
\end{figure}

% $\mathcal{G}=\{E_{att}, R_{att}\}$ are tool entities and relations with their attributes:
% \textit{\textbf{(1) Tool Entity $E_{att}$:}} The tool's details and its relevant information, i.e., \textit{Tool's Name}, \textit{Tool's Description}, \textit{Function Code}, and \textit{Relevant Information to Attack}.
% \textit{\textbf{(2) Relation $R_{att}$:}} The direction of the information theft, which starts from $\mathcal{T}_{vict}$ and ends at $\mathcal{T}_{att}$. 

% To better describe the attack details, we define three relation attributes as follows:
% \textit{\textbf{(1) Injected Command:}} The details of commands $\mathcal{C}$ that are used to steal the information.
% We regulate components that each command needs to contain as $\text{[\textit{ToolRecall}][\textit{Attack}][\textit{NotExpose}]}$,
% where [\textit{ToolRecall}] is the command for calling this tool again; [\textit{Attack}] indicates the task of information theft, and [\textit{NotExpose}] asks the LLM to hide the attack in the frontend.
% % and manually collect these commands in Section \ref{sec:data_preparation}.
% \textit{\textbf{(2) Attack Result:}} The results of whether the attack is successful and has stealthiness.
% \textit{\textbf{(3) Background Command Prompt:}} The target prompt that can be used to generate the injected commands. 
% We have illustrated more details of AttackKG in Section \ref{sec:gen_kg}.   


% \begin{equation}
% \begin{split}
%     &(HCase_i, \mathcal{T}^{A}_{att})\stackrel{LLM}{\longrightarrow} [\mathcal{C}^{A}_1,\mathcal{C}^{A}_2,...,\mathcal{C}^{A}_n]\\
%     &\mathcal{T}^{A}_{vict}\stackrel{\mathcal{O}^{A}_{att}\oplus\mathcal{C}^{A}}{\longrightarrow}[Res^{A}_1,Res^{A}_2...,Res^{A}_n]\\
%     &[Res^{A}_1,Res^{A}_2,...,Res^{A}_n]\stackrel{Update}{\longrightarrow}\mathcal{G}
% \end{split}
% \end{equation}

\paragraph{Attack Case Extractor.}
Given the historical case $H$ with the tool calling chain $\mathcal{T}_{1},\mathcal{T}_2,...,\mathcal{T}_N$, we construct ${N\times(N-1)}/2$ tool pairs $\langle\mathcal{T}_i,\mathcal{T}_j\rangle$.
Then, we treat $\mathcal{T}_j$ as $\mathcal{T}^{A}_{att}$, and $\mathcal{T}_i$ as $\mathcal{T}^{A}_{vict}$, then ask the GPT-4o to explore $K$ commands for each pair. We manually test each command and use the attack results to update the attack cases as follows:
\begin{equation}
\resizebox{.89\linewidth}{!}{$
    \displaystyle
\begin{rcases}
% \begin{split}
    &\langle\mathcal{T}^{A}_{vict}, \mathcal{T}^{A}_{att}\rangle\stackrel{LLM}{\longrightarrow} [\mathcal{C}^{A}_1,...,\mathcal{C}^{A}_K]\\
&\mathcal{T}^{A}_{vict}\stackrel{\mathcal{O}^{A}_{att}\oplus\mathcal{C}^{A}}{\longrightarrow}[\mathcal{R}^{A}_1,...,\mathcal{R}^{A}_K]\\
    % &[\mathcal{R}^{A}_1,\mathcal{R}^{A}_2...,\mathcal{R}^{A}_n]\stackrel{Update}{\longrightarrow}AttackDB
% \end{split}
\end{rcases}
\stackrel{Form}{\longrightarrow}AttackCase
$}
\end{equation}
where $[\mathcal{C}^{A}_1,\mathcal{C}^{A}_2,...,\mathcal{C}^{A}_K]$ are the generated commands of LLM.
%上面那个format应该已经删了吧,更新一下
Then, we manually inject these explored commands into the target $\mathcal{T}^{A}_{att}$ and observe $K$ attack results as $[\mathcal{R}^{A}_1,\mathcal{R}^{A}_2,...,\mathcal{R}^{A}_K]$.
Then, we utilize all the previous results to form the attack cases and update the $AttackDB$.
% \begin{center}
% \footnotesize
% \begin{tcolorbox}[colback=white,%gray background
%                   colframe=black,% black frame colour
%                   width=\columnwidth,% Use 8cm total width,
%                   arc=1mm, auto outer arc,
%                   boxrule=0.4pt,
%                   left=0.1pt,
%                   right=0.1pt,
%                   top=0.1pt,
%                   bottom=0.1pt,
%                 colbacktitle=white!80!gray, coltitle=black, %标题框的背景和线条颜色 
%                 title={{\textbf{Prompt for exploring the attack commands.}}}%标题
%                  ]
% {\textbf{Instruction:} We define the AttackKG with the following knowledge graph structure}

% \begin{itemize}[leftmargin=*]
%     \item {\textbf{AttackKG:} Introducing the }
% \end{itemize}
% \end{tcolorbox}
% \vspace{-0.2cm}
% \end{center}

\paragraph{Attack Case's Guidance Completer.}
With the generated commands and attack results, we introduce another GPT-4o model to output the guidance for the subsequent dynamic command generator.
This guidance includes the \textbf{key information} that GPT-4o observes between tools, and how to design a command that may have higher attack success rates, as the following equation:
\begin{equation}
\resizebox{.89\linewidth}{!}{$
    \displaystyle
    \langle\langle\mathcal{T}^{A}_{vict}, \mathcal{T}^{A}_{att},\mathcal{C}^{A},\mathcal{R}^{A}\rangle\stackrel{LLM}{\longrightarrow}\mathcal{G}^{A}
    \rangle\stackrel{Form}{\longrightarrow}AttackCase
$}
\end{equation}
where the guidance is mutated from the basic template, e.g., "\textit{The generated commands that may have the [ToolRecall][Attack][NotExpose] format, and will focus on the key information between tools}". We form the cases with all five tuples and insert this case to AttackDB: $AttackCases\rightarrow AttackDB$, which are references to guide the optimization of the dynamic command generator.

\subsection{RL-based Dynamic Command Generation}
Given inference examples $[E^O_1,E^O_2,..., E^O_m]$ that are used for model optimization, 
each tool can only access its relevant information and does not know the other invoked tools. 
% where we simulate the adversary's exploration steps for achieving the information theft and use the reward feedback to optimize this model dynamically.
We first incorporate the AttackDB to initialize the command generator.
Then, we randomly select one malicious tool $\mathcal{T}^{O}_{att}$ in $E^O_i$'s toolchain and generate the injected command.
Finally, we conduct the information theft attack with the command 
and calculate the rewards to optimize the AttackDB\&model with black-box attack cases.

\paragraph{Dynamic Command Generator.}
The dynamic command generator $f_{gen}$ is a model that simulates the adversary's learning ability (e.g., T5~\cite{DBLP:journals/jmlr/RaffelSRLNMZLL20}), which can be fine-tuned based on the current knowledge and the results of the observed attack results.
In the black-box attacks, the adversary can only access the $\mathcal{T}^{O}_{att}$'s relevant information, so we generate the command $\mathcal{C}_i^{O}$ as follows:
\begin{equation}
    P_{gen}(\mathcal{C}^{O}|Case,\mathcal{T}^{O}_{att})=f_{gen}(\hat{{Case}}\oplus\hat{\mathcal{T}}^{O}_{att})
\end{equation}
where $\hat{Case}$ is the \textbf{textual description} of the retrieved attack cases in the AttackDB with similar types of input/output values in the $\mathcal{T}^{A}_{att}$,
% We use the entity/relation lists and the adjacent matrix to describe the details of the AttackKG.
and $\hat{\mathcal{T}}^{O}_{att}$ is the text description of the current malicious tool.
We generate the command $\mathcal{C}^{O}$ with its probability $P(\mathcal{C}^{O})$, and inject it into the target tool-learning system and obtain the attack results: $(\hat{\mathcal{I}/\mathcal{O}}^{O}_{vict},\hat{Inf})$,
% \begin{equation}
% \textit{TarSys}\stackrel{\mathcal{O}^{O}_{att}\oplus\mathcal{C}^{O}}{\longrightarrow}(\hat{\mathcal{I}/\mathcal{O}}^{O}_{vict},\hat{Inf})
% \end{equation}
where $\hat{\mathcal{I}/\mathcal{O}}^{O}_{vict}$ and $\hat{Inf}$ are theft results and inference after the attack.
 % and $r_{Theft}$ indicates the rewards that are calculated based on the results of the information theft attack.



\paragraph{Command Sentiment Reviewer.}
Our manual analysis of the command's sentiment polarity shows that commands with neutral sentiments are likely to be executed by LLMs.
We calculate the absolute sentiment score $|S_{sent}|$ with NLTK tool~\cite{DBLP:conf/acl/Bird06} as the reward penalty, which indicates that if the command sentiment tends to be positive or negative, the reward will be lower.
% \begin{equation}
%     r(E_i)=r_{Theft}(E_i)-|S_{sent}|
% \end{equation}
% where $r(E_i)$ is the final reward for the model optimization. 
% The equation indicates that if the command sentiment tends to be positive or negative, the reward will be lower.

\paragraph{RL-Based Model Optimization.}
Based on the thought of RL, the command generator $f_{gen}$ is a policy that determines what the adversaries will do to maximize the rewards, so we choose the PPO reward model~\cite{DBLP:journals/corr/SchulmanWDRK17} to calculate two rewards, i.e., the theft ($r_t$) and exposed ($r_e$) reward, which obtains the State-of-the-Art (SOTA) performance in our task's optimization. The total reward can be calculated as follows:
\begin{equation}
\resizebox{.89\linewidth}{!}{$
    \displaystyle
r(E^O_i)=\underbrace{\sigma(\hat{\mathcal{I}/\mathcal{O}}^{O}_{vict},{\mathcal{I}/\mathcal{O}}^{O}_{vict})}_{r_t}+\underbrace{\sigma(\hat{Inf},Inf)}_{r_e}-|S_{sent}|
$}
\end{equation}
where $r(E_i)$ is the final reward for the model optimization, and function $\sigma(\hat{y},y)$ is the reward model, which is calculated based on the attack results.


To dynamically optimize the {\tool}, we update AttackBD by creating an attack case with $\mathcal{T}^{O}_{att}$'s attack results. 
Since the attack is black-box, adversaries cannot access the victim tool, so we create a new tool with the stolen information. 
The new knowledge can guide adversaries to design harmful commands in black-box attack scenarios.

\begin{algorithm}[t]
\small
	%\SetAlgoNoLine %控制有无竖线
	\caption{The online RL Optimization.} 
 \label{alg:rl_optimization}
	\KwIn{The command generator $f_{gen}$ and optimization examples $[E^O_1,...E^O_m]$.} 
	\KwOut{The optimized command generator $f^{'}_{gen}$.}
    \SetKwProg{Fn}{Function}{}{end}
    Initialize $Batch\_Size\rightarrow B$, $t=0$\;
    \While {$t\leq m$}{
        $\mathcal{D}_{t}=[ECase_{B\times(t-1)+1},...,ECase_{B\times t}]$\;
        Calculate policy loss at timestamp $t$: $\mathcal{L}^{t}_{gen}(\theta)=\text{Reinforce}(\mathcal{D}_t;\theta)$ with Equation \ref{equa:loss}\;\label{algo_line:loss_calculation}
        Optimize {\tool} with the policy gradient $\nabla_{\theta}\mathcal{L}^{t}_{gen}(\theta)$, $f_{gen}\stackrel{\nabla_{\theta}}{\longrightarrow}f^{'}_{gen}$\;\label{algo_line:optimize}
        $t=t+1$\;
    }
    return $f^{'}_{gen}$\;
    % \SetKwProg{Fn}{Function}{}{}
    % \Fn{AAA}{123} \tcc*[r]{AAA}
\end{algorithm}
% \vspace{-0.7cm}



Then, we use the rewards to estimate the policy losses and gradient.
We introduce Reinforce Loss~\cite{DBLP:journals/ml/Williams92}, the novel approach to bridge the gaps between rewards and the command generation probabilities. The loss is calculated as: 
\begin{equation}\label{equa:loss}
\resizebox{.89\linewidth}{!}{$
    \displaystyle    \mathcal{L}_{gen}=\mathbb{E}_{[\mathcal{C}^{O}_{1:m}]\sim f_{gen}}[-\eta\log P_{gen}(\mathcal{C}^{O}|\mathcal{G},\mathcal{T}^{O}_{att})\cdot r(E_i)]
$}
\end{equation}
where the $\mathcal{L}_{gen}$ is the loss for optimizing the {\tool}.
In practice, we introduce the thought of Online Learning~\cite{DBLP:conf/nips/BriegelT99} to optimize the model, as is shown in Algorithm \ref{alg:rl_optimization}.
It means the loss is calculated (Line \ref{algo_line:loss_calculation}) and {\tool} is continuously optimized (Line \ref{algo_line:optimize}) based on the new evaluation cases and feedback in $t_{th}$ timestamp, i.e., $\mathcal{D}_t$.
After optimization, we can apply {\tool} on the new LLM tool-learning systems by registering the malicious tools in the ecosystems and generating injected commands to steal the information of other tools. 


%\section{The Proposed Method: GDDSG}\label{sec4}
\begin{figure*}
    \centering
    \includegraphics[width=\linewidth]{figures/framework.pdf}
    \caption{Illustration of The Overall Framework. [best view in color]}
    \label{fig: framework}
\end{figure*}
\textbf{Overview.} \autoref{fig: framework} provides an overview of our proposed method. 
Using task \( t \) as an example, we begin by projecting all training samples into an embedding space utilizing a pre-trained backbone. In this space, we compute the centroids for each class. Next, we evaluate whether a new centroid \( \mathbf{c}_i \) should be integrated into an existing class group \( G_j \).
If \( \mathbf{c}_i \) is dissimilar to all classes within \( G_j \), it is added to the group. If it is similar to any class in an existing group, it remains unassigned.
For unassigned centroids, we construct new similarity graphs (SimGraphs) based on their pairwise similarities. We then apply graph coloring theory to these SimGraphs, forming new class groups by clustering dissimilar categories together.
Finally, we update the NCM-based classifier with all class groups, facilitating efficient model updates with minimal computational overhead.

\subsection{Class Grouping Based on Similarity}

\autoref{Corollary: cor} provides guidance for constructing a sequence of dissimilar tasks. A key idea is to dynamically assign each new class to a group during CIL, ensuring that the similarity between the new class and other classes within the group is minimized. This approach helps maintain the robustness of each group's incremental learning process to the order of tasks. For each group, a separate adapter can be trained, and the results from different adapters can be merged during prediction to enhance the model's overall performance. 

In a given CIL task sequence, we organize the classes into several groups. The group list is denoted as \( G = [G_1, \dots, G_k] \), where each \( G_i \) represents a distinct group of classes. For a specified task \( t \) and each class \( C \in CLS^t \), our objective is to assign class \( C \) to an optimal group \( G^* \), ensuring that the new class is dissimilar to all existing classes in that group.

To achieve this objective, we first define the similarity between classes.
The similarity between any two classes, \( CLS_i \) and \( CLS_j \), is determined using an adaptive similarity threshold \( \eta_{i,j} \).
This threshold is computed based on the mean distance between the training samples of each class and their respective centroids in a learned embedding space, as shown below:

\begin{align}
    \eta_{i,j} = \max [
    & \frac{\sum_{k = 1}^{|X^t|} \mathbb{I}(y^t_k = i) \, d(h(x_k^t), \mathbf{c_i}) }{\sum_{k = 1}^{|X^t|} \mathbb{I}(y^t_k = i)}, \nonumber \\
    & \frac{\sum_{k = 1}^{|X^t|} \mathbb{I}(y^t_k = j) \, d(h(x_k^t), \mathbf{c_j}) }{\sum_{k = 1}^{|X^t|} \mathbb{I}(y^t_k = j)} 
    ],
\end{align}
where \( \mathbf{x}^{(t)} \) denotes the t-th task instance, \( h(\cdot) \) is the feature extraction function defined in Equation \autoref{eq: feature}, \( d: \mathcal{X} \times \mathcal{X} \to \mathbb{R}^+ \) specifies the distance metric space, \( \mathbb{I}(\cdot) \) represents the characteristic function, and the class centroid \( \mathbf{c}_i \in \mathbb{R}^d \) is computed as \( \mathbf{c}_i = \frac{1}{|C_i|} \sum_{x_j \in C_i} \mathbf{x}_j \).


Building upon this framework, we define the condition under which two classes, \( CLS_i \) and \( CLS_j \), are considered dissimilar. Specifically, they are deemed dissimilar if the following condition holds:

\begin{equation}
    d(\mathbf{c_i}, \mathbf{c_j}) > \eta_{i,j}.
    \label{eq: sim}
\end{equation}

Thus, class \( C \) is assigned to group \( G^* \) only if it is dissimilar to all classes within \( G^* \), and \( G^* \) is the choice with the lowest average similarity:

\begin{equation}
    G^* = \arg\min_{G} \frac{1}{|G|} \sum_{C' \in G} d(C, C').
\end{equation}

This approach is consistent with the principles outlined in \autoref{Corollary: cor} and ensures the robustness of the model across the entire task sequence.


\subsection{Graph-Driven Class Grouping}

Graph algorithms provide an efficient method for dynamically grouping classes while minimizing intra-group similarity.
In a graph-theoretic framework, classes are represented as nodes, with edge weights quantifying the similarity between them.
The flexibility and analytical power of graph structures allow for dynamic adjustment of class assignments in CIL, facilitating optimal grouping in polynomial time.
This approach significantly enhances the model's robustness and adaptability in incremental learning tasks.

Therefore, we can leverage the similarity between classes to construct a SimGraph, defined as follows:
\begin{definition} \textbf{(SimGraph.)}
A SimGraph can be defined as an undirect graph $SimG = (V, E)$, where $V$ is the set of nodes that represent each class's centroid and $E$ is the set of edges connecting pair of nodes that represent classes that are determined as similar by \autoref{eq: sim}.
\label{SimGraph}
\end{definition}

Then, we aim to partition the vertex set of this graph into subsets, with each subset forming a maximal subgraph with no edges between vertices. This problem can be abstracted as the classic NP-hard combinatorial optimization problem of finding a minimum coloring of the graphs. Let $G^{-1}(\cdot)$ be an assignment of class group identities to each vertex of a graph such that no edge connects two identically labeled vertices (i.e. $G^{-1}(i) \neq G^{-1}(j)$ for all $(i,j) \in E$). We can formulate the minimum coloring for graph $SimG$ as follows:
\begin{equation}
    \mathcal{X}(SimG) = \min | \{ G^{-1}(k) | k \in V\} |, 
    \label{eq: graph}
\end{equation}
where $\mathcal{X}(SimG)$ is called the chromatic number of $SimG$ and $|\cdot|$ denotes the size of the set.

Brooks' theorem \cite{brooks1941colouring} offers an upper bound for the graph coloring problem. To apply this in our context, we must demonstrate that the similarity graphs constructed in CIL meet the conditions required by Brooks' theorem. By doing so, we can establish that the problem is solvable and that the solution converges, ensuring the effectiveness of our grouping and class coloring process in class incremental learning. Without loss of generality, we can make the following assumptions:

\begin{assumption} In the CIL task, class \( C_i \) is randomly sampled without replacement from the set \( \mathcal{U} = \bigcup_{i=1}^{\infty} C_i \), ensuring that \( C_i \neq C_j \) for all \( i \neq j \). The probability that any two classes \( C_i \) and \( C_j \) within the set \( \mathcal{U} \) meet the similarity condition (as described in \autoref{eq: sim}) is denoted by \( p \).
\end{assumption}

In the CIL scenario with \( N \) classes, the probability of forming an odd cycle is given by \(\left( p^2(1-p)^{(N-2)} \right)^N = p^{2N}(1-p)^{N^2-2N}\). Similarly, the probability of forming a complete graph is \(p^{\binom{N}{2}} = p^{\frac{1}{2}N(N-1)}\).
Thus, the probability that the CIL scenario satisfies Brooks' theorem can be expressed as:
\begin{equation}
    P_{\text{Satisfy Brooks}'} = 1 - p^{2N}(1-p)^{N^2-2N} - p^{\frac{1}{2}N(N-1)}.
\end{equation}
\begin{figure}[t]
    \centering
    \includegraphics[width=\linewidth]{figures/contour_plot.pdf}
    \caption{Contour plot delineating the subthreshold region where \( P_{\text{Satisfy Brooks}'} < 0.99 \). The horizontal axis spans \( p \in [0.9, 1.0] \), representing probability values, while the vertical axis specifies sample sizes \( N \in [10, 40] \). In regions not displayed, the corresponding \( P_{\text{Satisfy Brooks}'} \) values exceed 0.99.}
    \label{fig: probability}
\end{figure}

\autoref{fig: probability} illustrates the various values of \( N \) and \( p \) that satisfy Brooks' theorem with a probability of less than 0.99. Our findings indicate that when \( N > 35 \), the CIL scenario adheres to Brooks' theorem. Furthermore, even with fewer classes, as long as \( p \) does not exceed 0.9, the CIL scenario can still ensure that the similarity graph complies with Brooks' theorem at a confidence level of 0.99. We conclude that class grouping based on the similarity graph is convergent and can be solved efficiently in polynomial time.

For \autoref{eq: graph}, while no algorithm exists that can compute \(\mathcal{X}(SimG)\) in polynomial time for all cases, efficient algorithms have been developed that can handle most problems involving small to medium-sized graphs, particularly the similarity graph \(SimG\) discussed here. In practical scenarios, such graphs are typically sparse. Notably, in conjunction with the above analysis, the similarity graph \(SimG\) in the CIL scenario satisfies the non-odd cycle assumption in Brooks' theorem \cite{brooks1941colouring}. For non-complete similarity graphs \(SimG\), we have \(\mathcal{X}(SimG) \le \Delta(SimG)\), where \(\Delta(SimG)\) represents the maximum vertex degree in \(SimG\).

Therefore, we can apply a simple yet effective greedy method, the Welsh-Powell graph coloring algorithm \cite{welsh1967upper}. This algorithm first sorts all nodes in the graph in descending order based on their degree and then assigns a color to each node, prioritizing those with higher degrees. During the coloring process, the algorithm selects the minimum available color for each node that differs from its neighbors, creating new color classes when necessary. The time complexity of this algorithm is \( O(|V|^2) \), primarily due to the color conflict check between each node and its neighbors. In theory, the maximum number of groupings produced by this algorithm is \( \max_{i = 1}^n \min\{ \deg(v_i') + 1, i \} \), with an error margin of no more than 1, where \( V' \) is the sequence of nodes sorted by degree, derived from \( V \).

\subsection{Overall Process}

\noindent \textbf{Training Pipeline.}
Building upon the theoretical foundations in Section 3.1, we now formalize the complete training procedure. Our framework leverages a frozen pre-trained feature extractor $\phi(\cdot)$, augmented with trainable random projections $W \in \mathbb{R}^{L \times M}$ where $M \gg L$, to enhance representation capacity. For each input $x_i^t$ from class group $s$, we compute its expanded feature:
\begin{equation}
\label{eq: feature}
h(x_i^t) = g(\phi(x_i^t) W) \in \mathbb{R}^M,
\end{equation}
where $g(\cdot)$ denotes the nonlinear activation.

The core learning paradigm reframes classification as regularized least-squares regression. Let $H_s^t \in \mathbb{R}^{N_s^t \times M}$ be the feature matrix and $Y_s^t \in \mathbb{R}^{N_s^t \times L_s^t}$ denote the one-hot label matrix for class group $s$. We optimize the projection matrix $\Theta_s^t \in \mathbb{R}^{M \times L_s^t}$ through:
\begin{equation}
\label{eq: loss}
\min_{\Theta} \|Y_s^t - H_s^t \Theta_s^t\|_F^2 + \lambda \|\Theta_s^t\|_F^2,
\end{equation}
where $\lambda$ controls regularization strength. The closed-form solution is:
\begin{equation}
\label{eq: analytic}
\Theta_s^t = ( {H_s^t}^\top H_s^t + \lambda I )^{-1} {H_s^t}^\top Y_s^t.
\end{equation}

For incremental updates, we maintain two key components: the Gram matrix $Gram_s^t$ capturing feature correlations, and the prototype matrix $C_s^t$ encoding class centroids. When new task $t$ arrives with $N_s^t$ samples:
\begin{equation}
    \label{eq: gram}
    Gram_{s}^t = Gram_{s}^{t-1} + \sum_{n = 1}^{N_{s}^t} h(x^t_i)^\top h(x^t_i),
\end{equation}
\begin{equation}
    \label{eq: pro}
    C_{s}^t = \begin{bmatrix}C_{s}^{t-1} & \underbrace{\mathbf{0}_M \ \mathbf{0}_M \ \ldots \ \mathbf{0}_M}_{(L_{s}^t - L_s^{t-1})\text{ times}} \end{bmatrix} + \sum_{n = 1}^{N_s^t} h(x^t_i)^\top y(x^t_i).
\end{equation}

The regularization parameter $\lambda$ is adaptively selected from a candidate pool $\Lambda$ through cross-validation on a held-out calibration set, minimizing the empirical risk:
\begin{equation}
\lambda^* = \arg \min_{\lambda \in \Lambda} \|Y_{\text{val}} - H_{\text{val}} (Gram_{\text{val}} + \lambda I)^{-1} C_{\text{val}} \|_F^2.
\end{equation}

Additionally, group descriptors are constructed through prototype similarity analysis. For each training instance $(x, y) \in \mathcal{D}^t$, we generate meta-features dataset as:
\begin{equation}
\mathcal{D}_g = \left\{ \left( \rho(x),\ G^{-1}(y) \right) \right\}_{(x,y)\in \mathcal{D}^t},
\end{equation}
where $\rho(x) = \big[ d(h(x),\mathbf{c}_1), \ldots, d(h(x),\mathbf{c}_k) \big]^\top$ denotes the concatenated distance vector measuring similarity between the sample embedding and prototype vectors.

\noindent \textbf{Inference Pipeline.}
Given test sample $x^*$, its group identification can be learned via $\hat{g} = \mathcal{M}_g(\rho(x^*))$, where $\mathcal{M}_g$ is the class group predict model trained with $ \mathcal{D}_g$.
Then, the prediction will be performed within the selected group via $\hat{y} = \underset{c \in \mathcal{C}_{\hat{g}}}{\arg\max}\ ( g(\phi(x^*) W) (Gram_{\hat{g}} + \lambda I)^{-1} C_{\hat{g}}[:,c]$.
        
\iffalse
In the previous section, we introduced the motivation and core concepts behind the proposed algorithm. In this section, we will describe the entire training process in detail. Recent years have seen CIL methods based on pre-trained models achieve remarkable results \cite{panos2023first,zhou2023revisiting,zhou2023revisiting,mcdonnell2024ranpac}, largely due to their robust representation capabilities. Since our proposed class grouping method also relies heavily on the model's representation ability, we utilize a widely-adopted pre-trained model as a feature extractor. For each class group, we train independent classification heads, which enhances the model’s adaptability and generalization to different class groups.

As outlined above, we utilize a frozen random projection matrix \( W \in \mathbb{R}^{L \times M} \) to enhance features across all class groups, where \( L \) is the output dimension of the pre-trained model and \( M \gg L \) is the expanded dimensionality. Given a task \( t \) and a sample \( x^t_i \) belonging to a class group \( s \), the feature vector of the sample is denoted as \( h(x^t_i) \), and its one-hot encoded label as \( y(x^t_i) \). Specifically,

\begin{equation}
    \label{eq: feature}
    h(x^t_i) = g(\phi(x)^T W),
\end{equation}
where \( \phi(\cdot) \) represents the feature extractor, and \( g(\cdot) \) is a nonlinear activation function.
We define \( H_{s}^t \in \mathbb{R}^{N_{s}^t \times M} \) as the matrix containing feature vectors of \( N_{s}^t \) samples from group \( s \). The corresponding Gram matrix is defined as:
\begin{equation}
    \label{eq: grammatrix}
    Gram_{s}^t = {H_{s}^t}^T H_{s}^t \in \mathbb{R}^{M \times M}.
\end{equation}
Additionally, the matrix \( C_{s}^t \) consists of the concatenated column vectors of all classes within group \( s \), with dimensions \( M \times L_{s}^t \), where \( L_s^t \) represents the number of classes in group \( s \) for task \( t \). When a new task arrives, the model applies the GDDSG algorithm to assign new classes to their respective groups. The Gram matrix \( Gram \) and matrix \( C \) for each group are updated according to the following formulas:
\begin{equation}
    \label{eq: gram}
    Gram_{s}^t = Gram_{s}^{t-1} + \sum_{n = 1}^{N_{s}^t} h(x^t_i) \otimes h(x^t_i),
\end{equation}
\begin{equation}
    \label{eq: pro}
    C_{s}^t = \begin{bmatrix}C_{s}^{t-1} & \underbrace{\mathbf{0}_M \ \mathbf{0}_M \ \ldots \ \mathbf{0}_M}_{(L_{s}^t - L_s^{t-1})\text{ times}} \end{bmatrix} + \sum_{n = 1}^{N_s^t} h(x^t_i) \otimes y(x^t_i),
\end{equation}
where \( \mathbf{0}_M \) denotes a zero vector with \( M \) dimensions.

During the test phase, we combine the classification heads of all groups \( G = [G_1, G_2, \dots, G_k] \) to make a joint prediction for a given sample \( x \). For each class \( c' \) in a group, the score is computed as follows:
\begin{equation}
    \label{eq: predict}
    s_{c'} = g(\phi(x)^T W)(Gram_{i} + \lambda I)^{-1} C_{c'},
\end{equation}
where \( i = 1,\dots,k \) denotes the indices of each groups, and \( \lambda \) is the regularization parameter used to ensure that the \( Gram \) matrix remains invertible. The final classification result is then obtained by applying the following formula:
\begin{equation}
    \label{eq: predict_joint}
    \hat{c} = \mathop{\arg\max}\limits_{c' \in \cup_{i = 1}^k CLS^{G_i}} s_{c'},
\end{equation}
where \( \cup_{i = 1}^k CLS^{G_i} \) represents the set of possible classes across all class groups.
\fi

%\section{Perspectives}

\subsection{MADRL Should Leverage Direct Interpretability}

Engaging and expanding interpretability is an opportunity to address existing challenges in MADRL. Direct approaches are particularly well-suited for analysing communication dynamics, coordination strategies, and emergent behaviours in MAS. Graph-based analysis, for instance, could provide insights into inter-agent interactions, while feature importance techniques can identify biases and ensure fairness in decision-making. By systematically exploring and applying scalable direct methods to trained models, researchers can better address the inherent complexities of MADRL, enabling the development of more transparent, robust, and accountable systems for real-world applications.

Although previous calls to action are prone to integrate interpretability beforehand \cite{rodriguez2024explainable}, this paper claims that the interpretation of models post hoc is highly valuable. Direct interpretability offers greater flexibility, particularly for existing models where architectural modifications are impractical. 

\subsection{Robust Evaluation Protocols}

As repeatedly outlined, direct post-hoc methods are easily actionable and scalable.
However, their adoption requires acknowledging and addressing limitations such as the inherent shortcomings of saliency maps \cite{Adebayo2018SanityCF,Bilodeau2022ImpossibilityTF}, counterfactual explanations \cite{Laugel2019TheDO}, or other interpretability illusions \cite{Bolukbasi2021AnII,Friedman2023InterpretabilityII,Friedman2023InterpretabilityII}. In fact, these methods often generate metrics with limited predictive power, and thus, claims should be reasonable.


A key priority is the development of robust evaluation protocols for direct methods. Given the absence of ground-truth explanations, reliable metrics and standardized evaluation frameworks must be established to assess the quality and utility of these methods \cite{Gill2020ARM,Madsen2021PosthocIF,Amorim2023EvaluatingPI,Hedstrm2022QuantusAE,Wei2024RevisitingTR,Huang2024RAVELEI,Chaudhary2024EvaluatingOS}. 
Advancing evaluation thoroughly, e.g., by evaluating out of distribution, is especially important to develop scalable, effective, and actionable interpretability solutions.
  
%\section{Discussion}


\subsection{Implications}
Our investigation highlights the critical need to mitigate knowledge base poisoning. The findings from our study have several implications for enhancing the security of code generated by RACG systems.
Firstly, existing retrieval strategies naturally favor the most relevant examples from the knowledge base, which gives attackers the opportunity to successfully mislead the generation process with a small number of vulnerable examples. We argue that this risk can be mitigated by adjusting the retrieval strategy. For instance, an alternative strategy would be to select the second most similar example or randomly choose from a candidate pool containing several of the most similar examples, and we plan to explore such a way in future work.


Secondly, based on the results from RQ1, we found that hiding the programming intent (i.e., in Scenario II) increases the difficulty of successful poisoning. For instance, attackers can achieve a VRRC of 0.38 in Scenario I with only one poisoned vulnerable example. However, in Scenario II, attackers achieve a VRRC of 0.35 with a poisoning proportion of 0.8, meaning that they need to inject 9,642 vulnerable examples into the knowledge base (calculated as $\lfloor12,053 \times 0.8\rfloor$).
% which corresponds to a poisoning rate of 40\% (9,642/21,695). 
This indicates that concealing the programming intent makes poisoning more intricate and easier to detect.

Thirdly, according to the finding from \S\ref{subsec:vul_type}, the security of LLM-generated code vary across CWE types. This suggests that RACG systems could devise special strategies to check for the existence of several specific CWE types in the knowledge base, such as CWE-352, with the aim of improving the security of the generated code, as these vulnerabilities are more likely to induce the generation of vulnerable code.
% key types of vulnerabilities in the knowledge base, thereby improving the security of the generated code.

\subsection{Effectiveness of Judge}
\label{subsec:judge_effectivenss}
To assess the performance of using an LLM as a judge to label responses, we evaluate the effectiveness through both manual sampling inspection and automated inspection. For {\bf manual} inspection, we determine the sample size based on a 95\% confidence level and a 10 confidence interval, using a population size of 12,053 responses from GPT-4o. The final sample sizes for C, C++, Java, and Python are 95, 81, 93, and 91, respectively, as calculated using an off-the-shelf tool.\footnote{\url{https://www.surveysystem.com/sscalc.htm}} The evaluated code was sampled from GPT-4o in a moderated, one-shot setting using the JINA retriever in Scenario I with five poisoning vulnerabilities.
Two authors independently evaluated the samples through manual review, followed by a double-check to ensure consistency. For {\bf automated} inspection, we using a dataset containing both vulnerable code and its fixed version. Specifically, we evaluate pairs of items (i.e., the vulnerable version and its fixed counterpart) using our LLM-based judge to see if it can distinguish between them. This evaluation is performed on the full dataset.
We classify a code sample as positive when the judge correctly identifies it as vulnerable and the results are presented in Table~\ref{tab:dis_judge_combined}. 
\begin{table}[!t]
  \centering
  \caption{Performance of LLM-based judges under manual and automated inspection.}
  \resizebox{1\linewidth}{!}{
    \begin{tabular}{llrrrr}
    \toprule
    \multirow{2}{*}{\shortstack{\textbf{Inspection} \\ \textbf{Method}}} & \multirow{2}{*}{\textbf{Language}} & \multirow{2}{*}{\textbf{Accuracy}} & \multirow{2}{*}{\textbf{Precision}} & \multirow{2}{*}{\textbf{Recall}} & \multirow{2}{*}{\textbf{F1}} \\
    & & & & & \\
    \midrule
    \multirow{4}{*}{Manual} 
      & C     &   0.76  & 0.80 & 0.78  & 0.79 \\
      & C++   &  0.72   &  0.84  & 0.79    & 0.81 \\
      & Java  &  0.79 &  0.81  & 0.75 & 0.78 \\
      & Python &  0.84  &  0.85 & 0.78      & 0.81 \\
    \midrule
    \multirow{4}{*}{Automated} 
      & C     & 0.80 &  0.86 &  0.72 & 0.78 \\
      & C++   & 0.80 &  0.83 &  0.76 & 0.79 \\
      & Java  & 0.79 &  0.80 &  0.76 & 0.78 \\
      & Python & 0.82  & 0.83 & 0.80 & 0.81 \\
    \bottomrule
    \end{tabular}%
    }
  \label{tab:dis_judge_combined}%
\end{table}%
Overall, the LLM-based judge demonstrates commendable performance, consistently achieving high accuracy, precision, recall, and F1 scores across both inspection approaches. This indicates that the judge is capable of effectively detecting vulnerabilities in generated code, making it a reliable evaluation method. Among the four programming languages evaluated, the judge's performance remains generally consistent.

In the results under manual inspection, Python yields the highest performance, with accuracy and F1 scores reaching 0.84 and 0.81, respectively. The performance is slightly lower for C and Java, with F1 scores of 0.79 and 0.78, respectively, but still remains at a high level. In the results under automated inspection, the judge's F1 scores across all four programming languages hover around 0.8, aligning closely with the results from manual inspection. These evaluations demonstrate that the judge effectively detects vulnerabilities in generated code.


\subsection{Effectiveness of Retrievers}
% Table generated by Excel2LaTeX from sheet 'Sheet1'
\begin{table}[!t]
  \centering
  \caption{The effectiveness of retrievers}
  \resizebox{0.8\linewidth}{!}{
      \begin{tabular}{lrrrr}
    \toprule
    \textbf{Retriever} & \textbf{MRR} & \textbf{SR@1$\dagger$} & \textbf{SR@5} &\textbf{SR@10} \\
    \midrule
    JINA  & \bf{0.85} & \bf{0.79} & \bf{0.91} & \bf{0.93} \\
    BM25  & 0.20 & 0.14 & 0.19 & 0.24 \\
    \bottomrule
    \end{tabular}%
  }
  \label{tab:dis_retriever}%
  \caption*{\footnotesize $\dagger$ SR@k represents the SuccessRate@k\hspace{2.8cm}\textcolor{white}{.}}
\end{table}%
\vspace{-2mm}


Our influencing factors analysis of LLM-introduced vulnerabilities (\S\ref{subsec:cause_analysis}) reveals that different retrievers impact the security of generated code. Specifically, LLMs using the JINA retriever are more prone to generating vulnerable code than those using BM25 across various LLMs and scenarios. We attribute this to JINA's superior retrieval of relevant code. To validate this, we evaluate retriever effectiveness using MRR and SuccessRate@k (Table~\ref{tab:dis_retriever}), following prior work~\cite{liu2021opportunities,wang2024fusing}. MRR is the average reciprocal rank of results for a set of queries $q$, and SuccessRate@k is the percentage of queries where the relevant code snippet appears within the top-k results. As shown, JINA significantly outperforms BM25 across all metrics: MRR (0.85 vs. 0.20), SR@1 (0.79 vs. 0.14), SR@5 (0.91 vs. 0.19), and SR@10 (0.93 vs. 0.24). This confirms JINA's superior retrieval capability, which, while beneficial for general code generation, exposes LLMs to more potentially vulnerable code, thus increasing the likelihood of generating vulnerable code.

\subsection{The Difference with RAG Poisoning}
\label{subsec:diff_rag_poisoning}
RAG and RACG systems share the use of external knowledge to enhance content generation. However, they differ significantly in the nature of poisoning attacks and their consequences. Specifically, RAG poisoning targets the functional accuracy of the system~\cite{zou2024poisonedrag,zhang2024hijackrag}. In a RAG setup, attackers inject false or misleading examples into the knowledge base, causing the system to retrieve incorrect information. This disrupts the model’s ability to generate factually accurate outputs, undermining its usefulness in tasks requiring reliable information. The primary goal here is to compromise the system’s ability to produce correct content. 

In contrast, RACG poisoning aims to compromise the security of generated code without impacting functionality. Otherwise, the developer would discard the generated code and there would not be targeted vulnerability in the software.
By introducing vulnerable code examples into the knowledge base, attackers aim to influence the code generation process and lead to the creation of code with exploitable vulnerabilities, such as buffer overflows or SQL injection risks. This poisoning could propagate security vulnerabilities, creating potential real-world risks. RACG poisoning aims to infect the generated code with vulnerabilities that could be exploited.

This paper is the first comprehensive study examining how vulnerable code examples in the knowledge base impact the security of code generated by RACG systems. We focus on how these poisoned examples can lead to the generation of insecure code, introducing potential vulnerabilities. Our work highlights the need for securing RACG knowledge sources to prevent the propagation of security risks in generated code.

\subsection{Threats to Validity}

{\bf Query Generation through LLM.} In Section~\ref{subsec:dataset_cons}, we use DeepSeek-V2.5 to generate queries for code. However, there is a possibility that DeepSeek-V2.5 may produce inaccurate content. To mitigate this threat, we manually review the generated queries. Specifically, we randomly select 100 queries for each programming language and have them reviewed by the two authors. Any inconsistencies in the evaluation results were resolved through discussion between the authors. The manual review indicates that 86\% of the generated queries accurately reflect the functionality of the code on average. Therefore, the impact of this threat is minimal.

\noindent

{\bf Programming Languages Investigated.} In this study, we conduct experiments using four widely-used programming languages: C, C++, Java, and Python. One potential threat is that the selected languages may not fully represent real-world development scenarios. However, according to GitHub usage statistics (measured by the number of pull requests) for the first quarter of 2024~\cite{githut2024}, these four languages account for 42.7\% of the total activity. Among them, Python and Java are the most popular. Additionally, as other languages like Go gain popularity, we plan to extend our study to include more programming languages in future work.
%\input{8.ethicalState}


%
\newpage
\appendix
% \onecolumn

% \section{Pseudo-code for \Ours}

\section{LLM inference strategy and IR pipelines}

\begin{table}[h]
\caption{Correspondence between LLM inference and IR pipelines.}
  \label{tb:llm-retriever-reranker}
  \centering
%   \small
  \scalebox{0.8}{\begin{tabular}{lccc}
    \toprule
    Method & Retriever & Reranker & Pipeline       \\
    \midrule
    Greedy decoding     & LLM &  $\emptyset$ & Retriever-only  \\
    \midrule
    Best-of-N \citep{stiennon2020learning} & LLM & Reward model & Retriever-reranker  \\
    \midrule
    Majority voting  \citep{wang2022self}  & LLM & Majority & Retriever-reranker  \\
    \midrule
    Iterative refinement \citep{madaan2024self} & LLM & $\emptyset$ & Iterative retrieval  w. query rewriting \\
    \bottomrule
  \end{tabular}}
\end{table}


\section{How can SFT and preference optimization help the LLM from an IR perspective?}\label{apx:sft-rlhf-empirical}


We assess how well LLMs perform at two tasks: fine-grained reranking (using greedy decoding accuracy) and coarse-grained retrieval (using Recall@$N$).  
We focus on how SFT and DPO, affect these abilities.  
Using the Mistral-7b model, we evaluate on the GSM8k and MATH datasets with two approaches: SFT-only, and SFT followed by DPO (SFT $\rightarrow$ DPO).

In the SFT phase, the model is trained directly on correct answers. 
For DPO, we generate 20 responses per prompt and created preference pairs by randomly selecting one correct and one incorrect response.  
We use hyperparameter tuning and early stopping to find the best model checkpoints (see Appendix \ref{apx:sec:sft-rlhf} for details).


\begin{table}[h]
\caption{Retrieval (Recall@N) and reranking (greedy accuracy) metrics across dataset and training strategies, with Mistral-7b as the LLM. 0.7 is used as the temperature. Recall@N can also be denoted as pass@N.}\label{tb:sft-rlhf-result}
\vskip 1em
\centering
\small
\begin{tabular}{llcccc}
    \toprule
     & Metric & \textbf{init model} & \textbf{SFT} & \textbf{SFT $\rightarrow$ DPO} \\
    \midrule
    \multirow{4}{*}{\rotatebox{90}{GSM8K}} 
    & Greedy Acc & 0.4663 & 0.7680 & 0.7991  \\
    & Recall@20 & 0.8347 & 0.9462 & 0.9545  \\
    & Recall@50 & 0.9090 & 0.9629 & 0.9727  \\
    & Recall@100 & 0.9477 & 0.9735 & 0.9826   \\
    \midrule
    \multirow{4}{*}{\rotatebox{90}{Math}} 
    & Greedy Acc & 0.1004 & 0.2334 & 0.2502 \\
    & Recall@20 & 0.2600 & 0.5340 & 0.5416  \\
    & Recall@50 & 0.3354 & 0.6190 & 0.6258  \\
    & Recall@100 & 0.4036 & 0.6780 & 0.6846  \\
    \bottomrule
\end{tabular}
\end{table}

The results are shown in Table \ref{tb:sft-rlhf-result}.  
We observe that both SFT and DPO improve both retrieval and reranking, with SFT being more effective. Adding DPO after SFT further improves performance on both tasks.  
This is consistent with information retrieval principles that both direct retriever optimization and reranker-retrieval distillation can enhance the retriever performance, while the latter on top of the former can further improve the performance. Further discussions can be found in Appendices \ref{apx:discuss1} and \ref{apx:discuss2}.


\section{Discussion on the connection and difference between SFT and direct retriever optimization}\label{apx:discuss1}

As discussed in Section \ref{sec:llm-tuning-retriever}, the direct retriever optimization goal with InfoNCE is shown as:
\begin{gather*}
    \max \log P(d_{\text{gold}}|q) = \max \log \frac{\text{Enc}_d(d_{\text{gold}}) \cdot\text{Enc}_q(q)}{\sum^{|C|}_{j=1} \text{Enc}_d(d_j) \cdot\text{Enc}_q(q)},
\end{gather*}
while the SFT optimization goal is shown as:
\begin{gather}
    \max \log P(y_{\text{gold}}|x) = \max \log \prod^{|y_{\text{gold}}|}_i P(y_{\text{gold}}(i)|z_i) 
    = \max \sum^{|y_{\text{gold}}|}_i \log \frac{\text{Emb}(y_{\text{gold}}(i)) \cdot\text{LLM}(z_i)}{\sum^{|V|}_{j=1} \text{Emb}(v_j) \cdot\text{LLM}(z_i)}. \label{apx:eq:sft}
\end{gather}

As a result, the SFT objective can be seen as a summation of multiple retrieval optimization objectives, where $\text{LLM}(\cdot)$ and word embedding $\text{Emb}(\cdot)$ are query encoder and passage encoder respectively.

However, for direct retriever optimization with InfoNCE, $\text{Enc}_d(\cdot)$ is usually a large-scale pretrained language model which is computationally expensive on both time and memory.
In this case, it is unrealistic to calculate the $\text{Enc}_d(d_j)$ for all $d_j\in C$, when $C$ is large, because of the time constrain and GPU memory constrain.
As a result, a widely-adopted technique is to adopt ``in-batch negatives'' with ``hard negatives'' to estimate the $\log P(d_{\text{gold}}|q)$ function:
\begin{gather*}
    \max \log P(d_{\text{gold}}|q) = \max \log \frac{\text{Enc}_d(d_{\text{gold}}) \cdot\text{Enc}_q(q)}{\sum^{|C|}_{j=1} \text{Enc}_d(d_j) \cdot\text{Enc}_q(q)} \\
    \sim \max \log \frac{\text{Enc}_d(d_{\text{gold}}) \cdot\text{Enc}_q(q)}{\sum^{|B|}_{i=1} \text{Enc}_d(d_i) \cdot\text{Enc}_q(q) + \sum^{|H|}_{j=1} \text{Enc}_d(d_j) \cdot\text{Enc}_q(q)},
\end{gather*}
where $B$ is the in-batch negative set and $H$ is the hard negative set.
Note that $B\bigcup H \subset C$.
This objective is more efficient to optimize but is not the original optimization goal. As a result, the learned model after direct retriever optimization is not optimal.
It is also found that the hard negatives $H$ is the key to estimate the original optimization goal \citep{zhan2021optimizing}.
Thus, reranker-retriever distillation can further improve the retriever by introducing more hard negatives.

On the other hand, LLM optimization, as shown in Eq. (\ref{apx:eq:sft}), can be seen as a summation of multiple retrieval optimization function.
In each retrieval step, the passage can be seen as a token and the corpus is the vocabulary space $V$.
Given that the passage encoder $\text{Emb}(\cdot)$ (word embedding) here is cheap to compute and the vocabulary space $V$ ($<$100k) is usually not as large as $C$ ($>$1M) in IR, the objective in Eq. (\ref{apx:eq:sft}) can be directly optimized without any estimation.
In this case, the LLM as a retriever is more sufficiently trained compared with the retriever training in IR.


\section{Discussion on the connection and difference between preference optimization and reranker-retriever distillation}\label{apx:discuss2}

As discussed in Section \ref{sec:llm-tuning-retriever}, preference optimization with an online reward model $f_{\text{reward-model}}(\cdot) \overset{r}{\rightarrow} \text{data} \overset{g(\cdot)}{\rightarrow}  f_{\text{LLM}}(\cdot)$ can be seen as a reranker to retriever distillation process $f_{\text{rerank}}(\cdot) \overset{r}{\rightarrow} \text{data}\overset{g(\cdot)}{\rightarrow}   f_{\text{retrieval}}(\cdot)$, where the reward model is the reranker (\textit{i.e.}, cross-encoder) and the LLM is the retriever (\textit{i.e.}, bi-encoder).

However, there are two slight differences here:
\begin{itemize}[leftmargin=*]
\item The LLM after SFT is more sufficiently trained compared to a retriever after direct optimization. As discussed in Appendix \ref{apx:discuss1}, the SFT optimization function is not an estimated retriever optimization goal compared with the direct retrieval optimization. As a result, the LLM after SFT is suffienctly trained. In this case, if the reward model (reranker) cannot provide information other than that already in the SFT set (\textit{e.g.}, using the SFT prompts), this step may not contribute to significant LLM capability improvement.
\item The reward model may introduce auxiliary information than the reranker in IR. For a reranker in IR, it captures a same semantic with the retriever: semantic similarity between the query and the passage. However, in LLM post-training, the goal and data in SFT and preference optimization can be different. For example, the SFT phase could have query/response pairs which enable basic chat-based retrieval capability for the LLM. While the reward model may contain some style preference information or safety information which do not exist in SFT data. In this case, the preference optimization which is the reranker to retriever distillation step could also contribution to performance improvement.
\end{itemize}


\section{Evaluate LLMs as retrievers}\label{apx:llm-as-retriever}

In addition to Mathstral-7b-it on GSM8K in Figure \ref{fig:mathstral-gsm8k-infer}, we conduct extensive experiments to both Mistral-7b-it and Mathstral-7b-it on GSM8K and MATH. The results are shown in Figure \ref{apx:fig:empirical-llm-retriever}.
We have similar findings as in Figure \ref{fig:mathstral-gsm8k-infer} that:
(1) As $N$ increases, Recall@$N$ improves significantly, indicating that retrieving a larger number of documents increases the likelihood of including a correct one within the set.
(2) For smaller values of $N$ (e.g., $N=1$), lower temperatures yield higher Recall@$N$. This is because lower temperatures reduce response randomness, favoring the selection of the most relevant result.
(3) Conversely, for larger $N$ (e.g., $N>10$), higher temperatures enhance Recall@$N$. Increased temperature promotes greater response diversity, which, when combined with a larger retrieval set, improves the chances of capturing the correct answer within the results.

\begin{figure*}[h]
    \centering
    \subfigure[Mistral-7b-it on GSM8k]{\includegraphics[width=0.45\textwidth]{figure/LLM_alignment_gsm8k_mathstral7b_infer.pdf}}
    \subfigure[Mistral-7b-it on GSM8k]{\includegraphics[width=0.45\textwidth]{figure/LLM_alignment_gsm8k_mistral7b_infer.pdf}}
    \subfigure[Mathstral-7b-it on MATH]{\includegraphics[width=0.45\textwidth]{figure/LLM_alignment_math_mathstral7b_infer.pdf}} 
    \subfigure[Mistral-7b-it on MATH]{\includegraphics[width=0.45\textwidth]{figure/LLM_alignment_math_mistral7b_infer.pdf}}
    % \vspace{-0.1in}
    \vskip -1em
    \caption{Evaluate the LLM as a retriever with Recall@N (Pass@N). As the number (N) of retrieved responses increases, the retrieval recall increases. The higher the temperature is, the broader spectrum the retrieved responses are, and thus the higher the recall is.}\label{apx:fig:empirical-llm-retriever}
\end{figure*}


% \subsection{How SFT and RLHF benefit the LLM retriever?}\label{apx:sft-rlhf}

% In addition to the experiments with Gemma-1-7b-it in Table \ref{tb:sft-rlhf-result}, we also conduct experiments to study the effect of SFT and DPO on Deepseek-math-7b-base model \citep{shao2024deepseekmath}.
% The results on MATH dataset are shown in Table \ref{apx:tb:sft-rlhf-result}, where we have similar discovery with that in Table \ref{tb:sft-rlhf-result}:
% (1) Both SFT and DPO can improve the retrieval capability of the LLM, while SFT is more effective.
% (2) On top of SFT, DPO can slightly improve the reranking capability (greedy accuracy) but not the general retrieval capability.

% \begin{table}[h]
% \caption{Retrieval (Recall@N) and reranking (greedy accuracy) metrics across dataset and training strategies. LLM: Deepseek-math-7b. Temperature: 0.7. Recall@N can also be denoted as pass@N.}\label{apx:tb:sft-rlhf-result}
% \vskip 1em
% \centering
% \scalebox{0.8}{
% \begin{tabular}{llcccc}
%     \toprule
%      & Metric & \textbf{init model} & \textbf{DPO} & \textbf{SFT} & \textbf{SFT $\rightarrow$ DPO} \\
%     \midrule
%     \multirow{4}{*}{\rotatebox{90}{Math}} 
%     & Greedy Acc & 0.0972 & 0.1164 & 0.3078 & 0.312 \\
%     & Recall@20 & 0.4914 & 0.5136 & 0.6524 & 0.6558 \\
%     & Recall@50 & 0.6058 & 0.6278 & 0.7332 & 0.736 \\
%     & Recall@100 & 0.6728 & 0.6976 & 0.7844 & 0.7828 \\
%     \bottomrule
% \end{tabular}}
% \end{table}



\section{\Ours retriever optimization objective}\label{apx:proofs}

We provide the proof for different variants of \Ours's objective functions.

\subsection{Contrastive ranking}\label{apx:proof:contrastive}

\begin{theorem}
Let \( x \) be a prompt and \( (y_w, y^{(1)}_l, ..., y^{(m)}_l)  \) be the responses for \( x \) under the contrastive assumption (Eq.(\ref{eq:contrastive-assumption})).
Then the objective function to learn the LLM \( \pi_\theta \):
\end{theorem}

\begin{equation}
    \begin{aligned}
    \mathcal{L}_{\text{con}} = -\mathbb{E} & \biggl[
    \log \frac{\exp\bigl(\gamma(y_w \mid x)\bigr)}{
        \exp\bigl(\gamma(y_w \mid x)\bigr) + \sum_{i=1}^m \exp\bigl(\gamma(y_l^{(i)} \mid x)\bigr)}
    \biggr], \\
    \text{where } &\quad \gamma(y \mid x) = \beta \log \frac{\pi_\theta(y \mid x)}{\pi_{\mathrm{ref}}(y \mid x)}.
\end{aligned}\label{eq:contrastive}
\end{equation}

\textit{Proof.}
From \citep{rafailov2024direct}, we know that
\begin{gather}
    r(x, y) = \beta \text{log} \frac{\pi_{\text{llm}}(y|x)}{\pi_{\text{ref}}(y|x)} + \beta \text{log} Z,
\end{gather}
where $Z = \sum_{y'} \pi_{\text{ref}}(y'|x) \text{exp}(\frac{1}{\beta} r(x, y'))$.

Then,
\begin{equation}\label{eq:1-n}
\begin{aligned}
\mathbb{P}\text{r}(y_w & \succeq y^{(1)}_l, ..., y_w \succeq y^{(m)}_l) 
= \text{softmax}(r(x, y_w)) \\
&= \frac{\text{exp}(r(x,y_w))}{\text{exp}(r(x,y_w)) + \sum^m_{i=1}\text{exp}(r(x,y^{(i)}_l))} \\
&= \frac{1}{1 + \sum^m_{i=1}\text{exp}(r(x,y^{(i)}_l)-r(x,y_w))} \\
&= \frac{1}{1 + \sum^m_{i=1}\text{exp}(\gamma(y^{(i)}_l \mid x) + \beta \text{log} Z - \gamma(y_w \mid x) - \beta \text{log} Z)} \\
&= \frac{1}{1 + \sum^m_{i=1}\text{exp}(\gamma(y^{(i)}_l \mid x) - \gamma(y_w \mid x))} \\
&= \frac{\exp\bigl(\gamma(y_w \mid x)\bigr)}{
        \exp\bigl(\gamma(y_w \mid x)\bigr) + \sum_{i=1}^m \exp\bigl(\gamma(y_l^{(i)} \mid x)\bigr)}
\end{aligned}
\end{equation}

We can learn $\pi_\theta$ by maximizing the logarithm-likelihood: 
\begin{gather}
\max \log \mathbb{P}\text{r}(y_w \succeq y^{(1)}_l, \dots, y_w \succeq y^{(m)}_l) \Leftrightarrow 
\min - \log \mathbb{P}\text{r}(y_w \succeq y^{(1)}_l, \dots, y_w \succeq y^{(m)}_l) = \mathcal{L}, \\
 \therefore \mathcal{L}_{\text{con}} = -\mathbb{E} \biggl[
    \log \frac{\exp\bigl(\gamma(y_w \mid x)\bigr)}{
        \exp\bigl(\gamma(y_w \mid x)\bigr) + \sum_{i=1}^m \exp\bigl(\gamma(y_l^{(i)} \mid x)\bigr)}
    \biggr], \\
\text{where} \quad \gamma(y \mid x) = \beta \log \frac{\pi_\theta(y \mid x)}{\pi_{\mathrm{ref}}(y \mid x)}.
\end{gather}



\subsection{LambdaRank ranking}\label{apx:proof:lambdarank}

\begin{theorem}
Let \( x \) be a prompt and \( (y_1, ..., y_m)  \) be the responses for \( x \) under the LambdaRank assumption (Eq.(\ref{eq:lambdarank-assumption})).
Then the objective function to learn the LLM \( \pi_\theta \):
\end{theorem}

% \begin{gather}
%     \mathcal{L}_{\text{lamb}}=-\mathbb{E}\;\biggl[ \sum_{1<i<j<m}
%   w_{ij}\log \sigma\Bigl(
%      \gamma(y_i \mid x)-
%      \gamma(y_j \mid x)
%   \Bigr)
% \biggr]
% \end{gather}
\begin{gather}
    \mathcal{L}_{\text{lamb}}=-\mathbb{E}\;\biggl[ \sum_{1<i<j<m}
   \log \sigma\Bigl(
     \gamma(y_i \mid x)-
     \gamma(y_j \mid x)
   \Bigr)
\biggr].
\end{gather}
% where $w_{ij}$ is an adjustable weight.

\textit{Proof.}
\begin{equation}
\begin{aligned}
\mathbb{P}\text{r}(y_1 & \succeq ... \succeq y_m)
= \prod_{1<i<j<m} \sigma(r(x,y_i) - r(x,y_j)) \\
&= \prod_{1<i<j<m} \sigma(\gamma(x,y_i) + \beta \text{log} Z - \gamma(x,y_j) - \beta \text{log} Z)  \\
&= \prod_{1<i<j<m} \sigma(\gamma(y_i \mid x)-
     \gamma(y_j \mid x)).
\end{aligned}
\end{equation}

We can learn $\pi_\theta$ by maximizing the logarithm-likelihood: 
\begin{gather}
\max \log \mathbb{P}\text{r}(y_w \succeq y^{(1)}_l, \dots, y_w \succeq y^{(m)}_l) \Leftrightarrow 
\min - \log \mathbb{P}\text{r}(y_w \succeq y^{(1)}_l, \dots, y_w \succeq y^{(m)}_l) = \mathcal{L}, \\
 \therefore \mathcal{L}_{\text{lamb}}=-\mathbb{E}\;\biggl[ \sum_{1<i<j<m}
   \log \sigma\Bigl(
     \gamma(y_i \mid x)-
     \gamma(y_j \mid x)
   \Bigr)
\biggr], \\
\text{where} \quad \gamma(y \mid x) = \beta \log \frac{\pi_\theta(y \mid x)}{\pi_{\mathrm{ref}}(y \mid x)}.
\end{gather}
% $w_{ij}$ can be added to control the weight of each pair in the candidate list.


\subsection{ListMLE ranking}\label{apx:proof:listmle}

\begin{theorem}
Let \( x \) be a prompt and \( (y_1, ..., y_m)  \) be the responses for \( x \) under the ListMLE assumption (Eq.(\ref{eq:listmle-assumption})).
Then the objective function to learn the LLM \( \pi_\theta \):
\end{theorem}

\begin{equation}
\begin{aligned}
    \mathcal{L}_{\text{lmle}} &= -\mathbb{E} \biggl[
    \sum^m_{i=1} \log \frac{\exp\bigl(\gamma(y_i \mid x)\bigr)}{
        \exp\bigl(\gamma(y_i \mid x)\bigr) + \sum_{j=i}^m \exp\bigl(\gamma(y_j \mid x)\bigr)}
    \biggr].
\end{aligned}
\end{equation}

\textit{Proof.}
From Eq.(\ref{eq:1-n}),
\begin{gather}
\begin{aligned}
    \mathbb{P}\text{r}(y_1 & \succeq ... \succeq y_m) = \prod^m_{i=1} \mathbb{P}\text{r}(y_i \succeq y_{i+1}, ..., y_i \succeq y_m)  \\
    & = \prod^m_{i=1} \frac{\text{exp}(\gamma(y_i \mid x))}{\text{exp}(\gamma(y_i \mid x)) + \sum^m_{j=i+1}\text{exp}(\gamma(y_j \mid x))}
\end{aligned}.
\end{gather}
% The derivation above uses the result from Eq.(\ref{eq:1-n}).

We can learn $\pi_\theta$ by maximizing the logarithm-likelihood: 
\begin{gather}
\max \log \mathbb{P}\text{r}(y_w \succeq y^{(1)}_l, \dots, y_w \succeq y^{(m)}_l) \Leftrightarrow 
\min - \log \mathbb{P}\text{r}(y_w \succeq y^{(1)}_l, \dots, y_w \succeq y^{(m)}_l) = \mathcal{L}, \\
 \therefore \mathcal{L}_{\text{lmle}} = -\mathbb{E} \biggl[
    \sum^m_{i=1} \log \frac{\exp\bigl(\gamma(y_i \mid x)\bigr)}{
        \exp\bigl(\gamma(y_i \mid x)\bigr) + \sum_{j=i}^m \exp\bigl(\gamma(y_j \mid x)\bigr)}
    \biggr], \\
\text{where} \quad \gamma(y \mid x) = \beta \log \frac{\pi_\theta(y \mid x)}{\pi_{\mathrm{ref}}(y \mid x)}.
\end{gather}


\section{Baselines}\label{apx:sec:baselines}

We conduct detailed illustrations on the baselines compared with \Ours in Section \ref{sec:main-result} below.

\begin{itemize}[leftmargin=*]
  \item RRHF \citep{yuan2023rrhf} scores responses via a logarithm of conditional probabilities and learns to align these probabilities with human preferences through ranking loss.
  \item SLiC-HF \citep{zhao2023slic} proposes a sequence likelihood calibration method which can learn from human preference data.
  \item DPO \citep{guo2024direct} simplifies the PPO \citep{ouyang2022training} algorithms into an offline direct optimization objective with the pairwise Bradley-Terry assumption.
  \item IPO \citep{azar2024general} theoretically grounds pairwise assumption in DPO into a pointwise reward.
  \item CPO \citep{xu2024contrastive} adds a reward objective with sequence likelihood along with the SFT objective.
  \item KTO \citep{ethayarajh2024kto} adopts the Kahneman-Tversky model and proposes a method which directly maximizes the utility of generation instead of the likelihood of the preferences.
  \item RDPO \citep{park2024disentangling} modifies DPO by including an additional regularization term to disentangle the influence of length.
  \item SimPO \citep{meng2024simpo} further simplifies the DPO objective by using the average log probability of a sequence as the implicit reward and adding a target reward margin to the Bradley-Terry objective.
  \item Iterative DPO \citep{xiong2024iterative} identifies the challenge of offline preference optimization and proposes an iterative learning framework.
\end{itemize}


\section{Experiment settings}\label{apx:sec:main-result-setting}

\subsection{Table \ref{tab:main-performance}}\label{apx:sec:main}

We conduct evaluation on two widely used benchmark: AlpacaEval2 \citep{dubois2024length} and MixEval \citep{ni2024mixeval}.
We consider two base models: Mistral-7b-base and Mistral-7b-it. For Mistral-7b-base, we first conduct supervised finetuning following \citet{meng2024simpo} before the preference optimization.

The performance scores for offline preference optimization baselines are from SimPO \citep{meng2024simpo}.
To have a fair comparison with these baselines, we adopt the same off-the-shelf reward model \citep{jiang2023llm} as in SimPO for the iterative DPO baseline and \Ours.

For the iterative DPO baseline, we generate 2 responses for each prompt, score them with the off-the-shelf reward model and construct the preference pair data to tune the model.

For \Ours (contrastive $\mathcal{L}_{\text{con}}$), we generate 10 responses each iteration and score them with the reward model. The top-1 ranked response and the bottom-3 ranked responses are adopted as the chose response and rejected responses respectively.
Generation temperature is selected as 1 and 0.8 for Mistral-7b-base and Mistral-7b-it respectively (we search it among 0.8, 0.9, 1.0, 1.1, 1.2).

For \Ours (LambdaRank $\mathcal{L}_{\text{lamb}}$), we generate 10 responses each iteration and score them with the reward model. The top-2 ranked response and the bottom-2 ranked responses are adopted as the chose response and rejected responses respectively.
Generation temperature is selected as 1 and 0.8 for Mistral-7b-base and Mistral-7b-it respectively (we search it among 0.8, 0.9, 1.0, 1.1, 1.2).

For \Ours (ListMLE $\mathcal{L}_{\text{lmle}}$), we generate 10 responses each iteration and score them with the reward model. The top-2 ranked response and the bottom-2 ranked responses are adopted as the chose response and rejected responses respectively.
Generation temperature is selected as 1 and 0.8 for Mistral-7b-base and Mistral-7b-it respectively (we search it among 0.8, 0.9, 1.0, 1.1, 1.2).

\Ours can achieve even stronger performance with stronger off-the-shelf reward model \citep{dong2024rlhf}.
% Results with stronger a reward model can be found in Appendix \ref{apx:sec:stronger-rm}.

\subsection{Table \ref{tab:objective}}\label{apx:sec-objective-setting}

We conduct experiments on both Gemma2-2b-it \citep{team2024gemma} and Mistral-7b-it \citep{jiang2023mistral}.
Following \citet{Tunstall_The_Alignment_Handbook} and \citet{dong2024rlhf}, we perform training on UltraFeedback dataset for 3 iterations and show the performance of the final model checkpoint.
We use the pretrained reward model from \citet{dong2024rlhf}.
The learning rate is set as 5e-7 and we train the LLM for 2 epochs per iteration.

For the pairwise objective, we generate 2 responses for each prompt and construct the preference pair data with the reward model.
For the others, we generate 4 responses per prompt and rank them with the reward model.
For the contrastive objective, we construct the 1-vs-N data with the top-1 ranked response and the other responses.
For the listMLE and lambdarank objective, we take the top-2 as positives and the last-2 as the negatives.
Experiments with opensource LLM as the evaluator (\texttt{alpaca\_eval\_llama3\_70b\_fn}) can be found in Table \ref{tab:objective2}.



\begin{table*}[t]
    \centering
    % \renewcommand{\arraystretch}{1.2}
    \caption{Preference optimization objective study on AlpacaEval2 and MixEval. For AlpacaEval2, we report the result with both opensource LLM evaluator \texttt{alpaca\_eval\_llama3\_70b\_fn} and GPT4 evaluator \texttt{alpaca\_eval\_gpt4\_turbo\_fn}. SFT corresponds to the initial chat model.}\label{tab:objective2}
    \small
    \begin{tabular}{llccccccccc}
        \toprule
        & & \multicolumn{2}{c}{AlpacaEval 2 (opensource LLM)} & \multicolumn{2}{c}{AlpacaEval 2 (GPT-4)} & \multicolumn{1}{c}{MixEval} & \multicolumn{1}{c}{MixEval-Hard} \\
         \cmidrule(r){3-4} \cmidrule(r){5-6} \cmidrule(r){7-7} \cmidrule(r){8-8}
        & Method & LC Winrate & Winrate & LC Winrate & Winrate & Score & Score \\
        \midrule
        \multirow{6}{*}{\rotatebox{90}{Gemma2-2b-it}} & SFT & 47.03 & 48.38 & 36.39 & 38.26 & 0.6545 & 0.2980 \\
        \cmidrule{2-8}
        & pairwise & 55.06 & 66.56 & 41.39 & 54.60 & 0.6740 & 0.3375 \\
        & contrastive & 60.44 & 72.35 & 43.41 & 56.83 & 0.6745 & 0.3315 \\
        & ListMLE & 63.05 & 76.09 & 49.77 & 62.05 & 0.6715 & 0.3560 \\
        & LambdaRank & 58.73 & 74.09 & 43.76 & 60.56 & 0.6750 & 0.3560 \\
        \midrule
        \midrule
        \multirow{6}{*}{\rotatebox{90}{Mistral-7b-it}} & SFT & 27.04 & 17.41 & 21.14 & 14.22 & 0.7070 & 0.3610 \\
        \cmidrule{2-8}
        & pairwise & 49.75 & 55.07 & 36.43 & 41.86 & 0.7175 & 0.4105 \\
        & contrastive & 52.03 & 60.15 & 38.44 & 42.61 & 0.7260 & 0.4340 \\
        & ListMLE & 48.84 & 56.73 & 38.02 & 43.03 & 0.7360 & 0.4200 \\
        & LambdaRank & 51.98 & 59.73 & 40.29 & 46.21 & 0.7370 & 0.4400 \\
        \bottomrule
    \end{tabular}
\end{table*}


\subsection{Table \ref{fig:list-study}}\label{apx:sec-list-setting}

We adopt Gemma2-2b-it as the initial model. All the models are trained with iterative DPO for 3 iterations. We use the off-the-shelf reward model \citep{dong2024rlhf}.
We generate 2 responses for each prompt in each iteration.
For ``w. current'', we only use the scored responses in the current iteration for preference optimization data construction.
For ``w. current + prev'', we rank the responses in the current iteration and the previous one iteration, and construct the preference pair data with the top-1 and bottom-1 ranked responses.
For ``w. current + all prev'', we rank all the responses for the prompt in the current and previous iterations and construct the preference pair data.
For ``single temperature'', we only adopt temperature 1 and generate 2 responses for reward model scoring.
For ``diverse temperature'', we generate 2 responses with temperature 1 and 0.5 respective and rank the 4 responses to construct the preference data with the reward model.

\subsection{Table \ref{tb:sft-rlhf-result}}\label{apx:sec:sft-rlhf}

We use mistral-7b-it \citep{jiang2023mistral} as the initial model to alleviate the influence of the math related post-training data of the original model.
% For SFT, we conduct training on the training set of MATH \citep{hendrycks2021measuring} and GSM8K \citep{cobbe2021training} respectively.
For SFT, we conduct training on the meta-math dataset \citep{yu2023metamath}.
For DPO, we use the prompts in the training set of the two dataset and conduct online iterative preference optimization with the binary rule-based reward (measure if the final answer is correct or not with string match). 
The evaluation is performed on the test set of MATH and GSM8K respectively.
% For both SFT and DPO, we conduct careful hyper-parameter search.
For SFT, we follow the same training setting with \citet{yu2023metamath}.
For DPO, we search the learning rate in 1e-7, 2e-7, 5e-7, 2e-8, 5e-8 and train the LLM for 5 iterations with early stop (1 epoch per iteration for MATH and 2 epoch per iteration for GSM8K). The learning rate is set as 1e-7 and we select the checkpoint after the first and fourth iteration for GSM8K and MATH respectively.

\subsection{Figure \ref{fig:merge-study}(a)}\label{apx:sec-hard-neg-setting}

We conduct training with the prompts in the training set of GSM8K and perform evaluation on GSM8K testing set.
We conduct learning rate search and finalize it to be 2e-7.
The learning is performed for 3 iterations.

We make explanations of how we construct the four types of negative settings:
For (1) a random response not related to the given prompt, we select a response for a random prompt in Ultrafeedback.
For (2) a response to a related prompt, we pick up a response for a different prompt in the GSM8K training set.
For (3) an incorrect response to the given prompt with high temperature, we select the temperature to be 1.
For (4) an incorrect response to the given prompt with low temperature, we select the temperature to be 0.7.

\begin{figure}[t]
\centering
\includegraphics[scale=0.4]{figure/LLM_alignment_gemma_temperature_study.pdf}
\vskip -1em
\caption{Training temperature study with $\mathcal{L}_{\text{pair}}$ on Gemma2-2b-it and Alpaca Eval 2. Within a specific range ($>$ 0.9), lower temperature leads to harder negative and benefit the trained LLM. However, temperature lower than this range can cause preferred and rejected responses non-distinguishable and lead to degrade training.}\label{apx:tab:temp-hard}
\end{figure}

\subsection{Figure \ref{fig:merge-study}(b)}\label{apx:sec-hard-neg-setting-temp}

We conduct experiments on both Gemma2-2b-it and Mistral-7B-it models.
For both LLMs, we conduct iterative DPO for 3 iterations and report the performance of the final model.
We perform evaluation on Alpaca Eval2 with \texttt{alpaca\_eval\_llama3\_70b\_fn} as the evaluator.

For temperature study, we find that under a specific temperature threshold, repeatedly generated responses will be large identical for all LLMs and cannot be used to construct preference data, while the threshold varies for different LLMs.
% As a result, we select temperatures above the threshold for robust experiments.
The ``low'' and ``high'' refer to the value of those selected temperatures.
% For Gemma2-2b-it, we use temperature as 0.2, 0.5 and 0.7 to generate the responses, score the responses by the reward model and train the LLM with the newly labeled data.
% For Mistral-7b-it, we set the temperature as 1, 1.1 and 1.2 respectively.
We also conduct experiments on Gemma2-2b-it model and show the results in Figure \ref{apx:tab:temp-hard}.


\subsection{Figure \ref{fig:merge-study}(c)}\label{apx:sec-length-setting}

We adopt Mistral-7b-it as the initial LLM and the contrastive objective (Eq. \ref{eq:contrastive}) in iterative preference optimization.
We generate 4/6/8/10 responses with the LLM and score the responses with the off-the-shelf reward model \citep{dong2024rlhf}.
The top-1 scored response is adopted as the positive response and the other responses are treated as the negative responses to construct the 1-vs-N training data.
The temperature is set as 1 to generate the responses.


% \newpage
% \section{\Ours with a stronger reward model}\label{apx:sec:stronger-rm}

% In Section \ref{sec:lrpo}, we show the results with LLM-Blender \citep{jiang2023llm} as the reward model to have a fair comparison with the baseline methods.
% In this section, we would like to show that \Ours can achieve even stronger performance with stronger off-the-shelf reward model \citep{dong2024rlhf}.
% The results are shown in Table \ref{apx:tab:main-performance}, where we can find that a stronger reward model can further improve the performance of \Ours.


% \begin{table*}[h]
%     \centering
%     \caption{Method evaluation on AlpacaEval 2 and MixEval. LC WR and WR denote length-controlled win rate and win rate respectively. Offline baseline performances on AlpacaEval 2 are from \citept{meng2024simpo} with LLM-Blender reward model \citep{jiang2023llm}.}\label{apx:tab:main-performance}
%     \scalebox{0.78}{
%     \begin{tabular}{lcccccccccc}
%         \toprule
%         Model & \multicolumn{4}{c}{Mistral-Base (7B)} & \multicolumn{4}{c}{Mistral-Instruct (7B)} \\
%         \cmidrule(lr){2-5} \cmidrule(lr){6-9}
%         & \multicolumn{2}{c}{Alpaca Eval 2}  & \multirow{1}{*}{MixEval} & \multirow{1}{*}{MixEval-Hard} & \multicolumn{2}{c}{Alpaca Eval 2}  & \multirow{1}{*}{MixEval} & \multirow{1}{*}{MixEval-Hard} \\
%         \cmidrule(lr){2-3} \cmidrule(lr){4-4} \cmidrule(lr){5-5}  \cmidrule(lr){6-7} \cmidrule(lr){8-8} \cmidrule(lr){9-9}
%         & LC WR & WR  & Score & Score & LC WR  & WR & Score & Score \\
%         \midrule
%         SFT    & 8.4  & 6.2   &  0.602  & 0.279  & 17.1 & 14.7  & 0.707 & 0.361 \\
%         RRHF   & 11.6 & 10.2   &  0.600  & 0.312  & 25.3 & 24.8  &   0.700    & 0.380 \\
%         DPO    & 15.1 & 12.5  &  0.686  &  0.341 & 26.8 & 24.9  & 0.702 & 0.355 \\
%         KTO    & 13.1 & 9.1    & \textbf{0.704}  & 0.351   & 24.5 & 23.6  &   0.692    & 0.358 \\
%         RDPO   & 17.4 & 12.8  & 0.693  & 0.355   & 27.3 & 24.5  &   0.695    & 0.364 \\
%         SimPO  & 21.5 & 20.8  &  0.672  &  0.347 & 32.1 & 34.8  & 0.702  & 0.363 \\
%         Iterative DPO  & 18.9  & 16.7  & 0.660   & 0.341  & 20.4 & 24.84  & 0.719  & 0.389 \\
%         \midrule
%         \multicolumn{9}{c}{Reward model: LLM-Blender \citep{jiang2023llm}}  \\
%         \midrule
%         \Ours (contrastive) & 31.6 & 30.8  &   0.703 & 0.409  & 32.7 & 38.6  &  0.718 & \textbf{0.418} \\
%         \Ours (LambdaRank) &  \textbf{34.9} & \textbf{37.2} & 0.695 &  \textbf{0.452}  & \textbf{32.9} & \textbf{38.9}   & \textbf{0.720} & 0.417  \\
%         \Ours (ListMLE) & 31.1  &  32.1   &  0.669  & 0.390  &  29.7 & 36.2    & 0.709  & 0.397 \\
%         \midrule
%         \multicolumn{9}{c}{Reward model: FsfairX \citep{dong2024rlhf}}  \\
%         \midrule
%         \Ours (contrastive) & \textbf{41.5} & \textbf{42.9} & 0.718 & 0.417    & \textbf{43.0}  & \textbf{53.8} & 0.718 & 0.425   \\
%         \Ours (LambdaRank) & 35.8 & 34.1 & 0.717 & 0.431   & 41.9  & 48.1 & \textbf{0.740} & \textbf{0.440}  \\
%         \Ours (ListMLE) & 36.6 & 37.8 & \textbf{0.730} & \textbf{0.423}   & 39.6  & 48.1 & 0.717 & 0.397   \\
%         \bottomrule
%     \end{tabular}}
% \end{table*}

%\input{9.libraries}
%%%%%%%%%%%%%%%%%%%%%%%%%%%%%%%%%%%%%%%%%%%%%%%%%%%%%%%%%%%%%%%%%%%%%%%%%%%%%%%%
\end{document}

