\section{Related Work}
Gaussian Process (GP) is widely used for estimating scalar fields due to its ability to handle complex relationships between sensor measurements and the inherent uncertainty of the data~\citep{wang2017online, guerrero2021sparse}. Specifically, the GP leverages kernel functions to estimate relationships within the data, which offers the advantage of enabling online estimation for time-varying unknown scalar fields when appropriate kernel functions are employed. However, these classic GP models with the centralized framework face challenges in effectively managing a large amount of data from WSNs due to their computational complexity. Moreover, all wireless sensors need to transmit data to a single central node for estimating the unknown scalar field, leading to high communication burdens among nodes. Although the use of communication relay to enhance the communication bandwidth is a known solution, it still consumes considerable time and may not be robust against communication failures~\citep{manfredi2013design}. As a result, significant efforts have been made to develop a distributed Gaussian Process (DGP) framework to address challenges in scalability of WSNs.
\cite{deisenroth2015distributed} proposed a hierarchical DGP approach using the kernel-based GP, which focused primarily on reducing computational complexity. However, this method still struggles with the increasing size of the measurements and it requires an additional computational step of re-estimation whenever the map resolution changes.

Recently, a new type of GP approach has emerged, which employs finite dimensional approximation techniques to reduce the computational burden of the kernel-based GP by approximating the kernel function with a sum of eigenfunction and eigenvalue product~\citep{8387507, solin2014hilbert}. This advancement has transformed the kernel-based classic GP framework into a basis-function-based framework, paving the way for efficient DGP algorithms. Notably in~\citep{9144385}, a basis-function-based GP combined with the average consensus algorithm, multi-agent DGP (MADGP), is applied in distributed multi-agent systems. In this approach, the transmitted data increases quadratically with $\mathcal{O}(E^2)$, where $E$ represents the number of functions used to approximate the kernel. This relationship changes the computational complexity and communication load dependence of the GP from the number of sensor measurements to the number of functions $E$. However, accurately approximating the kernel requires a large number of eigenfunctions, which still imposes a significant computational and communication load. Furthermore, since finding an analytic form of basis functions for a time-varying kernel is challenging, it is difficult to apply this approximated DGP algorithm in estimating dynamic scalar fields.

As an alternative approach to reduce the computational burden of the GP dealing with all collected measurement data up to the current time step, a new approach called Kalman filter-based GP (K-GP) (or similarly recursive GP) method is introduced~\citep{solin2018modeling}. This method allows for the sequential estimation of unknown scalar fields within the Bayesian framework. Unlike the conventional GP, which relies on all historical data, the K-GP only requires data from the current time step, utilizing the Kalman filter's ability to process past information based on the Markov property. However, it has been only applied in single agent scenarios~\citep{8880505, viset2022extended}.
%~\citep{osborne2010bayesian, huber2014recursive}. 
Besides, a simple extension of the K-GP to multiple agents could result in the high computational and communication complexities found in existing DGP methods.