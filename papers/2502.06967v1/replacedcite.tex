\section{Prior Works}
\label{Section: Prior Works}
Recently, CAPA has attracted growing research interest. The performance of CAPA-based communications has been extensively studied in terms of spatial DoFs ____, array gains ____, and channel capacity ____, with most of these studies focusing on line-of-sight (LoS) channels. To address this limitation, researchers have developed a novel multipath spatial fading model for CAPA ____. Using this model, the work in ____ analyzed the spatial correlation and the diversity-multiplexing trade-off in CAPA-based communications. These studies have collectively validated CAPA's effectiveness in enhancing spatial DoFs and array gains, thereby providing a theoretical foundation for its application.

Another key research area is continuous beamforming design for CAPA-based communications. Since optimizing a continuous source current function involves challenging fractional programming, various approaches have been proposed. For example, a discretization-based approach (using Fourier series expansion) has been proposed to approximate continuous signals, which simplified the optimization problem ____. Additionally, a calculus of variations method was introduced to design low-complexity beamforming ____ and explore optimal beamforming structures ____.

In addition to its role in communications, CAPA has also been applied to enhance sensing performance. The studies in ____ and ____ derived Cram{\'e}r-Rao bounds (CRBs) and Ziv-Zakai bounds (ZZBs) for CAPA-based positioning. Building on these advancements, ____ addressed multi-target sensing and proposed low-complexity beamforming methods to improve CRBs. Initial efforts in CAPA-based ISAC have also been reported. For example, a Fourier series expansion-based beamforming approach was proposed to balance the communication signal-to-interference-plus-noise ratio (SINR) and sensing SINR in a downlink CAPA-based ISAC system ____.