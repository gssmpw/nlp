%%%%%%%% ICML 2025 EXAMPLE LATEX SUBMISSION FILE %%%%%%%%%%%%%%%%%

% \documentclass{article}
\documentclass{article}
% Recommended, but optional, packages for figures and better typesetting:
\usepackage{microtype}
\usepackage{graphicx}
\usepackage{subfigure}
\usepackage{booktabs} % for professional tables
\usepackage{hyperref}
% Attempt to make hyperref and algorithmic work together better:
\newcommand{\theHalgorithm}{\arabic{algorithm}}

% Use the following line for the initial blind version submitted for review:

% If accepted, instead use the following line for the camera-ready submission:
\usepackage[accepted]{icml2025}
\usepackage{amsmath}
\usepackage{mathrsfs}
\usepackage{amssymb}
\usepackage{mathtools}
\usepackage{amsthm}
\usepackage{url}            % simple URL typesetting
\usepackage{bm}
\usepackage{booktabs}       % professional-quality tables
\usepackage{amsfonts}       % blackboard math symbols
\usepackage{nicefrac}       % compact symbols for 1/2, etc.
\usepackage{microtype}      % microtypography
\usepackage{adjustbox}
\usepackage{blindtext}
\usepackage{enumitem}
\usepackage{listings}
\usepackage{pifont}
\usepackage{epsfig} % for postscript graphics files
\usepackage{times}
\usepackage{color}
% \usepackage[table]{xcolor}
\usepackage{multirow}
\usepackage{multicol}
\usepackage{xcolor}         % colors
\usepackage{makecell}
\usepackage{algorithm}
\usepackage{rotating}
\usepackage{subfigure}
\usepackage[T1]{fontenc}
\usepackage{algorithm}
\usepackage{algorithmic}
% \usepackage[noend]{algpseudocode}
\usepackage{colortbl}
\usepackage{color, soul}
\usepackage{arydshln} % For dashline
\usepackage{tcolorbox}
\definecolor{babyblue}{rgb}{0.54, 0.81, 0.94}
\definecolor{bisque}{rgb}{1.0, 0.89, 0.77}
\definecolor{bshade}{rgb}{0.55,0.75,0.95}
\def\eg{{\it e.g.}\xspace}
\def\ie{{\it i.e.}\xspace}
\definecolor{mygray}{gray}{.6}
\definecolor{myblue}{RGB}{89,158,254}
\definecolor{mygreen1}{RGB}{81,150,111}
\definecolor{mygreen2}{RGB}{93,174,86}
\definecolor{myred}{RGB}{160,0,0}
\definecolor{myyellow}{RGB}{227,207,87}
% For pictures
\usepackage{caption}
\usepackage[capitalize,noabbrev]{cleveref}
\usepackage{threeparttable}
\usepackage{wrapfig}

%%%%%%%%%%%%%%%%%%%%%%%%%%%%%%%%
% THEOREMS
%%%%%%%%%%%%%%%%%%%%%%%%%%%%%%%%
\theoremstyle{plain}
\newtheorem{theorem}{Theorem}[section]
\newtheorem{proposition}[theorem]{Proposition}
\newtheorem{lemma}[theorem]{Lemma}
\newtheorem{corollary}[theorem]{Corollary}
\theoremstyle{definition}
\newtheorem{definition}[theorem]{Definition}
\newtheorem{assumption}[theorem]{Assumption}
\theoremstyle{remark}
\newtheorem{remark}[theorem]{Remark}
\definecolor{babyblue}{rgb}{0.54, 0.81, 0.94}
\definecolor{bisque}{rgb}{1.0, 0.89, 0.77}
\definecolor{bshade}{rgb}{0.55,0.75,0.95}
\def\eg{{\it e.g.}\xspace}
\def\ie{{\it i.e.}\xspace}
\definecolor{mygray}{gray}{.6}
\definecolor{myblue}{RGB}{89,158,254}
\definecolor{mygreen1}{RGB}{81,150,111}
\definecolor{mygreen2}{RGB}{93,174,86}
\definecolor{myred}{RGB}{160,0,0}
\definecolor{myyellow}{RGB}{227,207,87}
\usepackage{bbding}
\newcommand{\cmark}{\ding{52}}%
\newcommand{\xmark}{\ding{55}}%
\newcommand{\mA}{\boldsymbol{A}}
\newcommand{\mE}{\boldsymbol{E}}
\newcommand{\mF}{\boldsymbol{F}}
\newcommand{\mH}{\boldsymbol{H}}
\newcommand{\mI}{\boldsymbol{I}}
\newcommand{\mM}{\boldsymbol{M}}
\newcommand{\mO}{\boldsymbol{O}}
\newcommand{\mP}{\boldsymbol{P}}
\newcommand{\mQ}{\boldsymbol{Q}}
\newcommand{\mR}{\boldsymbol{R}}
\newcommand{\mX}{\boldsymbol{X}}
\usepackage{enumitem}
\let\oldding\ding% Store old \ding in \oldding
\renewcommand{\ding}[2][1]{\scalebox{#1}{\oldding{#2}}}
% Todonotes is useful during development; simply uncomment the next line
%    and comment out the line below the next line to turn off comments
%\usepackage[disable,textsize=tiny]{todonotes}
\usepackage[textsize=tiny]{todonotes}
\newcommand{\ceil}[1]{\left\lceil #1 \right\rceil}
\newcommand{\floor}[1]{\left\lfloor #1 \right\rfloor}
\newcommand{\round}[1]{\left\lfloor #1 \right\rceil}
\newcommand{\sign}{\mathrm{sign}}
\newcommand{\tofloat}[1]{\mbox{float}\left(#1\right)}
\newcommand{\toint}[2]{\mbox{int}_{#1}\left(#2\right)}

\newcommand{\methodname}{MQuant}
\newcommand{\cc}{\color[rgb]{0,0.6,0.3}\checkmark}
\newcommand{\xx}{\color[rgb]{0.6,0,0}{\ding{55}}}

% The \icmltitle you define below is probably too long as a header.
% Therefore, a short form for the running title is supplied here:
% \icmltitlerunning{Submission and Formatting Instructions for ICML 2025}

\begin{document}

\twocolumn[
\icmltitle{MQuant: Unleashing the Inference Potential of Multimodal Large Language Models via Full Static Quantization}
%maybe need change our title:


% It is OKAY to include author information, even for blind
% submissions: the style file will automatically remove it for you
% unless you've provided the [accepted] option to the icml2025
% package.

% List of affiliations: The first argument should be a (short)
% identifier you will use later to specify author affiliations
% Academic affiliations should list Department, University, City, Region, Country
% Industry affiliations should list Company, City, Region, Country

% You can specify symbols, otherwise they are numbered in order.
% Ideally, you should not use this facility. Affiliations will be numbered
% in order of appearance and this is the preferred way.
\icmlsetsymbol{equal}{$\ast$}
\icmlsetsymbol{intern}{$\dagger$}
\icmlsetsymbol{corre}{$\Diamond$}

\begin{icmlauthorlist}
\icmlauthor{JiangYong Yu}{houmo,equal}
\icmlauthor{Sifan Zhou}{houmo,seu,equal,intern}
\icmlauthor{Dawei Yang}{houmo,corre}
\icmlauthor{Shuo Wang}{houmo} 
\icmlauthor{Shuoyu Li}{houmo,xjtu,intern} \\
\icmlauthor{Xing Hu}{houmo} 
\icmlauthor{Chen Xu}{houmo}
% \icmlauthor{}{sch}
\icmlauthor{Zukang Xu}{houmo}
\icmlauthor{Changyong Shu}{houmo}
\icmlauthor{Zhihang Yuan}{houmo}
\end{icmlauthorlist}

\icmlaffiliation{houmo}{Houmo AI}
\icmlaffiliation{seu}{Southeast University}
\icmlaffiliation{xjtu}{Xi'an Jiaotong University}

\icmlcorrespondingauthor{Dawei Yang}{}
% You may provide any keywords that you
% find helpful for describing your paper; these are used to populate
% the "keywords" metadata in the PDF but will not be shown in the document
% \icmlkeywords{Machine Learning, ICML}

\vskip 0.3in
]

\printAffiliationsAndNotice{\corresponding,\icmlEqualContribution,\intern}
% \printAffiliationsAndNotice{\intern}% otherwise use the standard text.
\begin{abstract}

% Recent works to jointly reconstruct 3D human and object from a single RGB image, are mostly model-based, that fail to capture the fine details of the clothed human body and object surface. In this paper, we introduce ReCHOR, a novel, model-free, first-method to produce realistic clothed human-object reconstructions from a monocular view. This is extremely challenging due to human-object occlusions, diverse interactions and depth ambiguity, as it needs to infer both 3D spatial awareness and high resolution details. Our core idea is based on estimating neural implicit representations for human and object respectively by an attention-based neural implicit model that attends to pixel-aligned features from both the global human-object image for spatial awareness and  the local separate view of human and object images for high quality details. Additionally, the network is conditioned on semantic features from an initial estimated human-object pose prior and a generative diffusion model that inpaints occluded regions, thus enabling the retrieval of details from them.
% We also propose a synthetic dataset with rendered scenes of diverse, inter-occluded 3D human and object scans, to train our network. We evaluate our method on the synthetic and real world BEHAVE dataset. Our experiments show that our method outperforms the SOTA in achieving realistic clothed human-object reconstructions.
Recent approaches to jointly reconstruct 3D humans and objects from a single RGB image represent 3D shapes with template-based or coarse models, which fail to capture details of loose clothing on human bodies. In this paper, we introduce a novel implicit approach for jointly reconstructing realistic 3D clothed humans and objects from a monocular view. For the first time, we model both the human and the object with an implicit representation, allowing to capture more realistic details such as clothing. This task is extremely challenging due to human-object occlusions and the lack of 3D information in 2D images, often leading to poor detail reconstruction and depth ambiguity. To address these problems, we propose a novel attention-based neural implicit model that leverages image pixel alignment from both the input human-object image for a global understanding of the human-object scene and from local separate views of the human and object images to improve realism with, for example, clothing details. Additionally, the network is conditioned on semantic features derived from an estimated human-object pose prior, which provides 3D spatial information about the shared space of humans and objects. To handle human occlusion caused by objects, we use a generative diffusion model that inpaints the occluded regions, recovering otherwise lost details. For training and evaluation, we introduce a synthetic dataset featuring rendered scenes of inter-occluded 3D human scans and diverse objects. Extensive evaluation on both synthetic and real-world datasets demonstrates the superior quality of the proposed human-object reconstructions over competitive methods.
\end{abstract}
\section{Introduction}
\label{sec:intro}
% Image editing methods in diffusion models depend on user-defined control directions - users can unlock their creativity using these methods by specifying the desired manipulation through prompts~\cite{gandikota2023concept}, reference images~\cite{ruiz2022dreambooth, kumari2022customdiffusion, gal2022image, chen2024trainingfreeregionalpromptingdiffusion}, or attribute vectors~\cite{parmar2023zero,hertz2022prompt}. In this work, we ask a fundamentally different question: \emph{Can we automatically discover the underlying visual structure of a concept within diffusion model's knowledge?} %Rather than requiring user-specified controls, we aim to decompose the model's internal knowledge into meaningful directions.

% This question touches on a fundamental limitation in how we interact with diffusion models. Current control methods ~\cite{zhang2023addingconditionalcontroltexttoimage, gandikota2023concept, ye2023ipadaptertextcompatibleimage,ye2023ipadaptertextcompatibleimage, hertz2024stylealignedimagegeneration, li2023photomaker, shi2024instantbooth, chen2024trainingfreeregionalpromptingdiffusion} require users to specify their desired manipulations in advance, limiting interactive creativity. This contrasts with natural human artistic workflows, where creators dynamically explore creative ideas while jointly refining them toward meaningful artistic outcomes~\cite{hoffmann2016modeling}. This synergy between specification and exploration is not new to generative models. Early GAN architectures naturally developed disentangled latent spaces that enabled continuous\cite{harkonen2020ganspace,radford2015unsupervised, wu2021stylespace, shen2020interfacegan}, compositional control over generated images. Users could explore these spaces to discover interesting variations that would be difficult to describe in words~\cite{wu2021stylespace}, then combine them to achieve their creative goals~\cite{grabe2022towards}. 


% While diffusion models have largely superseded GANs in conditional image synthesis~\cite{dhariwal2021diffusion},  their underlying structure remains less understood. Diffusion models achieve remarkable diversity through high-dimensional latents, unlike GANs' compact latent spaces.  With a single prompt, diffusion models can generate radically different variations through different random initializations of input noise. We ask - Is it possible to discover interpretable structure within this vast space of variations?

Text-to-image diffusion models are capable of generating remarkable visual variations from a single prompt through different random initializations. However, this vast creative potential remains largely opaque to users---while we can generate diverse images, we lack understanding of the underlying structure of these variations. This presents a fundamental challenge: how can we discover and expose the latent visual capabilities encoded within these models?

\let\thefootnote\relax \footnote{$^{*}$Correspondence to \texttt{gandikota.ro@northeastern.edu}}

The challenge touches on a key limitation in how we interact with diffusion models today. Current control methods require users to explicitly specify their desired edits in advance through prompts~\cite{gandikota2023concept}, reference images~\cite{zhang2023addingconditionalcontroltexttoimage, chen2024trainingfreeregionalpromptingdiffusion, ruiz2022dreambooth,kumari2022customdiffusion, Ryu_lora, hu2021lora}, or attribute vectors~\cite{ye2023ipadaptertextcompatibleimage, hertz2024stylealignedimagegeneration, li2023photomaker, shi2024instantbooth,parmar2023zero,hertz2022prompt}. That contrasts sharply with natural human creative workflows, where artists dynamically explore creative ideas and jointly refine them toward meaningful artistic outcomes~\cite{hoffmann2016modeling}. The need for pre-specified controls creates a barrier between users and the full creative potential of these models.

Interestingly, earlier generative models like GANs~\cite{gans,karras2019style,brock2018large} naturally developed more interpretable internal structures. Their compact latent spaces often exhibited emergent disentanglement~\cite{harkonen2020ganspace,radford2015unsupervised, wu2021stylespace, shen2020interfacegan}, enabling continuous and compositional control over generated images. Users could explore these spaces to discover interesting variations that would be difficult to describe in words~\cite{wu2021stylespace}, then combine them to achieve their creative goals~\cite{grabe2022towards}.

Diffusion models have largely superseded GANs in conditional image synthesis~\cite{dhariwal2021diffusion}, achieving greater diversity through much higher-dimensional latents. And yet an understanding of the underlying structure of these larger latent spaces has remained elusive. In this work, we ask a fundamental question: \emph{Can we automatically discover the visual structure within a diffusion model's knowledge of a concept?} Rather than requiring user-specified controls, we aim to decompose the model's internal representations into expressive directions that users can explore and combine.

To address these needs, we present \textbf{SliderSpace}, a framework that brings systematic explorability to diffusion models. Given just a text prompt, SliderSpace discovers a canonical set of meaningful, diverse, and controllable directions within the model's knowledge of that concept. Each direction is implemented as a low-rank adapter~\cite{hu2021lora} that can be scaled and composed with others, allowing users to explore and smoothly combine different aspects of variation, as shown in Figure~\ref{fig:intro}.

We ground SliderSpace discovery in three key requirements for meaningful decomposition of a diffusion model's visual manifold: 
\begin{enumerate}
    \item \textbf{Unsupervised Discovery:} The decomposition process should emerge from the intrinsic structure of the model's learned representation, rather than being guided by predefined attributes. This ensures we capture the true topology of the model's knowledge space rather than projecting our assumptions onto it.
    
    \item \textbf{Semantic Orthogonality:} Each discovered control must represent a distinct semantic direction. This is enforced in a semantic feature space, like CLIP, where every slider has an orthogonal effect in embeddings. This prevents discovering multiple controls that create similar semantic effects, making the system more efficient and easier.
    
    \item \textbf{Distribution Consistency:} Directions must induce consistent transformations across both random seeds and prompt variations. 
\end{enumerate}

These requirements naturally lead to our proposed framework, which we formalize in Section~\ref{sec:method}. As we show in our experiments, SliderSpace is architecture-agnostic, working with both conventional U-Net based models like Stable Diffusion~\cite{rombach2022high, rombach2022sd20, podell2023sdxl, turbo, dmd} and recent transformer-based architectures like Flux~\cite{flux}.

We demonstrate the expressiveness of SliderSpace through three applications: First, we show how SliderSpace can decompose high-level concepts into diverse and expressive components, revealing the natural axes of variation in the model's understanding. Second, we explore artistic style variation, where SliderSpace discovers directions that match or exceed the diversity of manually curated artist lists while being judged more useful by human evaluators. Finally, we show how SliderSpace can help reverse the mode collapse commonly observed in distilled diffusion models, restoring diversity while maintaining generation speed.

Beyond providing practical creative control, SliderSpace opens new avenues for understanding and utilizing the latent capabilities of diffusion models. By mapping these models' visual potential into intuitive, composable directions, we take a step toward making their creative possibilities more accessible and interpretable to users.

% Image editing methods in diffusion models unlock the creativity of users. In this work we ask an alternate question: \emph{Can we organize and expose what of the diffusion model is already capable of?}.
% Existing methods for controlling image generation typically require users to manually specify edit directions for desired changes. This process is time-consuming, requires technical expertise, and limits the spontaneity of the creative process. For instance, if a user wants to adjust the smile of a generated person, they must explicitly request this edit, often through imprecise prompt engineering or model fine-tuning. This approach of predefined controls or manual specifications restricts users from fully exploring the latent capabilities of the model. There may be interesting stylistic variations or attributes that the model can generate, but users have no easy way to discover or utilize these.

% Natural visual disentanglement was an emergent property in the latent space of Generative Adversarial Models (GANs) \cite{harkonen2020ganspace,radford2015unsupervised, wu2021stylespace, shen2020interfacegan}. In particular, it has been observed that StyleGAN~\cite{karras2019style} stylespace neurons offer detailed control over many meaningful aspects of images that would be difficult to describe in words~\cite{wu2021stylespace}. However, diffusion models do not share such a compact latent space~\cite{park2023unsupervised}; and efforts to uncover such a space in the semantic embeddings of the text conditioning have met with limited success \nik{Nick - is there a specific citation you were thinking about?}.

% In this work we introduce \textbf{SliderSpace}, which takes a step towards uncovering an analogous low dimensional representation of diffusion models' visual breadth; in essence treating the diffusion model as many generators sharing parameters, where a particular generator is defined by a specific prompt. For a given prompt we sample many random seeds (and optionally prompt expansions using an LLM), generate the corresponding images, and apply an off the shelf feature extractor (in this work CLIP, but our method can be applied to any differentiable feature extractor). We use PCA to analyze these features, and for each of the leading $k$ principal components we train a LoRA \cite{} which causes the diffusion model to produces images which increase the feature magnitude along that component when passed back through the same feature extractor. This leads to a 'Slider' for each principal component, because each LoRA can be scaled and applied to the original diffusion model, continuously varying those visual features in the generated results (as measured, in our case, by CLIP).

% There are many other works that enhance the controllability of diffusion models. One common approach is enabling users to add spatial constraints to a generation either manually, or via a reference image \cite{zhang2023addingconditionalcontroltexttoimage, chen2024trainingfreeregionalpromptingdiffusion}, a second is leveraging more abstract embeddings (e.g. identity, style) extracted from a reference image \cite{ye2023ipadaptertextcompatibleimage, hertz2024stylealignedimagegeneration, li2023photomaker, shi2024instantbooth}, a third is finetuning a foundation model to better generate a concept important to the user \cite{ruiz2022dreambooth, kumari2022customdiffusion, Ryu_lora, hu2021lora}, and a fourth (most relevant to this work) is finding low-rank adaptors of the model based on a prompt or small training set which can be scaled to provide continous control over one aspect of generated image (e.g. night vs day, basic vs luxury, etc.) \cite{gandikota2023concept}. SliderSpace is complementary to all of these methods and offers something distinct. All of the other methods we are aware require the user (and / or model designer) to know in advance what type of control they want. In contrast SliderSpace assists users in discovering and controlling hidden capabilities present in the diffusion model's distribution of possible generations.

%We propose that truly intuitive creative control in a text-to-image model should meet three key criteria: \emph{discoverability}, \emph{intuitiveness}, and \emph{specificity}. The model should reveal controllable attributes that may not be immediately obvious, offer controls that are easy to understand and manipulate, and ensure each control affects a distinct attribute of the generated image.

% We demonstrate the utility and power of SliderSpace using three applications built on top of SDXL-DMD \cite{dmd}, because its fast generation speed lends itself well to the continuous control offered by SliderSpace.

% First, we study concept decomposition (Section \ref{sec:concept_exp}), where we learn sliders for a specific concept (e.g. 'monster', 'waterfall', 'car'). Through quantitative metrics of diversity and text alignment we demonstrate that the learned sliders dramatically boost the diversity of generations when randomly applied without harming text alignment; we also ask humans to qualitatively judge these results in a user study where they find the SliderSpace results to be more 'Diverse', 'Useful', and 'Creative' than our baselines.

% Second, we attempt to compare the automatic discoveries of SliderSpace to a large scale manual study of artistic styles (Section \ref{sec:art_exp}), open-sourced by ParrotZone \cite{parrotzone}. In this study SDXL was prompted with over 4300 artist names,  and based on visual inspection the cases of successful stylistic mimicry recorded. Quantitatively SliderSpace more closely matches the distribution of artistic variation discovered by ParrotZone than other baselines, and in our user studies was judged to be significantly more 'Diverse' and 'Useful' than the baselines. To our surprise humans even judged SliderSpace results to be slightly more 'Diverse' than the results generated by the manually discovered artist names of \cite{parrotzone}.

% Third, we attempt to use SliderSpace to reverse the mode collapse commonly observed in distilled few-step diffusion models relative to the original teacher model (Section \ref{sec:diverse_exp}). We quantitatively demonstrate that applying SliderSpace to SDXL-DMD leads to more closely matching the distribution of images by the original teacher, SDXL.

%Through extensive experiments on various state-of-the-art text-to-image models, we demonstrate that SliderSpace significantly enhances user control and creative expression in AI-assisted image generation tasks. Our method enables a range of applications, including concept decomposition and control, diversity improvement in generated images, customization dissection and edits, and the exploration of artistic styles inherent in the model.

% SliderSpace goes beyond providing a practical tool for enhanced creative control. By mapping the visual potential of diffusion models it can open new avenues for generative creativity and deepens our understanding of each model's hidden potential.
\section{Preliminaries}
\noindent \textbf{Autoregressive Language Modeling}.
Language provides a versatile way to represent tasks, data, inputs, and outputs, all as a sequence of tokens. Autoregressive language modeling is the basis for LLMs like GPT~\citep{gpt2,gpt3}. This approach predicts the probability of a sequence of words or tokens, with each prediction conditioned on the previous elements in the sequence. 

Formally, given a language token sequence $\vx = (x_1, x_2, \cdots, x_n)$, autoregressive language modeling decomposes the joint distribution of the sequence as the product of a series of conditional probabilities: $p(\vx) = \prod_{i=1}^n p(x_i|x_1, ..., x_{i-1})$,
where $p(x_1|x_0) = p(x_1)$ is the marginal probability. With the factorized distribution and a parameterized model (e.g., Transformers~\cite{vaswani2017attention}), the parameterized distribution $p_{\theta}(x)$ can be optimized via minimizing the negative log-likelihood loss:
\begin{equation}\label{eqn:loss-ar}
    \gL(\theta) = -\log p_{\theta}(\vx) = - \sum\limits_{i=1}^n \log p(x_i|x_{1},\cdots, x_{i-1}).
\end{equation}

\vspace{3pt}
\noindent \textbf{Query-based Data Analytic Tasks}. This paper focuses on query-based data analytic tasks, represented as $\{task, data, query, answer\}$. Given a task description, the data to be analyzed, and a natural language query, the goal is to predict the answer, i,e, $p(answer|task,$ $data, query)$. For example, we can format the input text for a table selection task as: "$\texttt{find tables},$ $\texttt{table schema},~$ $\texttt{who was the}$ $\texttt{only athlete...}$". Then, the model is expected to return the table name(s) that can answer the question, e.g., "$\texttt{<tables>Final\_1},$ $\texttt{Athletics\_1</tables>}$".

\vspace{3pt}

\noindent \textbf{Supervised Instruction Tuning} is a critical fine-tuning process employed to enhance the performance of LLMs on specific tasks by leveraging labeled datasets. Supervised instruction tuning focuses on adapting the model to follow explicit instructions and produce task-specific outputs. This process involves training the model on input-output pairs, where the inputs are typically natural language instructions or prompts, and the outputs are the desired responses or completions. The loss function of supervised instruction tuning is computed only on the "output" tokens to optimize the ability to execute specific tasks and understand instructions.
\section{Causal IL as CMRs}\label{sec:method}

In this section, we demonstrate that performing causal IL in our framework is possible using trajectory histories as instruments. In the next step, we show that the problem can be described as CMRs and propose an effective algorithm to solve it.

The typical target for IL would be the expert policy $\pi_E$ itself. However, since the expert has access to information, namely $u^o_t$, which the imitator does not, the best thing an imitator can do is to learn a history-dependent policy $\pi_h$ that is the closest to the expert. A natural choice is the conditional expectation of $\pi_E(s_t,u^o_t)$ on the history $h_t$:
\begin{align}
\pi_h(h_t)\coloneqq \expectE_{\probP(u^o_t\mid h_t)}[\pi_E(s_t,u^o_t)]=\expectE[\pi_E(s_t,u^o_t)\mid h_t],\nonumber
\end{align}
% where $p(u^o_t\mid h_t)$ is a distribution over expert-observable confounders and captures the information about $u^o_t$ can be inferred from the trajectory history. 
because the conditional expectation minimizes the least squares criterion~\citep{hastie01statisticallearning} and $\pi_h$ is the best predictor of $\pi_E$ given $h_t$. In $\pi_h$, the distribution $\probP(u^o_t\mid h_t)$ captures the information about $u^o_t$ that can be inferred from trajectory histories.
\begin{remark}
\emph{Learning $\pi_h$ is not trivial. Policies learnt naively using behaviour cloning (i.e., $\expectE[a_t\mid h_t]$) fail to match $\pi_E$. In view of~\cref{eq:action}, we have that
\begin{align} 
\expectE[a_t\mid h_t]&=\expectE[\pi_E(s_t,u^o_t) \mid h_{t}]+\expectE[u^\epsilon_t\mid h_{t}]\nonumber\\
&=\pi_h(h_t)+\expectE[u^\epsilon_t\mid h_{t}],\label{eq:history_policy}
\end{align}
where $\expectE[u^\epsilon_t\mid h_{t}]\neq 0$ due to the spurious correlation between $u^\epsilon_t$ and the trajectory history $h_t$. As a result, $\expectE[a_t\mid h_t]$ becomes biased, which can lead to arbitrarily worse performance compared to $\pi_E$.   }
\end{remark}

\vspace{-5pt}
\paragraph{Derivation of CMRs.} 
Leveraging the confounding horizon from Assumption~\ref{assump:horizon}, it becomes possible to break the spurious correlation using the independence of $u^\epsilon_t$ and $u^\epsilon_{t-k}$. We propose to use the $k$-step trajectory history $h_{t-k}=(s_{1},a_{1},...,s_{t-k})$ as an instrument for the current state $s_t$. Taking the expectation conditional on $h_{t-k}$ in~\cref{eq:history_policy} yields
\begin{align*}
    \expectE[a_t\mid h_{t-k}] & = \expectE\left[\expectE[a_t\mid h_{t}]\mid h_{t-k}\right] \\ & = \expectE[\pi_h(h_t)\mid h_{t-k}]+\expectE[\expectE[u^\epsilon_t\mid h_{t}]\mid h_{t-k}] \\
    & = \expectE[\pi_h(h_t) \mid h_{t-k}]+\expectE[u^\epsilon_t\mid h_{t-k}]
\end{align*}
where we use the fact that $h_{t-k}$ is $\sigma(h_t)$-measurable because $h_{t-k}\subseteq h_t$. Next, recall that $u^\epsilon_t\indep u^\epsilon_{t-k}$ by Assumption~\ref{assump:horizon}, which implies $u^\epsilon_t\indep h_{t-k}$, so that % Hence, since $\expectE[u^\epsilon_t] = 0$, we obtain
\begin{align}
    \expectE[a_t\mid h_{t-k}] &= \expectE[\pi_h(h_t) \mid h_{t-k}]+\expectE[u^\epsilon_t]\nonumber\\
    &=\expectE[\pi_h(h_t) \mid h_{t-k}].
\end{align}

As a result, the problem of learning $\pi_h$ reduces to solving for $\pi_h$ that satisfies the following identity
\begin{align}
    \expectE[a_t-\pi_h(h_t)\mid h_{t-k}]=0,\label{eq:CMR}
\end{align}
which is a CMR problem as defined in~\cref{sec:cmr}. In this case, both $a_t$ and $h_t$ are observed in the confounded expert demonstrations, and $h_{t-k}$ acts as the instrument. 

To make sure the instrument $h_{t-k}$ is valid, we check that it satisfies the conditions of~\cref{assump:iv}. Firstly, we have checked that $u^\epsilon_t\indep h_{t-k}$. Secondly, the environment and the expert policy are non-trivial, which means $\probP(h_t\mid h_{t-k})$ is not constant in $h_{t-k}$. Finally, $h_{t-k}$ indeed only affects $a_t$ through $s_t$ by the Markovian property. However, the strength of the instrument, which informally represents the correlation between the instrument $h_{t-k}$ and $h_t$, plays an important role in how well we can identify $\pi_h(h_t)$ by solving the CMRs in~\cref{eq:CMR}. In particular, we see that, as the confounding horizon $k$ increases, the correlation between $h_{t-k}$ and $h_t$ weakens and $h_{t-k}$ becomes a weaker instrument. This means that it is less able to identify $\pi_h$ via the CMR in~\cref{eq:CMR} and the final learnt imitator will have poorer performance. This is confirmed theoretically in Proposition~\ref{prop:ill-posed} and experimentally in~\cref{sec:exps}, and we will formalise this notion of instrument strength in~\cref{sec:theory}.


% Note this problem is equivalent to solving an IV regression on~\cref{eq:history_policy}, where $Y=\expectE[a_t\lvert h_t]$, $f(x)=\pi_h(h_t)$, $\epsilon=\expectE[u^\epsilon_t$ and the instrument $Z=h_{t-k}$.




\subsection{Practical Algorithms for Solving the CMRs}

\begin{algorithm}[tb]
   \caption{DML-IL}
   \label{alg:DML-IL}
\begin{algorithmic}[1]
   \STATE {\bfseries input} Dataset $\dataset_E$ of expert demonstrations, Confounding noise horizon $k$
   \STATE Initialize the roll-out model $\hat{M}$ as a Gaussian mixture model\label{algo:roll_out_1}
    \REPEAT
   \STATE Sample $(h_{t},a_t)$ from data $\dataset_E$
   \STATE Fit the roll-out model $(h_t,a_t)\sim\hat{M}(h_{t-k})$ to maximize the log likelihood 
\UNTIL{convergence}\label{algo:roll_out_2}
   \STATE Initialize the expert model $\hat \pi_h$ as a neural network
   \REPEAT
   % \FOR{$k=1$ {\bfseries to} $K$}
   \STATE Sample $h_{t-k}$ from $\dataset_E$
   \STATE Generate $\hat{h}_t$ and $\hat{a}_t$ using the roll-out model $\hat{M}$
   \STATE Update $\hat \pi_h$ to minimise the loss $\ell:= \norm{\hat{a}_t - \hat{\pi}_h (\hat h_t)}_2$
   % \ENDFOR
    \UNTIL{convergence}
    \STATE {\bfseries return} A history-dependent imitator policy $\hat{\pi}_h$
\end{algorithmic}
\end{algorithm}

There are various techniques~\citep{Shao2024,Bennett2019,Xu2020,Dikkala2020} for solving the CMRs $\expectE[a_t\lvert h_{t-k}]=\expectE[\pi_h(h_t) \lvert h_{t-k}]$. Here, the \textit{CMR error} that we aim to minimise is given by 
\begin{align*}
\sqrt{\expectE\big[\expectE[a_t-\hat{\pi}_h(h_t)\lvert h_{t-k}]^2\big]}=\norm{\expectE[a_t-\hat{\pi}_h(h_t)\lvert h_{t-k}]}_{2}.    
\end{align*}
In~\cref{alg:DML-IL}, we introduce DML-IL, an algorithm adapted from the IV regression algorithm DML-IV~\citep{Shao2024}\footnote{DML stands for double machine learning~\citep{Chernozhukov2018Double}, which is a statistical technique to ensure fast convergence rate for two-step regression, as is the case in~\cref{alg:DML-IL}.}, which solves our CMRs by minimising the CMR error. The first part of the algorithm (line 3-7) learns a roll-out model $\hat{M}$ that generates a trajectory $k$ steps ahead given $h_{t-k}$. Then, the roll-out model $\hat{M}$ is used to train the policy model $\hat{\pi}_h$ (line 8-13). $\hat{\pi}_h$ takes the generated trajectory $\hat{h}_t$ from $\hat{M}(h_{t-k})$ as inputs, and minimises the mean squared error to the next action. Using generated trajectories is crucial in breaking the spurious correlation caused by $u^\epsilon_t$ between past states and actions, and using the trajectory history before $h_{t-k}$ allows the imitator to infer information about $u^o_t$.

DML-IL can also be implemented with $K$-fold cross-fitting, where the dataset is partitioned into $K$ folds, with each fold alternately used to train $\hat{\pi}_h$ and the remaining folds to train $\hat{M}$. This ensures unbiased estimation and improves the stability of training. The base IV algorithm DML-IV with $K$-fold cross-fitting is theoretically shown to converge at the rate of $O(N^{-1/2})$~\citep{Shao2024}, where $N$ is the sample size, under regularity conditions. DML-IL with $K$-fold cross-fitting (see~\cref{appendix:dmlil} for details) will thus inherit this convergence rate guarantee. 

Note that~\cref{alg:DML-IL} requires the confounding noise horizon $k$ as input. While the exact value of $k$ can be difficult to obtain in reality, any upper bound $\bar{k}$ of $k$ is sufficient to guarantee the correctness of ~\cref{alg:DML-IL}, since $h_{t-\bar{k}}$ is also a valid instrument. Ideally, we would like a data-driven approach to determine $k$. Unfortunately, it is generally intractable to empirically verify whether $h_{t-k}$ is a valid instrument from a static dataset, especially the unconfounded instrument condition (i.e., $h_{t-k}\indep u^\epsilon_t$). Therefore, we rely on the user to provide a sensible choice of $\bar{k}$ based on the environment that does not substantially overestimate $k$.


\subsection{Theoretical Analysis}\label{sec:theory}

% \begin{align}
% p(u_t\lvert do(a_{t-k+1}),...,do(a_{t-1}),s_{t-k+1},...,s_{t-1})&\propto p(h_t)p_{\mu_0}(s_{t-k+1})\prod_{i=t-k+1}^{t-1} \transitions(s_{i+1}\lvert s_i,a_i,u_i)
% \end{align}

% since $$(u_t\indep a_{(t-k+1)...(t-1)} \lvert s_{(t-k+1)...(t_1)})_{\mathcal{G}_{\underline{a{(t-k+1)...(t-1)}}}}$$
% on the causal graph $\mathcal{G}_{\underline{a{(t-k+1)...(t-1)}}}$ where the arrows going into $a_{(t-k+1)...(t-1)}$ are removed.



In this section, we derive theoretical guarantees for our algorithm, focusing on the imitation gap and its relationship with existing work.


On a high level, in order to bound the imitation gap of the learnt policy $\hat{\pi}_h$, i.e., $J(\pi_E)-J(\hat{\pi}_h)$, we need to control:
\begin{enumerate}
    \item[($i$)] The amount of information about the hidden confounders that can be inferred from trajectory histories;
    \item[($ii$)] The ill-posedness (or identifiability) of the set of CMRs, which intuitively measures the strength of the instrument $h_{t-k}$;
    \item[($iii$)] The disturbance of the confounding noise to the states and actions at test time.
\end{enumerate}
These factors are all determined by the environment and the expert policy. To control ($i$), we measure how much information about $u^o_t$ is captured by the trajectory history $h_t$ by analysing the Total Variation (TV) distance between the distribution of $u^o_t$ and $\expectE[u^o_t\lvert h_t]$ along the trajectories of $\pi_E$. To control ($ii$) and ($iii$), we need to introduce the following two key concepts.

\begin{definition}[The ill-posedness of CMRs~\citep{Dikkala2020,Chen2012}]

Given the derived CMRs in~\cref{eq:CMR}, for a policy $\pi\in\Pi$, $\norm{\pi_E-\pi}_2$ is the root mean squared error to the expert and $\norm{\expectE[a_t-\pi(s_t)\lvert s_{t-k}]}_2$ is the CMR error we aim to minimise. Then, the \emph{ill-posedness} $\ill(\Pi,k)$ of the policy space with confounding noise horizon $k$ is given by
\begin{align*}
    \ill(\Pi,k)=\sup_{\pi\in\Pi} \frac{\norm{\pi_E-\pi}_{2}}{\norm{\expectE[a_t-\pi(h_t)\lvert h_{t-k}]}_{2}}.
\end{align*}
\end{definition}
The ill-posedness $\ill(\Pi,k)$ measures the strength of the instrument where a higher $\ill(\Pi,k)$ indicates a weaker instrument. It bounds the ratio between the learning error of the imitator following our CMR objective and its $L_2$ error to the expert policy. 

As discussed previously, intuitively, the strength of the instrument would decrease as the confounding horizon $k$ increases. This is in fact true and is confirmed by the following proposition. The proof is deferred to~\cref{appendix:prop}. 
\begin{proposition}\label{prop:ill-posed}
The ill-posedness $\ill(\Pi,k)$ is monotonically increasing as the confounded horizon $k$ increases.
\end{proposition}

Next, we introduce the notion of c-TV stability.
\begin{definition}[c-total variation stability~\citep{Bassily2021,Swamy2022_temporal}]
Let $P(X)$ be the distribution of a random variable $X:\Omega\rightarrow \mathcal{X}$. $P(X)$ is c-TV stable if for $a_1,a_2\in \mathcal{X}$ and $\Delta>0$,
\begin{align*}
\norm{a_1-a_2}\leq\Delta \implies \delta_{TV}(a_1+X,a_2+X)\leq c\Delta.
\end{align*}
where $\norm{\cdot}$ is some norm defined on $\mathcal{X}$ and $\delta_{TV}$ is the total variation distance.
\end{definition}
A wide range of distributions are c-TV stable. For example, standard normal distributions are $\frac{1}{2}$-TV stable. We apply this notion to the distribution over $u^\epsilon_t$ to bound the disturbance it induces in the trajectory and the expected return.

With the notion of ill-posedness and c-TV stability, we can now analyse and upper bound the imitation gap $J(\pi_E)-J(\hat{\pi}_h)$ by controlling the three components $(i)-(iii)$ discussed above. 
% We present the main result for this paper, where t
The full proof is deferred to~\cref{appendix:gap}.

\begin{theorem}[Imitation Gap Bound]\label{thm:gap}
Let $\hat{\pi}_h$ be the learnt policy with CMR error $\epsilon$ and let $\ill(\Pi,k)$ be the ill-posedness of the problem. Assume that $\delta_{TV}(u^o_t,\expectE_{\pi_E}[u^o_t\lvert h_t])\leq\delta$ for $\delta\in\realNumber^+$, $P(u^\epsilon_t)$ is c-TV stable and $\pi_E$ is deterministic. Then, the imitation gap is upper bounded by 
\begin{align*}
    J(\pi_E)-J(\hat{\pi}_h)\leq T^2\big(c\epsilon\ill(\Pi,k)+2\delta\big)=\mathcal{O}\big(T^2(\delta+\epsilon)\big).
\end{align*}
\end{theorem}
This upper bound scales at the rate of $T^2$, which aligns with the expected behaviour of imitation learning without an interactive expert~\citep{Ross2010}.
Next, we show that the upper bounds on the imitation gap from prior work~\citep{Swamy2022_temporal, Swamy2022} are special cases of
% of  subsumed by the unifying causal IL framework introduced in Section~\ref{sec:setting} are special cases of 
Theorem~\ref{thm:gap}. The proofs are deferred to~\cref{appendix:corollaries}.
\begin{corollary}\label{corollary:noUo}
In the special case that $u^o_t = 0$, i.e., there are no expert-observable confounders, or $u^o_t=\expectE_{\pi_E}[u^o_t\lvert h_t]$, i.e., $u^o_t$ is $\sigma(h_t)$ measurable (all information about $u^o_t$ is contained in the history), the imitation gap is upper bounded by
\begin{align*}
    J(\pi_E)-J(\hat{\pi}_h)\leq T^2\big(c\epsilon\ill(\Pi,k)\big)=\mathcal{O}\big(T^2\epsilon\big),
\end{align*}
which coincides with Theorem 5.1 of~\citet{Swamy2022_temporal}.
\end{corollary}

When there are no hidden confounders, i.e, $u^\epsilon_t=0$, our framework is reduced to that of~\citet{Swamy2022}. However, \citet{Swamy2022} provided an abstract bound that directly uses the supremum of key components in the imitation gap over all possible Q functions to bound the imitation gap. We further extend and concretise the bound using the learning error $\epsilon$ and the TV distance bound $\delta$ instead of relying on the suprema.


\begin{corollary}\label{corollary:unconfounded}
In the special case that $u^\epsilon_t=0$, if the learnt policy has optimisation error $\epsilon$,  the imitation gap is upper bounded by
\begin{align*}
    J(\pi_E)-J(\hat{\pi}_h)\leq T^2\left(\frac{2}{\sqrt{\dim(A)}}\epsilon+2\delta \right),
\end{align*}
which is a concrete bound that extends the abstract bound in Theorem 5.4 of~\cite{Swamy2022}.
\end{corollary}

\begin{remark}
\emph{If both $u^\epsilon_t$ and $u^o_t$ are zero, we then recover the classic setting of IL without confounders~\citep{Ross2010}, and the imitation gap bound is $T^2\epsilon$, where $\epsilon$ is the optimisation error of the algorithm.}
\end{remark}

\begin{table*}[ht]
  \centering
  \small
  \begin{tabular}{lcccccccccc}
    \toprule
    \multirow{2}{*}{Method} & \multirow{2}{*}{\#Params} & \multicolumn{6}{c}{Pre-trained tasks} & \multicolumn{1}{c}{Target task} & \multicolumn{2}{c}{Metrics} \\
    \cmidrule(lr){3-8} \cmidrule(lr){9-9}  \cmidrule(lr){10-11}
                    &        & \textbf{GQA}              & \textbf{VizWiz}                & \textbf{SQA}                   & \textbf{TextVQA} & \textbf{POPE} & \textbf{MMBench} & \textbf{Flickr30k} & \textbf{Avg} & \textbf{Hscore} \\
    \midrule
    \textbf{Zero-shot}       & --      & 61.94            & 50.00                 & 66.80                 & 58.27 & 85.90 & 64.30 & 3.5 & 55.82  & 59.86 \\
    \midrule
    \textbf{Fine-tune}       & 1.2B   & 56.26            & 44.45                 & 28.34                 & 38.98 & 38.40 & 50.56 & \textbf{78.82} & 47.97 & 45.26 \\
    \textbf{LoRA}            & 29M   & 17.74            & 40.63                 & 5.38                  & 30.48 & 2.40  & 9.55  & 64.18 & 24.33 & 20.49 \\
    \textbf{Model Tailor}    & 273M   & 52.49            & 42.28                 & \underline{67.15}     & 43.89 & 82.88 & 63.40 & \underline{75.40} & 61.07 & 59.85 \\
    \midrule
    \textbf{MDGD}             & 1.2B   & \underline{67.71}  & \underline{48.18} & \textbf{69.05}         & \underline{57.32} & \textbf{85.12} & \underline{65.43} & 73.47 & \textbf{66.61} & \textbf{66.03} \\
    ~~w/o visual align     & 1.2B   & 57.64           & 36.95                 & 53.96                 & 32.84 & 30.43 & 56.66 & 65.58 & 47.72 & 46.19 \\
    \textbf{MDGD-GM } & 124M   & \textbf{69.89}  & \textbf{51.22}        & 65.87                 & \textbf{58.18} & \underline{84.39} & \textbf{66.42} & 64.18  & \underline{65.74} & \underline{65.86} \\
    \bottomrule
    \toprule

    \multirow{2}{*}{Method} & \multirow{2}{*}{\#Params} & \multicolumn{6}{c}{Pre-trained tasks} & \multicolumn{1}{c}{Target task} & \multicolumn{2}{c}{Metrics} \\
    \cmidrule(lr){3-8} \cmidrule(lr){9-9}  \cmidrule(lr){10-11}
    &                       & \textbf{GQA} & \textbf{VizWiz} & \textbf{SQA} & \textbf{TextVQA} & \textbf{POPE} & \textbf{MMBench} & \textbf{OKVQA} & \textbf{Avg} & \textbf{Hscore} \\
    \midrule
    \textbf{Zero-shot}       & --     & 61.94 & 50.00 & 66.80 & 58.27 & 85.90 & 64.30 & 0.14 & 55.34 & 59.58 \\
    \midrule
    \textbf{Fine-tune}     & 1.2B  & 62.98 & 40.59 & 59.84 & 48.38 & 71.42 & 51.98 & \underline{69.10} & 57.76 & 56.79 \\
    \textbf{LoRA}            & 29M   & 63.44 & 41.61 & 51.29 & 48.02 & 75.27 & 37.31 & \textbf{71.46} & 55.49 & 54.12 \\
    \textbf{Model Tailor}    & 273M   & 60.39              & \textbf{46.49} & \textbf{69.51} & \textbf{54.88} & \textbf{85.44} & \underline{63.32} & 38.10 & 59.73 & 61.48 \\
    \midrule
    \textbf{MDGD}          & 1.2B     & \textbf{66.55}     & 42.72 & 64.60 & 52.54 & \underline{85.17} & 61.73 & 62.29 & \underline{62.23} & \underline{62.22} \\
    ~~w/o visual align & 1.2B  & \underline{66.39} & 39.89 & 60.19 & 52.40 & 84.92 & 62.97 & 62.39 & 61.31 & 61.22 \\
    \textbf{MDGD-GM}  & 124M  & 66.02              & \underline{43.97} & \underline{67.91} & \underline{52.80} & 84.70 & \textbf{63.97} & 61.04 & \textbf{62.92} & \textbf{63.07} \\
    \bottomrule
  \end{tabular}
  \caption{
  Performance on various pre-trained tasks of LLaVA-1.5 models fine-tuned on Flickr30K and OKVQA. 
  We report the best performance for each task in a \textbf{bold font} while the second best performance \underline{underlined}. 
  }
  \label{tab:llava-main}
  \vspace{-.8cm}
\end{table*}

\section{Experiments}
In this section, we conduct experiments on various datasets and backbone MLLMs to investigate the following research questions:
\begin{enumerate}
    \item \textbf{RQ1 (Overall Performance)} Can MDGD prevent visual forgetting while improving downstream tasks?
    \item \textbf{RQ2 (Ablation Study)} How do the visual alignment and gradient masking affect the MDGD's performance?
    \item \textbf{RQ3 (Representation Learning)} How does MDGD benefit effective multimodal representation learning in MLLMs?
    \item \textbf{RQ4 (Sensitivity Study)} How does gradient masking ratio $\alpha$ affects the learning of MDGD?
\end{enumerate}


\noindent\textbf{Datasets}
To evaluate the effectiveness of MDGD in mitigating catastrophic forgetting, we used two models of different sizes. 
Our experimental design follows the settings from the work of \citet{zhu2024model}. 
For each model, datasets were categorized into two types: 
\textbf{pre-trained tasks}, which assess the model's ability to retain inherent knowledge after fine-tuning, 
and \textbf{fine-tuning tasks}, consisting of unseen datasets used to test adaptability. 
After fine-tuning, we evaluated performance on both task types to measure forgetting and generalization. 
Below, we detail the datasets used for each model. 
\textbf{LLaVA-1.5 (Vicuna-7B) \citep{liu2024improved}}: This model has 7 billion parameters. In line with \citet{liu2024improved}, we used the following datasets:
\begin{itemize}
    \item \textbf{Pre-trained Tasks}: VQAv2 \citep{goyal2017making}, GQA \citep{hudson2019gqa}, VizWiz \citep{gurari2018vizwiz}, SQA \citep{lu2022learn}, TextVQA \citep{singh2019towards}, POPE \citep{li2023evaluating}, and MM-Bench \citep{liu2023mmbench}.
        
    \item \textbf{Fine-tuning}: Flickr30k \citep{young2014image} and OKVQA \citep{marino2019ok}, which were not encountered in the pre-training stage.
\end{itemize}
\textbf{MiniCPM-V-2.0 \citep{yao2024minicpmvgpt4vlevelmllm}}: This model has 2.8 billion parameters. We evaluated its performance on:
\begin{itemize}
    \item \textbf{Pre-trained Tasks}: VizWiz, OKVQA, A-OKVQA \cite{schwenk2022okvqa}, Text-VQA, IconQA \cite{lu2021iconqa}, POPE, and MM-Bench.
    \item \textbf{Fine-tuning}: TextCaps \citep{sidorov2020textcaps} and PathVQA \citep{he2020pathvqa}, which were not part of its pre-training exposure.
\end{itemize}



\noindent\textbf{Baselines} We compare our approach against several baselines: 
\begin{itemize}
\item \textbf{Standard Fine-Tuning.} For a fair comparison, we follow the setting of Model-Tailor \cite{zhu2024model}, where LLaVA-1.5 is fine-tuned on the last 6 layers and its feature adapter, with a total of 1.2B parameters.
MiniCPM is fine-tuned on the last 8 layers and its feature resampler, with 517M parameters. 
\item \textbf{LoRA-based Fine-Tuning \citep{hu2021lora}.} LoRA introduces low-rank matrices to update only a small subset of parameters, reducing memory consumption and computational cost. In our experiments, LLaVA-1.5 and MiniCPM are fine-tuned by modifying the query and key projection layers within the attention mechanism. 
\item \textbf{Model Tailor \citep{zhu2024model}.} This baseline employs a hybrid strategy that mitigates catastrophic forgetting by identifying and adjusting the most critical parameters for adaptation. It has been evaluated through experiments on multimodal large language models (MLLMs). As the method is not open source, we report only the original results of the LLaVA-1.5 experiments provided in the original paper as a baseline.
\end{itemize}


\noindent\textbf{Implementation Details}
We use the official Huggingface implementations of the LLaVA-1.5 and the MiniCPM-V-2.0 models and their LoRA adapters. 
For model fine-tuning, we use BFloat16 precision for memory-efficient training. 
Experiments are conducted using 2 NVIDIA A100-SXM4-80GB GPUs.

\subsection{Overall Performance (RQ1)}\label{sec:main-results}
\subsubsection{Larger MLLM adapts better to downstream tasks but is more prone to visual forgetting.}
We study the visual forgetting problem on the LLaVA-1.5 MLLM which contains 7B model parameters 
and report performance comparison results in Table~\ref{tab:llava-main}.
We observe that the pre-trained LLaVA enables efficient instruction tuning on target tasks,
where the zero-shot performance is near zero.
When the model is fine-tuned on the image caption task, Flickr30K, which largely differs from the pre-trained tasks of visual question-answering,
the model can learn a degraded multimodal representation,
which causes visual forgetting in its projected visual representation space (in Section~\ref{sec:visual_forget}) 
and its average performance on pre-trained tasks drops 33.63\% compared with zero-shot performance. 
Fine-tuning on visual question-answering task OKVQA, which is similar to the pre-trained tasks, can also cause a 13.44\% performance drop,
due to the limited image-text pairs existing in the downstream task, which potentially leads to MLLM's visual understanding drift.

\subsubsection{Smaller MLLM also experiences visual forgetting while limited in downstream task improvements.}
To validate the observation on a smaller MLLM, we report the comparison results of MiniCPM-V-2.0 with 2.8B model parameters in Table~\ref{tab:minicpm}.
We observe that compared with the LLaVA MLLM, MiniCPM suffers from less prominent visual forgetting.
The average performance drop of the model limits to 6.28\% and 4.25\% when fine-tuning on PathVQA and TextCaps, respectively.
We attribute this observation to MiniCPM learning a more compact and constrained visual representation space during pre-training, 
causing the visual representations of target task images to be less aligned with those of the pre-trained MLLM. 
Consequently, MiniCPM exhibits limited improvement in downstream tasks, 
as its restricted ability to acquire additional visual knowledge leads to ineffective instruction tuning.

\subsubsection{MDGD prevents visual forgetting while maintaining downstream task improvements.}
By employing MDGD in MLLM instruction tuning,
we observe the LLaVA's average performance drop on pre-trained tasks reduces to 3.59\% when fine-tuned on OKVQA
and also achieves a 1.45\% improvement when fine-tuned on Flickr30K,
which demonstrates the efficiency of MDGD in mitigating visual forgetting.
For the smaller MLLM, MiniCPM, MDGD achieves comparable downstream task improvements with direct fine-tuning,
while completely eliminating visual forgetting in the pre-trained tasks.
MDGD and its variants consistently achieve the best average performance for both MLLMs,
demonstrating its great potential for incremental learning on individual downstream tasks.

\subsubsection{Comparison with baseline methods.}
In Table~\ref{tab:llava-main}, we compare MDGD with LoRA fine-tuning and Model Tailor \cite{zhu2024model} on LLaVA-1.5, 
which are designed for parameter-efficient fine-tuning.
We observe that LoRA fine-tuning can suffer from significant visual forgetting on Flickr30K and OKVQA.
Since LoRA introduces additional representation projections in intermediate layers,
the pre-trained multimodal representations can be projected into a lower-rank subspace leading to visual forgetting (in Section~\ref{sec:visual_forget}),
due to the limitation of image-text pairs in the target dataset.
Model Tailor is designed for MLLM anti-forgetting, 
which identifies ``patches'' of sub-model parameters significantly affected by fine-tuning on the target task.
However, since the method is not specifically designed for MLLMs, 
the unique challenge of visual forgetting cannot be effectively mitigated while maintaining robust performance on the target task.Thus, we observe that Model Tailor’s performance is sensitive to the target task datasets (e.g., better on Flickr30K than OKVQA), 
whereas MDGD consistently outperforms Model Tailor in terms of both average task scores and H-scores across the two datasets.
In Table~\ref{tab:minicpm}, we also report the results of the MDGD comparison with LoRA fine-tuning on MiniCPM.
We observe consistent improvements on the average task performance of MDGD when fine-tuned on PathVQA and TextCaps,
especially MDGD achieves 2.43\% and 1.83\% on the pre-trained tasks of PathVQA and TextCaps, respectively,
which demonstrates the effectiveness of MDGD in mitigating visual forgetting. 

\begin{table*}[ht]
  \centering
  \small
  \begin{tabular}{lccccccccccc}
    \toprule   
    \multirow{2}{*}{Method} & \multirow{2}{*}{\#Params} & \multicolumn{7}{c}{Pre-trained tasks} & \multicolumn{1}{c}{Target task} & \multicolumn{2}{c}{Metrics} \\
    \cmidrule(lr){3-9} \cmidrule(lr){10-10} \cmidrule(lr){11-12}
    &                       & \textbf{VizWiz} & \textbf{A-OKVQA} & \textbf{OKVQA} & \textbf{TextVQA} & \textbf{IconQA} & \textbf{POPE} & \textbf{MMBench} & \textbf{PathVQA} & \textbf{Avg} & \textbf{Hscore} \\
    \midrule
    \textbf{Zero-shot}       & -     & 55.27  & 79.39 & 64.86 & 77.98 & 79.01 & 88.93 & 70.98 & 5.44 & 65.23   & 10.04\\
    \midrule
    \textbf{Fine-tune}       & 517M  & 52.91  & 76.94 & 59.06 & 58.34 & 76.96 & \textbf{89.60} & 70.16 & \underline{11.04} & 61.88  & \underline{18.74}\\
    \textbf{LoRA}            & 35M & 52.95  & 76.24 & \textbf{64.45} & 77.18 & 77.80 & 88.08 & 67.47 & \textbf{15.03} & 64.90 & \textbf{24.41}\\
    \midrule
    \textbf{MDGD}           &  517M  & \textbf{55.73}  & 78.25 & \underline{64.33} & \underline{77.54} & \textbf{79.45} & \underline{89.19} & \textbf{71.94} & 9.09 & \textbf{65.69}   & 15.97\\
    ~~w/o visual align & 517M  & 54.92  & \underline{78.52} & 64.17 & 77.42 & \underline{79.37} & 89.10 & 70.96 & 8.49 & \underline{65.37}  & 15.03\\
   \textbf{ MDGD-GM}  & 52M  & \underline{55.04}  & \textbf{78.78} & 64.31 & \textbf{77.78} & 79.10 & 88.76 & \underline{70.98} & 5.72 & 65.06  & 10.52 \\
    
    \bottomrule
    \toprule
    
    \multirow{2}{*}{Method} & \multirow{2}{*}{\#Params} & \multicolumn{7}{c}{Pre-trained tasks} & \multicolumn{1}{c}{Target task} & \multicolumn{2}{c}{Metrics} \\
    \cmidrule(lr){3-9} \cmidrule(lr){10-10} \cmidrule(lr){11-12}
    &                        & \textbf{VizWiz} & \textbf{A-OKVQA} & \textbf{OKVQA} & \textbf{TextVQA} & \textbf{IconQA} & \textbf{POPE} & \textbf{MMBench} & \textbf{TextCaps} & \textbf{Avg} & \textbf{Hscore} \\
    \midrule
    \textbf{Zero-shot}        & -     & 55.27  & 79.39 & 64.86 & 77.98 & 79.01 & 88.93 & 70.98 & 15.77 & 66.52 & 25.50  \\
    \midrule
    \textbf{Fine-tune}        & 517M  & 52.03  & 77.73 & 59.16 & 67.24 & 78.67 & 88.20 & 71.42 & \textbf{33.85} & 66.04 & \textbf{44.76} \\
    \textbf{LoRA}             & 35M  & 53.30  & \underline{78.17}          & \underline{63.99} & \underline{77.68} & 78.28 & 87.31 & 69.23 & \underline{32.41} & 67.55 & \underline{43.80} \\
    \midrule
    \textbf{MDGD}             & 517M  & \textbf{55.17}  & \underline{78.17} & 63.67          & 76.08 & \underline{79.40} & \textbf{89.11} & \underline{71.58} & 28.90 & \underline{67.76} & 40.52 \\
    ~~w/o visual align & 517M  & 51.35           & 78.08            & 63.06          & 76.48 & 78.99                & \underline{88.98} & 71.30 & 25.93 & 66.77 & 37.35 \\
    \textbf{MDGD-GM}  & 52M   & \underline{55.04}  & \textbf{78.43} & \textbf{65.26} & \textbf{78.08} & \textbf{79.65} & 88.93           & \textbf{71.88} & 29.14 & \textbf{68.30} & 40.85 \\
    \bottomrule
  \end{tabular}
  \caption{
  Performance on various pre-trained tasks of MiniCPM-V2.5 models fine-tuned on PathVQA and TextCaps. 
  We report the best performance for each task in a \textbf{bold font} while the second best performance \underline{underlined}.
  }
  \label{tab:minicpm}
  \vspace{-.8cm}
\end{table*}


\section{Concluding Remarks}
In this paper, we proposed a novel approach utilizing multimodal LLMs to generate gesture-aware speech recognition transcripts for patients with language disorders. Our framework integrates verbal speech and iconic gestures, enabling the generation of enriched transcripts that capture the latent meaning conveyed through both modalities. Through extensive experimentation, we demonstrated that the proposed method effectively contextualizes incomplete or disfluent speech by incorporating gesture information, leading to more accurate and meaningful representations of the speaker's intent. These findings highlight the potential of our approach to significantly contribute to the field of speech and language therapy, offering innovative tools that can enhance the quality of life for individuals with language disorders by facilitating better communication and assessment methods.

\subsection{Ethical Statement} 
Our dataset was obtained from AphasiaBank with the approval of the Institutional Review Board (IRB) and adheres to the data sharing guidelines set by TalkBank\footnote{https://talkbank.org/share/ethics.html}. This includes complying with the Ground Rules for all TalkBank databases, which are based on the American Psychological Association Code of Ethics~\cite{american2002ethical}.

\subsection{Limitation \& Future Work} 
%This study represents a preliminary investigation into using multimodal LLMs to generate gesture-aware speech recognition transcripts. 
While the results are promising, we recognize several limitations and outline our plans to extend this work further.

One primary limitation is the absence of a definitive ground truth for quantitative evaluation. Since our model generates transcripts by synthesizing speech and gesture data from scratch, traditional benchmarks, such as comparisons with standard speech recognition outputs, are insufficient. Moreover, existing original transcripts lack gesture annotations, making direct comparisons challenging. In future work, we aim to address this gap by collaborating with certified pathologists to conduct qualitative assessments, such as A-B preference tests, to evaluate the effectiveness of gesture-enriched transcripts in accurately conveying the speaker's intentions.

To support quantitative evaluations, we plan to develop novel metrics that assess transcript quality, including grammar accuracy, semantic consistency, and the integration of multimodal information. Such metrics will provide a more objective basis for assessing our model's performance and facilitate comparisons with other multimodal and unimodal approaches.

Another limitation of this study is its focus on structured gestures from a specific task, the Peanut Butter Sandwich Task. While this task offers a controlled context for testing our approach, it does not encompass the diversity of gestures and communication patterns seen in everyday scenarios. As part of our future work, we plan to expand the scope of our model to include tasks such as the Cinderella Story Recall Task~\cite{bird1996cinderella}, which involves unstructured and complex narrative gestures. This expansion will allow us to evaluate the adaptability and robustness of our model in handling varied linguistic and gestural contexts.

In summary, while this study establishes a strong foundation for gesture-aware speech recognition, we aim to refine and extend our methods through collaborative qualitative evaluations, the development of robust quantitative metrics, and broader task applications. These efforts will ensure that our approach continues to evolve, ultimately contributing to more effective communication tools and interventions for individuals with language disorders.





% \clearpage
% \section*{Impact Statement}
% This paper aims to enhance the memory efficiency and reduce the computational burden of fine-tuning large language models (LLMs), with a particular focus on minimizing the environmental impact of on-device fine-tuning. The growing computational and memory demands of LLMs have led to significant energy consumption and carbon emissions, raising critical concerns about the sustainability of AI development. By enabling the fine-tuning of larger models on hardware with limited memory capacity, our approach not only alleviates hardware constraints but also significantly reduces energy consumption and carbon footprint. This work contributes to more environmentally responsible AI practices, making advanced language models more accessible while aligning with global sustainability goals.

\bibliography{example_paper}
\bibliographystyle{icml2025}


\section{Metric}
\label{sec:metric}

\textbf{Mean Squared Error (MSE)} Mean Squared Error (MSE) is a common statistical metric used to assess the difference between predicted and actual values. The formula is:
\begin{equation}
    MSE = \frac{1}{n} \sum_{i=1}^{n} (y_i - \hat{y}_i)^2
\end{equation}
where $ n $ is the number of samples, $ y_i $ is the actual value, and $ \hat{y}_i $ is the predicted value.

\textbf{Relative L2 Error} Relative L2 error measures the relative difference between predicted and actual values, commonly used in time series prediction. The formula is:
\begin{equation}
    \text{Relative L2 Error} = \frac{\| Y_{\text{pred}} - Y_{\text{true}} \|_2}{\| Y_{\text{true}} \|_2}
\end{equation}
where $ Y_{\text{pred}} $ is the predicted value and $ Y_{\text{true}} $ is the actual value.

\textbf{Structural Similarity Index Measure (SSIM)} The Structural Similarity Index (SSIM) measures the similarity between two images in terms of luminance, contrast, and structure. The formula is:
\begin{equation}
    SSIM(x, y) = \frac{(2\mu_x \mu_y + C_1)(2\sigma_{xy} + C_2)}{(\mu_x^2 + \mu_y^2 + C_1)(\sigma_x^2 + \sigma_y^2 + C_2)}
\end{equation}
where $ \mu_x $ and $ \mu_y $ are the mean values, $ \sigma_x $ and $ \sigma_y $ are the standard deviations, $ \sigma_{xy} $ is the covariance.

\section{Related Work}
\subsection{Deep Learning based Weather Forecasting}
\textbf{Global Weather Forecasting.} Global weather forecasting has seen significant progress with deep learning models. FourCastNet, based on Fourier neural operators, provides global forecasts comparable to traditional numerical methods like IFS, but at much higher speeds~\cite{pathak2022fourcastnet}. Pangu, utilizing the Swin Transformer, exceeds NWP methods, incorporating earth-specific location embeddings for better performance~\cite{bi2023accurate}. The Spherical Fourier Neural Operator (SFNO) extends Fourier methods using spherical harmonics, offering more stable long-term predictions~\cite{bonev2023spherical}. FuXi focuses on long-term forecasting, achieving a 15-day forecasts comparable to ECMWF~\cite{chen2023fuxi}. GraphCast leverages message-passing networks to improve efficiency and forecasting accuracy~\cite{lam2023learning}, and GenCast builds on this to enhance ensemble forecasting~\cite{price2023gencast}. Further, diffusion models like those in~\cite{li2024generative} generate probabilistic ensembles by sampling, while NeuralGCM~\cite{kochkov2024neural} focuses on atmospheric circulation with a dynamic core, offering climate simulation capabilities but at higher training and inference costs. 

\textbf{Regional Weather Forecasting.} The goal of regional weather forecasting is to enhance local prediction accuracy with high-resolution models. CorrDiff~\cite{mardani2023generative} combines U-Net and diffusion models to improve local forecasts. MetaWeather~\cite{kim2024metaweather} adapts global forecasts to regional contexts using meta-learning. GNNs are also widely applied in regional forecasting, with Graphcast~\cite{lam2023learning} enhancing accuracy by modeling complex spatial dependencies. MetNet-3~\cite{espeholt2022deep} offers high-accuracy forecasts for weather variables, such as precipitation, temperature, and wind speed, at 2-minute intervals and 1–4 km resolution, outperforming traditional models like HRRR. NowcastNet~\cite{zhang2023skilful} and DGMR~\cite{ravuri2021skilful} excel in short-term extreme precipitation forecasts using deep generative models and radar data. In spatiotemporal prediction, NMO~\cite{wu2024neural} models the evolution of physical dynamics, providing new insights for local weather forecasting. Similarly, SimVP~\cite{gao2022simvp} and PastNet~\cite{wu2024pastnet} achieve good results in forecasting local precipitation evolution using spatiotemporal convolution methods.
    
% Despite these advances, none of these methods effectively address the challenge of balancing global and regional high-resolution forecasts or capturing the fine-grained, dynamic interactions important for extreme event prediction.
    
\subsection{Numerical analysis methods}
Multigrid methods~\cite{mccormick1987multigrid,wesseling1995introduction,hackbusch2013multi,bramble2019multigrid,hiptmair1998multigrid,brandt1983multigrid,borzi2009multigrid} and nested grid strategies~\cite{miyakoda1977one,zhang2012nested,sullivan1996grid} are widely used to solve PDEs and handle multi-scale problems~\cite{debreu2008two,xue2000advanced}. Multigrid methods use grids of different resolutions to transfer information and accelerate iterations. They efficiently solve large-scale problems and improve computational accuracy. By eliminating low-frequency errors on coarse grids and high-frequency errors on fine grids, multigrid methods effectively handle error convergence at different scales~\cite{he2019mgnet,he2023mgno,shao2022fast}. Nested grid strategies embed higher-resolution fine grids into regions of interest based on a global coarse grid to capture local complex physical phenomena in detail. In weather forecasting, this method provides large-scale background fields on a global scale while refining the grid for target regions to accurately simulate the evolution of local weather systems and the occurrence of extreme events~\cite{bacon2000dynamically}. 

% Our proposed neural nested grid method helps address challenges like boundary information loss in regional forecasting and multi-scale feature capture.

\section{Additional Results}
%
We present more additional results in Figure \ref{fig_0.25-day}, \ref{fig_0.5-day}, \ref{fig_1.0-day} \ref{fig_1.5-day}, \ref{fig_2.0-day}, \ref{fig_2.5-day}, \ref{fig_3.0-day}, \ref{fig_3.5-day}, \ref{fig_4.0-day}, \ref{fig_4.5-day}, \ref{fig_5.0-day}, \ref{fig_5.5-day}, \ref{fig_6.0-day}, \ref{fig_6.5-day}, \ref{fig_7.0-day}, \ref{fig_7.5-day},
\ref{fig_8.0-day}, \ref{fig_8.5-day}, \ref{fig_9.0-day}, \ref{fig_9.5-day},
\ref{fig_10.0-day}, including 18 variables that are importmant to weather forecasting, each with results ranging from 6 hours to 10 days. These additional results further demonstrate the effectiveness of OneForecast. Same as the Figure \ref{fig:visual_results}
, the initial conditions is 00:00 UTC, 1 January 2020.


\begin{figure*}[h]
\centering
\includegraphics[width=1\linewidth]{figures/fig_0.25-day.jpg}
\vspace{-20pt}
\caption{6-hour forecast results of different models.}
\label{fig_0.25-day}
\end{figure*}

\begin{figure*}[h]
\centering
\includegraphics[width=1\linewidth]{figures/fig_0.5-day.jpg}
\vspace{-20pt}
\caption{0.5-day forecast results of different models.}
\label{fig_0.5-day}
\end{figure*}

\begin{figure*}[h]
\centering
\includegraphics[width=1\linewidth]{figures/fig_1.0-day.jpg}
\vspace{-20pt}
\caption{1-day forecast results of different models.}
\label{fig_1.0-day}
\end{figure*}

\begin{figure*}[h]
\centering
\includegraphics[width=1\linewidth]{figures/fig_1.5-day.jpg}
\vspace{-20pt}
\caption{1.5-day forecast results of different models.}
\label{fig_1.5-day}
\end{figure*}

\begin{figure*}[h]
\centering
\includegraphics[width=1\linewidth]{figures/fig_2.0-day.jpg}
\vspace{-20pt}
\caption{2-day forecast results of different models.}
\label{fig_2.0-day}
\end{figure*}


\begin{figure*}[h]
\centering
\includegraphics[width=1\linewidth]{figures/fig_2.5-day.jpg}
\vspace{-20pt}
\caption{2.5-day forecast results of different models.}
\label{fig_2.5-day}
\end{figure*}

\begin{figure*}[h]
\centering
\includegraphics[width=1\linewidth]{figures/fig_3.0-day.jpg}
\vspace{-20pt}
\caption{3-day forecast results of different models.}
\label{fig_3.0-day}
\end{figure*}

\begin{figure*}[h]
\centering
\includegraphics[width=1\linewidth]{figures/fig_3.5-day.jpg}
\vspace{-20pt}
\caption{3.5-day forecast results of different models.}
\label{fig_3.5-day}
\end{figure*}

\begin{figure*}[h]
\centering
\includegraphics[width=1\linewidth]{figures/fig_4.0-day.jpg}
\vspace{-20pt}
\caption{4-day forecast results of different models.}
\label{fig_4.0-day}
\end{figure*}

\begin{figure*}[h]
\centering
\includegraphics[width=1\linewidth]{figures/fig_4.5-day.jpg}
\vspace{-20pt}
\caption{4.5-day forecast results of different models.}
\label{fig_4.5-day}
\end{figure*}


\begin{figure*}[h]
\centering
\includegraphics[width=1\linewidth]{figures/fig_5.0-day.jpg}
\vspace{-20pt}
\caption{5.0-day forecast results of different models.}
\label{fig_5.0-day}
\end{figure*}

\begin{figure*}[h]
\centering
\includegraphics[width=1\linewidth]{figures/fig_5.5-day.jpg}
\vspace{-20pt}
\caption{5.5-day forecast results of different models.}
\label{fig_5.5-day}
\end{figure*}

\begin{figure*}[h]
\centering
\includegraphics[width=1\linewidth]{figures/fig_6.0-day.jpg}
\vspace{-20pt}
\caption{6.0-day forecast results of different models.}
\label{fig_6.0-day}
\end{figure*}

\begin{figure*}[h]
\centering
\includegraphics[width=1\linewidth]{figures/fig_6.5-day.jpg}
\vspace{-20pt}
\caption{6.5-day forecast results of different models.}
\label{fig_6.5-day}
\end{figure*}

\begin{figure*}[h]
\centering
\includegraphics[width=1\linewidth]{figures/fig_7.0-day.jpg}
\vspace{-20pt}
\caption{7.0-day forecast results of different models.}
\label{fig_7.0-day}
\end{figure*}

\begin{figure*}[h]
\centering
\includegraphics[width=1\linewidth]{figures/fig_7.5-day.jpg}
\vspace{-20pt}
\caption{7.5-day forecast results of different models.}
\label{fig_7.5-day}
\end{figure*}

\begin{figure*}[h]
\centering
\includegraphics[width=1\linewidth]{figures/fig_8.0-day.jpg}
\vspace{-20pt}
\caption{8.0-day forecast results of different models.}
\label{fig_8.0-day}
\end{figure*}

\begin{figure*}[h]
\centering
\includegraphics[width=1\linewidth]{figures/fig_8.5-day.jpg}
\vspace{-20pt}
\caption{8.5-day forecast results of different models.}
\label{fig_8.5-day}
\end{figure*}

\begin{figure*}[h]
\centering
\includegraphics[width=1\linewidth]{figures/fig_9.0-day.jpg}
\vspace{-20pt}
\caption{9.0-day forecast results of different models.}
\label{fig_9.0-day}
\end{figure*}

\begin{figure*}[h]
\centering
\includegraphics[width=1\linewidth]{figures/fig_9.5-day.jpg}
\vspace{-20pt}
\caption{9.5-day forecast results of different models.}
\label{fig_9.5-day}
\end{figure*}

\begin{figure*}[h]
\centering
\includegraphics[width=1\linewidth]{figures/fig_10.0-day.jpg}
\vspace{-20pt}
\caption{10.0-day forecast results of different models.}
\label{fig_10.0-day}
\end{figure*}


\section{Detailed Mathematical Proof}
\label{sec:proof}
\textbf{Proof of Theorem 1}

Now we have N augmented data and we need to select the best from them. We consider both the quality and the diversity of these data and get the sampling strategy from an optimization problem.

We model the sampling strategy as a multinomial distribution supported on all the augmented data $S = \{\mathbf{X}_j\}_{j=1}^N$, which means that the sampling strategy $\pi=(\pi_1,...,\pi_N)^\top$ is the corresponding probabilities of selecting $\mathbf{X}_1,...,\mathbf{X}_N$, then we can model the expectation of the similarity as:
$$\begin{aligned}
 & \mathbb{E}_{Y_x,Y_{x^{\prime}}\in\mathcal{C}}\{g(x,x^{\prime})\mid S\} \\
 & =\quad\int g(\mathbf{x},\mathbf{x}^{\prime})\boldsymbol{\pi}(\mathbf{x})\mathrm{Pr}_{S}(Y_{x}\in\mathcal{C}\mid\boldsymbol{x}=\mathbf{x})\boldsymbol{\pi}(\mathbf{x}^{\prime})\mathrm{Pr}_{S}(Y_{x}\in\mathcal{C}\mid\boldsymbol{x}=\mathbf{x}^{\prime})d\mathbf{x}d\mathbf{x}^{\prime} \\
 & =\quad\sum_{i,j=1}^Ng(\mathbf{X}_i,\mathbf{X}_j)\pi_i\pi_j\mathrm{Pr}_{S}(Y_x\in\mathcal{C}\mid\boldsymbol{x}=\mathbf{X}_i)\mathrm{Pr}_{S}(Y_x\in\mathcal{C}\mid\boldsymbol{x}=\mathbf{X}_j),
\end{aligned}$$
where the set $\mathcal{C}$ denotes the criterion of selection we are using, the function $g$ can be chosen as any similarity metric function and $x$ means a random variable.

The core to solving the above optimization problem is to use predictive inference to approximate the conditional probability of $\{Y_x\in\mathcal{C}\}$ given $x = \mathbf{X}$
Let $\mu ( \mathbf{x} ) : = \mathbb{E} ( Y\mid \mathbf{X} = \mathbf{x} )$ be the oracle associated with $( \mathbf{X} , Y) .$ Denote $\theta_j=\mathbb{I}\{Y_j\in\mathcal{C}\}$. As the augmented data
$\mathbf{X}_1,...,\mathbf{X}_N$ are independently identically distributed, $\theta_1,...,\theta_N$ can be regarded as independent Bernoulli($q)$ variables with $q=\Pr(Y_j\in\mathcal{C}).$ The probability distribution of the predicted result $W_j$ for $j=1,...,N$ is
$$\Pr(W_j\mid\theta_j)=(1-\theta_j)f_0+\theta_jf_1,\quad$$
where $f_0$ and $f_1$ are the conditional distributions of $W_j$ on $Y_j \in \mathcal{C}$ or not.

Denote $T(w) = \frac{(1-q)f_0(W_j)}{f(W_j)}$, we can rewrite the expectation of the similarity as
$$\mathbb{E}_{Y_x,Y_{x^{\prime}}\in\mathcal{C}}\{g(x,x^{\prime})|S\}=\sum_{i,j=1}^Ng(\mathbf{X}_i,\mathbf{X}_j)\pi_i\pi_j(1-T_i)(1-T_j)=\boldsymbol{\pi}^\top A_\mathbb{T}\boldsymbol{\pi},$$

Next, we use the expectation to control the quality of the data.
$$\mathbb{E}\{\mathbb{I}(Y_x\not\in\mathcal{C})\mid S\}=\sum_{i=1}^N\Pr(Y_i\not\in\mathcal{C}\mid\mathbf{X}_i)\pi_i=\sum_{i=1}^N\pi_iT_i\leq\alpha,$$

In all, the optimization problem can be modeled as 
\begin{align}
    & \arg\min_{\boldsymbol{\pi}}\quad h(\boldsymbol{\pi},\mathbb{T}):=\boldsymbol{\pi}^\top A_\mathbb{T}\boldsymbol{\pi}, \\
    & \text{subject to} \quad
        \begin{cases}
            \sum_{i = 1}^N\pi_iT_i\leq\alpha, \\
            \sum_{i = 1}^N\pi_i = 1, \\
            0\leq\pi_i\leq m^{-1}, \quad 1\leq i\leq N.
        \end{cases}
\end{align}

where $m$ is used to control the maximum selection.

The best selection of K is determined by the strategy $\pi$ which serves as the solution to the above optimization problem.

\section{Additional Experiments}
\label{sec:more_experiments}
\subsection{Long-term forecasting experiment expansion}

In the long-term forecasting experiments, we compare the performance of different backbone models on the SWE benchmark, evaluating the relative L2 error for three variables (U, V, and H). Our setup inputs 5 frames and predicts 50 frames. For the SimVP-v2 model, using \method{} reduces the relative L2 error for SWE (u) from 0.0187 to 0.0154, SWE (v) from 0.0387 to 0.0342, and SWE (h) from 0.0443 to 0.0397. We visualize SWE (h) in 3D as shown in Figure~\ref{fig:case} [\textcolor{red}{I}]. For the ConvLSTM model, applying \method{} reduces the relative L2 error for SWE (u) from 0.0487 to 0.0321, SWE (v) from 0.0673 to 0.0351, and SWE (h) from 0.0762 to 0.0432. For the FNO model, using \method{} reduces the relative L2 error for SWE (u) from 0.0571 to 0.0502, SWE (v) from 0.0832 to 0.0653, and SWE (h) from 0.0981 to 0.0911. Overall, \method{} significantly improves the long-term forecasting accuracy of different backbone models.

\begin{figure*}[h]
    \centering
    \includegraphics[width=\textwidth]{image/casestudy.pdf}
    \caption{
    \textcolor{red}{I.} 3D visualization of the SWE(h), showing Ground-truth, SimVP-V2+BeamVQ predictions, and Error at T=1, 10, 20, 30, 40, 50. The first row shows Ground-truth, the second SimVP-V2+BeamVQ predictions, and the third Error. \textcolor{red}{II.} A case study. Building fire simulation with ventilation settings added to Wu's Prometheus~\cite{wu2024prometheus}. (a) Layout and HRR growth. (b) Comparison of physical metrics for different methods. (c) Ground-truth, ResNet+BeamVQ, and ResNet predictions.
    }
    \label{fig:case} 
\end{figure*}


\subsection{Experiment Statistical Significance}
\label{sec:significance}
To measure the statistical significance of our main experiment results, we choose three backbones to train on two datasets to run 5 times. 
Table~\ref{tab:significance} records the average and standard deviation of the test MSE loss.
The results prove that our method is statistically significant to outperform the baselines
because our confidence interval is always upper than the confidence interval of the baselines. 
Due to limited computation resources, we do not cover all ten backbones and five datasets, 
but we believe these results have shown that our method has consistent advantages.


\begin{table}[h]
\label{tab:significance}
\centering
\begin{scriptsize}
    \begin{sc}
    \caption{ The average and standard deviation of MSE in 5 runs}
    \label{tab:significance}
    \centering
        \renewcommand{\multirowsetup}{\centering}
        \setlength{\tabcolsep}{10pt}
        \begin{tabular}{l|cc|cc}
            \toprule
            
            \multirow{4}{*}{Model} & \multicolumn{4}{c}{Benchmarks}  \\
            \cmidrule(lr){2-5}
            & \multicolumn{2}{c}{NSE} &   \multicolumn{2}{c}{SEVIR}   \\
            \cmidrule(lr){2-5}
           & Ori & + BeamVQ & Ori & + BeamVQ  \\
            \midrule
            ConvLSTM &0.4092$\pm$0.0002 &\textbf{0.1277$\pm$0.0001}  & 0.1762 0.0007  & \textbf{0.1279$\pm$0.0009}  \\
            FNO &  0.2227$\pm$0.0003 &\textbf{0.1007 $\pm$0.0002}& 0.0787$\pm$0.0012 & \textbf{ 0.0437$\pm$0.0013} \\
            CNO & 0.2192 $\pm$0.0008 &\textbf{ 0.1492$\pm$0.0011}& 0.0057$\pm$0.0005 & \textbf{ 0.0053$\pm$0.0006} \\
            \bottomrule
        \end{tabular}
    \end{sc}

\end{scriptsize}
\end{table}



\end{document}

