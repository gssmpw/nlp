%%%%%%%% ICML 2025 EXAMPLE LATEX SUBMISSION FILE %%%%%%%%%%%%%%%%%

% \documentclass{article}
\documentclass{article}
% Recommended, but optional, packages for figures and better typesetting:
\usepackage{microtype}
\usepackage{graphicx}
\usepackage{subfigure}
\usepackage{booktabs} % for professional tables
\usepackage{hyperref}
% Attempt to make hyperref and algorithmic work together better:
\newcommand{\theHalgorithm}{\arabic{algorithm}}

% Use the following line for the initial blind version submitted for review:

% If accepted, instead use the following line for the camera-ready submission:
\usepackage[accepted]{icml2025}
\usepackage{amsmath}
\usepackage{mathrsfs}
\usepackage{amssymb}
\usepackage{mathtools}
\usepackage{amsthm}
\usepackage{url}            % simple URL typesetting
\usepackage{bm}
\usepackage{booktabs}       % professional-quality tables
\usepackage{amsfonts}       % blackboard math symbols
\usepackage{nicefrac}       % compact symbols for 1/2, etc.
\usepackage{microtype}      % microtypography
\usepackage{adjustbox}
\usepackage{blindtext}
\usepackage{enumitem}
\usepackage{listings}
\usepackage{pifont}
\usepackage{epsfig} % for postscript graphics files
\usepackage{times}
\usepackage{color}
% \usepackage[table]{xcolor}
\usepackage{multirow}
\usepackage{multicol}
\usepackage{xcolor}         % colors
\usepackage{makecell}
\usepackage{algorithm}
\usepackage{rotating}
\usepackage{subfigure}
\usepackage[T1]{fontenc}
\usepackage{algorithm}
\usepackage{algorithmic}
% \usepackage[noend]{algpseudocode}
\usepackage{colortbl}
\usepackage{color, soul}
\usepackage{arydshln} % For dashline
\usepackage{tcolorbox}
\definecolor{babyblue}{rgb}{0.54, 0.81, 0.94}
\definecolor{bisque}{rgb}{1.0, 0.89, 0.77}
\definecolor{bshade}{rgb}{0.55,0.75,0.95}
\def\eg{{\it e.g.}\xspace}
\def\ie{{\it i.e.}\xspace}
\definecolor{mygray}{gray}{.6}
\definecolor{myblue}{RGB}{89,158,254}
\definecolor{mygreen1}{RGB}{81,150,111}
\definecolor{mygreen2}{RGB}{93,174,86}
\definecolor{myred}{RGB}{160,0,0}
\definecolor{myyellow}{RGB}{227,207,87}
% For pictures
\usepackage{caption}
\usepackage[capitalize,noabbrev]{cleveref}
\usepackage{threeparttable}
\usepackage{wrapfig}

%%%%%%%%%%%%%%%%%%%%%%%%%%%%%%%%
% THEOREMS
%%%%%%%%%%%%%%%%%%%%%%%%%%%%%%%%
\theoremstyle{plain}
\newtheorem{theorem}{Theorem}[section]
\newtheorem{proposition}[theorem]{Proposition}
\newtheorem{lemma}[theorem]{Lemma}
\newtheorem{corollary}[theorem]{Corollary}
\theoremstyle{definition}
\newtheorem{definition}[theorem]{Definition}
\newtheorem{assumption}[theorem]{Assumption}
\theoremstyle{remark}
\newtheorem{remark}[theorem]{Remark}
\definecolor{babyblue}{rgb}{0.54, 0.81, 0.94}
\definecolor{bisque}{rgb}{1.0, 0.89, 0.77}
\definecolor{bshade}{rgb}{0.55,0.75,0.95}
\def\eg{{\it e.g.}\xspace}
\def\ie{{\it i.e.}\xspace}
\definecolor{mygray}{gray}{.6}
\definecolor{myblue}{RGB}{89,158,254}
\definecolor{mygreen1}{RGB}{81,150,111}
\definecolor{mygreen2}{RGB}{93,174,86}
\definecolor{myred}{RGB}{160,0,0}
\definecolor{myyellow}{RGB}{227,207,87}
\usepackage{bbding}
\newcommand{\cmark}{\ding{52}}%
\newcommand{\xmark}{\ding{55}}%
\newcommand{\mA}{\boldsymbol{A}}
\newcommand{\mE}{\boldsymbol{E}}
\newcommand{\mF}{\boldsymbol{F}}
\newcommand{\mH}{\boldsymbol{H}}
\newcommand{\mI}{\boldsymbol{I}}
\newcommand{\mM}{\boldsymbol{M}}
\newcommand{\mO}{\boldsymbol{O}}
\newcommand{\mP}{\boldsymbol{P}}
\newcommand{\mQ}{\boldsymbol{Q}}
\newcommand{\mR}{\boldsymbol{R}}
\newcommand{\mX}{\boldsymbol{X}}
\usepackage{enumitem}
\let\oldding\ding% Store old \ding in \oldding
\renewcommand{\ding}[2][1]{\scalebox{#1}{\oldding{#2}}}
% Todonotes is useful during development; simply uncomment the next line
%    and comment out the line below the next line to turn off comments
%\usepackage[disable,textsize=tiny]{todonotes}
\usepackage[textsize=tiny]{todonotes}
\newcommand{\ceil}[1]{\left\lceil #1 \right\rceil}
\newcommand{\floor}[1]{\left\lfloor #1 \right\rfloor}
\newcommand{\round}[1]{\left\lfloor #1 \right\rceil}
\newcommand{\sign}{\mathrm{sign}}
\newcommand{\tofloat}[1]{\mbox{float}\left(#1\right)}
\newcommand{\toint}[2]{\mbox{int}_{#1}\left(#2\right)}

\newcommand{\methodname}{MQuant}
\newcommand{\cc}{\color[rgb]{0,0.6,0.3}\checkmark}
\newcommand{\xx}{\color[rgb]{0.6,0,0}{\ding{55}}}

% The \icmltitle you define below is probably too long as a header.
% Therefore, a short form for the running title is supplied here:
% \icmltitlerunning{Submission and Formatting Instructions for ICML 2025}

\begin{document}

\twocolumn[
\icmltitle{MQuant: Unleashing the Inference Potential of Multimodal Large Language Models via Full Static Quantization}
%maybe need change our title:


% It is OKAY to include author information, even for blind
% submissions: the style file will automatically remove it for you
% unless you've provided the [accepted] option to the icml2025
% package.

% List of affiliations: The first argument should be a (short)
% identifier you will use later to specify author affiliations
% Academic affiliations should list Department, University, City, Region, Country
% Industry affiliations should list Company, City, Region, Country

% You can specify symbols, otherwise they are numbered in order.
% Ideally, you should not use this facility. Affiliations will be numbered
% in order of appearance and this is the preferred way.
\icmlsetsymbol{equal}{$\ast$}
\icmlsetsymbol{intern}{$\dagger$}
\icmlsetsymbol{corre}{$\Diamond$}

\begin{icmlauthorlist}
\icmlauthor{JiangYong Yu}{houmo,equal}
\icmlauthor{Sifan Zhou}{houmo,seu,equal,intern}
\icmlauthor{Dawei Yang}{houmo,corre}
\icmlauthor{Shuo Wang}{houmo} 
\icmlauthor{Shuoyu Li}{houmo,xjtu,intern} \\
\icmlauthor{Xing Hu}{houmo} 
\icmlauthor{Chen Xu}{houmo}
% \icmlauthor{}{sch}
\icmlauthor{Zukang Xu}{houmo}
\icmlauthor{Changyong Shu}{houmo}
\icmlauthor{Zhihang Yuan}{houmo}
\end{icmlauthorlist}

\icmlaffiliation{houmo}{Houmo AI}
\icmlaffiliation{seu}{Southeast University}
\icmlaffiliation{xjtu}{Xi'an Jiaotong University}

\icmlcorrespondingauthor{Dawei Yang}{}
% You may provide any keywords that you
% find helpful for describing your paper; these are used to populate
% the "keywords" metadata in the PDF but will not be shown in the document
% \icmlkeywords{Machine Learning, ICML}

\vskip 0.3in
]

\printAffiliationsAndNotice{\corresponding,\icmlEqualContribution,\intern}
% \printAffiliationsAndNotice{\intern}% otherwise use the standard text.
\begin{abstract}


The choice of representation for geographic location significantly impacts the accuracy of models for a broad range of geospatial tasks, including fine-grained species classification, population density estimation, and biome classification. Recent works like SatCLIP and GeoCLIP learn such representations by contrastively aligning geolocation with co-located images. While these methods work exceptionally well, in this paper, we posit that the current training strategies fail to fully capture the important visual features. We provide an information theoretic perspective on why the resulting embeddings from these methods discard crucial visual information that is important for many downstream tasks. To solve this problem, we propose a novel retrieval-augmented strategy called RANGE. We build our method on the intuition that the visual features of a location can be estimated by combining the visual features from multiple similar-looking locations. We evaluate our method across a wide variety of tasks. Our results show that RANGE outperforms the existing state-of-the-art models with significant margins in most tasks. We show gains of up to 13.1\% on classification tasks and 0.145 $R^2$ on regression tasks. All our code and models will be made available at: \href{https://github.com/mvrl/RANGE}{https://github.com/mvrl/RANGE}.

\end{abstract}


\section{Introduction}
Backdoor attacks pose a concealed yet profound security risk to machine learning (ML) models, for which the adversaries can inject a stealth backdoor into the model during training, enabling them to illicitly control the model's output upon encountering predefined inputs. These attacks can even occur without the knowledge of developers or end-users, thereby undermining the trust in ML systems. As ML becomes more deeply embedded in critical sectors like finance, healthcare, and autonomous driving \citep{he2016deep, liu2020computing, tournier2019mrtrix3, adjabi2020past}, the potential damage from backdoor attacks grows, underscoring the emergency for developing robust defense mechanisms against backdoor attacks.

To address the threat of backdoor attacks, researchers have developed a variety of strategies \cite{liu2018fine,wu2021adversarial,wang2019neural,zeng2022adversarial,zhu2023neural,Zhu_2023_ICCV, wei2024shared,wei2024d3}, aimed at purifying backdoors within victim models. These methods are designed to integrate with current deployment workflows seamlessly and have demonstrated significant success in mitigating the effects of backdoor triggers \cite{wubackdoorbench, wu2023defenses, wu2024backdoorbench,dunnett2024countering}.  However, most state-of-the-art (SOTA) backdoor purification methods operate under the assumption that a small clean dataset, often referred to as \textbf{auxiliary dataset}, is available for purification. Such an assumption poses practical challenges, especially in scenarios where data is scarce. To tackle this challenge, efforts have been made to reduce the size of the required auxiliary dataset~\cite{chai2022oneshot,li2023reconstructive, Zhu_2023_ICCV} and even explore dataset-free purification techniques~\cite{zheng2022data,hong2023revisiting,lin2024fusing}. Although these approaches offer some improvements, recent evaluations \cite{dunnett2024countering, wu2024backdoorbench} continue to highlight the importance of sufficient auxiliary data for achieving robust defenses against backdoor attacks.

While significant progress has been made in reducing the size of auxiliary datasets, an equally critical yet underexplored question remains: \emph{how does the nature of the auxiliary dataset affect purification effectiveness?} In  real-world  applications, auxiliary datasets can vary widely, encompassing in-distribution data, synthetic data, or external data from different sources. Understanding how each type of auxiliary dataset influences the purification effectiveness is vital for selecting or constructing the most suitable auxiliary dataset and the corresponding technique. For instance, when multiple datasets are available, understanding how different datasets contribute to purification can guide defenders in selecting or crafting the most appropriate dataset. Conversely, when only limited auxiliary data is accessible, knowing which purification technique works best under those constraints is critical. Therefore, there is an urgent need for a thorough investigation into the impact of auxiliary datasets on purification effectiveness to guide defenders in  enhancing the security of ML systems. 

In this paper, we systematically investigate the critical role of auxiliary datasets in backdoor purification, aiming to bridge the gap between idealized and practical purification scenarios.  Specifically, we first construct a diverse set of auxiliary datasets to emulate real-world conditions, as summarized in Table~\ref{overall}. These datasets include in-distribution data, synthetic data, and external data from other sources. Through an evaluation of SOTA backdoor purification methods across these datasets, we uncover several critical insights: \textbf{1)} In-distribution datasets, particularly those carefully filtered from the original training data of the victim model, effectively preserve the model’s utility for its intended tasks but may fall short in eliminating backdoors. \textbf{2)} Incorporating OOD datasets can help the model forget backdoors but also bring the risk of forgetting critical learned knowledge, significantly degrading its overall performance. Building on these findings, we propose Guided Input Calibration (GIC), a novel technique that enhances backdoor purification by adaptively transforming auxiliary data to better align with the victim model’s learned representations. By leveraging the victim model itself to guide this transformation, GIC optimizes the purification process, striking a balance between preserving model utility and mitigating backdoor threats. Extensive experiments demonstrate that GIC significantly improves the effectiveness of backdoor purification across diverse auxiliary datasets, providing a practical and robust defense solution.

Our main contributions are threefold:
\textbf{1) Impact analysis of auxiliary datasets:} We take the \textbf{first step}  in systematically investigating how different types of auxiliary datasets influence backdoor purification effectiveness. Our findings provide novel insights and serve as a foundation for future research on optimizing dataset selection and construction for enhanced backdoor defense.
%
\textbf{2) Compilation and evaluation of diverse auxiliary datasets:}  We have compiled and rigorously evaluated a diverse set of auxiliary datasets using SOTA purification methods, making our datasets and code publicly available to facilitate and support future research on practical backdoor defense strategies.
%
\textbf{3) Introduction of GIC:} We introduce GIC, the \textbf{first} dedicated solution designed to align auxiliary datasets with the model’s learned representations, significantly enhancing backdoor mitigation across various dataset types. Our approach sets a new benchmark for practical and effective backdoor defense.




\section{Brief Review of 3D Gaussian Splatting}
\label{sec:prelim}
For the sake of clarity, we first briefly review 3D Gaussian Splatting (3DGS)~\cite{kerbl202333dgs}, an explicit representation of a 3D scene for providing effective image rendering. 
% We also provide brief reviews of two powerful extensions of 3DGS, Gaussian Grouping~\cite{ye2023gaussiangrouping} and Relightable Gaussian~\cite{gao2023relightable}, which equip 3DGS with segmentation and relighting abilities and are utilized together with 3DGS as the backbone representation in our work. 


Given $K$ multi-view images $I_{1:K} = \{I_1, I_2, ..., I_K\}$ with corresponding camera poses $\xi_{1:K} = \{\xi_1, \xi_2, ..., \xi_K\}$ of a 3D scene, a scene-specific 3DGS is applied to model the scene with $N$ learnable 3D Gaussian ellipsoids (i.e., $G_{1:N} = \{G_1, G_2, ..., G_N \}$). Each Gaussian $G_i$ is parameterized with its 3-dimensional centroid $\mathbf{p}_i \in \mathbb{R}^{3}$, a 3-dimensional standard deviation $\mathbf{s}_i \in \mathbb{R}^{3}$, a 4-dimensional rotational quaternion $\mathbf{q}_i \in \mathbb{R}^{4}$, an opacity ${\alpha}_i \in [0,1]$, and color coefficients $\mathbf{c}_i$ for spherical harmonics in degree of 3. Hence, $G_i$ is represented with a set of the above parameters (i.e., $G_i = \{\mathbf{p}_i, \mathbf{s}_i, \mathbf{q}_i, {\alpha}_i, \mathbf{c}_i\}$). To model the scene with $G_{1:N}$, 2D images $\hat{I_{1:K}} = \{\hat{I_1}, \hat{I_2}, ..., \hat{I_K}\}$ are sequentially rendered from $G_{1:N}$ using $\xi_{1:K} = \{\xi_1, \xi_2, ..., \xi_K\}$ (please refer to~\cite{kerbl202333dgs} for a detailed rendering process), and supervised with $I_{1:K}$ by the rendering loss:
\begin{equation} \label{Limage}
    \mathcal{L}_{image} = \sum_{k\in {1...K}}\lambda\| I_k - {\hat{I_k}}\|_1 + \mathcal{L}_{SSIM}(I_k, \hat{I_k}),
\end{equation}
where $\mathcal{L}_{SSIM}(\cdot)$ represents a SSIM loss and $\lambda$ is a hyper-parameter (set to $0.2$ as mentioned in~\cite{kerbl202333dgs}).



% \subsection{Gaussian Grouping}
% To overcome the lack of fine-grained scene understanding in 3DGS, Gaussian Grouping~\cite{ye2023gaussiangrouping} extends 3DGS by incorporating segmentation capabilities. Along with $I_{1:K}$, Gaussian Grouping additionally takes the Segment Anything Model (SAM) to produce 2D semantic segmentation masks $S_{1:K} = \{S_1, S_2, ..., S_K\}$ from multiple views as inputs, and an additional 16-dimensional parameter $\mathbf{e}_i \in \mathbb{R}^{16}$ is introduced to represent a 3D Identity Encoding for each Gaussian $G_i$. Therefore, each Gaussian $G_i$ is extended as $G_i = \{\mathbf{p}_i, \mathbf{s}_i, \mathbf{q}_i, {\alpha}_i, \mathbf{c}_i, \mathbf{e}_i\}$. To make sure $G_{1:K}$ learns to segment each object represented by $S_{1:K}$ in the scene, a 2D identity loss $\mathcal{L}_{id}$ is applied by calculating cross-entropy between $\hat{S}_{1:K}$ and $S_{1:K}$, where $\hat{S}_{1:K} = \{\hat{S}_1, \hat{S}_2, ... , S_K\}$ denotes the rendered segmentation maps from $G_{1:K}$. Additionally, to further ensure that the Gaussians having the same identities are grouped together, a 3D regularization loss $\mathcal{L}_{3D}$ is applied to enforce each $G_i$'s k-nearest 3D spatial neighbors to be close in their feature distance of Identity Encodings. Please refer to the original paper~\cite{ye2023gaussiangrouping} for detailed formulations of segmentation map rendering and $\mathcal{L}_{3D}$. The design of Gaussian Grouping ensures that the segmentation results are coherent across multiple views, enabling the automatic generation of binary masks for any queried object in the scene.

% \subsection{Relightable Gaussians}
% Different from Gaussian Grouping, Relightable Gaussians~\cite{gao2023relightable} extends the capabilities of Gaussian Splatting by incorporating Disney-BRDF~\cite{burley2012brdf} decomposition and ray tracing to achieve realistic point cloud relighting. 
% % Unlike traditional Gaussian Splatting, which primarily focuses on appearance and geometry modeling, Relightable Gaussians also aim to model the physical interaction of light with different surfaces in the scene.
% Specifically, for each Gaussian $G_i$, the original color coefficients $\mathbf{c}_i$ is decomposed into a 3-dimensional base color $\mathbf{b}_i \in [0,1]^3$, a 1-dimensional roughness $r \in [0,1]$, and incident light coefficients $\mathbf{l}_i$ for spherical harmonics in degree of 3. Subsequently, the Physical-Based Rendering (PBR) process and a point-based ray tracing are applied to obtain the colored 2D images $\hat{I}^{PBR}_{1:K}$ and supervised by $I_{1:K}$ using the aforementioned $\mathcal{L}_{image}$ in Eqn.~\ref{Limage}. Besides the above extensions on PBR for relighting, Relightable Gaussians also introduces a 3-dimensional normal $\mathbf{n}_i$ for $G_i$ and leverages several techniques, including an unsupervised estimation of a depth map $D_i$ from each input view $\xi_i$, to enhance the geometry accuracy and smoothness. Please refer to the original paper of Relightable Gaussians~\cite{gao2023relightable} for detailed explanations.  

\pj{Our pipeline for 3D Inpainting is built on top of the 3DGS model. Additionally, we incorporate the design of Gaussian Grouping~\cite{ye2023gaussiangrouping} to introduce a 16-dimensional semantic feature $\mathbf{e}_i \in \mathbb{R}^{16}$ for each Gaussian $G_i$, so that the 2D segmentation maps of the Gaussians $G_{1:K}$ is rendered and the object mask for the object to be removed can be directly produced, as mentioned in Sect.~\ref{subsec:3Dinpaint}.
By combining these methods as our backbone, we are able to perform an automatic inpainting mask generation and a reliable depth estimation for depth-guided 3D inpainting. Please refer to our Supplementary material for a more detailed explanation of our backbones.}

% the backbone representation by parameterizing each Gaussian $G_i$ as $G_i = \{\mathbf{p}_i, \mathbf{s}_i, \mathbf{q}_i, {\alpha}_i, \mathbf{c}_i, \mathbf{e}_i, \mathbf{b}_i, r,  \mathbf{l}_i, \mathbf{n}_i\}$. By combining these methods, we are able to perform an automatic inpainting mask generation and a reliable depth estimation for depth-guided 3D inpainting.
Effective human-robot cooperation in CoNav-Maze hinges on efficient communication. Maximizing the human’s information gain enables more precise guidance, which in turn accelerates task completion. Yet for the robot, the challenge is not only \emph{what} to communicate but also \emph{when}, as it must balance gathering information for the human with pursuing immediate goals when confident in its navigation.

To achieve this, we introduce \emph{Information Gain Monte Carlo Tree Search} (IG-MCTS), which optimizes both task-relevant objectives and the transmission of the most informative communication. IG-MCTS comprises three key components:
\textbf{(1)} A data-driven human perception model that tracks how implicit (movement) and explicit (image) information updates the human’s understanding of the maze layout.
\textbf{(2)} Reward augmentation to integrate multiple objectives effectively leveraging on the learned perception model.
\textbf{(3)} An uncertainty-aware MCTS that accounts for unobserved maze regions and human perception stochasticity.
% \begin{enumerate}[leftmargin=*]
%     \item A data-driven human perception model that tracks how implicit (movement) and explicit (image transmission) information updates the human’s understanding of the maze layout.
%     \item Reward augmentation to integrate multiple objectives effectively leveraging on the learned perception model.
%     \item An uncertainty-aware MCTS that accounts for unobserved maze regions and human perception stochasticity.
% \end{enumerate}

\subsection{Human Perception Dynamics}
% IG-MCTS seeks to optimize the expected novel information gained by the human through the robot’s actions, including both movement and communication. Achieving this requires a model of how the human acquires task-relevant information from the robot.

% \subsubsection{Perception MDP}
\label{sec:perception_mdp}
As the robot navigates the maze and transmits images, humans update their understanding of the environment. Based on the robot's path, they may infer that previously assumed blocked locations are traversable or detect discrepancies between the transmitted image and their map.  

To formally capture this process, we model the evolution of human perception as another Markov Decision Process, referred to as the \emph{Perception MDP}. The state space $\mathcal{X}$ represents all possible maze maps. The action space $\mathcal{S}^+ \times \mathcal{O}$ consists of the robot's trajectory between two image transmissions $\tau \in \mathcal{S}^+$ and an image $o \in \mathcal{O}$. The unknown transition function $F: (x, (\tau, o)) \rightarrow x'$ defines the human perception dynamics, which we aim to learn.

\subsubsection{Crowd-Sourced Transition Dataset}
To collect data, we designed a mapping task in the CoNav-Maze environment. Participants were tasked to edit their maps to match the true environment. A button triggers the robot's autonomous movements, after which it captures an image from a random angle.
In this mapping task, the robot, aware of both the true environment and the human’s map, visits predefined target locations and prioritizes areas with mislabeled grid cells on the human’s map.
% We assume that the robot has full knowledge of both the actual environment and the human’s current map. Leveraging this knowledge, the robot autonomously navigates to all predefined target locations. It then randomly selects subsequent goals to reach, prioritizing grid locations that remain mislabeled on the human’s map. This ensures that the robot’s actions are strategically focused on providing useful information to improve map accuracy.

We then recruited over $50$ annotators through Prolific~\cite{palan2018prolific} for the mapping task. Each annotator labeled three randomly generated mazes. They were allowed to proceed to the next maze once the robot had reached all four goal locations. However, they could spend additional time refining their map before moving on. To incentivize accuracy, annotators receive a performance-based bonus based on the final accuracy of their annotated map.


\subsubsection{Fully-Convolutional Dynamics Model}
\label{sec:nhpm}

We propose a Neural Human Perception Model (NHPM), a fully convolutional neural network (FCNN), to predict the human perception transition probabilities modeled in \Cref{sec:perception_mdp}. We denote the model as $F_\theta$ where $\theta$ represents the trainable weights. Such design echoes recent studies of model-based reinforcement learning~\cite{hansen2022temporal}, where the agent first learns the environment dynamics, potentially from image observations~\cite{hafner2019learning,watter2015embed}.

\begin{figure}[t]
    \centering
    \includegraphics[width=0.9\linewidth]{figures/ICML_25_CNN.pdf}
    \caption{Neural Human Perception Model (NHPM). \textbf{Left:} The human's current perception, the robot's trajectory since the last transmission, and the captured environment grids are individually processed into 2D masks. \textbf{Right:} A fully convolutional neural network predicts two masks: one for the probability of the human adding a wall to their map and another for removing a wall.}
    \label{fig:nhpm}
    \vskip -0.1in
\end{figure}

As illustrated in \Cref{fig:nhpm}, our model takes as input the human’s current perception, the robot’s path, and the image captured by the robot, all of which are transformed into a unified 2D representation. These inputs are concatenated along the channel dimension and fed into the CNN, which outputs a two-channel image: one predicting the probability of human adding a new wall and the other predicting the probability of removing a wall.

% Our approach builds on world model learning, where neural networks predict state transitions or environmental updates based on agent actions and observations. By leveraging the local feature extraction capabilities of CNNs, our model effectively captures spatial relationships and interprets local changes within the grid maze environment. Similar to prior work in localization and mapping, the CNN architecture is well-suited for processing spatially structured data and aligning the robot’s observations with human map updates.

To enhance robustness and generalization, we apply data augmentation techniques, including random rotation and flipping of the 2D inputs during training. These transformations are particularly beneficial in the grid maze environment, which is invariant to orientation changes.

\subsection{Perception-Aware Reward Augmentation}
The robot optimizes its actions over a planning horizon \( H \) by solving the following optimization problem:
\begin{subequations}
    \begin{align}
        \max_{a_{0:H-1}} \;
        & \mathop{\mathbb{E}}_{T, F} \left[ \sum_{t=0}^{H-1} \gamma^t \left(\underbrace{R_{\mathrm{task}}(\tau_{t+1}, \zeta)}_{\text{(1) Task reward}} + \underbrace{\|x_{t+1}-x_t\|_1}_{\text{(2) Info reward}}\right)\right] \label{obj}\\ 
        \subjectto \quad
        &x_{t+1} = F(x_t, (\tau_t, a_t)), \quad a_t\in\Ocal \label{const:perception_update}\\ 
        &\tau_{t+1} = \tau_t \oplus T(s_t, a_t), \quad a_t\in \Ucal\label{const:history_update}
    \end{align}
\end{subequations} 

The objective in~\eqref{obj} maximizes the expected cumulative reward over \( T \) and \( F \), reflecting the uncertainty in both physical transitions and human perception dynamics. The reward function consists of two components: 
(1) The \emph{task reward} incentivizes efficient navigation. The specific formulation for the task in this work is outlined in \Cref{appendix:task_reward}.
(2) The \emph{information reward} quantifies the change in the human’s perception due to robot actions, computed as the \( L_1 \)-norm distance between consecutive perception states.  

The constraint in~\eqref{const:history_update} ensures that for movement actions, the trajectory history \( \tau_t \) expands with new states based on the robot’s chosen actions, where \( s_t \) is the most recent state in \( \tau_t \), and \( \oplus \) represents sequence concatenation. 
In constraint~\eqref{const:perception_update}, the robot leverages the learned human perception dynamics \( F \) to estimate the evolution of the human’s understanding of the environment from perception state $x_t$ to $x_{t+1}$ based on the observed trajectory \( \tau_t \) and transmitted image \( a_t\in\Ocal \). 
% justify from a cognitive science perspective
% Cognitive science research has shown that humans read in a way to maximize the information gained from each word, aligning with the efficient coding principle, which prioritizes minimizing perceptual errors and extracting relevant features under limited processing capacity~\cite{kangassalo2020information}. Drawing on this principle, we hypothesize that humans similarly prioritize task-relevant information in multimodal settings. To accommodate this cognitive pattern, our robot policy selects and communicates high information-gain observations to human operators, akin to summarizing key insights from a lengthy article.
% % While the brain naturally seeks to gain information, the brain employs various strategies to manage information overload, including filtering~\cite{quiroga2004reducing}, limiting/working memory, and prioritizing information~\cite{arnold2023dealing}.
% In this context of our setup, we optimize the selection of camera angles to maximize the human operator's information gain about the environment. 

\subsection{Information Gain Monte Carlo Tree Search (IG-MCTS)}
IG-MCTS follows the four stages of Monte Carlo tree search: \emph{selection}, \emph{expansion}, \emph{rollout}, and \emph{backpropagation}, but extends it by incorporating uncertainty in both environment dynamics and human perception. We introduce uncertainty-aware simulations in the \emph{expansion} and \emph{rollout} phases and adjust \emph{backpropagation} with a value update rule that accounts for transition feasibility.

\subsubsection{Uncertainty-Aware Simulation}
As detailed in \Cref{algo:IG_MCTS}, both the \emph{expansion} and \emph{rollout} phases involve forward simulation of robot actions. Each tree node $v$ contains the state $(\tau, x)$, representing the robot's state history and current human perception. We handle the two action types differently as follows:
\begin{itemize}
    \item A movement action $u$ follows the environment dynamics $T$ as defined in \Cref{sec:problem}. Notably, the maze layout is observable up to distance $r$ from the robot's visited grids, while unexplored areas assume a $50\%$ chance of walls. In \emph{expansion}, the resulting search node $v'$ of this uncertain transition is assigned a feasibility value $\delta = 0.5$. In \emph{rollout}, the transition could fail and the robot remains in the same grid.
    
    \item The state transition for a communication step $o$ is governed by the learned stochastic human perception model $F_\theta$ as defined in \Cref{sec:nhpm}. Since transition probabilities are known, we compute the expected information reward $\bar{R_\mathrm{info}}$ directly:
    \begin{align*}
        \bar{R_\mathrm{info}}(\tau_t, x_t, o_t) &= \mathbb{E}_{x_{t+1}}\|x_{t+1}-x_t\|_1 \\
        &= \|p_\mathrm{add}\|_1 + \|p_\mathrm{remove}\|_1,
    \end{align*}
    where $(p_\mathrm{add}, p_\mathrm{remove}) \gets F_\theta(\tau_t, x_t, o_t)$ are the estimated probabilities of adding or removing walls from the map. 
    Directly computing the expected return at a node avoids the high number of visitations required to obtain an accurate value estimate.
\end{itemize}

% We denote a node in the search tree as $v$, where $s(v)$, $r(v)$, and $\delta(v)$ represent the state, reward, and transition feasibility at $v$, respectively. The visit count of $v$ is denoted as $N(v)$, while $Q(v)$ represents its total accumulated return. The set of child nodes of $v$ is denoted by $\mathbb{C}(v)$.

% The goal of each search is to plan a sequence for the robot until it reaches a goal or transmits a new image to the human. We initialize the search tree with the current human guidance $\zeta$, and the robot's approximation of human perception $x_0$. Each search node consists consists of the state information required by our reward augmentation: $(\tau, x)$. A node is terminal if it is the resulting state of a communication step, or if the robot reaches a goal location. 

% A rollout from the expanded node simulates future transitions until reaching a terminal state or a predefined depth $H$. Actions are selected randomly from the available action set $\mathcal{A}(s)$. If an action's feasibility is uncertain due to the environment's unknown structure, the transition occurs with probability $\delta(s, a)$. When a random number draw deems the transition infeasible, the state remains unchanged. On the other hand, for communication steps, we don't resolve the uncertainty but instead compute the expected information gain reward: \philip{TODO: adjust notation}
% \begin{equation}
%     \mathbb{E}\left[R_\mathrm{info}(\tau, x')\right] = \sum \mathrm{NPM(\tau, o)}.
% \end{equation}

\subsubsection{Feasibility-Adjusted Backpropagation}
During backpropagation, the rewards obtained from the simulation phase are propagated back through the tree, updating the total value $Q(v)$ and the visitation count $N(v)$ for all nodes along the path to the root. Due to uncertainty in unexplored environment dynamics, the rollout return depends on the feasibility of the transition from the child node. Given a sample return \(q'_{\mathrm{sample}}\) at child node \(v'\), the parent node's return is:
\begin{equation}
    q_{\mathrm{sample}} = r + \gamma \left[ \delta' q'_{\mathrm{sample}} + (1 - \delta') \frac{Q(v)}{N(v)} \right],
\end{equation}
where $\delta'$ represents the probability of a successful transition. The term \((1 - \delta')\) accounts for failed transitions, relying instead on the current value estimate.

% By incorporating uncertainty-aware rollouts and backpropagation, our approach enables more robust decision-making in scenarios where the environment dynamics is unknown and avoids simulation of the stochastic human perception dynamics.

\section{Experiments and analysis} \label{sec:5}
This section presents comprehensive experiments on the ATR2-HUTD dataset to evaluate the effectiveness of the proposed method. 
Section~\ref{sec:4.1} outlines the experimental metrics used. 
Section~\ref{sec:4.2} details the network architecture, comparison methods, experimental setup, and parameter configurations. 
To highlight the superiority of the proposed method, Section~\ref{sec:4.3} provides both quantitative analysis and visual evaluations across all comparison methods. 
Section~\ref{sec:4.4} includes ablation studies to assess the contributions of different model components, while Section~\ref{sec:4.5} presents a parameter sensitivity analysis.
\subsection{Evaluation Indicators}\label{sec:4.1}
To quantitatively assess the performance of the proposed method, we employ three widely recognized evaluation metrics in the HTD field.
\par
\textbf{(\romannumeral1) Receiver Operating Characteristic (ROC)~\cite{ROC, ROC3D}:} 
The ROC curve offers an unbiased, threshold-independent evaluation of detection performance. This paper presents three 2D ROC curves: $( \mathrm{P}_{\mathrm{d}}, \mathrm{P}_{\mathrm{f}})$, $( \mathrm{P}_{\mathrm{d}}, \tau)$, and $( \mathrm{P}_{\mathrm{f}}, \tau)$, along with a 3D ROC curve~\cite{ROC3D} of $(\tau, \mathrm{P}_{\mathrm{d}}, \mathrm{P}_{\mathrm{f}})$ for a comprehensive performance evaluation. A detector with ROC curves closer to the upper left, upper right, and lower left corners generally exhibits superior HTD performance.
\par
\textbf{(\romannumeral2) Area Under the ROC Curve (AUC)~\cite{Zhang2015}:} 
To address challenges in visually comparing ROC curves, we compute the area under each of the three 2D ROC curves: $\text{AUC}_{( \mathrm{P}_{\mathrm{d}}, \mathrm{P}_{\mathrm{f}})}$, $\text{AUC}_{( \mathrm{P}_{\mathrm{d}}, \tau)}$, and $\text{AUC}_{( \mathrm{P}_{\mathrm{f}}, \tau)}$. Larger AUC values indicate better performance, with $\text{AUC}_{( \mathrm{P}_{\mathrm{d}}, \mathrm{P}_{\mathrm{f}})} \to 1$, $\text{AUC}_{( \mathrm{P}_{\mathrm{d}}, \tau)} \to 1$, and $\text{AUC}_{( \mathrm{P}_{\mathrm{f}}, \tau)} \to 0$ signifying superior detection performance. Additionally, two AUC-based metrics are introduced for a more comprehensive evaluation:
\begin{equation}
    \mathrm{AUC}_{\mathrm{OA}} = \mathrm{AUC}_{\left(P_f, P_d\right)} + \mathrm{AUC}_{\left(\tau, P_d\right)} - \mathrm{AUC}_{\left(\tau, P_f\right)},
\end{equation}
\begin{equation}
    \mathrm{AUC}_{\mathrm{SNPR}} = \frac{\mathrm{AUC}_{\left(\tau, P_d\right)}}{\mathrm{AUC}_{\left(\tau, P_f\right)}},
\end{equation}
where higher values of $\mathrm{AUC}_{\mathrm{OA}} \to 2$ and $\mathrm{AUC}_{\mathrm{SNPR}} \to +\infty$ indicate improved detector performance.
% \textbf{(\romannumeral3) Separability Map~\cite{Liu2022}:} The degree of separation between the targets and backgrounds in the detection map is a critical performance indicator for UTD methods. 
% Thus, we also utilize the separability map for quantitative comparison in this study. 
% Specifically, the separability map uses green and blue boxes to represent the statistics of the target and background, respectively. 
% The horizontal line within each box indicates the median value, while the upper and lower whiskers denote the maximum and minimum values, providing a clear representation of the data range and central tendency. 
% \par
% A larger overlap between the two boxes suggests that the statistics of the target and background are similar, indicating poor separation between them. 
% Conversely, less overlap indicates better separation. 
% Moreover, background suppression is considered more effective when the blue box is closer to the ordinate 0, while higher target prominence is indicated when the green box is closer to ordinate 1.
% \clearpage
\subsection{Experimental Details and Settings}\label{sec:4.2}
\textbf{(\romannumeral1) Experimental Details:} 
The experimental setup and details of the proposed method are as follows. Unless otherwise specified, the parameters are applied consistently across all sub-datasets. The method consists of three core components: the RGC module, the HLCL module, and the SPL strategy, each contributing significantly to performance.

In the RGC module, unsupervised clustering is performed using the K-Means~\cite{Sinaga2020} algorithm, with cluster numbers set to 36, 39, and 42 for the lake, river, and sea sub-datasets, respectively, based on environmental complexity and waterbed characteristics.

The HLCL module employs the 3D-ResNet50~\cite{Jiang2019} network for spectral-spatial feature extraction. To enhance robustness and contrastive learning, untargeted FGSM~\cite{GoodfellowSS14} data augmentation is applied with a maximum perturbation of $\epsilon=0.1$ under the $l_{\infty}$ norm. The hybrid-level contrastive learning framework is trained for 50 epochs per SPL iteration. The Adam optimizer is used with a batch size of 256. The initial learning rate is $5\times10^{-3}$, decaying to $5\times10^{-5}$ through a cosine annealing schedule after 100 epochs, and a weight decay of $1\times10^{-4}$ is applied to reduce overfitting.

The SPL strategy is executed for 10 iterations across all sub-datasets to ensure convergence and computational efficiency.

For HUTD, as described in Section~\ref{sec3.4}, we use learned representations combined with basic hyperspectral detectors. To isolate the effect of detectors on performance, we employ two classic detectors, CEM~\cite{KRUSE1993145} and SAM~\cite{Manolakis2002}, as baseline methods.

\textbf{(\romannumeral2) Experimental Settings:} 
We compare the proposed method against several state-of-the-art (SOTA) HTD and HUTD methods, including two traditional HTD detectors (CEM and SAM), two advanced HTD methods (IEEPST~\cite{IEEPST} and MCLT~\cite{Wang2024}), and four HUTD methods (UTD-Net~\cite{Qi2021}, TUTDF~\cite{LiZheyong2023}, TDSS-UTD~\cite{Li2023}, and NUN-UTD~\cite{Liu2024}).

To ensure fairness, each method is executed with the original hyperparameter settings as specified in their respective publications. All experiments are conducted on a machine equipped with seven NVIDIA A6000 GPUs, an AMD Ryzen 5995WX CPU, and 128 GB of RAM, running Ubuntu 22.04.

\subsection{Main Results} \label{sec:4.3}
\textbf{(\romannumeral1) Detection Maps:} Figs.~\ref{fig:C1-1} to~\ref{fig:C1-2} present detection maps from the ATR2-HUTD-Lake sub-dataset, offering a qualitative comparison of the evaluated methods.
The detection maps of other sub-datasets are provided in the supplementary material.
\par
\begin{figure*}[!t]                 
    \centering                    
    \includegraphics[width=2\columnwidth]{images/C1-1.jpg}                     
    \caption{Detection maps of ATR2-HUTD Lake Scene1. (a) Pseudo-color image. (b) Ground truth. (c) CEM. (d) SAM. (e) IEEPST. (f) MCLT. (g) UTD-Net. (h) TUTDF. (i) TDSS-UTD. (j) NUN-UTD. (m) HUCLNet+CEM. (n) HUCLNet+SAM.}                  
    \label{fig:C1-1}    
\end{figure*}
\begin{figure*}[!t]                 
    \centering                    
    \includegraphics[width=2\columnwidth]{images/C1-2.jpg}                     
    \caption{Detection maps of ATR2-HUTD Lake Scene2. (a) Pseudo-color image. (b) Ground truth. (c) CEM. (d) SAM. (e) IEEPST. (f) MCLT. (g) UTD-Net. (h) TUTDF. (i) TDSS-UTD. (j) NUN-UTD. (m) HUCLNet+CEM. (n) HUCLNet+SAM.}                    
    \label{fig:C1-2}    
\end{figure*}
% \begin{figure*}[!t]                 
%     \centering                    
%     \includegraphics[width=2\columnwidth]{images/C1-3.jpg}                     
%     \caption{Detection maps of ATR2-HUTD River Scene1. (a) Pseudo-color image. (b) Ground truth. (c) CEM. (d) SAM. (e) IEEPST. (f) MCLT. (g) UTD-Net. (h) TUTDF. (i) TDSS-UTD. (j) NUN-UTD. (m) HUCLNet+CEM. (n) HUCLNet+SAM.}                      
%     \label{fig:C1-3}    
% \end{figure*}
% \begin{figure*}[!t]                 
%     \centering                    
%     \includegraphics[width=2\columnwidth]{images/C1-4.jpg}                     
%     \caption{Detection maps of ATR2-HUTD River Scene2. (a) Pseudo-color image. (b) Ground truth. (c) CEM. (d) SAM. (e) IEEPST. (f) MCLT. (g) UTD-Net. (h) TUTDF. (i) TDSS-UTD. (j) NUN-UTD. (m) HUCLNet+CEM. (n) HUCLNet+SAM.}                     
%     \label{fig:C1-4}    
% \end{figure*}
% \begin{figure*}[!t]                 
%     \centering                    
%     \includegraphics[width=2\columnwidth]{images/C1-5.jpg}                     
%     \caption{Detection maps of ATR2-HUTD Sea Scene1. (a) Pseudo-color image. (b) Ground truth. (c) CEM. (d) SAM. (e) IEEPST. (f) MCLT. (g) UTD-Net. (h) TUTDF. (i) TDSS-UTD. (j) NUN-UTD. (m) HUCLNet+CEM. (n) HUCLNet+SAM.}                 
%     \label{fig:C1-5}    
% \end{figure*}
% \begin{figure*}[!t]                 
%     \centering                    
%     \includegraphics[width=2\columnwidth]{images/C1-6.jpg}                     
%     \caption{Detection maps of ATR2-HUTD Sea Scene2. (a) Pseudo-color image. (b) Ground truth. (c) CEM. (d) SAM. (e) IEEPST. (f) MCLT. (g) UTD-Net. (h) TUTDF. (i) TDSS-UTD. (j) NUN-UTD. (m) HUCLNet+CEM. (n) HUCLNet+SAM.}                    
%     \label{fig:C1-2}    
% \end{figure*}
Traditional methods, such as CEM and SAM, exhibit significant limitations in underwater environments. CEM struggles with background noise suppression, resulting in false positives, while SAM fails to delineate target boundaries and often misses targets, especially in complex scenarios like the ATR2-HUTD River dataset. Its sensitivity to spectral noise and limited adaptability to spectral variations lead to incomplete detection and poor target-background separation.
\par
\begin{figure*}[!t]                 
    \centering                    
    \includegraphics[width=2\columnwidth]{images/C2-1.jpg}                     
    \caption{ROC curves comparison on ATR2-HUTD Lake Scene1. (a) 3-D ROC curve. (b) 2-D ROC curve of $(P_d, P_f)$. (c) 2-D ROC curve of $(P_f, \tau)$. (d) 2-D ROC curve of $(P_d, \tau)$.}                 
    \label{fig:C2-1}    
\end{figure*}
\begin{figure*}[!t]                 
    \centering                    
    \includegraphics[width=2\columnwidth]{images/C2-2.jpg}                     
    \caption{ROC curves comparison on ATR2-HUTD Lake Scene2. (a) 3-D ROC curve. (b) 2-D ROC curve of $(P_d, P_f)$. (c) 2-D ROC curve of $(P_f, \tau)$. (d) 2-D ROC curve of $(P_d, \tau)$.}                                  
    \label{fig:C2-2}    
\end{figure*}
Advanced land-cover detection methods, including IEEPST and MCLT, also underperform in underwater environments. IEEPST struggles to suppress background interference, particularly when water column spectral signatures overlap with target signatures in the ATR2-HUTD River sub-dataset. While MCLT leverages contrastive learning for feature enhancement, it shows reduced sensitivity to small or low-reflectance targets, hindered by the nonlinearities and spectral noise typical of underwater HSI data. These results underscore the necessity of specialized techniques for HUTD.

Among SOTA HUTD methods, UTD-Net demonstrates notable improvements by effectively unmixing target-water mixed pixels. However, it faces challenges with background interference in scenes with extensive non-target bottom areas, leading to high false positive rates. NUN-UTD improves target identification by preserving weak target spectral signals, yet remains susceptible to background interference when spectral characteristics of the background resemble those of the target, leading to false positives in spectrally overlapping environments.

Physical-based methods, such as TUTDF and TDSS-UTD, enhance background suppression using underwater imaging models and predicted depth values. However, TUTDF's performance declines in complex environments due to depth estimation inaccuracies, leading to inconsistent detection. Similarly, TDSS-UTD struggles in environments with substantial depth variation, such as the ATR2-HUTD River dataset, where depth errors degrade detection accuracy. Variations in underwater imaging mechanisms between deep and nearshore scenes further limit their effectiveness.

In contrast, HUCLNet-based methods consistently outperform the alternatives. By integrating instance-level and prototype-level contrastive learning, these methods effectively detect faint and deeply submerged targets with minimal false positives, enhancing background suppression and detection accuracy. HUCLNet+CEM and HUCLNet+SAM show resilience to spectral variability, capturing subtle target features while maintaining clear target-background separation, even under significant underwater bottom interference. These methods provide the most comprehensive target coverage and background suppression in challenging environments, such as the ATR2-HUTD River dataset, demonstrating the superior effectiveness of HUCLNet in mitigating spectral variability and improving detection accuracy.
\par 
\textbf{(\romannumeral2) ROC Curves:} Subjective analysis of detection maps may be insufficient for comprehensive evaluation. Therefore, 3-D ROC curves and their 2-D projections: ($P_d$, $P_f$), ($P_d$, $\tau$), and ($P_f$, $\tau$) were used to objectively assess detection performance on the ATR2-HUTD dataset, enabling a detailed evaluation of detection efficiency, target preservation, and background suppression. 
The ROC curves of ATR-HUTD-Lake sub-dataset are provided in Figs~\ref{fig:C2-1} to~\ref{fig:C2-2}, while those of the ATR-HUTD-River and ATR-HUTD-Sea sub-datasets are provided in the supplementary material.
\par
Figs.~\ref{fig:C2-1} (a) to~\ref{fig:C2-2} (a) show the 3-D ROC curves, highlighting the relationship between the true positive rate ($P_d$), false alarm probability ($P_f$), and detection threshold ($\tau$). HUCLNet+CEM and HUCLNet+SAM consistently outperform other methods, exhibiting higher $P_d$ and lower $P_f$ over a wide range of $\tau$, demonstrating superior adaptability.
\par
% \begin{figure*}[!t]                 
%     \centering                    
%     \includegraphics[width=2\columnwidth]{images/C2-3.jpg}                     
%     \caption{ROC curves comparison on ATR2-HUTD River Scene1. (a) 3-D ROC curve. (b) 2-D ROC curve of $(P_d, P_f)$. (c) 2-D ROC curve of $(P_f, \tau)$. (d) 2-D ROC curve of $(P_d, \tau)$.}                                   
%     \label{fig:C2-3}    
% \end{figure*}
% \begin{figure*}[!t]                 
%     \centering                    
%     \includegraphics[width=2\columnwidth]{images/C2-4.jpg}                     
%     \caption{ROC curves comparison on ATR2-HUTD River Scene2. (a) 3-D ROC curve. (b) 2-D ROC curve of $(P_d, P_f)$. (c) 2-D ROC curve of $(P_f, \tau)$. (d) 2-D ROC curve of $(P_d, \tau)$.}                                 
%     \label{fig:C2-4}    
% \end{figure*}
% \begin{figure*}[!t]                 
%     \centering                    
%     \includegraphics[width=2\columnwidth]{images/C2-5.jpg}                     
%     \caption{ROC curves comparison on ATR2-HUTD Sea Scene1. (a) 3-D ROC curve. (b) 2-D ROC curve of $(P_d, P_f)$. (c) 2-D ROC curve of $(P_f, \tau)$. (d) 2-D ROC curve of $(P_d, \tau)$.}                                   
%     \label{fig:C2-5}    
% \end{figure*}
% \begin{figure*}[!t]                 
%     \centering                    
%     \includegraphics[width=2\columnwidth]{images/C2-6.jpg}                     
%     \caption{ROC curves comparison on ATR2-HUTD Sea Scene2. (a) 3-D ROC curve. (b) 2-D ROC curve of $(P_d, P_f)$. (c) 2-D ROC curve of $(P_f, \tau)$. (d) 2-D ROC curve of $(P_d, \tau)$.}                                
%     \label{fig:C2-2}    
% \end{figure*}
Figs.~\ref{fig:C2-1} (b) to~\ref{fig:C2-2} (b) present the 2-D ROC curves of ($P_d$, $P_f$). HUCLNet-based methods occupy the top-left region, indicating superior detection accuracy. In contrast, traditional HTD methods, such as CEM and SAM, struggle to balance $P_d$ and $P_f$, particularly for targets with varying spectral properties. Although advanced HTD and SOTA HUTD methods show moderate performance, they fail to suppress false alarms in complex river environments, compromising detection accuracy.

Figs.~\ref{fig:C2-1}(c) to~\ref{fig:C2-2}(c) depict the 2-D ROC curves of ($P_f$, $\tau$), assessing background suppression. NUN-UTD shows high $P_f$ across thresholds, indicating poor background-target discrimination. While methods like MCLT and TUTDF show some improvement, they still struggle with high false alarm rates due to spectral overlap. \textbf{UTD-Net performs well in background suppression but largely by classifying all pixels as background}, as reflected in detection maps (Figs.~\ref{fig:C1-1} to~\ref{fig:C1-2}) and AUC$_{P_{d}, \tau}$ values (Tabs.~\ref{auc_lake} to~\ref{auc_sea}). In comparison, HUCLNet+CEM and HUCLNet+SAM exhibit superior background suppression with low $P_f$ and high AUC$_{P_{d}, \tau}$ values.

Figs.~\ref{fig:C2-1}(d) to~\ref{fig:C2-2}(d) present the 2-D ROC curves of ($P_d$, $\tau$), evaluating target preservation. Traditional methods, such as SAM, show significant drops in $P_d$ as $\tau$ increases, indicating poor target preservation. Advanced HTD and SOTA HUTD methods, such as MCLT and TDSS-UTD, show some improvement but still lag behind NUN-UTD and TUTDF. However, \textbf{the improved performance of NUN-UTD and TUTDF primarily results from misclassifying all pixels as targets}, as shown by high false alarm rates in detection maps (Figs.~\ref{fig:C1-1} to~\ref{fig:C1-2}) and increased AUC$_{P_{f}, \tau}$ values. In contrast, HUCLNet+CEM and HUCLNet+SAM maintain high $P_d$ at lower $\tau$, demonstrating robust and reliable target preservation.
\par
\begin{table*}[!t] 
    \centering
    \footnotesize   
    \caption{Quantitative comparison results on the ATR2-HUTD-Lake Sub-dataset. The best and second best results are in \textbf{bold} and with \underline{underline}.} \label{auc_lake}
    \renewcommand{\arraystretch}{1.5}
    \setlength{\tabcolsep}{1.85mm}
    \scalebox{0.875}
    {
        \begin{tabular}{ccccccccccc}
            \hline
            \multirow{2.4}{*}{\textbf{Method}} & \multicolumn{5}{c}{\cellcolor{tablecolor7!60}\textbf{ATR2-HUTD-Lake Scene1}}       & \multicolumn{5}{c}{\cellcolor{tablecolor8}\textbf{ATR2-HUTD-Lake Scene2}}       \\ \cmidrule(lr){2-6} \cmidrule(lr){7-11}
                                    & $\text{AUC}_{( \mathrm{P}_{\mathrm{d}},\mathrm{P}_{\mathrm{f}})}\textcolor{red}{\uparrow }$ & $\text{AUC}_{( \mathrm{P}_{\mathrm{f}}, \tau)}\textcolor{green}{\downarrow }$ & $\text{AUC}_{( \mathrm{P}_{\mathrm{d}},\tau)}\textcolor{red}{\uparrow }$ & $\mathrm{AUC}_{\mathrm{OA}} \textcolor{red}{\uparrow }$ & $\mathrm{AUC}_{\mathrm{SNPR}}\textcolor{red}{\uparrow }$ & $\text{AUC}_{( \mathrm{P}_{\mathrm{d}},\mathrm{P}_{\mathrm{f}})}\textcolor{red}{\uparrow }$ & $\text{AUC}_{( \mathrm{P}_{\mathrm{f}}, \tau)}\textcolor{green}{\downarrow }$ & $\text{AUC}_{( \mathrm{P}_{\mathrm{d}},\tau)}\textcolor{red}{\uparrow }$ & $\mathrm{AUC}_{\mathrm{OA}} \textcolor{red}{\uparrow }$ & $\mathrm{AUC}_{\mathrm{SNPR}}\textcolor{red}{\uparrow }$ \\ \hline
                                    CEM         & 0.671          & 0.250          & 0.258          & 0.678          & 1.028          & 0.489          & 0.524          & 0.520          & 0.485          & 0.994          \\
                                    SAM         & 0.670          & \underline{0.129}    & 0.151          & 0.692          & 1.170          & 0.480          & \underline{0.143}    & 0.025          & 0.362          & 0.172          \\
                                    IEEPST      & 0.424          & 0.204          & 0.075          & 0.295          & 0.369          & 0.417          & 0.187          & 0.036          & 0.266          & 0.193          \\
                                    MCLT        & 0.401          & 0.422          & 0.377          & 0.357          & 0.894          & 0.365          & 0.243          & 0.173          & 0.296          & 0.715          \\
                                    UTD-Net     & 0.846          & \textbf{0.013} & 0.019          & 0.853          & 1.510          & 0.944          & \textbf{0.041} & 0.073          & 0.976          & 1.773          \\
                                    TUTDF       & \underline{0.990}          & 0.634          & \underline{0.726}    & 1.081          & 1.145          & \underline{0.998}          & 0.461          & \underline{0.768}    & 1.306          & 1.667          \\
                                    TDSS-UTD    & 0.964          & 0.215          & 0.369          & 1.117          & 1.712          & \textbf{0.999} & 0.166          & 0.444          & 1.277          & 2.676          \\
                                    NUN-UTD     & 0.758          & 0.913          & \textbf{0.994} & 0.838          & 1.088          & 0.765          & 0.792          & \textbf{0.995} & 0.968          & 1.257          \\
            \rowcolor{tablecolor13!60}HUCLNet+CEM & 0.958    & 0.302          & 0.642          & \underline{1.298}    & \underline{2.126}    & 0.989    & 0.226          & 0.634          & \underline{1.397}    & \underline{2.805}    \\
            \rowcolor{tablecolor14!60}HUCLNet+SAM & \textbf{0.995} & 0.209          & 0.710          & \textbf{1.501} & \textbf{3.393} & \textbf{0.999} & 0.265          & 0.765          & \textbf{1.501} & \textbf{2.891} \\ \hline
        \end{tabular}}
\end{table*}
\begin{table*}[!t] 
    \centering
    \footnotesize   
    \caption{Quantitative comparison results on the ATR2-HUTD-River Sub-dataset. The best and second best results are in \textbf{bold} and with \underline{underline}.} \label{auc_river}
    \renewcommand{\arraystretch}{1.5}
    \setlength{\tabcolsep}{1.85mm}
    \scalebox{0.875}
    {
        \begin{tabular}{ccccccccccc}
            \hline
            \multirow{2.4}{*}{\textbf{Method}} & \multicolumn{5}{c}{\cellcolor{tablecolor9}\textbf{ATR2-HUTD-River Scene1}}       & \multicolumn{5}{c}{\cellcolor{tablecolor10}\textbf{ATR2-HUTD-River Scene2}}       \\ \cmidrule(lr){2-6} \cmidrule(lr){7-11}
                                    & $\text{AUC}_{( \mathrm{P}_{\mathrm{d}},\mathrm{P}_{\mathrm{f}})}\textcolor{red}{\uparrow }$ & $\text{AUC}_{( \mathrm{P}_{\mathrm{f}}, \tau)}\textcolor{green}{\downarrow }$ & $\text{AUC}_{( \mathrm{P}_{\mathrm{d}},\tau)}\textcolor{red}{\uparrow }$ & $\mathrm{AUC}_{\mathrm{OA}} \textcolor{red}{\uparrow }$ & $\mathrm{AUC}_{\mathrm{SNPR}}\textcolor{red}{\uparrow }$ & $\text{AUC}_{( \mathrm{P}_{\mathrm{d}},\mathrm{P}_{\mathrm{f}})}\textcolor{red}{\uparrow }$ & $\text{AUC}_{( \mathrm{P}_{\mathrm{f}}, \tau)}\textcolor{green}{\downarrow }$ & $\text{AUC}_{( \mathrm{P}_{\mathrm{d}},\tau)}\textcolor{red}{\uparrow }$ & $\mathrm{AUC}_{\mathrm{OA}} \textcolor{red}{\uparrow }$ & $\mathrm{AUC}_{\mathrm{SNPR}}\textcolor{red}{\uparrow }$ \\ \hline
                                    CEM         & 0.746          & 0.280          & 0.300          & 0.765          & 1.070          & 0.650          & 0.544          & 0.553          & 0.659          & 1.016          \\
                                    SAM         & 0.657          & 0.214          & 0.186          & 0.629          & 0.871          & 0.656          & \underline{0.078}    & 0.066          & 0.645          & 0.854          \\
                                    IEEPST      & 0.455          & 0.203          & 0.033          & 0.286          & 0.163          & 0.594          & 0.274          & 0.236          & 0.556          & 0.861          \\
                                    MCLT        & 0.550          & 0.989          & 0.990          & 0.552          & 1.001          & 0.531          & 0.970          & \underline{0.971}    & 0.533          & 1.002          \\
                                    UTD-Net     & \underline{0.843}          & \underline{0.080}    & 0.096          & 0.860          & 1.209          & \underline{0.889}          & \textbf{0.075} & 0.088          & \underline{0.903}          & 1.176          \\
                                    TUTDF       & 0.568          & 0.822          & \underline{0.824}    & 0.570          & 1.003          & 0.659          & 0.356          & 0.363          & 0.667          & 1.022          \\
                                    TDSS-UTD    & 0.402          & 0.438          & 0.415          & 0.379          & 0.948          & 0.539 & 0.179          & 0.174          & 0.534          & 0.974          \\
                                    NUN-UTD     & 0.632          & 0.968          & \textbf{0.999} & 0.663          & 1.032          & 0.503          & 0.977          & \textbf{0.980} & 0.505          & 1.002          \\
            \rowcolor{tablecolor13!60}HUCLNet+CEM & 0.794    & 0.353          & 0.518          & \underline{0.959}    & \underline{1.468}    & 0.753    & 0.354          & 0.481          & 0.880    & \underline{1.360}    \\
            \rowcolor{tablecolor14!60}HUCLNet+SAM & \textbf{0.966} & \textbf{0.055} & 0.175          & \textbf{1.086} & \textbf{3.206} & \textbf{0.924} & 0.178          & 0.327          & \textbf{1.073} & \textbf{1.837} \\ \hline
        \end{tabular}}
\end{table*}
\begin{table*}[!t] 
    \centering
    \footnotesize   
    \caption{Quantitative comparison results on the ATR2-HUTD-Sea Sub-dataset. The best and second best results are in \textbf{bold} and with \underline{underline}.} \label{auc_sea}
    \renewcommand{\arraystretch}{1.5}
    \setlength{\tabcolsep}{1.85mm}
    \scalebox{0.875}
    {
        \begin{tabular}{ccccccccccc}
            \hline
            \multirow{2.4}{*}{\textbf{Method}} & \multicolumn{5}{c}{\cellcolor{tablecolor11}\textbf{ATR2-HUTD-Sea Scene1}}       & \multicolumn{5}{c}{\cellcolor{tablecolor12!50}\textbf{ATR2-HUTD-Sea Scene2}}       \\ \cmidrule(lr){2-6} \cmidrule(lr){7-11}
                                    & $\text{AUC}_{( \mathrm{P}_{\mathrm{d}},\mathrm{P}_{\mathrm{f}})}\textcolor{red}{\uparrow }$ & $\text{AUC}_{( \mathrm{P}_{\mathrm{f}}, \tau)}\textcolor{green}{\downarrow }$ & $\text{AUC}_{( \mathrm{P}_{\mathrm{d}},\tau)}\textcolor{red}{\uparrow }$ & $\mathrm{AUC}_{\mathrm{OA}} \textcolor{red}{\uparrow }$ & $\mathrm{AUC}_{\mathrm{SNPR}}\textcolor{red}{\uparrow }$ & $\text{AUC}_{( \mathrm{P}_{\mathrm{d}},\mathrm{P}_{\mathrm{f}})}\textcolor{red}{\uparrow }$ & $\text{AUC}_{( \mathrm{P}_{\mathrm{f}}, \tau)}\textcolor{green}{\downarrow }$ & $\text{AUC}_{( \mathrm{P}_{\mathrm{d}},\tau)}\textcolor{red}{\uparrow }$ & $\mathrm{AUC}_{\mathrm{OA}} \textcolor{red}{\uparrow }$ & $\mathrm{AUC}_{\mathrm{SNPR}}\textcolor{red}{\uparrow }$ \\ \hline
                                    CEM         & 0.805          & 0.309          & 0.349          & 0.845          & 1.128           & 0.845          & 0.332          & 0.351          & 0.864          & 1.057          \\
                                    SAM         & 0.866          & 0.125    & 0.188          & 0.929          & 1.503           & 0.819          & 0.099          & 0.033          & 0.753          & 0.333          \\
                                    IEEPST      & 0.850          & 0.252          & 0.363          & 0.961          & 1.441           & 0.580          & 0.326          & 0.269          & 0.523          & 0.826          \\
                                    MCLT        & 0.895          & 0.980          & \underline{0.994} & 0.909          & 1.014           & 0.317          & 0.953          & \underline{0.944}    & 0.309          & 0.991          \\
                                    UTD-Net     & 0.762          & \underline{0.050}          & 0.083          & 0.796          & 1.682           & 0.774          & \textbf{0.043} & 0.070          & 0.801          & 1.634          \\
                                    TUTDF       & 0.952          & 0.841          & 0.872          & 0.984          & 1.037           & 0.903          & 0.426          & 0.482          & 0.959          & 1.131          \\
                                    TDSS-UTD    & 0.861          & 0.310          & 0.371          & 0.923          & 1.199           & 0.984          & 0.218          & 0.425          & 1.192          & 1.948          \\
                                    NUN-UTD     & \underline{0.979}    & 0.534          & \textbf{0.999} & \textbf{1.445} & 1.872           & 0.975          & 0.959          & \textbf{0.984} & 0.999          & 1.025          \\
            \rowcolor{tablecolor13!60}HUCLNet+CEM & 0.972          & 0.133          & 0.569          & \underline{1.409}    & \underline{4.284}     & \underline{0.987}    & 0.111          & 0.401          & \underline{1.287}    & \underline{3.620}    \\
            \rowcolor{tablecolor14!60}HUCLNet+SAM & \textbf{0.985} & \textbf{0.019} & 0.325          & 1.292          & \textbf{17.501} & \textbf{0.989} & \underline{0.053}    & 0.474          & \textbf{1.420} & \textbf{8.857} \\ \hline
        \end{tabular}}
\end{table*}
\textbf{(\romannumeral3) AUC Values:} The AUC values for each sub-dataset of the ATR2-HUTD dataset are computed using five key metrics: $\text{AUC}_{( \mathrm{P}_{\mathrm{d}}, \mathrm{P}_{\mathrm{f}})}$, $\text{AUC}_{( \mathrm{P}_{\mathrm{d}}, \tau)}$, $\text{AUC}_{( \mathrm{P}_{\mathrm{f}}, \tau)}$, $\text{AUC}_{SNPR}$, and $\text{AUC}_{OA}$, as detailed in Tabs.~\ref{auc_lake} to~\ref{auc_sea}. These metrics quantitatively assess detection accuracy, target preservation, background suppression, signal-to-noise ratio, and overall performance in varied underwater environments.
\par
\begin{table*}[!ht] 
    \centering
    \footnotesize   
    \caption{Quantitative results of ablation studies on the ATR2-HUTD dataset.} \label{ablation study}
    \renewcommand{\arraystretch}{2}
    \setlength{\tabcolsep}{2.5mm}
    \begin{threeparttable}
        \scalebox{0.975}
        { 
    \begin{tabular}{ccccccc}
        \hline
        \textbf{Module Name}                  & \textbf{Design}                                                      & $\text{AUC}_{( \mathrm{P}_{\mathrm{d}},\mathrm{P}_{\mathrm{f}})}\textcolor{red}{\uparrow }$ & $\text{AUC}_{( \mathrm{P}_{\mathrm{f}}, \tau)}\textcolor{green}{\downarrow }$ & $\text{AUC}_{( \mathrm{P}_{\mathrm{d}},\tau)}\textcolor{red}{\uparrow }$ & $\mathrm{AUC}_{\mathrm{OA}} \textcolor{red}{\uparrow }$ & $\mathrm{AUC}_{\mathrm{SNPR}}\textcolor{red}{\uparrow }$  \\ \hline
        \rowcolor{tablecolor0!50}
        \textbf{HUCLNet}                                          & N/A & 0.943 & 0.188 & 0.502 & 1.258 & 4.446 \\
        \rowcolor{tablecolor1!50}
        \cellcolor{tablecolor1!50}                             & w/o Cluster Refinement Strategy                             & 0.823 & 0.206 & 0.388 & 1.005 & 3.141 \\
        \rowcolor{tablecolor1!50}
        \multirow{-2}{*}{\cellcolor{tablecolor1!50}\textbf{RGC module}}  & w/o Reference Spectrum based Clustering Method & 0.737 & 0.211 & 0.375 & 0.901 & 2.616 \\
        \rowcolor{tablecolor2!50} 
        \cellcolor{tablecolor2!50}                              & w/o Instance-level Contrastive Learning                     & 0.878 & 0.199 & 0.438 & 1.117 & 3.513 \\
        \rowcolor{tablecolor2!50} 
        \cellcolor{tablecolor2!50}                              & w/o Prototype-level Contrastive Learning                    & 0.728 & 0.239 & 0.359 & 0.848 & 2.359 \\
        \rowcolor{tablecolor2!50}
        \cellcolor{tablecolor2!50}                              & w/o Hyperspectral-Oriented Data Augmentation                    & 0.883 & 0.195 & 0.452 & 1.165 & 3.584 \\
        \rowcolor{tablecolor2!50} 
        \multirow{-4}{*}{\cellcolor{tablecolor2!50}\textbf{HLCL module}} & w/o HLCL module$^{1}$                                             & 0.696 & 0.252 & 0.248 & 0.692 & 0.933 \\
        \rowcolor{tablecolor3!50} 
        \textbf{SPL Paradigm}                                          & w/o SPL Paradigm                                            & 0.743 & 0.217 & 0.388 & 0.914 & 2.864 \\ \hline
        \end{tabular}}
        \begin{tablenotes}
            \scriptsize
            \item[1] This experimental design is analogous to the baseline HTD methods, as the RGC module and SPL paradigm are dependent on the HLCL module for functionality.
        \end{tablenotes}
        \end{threeparttable}
\end{table*}
The $\text{AUC}_{( \mathrm{P}_{\mathrm{d}}, \mathrm{P}_{\mathrm{f}})}$ metric, which quantifies the trade-off between the true positive rate ($P_d$) and false alarm probability ($P_f$), is critical for evaluating detection performance. HUCLNet+SAM leads with an average score of 0.976, followed by HUCLNet+CEM at 0.909. Traditional methods, such as SAM (0.701) and MCLT (0.691), underperform significantly, while SOTA HUTD methods like TUTDF and NUN-UTD fall short of HUCLNet-based methods in detection capability.
\par
For background suppression, assessed by $\text{AUC}_{( \mathrm{P}_{\mathrm{f}}, \tau)}$, HUCLNet+SAM achieves the highest performance in the ATR2-HUTD-River Scene1 and ATR2-HUTD-Sea sub-datasets, the most complex nearshore environments. It also demonstrates robust performance across other sub-datasets. In contrast, SOTA HUTD methods, including TUTDF and NUN-UTD, show elevated values, suggesting overfitting due to high false positive rates.
\par
The $\text{AUC}_{( \mathrm{P}_{\mathrm{d}}, \tau)}$ metric, assessing target preservation, reveals HUCLNet-based methods performing well, though NUN-UTD leads. This may be attributed to the HLCL module in HUCLNet, which compromises target-background feature separation, impacting target preservation. Additionally, NUN-UTD's higher false positive rate boosts $P_d$ but hinders background suppression.
\par
The $\text{AUC}_{OA}$ metric, combining $\text{AUC}_{( \mathrm{P}_{\mathrm{d}}, \mathrm{P}_{\mathrm{f}})}$, $\text{AUC}_{( \mathrm{P}_{\mathrm{d}}, \tau)}$, and $\text{AUC}_{( \mathrm{P}_{\mathrm{f}}, \tau)}$, further emphasizes HUCLNet's superiority. HUCLNet+SAM achieves the highest average score of 1.312, with HUCLNet+CEM following at 1.205. In contrast, traditional and SOTA HUTD methods score between 0.492 and 0.928, underscoring HUCLNet's effectiveness in background suppression, target preservation, and detection accuracy in complex nearshore environments.
\par
Finally, the $\text{AUC}_{SNPR}$ metric, which measures robustness under varying signal-to-noise ratios, underscores HUCLNet+SAM's superior performance, achieving the highest scores across all sub-datasets, including 17.501 in ATR2-HUTD-Sea Scene1. HUCLNet+CEM consistently ranks second, while traditional HTD and SOTA HUTD methods show lower scores, indicating reduced robustness in fluctuating signal conditions.
\par
% \begin{figure*}[!t]                 
%     \centering                    
%     \includegraphics[scale=0.65]{images/C3-1.jpg}                     
%     \caption{Target-background separability boxplots for ATR2-HUTD Lake sub-dataset. (a) Scene1. (b) Scene2.}                    
%     \label{fig:C3-1}    
% \end{figure*}


% \begin{figure*}[!t]                 
%     \centering                    
%     \includegraphics[scale=0.65]{images/C3-2.jpg}                     
%     \caption{Target-background separability boxplots for ATR2-HUTD River sub-dataset. (a) Scene1. (b) Scene2.}                                 
%     \label{fig:C3-2}    
% \end{figure*}
% \begin{figure*}[!t]                 
%     \centering                    
%     \includegraphics[scale=0.65]{images/C3-3.jpg}                     
%     \caption{Target-background separability boxplots for ATR2-HUTD Sea sub-dataset. (a) Scene1. (b) Scene2.}                                      
%     \label{fig:C3-3}    
% \end{figure*}
% \textbf{(\romannumeral4) Separability Maps:} To assess the effectiveness of the comparison methods in distinguishing targets from the background, target-background separability is analyzed using boxplots, providing a clear visual representation of the separation. Figs.~\ref{fig:C3-1} to \ref{fig:C3-3} present these separability boxplots for all methods across the ATR2-HUTD sub-datasets.
% \par
% Traditional HTD methods, CEM and SAM, show limited separability, with SAM slightly outperforming CEM. In all sub-datasets, target boxes overlap with background boxes, despite some background suppression, indicating poor separation of targets from the background in underwater environments.
% Advanced HTD methods, IEEPST and MCLT, show minor improvement over traditional methods. 
% However, except for the ATR2-HUTD Sea sub-dataset (scene 1), target boxes still overlap with background boxes in most sub-datasets. 
% This suggests that even with advanced techniques, suppressing background noise and achieving clear target separation remains challenging in complex underwater environments.
% \par
% HUTD methods show improved separability. Specifically, UTD-Net achieves significant background suppression, though some overlap remains. 
% Additionally, UTD-Net exhibits a detection range near 0 in certain sub-datasets, indicating a high false positive rate. 
% NUN-UTD, an enhanced version of UTD-Net, improves target highlighting but still struggles with background noise suppression, leading to suboptimal performance in more complex scenes such as those in the ATR2-HUTD River and Sea sub-datasets.
% Compared to unmixing-based HUTD methods, TUTDF and TDSS-UTD demonstrate better separability, with detection ranges closer to 1, indicating more effective reduction of target-background correlation. 
% However, both methods still exhibit considerable target-background overlap and limited suppression in complex scenes, such as the ATR2-HUTD River sub-dataset.
% \par
% In contrast, the proposed HUCLNet-based methods, HUCLNet+CEM and HUCLNet+SAM, exhibit superior separability, with target boxes generally fully separated from the background. 
% These methods effectively suppress background noise, enabling reliable target detection in underwater environments. 
% The detection range for HUCLNet-based methods is close to 1, while the background range is near 0, indicating a low false positive rate. Compared to CEM and SAM, HUCLNet significantly enhances target-background separability, demonstrating its effectiveness in underwater hyperspectral target detection.
\subsection{Ablation Studies}\label{sec:4.4}
To evaluate the efficacy of each component in our method, we conducted ablation studies on the ATR2-HUTD dataset. These studies aim to confirm that the observed improvements stem not only from the increased number of parameters but also from the architectural design, which enhances the HUTD task. The HUCLNet framework is divided into three components for experimental validation. 
Corresponding results are presented in Tab.~\ref{ablation study}.
\par
\textbf{(\romannumeral1) Analysis of the RGC Module:} We validate the RGC with the following designs: 
\begin{itemize}
    \item \textbf{w/o Cluster Refinement Strategy:} This design excludes the cluster refinement strategy, relying solely on the reference spectrum-based clustering method. 
    \item \textbf{w/o Reference Spectrum-based Clustering:} This design omits the reference spectrum-based clustering approach from the RGC module.
\end{itemize}
\par
Without the cluster refinement strategy, the RGC module directly uses the original clustering results, often misclassifying pixels and negatively impacting prototype-level learning. As seen in Tab.~\ref{ablation study}, this leads to lower average metric values compared to the full HUCLNet-based methods, demonstrating the importance of refined pseudo-labels. Removing the RGC module entirely, the HLCL module uses pixel instances from the original HSI for instance-level contrastive learning, focusing on individual pixel spectra and neglecting the target-background relationships. Performance improves slightly over baseline HTD methods but remains significantly inferior to complete HUCLNet-based methods, highlighting the critical role of the RGC module in providing reliable pseudo-labels.
\par
\begin{figure*}[!t]                 
    \centering                    
    \includegraphics[width=2\columnwidth]{images/C4.jpg}                     
    \caption{Parameter analysis results on the ATR2-HUTD dataset. (a) Number of clusters in the RCG module; (b) Batch size in the HLCL module; (c) Attack method in the HLCL module. Red and yellow points indicate the maximum and minimum values, respectively.}                 
    \label{fig:C4-1}    
\end{figure*}
\textbf{(\romannumeral2) Analysis of the HLCL Module:} We evaluate the HLCL module with the following designs:
\begin{itemize}
    \item \textbf{w/o Instance-level Contrastive Learning:} This design removes instance-level contrastive learning, relying only on refined cluster labels from the RGC module.
    \item \textbf{w/o Prototype-level Contrastive Learning:} This design removes prototype-level contrastive learning, retaining only instance-level contrastive learning.
    \item \textbf{w/o Hyperspectral-Oriented Data Augmentation:} This design removes hyperspectral-specific data augmentation from the HLCL module.
    \item \textbf{w/o HLCL Module:} This design excludes the entire HLCL module.
\end{itemize}
\par
According to Tab.~\ref{ablation study}, we can draw the following conclusions.
When the HLCL module operates without instance-level contrastive learning, HUCLNet relies solely on the cluster labels, leading to performance degradation. However, prototype-level contrastive learning alone still outperforms baseline HTD methods, emphasizing the importance of target-background separability. The removal of prototype-level contrastive learning results in poorer performance compared to the instance-level design, indicating its greater impact on separability. When hyperspectral-oriented data augmentation is excluded, traditional augmentation methods lead to observable performance degradation, confirming the importance of hyperspectral-specific augmentation in enhancing feature discriminability and HUCLNet's performance. Finally, removing the HLCL module entirely reduces HUCLNet to baseline HTD methods, resulting in substantial performance loss, reinforcing the HLCL module's primary contribution to performance improvement.
\par
\textbf{(\romannumeral3) Analysis of the SPL Paradigm:} We evaluate the SPL paradigm with the following design: 
\begin{itemize}
    \item \textbf{w/o SPL Paradigm:} This design trains the model using the traditional self-supervised learning framework, which consists of a single reliable-guided clustering step followed by hybrid-level contrastive learning.
\end{itemize}
\par
Without the SPL paradigm, inaccurate clustering due to limited spectral discriminability hinders contrastive learning effectiveness, resulting in error propagation and performance degradation. Tab.~\ref{ablation study} confirms that the SPL paradigm significantly enhances HUCLNet's performance, underscoring the importance of the self-paced strategy in guiding model training and improving detection accuracy.
\par
\subsection{Parameter Analysis}\label{sec:4.5}
The key hyperparameters of the HUCLNet architecture, including the number of clusters in the RGC module, batch size, and attack method in the HLCL module, were analyzed through experiments on the ATR2-HUTD dataset. The results, primarily focusing on the $\text{AUC}_{\text{OA}}$ metric, are presented in Fig.~\ref{fig:C4-1}, as it is the most critical indicator of overall detection performance.
\par
\textbf{(\romannumeral1) Number of Clusters in the RGC Module:} The number of clusters in the RGC module plays a crucial role in clustering accuracy and overall HUCLNet performance. The number of clusters was varied between 30 and 48, with a step size of 3 (Fig.~\ref{fig:C4-1} (a)). Performance improves with an increasing number of clusters up to an optimal point, after which it deteriorates due to over-segmentation, where target pixels are fragmented into multiple clusters. This fragmentation hinders prototype-level contrastive learning, leading to inconsistent target representations. For the ATR2-HUTD Lake, River, and Sea sub-datasets, the optimal number of clusters was 36, 39, and 42, respectively. Even with suboptimal cluster numbers, HUCLNet outperforms baseline methods.

\textbf{(\romannumeral2) Batch Size in the HLCL Module:} The batch size in the HLCL module is another critical parameter affecting HUCLNet performance. Varying the batch size from 32 to 512 with a step size of 64, results (Fig.~\ref{fig:C4-1} (b)) show that larger batch sizes generally improve performance by increasing the number of negative samples, enhancing feature discriminability. This is consistent with prior work~\cite{Chen2020}, which indicates that larger batch sizes benefit contrastive learning. However, performance gains plateau at higher batch sizes, and larger sizes impose greater memory and computational demands. A batch size of 256 provides an optimal balance between performance and resource usage across all ATR2-HUTD sub-datasets.

\textbf{(\romannumeral3) Attack Method in the HLCL Module:} The choice of attack method in the HLCL module influences the generation of adversarial samples for contrastive learning. Four attack methods—FGSM~\cite{GoodfellowSS14}, PGD~\cite{MadryMSTV18}, FAB~\cite{Croce020}, and SPSA~\cite{SPSA}—were tested with a perturbation limit of $\epsilon = 0.1$. As shown in Fig.~\ref{fig:C4-1} (c), performance across attack methods is similar, suggesting that the specific choice of attack method has minimal impact, as long as the generated adversarial samples are effective. Given its computational efficiency and comparable performance, we adopt the FGSM attack method for HUCLNet.

% \subsection{Visualization of the effect of HUCLNet}\label{sec:4.5}
\paragraph{Summary}
Our findings provide significant insights into the influence of correctness, explanations, and refinement on evaluation accuracy and user trust in AI-based planners. 
In particular, the findings are three-fold: 
(1) The \textbf{correctness} of the generated plans is the most significant factor that impacts the evaluation accuracy and user trust in the planners. As the PDDL solver is more capable of generating correct plans, it achieves the highest evaluation accuracy and trust. 
(2) The \textbf{explanation} component of the LLM planner improves evaluation accuracy, as LLM+Expl achieves higher accuracy than LLM alone. Despite this improvement, LLM+Expl minimally impacts user trust. However, alternative explanation methods may influence user trust differently from the manually generated explanations used in our approach.
% On the other hand, explanations may help refine the trust of the planner to a more appropriate level by indicating planner shortcomings.
(3) The \textbf{refinement} procedure in the LLM planner does not lead to a significant improvement in evaluation accuracy; however, it exhibits a positive influence on user trust that may indicate an overtrust in some situations.
% This finding is aligned with prior works showing that iterative refinements based on user feedback would increase user trust~\cite{kunkel2019let, sebo2019don}.
Finally, the propensity-to-trust analysis identifies correctness as the primary determinant of user trust, whereas explanations provided limited improvement in scenarios where the planner's accuracy is diminished.

% In conclusion, our results indicate that the planner's correctness is the dominant factor for both evaluation accuracy and user trust. Therefore, selecting high-quality training data and optimizing the training procedure of AI-based planners to improve planning correctness is the top priority. Once the AI planner achieves a similar correctness level to traditional graph-search planners, strengthening its capability to explain and refine plans will further improve user trust compared to traditional planners.

\paragraph{Future Research} Future steps in this research include expanding user studies with larger sample sizes to improve generalizability and including additional planning problems per session for a more comprehensive evaluation. Next, we will explore alternative methods for generating plan explanations beyond manual creation to identify approaches that more effectively enhance user trust. 
Additionally, we will examine user trust by employing multiple LLM-based planners with varying levels of planning accuracy to better understand the interplay between planning correctness and user trust. 
Furthermore, we aim to enable real-time user-planner interaction, allowing users to provide feedback and refine plans collaboratively, thereby fostering a more dynamic and user-centric planning process.


% \clearpage
% \section*{Impact Statement}
% This paper aims to enhance the memory efficiency and reduce the computational burden of fine-tuning large language models (LLMs), with a particular focus on minimizing the environmental impact of on-device fine-tuning. The growing computational and memory demands of LLMs have led to significant energy consumption and carbon emissions, raising critical concerns about the sustainability of AI development. By enabling the fine-tuning of larger models on hardware with limited memory capacity, our approach not only alleviates hardware constraints but also significantly reduces energy consumption and carbon footprint. This work contributes to more environmentally responsible AI practices, making advanced language models more accessible while aligning with global sustainability goals.

\bibliography{example_paper}
\bibliographystyle{icml2025}

\section{Secure Token Pruning Protocols}
\label{app:a}
We detail the encrypted token pruning protocols $\Pi_{prune}$ in Figure \ref{fig:protocol-prune} and $\Pi_{mask}$ in Figure \ref{fig:protocol-mask} in this section.

%Optionally include supplemental material (complete proofs, additional experiments and plots) in appendix.
%All such materials \textbf{SHOULD be included in the main submission.}
\begin{figure}[h]
%vspace{-0.2in}
\begin{protocolbox}
\noindent
\textbf{Parties:} Server $P_0$, Client $P_1$.

\textbf{Input:} $P_0$ and $P_1$ holds $\{ \left \langle Att \right \rangle_{0}^{h}, \left \langle Att \right \rangle_{1}^{h}\}_{h=0}^{H-1} \in \mathbb{Z}_{2^{\ell}}^{n\times n}$ and $\left \langle x \right \rangle_{0}, \left \langle x \right \rangle_{1} \in \mathbb{Z}_{2^{\ell}}^{n\times D}$ respectively, where H is the number of heads, n is the number of input tokens and D is the embedding dimension of tokens. Additionally, $P_1$ holds a threshold $\theta \in \mathbb{Z}_{2^{\ell}}$.

\textbf{Output:} $P_0$ and $P_1$ get $\left \langle y \right \rangle_{0}, \left \langle y \right \rangle_{1} \in \mathbb{Z}_{2^{\ell}}^{n'\times D}$, respectively, where $y=\mathsf{Prune}(x)$ and $n'$ is the number of remaining tokens.

\noindent\rule{13.2cm}{0.1pt} % This creates the horizontal line
\textbf{Protocol:}
\begin{enumerate}[label=\arabic*:, leftmargin=*]
    \item For $h \in [H]$, $P_0$ and $P_1$ compute locally with input $\left \langle Att \right \rangle^{h}$, and learn the importance score in each head $\left \langle s \right \rangle^{h} \in \mathbb{Z}_{2^{\ell}}^{n} $, where $\left \langle s \right \rangle^{h}[j] = \frac{1}{n} \sum_{i=0}^{n-1} \left \langle Att \right \rangle^{h}[i,j]$.
    \item $P_0$ and $P_1$ compute locally with input $\{ \left \langle s \right \rangle^{i} \in \mathbb{Z}_{2^{\ell}}^{n}  \}_{i=0}^{H-1}$, and learn the final importance score $\left \langle S \right \rangle \in \mathbb{Z}_{2^{\ell}}^{n}$ for each token, where  $\left \langle S \right \rangle[i] = \frac{1}{H} \sum_{h=0}^{H-1} \left \langle s \right \rangle^{h}[i]$.
    \item  For $i \in [n]$, $P_0$ and $P_1$ invoke $\Pi_{CMP}$ with inputs  $\left \langle S \right \rangle$ and $ \theta $, and learn  $\left \langle M \right \rangle \in \mathbb{Z}_{2^{\ell}}^{n}$, such that$\left \langle M \right \rangle[i] = \Pi_{CMP}(\left \langle S \right \rangle[i] - \theta) $, where: \\
    $M[i] = \begin{cases}
        1  &\text{if}\ S[i] > \theta, \\
        0  &\text{otherwise}.
            \end{cases} $
    % \item If the pruning location is insensitive, $P_0$ and $P_1$ learn real mask $M$ instead of shares $\left \langle M \right \rangle$. $P_0$ and $P_1$ compute $\left \langle y \right \rangle$ with input $\left \langle x \right \rangle$ and $M$, where  $\left \langle x \right \rangle[i]$ is pruned if $M[i]$ is $0$.
    \item $P_0$ and $P_1$ invoke $\Pi_{mask}$ with inputs  $\left \langle x \right \rangle$ and pruning mask $\left \langle M \right \rangle$, and set outputs as $\left \langle y \right \rangle$.
\end{enumerate}
\end{protocolbox}
\setlength{\abovecaptionskip}{-1pt} % Reduces space above the caption
\caption{Secure Token Pruning Protocol $\Pi_{prune}$.}
\label{fig:protocol-prune}
\end{figure}




\begin{figure}[h]
\begin{protocolbox}
\noindent
\textbf{Parties:} Server $P_0$, Client $P_1$.

\textbf{Input:} $P_0$ and $P_1$ hold $\left \langle x \right \rangle_{0}, \left \langle x \right \rangle_{1} \in \mathbb{Z}_{2^{\ell}}^{n\times D}$ and  $\left \langle M \right \rangle_{0}, \left \langle M \right \rangle_{1} \in \mathbb{Z}_{2^{\ell}}^{n}$, respectively, where n is the number of input tokens and D is the embedding dimension of tokens.

\textbf{Output:} $P_0$ and $P_1$ get $\left \langle y \right \rangle_{0}, \left \langle y \right \rangle_{1} \in \mathbb{Z}_{2^{\ell}}^{n'\times D}$, respectively, where $y=\mathsf{Prune}(x)$ and $n'$ is the number of remaining tokens.

\noindent\rule{13.2cm}{0.1pt} % This creates the horizontal line
\textbf{Protocol:}
\begin{enumerate}[label=\arabic*:, leftmargin=*]
    \item For $i \in [n]$, $P_0$ and $P_1$ set $\left \langle M \right \rangle$ to the MSB of $\left \langle x \right \rangle$ and learn the masked tokens $\left \langle \Bar{x} \right \rangle \in Z_{2^{\ell}}^{n\times D}$, where
    $\left \langle \Bar{x}[i] \right \rangle = \left \langle x[i] \right \rangle + (\left \langle M[i] \right \rangle << f)$ and $f$ is the fixed-point precision.
    \item $P_0$ and $P_1$ compute the sum of $\{\Pi_{B2A}(\left \langle M \right \rangle[i]) \}_{i=0}^{n-1}$, and learn the number of remaining tokens $n'$ and the number of tokens to be pruned $m$, where $m = n-n'$.
    \item For $k\in[m]$, for $i\in[n-k-1]$, $P_0$ and $P_1$ invoke $\Pi_{msb}$ to learn the highest bit of $\left \langle \Bar{x}[i] \right \rangle$, where $b=\mathsf{MSB}(\Bar{x}[i])$. With the highest bit of $\Bar{x}[i]$, $P_0$ and $P_1$ perform a oblivious swap between $\Bar{x}[i]$ and $\Bar{x}[i+1]$:
    $\begin{cases}
        \Tilde{x}[i] = b\cdot \Bar{x}[i] + (1-b)\cdot \Bar{x}[i+1] \\
        \Tilde{x}[i+1] = b\cdot \Bar{x}[i+1] + (1-b)\cdot \Bar{x}[i]
    \end{cases} $ \\
    $P_0$ and $P_1$ learn the swapped token sequence $\left \langle \Tilde{x} \right \rangle$.
    \item $P_0$ and $P_1$ truncate $\left \langle \Tilde{x} \right \rangle$ locally by keeping the first $n'$ tokens, clear current MSB (all remaining token has $1$ on the MSB), and set outputs as $\left \langle y \right \rangle$.
\end{enumerate}
\end{protocolbox}
\setlength{\abovecaptionskip}{-1pt} % Reduces space above the caption
\caption{Secure Mask Protocol $\Pi_{mask}$.}
\label{fig:protocol-mask}
%\vspace{-0.2in}
\end{figure}

% \begin{wrapfigure}{r}{0.35\textwidth}  % 'r' for right, and the width of the figure area
%   \centering
%   \includegraphics[width=0.35\textwidth]{figures/msb.pdf}
%   \caption{Runtime of $\Pi_{prune}$ and $\Pi_{mask}$ in different layers. We compare different secure pruning strategies based on the BERT Base model.}
%   \label{fig:msb}
%   \vspace{-0.1in}
% \end{wrapfigure}

% \begin{figure}[h]  % 'r' for right, and the width of the figure area
%   \centering
%   \includegraphics[width=0.4\textwidth]{figures/msb.pdf}
%   \caption{Runtime of $\Pi_{prune}$ and $\Pi_{mask}$ in different layers. We compare different secure pruning strategies based on the BERT Base model.}
%   \label{fig:msb}
%   % \vspace{-0.1in}
% \end{figure}

\textbf{Complexity of $\Pi_{mask}$.} The complexity of the proposed $\Pi_{mask}$ mainly depends on the number of oblivious swaps. To prune $m$ tokens out of $n$ input tokens, $O(mn)$ swaps are needed. Since token pruning is performed progressively, only a small number of tokens are pruned at each layer, which makes $\Pi_{mask}$ efficient during runtime. Specifically, for a BERT base model with 128 input tokens, the pruning protocol only takes $\sim0.9$s on average in each layer. An alternative approach is to invoke an oblivious sort algorithm~\citep{bogdanov2014swap2,pang2023bolt} on $\left \langle \Bar{x} \right \rangle$. However, this approach is less efficient because it blindly sort the whole token sequence without considering $m$. That is, even if only $1$ token needs to be pruned, $O(nlog^{2}n)\sim O(n^2)$ oblivious swaps are needed, where as the proposed $\Pi_{mask}$ only need $O(n)$ swaps. More generally, for an $\ell$-layer Transformer with a total of $m$ tokens pruned, the overall time complexity using the sort strategy would be $O(\ell n^2)$ while using the swap strategy remains an overall complexity of $O(mn).$ Specifically, using the sort strategy to prune tokens in one BERT Base model layer can take up to $3.8\sim4.5$ s depending on the sorting algorithm used. In contrast, using the swap strategy only needs $0.5$ s. Moreover, alternative to our MSB strategy, one can also swap the encrypted mask along with the encrypted token sequence. However, we find that this doubles the number of swaps needed, and thus is less efficient the our MSB strategy, as is shown in Figure \ref{fig:msb}.

\section{Existing Protocols}
\label{app:protocol}
\noindent\textbf{Existing Protocols Used in Our Private Inference.}  In our private inference framework, we reuse several existing cryptographic protocols for basic computations. $\Pi_{MatMul}$ \citep{pang2023bolt} processes two ASS matrices and outputs their product in SS form. For non-linear computations, protocols $\Pi_{SoftMax}, \Pi_{GELU}$, and $\Pi_{LayerNorm}$\citep{lu2023bumblebee, pang2023bolt} take a secret shared tensor and return the result of non-linear functions in ASS. Basic protocols from~\citep{rathee2020cryptflow2, rathee2021sirnn} are also utilized. $\Pi_{CMP}$\citep{EzPC}, for example, inputs ASS values and outputs a secret shared comparison result, while $\Pi_{B2A}$\citep{EzPC} converts secret shared Boolean values into their corresponding arithmetic values.

\section{Polynomial Reduction for Non-linear Functions}
\label{app:b}
The $\mathsf{SoftMax}$ and $\mathsf{GELU}$ functions can be approximated with polynomials. High-degree polynomials~\citep{lu2023bumblebee, pang2023bolt} can achieve the same accuracy as the LUT-based methods~\cite{hao2022iron-iron}. While these polynomial approximations are more efficient than look-up tables, they can still incur considerable overheads. Reducing the high-degree polynomials to the low-degree ones for the less important tokens can imporve efficiency without compromising accuracy. The $\mathsf{SoftMax}$ function is applied to each row of an attention map. If a token is to be reduced, the corresponding row will be computed using the low-degree polynomial approximations. Otherwise, the corresponding row will be computed accurately via a high-degree one. That is if $M_{\beta}'[i] = 1$, $P_0$ and $P_1$ uses high-degree polynomials to compute the $\mathsf{SoftMax}$ function on token $x[i]$:
\begin{equation}
\mathsf{SoftMax}_{i}(x) = \frac{e^{x_i}}{\sum_{j\in [d]}e^{x_j}}
\end{equation}
where $x$ is a input vector of length $d$ and the exponential function is computed via a polynomial approximation. For the $\mathsf{SoftMax}$ protocol, we adopt a similar strategy as~\citep{kim2021ibert, hao2022iron-iron}, where we evaluate on the normalized inputs $\mathsf{SoftMax}(x-max_{i\in [d]}x_i)$. Different from~\citep{hao2022iron-iron}, we did not used the binary tree to find max value in the given vector. Instead, we traverse through the vector to find the max value. This is because each attention map is computed independently and the binary tree cannot be re-used. If $M_{\beta}[i] = 0$, $P_0$ and $P_1$ will approximate the $\mathsf{SoftMax}$ function with low-degree polynomial approximations. We detail how $\mathsf{SoftMax}$ can be approximated as follows:
\begin{equation}
\label{eq:app softmax}
\mathsf{ApproxSoftMax}_{i}(x) = \frac{\mathsf{ApproxExp}(x_i)}{\sum_{j\in [d]}\mathsf{ApproxExp}(x_j)}
\end{equation}
\begin{equation}
\mathsf{ApproxExp}(x)=\begin{cases}
    0  &\text{if}\ x \leq T \\
    (1+ \frac{x}{2^n})^{2^n} &\text{if}\ x \in [T,0]\\
\end{cases}
\end{equation}
where the $2^n$-degree Taylor series is used to approximate the exponential function and $T$ is the clipping boundary. The value $n$ and $T$ determines the accuracy of above approximation. With $n=6$ and $T=-13$, the approximation can achieve an average error within $2^{-10}$~\citep{lu2023bumblebee}. For low-degree polynomial approximation, $n=3$ is used in the Taylor series.

Similarly, $P_0$ or $P_1$ can decide whether or not to approximate the $\mathsf{GELU}$ function for each token. If $M_{\beta}[i] = 1$, $P_0$ and $P_1$ use high-degree polynomials~\citep{lu2023bumblebee} to compute the $\mathsf{GELU}$ function on token $x[i]$ with high-degree polynomial:
% \begin{equation}
% \mathsf{GELU}(x) = 0.5x(1+\mathsf{Tanh}(\sqrt{2/\pi}(x+0.044715x^3)))
% \end{equation}
% where the $\mathsf{Tanh}$ and square root function are computed via a OT-based lookup-table.

\begin{equation}
\label{eq:app gelu}
\mathsf{ApproxGELU}(x)=\begin{cases}
    0  &\text{if}\ x \leq -5 \\
    P^3(x), &\text{if}\ -5 < x \leq -1.97 \\
    P^6(x), &\text{if}\ -1.97 < x \leq 3  \\
    x, &\text{if}\ x >3 \\
\end{cases}
\end{equation}
where $P^3(x)$ and $P^6(x)$ are degree-3 and degree-6 polynomials respectively. The detailed coefficient for the polynomial is: 
\begin{equation*}
    P^3(x) = -0.50540312 -  0.42226581x - 0.11807613x^2 - 0.01103413x^3
\end{equation*}
, and
\begin{equation*}
    P^6(x) = 0.00852632 + 0.5x + 0.36032927x^2 - 0.03768820x^4 + 0.00180675x^6
\end{equation*}

For BOLT baseline, we use another high-degree polynomial to compute the $\mathsf{GELU}$ function.

\begin{equation}
\label{eq:app gelu}
\mathsf{ApproxGELU}(x)=\begin{cases}
    0  &\text{if}\ x < -2.7 \\
    P^4(x), &\text{if}\   |x| \leq 2.7 \\
    x, &\text{if}\ x >2.7 \\
\end{cases}
\end{equation}
We use the same coefficients for $P^4(x)$ as BOLT~\citep{pang2023bolt}.

\begin{figure}[h]
 % \vspace{-0.1in}
    \centering
    \includegraphics[width=1\linewidth]{figures/bumble.pdf}
    % \captionsetup{skip=2pt}
    % \vspace{-0.1in}
    \caption{Comparison with prior works on the BERT model. The input has 128 tokens.}
    \label{fig:bumble}
\end{figure}

If $M_{\beta}'[i] = 0$, $P_0$ and $P_1$ will use low-degree 
polynomial approximation to compute the $\mathsf{GELU}$ function instead. Encrypted polynomial reduction leverages low-degree polynomials to compute non-linear functions for less important tokens. For the $\mathsf{GELU}$ function, the following degree-$2$ polynomial~\cite{kim2021ibert} is used:
\begin{equation*}
\mathsf{ApproxGELU}(x)=\begin{cases}
    0  &\text{if}\ x <  -1.7626 \\
    0.5x+0.28367x^2, &\text{if}\ x \leq |1.7626| \\
    x, &\text{if}\ x > 1.7626\\
\end{cases}
\end{equation*}


\section{Comparison with More Related Works.}
\label{app:c}
\textbf{Other 2PC frameworks.} The primary focus of CipherPrune is to accelerate the private Transformer inference in the 2PC setting. As shown in Figure \ref{fig:bumble}, CipherPrune can be easily extended to other 2PC private inference frameworks like BumbleBee~\citep{lu2023bumblebee}. We compare CipherPrune with BumbleBee and IRON on BERT models. We test the performance in the same LAN setting as BumbleBee with 1 Gbps bandwidth and 0.5 ms of ping time. CipherPrune achieves more than $\sim 60 \times$ speed up over BOLT and $4.3\times$ speed up over BumbleBee.

\begin{figure}[t]
 % \vspace{-0.1in}
    \centering
    \includegraphics[width=1\linewidth]{figures/pumab.pdf}
    % \captionsetup{skip=2pt}
    % \vspace{-0.1in}
    \caption{Comparison with MPCFormer and PUMA on the BERT models. The input has 128 tokens.}
    \label{fig:pumab}
\end{figure}

\begin{figure}[h]
 % \vspace{-0.1in}
    \centering
    \includegraphics[width=1\linewidth]{figures/pumag.pdf}
    % \captionsetup{skip=2pt}
    % \vspace{-0.1in}
    \caption{Comparison with MPCFormer and PUMA on the GPT2 models. The input has 128 tokens. The polynomial reduction is not used.}
    \label{fig:pumag}
\end{figure}

\textbf{Extension to 3PC frameworks.} Additionally, we highlight that CipherPrune can be also extended to the 3PC frameworks like MPCFormer~\citep{li2022mpcformer} and PUMA~\citep{dong2023puma}. This is because CipherPrune is built upon basic primitives like comparison and Boolean-to-Arithmetic conversion. We compare CipherPrune with MPCFormer and PUMA on both the BERT and GPT2 models. CipherPrune has a $6.6\sim9.4\times$ speed up over MPCFormer and $2.8\sim4.6\times$ speed up over PUMA on the BERT-Large and GPT2-Large models.


\section{Communication Reduction in SoftMax and GELU.}
\label{app:e}

\begin{figure}[h]
    \centering
    \includegraphics[width=0.9\linewidth]{figures/layerwise.pdf}
    \caption{Toy example of two successive Transformer layers. In layer$_i$, the SoftMax and Prune protocol have $n$ input tokens. The number of input tokens is reduced to $n'$ for the Linear layers, LayerNorm and GELU in layer$_i$ and SoftMax in layer$_{i+1}$.}
    \label{fig:layer}
\end{figure}

\begin{table*}[h]
\captionsetup{skip=2pt}
\centering
\scriptsize
\caption{Communication cost (in MB) of the SoftMax and GELU protocol in each Transformer layer.}
\begin{tblr}{
    colspec = {c |c c c c c c c c c c c c},
    row{1} = {font=\bfseries},
    row{2-Z} = {rowsep=1pt},
    % row{4} = {bg=LightBlue},
    colsep = 2.5pt,
    }
\hline
\textbf{Layer Index} & \textbf{0}  & \textbf{1}  & \textbf{2} & \textbf{3} & \textbf{4} & \textbf{5} & \textbf{6} & \textbf{7} & \textbf{8} & \textbf{9} & \textbf{10} & \textbf{11} \\
\hline
Softmax & 642.19 & 642.19 & 642.19 & 642.19 & 642.19 & 642.19 & 642.19 & 642.19 & 642.19 & 642.19 & 642.19 & 642.19 \\
Pruned Softmax & 642.19 & 129.58 & 127.89 & 119.73 & 97.04 & 71.52 & 43.92 & 21.50 & 10.67 & 6.16 & 4.65 & 4.03 \\
\hline
GELU & 698.84 & 698.84 & 698.84 & 698.84 & 698.84 & 698.84 & 698.84 & 698.84 & 698.84 & 698.84 & 698.84 & 698.84\\
Pruned GELU  & 325.10 & 317.18 & 313.43 & 275.94 & 236.95 & 191.96 & 135.02 & 88.34 & 46.68 & 16.50 & 5.58 & 5.58\\
\hline
\end{tblr}
\label{tab:layer}
\end{table*}

{
In Figure \ref{fig:layer}, we illustrate why CipherPrune can reduce the communication overhead of both  SoftMax and GELU. Suppose there are $n$ tokens in $layer_i$. Then, the SoftMax protocol in the attention module has a complexity of $O(n^2)$. CipherPrune's token pruning protocol is invoked to select $n'$ tokens out of all $n$ tokens, where $m=n-n'$ is the number of tokens that are removed. The overhead of the GELU function in $layer_i$, i.e., the current layer, has only $O(n')$ complexity (which should be $O(n)$ without token pruning). The complexity of the SoftMax function in $layer_{i+1}$, i.e., the following layer, is reduced to $O(n'^2)$ (which should be $O(n^2)$ without token pruning). The SoftMax protocol has quadratic complexity with respect to the token number and the GELU protocol has linear complexity. Therefore, CipherPrune can reduce the overhead of both the GELU protocol and the SoftMax protocols by reducing the number of tokens. In Table \ref{tab:layer}, we provide detailed layer-wise communication cost of the GELU and the SoftMax protocol. Compared to the unpruned baseline, CipherPrune can effectively reduce the overhead of the GELU and the SoftMax protocols layer by layer.
}

\section{Analysis on Layer-wise redundancy.}
\label{app:f}

\begin{figure}[h]
    \centering
    \includegraphics[width=0.9\linewidth]{figures/layertime0.pdf}
    \caption{The number of pruned tokens and pruning protocol runtime in different layers in the BERT Base model. The results are averaged across 128 QNLI samples.}
    \label{fig:layertime}
\end{figure}

{
In Figure \ref{fig:layertime}, we present the number of pruned tokens and the runtime of the pruning protocol for each layer in the BERT Base model. The number of pruned tokens per layer was averaged across 128 QNLI samples, while the pruning protocol runtime was measured over 10 independent runs. The mean token count for the QNLI samples is 48.5. During inference with BERT Base, input sequences with fewer tokens are padded to 128 tokens using padding tokens. Consistent with prior token pruning methods in plaintext~\citep{goyal2020power}, a significant number of padding tokens are removed at layer 0.  At layer 0, the number of pruned tokens is primarily influenced by the number of padding tokens rather than token-level redundancy.
%In Figure \ref{fig:layertime}, we demonstrate the number of pruned tokens and the pruning protocol runtime in each layer in the BERT Base model. We averaged the number of pruned tokens in each layer across 128 QNLI samples and then tested the pruning protocol runtime in 10 independent runs. The mean token number of the QNLI samples is 48.5. During inference with BERT Base, input sequences with small token number are padded to 128 tokens with padding tokens. Similar to prior token pruning methods in the plaintext~\citep{goyal2020power}, a large number of padding tokens can be removed at layer 0. We remark that token-level redundancy builds progressively throughout inference~\citep{goyal2020power, kim2022LTP}. The number of pruned tokens in layer 0 mostly depends on the number of padding tokens instead of token-level redundancy.
}

{
%As shown in Figure \ref{fig:layertime}, more tokens are removed in the intermediate layers, e.g., layer $4$ to layer $7$. This suggests there is more redundant information in these intermediate layers. 
In CipherPrune, tokens are removed progressively, and once removed, they are excluded from computations in subsequent layers. Consequently, token pruning in earlier layers affects computations in later layers, whereas token pruning in later layers does not impact earlier layers. As a result, even if layers 4 and 7 remove the same number of tokens, layer 7 processes fewer tokens overall, as illustrated in Figure \ref{fig:layertime}. Specifically, 8 tokens are removed in both layer $4$ and layer $7$, but the runtime of the pruning protocol in layer $4$ is $\sim2.4\times$ longer than that in  layer $7$.
}

\section{Related Works}
\label{app:g}

{
In response to the success of Transformers and the need to safeguard data privacy, various private Transformer Inferences~\citep{chen2022thex,zheng2023primer,hao2022iron-iron,li2022mpcformer, lu2023bumblebee, luo2024secformer, pang2023bolt}  are proposed. To efficiently run private Transformer inferences, multiple cryptographic primitives are used in a popular hybrid HE/MPC method IRON~\citep{hao2022iron-iron}, i.e., in a Transformer, HE and SS are used for linear layers, and SS and OT are adopted for nonlinear layers. IRON and BumbleBee~\citep{lu2023bumblebee} focus on optimizing linear general matrix multiplications; SecFormer~\cite{luo2024secformer} improves the non-linear operations like the exponential function with polynomial approximation; BOLT~\citep{pang2023bolt} introduces the baby-step giant-step (BSGS) algorithm to reduce the number of HE rotations, proposes a word elimination (W.E.) technique, and uses polynomial approximation for non-linear operations, ultimately achieving state-of-the-art (SOTA) performance.
}

{Other than above hybrid HE/MPC methods, there are also works exploring privacy-preserving Transformer inference using only HE~\citep{zimerman2023converting, zhang2024nonin}. The first HE-based private Transformer inference work~\citep{zimerman2023converting} replaces \mysoftmax function with a scaled-ReLU function. Since the scaled-ReLU function can be approximated with low-degree polynomials more easily, it can be computed more efficiently using only HE operations. A range-loss term is needed during training to reduce the polynomial degree while maintaining high accuracy. A training-free HE-based private Transformer inference was proposed~\citep{zhang2024nonin}, where non-linear operations are approximated by high-degree polynomials. The HE-based methods need frequent bootstrapping, especially when using high-degree polynomials, thus often incurring higher overhead than the hybrid HE/MPC methods in practice.
}



\end{document}

