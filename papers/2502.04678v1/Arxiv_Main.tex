\documentclass{article} % For LaTeX2e
\usepackage{arxiv}

% Optional math commands from https://github.com/goodfeli/dlbook_notation.
%%%%% NEW MATH DEFINITIONS %%%%%

\usepackage{amsmath,amsfonts,bm}
\usepackage{derivative}
% Mark sections of captions for referring to divisions of figures
\newcommand{\figleft}{{\em (Left)}}
\newcommand{\figcenter}{{\em (Center)}}
\newcommand{\figright}{{\em (Right)}}
\newcommand{\figtop}{{\em (Top)}}
\newcommand{\figbottom}{{\em (Bottom)}}
\newcommand{\captiona}{{\em (a)}}
\newcommand{\captionb}{{\em (b)}}
\newcommand{\captionc}{{\em (c)}}
\newcommand{\captiond}{{\em (d)}}

% Highlight a newly defined term
\newcommand{\newterm}[1]{{\bf #1}}

% Derivative d 
\newcommand{\deriv}{{\mathrm{d}}}

% Figure reference, lower-case.
\def\figref#1{figure~\ref{#1}}
% Figure reference, capital. For start of sentence
\def\Figref#1{Figure~\ref{#1}}
\def\twofigref#1#2{figures \ref{#1} and \ref{#2}}
\def\quadfigref#1#2#3#4{figures \ref{#1}, \ref{#2}, \ref{#3} and \ref{#4}}
% Section reference, lower-case.
\def\secref#1{section~\ref{#1}}
% Section reference, capital.
\def\Secref#1{Section~\ref{#1}}
% Reference to two sections.
\def\twosecrefs#1#2{sections \ref{#1} and \ref{#2}}
% Reference to three sections.
\def\secrefs#1#2#3{sections \ref{#1}, \ref{#2} and \ref{#3}}
% Reference to an equation, lower-case.
\def\eqref#1{equation~\ref{#1}}
% Reference to an equation, upper case
\def\Eqref#1{Equation~\ref{#1}}
% A raw reference to an equation---avoid using if possible
\def\plaineqref#1{\ref{#1}}
% Reference to a chapter, lower-case.
\def\chapref#1{chapter~\ref{#1}}
% Reference to an equation, upper case.
\def\Chapref#1{Chapter~\ref{#1}}
% Reference to a range of chapters
\def\rangechapref#1#2{chapters\ref{#1}--\ref{#2}}
% Reference to an algorithm, lower-case.
\def\algref#1{algorithm~\ref{#1}}
% Reference to an algorithm, upper case.
\def\Algref#1{Algorithm~\ref{#1}}
\def\twoalgref#1#2{algorithms \ref{#1} and \ref{#2}}
\def\Twoalgref#1#2{Algorithms \ref{#1} and \ref{#2}}
% Reference to a part, lower case
\def\partref#1{part~\ref{#1}}
% Reference to a part, upper case
\def\Partref#1{Part~\ref{#1}}
\def\twopartref#1#2{parts \ref{#1} and \ref{#2}}

\def\ceil#1{\lceil #1 \rceil}
\def\floor#1{\lfloor #1 \rfloor}
\def\1{\bm{1}}
\newcommand{\train}{\mathcal{D}}
\newcommand{\valid}{\mathcal{D_{\mathrm{valid}}}}
\newcommand{\test}{\mathcal{D_{\mathrm{test}}}}

\def\eps{{\epsilon}}


% Random variables
\def\reta{{\textnormal{$\eta$}}}
\def\ra{{\textnormal{a}}}
\def\rb{{\textnormal{b}}}
\def\rc{{\textnormal{c}}}
\def\rd{{\textnormal{d}}}
\def\re{{\textnormal{e}}}
\def\rf{{\textnormal{f}}}
\def\rg{{\textnormal{g}}}
\def\rh{{\textnormal{h}}}
\def\ri{{\textnormal{i}}}
\def\rj{{\textnormal{j}}}
\def\rk{{\textnormal{k}}}
\def\rl{{\textnormal{l}}}
% rm is already a command, just don't name any random variables m
\def\rn{{\textnormal{n}}}
\def\ro{{\textnormal{o}}}
\def\rp{{\textnormal{p}}}
\def\rq{{\textnormal{q}}}
\def\rr{{\textnormal{r}}}
\def\rs{{\textnormal{s}}}
\def\rt{{\textnormal{t}}}
\def\ru{{\textnormal{u}}}
\def\rv{{\textnormal{v}}}
\def\rw{{\textnormal{w}}}
\def\rx{{\textnormal{x}}}
\def\ry{{\textnormal{y}}}
\def\rz{{\textnormal{z}}}

% Random vectors
\def\rvepsilon{{\mathbf{\epsilon}}}
\def\rvphi{{\mathbf{\phi}}}
\def\rvtheta{{\mathbf{\theta}}}
\def\rva{{\mathbf{a}}}
\def\rvb{{\mathbf{b}}}
\def\rvc{{\mathbf{c}}}
\def\rvd{{\mathbf{d}}}
\def\rve{{\mathbf{e}}}
\def\rvf{{\mathbf{f}}}
\def\rvg{{\mathbf{g}}}
\def\rvh{{\mathbf{h}}}
\def\rvu{{\mathbf{i}}}
\def\rvj{{\mathbf{j}}}
\def\rvk{{\mathbf{k}}}
\def\rvl{{\mathbf{l}}}
\def\rvm{{\mathbf{m}}}
\def\rvn{{\mathbf{n}}}
\def\rvo{{\mathbf{o}}}
\def\rvp{{\mathbf{p}}}
\def\rvq{{\mathbf{q}}}
\def\rvr{{\mathbf{r}}}
\def\rvs{{\mathbf{s}}}
\def\rvt{{\mathbf{t}}}
\def\rvu{{\mathbf{u}}}
\def\rvv{{\mathbf{v}}}
\def\rvw{{\mathbf{w}}}
\def\rvx{{\mathbf{x}}}
\def\rvy{{\mathbf{y}}}
\def\rvz{{\mathbf{z}}}

% Elements of random vectors
\def\erva{{\textnormal{a}}}
\def\ervb{{\textnormal{b}}}
\def\ervc{{\textnormal{c}}}
\def\ervd{{\textnormal{d}}}
\def\erve{{\textnormal{e}}}
\def\ervf{{\textnormal{f}}}
\def\ervg{{\textnormal{g}}}
\def\ervh{{\textnormal{h}}}
\def\ervi{{\textnormal{i}}}
\def\ervj{{\textnormal{j}}}
\def\ervk{{\textnormal{k}}}
\def\ervl{{\textnormal{l}}}
\def\ervm{{\textnormal{m}}}
\def\ervn{{\textnormal{n}}}
\def\ervo{{\textnormal{o}}}
\def\ervp{{\textnormal{p}}}
\def\ervq{{\textnormal{q}}}
\def\ervr{{\textnormal{r}}}
\def\ervs{{\textnormal{s}}}
\def\ervt{{\textnormal{t}}}
\def\ervu{{\textnormal{u}}}
\def\ervv{{\textnormal{v}}}
\def\ervw{{\textnormal{w}}}
\def\ervx{{\textnormal{x}}}
\def\ervy{{\textnormal{y}}}
\def\ervz{{\textnormal{z}}}

% Random matrices
\def\rmA{{\mathbf{A}}}
\def\rmB{{\mathbf{B}}}
\def\rmC{{\mathbf{C}}}
\def\rmD{{\mathbf{D}}}
\def\rmE{{\mathbf{E}}}
\def\rmF{{\mathbf{F}}}
\def\rmG{{\mathbf{G}}}
\def\rmH{{\mathbf{H}}}
\def\rmI{{\mathbf{I}}}
\def\rmJ{{\mathbf{J}}}
\def\rmK{{\mathbf{K}}}
\def\rmL{{\mathbf{L}}}
\def\rmM{{\mathbf{M}}}
\def\rmN{{\mathbf{N}}}
\def\rmO{{\mathbf{O}}}
\def\rmP{{\mathbf{P}}}
\def\rmQ{{\mathbf{Q}}}
\def\rmR{{\mathbf{R}}}
\def\rmS{{\mathbf{S}}}
\def\rmT{{\mathbf{T}}}
\def\rmU{{\mathbf{U}}}
\def\rmV{{\mathbf{V}}}
\def\rmW{{\mathbf{W}}}
\def\rmX{{\mathbf{X}}}
\def\rmY{{\mathbf{Y}}}
\def\rmZ{{\mathbf{Z}}}

% Elements of random matrices
\def\ermA{{\textnormal{A}}}
\def\ermB{{\textnormal{B}}}
\def\ermC{{\textnormal{C}}}
\def\ermD{{\textnormal{D}}}
\def\ermE{{\textnormal{E}}}
\def\ermF{{\textnormal{F}}}
\def\ermG{{\textnormal{G}}}
\def\ermH{{\textnormal{H}}}
\def\ermI{{\textnormal{I}}}
\def\ermJ{{\textnormal{J}}}
\def\ermK{{\textnormal{K}}}
\def\ermL{{\textnormal{L}}}
\def\ermM{{\textnormal{M}}}
\def\ermN{{\textnormal{N}}}
\def\ermO{{\textnormal{O}}}
\def\ermP{{\textnormal{P}}}
\def\ermQ{{\textnormal{Q}}}
\def\ermR{{\textnormal{R}}}
\def\ermS{{\textnormal{S}}}
\def\ermT{{\textnormal{T}}}
\def\ermU{{\textnormal{U}}}
\def\ermV{{\textnormal{V}}}
\def\ermW{{\textnormal{W}}}
\def\ermX{{\textnormal{X}}}
\def\ermY{{\textnormal{Y}}}
\def\ermZ{{\textnormal{Z}}}

% Vectors
\def\vzero{{\bm{0}}}
\def\vone{{\bm{1}}}
\def\vmu{{\bm{\mu}}}
\def\vtheta{{\bm{\theta}}}
\def\vphi{{\bm{\phi}}}
\def\va{{\bm{a}}}
\def\vb{{\bm{b}}}
\def\vc{{\bm{c}}}
\def\vd{{\bm{d}}}
\def\ve{{\bm{e}}}
\def\vf{{\bm{f}}}
\def\vg{{\bm{g}}}
\def\vh{{\bm{h}}}
\def\vi{{\bm{i}}}
\def\vj{{\bm{j}}}
\def\vk{{\bm{k}}}
\def\vl{{\bm{l}}}
\def\vm{{\bm{m}}}
\def\vn{{\bm{n}}}
\def\vo{{\bm{o}}}
\def\vp{{\bm{p}}}
\def\vq{{\bm{q}}}
\def\vr{{\bm{r}}}
\def\vs{{\bm{s}}}
\def\vt{{\bm{t}}}
\def\vu{{\bm{u}}}
\def\vv{{\bm{v}}}
\def\vw{{\bm{w}}}
\def\vx{{\bm{x}}}
\def\vy{{\bm{y}}}
\def\vz{{\bm{z}}}

% Elements of vectors
\def\evalpha{{\alpha}}
\def\evbeta{{\beta}}
\def\evepsilon{{\epsilon}}
\def\evlambda{{\lambda}}
\def\evomega{{\omega}}
\def\evmu{{\mu}}
\def\evpsi{{\psi}}
\def\evsigma{{\sigma}}
\def\evtheta{{\theta}}
\def\eva{{a}}
\def\evb{{b}}
\def\evc{{c}}
\def\evd{{d}}
\def\eve{{e}}
\def\evf{{f}}
\def\evg{{g}}
\def\evh{{h}}
\def\evi{{i}}
\def\evj{{j}}
\def\evk{{k}}
\def\evl{{l}}
\def\evm{{m}}
\def\evn{{n}}
\def\evo{{o}}
\def\evp{{p}}
\def\evq{{q}}
\def\evr{{r}}
\def\evs{{s}}
\def\evt{{t}}
\def\evu{{u}}
\def\evv{{v}}
\def\evw{{w}}
\def\evx{{x}}
\def\evy{{y}}
\def\evz{{z}}

% Matrix
\def\mA{{\bm{A}}}
\def\mB{{\bm{B}}}
\def\mC{{\bm{C}}}
\def\mD{{\bm{D}}}
\def\mE{{\bm{E}}}
\def\mF{{\bm{F}}}
\def\mG{{\bm{G}}}
\def\mH{{\bm{H}}}
\def\mI{{\bm{I}}}
\def\mJ{{\bm{J}}}
\def\mK{{\bm{K}}}
\def\mL{{\bm{L}}}
\def\mM{{\bm{M}}}
\def\mN{{\bm{N}}}
\def\mO{{\bm{O}}}
\def\mP{{\bm{P}}}
\def\mQ{{\bm{Q}}}
\def\mR{{\bm{R}}}
\def\mS{{\bm{S}}}
\def\mT{{\bm{T}}}
\def\mU{{\bm{U}}}
\def\mV{{\bm{V}}}
\def\mW{{\bm{W}}}
\def\mX{{\bm{X}}}
\def\mY{{\bm{Y}}}
\def\mZ{{\bm{Z}}}
\def\mBeta{{\bm{\beta}}}
\def\mPhi{{\bm{\Phi}}}
\def\mLambda{{\bm{\Lambda}}}
\def\mSigma{{\bm{\Sigma}}}

% Tensor
\DeclareMathAlphabet{\mathsfit}{\encodingdefault}{\sfdefault}{m}{sl}
\SetMathAlphabet{\mathsfit}{bold}{\encodingdefault}{\sfdefault}{bx}{n}
\newcommand{\tens}[1]{\bm{\mathsfit{#1}}}
\def\tA{{\tens{A}}}
\def\tB{{\tens{B}}}
\def\tC{{\tens{C}}}
\def\tD{{\tens{D}}}
\def\tE{{\tens{E}}}
\def\tF{{\tens{F}}}
\def\tG{{\tens{G}}}
\def\tH{{\tens{H}}}
\def\tI{{\tens{I}}}
\def\tJ{{\tens{J}}}
\def\tK{{\tens{K}}}
\def\tL{{\tens{L}}}
\def\tM{{\tens{M}}}
\def\tN{{\tens{N}}}
\def\tO{{\tens{O}}}
\def\tP{{\tens{P}}}
\def\tQ{{\tens{Q}}}
\def\tR{{\tens{R}}}
\def\tS{{\tens{S}}}
\def\tT{{\tens{T}}}
\def\tU{{\tens{U}}}
\def\tV{{\tens{V}}}
\def\tW{{\tens{W}}}
\def\tX{{\tens{X}}}
\def\tY{{\tens{Y}}}
\def\tZ{{\tens{Z}}}


% Graph
\def\gA{{\mathcal{A}}}
\def\gB{{\mathcal{B}}}
\def\gC{{\mathcal{C}}}
\def\gD{{\mathcal{D}}}
\def\gE{{\mathcal{E}}}
\def\gF{{\mathcal{F}}}
\def\gG{{\mathcal{G}}}
\def\gH{{\mathcal{H}}}
\def\gI{{\mathcal{I}}}
\def\gJ{{\mathcal{J}}}
\def\gK{{\mathcal{K}}}
\def\gL{{\mathcal{L}}}
\def\gM{{\mathcal{M}}}
\def\gN{{\mathcal{N}}}
\def\gO{{\mathcal{O}}}
\def\gP{{\mathcal{P}}}
\def\gQ{{\mathcal{Q}}}
\def\gR{{\mathcal{R}}}
\def\gS{{\mathcal{S}}}
\def\gT{{\mathcal{T}}}
\def\gU{{\mathcal{U}}}
\def\gV{{\mathcal{V}}}
\def\gW{{\mathcal{W}}}
\def\gX{{\mathcal{X}}}
\def\gY{{\mathcal{Y}}}
\def\gZ{{\mathcal{Z}}}

% Sets
\def\sA{{\mathbb{A}}}
\def\sB{{\mathbb{B}}}
\def\sC{{\mathbb{C}}}
\def\sD{{\mathbb{D}}}
% Don't use a set called E, because this would be the same as our symbol
% for expectation.
\def\sF{{\mathbb{F}}}
\def\sG{{\mathbb{G}}}
\def\sH{{\mathbb{H}}}
\def\sI{{\mathbb{I}}}
\def\sJ{{\mathbb{J}}}
\def\sK{{\mathbb{K}}}
\def\sL{{\mathbb{L}}}
\def\sM{{\mathbb{M}}}
\def\sN{{\mathbb{N}}}
\def\sO{{\mathbb{O}}}
\def\sP{{\mathbb{P}}}
\def\sQ{{\mathbb{Q}}}
\def\sR{{\mathbb{R}}}
\def\sS{{\mathbb{S}}}
\def\sT{{\mathbb{T}}}
\def\sU{{\mathbb{U}}}
\def\sV{{\mathbb{V}}}
\def\sW{{\mathbb{W}}}
\def\sX{{\mathbb{X}}}
\def\sY{{\mathbb{Y}}}
\def\sZ{{\mathbb{Z}}}

% Entries of a matrix
\def\emLambda{{\Lambda}}
\def\emA{{A}}
\def\emB{{B}}
\def\emC{{C}}
\def\emD{{D}}
\def\emE{{E}}
\def\emF{{F}}
\def\emG{{G}}
\def\emH{{H}}
\def\emI{{I}}
\def\emJ{{J}}
\def\emK{{K}}
\def\emL{{L}}
\def\emM{{M}}
\def\emN{{N}}
\def\emO{{O}}
\def\emP{{P}}
\def\emQ{{Q}}
\def\emR{{R}}
\def\emS{{S}}
\def\emT{{T}}
\def\emU{{U}}
\def\emV{{V}}
\def\emW{{W}}
\def\emX{{X}}
\def\emY{{Y}}
\def\emZ{{Z}}
\def\emSigma{{\Sigma}}

% entries of a tensor
% Same font as tensor, without \bm wrapper
\newcommand{\etens}[1]{\mathsfit{#1}}
\def\etLambda{{\etens{\Lambda}}}
\def\etA{{\etens{A}}}
\def\etB{{\etens{B}}}
\def\etC{{\etens{C}}}
\def\etD{{\etens{D}}}
\def\etE{{\etens{E}}}
\def\etF{{\etens{F}}}
\def\etG{{\etens{G}}}
\def\etH{{\etens{H}}}
\def\etI{{\etens{I}}}
\def\etJ{{\etens{J}}}
\def\etK{{\etens{K}}}
\def\etL{{\etens{L}}}
\def\etM{{\etens{M}}}
\def\etN{{\etens{N}}}
\def\etO{{\etens{O}}}
\def\etP{{\etens{P}}}
\def\etQ{{\etens{Q}}}
\def\etR{{\etens{R}}}
\def\etS{{\etens{S}}}
\def\etT{{\etens{T}}}
\def\etU{{\etens{U}}}
\def\etV{{\etens{V}}}
\def\etW{{\etens{W}}}
\def\etX{{\etens{X}}}
\def\etY{{\etens{Y}}}
\def\etZ{{\etens{Z}}}

% The true underlying data generating distribution
\newcommand{\pdata}{p_{\rm{data}}}
\newcommand{\ptarget}{p_{\rm{target}}}
\newcommand{\pprior}{p_{\rm{prior}}}
\newcommand{\pbase}{p_{\rm{base}}}
\newcommand{\pref}{p_{\rm{ref}}}

% The empirical distribution defined by the training set
\newcommand{\ptrain}{\hat{p}_{\rm{data}}}
\newcommand{\Ptrain}{\hat{P}_{\rm{data}}}
% The model distribution
\newcommand{\pmodel}{p_{\rm{model}}}
\newcommand{\Pmodel}{P_{\rm{model}}}
\newcommand{\ptildemodel}{\tilde{p}_{\rm{model}}}
% Stochastic autoencoder distributions
\newcommand{\pencode}{p_{\rm{encoder}}}
\newcommand{\pdecode}{p_{\rm{decoder}}}
\newcommand{\precons}{p_{\rm{reconstruct}}}

\newcommand{\laplace}{\mathrm{Laplace}} % Laplace distribution

\newcommand{\E}{\mathbb{E}}
\newcommand{\Ls}{\mathcal{L}}
\newcommand{\R}{\mathbb{R}}
\newcommand{\emp}{\tilde{p}}
\newcommand{\lr}{\alpha}
\newcommand{\reg}{\lambda}
\newcommand{\rect}{\mathrm{rectifier}}
\newcommand{\softmax}{\mathrm{softmax}}
\newcommand{\sigmoid}{\sigma}
\newcommand{\softplus}{\zeta}
\newcommand{\KL}{D_{\mathrm{KL}}}
\newcommand{\Var}{\mathrm{Var}}
\newcommand{\standarderror}{\mathrm{SE}}
\newcommand{\Cov}{\mathrm{Cov}}
% Wolfram Mathworld says $L^2$ is for function spaces and $\ell^2$ is for vectors
% But then they seem to use $L^2$ for vectors throughout the site, and so does
% wikipedia.
\newcommand{\normlzero}{L^0}
\newcommand{\normlone}{L^1}
\newcommand{\normltwo}{L^2}
\newcommand{\normlp}{L^p}
\newcommand{\normmax}{L^\infty}

\newcommand{\parents}{Pa} % See usage in notation.tex. Chosen to match Daphne's book.

\DeclareMathOperator*{\argmax}{arg\,max}
\DeclareMathOperator*{\argmin}{arg\,min}

\DeclareMathOperator{\sign}{sign}
\DeclareMathOperator{\Tr}{Tr}
\let\ab\allowbreak


\usepackage[utf8]{inputenc} % allow utf-8 input
\usepackage[T1]{fontenc}    % use 8-bit T1 fonts
\usepackage{hyperref}       % hyperlinks
\usepackage{url}            % simple URL typesetting
\usepackage{booktabs}       % professional-quality tables
\usepackage{amsfonts}       % blackboard math symbols
\usepackage{nicefrac}       % compact symbols for 1/2, etc.
\usepackage{microtype}      % microtypography
\usepackage{lipsum}
\usepackage{hyperref}
\usepackage{url}
\usepackage[round]{natbib}
\usepackage{bbm}
\usepackage{xcolor}

\usepackage{graphicx} % Required for inserting images
\usepackage{amsfonts,amsmath,amssymb,amsthm}
\usepackage{bbm}
\usepackage{cleveref}
\usepackage{algorithm}
\usepackage[algo2e, vlined, ruled, linesnumbered]{algorithm2e}



\newcommand{\ip}[1]{\left\langle #1 \right\rangle}
\newcommand{\bias}[1]{\mathbf{bias}_{#1}}
\newcommand{\Expect}[1]{\mathbb{E}\left[ #1 \right]}

\DeclareMathOperator{\indi}{\mathbbm{1}}
%\DeclareMathOperator{\E}{\mathbb{E}}

\DeclareMathOperator{\NodeIn}{N_{\text{in}}}
\DeclareMathOperator{\NodeOut}{N_{\text{out}}}
\DeclareMathOperator{\Prob}{\mathbb{P}}
\DeclareMathOperator{\Reg}{\mathrm{Reg}}
%\DeclareMathOperator{\argmin}{\text{argmin}}
%\DeclareMathOperator{\Var}{\mathrm{Var}}
\DeclareMathOperator{\Time}{\mathcal{T}}
\DeclareMathOperator{\Tl}{\mathcal{T}^{\ell}}
\DeclareMathOperator{\Tel}{\mathcal{T}_e^{\ell}}
\DeclareMathOperator{\Tf}{\mathcal{T}^{f}}
\DeclareMathOperator{\Tef}{\mathcal{T}_e^{f}}
\DeclareMathOperator{\His}{\mathcal{H}}

%%%%%%%%%%%%%%%%%%%%%%%%%%%%%%%%
% THEOREMS
%%%%%%%%%%%%%%%%%%%%%%%%%%%%%%%%
\theoremstyle{plain}
\makeatletter
\newtheorem*{rep@theorem}{\rep@title}
\newcommand{\newreptheorem}[2]{%
\newenvironment{rep#1}[1]{%
 \def\rep@title{#2 \ref{##1}}%
 \begin{rep@theorem}}%
 {\end{rep@theorem}}}
\makeatother

\newtheorem{theorem}{Theorem}
\newreptheorem{theorem}{Theorem}

\newtheorem{proposition}[theorem]{Proposition}
\newtheorem{lemma}[theorem]{Lemma}
\newtheorem{corollary}[theorem]{Corollary}
\theoremstyle{definition}
\newtheorem{definition}[theorem]{Definition}
\newtheorem{assumption}[theorem]{Assumption}
\theoremstyle{remark}
\newtheorem{remark}[theorem]{Remark}
\newtheorem{question}{Question}
\newtheorem*{question*}{Question}

\author{Ruiyuan Huang\\
School of Data Science\\
Fudan University\\
Shanghai, China\\
\texttt{RuiyuanHuang00@gmail.com} \\
\And
Zengfeng Huang\thanks{Corresponding Author}\\
School of Data Science\\
Fudan University\\
Shanghai, China\\
\texttt{huangzf@fudan.edu.cn} \\
% \AND
% Coauthor \\
% Affiliation \\
% Address \\
% \texttt{email}
}

% The \author macro works with any number of authors. There are two commands
% used to separate the names and addresses of multiple authors: \And and \AND.

% Using \And between authors leaves it to \LaTeX{} to determine where to break
% the lines. Using \AND forces a linebreak at that point. So, if \LaTeX{}
% puts 3 of 4 authors names on the first line, and the last on the second
% line, try using \AND instead of \And before the third author name.

\newcommand{\fix}{\marginpar{FIX}}
\newcommand{\new}{\marginpar{NEW}}

\title{Nearly Tight Bounds for Cross-Learning Contextual Bandits with Graphical Feedback}
%\iclrfinalcopy

% Authors must not appear in the submitted version. They should be hidden
% as long as the \iclrfinalcopy macro remains commented out below.
% Non-anonymous submissions will be rejected without review.

\author{Ruiyuan Huang\\
School of Data Science\\
Fudan University\\
Shanghai, China\\
\texttt{RuiyuanHuang00@gmail.com} \\
\And
Zengfeng Huang\thanks{Corresponding Author}\\
School of Data Science\\
Fudan University\\
Shanghai, China\\
\texttt{huangzf@fudan.edu.cn} \\
% \AND
% Coauthor \\
% Affiliation \\
% Address \\
% \texttt{email}
}


%\iclrfinalcopy % Uncomment for camera-ready version, but NOT for submission.
\begin{document}
\maketitle

\begin{abstract}
%Cross-Learning 很火,是什么。
%谁提出,有什么应用,理论趣味。
%无关context number的结果。
%Han further 考虑了这个带上 graphical feedback。
%这个问题有应用,进一步enrich了它的结构。
%但是,Han没有给出independent of context number的结果。
%最近的一篇新工作细致地研究了这个问题,并将能否在stochastic context下得到好bound列为open结果。
%我们解决了这个问题
%技术上我们对这个stochastic的问题用了adv的技术
%我们也得到了adversarial的结果

The cross-learning contextual bandit problem with graphical feedback has recently attracted significant attention.
%
In this setting, there is a contextual bandit with a feedback graph over the arms, and pulling an arm reveals the loss for all neighboring arms in the feedback graph across all contexts. 
%
Initially proposed by~\citet{Han24}, this problem has broad applications in areas such as bidding in first price auctions, and explores a novel frontier in the feedback structure of bandit problems.
%
A key theoretical question is whether an algorithm with $\widetilde{O}(\sqrt{\alpha T})$ regret exists, where $\alpha$ represents the independence number of the feedback graph. 
%
This question is particularly interesting because it concerns whether an algorithm can achieve a regret bound entirely independent of the number of contexts and matching the minimax regret of vanilla graphical bandits.
%
Previous work has demonstrated that such an algorithm is impossible for adversarial contexts, but the question remains open for stochastic contexts.
%
In this work, we affirmatively answer this open question by presenting an algorithm that achieves the minimax $\widetilde{O}(\sqrt{\alpha T})$ regret for cross-learning contextual bandits with graphical feedback and stochastic contexts.
%
Notably, although that question is open even for stochastic bandits, we directly solve the strictly stronger adversarial bandit version of the problem.
%
\end{abstract}


\section{Introduction}

Large language models (LLMs) have achieved remarkable success in automated math problem solving, particularly through code-generation capabilities integrated with proof assistants~\citep{lean,isabelle,POT,autoformalization,MATH}. Although LLMs excel at generating solution steps and correct answers in algebra and calculus~\citep{math_solving}, their unimodal nature limits performance in plane geometry, where solution depends on both diagram and text~\citep{math_solving}. 

Specialized vision-language models (VLMs) have accordingly been developed for plane geometry problem solving (PGPS)~\citep{geoqa,unigeo,intergps,pgps,GOLD,LANS,geox}. Yet, it remains unclear whether these models genuinely leverage diagrams or rely almost exclusively on textual features. This ambiguity arises because existing PGPS datasets typically embed sufficient geometric details within problem statements, potentially making the vision encoder unnecessary~\citep{GOLD}. \cref{fig:pgps_examples} illustrates example questions from GeoQA and PGPS9K, where solutions can be derived without referencing the diagrams.

\begin{figure}
    \centering
    \begin{subfigure}[t]{.49\linewidth}
        \centering
        \includegraphics[width=\linewidth]{latex/figures/images/geoqa_example.pdf}
        \caption{GeoQA}
        \label{fig:geoqa_example}
    \end{subfigure}
    \begin{subfigure}[t]{.48\linewidth}
        \centering
        \includegraphics[width=\linewidth]{latex/figures/images/pgps_example.pdf}
        \caption{PGPS9K}
        \label{fig:pgps9k_example}
    \end{subfigure}
    \caption{
    Examples of diagram-caption pairs and their solution steps written in formal languages from GeoQA and PGPS9k datasets. In the problem description, the visual geometric premises and numerical variables are highlighted in green and red, respectively. A significant difference in the style of the diagram and formal language can be observable. %, along with the differences in formal languages supported by the corresponding datasets.
    \label{fig:pgps_examples}
    }
\end{figure}



We propose a new benchmark created via a synthetic data engine, which systematically evaluates the ability of VLM vision encoders to recognize geometric premises. Our empirical findings reveal that previously suggested self-supervised learning (SSL) approaches, e.g., vector quantized variataional auto-encoder (VQ-VAE)~\citep{unimath} and masked auto-encoder (MAE)~\citep{scagps,geox}, and widely adopted encoders, e.g., OpenCLIP~\citep{clip} and DinoV2~\citep{dinov2}, struggle to detect geometric features such as perpendicularity and degrees. 

To this end, we propose \geoclip{}, a model pre-trained on a large corpus of synthetic diagram–caption pairs. By varying diagram styles (e.g., color, font size, resolution, line width), \geoclip{} learns robust geometric representations and outperforms prior SSL-based methods on our benchmark. Building on \geoclip{}, we introduce a few-shot domain adaptation technique that efficiently transfers the recognition ability to real-world diagrams. We further combine this domain-adapted GeoCLIP with an LLM, forming a domain-agnostic VLM for solving PGPS tasks in MathVerse~\citep{mathverse}. 
%To accommodate diverse diagram styles and solution formats, we unify the solution program languages across multiple PGPS datasets, ensuring comprehensive evaluation. 

In our experiments on MathVerse~\citep{mathverse}, which encompasses diverse plane geometry tasks and diagram styles, our VLM with a domain-adapted \geoclip{} consistently outperforms both task-specific PGPS models and generalist VLMs. 
% In particular, it achieves higher accuracy on tasks requiring geometric-feature recognition, even when critical numerical measurements are moved from text to diagrams. 
Ablation studies confirm the effectiveness of our domain adaptation strategy, showing improvements in optical character recognition (OCR)-based tasks and robust diagram embeddings across different styles. 
% By unifying the solution program languages of existing datasets and incorporating OCR capability, we enable a single VLM, named \geovlm{}, to handle a broad class of plane geometry problems.

% Contributions
We summarize the contributions as follows:
We propose a novel benchmark for systematically assessing how well vision encoders recognize geometric premises in plane geometry diagrams~(\cref{sec:visual_feature}); We introduce \geoclip{}, a vision encoder capable of accurately detecting visual geometric premises~(\cref{sec:geoclip}), and a few-shot domain adaptation technique that efficiently transfers this capability across different diagram styles (\cref{sec:domain_adaptation});
We show that our VLM, incorporating domain-adapted GeoCLIP, surpasses existing specialized PGPS VLMs and generalist VLMs on the MathVerse benchmark~(\cref{sec:experiments}) and effectively interprets diverse diagram styles~(\cref{sec:abl}).

\iffalse
\begin{itemize}
    \item We propose a novel benchmark for systematically assessing how well vision encoders recognize geometric premises, e.g., perpendicularity and angle measures, in plane geometry diagrams.
	\item We introduce \geoclip{}, a vision encoder capable of accurately detecting visual geometric premises, and a few-shot domain adaptation technique that efficiently transfers this capability across different diagram styles.
	\item We show that our final VLM, incorporating GeoCLIP-DA, effectively interprets diverse diagram styles and achieves state-of-the-art performance on the MathVerse benchmark, surpassing existing specialized PGPS models and generalist VLM models.
\end{itemize}
\fi

\iffalse

Large language models (LLMs) have made significant strides in automated math word problem solving. In particular, their code-generation capabilities combined with proof assistants~\citep{lean,isabelle} help minimize computational errors~\citep{POT}, improve solution precision~\citep{autoformalization}, and offer rigorous feedback and evaluation~\citep{MATH}. Although LLMs excel in generating solution steps and correct answers for algebra and calculus~\citep{math_solving}, their uni-modal nature limits performance in domains like plane geometry, where both diagrams and text are vital.

Plane geometry problem solving (PGPS) tasks typically include diagrams and textual descriptions, requiring solvers to interpret premises from both sources. To facilitate automated solutions for these problems, several studies have introduced formal languages tailored for plane geometry to represent solution steps as a program with training datasets composed of diagrams, textual descriptions, and solution programs~\citep{geoqa,unigeo,intergps,pgps}. Building on these datasets, a number of PGPS specialized vision-language models (VLMs) have been developed so far~\citep{GOLD, LANS, geox}.

Most existing VLMs, however, fail to use diagrams when solving geometry problems. Well-known PGPS datasets such as GeoQA~\citep{geoqa}, UniGeo~\citep{unigeo}, and PGPS9K~\citep{pgps}, can be solved without accessing diagrams, as their problem descriptions often contain all geometric information. \cref{fig:pgps_examples} shows an example from GeoQA and PGPS9K datasets, where one can deduce the solution steps without knowing the diagrams. 
As a result, models trained on these datasets rely almost exclusively on textual information, leaving the vision encoder under-utilized~\citep{GOLD}. 
Consequently, the VLMs trained on these datasets cannot solve the plane geometry problem when necessary geometric properties or relations are excluded from the problem statement.

Some studies seek to enhance the recognition of geometric premises from a diagram by directly predicting the premises from the diagram~\citep{GOLD, intergps} or as an auxiliary task for vision encoders~\citep{geoqa,geoqa-plus}. However, these approaches remain highly domain-specific because the labels for training are difficult to obtain, thus limiting generalization across different domains. While self-supervised learning (SSL) methods that depend exclusively on geometric diagrams, e.g., vector quantized variational auto-encoder (VQ-VAE)~\citep{unimath} and masked auto-encoder (MAE)~\citep{scagps,geox}, have also been explored, the effectiveness of the SSL approaches on recognizing geometric features has not been thoroughly investigated.

We introduce a benchmark constructed with a synthetic data engine to evaluate the effectiveness of SSL approaches in recognizing geometric premises from diagrams. Our empirical results with the proposed benchmark show that the vision encoders trained with SSL methods fail to capture visual \geofeat{}s such as perpendicularity between two lines and angle measure.
Furthermore, we find that the pre-trained vision encoders often used in general-purpose VLMs, e.g., OpenCLIP~\citep{clip} and DinoV2~\citep{dinov2}, fail to recognize geometric premises from diagrams.

To improve the vision encoder for PGPS, we propose \geoclip{}, a model trained with a massive amount of diagram-caption pairs.
Since the amount of diagram-caption pairs in existing benchmarks is often limited, we develop a plane diagram generator that can randomly sample plane geometry problems with the help of existing proof assistant~\citep{alphageometry}.
To make \geoclip{} robust against different styles, we vary the visual properties of diagrams, such as color, font size, resolution, and line width.
We show that \geoclip{} performs better than the other SSL approaches and commonly used vision encoders on the newly proposed benchmark.

Another major challenge in PGPS is developing a domain-agnostic VLM capable of handling multiple PGPS benchmarks. As shown in \cref{fig:pgps_examples}, the main difficulties arise from variations in diagram styles. 
To address the issue, we propose a few-shot domain adaptation technique for \geoclip{} which transfers its visual \geofeat{} perception from the synthetic diagrams to the real-world diagrams efficiently. 

We study the efficacy of the domain adapted \geoclip{} on PGPS when equipped with the language model. To be specific, we compare the VLM with the previous PGPS models on MathVerse~\citep{mathverse}, which is designed to evaluate both the PGPS and visual \geofeat{} perception performance on various domains.
While previous PGPS models are inapplicable to certain types of MathVerse problems, we modify the prediction target and unify the solution program languages of the existing PGPS training data to make our VLM applicable to all types of MathVerse problems.
Results on MathVerse demonstrate that our VLM more effectively integrates diagrammatic information and remains robust under conditions of various diagram styles.

\begin{itemize}
    \item We propose a benchmark to measure the visual \geofeat{} recognition performance of different vision encoders.
    % \item \sh{We introduce geometric CLIP (\geoclip{} and train the VLM equipped with \geoclip{} to predict both solution steps and the numerical measurements of the problem.}
    \item We introduce \geoclip{}, a vision encoder which can accurately recognize visual \geofeat{}s and a few-shot domain adaptation technique which can transfer such ability to different domains efficiently. 
    % \item \sh{We develop our final PGPS model, \geovlm{}, by adapting \geoclip{} to different domains and training with unified languages of solution program data.}
    % We develop a domain-agnostic VLM, namely \geovlm{}, by applying a simple yet effective domain adaptation method to \geoclip{} and training on the refined training data.
    \item We demonstrate our VLM equipped with GeoCLIP-DA effectively interprets diverse diagram styles, achieving superior performance on MathVerse compared to the existing PGPS models.
\end{itemize}

\fi 

\section{Preliminaries}
\label{Preliminaries}
\subsection{Multi-Agent Reinforcement Learning}
A MARL problem can be formulated as a decentralized partially observed Markov decision process (Dec-POMDP)~\cite{oliehoek2016concise}, which is described as a tuple $\langle n,\boldsymbol{S},\boldsymbol{A},P,R,\boldsymbol{O},\boldsymbol{\Omega},\gamma\rangle $, where $n$ represents the number of agents, $\boldsymbol{S}$ is the global state space. $\boldsymbol{A}$ is the action space. $\boldsymbol{O}=\{O_{i}\}_{i=1,\cdots,n}$ is the observation space. At timestep $t$, each agent $i$ receives an observation $o_{i}^t\in O_{i}$ according to the observation function $\boldsymbol{\Omega}(s^t,i):\boldsymbol{S}\to O_i$ and then selects an action $a_i^t\in\boldsymbol{A}$. The joint action $\boldsymbol{a}^t=(a_1^t,\ldots,a_n^t)$ is then applied to the environment, resulting in a transition to the next state $s^{t+1}$ and a global reward signal $r^{t}$ according to the transition function $P(s^{t+1}\mid s^{t},\boldsymbol{a}^t)$ and the reward function $R(s^t,\boldsymbol{a}^t)$. $\gamma\in[0,1]$ is the discount factor. The objective is to learn a joint policy $\pi$ that maximizes the expected cumulative reward $\mathbb{E}\left[\sum_{t=0}^{\infty}\gamma^{t}r^{t}\right|\pi]$.

\subsection{Centralized Training With Decentralized Execution}
Centralized Training with Decentralized Execution (CTDE) is a commonly employed architecture in MARL~\cite{lowe2017multi,rashid2020monotonic}. In CTDE, each agent utilizes an actor network to make decisions based on local observations. Additionally, the training process incorporates global information to train a centralized value function. The centralized value function provides a centralized gradient to update the actor network based on its outputs.

\subsection{Generalizable Model Structure in MARL}
To handle varying state/observation/action spaces, previous works like UPDeT~\cite{hu2021updet} and ASN~\cite{wang2019action} propose a generalizable model that treats all agents as entities. In such models, observation $o_i$ can be conducted as entity-observations: $[o_{i,1},o_{i,2},...,o_{i,m}]$, where $m$ denotes the number of all entities in the environment. Based on the criterion of whether entites can be observed, entity-observations can be splited into two subsets: observed entity-observations $o_{\mathrm{obs},i}$ and unobserved entity-observations $o_{\mathrm{mask},i}$. We denote the number of observed entities and masked entities as $n_{\mathrm{obs}}$ and $n_{\mathrm{mask}}$, respectively, and it holds that $m=n_{\mathrm{obs}}+n_{\mathrm{mask}}$.
Additionally, action space $\boldsymbol{A}$ can be decomposed into two subsets:$\boldsymbol{A}^{\mathrm{self}}$ containing actions that affect the environment or itself  and $\boldsymbol{A}^{\mathrm{out}}$ representing actions that directly interact with other entities.
\section{The Known Distribution Setting}
Here we consider the situation where the context distribution $\nu$ is known in advance. We use this situation to prepare the readers for the more involved situation where the context distribution $\nu$ is unknown.


\subsection{The Algorithm in the Known Distribution Setting}
When the distribution $\nu$ is known, our algorithm is fairly simple. 
%
Recall that we denote the probability of observing the loss of arm $a$ at round $t$ as $w_t(a) \triangleq \E_{c \sim \nu}[p_{t,c}(\NodeIn(a))]$.
%
We construct an importance-weighted estimator 
\[\widetilde{\ell}_{t,c}(a) = \frac{\ell_{t,c}(a)}{w_t(a)} \indi(a_t \rightarrow a)\]
and determine the playing distribution by the standard FTRL subroutine:
\[p_{t,c} = \underset{p \in \Delta([K])}{\argmin} \left\langle p, \sum_{s=1}^{t-1} \widetilde{\ell}_{s,c} \right\rangle - \frac{1}{\eta} F(p)\]
where $F(p) = \sum_{i=1}^K p_i \log(p_i)$ is the unnormalized negative entropy and $\eta$ is the learning rate.

\begin{algorithm}
\label{alg: known}
\caption{The algorithm when the distribution is known}
\textbf{Input:} Parameters $\eta > 0$\\
\For{$t=1,\dots,T$}{
Set $p_{t,c}= \argmin_{p \in\Delta([K])} \left(\ip{p,\sum_{s=1}^{t-1}\widetilde{\ell}_{s,c}}-\eta^{-1}F(x)\right)$\\
Observe $c_t$ \\
Play $A_t \sim p_{t,c_t}$ \\
Set $\widetilde{\ell}_{t,c}(a) \leftarrow \frac{\ell_{t,c}(a)}{w_t(a)} \mathbb{I}\left(A_{t} \rightarrow a\right)$ for each $c$
} 
\end{algorithm}

\subsection{The analysis in the Known Setting}
The analysis of the algorithm is rather straightforward. We only need to notice the regret decomposition
\[\E[\Reg(\pi)] = \sum_c \Pr(c) \E\left[\sum_t \ell_{t,c}(a_t) - \ell_{t,c}(\pi_c)\right].\]
And for each $c$, we consider
\begin{align*}
    &\E\left[\sum_t \ell_{t,c}(a_t) - \ell_{t,c}(\pi_c)\right]\\
%
   =& \E\left[\sum_t \ip{p_{t,c} - \pi_c,\ell_{t,c}} \right]\\
%
   \le& \frac{1}{\eta} \log K + \frac{\eta}{2} \E\left[\sum_t \ip{p_{t,c}, \widehat{\ell}_{t,c}^2} \right]  \\
   =& \frac{1}{\eta} \log K + \frac{\eta}{2} \E\left[\sum_t \sum_a p_{t,c}(a) \frac{\ell_{t,c}(a)}{w^2_t(a)} \indi(a_t \rightarrow a)  \right] \\
    =&     \frac{1}{\eta} \log K + \frac{\eta}{2} \E\left[\sum_t \sum_a p_{t,c}(a) \frac{\ell_{t,c}(a)}{w_t(a)}  \right] \\
    \le& \frac{1}{\eta} \log K + \frac{\eta}{2} \E\left[\sum_t \sum_a p_{t,c}(a) \frac{1}{w_t(a)}  \right] .
\end{align*}
Here the first inequality comes from the standard FTRL analysis.

We then have
\begin{align*}
    &\E\left[\Reg(\pi)\right]\\
    =& \sum_c \Pr(c) \E\left[\sum_t \ell_{t,c}(a_t) - \ell_{t,c}(\pi_c)\right]\\
%
    \le& \sum_c \Pr(c)  \left(  \frac{1}{\eta} \log K + \frac{\eta}{2} \E\left[\sum_t \sum_a p_{t,c}(a) \frac{1}{w_t(a)}  \right] \right) \\
% 
    =&  \frac{1}{\eta} \log K  + \frac{\eta}{2}  \E\left[\sum_t \sum_a \sum_c \Pr(c) p_{t,c}(a) \frac{1}{w_t(a)}    \right]\\
    =& \frac{1}{\eta} \log K  + \frac{\eta}{2}  \E\left[\sum_t \sum_a p_{t}(a)  \frac{1}{w_t(a)} \right] .
\end{align*}

A standard lemma in graphical bandit (\Cref{lem: graph_inverse}) shows that $\sum_t \sum_a p_{t}(a)  \frac{1}{w_t(a)} = \widetilde{O}(\alpha T)$. Taking $\eta = \Theta(\frac{1}{\sqrt{\alpha T}})$, we get that the algorithm satisfies
\[\Reg(\pi) \le \widetilde{O}(\sqrt{\alpha T}).\]
\section{The Unknown Distribution Setting}

Here we consider the situation where the context distribution $\nu$ is unknown. The basic idea is samilar to the situation where the  context distribution $\nu$ is known. We will form an importance-weighted estimator and feed the importance-weighted estimator to the standard FTRL subroutine to generate the playing distribution.

However, the unknown distribution situation is significantly more complicated than the known distribution situation. The main difficulty is that the importance $w_t(a)$ in the known distribution setting is now uncomputable  due to the distribution is now unknown. 
%
Recall that $p_{t,c_s}(\NodeIn(a)$ means  $\sum_{a' \in \NodeIn(a)} p_{t,c_s}(a')$.
%
A natural temptation is to replace the expected importance $w_t(a)$ with the empirical importance  $\frac{1}{t} \sum_{s=1}^t  p_{t,c_s}(\NodeIn(a))$.
%
However, this temptation does not work. The problem with the empirical importance $\frac{1}{t} \sum_{s=1}^t p_{t,c_s}(\NodeIn(a))$ is that the probability $p_t$ of playing arms in round $t$ is impacted by contexts $\{c_s\}_{s < t}$ in previous rounds. Thus, the  probability vector $p_t$ is not independent of the contexts $c_s$ for $s<t$, and the empirical importance  $\frac{1}{t} \sum_{s=1}^t  p_{t,c_s}(\NodeIn(a))$ has no concentration guarantees.

To overcome this difficulty, we employ techniques from \citet{Sch23}. We divide the time horizon $[T]$ into $T/L$ epochs, denoted as $\{\Time_e\}_{e = 1}^{L}$. All epochs have the same length $|\Time_e| = L$ . We also decouple the probability of playing arms from the probability of observing losses. During each epoch $e$, at each round $t \in \Time_e$, the probability $p_t$ of playing arms is determined by the FTRL subroutine as usual, while the probability of observing losses $ws_e$ remains unchanged in the entire epoch. This is possible because the probability $p_t$ only changes a constant factor within one epoch.

Since the probability $w_e$ of observing losses remains unchanged throughout the entire epoch, we only need to estimate the probability $w_e$ of observing losses once during epoch $e$. The fact that we only estimate the probability $w_e$ once per epoch allows us to obtain a number of independent samples of epoch length $L$  to construct an efficient estimator $\widehat{w}_e$. We use $\widehat{w}_e$ as the importance to form an importance-weighted  estimator $\widehat{\ell}$, and feed this estimator $\widehat{\ell}$ to the FTRL subroutine to compute the probability $p_t$ of playing arms. 

\subsection{The Algorithm in the Unknown Distribution Setting}
Our algorithm is similar to the algorithm in ~\citet{Sch23}. The main difference is that we utilize the graphical structure in constructing the importance. 
%

% FTRL, not p_t but q_t
In each round $t$, our algorithm generates a distribution using an FTRL subroutine: \[p_{t,c} = \argmin_{p \in \Delta([K])} \ip{p, \sum_{s=1}^{t-1} \widehat{\ell}_{s,c}} - \frac{1}{\eta} F(p)\]
for each context $c$, where $F(p) = \sum_{i=1}^K p_i \log(p_i)$ is the unnormalized negative entropy, $\eta$ is the learning rate, and $\widehat{\ell}$ are loss estimates to be defined later.
%
Due to some technical issues which we specify later, our algorithm does not sample the action $a_t$ played in round $t$ directly from $p_t$ but from a distribution $q_t$, which will be defined later.

% epoch, snaposhot
As mentioned at the beginning of this section, the algorithm divides the time horizon $[T]$ into $T/L$ epochs of equal length $L$, where $L$ is a tunable parameter. Let $\Time_e$ denote the set of rounds in the $e$-th epoch. At the end of each epoch, the algorithm takes a single snapshot of the underlying FTRL distribution $p_t$ for each context and arm. That is, the algorithm takes
\[s_{e+2,c}(a) = p_{eL,c}(a)\]
and
\[s_{1, c}(a) = s_{2, c}(a) = \frac{1}{K}.\]
The index of the snapshot $s_{e+2}$ for epoch $e$ is $e+2$ because the snapshot $s_{e+2}$ is used in epoch $e+2$.

%reject-sampling
As mentioned at the beginning of this section, our algorithm keeps the probability of observing losses unchanged within epoch $e$. 
%
Specifically, for every round $t \in \Time_e$, the algorithm observes the loss of arm $a$ with probability $w_e(a)=\mathbb{E}_{c \sim \nu}\left[ \sum_{a' \rightarrow a} s_{e,c}(a') / 2\right]$. Note that here $s_e$ is the snapshot of the probability distribution at the end of epoch $e-2$, i.e., $s_{e,c}(a) = p_{(e-2)L,c}(a)$.
%
We guarantee that the algorithm observes losses with probability $w_e(a)$ within epoch $e$ using the following rejection sampling procedure. Recall that $p_{t,c}(\NodeIn(a)) \triangleq \sum_{a' \in \NodeIn(a)} p_{t,c}(a')$ and $s_{e,c}(\NodeIn(a)) \triangleq \sum_{a' \in \NodeIn(a)} s_{e,c}(a')$. We first play an arm according to the distribution
\[
q_{t, c_t}= \begin{cases}p_{t, c_t} & \text { if } \forall a \in[K], 2 p_{t,c_t}(\NodeIn(a))  \geq s_{e, c_t}(\NodeIn(a))  \\ s_{e, c_t} & \text { otherwise.}\end{cases}
\]
Let ~$q_{t,c}(\NodeIn(a)) \triangleq \sum_{a' \in \NodeIn(a)} q_{t,c}(a')$. After playing arm $a_t$ according to $q_{t, c_t}$, for each arm $a \in \NodeOut(a_t)$, the learner samples a Bernoulli random variable $S_{t,a}$ with probability $\frac{s_{e,c_t}(\NodeIn(a))}{2 q_{t,c_t}(\NodeIn(a))}$. If $S_{t,a}=0$, the learner ignores the feedback of $a$ from this round; otherwise, the learner uses this loss. 

In the previous procedure, we introduce the new distribution $q_t$ to ensure that we can perform a rejection sampling procedure. The ctual rejection sampling procedure ensures that we observe the loss of arm $a$ with probability $w_e(a)=\mathbb{E}_{c \sim \nu}\left[  s_{e,c}(\NodeIn(a))/2\right]$.


% Estimation
The only remaining unspecified part is how to construct the loss estimators.
%
Due to technical problems encountered in the analysis later, we group all timesteps into consecutive pairs of two. In each pair of consecutive timesteps, we sample from the same distribution and randomly use one to calculate a loss estimator and the other to estimate the sampling frequency.
%
To be precise, let $\Time_e^f$ denote the timesteps selected for estimating the sampling frequency, and let $\Time_e^{\ell}$ denote the timesteps used to estimate the losses. Then we define
$$
\widehat{w}_{e}(a)=\frac{1}{\left|\Time_{e-1}^f\right|} \sum_{t \in \Time_{e-1}^f} \frac{s_{e,c_t}(a)}{2}
$$
which is an unbiased estimator of $w_e(a)$. The loss estimators are defined as follows:
$$
\widehat{\ell}_{t,c}(a)=\frac{2 \ell_{t,c}(\NodeIn(a))}{\widehat{w}_{e}(a)+\frac{3}{2} \gamma} \indi\left(A_t \rightarrow a \wedge S_{t,a} \wedge t \in \Time_e^{\ell}\right)
$$
where $\gamma$ is a confidence parameter to be specified later.

We note that in \Cref{alg:unknown}, the actual importance is $\widehat{w}_e(a) + \frac{3}{2} \gamma$ rather than vanilla $\widehat{w}_e(a)$. This additional $\frac{3}{2} \gamma$ parameter is an implicit exploration parameter. Implicit exploration parameters are widely used to derive concentration inequalities in bandit problems~\citep{neu2015}. We also introduce this implicit exploration parameter to derive the concentration inequalities needed for our arguments.


\begin{figure*}[hbt]
\begin{align*}
\underbrace{1,\dots,L}_{\small{\begin{matrix}\Time_1\quad \text{\small \color{purple}fix $s_3$}\\\text{\color{blue}compute $\widehat{w}_{2}$}\\\end{matrix}}},\underbrace{L+1,\dots, 2L}_{\small{\begin{matrix}\Time_2\quad\quad \text{\small \color{red}fix $s_4$}\\\text{\color{purple}compute $\widehat{w}_3$}\\\text{\color{blue}apply $\widehat{w}_2$}\\\text{\color{blue} sample with $s_2$}\end{matrix}}},\underbrace{2L+1,\dots, 3L}_{\small{\begin{matrix}\Time_3\quad\quad \text{\small \color{orange}fix $s_5$}\\\text{\color{red}compute $\widehat{w}_4$}\\\text{\color{purple}apply $\widehat{w}_3$}\\\text{\color{purple}sample with $s_3$}\end{matrix}}},
\underbrace{3L+1,\dots, 4L}_{\small{\begin{matrix}\Time_4\quad\quad \text{\small \color{yellow}fix $s_6$}\\\text{\color{orange}compute $\widehat{w}_5$}\\\text{\color{red}apply $\widehat{w}_4$}\\\text{\color{red}sample with $s_4$}\end{matrix}}},
\underbrace{4L+1,\dots, 5L}_{\small{\begin{matrix}\Time_5\quad\quad \text{\small \color{green}fix $s_7$}\\\text{\color{yellow}compute $\widehat{w}_6$}\\\text{\color{orange}apply $\widehat{w}_5$}\\\text{\color{orange}sample with $s_5$}\end{matrix}}},\dots, T
\end{align*}
\caption{A figure in~\citet{Sch23}. Here we use it to  illustrate the timeline of \Cref{alg:unknown}. At the end of epoch $\Time_e$, the snapshot $s_{e+2}$ is fixed. The contexts within epoch $\Time_{e}$ are used to compute loss estimators for epoch $\Time_{e+1}$, which are fed to the FTRL sub-algorithm.}
\end{figure*}

\begin{algorithm}
\caption{The algorithm for the unknown distribution setting}
\label{alg:unknown}
\textbf{Input:} Parameters $\eta, \gamma > 0$ and $L < T$. \\
$\widehat{w}_2\leftarrow 0$\\
$s_{1, c} \leftarrow \frac{1}{K}$ for each $c$\\
$s_{2, c} \leftarrow \frac{1}{K}$ for each $c$\\
\For{$t=1,\dots, L$}{
Observe $c_t$\\
Play $A_t\sim s_{1,c_t}$\\
    \For{$a \in [K]$}{
    $\widehat{w}_2(a) \leftarrow \widehat{w}_2(a) +  \frac{s_{2,c_t}(\NodeIn(a))}{2L}$
    }
}
\For{$e=2,\dots,T/L$}{
    $\widehat{w}_{e+1}\leftarrow 0$\\
    \For{$t=(e-1)L+1,t=(e-1)L+3,\dots,e L-1$}{
        Set $p_{t,c}= \underset{p \in \Delta{[K]}}{\argmin} \left(\ip{p,\sum_{s=1}^{t-1}\widehat{\ell}_{s}(c)}-\eta^{-1}F(p)\right)$\\
        \For{$t'=t,t+1$}{
            Observe $c_{t'}$\\
            \If{
                $p_{t,c_{t'}}(a) \geq s_{e,c_{t'}}(a)/2$ for all $a \in [K]$
            }{
                Set $q_{t',c_{t'}} = p_{t,c_{t'}}$
            }
            \Else{
                Set $q_{t',c_{t'}} = s_{e,c_{t'}}$
            }
            Play $A_{t'}\sim q_{t',c_{t'}}$ \\
            Observe $\ell_{t',A_{t'}}$
        }
        $t_{f},t_{\ell} \leftarrow \mathsf{RandPerm}(t,t+1)$\\
        \For{$a \in [K]$}{
            $\widehat{w}_{e+1}(a) \leftarrow\widehat{w}_{e+1}(a) + \frac{s_{e+1,c_{t_f}}(\NodeIn(a))}{2 (L/2)}$\\
            Sample $S_{t,a}\sim \mathcal{B}\left(\frac{s_{e,c_{t_\ell}}(\NodeIn(a)}{2q_{t,c_{t_\ell}}(\NodeIn(a))}\right)$\\
            $\widehat{\ell}_{t_\ell,c}(a) \leftarrow \frac{2\ell_{t_\ell,c}(a)}{\widehat{w}_{e}(\NodeIn(a))+\frac{3}{2}\gamma} \mathbb{I}\left(A_{t_\ell} \rightarrow a, S_{t,a}=1\right)$
        }
    }
    $s_{e+2} \leftarrow p_{t}$
}
\end{algorithm}

\subsection{The Algorithm Analysis}
We analyze the algorithm by combining the technique from \citet{Sch23} with the property of the graphical feedback structure. The main result is the following theorem.
\begin{theorem}
\label{thm:main}
For $\iota=2 \log (8 K T^2), L=\sqrt{\frac{\iota \alpha T}{\log (K)}}=\widetilde{\Theta}(\sqrt{\alpha T}), \gamma=\frac{16 \iota}{L}=\widetilde{\Theta}(1 / \sqrt{\alpha T})$, and $\eta=\frac{\gamma}{2(2 L \gamma+\iota)}=\widetilde{\Theta}(1 / \sqrt{\alpha T})$, \Cref{alg:unknown} yields a regret bound of
\[\Reg(\pi)= \widetilde{O}(\sqrt{\alpha T}).\]
\end{theorem}

Here we briefly overview the proof of \Cref{thm:main}. Technical details can be found in appendix.

The key of our analysis is to bound the bias $\ell - \widehat{\ell}$ in estimating the loss $\ell$. A critical problem in bounding the bias is that our estimator $\widehat{\ell}$ is only constructed from a number of epoch length $L$ many samples. As a result, the concentration of $\widehat{\ell}$ is not tight enough. We solve this problem by bounding only the expectation $\E[\ell - \widehat{\ell}]$ of the bias. To do so, we need the following technical preparations.  

We  introduce two sequences of random variables. With a slight abuse of notation, for each epoch $e$ and every round $t \in \Time_e$,  we define
\[\widetilde{\ell}_{t,c}(a) \triangleq \frac{2 \ell_{t,c}(a)}{w_e(a) + \gamma} \indi(a_t \rightarrow a, S_{t,a} = 1, t \in \Time_e^{\ell}).\]
The random variable $\widetilde{\ell}_{t,c}(a)$ is similar to the loss estimate $\widehat{\ell}_{t,c}(a)$. The main difference is that the importance for $\widetilde{\ell}_{t,c}(a)$ is $w_e(a) + \gamma$ rather than $\widehat{w}_e(a) + \frac{3}{2}\gamma$. The random variable $\widetilde{\ell}_{t,c}(a)$ can be seen as a pseudo-estimator of $\ell_{t,c}(a)$. 
%
The pseudo-estimator $\widetilde{\ell}_{t,c}(a)$ concentrates better than the actual loss estimator $\widehat{\ell}_{t,c}(a)$, but $\widetilde{\ell}_{t,c}(a)$ is not computable since $w_e(a)$ is not computable (hence the name pseudo).
%

We introduce another auxiliary sequence of random variables. For each epoch $e$, every round $t \in \Time_e$, and every context $c$, we define \[\widetilde{p}_{t,c} \propto p_{(e-1)L, c} \circ \exp \left(-\eta \sum_{t^{\prime} \in \mathcal{T}_e, t^{\prime}<t} \widetilde{\ell}_{t^{\prime} c}\right). \]
The random variable $\widetilde{p}$ can be thought of as a pseudo probability distribution computed by a counterfactual FTRL algorithm to which we feed the loss estimates $\widehat{\ell}$ up to epoch $e-1$, but feed our new pseudo-estimate $\widetilde{\ell}$ during epoch $e$. This pseudo probability distribution is crucial to our analysis. 

Now we are ready to state the sketch of the proof. We decompose the regret as follows.
\begin{align*}
\Reg(\pi) &= \E\left[\sum_{t=1}^T \ip{q_{t, c_t}- \pi_{c_t}, \ell_{t, c_t}}\right] \\
&=  \underbrace{\E\left[\sum_{t=1}^T \ip{q_{t, c_t} - p_{t, c_t}, \ell_{t, c_t}}\right]}_{\bias{1}} \\
&+ \underbrace{\E\left[\sum_{t=1}^T \ip{p_{t, c_t}- \pi_{c_t}, \ell_{t, c_t}-\widetilde{\ell}_{t, c_t}}\right]}_{\bias{2}} \\
& +\underbrace{\E\left[\sum_{t=1}^T \ip{\widetilde{p}_{t, c_t}-\pi_{c_t}, \widetilde{\ell}_{t, c_t}-\widehat{\ell}_{t, c_t}}\right]}_{\bias{3}} \\
& +\underbrace{\E\left[\sum_{t=1}^T\ip{ p_{t, c_t}-\widetilde{p}_{t, c_t}, \widetilde{\ell}_{t, c_t}-\widehat{\ell}_{t, c_t}}\right]}_{\bias{4}}\\
& +\underbrace{\E\left[\sum_{t=1}^T\ip{ p_{t, c_t}-\pi_{c_t}, \widehat{\ell}_{t, c_t}}\right]}_{\mathbf{ftrl}}.
\end{align*}

In this decomposition, the $\bias{1}$ term corresponds to the regret caused by replacing $p_t$ with $q_t$. We will argue that $q_t = p_t$ for all $t$ with high probability and thus we can safely work on $p_t$. 
%
The $\bias{2}$ term  corresponds to the bias of estimating $\ell$ using the pseudo estimator $\widetilde{\ell}$. It serves a bridge for bounding the bias of estimating $\ell$ by the real estimator $\widehat{\ell}$. The $\mathbf{ftrl}$ term corresponds to the regret of the FTRL subroutine on the fed loss. These three terms are not hard to bound.

%
The $\bias{3}$ and $\bias{4}$ terms are more subtle.
They represent a further decomposition of the bias term between $\widetilde{\ell}$ and $\widehat{\ell}$:
\[\E\left[\sum_{t=1}^T \ip{p_{t, c_t}- \pi_{c_t}, \widetilde{\ell}_{t, c_t}-\widehat{\ell}_{t, c_t}}\right] = \bias{3}+\bias{4}.\]
%
We perform this further decomposition because after the decomposition, for every time $t$ in epoch $e$, the random variables $\widetilde{\ell}_t$ and $\widehat{\ell}_t$ are independent of the random variable $\widetilde{p}_t$ conditional on the snapshot $s_{e}$ of epoch $e-2$.
%
In other words, the random variables $\widetilde{\ell}_t$ and $\widehat{\ell}_t$ are independent of the random variable $\widetilde{p}_t$ conditional on the sigma algebra $\His_{e-2}$ generated by all randomness until  epoch $e-2$.
%
This is because in epoch $e-1$, the random variables $\widetilde{\ell}_t$ and $\widehat{\ell}_t$ are only affected by randomness within round $s$ for $s \in \Time_{e-1}^{\ell}$, while the random variable $\widetilde{p}_t$  is only affected by randomness within round $s$ for $s \in \Time_{e-1}^{f}$.
These two sets $ \Time_{e-1}^{\ell}$ and $\Time_{e-1}^{f}$ are disjoint, thus the randomness within each is independent.
%
The reason  we group all timesteps into consecutive pairs of two in \Cref{alg:unknown} is exactly to ensure this conditional independence. 

Given the conditional independence, we have
\begin{align*}
    \bias{3} =&\E\left[\sum_{t=1}^T \ip{\widetilde{p}_{t, c_t}-\pi_{c_t}, \widetilde{\ell}_{t, c_t}-\widehat{\ell}_{t, c_t}}\right] \\
    =& \E\left[\sum_{t=1}^T \ip{\widetilde{p}_{t, c_t}-\pi_{c_t}, \E\left[\widetilde{\ell}_{t, c_t}-\widehat{\ell}_{t, c_t} | \His_{e-2} \right]}\right].
\end{align*}
Here we only need to bound the expected bias $\widetilde{\ell} - \widehat{\ell}$, which has a sufficiently small bound. 

The final $\bias{4}$ term can be bounded directly using concentration inequalities. That is, we can directly bound this term in high probability. The term $\bias{4}$ is a product of deviations $\left(p_{t} - \widetilde{p}_t\right) \cdot \left( \widetilde{\ell}_t - \widehat{\ell}_t\right)$, thus its concentration is tighter than other terms by a square. This square improvement allows us to bound it directly using concentration inequalities.
\section{Conclusion}
In this work, we propose a simple yet effective approach, called SMILE, for graph few-shot learning with fewer tasks. Specifically, we introduce a novel dual-level mixup strategy, including within-task and across-task mixup, for enriching the diversity of nodes within each task and the diversity of tasks. Also, we incorporate the degree-based prior information to learn expressive node embeddings. Theoretically, we prove that SMILE effectively enhances the model's generalization performance. Empirically, we conduct extensive experiments on multiple benchmarks and the results suggest that SMILE significantly outperforms other baselines, including both in-domain and cross-domain few-shot settings.
\bibliography{ICML_bib}
\bibliographystyle{icml2025}

\newpage
\appendix
\onecolumn
\subsection{Lloyd-Max Algorithm}
\label{subsec:Lloyd-Max}
For a given quantization bitwidth $B$ and an operand $\bm{X}$, the Lloyd-Max algorithm finds $2^B$ quantization levels $\{\hat{x}_i\}_{i=1}^{2^B}$ such that quantizing $\bm{X}$ by rounding each scalar in $\bm{X}$ to the nearest quantization level minimizes the quantization MSE. 

The algorithm starts with an initial guess of quantization levels and then iteratively computes quantization thresholds $\{\tau_i\}_{i=1}^{2^B-1}$ and updates quantization levels $\{\hat{x}_i\}_{i=1}^{2^B}$. Specifically, at iteration $n$, thresholds are set to the midpoints of the previous iteration's levels:
\begin{align*}
    \tau_i^{(n)}=\frac{\hat{x}_i^{(n-1)}+\hat{x}_{i+1}^{(n-1)}}2 \text{ for } i=1\ldots 2^B-1
\end{align*}
Subsequently, the quantization levels are re-computed as conditional means of the data regions defined by the new thresholds:
\begin{align*}
    \hat{x}_i^{(n)}=\mathbb{E}\left[ \bm{X} \big| \bm{X}\in [\tau_{i-1}^{(n)},\tau_i^{(n)}] \right] \text{ for } i=1\ldots 2^B
\end{align*}
where to satisfy boundary conditions we have $\tau_0=-\infty$ and $\tau_{2^B}=\infty$. The algorithm iterates the above steps until convergence.

Figure \ref{fig:lm_quant} compares the quantization levels of a $7$-bit floating point (E3M3) quantizer (left) to a $7$-bit Lloyd-Max quantizer (right) when quantizing a layer of weights from the GPT3-126M model at a per-tensor granularity. As shown, the Lloyd-Max quantizer achieves substantially lower quantization MSE. Further, Table \ref{tab:FP7_vs_LM7} shows the superior perplexity achieved by Lloyd-Max quantizers for bitwidths of $7$, $6$ and $5$. The difference between the quantizers is clear at 5 bits, where per-tensor FP quantization incurs a drastic and unacceptable increase in perplexity, while Lloyd-Max quantization incurs a much smaller increase. Nevertheless, we note that even the optimal Lloyd-Max quantizer incurs a notable ($\sim 1.5$) increase in perplexity due to the coarse granularity of quantization. 

\begin{figure}[h]
  \centering
  \includegraphics[width=0.7\linewidth]{sections/figures/LM7_FP7.pdf}
  \caption{\small Quantization levels and the corresponding quantization MSE of Floating Point (left) vs Lloyd-Max (right) Quantizers for a layer of weights in the GPT3-126M model.}
  \label{fig:lm_quant}
\end{figure}

\begin{table}[h]\scriptsize
\begin{center}
\caption{\label{tab:FP7_vs_LM7} \small Comparing perplexity (lower is better) achieved by floating point quantizers and Lloyd-Max quantizers on a GPT3-126M model for the Wikitext-103 dataset.}
\begin{tabular}{c|cc|c}
\hline
 \multirow{2}{*}{\textbf{Bitwidth}} & \multicolumn{2}{|c|}{\textbf{Floating-Point Quantizer}} & \textbf{Lloyd-Max Quantizer} \\
 & Best Format & Wikitext-103 Perplexity & Wikitext-103 Perplexity \\
\hline
7 & E3M3 & 18.32 & 18.27 \\
6 & E3M2 & 19.07 & 18.51 \\
5 & E4M0 & 43.89 & 19.71 \\
\hline
\end{tabular}
\end{center}
\end{table}

\subsection{Proof of Local Optimality of LO-BCQ}
\label{subsec:lobcq_opt_proof}
For a given block $\bm{b}_j$, the quantization MSE during LO-BCQ can be empirically evaluated as $\frac{1}{L_b}\lVert \bm{b}_j- \bm{\hat{b}}_j\rVert^2_2$ where $\bm{\hat{b}}_j$ is computed from equation (\ref{eq:clustered_quantization_definition}) as $C_{f(\bm{b}_j)}(\bm{b}_j)$. Further, for a given block cluster $\mathcal{B}_i$, we compute the quantization MSE as $\frac{1}{|\mathcal{B}_{i}|}\sum_{\bm{b} \in \mathcal{B}_{i}} \frac{1}{L_b}\lVert \bm{b}- C_i^{(n)}(\bm{b})\rVert^2_2$. Therefore, at the end of iteration $n$, we evaluate the overall quantization MSE $J^{(n)}$ for a given operand $\bm{X}$ composed of $N_c$ block clusters as:
\begin{align*}
    \label{eq:mse_iter_n}
    J^{(n)} = \frac{1}{N_c} \sum_{i=1}^{N_c} \frac{1}{|\mathcal{B}_{i}^{(n)}|}\sum_{\bm{v} \in \mathcal{B}_{i}^{(n)}} \frac{1}{L_b}\lVert \bm{b}- B_i^{(n)}(\bm{b})\rVert^2_2
\end{align*}

At the end of iteration $n$, the codebooks are updated from $\mathcal{C}^{(n-1)}$ to $\mathcal{C}^{(n)}$. However, the mapping of a given vector $\bm{b}_j$ to quantizers $\mathcal{C}^{(n)}$ remains as  $f^{(n)}(\bm{b}_j)$. At the next iteration, during the vector clustering step, $f^{(n+1)}(\bm{b}_j)$ finds new mapping of $\bm{b}_j$ to updated codebooks $\mathcal{C}^{(n)}$ such that the quantization MSE over the candidate codebooks is minimized. Therefore, we obtain the following result for $\bm{b}_j$:
\begin{align*}
\frac{1}{L_b}\lVert \bm{b}_j - C_{f^{(n+1)}(\bm{b}_j)}^{(n)}(\bm{b}_j)\rVert^2_2 \le \frac{1}{L_b}\lVert \bm{b}_j - C_{f^{(n)}(\bm{b}_j)}^{(n)}(\bm{b}_j)\rVert^2_2
\end{align*}

That is, quantizing $\bm{b}_j$ at the end of the block clustering step of iteration $n+1$ results in lower quantization MSE compared to quantizing at the end of iteration $n$. Since this is true for all $\bm{b} \in \bm{X}$, we assert the following:
\begin{equation}
\begin{split}
\label{eq:mse_ineq_1}
    \tilde{J}^{(n+1)} &= \frac{1}{N_c} \sum_{i=1}^{N_c} \frac{1}{|\mathcal{B}_{i}^{(n+1)}|}\sum_{\bm{b} \in \mathcal{B}_{i}^{(n+1)}} \frac{1}{L_b}\lVert \bm{b} - C_i^{(n)}(b)\rVert^2_2 \le J^{(n)}
\end{split}
\end{equation}
where $\tilde{J}^{(n+1)}$ is the the quantization MSE after the vector clustering step at iteration $n+1$.

Next, during the codebook update step (\ref{eq:quantizers_update}) at iteration $n+1$, the per-cluster codebooks $\mathcal{C}^{(n)}$ are updated to $\mathcal{C}^{(n+1)}$ by invoking the Lloyd-Max algorithm \citep{Lloyd}. We know that for any given value distribution, the Lloyd-Max algorithm minimizes the quantization MSE. Therefore, for a given vector cluster $\mathcal{B}_i$ we obtain the following result:

\begin{equation}
    \frac{1}{|\mathcal{B}_{i}^{(n+1)}|}\sum_{\bm{b} \in \mathcal{B}_{i}^{(n+1)}} \frac{1}{L_b}\lVert \bm{b}- C_i^{(n+1)}(\bm{b})\rVert^2_2 \le \frac{1}{|\mathcal{B}_{i}^{(n+1)}|}\sum_{\bm{b} \in \mathcal{B}_{i}^{(n+1)}} \frac{1}{L_b}\lVert \bm{b}- C_i^{(n)}(\bm{b})\rVert^2_2
\end{equation}

The above equation states that quantizing the given block cluster $\mathcal{B}_i$ after updating the associated codebook from $C_i^{(n)}$ to $C_i^{(n+1)}$ results in lower quantization MSE. Since this is true for all the block clusters, we derive the following result: 
\begin{equation}
\begin{split}
\label{eq:mse_ineq_2}
     J^{(n+1)} &= \frac{1}{N_c} \sum_{i=1}^{N_c} \frac{1}{|\mathcal{B}_{i}^{(n+1)}|}\sum_{\bm{b} \in \mathcal{B}_{i}^{(n+1)}} \frac{1}{L_b}\lVert \bm{b}- C_i^{(n+1)}(\bm{b})\rVert^2_2  \le \tilde{J}^{(n+1)}   
\end{split}
\end{equation}

Following (\ref{eq:mse_ineq_1}) and (\ref{eq:mse_ineq_2}), we find that the quantization MSE is non-increasing for each iteration, that is, $J^{(1)} \ge J^{(2)} \ge J^{(3)} \ge \ldots \ge J^{(M)}$ where $M$ is the maximum number of iterations. 
%Therefore, we can say that if the algorithm converges, then it must be that it has converged to a local minimum. 
\hfill $\blacksquare$


\begin{figure}
    \begin{center}
    \includegraphics[width=0.5\textwidth]{sections//figures/mse_vs_iter.pdf}
    \end{center}
    \caption{\small NMSE vs iterations during LO-BCQ compared to other block quantization proposals}
    \label{fig:nmse_vs_iter}
\end{figure}

Figure \ref{fig:nmse_vs_iter} shows the empirical convergence of LO-BCQ across several block lengths and number of codebooks. Also, the MSE achieved by LO-BCQ is compared to baselines such as MXFP and VSQ. As shown, LO-BCQ converges to a lower MSE than the baselines. Further, we achieve better convergence for larger number of codebooks ($N_c$) and for a smaller block length ($L_b$), both of which increase the bitwidth of BCQ (see Eq \ref{eq:bitwidth_bcq}).


\subsection{Additional Accuracy Results}
%Table \ref{tab:lobcq_config} lists the various LOBCQ configurations and their corresponding bitwidths.
\begin{table}
\setlength{\tabcolsep}{4.75pt}
\begin{center}
\caption{\label{tab:lobcq_config} Various LO-BCQ configurations and their bitwidths.}
\begin{tabular}{|c||c|c|c|c||c|c||c|} 
\hline
 & \multicolumn{4}{|c||}{$L_b=8$} & \multicolumn{2}{|c||}{$L_b=4$} & $L_b=2$ \\
 \hline
 \backslashbox{$L_A$\kern-1em}{\kern-1em$N_c$} & 2 & 4 & 8 & 16 & 2 & 4 & 2 \\
 \hline
 64 & 4.25 & 4.375 & 4.5 & 4.625 & 4.375 & 4.625 & 4.625\\
 \hline
 32 & 4.375 & 4.5 & 4.625& 4.75 & 4.5 & 4.75 & 4.75 \\
 \hline
 16 & 4.625 & 4.75& 4.875 & 5 & 4.75 & 5 & 5 \\
 \hline
\end{tabular}
\end{center}
\end{table}

%\subsection{Perplexity achieved by various LO-BCQ configurations on Wikitext-103 dataset}

\begin{table} \centering
\begin{tabular}{|c||c|c|c|c||c|c||c|} 
\hline
 $L_b \rightarrow$& \multicolumn{4}{c||}{8} & \multicolumn{2}{c||}{4} & 2\\
 \hline
 \backslashbox{$L_A$\kern-1em}{\kern-1em$N_c$} & 2 & 4 & 8 & 16 & 2 & 4 & 2  \\
 %$N_c \rightarrow$ & 2 & 4 & 8 & 16 & 2 & 4 & 2 \\
 \hline
 \hline
 \multicolumn{8}{c}{GPT3-1.3B (FP32 PPL = 9.98)} \\ 
 \hline
 \hline
 64 & 10.40 & 10.23 & 10.17 & 10.15 &  10.28 & 10.18 & 10.19 \\
 \hline
 32 & 10.25 & 10.20 & 10.15 & 10.12 &  10.23 & 10.17 & 10.17 \\
 \hline
 16 & 10.22 & 10.16 & 10.10 & 10.09 &  10.21 & 10.14 & 10.16 \\
 \hline
  \hline
 \multicolumn{8}{c}{GPT3-8B (FP32 PPL = 7.38)} \\ 
 \hline
 \hline
 64 & 7.61 & 7.52 & 7.48 &  7.47 &  7.55 &  7.49 & 7.50 \\
 \hline
 32 & 7.52 & 7.50 & 7.46 &  7.45 &  7.52 &  7.48 & 7.48  \\
 \hline
 16 & 7.51 & 7.48 & 7.44 &  7.44 &  7.51 &  7.49 & 7.47  \\
 \hline
\end{tabular}
\caption{\label{tab:ppl_gpt3_abalation} Wikitext-103 perplexity across GPT3-1.3B and 8B models.}
\end{table}

\begin{table} \centering
\begin{tabular}{|c||c|c|c|c||} 
\hline
 $L_b \rightarrow$& \multicolumn{4}{c||}{8}\\
 \hline
 \backslashbox{$L_A$\kern-1em}{\kern-1em$N_c$} & 2 & 4 & 8 & 16 \\
 %$N_c \rightarrow$ & 2 & 4 & 8 & 16 & 2 & 4 & 2 \\
 \hline
 \hline
 \multicolumn{5}{|c|}{Llama2-7B (FP32 PPL = 5.06)} \\ 
 \hline
 \hline
 64 & 5.31 & 5.26 & 5.19 & 5.18  \\
 \hline
 32 & 5.23 & 5.25 & 5.18 & 5.15  \\
 \hline
 16 & 5.23 & 5.19 & 5.16 & 5.14  \\
 \hline
 \multicolumn{5}{|c|}{Nemotron4-15B (FP32 PPL = 5.87)} \\ 
 \hline
 \hline
 64  & 6.3 & 6.20 & 6.13 & 6.08  \\
 \hline
 32  & 6.24 & 6.12 & 6.07 & 6.03  \\
 \hline
 16  & 6.12 & 6.14 & 6.04 & 6.02  \\
 \hline
 \multicolumn{5}{|c|}{Nemotron4-340B (FP32 PPL = 3.48)} \\ 
 \hline
 \hline
 64 & 3.67 & 3.62 & 3.60 & 3.59 \\
 \hline
 32 & 3.63 & 3.61 & 3.59 & 3.56 \\
 \hline
 16 & 3.61 & 3.58 & 3.57 & 3.55 \\
 \hline
\end{tabular}
\caption{\label{tab:ppl_llama7B_nemo15B} Wikitext-103 perplexity compared to FP32 baseline in Llama2-7B and Nemotron4-15B, 340B models}
\end{table}

%\subsection{Perplexity achieved by various LO-BCQ configurations on MMLU dataset}


\begin{table} \centering
\begin{tabular}{|c||c|c|c|c||c|c|c|c|} 
\hline
 $L_b \rightarrow$& \multicolumn{4}{c||}{8} & \multicolumn{4}{c||}{8}\\
 \hline
 \backslashbox{$L_A$\kern-1em}{\kern-1em$N_c$} & 2 & 4 & 8 & 16 & 2 & 4 & 8 & 16  \\
 %$N_c \rightarrow$ & 2 & 4 & 8 & 16 & 2 & 4 & 2 \\
 \hline
 \hline
 \multicolumn{5}{|c|}{Llama2-7B (FP32 Accuracy = 45.8\%)} & \multicolumn{4}{|c|}{Llama2-70B (FP32 Accuracy = 69.12\%)} \\ 
 \hline
 \hline
 64 & 43.9 & 43.4 & 43.9 & 44.9 & 68.07 & 68.27 & 68.17 & 68.75 \\
 \hline
 32 & 44.5 & 43.8 & 44.9 & 44.5 & 68.37 & 68.51 & 68.35 & 68.27  \\
 \hline
 16 & 43.9 & 42.7 & 44.9 & 45 & 68.12 & 68.77 & 68.31 & 68.59  \\
 \hline
 \hline
 \multicolumn{5}{|c|}{GPT3-22B (FP32 Accuracy = 38.75\%)} & \multicolumn{4}{|c|}{Nemotron4-15B (FP32 Accuracy = 64.3\%)} \\ 
 \hline
 \hline
 64 & 36.71 & 38.85 & 38.13 & 38.92 & 63.17 & 62.36 & 63.72 & 64.09 \\
 \hline
 32 & 37.95 & 38.69 & 39.45 & 38.34 & 64.05 & 62.30 & 63.8 & 64.33  \\
 \hline
 16 & 38.88 & 38.80 & 38.31 & 38.92 & 63.22 & 63.51 & 63.93 & 64.43  \\
 \hline
\end{tabular}
\caption{\label{tab:mmlu_abalation} Accuracy on MMLU dataset across GPT3-22B, Llama2-7B, 70B and Nemotron4-15B models.}
\end{table}


%\subsection{Perplexity achieved by various LO-BCQ configurations on LM evaluation harness}

\begin{table} \centering
\begin{tabular}{|c||c|c|c|c||c|c|c|c|} 
\hline
 $L_b \rightarrow$& \multicolumn{4}{c||}{8} & \multicolumn{4}{c||}{8}\\
 \hline
 \backslashbox{$L_A$\kern-1em}{\kern-1em$N_c$} & 2 & 4 & 8 & 16 & 2 & 4 & 8 & 16  \\
 %$N_c \rightarrow$ & 2 & 4 & 8 & 16 & 2 & 4 & 2 \\
 \hline
 \hline
 \multicolumn{5}{|c|}{Race (FP32 Accuracy = 37.51\%)} & \multicolumn{4}{|c|}{Boolq (FP32 Accuracy = 64.62\%)} \\ 
 \hline
 \hline
 64 & 36.94 & 37.13 & 36.27 & 37.13 & 63.73 & 62.26 & 63.49 & 63.36 \\
 \hline
 32 & 37.03 & 36.36 & 36.08 & 37.03 & 62.54 & 63.51 & 63.49 & 63.55  \\
 \hline
 16 & 37.03 & 37.03 & 36.46 & 37.03 & 61.1 & 63.79 & 63.58 & 63.33  \\
 \hline
 \hline
 \multicolumn{5}{|c|}{Winogrande (FP32 Accuracy = 58.01\%)} & \multicolumn{4}{|c|}{Piqa (FP32 Accuracy = 74.21\%)} \\ 
 \hline
 \hline
 64 & 58.17 & 57.22 & 57.85 & 58.33 & 73.01 & 73.07 & 73.07 & 72.80 \\
 \hline
 32 & 59.12 & 58.09 & 57.85 & 58.41 & 73.01 & 73.94 & 72.74 & 73.18  \\
 \hline
 16 & 57.93 & 58.88 & 57.93 & 58.56 & 73.94 & 72.80 & 73.01 & 73.94  \\
 \hline
\end{tabular}
\caption{\label{tab:mmlu_abalation} Accuracy on LM evaluation harness tasks on GPT3-1.3B model.}
\end{table}

\begin{table} \centering
\begin{tabular}{|c||c|c|c|c||c|c|c|c|} 
\hline
 $L_b \rightarrow$& \multicolumn{4}{c||}{8} & \multicolumn{4}{c||}{8}\\
 \hline
 \backslashbox{$L_A$\kern-1em}{\kern-1em$N_c$} & 2 & 4 & 8 & 16 & 2 & 4 & 8 & 16  \\
 %$N_c \rightarrow$ & 2 & 4 & 8 & 16 & 2 & 4 & 2 \\
 \hline
 \hline
 \multicolumn{5}{|c|}{Race (FP32 Accuracy = 41.34\%)} & \multicolumn{4}{|c|}{Boolq (FP32 Accuracy = 68.32\%)} \\ 
 \hline
 \hline
 64 & 40.48 & 40.10 & 39.43 & 39.90 & 69.20 & 68.41 & 69.45 & 68.56 \\
 \hline
 32 & 39.52 & 39.52 & 40.77 & 39.62 & 68.32 & 67.43 & 68.17 & 69.30  \\
 \hline
 16 & 39.81 & 39.71 & 39.90 & 40.38 & 68.10 & 66.33 & 69.51 & 69.42  \\
 \hline
 \hline
 \multicolumn{5}{|c|}{Winogrande (FP32 Accuracy = 67.88\%)} & \multicolumn{4}{|c|}{Piqa (FP32 Accuracy = 78.78\%)} \\ 
 \hline
 \hline
 64 & 66.85 & 66.61 & 67.72 & 67.88 & 77.31 & 77.42 & 77.75 & 77.64 \\
 \hline
 32 & 67.25 & 67.72 & 67.72 & 67.00 & 77.31 & 77.04 & 77.80 & 77.37  \\
 \hline
 16 & 68.11 & 68.90 & 67.88 & 67.48 & 77.37 & 78.13 & 78.13 & 77.69  \\
 \hline
\end{tabular}
\caption{\label{tab:mmlu_abalation} Accuracy on LM evaluation harness tasks on GPT3-8B model.}
\end{table}

\begin{table} \centering
\begin{tabular}{|c||c|c|c|c||c|c|c|c|} 
\hline
 $L_b \rightarrow$& \multicolumn{4}{c||}{8} & \multicolumn{4}{c||}{8}\\
 \hline
 \backslashbox{$L_A$\kern-1em}{\kern-1em$N_c$} & 2 & 4 & 8 & 16 & 2 & 4 & 8 & 16  \\
 %$N_c \rightarrow$ & 2 & 4 & 8 & 16 & 2 & 4 & 2 \\
 \hline
 \hline
 \multicolumn{5}{|c|}{Race (FP32 Accuracy = 40.67\%)} & \multicolumn{4}{|c|}{Boolq (FP32 Accuracy = 76.54\%)} \\ 
 \hline
 \hline
 64 & 40.48 & 40.10 & 39.43 & 39.90 & 75.41 & 75.11 & 77.09 & 75.66 \\
 \hline
 32 & 39.52 & 39.52 & 40.77 & 39.62 & 76.02 & 76.02 & 75.96 & 75.35  \\
 \hline
 16 & 39.81 & 39.71 & 39.90 & 40.38 & 75.05 & 73.82 & 75.72 & 76.09  \\
 \hline
 \hline
 \multicolumn{5}{|c|}{Winogrande (FP32 Accuracy = 70.64\%)} & \multicolumn{4}{|c|}{Piqa (FP32 Accuracy = 79.16\%)} \\ 
 \hline
 \hline
 64 & 69.14 & 70.17 & 70.17 & 70.56 & 78.24 & 79.00 & 78.62 & 78.73 \\
 \hline
 32 & 70.96 & 69.69 & 71.27 & 69.30 & 78.56 & 79.49 & 79.16 & 78.89  \\
 \hline
 16 & 71.03 & 69.53 & 69.69 & 70.40 & 78.13 & 79.16 & 79.00 & 79.00  \\
 \hline
\end{tabular}
\caption{\label{tab:mmlu_abalation} Accuracy on LM evaluation harness tasks on GPT3-22B model.}
\end{table}

\begin{table} \centering
\begin{tabular}{|c||c|c|c|c||c|c|c|c|} 
\hline
 $L_b \rightarrow$& \multicolumn{4}{c||}{8} & \multicolumn{4}{c||}{8}\\
 \hline
 \backslashbox{$L_A$\kern-1em}{\kern-1em$N_c$} & 2 & 4 & 8 & 16 & 2 & 4 & 8 & 16  \\
 %$N_c \rightarrow$ & 2 & 4 & 8 & 16 & 2 & 4 & 2 \\
 \hline
 \hline
 \multicolumn{5}{|c|}{Race (FP32 Accuracy = 44.4\%)} & \multicolumn{4}{|c|}{Boolq (FP32 Accuracy = 79.29\%)} \\ 
 \hline
 \hline
 64 & 42.49 & 42.51 & 42.58 & 43.45 & 77.58 & 77.37 & 77.43 & 78.1 \\
 \hline
 32 & 43.35 & 42.49 & 43.64 & 43.73 & 77.86 & 75.32 & 77.28 & 77.86  \\
 \hline
 16 & 44.21 & 44.21 & 43.64 & 42.97 & 78.65 & 77 & 76.94 & 77.98  \\
 \hline
 \hline
 \multicolumn{5}{|c|}{Winogrande (FP32 Accuracy = 69.38\%)} & \multicolumn{4}{|c|}{Piqa (FP32 Accuracy = 78.07\%)} \\ 
 \hline
 \hline
 64 & 68.9 & 68.43 & 69.77 & 68.19 & 77.09 & 76.82 & 77.09 & 77.86 \\
 \hline
 32 & 69.38 & 68.51 & 68.82 & 68.90 & 78.07 & 76.71 & 78.07 & 77.86  \\
 \hline
 16 & 69.53 & 67.09 & 69.38 & 68.90 & 77.37 & 77.8 & 77.91 & 77.69  \\
 \hline
\end{tabular}
\caption{\label{tab:mmlu_abalation} Accuracy on LM evaluation harness tasks on Llama2-7B model.}
\end{table}

\begin{table} \centering
\begin{tabular}{|c||c|c|c|c||c|c|c|c|} 
\hline
 $L_b \rightarrow$& \multicolumn{4}{c||}{8} & \multicolumn{4}{c||}{8}\\
 \hline
 \backslashbox{$L_A$\kern-1em}{\kern-1em$N_c$} & 2 & 4 & 8 & 16 & 2 & 4 & 8 & 16  \\
 %$N_c \rightarrow$ & 2 & 4 & 8 & 16 & 2 & 4 & 2 \\
 \hline
 \hline
 \multicolumn{5}{|c|}{Race (FP32 Accuracy = 48.8\%)} & \multicolumn{4}{|c|}{Boolq (FP32 Accuracy = 85.23\%)} \\ 
 \hline
 \hline
 64 & 49.00 & 49.00 & 49.28 & 48.71 & 82.82 & 84.28 & 84.03 & 84.25 \\
 \hline
 32 & 49.57 & 48.52 & 48.33 & 49.28 & 83.85 & 84.46 & 84.31 & 84.93  \\
 \hline
 16 & 49.85 & 49.09 & 49.28 & 48.99 & 85.11 & 84.46 & 84.61 & 83.94  \\
 \hline
 \hline
 \multicolumn{5}{|c|}{Winogrande (FP32 Accuracy = 79.95\%)} & \multicolumn{4}{|c|}{Piqa (FP32 Accuracy = 81.56\%)} \\ 
 \hline
 \hline
 64 & 78.77 & 78.45 & 78.37 & 79.16 & 81.45 & 80.69 & 81.45 & 81.5 \\
 \hline
 32 & 78.45 & 79.01 & 78.69 & 80.66 & 81.56 & 80.58 & 81.18 & 81.34  \\
 \hline
 16 & 79.95 & 79.56 & 79.79 & 79.72 & 81.28 & 81.66 & 81.28 & 80.96  \\
 \hline
\end{tabular}
\caption{\label{tab:mmlu_abalation} Accuracy on LM evaluation harness tasks on Llama2-70B model.}
\end{table}

%\section{MSE Studies}
%\textcolor{red}{TODO}


\subsection{Number Formats and Quantization Method}
\label{subsec:numFormats_quantMethod}
\subsubsection{Integer Format}
An $n$-bit signed integer (INT) is typically represented with a 2s-complement format \citep{yao2022zeroquant,xiao2023smoothquant,dai2021vsq}, where the most significant bit denotes the sign.

\subsubsection{Floating Point Format}
An $n$-bit signed floating point (FP) number $x$ comprises of a 1-bit sign ($x_{\mathrm{sign}}$), $B_m$-bit mantissa ($x_{\mathrm{mant}}$) and $B_e$-bit exponent ($x_{\mathrm{exp}}$) such that $B_m+B_e=n-1$. The associated constant exponent bias ($E_{\mathrm{bias}}$) is computed as $(2^{{B_e}-1}-1)$. We denote this format as $E_{B_e}M_{B_m}$.  

\subsubsection{Quantization Scheme}
\label{subsec:quant_method}
A quantization scheme dictates how a given unquantized tensor is converted to its quantized representation. We consider FP formats for the purpose of illustration. Given an unquantized tensor $\bm{X}$ and an FP format $E_{B_e}M_{B_m}$, we first, we compute the quantization scale factor $s_X$ that maps the maximum absolute value of $\bm{X}$ to the maximum quantization level of the $E_{B_e}M_{B_m}$ format as follows:
\begin{align}
\label{eq:sf}
    s_X = \frac{\mathrm{max}(|\bm{X}|)}{\mathrm{max}(E_{B_e}M_{B_m})}
\end{align}
In the above equation, $|\cdot|$ denotes the absolute value function.

Next, we scale $\bm{X}$ by $s_X$ and quantize it to $\hat{\bm{X}}$ by rounding it to the nearest quantization level of $E_{B_e}M_{B_m}$ as:

\begin{align}
\label{eq:tensor_quant}
    \hat{\bm{X}} = \text{round-to-nearest}\left(\frac{\bm{X}}{s_X}, E_{B_e}M_{B_m}\right)
\end{align}

We perform dynamic max-scaled quantization \citep{wu2020integer}, where the scale factor $s$ for activations is dynamically computed during runtime.

\subsection{Vector Scaled Quantization}
\begin{wrapfigure}{r}{0.35\linewidth}
  \centering
  \includegraphics[width=\linewidth]{sections/figures/vsquant.jpg}
  \caption{\small Vectorwise decomposition for per-vector scaled quantization (VSQ \citep{dai2021vsq}).}
  \label{fig:vsquant}
\end{wrapfigure}
During VSQ \citep{dai2021vsq}, the operand tensors are decomposed into 1D vectors in a hardware friendly manner as shown in Figure \ref{fig:vsquant}. Since the decomposed tensors are used as operands in matrix multiplications during inference, it is beneficial to perform this decomposition along the reduction dimension of the multiplication. The vectorwise quantization is performed similar to tensorwise quantization described in Equations \ref{eq:sf} and \ref{eq:tensor_quant}, where a scale factor $s_v$ is required for each vector $\bm{v}$ that maps the maximum absolute value of that vector to the maximum quantization level. While smaller vector lengths can lead to larger accuracy gains, the associated memory and computational overheads due to the per-vector scale factors increases. To alleviate these overheads, VSQ \citep{dai2021vsq} proposed a second level quantization of the per-vector scale factors to unsigned integers, while MX \citep{rouhani2023shared} quantizes them to integer powers of 2 (denoted as $2^{INT}$).

\subsubsection{MX Format}
The MX format proposed in \citep{rouhani2023microscaling} introduces the concept of sub-block shifting. For every two scalar elements of $b$-bits each, there is a shared exponent bit. The value of this exponent bit is determined through an empirical analysis that targets minimizing quantization MSE. We note that the FP format $E_{1}M_{b}$ is strictly better than MX from an accuracy perspective since it allocates a dedicated exponent bit to each scalar as opposed to sharing it across two scalars. Therefore, we conservatively bound the accuracy of a $b+2$-bit signed MX format with that of a $E_{1}M_{b}$ format in our comparisons. For instance, we use E1M2 format as a proxy for MX4.

\begin{figure}
    \centering
    \includegraphics[width=1\linewidth]{sections//figures/BlockFormats.pdf}
    \caption{\small Comparing LO-BCQ to MX format.}
    \label{fig:block_formats}
\end{figure}

Figure \ref{fig:block_formats} compares our $4$-bit LO-BCQ block format to MX \citep{rouhani2023microscaling}. As shown, both LO-BCQ and MX decompose a given operand tensor into block arrays and each block array into blocks. Similar to MX, we find that per-block quantization ($L_b < L_A$) leads to better accuracy due to increased flexibility. While MX achieves this through per-block $1$-bit micro-scales, we associate a dedicated codebook to each block through a per-block codebook selector. Further, MX quantizes the per-block array scale-factor to E8M0 format without per-tensor scaling. In contrast during LO-BCQ, we find that per-tensor scaling combined with quantization of per-block array scale-factor to E4M3 format results in superior inference accuracy across models. 


\end{document}