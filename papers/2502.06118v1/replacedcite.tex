\section{Related Works}
\subsection{Generative AI Meets Semantic Communications}
Goal-oriented and semantic communications have demonstrated significant potential to transform next-generation wireless networks using AI, making them more efficient, timely, and intelligent ____. 
With advancements in generative AI (GenAI), semantic communications can be enhanced by utilizing GenAI models to improve the comprehension and generation of high-fidelity communication content, a concept recently proposed as {\it generative semantic communications} ____.

GenAI has demonstrated significant potential to enhance point-to-point semantic communications across various dimensions, including source coding ____, channel coding ____, and deep joint source-channel coding (DeepJSCC) ____. Recent studies ____ have integrated DeepJSCC with non-orthogonal multiple access (NOMA), achieving impressive bandwidth savings especially in two-user scenarios. However, scaling these solutions to a massive number of devices presents challenges due to the complexity of end-to-end neural network training. Furthermore, the integration of powerful GenAI models, e.g., the potential of \gls*{mllm}s for token prediction based on multimodal context, into semantic multiple access schemes remains an under-explored area.

\subsection{Evolution of Multiple Access in 6G}

Advancements in multiple access towards 6G focus on efficient random access for numerous uncoordinated devices with sporadic activity and short data packets. Key 6G technologies like NOMA and grant-free random access (GFRA) are expected to meet these demands ____. For example, in GFRA, after receiving a beacon signal from the base station (BS), devices transmit their unique preambles over the same time-frequency resources. The BS then performs device activity detection, channel state information (CSI) estimation, and data detection. This protocol is commonly utilized in studies such as ____.
To accommodate a larger number of devices, unsourced massive access (UMA) introduces a shared preamble codebook, where messages are represented as indices of codewords within the codebook. The BS focuses on decoding the transmitted codewords from the shared codebook, without necessarily associating them to their transmitters ____. 

As devices become increasingly intelligent and computationally capable, 6G is expected to evolve toward the artificial intelligence of things (AIoT) paradigm, characterized by the deep integration of AI with massive communication protocols. For instance, the UMA protocol was specifically designed for efficient computation in ____, enabling communication-efficient distributed learning and inference. On the other hand, generative semantic communications can play a pivotal role in significantly reducing the latency in massive communication systems, serving as a key motivation for our work.

\begin{figure*}[t]
     \centering
     \includegraphics[width = 1.8\columnwidth,keepaspectratio]{SemMA_SystemModel_NEW4.pdf}
   %  {Fig1-eps-converted-to.pdf}
     \captionsetup{font={footnotesize, color = {black}}, singlelinecheck = off, justification = raggedright,name={Fig.},labelsep=period}
     \caption{The proposed Token-Domain Multiple Access (ToDMA) framework.}
     \label{fig2}
     % \vspace{-7mm}
     \vspace{-5mm}
\end{figure*}