\section{Semantic Analysis of Phishing emails}
\label{sec:semantic}

Phishing attacks leverage sophisticated social engineering techniques to deceive recipients into disseminating sensitive information or performing actions compromising security. A critical aspect of enhancing phishing detection mechanisms involves
understanding and identifying the semantic features commonly employed in phishing emails. This section details the comprehensive collection of semantic features used and the instrumentation activities that guide the LISA component of Cyri in recognizing them into email text.

\subsection{Collection of Phishing Semantic Features}

This activity aims to create a robust dataset of phishing semantic features and emails containing them
that can inform the development of more effective detection algorithms and improve the capabilities of LLMs in identifying phishing
emails and recognizing the presence of these features in the text. The semantic phishing features collected in our dataset derive from two primary activities:

\begin{itemize}
    \item Literature-Identified Features review: it aims at collecting semantic features previously recognized and documented in academic and professional cybersecurity literature.
    \item Methodology-Extracted Features: it aims at extracting novel semantic features through a systematic extraction using an automatic text analyzer (i.e., ChatGPT-4) to each element of a comprehensive email phishing dataset created specifically for this purpose.
\end{itemize}

Table~\ref{tab:semantic-features} shows the results of these two activities, reporting the list of all the semantic features collected along with their corresponding source.

\begin{table}[ht]
\centering % This centers the table
\caption{Overview of Cyri Semantic Features} % Title of the table
\label{tab:semantic-features} % Label for referencing
\begin{tabular}{p{5.5cm}p{1.5cm}} % Two columns, both left-aligned
\toprule % Top rule line
\textbf{Semantic Feature} & \textbf{Source} \\ % Column headings
\midrule % Middle rule line
Authority &  \cite{b8, b20}\\
Impersonation of Trusted Entities &  Extracted\\ 
Instant Gratification &  \cite{b10}\\
Exclusivity &  Extracted \\
Undesirable Consequences & \cite{b10}\\ 
Urgency (Scarcity) &  \cite{b20, b37}\\
Call to Action &  Extracted\\
False Dilemma &  Extracted\\
Assurance of Legitimacy & Extracted \\
Assurance of Security &  Extracted\\
Confidentiality Claims &  Extracted\\
Unsolicited Requests for Personal Information & Extracted \\
Appeal to Empathy/Altruism & \cite{b10} \\
Appeal to Values &  Extracted\\
Curiosity/Vagueness/Mystery & Extracted \\
Sense of Surprise/Confusion & Extracted \\
Reciprocation &  \cite{b20, b37}\\
Unity/Inclusivity/Sense of Community & \cite{b20} \\
Reinforcement of Positive Behavior &  Extracted\\
Appeal to Desires & Extracted \\
Motivational Language &  Extracted\\
Social Validation/Social Proof &  \cite{b20, b37} \\
\bottomrule % Bottom rule line
\end{tabular}
\end{table}

To identify and compile new semantic features, we first created a curated dataset of 300 phishing emails labeled by two experts in phishing analysis. This dataset was carefully assembled to include a
diverse range of phishing strategies and tactics. The sources from which  these emails were collected are:

\begin{itemize}
    \item Human-Generated Phishing Emails: Selected from the most recent ``Nazario'' and ``Nigerian Fraud'' collections~\cite{b33}, which are renowned repositories of real-world phishing emails that exhibit a variety of social engineering techniques;
    \item LLM-Generated Phishing Emails: collected by Greco et al.~\cite{b24}, which utilize advanced language models to generate realistic phishing emails that mimic human writing styles.
\end{itemize}

The dataset ensured comprehensive coverage of common and emerging phishing tactics by incorporating both human-generated and LLM-generated phishing emails.
Collecting the semantic features involved a meticulous analysis of each phishing email in the curated dataset being supported by experts and an automatic text analyzer (i.e., ChatGPT-4). Each email was input into the model with a carefully designed prompt that requested an in-depth examination of the email’s content, specifically focusing on the likelihood of it being a phishing attempt, the persuasion techniques employed, red flags, green flags, and potential countermeasures.
\\
From these responses, the identified persuasion techniques and red flags were first revised by experts and then documented. A validation process was undertaken to assess the significance and applicability of features not previously identified explicitly in the reviewed literature. This involved evaluating the consistency of these features across different phishing emails (validating their significance) and their effectiveness in deceiving recipients (validating their threat behavior). The validated features were then incorporated into the collection of semantic features, enhancing its coverage and utility. 

%\subsection{Descriptions of Cyri Semantic Features}
The semantic dataset provided to the Cyri LLM model comprises the feature names, an extensive description, and various examples. Table~\ref{tab:semantic-features-description} concisely describes each semantic feature we have identified.
\begin{table*}[ht]
\centering
\caption{Cyri Semantic Features Description}
\label{tab:semantic-features-description}
\begin{tabular}{|p{4cm}|p{12cm}|} % Two columns with specified width
\hline
\textbf{Semantic Feature} & \textbf{Description} \\ % Column headings

\hline
Authority & Impersonating authority figures to pressure recipients into complying with requests \\ [0.4em]
\hline
Impersonation of Trusted Entities & Mimicking trusted organizations deceives recipients into believing the email is genuine \\ [0.4em]
\hline
Instant Gratification & Offering tempting rewards prompts impulsive actions, exploiting the desire for quick benefits \\ [0.4em]
\hline
Exclusivity & Making recipients feel part of a select group increases compliance to avoid missing out on exclusive opportunities \\ [0.4em]
\hline
Undesirable Consequences & Threatening negative outcomes (e.g., account suspension) induces fear-driven responses without verification \\[0.4em]
\hline
Urgency (Scarcity) & Creating a sense of urgency forces recipients to act quickly, bypassing critical examination \\ [0.4em]
\hline
Call to Action & Clearly directing the recipient to perform a specific task (e.g., ``Click on the button below to verify your account'') \\[0.4em]
\hline
False Dilemma & Presenting only extreme choices pushes recipients toward the attacker’s desired course of action \\ [0.4em]
\hline
Assurance of Legitimacy & Convincing language and claims of authenticity are used to build trust and reduce suspicion \\ [0.4em]
\hline
Assurance of Security & Highlighting privacy and security reassures recipients \\ [0.4em]
\hline
Confidentiality Claims & Emphasizing the confidential nature of the information makes recipients feel they need to act without seeking advice or verification from others \\ [0.4em]
\hline
Unsolicited Requests for Personal Information & Requests for personal or financial data without prior authorization or legitimate justification \\ [0.4em]
\hline
Appeal to Empathy/Altruism & Exploiting the recipient's desire to help others or fulfill moral obligations \\ [0.4em]
\hline
Appeal to Values & Aligning with the recipient’s values builds trust and increases compliance with the attacker's requests \\ [0.4em]
\hline
Curiosity/Vagueness/Mystery & Vague or intriguing details induce recipients into taking action to satisfy their curiosity \\ [0.4em]
\hline
Sense of Surprise/Confusion & Unexpected scenarios create confusion, leading to unverified actions from recipients \\ [0.4em]
\hline
Reciprocation & Offering a benefit or favor creates a sense of obligation, leading recipients to fulfill follow-up requests \\ [0.4em]
\hline
Unity/Inclusivity/Sense of Community & Encouraging a sense of belonging or shared purpose motivates recipients to act in line with community goals \\ [0.4em]
\hline
Reinforcement of Positive Behavior & Praising the recipient for good behavior, and offering a reward reduces suspicion and increases engagement \\ [0.4em]
\hline
Appeal to Desires & Targeting personal goals or aspirations increases the chances of recipients ignoring warning signs \\ [0.4em]
\hline
Motivational Language & Evoking strong emotional responses, typically centered around desires for success, wealth, or security \\[0.4em]
\hline
Social Validation/Social Proof & Highlighting that others have taken the same action creates a sense of trust \\ [0.4em]
\hline
\end{tabular}
\end{table*}



\subsection{LISA: LLM-based Interactive Semantic Feature Analyzer}
\label{sec:lisa}

Traditional detection methods often rely on cloud-based services, which may not be suitable due to privacy concerns and dependence on external infrastructure. Deploying large LLMs raises privacy and data security concerns, as it requires sending sensitive emails to third-party servers. Users and organizations may be reluctant to adopt a system that necessitates sharing sensitive email content with external entities. 

To address these challenges, we developed a Python background process that performs in-depth phishing analysis and user query processing using a locally hosted LLM, specifically the Llama 3.1 8B model.
\\
We chose the Llama 3.1 8B model for its efficient reasoning capabilities and ability to handle contexts of up to 128,000 tokens, which is crucial when analyzing lengthy or complex emails. Additionally, this version is optimized for local deployment, balancing performance with resource demands, making it ideal for running on local machines without relying on cloud services.
\\
However, smaller models may not match the language comprehension of larger ones, making them more reliant on well-designed comprehensive prompts. For this reason, we defined an extensive prompt containing a large set of semantic social engineering techniques to improve the model's ability to detect diverse phishing tactics.

The LISA component performs two primary functions: analyzing incoming emails and handling user queries based on the analysis results during the conversation with the user. Each function is defined by a different prompt.

\subsubsection{Email Analysis Prompt}
\label{sed:eap}

The Email Analysis Prompt is a carefully constructed set of instructions designed to guide the LLM in performing a thorough analysis of an email to determine whether it is phishing or safe. The prompt employs several prompt engineering techniques to ensure that the LLM produces accurate, consistent, and user-friendly outputs. Due to the complexity and length of the prompt, we followed a Chain-of-Thought approach to make it more effective. Moreover, dissecting the prompt into its individual components ensures a thorough understanding of each aspect:\\

\noindent \textbf{1. Role Assignment}: ``\textit{You are an email phishing detector and analyzer. Your task is to identify whether an email is phishing or safe, explain why, and provide a detailed explanation.}''

The prompt begins by explicitly defining the LLM’s role as an ``email phishing detector and analyzer''. This sets the context for the model, focusing its capabilities on a specific task. By assigning a clear role, the LLM becomes ready to approach the subsequent instructions with the appropriate mindset.\\

\noindent \textbf{2. Presentation of the Email Content}: ``\textit{I want you to analyze the following email which could be phishing or safe: \{email\} I want you to tell me if this email is safe or phishing.}''

The prompt introduces the subject and the body of the email to be analyzed. Directly instructing the model to determine if the email is safe or phishing sets a clear objective.\\

\noindent \textbf{3. Base Reasoning Before Feature Consideration}: ``\textit{Use your base reasoning first to identify if the email is safe or phishing before considering the specified features.}''

The prompt instructs the LLM to use its inherent reasoning capabilities before relying on predefined features. This ensures that the model’s general understanding and language comprehension are utilized initially, potentially capturing nuances that this particular feature-based analysis might miss. \\

\noindent \textbf{4. Additional Information for Analysis and Guiding Questions:}: ``\textit{Here is additional information regarding the email for your analysis: \\
        1: Sender Information: \{sender\_email\} \\
        2: Google Safe Browsing API Result: \{google\_safe\_browsing\_output\}.\\
        3: AbuseIPDB Result: \{abuse\_ipdb\_output\}. \\
        - Is the sender domain or any URL found in the email reported as unsafe?\\
        - Identify if there is any impersonation of a well-known brand by comparing the sender’s email address with the claimed organization in the email content. If spoofing is detected, explain the inconsistencies. For example, if the email claims to be from 'Amazon' but the domain is not related to Amazon, highlight the inconsistency.\\
        Interpret the Google Safe Browsing API results: If threats are found, include the details. If no threats are found, note that.
        Interpret the AbuseIPDB results: If the domain is flagged as malicious, include the confidence score. If the domain is not flagged, note that as well. **Specify whether the domain refers to the sender or a link present in the email.** \\
        The sender's email address (\{sender\_email\}) is \{isSafeOutput\}.
        }'' \\
\indent The prompt provides external data such as the sender’s email, and results from security APIs enrich the context. These questions and instructions direct the LLM’s attention to specific aspects of the email, ensuring a comprehensive analysis. The LLM is instructed to determine whether the sender's domain or any URLs included in the email are reported as unsafe by the external APIs. The LLM is asked to compare the sender's email address with the organization mentioned in the email content to identify any impersonation. If spoofing is detected, for instance, the email claims to be from a reputable company like ``Amazon'' but the sender's domain does not match Amazon's official domain, the LLM should highlight these inconsistencies.
The variable \{isSafeOutput\} is set to indicate that the sender is ``present in the recipient's contact list and is trusted by the recipient'' or ``not present in the recipient's contact list''. This ensures that the model considers the trust relationship between the sender and the recipient. If the sender is recognized and trusted (i.e., in the contact list), the model lowers the phishing risk assessment for that email. \\

\noindent \textbf{5. Definition of Phishing/Safe Emails and Examples}: ``\textit{Here's a clear distinction for your analysis: \\ \\**Phishing Email**: Phishing emails are malicious attempts to deceive recipients into providing sensitive information or performing harmful actions.
            \\
            **Safe Email**: Safe emails are legitimate communications which typically have the following characteristics: Clear and concise language; Recognizable Sender Information; Content is relevant to the recipient's context (e.g., work-related updates, newsletters, transaction confirmations); Safe Links and Attachments. It includes, but is not limited to: routine communications like meeting requests, project updates, or a legitimate promotional email (Marketing email) from a company or organization offering products or services and it may contain offers, discounts, or promotional content.
            \\ \\
            I will provide examples of safe and phishing emails.\\
             This is a safe email:\\ \\
            \{example\_safe1\}.\\ \\
             This is a safe email:\\ \\ 
            \{example\_safe2\}.\\ \\ 
             This is a safe email:\\ \\
            \{example\_safe3\}.\\ \\
             This is a phishing email:\\ \\
            \{example\_phishing\}.
            }''
            
\indent Clear definitions and examples of phishing and safe emails are provided to help the model distinguish accurately between them for its classification process. We have added more examples of safe emails to improve the model’s ability to correctly identify safe emails and reduce false positives. \\
            
\noindent \textbf{6. Output Format Specification}: ``\textit{In the first line of the output, I want you to always respond with 'This email is [Likelihood Category] phishing ([percentage]\%)' or 'This email is [Likelihood Category] safe ([percentage]\%)' where you combine whether the email is phishing or safe with the likelihood description. \\ \\  Use these thresholds to categorize the likelihood of phishing: \\ \\
          - $0\% < x < 20\%$: Unlikely to be phishing \\ 
          - $20\% < y < 60\%$: Possibly phishing \\
          - $60\% < z < 90\%$: Likely phishing \\
          - $u > 90\%$: Almost certainly phishing \\ \\
    Also, categorize the likelihood of the email being safe:\\ \\
          - $0\% < x < 20\%$: Unlikely to be safe \\
          - $20\% < y < 60\%$: Possibly safe \\
          - $60\% < z < 90\%$: Likely safe \\
          - $u > 90\%$: Almost certainly safe}''

The prompt specifies the exact format for the output first line which will be composed by a percentage of the email being safe or phishing. By providing thresholds for likelihood categories, it ensures consistency in the model’s assessments and facilitates quantifiable evaluations. \\

\noindent \textbf{7. Feature Identification and Analysis}: ``\textit{You have to find the following features: \{features\}}''

With this step of the prompt we pass to the model the entire Cyri dataset of phishing semantic features composed by the features name, an extensive description and various examples to allow the model to perform a comprehensive analysis. \\

\noindent \textbf{8. Exact Output Format Instructions}: ``\textit{I want the output EXACTLY like this: \\  \\
- 'This email is [Likelihood Category] phishing ([percentage]\%)' or 'This email is [Likelihood Category] safe ([percentage]\%)'\\ \\
- Detailed Explanation: Provide a thorough explanation suitable for non-experts of why this email is phishing or safe. Clearly state your base reasoning for the classification, if spoofing is detected and if the sender is in the contact list or not (and how this impacts your assessment). Include references to specific elements of the email, the features of the email, the results from the Google Safe Browsing API, and the AbuseIPDB check, making sure to address how each contributes to your final assessment. \\ \\
- 'List of features found': [feature1; feature2; ...] **only the features present in the list below** for phishing emails. If the email is safe, define characteristics that make it safe (do not include any of the features present in the list below if the email is safe). \\ \\
- 'Analysis': \textless name of the feature \textgreater: '\textless specific part of the email \textgreater'. \textless explanation of why this part is linked to the feature \textgreater. **Only elements contained in 'List of features found' must be included**. \\ \\
- Countermeasures: where you offer practical recommendations on how the recipient should handle this email. These recommendations should be based on the identified risks and features, guiding the recipient on what actions to take next (e.g., verifying the sender, avoiding clicking on links, reporting the email as phishing, etc.).
}'' \\
\indent The prompt provides an exact template for the output, reducing variability and ensuring that all necessary components are included. Having a structured content analysis allows us to enhance the user interface design of the Cyri VAC component since it is possible to personalize the style of every section of the LLM analysis.\\

\noindent \textbf{9. Communication Style Guidelines}: ``\textit{Ensure the explanation is written in a conversational tone that directly addresses the recipient, making the analysis feel personalized. \\
Speak directly to the recipient using 'you' and 'your' when explaining why the email might be phishing or safe. \\
Provide clear, user-friendly explanations that are easy for non-experts to understand, directly addressing the recipient.}'' \\
\indent These instructions shape the tone and accessibility of the output, ensuring that it is appropriate for users without technical expertise.\\

\noindent \textbf{10. Feature Names and Weights}: ``\textit{Remember to use the exact names of the features listed below: \\ \\ \{list\_features\_names\} \\ \\ Weights are assigned to each feature indicating their importance in the classification: \\ \\ Authority, Impersonation of Trusted Entities: 0.6;
    Instant Gratification (False promise of reward): 0.9;
    Exclusivity: 0.8;
    Undesirable Consequences: 0.9;
    Urgency (Scarcity): 0.9;
    Call to Action: 0.9;
    False Dilemma: 0.8;
    Assurance of Legitimacy: 0.1;
    Assurance of Security: 0.3;
    Confidentiality Claims: 0.2;
    Unsolicited Requests for Personal Information/Financial Transactions: 0.9;
    Appeal to Empathy/Altruism: 0.4;
    Appeal to Values: 0.3;
    Curiosity/Vagueness/Mystery: 0.3;
    Sense of Surprise/Confusion: 0.3;
    Reciprocation: 0.3;
    Unity/Inclusivity/Sense of Community: 0.3;
    Reinforcement of Positive Behavior: 0.2;
    Appeal to Desires: 0.3;
    Motivational Language: 0.5;
    Social Validation/Social Proof: 0.5;
}''\\
\indent By specifying exact feature names, the prompt ensures consistency in terminology. Assigning weights to phishing features reflects their significance in identifying malicious emails. The weighting system guides the LLM’s reasoning process, emphasizing critical indicators.
Features assigned higher weights (e.g., 0.9) are considered strong indicators of phishing: Urgency (Scarcity); Undesirable Consequences; Unsolicited Requests. Features with moderate weights (e.g., 0.5 to 0.8) contribute significantly but may require the presence of additional indicators. Finally, features assigned lower weights (e.g., 0.1 to 0.4) may not strongly indicate phishing independently, but they can contribute to the overall assessment when combined with other indicators.


\subsubsection{Conversation Prompt}
The Conversation Prompt is designed to enable LISA to engage interactively with the user, providing detailed and context-aware responses to user queries based on prior email analysis. The prompt leverages previous interactions and analysis results to maintain continuity and relevance:
``\textit{You are an AI trained to analyze emails and interact with users to clarify and explain issues related to email security in simple terms. Use the analysis provided and the conversation history to inform your responses.\\
Given the user's query: (\{last\_user\_query\}), please provide a detailed and specific response that can help the user understand the steps to improve email security. For context I will give you also the initial instructions given to the model about how to analyze the email: \\ \\ \{initial\_prompt\} \\ \\ Here you can find the detailed analysis of the email generated by the model: \\  \{analysis\} \\ \\ All past interactions (questions and AI responses) related to this email analysis: \{conversation\_history\} \\ \\ Output only the response to this query: \{last\_user\_query\}
}''