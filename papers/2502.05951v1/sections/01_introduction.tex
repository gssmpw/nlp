\section{Introduction}
\label{sec:intro}
Among the various forms of cybercrime, phishing stands out as one of the most prevalent and harmful in modern time~\cite{b5}. According to the Anti-Phishing Working Group, over 1.2 million phishing attacks were recorded in the second quarter of 2023 alone~\cite{b6}. Every day, an estimated 3.4 billion phishing emails are sent by cybercriminals, amounting to over one trillion emails per year~\cite{b7}. Phishing serves as the entry point for 91\% of all cyber attacks and is involved in 36\% of all data breaches, making it the most common cause of such incidents. \\
Social engineering is at the core of phishing attacks, which exploits inherent human tendencies to trust and respond to certain stimuli~\cite{b1, b3, b19}. Attackers use psychological manipulation to deceive individuals into actions compromising security, such as clicking on malicious links or providing confidential information such as login credentials, financial details, or personal data~\cite{b2, b21}. This manipulation can have serious consequences, leading to identity theft, financial loss, and compromised personal or organizational security. \\
Human error is a significant factor in the success of phishing attacks, contributing to 95\% of successful cybersecurity breaches~\cite{b8}. Factors such as lack of awareness, inadequate training, stress, fatigue, and cognitive overload can impair an individual's ability to recognize and respond appropriately to phishing attempts~\cite{b9, b10, b17}. Attackers exploit these vulnerabilities by crafting messages that capitalize on distraction, curiosity, or the tendency to comply with authority figures~\cite{b4, b11, b20, b37}. This underlines the importance of implementing automated detection systems, which act as a first line of defense against human weaknesses. On the other hand, many of these solutions focus more on the technical identification of phishing without focusing on providing the human user explanations (apart from shallow ones or very technical ones, excluding non-technology expert people who represent the majority of targets) they can relate with, understand, and eventually use this information for improving awareness.\\
To better support non-expert human users in protecting against these threats, in this work, we propose Cyri, an AI-based solution designed to empower users in detecting, understanding, and responding to phishing emails. Cyri leverages Large Language Models (LLMs) to extract semantic features from email text, identifying subtle cues and psychological manipulation tactics that traditional detection systems might overlook or not target at all. By focusing on the meaning and context of the messages, Cyri provides users with an accurate assessment of potential threats at sentence-level granularity. Furthermore, users can clarify any doubts about the analysis by engaging in a conversation with Cyri, allowing them to gain deeper insights into why an email was flagged as phishing or safe. Cyri addresses the primary limitations of previous phishing detection solutions by integrating a user-centered interface that effectively communicates risks mitigating user desensitization from purely technical-based detectors~\cite{b22, b23}, while safeguarding user privacy through local data processing. These enhancements are crucial for improving phishing detection systems' overall effectiveness and acceptance in real-world applications.

Summarizing, the main contributions of this work are:
\begin{itemize}
    \item Instrumentation of local LLM to classify semantic features of email text linked to phishing attacks;
    \item A system, in the form of a plugin and web-based interface, which allows to directly connect Cyri to an existing email account without disturbing classic user experience while exploiting conversational capabilities through speech and interactive analysis through usable visual representations;
    \item Deep evaluation activities showing very good accuracy from the semantic features classifier and a user study with 10 participants, equally split into expert and non-expert, which tested Cyri directly on their email accounts and reported on the efficacy and usability of the proposed solution.
\end{itemize}


