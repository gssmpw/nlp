\section{Discussion}
\label{sec:discussion}
%include una discussione più ampia dei risultati pottentuti in termini di loro validità, eventuali limitazioni, ed opportunità di ricerca abilitate
While Cyri enhances phishing detection, management, and understanding from human users, its current implementation also presents a set of limitations that offer avenues for further improvement.
Improvements in detecting low-accuracy semantic features could be achieved by fine-tuning activities leveraging the produced phishing email dataset. These features, while less critical than others, still contribute to the overall understanding of phishing tactics. This solution would not be in substitution but would complement the current Chain-of-thought approach. Retrieval Augmented Generation can even be exploited, using currently detected emails as additional context for more accurate detection.
Another limitation we foresee is the need for a longitudinal study with users that lasts longer and collects usage data on a higher quantity of tested emails and in real-pressure conditions. We are planning this activity in the near future.

%A well-structured fine-tuning process could enhance the model’s ability to identify these less-detected features, improving the model’s comprehensiveness and educational value. Fine-tuning would involve training the model further on examples specifically designed to highlight these features, thereby increasing its sensitivity to a broader range of phishing strategies.

As interesting future possibilities enabled by this research, we foresee integrating Cyri into existing mobile email clients, which would enhance accessibility and provide real-time phishing detection and education on the devices most commonly used for email communication. Limits and possibilities in this scenario could be provided by quantized versions of small LLMs capable of being run on smartphones with similar accuracy to 8 billion models. The use of information distillation techniques with a teacher-student approach using LISA as the teacher model may be beneficial for this effort.
%Another promising direction is the integration of speech-to-text capabilities into Cyri. This functionality would make Cyri more accessible facilitating individuals who prefer voice interaction.

%This project laid the foundation for these future developments, demonstratingì the efficacy of combining advanced AI technologies with user-centered design. By addressing current limitations and exploring new frontiers in phishing detection, subsequent research can build upon this work to create even more robust and comprehensive cybersecurity solutions.
