\section{User Evaluation}
\label{sec:eval}
%include metodo, setup, ipotesi, risultati e grafici per i test con le due coorti di utenti
To assess the effectiveness and usability of Cyri as a tool for phishing detection and management from a human user, a user study was conducted involving ten participants with varying levels of expertise in computer security. This section details the methodology of the user study, the setup, and discusses the findings.

\subsection{Experiment Setup}
\label{sec:setup}

The study involved 10 participants, split equally between computer security
experts (meaning having at least two years of expertise and being knowledgeable of phishing tactics and techniques) and non-experts (meaning not being knowledgeable of phishing tactics and techniques but capable of using an email account).\\
The study duration for each participant was 60 minutes, split into 15 minutes of initial explanation on what are the most important features of Cyri and how to install it. This first step was then followed by two main tasks:
\begin{itemize}
    \item Controlled Email Identification Task: Participants were put in front of a preconfigured installation of Cyri and received five emails, four safe emails, and one phishing email sent by us. They were instructed to review these emails with Cyri and identify the phishing emails among them and the motivating factors for their final decision.
    This test was used both to let the participants gain confidence with Cyri usage and interface and as a controlled experiment where to evaluate how users interpreted and used the different results and functionalities Cyri exposes in a controlled situation equals for all of them. This step lasted, on average, from 10 to 15 minutes.
    \item Exploration with Personal Emails: Participants were then tasked to use Cyri to analyze their inbox emails from one personal account, such as those in their spam folder, unopened ones, or newly received messages. This allowed them to interact with the application in a context familiar to them and to assess its usefulness beyond the controlled task of provided emails, resulting in a more personal experience capable of letting them assess the degree of support they received from Cyri. This task lasted, on average, 25 minutes.
\end{itemize}
After completing the second task, participants were asked to compile a survey
comprising several questions aimed at evaluating Cyri’s effectiveness in assisting users in identifying phishing emails, usability and intuitiveness of the application interface, impact on users' understanding of phishing tactics, the likelihood of continued use, and preference for platform availability.
In particular, the questions proposed to the participants were the following:

\begin{enumerate}
    \item Are you a computer security expert? (Yes or No)
    \item How confident are you in identifying phishing emails without assistance?  (Scale 1 to 5)
    \item How useful was Cyri in helping you identify the phishing email?  (Scale 1 to 5)
    \item Did Cyri provide information that you wouldn't have noticed on your own? (Yes or No)
    \item How would you rate the overall usability of Cyri? (Scale 1 to 5)
    \item How intuitive did you find the Cyri interface?  (Scale 1 to 5)
    \item How much do you think using Cyri would improve your understanding of phishing tactics? (Scale 1 to 5)
    \item Would you use Cyri regularly as part of your email routine?  (Yes or No)
    \item Would you prefer if Cyri was available on your mobile phone instead of your computer?  (Yes or No)
\end{enumerate}

A final free text form allows the insertion of open comments and suggestions. Overall 5 minutes were dedicated on average to this activity.

\subsection{Results}
\label{sec:userresults}

We analyzed the survey results by splitting participants into their expertise level into two groups: Figure~\ref{fig:secexp} reports results for computer security experts while Figure~\ref{fig:nonsecexp} reports them for non-expert users. This distinction allowed us to understand how Cyri is perceived by users with different levels of expertise.

\begin{figure}[htbp]
  \centering
  \includegraphics[width=0.45\textwidth]{figures/SecurityExpertsBarchart.PNG}
  \caption{Security Experts Average Scores}
  \label{fig:secexp}
\end{figure}

\begin{figure}[htbp]
  \centering
  \includegraphics[width=0.45\textwidth]{figures/NonSecurityExpertsBarchart.PNG}
  \caption{Non-Security Experts Average Scores}
  \label{fig:nonsecexp}
\end{figure}

Cyri has been declared to be highly beneficial by non-expert participants, reporting generally low confidence in their ability to identify phishing emails without Cyri's assistance (Q2). All non-expert participants affirmed that Cyri provided information they would not have noticed on their own (Q4). This suggests that Cyri effectively highlights phishing indicators that might be overlooked, adding significant value in assisting them in identifying potential threats. 
Furthermore, non-experts provided very high ratings for both the usability (Q5) and intuitiveness (Q6) of Cyri. These results indicate that they found the application user-friendly and accessible. Non-experts believed using Cyri would significantly improve their understanding of phishing tactics (Q7), underscoring the application’s educational value.

Expert participants, even if they declared an expected good capability of identification and management of the phishing email with and without Cyri support (Q2), acknowledged that Cyri can enhance analysis capabilities by providing an additional layer of information (Q3). Interestingly, all expert participants also affirmed that Cyri provided information they would not have noticed on their own (Q4). This indicates that Cyri can uncover subtle phishing indicators and offer insights that even experienced users might overlook. Experts rated the overall usability (Q5) and intuitiveness (Q6) of Cyri highly, similar to non-experts, suggesting that the application is well-designed for users across different expertise levels. Moreover, experts believed that using Cyri could further improve their understanding of phishing tactics (Q7).
All participants expressed their willingness to use Cyri regularly as part of their email routine (Q8) and showed a clear preference for having it available also on their mobile devices (Q9).