\section{Cyri Architectural Design}
\label{sec:design}
Cyri represents an innovative AI-powered conversational assistant designed to
help users detect and analyze phishing attacks within email communications. By leveraging a refined Large Language Model (LLM) through prompt
engineering and Chain of Thought techniques, Cyri provides users with detailed explanations of the suspicious features that could make an email potentially malicious, as well as the necessary countermeasures. Cyri’s architecture is composed of three main components:
\begin{enumerate}
\item LLM-based Interactive Semantic Analyzer component (LISA): it performs in-depth email analysis using a local Large Language Model (LLM) through APIs.
\item E-mail client plugin: it captures incoming emails and communicates with the
Electron application. A demonstrator is implemented for Thunderbird since It is an open-source email client that offers extensive customization capabilities through its support for add-ons~\cite{b32}.
\item Visual and Audio Conversational interface (VAC): it serves as the user interface, manages data storage, and allows a non-expert user to analyze the classification of e-mails, the main semantic reasons, and inquire more on it in both interactive visual and audio means. It is implemented as an Electron web application for generality and usability.
\end{enumerate}

Cyri continuously monitors incoming emails through the email client plugin. When a new, unseen email arrives, the plugin extracts essential data such as the sender's information, subject, body content, a flag indicating whether the sender is in the user's contacts, the message ID, and the timestamp. This data is then transmitted to both the VAC interface application and to the LISA component to perform an in-depth analysis of the email (locally hosted), specifically using the Meta-Llama-3.1-8B-Instruct model~\cite{b29}. LISA evaluates the email for semantic features such as urgency, authority, instant gratification, and others, all collected from the literature or extracted by LISA itself. LISA is helped by a sub-component for links checking that, using external APIs, specifically Google Safe Browsing~\cite{b30} and AbuseIPDB~\cite{b31}, enhance detection capabilities by checking only links and domains against known malicious entities and provides additional context to the semantic analysis.\\
Upon completion of the analysis, the results are stored in the VAC application and sent to the e-mail client plugin for e-mail text tagging and classification as ``Phishing'' or ``Safe'' (see Figure~\ref{fig:thunderbirdexample}). Finally, the user is presented with this information in the VAC interface and can explore it as explanations and converse with Cyri with a mix of visual cues and audio.

\begin{figure}[htbp]
  \centering
  \includegraphics[width=0.48\textwidth]{figures/ThunderbirdExample.PNG}
  \caption{Cyri email Plugin Example using the Thunderbird email client}
  \label{fig:thunderbirdexample}
\end{figure}

Figure~\ref{fig:EmailAnalysisArchitecture} illustrates the process by which Cyri analyzes the semantically tagged email in the VAC interface.

\begin{figure}[htbp]
  \centering
  \includegraphics[width=0.45\textwidth]{figures/EmailAnalysisArchitecture.pdf}
  \caption{Cyri Architecture and Data Flow for Email Analysis}
  \label{fig:EmailAnalysisArchitecture}
\end{figure}

The user can monitor newly tagged emails, interact with the detailed analysis through visual means, and issue further queries. User queries are processed interactively by the LISA component, which generates responses based on the conversation history with the user and initial semantic analysis of the e-mails, taking into account the user's inputs and questions. The whole process is visible in Figure~\ref{fig:ConversationArchitecture}.

%The diagram below  represents the interaction between the user and Cyri during the query process.

\begin{figure}[htbp]
  \centering
  \includegraphics[width=0.45\textwidth]{figures/ConversationArchitecture.pdf}
  \caption{User Interaction and Query Processing Flow}
  \label{fig:ConversationArchitecture}
\end{figure}

%The Thunderbird plugin communicates with the Electron application using HTTP POST and GET requests. This method is employed to send email data and to poll for analysis results, providing a reliable and straightforward communication channel.

%The Electron application and the Python background LLM processing component interact via the CyriShared folder. This shared storage serves as a repository for email data, analysis results, and user queries and responses. A strict file naming convention is followed to ensure proper identification and handling of files.

Finally, security and privacy are integral to Cyri’s design, especially given the sensitive
nature of email content. Data privacy is ensured thanks to local processing and
minimal external data sharing (URLs and domains) for safety checks.
All email analyses are conducted locally on the user’s machine. Using a locally
hosted LLM ensures that sensitive information remains within the
user’s environment, mitigating the risk of data breaches. The LISA component implements the Hugging Face Transformers library to load and utilize the Llama 3.1 8B model locally.
