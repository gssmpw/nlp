\section{Cyri Usable Interface }
\label{sec:ui}
%include plugin e UI con electron app per easy deploy
The VAC application serves as the central hub of Cyri, providing a user-friendly interface for email analysis and interaction. It is implemented using web technology and Electron. 
%It enables the creation of a desktop application that feels native to the user’s operating system but is developed with the ease and speed of web technologies. 
By leveraging Electron’s capabilities, the application offers a cross-platform, user-friendly interface that integrates seamlessly with LISA and the email client plugin.
Several studies~\cite{b22, b23, b36} have consistently demonstrated that traditional warning dialogs are often ineffective in alerting users to phishing threats due to a lack of user understanding and the habituation effect. The habituation effect occurs when users become desensitized to repetitive visual stimuli, such as generic phishing warnings, leading them to ignore these alerts over time and diminishing their vigilance. This desensitization results in users dismissing important security warnings without adequate consideration, thereby increasing their susceptibility to phishing attacks.
To address this critical challenge, research on usable security has emphasized the importance of creating polymorphic warning interfaces in phishing detection. These interfaces dynamically alter their appearance and content each time they are presented to the user, aiming to reduce habituation and encourage users to pay closer attention to each alert. By introducing variability in warnings, the polymorphic approach enhances user engagement and prompts cautious behavior, making security warnings more effective.\\
Cyri thoughtfully incorporates these research findings to enhance its phishing detection efficacy. It employs a user-centered interface that moves beyond generic warning dialogs by providing clear, contextualized explanations of potential phishing threats within the user's email text and exchanges. By offering detailed analyses and actionable advice articulated in clear and understandable language without excessive technical terminology, Cyri enhances user understanding of the potential risk causes in the email text and encourages proactive behavior in managing suspicious emails. Moreover, Cyri offers actionable advice based on the identified risks and features, providing practical recommendations on how the recipient should handle the suspicious email, such as verifying the sender's identity through alternative channels, avoiding clicking on embedded links, or reporting the email to the appropriate authorities, following best practices in phishing management~\cite{b22}.
\\
Cyri utilizes dynamic visual feedback by changing the interface's background color and icons based on the analysis results. For instance, when an email is identified as phishing, the background color shifts toward red color (see Figure~\ref{fig: examplePhishingInterface}) to indicate danger, with the intensity increasing based on the phishing likelihood percentage and the ``Feature Score'', which depends on the weights of the features found.

\begin{figure*}[htbp]
  \centering
  \includegraphics[width=0.85\linewidth]{figures/PhishingExampleInterface.PNG}
  \caption{The VAC interface of Cyri in action: red background identifies a phishing mail, with semantic features highlighted in the email text (a) and the list below (b). Conversation with LISA happens on the right (c) through the query interface (d) or by audio (e)}
  \label{fig: examplePhishingInterface}
\end{figure*}

Conversely, if an email is deemed safe (see Figure~\ref{fig: exampleSafeInterface}), the background shifts to a calming blue, with the intensity depending on the safe likelihood percentage.

\begin{figure}[htbp]
  \centering
  \includegraphics[width=0.48\textwidth]{figures/SafeExampleInterface.PNG}
  \caption{The VAC interface of Cyri for a safe email}
  \label{fig: exampleSafeInterface}
\end{figure}

The Cyri VAC interface is designed to offer an intuitive and user-friendly experience. The interface presents a clean and organized layout divided into two primary sections. On the left side, users encounter email details (see Figure~\ref{fig: examplePhishingInterface}.a) and analysis results (see Figure~\ref{fig: examplePhishingInterface}.b), with a date picker feature allowing effortless navigation through emails by selecting specific dates. The e-mail text keeps the same indentation and style as the email client used. Phishing features detected in the email are highlighted by utilizing unique color combinations and text styles to draw attention to these elements (see Figure~\ref{fig: exampleFeaturesInterface}).
At the bottom, there is a list of all Cyri semantic features: users can select or remove a particular feature from the analysis by clicking on it, giving them control over the information presented if they consider a specific part of the analysis wrong or not accurate enough. 

On the right side column, the application focuses on facilitating user interaction and query handling. The conversation history displays all past interactions (see Figure~\ref{fig: examplePhishingInterface}.c) and responses related to the selected email analysis, maintaining continuity and context. Users can submit new questions for additional clarification, and when a query is entered (see Figure~\ref{fig: examplePhishingInterface}.d), the application processes it and monitors for a corresponding response generated by LISA. 

\begin{figure}[htbp]
  \centering
  \includegraphics[width=0.48\textwidth]{figures/PhishingEmailFeaturesAnalysis.PNG}
  \caption{Phishing Email Features Analysis Example}
  \label{fig: exampleFeaturesInterface}
\end{figure}

The user can select tags representing phishing semantic features to navigate directly to the portions of the email body and analysis characterized by that feature (navigating them by order of occurrences or severity). 
Cyri also incorporates a text-to-speech functionality (see Figure~\ref{fig: examplePhishingInterface}.e), allowing users to have the contents of the email and the analysis results read aloud, further enhancing accessibility and allowing them to listen to Cyri's recommendations while interacting with visual cues, taking advantage of a multi-modal interaction.
\\
By utilizing dynamic and context-specific visual cues and explanations, Cyri effectively reduces habituation. Each interaction feels unique and tailored to the specific situation, maintaining the user's engagement and attentiveness to security warnings. This approach aligns with best practices in user interface design for security applications~\cite{b22, b23}, where the goal is to balance alerting users to potential threats without causing alarm fatigue.
A video demonstration of Cyri is available in the GitHub repository reported in Appendix A.
