\documentclass[compsoc, conference, a4paper, 10pt, times]{IEEEtran}
\IEEEoverridecommandlockouts
% The preceding line is only needed to identify funding in the first footnote. If that is unneeded, please comment it out.
\usepackage{cite}
\usepackage{amsmath,amssymb,amsfonts}
\usepackage{algorithmic}
\usepackage{graphicx}
\usepackage{textcomp}
\usepackage{bmpsize}
\usepackage{xcolor}
\usepackage{lipsum}


\usepackage{booktabs}

\usepackage[colorlinks=true,urlcolor=black]{hyperref}
\def\BibTeX{{\rm B\kern-.05em{\sc i\kern-.025em b}\kern-.08em
    T\kern-.1667em\lower.7ex\hbox{E}\kern-.125emX}}
\begin{document}

\title{Cyri: A Conversational AI-based Assistant for Supporting the Human User in Detecting and Responding to Phishing Attacks}
%{\footnotesize
%    Submission 352}  }

%\iffalse
\author{\IEEEauthorblockN{1\textsuperscript{st} Antonio La Torre}
\IEEEauthorblockA{
\textit{Sapienza University of Rome}\\
Rome, Italy \\
latorre.2067686@studenti.uniroma1.it}
\and
\IEEEauthorblockN{2\textsuperscript{nd} Marco Angelini}
\IEEEauthorblockA{
\textit{Link Campus University of Rome)}\\
Rome, Italy \\
m.angelini@unilink.it}
\iffalse
\and
\IEEEauthorblockN{3\textsuperscript{rd} Given Name Surname}
\IEEEauthorblockA{\textit{dept. name of organization (of Aff.)} \\
\textit{name of organization (of Aff.)}\\
City, Country \\
email address}
\and
\IEEEauthorblockN{4\textsuperscript{th} Given Name Surname}
\IEEEauthorblockA{\textit{dept. name of organization (of Aff.)} \\
\textit{name of organization (of Aff.)}\\
City, Country \\
email address}
\and
\IEEEauthorblockN{5\textsuperscript{th} Given Name Surname}
\IEEEauthorblockA{\textit{dept. name of organization (of Aff.)} \\
\textit{name of organization (of Aff.)}\\
City, Country \\
email address}
\fi
}
%\fi

\maketitle

\begin{abstract}
Phishing attacks have become increasingly sophisticated, exploiting human vulnerabilities through social engineering tactics to deceive individuals into revealing sensitive information. Traditional detection methods, such as blacklist-based and heuristic approaches, often fail to identify new or cleverly disguised phishing attempts due to their reliance on known patterns and technical indicators and not on the semantic characteristics of the attack attempt. This work introduces Cyri, an AI-powered conversational assistant designed to support a human user in detecting and analyzing phishing emails by leveraging Large Language Models (LLMs).\\
Cyri has been designed to scrutinize emails for semantic features used in phishing attacks, such as urgency, authority, impersonation, exclusivity, and undesirable consequences, using an approach that unifies features already established in the literature with others by Cyri features extraction methodology. Cyri can be directly plugged into a client mail or webmail, ensuring seamless integration with the user's email workflow while maintaining data privacy through local processing. By performing all analyses on the user's machine, Cyri eliminates the need to transmit sensitive email data over the internet, reducing security risks associated with external data breaches. 
The Cyri user interface has been designed to reduce habituation effects and enhance user engagement. It employs dynamic visual cues and context-specific explanations to keep users alert and informed while maintaining their experience in using emails.
Additionally, it allows users to explore identified malicious semantic features both through conversation with the agent and visual exploration, obtaining the advantages of both modalities for expert or non-expert users. It also allows users to keep track of the conversation, supports the user in solving additional questions on both computed features or new parts of the mail, and applies its detection on demand.\\
To evaluate Cyri's ability to distinguish between phishing and safe communications, we crafted a comprehensive dataset of 420 phishing emails and 420 legitimate emails. 
%The phishing emails were generated using WormGPT V3.0, guided by carefully designed prompts to embody specific semantic features. Each email underwent meticulous manual review to ensure authenticity and effectiveness.
Through iterative evaluation, Cyri was optimized to reach an accuracy of 95.24\%, a precision of 96.8\%, a recall of 93.56\%, and an F1-score of 95.15\%, demonstrating high effectiveness in identifying critical phishing semantic features fundamental to phishing detection. A user study involving 10 participants, both experts and non-experts, evaluated Cyri's effectiveness and usability in real use, where the participants tested the system on their mail accounts. Results indicated that Cyri significantly aided users in identifying phishing emails and enhanced their understanding of phishing tactics.
%The results underscore the potential of advanced AI models in enhancing cybersecurity measures by not only detecting phishing attempts but also educating users about the tactics employed by attackers. Cyri's integration of detailed explanations and user interaction positions it as both a detection tool and an educational resource, aiming to reduce human error and improve overall resilience against phishing attacks.

\end{abstract}

\begin{IEEEkeywords}
Usable security, Phishing, LLM, Mixed initiative, Security Awareness and Training
\end{IEEEkeywords}

\section{Introduction}
\label{section:introduction}

% redirection is unique and important in VR
Virtual Reality (VR) systems enable users to embody virtual avatars by mirroring their physical movements and aligning their perspective with virtual avatars' in real time. 
As the head-mounted displays (HMDs) block direct visual access to the physical world, users primarily rely on visual feedback from the virtual environment and integrate it with proprioceptive cues to control the avatar’s movements and interact within the VR space.
Since human perception is heavily influenced by visual input~\cite{gibson1933adaptation}, 
VR systems have the unique capability to control users' perception of the virtual environment and avatars by manipulating the visual information presented to them.
Leveraging this, various redirection techniques have been proposed to enable novel VR interactions, 
such as redirecting users' walking paths~\cite{razzaque2005redirected, suma2012impossible, steinicke2009estimation},
modifying reaching movements~\cite{gonzalez2022model, azmandian2016haptic, cheng2017sparse, feick2021visuo},
and conveying haptic information through visual feedback to create pseudo-haptic effects~\cite{samad2019pseudo, dominjon2005influence, lecuyer2009simulating}.
Such redirection techniques enable these interactions by manipulating the alignment between users' physical movements and their virtual avatar's actions.

% % what is hand/arm redirection, motivation of study arm-offset
% \change{\yj{i don't understand the purpose of this paragraph}
% These illusion-based techniques provide users with unique experiences in virtual environments that differ from the physical world yet maintain an immersive experience. 
% A key example is hand redirection, which shifts the virtual hand’s position away from the real hand as the user moves to enhance ergonomics during interaction~\cite{feuchtner2018ownershift, wentzel2020improving} and improve interaction performance~\cite{montano2017erg, poupyrev1996go}. 
% To increase the realism of virtual movements and strengthen the user’s sense of embodiment, hand redirection techniques often incorporate a complete virtual arm or full body alongside the redirected virtual hand, using inverse kinematics~\cite{hartfill2021analysis, ponton2024stretch} or adjustments to the virtual arm's movement as well~\cite{li2022modeling, feick2024impact}.
% }

% noticeability, motivation of predicting a probability, not a classification
However, these redirection techniques are most effective when the manipulation remains undetected~\cite{gonzalez2017model, li2022modeling}. 
If the redirection becomes too large, the user may not mitigate the conflict between the visual sensory input (redirected virtual movement) and their proprioception (actual physical movement), potentially leading to a loss of embodiment with the virtual avatar and making it difficult for the user to accurately control virtual movements to complete interaction tasks~\cite{li2022modeling, wentzel2020improving, feuchtner2018ownershift}. 
While proprioception is not absolute, users only have a general sense of their physical movements and the likelihood that they notice the redirection is probabilistic. 
This probability of detecting the redirection is referred to as \textbf{noticeability}~\cite{li2022modeling, zenner2024beyond, zenner2023detectability} and is typically estimated based on the frequency with which users detect the manipulation across multiple trials.

% version B
% Prior research has explored factors influencing the noticeability of redirected motion, including the redirection's magnitude~\cite{wentzel2020improving, poupyrev1996go}, direction~\cite{li2022modeling, feuchtner2018ownershift}, and the visual characteristics of the virtual avatar~\cite{ogawa2020effect, feick2024impact}.
% While these factors focus on the avatars, the surrounding virtual environment can also influence the users' behavior and in turn affect the noticeability of redirection.
% One such prominent external influence is through the visual channel - the users' visual attention is constantly distracted by complex visual effects and events in practical VR scenarios.
% Although some prior studies have explored how to leverage user blindness caused by visual distractions to redirect users' virtual hand~\cite{zenner2023detectability}, there remains a gap in understanding how to quantify the noticeability of redirection under visual distractions.

% visual stimuli and gaze behavior
Prior research has explored factors influencing the noticeability of redirected motion, including the redirection's magnitude~\cite{wentzel2020improving, poupyrev1996go}, direction~\cite{li2022modeling, feuchtner2018ownershift}, and the visual characteristics of the virtual avatar~\cite{ogawa2020effect, feick2024impact}.
While these factors focus on the avatars, the surrounding virtual environment can also influence the users' behavior and in turn affect the noticeability of redirection.
This, however, remains underexplored.
One such prominent external influence is through the visual channel - the users' visual attention is constantly distracted by complex visual effects and events in practical VR scenarios.
We thus want to investigate how \textbf{visual stimuli in the virtual environment} affect the noticeability of redirection.
With this, we hope to complement existing works that focus on avatars by incorporating environmental visual influences to enable more accurate control over the noticeability of redirected motions in practical VR scenarios.
% However, in realistic VR applications, the virtual environment often contains complex visual effects beyond the virtual avatar itself. 
% We argue that these visual effects can \textbf{distract users’ visual attention and thus affect the noticeability of redirection offsets}, while current research has yet taken into account.
% For instance, in a VR boxing scenario, a user’s visual attention is likely focused on their opponent rather than on their virtual body, leading to a lower noticeability of redirection offsets on their virtual movements. 
% Conversely, when reaching for an object in the center of their field of view, the user’s attention is more concentrated on the virtual hand’s movement and position to ensure successful interaction, resulting in a higher noticeability of offsets.

Since each visual event is a complex choreography of many underlying factors (type of visual effect, location, duration, etc.), it is extremely difficult to quantify or parameterize visual stimuli.
Furthermore, individuals respond differently to even the same visual events.
Prior neuroscience studies revealed that factors like age, gender, and personality can influence how quickly someone reacts to visual events~\cite{gillon2024responses, gale1997human}. 
Therefore, aiming to model visual stimuli in a way that is generalizable and applicable to different stimuli and users, we propose to use users' \textbf{gaze behavior} as an indicator of how they respond to visual stimuli.
In this paper, we used various gaze behaviors, including gaze location, saccades~\cite{krejtz2018eye}, fixations~\cite{perkhofer2019using}, and the Index of Pupil Activity (IPA)~\cite{duchowski2018index}.
These behaviors indicate both where users are looking and their cognitive activity, as looking at something does not necessarily mean they are attending to it.
Our goal is to investigate how these gaze behaviors stimulated by various visual stimuli relate to the noticeability of redirection.
With this, we contribute a model that allows designers and content creators to adjust the redirection in real-time responding to dynamic visual events in VR.

To achieve this, we conducted user studies to collect users' noticeability of redirection under various visual stimuli.
To simulate realistic VR scenarios, we adopted a dual-task design in which the participants performed redirected movements while monitoring the visual stimuli.
Specifically, participants' primary task was to report if they noticed an offset between the avatar's movement and their own, while their secondary task was to monitor and report the visual stimuli.
As realistic virtual environments often contain complex visual effects, we started with simple and controlled visual stimulus to manage the influencing factors.

% first user study, confirmation study
% collect data under no visual stimuli, different basic visual stimuli
We first conducted a confirmation study (N=16) to test whether applying visual stimuli (opacity-based) actually affects their noticeability of redirection. 
The results showed that participants were significantly less likely to detect the redirection when visual stimuli was presented $(F_{(1,15)}=5.90,~p=0.03)$.
Furthermore, by analyzing the collected gaze data, results revealed a correlation between the proposed gaze behaviors and the noticeability results $(r=-0.43)$, confirming that the gaze behaviors could be leveraged to compute the noticeability.

% data collection study
We then conducted a data collection study to obtain more accurate noticeability results through repeated measurements to better model the relationship between visual stimuli-triggered gaze behaviors and noticeability of redirection.
With the collected data, we analyzed various numerical features from the gaze behaviors to identify the most effective ones. 
We tested combinations of these features to determine the most effective one for predicting noticeability under visual stimuli.
Using the selected features, our regression model achieved a mean squared error (MSE) of 0.011 through leave-one-user-out cross-validation. 
Furthermore, we developed both a binary and a three-class classification model to categorize noticeability, which achieved an accuracy of 91.74\% and 85.62\%, respectively.

% evaluation study
To evaluate the generalizability of the regression model, we conducted an evaluation study (N=24) to test whether the model could accurately predict noticeability with new visual stimuli (color- and scale-based animations).
Specifically, we evaluated whether the model's predictions aligned with participants' responses under these unseen stimuli.
The results showed that our model accurately estimated the noticeability, achieving mean squared errors (MSE) of 0.014 and 0.012 for the color- and scale-based visual stimili, respectively, compared to participants' responses.
Since the tested visual stimuli data were not included in the training, the results suggested that the extracted gaze behavior features capture a generalizable pattern and can effectively indicate the corresponding impact on the noticeability of redirection.

% application
Based on our model, we implemented an adaptive redirection technique and demonstrated it through two applications: adaptive VR action game and opportunistic rendering.
We conducted a proof-of-concept user study (N=8) to compare our adaptive redirection technique with a static redirection, evaluating the usability and benefits of our adaptive redirection technique.
The results indicated that participants experienced less physical demand and stronger sense of embodiment and agency when using the adaptive redirection technique. 
These results demonstrated the effectiveness and usability of our model.

In summary, we make the following contributions.
% 
\begin{itemize}
    \item 
    We propose to use users' gaze behavior as a medium to quantify how visual stimuli influences the noticebility of redirection. 
    Through two user studies, we confirm that visual stimuli significantly influences noticeability and identify key gaze behavior features that are closely related to this impact.
    \item 
    We build a regression model that takes the user's gaze behavioral data as input, then computes the noticeability of redirection.
    Through an evaluation study, we verify that our model can estimate the noticeability with new participants under unseen visual stimuli.
    These findings suggest that the extracted gaze behavior features effectively capture the influence of visual stimuli on noticeability and can generalize across different users and visual stimuli.
    \item 
    We develop an adaptive redirection technique based on our regression model and implement two applications with it.
    With a proof-of-concept study, we demonstrate the effectiveness and potential usability of our regression model on real-world use cases.

\end{itemize}

% \delete{
% Virtual Reality (VR) allows the user to embody a virtual avatar by mirroring their physical movements through the avatar.
% As the user's visual access to the physical world is blocked in tasks involving motion control, they heavily rely on the visual representation of the avatar's motions to guide their proprioception.
% Similar to real-world experiences, the user is able to resolve conflicts between different sensory inputs (e.g., vision and motor control) through multisensory integration, which is essential for mitigating the sensory noise that commonly arises.
% However, it also enables unique manipulations in VR, as the system can intentionally modify the avatar's movements in relation to the user's motions to achieve specific functional outcomes,
% for example, 
% % the manipulations on the avatar's movements can 
% enabling novel interaction techniques of redirected walking~\cite{razzaque2005redirected}, redirected reaching~\cite{gonzalez2022model}, and pseudo haptics~\cite{samad2019pseudo}.
% With small adjustments to the avatar's movements, the user can maintain their sense of embodiment, due to their ability to resolve the perceptual differences.
% % However, a large mismatch between the user and avatar's movements can result in the user losing their sense of embodiment, due to an inability to resolve the perceptual differences.
% }

% \delete{
% However, multisensory integration can break when the manipulation is so intense that the user is aware of the existence of the motion offset and no longer maintains the sense of embodiment.
% Prior research studied the intensity threshold of the offset applied on the avatar's hand, beyond which the embodiment will break~\cite{li2022modeling}. 
% Studies also investigated the user's sensitivity to the offsets over time~\cite{kohm2022sensitivity}.
% Based on the findings, we argue that one crucial factor that affects to what extent the user notices the offset (i.e., \textit{noticeability}) that remains under-explored is whether the user directs their visual attention towards or away from the virtual avatar.
% Related work (e.g., Mise-unseen~\cite{marwecki2019mise}) has showcased applications where adjustments in the environment can be made in an unnoticeable manner when they happen in the area out of the user's visual field.
% We hypothesize that directing the user's visual attention away from the avatar's body, while still partially keeping the avatar within the user's field-of-view, can reduce the noticeability of the offset.
% Therefore, we conduct two user studies and implement a regression model to systematically investigate this effect.
% }

% \delete{
% In the first user study (N = 16), we test whether drawing the user's visual attention away from their body impacts the possibility of them noticing an offset that we apply to their arm motion in VR.
% We adopt a dual-task design to enable the alteration of the user's visual attention and a yes/no paradigm to measure the noticeability of motion offset. 
% The primary task for the user is to perform an arm motion and report when they perceive an offset between the avatar's virtual arm and their real arm.
% In the secondary task, we randomly render a visual animation of a ball turning from transparent to red and becoming transparent again and ask them to monitor and report when it appears.
% We control the strength of the visual stimuli by changing the duration and location of the animation.
% % By changing the time duration and location of the visual animation, we control the strengths of attraction to the users.
% As a result, we found significant differences in the noticeability of the offsets $(F_{(1,15)}=5.90,~p=0.03)$ between conditions with and without visual stimuli.
% Based on further analysis, we also identified the behavioral patterns of the user's gaze (including pupil dilation, fixations, and saccades) to be correlated with the noticeability results $(r=-0.43)$ and they may potentially serve as indicators of noticeability.
% }

% \delete{
% To further investigate how visual attention influences the noticeability, we conduct a data collection study (N = 12) and build a regression model based on the data.
% The regression model is able to calculate the noticeability of the offset applied on the user's arm under various visual stimuli based on their gaze behaviors.
% Our leave-one-out cross-validation results show that the proposed method was able to achieve a mean-squared error (MSE) of 0.012 in the probability regression task.
% }

% \delete{
% To verify the feasibility and extendability of the regression model, we conduct an evaluation study where we test new visual animations based on adjustments on scale and color and invite 24 new participants to attend the study.
% Results show that the proposed method can accurately estimate the noticeability with an MSE of 0.014 and 0.012 in the conditions of the color- and scale-based visual effects.
% Since these animations were not included in the dataset that the regression model was built on, the study demonstrates that the gaze behavioral features we extracted from the data capture a generalizable pattern of the user's visual attention and can indicate the corresponding impact on the noticeability of the offset.
% }

% \delete{
% Finally, we demonstrate applications that can benefit from the noticeability prediction model, including adaptive motion offsets and opportunistic rendering, considering the user's visual attention. 
% We conclude with discussions of our work's limitations and future research directions.
% }

% \delete{
% In summary, we make the following contributions.
% }
% % 
% \begin{itemize}
%     \item 
%     \delete{
%     We quantify the effects of the user's visual attention directed away by stimuli on their noticeability of an offset applied to the avatar's arm motion with respect to the user's physical arm. 
%     Through two user studies, we identified gaze behavioral features that are indicative of the changes in noticeability.
%     }
%     \item 
%     \delete{We build a regression model that takes the user's gaze behavioral data and the offset applied to the arm motion as input, then computes the probability of the user noticing the offset.
%     Through an evaluation study, we verified that the model needs no information about the source attracting the user's visual attention and can be generalizable in different scenarios.
%     }
%     \item 
%     \delete{We demonstrate two applications that potentially benefit from the regression model, including adaptive motion offsets and opportunistic rendering.
%     }

% \end{itemize}

\begin{comment}
However, users will lose the sense of embodiment to the virtual avatars if they notice the offset between the virtual and physical movements.
To address this, researchers have been exploring the noticing threshold of offsets with various magnitudes and proposing various redirection techniques that maintain the sense of embodiment~\cite{}.

However, when users embody virtual avatars to explore virtual environments, they encounter various visual effects and content that can attract their attention~\cite{}.
During this, the user may notice an offset when he observes the virtual movement carefully while ignoring it when the virtual contents attract his attention from the movements.
Therefore, static offset thresholds are not appropriate in dynamic scenarios.

Past research has proposed dynamic mapping techniques that adapted to users' state, such as hand moving speed~\cite{frees2007prism} or ergonomically comfortable poses~\cite{montano2017erg}, but not considering the influence of virtual content.
More specifically, PRISM~\cite{frees2007prism} proposed adjusting the C/D ratio with a non-linear mapping according to users' hand moving speed, but it might not be optimal for various virtual scenarios.
While Erg-O~\cite{montano2017erg} redirected users' virtual hands according to the virtual target's relative position to reduce physical fatigue, neglecting the change of virtual environments. 

Therefore, how to design redirection techniques in various scenarios with different visual attractions remains unknown.
To address this, we investigate how visual attention affects the noticing probability of movement offsets.
Based on our experiments, we implement a computational model that automatically computes the noticing probability of offsets under certain visual attractions.
VR application designers and developers can easily leverage our model to design redirection techniques maintaining the sense of embodiment adapt to the user's visual attention.
We implement a dynamic redirection technique with our model and demonstrate that it effectively reduces the target reaching time without reducing the sense of embodiment compared to static redirection techniques.

% Need to be refined
This paper offers the following contributions.
\begin{itemize}
    \item We investigate how visual attractions affect the noticing probability of redirection offsets.
    \item We construct a computational model to predict the noticing probability of an offset with a given visual background.
    \item We implement a dynamic redirection technique adapting to the visual background. We evaluate the technique and develop three applications to demonstrate the benefits. 
\end{itemize}



First, we conducted a controlled experiment to understand how users perceived the movement offset while subjected to various distractions.
Since hand redirection is one of the most frequently used redirections in VR interactions, we focused on the dynamic arm movements and manually added angular offsets to the' elbow joint~\cite{li2022modeling, gonzalez2022model, zenner2019estimating}. 
We employed flashing spheres in the user's field of view as distractions to attract users' visual attention.
Participants were instructed to report the appearing location of the spheres while simultaneously performing the arm movements and reporting if they perceived an offset during the movement. 
(\zhipeng{Add the results of data collection. Analyze the influence of the distance between the gaze map and the offset.}
We measured the visual attraction's magnitude with the gaze distribution on it.
Results showed that stronger distractions made it harder for users to notice the offset.)
\zhipeng{Need to rewrite. Not sure to use gaze distribution or a metric obtained from the visual content.}
Secondly, we constructed a computational model to predict the noticing probability of offsets with given visual content.
We analyzed the data from the user studies to measure the influence of visual attractions on the noticing probability of offsets.
We built a statistical model to predict the offset's noticing probability with a given visual content.
Based on the model, we implement a dynamic redirection technique to adjust the redirection offset adapted to the user's current field of view.
We evaluated the technique in a target selection task compared to no hand redirection and static hand redirection.
\zhipeng{Add the results of the evaluation.}
Results showed that the dynamic hand redirection technique significantly reduced the target selection time with similar accuracy and a comparable sense of embodiment.
Finally, we implemented three applications to demonstrate the potential benefits of the visual attention adapted dynamic redirection technique.
\end{comment}

% This one modifies arm length, not redirection
% \citeauthor{mcintosh2020iteratively} proposed an adaptation method to iteratively change the virtual avatar arm's length based on the primary tasks' performance~\cite{mcintosh2020iteratively}.



% \zhipeng{TO ADD: what is redirection}
% Redirection enables novel interactions in Virtual Reality, including redirected walking, haptic redirection, and pseudo haptics by introducing an offset to users' movement.
% \zhipeng{TO ADD: extend this sentence}
% The price of this is that users' immersiveness and embodiment in VR can be compromised when they notice the offset and perceive the virtual movement not as theirs~\cite{}.
% \zhipeng{TO ADD: extend this sentence, elaborate how the virtual environment attracts users' attention}
% Meanwhile, the visual content in the virtual environment is abundant and consistently captures users' attention, making it harder to notice the offset~\cite{}.
% While previous studies explored the noticing threshold of the offsets and optimized the redirection techniques to maintain the sense of embodiment~\cite{}, the influence of visual content on the probability of perceiving offsets remains unknown.  
% Therefore, we propose to investigate how users perceive the redirection offset when they are facing various visual attractions.


% We conducted a user study to understand how users notice the shift with visual attractions.
% We used a color-changing ball to attract the user's attention while instructing users to perform different poses with their arms and observe it meanwhile.
% \zhipeng{(Which one should be the primary task? Observe the ball should be the primary one, but if the primary task is too simple, users might allocate more attention on the secondary task and this makes the secondary task primary.)}
% \zhipeng{(We need a good and reasonable dual-task design in which users care about both their pose and the visual content, at least in the evaluation study. And we need to be able to control the visual content's magnitude and saliency maybe?)}
% We controlled the shift magnitude and direction, the user's pose, the ball's size, and the color range.
% We set the ball's color-changing interval as the independent factor.
% We collect the user's response to each shift and the color-changing times.
% Based on the collected data, we constructed a statistical model to describe the influence of visual attraction on the noticing probability.
% \zhipeng{(Are we actually controlling the attention allocation? How do we measure the attracting effect? We need uniform metrics, otherwise it is also hard for others to use our knowledge.)}
% \zhipeng{(Try to use eye gaze? The eye gaze distribution in the last five seconds to decide the attention allocation? Basically constructing a model with eye gaze distribution and noticing probability. But the user's head is moving, so the eye gaze distribution is not aligned well with the current view.)}

% \zhipeng{Saliency and EMD}
% \zhipeng{Gaze is more than just a point: Rethinking visual attention
% analysis using peripheral vision-based gaze mapping}

% Evaluation study(ideal case): based on the visual content, adjusting the redirection magnitude dynamically.

% \zhipeng{(The risk is our model's effect is trivial.)}

% Applications:
% Playing Lego while watching demo videos, we can accelerate the reaching process of bricks, and forbid the redirection during the manipulation.

% Beat saber again: but not make a lot of sense? Difficult game has complicated visual effects, while allows larger shift, but do not need large shift with high difficulty



\section{Background and Related Work}\label{sec:related}

\paragraph{\textbf{Privacy of Human-Centered Systems}}
Ensuring privacy in human-centric ML-based systems presents inherent conflicts among service utility, cost, and personal and institutional privacy~\cite{sztipanovits2019science}. Without appropriate incentives for societal information sharing, we may face decision-making policies that are either overly restrictive or that compromise private information, leading to adverse selection~\cite{jin2016enabling}. Such compromises can result in privacy violations, exacerbating societal concerns regarding the impact of emerging technology trends in human-centric systems~\cite{mulligan2016privacy,fox2021exploring,goldfarb2012shifts}. Consequently, several studies have aimed to establish privacy guarantees that allow auditing and quantifying compromises to make these systems more acceptable~\cite{jagielski2020auditing, raji2020saving}. ML models in decision-making systems have also been shown to leak significant amounts of private information that requires auditing platforms~\cite{hamon2022bridging}. Various studies focused on privacy-preserving machine learning techniques targeting decision-making systems~\cite{abadi2016deep, cummings2019compatibility, taherisadr2023adaparl, taherisadr2024hilt}. Recognizing that perfect privacy is often unattainable, this paper examines privacy from an equity perspective. We investigate how to ensure a fair distribution of harm when privacy leaks occur, addressing the technical challenges alongside the ethical imperatives of equitable privacy protection.


\paragraph{\textbf{\acf{fl}}}
\ac{fl} is an approach in machine learning that enables the collaborative training of models across multiple devices or institutions without requiring data to be centralized. This decentralized setup is particularly beneficial in fields where data-sharing restrictions are enforced by privacy regulations, such as healthcare and finance. \ac{fl} allows organizations to derive insights from data distributed across various locations while adhering to legal constraints, including the General Data Protection Regulation (GDPR) \cite{BG_Survey2,BG_Survey1}.

One of the most widely adopted methods in \ac{fl} is \ac{fedavg}, which operates through iterative rounds of communication between a central server and participating clients to collaboratively train a shared model. During each communication round, the server sends the current global model to each client, which uses their locally stored data to perform optimization steps. These optimized models are subsequently sent back to the server, where they are aggregated to update the global model. The process repeats until the model converges. Known for its simplicity and effectiveness, \ac{fedavg} serves as the primary technique for coordinating model updates across distributed clients in our work. Additionally, we specifically employ horizontal federated learning, where data is distributed across entities with similar feature spaces but distinct user groups \cite{BG_HorizontalFL}.

\paragraph{\textbf{Privacy Risks in \ac{fl}}}
Privacy risks are a critical concern in \ac{fl}, as collaborative training on decentralized data can inadvertently expose sensitive information. A primary threat is the \ac{mia}, where adversaries determine whether specific data records were part of the model's training set \cite{shokri2017membership,BG_MIA}. Researchers have since demonstrated \ac{mia}'s effectiveness across various machine learning models, including \ac{fl}, showing, for example, that adversaries can infer if a specific location profile contributed to an FL model \cite{BG_MIA_1,BG_MIA_2}. However, while \ac{mia} identifies training members, it does not reveal the client that contributed the data. \ac{sia}, introduced in \cite{BG_SIA_2}, extends \ac{mia} by identifying which client owns a training record, thus posing significant security risks by exposing client-specific information in \ac{fl} settings.

The \ac{noniid} nature of data in federated learning presents additional privacy challenges, as variations in data distributions across clients heighten the risk of privacy leakage. When data distributions differ widely among clients, individual model updates become more distinguishable, potentially allowing attackers to infer sensitive information \cite{BG_NON_IID}. This distinctiveness in updates can make federated models more susceptible to inference attacks, such as \ac{mia} and \ac{sia}, as malicious actors may exploit these distributional differences to trace updates back to specific clients. This vulnerability is especially relevant in our work, as we use the \ac{har} dataset, which is inherently \ac{noniid} across clients, thus posing an increased risk for privacy leakage.




\paragraph{\textbf{Fairness in \ac{fl}}}
Fairness in \ac{fl} is crucial due to the varied data distributions among clients, which can lead to biased outcomes favoring certain groups \cite{BG_Fairness_2}. Achieving fairness involves balancing the global model's benefits across clients despite the decentralized nature of the data. Approaches include group fairness, ensuring performance equity across client groups, and performance distribution fairness, which focuses on fair accuracy distribution~\cite{selialia2024mitigating}. Additional types are selection fairness (equitable client participation), contribution fairness (rewards based on contributions), and expectation fairness (aligning performance with client expectations) \cite{BG_Fairness}. Achieving fairness in \ac{fl} across these various dimensions remains challenging due to the inherent heterogeneity of client data and environments. In response to this heterogeneity, personalization has emerged as a strategy to tailor models to individual clients, enhancing local performance~\cite{BG_Personalization,BG_Personalization_2, BG_FairnessPrivacy}.   

When considering fairness in FL, it is crucial to address the interplay with privacy. Specifically, ensuring an equitable distribution of privacy risks across clients is paramount, preventing any group from being disproportionately vulnerable to privacy leakage, particularly under attacks such as source inference attacks (SIAs).





\begin{figure*}[htbp]
    \includegraphics[width=\linewidth]{document/figures/sensing_principle.png}
  \hfill
    \includegraphics[width=\linewidth]{document/figures/sensing_principle_sequential2.png}
  \caption{Acoustic and IMU data over 26 isolated English/ASL alphabet letters and continuously fingerspelled words. Continuous fingerspelling adds complexity due to natural flow and quick transitions between letters, which alter sensor values depending on adjacent letters. }
\Description{Acoustic and IMU data over Isolated 26 English manual alphabets and continuous fingerspelled words: Continuous fingerspelling involves added complexity due to the natural flow and quick transitions between letters, which alter sensor values depending on adjacent letters. }
  \label{fig:sensing_principle}
\end{figure*}



\section{SpellRing}

SpellRing is an AI-powered sensing system designed to recognize continuously fingerspelled words using a single ring. This section elaborates on the challenges of designing such a wearable device and details how we developed an AI-powered ring with intelligent sensing methods to achieve accurate recognition.


\subsection{Challenges}
Recognizing continuous fingerspelling poses several unique challenges that make it significantly more complex than recognizing isolated ASL letters:

\subsubsection{Complexity of Handshape, Movement, and Palm Orientation}
American Sign Language (ASL) fingerspelling involves complex combinations of different handshapes, movements, and palm orientations. This poses challenges for accurate fingerspelling recognition. Some letters, such as `A', `S', `M', `N', and `T' (see Figure \ref{fig:sensing_principle}), appear visually similar, while others like `K' and `P', `G' and `Q', or `H' and `U' share the same handshape while differing in palm orientation. Additionally, certain letters (e.g., `J' and `Z') involve specific hand movements, further complicating the recognition process.


\subsubsection{User-Dependent Customized Transitions}
Continuous fingerspelling introduces an additional layer of complexity due to its natural flow and quick transitions between letters (see Figure \ref{fig:sensing_principle}). These transitions vary significantly depending on letter sequences and individual signing behaviors \cite{keane2016fingerspelling, keane2015segmentation}. For instance, the letter `E' can be signed as either a closed or open form, with the open `E' more commonly used during faster fingerspelling, particularly at the beginning or end of a word.
Fluent signers can fingerspell at speeds of 5--8 letters per second \cite{hanson1982use, quinto2010rates, keane2016fingerspelling}, often blending adjacent letters \cite{hassan2023tap, shi2018american, keane2015segmentation}. This high speed increases both efficiency and user-specific signing behaviors, making the accurate recognition of continuous fingerspelling much more challenging than recognizing isolated ASL letters.


\subsubsection{Form Factor vs. Recognition Accuracy Trade-off}
Designing a wearable device for ASL recognition presents a significant challenge in balancing form factor with user experience and recognition accuracy. Glove-based devices with sensors on all fingers can capture detailed poses but are bulky and impractical for daily use, often hindering dexterity. Wristbands, such as EMG sensor bands, offer better usability but struggle with performance issues due to the need for extensive training data across sessions and muscle variability.
Rings with embedded IMUs are more user-friendly, but reliable recognition often requires multiple rings, which can still compromise simplicity. Capturing complex ASL handshapes and movements with a single ring remains a significant challenge, as it must balance unobtrusive design with the ability to capture detailed and reliable data for recognition.


 \begin{table}[t]

\textcolor{black}{
\caption{\textcolor{black}{Performance over Fingerspelling Speed and Sampling Rate. FPS (Frame Per Second), H (Hearing), CODA (Child of Deaf Adults), G (Gender), \# (Max Letters Per Second)}}
\Description{Performance Analysis: Fingerspelling Experience and Sampling Rate}
\begin{tabular}{c|c|c|c|c|c|c}
\hline
   & Experience    & G & Year & \# & FPS - 87     & FPS - 490    \\ \hline
P1 & Leaner, H & M      & 1    & 2                                                                       & 92.99 (2.01) & 93.32 (1.89) \\ \hline
P2 & Leaner, H & F      & 5    & 4                                                                       & 62.67 (3.49) & 86.86 (4.33) \\ \hline
P3 & CODA, H       & M      & 10   & 5                                                                       & 57.93 (3.61) & 86.03 (3.87) \\ \hline
\end{tabular}
}
\end{table}
\label{fig:performance_pilot}
\subsection{Hardware Prototype Design}
To address these challenges, we developed SpellRing, a single-ring system capable of recognizing fingerspelled words at the word level. Our design incorporates two key sensing modalities: acoustic sensing for handshape and IMU sensors for movement.



\begin{figure*}[t]
  \includegraphics[width=0.8\linewidth]{document/figures/model.png}
  \caption{Fusion Model Framework}
  \Description{Fusion Model Framework}
  \label{fig:model}
\end{figure*}

\subsubsection{Single Ring Approach}

SpellRing is designed specifically for the thumb. While Ring-a-Pose \cite{yu2024ring} showed that rings could potentially be placed on all five fingers to track handshape, thumb placement is ideal for ASL recognition. Placing the sensor on other fingers leads to blockage issues, especially when fingerspelling letters such as `A', `S', `M', `N', `L', and `I', and during transitions between these letters. This blockage makes it difficult both to fingerspell and to capture handshape using acoustic sensing. We chose to position the ring on the thumb to minimize these blockage issues. \textcolor{black}{We further discuss the ring's placement and user experience in Section \ref{future}.}

\subsubsection{Sensing Modalities} \textit{1) Active Acoustic Sensing:} For handshape sensing, we chose to adpot active acoustic sensing on the ring. Only requiring low-power and miniature microphone and speakers, this sensing method has shown promising performance in tracking and understanding various body postures on wearables\cite{yu2024ring,li2022eario,li2024eyeecho,li2024gazetrak,mahmud2023posesonic,li2024sonicid,mahmud2024actsonic,mahmud2024munchsonic,sun2023echonose,lee2024echowrist,zhang2023echospeech,mahmud2024wristsonic,zhang2023hpspeech,parikh2024echoguide}. Similar to Ring-a-Pose\cite{yu2024ring}, the ring acts as a `scanner' by emitting inaudible sound waves (frequency range of 20-24 kHz) to scan hand shapes. These sound waves are reflected and refracted by the fingers and received by the microphone on the ring. The preprocessed reflected acoustic signal patterns vary with different hand shapes, leading to precise estimation of handshape \cite{yu2024ring}. \textcolor{black}{However, our earlier experiments (see Table 2) with users of varying fingerspelling skills—especially in speed—using Ring-a-Pose showed that the system struggled to handle rapid fingerspelling of a participant with 10 years of ASL signing experience with a fingerspelling speed of up to 5 letters per second, resulting in an accuracy of 57.93\% on 1,164 words; our experiment is described in detail in Section \ref{experiment}. To capture sufficient information during fast fingerspelling, we increased the sampling rate by six times based on our hardware capabilities. This adjustment reduced the sensing range from 2.06 m (as with Ring-a-Pose \cite{yu2024ring}) to 34.3 cm, focusing more on finger and hand movements and capturing information every 0.12 seconds to classify letters. These changes led to improved performance, achieving an accuracy of 86.03\%---we used this setup for our full experiment.}



\textit{2) IMU for Hand Movement:} To track hand movement and palm orientation, we utilize a gyroscope from the IMU module \cite{zhang2017fingersound,zhang2017fingorbits}. This allows us to measure changes in rotational velocity (angular velocity) around three axes (x, y, and z). By integrating these measurements over time, we can track changes in hand movement, making it easy to distinguish letters with similar handshapes but different palm orientations.
\begin{figure}[b!]
  \includegraphics[width=\linewidth]{document/figures/prototyping.png}
  \caption{Hardware Prototype: (a) a 3.7V 70mAh LiPo battery, (b) an nRF MCU, (c) a customized Flexible Printed Circuit Board (FPCB) with a microphone and speaker, (d) an IMU sensor board (MPU6050), (e) an ESP32 Feather Board, and (f) a 3D-printed ring case.}
  \Description{Hardware Prototype}
  \label{fig:prototype}
\end{figure}

\subsubsection{Hardware Components}
As shown in Figure \ref{fig:prototype}, the ring incorporates a microphone (TDK-ICS-43434), a speaker (USound UT-P2019), and a customized Flexible Printed Circuit Board (FPCB) \textcolor{black}{(c)} enclosed within a 3D-printed Polylactic Acid (PLA) case \textcolor{black}{(f)}. It also features a microcontroller unit (MCU) \textcolor{black}{(e)}, an SD card for data storage \textcolor{black}{(b)}, and a 3.7V 70mAh LiPo battery \textcolor{black}{(a)}. The ring is powered by the battery, which has a switch for toggling it ON/OFF. Once powered on, the acoustic sensing system initiates and automatically saves data to the SD card until powered off. The IMU sensor board (MPU6050) \textcolor{black}{(d)} includes 6-axis inertial motion sensors (accelerometer and gyroscope), providing three-axis data output at a rate of 150Hz. The IMU is connected to the microcontroller on the wrist via a flexible wire, and the microcontroller transmits the data to an external PC through a flexible USB cable. \textcolor{black}{The acoustic data on the SD card and the IMU data on the PC are then synchronized based on timestamped records.}





% The SpellRing prototype includes a 3.7V 70mAh LiPo battery for powering the acoustic data acquisition, an onboard SD card for storing acoustic data, an ESP32 feather strapped on the wrist for reading IMU data, and an MPU6050 IMU sensor for capturing finger movement. The ESP32 feather reads the IMU data at 150 Hz and transmits it to a PC via serial-over-USB. While our current prototype includes a wrist-mounted component to reduce bulk on the thumb, further miniaturization is possible to eliminate this component in future iterations.


\subsection{Algorithms and Data Processing Pipeline}
SpellRing's software pipeline is designed to process multimodal data from the acoustic and IMU sensors and recognize fingerspelled words accurately. Our approach incorporates sophisticated data processing techniques and a deep learning model optimized for sequence recognition.

\subsubsection{Acoustic Data Processing}

Correlation-based frequency modulated continuous wave (C-FMCW) \cite{wang2018c} is used as the transmitted signal for acoustic sensing. The received signals are processed to calculate an echo profile, following methods specified in prior work \cite{yu2024ring,li2022eario,zhang2023echospeech}. These echo profiles encode temporal and spatial information of reflection and diffraction strengths, representing different handshape patterns. To isolate handshape changes from constant environmental reflections, we calculate the difference between consecutive echo frames, generating differential echo profiles. These profiles serve as the input representation of handshape patterns for our deep learning pipeline.

\subsubsection{IMU Data Processing}
Tri-axial gyroscope data, sampled at 150 Hz, is used to track palm orientation and rotational movement. Before feeding them into the deep learning model, we normalize the x, y, and z values and upsample them to synchronize with the acoustic data. This preprocessing step ensures that we can extract synchronized features from our multimodal deep learning pipeline.

\subsubsection{Deep Learning Pipeline}
Our deep learning pipeline leverages Connectionist Temporal Classification (CTC) \cite{graves2006connectionist,zhang2023echospeech}, a method widely employed in sequence labeling tasks, to recognize fingerspelled words continuously without needing to label or segment each letter. Aa shwon in Figure \ref{fig:model}, the model architecture comprises two main components: an acoustic sensing model and an IMU sensing model.

% \subsubsection{Model Framework}

For the acoustic sensing model, \textcolor{black}{we process differential echo profiles using a convolutional neural network (CNN) with ResNet-18 as the backbone. During pooling steps, we apply one-dimensional average pooling along the temporal axis only, rather than both axes, to preserve sequential information.} The IMU sensing model employs a 2D CNN architecture to process IMU data, as our pilot study demonstrated that this approach slightly outperformed a 1D CNN in terms of CTC loss.

The embeddings generated from both modalities are concatenated and then fed into a fully connected dense layer. This is followed by a dropout layer to prevent overfitting, and finally, a softmax function to produce the output probabilities. This multimodal approach allows our system to effectively combine information from both acoustic and motion sensors, enhancing the accuracy of fingerspelling recognition.

\subsubsection{Data Augmentation and Training Scheme}
To enhance performance and streamline training, we adopted several techniques. \textcolor{black}{To enhance the model's adaptability to varying fingerspelling speeds with a fixed window size, we augment the dataset by merging consecutive fingerspelled words, simply concatenating up to four words.} We also apply random noise during training to prevent overfitting and use random padding to handle variable-length inputs. Our training process involves a two-step approach: first training with data from all participants except one, then retraining with the specific participant's data for leave-one-session-out cross-validation. \textcolor{black}{We note that this two-step approach results in a user-dependent model, using 20 sessions collected from each participant over two to three different days, and the following reported results are based on this setup. User-independent results are further discussed in Section \ref{pretrained model}. }

\subsubsection{Word Correction}
To correct potential errors in the model's character sequence predictions, we compute the Levenshtein distance \cite{Levenshtein1965BinaryCC} between the predicted sequence and each unique word in a reference dictionary. The word with the smallest Levenshtein distance was selected as the corrected word, enhancing the overall accuracy of our system. \textcolor{black}{To align our system evaluation with prior literature, specifically for performance comparison with FingerSpeller \cite{martin2023fingerspeller}—such as multi-ring versus single-ring setups—we adopted their evaluation method by using the MacKenzie-Soukoreff phrase set \cite{mackenzie2003phrase} as the reference dictionary.}
\textcolor{black}{However, since the choice of reference dictionary affects the performance of the auto-correction, we discuss its impact using different dictionary sets in Section \ref{auto_correction}.}




\section{Semantic Analysis of Phishing emails}
\label{sec:semantic}

Phishing attacks leverage sophisticated social engineering techniques to deceive recipients into disseminating sensitive information or performing actions compromising security. A critical aspect of enhancing phishing detection mechanisms involves
understanding and identifying the semantic features commonly employed in phishing emails. This section details the comprehensive collection of semantic features used and the instrumentation activities that guide the LISA component of Cyri in recognizing them into email text.

\subsection{Collection of Phishing Semantic Features}

This activity aims to create a robust dataset of phishing semantic features and emails containing them
that can inform the development of more effective detection algorithms and improve the capabilities of LLMs in identifying phishing
emails and recognizing the presence of these features in the text. The semantic phishing features collected in our dataset derive from two primary activities:

\begin{itemize}
    \item Literature-Identified Features review: it aims at collecting semantic features previously recognized and documented in academic and professional cybersecurity literature.
    \item Methodology-Extracted Features: it aims at extracting novel semantic features through a systematic extraction using an automatic text analyzer (i.e., ChatGPT-4) to each element of a comprehensive email phishing dataset created specifically for this purpose.
\end{itemize}

Table~\ref{tab:semantic-features} shows the results of these two activities, reporting the list of all the semantic features collected along with their corresponding source.

\begin{table}[ht]
\centering % This centers the table
\caption{Overview of Cyri Semantic Features} % Title of the table
\label{tab:semantic-features} % Label for referencing
\begin{tabular}{p{5.5cm}p{1.5cm}} % Two columns, both left-aligned
\toprule % Top rule line
\textbf{Semantic Feature} & \textbf{Source} \\ % Column headings
\midrule % Middle rule line
Authority &  \cite{b8, b20}\\
Impersonation of Trusted Entities &  Extracted\\ 
Instant Gratification &  \cite{b10}\\
Exclusivity &  Extracted \\
Undesirable Consequences & \cite{b10}\\ 
Urgency (Scarcity) &  \cite{b20, b37}\\
Call to Action &  Extracted\\
False Dilemma &  Extracted\\
Assurance of Legitimacy & Extracted \\
Assurance of Security &  Extracted\\
Confidentiality Claims &  Extracted\\
Unsolicited Requests for Personal Information & Extracted \\
Appeal to Empathy/Altruism & \cite{b10} \\
Appeal to Values &  Extracted\\
Curiosity/Vagueness/Mystery & Extracted \\
Sense of Surprise/Confusion & Extracted \\
Reciprocation &  \cite{b20, b37}\\
Unity/Inclusivity/Sense of Community & \cite{b20} \\
Reinforcement of Positive Behavior &  Extracted\\
Appeal to Desires & Extracted \\
Motivational Language &  Extracted\\
Social Validation/Social Proof &  \cite{b20, b37} \\
\bottomrule % Bottom rule line
\end{tabular}
\end{table}

To identify and compile new semantic features, we first created a curated dataset of 300 phishing emails labeled by two experts in phishing analysis. This dataset was carefully assembled to include a
diverse range of phishing strategies and tactics. The sources from which  these emails were collected are:

\begin{itemize}
    \item Human-Generated Phishing Emails: Selected from the most recent ``Nazario'' and ``Nigerian Fraud'' collections~\cite{b33}, which are renowned repositories of real-world phishing emails that exhibit a variety of social engineering techniques;
    \item LLM-Generated Phishing Emails: collected by Greco et al.~\cite{b24}, which utilize advanced language models to generate realistic phishing emails that mimic human writing styles.
\end{itemize}

The dataset ensured comprehensive coverage of common and emerging phishing tactics by incorporating both human-generated and LLM-generated phishing emails.
Collecting the semantic features involved a meticulous analysis of each phishing email in the curated dataset being supported by experts and an automatic text analyzer (i.e., ChatGPT-4). Each email was input into the model with a carefully designed prompt that requested an in-depth examination of the email’s content, specifically focusing on the likelihood of it being a phishing attempt, the persuasion techniques employed, red flags, green flags, and potential countermeasures.
\\
From these responses, the identified persuasion techniques and red flags were first revised by experts and then documented. A validation process was undertaken to assess the significance and applicability of features not previously identified explicitly in the reviewed literature. This involved evaluating the consistency of these features across different phishing emails (validating their significance) and their effectiveness in deceiving recipients (validating their threat behavior). The validated features were then incorporated into the collection of semantic features, enhancing its coverage and utility. 

%\subsection{Descriptions of Cyri Semantic Features}
The semantic dataset provided to the Cyri LLM model comprises the feature names, an extensive description, and various examples. Table~\ref{tab:semantic-features-description} concisely describes each semantic feature we have identified.
\begin{table*}[ht]
\centering
\caption{Cyri Semantic Features Description}
\label{tab:semantic-features-description}
\begin{tabular}{|p{4cm}|p{12cm}|} % Two columns with specified width
\hline
\textbf{Semantic Feature} & \textbf{Description} \\ % Column headings

\hline
Authority & Impersonating authority figures to pressure recipients into complying with requests \\ [0.4em]
\hline
Impersonation of Trusted Entities & Mimicking trusted organizations deceives recipients into believing the email is genuine \\ [0.4em]
\hline
Instant Gratification & Offering tempting rewards prompts impulsive actions, exploiting the desire for quick benefits \\ [0.4em]
\hline
Exclusivity & Making recipients feel part of a select group increases compliance to avoid missing out on exclusive opportunities \\ [0.4em]
\hline
Undesirable Consequences & Threatening negative outcomes (e.g., account suspension) induces fear-driven responses without verification \\[0.4em]
\hline
Urgency (Scarcity) & Creating a sense of urgency forces recipients to act quickly, bypassing critical examination \\ [0.4em]
\hline
Call to Action & Clearly directing the recipient to perform a specific task (e.g., ``Click on the button below to verify your account'') \\[0.4em]
\hline
False Dilemma & Presenting only extreme choices pushes recipients toward the attacker’s desired course of action \\ [0.4em]
\hline
Assurance of Legitimacy & Convincing language and claims of authenticity are used to build trust and reduce suspicion \\ [0.4em]
\hline
Assurance of Security & Highlighting privacy and security reassures recipients \\ [0.4em]
\hline
Confidentiality Claims & Emphasizing the confidential nature of the information makes recipients feel they need to act without seeking advice or verification from others \\ [0.4em]
\hline
Unsolicited Requests for Personal Information & Requests for personal or financial data without prior authorization or legitimate justification \\ [0.4em]
\hline
Appeal to Empathy/Altruism & Exploiting the recipient's desire to help others or fulfill moral obligations \\ [0.4em]
\hline
Appeal to Values & Aligning with the recipient’s values builds trust and increases compliance with the attacker's requests \\ [0.4em]
\hline
Curiosity/Vagueness/Mystery & Vague or intriguing details induce recipients into taking action to satisfy their curiosity \\ [0.4em]
\hline
Sense of Surprise/Confusion & Unexpected scenarios create confusion, leading to unverified actions from recipients \\ [0.4em]
\hline
Reciprocation & Offering a benefit or favor creates a sense of obligation, leading recipients to fulfill follow-up requests \\ [0.4em]
\hline
Unity/Inclusivity/Sense of Community & Encouraging a sense of belonging or shared purpose motivates recipients to act in line with community goals \\ [0.4em]
\hline
Reinforcement of Positive Behavior & Praising the recipient for good behavior, and offering a reward reduces suspicion and increases engagement \\ [0.4em]
\hline
Appeal to Desires & Targeting personal goals or aspirations increases the chances of recipients ignoring warning signs \\ [0.4em]
\hline
Motivational Language & Evoking strong emotional responses, typically centered around desires for success, wealth, or security \\[0.4em]
\hline
Social Validation/Social Proof & Highlighting that others have taken the same action creates a sense of trust \\ [0.4em]
\hline
\end{tabular}
\end{table*}



\subsection{LISA: LLM-based Interactive Semantic Feature Analyzer}
\label{sec:lisa}

Traditional detection methods often rely on cloud-based services, which may not be suitable due to privacy concerns and dependence on external infrastructure. Deploying large LLMs raises privacy and data security concerns, as it requires sending sensitive emails to third-party servers. Users and organizations may be reluctant to adopt a system that necessitates sharing sensitive email content with external entities. 

To address these challenges, we developed a Python background process that performs in-depth phishing analysis and user query processing using a locally hosted LLM, specifically the Llama 3.1 8B model.
\\
We chose the Llama 3.1 8B model for its efficient reasoning capabilities and ability to handle contexts of up to 128,000 tokens, which is crucial when analyzing lengthy or complex emails. Additionally, this version is optimized for local deployment, balancing performance with resource demands, making it ideal for running on local machines without relying on cloud services.
\\
However, smaller models may not match the language comprehension of larger ones, making them more reliant on well-designed comprehensive prompts. For this reason, we defined an extensive prompt containing a large set of semantic social engineering techniques to improve the model's ability to detect diverse phishing tactics.

The LISA component performs two primary functions: analyzing incoming emails and handling user queries based on the analysis results during the conversation with the user. Each function is defined by a different prompt.

\subsubsection{Email Analysis Prompt}
\label{sed:eap}

The Email Analysis Prompt is a carefully constructed set of instructions designed to guide the LLM in performing a thorough analysis of an email to determine whether it is phishing or safe. The prompt employs several prompt engineering techniques to ensure that the LLM produces accurate, consistent, and user-friendly outputs. Due to the complexity and length of the prompt, we followed a Chain-of-Thought approach to make it more effective. Moreover, dissecting the prompt into its individual components ensures a thorough understanding of each aspect:\\

\noindent \textbf{1. Role Assignment}: ``\textit{You are an email phishing detector and analyzer. Your task is to identify whether an email is phishing or safe, explain why, and provide a detailed explanation.}''

The prompt begins by explicitly defining the LLM’s role as an ``email phishing detector and analyzer''. This sets the context for the model, focusing its capabilities on a specific task. By assigning a clear role, the LLM becomes ready to approach the subsequent instructions with the appropriate mindset.\\

\noindent \textbf{2. Presentation of the Email Content}: ``\textit{I want you to analyze the following email which could be phishing or safe: \{email\} I want you to tell me if this email is safe or phishing.}''

The prompt introduces the subject and the body of the email to be analyzed. Directly instructing the model to determine if the email is safe or phishing sets a clear objective.\\

\noindent \textbf{3. Base Reasoning Before Feature Consideration}: ``\textit{Use your base reasoning first to identify if the email is safe or phishing before considering the specified features.}''

The prompt instructs the LLM to use its inherent reasoning capabilities before relying on predefined features. This ensures that the model’s general understanding and language comprehension are utilized initially, potentially capturing nuances that this particular feature-based analysis might miss. \\

\noindent \textbf{4. Additional Information for Analysis and Guiding Questions:}: ``\textit{Here is additional information regarding the email for your analysis: \\
        1: Sender Information: \{sender\_email\} \\
        2: Google Safe Browsing API Result: \{google\_safe\_browsing\_output\}.\\
        3: AbuseIPDB Result: \{abuse\_ipdb\_output\}. \\
        - Is the sender domain or any URL found in the email reported as unsafe?\\
        - Identify if there is any impersonation of a well-known brand by comparing the sender’s email address with the claimed organization in the email content. If spoofing is detected, explain the inconsistencies. For example, if the email claims to be from 'Amazon' but the domain is not related to Amazon, highlight the inconsistency.\\
        Interpret the Google Safe Browsing API results: If threats are found, include the details. If no threats are found, note that.
        Interpret the AbuseIPDB results: If the domain is flagged as malicious, include the confidence score. If the domain is not flagged, note that as well. **Specify whether the domain refers to the sender or a link present in the email.** \\
        The sender's email address (\{sender\_email\}) is \{isSafeOutput\}.
        }'' \\
\indent The prompt provides external data such as the sender’s email, and results from security APIs enrich the context. These questions and instructions direct the LLM’s attention to specific aspects of the email, ensuring a comprehensive analysis. The LLM is instructed to determine whether the sender's domain or any URLs included in the email are reported as unsafe by the external APIs. The LLM is asked to compare the sender's email address with the organization mentioned in the email content to identify any impersonation. If spoofing is detected, for instance, the email claims to be from a reputable company like ``Amazon'' but the sender's domain does not match Amazon's official domain, the LLM should highlight these inconsistencies.
The variable \{isSafeOutput\} is set to indicate that the sender is ``present in the recipient's contact list and is trusted by the recipient'' or ``not present in the recipient's contact list''. This ensures that the model considers the trust relationship between the sender and the recipient. If the sender is recognized and trusted (i.e., in the contact list), the model lowers the phishing risk assessment for that email. \\

\noindent \textbf{5. Definition of Phishing/Safe Emails and Examples}: ``\textit{Here's a clear distinction for your analysis: \\ \\**Phishing Email**: Phishing emails are malicious attempts to deceive recipients into providing sensitive information or performing harmful actions.
            \\
            **Safe Email**: Safe emails are legitimate communications which typically have the following characteristics: Clear and concise language; Recognizable Sender Information; Content is relevant to the recipient's context (e.g., work-related updates, newsletters, transaction confirmations); Safe Links and Attachments. It includes, but is not limited to: routine communications like meeting requests, project updates, or a legitimate promotional email (Marketing email) from a company or organization offering products or services and it may contain offers, discounts, or promotional content.
            \\ \\
            I will provide examples of safe and phishing emails.\\
             This is a safe email:\\ \\
            \{example\_safe1\}.\\ \\
             This is a safe email:\\ \\ 
            \{example\_safe2\}.\\ \\ 
             This is a safe email:\\ \\
            \{example\_safe3\}.\\ \\
             This is a phishing email:\\ \\
            \{example\_phishing\}.
            }''
            
\indent Clear definitions and examples of phishing and safe emails are provided to help the model distinguish accurately between them for its classification process. We have added more examples of safe emails to improve the model’s ability to correctly identify safe emails and reduce false positives. \\
            
\noindent \textbf{6. Output Format Specification}: ``\textit{In the first line of the output, I want you to always respond with 'This email is [Likelihood Category] phishing ([percentage]\%)' or 'This email is [Likelihood Category] safe ([percentage]\%)' where you combine whether the email is phishing or safe with the likelihood description. \\ \\  Use these thresholds to categorize the likelihood of phishing: \\ \\
          - $0\% < x < 20\%$: Unlikely to be phishing \\ 
          - $20\% < y < 60\%$: Possibly phishing \\
          - $60\% < z < 90\%$: Likely phishing \\
          - $u > 90\%$: Almost certainly phishing \\ \\
    Also, categorize the likelihood of the email being safe:\\ \\
          - $0\% < x < 20\%$: Unlikely to be safe \\
          - $20\% < y < 60\%$: Possibly safe \\
          - $60\% < z < 90\%$: Likely safe \\
          - $u > 90\%$: Almost certainly safe}''

The prompt specifies the exact format for the output first line which will be composed by a percentage of the email being safe or phishing. By providing thresholds for likelihood categories, it ensures consistency in the model’s assessments and facilitates quantifiable evaluations. \\

\noindent \textbf{7. Feature Identification and Analysis}: ``\textit{You have to find the following features: \{features\}}''

With this step of the prompt we pass to the model the entire Cyri dataset of phishing semantic features composed by the features name, an extensive description and various examples to allow the model to perform a comprehensive analysis. \\

\noindent \textbf{8. Exact Output Format Instructions}: ``\textit{I want the output EXACTLY like this: \\  \\
- 'This email is [Likelihood Category] phishing ([percentage]\%)' or 'This email is [Likelihood Category] safe ([percentage]\%)'\\ \\
- Detailed Explanation: Provide a thorough explanation suitable for non-experts of why this email is phishing or safe. Clearly state your base reasoning for the classification, if spoofing is detected and if the sender is in the contact list or not (and how this impacts your assessment). Include references to specific elements of the email, the features of the email, the results from the Google Safe Browsing API, and the AbuseIPDB check, making sure to address how each contributes to your final assessment. \\ \\
- 'List of features found': [feature1; feature2; ...] **only the features present in the list below** for phishing emails. If the email is safe, define characteristics that make it safe (do not include any of the features present in the list below if the email is safe). \\ \\
- 'Analysis': \textless name of the feature \textgreater: '\textless specific part of the email \textgreater'. \textless explanation of why this part is linked to the feature \textgreater. **Only elements contained in 'List of features found' must be included**. \\ \\
- Countermeasures: where you offer practical recommendations on how the recipient should handle this email. These recommendations should be based on the identified risks and features, guiding the recipient on what actions to take next (e.g., verifying the sender, avoiding clicking on links, reporting the email as phishing, etc.).
}'' \\
\indent The prompt provides an exact template for the output, reducing variability and ensuring that all necessary components are included. Having a structured content analysis allows us to enhance the user interface design of the Cyri VAC component since it is possible to personalize the style of every section of the LLM analysis.\\

\noindent \textbf{9. Communication Style Guidelines}: ``\textit{Ensure the explanation is written in a conversational tone that directly addresses the recipient, making the analysis feel personalized. \\
Speak directly to the recipient using 'you' and 'your' when explaining why the email might be phishing or safe. \\
Provide clear, user-friendly explanations that are easy for non-experts to understand, directly addressing the recipient.}'' \\
\indent These instructions shape the tone and accessibility of the output, ensuring that it is appropriate for users without technical expertise.\\

\noindent \textbf{10. Feature Names and Weights}: ``\textit{Remember to use the exact names of the features listed below: \\ \\ \{list\_features\_names\} \\ \\ Weights are assigned to each feature indicating their importance in the classification: \\ \\ Authority, Impersonation of Trusted Entities: 0.6;
    Instant Gratification (False promise of reward): 0.9;
    Exclusivity: 0.8;
    Undesirable Consequences: 0.9;
    Urgency (Scarcity): 0.9;
    Call to Action: 0.9;
    False Dilemma: 0.8;
    Assurance of Legitimacy: 0.1;
    Assurance of Security: 0.3;
    Confidentiality Claims: 0.2;
    Unsolicited Requests for Personal Information/Financial Transactions: 0.9;
    Appeal to Empathy/Altruism: 0.4;
    Appeal to Values: 0.3;
    Curiosity/Vagueness/Mystery: 0.3;
    Sense of Surprise/Confusion: 0.3;
    Reciprocation: 0.3;
    Unity/Inclusivity/Sense of Community: 0.3;
    Reinforcement of Positive Behavior: 0.2;
    Appeal to Desires: 0.3;
    Motivational Language: 0.5;
    Social Validation/Social Proof: 0.5;
}''\\
\indent By specifying exact feature names, the prompt ensures consistency in terminology. Assigning weights to phishing features reflects their significance in identifying malicious emails. The weighting system guides the LLM’s reasoning process, emphasizing critical indicators.
Features assigned higher weights (e.g., 0.9) are considered strong indicators of phishing: Urgency (Scarcity); Undesirable Consequences; Unsolicited Requests. Features with moderate weights (e.g., 0.5 to 0.8) contribute significantly but may require the presence of additional indicators. Finally, features assigned lower weights (e.g., 0.1 to 0.4) may not strongly indicate phishing independently, but they can contribute to the overall assessment when combined with other indicators.


\subsubsection{Conversation Prompt}
The Conversation Prompt is designed to enable LISA to engage interactively with the user, providing detailed and context-aware responses to user queries based on prior email analysis. The prompt leverages previous interactions and analysis results to maintain continuity and relevance:
``\textit{You are an AI trained to analyze emails and interact with users to clarify and explain issues related to email security in simple terms. Use the analysis provided and the conversation history to inform your responses.\\
Given the user's query: (\{last\_user\_query\}), please provide a detailed and specific response that can help the user understand the steps to improve email security. For context I will give you also the initial instructions given to the model about how to analyze the email: \\ \\ \{initial\_prompt\} \\ \\ Here you can find the detailed analysis of the email generated by the model: \\  \{analysis\} \\ \\ All past interactions (questions and AI responses) related to this email analysis: \{conversation\_history\} \\ \\ Output only the response to this query: \{last\_user\_query\}
}''
\section{Cyri Usable Interface }
\label{sec:ui}
%include plugin e UI con electron app per easy deploy
The VAC application serves as the central hub of Cyri, providing a user-friendly interface for email analysis and interaction. It is implemented using web technology and Electron. 
%It enables the creation of a desktop application that feels native to the user’s operating system but is developed with the ease and speed of web technologies. 
By leveraging Electron’s capabilities, the application offers a cross-platform, user-friendly interface that integrates seamlessly with LISA and the email client plugin.
Several studies~\cite{b22, b23, b36} have consistently demonstrated that traditional warning dialogs are often ineffective in alerting users to phishing threats due to a lack of user understanding and the habituation effect. The habituation effect occurs when users become desensitized to repetitive visual stimuli, such as generic phishing warnings, leading them to ignore these alerts over time and diminishing their vigilance. This desensitization results in users dismissing important security warnings without adequate consideration, thereby increasing their susceptibility to phishing attacks.
To address this critical challenge, research on usable security has emphasized the importance of creating polymorphic warning interfaces in phishing detection. These interfaces dynamically alter their appearance and content each time they are presented to the user, aiming to reduce habituation and encourage users to pay closer attention to each alert. By introducing variability in warnings, the polymorphic approach enhances user engagement and prompts cautious behavior, making security warnings more effective.\\
Cyri thoughtfully incorporates these research findings to enhance its phishing detection efficacy. It employs a user-centered interface that moves beyond generic warning dialogs by providing clear, contextualized explanations of potential phishing threats within the user's email text and exchanges. By offering detailed analyses and actionable advice articulated in clear and understandable language without excessive technical terminology, Cyri enhances user understanding of the potential risk causes in the email text and encourages proactive behavior in managing suspicious emails. Moreover, Cyri offers actionable advice based on the identified risks and features, providing practical recommendations on how the recipient should handle the suspicious email, such as verifying the sender's identity through alternative channels, avoiding clicking on embedded links, or reporting the email to the appropriate authorities, following best practices in phishing management~\cite{b22}.
\\
Cyri utilizes dynamic visual feedback by changing the interface's background color and icons based on the analysis results. For instance, when an email is identified as phishing, the background color shifts toward red color (see Figure~\ref{fig: examplePhishingInterface}) to indicate danger, with the intensity increasing based on the phishing likelihood percentage and the ``Feature Score'', which depends on the weights of the features found.

\begin{figure*}[htbp]
  \centering
  \includegraphics[width=0.85\linewidth]{figures/PhishingExampleInterface.PNG}
  \caption{The VAC interface of Cyri in action: red background identifies a phishing mail, with semantic features highlighted in the email text (a) and the list below (b). Conversation with LISA happens on the right (c) through the query interface (d) or by audio (e)}
  \label{fig: examplePhishingInterface}
\end{figure*}

Conversely, if an email is deemed safe (see Figure~\ref{fig: exampleSafeInterface}), the background shifts to a calming blue, with the intensity depending on the safe likelihood percentage.

\begin{figure}[htbp]
  \centering
  \includegraphics[width=0.48\textwidth]{figures/SafeExampleInterface.PNG}
  \caption{The VAC interface of Cyri for a safe email}
  \label{fig: exampleSafeInterface}
\end{figure}

The Cyri VAC interface is designed to offer an intuitive and user-friendly experience. The interface presents a clean and organized layout divided into two primary sections. On the left side, users encounter email details (see Figure~\ref{fig: examplePhishingInterface}.a) and analysis results (see Figure~\ref{fig: examplePhishingInterface}.b), with a date picker feature allowing effortless navigation through emails by selecting specific dates. The e-mail text keeps the same indentation and style as the email client used. Phishing features detected in the email are highlighted by utilizing unique color combinations and text styles to draw attention to these elements (see Figure~\ref{fig: exampleFeaturesInterface}).
At the bottom, there is a list of all Cyri semantic features: users can select or remove a particular feature from the analysis by clicking on it, giving them control over the information presented if they consider a specific part of the analysis wrong or not accurate enough. 

On the right side column, the application focuses on facilitating user interaction and query handling. The conversation history displays all past interactions (see Figure~\ref{fig: examplePhishingInterface}.c) and responses related to the selected email analysis, maintaining continuity and context. Users can submit new questions for additional clarification, and when a query is entered (see Figure~\ref{fig: examplePhishingInterface}.d), the application processes it and monitors for a corresponding response generated by LISA. 

\begin{figure}[htbp]
  \centering
  \includegraphics[width=0.48\textwidth]{figures/PhishingEmailFeaturesAnalysis.PNG}
  \caption{Phishing Email Features Analysis Example}
  \label{fig: exampleFeaturesInterface}
\end{figure}

The user can select tags representing phishing semantic features to navigate directly to the portions of the email body and analysis characterized by that feature (navigating them by order of occurrences or severity). 
Cyri also incorporates a text-to-speech functionality (see Figure~\ref{fig: examplePhishingInterface}.e), allowing users to have the contents of the email and the analysis results read aloud, further enhancing accessibility and allowing them to listen to Cyri's recommendations while interacting with visual cues, taking advantage of a multi-modal interaction.
\\
By utilizing dynamic and context-specific visual cues and explanations, Cyri effectively reduces habituation. Each interaction feels unique and tailored to the specific situation, maintaining the user's engagement and attentiveness to security warnings. This approach aligns with best practices in user interface design for security applications~\cite{b22, b23}, where the goal is to balance alerting users to potential threats without causing alarm fatigue.
A video demonstration of Cyri is available in the GitHub repository reported in Appendix A.

\section{Validation}
\label{sec:validation}
%include la validazione dei risultati del modello, la generazione del dataset con analisi delle sue carattertistiche e considerazioni sulla bontà
The evaluation involved a series of tests designed to assess the model’s performance in classifying emails and identifying phishing features. The tests were structured to incrementally refine the model’s prompt and configuration presented in Section~\ref{sec:lisa} and iteratively analyze how changes affected outcomes. Key metrics such as the number of false positives, number of false negatives, Precision, Recall, and F1-score were computed to measure performance quantitatively.

\begin{itemize}
    \item \textbf{False Positives (FP)}: The number of safe emails incorrectly classified as phishing. This metric reflects instances where the model raises unnecessary alarms, causing inconvenience or mistrust.
    \item \textbf{False Negatives (FN)}: The number of phishing emails incorrectly classified as safe. This is critical as it represents missed detections, allowing potential threats to go unnoticed.
    \item \textbf{Precision}: The proportion of correctly identified phishing emails out of all emails the model classified as phishing. It measures the model's accuracy in predicting phishing emails and is calculated as: \[
    \text{Precision} = \frac{TP}{TP + FP}
    \]
    \item \textbf{Recall}: The proportion of correctly identified phishing emails out of all actual phishing emails. It assesses the model's ability to detect all phishing attempts and is calculated as:     \[
    \text{Recall} = \frac{TP}{TP + FN}.
    \]
    \item \textbf{F1-Score}: The harmonic mean of precision and recall, providing a single metric that balances false positives and false negatives. It is calculated as:     \[
    \text{F1-Score} = 2 \times \frac{\text{Precision} \times \text{Recall}}{\text{Precision} + \text{Recall}}.
    \]
\end{itemize}

\subsection{Validation Datasets}
\label{sec:valdataset}

To evaluate the model's effectiveness in detecting phishing emails, we required a comprehensive and diverse dataset of phishing emails that accurately reflect real-world phishing strategies. This dataset was meticulously crafted to encompass various topics and persuasion techniques, ensuring heterogeneity and coverage of most phishing scenarios.
\\
To create high-quality phishing emails, we utilized WormGPT V3.0~\cite{b34} since it is an advanced language model fine-tuned for crafting phishing emails (among other supported tasks). To guide the generation of phishing emails that embody the collected semantic features (both in isolation or in combination with others), we have utilized the following prompt:\\ \\
``\textit{Pretend to be a hacker planning a phishing campaign. Your goal is to generate one very detailed phishing email about this topic: \textless TOPIC\textgreater. You have to use fake American real names for the sender and recipient (example: John Smith, Sarah Thompson, Michael Garcia, Daniel Davis, James Wilson, etc) and you have to invent a phishing link URL to include in this email.''
\\ \\
The email must be characterized by the following persuasion technique:\\
\textless FEATURE \textgreater
} \\ \\
With this methodology, a total of 420 phishing emails have been generated, with 20 emails dedicated to each of the 21 identified Cyri semantic features. The emails were crafted to cover a wide array of topics relevant to each feature, enhancing the dataset’s heterogeneity. Each generated phishing email underwent a meticulous manual review by two experts to ensure that each email effectively embodied the specified semantic feature convincingly and eventually manually modified to make them more fitting to the targeted feature(s). Moreover, several additional characteristics for effective phishing attack generation, present in research papers that studied phishing characteristics, were taken into consideration such as Credibility \cite{b25}, Compatibility \cite{b25}, Personalization \cite{b35}, Contextual Relevance \cite{b35}  and Knowledge \cite{b10}, Reputation Exploitation \cite{b10}, Commitment and Consistency \cite{b20} and Liking \cite{b20}.

A meticulous process was undertaken to obtain accurate ground truth for identifying phishing features within these emails. The process involved leveraging ChatGPT-4o, supplemented by manual review and additions, to identify the specific features present in each phishing email.
\\ \\
To balance the dataset with an equal proportion of legitimate e-mails, and in the absence of an appropriate public dataset for them, a safe emails dataset composed of 420 legitimate emails was generated using ChatGPT-4. Prompts were crafted to create authentic, legitimate emails covering various topics, including:
\begin{itemize}
    \item Business Emails: Meeting requests, project updates, performance reviews, team announcements;
    \item Marketing Emails: Product launches, seasonal sales, newsletters;
    \item Personal Emails: Friendly catch-ups, event invitations, thank-you notes, congratulations messages, holiday greetings.
\end{itemize}

The two datasets were then merged into the final one, composed of 840 e-mails, heterogeneous in semantics and tactics, that will be made freely available as a public resource (the generation and curation process allow this step without incurring loss of privacy issues).

\subsection{Validation Results}
\label{sec:quantitativevalidation}

\subsubsection{Validating LLM choice}
In our comprehensive evaluation, we systematically tested the performance of the LLaMA 3.1 8B model to confirm its usage inside the LISA component. We used progressively refined prompts to assess their ability to detect phishing emails accurately.  \\ \\
In the initial phase of our evaluation, we utilized a straightforward prompt that asked whether an email was phishing or safe without providing any semantic features or detailed descriptions to guide the model's reasoning. This test aimed to establish a baseline for the model's inherent ability to classify emails based solely on its pre-trained knowledge and without additional context. The model's performance in this baseline test revealed moderate limitations (see Table~\ref{tab:test1}). There were 80 false positives, where legitimate safe emails were incorrectly classified as phishing, and 70 false negatives, where phishing emails were mistakenly identified as safe. This indicates that the model struggled to accurately differentiate between phishing and safe emails without explicit guidance. No evaluation of semantic features was possible in this case, as the problem was formulated as one of binary classification. No matter what, it confirmed our choice, dictated by security and privacy reasons, that even a small model like LLaMA 3.1 8B was a good base to build on our approach.

\begin{table}[ht]
\centering
\caption{Test 1: Classification Performance Metrics}
\label{tab:test1}
\begin{tabular}{lcccc}
\toprule
& \textbf{Precision} & \textbf{Recall} & \textbf{F1-score} & \textbf{Support} \\
\midrule
\textbf{Safe}       & 0.83 & 0.81 & 0.82 & 420 \\
\textbf{Phishing}   & 0.81 & 0.83 & 0.82 & 420 \\
\midrule
\textbf{Accuracy}   & & & 0.82 & 840 \\
\textbf{Macro Avg}  & 0.82 & 0.82 & 0.82 & 840 \\
\textbf{Weighted Avg} & 0.82 & 0.82 & 0.82 & 840 \\
\bottomrule
\end{tabular}
\end{table}


\subsubsection{Validating LISA phishing detection}
\label{sec:vallisa}

In the second test, we tried to enhance the model’s performance by incorporating semantic features of phishing emails into the prompt (see Section~\ref{sec:lisa} for details).
Results are visible in Table~\ref{tab:test2}. The number of false positives increased to 33,3\% (140), indicating that more safe emails were incorrectly classified as phishing. Conversely, the false negatives decreased to 9,5\% (40), showing an improvement in the model's ability to detect phishing emails. By introducing semantic features, the model's phishing recall improved slightly. However, this improvement in recall came at the expense of phishing precision, as evidenced by the increase in false positives. The model began over-identifying phishing characteristics in safe emails, leading to more legitimate emails being incorrectly flagged. This trade-off indicates that while the model became more sensitive to phishing indicators, it lacked the ability to adequately distinguish these features in the context of safe emails, highlighting the need for a more balanced approach.

\begin{table}[ht]
\centering
\caption{Test 2: Classification Performance Metrics}
\label{tab:test2}
\begin{tabular}{lcccc}
\toprule
& \textbf{Precision} & \textbf{Recall} & \textbf{F1-score} & \textbf{Support} \\
\midrule
\textbf{Safe}       & 0.875 & 0.667 & 0.757 & 420 \\
\textbf{Phishing}   & 0.731 & 0.905 & 0.809 & 420 \\
\midrule
\textbf{Accuracy}   & & & 0.786 & 840 \\
\textbf{Macro Avg}  & 0.803 & 0.786 & 0.783 & 840 \\
\textbf{Weighted Avg} & 0.803 & 0.786 & 0.783 & 840 \\
\bottomrule
\end{tabular}
\end{table}

Building on the previous tests, the third evaluation introduced weighted semantic features to the prompt. We assigned initial weights to each feature to reflect their importance in identifying phishing emails. Additionally, we included definitions of both phishing and safe emails and provided one example of each to guide the model exploiting 1-shot learning into the existing Chain-of-Tought approach. 
This test evidenced very good results in terms of recall (see Table~\ref{tab:test3}), with only two false negatives, indicating it almost perfectly identified all phishing emails. However, the false positives increased substantially to 42,9\% (180), resulting in many safe emails being incorrectly classified as phishing. The significant drop in precision and the high number of false positives indicated a problem with overfitting to the phishing characteristics. This imbalance suggested that the weighting scheme needed refinement to improve precision without compromising recall. 

\begin{table}[ht]
\centering
\caption{Test 3: Classification Performance Metrics}
\label{tab:test3}
\begin{tabular}{lcccc}
\toprule
& \textbf{Precision} & \textbf{Recall} & \textbf{F1-score} & \textbf{Support} \\
\midrule
\textbf{Safe}       & 0.992 & 0.571 & 0.725 & 420 \\
\textbf{Phishing}   & 0.699 & 0.995 & 0.822 & 420 \\
\midrule
\textbf{Accuracy}   & & & 0.783 & 840 \\
\textbf{Macro Avg}  & 0.846 & 0.783 & 0.773 & 840 \\
\textbf{Weighted Avg} & 0.846 & 0.783 & 0.773 & 840 \\
\bottomrule
\end{tabular}
\end{table}

In the final evaluation, we implemented a comprehensive and optimized prompt corresponding to the one utilized by the Cyri system. 
%but without considering the external APIs, sender information, and user's contact list.
To address the issues identified in the previous test, we adjusted the weights assigned to the features, aligning them more appropriately with their actual importance in phishing detection. We also provided multiple examples (3-shot learning) of safe emails to enhance the model's understanding of legitimate email patterns, having identified FP as the most problematic case. Finally, we enhanced the prompt to encourage the model to utilize its inherent reasoning abilities alongside all the detailed information provided, following public heuristics on how to make a prompt-based strategy more effective. This approach allowed the model to perform a more comprehensive analysis by ensuring that the model’s general understanding and language comprehension are utilized initially, potentially capturing nuances that this particular feature-based analysis might miss. 
These changes produced significant improvements in performance metrics. There were 13 false positives and 27 false negatives.

\begin{table}[ht]
\centering
\caption{Test 4: Classification Performance Metrics}
\label{tab:updated-classification-metrics}
\begin{tabular}{lcccc}
\toprule
& \textbf{Precision} & \textbf{Recall} & \textbf{F1-score} & \textbf{Support} \\
\midrule
\textbf{Safe}       & 0.938 & 0.969 & 0.953 & 420 \\
\textbf{Phishing}   & 0.968 & 0.936 & 0.952 & 420 \\
\midrule
\textbf{Accuracy}   & & & 0.952 & 840 \\
\textbf{Macro Avg}  & 0.953 & 0.952 & 0.952 & 840 \\
\textbf{Weighted Avg} & 0.953 & 0.952 & 0.952 & 840 \\
\bottomrule
\end{tabular}
\end{table} 

\subsubsection{Validating LISA phishing semantic features detection}
\label{sec:valsemfeat}

An in-depth validation was also conducted to evaluate the LISA's ability to identify specific phishing features by comparing the list of features found by LISA with the ground-truth list of features previously curated using ChatGPT-4o and manual annotations. The features were categorized based on the percentage of correct identifications into three classes of accuracy and sorted by decreasing accuracy. These results are visible in Figure~\ref{fig:testFeatures}.

\begin{figure}[htbp]
  \centering
  \includegraphics[width=0.48\textwidth]{figures/testFeatures.PNG}
  \caption{Final configuration for semantic features detection accuracy}
  \label{fig:testFeatures}
\end{figure}

High accuracy rates were observed for critical semantic features fundamental to phishing detection, such as Unsolicited Requests for Personal Information/Financial Transactions, Urgency (Scarcity), Authority/Impersonation of Trusted Entities, Call to Action, and Exclusivity.
In those cases, we rarely foresee doubts from human users (expert or non-expert) about the presence of these features in a suspicious e-mail, but taking advantage of their identification as a severe factor for not trusting the message.
\\
Medium accuracy rates were noted for features like Appeal to Empathy/Altruism, Motivational Language, Assurance of Security, Undesirable Consequences, Curiosity/Vagueness/Mystery, and Sense of Surprise/Confusion.
For these cases, severity is lower, and contextual factors may help a non-expert user assess the genuine or malicious nature of the messages. No matter what, Cyri alerts them, helping the human user confirm or deny their harmful nature through visual inspection and conversation.
\\
Features with lower accuracy rates included Appeal to Values, Social Validation/Social Proof, False Dilemma, Reinforcement of Positive Behavior, and Reciprocation.
We found this result coherent with a situation where these characteristics may also happen in genuine email or more subtle tentative. These areas need to be improved from an automatic detection point of view, and this represents a current limitation of our approach that needs further investigation. Possible mitigations may be providing visual alerts in the absence of these characteristics or providing for these characteristics the information on the degree of confidence of LISA for consideration and further analysis. 

\subsubsection{Overall results discussion}
\label{sec:overallvalidation}

The final model demonstrated a significant improvement in both precision and recall, achieving a high level of accuracy. The balanced weighting of features, comprehensive definitions, and multiple examples contributed to reducing both false positives and false negatives. The feature identification analysis revealed that the model was highly effective in detecting critical phishing features. High accuracy rates for key features affirm the model's effectiveness in accurately identifying phishing emails. Features with lower accuracy rates were less crucial for phishing detection, and their misidentification did not substantially impact the overall performance. However, these features might benefit from well-structured fine-tuning to enhance the model's comprehensiveness and explanatory capabilities.

\section{User Evaluation}
\label{sec:eval}
%include metodo, setup, ipotesi, risultati e grafici per i test con le due coorti di utenti
To assess the effectiveness and usability of Cyri as a tool for phishing detection and management from a human user, a user study was conducted involving ten participants with varying levels of expertise in computer security. This section details the methodology of the user study, the setup, and discusses the findings.

\subsection{Experiment Setup}
\label{sec:setup}

The study involved 10 participants, split equally between computer security
experts (meaning having at least two years of expertise and being knowledgeable of phishing tactics and techniques) and non-experts (meaning not being knowledgeable of phishing tactics and techniques but capable of using an email account).\\
The study duration for each participant was 60 minutes, split into 15 minutes of initial explanation on what are the most important features of Cyri and how to install it. This first step was then followed by two main tasks:
\begin{itemize}
    \item Controlled Email Identification Task: Participants were put in front of a preconfigured installation of Cyri and received five emails, four safe emails, and one phishing email sent by us. They were instructed to review these emails with Cyri and identify the phishing emails among them and the motivating factors for their final decision.
    This test was used both to let the participants gain confidence with Cyri usage and interface and as a controlled experiment where to evaluate how users interpreted and used the different results and functionalities Cyri exposes in a controlled situation equals for all of them. This step lasted, on average, from 10 to 15 minutes.
    \item Exploration with Personal Emails: Participants were then tasked to use Cyri to analyze their inbox emails from one personal account, such as those in their spam folder, unopened ones, or newly received messages. This allowed them to interact with the application in a context familiar to them and to assess its usefulness beyond the controlled task of provided emails, resulting in a more personal experience capable of letting them assess the degree of support they received from Cyri. This task lasted, on average, 25 minutes.
\end{itemize}
After completing the second task, participants were asked to compile a survey
comprising several questions aimed at evaluating Cyri’s effectiveness in assisting users in identifying phishing emails, usability and intuitiveness of the application interface, impact on users' understanding of phishing tactics, the likelihood of continued use, and preference for platform availability.
In particular, the questions proposed to the participants were the following:

\begin{enumerate}
    \item Are you a computer security expert? (Yes or No)
    \item How confident are you in identifying phishing emails without assistance?  (Scale 1 to 5)
    \item How useful was Cyri in helping you identify the phishing email?  (Scale 1 to 5)
    \item Did Cyri provide information that you wouldn't have noticed on your own? (Yes or No)
    \item How would you rate the overall usability of Cyri? (Scale 1 to 5)
    \item How intuitive did you find the Cyri interface?  (Scale 1 to 5)
    \item How much do you think using Cyri would improve your understanding of phishing tactics? (Scale 1 to 5)
    \item Would you use Cyri regularly as part of your email routine?  (Yes or No)
    \item Would you prefer if Cyri was available on your mobile phone instead of your computer?  (Yes or No)
\end{enumerate}

A final free text form allows the insertion of open comments and suggestions. Overall 5 minutes were dedicated on average to this activity.

\subsection{Results}
\label{sec:userresults}

We analyzed the survey results by splitting participants into their expertise level into two groups: Figure~\ref{fig:secexp} reports results for computer security experts while Figure~\ref{fig:nonsecexp} reports them for non-expert users. This distinction allowed us to understand how Cyri is perceived by users with different levels of expertise.

\begin{figure}[htbp]
  \centering
  \includegraphics[width=0.45\textwidth]{figures/SecurityExpertsBarchart.PNG}
  \caption{Security Experts Average Scores}
  \label{fig:secexp}
\end{figure}

\begin{figure}[htbp]
  \centering
  \includegraphics[width=0.45\textwidth]{figures/NonSecurityExpertsBarchart.PNG}
  \caption{Non-Security Experts Average Scores}
  \label{fig:nonsecexp}
\end{figure}

Cyri has been declared to be highly beneficial by non-expert participants, reporting generally low confidence in their ability to identify phishing emails without Cyri's assistance (Q2). All non-expert participants affirmed that Cyri provided information they would not have noticed on their own (Q4). This suggests that Cyri effectively highlights phishing indicators that might be overlooked, adding significant value in assisting them in identifying potential threats. 
Furthermore, non-experts provided very high ratings for both the usability (Q5) and intuitiveness (Q6) of Cyri. These results indicate that they found the application user-friendly and accessible. Non-experts believed using Cyri would significantly improve their understanding of phishing tactics (Q7), underscoring the application’s educational value.

Expert participants, even if they declared an expected good capability of identification and management of the phishing email with and without Cyri support (Q2), acknowledged that Cyri can enhance analysis capabilities by providing an additional layer of information (Q3). Interestingly, all expert participants also affirmed that Cyri provided information they would not have noticed on their own (Q4). This indicates that Cyri can uncover subtle phishing indicators and offer insights that even experienced users might overlook. Experts rated the overall usability (Q5) and intuitiveness (Q6) of Cyri highly, similar to non-experts, suggesting that the application is well-designed for users across different expertise levels. Moreover, experts believed that using Cyri could further improve their understanding of phishing tactics (Q7).
All participants expressed their willingness to use Cyri regularly as part of their email routine (Q8) and showed a clear preference for having it available also on their mobile devices (Q9).

\section{Discussion and Conclusion}
\vspace{-5pt}
Drawing ideas from the contextual integrity theory, we defined the notion of contextual privacy for users interacting with LLM-based conversation agents.
We proposed a framework, grounded in our contextual privacy formulation, that acts as an intermediary between the user and the agent, and carefully reformulates user prompts to preserve contextual privacy while preserving the utility.  






This work serves as an initial step in exploring privacy protection in user interactions with conversational agents. There are several directions that future research can further investigate. 
First, our framework may not be suitable for user prompts that require preserving exact content, such as document translation or verbatim summarization. For example, translating a legal document demands keeping the original content intact, making it challenging to reformulate while preserving contextual privacy. For such tasks, alternative approaches like using placeholders or pseudonyms for sensitive information could help protect privacy without compromising accuracy, though this is beyond our current implementation. 
Second, our framework relies on LLM-based assessment of privacy violations which, while effective for demonstrating the approach, lacks formal privacy guarantees and can be sensitive to the prompt. Future work could explore combining our contextual approach with deterministic rules or provable privacy properties. 
Third, while we demonstrate how users can adjust reformulations to balance privacy and utility, developing precise metrics to quantify this trade-off remains an open research challenge. This is particularly important as the relationship between privacy preservation and task effectiveness can vary significantly across different contexts and user preferences. 
Finally, while our evaluation using selected ShareGPT conversations demonstrates the potential of our approach, broader testing across diverse contexts and user groups would better establish the framework's general applicability.










\section{Conclusions}
\label{sec:conclusions}

This paper contributes Cyri, a significant advancement in applying AI-driven
solutions for phishing detection and management for human users. 
%Throughout this paper, we addressed the pressing issue of sophisticated phishing attacks that exploit human vulnerabilities through social engineering tactics. Traditional detection methods often fall short in identifying these advanced threats due to their reliance on known patterns and technical indicators. To overcome these limitations, we introduced
Through a systematic collection of semantic features and the tuning of a local LLM to detect them directly in email text, we created a system capable of detecting subtle cues indicative of malicious intent that traditional methods might overlook.
Extensive iterative testing and prompt refinement were conducted to optimize
Cyri's performance. By providing detailed explanations and engaging in conversational interactions through its visual and conversational interface, Cyri helps users understand why an email is potentially malicious and what steps they should take.
This addresses the human factors contributing to phishing success, such as lack of awareness and susceptibility to psychological manipulation.
Cyri achieved good results in detecting and explaining phishing emails and very positive results for efficacy and usability by ten experts and non-expert human users in a task-based evaluation.
In future work, we plan to mitigate the reported limitations of Cyri and explore its use of Cyri not only as a detection tool but also as an educational tool for training and awareness activities.

%with an accuracy of 95.24\%, a precision of 96.8\%, a recall of 93.56\%, and an F1-score of 95.15\%. The model demonstrated exceptional effectiveness in identifying critical phishing features, particularly those fundamental to phishing detection, such as Authority/Impersonation of Trusted Entities, Urgency, and Unsolicited Requests for Personal Information. \\
%A user study involving participants with varying levels of expertise further validated Cyri’s effectiveness. Both experts and non-experts found the application user-friendly and valuable in enhancing their ability to detect phishing emails. Participants noted that Cyri provided information they wouldn’t have noticed on their own, underscoring its educational value.




%\subsection{Figures and Tables}
%\paragraph{Positioning Figures and Tables} Place figures and tables at the top and 
%bottom of columns. Avoid placing them in the middle of columns. Large 
%figures and tables may span across both columns. Figure captions should be 
%below the figures; table heads should appear above the tables. Insert 
%figures and tables after they are cited in the text. Use the abbreviation 
%``Fig.~\ref{fig}'', even at the beginning of a sentence.

%\begin{table}[htbp]
%\caption{Table Type Styles}
%\begin{center}
%\begin{tabular}{|c|c|c|c|}
%\hline
%\textbf{Table}&\multicolumn{3}{|c|}{\textbf{Table Column Head}} \\
%\cline{2-4} 
%\textbf{Head} & \textbf{\textit{Table column subhead}}& %\textbf{\textit{Subhead}}& \textbf{\textit{Subhead}} \\
%\hline
%copy& More table copy$^{\mathrm{a}}$& &  \\
%\hline
%\multicolumn{4}{l}{$^{\mathrm{a}}$Sample of a Table footnote.}
%\end{tabular}
%\label{tab1}
%\end{center}
%\end{table}

%\begin{figure}[htbp]
%\centerline{\includegraphics[width=0.8\columnwidth]{fig1.png}}
%\caption{Example of a figure caption.}
%\label{fig}
%\end{figure}

%Figure Labels: Use 8 point Times New Roman for Figure labels. Use words 
%rather than symbols or abbreviations when writing Figure axis labels to 
%avoid confusing the reader. As an example, write the quantity 
%``Magnetization'', or ``Magnetization, M'', not just ``M''. If %including 
%units in the label, present them within parentheses. Do not label axes only 
%with units. In the example, write ``Magnetization (A/m)'' or ``Magnetization 
%\{A[m(1)]\}'', not just ``A/m''. Do not label axes with a ratio of 
%quantities and units. For example, write ``Temperature (K)'', not 
%``Temperature/K''.

%\section*{Acknowledgment}
%{
%\color{red}Remove this section for submission
%}
%The preferred spelling of the word ``acknowledgment'' in America is without 
%an ``e'' after the ``g''. Avoid the stilted expression ``one of us (R. B. 
%G.) thanks $\ldots$''. Instead, try ``R. B. G. thanks$\ldots$''. Put sponsor 
%acknowledgments in the unnumbered footnote on the first page.



\begin{thebibliography}{00}
\bibitem{b1} Bansla, N., Kunwar, S., and Jain, K. Social engineering: A technique for managing human behavior. (2019). doi:10.5281/zenodo.2580822. 
\bibitem{b2} Salahdine, F. and Kaabouch, N. Social engineering attacks: A survey.
Future Internet, 11 (2019). doi:10.3390/fi11040089.
\bibitem{b3} Sprinto.com. Social engineering statistics (2023). Available from: https://sprinto.com/blog/social-engineering-statistics.
\bibitem{b4}  Ferreira, A. and Teles, S. Persuasion: How phishing emails can influence users and bypass security measures. International Journal of Human-Computer Studies, 125 (2019), 19. Available from: https://www.sciencedirect.com/science/article/pii/S1071581918306827, doi:https://doi.org/10. 1016/j.ijhcs.2018.12.004.
\bibitem{b5}  Irwin, L. The 5 biggest phishing scams of all time. https://www.
itgovernance.eu/blog/en/the-5-biggest-phishing-scams-of-all-time
(2022).
\bibitem{b6} APWG. Phishing activity trends reports. https://apwg.org/
trendsreports/ (2023).
\bibitem{b7} Smith, G. Top phishing statistics for 2024: Latest figures and trends. https://www.stationx.net/phishing-statistics/ (2024)
\bibitem{b8} Hadnagy, C. Social Engineering: The Science of Human Hacking. John Wiley \& Sons (2018).
\bibitem{b9} Sawyer, B. D. and Hancock, P. A. Hacking the human: The prevalence
paradox in cybersecurity. Human Factors, 60 (2018), 597. doi:10.1177/
0018720818780472.
\bibitem{b10} Desolda, G., Ferro, L., Marrella, A., Costabile, M., and Catarci,
T. Human factors in phishing attacks: A systematic literature review. ACM
Computing Surveys, 54 (2022), 35. doi:10.1145/3469886.
\bibitem{b11} Taib, R., Yu, K., Berkovsky, S., Wiggins, M., and Bayl-Smith, P.
Social engineering and organisational dependencies in phishing attacks. In
Human-Computer Interaction – INTERACT 2019: 17th IFIP TC 13 International Conference, Paphos, Cyprus, September 2–6, 2019, Proceedings, Part
I, p. 564–584. Springer-Verlag, Berlin, Heidelberg (2019). ISBN 978-3-030-
29380-2. Available from: https://doi.org/10.1007/978-3-030-29381-9\_35,
doi:10.1007/978-3-030-29381-9\_35.
\bibitem{b12}  Muneer, A., Ali, R. F., Al-Sharai, A. A., and Fati, S. M. A survey
on phishing emails detection techniques. In 2021 International Conference on
Innovative Computing (ICIC), pp. 1–6 (2021). doi:10.1109/ICIC53490.2021.
9692960.
\bibitem{b13} Pujara, Er Purvi \& Chaudhari, M. (2019). Phishing Website Detection using Machine Learning : A Review. 
\bibitem{b14}  Altwaijry, N., Al-Turaiki, I., Alotaibi, R., and Alakeel, F. Advancing phishing email detection: A comparative study of deep learning models. Sensors, 24 (2024). Available from: https://www.mdpi.com/1424-8220/24/7/2077,
doi:10.3390/s24072077.
\bibitem{b15} Sahingoz, O. K., Buber, E., Demir, O., and Diri, B. Machine learning based phishing detection from urls. Expert Systems with Applications,
117 (2019), 345. Available from: https://www.sciencedirect.com/science/
article/pii/S0957417418306067, doi:https://doi.org/10.1016/j.eswa.
2018.09.029.
\bibitem{b16} Salahdine, F., El Mrabet, Z., and Kaabouch, N. Phishing attacks detection a machine learning-based approach. In 2021 IEEE 12th Annual Ubiquitous
Computing, Electronics Mobile Communication Conference (UEMCON), pp.
0250–0255 (2021). doi:10.1109/UEMCON53757.2021.9666627.
\bibitem{b17}  Avery, J., Almeshekah, M., and Spafford, E. Offensive deception in
computing. In 12th International Conference on Cyber Warfare and Security
(ICCWS’17), pp. 23–31 (2017).
\bibitem{b18} Beaman, C. and Isah, H. Anomaly detection in emails using machine
learning and header information. CoRR, abs/2203.10408 (2022). Available
from: https://arxiv.org/abs/2203.10408.
\bibitem{b19}  Salahdine, F. and Kaabouch, N. Social engineering attacks: A survey.
Future Internet, 11 (2019). doi:10.3390/fi11040089
\bibitem{b20} van der Laan, J. J. The semantics of persuasion: A case
study using phishing emails. https://unbscholar.lib.unb.ca/items/
716c75b0-cb66-4575-ada2-fedd2ea9ceeb (2021).
\bibitem{b21}  Vishwanath, A. The Weakest Link: How to Diagnose, Detect, and Defend
Users from Phishing. MIT Press (2022).
\bibitem{b22}  Buono, P., Desolda, G., Greco, F., and Piccinno, A. Let warnings
interrupt the interaction and explain: designing and evaluating phishing email
warnings (2023). doi:10.1145/3544549.3585802.
\bibitem{b23} Desolda, G., Aneke, J., Ardito, C., Lanzilotti, R., and Costabile, M.
Explanations in warning dialogs to help users defend against phishing attacks
(2023). doi:10.1016/j.ijhcs.2023.103056.
\bibitem{b24} Greco, F., Desolda, G., Esposito, A., and Carelli, A. David versus
goliath: Can machine learning detect llm-generated text? a case study in the
detection of phishing emails (2024).
\bibitem{b25} Heiding, F., Schneier, B., Vishwanath, A., Bernstein, J., and Park,
P. S. Devising and detecting phishing: Large language models vs. smaller
human models (2023). Available from: https://arxiv.org/abs/2308.12287,
arXiv:2308.12287.
\bibitem{b26} Roy, S. S., Thota, P., Naragam, K. V., and Nilizadeh, S. From chatbots
to phishbots? – preventing phishing scams created using chatgpt, google bard
and claude (2024). Available from: https://arxiv.org/abs/2310.19181,
arXiv:2310.19181.
\bibitem{b27} Koide, T., Fukushi, N., Nakano, H., and Chiba, D. Chatspamdetector:
Leveraging large language models for effective phishing email detection (2024).
Available from: https://arxiv.org/abs/2402.18093, arXiv:2402.18093.
\bibitem{b28} Li, Y., Huang, C., Deng, S., Lock, M. L., Cao, T., Oo, N., Lim, H. W., and Hooi, B. Knowphish: Large language models meet multimodal knowledge
graphs for enhancing reference-based phishing detection (2024). Available from:
https://arxiv.org/abs/2403.02253, arXiv:2403.02253.
\bibitem{b29} AI, M. Llama 3.1 8b instruct (2024). https://huggingface.co/meta-llama/Llama-3.1-8B-Instruct.
\bibitem{b30} Google Developers. Google Safe Browsing. Available at: https://developers.google.com/safe-browsing.
\bibitem{b31} AbuseIPDB. AbuseIPDB - IP Address Abuse Reports. Available at: https://www.abuseipdb.com/.
\bibitem{b32} Thunderbird Developers. WebExtension APIs for Thunderbird. Available at: https://webextension-api.thunderbird.net/en/stable/.
\bibitem{b33}  Unknown. Phishing email curated dataset (2023). Available from: https: //zenodo.org/records/8339691, doi:10.5281/zenodo.8339691.
\bibitem{b34} Eternanet. Wormgpt v3.0 (2023). Available from: https://flowgpt.com/
chat/wormgpt-v30.
\bibitem{b35} Hazell, J. Spear phishing with large language models (2023). Available from: https://arxiv.org/abs/2305.06972, arXiv:2305.06972.
\bibitem{b36} Anderson, B., Kirwan, B., Jenkins, J., Eargle, D., Howard, S., and
Vance, A. How polymorphic warnings reduce habituation in the brain. pp.
2883–2892 (2015). doi:10.1145/2702123.2702322.
\bibitem{b37}  Cialdini, R. B. Influence: The Psychology of Persuasion. Collins Business Essentials, Harper Collins, revised edn. (2009).
\bibitem{b38} Muhammad Usman Hadi, Qasem Al Tashi, Rizwan Qureshi, et al. Large Language Models: A Comprehensive Survey of its Applications, Challenges, Limitations, and Future Prospects. TechRxiv. September 05, 2024.
\bibitem{b39} Desolda, G., Greco, F., and Viganò, L. Apollo: A gpt-based tool to detect
phishing emails and generate explanations that warn users (2024). Available
from: https://arxiv.org/abs/2410.07997, arXiv:2410.07997.
\bibitem{b40} Graziano, G., Ucci, D., Bisio, F., and Oneto, L. Phishvision: A deep
learning based visual brand impersonation detector for identifying phishing
attacks. pp. 123–134. Springer Nature (2024).
\end{thebibliography}

\appendices
\section{Cyri Materials and Components}
All materials and source code of Cyri, including the created datasets for training, the video demonstration, and the results of different evaluation activities, are available at the following GitHub repository:\\\url{https://github.com/AntoReddy/Cyri}
\section{Topics Used for Phishing Email Generation and Example}
  We generated a total of 420 phishing emails, with 20 emails dedicated to each of
 the 21 identified semantic features. The emails were crafted to cover a wide array of topics relevant to each feature, enhancing the dataset’s heterogeneity.
 In the figure \ref{fig:topicsgenerationemails} are represented the topics utilized for the following features: Authority/Impersonation of Trusted Entities, Instant Gratification(False promise of reward), Exclusivity, Undesirable Consequences, Urgency (Scarcity) and Call to Action. 

 \begin{figure}[htbp]
  \centering
  \includegraphics[width=0.48\textwidth]{figures/topicsused.png}
  \caption{Topics for Generation of Phishing Emails, First Part}
  \label{fig:topicsgenerationemails}
\end{figure}
In the figure \ref{fig:topicsgenerationemails2} are represented the topics utilized for the following features:     False Dilemma,
    Assurance of Legitimacy,
    Assurance of Security,
    Confidentiality Claims,
    Unsolicited Requests for Personal Information/Financial Transactions,
    Appeal to Empathy/Altruism,
    Appeal to Values and
    Curiosity/Vagueness/Mystery.
\begin{figure}[htbp]
  \centering
  \includegraphics[width=0.48\textwidth]{figures/topicsused2.png}
  \caption{Topics for Generation of Phishing Emails, Second Part}
  \label{fig:topicsgenerationemails2}
\end{figure}
In the figure \ref{fig:topicsgenerationemails3} are represented the topics utilized for the following features:     Sense of Surprise/Confusion,
    Reciprocation,
    Unity/Inclusivity/Sense of Community,
    Reinforcement of Positive Behavior,
    Appeal to Desires,
    Motivational Language and
    Social Validation/Social Proof.
\begin{figure}[htbp]
  \centering
  \includegraphics[width=0.48\textwidth]{figures/topicsused3.png}
  \caption{Topics for Generation of Phishing Emails, Third Part}
  \label{fig:topicsgenerationemails3}
\end{figure}

 



\vspace{12pt}
%\color{red}
%IEEE conference templates contain guidance text for composing and formatting conference papers. Please ensure that all template text is removed from your conference paper prior to submission to the conference. Failure to remove the template text from your paper may result in your paper not being published.

\end{document}
