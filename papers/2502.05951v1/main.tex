\documentclass[compsoc, conference, a4paper, 10pt, times]{IEEEtran}
\IEEEoverridecommandlockouts
% The preceding line is only needed to identify funding in the first footnote. If that is unneeded, please comment it out.
\usepackage{cite}
\usepackage{amsmath,amssymb,amsfonts}
\usepackage{algorithmic}
\usepackage{graphicx}
\usepackage{textcomp}
\usepackage{bmpsize}
\usepackage{xcolor}
\usepackage{lipsum}


\usepackage{booktabs}

\usepackage[colorlinks=true,urlcolor=black]{hyperref}
\def\BibTeX{{\rm B\kern-.05em{\sc i\kern-.025em b}\kern-.08em
    T\kern-.1667em\lower.7ex\hbox{E}\kern-.125emX}}
\begin{document}

\title{Cyri: A Conversational AI-based Assistant for Supporting the Human User in Detecting and Responding to Phishing Attacks}
%{\footnotesize
%    Submission 352}  }

%\iffalse
\author{\IEEEauthorblockN{1\textsuperscript{st} Antonio La Torre}
\IEEEauthorblockA{
\textit{Sapienza University of Rome}\\
Rome, Italy \\
latorre.2067686@studenti.uniroma1.it}
\and
\IEEEauthorblockN{2\textsuperscript{nd} Marco Angelini}
\IEEEauthorblockA{
\textit{Link Campus University of Rome)}\\
Rome, Italy \\
m.angelini@unilink.it}
\iffalse
\and
\IEEEauthorblockN{3\textsuperscript{rd} Given Name Surname}
\IEEEauthorblockA{\textit{dept. name of organization (of Aff.)} \\
\textit{name of organization (of Aff.)}\\
City, Country \\
email address}
\and
\IEEEauthorblockN{4\textsuperscript{th} Given Name Surname}
\IEEEauthorblockA{\textit{dept. name of organization (of Aff.)} \\
\textit{name of organization (of Aff.)}\\
City, Country \\
email address}
\and
\IEEEauthorblockN{5\textsuperscript{th} Given Name Surname}
\IEEEauthorblockA{\textit{dept. name of organization (of Aff.)} \\
\textit{name of organization (of Aff.)}\\
City, Country \\
email address}
\fi
}
%\fi

\maketitle

\begin{abstract}
Phishing attacks have become increasingly sophisticated, exploiting human vulnerabilities through social engineering tactics to deceive individuals into revealing sensitive information. Traditional detection methods, such as blacklist-based and heuristic approaches, often fail to identify new or cleverly disguised phishing attempts due to their reliance on known patterns and technical indicators and not on the semantic characteristics of the attack attempt. This work introduces Cyri, an AI-powered conversational assistant designed to support a human user in detecting and analyzing phishing emails by leveraging Large Language Models (LLMs).\\
Cyri has been designed to scrutinize emails for semantic features used in phishing attacks, such as urgency, authority, impersonation, exclusivity, and undesirable consequences, using an approach that unifies features already established in the literature with others by Cyri features extraction methodology. Cyri can be directly plugged into a client mail or webmail, ensuring seamless integration with the user's email workflow while maintaining data privacy through local processing. By performing all analyses on the user's machine, Cyri eliminates the need to transmit sensitive email data over the internet, reducing security risks associated with external data breaches. 
The Cyri user interface has been designed to reduce habituation effects and enhance user engagement. It employs dynamic visual cues and context-specific explanations to keep users alert and informed while maintaining their experience in using emails.
Additionally, it allows users to explore identified malicious semantic features both through conversation with the agent and visual exploration, obtaining the advantages of both modalities for expert or non-expert users. It also allows users to keep track of the conversation, supports the user in solving additional questions on both computed features or new parts of the mail, and applies its detection on demand.\\
To evaluate Cyri's ability to distinguish between phishing and safe communications, we crafted a comprehensive dataset of 420 phishing emails and 420 legitimate emails. 
%The phishing emails were generated using WormGPT V3.0, guided by carefully designed prompts to embody specific semantic features. Each email underwent meticulous manual review to ensure authenticity and effectiveness.
Through iterative evaluation, Cyri was optimized to reach an accuracy of 95.24\%, a precision of 96.8\%, a recall of 93.56\%, and an F1-score of 95.15\%, demonstrating high effectiveness in identifying critical phishing semantic features fundamental to phishing detection. A user study involving 10 participants, both experts and non-experts, evaluated Cyri's effectiveness and usability in real use, where the participants tested the system on their mail accounts. Results indicated that Cyri significantly aided users in identifying phishing emails and enhanced their understanding of phishing tactics.
%The results underscore the potential of advanced AI models in enhancing cybersecurity measures by not only detecting phishing attempts but also educating users about the tactics employed by attackers. Cyri's integration of detailed explanations and user interaction positions it as both a detection tool and an educational resource, aiming to reduce human error and improve overall resilience against phishing attacks.

\end{abstract}

\begin{IEEEkeywords}
Usable security, Phishing, LLM, Mixed initiative, Security Awareness and Training
\end{IEEEkeywords}

\begin{figure}[ht]
    \centering
    \includegraphics[width=0.8\linewidth]{graphs/greater_than_naive.pdf}
    \vspace{0.5cm}
    \includegraphics[width=0.8\linewidth]{graphs/p1_bottom.png}
    \vspace{-5pt}
    \caption{\textcolor{positional}{Positional} vs.\ \textcolor{nonpositional}{non-positional} circuits. In a \textcolor{nonpositional}{non-positional} circuit, the same edges must be included at all positions. A \textcolor{positional}{positional} circuit can distinguish between the same edge at different positions. This specificity yields better trade-offs between circuit size and faithfulness. It can also increase both precision and recall.}
    \label{fig:p1}
    \vspace{-5pt}
\end{figure}

\section{Introduction}

\looseness=-1
A primary goal of interpretability research is to characterize the internal mechanisms in language models (LMs) and other NLP models. 
A core approach in this area is \textbf{circuit discovery}---identifying the minimal subgraph within the model's computation graph that performs a specific task \citep{olah2021framework,olah-mech}.
Typically, the nodes of a circuit represent model components (e.g., attention heads, neurons, or layers).
While manual circuit discovery methods can yield position-specific insights \citep{wanginterpretability,goldowskydill2023localizingmodelbehaviorpath}, \emph{automatic methods often overlook positional information}, treating components as uniformly relevant across all input token positions \citep{conmytowards,syed2023attribution}. 
For instance, if an attention head is included in a circuit, it is assumed to contribute equally to the computation for every position in the input sequence.
The assumption that circuits are position-invariant ignores the fact that different positions often require distinct computations.
By ignoring positions, current methods limit their ability to capture mechanisms that operate across positions, such as interactions between attention heads across positions.

In this study, we start by demonstrating that positional agnosticism is a significant limitation (\S\ref{sec:motivating}). Then, to address these limitations, we introduce a new approach: position-aware edge attribution patching (PEAP; \S\ref{sec:full_circ_discovery}; Figure~\ref{fig:p1}). Current approaches  assume that if an edge is in a circuit, then the same edge will be in the circuit at all positions, thus leading to low precision. It is also assumed that an edge's importance should be aggregated across positions before deciding whether it should be included in the circuit; this can lead to cancellation effects, and thus low recall. PEAP instead allows us to compute the importance of cross-positional edges, and separately evaluates edge importance at each position. We show that this leads to smaller and more accurate circuits; see Figure~\ref{fig:p1}.

Incorporating positional information into circuit discovery is straightforward when inputs have the same length and structure across examples.

However, realistic datasets are not nearly this templatic.
How, then, can we incorporate positional information into automatic circuit discovery?
To address this challenge, we propose \textbf{schemas} (\S\ref{sec:schema}). 
Schemas assign semantic labels to spans of tokens, enabling information aggregation across examples even when the spans differ in length.

For example, in the input ``The \textcolor{positional}{war} lasted from 1453 to 14\underline{\hspace{1em}},'' the span ``\textcolor{positional}{war}'' could be labeled as ``\emph{Subject}''.
This enables handling spans with varying lengths: the phrase ``\textcolor{positional}{Black Plague}'' in another example can be treated as a single positional span with the same role as ``\textcolor{positional}{war}''.
In experiments with two LMs and three tasks, we find that circuits discovered using schemas achieve a better trade-off between circuit size and faithfulness to the model's behavior than position-agnostic circuits.
Importantly, position-aware circuits offer a more precise representation of the underlying mechanisms, providing a more concise foundation for mechanistic explanations.

We also present a fully automated pipeline for schema generation and application (\S\ref{sec:schema-generation}) using large language models (LLMs). 
We evaluate the quality of the generated schemas and their utility in discovering position-aware circuits (\S\ref{sec:schema-eval}).
Notably, circuits derived using automatically generated and applied schemas achieve comparable faithfulness scores to circuits discovered with human-designed and manually applied schemas.

We summarize our contributions as follows:
\begin{itemize}[noitemsep,leftmargin=*,topsep=1pt,parsep=1pt]
    \item Introduce a position-aware circuit discovery method, which obtains better faithfulness than position-agnostic discovery.  
    \item Introduce dataset schemas,  facilitating positional circuit discovery in more naturalistic settings. 
    \item Develop an automated schema generation and application pipeline with LLMs, yielding schemas that are comparable to manually-annotated ones.
\end{itemize}

\section{Related Work}

\subsection{Penetration Depth Computation}

The computation of penetration depth often utilizes the Minkowski sum, a well-regarded algorithm documented in Dobkin et al.'s work~\cite{dobkin1993computing}.
This method shows high efficacy for convex shapes, where the simplicity of the objects allows for accurate and computationally efficient penetration depth calculations~\cite{dobkin1993computing,varadhan2004accurate,hachenberger2009exact}.
However, applying this algorithm to concave shapes significantly increases computational complexity.  
As a result, research has focused on developing methods to approximate penetration depth more efficiently for these shapes~\cite{cameron1997enhancing,bergen1999fast,lien2010simple,je2012polydepth}.  

Beyond the Minkowski sum, other methods have been explored, including techniques such as utilizing distance fields or the Hausdorff distance for penetration depth calculations~\cite{fisher2001fast,sud2006fast,SIG09HIST}.

Tang et al.\cite{SIG09HIST} devised an efficient algorithm for calculating the Hausdorff distance between two objects within a given error bound.
They also demonstrated that the proposed algorithm can accelerate penetration depth computation by focusing on the Hausdorff distance in overlapping regions of objects.
Building upon Tang et al.'s method, Zheng et al.\cite{zheng2022economic} improved performance using a BVH-based framework with a four-point strategy.
This method has achieved a performance improvement of up to 20 times compared to Tang et al.'s technique~\cite{SIG09HIST}.
\revision{A common feature of these works, known as the culling-based method, is computing bounds for the Hausdorff distance and reducing the search space.}

\revision{Although culling-based methods have demonstrated significant performance gains, they face challenges in leveraging parallel hardware.  
Updating and sharing bounds require synchronization, which is not well-suited for massively parallel processing architectures such as GPUs.}

\revision{In this work, we propose a GPU-based penetration depth algorithm that specifically accelerates two key processes using RT core technology:  
(1) detecting the overlapping volume and (2) calculating the Hausdorff distance.  
To highlight the effectiveness of our approach, we also implemented a CPU-based penetration depth algorithm based on Tang et al.~\cite{SIG09HIST} and Zheng et al.~\cite{zheng2022economic} for performance comparison.}

%In this work, we propose a GPU-based penetration depth algorithm, specifically accelerating two key processes with RT core technology:  
%first, detecting the overlapping volume; and second, calculating the Hausdorff distance.  
%To highlight our method's effectiveness, we also implemented a CPU-based penetration depth algorithm based on Tang et al.~\cite{SIG09HIST} and Zheng et al.~\cite{zheng2022economic} for performance comparison.  

%utilize a Hausdorff distance-based method for penetration depth calculation, accelerating two key processes with RT core technology: 

%A notable development in this area is the work of Tang et al., who devised algorithms for the rapid calculation of the Hausdorff distance between two objects~\cite{SIG09HIST}.
%Their approach is geared towards efficient penetration depth calculation by focusing on the Hausdorff distance in overlapping object regions.


%One of the algorithms for calculating penetration depth is the Minkowski sum.\cite{dobkin1993computing} The Minkowski sum is useful to compute penetration depth between two convex objects because they have a simple shape so the Minkowski sum can calculate accurate penetration depth with low computational complexity~\cite{dobkin1993computing,varadhan2004accurate,hachenberger2009exact}.
%However, applying the Minkowski sum in cases involving concave objects is challenging due to higher computational complexity. As a result, prior research has focused on quickly computing an approximate penetration depth in these scenarios~\cite{cameron1997enhancing,bergen1999fast,lien2010simple,je2012polydepth}.

%Instead of the Minkowski sum method, there have also been attempts to calculate the penetration depth based on the distance field or the vertices that make up the objects~\cite{fisher2001fast,sud2006fast,SIG09HIST}. Tang et al.~\cite{SIG09HIST} proposed the algorithms that compute the Hausdorff distance between two objects quickly and showed that can be computed penetration depth to fast by calculating the Hausdorff distance for the overlapping area of two objects.

%In this paper, the proposed method is based on Tang's methods~\cite{SIG09HIST}, and then partially divided into steps detecting overlapping volume step and the Hausdorff distance step. These two steps accelerated with RT core.

\subsection{Ray-Tracing Core-Based Acceleration}

\revision{Recent advancements in GPU technology have led to the integration of dedicated ray-tracing cores (RT cores), enabling hardware-accelerated ray tracing.
These cores optimize intersection checks between rays and objects, allowing for efficient ray-bounding box and ray-triangle intersection tests.
To utilize RT cores, various frameworks such as DXR, OptiX~\cite{parker2010optix}, and Vulkan have been developed.
RT cores primarily accelerate ray intersection tasks by efficiently traversing acceleration hierarchies.}

%The Ray-Tracing Core (RT-core) is NVIDIA’s specialized hardware for accelerating ray tracing.
%Integrated into RTX GPUs like the GeForce RTX series

%\revision{Notably, OptiX~\cite{parker2010optix} is an NVIDIA-supported SDK.
%The ray-tracing core primarily facilitates two tasks: building an acceleration hierarchy and executing ray intersection tasks with traversal.}
%OptiX operates by launching a CUDA kernel and invoking a ray generation ($ray_{gen}$) shader.
%Each CUDA core thread makes requests to the ray-tracing core, which then executes appropriate shaders like intersection ($IS$), miss($miss_{hit}$), closest hit($closest$), and any hit($any_{hit}$).
%Consequently, OptiX enables access to the results of ray-primitive intersection tests.

While the core purpose of ray-tracing cores is to expedite ray tracing, recent studies have explored their application beyond this traditional scope~\cite{wald2019rtx,zhu2022rtnn,thoman2022multi,nagarajan2023rt,meneses2023accelerating,morrical2023attribute}.
Wald et al.~\cite{wald2019rtx} addressed the problem of locating points within tetrahedra using ray-tracing cores.
Zhu et al.~\cite{zhu2022rtnn} introduced a K-Nearest Neighbor (K-NN) algorithm utilizing ray-tracing cores, achieving performance improvements of 2.2 to 65.0 times compared to previous GPU-based nearest neighbor search algorithms.
Thoman et al.~\cite{thoman2022multi} employed RT cores for Room Impulse Response (RIR) simulation.
Nagarajan et al.~\cite{nagarajan2023rt} implemented RT core-based DBSCAN clustering, reporting up to 4 times higher performance enhancement.
Meneses et al.~\cite{meneses2023accelerating} proposed RT core-based Range Minimum Query (RMQ) algorithms, yielding performance up to 2.3 times faster than existing RMQ methods.

\revision{
For collision detection between objects, one of the fundamental proximity queries, researchers have explored ray-tracing approaches even before the introduction of RT-core technology.
Hermann et al.\cite{hermann2008ray} proposed ray-tracing-based collision detection methods for deformable bodies.
Youngjun et al.\cite{kim2010mesh} applied Hermann's idea to medical simulation.
Lehericey et al.\cite{lehericey2015gpu} introduced GPU ray-traced collision detection algorithms for cloth simulation.
Recently, these approaches have been extended to utilize RT cores, as demonstrated by Sui et al.\cite{sui2024hardware}, who proposed discrete and continuous collision detection algorithms using ray-tracing cores.
Unlike these works, which focus on determining when and where collisions occur, our work focuses on calculating penetration depth.
}

In line with these advancements, this study uniquely applies RT-core technology to compute penetration depth, diverging from traditional ray-tracing applications and thereby contributing a novel approach to this field.

%\subsection{Collision detection with Ray-tracing}

%\YW{There have been attempts to apply the ray tracing approaches for collision detection~\cite{hermann2008ray, kim2010mesh, lehericey2015gpu}. Hermann et al~\cite{hermann2008ray} proposed ray tracing collision detection methods for deformable bodies. Youngjun et al~\cite{kim2010mesh} apply Hermann's idea for Medical simulation. Lehericey et al~\cite{lehericey2015gpu} introduced GPU ray-traced collision detection algorithms for cloth simulation.
%However, these methods proposed deformable objects, not solid- or discrete- objects, and there is no report about the result using ray tracing core yet. Therefore, our research implements the penetration depth algorithm with ray tracing methods and reports the benefit of ray tracing core.}

%\YW{Sui et al~\cite{sui2024hardware} proposed the method for discrete and continuous collision detection with ray tracing core. They generate the ray candidate as much as the edge of the source mesh and investigate the intersections to solve discrete collision detection. And also, to solve continuous collision detection, they build sphere-swept volumes with OptiX B-Spline curves using continuous trajectory points that are pre-computed and trace the ray samely. However, their implementation only considers non-penetrating collision, and because of that reason, there need for other approaches to compute penetration cases.}

%\YW{To address this issue, our approacthe has propose the methods to find penetration surface with RT core (that called RT-PPE). Not only that, our methods report the penetration depth as computing the Hausdorff distance between the penetration surface.}

%Recently, modern GPU embedded ray tracing core for hardware accelerated ray tracing.

%To access the ray tracing core, we can use DXR, OptiX~\cite{parker2010optix}, and Vulkan.
%Above all, OptiX~\cite{parker2010optix} is NVIDIA NVIDIA-supported SDK. The ray tracing core actually works about two tasks. One is a built acceleration hierarchy, and another is ray intersection task with traversal. Therefore OptiX launches one CUDA kernel and called $ray\_gen$ shader. Each CUDA core thread requests to ray tracing core, and then ray tracing core executes a suitable shader such as $IS$, $miss\_hit$, $closest$, $any\_hit$ shader.
%Finally, we can access ray-primitive intersection test results using OptiX shader.

%While the ray tracing core is designed for accelerating ray tracing, recent research tried using the ray tracing core for other purposes~\cite{wald2019rtx,zhu2022rtnn,thoman2022multi,nagarajan2023rt,meneses2023accelerating,morrical2023attribute}.
%%Beyond ray tracing
%Wald et al~\cite{wald2019rtx} solved the point in location of tetrahedron problem using ray tracing cores.
%%RTNN
%Zhu et al~\cite{zhu2022rtnn} proposed K-NN(K-Nearest Neighbor) algorithms using ray tracing cores. They achieved a performance of 2.2-65.0 times faster than prior GPU-based nearest neighbor search algorithms.
%%RIR Simulation
%Thoman et al~\cite{thoman2022multi} utilized the RT core to RIR(Room impulse response) simulation,
%%RT-DBSCAN
%Nagarajan et al~\cite{nagarajan2023rt} implemented DBSCAN clustering with RT core and achieved performance up to 4x times.
%%RTX-RMQ
%Meneses et al~\cite{meneses2023accelerating} proposed RT core-based RMQ(Range minimum query) algorithms, and they got performance up to 2.3x than state-of-the-art RMQ algorithms.

%%
%Similar to prior research, this study is distinguished by utilizing RT-core for computing penetration depth, as opposed to conventional ray tracing problems.



\section{Cyri Architectural Design}
\label{sec:design}
Cyri represents an innovative AI-powered conversational assistant designed to
help users detect and analyze phishing attacks within email communications. By leveraging a refined Large Language Model (LLM) through prompt
engineering and Chain of Thought techniques, Cyri provides users with detailed explanations of the suspicious features that could make an email potentially malicious, as well as the necessary countermeasures. Cyri’s architecture is composed of three main components:
\begin{enumerate}
\item LLM-based Interactive Semantic Analyzer component (LISA): it performs in-depth email analysis using a local Large Language Model (LLM) through APIs.
\item E-mail client plugin: it captures incoming emails and communicates with the
Electron application. A demonstrator is implemented for Thunderbird since It is an open-source email client that offers extensive customization capabilities through its support for add-ons~\cite{b32}.
\item Visual and Audio Conversational interface (VAC): it serves as the user interface, manages data storage, and allows a non-expert user to analyze the classification of e-mails, the main semantic reasons, and inquire more on it in both interactive visual and audio means. It is implemented as an Electron web application for generality and usability.
\end{enumerate}

Cyri continuously monitors incoming emails through the email client plugin. When a new, unseen email arrives, the plugin extracts essential data such as the sender's information, subject, body content, a flag indicating whether the sender is in the user's contacts, the message ID, and the timestamp. This data is then transmitted to both the VAC interface application and to the LISA component to perform an in-depth analysis of the email (locally hosted), specifically using the Meta-Llama-3.1-8B-Instruct model~\cite{b29}. LISA evaluates the email for semantic features such as urgency, authority, instant gratification, and others, all collected from the literature or extracted by LISA itself. LISA is helped by a sub-component for links checking that, using external APIs, specifically Google Safe Browsing~\cite{b30} and AbuseIPDB~\cite{b31}, enhance detection capabilities by checking only links and domains against known malicious entities and provides additional context to the semantic analysis.\\
Upon completion of the analysis, the results are stored in the VAC application and sent to the e-mail client plugin for e-mail text tagging and classification as ``Phishing'' or ``Safe'' (see Figure~\ref{fig:thunderbirdexample}). Finally, the user is presented with this information in the VAC interface and can explore it as explanations and converse with Cyri with a mix of visual cues and audio.

\begin{figure}[htbp]
  \centering
  \includegraphics[width=0.48\textwidth]{figures/ThunderbirdExample.PNG}
  \caption{Cyri email Plugin Example using the Thunderbird email client}
  \label{fig:thunderbirdexample}
\end{figure}

Figure~\ref{fig:EmailAnalysisArchitecture} illustrates the process by which Cyri analyzes the semantically tagged email in the VAC interface.

\begin{figure}[htbp]
  \centering
  \includegraphics[width=0.45\textwidth]{figures/EmailAnalysisArchitecture.pdf}
  \caption{Cyri Architecture and Data Flow for Email Analysis}
  \label{fig:EmailAnalysisArchitecture}
\end{figure}

The user can monitor newly tagged emails, interact with the detailed analysis through visual means, and issue further queries. User queries are processed interactively by the LISA component, which generates responses based on the conversation history with the user and initial semantic analysis of the e-mails, taking into account the user's inputs and questions. The whole process is visible in Figure~\ref{fig:ConversationArchitecture}.

%The diagram below  represents the interaction between the user and Cyri during the query process.

\begin{figure}[htbp]
  \centering
  \includegraphics[width=0.45\textwidth]{figures/ConversationArchitecture.pdf}
  \caption{User Interaction and Query Processing Flow}
  \label{fig:ConversationArchitecture}
\end{figure}

%The Thunderbird plugin communicates with the Electron application using HTTP POST and GET requests. This method is employed to send email data and to poll for analysis results, providing a reliable and straightforward communication channel.

%The Electron application and the Python background LLM processing component interact via the CyriShared folder. This shared storage serves as a repository for email data, analysis results, and user queries and responses. A strict file naming convention is followed to ensure proper identification and handling of files.

Finally, security and privacy are integral to Cyri’s design, especially given the sensitive
nature of email content. Data privacy is ensured thanks to local processing and
minimal external data sharing (URLs and domains) for safety checks.
All email analyses are conducted locally on the user’s machine. Using a locally
hosted LLM ensures that sensitive information remains within the
user’s environment, mitigating the risk of data breaches. The LISA component implements the Hugging Face Transformers library to load and utilize the Llama 3.1 8B model locally.

\section{Semantic Analysis of Phishing emails}
\label{sec:semantic}

Phishing attacks leverage sophisticated social engineering techniques to deceive recipients into disseminating sensitive information or performing actions compromising security. A critical aspect of enhancing phishing detection mechanisms involves
understanding and identifying the semantic features commonly employed in phishing emails. This section details the comprehensive collection of semantic features used and the instrumentation activities that guide the LISA component of Cyri in recognizing them into email text.

\subsection{Collection of Phishing Semantic Features}

This activity aims to create a robust dataset of phishing semantic features and emails containing them
that can inform the development of more effective detection algorithms and improve the capabilities of LLMs in identifying phishing
emails and recognizing the presence of these features in the text. The semantic phishing features collected in our dataset derive from two primary activities:

\begin{itemize}
    \item Literature-Identified Features review: it aims at collecting semantic features previously recognized and documented in academic and professional cybersecurity literature.
    \item Methodology-Extracted Features: it aims at extracting novel semantic features through a systematic extraction using an automatic text analyzer (i.e., ChatGPT-4) to each element of a comprehensive email phishing dataset created specifically for this purpose.
\end{itemize}

Table~\ref{tab:semantic-features} shows the results of these two activities, reporting the list of all the semantic features collected along with their corresponding source.

\begin{table}[ht]
\centering % This centers the table
\caption{Overview of Cyri Semantic Features} % Title of the table
\label{tab:semantic-features} % Label for referencing
\begin{tabular}{p{5.5cm}p{1.5cm}} % Two columns, both left-aligned
\toprule % Top rule line
\textbf{Semantic Feature} & \textbf{Source} \\ % Column headings
\midrule % Middle rule line
Authority &  \cite{b8, b20}\\
Impersonation of Trusted Entities &  Extracted\\ 
Instant Gratification &  \cite{b10}\\
Exclusivity &  Extracted \\
Undesirable Consequences & \cite{b10}\\ 
Urgency (Scarcity) &  \cite{b20, b37}\\
Call to Action &  Extracted\\
False Dilemma &  Extracted\\
Assurance of Legitimacy & Extracted \\
Assurance of Security &  Extracted\\
Confidentiality Claims &  Extracted\\
Unsolicited Requests for Personal Information & Extracted \\
Appeal to Empathy/Altruism & \cite{b10} \\
Appeal to Values &  Extracted\\
Curiosity/Vagueness/Mystery & Extracted \\
Sense of Surprise/Confusion & Extracted \\
Reciprocation &  \cite{b20, b37}\\
Unity/Inclusivity/Sense of Community & \cite{b20} \\
Reinforcement of Positive Behavior &  Extracted\\
Appeal to Desires & Extracted \\
Motivational Language &  Extracted\\
Social Validation/Social Proof &  \cite{b20, b37} \\
\bottomrule % Bottom rule line
\end{tabular}
\end{table}

To identify and compile new semantic features, we first created a curated dataset of 300 phishing emails labeled by two experts in phishing analysis. This dataset was carefully assembled to include a
diverse range of phishing strategies and tactics. The sources from which  these emails were collected are:

\begin{itemize}
    \item Human-Generated Phishing Emails: Selected from the most recent ``Nazario'' and ``Nigerian Fraud'' collections~\cite{b33}, which are renowned repositories of real-world phishing emails that exhibit a variety of social engineering techniques;
    \item LLM-Generated Phishing Emails: collected by Greco et al.~\cite{b24}, which utilize advanced language models to generate realistic phishing emails that mimic human writing styles.
\end{itemize}

The dataset ensured comprehensive coverage of common and emerging phishing tactics by incorporating both human-generated and LLM-generated phishing emails.
Collecting the semantic features involved a meticulous analysis of each phishing email in the curated dataset being supported by experts and an automatic text analyzer (i.e., ChatGPT-4). Each email was input into the model with a carefully designed prompt that requested an in-depth examination of the email’s content, specifically focusing on the likelihood of it being a phishing attempt, the persuasion techniques employed, red flags, green flags, and potential countermeasures.
\\
From these responses, the identified persuasion techniques and red flags were first revised by experts and then documented. A validation process was undertaken to assess the significance and applicability of features not previously identified explicitly in the reviewed literature. This involved evaluating the consistency of these features across different phishing emails (validating their significance) and their effectiveness in deceiving recipients (validating their threat behavior). The validated features were then incorporated into the collection of semantic features, enhancing its coverage and utility. 

%\subsection{Descriptions of Cyri Semantic Features}
The semantic dataset provided to the Cyri LLM model comprises the feature names, an extensive description, and various examples. Table~\ref{tab:semantic-features-description} concisely describes each semantic feature we have identified.
\begin{table*}[ht]
\centering
\caption{Cyri Semantic Features Description}
\label{tab:semantic-features-description}
\begin{tabular}{|p{4cm}|p{12cm}|} % Two columns with specified width
\hline
\textbf{Semantic Feature} & \textbf{Description} \\ % Column headings

\hline
Authority & Impersonating authority figures to pressure recipients into complying with requests \\ [0.4em]
\hline
Impersonation of Trusted Entities & Mimicking trusted organizations deceives recipients into believing the email is genuine \\ [0.4em]
\hline
Instant Gratification & Offering tempting rewards prompts impulsive actions, exploiting the desire for quick benefits \\ [0.4em]
\hline
Exclusivity & Making recipients feel part of a select group increases compliance to avoid missing out on exclusive opportunities \\ [0.4em]
\hline
Undesirable Consequences & Threatening negative outcomes (e.g., account suspension) induces fear-driven responses without verification \\[0.4em]
\hline
Urgency (Scarcity) & Creating a sense of urgency forces recipients to act quickly, bypassing critical examination \\ [0.4em]
\hline
Call to Action & Clearly directing the recipient to perform a specific task (e.g., ``Click on the button below to verify your account'') \\[0.4em]
\hline
False Dilemma & Presenting only extreme choices pushes recipients toward the attacker’s desired course of action \\ [0.4em]
\hline
Assurance of Legitimacy & Convincing language and claims of authenticity are used to build trust and reduce suspicion \\ [0.4em]
\hline
Assurance of Security & Highlighting privacy and security reassures recipients \\ [0.4em]
\hline
Confidentiality Claims & Emphasizing the confidential nature of the information makes recipients feel they need to act without seeking advice or verification from others \\ [0.4em]
\hline
Unsolicited Requests for Personal Information & Requests for personal or financial data without prior authorization or legitimate justification \\ [0.4em]
\hline
Appeal to Empathy/Altruism & Exploiting the recipient's desire to help others or fulfill moral obligations \\ [0.4em]
\hline
Appeal to Values & Aligning with the recipient’s values builds trust and increases compliance with the attacker's requests \\ [0.4em]
\hline
Curiosity/Vagueness/Mystery & Vague or intriguing details induce recipients into taking action to satisfy their curiosity \\ [0.4em]
\hline
Sense of Surprise/Confusion & Unexpected scenarios create confusion, leading to unverified actions from recipients \\ [0.4em]
\hline
Reciprocation & Offering a benefit or favor creates a sense of obligation, leading recipients to fulfill follow-up requests \\ [0.4em]
\hline
Unity/Inclusivity/Sense of Community & Encouraging a sense of belonging or shared purpose motivates recipients to act in line with community goals \\ [0.4em]
\hline
Reinforcement of Positive Behavior & Praising the recipient for good behavior, and offering a reward reduces suspicion and increases engagement \\ [0.4em]
\hline
Appeal to Desires & Targeting personal goals or aspirations increases the chances of recipients ignoring warning signs \\ [0.4em]
\hline
Motivational Language & Evoking strong emotional responses, typically centered around desires for success, wealth, or security \\[0.4em]
\hline
Social Validation/Social Proof & Highlighting that others have taken the same action creates a sense of trust \\ [0.4em]
\hline
\end{tabular}
\end{table*}



\subsection{LISA: LLM-based Interactive Semantic Feature Analyzer}
\label{sec:lisa}

Traditional detection methods often rely on cloud-based services, which may not be suitable due to privacy concerns and dependence on external infrastructure. Deploying large LLMs raises privacy and data security concerns, as it requires sending sensitive emails to third-party servers. Users and organizations may be reluctant to adopt a system that necessitates sharing sensitive email content with external entities. 

To address these challenges, we developed a Python background process that performs in-depth phishing analysis and user query processing using a locally hosted LLM, specifically the Llama 3.1 8B model.
\\
We chose the Llama 3.1 8B model for its efficient reasoning capabilities and ability to handle contexts of up to 128,000 tokens, which is crucial when analyzing lengthy or complex emails. Additionally, this version is optimized for local deployment, balancing performance with resource demands, making it ideal for running on local machines without relying on cloud services.
\\
However, smaller models may not match the language comprehension of larger ones, making them more reliant on well-designed comprehensive prompts. For this reason, we defined an extensive prompt containing a large set of semantic social engineering techniques to improve the model's ability to detect diverse phishing tactics.

The LISA component performs two primary functions: analyzing incoming emails and handling user queries based on the analysis results during the conversation with the user. Each function is defined by a different prompt.

\subsubsection{Email Analysis Prompt}
\label{sed:eap}

The Email Analysis Prompt is a carefully constructed set of instructions designed to guide the LLM in performing a thorough analysis of an email to determine whether it is phishing or safe. The prompt employs several prompt engineering techniques to ensure that the LLM produces accurate, consistent, and user-friendly outputs. Due to the complexity and length of the prompt, we followed a Chain-of-Thought approach to make it more effective. Moreover, dissecting the prompt into its individual components ensures a thorough understanding of each aspect:\\

\noindent \textbf{1. Role Assignment}: ``\textit{You are an email phishing detector and analyzer. Your task is to identify whether an email is phishing or safe, explain why, and provide a detailed explanation.}''

The prompt begins by explicitly defining the LLM’s role as an ``email phishing detector and analyzer''. This sets the context for the model, focusing its capabilities on a specific task. By assigning a clear role, the LLM becomes ready to approach the subsequent instructions with the appropriate mindset.\\

\noindent \textbf{2. Presentation of the Email Content}: ``\textit{I want you to analyze the following email which could be phishing or safe: \{email\} I want you to tell me if this email is safe or phishing.}''

The prompt introduces the subject and the body of the email to be analyzed. Directly instructing the model to determine if the email is safe or phishing sets a clear objective.\\

\noindent \textbf{3. Base Reasoning Before Feature Consideration}: ``\textit{Use your base reasoning first to identify if the email is safe or phishing before considering the specified features.}''

The prompt instructs the LLM to use its inherent reasoning capabilities before relying on predefined features. This ensures that the model’s general understanding and language comprehension are utilized initially, potentially capturing nuances that this particular feature-based analysis might miss. \\

\noindent \textbf{4. Additional Information for Analysis and Guiding Questions:}: ``\textit{Here is additional information regarding the email for your analysis: \\
        1: Sender Information: \{sender\_email\} \\
        2: Google Safe Browsing API Result: \{google\_safe\_browsing\_output\}.\\
        3: AbuseIPDB Result: \{abuse\_ipdb\_output\}. \\
        - Is the sender domain or any URL found in the email reported as unsafe?\\
        - Identify if there is any impersonation of a well-known brand by comparing the sender’s email address with the claimed organization in the email content. If spoofing is detected, explain the inconsistencies. For example, if the email claims to be from 'Amazon' but the domain is not related to Amazon, highlight the inconsistency.\\
        Interpret the Google Safe Browsing API results: If threats are found, include the details. If no threats are found, note that.
        Interpret the AbuseIPDB results: If the domain is flagged as malicious, include the confidence score. If the domain is not flagged, note that as well. **Specify whether the domain refers to the sender or a link present in the email.** \\
        The sender's email address (\{sender\_email\}) is \{isSafeOutput\}.
        }'' \\
\indent The prompt provides external data such as the sender’s email, and results from security APIs enrich the context. These questions and instructions direct the LLM’s attention to specific aspects of the email, ensuring a comprehensive analysis. The LLM is instructed to determine whether the sender's domain or any URLs included in the email are reported as unsafe by the external APIs. The LLM is asked to compare the sender's email address with the organization mentioned in the email content to identify any impersonation. If spoofing is detected, for instance, the email claims to be from a reputable company like ``Amazon'' but the sender's domain does not match Amazon's official domain, the LLM should highlight these inconsistencies.
The variable \{isSafeOutput\} is set to indicate that the sender is ``present in the recipient's contact list and is trusted by the recipient'' or ``not present in the recipient's contact list''. This ensures that the model considers the trust relationship between the sender and the recipient. If the sender is recognized and trusted (i.e., in the contact list), the model lowers the phishing risk assessment for that email. \\

\noindent \textbf{5. Definition of Phishing/Safe Emails and Examples}: ``\textit{Here's a clear distinction for your analysis: \\ \\**Phishing Email**: Phishing emails are malicious attempts to deceive recipients into providing sensitive information or performing harmful actions.
            \\
            **Safe Email**: Safe emails are legitimate communications which typically have the following characteristics: Clear and concise language; Recognizable Sender Information; Content is relevant to the recipient's context (e.g., work-related updates, newsletters, transaction confirmations); Safe Links and Attachments. It includes, but is not limited to: routine communications like meeting requests, project updates, or a legitimate promotional email (Marketing email) from a company or organization offering products or services and it may contain offers, discounts, or promotional content.
            \\ \\
            I will provide examples of safe and phishing emails.\\
             This is a safe email:\\ \\
            \{example\_safe1\}.\\ \\
             This is a safe email:\\ \\ 
            \{example\_safe2\}.\\ \\ 
             This is a safe email:\\ \\
            \{example\_safe3\}.\\ \\
             This is a phishing email:\\ \\
            \{example\_phishing\}.
            }''
            
\indent Clear definitions and examples of phishing and safe emails are provided to help the model distinguish accurately between them for its classification process. We have added more examples of safe emails to improve the model’s ability to correctly identify safe emails and reduce false positives. \\
            
\noindent \textbf{6. Output Format Specification}: ``\textit{In the first line of the output, I want you to always respond with 'This email is [Likelihood Category] phishing ([percentage]\%)' or 'This email is [Likelihood Category] safe ([percentage]\%)' where you combine whether the email is phishing or safe with the likelihood description. \\ \\  Use these thresholds to categorize the likelihood of phishing: \\ \\
          - $0\% < x < 20\%$: Unlikely to be phishing \\ 
          - $20\% < y < 60\%$: Possibly phishing \\
          - $60\% < z < 90\%$: Likely phishing \\
          - $u > 90\%$: Almost certainly phishing \\ \\
    Also, categorize the likelihood of the email being safe:\\ \\
          - $0\% < x < 20\%$: Unlikely to be safe \\
          - $20\% < y < 60\%$: Possibly safe \\
          - $60\% < z < 90\%$: Likely safe \\
          - $u > 90\%$: Almost certainly safe}''

The prompt specifies the exact format for the output first line which will be composed by a percentage of the email being safe or phishing. By providing thresholds for likelihood categories, it ensures consistency in the model’s assessments and facilitates quantifiable evaluations. \\

\noindent \textbf{7. Feature Identification and Analysis}: ``\textit{You have to find the following features: \{features\}}''

With this step of the prompt we pass to the model the entire Cyri dataset of phishing semantic features composed by the features name, an extensive description and various examples to allow the model to perform a comprehensive analysis. \\

\noindent \textbf{8. Exact Output Format Instructions}: ``\textit{I want the output EXACTLY like this: \\  \\
- 'This email is [Likelihood Category] phishing ([percentage]\%)' or 'This email is [Likelihood Category] safe ([percentage]\%)'\\ \\
- Detailed Explanation: Provide a thorough explanation suitable for non-experts of why this email is phishing or safe. Clearly state your base reasoning for the classification, if spoofing is detected and if the sender is in the contact list or not (and how this impacts your assessment). Include references to specific elements of the email, the features of the email, the results from the Google Safe Browsing API, and the AbuseIPDB check, making sure to address how each contributes to your final assessment. \\ \\
- 'List of features found': [feature1; feature2; ...] **only the features present in the list below** for phishing emails. If the email is safe, define characteristics that make it safe (do not include any of the features present in the list below if the email is safe). \\ \\
- 'Analysis': \textless name of the feature \textgreater: '\textless specific part of the email \textgreater'. \textless explanation of why this part is linked to the feature \textgreater. **Only elements contained in 'List of features found' must be included**. \\ \\
- Countermeasures: where you offer practical recommendations on how the recipient should handle this email. These recommendations should be based on the identified risks and features, guiding the recipient on what actions to take next (e.g., verifying the sender, avoiding clicking on links, reporting the email as phishing, etc.).
}'' \\
\indent The prompt provides an exact template for the output, reducing variability and ensuring that all necessary components are included. Having a structured content analysis allows us to enhance the user interface design of the Cyri VAC component since it is possible to personalize the style of every section of the LLM analysis.\\

\noindent \textbf{9. Communication Style Guidelines}: ``\textit{Ensure the explanation is written in a conversational tone that directly addresses the recipient, making the analysis feel personalized. \\
Speak directly to the recipient using 'you' and 'your' when explaining why the email might be phishing or safe. \\
Provide clear, user-friendly explanations that are easy for non-experts to understand, directly addressing the recipient.}'' \\
\indent These instructions shape the tone and accessibility of the output, ensuring that it is appropriate for users without technical expertise.\\

\noindent \textbf{10. Feature Names and Weights}: ``\textit{Remember to use the exact names of the features listed below: \\ \\ \{list\_features\_names\} \\ \\ Weights are assigned to each feature indicating their importance in the classification: \\ \\ Authority, Impersonation of Trusted Entities: 0.6;
    Instant Gratification (False promise of reward): 0.9;
    Exclusivity: 0.8;
    Undesirable Consequences: 0.9;
    Urgency (Scarcity): 0.9;
    Call to Action: 0.9;
    False Dilemma: 0.8;
    Assurance of Legitimacy: 0.1;
    Assurance of Security: 0.3;
    Confidentiality Claims: 0.2;
    Unsolicited Requests for Personal Information/Financial Transactions: 0.9;
    Appeal to Empathy/Altruism: 0.4;
    Appeal to Values: 0.3;
    Curiosity/Vagueness/Mystery: 0.3;
    Sense of Surprise/Confusion: 0.3;
    Reciprocation: 0.3;
    Unity/Inclusivity/Sense of Community: 0.3;
    Reinforcement of Positive Behavior: 0.2;
    Appeal to Desires: 0.3;
    Motivational Language: 0.5;
    Social Validation/Social Proof: 0.5;
}''\\
\indent By specifying exact feature names, the prompt ensures consistency in terminology. Assigning weights to phishing features reflects their significance in identifying malicious emails. The weighting system guides the LLM’s reasoning process, emphasizing critical indicators.
Features assigned higher weights (e.g., 0.9) are considered strong indicators of phishing: Urgency (Scarcity); Undesirable Consequences; Unsolicited Requests. Features with moderate weights (e.g., 0.5 to 0.8) contribute significantly but may require the presence of additional indicators. Finally, features assigned lower weights (e.g., 0.1 to 0.4) may not strongly indicate phishing independently, but they can contribute to the overall assessment when combined with other indicators.


\subsubsection{Conversation Prompt}
The Conversation Prompt is designed to enable LISA to engage interactively with the user, providing detailed and context-aware responses to user queries based on prior email analysis. The prompt leverages previous interactions and analysis results to maintain continuity and relevance:
``\textit{You are an AI trained to analyze emails and interact with users to clarify and explain issues related to email security in simple terms. Use the analysis provided and the conversation history to inform your responses.\\
Given the user's query: (\{last\_user\_query\}), please provide a detailed and specific response that can help the user understand the steps to improve email security. For context I will give you also the initial instructions given to the model about how to analyze the email: \\ \\ \{initial\_prompt\} \\ \\ Here you can find the detailed analysis of the email generated by the model: \\  \{analysis\} \\ \\ All past interactions (questions and AI responses) related to this email analysis: \{conversation\_history\} \\ \\ Output only the response to this query: \{last\_user\_query\}
}''
\section{Cyri Usable Interface }
\label{sec:ui}
%include plugin e UI con electron app per easy deploy
The VAC application serves as the central hub of Cyri, providing a user-friendly interface for email analysis and interaction. It is implemented using web technology and Electron. 
%It enables the creation of a desktop application that feels native to the user’s operating system but is developed with the ease and speed of web technologies. 
By leveraging Electron’s capabilities, the application offers a cross-platform, user-friendly interface that integrates seamlessly with LISA and the email client plugin.
Several studies~\cite{b22, b23, b36} have consistently demonstrated that traditional warning dialogs are often ineffective in alerting users to phishing threats due to a lack of user understanding and the habituation effect. The habituation effect occurs when users become desensitized to repetitive visual stimuli, such as generic phishing warnings, leading them to ignore these alerts over time and diminishing their vigilance. This desensitization results in users dismissing important security warnings without adequate consideration, thereby increasing their susceptibility to phishing attacks.
To address this critical challenge, research on usable security has emphasized the importance of creating polymorphic warning interfaces in phishing detection. These interfaces dynamically alter their appearance and content each time they are presented to the user, aiming to reduce habituation and encourage users to pay closer attention to each alert. By introducing variability in warnings, the polymorphic approach enhances user engagement and prompts cautious behavior, making security warnings more effective.\\
Cyri thoughtfully incorporates these research findings to enhance its phishing detection efficacy. It employs a user-centered interface that moves beyond generic warning dialogs by providing clear, contextualized explanations of potential phishing threats within the user's email text and exchanges. By offering detailed analyses and actionable advice articulated in clear and understandable language without excessive technical terminology, Cyri enhances user understanding of the potential risk causes in the email text and encourages proactive behavior in managing suspicious emails. Moreover, Cyri offers actionable advice based on the identified risks and features, providing practical recommendations on how the recipient should handle the suspicious email, such as verifying the sender's identity through alternative channels, avoiding clicking on embedded links, or reporting the email to the appropriate authorities, following best practices in phishing management~\cite{b22}.
\\
Cyri utilizes dynamic visual feedback by changing the interface's background color and icons based on the analysis results. For instance, when an email is identified as phishing, the background color shifts toward red color (see Figure~\ref{fig: examplePhishingInterface}) to indicate danger, with the intensity increasing based on the phishing likelihood percentage and the ``Feature Score'', which depends on the weights of the features found.

\begin{figure*}[htbp]
  \centering
  \includegraphics[width=0.85\linewidth]{figures/PhishingExampleInterface.PNG}
  \caption{The VAC interface of Cyri in action: red background identifies a phishing mail, with semantic features highlighted in the email text (a) and the list below (b). Conversation with LISA happens on the right (c) through the query interface (d) or by audio (e)}
  \label{fig: examplePhishingInterface}
\end{figure*}

Conversely, if an email is deemed safe (see Figure~\ref{fig: exampleSafeInterface}), the background shifts to a calming blue, with the intensity depending on the safe likelihood percentage.

\begin{figure}[htbp]
  \centering
  \includegraphics[width=0.48\textwidth]{figures/SafeExampleInterface.PNG}
  \caption{The VAC interface of Cyri for a safe email}
  \label{fig: exampleSafeInterface}
\end{figure}

The Cyri VAC interface is designed to offer an intuitive and user-friendly experience. The interface presents a clean and organized layout divided into two primary sections. On the left side, users encounter email details (see Figure~\ref{fig: examplePhishingInterface}.a) and analysis results (see Figure~\ref{fig: examplePhishingInterface}.b), with a date picker feature allowing effortless navigation through emails by selecting specific dates. The e-mail text keeps the same indentation and style as the email client used. Phishing features detected in the email are highlighted by utilizing unique color combinations and text styles to draw attention to these elements (see Figure~\ref{fig: exampleFeaturesInterface}).
At the bottom, there is a list of all Cyri semantic features: users can select or remove a particular feature from the analysis by clicking on it, giving them control over the information presented if they consider a specific part of the analysis wrong or not accurate enough. 

On the right side column, the application focuses on facilitating user interaction and query handling. The conversation history displays all past interactions (see Figure~\ref{fig: examplePhishingInterface}.c) and responses related to the selected email analysis, maintaining continuity and context. Users can submit new questions for additional clarification, and when a query is entered (see Figure~\ref{fig: examplePhishingInterface}.d), the application processes it and monitors for a corresponding response generated by LISA. 

\begin{figure}[htbp]
  \centering
  \includegraphics[width=0.48\textwidth]{figures/PhishingEmailFeaturesAnalysis.PNG}
  \caption{Phishing Email Features Analysis Example}
  \label{fig: exampleFeaturesInterface}
\end{figure}

The user can select tags representing phishing semantic features to navigate directly to the portions of the email body and analysis characterized by that feature (navigating them by order of occurrences or severity). 
Cyri also incorporates a text-to-speech functionality (see Figure~\ref{fig: examplePhishingInterface}.e), allowing users to have the contents of the email and the analysis results read aloud, further enhancing accessibility and allowing them to listen to Cyri's recommendations while interacting with visual cues, taking advantage of a multi-modal interaction.
\\
By utilizing dynamic and context-specific visual cues and explanations, Cyri effectively reduces habituation. Each interaction feels unique and tailored to the specific situation, maintaining the user's engagement and attentiveness to security warnings. This approach aligns with best practices in user interface design for security applications~\cite{b22, b23}, where the goal is to balance alerting users to potential threats without causing alarm fatigue.
A video demonstration of Cyri is available in the GitHub repository reported in Appendix A.

\section{Validation}
\label{sec:validation}
%include la validazione dei risultati del modello, la generazione del dataset con analisi delle sue carattertistiche e considerazioni sulla bontà
The evaluation involved a series of tests designed to assess the model’s performance in classifying emails and identifying phishing features. The tests were structured to incrementally refine the model’s prompt and configuration presented in Section~\ref{sec:lisa} and iteratively analyze how changes affected outcomes. Key metrics such as the number of false positives, number of false negatives, Precision, Recall, and F1-score were computed to measure performance quantitatively.

\begin{itemize}
    \item \textbf{False Positives (FP)}: The number of safe emails incorrectly classified as phishing. This metric reflects instances where the model raises unnecessary alarms, causing inconvenience or mistrust.
    \item \textbf{False Negatives (FN)}: The number of phishing emails incorrectly classified as safe. This is critical as it represents missed detections, allowing potential threats to go unnoticed.
    \item \textbf{Precision}: The proportion of correctly identified phishing emails out of all emails the model classified as phishing. It measures the model's accuracy in predicting phishing emails and is calculated as: \[
    \text{Precision} = \frac{TP}{TP + FP}
    \]
    \item \textbf{Recall}: The proportion of correctly identified phishing emails out of all actual phishing emails. It assesses the model's ability to detect all phishing attempts and is calculated as:     \[
    \text{Recall} = \frac{TP}{TP + FN}.
    \]
    \item \textbf{F1-Score}: The harmonic mean of precision and recall, providing a single metric that balances false positives and false negatives. It is calculated as:     \[
    \text{F1-Score} = 2 \times \frac{\text{Precision} \times \text{Recall}}{\text{Precision} + \text{Recall}}.
    \]
\end{itemize}

\subsection{Validation Datasets}
\label{sec:valdataset}

To evaluate the model's effectiveness in detecting phishing emails, we required a comprehensive and diverse dataset of phishing emails that accurately reflect real-world phishing strategies. This dataset was meticulously crafted to encompass various topics and persuasion techniques, ensuring heterogeneity and coverage of most phishing scenarios.
\\
To create high-quality phishing emails, we utilized WormGPT V3.0~\cite{b34} since it is an advanced language model fine-tuned for crafting phishing emails (among other supported tasks). To guide the generation of phishing emails that embody the collected semantic features (both in isolation or in combination with others), we have utilized the following prompt:\\ \\
``\textit{Pretend to be a hacker planning a phishing campaign. Your goal is to generate one very detailed phishing email about this topic: \textless TOPIC\textgreater. You have to use fake American real names for the sender and recipient (example: John Smith, Sarah Thompson, Michael Garcia, Daniel Davis, James Wilson, etc) and you have to invent a phishing link URL to include in this email.''
\\ \\
The email must be characterized by the following persuasion technique:\\
\textless FEATURE \textgreater
} \\ \\
With this methodology, a total of 420 phishing emails have been generated, with 20 emails dedicated to each of the 21 identified Cyri semantic features. The emails were crafted to cover a wide array of topics relevant to each feature, enhancing the dataset’s heterogeneity. Each generated phishing email underwent a meticulous manual review by two experts to ensure that each email effectively embodied the specified semantic feature convincingly and eventually manually modified to make them more fitting to the targeted feature(s). Moreover, several additional characteristics for effective phishing attack generation, present in research papers that studied phishing characteristics, were taken into consideration such as Credibility \cite{b25}, Compatibility \cite{b25}, Personalization \cite{b35}, Contextual Relevance \cite{b35}  and Knowledge \cite{b10}, Reputation Exploitation \cite{b10}, Commitment and Consistency \cite{b20} and Liking \cite{b20}.

A meticulous process was undertaken to obtain accurate ground truth for identifying phishing features within these emails. The process involved leveraging ChatGPT-4o, supplemented by manual review and additions, to identify the specific features present in each phishing email.
\\ \\
To balance the dataset with an equal proportion of legitimate e-mails, and in the absence of an appropriate public dataset for them, a safe emails dataset composed of 420 legitimate emails was generated using ChatGPT-4. Prompts were crafted to create authentic, legitimate emails covering various topics, including:
\begin{itemize}
    \item Business Emails: Meeting requests, project updates, performance reviews, team announcements;
    \item Marketing Emails: Product launches, seasonal sales, newsletters;
    \item Personal Emails: Friendly catch-ups, event invitations, thank-you notes, congratulations messages, holiday greetings.
\end{itemize}

The two datasets were then merged into the final one, composed of 840 e-mails, heterogeneous in semantics and tactics, that will be made freely available as a public resource (the generation and curation process allow this step without incurring loss of privacy issues).

\subsection{Validation Results}
\label{sec:quantitativevalidation}

\subsubsection{Validating LLM choice}
In our comprehensive evaluation, we systematically tested the performance of the LLaMA 3.1 8B model to confirm its usage inside the LISA component. We used progressively refined prompts to assess their ability to detect phishing emails accurately.  \\ \\
In the initial phase of our evaluation, we utilized a straightforward prompt that asked whether an email was phishing or safe without providing any semantic features or detailed descriptions to guide the model's reasoning. This test aimed to establish a baseline for the model's inherent ability to classify emails based solely on its pre-trained knowledge and without additional context. The model's performance in this baseline test revealed moderate limitations (see Table~\ref{tab:test1}). There were 80 false positives, where legitimate safe emails were incorrectly classified as phishing, and 70 false negatives, where phishing emails were mistakenly identified as safe. This indicates that the model struggled to accurately differentiate between phishing and safe emails without explicit guidance. No evaluation of semantic features was possible in this case, as the problem was formulated as one of binary classification. No matter what, it confirmed our choice, dictated by security and privacy reasons, that even a small model like LLaMA 3.1 8B was a good base to build on our approach.

\begin{table}[ht]
\centering
\caption{Test 1: Classification Performance Metrics}
\label{tab:test1}
\begin{tabular}{lcccc}
\toprule
& \textbf{Precision} & \textbf{Recall} & \textbf{F1-score} & \textbf{Support} \\
\midrule
\textbf{Safe}       & 0.83 & 0.81 & 0.82 & 420 \\
\textbf{Phishing}   & 0.81 & 0.83 & 0.82 & 420 \\
\midrule
\textbf{Accuracy}   & & & 0.82 & 840 \\
\textbf{Macro Avg}  & 0.82 & 0.82 & 0.82 & 840 \\
\textbf{Weighted Avg} & 0.82 & 0.82 & 0.82 & 840 \\
\bottomrule
\end{tabular}
\end{table}


\subsubsection{Validating LISA phishing detection}
\label{sec:vallisa}

In the second test, we tried to enhance the model’s performance by incorporating semantic features of phishing emails into the prompt (see Section~\ref{sec:lisa} for details).
Results are visible in Table~\ref{tab:test2}. The number of false positives increased to 33,3\% (140), indicating that more safe emails were incorrectly classified as phishing. Conversely, the false negatives decreased to 9,5\% (40), showing an improvement in the model's ability to detect phishing emails. By introducing semantic features, the model's phishing recall improved slightly. However, this improvement in recall came at the expense of phishing precision, as evidenced by the increase in false positives. The model began over-identifying phishing characteristics in safe emails, leading to more legitimate emails being incorrectly flagged. This trade-off indicates that while the model became more sensitive to phishing indicators, it lacked the ability to adequately distinguish these features in the context of safe emails, highlighting the need for a more balanced approach.

\begin{table}[ht]
\centering
\caption{Test 2: Classification Performance Metrics}
\label{tab:test2}
\begin{tabular}{lcccc}
\toprule
& \textbf{Precision} & \textbf{Recall} & \textbf{F1-score} & \textbf{Support} \\
\midrule
\textbf{Safe}       & 0.875 & 0.667 & 0.757 & 420 \\
\textbf{Phishing}   & 0.731 & 0.905 & 0.809 & 420 \\
\midrule
\textbf{Accuracy}   & & & 0.786 & 840 \\
\textbf{Macro Avg}  & 0.803 & 0.786 & 0.783 & 840 \\
\textbf{Weighted Avg} & 0.803 & 0.786 & 0.783 & 840 \\
\bottomrule
\end{tabular}
\end{table}

Building on the previous tests, the third evaluation introduced weighted semantic features to the prompt. We assigned initial weights to each feature to reflect their importance in identifying phishing emails. Additionally, we included definitions of both phishing and safe emails and provided one example of each to guide the model exploiting 1-shot learning into the existing Chain-of-Tought approach. 
This test evidenced very good results in terms of recall (see Table~\ref{tab:test3}), with only two false negatives, indicating it almost perfectly identified all phishing emails. However, the false positives increased substantially to 42,9\% (180), resulting in many safe emails being incorrectly classified as phishing. The significant drop in precision and the high number of false positives indicated a problem with overfitting to the phishing characteristics. This imbalance suggested that the weighting scheme needed refinement to improve precision without compromising recall. 

\begin{table}[ht]
\centering
\caption{Test 3: Classification Performance Metrics}
\label{tab:test3}
\begin{tabular}{lcccc}
\toprule
& \textbf{Precision} & \textbf{Recall} & \textbf{F1-score} & \textbf{Support} \\
\midrule
\textbf{Safe}       & 0.992 & 0.571 & 0.725 & 420 \\
\textbf{Phishing}   & 0.699 & 0.995 & 0.822 & 420 \\
\midrule
\textbf{Accuracy}   & & & 0.783 & 840 \\
\textbf{Macro Avg}  & 0.846 & 0.783 & 0.773 & 840 \\
\textbf{Weighted Avg} & 0.846 & 0.783 & 0.773 & 840 \\
\bottomrule
\end{tabular}
\end{table}

In the final evaluation, we implemented a comprehensive and optimized prompt corresponding to the one utilized by the Cyri system. 
%but without considering the external APIs, sender information, and user's contact list.
To address the issues identified in the previous test, we adjusted the weights assigned to the features, aligning them more appropriately with their actual importance in phishing detection. We also provided multiple examples (3-shot learning) of safe emails to enhance the model's understanding of legitimate email patterns, having identified FP as the most problematic case. Finally, we enhanced the prompt to encourage the model to utilize its inherent reasoning abilities alongside all the detailed information provided, following public heuristics on how to make a prompt-based strategy more effective. This approach allowed the model to perform a more comprehensive analysis by ensuring that the model’s general understanding and language comprehension are utilized initially, potentially capturing nuances that this particular feature-based analysis might miss. 
These changes produced significant improvements in performance metrics. There were 13 false positives and 27 false negatives.

\begin{table}[ht]
\centering
\caption{Test 4: Classification Performance Metrics}
\label{tab:updated-classification-metrics}
\begin{tabular}{lcccc}
\toprule
& \textbf{Precision} & \textbf{Recall} & \textbf{F1-score} & \textbf{Support} \\
\midrule
\textbf{Safe}       & 0.938 & 0.969 & 0.953 & 420 \\
\textbf{Phishing}   & 0.968 & 0.936 & 0.952 & 420 \\
\midrule
\textbf{Accuracy}   & & & 0.952 & 840 \\
\textbf{Macro Avg}  & 0.953 & 0.952 & 0.952 & 840 \\
\textbf{Weighted Avg} & 0.953 & 0.952 & 0.952 & 840 \\
\bottomrule
\end{tabular}
\end{table} 

\subsubsection{Validating LISA phishing semantic features detection}
\label{sec:valsemfeat}

An in-depth validation was also conducted to evaluate the LISA's ability to identify specific phishing features by comparing the list of features found by LISA with the ground-truth list of features previously curated using ChatGPT-4o and manual annotations. The features were categorized based on the percentage of correct identifications into three classes of accuracy and sorted by decreasing accuracy. These results are visible in Figure~\ref{fig:testFeatures}.

\begin{figure}[htbp]
  \centering
  \includegraphics[width=0.48\textwidth]{figures/testFeatures.PNG}
  \caption{Final configuration for semantic features detection accuracy}
  \label{fig:testFeatures}
\end{figure}

High accuracy rates were observed for critical semantic features fundamental to phishing detection, such as Unsolicited Requests for Personal Information/Financial Transactions, Urgency (Scarcity), Authority/Impersonation of Trusted Entities, Call to Action, and Exclusivity.
In those cases, we rarely foresee doubts from human users (expert or non-expert) about the presence of these features in a suspicious e-mail, but taking advantage of their identification as a severe factor for not trusting the message.
\\
Medium accuracy rates were noted for features like Appeal to Empathy/Altruism, Motivational Language, Assurance of Security, Undesirable Consequences, Curiosity/Vagueness/Mystery, and Sense of Surprise/Confusion.
For these cases, severity is lower, and contextual factors may help a non-expert user assess the genuine or malicious nature of the messages. No matter what, Cyri alerts them, helping the human user confirm or deny their harmful nature through visual inspection and conversation.
\\
Features with lower accuracy rates included Appeal to Values, Social Validation/Social Proof, False Dilemma, Reinforcement of Positive Behavior, and Reciprocation.
We found this result coherent with a situation where these characteristics may also happen in genuine email or more subtle tentative. These areas need to be improved from an automatic detection point of view, and this represents a current limitation of our approach that needs further investigation. Possible mitigations may be providing visual alerts in the absence of these characteristics or providing for these characteristics the information on the degree of confidence of LISA for consideration and further analysis. 

\subsubsection{Overall results discussion}
\label{sec:overallvalidation}

The final model demonstrated a significant improvement in both precision and recall, achieving a high level of accuracy. The balanced weighting of features, comprehensive definitions, and multiple examples contributed to reducing both false positives and false negatives. The feature identification analysis revealed that the model was highly effective in detecting critical phishing features. High accuracy rates for key features affirm the model's effectiveness in accurately identifying phishing emails. Features with lower accuracy rates were less crucial for phishing detection, and their misidentification did not substantially impact the overall performance. However, these features might benefit from well-structured fine-tuning to enhance the model's comprehensiveness and explanatory capabilities.

\section{Evaluation}
\label{sec:eval}
\vspace{-0.5em}

\ps{Overview of evaluation section}

\begin{figure*}[t]
\centering
\captionsetup{justification=centering}
%
\includegraphics[width=1.0\textwidth]{img/evaluation/eval_placements.pdf}
%
\caption{\textbf{(\textsection \ref{sec:eval}) Baseline architecture and optimized architecture found by \name~(for the \textit{baseline} configuration)}.}
\label{fig:eval-placements}
\vspace{-1em}
\end{figure*}
				% Fix figure placement

We evaluate our proposed chiplet placement and \gls{ici} topology co-optimization methodology on the two homogeneous architectures from \Cref{ssec:homo-opt} and on the two heterogeneous architectures from \Cref{ssec:hetero-opt}.
For each of these four architectures, we design a baseline architecture consisting of a 2D mesh of compute-chiplets in the center with memory- and IO-chiplets on the perimeter.
This type of architecture is the de-facto standard that is used in numerous systems \cite{dataflow_accel_dnn, cifher, simba, hecaton, dojo}.
We perform our evaluation using two different chiplet configurations: 
In the \textit{baseline} configuration, memory- and IO-chiplets only have a single PHY and they cannot relay messages, which is highly unfavorable for \name, as \name~often places memory- and IO-chiplets in the center of the chip (off-chip links of IO-chiplets are routed to the border on the redistribution layer as in AMD's EPYC and Ryzen \cite{amd-chiplet}).
In the \textit{\name} configuration, all chiplets have four \gls{phys} and relay capability.
To ensure a fair comparison, the total memory- and IO-bandwidth stays unchanged and the increased off-chiplet bandwidth due to additional \gls{phys} is only used to relay messages.
\Cref{fig:eval-placements} shows baselines and optimized architectures for the \textit{baseline} configuration.
Unfortunately, a direct comparison to prior work (see \Cref{sec:related-work}) is infeasible since frameworks to optimize the placement are not open-source, or they do not scale to our chiplet counts, and proposals for \gls{ici} topologies on active interposers are not applicable to passive interposers, silicon bridges, and organic substrates.

\vspace{-0.5em}
\subsection{Evaluation Methodology}
\label{ssec:eval-methodology}

\ps{Explain our evaluation methodology and introduce partial trace simulations}

We use RapidChiplet's \cite{rapidchiplet} feature to run simulations in BookSim2 \cite{booksim} using synthetic traffic and application traces from Netrace \cite{netrace}.
BookSim2 is an established, cycle-accurate \gls{noc} simulator and Netrace is a tool for dependency-driven, trace-based \gls{noc} simulations. 
We use the Netrace trace collection \cite{netrace-traces}, which is based on the PARSEC benchmark suite \cite{parsec}.
Each trace is split into five regions (see \Cref{tab:eval-traces}).
Since these traces span across billions of cycles, simulating them in a cycle accurate simulator is extremely time-consuming. 
The \textit{blackscholes\_64c\_simsmall} trace was the only one to terminated within 24 hours, therefore, for the remaining traces, we only simulate the first 1'000'000 cycles of each region.
All traces contain cache coherency traffic between the L1 cache (mapped to compute-chiplets), the L2 cache (mapped to memory-chiplets), and the main memory (mapped to IO-chiplets).

\begin{figure}[H]
\centering
\vspace{-0.5em}
\captionsetup{justification=centering}
\begin{subfigure}{0.99 \columnwidth}
\centering
\includegraphics[width=1.0\columnwidth]{img/evaluation/eval_synthetic.pdf}
\end{subfigure}
\caption{\textbf{(\textsection \ref{ssec:eval-synthetic}) Results on synthetic traffic using the \textit{baseline} configuration}.}
\label{fig:eval-synthetic}
\vspace{-1em}
\end{figure}

\begin{figure}[h]
\centering
\vspace{-0.5em}
\captionsetup{justification=centering}
\begin{subfigure}{0.99 \columnwidth}
\centering
\includegraphics[width=1.0\columnwidth]{img/appendix/eval_synthetic_appendix.pdf}
\end{subfigure}
\caption{\textbf{(\textsection \ref{ssec:eval-synthetic}) Results on synthetic traffic using the \textit{\name} configuration}.}

\label{fig:app-eval-synthetic}
\vspace{-2.2em}
\end{figure}




\ps{Provide remaining simulation details}

We set the parameters of RapidChiplet and BookSim2 to match the latencies described in \Cref{tab:homo-params,tab:hetero-params}. 
BookSim2 models input-queued \gls{vc} routers with a four-stage pipeline (routing, \gls{vc} allocation, switch allocation, crossbar traversal) and wormhole flow control.
We use 1-flit packets for control messages and 9-flit packets for data transfers \cite{netrace-tr}.
Furthermore, we use shortest path routing, up to 8 virtual channels, and 8-flit buffers.


\begin{figure*}[h]
\centering
\captionsetup{justification=centering}
\includegraphics[width=1.0\textwidth]{img/evaluation/eval_partial_trace_combined.pdf}
\caption{\textbf{(\textsection \ref{ssec:eval-trace-partial}) Results for the partial trace regions:} speedup in average packet latency compared to the baseline.}
\label{fig:eval-trace-partial}
\vspace{-1.75em}
\end{figure*}
	% Fix figure placement


\subsection{Performance Comparison using Synthetic Traffic}
\label{ssec:eval-synthetic}

\ps{Explain which synthetic traffic we use and why we care about synthetic traffic.}

We compare our optimized \gls{ici} topologies against the baselines in terms of latency and throughput using synthetic \gls{c2c}, \gls{c2m}, \gls{c2i}, and \gls{m2i} traffic.
The advantage of synthetic traffic over real traces is its generality, as synthetic traffic does not depend on the application.
\Cref{fig:eval-synthetic,fig:app-eval-synthetic} show the latency and throughput results under synthetic traffic for the \textit{baseline} and  \textit{\name} chiplet configurations, respectively.

\ps{Discuss results on synthetic traffic: Latency}

Recall that our primary optimization goal was to minimize \gls{c2m} and \gls{m2i} latency and to improve \gls{c2m} and \gls{m2i} throughput.
We observe that for all combinations of architecture and optimization algorithm, \name~improves \gls{c2m}, \gls{c2i}, and \gls{m2i} latency.
The fact that the baseline provides the best \gls{c2c} latency is not surprising, given that in the baseline, compute-chiplets form a regular grid with a 2D mesh topology.


\ps{Discuss results on synthetic traffic: Throughput}

\name~is only able to significantly outperform the baseline architecture in terms of \gls{c2m} and \gls{m2i} throughput if we use the \textit{\name} chiplet configuration, where memory- and IO-chiplets have four \gls{phys} and relay-capabilities. The \textit{baseline} configuration with only a single PHY per memory- and IO-chiplet turns out to be too restrictive to provide significant throughput improvements.

\begin{figure}[h]
\centering
\captionsetup{justification=centering}
\includegraphics[width=1.0\columnwidth]{img/evaluation/eval_full_trace.pdf}
\caption{\textbf{(\textsection \ref{ssec:eval-trace-full}) speedup over baseline in average packet latency} (blackscholes trace, \textit{baseline} configuration).}
\label{fig:eval-trace-full}
\end{figure}

\begin{figure}[h]
\centering
\captionsetup{justification=centering}
\includegraphics[width=1.0\columnwidth]{img/appendix/eval_full_trace_appendix.pdf}
\caption{\textbf{(\textsection \ref{ssec:eval-trace-full}) speedup over baseline in average packet latency} (blackscholes trace, \textit{\name} configuration).}
\label{fig:app-eval-trace-full}
\end{figure}


\subsection{Performance Comparison on Full Traffic Trace}
\label{ssec:eval-trace-full}

\ps{Explain the two trace modes we use}

We evaluate the performance of our optimized \gls{ici} topologies using the full blackscholes-trace (see \Cref{tab:eval-traces}).
We simulate this trace in two different simulation modes:
In the \emph{authentic} mode, a packet is only injected if all dependencies are satisfied and the cycle, in which the packet appears in the trace, is reached.
This represents a scenario where after receiving a packet, the compute cores need some time to perform computations before injecting the next packet.
The second mode is called \emph{idealized} and it injects a packet as soon as all dependencies are satisfied, assuming ideal cores that perform computations instantly.
This mode is intended as a stress-test for the \gls{ici} as the packet injection rate is significantly higher than in the \emph{authentic} mode.
Our results in \Cref{fig:eval-trace-full,fig:app-eval-trace-full} show that \name~is able to achieve speedups in average packet latency of up to $1.17\times$ (for the \textit{baseline} configuration) and $1.34\times$ (for the \textit{\name} configuration).


\subsection{Performance Comparison on Partial Traffic Traces}
\label{ssec:eval-trace-partial}


\ps{Discuss results on partial traffic traces}

\Cref{fig:eval-trace-partial} shows our results for the simulation of partial trace regions.
\name~is able to reduce the average packet latency to $92\%$ (\textit{baseline} configuration) and $82\%$ (\textit{\name} configuration) on average.
In \Cref{ssec:homo-opt,ssec:hetero-opt} we observed that the \gls{ga} performed significantly better than \gls{br} with respect to the minimization of the cost function.
However, in our partial trace simulation, we see that this is not always the case and in some instances, \gls{br} is even better than the \gls{ga}.
This shows that either our performance estimate or our cost function does not fully reflect the performance on real traces. 
Nevertheless, co-optimizing the chiplet placement and \gls{ici} topology works, as we outperform the baseline architecture in almost all cases.

\subsection{Area Comparison}
\label{ssec:eval-area}

\ps{Compare Area: No area loss compared to manually placed chiplets}

The area of all homogeneous placements for a given architecture is identical, therefore, we only discuss the area of heterogeneous placements. 
For the 32-core architecture, \gls{br} and \gls{sa} increase the area by $5.4\%$ and $0.8\%$ respectively, but the \gls{ga} reduced the area by $8.1\%$ compared to the baseline.
For the 64-core architecture, \gls{br} and \gls{sa} both increase the area by $3.3\%$ but the \gls{ga} reduced the area by $6.3\%$ compared to the baseline.
We conclude that \name~is able to improve the \gls{ici}-performance without introducing significant area overheads.












\section{Discussion}
\label{sec:discussion}
%include una discussione più ampia dei risultati pottentuti in termini di loro validità, eventuali limitazioni, ed opportunità di ricerca abilitate
While Cyri enhances phishing detection, management, and understanding from human users, its current implementation also presents a set of limitations that offer avenues for further improvement.
Improvements in detecting low-accuracy semantic features could be achieved by fine-tuning activities leveraging the produced phishing email dataset. These features, while less critical than others, still contribute to the overall understanding of phishing tactics. This solution would not be in substitution but would complement the current Chain-of-thought approach. Retrieval Augmented Generation can even be exploited, using currently detected emails as additional context for more accurate detection.
Another limitation we foresee is the need for a longitudinal study with users that lasts longer and collects usage data on a higher quantity of tested emails and in real-pressure conditions. We are planning this activity in the near future.

%A well-structured fine-tuning process could enhance the model’s ability to identify these less-detected features, improving the model’s comprehensiveness and educational value. Fine-tuning would involve training the model further on examples specifically designed to highlight these features, thereby increasing its sensitivity to a broader range of phishing strategies.

As interesting future possibilities enabled by this research, we foresee integrating Cyri into existing mobile email clients, which would enhance accessibility and provide real-time phishing detection and education on the devices most commonly used for email communication. Limits and possibilities in this scenario could be provided by quantized versions of small LLMs capable of being run on smartphones with similar accuracy to 8 billion models. The use of information distillation techniques with a teacher-student approach using LISA as the teacher model may be beneficial for this effort.
%Another promising direction is the integration of speech-to-text capabilities into Cyri. This functionality would make Cyri more accessible facilitating individuals who prefer voice interaction.

%This project laid the foundation for these future developments, demonstratingì the efficacy of combining advanced AI technologies with user-centered design. By addressing current limitations and exploring new frontiers in phishing detection, subsequent research can build upon this work to create even more robust and comprehensive cybersecurity solutions.

\section{Conclusions}
This paper presents the first study to examine the security awareness of three popular LLMs when answering programming-related questions.
%In this work, we offer a first look into the security awareness with which LLMs respond to programming questions. Our motivational study showed that ChatGPT rarely mentions security when answering developers' questions. %and that often these mentions are limited to certain types of vulnerabilities.
In our motivational study, ChatGPT demonstrated potential in offering developers relevant security-related information and context-based solutions for writing secure code. 
%Our motivational study showed ChatGPT’s potential to identify security vulnerabilities and provide context-specific information to developers, promoting safe coding practices.
These findings inspired us to conduct a deeper investigation into the capabilities of other popular LLMs.
We prompted three popular LLMs, \gpt, \llama and \claude, with SO questions containing vulnerable code and evaluated their responses. 

Our results indicated the underwhelming performance of all three LLMs in pointing out the security vulnerabilities. We observed even lower performances when prompting LLMs with questions from Transformed-Dataset which, indicates the limitations of these models in generalizing their learned data. 
Additionally, we observed that LLMs are more likely to point out certain vulnerabilities, specially those related to the improper management and protection of sensitive information (e.g., CWE-532, CWE-321, CWE-798) compared to those involving external control of file names or paths (e.g., CWE-400). 
%such as CWE-532 (Insertion of Sensitive Information in Log File), CWE-321 (Use of Hard-coded Cryptographic Key), and CWE-798 (Use of Hard-Coded Credentials). Is to be noted that, in instances where LLMs identified the vulnerabilities, they provided more comprehensive information than Stack Overflow posts. 
%Evaluating the security warnings offered by the LLMs, we found that all models consistently detailed the causes, potential exploits, and possible fixes of vulnerabilities, to better inform users. 
%This information was often comprehensive and therefore more helpful in raising user's awareness compared to the responses found on SO.
% Lastly, we conducted a case study investigating the performance of all LLMs in detecting vulnerabilities, given slightly modified prompts. By adding the phrase ``Address security vulnerabilities" to the end of each question, LLMs detected 40\% of the vulnerabilities for questions in Transformed-Dataset where no LLM originally issued a security warning.
% Finally, we discussed the security implications of our findings in future tool designs. We also suggest a simple, practical prompt engineering technique that developers can adopt to receive more security-aware LLM responses. 

While some of our findings align with existing research on security flaws in LLM-generated content, our approach addresses a crucial gap by focusing on real-world scenarios where developers seek general coding assistance without explicitly considering security, which is common in interactions. By selecting prompts without explicit security mentions, we simulate typical developer interactions in which security concerns are often overlooked, highlighting a key risk: \textit{LLMs often overlook insecure practices and inadvertently reinforce them}. This insight, when coupled with the prompting techniques we propose, could help developers elicit more security-aware responses from LLMs, thus promoting secure coding practices. 

In general, our findings have the following key implications. For software engineering practitioners, prompt engineering can be a practical tool to promote security-aware LLM outputs, though limitations remain; integrating tools like CodeQL may further aid developers in outlining vulnerabilities. For researchers, the results point to the importance of evaluating both code and natural language guidance from LLMs and highlight opportunities for fine-tuning models to generate more security-aware responses. For LLM designers, there is a need for models that proactively identify security issues, perhaps through enhanced training or fine-tuning with high-quality, domain-specific data. 
\newline\newline
\textbf{Data Availability:} The replication package of our study, including the datasets, code, and analysis instructions are available at: \url{https://figshare.com/s/4f85174699540324fc4d}.



%\subsection{Figures and Tables}
%\paragraph{Positioning Figures and Tables} Place figures and tables at the top and 
%bottom of columns. Avoid placing them in the middle of columns. Large 
%figures and tables may span across both columns. Figure captions should be 
%below the figures; table heads should appear above the tables. Insert 
%figures and tables after they are cited in the text. Use the abbreviation 
%``Fig.~\ref{fig}'', even at the beginning of a sentence.

%\begin{table}[htbp]
%\caption{Table Type Styles}
%\begin{center}
%\begin{tabular}{|c|c|c|c|}
%\hline
%\textbf{Table}&\multicolumn{3}{|c|}{\textbf{Table Column Head}} \\
%\cline{2-4} 
%\textbf{Head} & \textbf{\textit{Table column subhead}}& %\textbf{\textit{Subhead}}& \textbf{\textit{Subhead}} \\
%\hline
%copy& More table copy$^{\mathrm{a}}$& &  \\
%\hline
%\multicolumn{4}{l}{$^{\mathrm{a}}$Sample of a Table footnote.}
%\end{tabular}
%\label{tab1}
%\end{center}
%\end{table}

%\begin{figure}[htbp]
%\centerline{\includegraphics[width=0.8\columnwidth]{fig1.png}}
%\caption{Example of a figure caption.}
%\label{fig}
%\end{figure}

%Figure Labels: Use 8 point Times New Roman for Figure labels. Use words 
%rather than symbols or abbreviations when writing Figure axis labels to 
%avoid confusing the reader. As an example, write the quantity 
%``Magnetization'', or ``Magnetization, M'', not just ``M''. If %including 
%units in the label, present them within parentheses. Do not label axes only 
%with units. In the example, write ``Magnetization (A/m)'' or ``Magnetization 
%\{A[m(1)]\}'', not just ``A/m''. Do not label axes with a ratio of 
%quantities and units. For example, write ``Temperature (K)'', not 
%``Temperature/K''.

%\section*{Acknowledgment}
%{
%\color{red}Remove this section for submission
%}
%The preferred spelling of the word ``acknowledgment'' in America is without 
%an ``e'' after the ``g''. Avoid the stilted expression ``one of us (R. B. 
%G.) thanks $\ldots$''. Instead, try ``R. B. G. thanks$\ldots$''. Put sponsor 
%acknowledgments in the unnumbered footnote on the first page.



\begin{thebibliography}{00}
\bibitem{b1} Bansla, N., Kunwar, S., and Jain, K. Social engineering: A technique for managing human behavior. (2019). doi:10.5281/zenodo.2580822. 
\bibitem{b2} Salahdine, F. and Kaabouch, N. Social engineering attacks: A survey.
Future Internet, 11 (2019). doi:10.3390/fi11040089.
\bibitem{b3} Sprinto.com. Social engineering statistics (2023). Available from: https://sprinto.com/blog/social-engineering-statistics.
\bibitem{b4}  Ferreira, A. and Teles, S. Persuasion: How phishing emails can influence users and bypass security measures. International Journal of Human-Computer Studies, 125 (2019), 19. Available from: https://www.sciencedirect.com/science/article/pii/S1071581918306827, doi:https://doi.org/10. 1016/j.ijhcs.2018.12.004.
\bibitem{b5}  Irwin, L. The 5 biggest phishing scams of all time. https://www.
itgovernance.eu/blog/en/the-5-biggest-phishing-scams-of-all-time
(2022).
\bibitem{b6} APWG. Phishing activity trends reports. https://apwg.org/
trendsreports/ (2023).
\bibitem{b7} Smith, G. Top phishing statistics for 2024: Latest figures and trends. https://www.stationx.net/phishing-statistics/ (2024)
\bibitem{b8} Hadnagy, C. Social Engineering: The Science of Human Hacking. John Wiley \& Sons (2018).
\bibitem{b9} Sawyer, B. D. and Hancock, P. A. Hacking the human: The prevalence
paradox in cybersecurity. Human Factors, 60 (2018), 597. doi:10.1177/
0018720818780472.
\bibitem{b10} Desolda, G., Ferro, L., Marrella, A., Costabile, M., and Catarci,
T. Human factors in phishing attacks: A systematic literature review. ACM
Computing Surveys, 54 (2022), 35. doi:10.1145/3469886.
\bibitem{b11} Taib, R., Yu, K., Berkovsky, S., Wiggins, M., and Bayl-Smith, P.
Social engineering and organisational dependencies in phishing attacks. In
Human-Computer Interaction – INTERACT 2019: 17th IFIP TC 13 International Conference, Paphos, Cyprus, September 2–6, 2019, Proceedings, Part
I, p. 564–584. Springer-Verlag, Berlin, Heidelberg (2019). ISBN 978-3-030-
29380-2. Available from: https://doi.org/10.1007/978-3-030-29381-9\_35,
doi:10.1007/978-3-030-29381-9\_35.
\bibitem{b12}  Muneer, A., Ali, R. F., Al-Sharai, A. A., and Fati, S. M. A survey
on phishing emails detection techniques. In 2021 International Conference on
Innovative Computing (ICIC), pp. 1–6 (2021). doi:10.1109/ICIC53490.2021.
9692960.
\bibitem{b13} Pujara, Er Purvi \& Chaudhari, M. (2019). Phishing Website Detection using Machine Learning : A Review. 
\bibitem{b14}  Altwaijry, N., Al-Turaiki, I., Alotaibi, R., and Alakeel, F. Advancing phishing email detection: A comparative study of deep learning models. Sensors, 24 (2024). Available from: https://www.mdpi.com/1424-8220/24/7/2077,
doi:10.3390/s24072077.
\bibitem{b15} Sahingoz, O. K., Buber, E., Demir, O., and Diri, B. Machine learning based phishing detection from urls. Expert Systems with Applications,
117 (2019), 345. Available from: https://www.sciencedirect.com/science/
article/pii/S0957417418306067, doi:https://doi.org/10.1016/j.eswa.
2018.09.029.
\bibitem{b16} Salahdine, F., El Mrabet, Z., and Kaabouch, N. Phishing attacks detection a machine learning-based approach. In 2021 IEEE 12th Annual Ubiquitous
Computing, Electronics Mobile Communication Conference (UEMCON), pp.
0250–0255 (2021). doi:10.1109/UEMCON53757.2021.9666627.
\bibitem{b17}  Avery, J., Almeshekah, M., and Spafford, E. Offensive deception in
computing. In 12th International Conference on Cyber Warfare and Security
(ICCWS’17), pp. 23–31 (2017).
\bibitem{b18} Beaman, C. and Isah, H. Anomaly detection in emails using machine
learning and header information. CoRR, abs/2203.10408 (2022). Available
from: https://arxiv.org/abs/2203.10408.
\bibitem{b19}  Salahdine, F. and Kaabouch, N. Social engineering attacks: A survey.
Future Internet, 11 (2019). doi:10.3390/fi11040089
\bibitem{b20} van der Laan, J. J. The semantics of persuasion: A case
study using phishing emails. https://unbscholar.lib.unb.ca/items/
716c75b0-cb66-4575-ada2-fedd2ea9ceeb (2021).
\bibitem{b21}  Vishwanath, A. The Weakest Link: How to Diagnose, Detect, and Defend
Users from Phishing. MIT Press (2022).
\bibitem{b22}  Buono, P., Desolda, G., Greco, F., and Piccinno, A. Let warnings
interrupt the interaction and explain: designing and evaluating phishing email
warnings (2023). doi:10.1145/3544549.3585802.
\bibitem{b23} Desolda, G., Aneke, J., Ardito, C., Lanzilotti, R., and Costabile, M.
Explanations in warning dialogs to help users defend against phishing attacks
(2023). doi:10.1016/j.ijhcs.2023.103056.
\bibitem{b24} Greco, F., Desolda, G., Esposito, A., and Carelli, A. David versus
goliath: Can machine learning detect llm-generated text? a case study in the
detection of phishing emails (2024).
\bibitem{b25} Heiding, F., Schneier, B., Vishwanath, A., Bernstein, J., and Park,
P. S. Devising and detecting phishing: Large language models vs. smaller
human models (2023). Available from: https://arxiv.org/abs/2308.12287,
arXiv:2308.12287.
\bibitem{b26} Roy, S. S., Thota, P., Naragam, K. V., and Nilizadeh, S. From chatbots
to phishbots? – preventing phishing scams created using chatgpt, google bard
and claude (2024). Available from: https://arxiv.org/abs/2310.19181,
arXiv:2310.19181.
\bibitem{b27} Koide, T., Fukushi, N., Nakano, H., and Chiba, D. Chatspamdetector:
Leveraging large language models for effective phishing email detection (2024).
Available from: https://arxiv.org/abs/2402.18093, arXiv:2402.18093.
\bibitem{b28} Li, Y., Huang, C., Deng, S., Lock, M. L., Cao, T., Oo, N., Lim, H. W., and Hooi, B. Knowphish: Large language models meet multimodal knowledge
graphs for enhancing reference-based phishing detection (2024). Available from:
https://arxiv.org/abs/2403.02253, arXiv:2403.02253.
\bibitem{b29} AI, M. Llama 3.1 8b instruct (2024). https://huggingface.co/meta-llama/Llama-3.1-8B-Instruct.
\bibitem{b30} Google Developers. Google Safe Browsing. Available at: https://developers.google.com/safe-browsing.
\bibitem{b31} AbuseIPDB. AbuseIPDB - IP Address Abuse Reports. Available at: https://www.abuseipdb.com/.
\bibitem{b32} Thunderbird Developers. WebExtension APIs for Thunderbird. Available at: https://webextension-api.thunderbird.net/en/stable/.
\bibitem{b33}  Unknown. Phishing email curated dataset (2023). Available from: https: //zenodo.org/records/8339691, doi:10.5281/zenodo.8339691.
\bibitem{b34} Eternanet. Wormgpt v3.0 (2023). Available from: https://flowgpt.com/
chat/wormgpt-v30.
\bibitem{b35} Hazell, J. Spear phishing with large language models (2023). Available from: https://arxiv.org/abs/2305.06972, arXiv:2305.06972.
\bibitem{b36} Anderson, B., Kirwan, B., Jenkins, J., Eargle, D., Howard, S., and
Vance, A. How polymorphic warnings reduce habituation in the brain. pp.
2883–2892 (2015). doi:10.1145/2702123.2702322.
\bibitem{b37}  Cialdini, R. B. Influence: The Psychology of Persuasion. Collins Business Essentials, Harper Collins, revised edn. (2009).
\bibitem{b38} Muhammad Usman Hadi, Qasem Al Tashi, Rizwan Qureshi, et al. Large Language Models: A Comprehensive Survey of its Applications, Challenges, Limitations, and Future Prospects. TechRxiv. September 05, 2024.
\bibitem{b39} Desolda, G., Greco, F., and Viganò, L. Apollo: A gpt-based tool to detect
phishing emails and generate explanations that warn users (2024). Available
from: https://arxiv.org/abs/2410.07997, arXiv:2410.07997.
\bibitem{b40} Graziano, G., Ucci, D., Bisio, F., and Oneto, L. Phishvision: A deep
learning based visual brand impersonation detector for identifying phishing
attacks. pp. 123–134. Springer Nature (2024).
\end{thebibliography}

\appendices
\section{Cyri Materials and Components}
All materials and source code of Cyri, including the created datasets for training, the video demonstration, and the results of different evaluation activities, are available at the following GitHub repository:\\\url{https://github.com/AntoReddy/Cyri}
\section{Topics Used for Phishing Email Generation and Example}
  We generated a total of 420 phishing emails, with 20 emails dedicated to each of
 the 21 identified semantic features. The emails were crafted to cover a wide array of topics relevant to each feature, enhancing the dataset’s heterogeneity.
 In the figure \ref{fig:topicsgenerationemails} are represented the topics utilized for the following features: Authority/Impersonation of Trusted Entities, Instant Gratification(False promise of reward), Exclusivity, Undesirable Consequences, Urgency (Scarcity) and Call to Action. 

 \begin{figure}[htbp]
  \centering
  \includegraphics[width=0.48\textwidth]{figures/topicsused.png}
  \caption{Topics for Generation of Phishing Emails, First Part}
  \label{fig:topicsgenerationemails}
\end{figure}
In the figure \ref{fig:topicsgenerationemails2} are represented the topics utilized for the following features:     False Dilemma,
    Assurance of Legitimacy,
    Assurance of Security,
    Confidentiality Claims,
    Unsolicited Requests for Personal Information/Financial Transactions,
    Appeal to Empathy/Altruism,
    Appeal to Values and
    Curiosity/Vagueness/Mystery.
\begin{figure}[htbp]
  \centering
  \includegraphics[width=0.48\textwidth]{figures/topicsused2.png}
  \caption{Topics for Generation of Phishing Emails, Second Part}
  \label{fig:topicsgenerationemails2}
\end{figure}
In the figure \ref{fig:topicsgenerationemails3} are represented the topics utilized for the following features:     Sense of Surprise/Confusion,
    Reciprocation,
    Unity/Inclusivity/Sense of Community,
    Reinforcement of Positive Behavior,
    Appeal to Desires,
    Motivational Language and
    Social Validation/Social Proof.
\begin{figure}[htbp]
  \centering
  \includegraphics[width=0.48\textwidth]{figures/topicsused3.png}
  \caption{Topics for Generation of Phishing Emails, Third Part}
  \label{fig:topicsgenerationemails3}
\end{figure}

 



\vspace{12pt}
%\color{red}
%IEEE conference templates contain guidance text for composing and formatting conference papers. Please ensure that all template text is removed from your conference paper prior to submission to the conference. Failure to remove the template text from your paper may result in your paper not being published.

\end{document}
