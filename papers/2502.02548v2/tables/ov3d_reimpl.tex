\begin{table}[h]
    \centering
        \resizebox{\linewidth}{!}{
        \begin{tabular}{ll|rr|rr}
            \toprule
            \multirow{2}{*}{Method} & \multirow{2}{*}{Loss} & \multicolumn{2}{c|}{ScanNet20~\cite{dai2017scannet}} & \multicolumn{2}{c}{ScanNet200~\cite{scannet200}} \\
            & & f-mIoU & f-mAcc & f-mIoU & f-mAcc \\
            \midrule
            OV3D~\cite{jiang2024open} & \texttt{DenseAlign} & 64.0 & 76.3 & 8.7 & - \\
            \midrule
            OV3D-rep & \texttt{DenseAlign} & 34.7 & 54.9 & 4.6 & 8.3 \\
            OV3D-rep & \texttt{Align} & 20.0 & 34.0 & 2.4 & 5.4 \\
            OV3D-rep & \texttt{Contrastive} & 45.6 & 69.8 & 6.9 & 14.3 \\
            \midrule
            OV3D++ & \texttt{DenseAlign} & 54.3 & 71.6 & 7.0 & 12.0 \\
            OV3D++ & \texttt{Align} & 22.5 & 37.6 & 3.1 & 5.6 \\
            OV3D++ & \texttt{Contrastive} & 58.4 & 76.7 & 9.2 & 16.7 \\
            \midrule
            \rowcolor{gray!15}\nickname & \texttt{Contrastive} & \textbf{65.0} & \textbf{82.5} & \textbf{13.0} & \textbf{24.5} \\
            \bottomrule
        \end{tabular}
        }
        \caption{
            \textbf{Re-implementation and improvement of OV3D~\cite{jiang2024open}.}
            We present our re-implementation results of OV3D with three different training objectives: \texttt{DenseAlign}, \texttt{Align}, and \texttt{Contrastive}.
            OV3D-rep denotes our re-implementation, while OV3D++ is our improved version using RAM++~\cite{ram_pp} tagging.
        }
    \label{tab:ov3d_reimpl}
\end{table}


