\section{Dataset Details}
Below we report the data statistics of our \dataname dataset, detail the data preprocessing steps, pipeline configurations used for each dataset in our experiments, and additional data pipeline experiments that utilize 3D instance mask predictions in caption generation process.

\subsection{Data Statistics}
In Tab.~\ref{tab:datasets}, we report the statistics of our generated dataset, including the number of scenes, RGB-D frames, generated captions, and total tokens in captions for each source dataset. Our dataset contains over 30K scenes, and 5.6M captions with a total of 30M tokens across both real and synthetic indoor environments.
\section{Dataset}
\label{sec:dataset}

\subsection{Data Collection}

To analyze political discussions on Discord, we followed the methodology in \cite{singh2024Cross-Platform}, collecting messages from politically-oriented public servers in compliance with Discord's platform policies.

Using Discord's Discovery feature, we employed a web scraper to extract server invitation links, names, and descriptions, focusing on public servers accessible without participation. Invitation links were used to access data via the Discord API. To ensure relevance, we filtered servers using keywords related to the 2024 U.S. elections (e.g., Trump, Kamala, MAGA), as outlined in \cite{balasubramanian2024publicdatasettrackingsocial}. This resulted in 302 server links, further narrowed to 81 English-speaking, politics-focused servers based on their names and descriptions.

Public messages were retrieved from these servers using the Discord API, collecting metadata such as \textit{content}, \textit{user ID}, \textit{username}, \textit{timestamp}, \textit{bot flag}, \textit{mentions}, and \textit{interactions}. Through this process, we gathered \textbf{33,373,229 messages} from \textbf{82,109 users} across \textbf{81 servers}, including \textbf{1,912,750 messages} from \textbf{633 bots}. Data collection occurred between November 13th and 15th, covering messages sent from January 1st to November 12th, just after the 2024 U.S. election.

\subsection{Characterizing the Political Spectrum}
\label{sec:timeline}

A key aspect of our research is distinguishing between Republican- and Democratic-aligned Discord servers. To categorize their political alignment, we relied on server names and self-descriptions, which often include rules, community guidelines, and references to key ideologies or figures. Each server's name and description were manually reviewed based on predefined, objective criteria, focusing on explicit political themes or mentions of prominent figures. This process allowed us to classify servers into three categories, ensuring a systematic and unbiased alignment determination.

\begin{itemize}
    \item \textbf{Republican-aligned}: Servers referencing Republican and right-wing and ideologies, movements, or figures (e.g., MAGA, Conservative, Traditional, Trump).  
    \item \textbf{Democratic-aligned}: Servers mentioning Democratic and left-wing ideologies, movements, or figures (e.g., Progressive, Liberal, Socialist, Biden, Kamala).  
    \item \textbf{Unaligned}: Servers with no defined spectrum and ideologies or opened to general political debate from all orientations.
\end{itemize}

To ensure the reliability and consistency of our classification, three independent reviewers assessed the classification following the specified set of criteria. The inter-rater agreement of their classifications was evaluated using Fleiss' Kappa \cite{fleiss1971measuring}, with a resulting Kappa value of \( 0.8191 \), indicating an almost perfect agreement among the reviewers. Disagreements were resolved by adopting the majority classification, as there were no instances where a server received different classifications from all three reviewers. This process guaranteed the consistency and accuracy of the final categorization.

Through this process, we identified \textbf{7 Republican-aligned servers}, \textbf{9 Democratic-aligned servers}, and \textbf{65 unaligned servers}.

Table \ref{tab:statistics} shows the statistics of the collected data. Notably, while Democratic- and Republican-aligned servers had a comparable number of user messages, users in the latter servers were significantly more active, posting more than double the number of messages per user compared to their Democratic counterparts. 
This suggests that, in our sample, Democratic-aligned servers attract more users, but these users were less engaged in text-based discussions. Additionally, around 10\% of the messages across all server categories were posted by bots. 

\subsection{Temporal Data} 

Throughout this paper, we refer to the election candidates using the names adopted by their respective campaigns: \textit{Kamala}, \textit{Biden}, and \textit{Trump}. To examine how the content of text messages evolves based on the political alignment of servers, we divided the 2024 election year into three periods: \textbf{Biden vs Trump} (January 1 to July 21), \textbf{Kamala vs Trump} (July 21 to September 20), and the \textbf{Voting Period} (after September 20). These periods reflect key phases of the election: the early campaign dominated by Biden and Trump, the shift in dynamics with Kamala Harris replacing Joe Biden as the Democratic candidate, and the final voting stage focused on electoral outcomes and their implications. This segmentation enables an analysis of how discourse responds to pivotal electoral moments.

Figure \ref{fig:line-plot} illustrates the distribution of messages over time, highlighting trends in total messages volume and mentions of each candidate. Prior to Biden's withdrawal on July 21, mentions of Biden and Trump were relatively balanced. However, following Kamala's entry into the race, mentions of Trump surged significantly, a trend further amplified by an assassination attempt on him, solidifying his dominance in the discourse. The only instance where Trump’s mentions were exceeded occurred during the first debate, as concerns about Biden’s age and cognitive abilities temporarily shifted the focus. In the final stages of the election, mentions of all three candidates rose, with Trump’s mentions peaking as he emerged as the victor.

In Tab.~\ref{tab:existing_dataset}, we evaluate caption and 3D mask quality across datasets using three metrics. 
The \textit{unique normalized nouns count} measures the total number of unique normalized nouns in captions, with higher count indicating richer and more diverse captions.
\textit{Mask coverage} (\%) calculates the mean percentage of 3D points with associated captions per scene, where higher coverage enables more effective training.
\textit{Mask entropy} (bits) measures mask quality for datasets with partial masks generated from multi-view images (\ie OV3D, RegionPLC, and Mosaic3D-5.6M) without using GT.
It calculates Shannon entropy of GT instance ID distributions within each mask--higher entropy indicates that a mask contains multiple GT instances, suggesting less accurate mask boundaries.
Mosaic3D-5.6M demonstrates superior caption diversity and mask quality compared to both existing large-scale 3D-text datasets and previous open-vocabulary 3D segmentation datasets, validating its value as a new dataset.
\begin{table*}[]
    \centering
        \caption{
    \textbf{Dataset Comparison with Existing Text-Rich Image Generation Datasets}.
    The last two columns detail the sources of automatically generated labels, while the final column presents the average text token length derived from OCR applied to the images.}
    \begin{tabular}{ccccc|c}
    \hline
        Dataset Name  & Samples & Annotations & Domain & Labels & Token Length   \\
        \hline
    TextCaps~\cite{textcaps} & 28K & Caption & Real Image & Human & 26.36 \\
    SynthText~\cite{SynthText} & 0.8M& OCR & Synthetic Image & Auto & 13.75 \\
    Marion10M~\cite{textdiffuser} & 10M &  Caption+OCR  & Real Image & Auto & 16.13\\
        AnyWords3M~\cite{anytext}  & 3M & Caption+OCR & Real Image & Auto  & 9.92\\
        RenderedText~\cite{renderedtext} &12M & Text & Synthetic Image & Auto & 21.21 \\
        \hline
        \DatasetName &5M & Caption/OCR& Real\&Synthetic Image & Auto/Human & \textbf{148.82} \\
        \hline
    \end{tabular}
    \label{tab:dataset_comparison}
\end{table*}

\subsection{Data Preprocessing}
\begin{itemize}[leftmargin=*,itemsep=1pt]
    \item \textbf{ScanNet}~\cite{dai2017scannet} To optimize computational efficiency while maintaining adequate spatial coverage, we process every 20th RGB-D frame from each scene. Prior to processing, we resize all RGB-D frames to 640$\times$480 resolution.
    \item \textbf{ScanNet++}~\cite{yeshwanth2023scannet++} From the official dataset, we utilize the \textit{``DSLR''} image collection. Following repository guidelines, we generate synthetic depth images using the reconstructed mesh and camera parameters. After correcting for distortion in both RGB and depth images and adjusting camera intrinsics, we process every 10th frame through our annotation pipeline. Point clouds are generated via surface sampling on the reconstructed meshes.
    \item \textbf{ARKitScenes}~\cite{baruch2021arkitscenes} We leverage the \textit{``3D Object Detection (3DOD)''} subset, utilizing its RGB-D frames and reconstructed meshes. We use every 10th frame at low resolution (256$\times$192), and apply surface point sampling on mesh for point clouds.
    \item \textbf{Matterport3D}~\cite{chang2017matterport3d} We use preprocesed RGB-D frames and point clouds provided by the author of OpenScene~\cite{Peng2023OpenScene}.
    \item \textbf{Structured3D}~\cite{zheng2020structured3d} We utilize RGB-D frames from both perspective and panoramic camera. We utilize preprocessed point clouds from the \textit{Pointcept}~\cite{pointcept2023} library, which fuses multi-view depth unprojection with voxel downsampling to get point clouds.
\end{itemize}

\subsection{Pipeline Configurations}
Our data generation pipeline leverages multiple Visual Foundation Models to automate the data annotation process. Below we detail the configuration of each model in our pipeline.
\begin{itemize}[leftmargin=*,itemsep=1pt]
    \item \textbf{RAM++~\cite{ram_pp}}: we utilize the official pretrained checkpoint \texttt{ram\_plus\_swin\_large\_14m} available at \url{https://huggingface.co/xinyu1205/recognize-anything-plus-model}.
    \item \textbf{Grounded-SAM~\cite{ren2024grounded}}: We employ the official checkpoint of Grounding-DINO~\cite{liu2023grounding} \texttt{IDEA-Research/grounding-dino-tiny} accessed through HuggingFace at \url{https://huggingface.co/IDEA-Research/grounding-dino-tiny}, together with SAM2~\cite{ravi2024sam} with checkpoint \texttt{sam2\_hiera\_l}, available at \url{https://huggingface.co/facebook/sam2-hiera-large}. For the postprocessing, we process the output bounding boxes from Grounding-DINO using a box score threshold of 0.25 and a text score threshold of 0.2. We then apply non-maximum suppression (NMS) with an IoU threshold of 0.5 to remove redundancy. To ensure meaningful region proposals, we filter out excessively large boxes that occupy more than 95\% of the image area. These refined bounding boxes are then passed to SAM2 for mask prediction.
    \item \textbf{Osprey~\cite{yuan2024osprey}}: We utilize the official pretrained \texttt{sunshine-lwt/Osprey-Chat-7b} checkpoint, available at \url{https://huggingface.co/sunshine-lwt/Osprey-Chat-7b}. The generation parameters are set with a temperature of 1.0, top\_p of 1.0, beam search size of 1, and the maximum number of new tokens to 512.
\end{itemize}

\begin{table}[ht]
    \centering
    \begin{minipage}{0.99\columnwidth}\vspace{0mm}    \centering
    \begin{tcolorbox} 
        \raggedright
        \small
        $\texttt{\textbf{System}}$: \texttt{A chat between a curious human and an artificial intelligence assistant. The assistant gives helpful, detailed, and polite answers to the human's questions.} \\
        $\texttt{\textbf{User}}$: \PredSty{\texttt{<image>}} \texttt{This provides an overview of the picture. Please give me a short description of} \PredSty{\texttt{<mask><pos>}} \texttt{, using a short phrase.}
    \end{tcolorbox}
        \label{tab:osprey_prompt}
    \end{minipage}
    \caption{\textbf{Osprey region caption prompt}. Osprey~\cite{yuan2024osprey} utilizes this prompt along with segmentation masks generated by Grounded-SAM to produce descriptive captions for each region.}
    \vspace{-4mm}
\end{table}



\subsection{Additional Pipeline Experiments}
We explore two additional data pipeline configurations that use Segment3D~\cite{huang2024segment3d} masks for segmentation while maintaining Osprey~\cite{yuan2024osprey} for captioning:
\begin{itemize}[leftmargin=*,itemsep=1pt]
    \item \textbf{Segment3D}: We utilize complete Segment3D masks and obtain captions by aggregating descriptions from multiple projected views of each mask. This approach maintains mask completeness but may result in multiple captions being assigned to a single mask from different viewpoints.
    \item \textbf{Segment3D - Mosaic}: We use partial Segment3D masks as seen from individual views and generate captions based on these view-specific projections. While masks are partial, each mask-caption pair is aligned since it represents the exact visible region from a specific viewpoint.
\end{itemize}
The results in Tab.~\ref{tab:segment3d_pipeline} demonstrate that Segment3D - Mosaic outperforms the baseline Segment3D approach, highlighting the importance of precise mask-text pair alignment.
However, both Segment3D variants are outperformed by our \nickname pipeline, which suggests that our combination of RAM++~\cite{ram_pp}, Grounded-SAM~\cite{ren2024grounded}, and SEEM~\cite{zou2024segment} provides superior segmentation quality.

\begin{table}[h]
\centering
\resizebox{\linewidth}{!}{
\begin{tabular}{l|rr|rr}
\midrule
\multirow{2}{*}{Pipeline} & \multicolumn{2}{c|}{ScanNet20~\cite{dai2017scannet}} & \multicolumn{2}{c}{ScanNet200~\cite{scannet200}} \\
 & f-mIoU & f-mAcc & f-mIoU & f-mAcc \\
\midrule
Segment3D~\cite{huang2024segment3d} & 50.6 & 76.6 & 8.3 & 19.1 \\
Segment3D~\cite{huang2024segment3d} - Mosaic & 57.3 & 79.6 & 10.6 & 22.8 \\
\rowcolor{gray!15} \nickname & \textbf{65.0} & \textbf{82.5} & \textbf{13.0} & \textbf{24.5} \\
\midrule
\end{tabular}
}
\caption{\textbf{Segment3D data pipeline evaluation results.}}
\label{tab:segment3d_pipeline}
\end{table}

