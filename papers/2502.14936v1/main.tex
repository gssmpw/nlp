
\documentclass[fleqn,usenatbib]{mnras}

\usepackage{fix-cm}
\usepackage{lmodern}
\usepackage{newtxtext,newtxmath}
\usepackage{colortbl}


\DeclareRobustCommand{\VAN}[3]{#2}
\let\VANthebibliography\thebibliography
\def\thebibliography{\DeclareRobustCommand{\VAN}[3]{##3}\VANthebibliography}


\usepackage{graphicx}	% Including figure files
\usepackage{amsmath}	% Advanced maths commands
\usepackage{mathtools}
\usepackage{xcolor}
\usepackage{soul}
\usepackage{tikz}
\usepackage{graphicx}
\usepackage{subcaption}
\usepackage[normalem]{ulem}
\usepackage{multirow}
\usepackage{hyperref}
\hypersetup{
    colorlinks = true,
    linkcolor = blue
    }
\usepackage[
        noabbrev,
        capitalise,
        nameinlink,]{cleveref}
\usetikzlibrary{positioning, arrows.meta, shapes.multipart}

\title[Data-driven solutions for lens detection]{Reducing false positives in strong lens detection through effective augmentation and ensemble learning}

\author[S. Rezaei et al.]{Samira Rezaei$^{1,2}$\thanks{rezaei@strw.leidenuniv.nl}, Amirmohammad Chegeni$^{1,3,4}$, Bharath Chowdhary Nagam$^{5}$, J. P. McKean$^{5,6,7}$,
\newauthor Mitra Baratchi$^{2}$, Koen Kuijken$^{1}$, and Léon V. E. Koopmans$^{5}$\\
$^{1}$Leiden Observatory, Leiden University,  2333 CC Leiden, the Netherlands.\\
$^{2}$Leiden Institute of Advanced Computer Science (LIACS), Leiden University, 2333 CC Leiden, the Netherlands.\\
$^{3}$Dipartimento di Fisica e Astronomia "G. Galilei", Università di
Padova, Via Marzolo 8, 35131 Padova, Italy\\
$^{4}$INFN-Padova, Via Marzolo 8, 35131 Padova, Italy\\
$^{5}$Kapteyn Astronomical Institute, University of Groningen, Postbus 800, NL9700 AV Groningen, the Netherlands\\
$^{6}$South African Radio Astronomy Observatory (SARAO), P.O. Box 443, Krugersdorp 1740, South Africa\\
$^{7}$Department of Physics, University of Pretoria, Lynnwood Road, Hatfield, Pretoria, 0083, South Africa
}

% These dates will be filled out by the publisher
\date{Accepted XXX. Received YYY; in original form 2024 September 2}


\begin{document}
\label{firstpage}
\pagerange{\pageref{firstpage}--\pageref{lastpage}}
\maketitle


\begin{abstract}
This research studies the impact of high-quality training datasets on the performance of Convolutional Neural Networks (CNNs) in detecting strong gravitational lenses. We stress the importance of data diversity and representativeness, demonstrating how variations in sample populations influence CNN performance. In addition to the quality of training data, our results highlight the effectiveness of various techniques, such as data augmentation and ensemble learning, in reducing false positives while maintaining model completeness at an acceptable level. This enhances the robustness of gravitational lens detection models and advancing capabilities in this field. Our experiments, employing variations of DenseNet and EfficientNet, achieved a best false positive rate (FP rate) of $10^{-4}$, while successfully identifying over 88 per cent of genuine gravitational lenses in the test dataset. This represents an 11-fold reduction in the FP rate compared to the original training dataset. Notably, this substantial enhancement in the FP rate is accompanied by only a 2.3 per cent decrease in the number of true positive samples. Validated on the KiDS dataset, our findings offer insights applicable to ongoing missions, like Euclid.
\end{abstract}


\begin{keywords}
gravitational lensing: strong -- methods: data analysis -- techniques: image processing
\end{keywords}


\section{Introduction}

Strong gravitational lensing occurs when light from a distant galaxy gets bent by the curvature of space-time that is caused by another galaxy along our line of sight. This phenomenon significantly impacts our understanding of the Universe. Strong gravitational lensing, as predicted by the theory of general relativity, occurs when the foreground galaxy acts as a lens, creating multiple magnified and distorted images of the background galaxy (see \citealt{Treu2010} for a review). Strong lensing serves as a unique tool for testing models for galaxy formation \citep{SLandGF} and cosmology \citep{cao2015cosmology}. Measuring the mass components of early-type galaxies, constraining their stellar initial mass function, and determining their inner mass density profiles \citep{Treu2006,Bolton2008,Auger2009,Auger2010,Spiniello2012,Spiniello2014,Wucknitz2004,Koopmans2006,Spingola2018}, probing the nature of dark matter through detailed modelling of the surface brightness distribution of lensed images \citep{Vegetti2012,Vegetti2014,Ritondale2019,Hsueh2020,Gilman2020}, testing models for the expansion of the Universe and dark energy \citep{Suyu2010,Suyu2013,Bonvin2017,Wong2020}, and measuring the Hubble constant \citep{Birrer2021} are some applications of studying this phenomenon.

Despite the profound insights gained by strong lensing, identifying gravitational lenses remains a challenge, as these events are rare, with the probability of one gravitational lens being found in about a thousand observed galaxies \citep{Chae2002,Wardlow2013,Amante2020}. Visual inspection of the extensive parent population is both time-consuming and susceptible to incompleteness \citep{Jackson2008,Marshall2016,More2016}. Consequently, the discovery of gravitational lenses often relies on the application of selection criteria in catalogue space, considering factors such as optical colour, radio spectral index, total flux density, and the morphology of candidate lensed images (see \citealt{Spingola2019} for an illustrative example). Despite the use of these criteria, some degree of visual inspection remains necessary to validate potential lens candidates. As we anticipate parent samples to surpass $10^7$ galaxies in size, driven by the data volumes from wide-field surveys conducted by the Vera C. Rubin Observatory \citep{rubin}, the Nancy Grace Roman Space Telescope \citep{roman}, and Euclid \citep{euclid}, the imperative for sophisticated automated search techniques becomes evident. Developing such techniques is crucial to efficiently navigate and analyze datasets of this scale. The literature has seen the emergence of several intelligent approaches tailored to diverse imaging surveys, reflecting the need for innovative and effective solutions in the face of increasingly massive datasets. \citet{Metcalf2019} conducted a lens finding challenge focused on optical/infrared datasets, revealing that automated approaches, including machine learning algorithms, outperformed traditional methods.

Convolutional Neural Networks (CNNs), a subset of machine learning models, have emerged as significant tools for cosmological and astronomical applications. These models excel at identifying patterns in complex datasets, making them highly effective for a range of scientific tasks. CNN methods have been applied in the study of exoplanet detection, where they enhance the identification and analysis of potential exoplanets from large datasets \citep{cnnast1, cnnast2}. In the field of radio astronomy, CNNs have been instrumental in classifying radio galaxies by analyzing their morphological features \citep{cnnast3, cnnast4}. They have also been used in the detection and analysis of gravitational waves, helping to filter noise and improve signal detection from data collected by observatories such as LIGO and Virgo \citep{cnnast5, cnnast6}. Furthermore, CNNs have contributed significantly to distinguishing alternative dark energy \citep{chegeni2024clusternets} and dark matter \citep{cnnast7} scenarios compared to the standard model of cosmology. In the study of the Cosmic Microwave Background (CMB) maps, CNNs have been employed to fill the masked regions of the CMB \citep{Sadr:2020rje}, extract cosmological parameters and identify subtle features in the CMB data \citep{Sadr:2020rje, cnnast8}. These diverse applications underscore the versatility and power of CNNs in enhancing our understanding of the Universe. A comprehensive review of CNN applications across various astronomical problems was presented by \citet{REZAEI2025100921}.

Numerous studies have investigated the application of CNNs for identifying strong gravitational lenses in vast datasets, automating a process that traditionally required extensive manual effort. For instance, \citet{Petrillo2019} and \citet{Rojas2021} showcased the efficiency of CNNs in detecting gravitational lenses in the Kilo Degree Survey (KiDS) data, while \citet{Nagam2023} recently proposed a new pipeline-ensemble model using Densely Connected Neural Networks (DenseNets) to reduce false positives in the identification of strong gravitational lenses, making it more suitable for large-scale astronomical surveys like Euclid. An additional example of using ensemble techniques for identifying gravitationally lensed quasars was presented by \citet{Andika2023}, where an ensemble averaging approach combines state-of-the-art convolutional and transformer-based neural networks applied to multi-band images.

Utilizing a CNN classifier on both single-band and multi-band KiDS images, \citet{Li2021} identified a sample of 97 new high-quality strong lensing candidates. This effort contributes to a total of 268 high-quality candidates from KiDS, optimizing classifier efficiency for future large-scale surveys. In addition, \citet{Rezaei2022} developed and validated CNNs for detecting galaxy-scale gravitational lenses in interferometric data from the International LOFAR Telescope (ILT), aiming for predicting a pure selection of lens candidates. The effectiveness of CNNs in these diverse settings highlights their robustness and adaptability, making them indispensable tools for the automated detection of strong gravitational lenses across various astronomical surveys.

Current research on CNNs has largely focused on refining model architectures and optimizing hyperparameters to improve performance metrics, as highlighted in recent reviews (e.g., \citealt{2018arXiv180708169R, menghani2023efficient}). While these technical improvements are essential, the quality of training data is equally crucial for CNN effectiveness. High-quality, diverse, and representative datasets significantly boost a model’s generalization ability and robustness. For instance, \citet{Canameras2023}
demonstrates that tailoring the training dataset in both the lensed and non-lensed samples can lead to improved results. That study shows how CNNs perform best when the dataset includes mock lenses closely resembling real, detectable systems with multiple lensed images that are bright and deblended. Such samples help the model to clearly differentiate between lensed and non-lensed objects. Likewise, including a high proportion of non-lensed contaminants (images resembling lenses without strong lensing features), improves the model's performance by learning to filter out such non-lensed objects.

By exploring the relationship between training data quality and the performance of a CNN, with a particular architecture performance, we aim to uncover insights that go beyond conventional optimization strategies. Specifically, we seek to understand how variations in training data characteristics, such as data distribution, labeling accuracy, and dataset size, can influence a CNN's ability to learn and generalize from the provided data. Additionally, we investigate whether certain architectures exhibit greater resilience to variations in training data.

The structure of this paper is organized as follows. Section~\ref{training_data} outlines the methodology for generating the training, validation, and test datasets, focusing on the creation of two primary classes: lensed and non-lensed data. In Section~\ref{method}, we delve into the exploration of different training strategies and their respective definitions. Moreover, we describe the CNN architectures utilized in this study. We conduct an analysis of several CNN architectures to demonstrate that the dependence on training data is not specific to any particular CNN architecture. Section~\ref{results} presents a comparative analysis of the results obtained from each training strategy, considering both the training data and the CNN architecture employed. Lastly, in Section~\ref{conclusion}, we engage in a thorough discussion of the results obtained in this study, providing insights and reflections on the methodology employed and suggesting potential avenues for future research and development.

\section{Training and Test Dataset}\label{training_data}

In order to assemble the training, validation, and test datasets for this research, it is necessary to establish two main categories of data: "lensed" and "non-lensed" samples. A lens system is created by pairing an elliptical galaxy, specifically a Luminous Red Galaxy (LRG) from the actual KiDS data release 4 (DR4; \citealt{Kuijken2019}), denoted as the "lensing galaxy" or "foreground galaxy", with a simulated lens configuration (mock lens). The same LRGs can also be used in the non-lensed class together with a selection of spiral galaxies and contaminants collected from the KiDS catalog. A visual representation of both classes is provided in Fig.~\ref{fig:lens_nonlens_samples}. The top row exhibits a diverse array of lensed samples, showcasing various morphologies and configurations, offering a representative glimpse into the spectrum of lensed phenomena within our dataset. The middle panel displays samples of LRGs, serving as both foreground lensing galaxies and non-lensed instances. Furthermore, the non-lensed category encompasses not only LRGs, but also spiral galaxies depicted in the bottom panel of Fig.~\ref{fig:lens_nonlens_samples}.

The lensed and non-lensed samples within our dataset display a wide range of surface brightness distributions. To ensure consistency in pixel value ranges across both classes of data, we implement a \textit{MinMax} normalization. The objective of this normalization is to standardize the model by aligning the distribution of inputs. Although this normalization method results in scaling the absolute loss amplitude in training, it does not affect our capability to identify lens candidates. This is because our analysis focuses on the relative surface brightness of the lensed images and non-lensed source emission within each simulated sample. Mathematically, this normalization can be expressed as,
%
\begin{equation}
\centering
    x_{\rm normalized}= \frac{x-\min(x_d)}{\max(x_d)-\min(x_d)},
\label{eq:minmax}
\end{equation}
%
where \(x\) represents the value of a specific pixel, and \(x_d\) encompasses all pixels in an image. Through this process, all pixel values within a given image are mapped to the range of \((0, 1)\). A square-root stretch is applied to this normalized image to enhance features with lower surface brightness. 

\begin{figure*}
    \centering
    \begin{subfigure}{0.16\textwidth}
        \centering
        \includegraphics[width=\linewidth]{figs/lens_samples/fig1-1.pdf}
    \end{subfigure}%
    \hfill
    \begin{subfigure}{0.16\textwidth}
        \centering
        \includegraphics[width=\linewidth]{figs/lens_samples/fig1-2.pdf}
    \end{subfigure}%
    \hfill
    \begin{subfigure}{0.16\textwidth}
        \centering
        \includegraphics[width=\linewidth]{figs/lens_samples/fig1-3.pdf}
    \end{subfigure}%
    \hfill
    \begin{subfigure}{0.16\textwidth}
        \centering
        \includegraphics[width=\linewidth]{figs/lens_samples/fig1-4.pdf}
    \end{subfigure}%
    \hfill
    \begin{subfigure}{0.16\textwidth}
        \centering
        \includegraphics[width=\linewidth]{figs/lens_samples/fig1-5.pdf}
    \end{subfigure}%
    \hfill
    \begin{subfigure}{0.16\textwidth}
        \centering
        \includegraphics[width=\linewidth]{figs/lens_samples/fig1-6.pdf}
    \end{subfigure}

    %\vspace{1em} % add some vertical space between rows

    \begin{subfigure}{0.16\textwidth}
        \centering
        \includegraphics[width=\linewidth]{figs/LRG_samples/fig1-7.pdf}
    \end{subfigure}%
    \hfill
    \begin{subfigure}{0.16\textwidth}
        \centering
        \includegraphics[width=\linewidth]{figs/LRG_samples/fig1-8.pdf}
    \end{subfigure}%
    \hfill
    \begin{subfigure}{0.16\textwidth}
        \centering
        \includegraphics[width=\linewidth]{figs/LRG_samples/fig1-9.pdf}
    \end{subfigure}%
    \hfill
    \begin{subfigure}{0.16\textwidth}
        \centering
        \includegraphics[width=\linewidth]{figs/LRG_samples/fig1-10.pdf}
    \end{subfigure}%
    \hfill
    \begin{subfigure}{0.16\textwidth}
        \centering
        \includegraphics[width=\linewidth]{figs/LRG_samples/fig1-11.pdf}
    \end{subfigure}%
    \hfill
    \begin{subfigure}{0.16\textwidth}
        \centering
        \includegraphics[width=\linewidth]{figs/LRG_samples/fig1-12.pdf}
    \end{subfigure}

    %\vspace{1em} % add some vertical space between rows

    \begin{subfigure}{0.16\textwidth}
        \centering
        \includegraphics[width=\linewidth]{figs/spiral_samples/fig1-13.pdf}
    \end{subfigure}%
    \hfill
    \begin{subfigure}{0.16\textwidth}
        \centering
        \includegraphics[width=\linewidth]{figs/spiral_samples/fig1-14.pdf}
    \end{subfigure}%
    \hfill
    \begin{subfigure}{0.16\textwidth}
        \centering
        \includegraphics[width=\linewidth]{figs/spiral_samples/fig1-15.pdf}
    \end{subfigure}%
    \hfill
    \begin{subfigure}{0.16\textwidth}
        \centering
        \includegraphics[width=\linewidth]{figs/spiral_samples/fig1-16.pdf}
    \end{subfigure}%
    \hfill
    \begin{subfigure}{0.16\textwidth}
        \centering
        \includegraphics[width=\linewidth]{figs/spiral_samples/fig1-17.pdf}
    \end{subfigure}%
    \hfill
    \begin{subfigure}{0.16\textwidth}
        \centering
        \includegraphics[width=\linewidth]{figs/spiral_samples/fig1-18.pdf}
    \end{subfigure}

    \caption{These images illustrate the dataset's diversity, presenting lensed phenomena and non-lensed galaxies. The top row showcases various lensed samples, offering insights into their morphology and configurations. The middle panel displays a selection of LRGs, serving as both foreground and non-lensed instances, while the bottom panel includes spiral galaxies. More details about the properties of these galaxies are available in Section~\ref{training_data} and Table~\ref{tab:sim_params}. Each image has a size of $101 \times 101$ pixels, which corresponds to an area of $20 \times 20$~arcsec.}
    \label{fig:lens_nonlens_samples}
\end{figure*}

In the following subsections, we provide detailed information on how both lensed and non-lensed classes of data are generated. 

\subsection{Non-lensed samples} \label{data_non-lens}

The task of distinguishing genuine gravitational lenses from various celestial objects in surveys like KiDS is challenging due to factors such as the diverse range of objects observed and variations in colour. Objects spanning different morphologies, including galaxies and artifacts, can potentially mimic lensing features, while limitations in survey resolution and depth may obscure faint lensed signals. To effectively train CNNs for lens detection, it is crucial to compile a comprehensive dataset that encompasses a wide variety of objects with diverse characteristics. By exposing the CNNs to this diverse dataset during training, they can learn to distinguish between genuine lensing events and contaminants, thereby improving the accuracy and reliability of lens detection methods in astronomical surveys. 

Our approach in selecting non-lensed samples aligns with previous studies, such as by \citet{Nagam2023} and \citet{Petrillo2019}. It comprises three distinct classes of data: a) 3000 LRGs with an $r$ magnitude below 21; b) 2000 sources that were previously misidentified as mock lenses in earlier tests conducted by \citet{Petrillo2017}; and c) 1000 galaxies visually classified as spiral galaxies through the GalaxyZoo project \citep{2014MNRAS.438.2882M, 2013MNRAS.435.2835W}. This diverse selection introduces a wide range of objects with different characteristics, enhancing the robustness of our training dataset for the CNNs.

We use the same backbone of data as in previous studies \citep{Petrillo2019, Nagam2023}, but we make changes to improve the CNN performance. In order to understand the data, we need to investigate deeper into the properties of the training dataset. For those 3000 LRGs (see above), we calculate the S{\'e}rsic profile \citep{sersic1968atlas}, which is a mathematical function used to describe the distribution of light in galaxies. The parameters of this profile provide insights into their structural properties such as size, magnitude/flux and morphology. By fitting the observed intensity profiles of LRGs, cropped to $20 \times 20$ pixel cutouts, with the S{\'e}rsic function, we can extract parameters such as the effective radius and S{\'e}rsic index. We use these cutouts to eliminate the effects of contamination in our S{\'e}rsic profile calculations. Our aim is to shed light on the nature and distribution of galaxies in our sample. Moreover, we introduced an additional parameter referred to as \textit{galaxy complexity}, which is defined as,
%
\begin{equation}
\centering
    {\rm Complexity} = \frac{S_I}{S_P},
\label{eq:compactness}
\end{equation}
%
where $S_I$ denotes the galaxy's integrated surface brightness, and $S_P$ represents the galaxy's peak surface brightness. This metric provides insights into the extent of an elliptical galaxy, helping us assess its size. In our experiments, we considered this property to enhance our understanding of the lensing galaxy's dimensions, a critical factor when selecting a suitable mock lensed system to be added to a lensing galaxy.

Fig.~\ref{fig:lensingGalaxyInfo} shows the distribution of the effective radius (top) and compactness (bottom) for the LRG samples to be considered as both non-lensed samples and foreground lensing galaxies. Notably, the selection of foreground lensing galaxies exhibits a significant bias toward effective radii in the range of 0.65 to 0.85 arcsec and complexity between 70 to 80. This observation proves invaluable, particularly in understanding the training data and its impact on the performance of trained CNNs. Later in Sections~\ref{applied1} and \ref{Applied2} we show how the properties of training data affect the completeness and purity of the generated lens candidates. 

\subsection{Lensed samples}

Lensed samples are created by combining simulated gravitational arcs, rings, quads, and doubles and a selection of foreground elliptical galaxies. These elliptical galaxies, which also act as the foreground lensing galaxies, are collected from the KiDS DR4, and therefore, provide a varied and representative collection of instances for both training and evaluation purposes. The process entails producing artificial distortions, such as arcs, rings, quads, and doubles, around selected foreground galaxies to replicate the appearance of gravitational lensing. The goal of these simulated lens configurations is to mimic the gravitational lensing characteristics observed in actual astronomical data, thereby providing a comprehensive and diverse dataset for training and evaluation. In the following, we explain how the "lensing galaxies" and the "mock lenses" are selected for the purpose of this work. 

\subsubsection{Foreground Galaxies} \label{lensing galaxy}

Similar to previous work, such as by \citet{Petrillo2017, Petrillo2019b} and \citet{Nagam2023}, we focus on low-redshift ($z \leq 0.4$), massive early-type galaxies (i.e. LRGs), which have been established as the predominant contributors to the lensing galaxy population (e.g. \citealt{Moller2007, Oguri2006}). The training dataset comprises 4,411 unique LRGs, with an additional 511 samples allocated for validation and another 511 for samples for testing. Further details on the selection process of LRGs to be considered as foreground galaxies is presented by  \citet{Nagam2023}.

In contrast to previous studies (e.g. \citealt{Nagam2023, Li2021, Petrillo2019}), which randomly select a mock lensed source and a foreground galaxy to form a lens system, we consider the properties of both components. This approach enables us to generate realistic gravitational lens samples, but to also provide our deep learning model with clear samples that minimize the risk of misinterpretation, which should improve the accuracy of our analysis. This would also enhance the model's ability to generalize to unseen data. More details on how we have adjusted the selection of LRGs to be used in combination with a mock lens is provided in Section~\ref{Applied2}.

\begin{figure}
    \centering

    \begin{minipage}{0.45\textwidth}
        \centering
        \includegraphics[width=\linewidth]{figs/fig2-1.pdf}
    \end{minipage}%
    \hfill
    \vspace{0.25cm}
    \begin{minipage}{0.45\textwidth}
        \centering
        \includegraphics[width=\linewidth]{figs/fig2-2.pdf}
        %\caption{Caption for Plot 2}

    \end{minipage}%
  

    \caption{The distribution of the lensing galaxy model parameters used for the training dataset; lensing galaxy effective radius (top) and the lensing galaxy complexity (bottom).} 
    \label{fig:lensingGalaxyInfo}
\end{figure}

\subsubsection{Mock Lenses} \label{mock_lens}
         
Developing a robust training dataset for strong gravitational lensing detection poses unique challenges, demanding representation across a diverse set of lensing configurations, while maintaining the spatial sampling of KiDS ($\sim0.2$~arcsec~pixel$^{-1}$). The selection of lens and source parameters is detailed in Table~\ref{tab:sim_params}.

We model the sources using a S\'ersic profile, and we exclude highly elliptical sources by restricting to axis ratios greater than 0.3 \citep{Petrillo2017}. The effective radius is randomly assigned within the range of 0.2 to 0.6 arcsec, while the S\'ersic index is randomly selected between 0.5 and 5, which extends lower when compared to the range considered by \citet{Petrillo2017}. This is done in order to consider a wider range of source morphologies in our samples. The range of S\'ersic indices and effective radii for the source galaxies shows a slight bias toward early-type galaxies \citep{he2020deep}. The major-axis position angle is also randomly assigned across the entire range of 0 to 180 deg, ensuring a robust and realistic dataset for source modelling. As evident from the mock lens parameter distribution shown in Fig.~\ref{fig:mock_lens_params}, the objective is not to replicate a statistically accurate representation of the real lens population. Instead, the emphasis is on densely populating the training dataset within the considered parameter space. This strategic approach, which has been previously employed by other studies \citep{Petrillo2019, Rezaei2022}, empowers developed model architectures to learn various configurations, even those that may be rare or currently unknown in real distributions. This approach enhances the model's ability to generalize beyond common scenarios.

In total, \(10^6\) mock lenses are generated, each covering a $20 \times 20$ arcsec field, introducing heightened complexity in both source and lens planes. Among these, 800,000 are randomly selected for the training phase, while the validation and test datasets each encompass 100,000 unique lens configurations.

\begin{table}
  \caption{The parameter ranges for the Singular Isothermal Ellipsoid (SIE) representing the lens and the S\'ersic model representing the source used in our simulations. These parameters contribute to the diversity of lensed images generated for training and testing purposes. The units and the range of values for each parameter are also indicated for reference.}
\begin{center}
\begin{tabular}{l l c}
Parameter              & Range & Unit \\
\hline
\multicolumn{3}{c}{Lens (SIE)}\\
\hline
Einstein radius      & 0.5 -- 5.0 & arcsec\\
Axis ratio           & 0.3 -- 1.0  & -\\
Major-axis angle     & 0.0 -- 180 & deg\\
External shear       & 0.0 -- 0.05 & -\\
External-shear angle & 0.0 -- 180 & deg\\
\hline
\multicolumn{3}{c}{Source (S\'ersic)}\\
\hline
Effective radius     & 0.2 -- 0.6 & arcsec\\
Axis ratio           & 0.3 -- 1.0 & -\\
Major-axis angle     & 0.0 -- 180 & deg\\
S\'ersic index       & 0.5 -- 5.0 & -\\
\hline
\end{tabular}

 \label{tab:sim_params}
	\end{center}
\end{table} 

\begin{figure*}
    \centering

    \begin{minipage}{0.33\textwidth}
        \centering
        \includegraphics[width=\linewidth]{figs/fig3-1.pdf}
       % \caption{Caption for Plot 1}
        \label{fig:sub1}
    \end{minipage}%
    \hfill
    \begin{minipage}{0.33\textwidth}
        \centering
        \includegraphics[width=\linewidth]{figs/fig3-2.pdf}
        %\caption{Caption for Plot 2}
        \label{fig:sub2}
    \end{minipage}%
    \hfill
    \begin{minipage}{0.33\textwidth}
        \centering
        \includegraphics[width=\linewidth]{figs/fig3-3.pdf}
        %\caption{Caption for Plot 3}
        \label{fig:sub3}
    \end{minipage}

    \caption{The distribution of the lens model parameters used for the training dataset; these are (left panel) the lens axis ratio (b/a), (middle panel) the lens external shear ($\gamma_{\rm ext}$ ), and (right panel) the lens Einstein radius ($\theta_{\rm E}$ ). Notably, the Einstein radii follow a logarithmic distribution, while the other parameters adhere to a flat distribution. The position angles of the ellipsoidal mass distribution and the external shear were set randomly between $\pm 90$ deg.}
    \label{fig:mock_lens_params}
\end{figure*}


\subsubsection{Creating real-looking lens systems}

To create realistic lens systems, we employ the method described by \citet{Petrillo2017}, which combines a chosen mock lens (detailed in Section~\ref{mock_lens}) with a potential lensing galaxy (LRGs; as explained in Section~\ref{lensing galaxy}). When combining a LRG and mock lensed emission, we adjust their peak brightness using a scaling factor \( ( 0.03 \leq K \leq 0.5) \), ensuring the lower magnitudes typically observed in lensing features relative to LRGs are preserved. Specifically, we scaled the brightness of the mock lens to \(K\) times the peak brightness of the selected LRG, allowing it to resemble the lensing galaxy in the system. Additional steps to create realistic lens systems include clipping negative pixel values to zero, which eliminates non-physical intensities, and applying a square-root stretch to emphasize low-brightness features, such as extended gravitational arcs. Finally, the images are normalized to a pixel value range of 0 and 1, ensuring uniformity across the dataset and optimizing the training process for the CNN.

Previous studies, such as by \citet{Petrillo2017}, have often employed a random selection strategy to pair mock lenses and foreground galaxies; however, in our investigation of the influence of training data on CNN performance, we identified potential limitations in this approach. Fig.~\ref{fig:confusing_radii} illustrates an example in which such a random strategy could cause potential problems by generating confusing training samples. Two scenarios are depicted: in one, a mock lens with a small Einstein radius is added to a LRG, resulting in a lens sample where the ring configuration of the lens is entirely hidden by the LRG emission. Conversely, incorporating a mock lens with a larger Einstein radius offers a more suitable match for the same LRG. This observation aligns with the equation \( M = \pi \rho \theta^2 \), where $\rho$ is the average surface mass density inside the Einstein radius (\( \theta \)). It is proportional to the square root of the enclosed mass (\( M \)). Assuming the enclosed mass correlates with the total integrated surface brightness, the Einstein radius is linked to the total flux of a galaxy. 

Avoiding the generation of samples similar to the one on the left side of Fig.~\ref{fig:confusing_radii} enhances quality assurance for the CNN. This sample, which would be labeled as lensed despite presenting no clear lensing emission, can be considered a non-lensed sample. In other words, incorporating nearly identical samples with differing labels into the training dataset leads to confusion for the model. This underscores the importance of carefully selecting the pair of foreground galaxy and mock lensed emission, as it significantly influences the composition of the generated lens population in the training dataset.

By avoiding such confusing samples, the model trains more effectively and produces the expected output, translating to a lower false positive (FP) rate. Reducing the FP rate is particularly important in large-scale surveys such as KiDS and Euclid, where the volume of data is immense, and manual inspection of each candidate lens is impractical. High FP rates make it challenging to accurately identify true lenses. This not only wastes valuable time and resources, but also introduces uncertainties that can propagate into subsequent analyses, leading to erroneous conclusions about the properties and distribution of galaxies and dark matter. 

\begin{figure}
    \centering
    \includegraphics[width=\columnwidth, trim=1cm 0cm 2cm 0cm, clip]{figs/fig4.pdf}
    \caption{An example of a potential issue arising when the lensed emission is faint with respect to the brightness of the foreground lensing galaxy, and has an Einstein radius that is significantly lower than the effective radius of the foreground lensing galaxy. Two scenarios employing the same LRG as the foreground galaxy are shown, but with different lens configurations. As depicted, when lensed emission with a smaller Einstein radius is introduced to the selected LRG, the ring configuration of the lensed emission is entirely obscured by the LRG emission. The inclusion of lensed samples, such as the example on the left (labeled as 1 or lensed), may confuse the CNN model, as this sample resembles a non-lensed sample (labeled as 0) in our training dataset.}
    \label{fig:confusing_radii}
\end{figure}

\section{Method}\label{method}
In this section, we provide an overview of the architecture employed in our deep learning algorithm designed for the detection and ranking of gravitational lensing candidates. CNNs recognized for their adeptness in processing input imaging data with a topological structure, stand out as the primary approach for object detection and classification. The strength of CNNs lies in their capacity to utilize multiple layers, each serving a distinct function, and the arrangement of these layers can produce various convolutional components. As a consequence, the efficacy of a CNN is closely tied to the specific components implemented, and the performance may vary based on the application's requirements. 

To quantify the performance of our network in terms of dissimilarity between estimated and true class labels, we employ loss functions. In our task of binary classification to distinguish between lensed and non-lensed objects, we have chosen the Binary Cross Entropy (BCE) form of the loss function for our evaluations. The BCE loss function, represented as,
\begin{equation}
{\rm BCE} = -\frac{1}{N} \sum_{i=1}^{N} \left[ y_i \log p_i + (1 - y_i) \log (1 - p_i) \right],
\end{equation}
is employed where \( y_i \) denotes the given class label for the \( i \)th sample in our dataset of \( N \) training samples, and \( p_i \) represents the estimated probability of the model indicating the \( i \)th sample as a strong gravitational lens system.

While some studies have developed custom architectures for strong lens detection \citep{Rezaei2022}, leveraging existing network structures is a common practice in the field. Among the widely utilized architectures, \textit{ResNet} \citep{He2016} stands out as one of the most popular choices in the strong lensing literature, as evidenced by studies such as from \citet{Petrillo2017} and \citet{Lanusse2018}.  The \textit{ResNet} model is built on the concept of training deeper CNNs by incorporating shortcuts or by skipping connections between the front and back layers. This strategy helps in facilitating the backpropagation of gradients during training, allowing for better optimization of the model. In a comparative analysis conducted by \citet{Nagam2023}, the performance of \textit{DenseNet} \citep{Huang2017} in comparison to \textit{ResNet} was assessed. The findings revealed that \textit{DenseNet} achieved comparable true positive rates while exhibiting lower false positive rates. The \textit{DenseNet} model builds upon the skipped connections concept, but introduces dense connections between all previous and subsequent layers. 

This unique characteristic allows \textit{DenseNet} to achieve superior performance compared to \textit{ResNet}, all while requiring fewer parameters and incurring less computational cost. Motivated by those results, we have opted for \textit{DenseNet} in our further analysis. Specifically, we explore multiple variants, including \textit{DenseNet-121} and \textit{DenseNet-169}. 


Furthermore, we investigate various architectures based on the understanding that the selection of model architecture plays a crucial role in determining the overall performance in tasks such as strong lens detection.
Taking this direction further, we consider another branch of CNN architectures, called  \textit{EfficientNet} \citep{tan2019efficientnet}, which achieve state-of-the-art performance on image classification tasks while also being computationally efficient. In particular, we investigate various versions, such as \textit{EfficientNet-B3} and \textit{EfficientNet-B4}. This comprehensive analysis aims to uncover insights into how different neural network architectures influence the accuracy and reliability of strong lens detection models. In the following, we provide an overview of these selected architectures. 

\subsection{\textit{DenseNet}}
\textit{DenseNet}, short for Densely Connected Convolutional Networks, is a type of neural network architecture that emphasizes dense connectivity within "dense blocks". In each dense block, every layer receives the feature maps generated by all preceding layers as input, while also passing on its own feature maps to every subsequent layer in the block. This dense connectivity structure is distinct from traditional CNNs, where each layer is only connected to the immediately following layer. By establishing direct connections between all layers in a block, \textit{DenseNet} allows each layer to directly access the features of all previous layers, promoting both efficient information flow and rich feature representation. \textit{DenseNet} comes in various versions, such as \textit{DenseNet-121}, \textit{DenseNet-169}, and \textit{DenseNet-201}, where the numbers represent the total number of layers in each network. The choice of model variant depends on the complexity of the task and the available computational resources. For tasks that demand a balance between model complexity and computational efficiency, \textit{DenseNet-121} and \textit{DenseNet-169} are widely adopted due to their favorable trade-off between performance and resource consumption. We refer the interested reader to the review by \citet{Huang2017} for a detailed discussion on the architecture of \textit{DenseNet}.

\subsection{\textit{EfficientNet}}

\textit{EfficientNet} is a family of CNNs with the key innovation of compound scaling, which enables an optimal trade-off between model size and performance by uniformly scaling the network's depth, width, and resolution. \textit{EfficientNet} has been widely used in a variety of scientific applications, such as galaxy morphology classification, \citep{kalvankar2020galaxy}, spectral classification of astronomical objects \citep{wu2023automatic}, skin cancer detection \citep{VENUGOPAL2023100278}, brain tumor detection \citep{nayak2022brain}, and lung cancer detection
\citep{raza2023lung}. 

The key innovation of \textit{EfficientNet} lies in the balance between model depth, width, and resolution, as governed by a compound scaling method. This approach ensures that the network scales efficiently across these dimensions, making it well-suited for diverse tasks. Each variant of \textit{EfficientNet} is denoted by a scaling factor (e.g., \textit{EfficientNet-B3}, \textit{EfficientNet-B4}), reflecting its capacity for increased depth and width. These scaling factors allow users to choose a model that aligns with the specific requirements of their task and computational resources. This concept can be mathematically expressed as,
\begin{equation}
    \alpha .\, \beta^2.\, \gamma^2 \approx 2
\end{equation}
for $\alpha \geqslant 1$, $\beta \geqslant 1$, and $\gamma \geqslant 1$. Here, the depth is $\alpha^{\phi}$, the width is $\beta^{\phi}$ and the resolution is $\gamma^{\phi}$, where $\phi$ denotes the scaling coefficient that uniformly scales the network.
The choice of scaling coefficients impacts the trade-off between model complexity and computational efficiency, making \textit{EfficientNet} a versatile architecture that is adaptable to different resource constraints and task requirements. Further details on the structure of \textit{EfficientNet} are given by \citet{tan2019efficientnet}. 

 In our investigation, we incorporate \textit{EfficientNet-B3} and \textit{EfficientNet-B4}, considering their widespread adoption and ability to strike a suitable balance between model complexity and computational efficiency in various applications. The initial weights were randomly set using a uniform distribution. While different weight initializations can impact the training process during the early epochs, our experiments show that the network converges shortly after this period. Thus, the observed performance remains largely unaffected by the initial weight settings.


\subsection{Evaluation criteria}

To assess and compare the effectiveness of different methodologies and training data strategies on the test dataset, it's essential to establish appropriate evaluation criteria. Given that lens detection is treated as a classification problem, where samples are categorized as lensed or non-lensed, the following representation can be utilized for evaluation purposes:
\[
\begin{array}{cc|cc}
\multicolumn{2}{c}{} & \multicolumn{2}{c}{\text{True Data}} \\
\cline{3-4}
\multicolumn{2}{c|}{} & \text{Lens} & \text{Not Lens} \\
\cline{2-4}
\text{Test Results} & \text{Lens} & TP & FP \\
\cline{2-4}
 & \text{Not Lens} & FN & TN \\
\cline{2-4}
\end{array}
\label{tab:sample_confusion}
\]

In this representation, True Positive (TP) indicates correctly identified gravitational lens systems, True Negative (TN) corresponds to accurately recognized non-lensed sources, False Positive (FP) denotes mis-classification of non-lensed sources as gravitational lenses, and False Negative (FN) refers to gravitational lensing events missed by the algorithm and classified as non-lensed sources.

Based on these terms, several evaluation criteria can be defined. Accuracy measures the proportion of correctly identified samples (TP and TN) out of the total number of samples. Precision quantifies the ratio of true positives to the sum of true positives and false positives, indicating the reliability of positive predictions. Recall assesses the fraction of true positives correctly identified by the algorithm, reflecting the model's completeness. Fall-out, also known as the false positive rate, calculates the proportion of negative samples incorrectly classified as positive, providing insight into the purity of detected lens candidates. Mathematically, these metrics are represented as:
\begin{equation}
\begin{aligned}
\text{Accuracy} &= \frac{TP + TN}{TP + FN + TN + FP}, \\
\text{Precision} &= \frac{TP}{TP + FP}, \\
\text{Recall} &= \frac{TP}{TP + FN}, \\
\text{Fall-out or FP rate} &= \frac{FP}{FP + TN}.
\end{aligned}
\end{equation}
In assessing gravitational lens search algorithms, the emphasis extends beyond completeness alone, especially given the anticipation of a large number of gravitational lenses to be detected with upcoming all-sky surveys. Rather, the focus often centres on achieving a low FP rate, with the aim of identifying a high number of genuine lens candidates in the ranked list. Additionally, the Receiver Operating Characteristic (ROC) curve serves as another valuable metric. The ROC plot visually represents the trade-off between the TP rate and the FP rate for each model. Each point on the ROC curve signifies a different threshold for classifying samples as positive or negative based on their predicted probabilities. By examining the ROC plot, we can discern how well each model discriminates between positive (lensed) and negative (non-lensed) samples. A model with better performance will exhibit a curve that closely approaches the top-left corner of the plot, indicating higher TP rate and lower FP rate across various threshold values. The comparison of ROC curves for different models provides insights into their relative effectiveness in identifying lensed samples. This analysis aids in selecting the most suitable model for the task at hand, considering both sensitivity to TPs and robustness against FPs.

\section{Results} \label{results}
The success of any machine learning model relies on the quality of the training dataset. This section examines various properties of training datasets, focusing on their impact in strong lens detection projects. The relevance and representation of the training dataset are crucial. A well-curated dataset must encompass a diverse and representative set of examples that reflect the variety and complexity of real-world data. In the context of strong lens detection, this means including images with different types of strong lensing phenomena, as well as a high variety in the non-lensed samples. Also, a representative dataset helps the model learn to generalize from the training data to unseen data, improving its robustness and accuracy. Another important factor is accurate labeling as the model relies on these labels to learn the correct associations between input data and the desired output. An example of the importance of accurate labeling in the context of strong lens detection is provided in Fig.~\ref{fig:confusing_radii}.

Considering the significant impact of training data on the performance of CNNs, we have provided two novel strategies to handle the training dataset, complementing the conventional approach typically adopted in the literature (e.g. \citealt{Petrillo2019,Nagam2023}), which we refer to as the Vanilla strategy. Our aim is to craft a dataset that enhances the purity of detected candidates. While completeness is still a consideration, in this study, we prioritize the necessity of mitigating FPs.

Our first approach, termed "Applied1", prioritizes the treatment of non-lensed samples within the training dataset. This strategy is tailored to address specific challenges associated with the characterization of non-lensed objects. Conversely, our second approach, "Applied2", focuses on optimizing the representation of the lens population within the training dataset, thereby enhancing the CNN's ability to accurately identify and classify gravitational lensing events. Further elaboration on these strategies is provided below, including their respective methodologies and rationales for effectively training CNNs in gravitational lens detection tasks.

Although we have tested CNN architectures on various training datasets, we ensured that each round incorporated the same number of samples to provide a fair analysis of model performance. Each CNN architecture was trained on a total of 500,000 samples. This training dataset was balanced with 250,000 samples labeled as lensed and 250,000 as non-lensed.


\subsection{Vanilla} \label{van}

\begin{figure*}
    \centering
    \includegraphics[ scale=0.56]{figs/fig5.pdf}
    \caption{The ROC plot demonstrates how different machine learning models, trained with the Vanilla setting, perform in distinguishing between positive (lensed) and negative (non-lensed) samples across various thresholds balancing true positive (TP) and false positive (FP) rates. For presentation purposes, the range of the ROC plot has changed from [0,1] to the current display. By comparing these curves, we can identify the most efficient model for detecting lensed samples, while minimizing FPs. The plotted results show the improvement in FP rate as we use an ensemble technique, by averaging the predicted lens probability of individual models. The best performing ensemble belongs to averaging the output of \textit{EfficientNet-B3}, \textit{EfficientNet-B4} and \textit{DenseNet-121} with the FP rate of $1.1 \times 10^{-3}$ and a TP rate of 0.906.}
    \label{fig:roc_vanila}
\end{figure*}

\begin{table}
    \caption{The main evaluation metrics, such as true positive (TP) and false positive (FP) rates, derived from training the featured models, using the Vanilla setting on the training dataset. The predicted lensing probability from each model is a value between 0 and 1. However, for calculating the evaluation metrics, we have established a threshold of 0.99 to differentiate between lensed and non-lensed samples. The test dataset consists of 96,072 samples equally distributed between lensed and non-lensed categories.}
    \centering
    \begin{tabular}{ccc}
    Model & TP & FP\\
    \hline
       DenseNet-121  &0.925& $5.3 \times 10^{-3}$ \\
%       \hline
       DenseNet-169 &0.925&$6.4 \times 10^{-3}$ \\
%       \hline
       EfficientNet-B3 &0.929&$3.6 \times 10^{-3}$ \\
%       \hline
       EfficientNet-B4 &0.937&$7.1 \times 10^{-3}$ \\
%       \hline
       DenseNet-121, EfficientNet-B3 &0.908&$1.3 \times 10^{-3}$ \\
%       \hline
       DenseNet-169, EfficientNet-B3 &0.910 &$1.8 \times 10^{-3}$ \\
%       \hline
       DenseNet-121, EfficientNet-B4 &0.916&$2.7 \times 10^{-3}$ \\
%       \hline
       DenseNet-169, EfficientNet-B4 &0.915&$3.1 \times 10^{-3}$ \\
%       \hline
       EfficientNet-B3, EfficientNet-B4 &0.921&$2.1 \times 10^{-3}$ \\
%       \hline
       DenseNet-121, DenseNet-169 &0.910 &$2.7 \times 10^{-3}$ \\
%\hline
    DenseNet-121, EfficientNet-B3, EfficientNet-B4&0.906 &$1.1 \times 10^{-3}$\\
%    \hline
    DenseNet-169, EfficientNet-B3, EfficientNet-B4&0.906 &$1.2 \times 10^{-3}$\\
    \hline
    \end{tabular}

    \label{tab:roc_vanilla}
\end{table}


\begin{figure}
\begin{center}
\begin{minipage}[b]{\columnwidth}
\centering
\includegraphics[height=2.85cm]{figs/fig6-1.pdf}
\end{minipage}
\begin{minipage}[b]{\columnwidth}
\centering
\includegraphics[height=2.85cm]{figs/fig6-2.pdf}
\end{minipage}
\newpage
\caption{A selection of LRGs falsely detected as strong lens systems by the Vanilla setting. These samples (FPs) are having a high complexity value of 200 or more, which are not well represented in the Vanilla setting. More details regarding these samples are provided in Table~\ref{tab:high_comp_details}.}
\label{fig:sample_high_compactness}
\end{center}
\end{figure}

The Vanilla setting, pioneered by \citet{Petrillo2017} and recently utilized by \citet{Nagam2023} with an expanded repertoire of non-lensed samples, stands as a benchmark methodology for gravitational lens detection tasks using KiDS data. As described above, the generation of lensed samples entails a random pairing of a lens and an LRG sample to create a realistic-looking strong lensing sample. Within this framework, non-lensed samples are predominantly drawn from a pool of observed spiral (contaminations) and elliptical galaxies. These samples, originating from authentic KiDS observations, undergo augmentation to mitigate overfitting and enhance model performance. While \citet{Nagam2023} maintains a ratio of 20 per cent LRGs to 80 per cent contamination samples in their non-lensed selection process, we have opted for a balanced approach, incorporating an equal number of LRGs and contaminations in our dataset. This deliberate choice aims to provide the CNNs with a more comprehensive representation of LRGs, thereby strengthening their capacity to differentiate between LRGs acting as foreground galaxies in gravitational lens systems and those exhibiting typical LRG characteristics devoid of lensing effects.

Fig.~\ref{fig:roc_vanila} shows the performance comparison of the four different models: \textit{DenseNet-121}, \textit{DenseNet-169}, \textit{EfficientNet-B3}, and \textit{EfficientNet-B4}. Since the model output represents the probability of an object being a lens, ranging between 0 and 1, setting a threshold is necessary to classify objects into two classes: lensed and non-lensed. To ensure a stringent selection of strong lenses, a threshold of 0.99 is set for further analysis. This decision is driven by our primary goal of reducing the FP rate. As illustrated in Fig.~\ref{fig:roc_vanila}, increasing the threshold does not significantly impact the TP rate. In other words, it is acceptable to sacrifice a small percentage of the TP rate to achieve a better FP rate.

Building upon findings of \citet{Rezaei2022, Andika2023}, it has been observed that averaging the output probabilities of multiple models can effectively mitigate FP rates. This improvement stems from the models' tendency to disagree on suspicious cases, thus allowing for a voting scheme to be applied. As depicted in Fig.~\ref{fig:roc_vanila}, averaging the model outputs indeed yields enhanced results. 

\cref{tab:roc_vanilla} provides a detailed overview of each method's performance when the threshold is set at 0.99. Notably, the average of \textit{EfficientNet-B3}, \textit{EfficientNet-B4}, and \textit{DenseNet-121} demonstrates the most promising outcomes, with a FP rate of $1.1 \times 10^{-3}$ and an acceptable TP rate of 0.90. Following closely, the average of \textit{EfficientNet-B3}, \textit{EfficientNet-B4}, and \textit{DenseNet-169} achieves a similar TP rate and a slightly higher FP rate of $1.2 \times 10^{-3}$. These results underscore the effectiveness of leveraging ensemble methods to enhance the performance of gravitational lens detection algorithms. 

In order to comprehensively understand the model behavior and analyze its performance, we conducted a thorough investigation into the samples contributing to the FP rate of $1.1 \times 10^{-3}$ by our best model. Several examples are provided in Fig.~\ref{fig:sample_high_compactness} and their properties are given in Table~\ref{tab:high_comp_details}. Our analysis revealed that a significant portion of these FPs stem from the misclassification of extended LRGs as strong gravitational lensed samples. This observation, coupled with the complexity distribution shown in the lower panel of Fig.~\ref{fig:lensingGalaxyInfo}, indicates that our training data for LRG samples lacks a sufficient number of instances representing LRGs with extended emission or those with nearby contaminants. This realization prompted us to introduce the first variation of this study, which is detailed in the next sub-section.

This investigation highlights the importance of ensuring the diversity and representativeness of training data not only in lensed samples, but also in non-lensed samples. By addressing the imbalance in the representation of extended LRGs in our training dataset, we aim to enhance the robustness of our model in distinguishing between genuine gravitational lenses and other astronomical objects.

\begin{table}
    \centering
     \caption{Properties of FPs detected by the best performing model using the Vanilla setting. The indices correspond to the images depicted in Fig.~\ref{fig:sample_high_compactness}. The lens probability and uncertainty metrics were derived from the averaged predictions of the lensing probability across the \textit{EfficientNet-B3}, \textit{EfficientNet-B4}, and \textit{DenseNet-121} models. These findings indicate that all three models consistently predict a probability exceeding 99 per cent for this selection of LRGs, even though they are non-lensed samples.}

    \begin{tabular}{cccc}
       Index &  Lens probability & Complexity & Effective radii\\
       \hline 

       a &  0.997  & 271 & 21.28\\ 

        b&  0.999 & 282 & 24.73\\ 

        c&  0.995 &283 & 32.41\\ 
        d&  0.999&  297 & 21.17\\ 

        e&  0.999 &  285 & 25.38\\ 

        f&  0.997 & 400 & 50.23\\ \hline

    \end{tabular}
       \label{tab:high_comp_details}
\end{table}


\begin{figure}
    \centering
    \includegraphics[width=\columnwidth]{figs/fig7.pdf}
    \caption{A comparison of the complexity between observational KiDS LRGs data provided by \citet{RuiLi2020} and the LRG samples within our non-lensed population from the training dataset. It highlights the adjustments made to the LRG population, achieved through strategic augmentation of LRG samples (in blue). These modifications aim to enhance the visibility of sources with less representative samples, particularly those with complexity values exceeding 100. By augmenting these samples, the CNN is exposed more to the complex LRGs and, thereby enhancing its ability in identifying and analyzing LRGs as non-lensed samples.}
    \label{fig:compactness}
\end{figure}


\subsection{Applied1} \label{applied1}
\begin{figure*}
    \centering
    \includegraphics[scale=0.6]{figs/fig8.pdf}
    \caption{The ROC plot comparing the performance of trained CNN models under the Applied1 scenario in distinguishing between lensed and non-lensed samples. Given the significance of minimizing FPs, the most effective model is one that can provides a purer selection of candidates. Consequently, the ensemble comprising \textit{EfficientNet-B3}, \textit{EfficientNet-B4}, and \textit{DenseNet-169}, with an FP rate of $3.5 \times 10^{-3}$ and a TP rate of 0.906, emerges as the top-performing model.}
    \label{fig:roc_app1}
\end{figure*}

\begin{table}
    \centering
\caption{The primary evaluation metrics obtained from training the CNN models using the Applied1 setting on the training dataset. The models predict lensing probability values ranging between 0 and 1. However, to calculate the evaluation metrics, we have set a threshold of 0.99 to distinguish between lensed and non-lensed samples. The test dataset comprises 96,072 samples evenly split between lensed and non-lensed categories.}
    \begin{tabular}{ccc}
    Model & TP & FP\\
    \hline
       DenseNet121  &0.899&$1.4 \times 10^{-3}$ \\

       DenseNet169 &0.914&$2.9 \times 10^{-3}$ \\

       EfficientNetB3 &0.946&$2.2 \times 10^{-3}$ \\

       EfficientNetB4 &0.951&$2.4 \times 10^{-3}$ \\

       DenseNet121, EfficientNetB3 &0.898& $6.8 \times 10^{-4}$ \\

       DenseNet169, EfficientNetB3 &0.900&$5.4 \times 10^{-4}$ \\

       DenseNet121, EfficientNetB4 &0.898&$6.2 \times 10^{-4}$ \\

       DenseNet169, EfficientNetB4 &0.910&$4.9 \times 10^{-4}$ \\

       EfficientNetB3, EfficientNetB4 &0.938&$9.5 \times 10^{-4}$ \\

       DenseNet121, DenseNet169 &0.886&$6.2 \times 10^{-4}$ \\

    DenseNet121, EfficientNetB3, EfficientNetB4&0.897 &$4.7 \times 10^{-4}$\\

    DenseNet169, EfficientNetB3, EfficientNetB4&0.906 &$3.5 \times 10^{-4}$\\
    \hline
    \end{tabular}
    
    \label{tab:roc_app1}
\end{table}

\begin{table}
    \centering
     \caption{The lens probability of the FPs detected using the Vanilla setting that were not detected as FPs using the Applied1 scenario. Alongside the ensemble predicted lens probability, the individual lens probabilities for \textit{EfficientNet-B3} (ENet-B3), \textit{EfficientNet-B4} (ENet-B4), and \textit{DenseNet-169} (DNet-169) are provided. All of these predictions are based on the Applied1 settting. A visual representation of these samples can be found in Fig.~\ref{fig:sample_high_compactness}, while additional details regarding the Vanilla setting predictions are presented in Table~\ref{tab:high_comp_details}.}
    \begin{tabular}{ccccc}
       index & Ensemble &ENet-B3 &ENet-B4&DNet-169\\
       \hline
       a &  0.65 &  0.950 & 0.997 & $3.3 \times 10^{-6}$\\

        b&  0.96 & 0.999 & 1 & 0.89\\

        c&  0.64 & 0.950 & 0.98 & $1.5 \times 10^{-3}$, \\

        d&  0.98& 0.999 & 1  &0.95\\

        e&  0.67 & 0.999 & 0.999 & $1.1 \times 10^{-2}$, \\

        f&  0.70 & 0.991 & 0.998 &0.12\\ \hline

    \end{tabular}
   
    \label{tab:high_comp_details_app1}
\end{table}

We now demonstrate how the properties of the training dataset can significantly influence the achieved results and the overall performance of the model. This is done by changing the characteristics and sample population within the training dataset, while maintaining consistency in the test dataset. 
Drawing insight from our analysis of the FP samples presented in Section~\ref{van}, 
here we focus on improving the representation of the training dataset for non-lensed samples. By incorporating this solution, we have achieved compelling results, which we detail below. 

Fig.~\ref{fig:compactness} shows the distribution of complexity values among the LRG samples within our training dataset. Notably, this distribution exhibits non-uniformity, which is particularly evident in the complexity range of 120 and higher, where less than 1.4 per cent of the LRG samples are represented. To validate these findings, we compare them with the complexity distribution of a larger LRG dataset produced by \citet{RuiLi2020}. This dataset comprises low-redshift samples $(z \leq 0.4)$ selected based on their $r$-band magnitude, limited to values below 20. The complexity distribution of these LRG samples is also shown in Fig.~\ref{fig:compactness}. Remarkably, the comparison reveals that the distribution of LRG complexity in our training dataset does not accurately reflect the distribution observed in real observational data.

Even when augmentation techniques are applied to the current LRG samples, the resulting distribution tends to mirror the original distribution, which worsens the imbalance between compactness categories. Therefore, augmentation alone does not offer a viable solution to this issue. To address this challenge, we propose a strategic augmentation approach aimed at achieving a more balanced distribution of samples across various complexity bins. Specifically, we advocate for augmenting the less frequent sources with greater variations compared to the more densely populated regions in the complexity distribution. This strategic augmentation aims to ensure a semi-uniform distribution of samples across various complexity categories, thereby enhancing the representativeness of our training dataset. The distribution of the generated LRG population with this strategy is also shown in Fig.~\ref{fig:compactness}.

Following this, we adapted our training dataset to align with the adjusted LRG population. However, we have kept other training parameters, such as the total number of training data points, the distribution of lensed samples, the learning rate, and other hyperparameters consistent with the Vanilla setting. The results obtained from this training data configuration, referred to as "Applied1", are presented in Fig.~\ref{fig:roc_app1}. It is evident that the best achieved FP rate comes from the ensemble of \textit{EfficientNet-B3}, \textit{EfficientNet-B4}, and \textit{DenseNet-169}, with a value of $3.5 \times 10^{-4}$, marking a notable decrease from the FP rate of $1.1 \times 10^{-3}$ observed for the Vanilla setting. With a total of 48,036 non-lensed samples in our test dataset, the number of FP detections has decreased from 53 to 17, which represents more than a threefold reduction. This effect will likely be further magnified when dealing with a larger test dataset, such as the 126,884 LRGs from the KiDS DR4, as described by \citet{RuiLi2020}.

Moreover, this reduction in the number of FPs has not compromised the TP rate. The specific details of the FP and TP rates are presented in Table~\ref{tab:roc_app1}. In comparison to the Vanilla setting, where the ensemble of \textit{DenseNet-121}, \textit{EfficientNet-B3}, and \textit{EfficientNet-B4} yielded the best results, in this scenario, \textit{DenseNet-169} has replaced \textit{DenseNet-121} in the ensemble. The combination of \textit{EfficientNet-B3}, \textit{EfficientNet-B4} and \textit{DenseNet-169} achieves a TP rate that is 0.9 per cent better, while exhibiting a slightly higher FP rate of $1.2 \times 10^{-4}$,  when compared to the ensemble of \textit{EfficientNet-B3}, \textit{EfficientNet-B4} and \textit{DenseNet-121}.


Table~\ref{tab:high_comp_details_app1} offers some insight to the variability of individual model predictions and their influence on the ensemble probability. Also, the recorded samples are shown in Fig.~\ref{fig:sample_high_compactness} and their predicted lensing probabilities under the Vanilla setting are given in Table~\ref{tab:high_comp_details}. A comparison between Tables~\ref{tab:high_comp_details} and \ref{tab:high_comp_details_app1} highlights the better performance of the training strategy implemented in Applied1, when compared to the results using the Vanilla setting.

\subsection{Applied2} \label{Applied2}
\begin{figure*}
    \centering
    \includegraphics[scale=0.56]{figs/fig9.pdf}
    \caption{The ROC plot comparing the performance of trained CNN models under the Applied1 scenario in distinguishing between lensed and non-lensed samples. In this scenario, both the lensed and non-lensed populations have been altered compared to the Vanilla setting. The ensemble of \textit{EfficientNet-B3}, \textit{EfficientNet-B4}, and \textit{DenseNet-121} achieves a TP rate of 0.9112 with a FP rate of $4.16 \times 10^{-4}$. The achieved TP rate surpasses that of both the Vanilla and Applied1 settings.}
    \label{fig:roc_app2}
\end{figure*}

We now introduce our second modification to the training dataset. While the previous section addressed issues concerning the non-lensed population, the focus here is on potential challenges within the lensed population. As discussed previously in Fig.~\ref{fig:confusing_radii}, it is critical to consider the morphology of the foreground galaxy when pairing it with a mock lens. Our adjustments target how the lens population is constructed within the training dataset. As outlined in Section~\ref{training_data}, this process involves pairing mock lenses with foreground galaxies to simulate realistic strong gravitational lens systems. Through visual examination, we identified potential issues, particularly with small-separation lensed emission ($\theta_E < 0.85$ arcsec). An analysis of the mock lens parameters (see the right panel of Fig.~\ref{fig:mock_lens_params}) reveals that a considerable portion of existing samples in the training dataset belong to the small Einstein radii population, with approximately 23 per cent having Einstein radii below 0.85 arcsec.

Our primary focus is on selecting appropriate lensing galaxies, specifically tailored for these smaller Einstein radii lenses. An example illustrating the potential issue is presented in Fig.~\ref{fig:confusing_radii}, where an unsuitable choice of lensing galaxy results in a perplexing sample that lacks distinct lensed emission. Given the considerable likelihood (approximately 23 per cent) of such mock lenses being incorporated into the training dataset, it is crucial to address this challenge. Therefore, we propose selecting only a subset of LRGs as potential foreground galaxies for small Einstein radii lenses. This approach ensures that the distribution of source parameters remains consistent with the previous analysis. The variable aspect here is the selection process for the corresponding lensing galaxy associated with the mock lensed emission that have smaller Einstein radii. In Section~\ref{training_data}, we discuss the assumption that the Einstein radius ($\theta_E$) is proportional to the total integrated flux of a foreground galaxy. Here, we use the LRGs' effective radii as a proxy for their integrated flux. Under the "Applied2" setting, the training data is modified to include only LRGs with effective radii below 0.5 arcsec, paired with mock lensed emission with Einstein radii below 0.85 arcsec. However, it is important to note that the same test dataset has been utilized here as in the Vanilla and Applied1 settings. 

Fig.~\ref{fig:roc_app2} and Table~\ref{tab:roc_app2} show the results obtained from employing this strategy. We see that the most optimal performance is achieved by the ensemble of \textit{EfficientNet-B3}, \textit{EfficientNet-B4}, and \textit{DenseNet-121}, with a FP rate of $4.16 \times 10^{-4}$ when a detection threshold of 0.99 is used. This translates to the detection of 20 FPs within the 48,036 total non-lensed samples in our dataset, which is three more than the number of FPs detected using the Applied1 setting (see Figs.~\ref{fig:roc_app1} and \cref{tab:roc_app1}). Interestingly, this variation in the training dataset, by solely modifying the lens class of the training data, has influenced the number of FP detections by the model. This indicates that the challenges of strong lensing detection are complex, relying on multiple parameters. Another notable point is the 222 additional TP samples achieved using the Applied2 setting, when compared to the Applied1 and Vanilla settings. This corresponds to approximately a 0.46 per cent improvement in TPs. This presents a trade-off between selecting a model with a higher TP rate or a lower FP rate, considering that both models significantly outperform the Vanilla setting in terms of the FP rate.


\begin{table}
    \centering
        \caption{A comparison of the TP and FP rates for different CNN architectures, when the training data follows the Applied2 scenario. The best results are obtained using an ensemble of \textit{DenseNet-121}, \textit{EfficientNet-B3} and \textit{EfficientNet-B4} with a TP rate of 0.9112 and a FP rate of $4.16 \times 10^{-4}$.}
    \begin{tabular}{ccc}
    Model & TP & FP\\
    \hline
       DenseNet-121  &0.931& $4.1 \times 10^{-3}$ \\
%       \hline
       DenseNet-169 &0.932&$3.7 \times 10^{-3}$ \\
%       \hline
       EfficientNet-B3 &0.941&$3.2 \times 10^{-3}$ \\
%       \hline
       EfficientNet-B4 &0.940&$1.9 \times 10^{-3}$ \\
%       \hline
       DenseNet-121, EfficientNet-B3 &0.917&$9.3 \times 10^{-4}$ \\
%       \hline
       DenseNet-169, EfficientNet-B3 &0.918& $1.1 \times 10^{-3}$ \\
%       \hline
       DenseNet-121, EfficientNet-B4 &0.817& $6.2 \times 10^{-4}$ \\
%       \hline
       DenseNet-169, EfficientNet-B4 &0.918& $8.7 \times 10^{-4}$ \\
%       \hline
       EfficientNet-B3, EfficientNet-B4 &0.927&$8.5 \times 10^{-4}$ \\
%       \hline
       DenseNet-121, DenseNet-169 &0.916&$1.2 \times 10^{-3}$ \\
%\hline
    DenseNet-121, EfficientNet-B3, EfficientNet-‌B4&0.911 &$4.2 \times 10^{-4}$\\
%    \hline
    DenseNet-169, EfficientNet-B3, EfficientNet-B4&0.919 &$6.2 \times 10^{-4}$\\
    \hline
    \end{tabular}
    \label{tab:roc_app2}
\end{table}

\begin{figure}
\begin{center}
\begin{minipage}[b]{\columnwidth}
\centering
\includegraphics[height=2.25cm]{figs/fig10-1.pdf}
\end{minipage}
\begin{minipage}[b]{\columnwidth}
\centering
\includegraphics[height=2.25cm]{figs/fig10-2.pdf}
\end{minipage}
\newpage
\caption{The 8 FP samples within the test dataset of 48,036 non-lensed samples, when considering the prediction of the Applied1 and Applied2 settings. All of the samples belong to the LRG population, which indicates that the models have learned to distinguish between spiral galaxies and lensed samples. The remaining issue is in separating the FP samples belonging to non-lensed LRGs with those that exhibit lensed emission.}
\label{fig:app1_app2_FPs}
\end{center}
\end{figure}


\subsection{Combined model} \label{discussion}
\begin{figure*}
    \centering
    \includegraphics[scale=0.7]{figs/fig11.pdf}
    \caption{The ROC plot comparing the performance of each training dataset setting (Vanilla, Applied1 and Applied2). This illustrates the dependency of the TP and FP rates on the chosen threshold that separates lensed and non-lensed samples. The combined model incorporates the average lensing probability of all three training settings, demonstrating an improved FP rate, albeit with a slight reduction of the TP rate by a few per cent.}
    \label{fig:roc_combined}
\end{figure*}

Our findings have thus far highlighted the impact of modifications in the training dataset on the model's performance. We have observed that adjustments to either the lensed or non-lensed classes of data can influence the TP and FP rates, thus affecting the overall quality and reliability of the results obtained. A comparison between Tables~\ref{tab:roc_app1} and \ref{tab:roc_app2} reveals that the Applied2 setting exhibits a higher FP rate, resulting in three more FP samples compared to the Applied1 setting, within our non-lensed test dataset of 48,036 samples. Upon examination, it becomes apparent that the two models only agree on 8 FP samples, which are shown in Fig.~\ref{fig:app1_app2_FPs}. Among these, the predicted lensing probability of only 4 samples exceeds 0.99 for the Vanilla setting, indicating that if a voting strategy were employed across all models, we would identify only 4 FP samples within the entire non-lensed sample; these are labeled as (b), (c), (f), and (h) in Fig.~\ref{fig:app1_app2_FPs}. However, incorporating a strategy that averages out the predicted lensing probability for the Vanilla, Applied1 and Applied2 settings results in 5 FP samples, which now includes sample (a) in Fig.~\ref{fig:app1_app2_FPs}. This intriguing result also underscores how different ensemble approaches can impact the actual number of FPs encountered. Another interesting finding is that all of the FPs shown in Fig.~\ref{fig:app1_app2_FPs} are LRGs, which do not exhibit any spiral emission. This suggests that the primary challenge in reducing the number of FPs from the KiDS data lies not in spiral emission, but rather in distinguishing between the LRG population and contaminates that may mislead the model into interpreting them as lensed emission.

Fig.~\ref{fig:roc_combined} presents a comparison of different settings in the training dataset, namely the Vanilla, Applied1 and Applied2 settings. As previously discussed, averaging through all of these predictions yields the combined lens probability prediction, which demonstrates a better FP rate, as the models may not agree on identifying challenging samples as lensed. A more detailed view of these results is provided in Table~\ref{tab:roc_combined}, which shows the efficacy of ensemble techniques. According to our results, the combined model exhibits a FP rate of $10^{-4}$, representing an 11-fold decrease in the number of FPs compared to the Vanilla setting, a 3.5-fold improvement compared to Applied1, and a 4.1-fold improvement compared to Applied2. This significant reduction in FPs comes at the cost of a 2.3 per cent decrease in TPs when compared to the Vanilla and Applied1 settings, while compared to the Applied2 setting, the TP rate has decreased by 2.8 per cent. In the following, we provide details on how these TP and FP rates are related to the population of lensed and non-lensed samples and their underlying properties.



\begin{table}
    \centering
    \caption{A comparison of TP and FP rates for different training dataset settings (Vanilla, Applied1, and Applied2), alongside the combined model. These results highlight the superior performance of the combined model in detecting fewer FPs compared to each of the investigated training settings. The selected threshold is 0.99.}
    \begin{tabular}{ccc}
      Model   & TP & FP\\
      \hline
      Vanilla &0.906 & $1.1 \times 10^{-3}$ \\

      Applied1&0.906 & $3.5 \times 10^{-4}$ \\

      Applied2&0.911 &$4.2 \times 10^{-4}$\\

      Combined & 0.883 & $1.0 \times 10^{-4}$\\
      \hline
    \end{tabular}
    
    \label{tab:roc_combined}
\end{table}

\subsection{Parameter-space analysis} \label{parameterspace}

Utilizing the results obtained from Fig.~\ref{fig:roc_combined} and Table~\ref{tab:roc_combined}, we now investigate the parameter-space for detection, specifically examining how the Einstein radius of a lensed object or the complexity of the LRG influences the TP and FP rates. Through these experiments, our aim is to understand how different training datasets affect the types of lenses to which the trained CNN architectures are sensitive, as well as the parameters that may impact the model's ability to accurately label samples in the test dataset.

Fig.~\ref{fig:tp_er} shows the TP rate as a function of the lens Einstein radius. This result indicates that although the Applied2 setting shows a slightly higher TP rate when compared to the Vanilla and Applied1 settings, this difference is consistent across all Einstein radius bins, as opposed to belonging to any specific range. On the other hand, when examining the behavior of the combined model, it becomes evident that this model exhibits a lower TP rate, when compared to the individual models. This outcome is expected because in the combined model, a sample is labeled as a lens if all three settings of Vanilla, Applied1 and Applied2 agree. This ensemble technique, achieved by averaging the predicted lens probabilities of each setting, results in a smaller set of final candidates, as is also demonstrated in Table~\ref{tab:roc_combined}.

\begin{figure}
    \centering
    \includegraphics[width=\linewidth]{figs/fig12.pdf}
    \caption{The distribution of TP rates as a function of the Einstein radius of the mock lenses. This reveals that the effectiveness of the detection method does not seem to be directly influenced by the size of the Einstein radius of the mock lenses. As expected from the results presented in Table~\ref{tab:roc_combined} and Fig.~\ref{fig:roc_combined}, the Combined model has the lowest TP rate, when compared to the individual models.}
    \label{fig:tp_er}
\end{figure}

Another noteworthy comparison lies in examining the FP rate alongside the complexity of the foreground lens galaxies. As previously demonstrated, the adjustment made in the training dataset for Applied1 aims to balance the non-lensed population, ensuring that extended, complex sources are adequately represented in that class of data. A comparison between the Vanilla, Applied1 and Applied2 settings clearly highlights the impact of such adjustments on the achieved results. Fig.~\ref{fig:fp_com} illustrates a significant difference in the FP rate obtained by the Vanilla, Applied1 and Applied2 settings. An intriguing observation from Fig.~\ref{fig:fp_com} is the disparity between the FP rates of the Applied1 and Applied2 settings, despite both utilizing the exact same non-lensed samples. The sole difference between these two settings lies in the distribution of lensed samples. This observation underscores the complexity of the lens detection problem, where the behavior of the model can be influenced by numerous parameters that may initially seem unrelated.

\begin{figure}
    \centering
    \includegraphics[width=\linewidth]{figs/fig13.pdf}
    \caption{The FP rate as a function of the LRG compactness employed within the non-lensed population. This illustrates the influence of the distribution of the implemented training dataset on the behavior of the Vanilla, Applied1 and Applied2 settings.}
    \label{fig:fp_com}
\end{figure}

\begin{figure*}
    \centering
    \includegraphics[width=\linewidth]{figs/fig14.pdf}
    \caption{A comparison of the predicted lens probability for 126,000 KiDS LRG samples, as a real test dataset. As expected, the Combined model has the lowest number of predicted lens candidates, when compared to the other scenarios that have been tested in this study.}
    \label{fig:all_DR4}
\end{figure*}

\subsection{Evaluation on real KiDS data}

To evaluate the performance of our methodology on real KiDS data, we have applied the Vanilla, Applied1, Applied2 and Combined strategies to 126,000 LRGs from the KiDS DR4 \citep{RuiLi2020}. The predictions for each lensing probability bin are presented in Fig.~\ref{fig:all_DR4}. We find that the results align well with our previous analyses, such as those shown in Fig.~\ref{fig:roc_combined}. The Vanilla setting predicts the highest number of lens candidates for the probability range of [0.9--1], whereas the Applied2, Applied1, and Combined settings predict fewer samples with a high lensing probability. When using a threshold of 0.99 to identify potential lenses, the Combined strategy identifies 347 samples, whereas the Vanilla setting identifies 997 samples, indicating a threefold reduction in the number of lens candidates. This reduction is advantageous as it minimizes the need for expert visual inspection and reduces the time required for follow-up observations to confirm the strong gravitational lensing nature of these candidates. The Applied1 and Applied2 settings detect 551 and 710 samples, respectively, with a lensing probability higher than 0.99. It is crucial to assess how many genuine lensed samples are being missed among the detected candidates and whether some lenses are being overlooked in this selection process. This aspect will be addressed in our future work, where we will verify the predicted lens candidates through visual inspection.

\section{Conclusions} 
\label{conclusion}

In this study, we have investigated the complicated task of detecting gravitational lens systems through analyzing the intricate relationship between the composition of the training dataset and the performance of the detection models. Through a comprehensive analysis and experimentation, we uncovered vital insights that shed light on the multifaceted challenges and opportunities within this domain. Our findings underscored the importance of data diversity and representativeness, revealing how variations in the sample populations can cause significant influence on the behavior and efficacy of the detection models. Our study highlighted the importance of understanding the underlying reasons for model output, beyond merely assessing its performance. This iterative process often necessitates a return to the foundational step of data collection and analysis. For instance, our research revealed that the underlying distribution of LRGs in our non-lensed sample affected the number of FPs. By continuously refining the data collection methods, reassessing the dataset properties, and fine-tuning the model architectures, based on the insights gleaned from the model behaviors, we iteratively enhanced the accuracy and robustness of the detection models. 

One pivotal discovery from our research revolves around the critical need to address imbalances within the training dataset. In particular, distinguishing between extended, complex LRGs and genuine gravitational lenses posed a significant challenge. We noticed that the original test dataset, which has been widely used in the literature, has an un-balanced population of LRGs as non-lensed samples in terms of complexity. The number of complex LRGs are significantly lower in the original dataset, when compared to the compact ones. By strategically mitigating these imbalances, through techniques such as data augmentation and ensemble learning approaches, we were able to achieve notable reductions in the number of FPs, which enhanced the overall reliability of our detection model.

Beside the population of non-lensed samples in the training dataset, we also made modifications that focused on the lensed population. The adjustments changed how the lensed population is constructed, particularly concerning lenses with Einstein radii below 0.85 arcsec. We proposed to only use a subset of LRGs with a small effective radius, to act as potential foreground galaxies for these small Einstein radii lenses. The effectiveness of this modification was evaluated through experimentation, which showed an improved performance in terms of the TP and FP rates, when compared to the standard Vanilla setting.

Our examination of the FP samples obtained from the Applied1 and Applied2 settings revealed that within our test dataset of 48,036 non-lensed samples, Applied1 incorrectly labeled 17 samples as strong lenses, while Applied2 identified 20 samples as strong lenses. Further analysis showed that among those samples, Applied1 and Applied2 shared only 8 common FP samples. Interestingly, when incorporating the lensing probability predicted by the Vanilla setting into the ensemble average, only 5 FP samples remained, all of which were found to be LRGs without any evidence of spiral emission. This underscored the effectiveness of ensemble methods and highlighted the challenge of distinguishing LRG populations from potential contaminants in reducing the FP rate.

The FP rate achieved when we used the ensemble method of the Vanilla, Applied1 and Applied2 (Combined) setting was found to be $10^{-4}$, which represented an 11-fold improvement, when compared to the Vanilla setting. While this reduction in FPs came with a 2.3 per cent decrease in TPs, when compared to the Vanilla and Applied1 settings, it signified an advancement considering our primary goal of minimizing FP rates in strong gravitational lens detection algorithms.

We then evaluated our methodology using real KiDS data by applying the various strategies (Vanilla, Applied1, Applied2, and Combined) to a dataset of 126\,000 LRGs. The results showed that different strategies yielded varying numbers of high-probability lens candidates, with the Combined strategy significantly reducing the number of candidates, when compared to the Vanilla setting. This reduction is beneficial for minimizing the effort required for expert visual inspection and follow-up observations. However, it remains essential to investigate how many genuine lenses may be missed and whether some lenses are overlooked from this process. In a future work, we will focus on addressing these concerns through detailed visual inspection of the predicted lens candidates.

In conclusion, our research contributes significantly to advancing the field of gravitational lens detection by offering insights into the interplay between the training dataset, model performance, and the iterative refinement process. By emphasizing the importance of data diversity, imbalance mitigation, and continuous refinement through iterative analysis, our study provides a road-map for developing accurate and reliable detection models that are capable of unraveling the mysteries of the Universe's most intriguing phenomena.

\section*{Acknowledgements}
This work was performed using the compute resources from the Academic Leiden Interdisciplinary Cluster Environment (ALICE) provided by Leiden University.
AC was supported by the MUR PRIN2022 project 20222JBEKN with title "LaScaLa" - funded by the European Union - NextGenerationEU. This work is based on the research supported in part by the National Research Foundation of South Africa (Grant Number: 128943).


%%%%%%%%%%%%%%%%%%%%%%%%%%%%%%%%%%%%%%%%%%%%%%%%%%
\section*{DATA AVAILABILITY}
Upon reasonable request, the underlying data used for this article will be shared by the corresponding author.

%%%%%%%%%%%%%%%%%%%% REFERENCES %%%%%%%%%%%%%%%%%%

% The best way to enter references is to use BibTeX:

\bibliographystyle{mnras}
%\bibliography{bibliography} 
\documentclass{MITstyle}

%\usepackage[table]{xcolor}
\usepackage{chngcntr}
\usepackage{hyperref}
\usepackage{microtype}

\title{A Lightweight and Extensible Cell Segmentation and Classification Model for Whole Slide Images}

\author{Nikita Shvetsov~$^{1, }$\footnote{Correspondence e-mail: nikita.shvetsov@uit.no}, Thomas K. Kilvaer~$^{2, 3}$, Masoud Tafavvoghi~$^{4}$, Anders Sildnes~$^{1}$, \\ Kajsa Møllersen~$^{4}$, Lill-Tove Rasmussen Busund~$^{5, 6}$, Lars Ailo Bongo~$^{1}$ \\
%
\vspace{1em} % Space between authors and afilliations
%
\normalfont{\small $^{1}$Department of Computer Science, UiT The Arctic University of Norway}\\
\normalfont{\small $^{2}$Department of Oncology, University Hospital of North Norway}\\
\normalfont{\small $^{3}$Department of Clinical Medicine, UiT The Arctic University of Norway}\\
\normalfont{\small $^{4}$Department of Community Medicine, UiT The Arctic University of Norway}\\
\normalfont{\small $^{5}$Department of Medical Biology, UiT The Arctic University of Norway} \\
\normalfont{\small $^{6}$Department of Clinical Pathology, University Hospital of North Norway} %\vspace{2em}
}

\begin{document}
\maketitle

\section*{Abstract}

% \begin{abstract}
% Developing clinically useful cell-level analysis tools in digital pathology remains challenging due to limitations in dataset granularity, inconsistent annotations, computational demands of advanced models, and difficulties in integrating new technologies into clinical workflows. To address these challenges, we propose a multi-faceted solution that enhances data quality, model performance, and usability to create a lightweight and extensible cell segmentation and classification model.

% First, we update data labels by employing a cross-relabeling process that refines the labels of two existing datasets, PanNuke and MoNuSAC, to create a new unified dataset with enhanced granularity, encompassing seven distinct cell types. Second, we leverage the H-Optimus foundation model as a fixed encoder to improve feature representation for simultaneous cell segmentation and classification tasks. Third, to address the computational demands of foundation models, we employ knowledge distillation to reduce model size and complexity while maintaining comparable performance. Finally, to facilitate integration into clinical workflows, we integrate the distilled model into the QuPath software, a widely used open-source platform in digital pathology.

% Our results demonstrate improvements in cell segmentation and classification performance using the H‑Optimus-based model compared to a CNN-based model. Specifically, the average $R^2$ improved from 0.575 to 0.871, and the average $PQ$ score improved from 0.450 to 0.492, indicating better alignment with actual cell counts and enhanced segmentation and classification quality. Furthermore, the distilled student model maintains performance comparable to the larger foundation model while reducing the parameter count by a factor of 48.
% Overall, by reducing computational complexity and integrating it into existing workflows, the proposed approach may significantly impact diagnostic processes, reduce the workload of pathologists, and contribute to improved patient outcomes. Though our approach shows potential enhancements in efficiency and usability of cell segmentation and classification models in digital pathology, extensive validation is needed to deploy these models in clinical practice.
% \end{abstract}

%%% shortened abstract
\begin{abstract}
Developing clinically useful cell-level analysis tools in digital pathology remains challenging due to limitations in dataset granularity, inconsistent annotations, high computational demands, and difficulties integrating new technologies into workflows. To address these issues, we propose a solution that enhances data quality, model performance, and usability by creating a lightweight, extensible cell segmentation and classification model. 

First, we update data labels through cross-relabeling to refine annotations of PanNuke and MoNuSAC, producing a unified dataset with seven distinct cell types. Second, we leverage the H-Optimus foundation model as a fixed encoder to improve feature representation for simultaneous segmentation and classification tasks. Third, to address foundation models' computational demands, we distill knowledge to reduce model size and complexity while maintaining comparable performance. Finally, we integrate the distilled model into QuPath, a widely used open-source digital pathology platform. 

Results demonstrate improved segmentation and classification performance using the H-Optimus-based model compared to a CNN-based model. Specifically, average $R^2$ improved from 0.575 to 0.871, and average $PQ$ score improved from 0.450 to 0.492, indicating better alignment with actual cell counts and enhanced segmentation quality. The distilled model maintains comparable performance while reducing parameter count by a factor of 48. By reducing computational complexity and integrating into workflows, this approach may significantly impact diagnostics, reduce pathologist workload, and improve outcomes. Although the method shows promise, extensive validation is necessary prior to clinical deployment.
\end{abstract}
\clearpage

\section{Introduction}
In digital pathology, accurate segmentation and classification of cells are crucial for many diagnostic, prognostic, and predictive analyses \cite{Jaber_Beziaeva_etal._2019,Lin_Pan_etal._2022,Park_Ock_etal._2022,Shen_Choi_etal._2024}. Nowadays, developments in computational pathology offer multiple solutions \cite{H._Qu_P._Wu_etal._2020,Javed_Mahmood_etal._2020} to utilize cell-level datasets to train machine learning models that solve these problems. The quality and specificity of training datasets are critical for robust and accurate models. Adhering to the principle of "garbage in, garbage out", it is essential to ensure that these datasets are extensively and accurately labeled with distinct classes that reflect the diverse biological characteristics of different cell types. Unfortunately, the number of open-source datasets comprising such high-quality annotations is limited. Existing cell segmentation datasets \cite{Gamper_Koohbanani_etal._2019,Graham_Vu_etal._2019,Verma_Kumar_etal._2021} may offer extensive annotations for certain cell types while providing more general labels for others. For example, in PanNuke, which is one of the largest open-source datasets comprising labeled cells, various types of morphologically and functionally different inflammatory cells like macrophages and lymphocytes are clustered in a broad "inflammatory" class. Consequently, these classes are frequently omitted from analyses or aggregated into broader meta-classes \cite{Gamper_Koohbanani_etal._2020} and likely interfere with other cell classes included in the dataset. This and similar inconsistencies in annotation granularity limit the ability of machine learning models to learn the comprehensive and nuanced features necessary for accurate cell segmentation and classification. To address these challenges, methods for refining and standardizing dataset annotations are essential to enhance the quality of training data.

A complementary approach to mitigate the absence of high-quality training data is the use of foundation models. Foundation models as encoders are defined as large-scale, versatile networks pre-trained on vast, diverse datasets using self-supervised learning, contrasting with convolutional neural network (CNN) pre-trained encoders that rely on supervised learning with labeled data. In practice, foundation models leverage enormous amounts of weakly or unlabeled data from millions of whole slide images (WSIs) and employ self-attention mechanisms to capture long-range dependencies and global context \cite{Chen_Ding_etal._2024,Saillard_Jenatton_etal._2024,Vorontsov_Bozkurt_etal._2024,Xu_Usuyama_etal._2024}. As a consequence, foundation models are able to produce transferable feature representations across different cell types and tissue environments. The feature representations can be leveraged by decoder networks to produce segmentation masks and pixel-level classifications. Because foundation models have comprehensive feature representations, they can be effectively fine-tuned using much smaller amounts of cell-level data compared to the large datasets needed to train models from scratch. Furthermore, foundation models incorporate adversarial training elements or contrastive learning \cite{Chen_Ding_etal._2024,Xu_Usuyama_etal._2024}, enhancing their resilience and adaptability by exposing them to challenging and varied scenarios during training. This may result in more generalizable models, often making them well-suited for diverse and complex tasks in digital pathology.

Despite the inherent advantages of foundation models, their deployment for practical use faces its own obstacles. In particular, they require substantial computational power, financial investments and rigorous testing to ensure reliability and efficacy for a given task \cite{Akkus_Dangott_etal._2022,Dragomir_Cocuz_etal._2022,Go_2022,Jafri_Farooqui_etal._2024}. Moreover, while foundation models enhance feature representation and performance, they depend on the quality of available annotations for decoder fine-tuning and, like any other model, cannot resolve existing inconsistencies or ambiguities in data labels. Therefore, there remains a critical need for solutions that address both data quality and practical deployment considerations.
Further, integrating new technologies into existing clinical workflows often encounters resistance, as it necessitates adjustments to established diagnostic processes. So, there is a need to develop solutions that could be integrated into current practices, minimizing the burden on medical professionals to adopt new tools \cite{King_Williams_etal._2023}.

Existing solutions \cite{Goldsborough_Philps_etal._2024,Hörst_Rempe_etal._2024}, while addressing some aspects of these challenges, fall short in providing a comprehensive approach. To address the data quality and clinical deployment issues, we propose a multi-faceted solution that encompasses data refinement, model optimization, and integration with existing pathology tools (\hyperref[fig:fig1]{Figure 1}). The outcome is a lightweight cell segmentation and classification model that can be integrated into digital pathology workflows for practical clinical use.

\begin{figure}[h!]
    \centering
    \includegraphics[width=\textwidth, height=0.82\textheight, keepaspectratio]{images/Figure_1.pdf}
    \caption{Overview of the proposed solution, including 1) Data refinement using cross-relabeling, 2) Teacher model development and fine tuning, 3) Student model optimization with knowledge distillation and 4) Student model and QuPath integration}
    \label{fig:fig1}
\end{figure}
\clearpage

Our approach begins with preparing the data for the fine-tuning and training of the machine learning models. We create a refined dataset, acquired via cross-relabeling two cell-level datasets, enhancing annotation specificity and consistency of the labeled data. Subsequently, we create a cell segmentation and classification model based on the foundation model. We leverage the foundation model as a fixed encoder and fine-tune a decoder using the refined dataset to improve generalization across diverse tissue- and cell types.
To ensure that the model remains lightweight and deployable in a possibly resource-constrained environment, we employ knowledge distillation to approximate the functionality of the foundation model. Finally, to facilitate the practical application of our model in digital pathology workflows, we integrate it with the QuPath \cite{Bankhead_Loughrey_etal._2017} application. Each methodological component contributes to the overarching goal of enhancing model performance, generalizability, and usability in clinical settings.

The primary contributions of this paper are:
\begin{enumerate}
    \item \textit{Data labels refinement through cross-relabeling:}
    
    We propose a new method for refining labels of cell-level datasets through cross-relabeling. This method employs classification models to re-label broad and ambiguous instances, resulting in a more diverse dataset. Our evaluation demonstrates that these classification models achieve high accuracy on test subsets, indicating the reliability of the method for label refinement.

    \item \textit{Enhanced model performance via foundation models:}
    
    We employ a foundation model as a feature extractor for the cell segmentation and classification task. In comparison with training a CNN model from scratch, the foundation model backbone only needs fine-tuning, which significantly reduces training time, computational resources and data requirements. We show that using a foundation model encoder leads to better performance in cell segmentation and classification networks than using a CNN-based encoder. This improvement may enable the model to generalize more effectively across various tissue types and imaging methods.
    
    \item \textit{Model optimization through knowledge distillation:}
    
    We show that a smaller student model trained using knowledge distillation on the refined dataset obtained via our cross-relabeling approach from a foundation model achieves comparable performance in cell segmentation and quantification tasks. As a result, this model is more suitable for deployment in environments without high-performance computing resources.
    
    \item \textit{Integration with QuPath:}
    
    We integrate the distilled cell segmentation and classification model into QuPath, a widely used open-source digital pathology platform, to accelerate clinical adaptation by enabling pathologists to more easily incorporate advanced computational tools into their existing workflows.
\end{enumerate}

Through these methodological steps, we aim to bridge the gap between advanced machine learning techniques and practical clinical applications, making accurate and efficient digital pathology accessible in a broader range of healthcare settings.

\section{Refining Existing Datasets Using Cross-Relabeling}
To address the limitations of sparse and ambiguous labeling of cell-level datasets, we propose a generalizable cross-relabeling strategy that can be applied to any dataset containing broadly categorized or imprecisely labeled cell types. This approach involves training and subsequently leveraging classification models to refine broad categories into more specific or biologically relevant classes.
When applied to cell-level data, the methodology includes extracting individual cell images from the dataset patches, preprocessing these images to standardize the size and accommodate partial cells, and then training deep learning classifiers capable of distinguishing between the finer cell subtypes within the coarser categories. 
To illustrate our approach, we focus on the PanNuke \cite{Gamper_Koohbanani_etal._2020, Gamper_Koohbanani_etal._2019} and MoNuSAC \cite{Verma_Kumar_etal._2021} datasets that we have used to train models for cell quantification in our previous works \cite{Shvetsov_Grønnesby_etal._2022,Shvetsov_Sildnes_etal._2024}. We find that for better cell differentiation we have to introduce more granular labels. PanNuke includes a broad classification of "inflammatory" cells, encompassing lymphocytes, macrophages, and neutrophils. Each cell type differs significantly in structure, function, and clinical relevance. Conversely, MoNuSAC uses the label "epithelial" for a class that comprises both benign epithelial cells and malignant neoplastic cells. This practice makes it challenging to differentiate between benign and malignant epithelial cells in the dataset, which is a critical distinction when identifying tumor areas within tissue samples. To address these issues, we implement a cross-relabeling strategy as shown in \hyperref[fig:fig2]{Figure 2}. The key components are two classification models: one is trained on singular cell images from PanNuke data to classify the epithelial meta-class into epithelial and neoplastic classes. The other is trained on MoNuSAC to refine the inflammatory class into lymphocytes, neutrophils, and macrophages.

\begin{figure}[h!]
    \centering
    \includegraphics[width=\textwidth]{images/Figure_2.pdf}
    \caption{Refined dataset generation via cross relabeling}
    \label{fig:fig2}
\end{figure}

The refining approach consists of three consecutive steps. The first is the preprocessing step, in which we extract individual cells from both datasets (\hyperref[fig:fig3]{Figure 3}). The specifics of PanNuke and MoNuSAC patch preparation before cell preprocessing are provided in \hyperref[chap:S1]{Appendix S1}.

\begin{figure}[h!]
    \centering
    \includegraphics[width=\textwidth]{images/Figure_3.pdf}
    \caption{Cell instances preprocessing including (1) cell map extraction, (2) bounding box delineation, (3) adjusting cell boxes and (4) cropping and resizing of cell images}
    \label{fig:fig3}
\end{figure}

During preprocessing, we extract cell type maps from the ground truth label mask and calculate bounding boxes around each cell instance. To accommodate partial cells at patch borders, a common issue in cropped patch images, we employ mirror padding and extend the field of view of the cell label by 15 pixels to capture adjacent cells. We then crop and resize the identified regions to $64 \times 64$ pixels using bicubic interpolation.

The preprocessed PanNuke dataset comprises 68,031 neoplastic and 23,207 epithelial cell images, while MoNuSAC comprises  33,104 lymphocytes, 1,252 neutrophils, and 1,695 macrophages, which we subsequently use in training cell classification models and classifying the cell image data \hyperref[fig:S2]{Appendix Figure S2 (1)}. 

The next step is to train two distinct ResNet50-based classifiers tailored to address the specific labeling challenges inherent in each dataset. We use ResNet50 for classification models due to its proven effectiveness for image classification tasks in histopathology \cite{pan2022reviewmachinelearningapproaches}, and its compatibility with small images. For the PanNuke dataset, we design the classifier, trained on MoNuSAC data, to disaggregate the heterogeneous "inflammatory" cell category into distinct subtypes: lymphocytes, macrophages, and neutrophils. Similarly, for the MoNuSAC dataset, the classifier is trained on PanNuke data and distinguishes between benign and malignant epithelial cells within the overarching "epithelial" label. By applying these targeted classifiers to their respective datasets, we assign more specific labels to individual cell instances, thus enabling us to create a unified dataset.
To ensure a balanced representation of classes, we train both models on datasets that had been equalized to match the size of the least represented class. Thus, we obtain datasets comprising 23,207 samples per class for PanNuke and 1,252 samples per class for MoNuSAC data. Next, we partition both of them into training (70\%), validation (20\%), and testing (10\%) subsets. To mitigate the risk of overfitting, we use a single dropout layer with a rate of p=0.5 in both models and data augmentation using randomized color perturbations, rotation, and horizontal and vertical flipping. We employ AdamW optimizer and the cross-entropy loss function for the training criterion.

To evaluate the two trained models, we measure the classification accuracy on the respective test subsets. The accuracies on the test subset for both classifiers are presented in \hyperref[tab:1]{Table 1}. The PanNuke model achieves an average accuracy of 93.57\%, with higher accuracy for neoplastic cells (96.06\%) compared to epithelial cells (86.26\%). The confusion matrix in Figure A3.1 shows that the model predominantly distinguishes accurately between epithelial and neoplastic tissues, with a substantial number of correct classifications and relatively few misclassifications. The MoNuSAC model demonstrates an average accuracy of 98.92\%, excelling in classifying lymphocytes (99.67\%) and macrophages (94.12\%), with lower performance for neutrophils (85.71\%). The confusion matrix in Figure A3.2 shows that the model identifies lymphocytes and performs reasonably well with macrophages and neutrophils.

\begin{table}[h!]
\renewcommand{\arraystretch}{1.5}
  \centering
  \caption{Cell classification results for PanNuke and MoNuSAC trained models (CI 95\%).}
  \label{tab:1}
  \begin{tabular}{|l|c|c|}
   \hline
   %\rowcolor{gray!30}
    Accuracy               & PanNuke model              & MoNuSAC model              \\
    \hline
    Average      & 0.936 (0.931--0.941)         & 0.989 (0.986--0.993)        \\
    \hline
    Neoplastic   & 0.961 (0.956--0.965)         & -                          \\
    \hline
    Epithelial   & 0.863 (0.849--0.877)         & -                          \\
    \hline
    Lymphocytes  & -                          & 0.997 (0.995--0.999)        \\
    \hline
    Neutrophils  & -                          & 0.857 (0.796--0.918)        \\
    \hline
    Macrophages  & -                          & 0.941 (0.906--0.976)        \\
    \hline
  \end{tabular}
\end{table}

Finally, during the last step, we use the model trained on PanNuke data for epithelial cells in MoNuSAC and the model trained on MoNuSAC for the inflammatory cells class in PanNuke. Specifically, we use classifier models to relabel epithelial cells in MoNuSAC and inflammatory cells in PanNuke data. Then we combine cells with refined labels and the rest of the cells in both datasets to create a refined dataset (\hyperref[fig:S2]{Appendix Figure S2 (2)}). The process of relabeling cells and visualizing them on a patch is shown in \hyperref[fig:fig4]{Figure 4}. The cell counts in the refined dataset are provided in \hyperref[tab:S4]{Appendix Table S4}.

\begin{figure}[h!]
    \centering
    \includegraphics[width=\textwidth, height=0.42\textheight, keepaspectratio]{images/Figure_4.pdf}
    \caption{Cell relabeling procedure for epithelial and inflammatory cell classes}
    \label{fig:fig4}
\end{figure}

%\hfill

Relabeling and combining datasets have been explored in a prior study \cite{Parulekar_Kanwat_etal._2023}, where consecutive fine-tuning on multiple datasets was employed to account for hierarchical class label structures. While the method presented in \cite{Parulekar_Kanwat_etal._2023} is intuitive, it often lacks consistency and requires multiple fine-tuning runs, which can be cumbersome and time-consuming. 
In contrast, cross-relabeling simplifies this process by using specialized classification models tailored to each dataset's specific labeling challenges. This approach provides better transparency and produces a unified dataset encompassing seven distinct cell types across multiple tissue samples, enhancing data diversity for further model training or fine-tuning.

Despite these improvements, cross-relabeling does not entirely resolve issues related to poor labeling quality or the amount of labeled data. Specifically, our results show lower accuracies persist for underrepresented classes, such as macrophages, which may stem from a limited sample availability and intrinsic challenges in distinguishing these cells based solely on H\&E staining. Furthermore, while our method enhances label specificity, it relies on the initial quality of the broad labels; thus, any fundamental inaccuracies in the original annotations can propagate through the relabeling process. Addressing the overall problem of limited data labels may require integrating additional data sources or utilizing complementary immunohistochemical staining methods.
Although the reported performance metrics are obtained from evaluations on the native test sets of each dataset, it is important to note that the primary application of these classifiers is to perform cross-relabeling, where a model trained on one dataset (e.g., PanNuke) is applied to another (e.g., MoNuSAC) and vice versa. We acknowledge that a more systematic evaluation of cross-dataset generalization is needed and could be performed in future work.

Overall, the refined dataset produced by our approach can enhance the supervised training or fine-tuning of cell segmentation and classification models, especially those that utilize pre-trained foundation models to improve feature extraction robustness. In addition, these models can detect nuanced classes that enable researchers to conduct more detailed analyses of biological processes in computational pathology.

\section{Foundation models for robust cell segmentation and classification}

Accurate cell segmentation and classification in digital pathology are hindered by limited labeled data and the fact that conventional CNNs are unable to capture global contextual information due to their local receptive field constraints \cite{Gheflati_Rivaz_2022,Yang_Marcus_etal.}. Traditional approaches in cell quantification have predominantly relied on CNN encoders, such as ResNet50, given their proven effectiveness in semantic segmentation tasks \cite{Deshmane_2023,Graham_Vu_etal._2019,Mukasheva_Koishiyeva_etal._2024,Stringer_Wang_etal._2021}. However, approaches that include fine-tuning of pretrained CNNs, data augmentation, and stain normalization to partially increase data variability and address staining differences often fail to achieve the necessary generalization and robustness across diverse tissue types and staining conditions \cite{G._Wang_W._Li_etal._2018,Gao_Bagci_etal._2018,Karim_El_Khoury_Martin_Fockedey_etal._2021}.

To overcome these challenges, we leverage an encoder-decoder network that uses a foundation model as the encoder and a CNN upsampling decoder (\hyperref[fig:fig5]{Figure 5}) for simultaneous cell segmentation and classification in 2D patches extracted from WSIs. Foundation models with transformer-based architectures are viable alternatives to CNN-based encoders \cite{Shamshad_Khan_etal._2023,Sourget_2023}. They enable the creation of more advanced architectures that can decode or transform learned features more effectively \cite{Chen_Duan_etal._2023,Cheng_Misra_etal._2022,Xie_Wang_etal._2021}.

\begin{figure}[h!]
    \centering
    \includegraphics[width=\textwidth]{images/Figure_5.pdf}
    \caption{UNETR-like model with foundational model as backbone}
    \label{fig:fig5}
\end{figure}

By utilizing a transformer-based encoder, we incorporate global contextual information into the feature extraction process, which is a key advantage of such architectures \cite{Chen_Lu_etal._2021}. This foundation model integration facilitates accurate pixel-wise segmentation and classification without the need for extensive encoder training, thereby potentially improving generalization across varied cellular structures and tissue types.
In our implementation, we employ a modified UNETR \cite{Hatamizadeh_Tang_etal._2021} architecture that combines a vision transformer (ViT) \cite{Dosovitskiy_Beyer_etal._2021} encoder with a CNN-based decoder. The encoder utilizes the pretrained H-Optimus foundation model, which contains 1.1 billion parameters and is trained on over 500,000 H\&E stained WSIs \cite{Saillard_Jenatton_etal._2024}. We extract outputs from four evenly spaced transformer blocks $Z_i$, where $i \in [1, 14, 26, 38]$, to serve as residual connections for the CNN decoder. We select these blocks based on our observation that features from non-adjacent levels of the encoder lead to better overall performance on the test subset.

The CNN decoder upsamples the feature representations, acquired from the transformer blocks, to generate an intermediate vector that is handled by two task-specific layers that generate cell segmentation and classification masks. The first task-specific layer is the ‘Cellpose head’,  which is used to delineate cell instances. The layer generates horizontal and vertical gradient maps to form vector fields that are refined through gradient tracking in a post-processing step using the Cellpose algorithm \cite{Stringer_Wang_etal._2021}, known for its efficacy in cell segmentation tasks and generalizability across multiple domains \cite{Pachitariu_Stringer_2022,Stringer_Pachitariu_2024}. The second task-specific layer is the "Cell type head", which assigns labels to individual pixels. In the post-processing step, we determine the output classification label of each segmented cell instance by majority voting over the labeled pixels that comprise the cell in the segmentation map.

To evaluate model performance and measure the impact of adding a foundation model as backbone, we compare it to a ResNet50-based model. ResNet50 is a widely used solution for encoders in segmentation architectures in the medical domain \cite{Deshmane_2023,Graham_Vu_etal._2019,Mukasheva_Koishiyeva_etal._2024,Stringer_Wang_etal._2021}. For the H-Optimus-based model, we utilize frozen weights for the encoder and only fine-tune the decoder to take advantage of the extensive pre-training of the foundation model. For the ResNet50-based model we start with ImageNet \cite{Deng_Dong_etal.} weights and train both encoder and decoder parts. Hyperparameters for the training step are set to be identical, where possible, for comparable evaluation. 
For this evaluation, we deliberately use the PanNuke dataset to provide a standardized and controlled comparison between the H‑Optimus and ResNet50-based models (\hyperref[fig:S2]{Appendix Figure S2 (3)}). Specifically, we use two of the default PanNuke dataset splits (66\%) for training and validation, and reserve the third split (33\%) for testing.

To address the challenge of cell class imbalance in the PanNuke dataset, which is a common characteristic in most cell-level H\&E patch datasets, both models’ training processes employ a weighted loss function comprising cross-entropy and focal loss \cite{Lin_Goyal_etal._2018}. The focal loss component is adjusted with coefficients derived from each cell class' instance frequency, emphasizing learning from underrepresented classes and enhancing the model's sensitivity to rare but significant cellular patterns. The cross-entropy loss is augmented with spectral decoupling regularization \cite{Pezeshki_Kaba_etal._2021,Pohjonen_Stürenberg_etal._2022} and spatially varying label smoothing \cite{Islam_Glocker_2021}, which potentially stabilizes training and improves generalization in case of complex tissue morphologies. For optimization, we employ the \textit{AdamW} \cite{Loshchilov_Hutter_2019} to counter unbalanced class scenarios, with cosine annealing learning rate scheduler.

We utilize the scikit-learn library \cite{Van_der_Walt_Schönberger_etal._2014} and HoVer-Net \cite{Graham_Vu_etal._2019} implementations of $R^2$ (the coefficient of determination) and $PQ$ (panoptic quality) to evaluate our experiments. Complete mathematical formulations and detailed explanations of these metrics are provided in \hyperref[chap:S5]{Appendix S5}. To compute confidence intervals, we use nonparametric bootstrapping, where after calculating the metric on the full sample, we generated 1000 bootstrap replicates by resampling with replacement and then determined the 95\% confidence intervals as the 2.5th and 97.5th percentiles of the resulting empirical distribution.

%\hfill

The model comparisons are summarized in \hyperref[tab:2]{Table 2}. The H‑Optimus-based model achieves higher $R^2$ across all cell classes compared to the ResNet50-based model, which means that its predictions are more closely aligned with the PanNuke cell counts, indicating a stronger correlation with the observed data. Notably, the improvement of $R^2_{dead}$ may be an indicator of better global contextual representations provided by the foundation model backbone. In terms of segmentation and classification quality combined, measured by the PQ score, the H‑Optimus-based model demonstrates notable improvements across most cell classes. Overall, the average $R^2$ improved from 0.575 to 0.871, while the average $PQ$ score improved from 0.450 to 0.492, demonstrating better performance of the H-Optimus-based model.

\begin{table}[h!]
\renewcommand{\arraystretch}{1.5}
  \centering
  \caption{Cell quantification metrics for baseline and proposed models (CI 95\%).}
  \label{tab:2}
  \begin{tabular}{|l|c|c|}
    \hline
    %\rowcolor{gray!30}
    Metric             & Resnet50-based            & H-optimus-based              \\
    \hline
    $R^2_{neoplastic}$    & 0.681 (0.576--0.769)       & \textbf{0.941 (0.917--0.960)} \\
    \hline
    $R^2_{inflammatory}$  & 0.863 (0.778--0.903)       & \textbf{0.949 (0.918--0.966)} \\
    \hline
    $R^2_{connective}$    & 0.600 (0.488--0.698)       & 0.609 (0.436--0.772)          \\
    \hline
    $R^2_{dead}$          & 0.097 (-11.389--0.669)     & 0.925 (0.404--0.982)          \\
    \hline
    $R^2_{epithelial}$    & 0.635 (0.490--0.747)       & \textbf{0.930 (0.886--0.964)} \\
    \hline
    $PQ_{neoplastic}$       & 0.517 (0.499--0.535)       & \textbf{0.589 (0.575--0.604)} \\
    \hline
    $PQ_{inflammatory}$     & 0.455 (0.429--0.482)       & \textbf{0.528 (0.507--0.549)} \\
    \hline
    $PQ_{connective}$       & 0.416 (0.400--0.431)       & \textbf{0.451 (0.436--0.465)} \\
    \hline
    $PQ_{dead}$             & 0.374 (0.342--0.408)       & 0.292 (0.209--0.365)          \\
    \hline
    $PQ_{epithelial}$       & 0.488 (0.460--0.519)       & \textbf{0.599 (0.579--0.618)} \\
    \hline
  \end{tabular}
\end{table}

Our results  show that integrating the H‑Optimus foundation model within the UNETR architecture enhances the model's ability to segment and classify cells across diverse tissues from PanNuke data. The pretrained transformer encoder provides robust feature representations, resulting in higher average $R^2$ and $PQ$ scores compared to the CNN-based model. This leads to more reliable cell quantification and more accurate downstream analysis. Additionally, the streamlined fine-tuning process reduces computational overhead and training time, making the model more adaptable for new data.

Despite these advancements, the foundation model-based approach does not fully resolve all challenges related to cell segmentation and classification. We observe lower metric scores for underrepresented classes in the training data. Furthermore, foundation models typically encompass billions of parameters, resulting in substantial computational and memory requirements. It therefore poses challenges for deployment in resource-constrained environments, limiting their practical applicability in certain clinical settings.

\section{Model optimization via Knowledge Distillation}

To address the limitations posed by the extensive size of foundation models, we implement knowledge distillation — a model compression technique that leverages the teacher-student paradigm \cite{Hinton_Vinyals_etal._2015}. By training a smaller, more efficient student model to replicate the output of a larger, pre-trained teacher model, we retain performance while significantly reducing the model's complexity and resource requirements (\hyperref[fig:fig6]{Figure 6}).

\begin{figure}[h!]
    \centering
    \includegraphics[width=\textwidth, height=0.45\textheight, keepaspectratio]{images/Figure_6.pdf}
    \caption{Knowledge distillation framework for training a student model using a pre-trained teacher}
    \label{fig:fig6}
\end{figure}

We employ knowledge distillation to compress the H‑Optimus-based teacher model into a more efficient student model. The teacher model is the modified UNETR architecture with the H‑Optimus foundation model described in the previous chapter. The student model is based on a UNet architecture augmented with residual connections and incorporates a smaller ViT encoder with 9 million parameters \cite{Steiner_Kolesnikov_etal._2022,Wightman_2019}. 

First, we fine-tune the teacher model using the refined dataset from the cross-relabeling procedure (Section 2). Initially we train the decoder of the teacher model while keeping the encoder weights frozen. We split the refined dataset into train (70\%), validation (20\%) and test (10\%) subsets (\hyperref[fig:S2]{Appendix Figure S2 (4)}). During fine-tuning, we use the train and validation subsets, while leaving the test subset for model evaluation. We set the training procedure and model hyperparameters to be identical to those that were used to demonstrate the utility of foundation models for the simultaneous cell segmentation and classification task.

Next, we perform knowledge distillation from teacher to student using the refined dataset used to fine-tune the teacher model. The student model is trained to replicate the teacher model's outputs. We utilize a specialized loss function that aligns the student's predicted probability distribution with the teacher's, incorporating the teacher's class probability distribution derived from the output. Following the methodology of Hinton et al. \cite{Hinton_Vinyals_etal._2015}, we experiment with various hyperparameter settings for the temperature ($T$) and the balancing coefficients ($\alpha$ and $\beta$) in the loss function. We vary $T$ from 1 to 20 and adjust $\alpha$ and $\beta$ to balance the distillation and student losses. Through iterative tuning and evaluation, we identify that setting $T=14$, $\alpha=0.3$, and $\beta=0.7$ yields a configuration that converges and closely approximates the teacher model's performance during training.

Finally, we assess the performance of both models using the $R^2$ and $PQ$ (defined in \hyperref[chap:S5]{Appendix S5}) on the test set of the refined dataset (\hyperref[tab:3]{Table 3}). We observe that the 95\% confidence intervals overlap for most cell types, so we cannot claim statistically significant performance differences between the teacher and student models. One exception appears in the neoplastic class. The teacher model produces an $R^2$ of 0.919, while the student model shows an $R^2$ of 0.852. In addition, the student model achieves higher $PQ$ values for the neoplastic and connective classes, though the confidence intervals show overlap.

\begin{table}[h!]
\renewcommand{\arraystretch}{1.5}
  \centering
  \caption{Cell quantification metrics for teacher and distilled student models (CI 95\%).}
  \label{tab:3}
  \begin{tabular}{|l|c|c|}
    \hline
    %\rowcolor{gray!30}
    Metric & Teacher & Student \\
    \hline
    $R^2_{neoplastic}$    & \textbf{0.919} (0.898--0.939) & 0.852 (0.800--0.891) \\
    \hline
    $R^2_{lymphocyte}$    & 0.969 (0.956--0.977)         & 0.969 (0.956--0.978) \\
    \hline
    $R^2_{connective}$    & 0.694 (0.548--0.809)         & 0.618 (0.469--0.741) \\
    \hline
    $R^2_{dead}$          & 0.755 (0.400--0.908)         & 0.424 (0.100--0.731) \\
    \hline
    $R^2_{epithelial}$    & 0.922 (0.870--0.958)         & 0.843 (0.738--0.917) \\
    \hline
    $R^2_{macrophage}$    & 0.384 (-0.369--0.724)        & 0.704 (0.352--0.859) \\
    \hline
    $R^2_{neutrofil}$     & 0.854 (0.578--0.929)         & 0.833 (0.502--0.925) \\
    \hline
    $PQ_{neoplastic}$       & 0.581 (0.569--0.593)         & 0.601 (0.588--0.613) \\
    \hline
    $PQ_{lymphocyte}$       & 0.536 (0.520--0.553)         & 0.563 (0.544--0.579) \\
    \hline
    $PQ_{connective}$       & 0.436 (0.421--0.451)         & 0.457 (0.441--0.474) \\
    \hline
    $PQ_{dead}$             & 0.272 (0.235--0.315)         & 0.279 (0.201--0.369) \\
    \hline
    $PQ_{epithelial}$       & 0.522 (0.500--0.545)         & 0.530 (0.506--0.555) \\
    \hline
    $PQ_{macrophage}$       & 0.524 (0.459--0.588)         & 0.474 (0.405--0.543) \\
    \hline
    $PQ_{neutrofil}$        & 0.541 (0.490--0.592)         & 0.565 (0.522--0.607) \\
    \hline
  \end{tabular}
\end{table}


We further decompose the $PQ$ metric into its $SQ$ and $DQ$ components (\hyperref[tab:S6]{Appendix Table S6}). Both models produce nearly identical $SQ$ values, which indicates that they predict instance boundaries with similar precision. Although the student model shows some improvement in $DQ$ scores for certain classes, the confidence intervals overlap and do not confirm a statistically significant difference.

We observe that the student and teacher models yield comparable detection performance despite the student model using a much smaller and simpler architecture. A model with fewer parameters reduces the risk of overfitting when training data are scarce relative to the model’s complexity \cite{Farias_Ludermir_etal._2022}. The knowledge distillation process also encourages the student model to focus on the most generalizable detection features learned from the teacher. These factors enable the student model to achieve similar detection performance across different cell types.

Additionally, considering the model sizes reported in \hyperref[tab:4]{Table 4}, the distilled model achieves a significant reduction compared to the teacher model, with a 48-fold decrease in parameter count and a 5.5-fold reduction in on-disk size. In inference mode, the teacher model requires 16 GB of VRAM for a batch size of 32, while the distilled model only needs 3 GB of VRAM for the same batch size. These reductions make the distilled model significantly more practical for fine-tuning and deployment in resource-constrained environments.

\begin{table}[h!]
\renewcommand{\arraystretch}{1.5}
  \centering
  \caption{Parameter counts and size of teacher and distilled model}
  \label{tab:4}
  \adjustbox{max width=\textwidth}{%
  \begin{tabular}{|l|c|c|c|}
    \hline
    %\rowcolor{gray!30}
    Metric & H-optimus-based (Teacher) & mobileViT-based (Student) & Magnitude of difference \\
    \hline
    Parameters count       & 1,158,917,906   & \textbf{24,093,393}   & \textbf{48x}  \\
    \hline
    Estimated Total Size (MB) & 87,912       & \textbf{15,935}    & \textbf{5.5x} \\
    \hline
  \end{tabular}%
}
\end{table}

%\hfill

With recent advancements in complex network architectures and the use of pretrained encoders to achieve state-of-the-art performance \cite{Baumann_Dislich_etal._2024,Hörst_Rempe_etal._2024} in cell segmentation and classification tasks, model size, computational complexity, and processing times have increased. This limits the scalability and accessibility of these models. As we demonstrate, this may be mitigated using knowledge distillation. Studies in the field of natural language processing have demonstrated the efficacy of knowledge distillation in retaining the capabilities of the teacher model while achieving significant reductions in size and complexity \cite{Huangpu_Gao_2024,Sun_Yu_etal.}. 

We demonstrate the feasibility of knowledge distillation in digital pathology, specifically for cell segmentation and classification tasks. Moreover, we achieve this performance while also significantly reducing the parameter count. In addressing the challenge of knowledge transfer, we found that distillation from a transformer-based model to a smaller transformer is more straightforward than attempting to map transformer features to CNN blocks. In our experiments, using a CNN-based network as a student results in worse cell quantification performance due to the structural constraints of CNN feature space dimensions. 

Although our primary approach relies on a transformer-based student model that performs well, it can be further optimized to incorporate advantages from CNN architectures. For example, employing alternative techniques such as using ViT adapters \cite{Chen_Duan_etal._2023} or $1 \times 1$ convolutions to adjust feature map sizes may be beneficial for harnessing CNN advantages like enhanced local feature extraction. Moreover, if additional performance improvements are desired, the process can be further enhanced by applying supplementary knowledge distillation techniques, such as self-distillation \cite{Zhang_Song_etal._2019} or online distillation \cite{Houyon_Cioppa_etal._2023}.

Despite these promising results, further validation on independent datasets is necessary to fully understand the model's limitations. Underrepresented classes may pose challenges when addressing complex cases. Pathologists need to validate these models to adopt them in clinical settings. While the distilled models are smaller and more deployable, a technological gap persists because pathologists traditionally rely on established methods for inspecting WSIs and diagnosing diseases. Addressing the complexities involved in deploying models for inference and supporting pathologists in adopting new tools is essential for integrating these models into clinical workflows.

\section{Model integration with QuPath}
Digital pathology tools with graphical user interfaces are essential for visualizing and analyzing WSIs. To make our student model useful in clinical pathology workflows, it needs to be integrated into a tool that enables inspecting regions, creating annotations, and providing quantitative analyses of biomarkers. Therefore, we integrate the trained student model from the previous chapter into the QuPath open‑source platform \cite{Bankhead_Loughrey_etal._2017}. QuPath provides the required annotation, visualization, and analysis tools to interpret complex histological data, including workflows for cell segmentation, classification, and quantification (\hyperref[fig:fig7]{Figure 7}). 

\begin{figure}[h!]
    \centering
    \includegraphics[width=\textwidth]{images/Figure_7.pdf}
    \caption{Visualization of model-generated cell quantification annotations (left) and the corresponding unannotated slide (right) in QuPath}
    \label{fig:fig7}
\end{figure}

To identify the regions in a WSI critical for prognosticating tumor development, such as specific tumor areas or border regions without overlapping healthy tissue, the pathologist uses QuPath to outline these regions. Then, the pathologist initiates a cell segmentation and classification script through the QuPath interface for the selected regions. The resulting annotations and quantified cell information are then directly overlaid onto the WSI in the QuPath interface. Additional design and implementation details are in \hyperref[chap:S7]{Appendix S7}. 

Two common approaches for integrating deep learning models into QuPath are Java‑based native QuPath extensions \cite{Goldsborough_Philps_etal._2024} and the execution of RESTful API requests to a model server coupled with handling the response via an extension, as demonstrated in the application of cell segmentation models applied to immunofluorescence images \cite{Sugawara_2023}. While the community is actively working on these integration strategies, there is currently no universal solution that fully addresses all integration and performance requirements.

Extensions may offer better integration with QuPath, allowing slightly improved performance and more widespread usage of the built-in QuPath models, but they lack the flexibility to customize models and modify their behavior. For example, the newest version of QuPath includes models such as StarDist \cite{Weigert_Schmidt} and InstanSeg \cite{Goldsborough_Philps_etal._2024} that can perform cell segmentation. Both models pose limitations when applied to simultaneous cell segmentation and classification. StarDist performs well only on convex, round shapes by design, whereas some neoplastic, inflammatory, and connective cells exhibit complex and non-convex shapes. InstanSeg provides only semantic segmentation without assigning classes to the segmented cells.

%\hfill

In contrast, our approach offers an alternative integration strategy. It utilizes the paquo library to directly interact with QuPath’s internal application programming interface from within Python. This enables data exchange and processing without the need for intermediate conversion steps and provides greater control over model customization, retraining, and the incorporation of custom processing steps.

The integration of our custom model with QuPath underscores its potential to significantly enhance the diagnostic process by reducing the time burden on pathologists and enabling them to focus on more complex interpretative tasks using familiar software. Leveraging a tool that is already well-established among pathologists increases the likelihood of its adoption into daily clinical workflows. The quantitative data generated through the automated workflow is critical for both clinical decision-making and research, facilitating more accurate biomarker analysis, enabling robust statistical evaluations, and supporting hypothesis generation and testing. Additionally, by streamlining cell segmentation and classification, the tool enhances the scalability and reproducibility of pathological assessments, ultimately contributing to improved diagnostic accuracy and patient outcomes.

\section{Conclusion and future work}

In this study, we address critical challenges in digital pathology and tackle the usability and deployment issues of the developed models in standard computing environments without the need for high-performance computing systems. Our multi-faceted approach encompasses data refinement through cross-relabeling, leveraging foundation models for robust cell segmentation and classification, optimizing model performance via knowledge distillation, and integrating the optimized model into the QuPath software for practical application. This approach is used to construct a capable, versatile, and adjustable model for cell segmentation and classification, with enhanced performance and usability.

\begin{sloppypar}
While our approach shows potential in the field of computational pathology, certain limitations persist. 
For example, our implementation currently exhibits lower performance in detecting macrophages. 
This serves as an instance of the broader challenge of accurately identifying complex cell types. In order to address this issue, extending our approach to incorporate additional data sources, exploring alternative modeling approaches, and integrating other imaging modalities such as immunohistochemical staining may help improve detection accuracy. Moreover, although the distilled model reduces computational demands, integrating advanced deep learning models into clinical practice requires addressing technological gaps and potential resistance to adopting new tools within established diagnostic processes.
\end{sloppypar}

Future work could focus on several key areas to refine the proposed approach and facilitate its adoption in clinical environments. Enhancing the cell-relabeling process with additional datasets \cite{Graham_Jahanifar_etal._2021} could improve the representation of underrepresented cell types and enhance overall model performance. Also, incorporating additional data sources, such as multi-modal imaging or complementary staining methods, may address limitations related to cell type differentiation and class imbalance. Exploring other foundation models \cite{Vorontsov_Bozkurt_etal._2024,Zimmermann_Vorontsov_etal._2024} or introducing additional modalities \cite{Ding_Wagner_etal._2024,Vaidya_Zhang_etal._2025} may provide alternative architectures better suited to specific tasks or offer improved efficiency. Implementing more complex knowledge distillation techniques \cite{Houyon_Cioppa_etal._2023,Zhang_Song_etal._2019} could further optimize the model's performance and adaptability. Additionally, deeper integration with QuPath or other digital pathology software could provide pathologists more control over cell quantification analysis directly within the QuPath interface, thereby increasing accessibility and usability. Such enhancements would not only refine model performance but also ensure greater adaptability and scalability within various clinical environments. Finally, extensive validation of the model by pathologists and benchmarking against independent datasets are essential steps toward establishing the model's reliability and fostering confidence in its clinical utility.

\section*{Acknowledgments} 
This work was funded in part by the Research Council of Norway grant no. 309439 SFI Visual Intelligence, and the North Norwegian Health Authority grant no. HNF1521-20.

\bibliographystyle{IEEEtran}
\begin{sloppypar}
\begin{thebibliography}{99}

\bibitem{chaplot2020neural} Chaplot, Devendra Singh, et al. "Neural topological slam for visual navigation." Proceedings of the IEEE/CVF conference on computer vision and pattern recognition. 2020.

\bibitem{maksymets2021thda} Maksymets, Oleksandr, et al. "Thda: Treasure hunt data augmentation for semantic navigation." Proceedings of the IEEE/CVF International Conference on Computer Vision. 2021.

\bibitem{mezghan2022memory} Mezghan, Lina, et al. "Memory-augmented reinforcement learning for image-goal navigation." 2022 IEEE/RSJ International Conference on Intelligent Robots and Systems (IROS). IEEE, 2022.

\bibitem{al2022zero} Al-Halah, Ziad, Santhosh Kumar Ramakrishnan, and Kristen Grauman. "Zero experience required: Plug \& play modular transfer learning for semantic visual navigation." Proceedings of the IEEE/CVF Conference on Computer Vision and Pattern Recognition. 2022.

\bibitem{ye2021auxiliary} Ye, Joel, et al. "Auxiliary tasks and exploration enable objectgoal navigation." Proceedings of the IEEE/CVF international conference on computer vision. 2021.

\bibitem{chaplot2020object} Chaplot, Devendra Singh, et al. "Object goal navigation using goal-oriented semantic exploration." Advances in Neural Information Processing Systems 33 (2020)

\bibitem{ramakrishnan2022poni} Ramakrishnan, Santhosh Kumar, et al. "Poni: Potential functions for objectgoal navigation with interaction-free learning." Proceedings of the IEEE/CVF Conference on Computer Vision and Pattern Recognition. 2022.

\bibitem{ramrakhya2022habitat} Ramrakhya, Ram, et al. "Habitat-web: Learning embodied object-search strategies from human demonstrations at scale." Proceedings of the IEEE/CVF Conference on Computer Vision and Pattern Recognition. 2022.

\bibitem{mousavian2019visual} Mousavian, Arsalan, et al. "Visual representations for semantic target driven navigation." 2019 International Conference on Robotics and Automation (ICRA). IEEE, 2019.

\bibitem{dhariwal2021diffusion} Dhariwal, Prafulla, and Alexander Nichol. "Diffusion models beat gans on image synthesis." Advances in neural information processing systems 34 (2021)

\bibitem{ho2022classifier} Ho, Jonathan, and Tim Salimans. "Classifier-free diffusion guidance." arXiv preprint arXiv:2207.12598 (2022).

\bibitem{nichol2021glide} Nichol, Alex, et al. "Glide: Towards photorealistic image generation and editing with text-guided diffusion models." arXiv preprint arXiv:2112.10741 (2021)

\bibitem{brooks2023instructpix2pix} Brooks, Tim, Aleksander Holynski, and Alexei A. Efros. "Instructpix2pix: Learning to follow image editing instructions." Proceedings of the IEEE/CVF Conference on Computer Vision and Pattern Recognition. 2023.

\bibitem{fu2023guiding} Fu, Tsu-Jui, et al. "Guiding instruction-based image editing via multimodal large language models." arXiv preprint arXiv:2309.17102 (2023).

\bibitem{geng2024instructdiffusion} Geng, Zigang, et al. "Instructdiffusion: A generalist modeling interface for vision tasks." Proceedings of the IEEE/CVF Conference on Computer Vision and Pattern Recognition. 2024.

\bibitem{zhou2024minedreamer} Zhou, Enshen, et al. "Minedreamer: Learning to follow instructions via chain-of-imagination for simulated-world control." arXiv preprint arXiv:2403.12037 (2024).

\bibitem{zhou2023esc} Zhou, Kaiwen, et al. "Esc: Exploration with soft commonsense constraints for zero-shot object navigation." International Conference on Machine Learning. PMLR, 2023.

\bibitem{yu2023l3mvn} Yu, Bangguo, Hamidreza Kasaei, and Ming Cao. "L3mvn: Leveraging large language models for visual target navigation." 2023 IEEE/RSJ International Conference on Intelligent Robots and Systems (IROS). IEEE, 2023.

\bibitem{gadre2023cows} Gadre, Samir Yitzhak, et al. "Cows on pasture: Baselines and benchmarks for language-driven zero-shot object navigation." Proceedings of the IEEE/CVF Conference on Computer Vision and Pattern Recognition. 2023.

\bibitem{shah2023navigation} Shah, Dhruv, et al. "Navigation with large language models: Semantic guesswork as a heuristic for planning." Conference on Robot Learning. PMLR, 2023.

\bibitem{cai2024bridging} Cai, Wenzhe, et al. "Bridging zero-shot object navigation and foundation models through pixel-guided navigation skill." 2024 IEEE International Conference on Robotics and Automation (ICRA). IEEE, 2024.

\bibitem{yu2023co} Yu, Bangguo, Hamidreza Kasaei, and Ming Cao. "Co-NavGPT: Multi-robot cooperative visual semantic navigation using large language models." arXiv preprint arXiv:2310.07937 (2023).

\bibitem{wu2024voronav} Wu, Pengying, et al. "Voronav: Voronoi-based zero-shot object navigation with large language model." arXiv preprint arXiv:2401.02695 (2024).

\bibitem{qin2023mp5} Qin, Yiran, et al. "Mp5: A multi-modal open-ended embodied system in minecraft via active perception." arXiv preprint arXiv:2312.07472 (2023).

\bibitem{du2024learning} Du, Yilun, et al. "Learning universal policies via text-guided video generation." Advances in Neural Information Processing Systems 36 (2024).

\bibitem{ajay2024compositional} Ajay, Anurag, et al. "Compositional foundation models for hierarchical planning." Advances in Neural Information Processing Systems 36 (2024).

\bibitem{liang2024skilldiffuser} Liang, Zhixuan, et al. "Skilldiffuser: Interpretable hierarchical planning via skill abstractions in diffusion-based task execution." Proceedings of the IEEE/CVF Conference on Computer Vision and Pattern Recognition. 2024.

\bibitem{heusel2017gans} Heusel, Martin, et al. "Gans trained by a two time-scale update rule converge to a local nash equilibrium." Advances in neural information processing systems 30 (2017).

\bibitem{zhang2018unreasonable} Zhang, Richard, et al. "The unreasonable effectiveness of deep features as a perceptual metric." Proceedings of the IEEE conference on computer vision and pattern recognition. 2018.

\bibitem{brown2020language} Brown, Tom B. "Language models are few-shot learners." arXiv preprint arXiv:2005.14165 (2020).

\bibitem{podell2023sdxl} Podell, Dustin, et al. "Sdxl: Improving latent diffusion models for high-resolution image synthesis." arXiv preprint arXiv:2307.01952 (2023).

\bibitem{brohan2022rt} Brohan, Anthony, et al. "Rt-1: Robotics transformer for real-world control at scale." arXiv preprint arXiv:2212.06817 (2022).

\bibitem{brohan2023rt} Brohan, Anthony, et al. "Rt-2: Vision-language-action models transfer web knowledge to robotic control." arXiv preprint arXiv:2307.15818 (2023).

\bibitem{li2024manipllm} Li, Xiaoqi, et al. "Manipllm: Embodied multimodal large language model for object-centric robotic manipulation." Proceedings of the IEEE/CVF Conference on Computer Vision and Pattern Recognition. 2024.

\bibitem{shah2023vint} Shah, Dhruv, et al. "ViNT: A foundation model for visual navigation." arXiv preprint arXiv:2306.14846 (2023).

\bibitem{liu2024visual} Liu, Haotian, et al. "Visual instruction tuning." Advances in neural information processing systems 36 (2024).

\bibitem{hu2021lora} Hu, Edward J., et al. "Lora: Low-rank adaptation of large language models." arXiv preprint arXiv:2106.09685 (2021).

\bibitem{qin2023supfusion} Qin, Yiran, et al. "SupFusion: Supervised LiDAR-camera fusion for 3D object detection." Proceedings of the IEEE/CVF International Conference on Computer Vision. 2023.

\bibitem{qin2024worldsimbench} Qin, Yiran, et al. "Worldsimbench: Towards video generation models as world simulators." arXiv preprint arXiv:2410.18072 (2024).

\bibitem{yu2025gamefactory} Yu, Jiwen, et al. "GameFactory: Creating New Games with Generative Interactive Videos." arXiv preprint arXiv:2501.08325 (2025).

\bibitem{zhou2024code} Zhou, Enshen, et al. "Code-as-Monitor: Constraint-aware Visual Programming for Reactive and Proactive Robotic Failure Detection." arXiv preprint arXiv:2412.04455 (2024).

\bibitem{zhang2024ad} Zhang, Zaibin, et al. "AD-H: Autonomous Driving with Hierarchical Agents." arXiv preprint arXiv:2406.03474 (2024).

\bibitem{wang2024toward} Wang, Chaoqun, et al. "Toward Accurate Camera-based 3D Object Detection via Cascade Depth Estimation and Calibration." arXiv preprint arXiv:2402.04883 (2024).

\bibitem{huang2024story3d} Huang, Yuzhou, et al. "Story3d-agent: Exploring 3d storytelling visualization with large language models." arXiv preprint arXiv:2408.11801 (2024).

\bibitem{savinov2018semi} Savinov, Nikolay, Alexey Dosovitskiy, and Vladlen Koltun. "Semi-parametric topological memory for navigation." arXiv preprint arXiv:1803.00653 (2018).

\bibitem{majumdar2022zson} Majumdar, Arjun, et al. "Zson: Zero-shot object-goal navigation using multimodal goal embeddings." Advances in Neural Information Processing Systems 35 (2022): 32340-32352.

\bibitem{yadav2023offline} Yadav, Karmesh, et al. "Offline visual representation learning for embodied navigation." Workshop on Reincarnating Reinforcement Learning at ICLR 2023. 2023.

\bibitem{yadav2023ovrl} Yadav, Karmesh, et al. "Ovrl-v2: A simple state-of-art baseline for imagenav and objectnav." arXiv preprint arXiv:2303.07798 (2023).

\bibitem{sun2024fgprompt} Sun, Xinyu, et al. "FGPrompt: fine-grained goal prompting for image-goal navigation." Advances in Neural Information Processing Systems 36 (2024).

\bibitem{zhu2017target} Zhu, Yuke, et al. "Target-driven visual navigation in indoor scenes using deep reinforcement learning." 2017 IEEE international conference on robotics and automation (ICRA). IEEE, 2017.

\bibitem{koh2024generating} Koh, Jing Yu, Daniel Fried, and Russ R. Salakhutdinov. "Generating images with multimodal language models." Advances in Neural Information Processing Systems 36 (2024).

\bibitem{krantz2022instance} Krantz, Jacob, et al. "Instance-specific image goal navigation: Training embodied agents to find object instances." arXiv preprint arXiv:2211.15876 (2022).

\bibitem{schulman2017proximal} Schulman, John, et al. "Proximal policy optimization algorithms." arXiv preprint arXiv:1707.06347 (2017).

\bibitem{anderson2018evaluation} Anderson, Peter, et al. "On evaluation of embodied navigation agents." arXiv preprint arXiv:1807.06757 (2018).

\bibitem{lin2024navcot} Lin, Bingqian, et al. "NavCoT: Boosting LLM-Based Vision-and-Language Navigation via Learning Disentangled Reasoning." arXiv preprint arXiv:2403.07376 (2024).

\bibitem{NavGPT} Zhou, Gengze, Yicong Hong, and Qi Wu. "Navgpt: Explicit reasoning in vision-and-language navigation with large language models." Proceedings of the AAAI Conference on Artificial Intelligence.

\bibitem{hahn2021no} Hahn, Meera, et al. "No rl, no simulation: Learning to navigate without navigating." Advances in Neural Information Processing Systems 34 (2021): 26661-26673.

\bibitem{li2025t2isafety} Li, Lijun, et al. "T2ISafety: Benchmark for Assessing Fairness, Toxicity, and Privacy in Image Generation." arXiv preprint arXiv:2501.12612 (2025).

\bibitem{an2024agfsync} An, Jingkun, et al. "AGFSync: Leveraging AI-Generated Feedback for Preference Optimization in Text-to-Image Generation." arXiv preprint arXiv:2403.13352 (2024).


\end{thebibliography}
\end{sloppypar}

\clearpage
\beginsupplement
\section*{Appendix}
\renewcommand{\thesubsection}{S\arabic{subsection}}

\subsection{\label{chap:S1}PanNuke and MoNuSAC preprocessing}
The PanNuke dataset comprises a set of 7,901 RGB patches, each with dimensions of $256 \times 256$ pixels, which we set as the standard patch size for our analysis. In contrast, the MoNuSAC dataset encompasses 294 images of heterogeneous dimensions. To standardize the MoNuSAC images with our experiments, we implement a standardization protocol. Specifically, for images exceeding the dimensions of $256 \times 256$ pixels, we segment them into equal-sized patches and apply mirror padding to the remaining portions to avoid information loss at the peripherals. Patches with dimensions less than $128 \times 128$ pixels are excluded from the dataset due to the insufficient resolution to capture relevant cellular details. For patches where either dimension falls between 128 and 256 pixels, we employ upsampling to achieve the standard patch size. As a result, we obtain a total of 2,823 RGB patches derived from the MoNuSAC dataset for subsequent analysis. For additional details on the MoNuSAC data preparation process, refer to the source code \cite{Shvetsov_2025a}.
\clearpage

\subsection{\label{chap:S2}Data usage for the methodology}

\counterwithin{figure}{subsection}
\renewcommand{\thefigure}{S\arabic{subsection}}

\begin{figure}[h!]
    \centering
    \includegraphics[width=\textwidth, height=0.85\textheight, keepaspectratio]{images/A2.pdf}
    \caption{Overview of the methodology for cross-labeling, dataset refinement, and model comparison. (1) Cross-relabeling - training and testing cell classification models, (2) Cross-relabeling - using cell classification models to create refined dataset, (3) Fine-tuning and training models for comparison, (4) Student knowledge distillation with refined dataset}
    \label{fig:S2}
\end{figure}
\clearpage

\subsection{\label{chap:S3}Confusion matrices for classification models}
\counterwithin{figure}{subsection}
\renewcommand{\thefigure}{S\arabic{subsection}.\arabic{figure}}

\begin{figure}[h!]
    \centering
    \includegraphics[width=\textwidth, height=0.4\textheight, keepaspectratio]{images/A3_1.pdf}
    \caption{Confusion matrix for PanNuke trained model}
    \label{fig:S3.1}
\end{figure}

\begin{figure}[h!]
    \centering
    \includegraphics[width=\textwidth, height=0.4\textheight, keepaspectratio]{images/A3_2.pdf}
    \caption{Confusion matrix for MoNuSAC trained model}
    \label{fig:S3.2}
\end{figure}

\clearpage

\subsection{\label{chap:S4}Datasets cell counts}

\counterwithin{table}{subsection}
\renewcommand{\thetable}{S\arabic{subsection}}

\begin{table}[h!]
\renewcommand{\arraystretch}{2.0}
\centering
\caption{\label{tab:S4}Cell counts for PanNuke, MoNuSAC and refined datasets. Numbers in parentheses indicate preprocessed cell counts for cell classifier models training and testing.}
%\adjustbox{max width=\textwidth}{%
\begin{tabular}{|l|c|c|c|}
\hline
%\rowcolor{gray!30}
Cell type & PanNuke & MoNuSAC & Refined \\
\hline
Neoplastic & 77,403 (68,031) & - & 105,451 \\
\hline
Epithelial & 26,572 (23,207) & - & 29,926 \\
\hline
Epithelial (benign and malignant) & - & 31,402 & - \\
\hline
Inflammatory & 32,276 & - & - \\
\hline
Lymphocytes & - & 37,045 (33,104) & 65,275 \\
\hline
Neutrophils & - & 1,355 (1,252) & 3,833 \\
\hline
Macrophage & - & 1,842 (1,695) & 3,410 \\
\hline
Dead & 2,908 & - & 2,908 \\
\hline
Connective & 50,585 & - & 50,585 \\
\hline
\end{tabular}
%
%}
\end{table}



\clearpage

\subsection{\label{chap:S5}Definition of validation metrics}
\counterwithin{equation}{subsection}
\renewcommand{\theequation}{\arabic{equation}}

\subsubsection{\label{chap:S5.1}R\textsuperscript{2}}
The coefficient of determination, denoted as $R^2$, is a statistical measure that represents the proportion of variance in the dependent variable that is predictable from the independent variables. In the context of cell quantification in pathology, $R^2$ is used to assess how well the predicted quantities of different cell types in a patch align with the actual quantities observed in the ground truth data, with higher values representing more accurate quantification. $R^2$ is defined as
\begin{equation*}
R^2 = 1 - \frac{\sum_{i=1}^n (y_i - \hat{y}_i)^2}{\sum_{i=1}^n (y_i - \bar{y})^2},
\end{equation*}
where $y_i$ represents the actual number of cells of a specific type in the $i$-th image, $\hat{y}_i$ represents the predicted number of cells of that type in the $i$-th image, $\bar{y}$ is the mean of the actual numbers across all images, and $n$ is the total number of images in the dataset.

The $R^2$ metric has a range of $(-\infty, 1]$. An $R^2$ of 1 indicates perfect prediction, where all predicted values exactly match the actual values. An $R^2$ of 0 suggests that the model explains none of the variability of the response data around its mean. If $R^2$ is negative, it indicates that the model performs worse than a model that simply predicts the mean of the actual values for all observations.

\subsubsection{\label{chap:S5.2}PQ}
Panoptic Quality ($PQ$) is a comprehensive metric used to evaluate the performance of segmentation models in tasks that require both instance segmentation and classification. $PQ$ provides a single score that encapsulates both the detection accuracy (i.e., how many objects were correctly identified) and the segmentation quality (i.e., how accurately the objects' boundaries were delineated). This metric is particularly useful in multiclass scenarios where each pixel is classified into distinct categories, such as different cell types in pathology images.

$PQ$ is calculated as the product of two terms: Detection Quality ($DQ$) and Segmentation Quality ($SQ$). It can be expressed as
\begin{equation*}
PQ = DQ \cdot SQ,
\end{equation*}
where
\begin{equation*}
DQ = \frac{TP}{TP + 0.5\, FP + 0.5\, FN},
\end{equation*}
\begin{equation*}
SQ = \frac{\sum_{(p, g) \in \mathcal{M}} IoU(p, g)}{TP}.
\end{equation*}
In these formulas, $TP$ denotes the number of correctly matched instances between ground truth and prediction, $FP$ denotes the predicted instances that have no corresponding ground truth, $FN$ denotes the ground truth instances that were not detected, $IoU(p, g)$ is the Intersection over Union for a pair of matched instances $p$ (prediction) and $g$ (ground truth), and $\mathcal{M}$ is the set of matched pairs.

The $PQ$ metric is calculated for each class and is averaged across classes to provide a global performance measure.

The $PQ$ score has a range of $[0, 1.0]$, where a higher score indicates better performance in both detecting and segmenting the instances correctly. A $PQ$ of 1 signifies perfect identification and segmentation of all instances, whereas a $PQ$ of 0 indicates that no instances were correctly identified and segmented.

\clearpage

\subsection{\label{chap:S6}Segmentation and Detection quality metrics for teacher and student models}

\begin{table}[h!]
\renewcommand{\arraystretch}{2.0}
\centering
\caption{Segmentation and detection quality for student and teacher models (CI 95\%)}
\label{tab:S6}
%\adjustbox{max width=\textwidth}{%
\begin{tabular}{|l|c|c|}
\hline
%\rowcolor{gray!30}
Metric & Teacher & Student \\
\hline
$SQ_{neoplastic}$ & 0.819 (0.815--0.823) & 0.824 (0.819--0.828) \\
\hline
$SQ_{lymphocyte}$ & 0.795 (0.788--0.802) & 0.790 (0.783--0.796) \\
\hline
$SQ_{connective}$ & 0.770 (0.762--0.776) & 0.780 (0.772--0.786) \\
\hline
$SQ_{dead}$ & 0.659 (0.623--0.688) & 0.657 (0.624--0.695) \\
\hline
$SQ_{epithelial}$ & 0.780 (0.770--0.790) & 0.788 (0.779--0.797) \\
\hline
$SQ_{macrophage}$ & 0.788 (0.760--0.810) & 0.757 (0.730--0.783) \\
\hline
$SQ_{neutrofil}$ & 0.782 (0.761--0.801) & 0.775 (0.759--0.792) \\
\hline
$DQ_{neoplastic}$ & 0.706 (0.692--0.719) & 0.727 (0.712--0.741) \\
\hline
$DQ_{lymphocyte}$ & 0.675 (0.656--0.698) & 0.713 (0.691--0.734) \\
\hline
$DQ_{connective}$ & 0.566 (0.546--0.584) & 0.583 (0.565--0.602) \\
\hline
$DQ_{dead}$ & 0.410 (0.361--0.465) & 0.435 (0.306--0.561) \\
\hline
$DQ_{epithelial}$ & 0.668 (0.639--0.694) & 0.673 (0.644--0.702) \\
\hline
$DQ_{macrophage}$ & 0.657 (0.583--0.727) & 0.615 (0.531--0.703) \\
\hline
$DQ_{neutrofil}$ & 0.691 (0.625--0.753) & 0.729 (0.679--0.778) \\
\hline
\end{tabular}
%
%}
\end{table}

\clearpage

\subsection{\label{chap:S7}QuPath integration method}
We adopt an integration strategy leveraging the paquo \cite{Bayer_AG} library, a Python package that enables direct interaction with QuPath’s internal API, thereby facilitating seamless data exchange without intermediate conversion steps. The data processing pipeline (\hyperref[fig:S7]{Appendix Figure S7}) begins with the acquisition of WSIs and their associated annotations from QuPath, which are represented as Shapely \cite{Gillies_Wel_etal._2024} polygons. Utilizing paquo, we directly read, create, and modify these annotations and detections within a QuPath project in the Python environment. Images are then cropped using these polygons and processed by cell segmentation and classification models employing standard vision processing toolkits such as OpenCV, pyvips, and PyTorch. Additionally, QuPath employs Groovy scripts to initiate a Python process that starts the entire pipeline from QuPath graphical interface: fetching polygons, extracting images from them, and running deep learning model inference on the cropped images. 
The results are returned to QuPath, leveraging paquo's Python bindings to manipulate QuPath data while minimizing the computational overhead typically associated with cross-environment communication.

\counterwithin{figure}{subsection}
\renewcommand{\thefigure}{S\arabic{subsection}}

\begin{figure}[h!]
    \centering
    \includegraphics[width=\textwidth]{images/A7.pdf}
    \caption{QuPath integration workflow using Python environment}
    \label{fig:S7}
\end{figure}

Compared to traditional workflows that involve exporting annotations as GeoJSON, classifying them in Python, and reimporting them into QuPath, our approach offers several advantages. We eliminate the need to switch between programming languages, providing a cohesive and streamlined development process entirely within QuPath software and removing the necessity to use other tools. Meanwhile, we avoid storing annotations as intermediate JSON files unless required for external use or archiving. By conducting the entire inference and post-processing workflow within the Python environment, we leverage the power and flexibility of Python libraries for image processing and machine learning. This approach also enables adjustments to any set of labels and models, thereby improving its applicability.

%\hfill

The distilled model and QuPath integration code are packaged into a Docker container, enabling streamlined execution with the Docker engine. Detailed integration code and deployment instructions can be found in the GitHub repository \cite{Shvetsov_2025b}.

Despite these benefits, we acknowledge that the paquo library is a proof‑of‑concept project in its early development stage and has not been tested across all versions of QuPath.

\clearpage

\subsection{\label{chap:S8}Data and code availability statement}
All datasets, models, and code used in this study are publicly available and can be obtained from the repositories listed below. 
The PanNuke \cite{Gamper_Koohbanani_etal._2019} and MoNuSAC \cite{Verma_Kumar_etal._2021} datasets are publicly accessible, and download information along with detailed descriptions can be found in their respective articles. Preprocessing scripts for PanNuke and MoNuSAC data, as well as individual cell extraction scripts, are available on GitHub \cite{Shvetsov_2025a}. The H-Optimus foundation model used in our experiments can be downloaded from the HuggingFace repository \cite{hoptimus2024}, and model information is available on GitHub \cite{Saillard_Jenatton_etal._2024}. In addition, the integration code for QuPath and the distilled model packaged in a Docker container are provided in the repository \cite{Shvetsov_2025b}, and paquo Python library is available from the authors GitHub repository \cite{Bayer_AG}.
\clearpage

\end{document}


\bsp	% typesetting comment
\label{lastpage}
\end{document}

% End of mnras_template.tex
