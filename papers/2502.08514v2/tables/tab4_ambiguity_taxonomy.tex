% \usepackage{booktabs}
% \usepackage[table,xcdraw]{xcolor}
% Beamer presentation requires \usepackage{colortbl} instead of \usepackage[table,xcdraw]{xcolor}

\begin{table*}
\scriptsize
\centering
\begin{tabular}{@{}p{1.2cm}p{2.5cm}p{10cm}p{1cm}@{}}
\toprule
\textbf{Category}& \textbf{Type} & \textbf{Definition} & \textbf{Example}\\ 
\midrule
Implicit \newline reasoning \newline phenomena & Deduction                                                                       &   The summarizer has made a logical deduction, utilizing premises solely from the source document to draw a conclusion that cannot be directly traced to the source document. The conclusion can be accurate or inaccurate, but the key aspect of this label is that individual premises were accurately identified in the source document, and were used to form a conclusion that is stated in the summary sentence.Importantly, this deduction must be significant enough to make it noticeably harder for someone to assess the factuality of the summary sentence. Minor or obvious deductions that don't create evaluation challenges should not be included.   & Table \ref{tab:deduction_example}                                                                            \\
\cmidrule{2-4}
                             & Inference: \newline Common-sense                                                         &   The summarizer appears to have made an inference based on at least one premise from the source document, in addition to at least one premise that is based on a common-sense notion that is not stated explicitly in the source document. This common-sense notion should be widely accepted but not so universally obvious that it doesn't create any evaluation challenge. The resulting inference should make it noticeably harder to assess the factuality of the summary sentence against the source document alone.                                              & Table \ref{tab:commonsense_example}                               \\
                             \cmidrule{2-4}
                             & Inference: \newline Value-based                                                          &     The summarizer appears to have made an inference based on at least one premise from the source document in addition to at least one premise that is based on a value assumption.'Value' here specifically refers to a moral, ethical, or societal belief, principle, or ideal that guides or motivates attitudes and actions. This is distinct from common-sense notions or industry-standard evaluations. The value-based premise should be significant enough that it creates a notable challenge in evaluating the factuality of the summary sentence against the source document alone. The value assumption should not be explicitly stated in the source document but should be a recognizable societal or cultural value that the summarizer has applied to interpret the information.                                       & Table \ref{tab:value_based_example}                                           \\
                             \cmidrule{2-4}
                             & Other implicit \newline reasoning                                             &     The summarizer appears to have employed a form of implicit reasoning that goes beyond simple deduction, common-sense inference, or value-based inference, and significantly affects the summary sentence's evaluability. This category could include complex pattern recognition, synthesis of diverse information sources, experiential reasoning, or other sophisticated cognitive processes that are not explicitly traceable to the source document but are likely to have taken place in the summarizer's "mind" in order for the summary sentence to have been written. The (estimated) reasoning should be substantial enough that it creates a notable challenge in evaluating the factuality of the summary sentence against the source document alone. This category is reserved for cases that don't fit neatly into other categories of implicit reasoning but still present a clear evaluability issue due to the complexity or opacity of the reasoning process involved.                                      & Table \ref{tab:other_reasoning_example}                                            \\
\midrule
Meaning phenomena            &Semantic relations: \newline Hypernymy/Generalization                                &         A more general meaning (hypernym) is used in the summary sentence than is observed in the source document (for the same topic). This generalization should be significant enough to potentially affect the evaluability of the summary sentence.Minor generalizations that don't create evaluation challenges should not be included. The key aspect is that the generalization makes it harder to directly map the summary's claim to the specific information in the source document.   
& Table \ref{tab:hypernymy_example} \\
\cmidrule{2-4}
                             &Semantic relations: \newline Hyponymy/Specialization                                 &      A more specific meaning (hyponym) is used in the summary sentence than is observed in the source document (for the same topic). This specialization should be significant enough to potentially affect the evaluability of the summary sentence.The key aspect is that the specialization introduces details not explicitly mentioned in the source document, making it challenging to directly verify the summary's claim against the source information. Minor or widely known specializations that don't create evaluation challenges should not be included.The specialization should create a notable difficulty in assessing the factual accuracy of the summary based solely on the source document.                                     & Table  \ref{tab:hyponymy_example}                                           \\
                             \cmidrule{2-4}
                             &Semantic relations: \newline Synonymy/Paraphrasing                                   &      Meaning from the source document is paraphrased or expressed using synonyms in such a way that the summary sentence's evaluability is significantly affected. While the core meaning has not technically changed, the way the meaning is constructed or expressed has changed substantially. This paraphrasing or use of synonyms should be extensive or complex enough to create a notable challenge in directly mapping the summary's claims to the source document's information. Minor or straightforward paraphrasing that doesn't meaningfully impact evaluability should not be included. The key aspect is that the rephrasing makes it noticeably more difficult to assess the factual accuracy of the summary based solely on the source document.                                        & Table  \ref{tab:synonymy_example}                                        \\
                             \cmidrule{2-4}
                             &Linguistic ambiguity: \newline Structural                                              &     A phrase or sentence in the summary sentence has multiple valid parses (multiple valid syntactic structures), and it is not obvious which parse is intended. This ambiguity should significantly affect the evaluability of the summary sentence by creating notably different interpretations of the information presented. The key aspect is that the different possible syntactic structures lead to meaningfully different readings of the sentence, making it challenging to assess the factual accuracy of the summary against the source document. Minor ambiguities that don't substantially affect the meanings or create significant evaluation challenges should not be included, and neither should major ambiguities in which the additional interpretation is extremely implausible. The structural ambiguity should create a clear obstacle in determining which interpretation to evaluate against the source information.             & Table   \ref{tab:structural_example}                                                                  \\
                             \cmidrule{2-4}
                             &Linguistic ambiguity: \newline Lexical                                                 &     A word or phrase in the summary sentence has multiple valid interpretations in the given context, and it is not obvious which meaning is intended. This ambiguity should significantly affect the evaluability of the summary sentence by creating notably different interpretations of the information presented. The key aspect is that the different possible meanings of the word or phrase lead to meaningfully different understandings of the sentence, making it challenging to assess the factual accuracy of the summary against the source document. Minor ambiguities that don't substantially affect the overall meaning or create significant evaluation challenges should not be included. Similarly, cases where one interpretation is extremely implausible in the given context should also be excluded. The lexical ambiguity should create a clear obstacle in determining which interpretation to evaluate against the source information.      & Table  \ref{tab:lexical_example}                                                                          \\
                             \cmidrule{2-4}
                             & Vagueness                                               &         The meaning of part of the summary sentence is significantly underspecified compared to the source document, resulting in many different realities being compatible with the claim made.This vagueness should be substantial enough that:
                             \newline
1. There is confusion about what specific claim is actually being made.
\newline
2. The claim cannot be evaluated reliably against the source document.
\newline
3. The range of possible interpretations is so broad that it becomes challenging to determine if the summary accurately represents the source information.
\newline
The key aspect is that the vagueness creates a meaningful obstacle in assessing the factual accuracy of the summary.Minor instances of underspecification that don't significantly impact evaluability should not be included. The vagueness should go beyond simple generalization or summarization and create a genuine challenge in mapping the summary's claims to the specific information provided in the source document.  
& Table \ref{tab:vagueness_example}\\
                              \cmidrule{2-4}
\end{tabular}
\caption*{}
\label{tab:ambiguity}
\end{table*}