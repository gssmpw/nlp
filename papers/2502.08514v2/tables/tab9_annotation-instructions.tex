\begin{table*}
\centering
\small
\begin{tabular}{@{}p{14cm}@{}}
\toprule
\textbf{Overview:}
\newline
The goal of this task is to gather information about the factors or characteristics in a set of summaries (analyzed on a sentence-by-sentence basis) that impact the summary’s evaluability — the ease with which its factuality can be assessed while utilizing the summary’s source document as the source of truth. There is a diverse set of characteristics that might impact a summary’s evaluability, and we have done our best to capture a wide range of these characteristics in a Taxonomy of evaluability issues.These have been built mostly from the ground up (based on real examples), so they may not cover all kinds of evaluability issues.
\newline
\newline
This task will ultimately support the development of an automatic LLM-based summary evaluation pipeline. LLM-based factuality evaluation still lags behind humans, and one of the kinds of cases LLMs struggle with are summaries (or summary sentences)with low evaluability.
\newline
\newline

\textbf{Evaluability vs Factuality}
\newline
These two concepts are related but fundamentally distinct, and for this task, it is essential that you grasp the difference between them.
\newline
\newline
\textbf{Factuality:}
\newline
The truthfulness of a statement (or a union of statements) relative to a source of truth.
\newline
\newline
\textbf{Evaluability:}
How readily a factual evaluator can assess the factuality of the statement relative to the designated source of truth.
\newline
\newline
Along these lines, there are three pre-requisites to assessing factuality in this way:
\newline
\newline
1. A sufficiently sophisticated evaluator, with the ability for (2) and (3)
\newline
2. Comprehension of the source of truth
\newline
3. Comprehension of the statement
\newline
\newline
In principle, then, there are three variables that could prevent successful factual evaluation:
\newline
\newline
1. The evaluator is not sophisticated enough for complete comprehension/evaluation
\newline
2. The source of truth poses barriers to comprehension/evaluation
\newline
3. The statement poses barriers to comprehension/evaluation
\newline
\newline
These three variables are inter-related. For instance, (1) depends on the severity of the barriers noted in (2) and (3). But for the current task, the primary focus will be on (3), although you will also be registering your impressions for (2), and to a lesser extent,(1). (Which of the above three items is the variable that currently prevents us from using LLMs to check factuality?)
\newline
\newline
\textbf{“The statement poses barriers to comprehension”}
\newline
\newline
As mentioned above, the main focus of the current task is (3) above, so we need a way to talk about and categorize these barriers. We will use the term evaluability issue to distinguish the kinds of barriers described by (3) from the kinds of barriers described by all of (1-3). The evaluability issues we've devised fall under three main categories, plus an overflow category:
\newline
\newline
1. Implicit reasoning phenomena
\newline
2. Meaning phenomena
\newline
3. Context phenomena
\newline
4. (Other phenomena)
\newline
\newline
Refer to all the subtypes of these categories in the table below (taxonomy table as shown in Table \ref{tab:ambiguity2}). Carefully read all the definitions and refer to the examples provided in the last column. When referring to examples, ambiguous examples are examples that are an instance of the subtype, and non-ambiguous examples are examples that are not
a good example of the subtype.
    \\ 
 \bottomrule
\end{tabular}
\caption{Instructions provided to the expert annotators for ambiguity annotation.}
\label{tab:annotator_intsructs}
\end{table*}