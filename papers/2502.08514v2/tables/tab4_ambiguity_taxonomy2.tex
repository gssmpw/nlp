% \usepackage{booktabs}
% \usepackage[table,xcdraw]{xcolor}
% Beamer presentation requires \usepackage{colortbl} instead of \usepackage[table,xcdraw]{xcolor}

\begin{table*}
\scriptsize
\centering
\begin{tabular}{@{}p{1.2cm}p{2.5cm}p{10cm}p{1cm}@{}}
% \toprule
% \textbf{Category}& \textbf{Type} & \textbf{Definition} \\ 
% \midrule
\cmidrule{2-4}
                             & Non-assertion                                &  The summary sentence does not make a clear claim or assert anything as definitively true because it is not a standard declarative sentence. Instead, it may be:
                             \newline
1. A sentence fragment or incomplete thought
\newline
2. A question (rhetorical or otherwise)
\newline
3. A plain description without any claim
\newline
4. An exclamation or interjection
\newline
5. A command or request
\newline
6. Any other type of non-declarative expression
\newline
The key aspect is that the summary does not present a statement that can be directly evaluated for factual accuracy against the source document. This creates an evaluability issue because there's no clear assertion to assess for truthfulness or correspondence with the source information, or if there is an implicit assertion, the sentence's non-declarative nature makes it hard to identify the assertion.
\newline
Note that sentences in headline style (e.g., with articles omitted) should still be considered assertions if they convey a clear, evaluable claim despite their condensed format. The focus should be on the content and function of the sentence,not just its grammatical form.    
& Table \ref{tab:non_asserstion_example}\\
% \midrule
                             \cmidrule{2-4}
                             & Other meaning \newline phenomenon                                &  There is something else about the literal meaning of the summary sentence that may make it challenging to assess its factuality, which is not covered by other categories in the meaning phenomena taxonomy. This could include, but is not limited to:
                             \newline
1. Use of metaphorical or highly figurative language that doesn't have a clear, literal correspondence to the source document's content.
  \newline
2. Referential ambiguities not covered by the existing ambiguity categories, such as unclear pronoun references without clear antecedents in the summary or its context.
  \newline
3. Unusual or creative uses of language that introduce interpretive challenges not captured by other categories.
  \newline
The key aspect is that these phenomena should create a significant obstacle in evaluating the factual accuracy of the summary against the source document. The issue should be substantial enough that it genuinely impedes the ability to determine if the summary accurately represents the source information.
Note that this category should only be used when the meaning phenomenon doesn't fit clearly into any other category in the taxonomy. Minor stylistic choices or common figures of speech that don't significantly impact evaluability should not be included.    
& Table \ref{tab:other_meaning_example}\\
\midrule
Context \newline phenomena            & Decontextualization                                     &                     The summary sentence presents information from the source document in a way that significantly alters its intended meaning or interpretation by removing crucial contextual elements. This can include presenting hypothetical scenarios as facts, stripping statements of important qualifications, or omitting key background information. The removal of context should create a meaningful evaluability issue by changing the meaning, losing important nuances, or making it difficult to assess the summary's factuality against the source. This issue should be substantial enough to alter the interpretation of the information, not merely a simplification that maintains the core meaning and context.                                   
& Table \ref{tab:decontextualization_example}\\
\cmidrule{2-4}
                             & Conflation                                              &    Conflation occurs when the summary sentence inappropriately combines or merges distinct pieces of information from the source document in a way that significantly affects the evaluability of the summary's factuality. This can include:
                             \newline
1. Merging separate topics or events as if they were a single issue.
 \newline
2. Combining attributes or characteristics of different entities or concepts.
 \newline
3. Blending outcomes or decisions related to distinct matters.
 \newline
The key aspect is that this merging of information creates a meaningful evaluability issue by misrepresenting relationships between different pieces of information or making it challenging to accurately assess the factuality of the combined statement. Conflation should be substantial enough to create genuine difficulty in evaluating the summary against the source document, beyond minor simplifications or simple misattributions that don't involve merging distinct concepts or information.  
& Table \ref{tab:conflation_example}\\
                             \cmidrule{2-4}
                             & Other context \newline phenomenon                                &   This category covers context-related challenges in evaluating the summary sentence's factuality that don't fit neatly into the Decontextualization or Conflation categories. These issues arise from how the summary interprets or presents the relationships between different pieces of information in the source document. Examples might include:
                             \newline
1. Inferring causal relationships not explicitly stated in the source.
\newline
2. Reordering information in a way that implies a different significance or relationship than in the original context.
\newline
3. Drawing conclusions about the overall meaning or importance of information based on its placement or context in the source document.
\newline
The key aspect is that these phenomena should create a significant obstacle in evaluating the factual accuracy of the summary against the source document, stemming from how the summary interprets or represents the context of the information. The issue should be substantial enough that it genuinely impedes the ability to determine if the summary accurately represents the source information and its intended meaning or significance.
This category should only be used when the context-related issue doesn't clearly fit into Decontextualization or Conflation,and when it creates a genuine evaluability challenge. 
& Table \ref{tab:other_context_example}
\\
\midrule
                       Other &  Other evaluability \newline issue    & This category covers evaluability challenges that don't fit into any other category in the taxonomy. These issues should significantly impede the ability to assess the factual accuracy of the summary sentence against the source document.Examples might include:
                       \newline
1. Subjective interpretations of objective information that make factual assessment difficult.
\newline
2. Reliance on specialized cultural or contextual knowledge not provided in the source document and not common enough to be considered general knowledge.
\newline
3. Novel or unique challenges in comparing the summary to the source that aren't captured by existing categories.
\newline
It's crucial to note that factual errors alone do not create an evaluability issue. The key aspect is that these phenomena should create a significant obstacle in determining whether the summary accurately represents the information in the source document.
This category should only be used when the evaluability issue doesn't clearly fit into any other category in the taxonomy and when it creates a genuine, substantial challenge in assessment.
& Table \ref{tab:other_example}
\\ \bottomrule
\end{tabular}
\caption{Ambiguity taxonomy: categories and definitions}
\label{tab:ambiguity2}
\end{table*}