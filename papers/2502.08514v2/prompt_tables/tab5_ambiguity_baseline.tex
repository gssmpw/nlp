\begin{table*}
\centering
\small
\begin{tabular}{@{}p{14cm}@{}}
\toprule
You are given a document and a summary. You will go over the document in the <doc></doc> tags carefully and try to understand it fully. Then you look at the summary in <summary></summary> tags carefully. Your task is to identify whether the summary contains an ambiguity according to the provided ambiguity taxonomy in <taxonomy></taxonomy> tags. A summary is ambiguous if it can have multiple correct interpretations.
\newline
\newline
 <doc>
    \newline
    \%s
    \newline
    </doc>
    \newline
    \newline
    <summary>
    \newline
    \%s
    \newline
    </summary>
    \newline
    \newline
<taxonomy>
\newline
    1. Deduction: The summarizer has made a logical deduction (well or poorly), utilizing premises from the source document to draw a conclusion that cannot be directly traced to the source document.
    \newline
    2. Common-sense inference: The summarizer appears to have made an inference based on common sense notions.
    \newline
    3. Value-based inference: The summarizer appears to have made an inference based on assumed values.
    \newline
    4. Other implicit reasoning phenomenon: Some other kind of implicit reasoning took place that affects the summary's evaluability.
    \newline
    5. Hypernymy/Generalization: A more general meaning is used in the summary than is observed in the source document (for the same topic).
    \newline
    6. Hyponymy/Specialization: A more specific meaning is used in the summary than is observed in the source document (for the same topic).
    \newline
    7. Synonymy/Paraphrasing: Meaning from the source document is paraphrased in such a way that interpretation is challenged. The meaning has not technically changed, but the way the meaning is built changed.
    \newline
    8. Structural ambiguity: A phrase or sentence in the summary has multiple valid parses (multiple valid syntactic structures), and it is not obvious which parse is intended.
    \newline
    9. Lexical ambiguity: A word in the summary has multiple valid interpretations, and it is not obvious which meaning is intended.
    \newline
    10. Other ambiguity phenomenon: There is another type of linguistic ambiguity in the summary that is likely to cause difficulty in interpretation. Other types of ambiguity include scope ambiguity and pronoun reference ambiguity.
    \newline
    11. Vagueness: The meaning of part of the summary is underspecified, resulting in many realities being compatible with the claim made. For this use case, it would be so many realities that there is confusion about what claim is actually being made and whether the claim can be evaluated reliably.
    \newline
    12. Other meaning phenomenon: There is something else about the literal meaning of the summary that may have made it challenging to assess its factuality.
    \newline
    13. Decontextualization: The summary puts forth or describes something outside of the context in which its meaning was meant to be interpreted. It takes on new meaning or loses its meaning outside of that context.
    \newline
    14. Conflation: The summary joins or synthesizes pieces of information that were independently relevant in the source document. (It may have done this to good effect or to bad effect.)
    \newline
    15. Other context phenomenon: Some other challenge related to the relationship between the summary's meaning and the context(s) in the source document.
    \newline
    </taxonomy>
\newline
\newline
Go over the agents responses, summarize them by saying who agrees/disagrees. Then looking at the agents responses, how well they are associated with the guidelines and finally your own judgement of the summary using the provided guidelines, determine if the summary is factually consistent with the document. Provide your evaluation between <label></label> keys with values 1 (consistent) or 0 (inconsistent) and add your explanations in <explanation></explanation> XML tags.
    \\ 
 \bottomrule
\end{tabular}
\caption{Prompt used for ambiguity detection baseline.}
\label{tab:ambiguity_baseline_prompt}
\end{table*}