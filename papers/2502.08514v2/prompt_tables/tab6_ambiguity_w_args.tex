\begin{table*}
\centering
\scriptsize
\begin{tabular}{@{}p{14cm}@{}}
\toprule
You are given a document and a summary. You will go over the document in the <doc></doc> tags carefully and try to understand it fully. Then you look at the summary in <summary></summary> tags carefully. Evaluator agents have had rounds of discussion to identify whether the summary is factual or not and you can see their arguments in <arguments></arguments> tags. Different agents might have contrasting reasonings on whether the summary is factual or not and they might be correct in their judgement even though they have opposing views. Your task is to go over the arguments and identify whether the summary contains an ambiguity using the provided ambiguity taxonomy in <taxonomy></taxonomy> tags that can cause opposing views of the factuality. An ambiguity is present when the summary can be correctly classified as both factual and non-factual at the same time. Please note that the arguments might not be correct as the agents might have misused the provided guidelines in <guidelines></guidelines> tags so first make sure the agents’ arguments indeed follow the guidelines and then only consider the ones that are sound in your ambiguity evaluation.
\newline
\newline
 <doc>
    \newline
    \%s
    \newline
    </doc>
    \newline
    \newline
    <summary>
    \newline
    \%s
    \newline
    </summary>
    \newline
    \newline
    <arguments>
    \newline
    \%s
    \newline
    </arguments>
    \newline
    \newline
<guidelines>
\newline
    1. You should aim for accuracy and not comprehensiveness. If individual facts are correct, the summary is factually consistent regardless of its comprehensiveness.
    \newline
    2. A summary does not imply that its facts are the only ones mentioned in the dialogue.
    \newline
    3. The summary is factually inconsistent if it makes an assumption that is not supported (explicitly or implicitly) by the document.
    \newline
    4. The summary is factually inconsistent if it includes any information (even a minor detail) that is not present in the document or can not be entailed from the document.
    \newline
    5. The summary is factually consistent if it is a paraphrase of the document and it does not change the meaning of what is stated in the document.
    \newline
    6. Details (even crucial) that are present in the document but omitted in the summary do not lead to factual inconsistency.
    \newline
    7. lack of coherence between summary sentences does not necessarily lead to factual inconsistency.
    \newline
    8. The summary should not hallucinate new entities such as new people or locations not mentioned in the document otherwise it is factually inconsistent.
    \newline
    9. The summary does not have to provide the context or focus only on the main points of the document, it can only focus on a minor concept.
    \newline
    10. The summary is factually consistent even if it omits crucial details from document.
    \newline
    11. The addition of details that are not mentioned in the document or can not be entailed from it, makes the summary factually inconsistent.
    \newline
    12. Every word or phrase of the summary (or its paraphrase) should be present in the document otherwise the summary is factually inconsistent.
    \newline
    13. If even a single part of the summary is factually inconsistent, then the whole summary is factually inconsistent.
    \newline
    </guidelines>
\newline
\newline
<taxonomy>
\newline
    1. Deduction: The summarizer has made a logical deduction (well or poorly), utilizing premises from the source document to draw a conclusion that cannot be directly traced to the source document.
    \newline
    2. Common-sense inference: The summarizer appears to have made an inference based on common sense notions.
    \newline
    3. Value-based inference: The summarizer appears to have made an inference based on assumed values.
    \newline
    4. Other implicit reasoning phenomenon: Some other kind of implicit reasoning took place that affects the summary's evaluability.
    \newline
    5. Hypernymy/Generalization: A more general meaning is used in the summary than is observed in the source document (for the same topic).
    \newline
    6. Hyponymy/Specialization: A more specific meaning is used in the summary than is observed in the source document (for the same topic).
    \newline
    7. Synonymy/Paraphrasing: Meaning from the source document is paraphrased in such a way that interpretation is challenged. The meaning has not technically changed, but the way the meaning is built changed.
    \newline
    8. Structural ambiguity: A phrase or sentence in the summary has multiple valid parses (multiple valid syntactic structures), and it is not obvious which parse is intended.
    \newline
    9. Lexical ambiguity: A word in the summary has multiple valid interpretations, and it is not obvious which meaning is intended.
    \newline
    10. Other ambiguity phenomenon: There is another type of linguistic ambiguity in the summary that is likely to cause difficulty in interpretation. Other types of ambiguity include scope ambiguity and pronoun reference ambiguity.
    \newline
    11. Vagueness: The meaning of part of the summary is underspecified, resulting in many realities being compatible with the claim made. For this use case, it would be so many realities that there is confusion about what claim is actually being made and whether the claim can be evaluated reliably.
    \newline
    12. Other meaning phenomenon: There is something else about the literal meaning of the summary that may have made it challenging to assess its factuality.
    \newline
    13. Decontextualization: The summary puts forth or describes something outside of the context in which its meaning was meant to be interpreted. It takes on new meaning or loses its meaning outside of that context.
    \newline
    14. Conflation: The summary joins or synthesizes pieces of information that were independently relevant in the source document. (It may have done this to good effect or to bad effect.)
    \newline
    15. Other context phenomenon: Some other challenge related to the relationship between the summary's meaning and the context(s) in the source document.
    \newline
    </taxonomy>
\newline
\newline
Go over the agents responses, summarize them by saying who agrees/disagrees. Then looking at the agents responses, how well they are associated with the guidelines and finally your own judgement of the summary using the provided guidelines, determine if the summary is factually consistent with the document. Provide your evaluation between <label></label> keys with values 1 (consistent) or 0 (inconsistent) and add your explanations in <explanation></explanation> XML tags.
    \\ 
 \bottomrule
\end{tabular}
\caption{Prompt used for ambiguity detection with debate arguments.}
\label{tab:ambiguity_w_args_prompt}
\end{table*}