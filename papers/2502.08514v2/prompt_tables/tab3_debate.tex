\begin{table*}
\centering
\small
\begin{tabular}{@{}p{14cm}@{}}
\toprule
You are given a document and a summary (summarizing only a part of the document). You will go over the document in the <doc></doc> tags carefully and try to understand it fully. Then you look at the summary sentence in <summary></summary> tags. You have to identify whether the summary is factually consistent with the given document. There are also other evaluator agents assigned the same task as you and you can also see the discussion history in <chat\_history></chat\_history> tags below. You are also given a set of guidelines in <guideline></guidelines> that you can refer to when making your arguments. Go over them carefully and make sure you remember them.
\newline
\newline
<guidelines>
\newline
    1. You should aim for accuracy and not comprehensiveness. If individual facts are correct, the summary is factually consistent regardless of its comprehensiveness.
    \newline
    2. A summary does not imply that its facts are the only ones mentioned in the dialogue.
    \newline
    3. The summary is factually inconsistent if it makes an assumption that is not supported (explicitly or implicitly) by the document.
    \newline
    4. The summary is factually inconsistent if it includes any information (even a minor detail) that is not present in the document or can not be entailed from the document.
    \newline
    5. The summary is factually consistent if it is a paraphrase of the document and it does not change the meaning of what is stated in the document.
    \newline
    6. Details (even crucial) that are present in the document but omitted in the summary do not lead to factual inconsistency.
    \newline
    7. lack of coherence between summary sentences does not necessarily lead to factual inconsistency.
    \newline
    8. The summary should not hallucinate new entities such as new people or locations not mentioned in the document otherwise it is factually inconsistent.
    \newline
    9. The summary does not have to provide the context or focus only on the main points of the document, it can only focus on a minor concept.
    \newline
    10. The summary is factually consistent even if it omits crucial details from document.
    \newline
    11. The addition of details that are not mentioned in the document or can not be entailed from it, makes the summary factually inconsistent.
    \newline
    12. Every word or phrase of the summary (or its paraphrase) should be present in the document otherwise the summary is factually inconsistent.
    \newline
    13. If even a single part of the summary is factually inconsistent, then the whole summary is factually inconsistent.
    \newline
    </guidelines>
\newline
\newline
    <doc>
    \newline
    \%s
    \newline
    </doc>
    \newline
    \newline
    <summary>
    \newline
    \%s
    \newline
    </summary>
    \newline
    \newline
    <chat\_history>
    \newline
    \%s
    \newline
    </chat\_history>
    \newline
    \newline
    Determine if the summary is factually consistent with the document provided above. Provide your evaluation between <label></label> tags with values 1 (consistent) or 0 (inconsistent) and add your explanations in <explanation></explanation> XML tags. Before answering, please think about the question within <thinking></thinking> XML tags.
    \\ 
 \bottomrule
\end{tabular}
\caption{Prompt used for evaluator agents during debate for faithfulness evaluation}
\label{tab:debate_prompt}
\end{table*}