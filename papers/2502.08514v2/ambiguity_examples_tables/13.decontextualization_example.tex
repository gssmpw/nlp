\begin{table*}
\centering
\small
\begin{tabular}{@{}p{14cm}@{}}
\toprule
\textbf{\textcolor{red}{Ambiguous example: Cyberattack Warning}}
\newline
\textbf{Source document excerpt:}
In a hypothetical scenario presented during a cybersecurity conference, Dr. Jane Smith, a leading expert in the field, stated, “If a nation-state were to launch a coordinated cyberattack on our power grid, it could potentially leave millions without electricity for weeks.” She emphasized that this was a worst-case scenario used to illustrate the importance of robust cybersecurity measures, not a prediction of an imminent threat.
\newline
\textbf{Summary sentence:}
An expert warned that millions could be left without power for weeks due to cyberattacks.
\newline
\textbf{Explanation:}
This is a positive example of Decontextualization because the summary sentence presents Dr. Smith's statement outside of its crucial context. In the source, it’s clear that she was discussing a hypothetical scenario in a specific setting (a cybersecurity conference) to illustrate a point. The summary, however, presents it as a general warning about a real threat,stripping away the hypothetical nature and the purpose of the example. This decontextualization significantly changes the meaning and urgency of the statement, making it challenging to evaluate the factuality of the summary against the source document. The summary takes on a new, more alarming meaning when removed from its original context, creating a clear evaluability issue.
\\
\midrule
\textbf{\textcolor{teal}{Non-ambiguous example: Coffee Memory Study}}
\newline
\textbf{Source document excerpt:}
new study published in the Journal of Nutrition examined the effects of coffee consumption on cognitive function in adults over 65. The researchers found that participants who drank 2-3 cups of coffee daily showed a 15\%improvement in short-term memory tests compared to non-coffee drinkers. However, the study’s authors cautioned that more research is needed to establish a causal relationship and to account for other lifestyle factors that might influence cognitive health.
\newline
\textbf{Summary sentence:}
Recent research suggests moderate coffee consumption may boost short-term memory in older adults.
\newline
\textbf{Explanation:}
An annotator might be tempted to label this as Decontextualization because the summary doesn't mention the study’s limitations or the need for further research. However, this is not a true case of Decontextualization. The summary sentence accurately represents the main finding of the study within its proper context. It uses the word “suggests” to indicate that the relationship is not definitively proven, which aligns with the cautionary note in the source. The summary doesn't remove the information from its research context or change its meaning. While it doesn't include all the details from the source, it presents a fair and contextualized summary of the key finding. Therefore, this example doesn't create an evaluability issue due to Decontextualization and shouldn't’t be labeled as such.
    \\ 
 \bottomrule
\end{tabular}
\caption{Examples for context phenomena: Decontextualization}
\label{tab:decontextualization_example}
\end{table*}