\begin{table*}
\centering
\small
\begin{tabular}{@{}p{14cm}@{}}
\toprule
\textbf{\textcolor{red}{Ambiguous example: Zoning Question}}
\newline
\textbf{Source document excerpt:}
The city council voted 7-2 in favor of the new zoning ordinance, which will allow for the construction of multi-family housing units in previously single-family residential areas. Proponents argue this will help address the city’s housing shortage, while opponents express concerns about increased traffic and changes to neighborhood character.
\newline
\textbf{Summary sentence:}
Will the new zoning law really solve the housing crisis?
\newline
\textbf{Explanation:}
This is a positive example of Non-assertion because the summary sentence is a question rather than a declarative statement. It doesn’t assert any facts or make any claims about the zoning law or its effects. Instead, it poses a rhetorical question that cannot be evaluated for factual accuracy against the source document. This creates an evaluability issue because there’s no clear assertion to assess – the question merely raises a point for consideration without providing any information that can be judged as true or false.
\\
\midrule
\textbf{\textcolor{red}{Ambiguous example: Economic Data}}
\newline
\textbf{Source document excerpt:}
The annual economic report released by the Federal Reserve indicates that inflation rates have decreased from 6.5\% to 3.2\% over the past year, while unemployment has remained stable at 3.8\%. The report suggests that these trends reflect a gradual stabilization of the economy following recent global disruptions.
\newline
\textbf{Summary sentence:}
Promising economic indicators
\newline
\textbf{Explanation:}
This is a positive example of Non-assertion because the summary sentence is a mere phrase rather than a complete sentence. It doesn’t make any explicit claims or assertions about the economic situation. While it implies that there are economic indicators and that they are promising, it doesn’t actually state this as a fact. The phrase could be a title, a category label, or a fragment of a longer thought. Without a verb or a complete sentence structure, there’s no clear assertion that can be evaluated for factual accuracy against the source document. This creates an evaluability issue because there’s no specific claim being made – the phrase merely suggests a topic without providing any information that can be judged as true or false.
\\
\midrule
\textbf{\textcolor{teal}{Non-ambiguous example: Whale Recovery}}
\newline
\textbf{Source document excerpt:}
A recent study by marine biologists has found that the population of blue whales in the Eastern Pacific has increased by 20\% over the past decade, attributed to strict international whaling bans and protected marine areas.
\newline
\textbf{Summary sentence:}
Scientists Report Encouraging Trend in Blue Whale Numbers
\newline
\textbf{Explanation:}
An annotator might be tempted to mark this as a Non-assertion issue because it’s written in headline style, lacking the article “an” before “encouraging trend.” However, despite its condensed form, this sentence still functions as a declarative statement with clear assertive force. It makes a specific claim that can be evaluated against the source document: scientists have reported a trend, and this trend is encouraging for blue whale populations. The omission of the article is a common feature in headlines, and while this might not be a typical style for a summary, its style doesn’t negate the sentence’s declarative nature.The increase in whale population described in the source can reasonably be characterized as “encouraging.” Therefore, this is not a case of Non-assertion, as it does make a claim that can be evaluated for accuracy, despite its headline-style formatting.
    \\ 
 \bottomrule
\end{tabular}
\caption{Examples for meaning phenomena: non-assertion}
\label{tab:non_asserstion_example}
\end{table*}