\begin{table*}
\centering
\small
\begin{tabular}{@{}p{14cm}@{}}
\toprule
\textbf{\textcolor{red}{Ambiguous example: Exercise Effects}}
\newline
\textbf{Source document excerpt:}
A longitudinal study tracking 10,000 adults over 20 years found that those who engaged in regular physical activity (defined as at least 150 minutes of moderate exercise per week) had a 37\% lower risk of developing type 2 diabetes compared to sedentary individuals. The study controlled for factors such as diet, family history, and initial BMI.
\newline
\textbf{Summary sentence:}
Research suggests that being active may have health benefits.
\newline
\textbf{Explanation:}
This is a positive example of vagueness because the summary sentence is extremely underspecified compared to the source. The terms “being active,” “may have,” and “health benefits” are so vague that they’re compatible with an enormous range of realities. What counts as “being active”? How probable is “may”? What specific “health benefits” are we talking about? The vagueness is so significant that it’s unclear what specific claim is being made, making it very difficult to evaluate the accuracy of the summary against the precise information in the source document. This level of vagueness creates a genuine evaluability issue, as the summary could be considered technically true for even minimal activity providing any small health benefit, but the source document hones in on much more specific health benefits for specific conditions.
\\
\midrule
\textbf{\textcolor{teal}{Non-ambiguous example: EV Performance}}
\newline
\textbf{Source document excerpt:}
The new electric vehicle model has a range of 400 miles on a single charge, compared to the current industry average of 250 miles. It can accelerate from 0 to 60 mph in 3.5 seconds, while the average for electric cars in its class is 6 seconds. These specifications place it in the top 1\% of electric vehicles currently on the market in terms of both range and acceleration.
\newline
\textbf{Summary sentence:}
The latest electric car boasts impressive range and acceleration capabilities.
\newline
\textbf{Explanation:}
An annotator might initially consider marking this as a vagueness issue because terms like “impressive” and“capabilities” are not as specific as the numbers in the source document. However, upon examination of the source, it’s clear that this level of vagueness doesn’t create a significant evaluability issue. The use of “impressive” is justified by the explicit comparisons to industry averages and the car’s placement in the top 1\% for both metrics. While the summary doesn’t provide exact figures, it doesn’t create confusion about what claim is being made or make evaluation unreliable. The characterization of the range and acceleration as “impressive” is fair and evaluable against the specific comparisons provided in the source document. This example demonstrates acceptable summarization rather than problematic vagueness.
    \\ 
 \bottomrule
\end{tabular}
\caption{Examples for meaning phenomena: Vagueness}
\label{tab:vagueness_example}
\end{table*}