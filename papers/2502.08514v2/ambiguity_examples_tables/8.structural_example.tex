\begin{table*}
\centering
\small
\begin{tabular}{@{}p{14cm}@{}}
\toprule
\textbf{\textcolor{red}{Ambiguous example: University Changes}}
\newline
\textbf{Source document excerpt:}
The university announced new funding for research projects in biology and chemistry departments. Separately, they introduced stricter guidelines for laboratory safety procedures and a new online course registration system.
\newline
\textbf{Summary sentence:}
The university implemented new laboratory safety guidelines and online registration for biology courses.
\newline
\textbf{Explanation:}
This is a positive example of Structural ambiguity because the summary sentence can be parsed in two distinctly different and plausible ways: 1) The university implemented [new laboratory safety guidelines] and [online registration for biology courses]. 2) The university implemented [new laboratory safety guidelines and online registration] for [biology courses]. In the first interpretation, the university did two separate things: implemented new lab safety guidelines (potentially for all departments) and introduced online registration specifically for biology courses. In the second interpretation, both the new safety guidelines and the online registration system are specifically for biology courses. This significant difference in meaning based on the syntactic structure creates a genuine evaluability challenge, as it’s not clear which interpretation is intended and they lead to very different factual claims about the scope of the implementations.
\\
\midrule
\textbf{\textcolor{red}{Ambiguous example: Merkel’s Statement}}
\newline
\textbf{Source document excerpt:}
Mrs Merkel said she wanted friendly relations with both countries as well as Russia but Europe now had to “fight for its own destiny”. Her comments come after Mr Trump refused to re-commit to the 2015 Paris climate deal at the G7 summit. “The times in which we could completely depend on others are on the way out. I’ve experienced that in the last few days,” Mrs Merkel told a crowd at an election rally in Munich, southern Germany. The relationship between Berlin and new French President Emmanuel Macron had to be a priority, Mrs Merkel said, adding: “We Europeans have to take our destiny into our own hands.” Mr Trump has previously pledged to abandon the Paris deal, and expressed doubts about climate change.Speaking in Brussels last week, Mr Trump also told Nato members to spend more money on defence and did not re-state his administration’s commitment to Nato’s mutual security guarantees.
\newline
\textbf{Summary sentence:}
German Chancellor Angela Merkel has said the times when Europe could “completely depend on others” are “on the way out” after US President Donald Trump’s comments on climate change and Brexit.
\newline
\textbf{Explanation:}
This is a positive example of Structural ambiguity because the summary sentence can be parsed in two distinctly different ways: 1) … after [[US President Donald Trump’s comments on climate change] and [Brexit]] In this interpretation, Merkel’s statement is in response to two separate things: Trump’s comments on climate change, and Brexit. 2) … after [US President Donald Trump’s comments on [[climate change] and [Brexit]]] In this interpretation, Merkel’s statement is in response to Trump’s comments on two topics: climate change and Brexit. This ambiguity in the syntactic structure creates a significant evaluability challenge. The first interpretation suggests that Brexit itself (not Trump’s comments on it) is part of the reason for Merkel’s statement, while the second interpretation attributes both climate change and Brexit comments to Trump. These different parses lead to different factual claims about the reasons behind Merkel’s statement, making it difficult to evaluate the accuracy of the summary without clarification.
\\
\midrule
\textbf{\textcolor{teal}{Non-ambiguous example: Company Growth}}
\newline
\textbf{Source document excerpt:}
The annual report shows that the company’s profits increased by 10\% in the technology sector and 5\% in the retail sector.
\newline
\textbf{Summary sentence:}
The company saw growth in both its technology and retail divisions.
\newline
\textbf{Explanation:}
An annotator might initially consider marking this as a Structural ambiguity issue, but upon closer examination, it becomes clear that this is actually an example of Vagueness. The sentence structure itself is not ambiguous; there’s only one valid parse. However, the term “growth” is vague and underspecified compared to the precise percentages given in the source document. This vagueness allows for many realities to be compatible with the claim (any positive growth in both sectors would satisfy the summary). While this does create some evaluability challenges, it’s due to the lack of specificity rather than multiple possible syntactic structures. Therefore, this example would be better categorized under the Vagueness evaluability issue rather than Structural ambiguity.
    \\ 
\midrule
\textbf{\textcolor{teal}{Non-ambiguous example: City Achievements}}
\newline
\textbf{Source document excerpt:}
The city’s annual report highlighted two major achievements: the completion of the new public library and the successful implementation of a city-wide recycling program. Mayor Johnson praised the efforts of city employees and volunteers who contributed to these projects.
\newline
\textbf{Summary sentence:}
The mayor commended city workers and volunteers who built the library and implemented the recycling program.
\newline
\textbf{Explanation:}
An annotator might initially be tempted to mark this as a Structural ambiguity issue because the sentence could theoretically be parsed in two ways: 1) The mayor commended ((city workers and volunteers) who built the library and implemented the recycling program), or 2) The mayor (commended city workers and volunteers who built the library) and(implemented the recycling program). However, this should not be considered a case of Structural ambiguity that creates an evaluability issue. While there is technically an ambiguity in the syntactic structure, the second interpretation in which it was the mayor herself who implemented the recycling program is extremely implausible. No reasonable reader would assume this interpretation, given common knowledge about how city projects typically work. The much more likely and sensible interpretation is that it was the city workers and volunteers who built the library and implemented the recycling program. Therefore, this example doesn’t create a significant evaluability challenge and shouldn’t be labeled as having a Structural ambiguity issue.
    \\ 
 \bottomrule
\end{tabular}
\caption{Examples for meaning phenomena: Structural ambiguity}
\label{tab:structural_example}
\end{table*}