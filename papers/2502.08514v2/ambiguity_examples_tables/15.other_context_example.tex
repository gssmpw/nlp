\begin{table*}
\centering
\small
\begin{tabular}{@{}p{14cm}@{}}
\toprule
\textbf{\textcolor{red}{Ambiguous example: Parking Solution}}
\newline
\textbf{Source document excerpt:}
The city’s annual report included several sections. In the “Challenges” section, it stated: “Downtown parking remains a significant issue, with demand often exceeding supply during peak hours.” Later, in the “Future Plans” section,it mentioned: “A new multi-story parking garage is scheduled to begin construction next year, which will add 500 parking spaces to the downtown area.”
\newline
\textbf{Summary sentence:}
The city is building a new parking garage downtown because parking demand exceeds supply during peak hours.
\newline
\textbf{Explanation:}
This is a positive example of an “Other context phenomenon” because the summary sentence creates a causal relationship between two pieces of information that were presented in separate contexts within the source document. While the parking issue and the new garage construction are both mentioned, the source does not explicitly state that one is the direct cause of the other. The summary’s assertion of causality (“because”) goes beyond what’s stated in the source and creates an evaluability issue. This doesn't fit neatly into Decontextualization or Conflation categories. Instead, it represents a different kind of context-related challenge where the summarizer has inferred a relationship between separate pieces of information based on their appearance in the same document, even though they were presented in different sections with different purposes (one describing challenges, the other outlining future plans).
\\
\midrule
\textbf{\textcolor{teal}{Non-ambiguous example: Company Outlook}}
\newline
\textbf{Source document excerpt:}
In a press conference, the CEO stated: “Our company’s profits have increased by 15\% this quarter.”Later, in response to a journalist’s question about potential layoffs, she said: “We have no plans for layoffs at this time. In fact,we’re looking to expand our workforce in the coming months.”
\newline
\textbf{Summary sentence:}
Despite increased profits, the company has no plans for layoffs and is looking to hire more employees.
\newline
\textbf{Explanation:}
An annotator might be tempted to label this as an “Other context phenomenon” because the summary sentence brings together information from two different parts of the press conference and seems to imply a contrast (“Despite”). However,this is not a true case of a context-related evaluability issue. The summary accurately represents the information provided in the source document without changing its meaning or creating any challenges in evaluation. The use of “Despite” to connect the two pieces of information is a reasonable interpretation that doesn't misrepresent the context or create any evaluability issues. The relationship between the profit increase and the hiring plans is implied but doesn't distort the original information or its context.Therefore, this example shouldn't be labeled as having any context-related evaluability issue.
    \\ 
 \bottomrule
\end{tabular}
\caption{Examples for context phenomena: Other context phenomenon}
\label{tab:other_context_example}
\end{table*}