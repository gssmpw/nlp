\begin{table*}
\centering
\small
\begin{tabular}{@{}p{14cm}@{}}
\toprule
\textbf{\textcolor{red}{Ambiguous example: Economic Slowdown}}
\newline
\textbf{Source document excerpt:}
The latest economic report shows that the country's GDP growth has slowed to 1.2\% in the last quarter, down from 2.8\% in the previous quarter. Analysts attribute this decrease to global supply chain disruptions and increasing energy costs. The central bank has indicated it may consider adjusting interest rates in response to these trends.
\newline
\textbf{Summary sentence:}
The economy is navigating choppy waters as growth figures take a dive.
\newline
\textbf{Explanation:}
This is a positive example of an "other meaning phenomenon" because it uses metaphorical language that creates an evaluability issue. The phrases "navigating choppy waters" and "take a dive" are figurative expressions that, while evocative,don't have a clear, literal correspondence to the economic data presented in the source document. This use of metaphor makes it challenging to assess the factuality of the summary sentence against the precise figures and factual statements in the source.While the metaphors generally align with the idea of economic difficulty and declining growth, they introduce a level of interpretive ambiguity that goes beyond the categories we've discussed. It's not a case of generalization, specialization, paraphrasing,ambiguity, vagueness, or non-assertion, but rather a use of figurative language that affects the ability to directly evaluate the claim against the source material.
\\
\midrule
\textbf{\textcolor{red}{Ambiguous example: Surprising Endorsement}}
\newline
\textbf{Source document excerpt:}
In a surprising turn of events, Senator Jane Smith publicly endorsed her long-time rival, Governor Tom Brown, for the upcoming presidential election. This endorsement comes just weeks after Brown’s campaign criticized Smith’s voting record on healthcare reform. Political analysts suggest this move could significantly impact voter perceptions in key swing states.
\newline
\textbf{Summary sentence:}
The two politicians have a history of disagreement, so her endorsement of him shocked many.
\newline
\textbf{Explanation:}
This is a positive example of an “other meaning phenomenon,” specifically a kind of referential ambiguity, which is not included in the list of ambiguity types. The summary sentence uses pronouns “her” and “him” without clear antecedents within the sentence itself or the preceding context of the summary. While readers familiar with the source document could figure out that“her” refers to Senator Smith and “him” to Governor Brown, this is not explicit in the summary sentence or preceding context.This ambiguity in pronoun reference creates an evaluability issue because it’s not immediately clear who is endorsing whom,making it challenging to assess the factual accuracy of the statement against the source document. This type of referential ambiguity doesn’t fit neatly into the previously discussed categories but significantly affects the ability to evaluate the summary’s factuality, thus qualifying as an “other meaning phenomenon.”
\\
\midrule
\textbf{\textcolor{teal}{Non-ambiguous example: Whale Population}}
\newline
\textbf{Source document excerpt:}
A recent study conducted by marine biologists at the Pacific Oceanic Institute has revealed that the population of blue whales in the Eastern Pacific Ocean has increased by approximately 20\% over the past decade. Researchers attribute this growth to the effectiveness of international whaling bans and the establishment of protected marine areas. Dr. Sarah Johnson, lead author of the study, stated, “This is a promising sign for the species’ recovery, but continued conservation efforts are crucial.”
\newline
\textbf{Summary sentence:}
Blue whale numbers in the Eastern Pacific have shown a significant uptick, according to new research.
\newline
\textbf{Explanation:}
An annotator might be tempted to categorize this as “Other meaning phenomenon” due to the use of the colloquial term “uptick” to describe the population increase. They might argue that this informal term creates an evaluability issue because it’s not as precise as the percentage given in the source document. However, this is not a valid case for the “Other meaning phenomenon” category, nor does it present a significant evaluability issue. While the use of “uptick” is an example of paraphrasing, it is not so substantial as to warrant categorization under the Synonymy/Paraphrasing evaluability issue. The term“uptick” clearly conveys the idea of increase, and “significant” accurately reflects the 20\% growth mentioned in the source. The summary sentence makes a clear, evaluable claim that aligns with the information provided in the source document without introducing any meaningful ambiguity or difficulty in assessment. Therefore, this example does not qualify as an “other meaning phenomenon” and should not be marked as having any evaluability issue.
    \\ 
\midrule
\textbf{\textcolor{teal}{Non-ambiguous example: Job Market Trends}}
\newline
\textbf{Source document excerpt:}
A recent economic report shows that the unemployment rate has dropped from 5.2\% to 4.8\% over the past six months. During the same period, the number of job openings increased by 15\%, with the technology and healthcare sectors showing the strongest growth. However, wage growth remained stagnant at 1.5\% annually, barely keeping pace with inflation.
\newline
\textbf{Summary sentence:}
The job market is improving, but workers aren’t feeling the benefits yet.
\newline
\textbf{Explanation:}
An annotator might be tempted to categorize this as an “other meaning phenomenon” due to the somewhat abstract nature of “feeling the benefits.” They might argue that this creates an evaluability issue because it’s not a direct representation of the data in the source document. However, this is not a case of an “other meaning phenomenon.” Instead, this summary sentence contains a clear example of a common-sense inference. The first part of the sentence, “The job market is improving,” is a reasonable inference based on the lower unemployment rate and increased job openings mentioned in the source. The second part, “workers aren’t feeling the benefits yet,” is an inference based on the stagnant wage growth information. These inferences rely on common-sense connections between economic indicators and their real-world impacts.Therefore, this example should be labeled as a common-sense inference under the implicit reasoning category, rather than an“other meaning phenomenon” or any other category in the meaning phenomena taxonomy.
    \\ 
 \bottomrule
\end{tabular}
\caption{Examples for meaning phenomena: Other meaning phenomenon}
\label{tab:other_meaning_example}
\end{table*}