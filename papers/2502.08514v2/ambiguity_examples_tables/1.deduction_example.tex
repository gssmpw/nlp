\begin{table*}
\centering
\small
\begin{tabular}{@{}p{14cm}@{}}
\toprule
\textbf{\textcolor{red}{Ambiguous example: Street Closures}}
\newline
\textbf{Source document excerpt:}
The annual Bloomington Street Festival will be held on February 2nd and February 3rd this year.The event attracts thousands of visitors and features local artisans, food vendors, and live music performances.
\newline
\textbf{Summary sentence:}
Some streets will be closed on February 2-3.
\newline
\textbf{Explanation:}
This is a positive example of a Deduction evaluability issue. The summary sentence makes a claim that is not directly stated in the source document. Instead, it appears to be the result of a deduction based on two premises from the source:1) The festival is held on February 2nd and 3rd, and 2) It’s referred to as a “street festival”. While this deduction is logically valid, it creates an evaluability issue because an evaluator would likely try to verify this claim directly in the source document, where it’s not explicitly stated. The deductive leap, while reasonable, makes it harder to assess the factuality of the summary sentence against the source document. Therefore, this summary sentence should be marked with the `Deduction' label.
\\
\midrule
\textbf{\textcolor{teal}{Non-ambiguous example: Music Budget}}
\newline
\textbf{Source document excerpt:}
The Smithville School Board meeting on Tuesday addressed several topics. The board unanimously approved a budget increase of \$500,000 for the music department. This additional funding will be used to purchase new instruments, hire two part-time music teachers, and expand the after-school music program to include elementary school students.
\newline
\textbf{Summary sentence:}
The Smithville School Board has decided to invest more in music education.
\newline
\textbf{Explanation:}
This is a negative example for the Deduction evaluability issue. While the summary sentence does involve a minor deduction - connecting the approval of a budget increase for the music department with “investing more in music education” - this deduction is so straightforward and closely tied to the explicit information in the source that it doesn’t create an evaluability issue. The connection between increasing the budget for instruments, hiring teachers, and expanding programs, and “investing more in music education” is immediate and obvious. An evaluator would have no difficulty assessing the factuality of this summary sentence based on the information provided in the source document. Therefore, despite involving a slight deductive step, this summary sentence should not be marked with the ‘Deduction’ label as it doesn’t create any significant challenge in evaluation.
    \\ 
 \bottomrule
\end{tabular}
\caption{Examples for implicit reasoning phenomena: deduction. The \textcolor{red}{red example} is the one with ambiguity and the \textcolor{teal}{teal example} is a case that the deduction would not lead to ambiguities.}
\label{tab:deduction_example}
\end{table*}