\begin{table*}
\centering
\small
\begin{tabular}{@{}p{14cm}@{}}
\toprule
\textbf{\textcolor{red}{Ambiguous example: Economic Impact}}
\newline
\textbf{Source document excerpt:}
The annual Bloomington Street Festival will be held next month in the downtown area. Organizers expect over 50,000 visitors from across the state to attend the two-day event, which features local artisans, food vendors, and live music performances.
\newline
\textbf{Summary sentence:}
The festival will significantly boost local business in downtown Bloomington.
\newline
\textbf{Explanation:}
This is a positive example of a Common-sense inference evaluability issue. The summary sentence makes a claim that is not explicitly stated in the source document. Instead, it appears to be an inference based on two premises: 1) The festival will bring over 50,000 visitors to downtown Bloomington (from the source), and 2) Large events that bring many people to an area tend to increase business for local businesses (a common-sense notion). While this inference is likely valid, it creates an evaluability issue because an evaluator would need to rely on the same common-sense notion to assess its accuracy, rather than finding the information directly stated in the source. This reliance on common-sense reasoning, while often accurate, makes it harder to objectively evaluate the factuality of the summary sentence against the source document.
\\
\midrule
\textbf{\textcolor{teal}{Non-ambiguous example: Voter Influence}}
\newline
\textbf{Source document excerpt:}
Chancellor Angela Merkel delivered a speech at an election rally in Munich on Sunday, addressing key policy issues ahead of the September elections. The event was attended by thousands of supporters and local party members.
\newline
\textbf{Summary sentence:}
Merkel’s speech may influence some German voters’ opinions.
\newline
\textbf{Explanation:}
This is a negative example for the Common-sense inference evaluability issue. While the summary sentence does involve an inference based on information from the source (Merkel made a speech at an election rally) and a common-sense notion (political speeches can affect voters’ opinions), this inference doesn’t create a significant evaluability issue. The common-sense premise that political speeches can influence voters is so widely accepted and fundamental to the understanding of political rallies that it doesn’t introduce any real challenge in evaluating the summary’s factuality. The connection between giving a speech at a rally and potentially influencing voters is immediate and obvious. Therefore, despite being an inference involving common sense, this summary sentence should not be marked as having an evaluability issue, as it doesn’t create any meaningful difficulty in assessment.
    \\ 
 \bottomrule
\end{tabular}
\caption{Examples for implicit reasoning phenomena: commonsense}
\label{tab:commonsense_example}
\end{table*}