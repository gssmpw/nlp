\begin{table*}
\centering
\small
\begin{tabular}{@{}p{14cm}@{}}
\toprule
\textbf{\textcolor{red}{Ambiguous example: Menu Calories}}
\newline
\textbf{Source document excerpt:}
The city’s new ordinance requires all restaurants to clearly display calorie information for each dish on their menus.
\newline
\textbf{Summary sentence:}
Local eateries must now post dishes’ calorie counts.
\newline
\textbf{Explanation:}
This is a positive example of Lexical ambiguity because the word “post” in the summary sentence has multiple valid interpretations. It could mean: 1) to physically display or affix the information in the restaurant, or 2) to publish or upload the information online. Without additional context, it’s not obvious which meaning is intended. This ambiguity makes it challenging to evaluate the factual accuracy of the summary, as the source document specifically mentions displaying the information on menus, but the summary’s use of “post” leaves room for different interpretations.
\\
\midrule
\textbf{\textcolor{teal}{Non-ambiguous example: Ancient Knowledge}}
\newline
\textbf{Source document excerpt:}
The latest archaeological findings suggest that the ancient civilization had advanced knowledge of astronomy and mathematics.
\newline
\textbf{Summary sentence:}
Recent discoveries indicate the early society was versed in celestial and numerical sciences.
\newline
\textbf{Explanation:}
An annotator might be tempted to mark this as a Lexical ambiguity issue because of the use of less common terms like “versed” or “celestial sciences”. However, this is not a true case of lexical ambiguity. While the words used in the summary are more formal or academic, they don’t have multiple valid interpretations in this context. “Versed” clearly means knowledgeable or skilled, and “celestial sciences” is an obvious reference to astronomy. The words, though perhaps less common, have singular,clear meanings in this context. Any perceived difficulty in evaluation comes from the use of synonyms or domain-specific language, not from words having multiple possible interpretations. Therefore, this example doesn’t present a lexical ambiguity evaluability issue.
    \\ 
 \bottomrule
\end{tabular}
\caption{Examples for meaning phenomena: Lexical ambiguity}
\label{tab:lexical_example}
\end{table*}