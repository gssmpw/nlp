\begin{table*}
\centering
\small
\begin{tabular}{@{}p{14cm}@{}}
\toprule
\textbf{\textcolor{red}{Ambiguous example: Zoning Vote}}
\newline
\textbf{Source document excerpt:}
The city council voted 7-2 in favor of the new zoning ordinance. Council member Johnson, who had previously expressed reservations, was absent due to illness.
\newline
\textbf{Summary sentence:}
The controversial zoning ordinance passed despite opposition.
\newline
\textbf{Explanation:}
This example creates an evaluability issue while also containing factual inaccuracies. The summary’s claim that the ordinance was “controversial” is not explicitly stated in the source and should be judged as factually inaccurate (separately from the evaluability issue). The phrase “despite opposition” creates an evaluability issue because while the 7-2 vote and mention of a councilmember’s previous reservations imply some level of opposition, the extent and nature of this opposition are not clearly defined in the source. This makes it challenging to evaluate the accuracy of the “despite opposition” claim. The summary thus combines a factual inaccuracy with an evaluability issue, demonstrating how implied information can complicate the assessment of a summary’s factual accuracy.
\\
\midrule
\textbf{\textcolor{red}{Ambiguous example: Art Sale}}
\newline
\textbf{Source document excerpt:}
The painting, a large canvas covered entirely in red paint, sold for \$3 million at auction.
\newline
\textbf{Summary sentence:}
A simplistic artwork fetched an outrageously high price at the recent sale.
\newline
\textbf{Explanation:}
This example creates an evaluability issue because it interprets objective information (the description and price of the painting) in a subjective manner. The terms “simplistic” and “outrageously high” are subjective judgments not present in the source, and while it’s probably true that most people would agree with these subjective statements, their presence in the summary sentence makes it difficult to evaluate the factual accuracy of the summary sentence.
\\
\midrule
\textbf{\textcolor{red}{Ambiguous example: Festival Timing}}
\newline
\textbf{Source document excerpt:}
The local council of Whittlesea, Victoria, has announced that this year’s Whittlesea Country Music Festival will take place on the second weekend of February, as is tradition.
\newline
\textbf{Summary sentence:}
Whittlesea’s annual music event is scheduled for the height of Australian summer, potentially impacting attendance.
\newline
\textbf{Explanation:}
This example creates an evaluability issue because it requires cultural and geographical knowledge not provided in the source document. The summary makes claims about the Australian summer season and its potential impact on the event,which aren't mentioned in the source. To evaluate the accuracy of this summary, one would need to know:
\newline
1. That February is indeed summer in Australia (opposite to the Northern Hemisphere).
\newline
2. That summer in Victoria can be extremely hot, potentially affecting outdoor events.
\newline
3. The typical weather patterns in Whittlesea specifically.
This information isn't common knowledge for many people outside Australia, and it’s not provided in the source. An evaluator would need to do external research to verify these claims, making it challenging to assess the summary’s factual accuracy based solely on the given source document. This goes beyond simple paraphrasing or inference, creating a unique evaluability issue related to cultural and contextual knowledge.
\\
\midrule
\textbf{\textcolor{teal}{Non-ambiguous example: Brain Chemistry}}
\newline
\textbf{Source document excerpt:}
The study found a significant increase in dopamine levels in the nucleus accumbens following the experimental treatment.
\newline
\textbf{Summary sentence:}
The research showed the treatment boosted feel-good neurotransmitters in the brain’s reward center.
\newline
\textbf{Explanation:}
An annotator might be tempted to label this as an “Other evaluability issue” due to the use of lay terms to describe technical concepts. They might argue that this creates a unique challenge in evaluating the summary’s accuracy. However, this is actually a clear case of Synonymy/Paraphrasing. The summary translates technical terms (“dopamine” and “nucleus accumbens”) into more accessible language (“feel-good neurotransmitters” and “brain’s reward center”). While this does require some specialized knowledge to evaluate, it doesn't create a new type of evaluability issue. Instead, it falls squarely within the existing category of Synonymy/Paraphrasing, where the challenge is in determining if the paraphrased terms accurately represent the original technical language.
    \\ 
\midrule
\textbf{\textcolor{teal}{Non-ambiguous example: Product Availability}}
\newline
\textbf{Source document excerpt:}
The company announced its new product line on March 15, 2023. CEO Jane Smith stated, “We expect to begin shipping these products to customers within six months.”
\newline
\textbf{Summary sentence:}
The company’s new products are now available to customers.
\newline
\textbf{Explanation:}
An annotator might be inclined to classify this as an “Other evaluability issue” due to the temporal ambiguity created by the word “now” in the summary. They might argue that this creates a unique challenge in evaluation because the accuracy of the statement depends on when the summary was written or read. However, this is actually an example of an “Other meaning phenomenon” rather than a distinct evaluability issue. The challenge here stems from the context-dependent meaning of “now,” which is a semantic issue related to deixis (words whose meaning depends on the context of utterance). This fits within the existing category of meaning-related phenomena and doesn't constitute a new type of evaluability issue outside the current taxonomy.
    \\ 
 \bottomrule
\end{tabular}
\caption{Examples for other ambiguities: Other evaluability issue}
\label{tab:other_example}
\end{table*}