\begin{table*}
\centering
\small
\begin{tabular}{@{}p{14cm}@{}}
\toprule
\textbf{\textcolor{red}{Ambiguous example: Library Funding}}
\newline
\textbf{Source document excerpt:}
A recent city council meeting addressed two separate issues. First, the council discussed plans to increase funding for public libraries by 10\% in the next fiscal year. In the next agenda item, they debated a proposal to extend park hours during summer months. The library funding increase was approved unanimously, while the park hours extension was tabled for further discussion due to concerns about increased maintenance costs.
\newline
\textbf{Summary sentence:}
The city council approved a measure to enhance public library access, extending their hours.
\newline
\textbf{Explanation:}
This is a positive example of Conflation because the summary sentence incorrectly combines two separate pieces of information from the source document. While the council did approve increased funding for libraries, there was no mention of extending library hours. The idea of extending hours was actually related to parks, not libraries, and that proposal was tabled, not approved. By conflating the approved library funding with the unapproved (and unrelated) extension of park hours, and then misapplying this to libraries, the summary creates a statement that is difficult to evaluate against the source document. This conflation of distinct issues and their outcomes results in a summary that misrepresents the council’s actions, making it challenging to assess its factuality. The synthesis of these separate pieces of information, along with the misattribution of the hours extension, creates a clear evaluability issue.
\\
\midrule
\textbf{\textcolor{teal}{Non-ambiguous example: Climate Change Effects}}
\newline
\textbf{Source document excerpt:}
A new study on climate change impacts has found that average global temperatures have risen by 1.1°C since pre-industrial times. The same research indicates that sea levels have risen by an average of 8 inches over the past century. The study’s authors emphasize that these two phenomena are interconnected, with rising temperatures contributing to thermal expansion of the oceans and melting of land-based ice.
\newline
\textbf{Summary sentence:}
Research shows that global warming has led to both higher temperatures and rising sea levels.
\newline
\textbf{Explanation:}
An annotator might be tempted to label this as Conflation because the summary sentence combines information about temperature rise and sea level increase into a single statement. However, this is not a true case of Conflation that creates an evaluability issue. The summary accurately represents the relationship between these phenomena as presented in the source document. The source explicitly states that these issues are interconnected, and the summary maintains this context. The synthesis of this information in the summary doesn't make it harder to evaluate its factuality against the source; rather, it provides a concise and accurate representation of the key findings and their relationship. Therefore, this example doesn't create an evaluability issue due to Conflation and shouldn't be labeled as such.
    \\ 
    \midrule
\textbf{\textcolor{teal}{Non-ambiguous example: Emissions Statement}}
\newline
\textbf{Source document excerpt:}
Speaker 0: Good evening, everyone. Today we’re joined by Dr. Emily Chen, a climate scientist, and Mr. John Davis, an energy policy expert. Speaker 1: Thank you for having me. Recent data shows that global carbon emissions have increased by 2\% in the past year, despite international efforts to reduce them. Speaker 2: That’s concerning. From a policy perspective, we need to incentivize a faster transition to renewable energy sources. Speaker 1: I agree. Our models predict that if this trend continues, we could see a 3°C rise in global temperatures by 2100. Speaker 2: That would have devastating consequences. We need to act now to prevent this.
\newline
\textbf{Summary sentence:}
John Davis stated that recent data shows a 2\% increase in global carbon emissions over the past year.
\newline
\textbf{Explanation:}
An annotator might be tempted to label this as Conflation because the summary sentence attributes a statement tot he wrong speaker, seemingly combining information from different parts of the conversation. However, this is not a true case of Conflation that creates an evaluability issue. While the summary incorrectly attributes Dr. Chen’s statement about carbon emissions to Mr. Davis, this misattribution doesn't make the sentence inherently harder to evaluate for factuality. An evaluator can easily check the source document to see who actually made the statement about the 2\% increase in emissions. The content of the statement itself is accurately represented; it’s only the speaker attribution that’s incorrect. This type of error is more akin to a simple factual mistake rather than a conflation of information that creates an evaluability issue.
Moreover, it’s important to note that the Conflation label should be reserved for cases in which ideas, concepts, topics, or other substantive content are merged or synthesized in a way that creates evaluability issues. It should not be applied to lower-level misattributions such as incorrect speaker identification or other similar factual errors. In this case, the core information and ideas presented by the speakers remain distinct and are not conflated; only the attribution of who said what is incorrect. Therefore, this example shouldn't’t be labeled as Conflation or as having any other evaluability issue.
    \\
 \bottomrule
\end{tabular}
\caption{Examples for context phenomena: Conflation}
\label{tab:conflation_example}
\end{table*}