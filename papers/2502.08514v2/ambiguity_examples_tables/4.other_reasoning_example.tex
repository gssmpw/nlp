\begin{table*}
\centering
\small
\begin{tabular}{@{}p{14cm}@{}}
\toprule
\textbf{\textcolor{red}{Ambiguous example: Investor Sentiment}}
\newline
\textbf{Source document excerpt:}
In the quarterly earnings call, CEO Jane Smith projected a 15\% revenue growth for the next fiscal year. She cited strong product performance and expansion into new markets as key drivers for this optimistic forecast. The company’s stock price saw a modest 2\% increase following the announcement.
\newline
\textbf{Summary sentence:}
Despite the CEO’s optimistic forecast, investors remain cautious about the company’s future performance.
\newline
\textbf{Explanation:}
This is a positive example of an Other implicit reasoning phenomenon. The summary sentence makes a claim about investor sentiment that isn’t explicitly stated in the source document. This inference appears to be based on a complex synthesis of information and experience that goes beyond simple deduction or common-sense reasoning. The summarizer seems to have considered factors such as past experiences with CEO projections, general market conditions, the company’s  recent performance history, and the tendency of executives to present optimistic forecasts. This type of pattern recognition and experiential reasoning creates an evaluability issue because it relies on implicit knowledge and interpretation that isn’t directly traceable to the source document. An evaluator would find it challenging to assess the factuality of the claim about investor caution based solely on the information provided in the source.
\\
\midrule
\textbf{\textcolor{teal}{Non-ambiguous example: Market Challenges}}
\newline
\textbf{Source document excerpt:}
TechCorp announced the release of its new smartphone, featuring holographic display technology.The company plans to launch the product in major markets next quarter.
\newline
\textbf{Summary sentence:}
The company’s new product launch is likely to face significant challenges in the market.
\newline
\textbf{Explanation:}
This is a negative example for the Other implicit reasoning phenomenon. While the summary sentence does involve an inference not explicitly stated in the source, it doesn’t require complex market analysis or specialized industry expertise. An annotator might mistakenly think this involves sophisticated prediction about market dynamics. However, this is actually a case of common-sense inference. The summarizer is likely basing their conclusion on the general notion that new and unusual technologies often face challenges when first introduced to the market. This is a straightforward observation based on general experience, not a complex reasoning process. Therefore, this example should be labeled as a common-sense inference rather than an Other implicit reasoning phenomenon.
\\
\midrule
\textbf{\textcolor{red}{Negative example: Recycling Reception}}
\newline
\textbf{Source document excerpt:}
The city launched a mandatory recycling program requiring residents to separate their waste into three categories: recyclables, compostables, and landfill waste. In a survey conducted by the city, 45\% of residents expressed support for the program, while 40\% opposed it, and 15\% were undecided.
\newline
\textbf{Summary sentence:}
The city’s new recycling program has been met with mixed reactions from residents.
\newline
\textbf{Explanation:}
This is a negative example for the Other implicit reasoning phenomenon. An annotator might initially think this summary involves complex analysis of public opinion or implicit knowledge of local attitudes. However, the summary sentence is actually a straightforward interpretation of explicit information provided in the source document. The survey results directly support the claim of “mixed reactions” without requiring any significant inference or implicit reasoning. This summary doesn’t create an evaluability issue because it’s directly supported by the source material. Therefore, it should not be labeled as having any evaluability issue, let alone an Other implicit reasoning phenomenon.
    \\ 
 \bottomrule
\end{tabular}
\caption{Examples for implicit reasoning phenomena: other}
\label{tab:other_reasoning_example}
\end{table*}