\begin{table*}
\centering
\small
\begin{tabular}{@{}p{14cm}@{}}
\toprule
\textbf{\textcolor{red}{Ambiguous example: Park Equipment}}
\newline
\textbf{Source document excerpt:}
The city council approved funding for new playground equipment in several local parks.
\newline
\textbf{Summary sentence:}
The council has allocated money for new swings and slides in the city’s parks.
\newline
\textbf{Explanation:}
This is a positive example of Hyponymy/Specialization because the summary sentence uses more specific terms(“swings and slides”) than the general “playground equipment” mentioned in the source document. This specialization makes it challenging to evaluate the factual accuracy of the summary, as the specific types of equipment were not mentioned in the original text.
\\
\midrule
\textbf{\textcolor{teal}{Non-ambiguous example: Apple Revenue}}
\newline
\textbf{Source document excerpt:}
Apple’s annual report showed a 15\% increase in revenue from its smartphone division.
\newline
\textbf{Summary sentence:}
Apple’s latest financial statement reveals significant growth in iPhone sales.
\newline
\textbf{Explanation:}
An annotator might consider marking this as a Hyponymy/Specialization issue because “iPhone” is indeed a more specific term than “smartphone.” However, this is not a significant enough case to cause an evaluability issue. While it’s technically a hyponym, the relationship between “iPhone” and “Apple’s smartphone” is so well-known that it doesn’t meaningfully impact the ability to evaluate the factual accuracy of the summary. The vast majority of people know that the iPhone is Apple’s only smartphone, so this specialization doesn’t introduce any real ambiguity or difficulty in assessing the statement’s accuracy.
    \\ 
 \bottomrule
\end{tabular}
\caption{Examples for meaning phenomena: Hyponymy/Specialization}
\label{tab:hyponymy_example}
\end{table*}