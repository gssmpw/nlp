\begin{table*}
\centering
\small
\begin{tabular}{@{}p{14cm}@{}}
\toprule
\textbf{\textcolor{red}{Ambiguous example: Dog Popularity}}
\newline
\textbf{Source document excerpt:}
The study found that golden retrievers and labrador retrievers were the most popular dog breeds in the United States last year.
\newline
\textbf{Summary sentence:}
Retriever breeds were the most popular dogs in the U.S. in the previous year.
\newline
\textbf{Explanation:}
This is a positive example for Hypernymy/Generalization because the summary uses the more general term“retriever breeds” instead of the specific breeds mentioned in the source document. This generalization makes it slightly more challenging to evaluate the factual accuracy of the summary, as it doesn’t preserve the exact level of detail from the source.
\\
\midrule
\textbf{\textcolor{teal}{Non-ambiguous example: Emissions Policy}}
\newline
\textbf{Source document excerpt:}
The new environmental policy aims to reduce carbon emissions from factories by 30\% over the next decade.
\newline
\textbf{Summary sentence:}
The recently introduced policy targets a significant decrease in industrial pollution in the coming years.
\newline
\textbf{Explanation:}
An annotator might mistakenly identify this as a Hypernymy/Generalization issue because “industrial pollution” is a broader term than “carbon emissions from factories.” However, this is not an accurate example of the evaluability issue. The summary introduces new information (“significant decrease” and “coming years”) that isn’t directly generalizing from the source. The challenge in evaluating this summary comes more from paraphrasing and potential exaggeration rather than from using a hypernym or more general meaning.
    \\ 
 \bottomrule
\end{tabular}
\caption{Examples for meaning phenomena: Hypernymy/Generalization}
\label{tab:hypernymy_example}
\end{table*}