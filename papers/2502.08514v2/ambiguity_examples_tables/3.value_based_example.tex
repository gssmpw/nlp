\begin{table*}
\centering
\small
\begin{tabular}{@{}p{14cm}@{}}
\toprule
\textbf{\textcolor{red}{Ambiguous example: CEO Conduct}}
\newline
\textbf{Source document excerpt:}
During his keynote speech at the industry conference, CEO John Smith of Tech Innovations Inc.referred to rival company XYZ’s new smartphone as ‘a glorified paperweight’ and mimicked throwing it into a trash can. His remarks drew gasps from some attendees, while others chuckled uncomfortably.
\newline
\textbf{Summary sentence:}
The CEO’s public ridicule of a competitor’s product at the industry conference was widely viewed as unprofessional and damaging to the company’s reputation.
\newline
\textbf{Explanation:}
This is a positive example of a Value-based inference evaluability issue. The summary sentence makes claims about how the CEO’s actions were perceived, which are not explicitly stated in the source document. This inference appears to be based on the factual information from the source (the CEO’s mocking remarks) and a value-based premise that professional conduct in business, especially for high-level executives, should involve treating competitors with respect and focusing on one’s own products rather than denigrating others’. This commonly-held value is not stated in the source but is crucial to the inference.The reliance on this value-based premise creates an evaluability issue because it requires the evaluator to share or recognize this value to assess the accuracy of the summary’s claims about how the CEO’s actions were perceived.
\\
\midrule
\textbf{\textcolor{teal}{Non-ambiguous example: Pothole Repairs}}
\newline
\textbf{Source document excerpt:}
At yesterday’s city council meeting, members unanimously approved a \$500,000 budget to fix potholes on Main Street over the next month. The decision came after numerous complaints from residents about the road’s condition.
\newline
\textbf{Summary sentence:}
The city’s decision to repair potholes on Main Street will improve road safety for drivers.
\newline
\textbf{Explanation:}
This is a negative example for the Value-based inference evaluability issue. While the summary sentence does involve an inference not explicitly stated in the source, it’s not primarily based on a value judgment. An annotator might mistakenly think this involves a value-based premise like “Safety is important and should be maintained in public spaces.” However, the inference is more accurately based on the common-sense notion that “Pothole repairs make roads safer.” This is not a moral or ethical value, but rather a widely accepted understanding of road maintenance effects. Therefore, this should be considered a common-sense inference rather than a value-based one. It doesn’t create the kind of evaluability challenge associated with value-based inferences, where personal or societal values significantly influence the interpretation of facts.
\\
\midrule
\textbf{\textcolor{teal}{Non-ambiguous example: Phone Upgrade}}
\newline
\textbf{Source document excerpt:}
The upcoming smartphone from TechCorp features a processor that is 20\% faster, a camera with 5 megapixels more resolution, and a battery that lasts 2 hours longer than the previous model. The company plans to release the new model next month.
\newline
\textbf{Summary sentence:}
The new smartphone model is expected to be a significant upgrade from its predecessor.
\newline
\textbf{Explanation:}
This is another negative example for the Value-based inference evaluability issue. An annotator might mistakenly label this as a value-based inference, thinking that “significant upgrade” involves a value judgment. However, this inference is based on commonly accepted standards in the tech industry rather than moral, ethical, or personal values. The judgment that faster processors, better cameras, and longer battery life constitute an upgrade is a technical evaluation, not a value-based one. This should be considered a common-sense inference within the context of technology advancements. The example highlights the importance of distinguishing between technical or industry-standard evaluations and true value judgments based on moral,ethical, or societal values.
    \\ 
 \bottomrule
\end{tabular}
\caption{Examples for implicit reasoning phenomena: value-based inference.}
\label{tab:value_based_example}
\end{table*}