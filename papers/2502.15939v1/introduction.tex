\section{INTRODUCTION}

There has been significant interest in the potential for LLMs in healthcare---to support health information seeking, data summarization, delivery of care services, and more ~\cite{singhal2023large, shahsavar2023role, zhou2023survey}. Several of these efforts focus on leveraging this technology to improve healthcare delivery, including in underserved settings in the Global South ~\cite{kimblack2022, tsengunder, 10.1007/978-3-030-60114-0_10}.
Most LLMs, however, are primarily trained on data and literature in English from the Global North, and inherit racial, gender, and cultural biases persistent in these databases ~\cite{8418155, denecke2021evaluation}. 
They can fail to recognize local dialects, cultural nuances, and speaking patterns, especially for populations that are less represented online and do not speak English, furthering the technology access gap ~\cite{adilazuarda2024measuringmodelingculturellms, ondrejova2024can, li2024culturellm}. 
Several ongoing efforts have focused on both adapting existing models and developing entirely new language models (including more efficient and sustainable ones) to improve performance in non-English languages ~\cite{10.1007/978-3-030-60114-0_10,  Ożegalska-Łukasik_Łukasik_2023,10.1145/3639856.3639890}. Our paper aims to contribute to this landscape by informing the design of \textit{culturally sensitive} LLMs in healthcare, by considering the design of a chatbot in an underserved setting in India.

% This includes several ongoing efforts on the use of LLMs reproductive health \cite{}. 
% and chatbots are designed to cater to the medical systems in the west, they render useless when evaluated in different contexts. 

Our interest in the role of culture \chiadd{in LLM-driven text generation} is guided by its close relationship to language and \chiadd{health} communication; how it both shapes and is shaped by it.
% \textbf{}
\chiadd{LLMs frequently struggle to accurately interpret health-related queries for users from diverse backgrounds, partly due to insufficient training data ~\cite{banerjee2024navigating}. This can lead to misunderstanding culturally specific health practices, which we hope to address through our work. } % Additionally, LLMs face persistent challenges with question awareness, resulting in inappropriate responses to medical queries \cite{yang2024llms}.
The term ``culture'' has a long history of being studied in relation to technology use within the field of Human-Computer Interaction (HCI), though it has rarely been defined concretely (e.g. ~\cite{ge2024culture,clemmensen2010overview,lofstrom2010culture}). 
% It has been used as a catch-all phrase to describe research done with non-WEIRD (Western and mostly white, Educated, Industrial, Rich, Democracies) populations, and 
It has frequently been used interchangeably with local context, community, nationality, religion, as well as race and ethnicity---\chiadd{and to distinguish from most research with white, privileged, or university-educated populations}. 
In this paper, \chiadd{we draw on a rich body of work in health communication literature on the role of culture}. We define culture as \textit{``shared values, norms, codes, roles, and assumptions that shape a group's beliefs, attitudes, and behavior through their interactions in and with their environments''} ~\cite{griffith2024cultural}.
\chiadd{Building on this understanding, we rely on Resnicow et al.'s definition of cultural sensitivity as \textit{``the extent to which ethnic, cultural, and other factors are incorporated in the design, delivery, and evaluation of health communication, health promotion materials, and health promotion programs''}~\cite{resnicow1999cultural}. However, LLMs offer unique challenges to design in comparison to more traditional group and individual-level health messaging, which we discuss in this paper.}

We focus on Sexual and Reproductive Health (SRH), a topic that has been highly taboo in contexts around the world \cite{giritharan2020socio,likith2024exploring}, and that surfaces many of the tensions around culture in the development of LLM-based healthcare interventions. 
\RD{SRH has been deeply shaped by cultural norms---reflected in normative pressures, media discourse, and spousal dynamics \cite{feriani2024systematic}. 
%At times restricting open conversations and access to accurate information, especially among women \cite{giritharan2020socio,likith2024exploring}. 
Stigmas surrounding ``cleanliness'' during menstruation and sexual health behaviors can lead to social exclusion and discrimination \cite{espinosa2019breaking}, while harmful practices like virginity testing in some communities may reinforce restrictive societal expectations \cite{robatjazi2015virginity}. The fear of judgment and shame can prevent individuals from seeking medical care for sexually transmitted infections (STIs) or unplanned pregnancies \cite{cook2014reducing}, impacting both physical and mental health.} 
%Moreover, reproductive and contraceptive choices are influenced by normative pressures, media discourse, and spousal dynamics \cite{feriani2024systematic}.}
\chiadd{Finally, colloquial language may be used locally to refer to SRH concerns instead of medical terms in English.} \RD{LLMs risk magnifying these barriers if they fail to account for the culturally specific language people use to navigate stigma.}

We partnered with a Non-Governmental Organization (NGO) called Myna Mahila Foundation (Myna) based in Mumbai (India). We studied the design and use of an LLM-based chatbot that Myna had developed for providing information and services on SRH. \chiadd{Myna used a multilingual LLM (preview model of GPT-4) that offers support for Hinglish (transliterated Hindi, \textit{i.e.} Hindi written with the Roman script, with code-mixing between Hindi and English). Based on initial testing with community members, Hinglish was identified as the most preferred language for chat interactions and was used in the final chatbot.}
The chatbot was extensively tested with a largely Hindi-speaking, but culturally and religiously diverse group of women \chirm{workers} in an underserved migrant urban community in Mumbai, within Myna's outreach area.
We \chiadd{primarily} studied 2118 message logs (question-answer pairs) of the \chirm{workers'} \chiadd{women's} interactions with the chatbot, analyzing both the types of questions asked and the chatbot's effectiveness in responding to them. We also supplemented this with data from two focus groups with the\chirm{workers} \chiadd{women}, and focus groups, interviews, and online WhatsApp and Slack discussions with healthcare professionals, and technology developers at Myna to understand design decisions that were made to improve the cultural sensitivity of the chatbot. 
% We analyzed this data to consider the strengths and limitations of the chatbot in delivering culturally appropriate health advice.
We paid special attention to the role of stigma and taboos, linguistic context, nuances around social norms, and the role of family dynamics. 
% Our research thus aims to offer a perspective on the design of LLM-based chatbots that meet the needs of a community.

Our paper is structured as follows. We begin by presenting literature on LLMs and chatbots in healthcare and on culturally sensitive design. This is followed by details about our study context and methods, and the design of the chatbot. 
Our findings reveal the strengths and limitations of the system in capturing local context and complexities around
what constitutes ``culture''.
We also identify the elements of culture that emerged as being relevant and highly critical to address in an LLM-based intervention. 
We then discuss the implications of these findings for the future development of personalized LLM-driven tools in public health in diverse settings.
Finally, we consider how we might integrate the local context into chatbots, and present a framework to inform the design of culturally-sensitive LLMs for community health.
