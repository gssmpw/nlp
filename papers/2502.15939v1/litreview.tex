\section{RELATED WORK}
Below we present literature on LLMs and chatbots in healthcare, design principles for chatbots, and research on cultural sensitivity in HCI and with LLMs. 

\subsection{Information-Seeking and Chatbots in Healthcare}
\chiadd{HCI literature has well-established that information seeking and processing are deeply influenced by people's daily lives and their social contexts--both in terms of how one finds and searches for accurate or verifiable information, and how they process and apply it \cite{savolainen1995everyday,savolainen2022assessing}. 
In high-stakes environments such as healthcare, getting access to timely and accurate information becomes even more critical. HCI has long documented how people make sense of health information \cite{raj2019clinical,young2019girl,nakikj2023alerts}, such as through social media \cite{mcdowall2024sensemaking}. 
% X \cite{}. 
For underserved communities specifically, information gaps are amplified by systemic barriers like limited digital literacy \cite{arias2023digital, ehrari2022digital}, distrust in automated systems \cite{ehrari2022digital}, contradictory nature of certain cultural and healthcare practices \cite{azongo2014complementary, craffert1997opposing},  and general difficulties in navigating through existing healthcare systems \cite{pervez2023systemic, ahmed2001barriers}. Prior work by Yadav et al. \RD{ and Wang et al.} has aimed to bridge this gap using chatbots designed to meet the needs of breastfeeding women and adolescent sex education while keeping in mind these constraints \cite{yadav_feedpal_2019, wang2022artificial}.}  
\RD{By incorporating local beliefs and considering social dynamics like family influence \cite{yadav_feedpal_2019, wang2022artificial}, they aim to provide culturally congruent health education.
%Their chatbot Feedpal, designed to serve as the first point of contact for breastfeeding mothers \cite{yadav_feedpal_2019}, and SnehAI,  designed for SRH awareness among youth \cite{wang2022artificial}, adapted their responses to align with local beliefs. By incorporating culturally contextualized responses and considering social dynamics such as family influence their approach highlights how chatbots can be adapted for high-impact, community-driven health education. 
%Yadav et al. emphasize the significance of culturally aware chatbot design in healthcare. 
}
 
% These patterns become more nuanced as one seeks information about more sensitive subjects such as health. 
% These studies further strengthen our proposal of incorporating sociocultural contexts into LLM-based chatbots to account for diverse information-seeking patterns across different communities. 
% Incorporation of such contextualized healthcare information-seeking patterns \cite{savolainen1995everyday, savolainen2022assessing} onto LLMs thus becomes very important in providing care and help. 

Recent studies in HCI and health informatics show a growing interest in utilizing chatbots in configurable healthcare interventions to improve patient engagement due to their anthropomorphic communication styles~\cite{singhal2023large, shahsavar2023role, zhou2023survey, sallam2023chatgpt, parmar2022health}. Usage of LLM-based chatbots could possibly mark a new era of digital patient engagement and care delivery ~\cite{ruggiano2021chatbots, shahsavar2023role}. The use of LLM-based chatbots for healthcare started with a lot of attention on aiding conversations around mental health ~\cite{song2024typing, althoff2016large, yang2023towards,10.1145/3613904.3642761}, virtual caregiving ~\cite{ruggiano2021chatbots, miura2022assisting, wang2021evaluation} and personalized health information delivery ~\cite{harrington2023trust, almalki2020health, skjuve2018chatbots}. LLMs have been particularly beneficial for these applications, as they can provide desired information while allowing for open-ended questions and conversations ~\cite{jovanovic2020chatbots, holmes2019usability, dolianiti2020chatbots,10.1145/3544548.3581503}. Their multilingual and personalization capabilities ~\cite{kocaballi2019personalization, liu2022effects, ait2023power,10.1145/3613905.3651093} also allow tailoring of interactions to the individual linguistic background and their health literacy levels ~\cite{marzo2024bridging}. Studies have shown that these models, when paired with factual clinical knowledge, have the potential to offer contextually relevant information in response to user's queries, making them suitable for complex healthcare interactions. 

The rapid advancement of LLMs in healthcare requires frameworks to address ethical concerns ~\cite{song2024typing, sepahpour2020ethical, chen2023chatbots, 10.1145/3613904.3642385}.
Major issues include ensuring clinically-accurate medical responses ~\cite{abbasian2024foundation, goodman2023accuracy}, privacy concerns ~\cite{li2023security, kanter2023health} and potential biases ~\cite{kim2023assessing}. In response, the World Health Organization has released ethics and governance guidelines for the safe use of LLMs in healthcare~\cite{whoguidelines}. Though these chatbots might be able to help people who have no other options at their disposal ~\cite{song2024typing}, evidence suggests that they have notable risks~\cite{song2024typing, sepahpour2020ethical}. Hence, there has been a growing research interest around the design ~\cite{wilson2022development, tsengunder, griffin2023chatbot} and evaluation of chatbots in healthcare ~\cite{cascella2023evaluating, denecke2021evaluation}, and documentation of ethical challenges that they may pose ~\cite{sepahpour2020ethical, li2023security}. 

\subsection{Design of Chatbots}
The design of healthcare chatbots prioritizes usage of high-quality clinical knowledge ~\cite{singhal2023large, denecke2021evaluation}, security, safety ~\cite{li2023security, abbasian2024foundation} and accurate comprehension of the user's query. The HCI community has placed a lot of emphasis on the importance of appropriate user experience ~\cite{hornbaek2017technology, hassenzahl2006user}, by assessment and incorporation of emotional intelligence ~\cite{bilquise2022emotionally, ghandeharioun2019towards, pamungkas2019emotionally} and adaptability ~\cite{nadarzynski2023but} to these chatbots. These factors are crucial to the design of the bot given the risks involved in healthcare interactions ~\cite{sepahpour2020ethical, li2023security} and the empathy required ~\cite{alam2022examining, seitz2024artificial, seitz2021empathic}. As with other HCI research, there is a strong emphasis on user-centered design ~\cite{abras2004user, soloway1994learner}. One of the primary challenges is ensuring clinical accuracy while maintaining conversational fluency based on the user’s linguistic context ~\cite{chavesling, spilnertalk2021} and health information literacy ~\cite{abreu2024utilization}. This not only involves design modifications during development but also tailoring the medical knowledge base ~\cite{8418155, chung2019chatbot}, incorporating real-world scenarios and patient-doctor dialogues ~\cite{chaix2019chatbots, chung2019chatbot} to capture the nuances of medical communication. 

The user experience is also designed to encourage user’s trust, with incorporation of empathy ~\cite{seitz2021empathic, seitz2024artificial,10.1145/3613904.3642336}, accessibility ~\cite{habicht2024closing, stanley2022chatbot} and clarity.  A persona is provided to the bot based on the use case and the user’s context, and personalization of responses is promoted~\cite{nissen2022effects, hwang2021applying}. Studies have shown that users are more likely to engage with and trust bots that present a professional appearance while maintaining a warm, empathetic tone ~\cite{seitz2021empathic, seitz2024artificial}. 
\RD{Yadav et al. and Wang et al. emphasized the significance of culturally aware chatbot design in fostering comfort, trust, and cultural relevance ~\cite{yadav_feedpal_2019, wang2022artificial}. SnehAI featured a female persona based on a popular television character and adopted a narrative approach aligning with the Indian context \cite{wang2022artificial}, whereas FeedPal modeled its female persona after trusted maternal health workers, to answer user questions from the knowledge base \cite{yadav_feedpal_2019}.}

%Acknowledging that women may be reluctant to discuss maternal and sexual health due to stigma, the significance of culturally aware chatbot design in healthcare was emphasized \cite{wang2022artificial, yadav_feedpal_2019}. To foster comfort, trust, and cultural relevance, ensuring women felt at ease engaging in discussions around sensitive topics---a female persona modeled after trusted maternal health workers, Accredited Social Health Activists (ASHAs), and a persona based on popular television character that takes a narrative approach aligning with the indian context were featured \cite{wang2022artificial, yadav_feedpal_2019}.}

LLM-based chatbots are pivotal for the level of personalization they provide, which has historically not been possible ~\cite{kocaballi2019personalization}. They leverage the user's profile, interaction history and their health information to provide medical information that is the most appropriate for that user~\cite{openai2024gpt4technicalreport}. Personalization also helps in tailoring the interactions based on the user's conversation style ~\cite{kocaballi2019personalization}, preferences, and their health literacy level, hence adapting to their tone and approach. Explainability is also promoted and utmost importance is given to addressing the risks and ethical concerns~\cite{ai4030034, shuchaoreview}. Building on top of personalization and explainability, there is work on incorporating the user’s local and cultural context ~\cite{adilazuarda2024measuringmodelingculturellms, 10.1145/1878450.1878481, Ożegalska-Łukasik_Łukasik_2023}. This becomes even more important when working with underserved communities ~\cite{harrington2023trust, kimblack2022, tsengunder}. 

\subsection{Cultural Sensitivity}
\chiadd{In health communication literature, Resnicow et al. define cultural sensitivity as \textit{``the extent to which ethnic, cultural, and other factors are incorporated in the design, delivery, and evaluation of health communication, health promotion materials, and health promotion programs''} \cite{resnicow1999cultural}. They go on to highlight two primary dimensions of cultural sensitivity---surface structure and deep structure \cite{resnicow1999cultural} \textit{Surface structure} focuses on the inclusion of visible and identifiable elements of a specific population, such as language, imagery, music, food, clothing, and other cultural symbols that resonate with the group.
In contrast, \textit{deep structure} delves into the underlying cultural factors that shape behaviors, including values, beliefs, norms, and stressors. For example, it involves tailoring content to reflect core cultural values like family commitment, spirituality, or respect for elders, as well as addressing unique stressors such as discrimination or racism. By incorporating these deeper cultural drivers, health messages and interventions can more effectively influence behaviors and outcomes \cite{resnicow1999cultural}. While these theoretical understandings of cultural sensitivity pre-date the use of LLMs (which raise a unique set of challenges), they offer us a starting point to expand and build a framework for integrating cultural sensitivity in LLM-based interventions in healthcare settings.}

%\RD{Yadav et al. and Wang et al.’s work closely align with these principles. They accounted for surface structure by ensuring that their chatbots, supported code-mix and voice input, enhancing accessibility for users with diverse literacy skills. More significantly, their approach addressed deep structure by considering social influences on breastfeeding practices, particularly the role of mothers-in-law in shaping maternal health decisions, and tackled social stigma surrounding SRH discussions. Instead of opposing traditional beliefs, both employed affirmative interaction; however, their implementation differed. Feedpal used structured guidance and gentle cues, while SnehAI relied on narrative-driven engagement to counter misinformation with medically validated responses by identifying cultural myths. Despite their strengths, these approaches have restricted adaptability, pointing to the need for further research into chatbot models that can flexibly and contextually respond to cultural sensitivities.}

The capability of LLMs to appropriately respond to the cultural context of the user is a crucial aspect of personalization needed for healthcare information seeking ~\cite{adilazuarda2024measuringmodelingculturellms, 10.1007/978-3-030-60114-0_10, Ożegalska-Łukasik_Łukasik_2023}. In particular, communities that are underserved and less represented online may have less trust in AI and the healthcare system ~\cite{harrington2023trust, kimblack2022, tsengunder}. Taking into account linguistic diversity and responding to the user’s query in their preferred language is also critical. But being sensitive to the culture goes beyond the linguistic translation ~\cite{orosoo2024enhancing, heim2013}. It requires a deep understanding of cultural differences, their local dialects, nuances, behavioral patterns, and their sociocultural context ~\cite{10.1007/978-3-030-60114-0_10}. Despite advancements in current LLMs, they often struggle with cultural misunderstandings given that the training data is predominantly in English from Western contexts ~\cite{tsengunder, 10.1145/1878450.1878481, griffith2024cultural}. Recent work by Adilazuarda et \textit{al.} has defined proxies of culture, which can be mapped to certain measurable aspects of LLM’s responses and language ~\cite{adilazuarda2024measuringmodelingculturellms}. Based on modeling and linguistic analysis, they give guidelines and techniques to improve cultural and linguistic sensitivity ~\cite{griffith2024cultural, 10.1007/978-3-030-60114-0_10, article}. 

% \chiadd{While there are various challenges and misalignments in usage of LLMs in these contexts, there are some specific to healthcare information access \cite{yang2023large} which require additional focus. There are differences in how health conditions are understood and described across cultures \cite{hogg2010cultural, bakic2012conceptual}, hence designing these system requires varying the preferences of array of various qualitative parameters \cite{bakic2012conceptual, bakic2018communicating} like directness \cite{singleton2009understanding, bakic2018communicating}, expectations \cite{singleton2009understanding, bakic2018communicating} , level of information \cite{singleton2009understanding, bakic2018communicating}, the tone of interactions \cite{singleton2009understanding, bakic2018communicating}, etc.}



Within the HCI community, researchers have defined cultural sensitivity as the extent to which ethnic, cultural, and local contexts are incorporated in the design, delivery, and evaluation of these chatbots ~\cite{adilazuarda2024measuringmodelingculturellms, shi2024culturebank, li2024culture}. This may entail aligning with different communication styles, e.g. directness, tone, dialect ~\cite{adilazuarda2024measuringmodelingculturellms, ondrejova2024can, li2024culturellm}, and adapting to what is preferred locally and culturally. It also includes recognizing cultural variations in family roles in healthcare, including gender norms ~\cite{anderson2003culturally, vaughn2009cultural,cerezo2023healthcare}, differing levels of healthcare access, and health and data literacies~\cite{almalki2020health, 10.1007/978-3-030-60114-0_10}. 
% Recognizing cultural barriers to healthcare becomes crucial, especially if the goal is to improve access. 
% Based on all of these considerations, user research is done with direct user involvement throughout the process ~\cite{abras2004user, soloway1994learner}. 
Based on this understanding, appropriate interface and interaction design can be carried out, with localization and accessibility considerations ~\cite{seitz2021empathic, habicht2024closing}. 
These cultural contexts are kept at the forefront during evaluation ~\cite{10.1007/978-3-030-60114-0_10, adilazuarda2024measuringmodelingculturellms}. 

%Despite advancements in technology, existing AI systems often struggle with cultural misunderstandings due to exclusion of language diversity in the training dataset since the majority of the LLMs are trained in English. Accurate interpretation of user queries requires not only linguistic translation but also a deep understanding of cultural differences that influence interaction patterns. Researchers proposed the inclusion of various techniques such as comprehensive linguistic analysis, cultural context modeling, and user feedback mechanisms to achieve better performance excelling in both cultural sensitivity and linguistic accuracy by reducing potential biases in training data, increasing the adaptability to low-resource languages, and the need for continuous development based on user interactions. Study showed that users find the augmented translated text to be culturally sensitive and provide better interaction experiences to the user.

%\subsection{HCI and Cultural Sensitivity }
%Researchers defined cultural sensitivity as the extent to which ethnic, cultural, and other factors are incorporated in the design, delivery, and evaluation of health communication. In healthcare applications, cultural sensitivity in LLMs-based chatbots plays a critical role, as it helps bridge the gap between patient needs and AI-provided information. This sensitivity ensures that diverse populations are catered to appropriately, with tailored, relevant healthcare advice. By addressing cultural biases and adapting to different languages, chatbots can provide more accurate and culturally appropriate responses, ultimately improving the quality of healthcare interactions.