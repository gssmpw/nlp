\section{METHODS}
Our study aimed to understand the cultural appropriateness of an LLM-based chatbot designed to provide information on sexual and reproductive health (SRH). 
% to women in urban India, and identify elements that constitute ``culture'' in the local context. 
We analyzed log data from \chiadd{women from an underserved community in Mumbai who were recruited by} \chirm{workers at}Myna to test the application. We also conducted focus groups with them\chirm{workers}, and interviews and focus groups with developers, healthcare professionals, and the program team at Myna. Below, we detail the \chiadd{background of the women from the community,} participant recruitment process, data collection, and analysis methods. We obtained approval to conduct the research from the Institutional Review Board at Emory University in the United States.

\begin{figure*}[h]
  \centering
\includegraphics[width=1\textwidth]{phases.png} 
  \caption{\textbf{Various phases of the study}.}
  \label{fig:methods}
  \Description{Six different phases of our research study. Starting with Recruiting Women from the Community Via door-door surveys, Training Women from the community to understand SRH topics, Knowledge Base Development using queries gathered from women, Chatbot Development followed by Iterative testing of the chatbot with community women, and finally Evaluation and Improvements.}
\end{figure*}

\subsection{Study Context and Community Engagement}
We were introduced to women from the community, technology developers, and healthcare professionals through Myna, a non-profit working on sexual and reproductive health in Mumbai, India, since 2016.
\chiadd{The women who participated in testing the chatbot in our study were initially employed by Myna as part of a micro-tasking employment program called Rani Jobs. Despite several of them having previously engaged with Myna through their menstrual health programs, they were initially recruited via door-to-door surveys by Myna's program team to ensure a diverse group of participants representing a broad spectrum of SRH concerns and language use. Many of them had little to no prior knowledge about SRH topics and chatbots when recruited. As a representative sample, the women were well-positioned to understand and articulate SRH issues women face in their community. }

\chiadd{Prioritizing a community-centric approach, Myna engaged the women across various phases of the chatbot development process, including the pre-development phase, iterative testing, and evaluation. 
Figure \ref{fig:methods} outlines all the phases and the community's involvement.
The women were recruited even before the chatbot was developed and were paid to create a question bank to help potentially fine-tune the LLM (an approach that was later abandoned in favor of pre-trained models with RAG, detailed in Section 4).  
They were called to Myna's local centers for digital training to learn to use a spreadsheet to record their questions, and to build awareness on SRH topics such as family planning, pregnancy, infertility, and menstrual health management to help with question generation.
Follow-up sessions were set up at Myna's center after the chatbot was developed to onboard them to the platform.}
% Peer-to-peer learning and interactive sessions with health experts was used to create a safe and supportive environment for discussing such sensitive topics. 
% , to provide more context into who the women were, their specific roles in each phase, and how their active participation was key in shaping the study’s outcomes. However, the chatbot interactions presented in this paper focus on the data collected during the testing and evaluation phase. Informed consent was obtained from all participants, and confidentiality was strictly maintained to protect their identities and personal information. The women were compensated for their contribution, and treated by Myna as paid contract workers.}
%The study adopted a deliberate, iterative approach, with input from women at every phase playing an important role in shaping the dvelopemnt process. %we will not be including details about the pre-development and development phase of the knowledge base and chatbot.



\subsection{Data Collection}
\chirm{The recruitment strategy aimed to include a diverse group of women to reflect a range of SRH concerns and language use. The community women were recruited for this study via door-to-door surveys by the Myna team. Being members of the same community, they were capable of understanding the issues faced by women in these areas.}  
% \subsection{Data Collection}
\chiadd{As part of the testing phase, women from the community were first invited to digital centers where they engaged with the system under the supervision of the program team. The team monitored these interactions and assisted with prompt generation if needed. Observations made by the team during this phase were recorded for analysis.
% Participants showed a strong interest in asking questions and seeking information from the chatbot. They asked their family members and neighbors about their concerns and started prompting those questions in the interactions from their homes. 
The testing phase is iterative and ongoing, however, for the scope of our study, we specifically analyzed interactions with the chatbot over a period of three months, from late December 2023 to mid-March 2024.} 
All interactions during this period were logged in a database, including timestamps, questions asked, chatbot responses, conversation and message identifiers, and the language used. 
Additionally, feedback from women on various aspects such as response length, understandability, the difficulty of terms used in responses, and latency was collected by the Myna team to support improvements to the knowledge base.
% In the later phase, all the logged responses were reviewed and commented on by healthcare professionals for metrics such as medical accuracy, social context accuracy, interpretation of questions, breaking of misconceptions, and simple medical terminology. 
% Additionally, quantitative feedback from the women on various aspects such as response length, understandability, the difficulty of terms used in responses, and latency was collected by the Myna team to support improvements to the knowledge base.
\chiadd{After the chatbot testing, two focus groups were conducted with the women to understand their backgrounds and perspectives better.}
Our participants included a diverse group of women, in terms of age, religion, family status, and education. Their demographic data is presented in Table \ref{tab:demo}.

We also interviewed two developers, one of the healthcare professionals, and the program team virtually,
and recorded the conversations for further analysis. % on Google Meet 
The healthcare professional we interviewed is the Associate Director of the health vertical of Myna, and has been associated with the organization for six years. 
The healthcare professionals had varying qualifications; all of them had either a diploma certificate in gynecology and obstetrics or more than ten years of experience in the field of sexual and reproductive health while working with underserved communities. 
\chirm{They not only have professional medical training but have had experience working with underserved communities, are able to understand what interventions help those people, and understand the language and cultural taboos that exist among women in such communities.}
They were responsible for testing and evaluating the medical accuracy and contextual relevance of the chatbot's responses, \chiadd{leveraging their experience and understanding of SRH language and taboos}. The developers had experience working on AI applications, they were responsible for enhancing the chatbot in multiple phases. 
All the healthcare professionals were women, while the chatbot developers were men.
% We also conducted interviews and focus groups with healthcare professionals, Myna's program team, and technology developers.  
We also had access to WhatsApp and Slack chat communications with five healthcare professionals and two chatbot developers. 
They actively communicated on these channels throughout the processes of chatbot development, testing, evaluation, and improvement. These chats were also recorded for analysis. 
As we analyzed the chatbot interactions, the other data sources provided important context on complexities designing a culturally sensitive chatbot.


% We analyzed the log data from the interactions between these workers and the chatbot and conducted two focus groups with the workers to better understand their backgrounds and perspectives.
% We also conducted interviews and focus groups with healthcare professionals, Myna's program team, and technology developers.  The WhatsApp and Slack communications among them were also recorded for analysis. 
% As we analyzed the log data, the other sources of data provided important context on the complexities of designing a culturally sensitive chatbot.

\chirm{As part of the testing process, community workers were called to Myna's in-house digital centers to onboard to the chatbot in group sessions, and to encourage them to ask questions about SRH by creating a safe and supporting environment for discussing sensitive topics.
% as well as get training on SRH, considering cultural taboos around the topic. 
Peer-to-peer learning was used to create a safe and supportive environment for discussing sensitive topics. 
Once the community workers felt comfortable asking questions related to SRH, they were encouraged to interact with the chatbot by asking questions on family planning, pregnancy, and menstrual health-related topics using Hinglish (or transliterated Hindi). This interaction helped in generating a training dataset for the chatbot. The community workers played a crucial role throughout the chatbot's development, participating from the early stages of dataset generation to the validation phase. Their involvement ensured the chatbot was trained with relevant, culturally sensitive queries, enhancing its effectiveness in addressing SRH topics. Initially, they were invited to digital centers where they engaged with the system under the supervision of the program team. The team monitored these interactions and assisted with prompt generation. Observations made by the team during this phase were also recorded for analysis. 

In the later phases, participants showed a strong interest in asking questions and seeking information from the chatbot. They asked their family members and neighbors about their concerns and started prompting those questions in the interactions from their home. Interaction with the chatbot continued over a period of nearly three months, from late December 2023 to mid-March 2024. }
% During this period, all interactions were logged in a database, including timestamps, questions asked, chatbot responses, conversation and message identifiers, and the language used. All the logged responses were later reviewed and commented on by healthcare professionals. 
% They were asked to evaluate for the metrics of medical accuracy, social context and terminology used, 


\subsection{Data Analysis}
The analysis phase was done in several stages. First, the transcribed log data from user chatbot interactions and WhatsApp and Slack communications were translated into English. However, the Hinglish text was analyzed alongside the English text to retain meaning. A total of 2118 question-answer pairs from the message logs were further analyzed using open coding to identify key issues related to the chatbot’s effectiveness and cultural sensitivity.
% \chirm{ We conducted an inductive analysis and coding process on the chatbot log data to identify patterns and common themes in user queries and chatbot responses. The coding process followed an iterative approach, beginning with line-by-line coding to identify the SRH topics of the user prompts. We further filtered out the prompts and responses that had a cultural relevance. }
\chiadd{Given that the log data was in the form of spreadsheets and involved analyzing chatbot conversations in both Hinglish and English, we chose to do manual analysis over using analysis software. This ensured that each user interaction, including prompts and chatbot responses, was examined in detail in time sequence, capturing complexities that automated tools might overlook.} 
%Our analysis aimed to capture not only the explicit content but also the implicit meanings and cultural relevance situated in the interactions, ensuring a comprehensive understanding of the data. 
We conducted an inductive analysis and iterative coding process to identify patterns and common themes in user queries and chatbot responses, and with the interview data.
We started with line-by-line coding, and categorized user prompts by SRH topics and type of questions. 
\chiadd{The coding was conducted by the first two authors independently. Discussions were held with the whole team regularly to compare, refine, and consolidate the codes.
In case of conflicting codes, the last author weighed in and helped reach a resolution through discussion.
Through this iterative process, codes were carefully evaluated in the context of surrounding text, preserving the intent of user prompts and the chatbot's responses. This collaborative approach helped ensure inter-coder reliability.}

\chiadd{In particular, we analyzed user prompts and chatbot responses for the role of culture. To identify relevant cultural components, we relied on the definition of culture presented earlier, as ``\textit{shared values} (e.g. community cohesiveness and support), \textit{norms} (e.g. community dynamics, laws, and regulations), \textit{codes} (e.g. colloquial language on SRH instead of medical terms), \textit{roles} (e.g. gender roles), \textit{and assumptions} (e.g. medical consensus) \textit{that shape a group's beliefs} (e.g. on the importance of vaccination), \textit{attitudes} (e.g. taboos around SRH), \textit{and behavior} (e.g. menstrual practices, diet)'' \cite{griffith2024cultural}.
We considered both the literal meaning of the text and its potential meaning in the broader sociocultural context and analyzed to what extent this was understood by the chatbot.}  
During this process, the WhatsApp and Slack conversations, along with data from interviews and focus groups, provided context on how healthcare professionals perceived the accuracy and cultural sensitivity of chatbot responses.

\subsection{Study Limitations}
A limitation of our study is that we could not link user IDs to their demographic data during analysis.
Also, the participants were women from the community who were employed and monetarily compensated by Myna specifically to generate data, which may have influenced user engagement. This testing stage, which took place both on Myna's premises and in the \chirm{workers'} \chiadd{women's} homes, is critical before implementing with the community as the chatbot's performance was uncertain. In future research, we plan to test the chatbot with community members accessing the bot only from their homes, to better understand organic uptake and address user concerns before launching the application more broadly.

\subsection{Positionality}
Our team includes individuals from academia and the program team at Myna. All of us are of Indian origin, working or living in India or the United States. As a group, we come from diverse cultural and religious backgrounds. One of us identifies as a cis-man, and the rest as cis-women. We all have lived experiences and observations around sexual and reproductive health in Indian settings, with a collective experience of almost two decades working on gendered health and wellbeing, with and without use of technology. \chiadd{
% Our lived experiences and professional insights have shaped our understanding of the sociocultural and systemic factors affecting reproductive health outcomes. 
Our lived experiences with the cultural and social context and professional insights guided chatbot development and interpretation of findings.
% , as this type of analysis necessitates an understanding of the cultural and social context. 
We acknowledge that our own positionality may have introduced certain assumptions and biases. 
In particular, we are committed to centering women's agency when making decisions about their own reproductive health. 
To address biases, we actively engaged healthcare professionals and women from the community to check our assumptions during analysis.}

\begin{table}
\centering
\small
\begin{tabular}{l l l l l}
\toprule
\textbf{Age} & \textbf{Religion} & \textbf{Married Status} & \textbf{Children} & \textbf{Education}  \\ 
\toprule
32 & Hindu & Married & 2 & 10th  \\ 
32 & Hindu & Married & 2 & SY.BA \\ 
- & Hindu & Married & 0 & 10th  \\ 
19 & Hindu & Married & 0 & SY.BA  \\ 
33 & Hindu & Married & 2 & 12th  \\ 
27 & Muslim & Married & 1 & 10th  \\ 
28 & Muslim & Married & 2 & 11th \\ 
37 & Buddhist & Married & 1 & FY.BA \\ 
34 & Muslim & Married & 2 & FY.BA \\ 
45 & Hindu & Married & 2 & 12th \\ 
38 & Muslim & Married & 4 & 10th \\ 
32 & Hindu & Married & 2 & 12th \\ 
37 & Muslim & Married & 2 & 10th \\ 
30 & Muslim & Married & 2 & 10th  \\ 
31 & Hindu & Married & 2 & 12th  \\ 
34 & Hindu & Married & 2 & Graduation  \\ 
42 & Hindu & Married & 2 & 12th \\ 
28 & Hindu & Married & 2 & BA Completed  \\ 
\bottomrule
  \end{tabular}\\
    % \vspace{10pt}
 \caption{\textbf{Demographic information about our study participants.} Fields marked as “-” indicate that the information was not collected. Participant IDs were assigned to ensure anonymization. FY.BA and SY.BA refers to the completion of one or two years respectively of a three-year BA degree.}
  \label{tab:demo}
  \end{table}
