% \newpage
\section{APPENDIX}
In this section, we include the tables for the description of prompts given to the LLM for response generation and translation, categorization of topics and type of question examples, and original Hinglish text of the chat interactions presented in Section 5 followed by more examples of user queries and chatbot responses.

% \newpage
\begin{table*}[ht]
\centering
%\small
\fontsize{7}{8}\selectfont
   \begin{tabular}{>{\raggedright\arraybackslash}p{2cm} p{2.5cm}p{2cm} p{6.5cm}}
%\begin{tabular}{llll}
\toprule
\chiadd{\textbf{Topic}}& \chiadd{\textbf{Type of Question}} & \chiadd{\textbf{Language}} & \chiadd{\textbf{User Query}} \\
\toprule
\chiadd{Contraceptive methods} & \chiadd{Basic Conceptual Inquiry} & \chiadd{Hinglish} & \chiadd{Condom Kya hota hai?}\\
 & &\chiadd{English
(Translated)}& \chiadd{What is a condom?}\\ 
 \hline
 \chiadd{Family planning} & \chiadd{Complex Queries} & \chiadd{Hinglish} & \chiadd{Family planning ke liye Copper-T lagate hain, vah lagane ke liye kitna time lagta hai??}\\
 & &\chiadd{English
(Translated)}& \chiadd{How much time does it take to insert copper-T for family planning?}\\ 
 \hline
\chiadd{Family planning} & \chiadd{Health/Safety concerns and wellness}  & \chiadd{Hinglish} &\chiadd{ Family planning mein diaphragm use karne se UTI ke problem ho sakte hai kay?} \\
 & &\chiadd{English
(Translated)}& \chiadd{Can using a diaphragm for family planning cause UTI problems?}\\
 \hline
\chiadd{Sexual Health} & \chiadd{Advice/Opinion}  & \chiadd{Hinglish} & \chiadd{agar mai drink kar ke sax karti hu to kya ho sakta hai?} \\
 & &\chiadd{English
(Translated)}& \chiadd{What can happen if I have sex after drinking?}\\
 \hline
 \chiadd{Pregnancy} & \chiadd{Advice/Opinion} & \chiadd{Hinglish} & \chiadd{Konsi aisi bimari hai jis mai hame pregnancy avoid karna chahiye?} \\
 & &\chiadd{English
(Translated)}& \chiadd{What are the diseases during which we should avoid pregnancy?}\\
 \hline
%\chiadd{Pregnancy}& \chiadd{Misconception} & \chiadd{Hinglish} & \chiadd{Papita khane se baccha girta hai kya?} \\
 %& &\chiadd{English
%(Translated)}& \chiadd{Does eating papaya cause a miscarriage?}\\
 %\hline
%\chiadd{Sterilization} & \chiadd{Sexual functionality}& \chiadd{Hinglish} & \chiadd{Mahila nasbandhi ke bad ky sex ki ichha khatm ho jati he ?}\\
 %& &\chiadd{English
%(Translated)}& \chiadd{Does a woman's desire for sex end after sterilization?}\\
 %\hline
%\chiadd{Contraceptive methods} & \chiadd{Menstrual changes} & \chiadd{Hinglish} & \chiadd{Kya Ipill lene ke baad periods jaldi aate hai?}\\
 %& &\chiadd{English
%(Translated)}& \chiadd{Do periods come early after taking an i-pill?}\\
 %\hline
\chiadd{Reproductive anatomy} & \chiadd{Informational} & \chiadd{Hinglish} & \chiadd{Sperm ko kaun control karta hai?} \\
 & &\chiadd{English
(Translated)}& \chiadd{Who controls sperm?}\\
 \hline
%\chiadd{Family planning} & \chiadd{Gender Roles} & \chiadd{Hinglish} & \chiadd{Family planning ki responsibility gents kyu nahi hoti hai?}\\
% & &\chiadd{English
%(Translated)}& \chiadd{Why is family planning not considered the responsibility of men?}\\
 %\hline
\chiadd{Menstrual health} & \chiadd{Health/Safety concerns and wellness} & \chiadd{Hinglish} & \chiadd{Jab period main hote hai to kya chiz nhi khana chahiye} \\
 & &\chiadd{English
(Translated)}& \chiadd{What should not be eaten during periods?}\\
 \hline
\chiadd{Sterilization} & \chiadd{Health/Safety concerns and wellness} & \chiadd{Hinglish} & \chiadd{Mahila nasbandi ke tanke sukhne ke liye Kya Karen?}\\
 & &\chiadd{English
(Translated)}& \chiadd{What should be done to help the stitches from female sterilization heal properly?}\\
 \hline
 \chiadd{Abortion} & \chiadd{Misconception} & \chiadd{Hinglish} & \chiadd{Adrak ka juice peene se kya abortion hota hai?} \\
 & &\chiadd{English
(Translated)}& \chiadd{Does drinking ginger juice cause abortion?}\\
 \hline
 \chiadd{Fertility support} & \chiadd{Informational:Basic Conceptual Inquiry} & \chiadd{Hinglish} & \chiadd{IVF india me bhi hota hai kya?} \\
 & &\chiadd{English
(Translated)}& \chiadd{Is IVF also available in India?}\\
 \hline
 \chiadd{Marriage \& Relationships} & \chiadd{Advice/Opinion} & \chiadd{Hinglish} & \chiadd{Shadi ke bad Hasband ke satha nhi raha hai to Kya  divorce lena sahi hai?}  \\
 & &\chiadd{English
(Translated)}& \chiadd{If I don't stay with my husband after marriage, is it right to get a divorce?}\\
 \hline
 \chiadd{Miscarriage} & \chiadd{Advice/Opinion} & \chiadd{Hinglish} & \chiadd{Pregnancy rukne ke bad bar bar Miscarriage hone ka kya karan ho sakti hai?}  \\
 & &\chiadd{English
(Translated)}& \chiadd{What could be the reasons for repeated miscarriages after conceiving?}\\
 \hline
\chiadd{Family planning} & \chiadd{Norms and Ethics}  & \chiadd{Hinglish} & \chiadd{Religion m operation karna mana hai to kya kare} \\
 & &\chiadd{English
(Translated)}& \chiadd{What to do if religion prohibits surgery?}\\
 \hline
\chiadd{Family planning} & \chiadd{Health Access}  & \chiadd{Hinglish} & \chiadd{Paisa Na Ho to Kya family planning ho sakti hai?} \\
 & &\chiadd{English
(Translated)}& \chiadd{Can family planning be done if there is no money?}\\
 \hline
%\chiadd{Family planning} & \chiadd{Decision Making} & \chiadd{Hinglish} & \chiadd{family planning ke liye kiski Raza Mandi lena zaroori hai ya FIR is kiska Raji hona jaruri hai?} \\
% & &\chiadd{English
%(Translated)}& \chiadd{Whose consent is necessary for family planning, or whose approval is essential for it?}\\
 %\hline
\chiadd{Follow-up} & \chiadd{Follow-up}  & \chiadd{Hinglish} & \chiadd{main doctors se puchna bhul gai thi} \\
 & &\chiadd{English
(Translated)}& \chiadd{I forgot to ask the doctors}\\
\bottomrule
  \end{tabular}
 \caption{\textbf{Categorization of SRH Topics and Type of Questions Covered by Users.} Categorization of topics and the type of questions with examples of how they were grouped.}
     % \vspace{10pt}
  \label{tab:examples}
  \Description{Examples of how the user questions were grouped for topic and type of question are given in both original Hinglish and English text.}
  \end{table*}


\begin{table*}[ht]
\fontsize{7}{8}\selectfont
  \centering
   \begin{tabular}{>{\raggedright\arraybackslash}p{2cm} p{12cm}}
    \toprule
    \textbf{Objective using LLM (GPT-4)} & \textbf{\chiadd{Simplified description of } the prompt given} \\
 \midrule
    Interpret and translate the user query to English & \chirm{You are an expert interpreter and a translator who assists rural Indian women by translating their Hinglish messages related to gynecological health issues into standard English. You should be sensitive to the privacy and cultural context of the users, capable of interpreting common mistakes and conversational and vernacular language, and focused on providing clear, accurate information to improve healthcare communication and outcomes. Please help me interpret and translate the user query considering the things discussed before.

Please return the output in JSON format with the key "translated text".}
\chiadd{A prompt designed to translate colloquial messages into clear English, ensuring respect for cultural nuances and user privacy while preserving the intent of the original message.}
   \\
    \midrule
    Gets the answer from the knowledge base
 & \chirm{You are a seasoned female gynecologist and obstetrician from india, boasting over 25 years of expertise. Ensuring the explanation and sentence formation are easy to understand by a 10-year-old. You understand the importance of cultural sensitivity and will offer you guidance with warmth, empathy, and compassion.

You're guiding rural Indian women, many of whom may lack in-depth biological knowledge and might be apprehensive.

Your core objective is to provide empathetic and dear counsel Prioritize medical accuracy and swiftly correct any misunderstandings cultural nuances and recalling past interactions.

\textbf{Steps :} 
Engage in a step-by-step conversation, mirroring the approach a doctor would take. Identify potential causes for any mentioned ailments. When you need more information, systematically gather patient information, focusing on one follow-up question based on a specific aspect at a time. Give precise guidance to the questions when you understand the user's problem clearly, referring to the information by
our doctors in triple quotes below

"""
\$(vector search results)
"""
When applicable, suggest immediate home-based remedies and precautions.

\textbf{Important rules :}

Avoid recommending tests outright. Ensure your responses. are capped at 150 words, and easily comprehensible. Use easy-to-understand and commonly used words don't use complex medical terms 
Emphasize the importance of consulting a doctor for further diagnosis and treatment. If you perceive reluctance or need to wrap up conversation, direct them to ``Please consult a doctor with Telehealth"

Please return response in JSON format with the key medical answer"}
\chiadd{A prompt crafted to simulate expert guidance in a conversational tone, providing accurate and empathetic answers tailored to user needs. It emphasizes clarity, simplicity, and actionable advice, encouraging professional consultation when appropriate. Responses are concise and accessible.}
  \\
 \midrule
    Fixes the grammar
 & \chirm{Your job is to simplify text and fix grammar without changing the language or script. If the text mixes English with another language or uses a non-traditional script, just simplify and correct grammar. Keep the original meaning, Provide the simplified version in JSON format with 'updated text' as the key.

Simplifying also means replacing words which are heavy in the language by the user.

For example of Hinglish language:

Original message:

Yadi aapka pati ek bachche ke liye tayyar nahi hai, to pitrut ke baare mein aapki bhavnaon, apekshaon aur bhay donon like baare mein ek khull aur imandar baatchit kama mahatvapurn hal

New message:

Agar aapka pati bachche ke liye raazi nahl hal, to bachche aur pita banne ke vishay par aap dono ke vichar, ummeed, aur dar ke baare mein khul kar aur sachchi baat karni zaroori hai.}
\chiadd{A prompt to refine grammar and ensure readability while maintaining the original tone and meaning. It focuses on simplifying language to enhance understanding for the target audience.}
  \\

  \bottomrule
  \end{tabular}
%    \vspace{10pt}
  \caption{\textbf{Prompts given to the LLM.} Response generation and translation prompts given to the LLM.}
    \label{tab:prompt}   
    \Description{Table shows the prompts given to the LLM for response generation and translation. It has two columns with objectives using LLM and a simplified description of the prompt given.}
\end{table*}



\begin{table*}[ht]
 \centering
    \fontsize{7}{8}\selectfont
  \begin{tabular}{>{\raggedright\arraybackslash}p{1.5cm}p{2cm} p{2.5cm} p{8cm}}
    \toprule
 \textbf{Topic}  & \textbf{Language} & \textbf{User Query} & \textbf{Chatbot Response} \\
     \midrule
    \chiadd{Misconceptions and Gender Norms} & Hinglish & Family planning main muje sirf ladka chahiye to uske liye khuch upay hai kya?
  &
   Bacche ka janm chunne ka koi pakka tarika nahi hai. Parivaar niyojan ka mukhya uddeshya garbhavastha ko niyantrit karna hota hai, na ki bacche ke janm ko chunna. Kuch log koshish karte hain ki timing aur tarah-tarah ke tareeke apna kar ladka ya ladki hone ka anumaan lagaye, lekin yeh vidhiyan bilkul bhi pakki nahi hoti hain aur inka koi vaigyanik aadhar nahi hota. Agar aap parivaar niyojan ke vikalpon ke baare mein soch rahe hain, toh aapko un vikalpon par vichar karna chahiye jo aapke aur aapke parivaar ke liye sahi ho. Agar aap aur jaankari chahte hain ya kisi vishesh salah ki zaroorat hai, toh kripya Myna's Telehealth ke saath doctor se salah lein.\\
   \midrule
    \chiadd{Cultural taboo} & Hinglish & Family planning me sex word ko family ke samne kyu Nhi bol na chaya? 
 &
  Har parivaar aur samaj mein baatcheet ke tareeke alag hote hain. Kuch parivaaron mein 'sex' shabd ka istemal khule aam nahi kiya jata kyunki yeh unke liye asahaj ho sakta hai. Yeh samajik ya sanskritik maanyataon ke karan ho sakta hai. Jab family planning ki baat aati hai, toh kai baar log is vishay ko aur adhik samajhdaar tareeke se, jaise 'sambandh banana', 'parivaar niyojan' ya 'bachche ki planning' ke roop mein discuss karte hain. Yeh zaroori hai ki jab aap apne parivaar ke saath is vishay par charcha karte hain, toh aap unki bhavnaon ka samman karein aur aise shabdon ka chayan karein jo sabke liye suvidhajanak ho. Isse baatcheet ko aage badhane mein madad milti hai aur sabhi ko asahaj mehsoos nahi hota.  \\
  \midrule
  \chiadd{Dietary recommendations}& Hinglish & Parivar niyojan nasbandi ke ghav bharane ke liye kya khana chahie
  &
    Nasbandi ke ghav ko bharne ke liye poshtik aahar ka sevan karna bahut zaroori hai. Yahaan kuch aahar sambandhi sujhav diye ja rahe hain jo ghav bharne mein madadgar ho sakte hain:

1. Protein Yukt Aahar: Protein ghav bharne ke liye mahatvapurn hota hai. Daal, chana, soybean, paneer, dahi, ande, chicken, aur machli jaise protein se bharpoor aahar khayein.

2. Vitamin C aur Zinc: Ye dono poshak tatva ghav bharne mein sahayak hote hain. Santre, nimbu, kiwi, tamatar, bell peppers, akhrot, aur beej jaise aahar mein inka sevan badhayein.

3. Harit Sabjiyan: Palak, methi, sarson, aur anya harit sabjiyan vitamins aur minerals se bharpoor hoti hain jo sharir ki marammat mein madad karte hain.

4. Phal: Taaza phal jaise seb, kela, papita, aur berries antioxidants se bharpoor hote hain jo sharir ko swasth rakhte hain.

5. Paryapt Paani: Paryapt matra mein paani piyein kyunki hydration ghav bharne ke liye zaroori hai.

6. Whole Grains: Gehun, jau, aur oats jaise whole grains fiber se bharpoor hote hain jo pachan tantra ko durust rakhte hain.

In aahar ke alava, sharab, caffeine, aur atyadhik chini yukt aahar se bachna chahiye kyunki ye ghav bharne ki prakriya ko dhima kar sakte hain. Kya aapko aur koi sujhav ya jaankari chahiye?\\
\midrule
\chiadd{Religion}& Hinglish & Religion m operation karna mana hai to kya kare &
    Agar aapka dharm ya vishwas operation ya surgical procedures ke khilaf hai, to aapko kuch non-surgical birth control methods ke baare mein sochna chahiye. Kuch non-surgical methods hain:
1. Condoms: Yeh ek prabhavi tarika hai jo pregnancy aur sexually transmitted infections (STIs) se bachav karta hai.
2. Oral contraceptive pills: Mahilao ke liye daily leni padti hain.
3. Vaginal rings: Yeh mahine bhar ke liye insert ki jaati hain.
4. Contraceptive patches: Yeh skin par chipkane wale patches hote hain jo weekly badalne padte hain.
5. Natural family planning methods: Jaise ki ovulation tracking aur withdrawal method.
In sabhi options mein se koi bhi aapke liye sahi ho sakta hai. Lekin, inka istemal karne se pehle, yeh samajhna zaroori hai ki kuch methods jaise ki natural family planning, kam effective ho sakte hain. 
Aapko apne dharmik leader ya samuday ke margdarshan ke saath-saath ek doctor se bhi salah leni chahiye. Aap ``Myna's Telehealth'' se bhi sampark kar sakti hain jahan aapko doctor se salah mil sakti hai jo aapke dharmik vishwas ko samajhte hue aapko sahi margdarshan de sakte hain.\\
   
   \bottomrule
  \end{tabular}
%    \vspace{10pt}
  \caption{\textbf{Taking Social and Cultural Norms into Account (Hinglish).} Chatbot response to questions on cultural beliefs, taboo, food recommendations and role of religion. Find the English text in Table \ref{tab:Baby Sex}}
    \label{tab:stigma}
 \Description{Chatbot response to user questions on cultural beliefs, taboos,
food recommendations, and the role of religion is presented in the table with original Hinglish text. The English text is referenced to Table \ref{tab:Baby Sex}}
  
\end{table*} 



\begin{table*}[ht]
 \centering
    \fontsize{7}{8}\selectfont
  \begin{tabular}{>{\raggedright\arraybackslash}p{1.5cm}p{2cm} p{2.5cm} p{8cm}}
    \toprule
    \textbf{Topic} &  \textbf{Language} & \textbf{User Query} & \textbf{Chatbot Response} \\
     
    \midrule
   \chiadd{Age for sex}& Hinglish &  Proper age kya hai first time sex karne ka? &
  Sex ke liye `proper age' ka concept legal aur emotional maturity par depend karta hai. Har desh mein, sex karne ki legal age (jise `age of consent' bhi kaha jata hai) alag hoti hai, aur yeh aam taur par 16 se 18 saal ke beech hoti hai. Ye legal umar yeh sunishchit karni ke liye hoti hai ki vyakti emotionally aur physically mature ho chuke hain apne decisions samajhne aur unke parinaam ko sambhalne ke liye. Lekin, sirf legal age ka hona hi kafi nahi hai. Yeh bhi zaroori hai ki vyakti emotionally ready ho, unke paas sahi jankari ho, aur wo apne partner ke saath ek swasth aur samajhdar rishta banane ke liye taiyar ho. Safe sex aur consent (aapsi sahmati) ke mahatva ko samajhna bhi zaroori hai. Isliye, pehli baar sex karne ka `sahi samay' vyakti ke apne vichar, bhavnaon, aur paristhitiyon par nirbhar karta hai. Agar kisi ko is vishay mein aur madad ya jankari ki zarurat ho, to ek vishwasniya doctor ya healthcare provider se baat karna uchit hoga.  \\
  \midrule
  \chiadd{Family planning}& Hinglish & 15 saal ke ladki family planning kar sakti hai? &
  Haan, 15 saal ki ladki family planning ke baare mein soch sakti hai, lekin is umar mein, zyadatar focus education aur personal development par hota hai. Agar family planning ki zarurat ho, toh sabse pehle non-invasive methods jaise ki barrier methods (condom ka istemal) ya oral contraceptives (goliyan) ke baare mein sochna chahiye. Ye dono tareeke surakshit aur reversible hote hain, matlab aap jab chahein toh inhe band kar sakti hain. Lekin, kisi bhi tarah ka decision lene se pehle, ek visheshagya ya Myna's Telehealth ke doctor se salah zaroor lein. Unka margdarshan aapko sahi disha mein le jayega. \\
  \midrule
   \chiadd{Divorce}&Hinglish & Shadi ke bad Hasband ke satha nhi raha hai to Kya  divorce lena sahi hai? 

  &
   Shaadi ke baad agar aap apne pati ke saath nahi reh paayi hain aur aapko lagta hai ki aapke beech samasyaayein hain, toh pehle yeh samajhna zaroori hai ki samasya kya hai. Har rishta alag hota hai aur kabhi-kabhi samajhauta ya samadhan dhoondhna mushkil ho sakta hai. Lekin, yeh bhi zaroori hai ki aap dono ek dusre se khule mann se baat karein aur samasyaon ka hal dhoondhne ki koshish karein. Agar aapko lagta hai ki aapne sab kuch try kar liya hai aur phir bhi samadhan nahi mil raha, toh aapko apne dil ki sunni chahiye. Yeh ek bahut bada faisla hai, isliye is par salah karne ke liye samay lein aur zaroorat padne par kisi visheshagya ya counsellor se bhi salah lein. Aapke khushi aur swasthya sabse pehle aate hain.\\
   \midrule
   \chiadd{Gender roles}& Hinglish & Purush nasbandi kyun  nahi karte &
   Purush nasbandi, ya vasectomy, ek sthayi garbhnirodhak prakriya hai jo ki bahut surakshit aur prabhavi hoti hai. Lekin, kai baar purush is prakriya ko nahi karwate hain kuch karanon ki wajah se:

1. Jankari ka abhav: Kai purushon ko is prakriya ke baare mein sahi jankari nahi hoti hai.
2. Galatfahmi: Kuch purush sochte hain ki nasbandi se unki yon himmat ya mardangi prabhavit hogi, jo ki galat hai.
3. Samajik Dabav: Samaj mein kai baar aise vichar hote hain ki nasbandi mahilaon ke liye hi hai.
4. Darr ya Bhay: Operation ya surgery ka bhay bhi ek karan ho sakta hai.

Vasectomy ke baad purushon ki yon himmat mein koi kami nahi aati aur na hi unke purush hormones mein koi parivartan hota hai. Yeh ek surakshit aur kam takleefdeh prakriya hai. Agar aap ya aapke parichit is vishay mein aur jaankari chahte hain, toh kripya sthaniya swasthya kendra ya doctor se sampark karein. Aapko aur kuch poochna hai is vishay mein?\\
   \bottomrule
  \end{tabular}
%    \vspace{10pt}
  \caption{\textbf{Responding to Social Dynamics (Hinglish).} Chatbot responses to questions on legal implications, and gender roles. Find the English text in Table \ref{tab:planning}}
    \label{tab:mar}
     \Description{Chatbot response to user questions on legal implications, and gender roles is presented in the table with original Hinglish text. The English text is referenced to Table \ref{tab:planning}}
  
\end{table*} 


\begin{table*}[ht]
\centering
\fontsize{7}{8}\selectfont
\begin{tabular}{>{\raggedright\arraybackslash}p{2cm} p{2.5cm} p{9.5cm}}
\toprule
\textbf{Language} & \textbf{User Query} & \textbf{Chatbot Response} \\
\midrule
\multirow{2}{*}{English (Translated)}  & If I'm not getting pregnant, what could be the main reason? & I understand that you might be worried about your problem. There can be several reasons for not getting pregnant. Have you noticed any changes in your menstrual cycle recently? Has your weight increased suddenly or do you have any other health issues like thyroid problems or PCOS? Also, have you ever considered family planning methods? This information will help me better understand your issue. \\
\cmidrule{2-3}
& I don’t know anything, please tell me what could be the reasons. & I understand that you don’t know much about your health, but you’re worried. There could be several reasons for your issue:
\begin{itemize}
    \item \textbf{Hormonal Changes}: Sometimes stress or weight gain/loss can cause hormonal imbalances, leading to irregular periods.
    \item \textbf{Anemia}: Low hemoglobin (iron deficiency) can also cause irregular periods.
    \item \textbf{Thyroid Issues}: Abnormal thyroid gland activity can affect your menstrual cycle.
    \item \textbf{Polycystic Ovary Syndrome (PCOS)}: This condition, where cysts form on the ovaries, can lead to irregular periods.
\end{itemize}
Are you experiencing any other symptoms, such as abdominal pain or mood swings? And have you been involved in any activities that might be causing you stress? This information will help me better understand your problem. \\
\midrule
\multirow{2}{*}{Hinglish} & Agar bacha nahi rukh raha hbai to uska main karan kya ho sakta hai? & Aapki samasya ke liye main samajh sakti hoon ki aapko chinta ho rahi hai. Bacha na rukhne ke kai karan ho sakte hain. Kya aapko pichhle kuch samay se periods mein koi badlav mehsoos ho raha hai? Kya aapka weight sudden mein badh gaya hai ya fir koi aur health issue hai jaise ki thyroid ya PCOS? Aur kya aapne kabhi family planning methods ke baare mein socha hai? Ye jaankari mujhe aapki samasya ko samajhne mein madad karegi. \\
\cmidrule{2-3}
& Mujhe kuch nahi malum muje kya huva hai please aap bataye ki kya karan ho sakte hai & Mujhe samajh hai ki aapko apne swasthya ke baare mein kuch pata nahi hai, lekin aapko chinta toh hai. Aapki samasya ke kuch karan ho sakte hain:
\begin{itemize}
    \item Hormonal badlav: Kabhi-kabhi stress ya wajan badhna/loss ke karan hormones mein asantulan ho sakta hai, jisse periods regular nahi rehte.
    \item Khoon ki kami: Kam hemoglobin (blood mein iron ki kami) bhi periods ko asamanya bana sakta hai.
    \item Thyroid samasya: Thyroid gland ki asamanya gatividhi bhi periods ko prabhavit kar sakti hai.
    \item Polycystic Ovary Syndrome (PCOS): Ismein ovary mein cysts bante hain, jo periods ko asamanya kar dete hain.
\end{itemize}
Aapko kya koi aur lakshan mehsoos ho rahe hain, jaise ki pet mein dard ya chidchidapan? Aur kya aapne koi aisi gatividhi ki hai jisse aapko stress ho raha ho? Ye jaankari mujhe aapki samasya ko samajhne mein madad karegi. \\
\bottomrule
\end{tabular}
\caption{\textbf{Pregnancy query with Follow-up.} Responses to queries regarding reasons for not getting pregnant and follow-up queries and responses for the same.}
\label{tab:reasons}
\Description{Chatbot response to user question regarding reasons for not getting pregnant and follow-up query
and response for the same is presented in a table with both original Hinglish and English text.}
\end{table*}

\begin{table*}[ht]
\centering
\fontsize{7}{8}\selectfont
\begin{tabular}{>{\raggedright\arraybackslash}p{1.5cm} p{2cm} p{2.5cm} p{8cm}}
\toprule
\textbf{Topic}&\textbf{Language} & \textbf{User Query} & \textbf{Chatbot Response} \\
\midrule
 Employment& English(Translated) & How important is one's job for family planning? &
My work is very important in family planning. I provide women and their partners with accurate
information and guidance about their family planning options. This helps them decide when and
how many children they want. I would tell them about different types of contraceptives, like the pill,
IUD, or condoms, and also explain their advantages, disadvantages, and contraindications. Apart from
this, I also give them advice regarding pregnancy related problems and their solutions. My aim is that
every person and every family should make informed decisions regarding their health and future. In
this way, I can contribute to improving the health and well-being of the society.\\
\cmidrule{2-4}
  &Hinglish &Apni job family planning ke liye kitni important hai?&Mera kaam parivar niyojan mein bahut mahatvapurna hai. Main mahilaon aur unke sathiyon ko unke parivar niyojan ke vikalpon ke baare mein sahi jankari aur margadarshan pradan karta hoon. Ye unhe ye tay karne mein madad karta hai ki woh kab aur kitne bachche chahte hain. Main unhe alag-alag garbh nirodhak ke baare mein batata hoon, jaise ki goliyan, IUD, ya Condom, aur unke fayde, nuksan, aur dushparinam ke baare mein bhi samjhaata hoon. Iske alava, main unhe garbhavastha se sambandhit samasyaon aur unke samadhanon ke baare mein bhi salah deta hoon. Mera uddeshya hai
ki har vyakti aur har parivar apne swasthya aur bhavishya ke nirnay soch-samajh kar le. Is tarah,
main samaj mein swasthya aur kalyan ko behtar dene mein yogdan deta hoon.\\
\bottomrule
\end{tabular}
\caption{\textbf{Understanding Information-seeking behaviors around SRH.} Chatbot response to a question on the relationship between having a job and family planning.}
\label{tab:emp}
\Description{User query and chatbot response on the relationship between job and family planning are given in both original Hinglish and English text.}
  
\end{table*}