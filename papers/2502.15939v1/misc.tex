\chirm{We analyzed the log data from the interactions between these workers and the chatbot and conducted two focus groups with the workers to better understand their backgrounds and perspectives.
We also conducted interviews and focus groups with healthcare professionals, HealthNGO's program team, and technology developers.  The WhatsApp and Slack communications among them were also recorded for analysis. 
As we analyzed the log data, the other sources of data provided important context on the complexities of designing a culturally sensitive chatbot.

As part of the testing process, community workers were called to HealthNGO's in-house digital centers to onboard them to the chatbot in group sessions, and to encourage them to ask questions about SRH by creating a safe and supporting environment for discussing sensitive topics.
% as well as get training on SRH, considering cultural taboos around the topic. 
% Peer-to-peer learning was used to create a safe and supportive environment for discussing sensitive topics. 
Once the community workers felt comfortable asking questions related to SRH, they were encouraged to interact with the chatbot by asking questions on family planning, pregnancy, and menstrual health-related topics using Hinglish (or transliterated Hindi). 
% This interaction helped in generating a training dataset for the chatbot.
% The community workers played a crucial role throughout the chatbot's development, participating from the early stages of dataset generation to the validation phase. 
% Their involvement ensured the chatbot was trained with relevant, culturally sensitive queries, enhancing its effectiveness in addressing SRH topics. 
Additional follow-up sessions were scheduled where they were invited back to the digital centers. 
The HealthNGO team monitored the chatbot interactions during these sessions, and assisted with initial prompting. Subsequent chatbot interactions were from the women's homes.
% Observations made by the team during this phase were also recorded for analysis. 

\textbf{MOVE TO FINDINGS - As participants got comfortable with the system, they showed strong interest in asking questions and seeking information from the chatbot. They asked their family members and neighbors about their concerns and started prompting those questions in the interactions from their home. }

Interaction with the chatbot continued over a period of nearly three months, from late December 2023 to mid-March 2024. During this period, all interactions were logged in a database, including timestamps, questions asked, chatbot responses, conversation and message identifiers, and the language used. All the logged responses were later reviewed and commented on by healthcare professionals. 
% They were asked to evaluate for the metrics of medical accuracy, social context and terminology used, 
Additionally, feedback from community workers on various aspects such as response length, understandability, the difficulty of terms used in responses, and latency was collected by the HealthNGO team to support improvements to the knowledge base.}



\begin{table*}[h]
% \fontsize{6}{5}\selectfont
% \footnote
\centering
\small
  % \begin{tabular}{>{\raggedright\arraybackslash}p{1.5cm} p{2.7cm} p{3.5cm} p{8.5cm}}
\begin{tabular}{l l l l}
\toprule
\textbf{Context Layers} &\textbf{Dimension} & \textbf{Example of Relevance to SRH or LLM Choices} & \textbf{Implications for Design} \\
\toprule

Societal & Laws and Regulations & Legal age of consent for sex & Include in Knowledge Base (KB)\\
  & Medical Consensus & Consensus among medical professionals on how to  & Include in KB\\
  & &  prevent spread of HIV & \\ 
 \hline
Regional & Spoken Language & Use of Hindi, Marathi, and Urdu widely in Mumbai & Pick LLM model or translation service to \\
  & & & support that language \\ 
 & Written Script & Use of transliterated Hindi & Gather and study examples of everyday chat \\
& & widely in WhatsApp communications &  communications; consider fine-tuning with \\
& & &  this data if translation does not work well \\
& & & or existing LLMs for that language perform poorly \\
 & Healthcare Access & Lack of affordable clinics in the area & Consider when suggesting actions to take \\
& & & such as offering teleconsulation or referring \\
& & & to free or local services \\
  \hline
  
Community & Community Dynamics & Taking a neighbor's health advice on how to & Update KB to recognize importance of community \\
 & & reduce pain during periods &  and encourage consulting a community leader for  \\
  & & & advice, while prioritizing verified medical information \\
   & Community Beliefs & Lack of belief in vaccinations & Counter misconceptions, counter harmful practices,  \\  
   % & & & any harmful practices, \\
  & & & maintain a neutral tone if it is a benign practice \\
 & Religion & Belief that sterilization is not allowed in their religion & Update KB to recognize religious and communal  \\
   &  &  &   beliefs, encourage talking to a religious/community  \\
 & Caste \& Tribe & Belief that a woman is impure while menstruating &  leader, and offer alternatives if \\
 & & & a recommendation goes against their beliefs \\
     % &  &  &  if they believe that a certain recommendation 
  % &  & eligibility for government schemes & \\  
 & Dialect & Specific manner of speaking or writing & Include a dictionary to swap out with words  \\
  &  &  & that are used locally \\
& Gender Roles & Limited mobility of women shaping healthcare access & Prompt asks LLM to acknowledge dominant gender \\
& & &   roles, but also that they can change. Center  \\
& & & women's agency and offer strategies to negotiate  \\
& & &  power over one's health and in their relationships \\
 & Diet & Dietary preferences & Update KB with recommendations based on \\
 & & &  local dietary practices \\
   \hline
   
Individual & Household Dynamics & Discomfort talking about sex with partner & Include suggestions for navigating dynamics in KB \\
 & Privacy Practices & Woman shares device with her son & Design the LLM prompt to generate text \\
 & &  & with a tone that is respectful and formal,\\
 & &  &  similar to a medical professional \\ 
 & Age & Experiencing menopausal symptoms & Ask follow-up questions on age when \\
   &  &  &  relevant to a health symptom, and factor \\
      &  &  &  age in prompt when brought up by the user \\
 & Income \& Occupation & Stigma around accessing free government services & Consider when suggesting actions to take \\
& & & such as offering teleconsulation or referrals \\
 & Marital Status & Taboo on using contraceptives (having sex) & Update KB to provide information while  \\
    &  &  before marriage & acknowledging taboos \\
 % &  &   &\\ 
%  & Occupation & Works in a labor-intensive job and cannot rest & \\
% & &   during pregnancy  & \\
 & (Dis)Abilities & Mobility impacting healthcare access & Offer accessible services like teleconsultation \\ 
 % & Education & Learned reproductive health in 10th grade &\\
 & Digital Literacy & Discomfort with typing resulting in grammatical errors & Do grammar correction, offer voice capabilities \\
 & Health Literacy & Unfamiliar with a medical condition or term & Continually update dictionary and KB\\
 & & & with simpler language\\
 & Medical History & Having a previous miscarriage; or experiences with & Ask follow-up questions on medical history relevant\\
  &  & heavy and painful periods &  to a health symptom, and include medical history \\
  &  &  & in prompt when brought up by the user \\
\bottomrule
  \end{tabular}
    \vspace{-10pt}
 \caption{\textbf{Framework for Culturally Sensitive Design of LLMs in Healthcare.} The acronym KB refers to the Knowledge Base.}
  \label{tab:framework}
  \end{table*}