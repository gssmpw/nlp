\section{DESIGN OF THE CHATBOT}

We now share more details about the final design and architecture of the chatbot. The chatbot's design was aimed towards providing non-judgmental, confidential, and medically accurate advice on sexual and reproductive health (SRH). It underwent a series of iterative improvements to enhance its effectiveness, cultural sensitivity, and overall user experience in four phases with different GPT models, including a fine-tuned GPT-3.5 model, GPT-3.5, and GPT-4. The final system was developed using the preview model of GPT-4. The chatbot was hosted on a custom web application, as shown in Figure \ref{fig:interface}. It interfaced in Hinglish \chirm{(or transliterated Hindi)} which was common in digital interactions in the study context. In addition to receiving responses to questions as text, users could choose the text-to-speech button to listen to an audio recording of the text. This was included to accommodate users with a range of literacies. The final architecture of the bot consists of three main stages. A brief overview of the system's final architecture is described in Figure \ref{fig:design1}.



\begin{figure}[h]
  \centering
  \includegraphics[width=0.25\textwidth]{interface.png} 
  \caption{\textbf{Chatbot Interface on Mobile Web Platform}. The figure presents a greeting in Hinglish and an explanation of what the chatbot is capable of. It also presents buttons below the text with suggested questions to click on to get started. On the right is a button with the speaker icon which plays automated text-to-speech on clicking.}
  \label{fig:interface}
  \Description{Chatbot Interface screenshot on Mobile Web Platform. The top right end has a logout button for the chatbot. The interface has a welcome message in Hinglish for the user, followed by some question suggestions. Read aloud button next to the welcome note. The interface consists of an input field to type in a prompt for the chatbot. Microphone icon in the input field for the user to speak. The menu bar is at the bottom end of the screen.}
\end{figure}



 
In the \textit{first stage}, the user's query in Hinglish, is interpreted and translated into English using OpenAI's LLM translation services. This process leverages LLMs with prompts designed to accommodate the linguistic and cultural backgrounds of the target population. The translated query is then passed further to capture the correct medical context around the query. This is done by consulting medical documents (contextualized and developed) by MMF through a retriever model like RAG (Retrieval Augmented Generation) that draws on a vector store of text embeddings of these documents. \chiadd{Our approach follows a widely adopted methodology, which involves the following steps: First, the documents in english are split into smaller chunks and stored as vector embeddings in a vector database. A query is then used to find similar chunks in the database, and the retrieved chunks are added to the prompt.} This ensures the responses are medically informed and tailored to the user's needs and context, and reduces the possibility of hallucinations and inaccuracies in the text generated.

The \textit{second stage} involves generating the medical answer from the bot's knowledge base. This is facilitated by prompting an LLM with relevant background information and instructions. The prompt asks the LLM to generate a response as a seasoned female gynecologist and obstetrician from India with over 25 years of expertise (see Table \ref{tab:prompt} in Appendix for the detailed prompt). Emphasis is placed on the importance of cultural sensitivity, empathy, and compassion when interacting with underserved Indian women who may lack in-depth biological knowledge. The LLM is instructed to engage in a step-by-step conversation, mirroring the approach that a doctor would take. This includes identifying potential causes for any mentioned ailments, systematically gathering patient information, and providing precise guidance based on a clear understanding of the user's problem. The LLM's response is then passed to the third stage for localization and final output generation.

In the \textit{third stage}, the medical answer from the LLM undergoes a localization process to ensure its cultural and linguistic relevance. The system adapts the language of the response to match local dialects and usage patterns, employing a service that performs fuzzy matching to find and replace words with their localized counterparts. Before delivering the final response to the user, the chatbot simplifies the text to ensure grammatical correctness and ease of understanding. 
The length of the chatbot responses ranged from 25 words for the shortest to 393 words for the longest. On average, responses contained 123 words. Throughout the workflow, various safeguards are implemented to ensure the chatbot operates within its intended scope. For instance, the chatbot is instructed to avoid recommending tests outright, refrain from prescribing medications, and emphasize the importance of consulting a healthcare professional for further diagnosis and treatment. 
\begin{figure*}[h]
  \centering
  \includegraphics[width=1\textwidth]{chatbotdesign.png} 
  \caption{\textbf{Final system architecture of the chatbot}. The three stages of the chatbot flow include---Translation module, Generating the medical answer, and Localization module. The translation module involves OpenAI’s LLM model which interprets and translates the user’s query in Hinglish into English. Generating the medical answer involves generating the medical answer from the chatbot’s knowledge base by prompting the LLM with a predefined prompt and translating it back to the user language. The localization module involves replacing complex medical words with colloquial terms.
}
  \label{fig:design1}
   \Description{Three stages of the chatbot: Translation module, Generating the medical answer, Localization module. Translation module involves OpenAI’s LLM which interprets and translates the user’s query in Hinglish into English. Generating the medical answer involves generating the medical answer from the chatbot’s knowledge base by prompting the LLM with a predefined prompt and translating it back to the user language. Localization module involves localization process of medical answers generated where words are replaced with their localized counterparts.}
\end{figure*}