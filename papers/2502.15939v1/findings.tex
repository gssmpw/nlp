\section{FINDINGS}
% In this section we discuss the findings of our study, focusing on cultural elements of the local context. 
We now present learnings from our analysis of both the questions asked by the users, and the chatbot responses to sensitive and highly contextual queries on sexual and reproductive health.
\DR{We interpret these with the support of data gathered and analyzed from interviews, focus groups, and online conversations with the healthcare professionals, developers, and the program team at Myna.} 
\RD{We organize our findings by first noting broader trends around information seeking behaviors and focusing on \textit{surface structure} ~\cite{resnicow1999cultural}---such as the topics, timings, and types of questions asked, and the language of interaction---based on user log data. We then moved our focus to the \textit{deep structure}---the underlying cultural factors that shape behaviors, including values, beliefs, norms, and
stressors.
This led us to note the role of cultural beliefs and stigma, and legal and social dynamics in shaping interactions. 
% Here, we drew on the definition of culture stated earlier, \textit{i.e.} \textit{``shared values, norms, codes, roles, and assumptions that shape a group’s beliefs, attitudes, and behavior through their interactions''} \cite{griffith2024cultural}. 
Based on the chat timestamps, we further reviewed Slack and WhatsApp conversations within the development and program teams and used insights from interviews and focus groups. This gave us more context on when and why the system failed to respond to a user query correctly, and to understand the role of design decisions.} 
%We structured our findings using the definition of culture we adopted. We grouped our findings based on the analysis of chatbot interactions, to identify indicators of cultural relevance in user engagement. These themes were then contextualized with insights from interviews, focus groups, and online conversations with the healthcare professionals, developers, and the program team at Myna. By integrating these diverse data sources, we aimed to evaluate how cultural factors and design changes over time influenced chatbot performance and to interpret the patterns observed in user interactions.}
Our findings highlight the strengths and limitations of the system in appropriately recognizing and responding to diverse cultural elements in its interactions. 
% We identified several culturally sensitive elements from the user interactions presented graphically in Fig.1.

% In this section, we present sample user messages and bot responses. 
In several places, we have included both the original Hinglish text as well as the translated English text. This is done in an effort to retain the cultural context and showcase how questions and responses were framed, which can be lost in translation. For the English translations, we have aimed to provide literal translations as much as possible, which may result in some oddly framed sentences, but try to retain the original style and meaning of the user messages. Finally, when quoting from WhatsApp or Slack conversations between team members at Myna, we have shared the original text (which was mostly in English), including any typos. All names mentioned in the paper are pseudonyms.


\subsection{Understanding Information-Seeking Behaviors Around SRH}
A major concern for Myna was ensuring that the chatbot took into account the literacies and socioeconomic realities of the participants. 
Below we first summarize patterns we observed in the questions that users were interested in asking of the chatbot. We then reflect on the conversation styles of both users and chatbots and the challenges those could present. We then discuss whether the chatbot could provide easily comprehensible text to users. We also consider how demographic factors such as income and education played a role in conversations.

\subsubsection{Topics, types, and timing of questions asked by users}
We analyzed the log data to identify the various topics and types of questions asked by users. These categorizations are presented in Tables \ref{tab:topic} and \ref{tab:type}, with the number of queries that fell under each category. Myna had instructed the \chiadd{women recruited from the community} \chirm{workers} to generate questions about family planning, pregnancy, and menstrual health---which are typically the focus of women's health programs in India. However, our analysis revealed that the questions asked on the chatbot extended beyond these topics, indicating a strong interest and need in other SRH areas. Table \ref{tab:topic} shows that questions also revolved around the use of contraceptive pregnancy, reproductive anatomy, and marriage and relationships. \chiadd{
% While all the topics can be classified under one broad category, they have been distinguished to account for the specificity and frequency of user queries, allowing for a more detailed and clear analysis. 
Table \ref{tab:type} presents the types of questions asked by users, and most were basic conceptual inquiries (577), followed by 504 number of more advice/opinion queries.
This differentiation was made to understand the SRH literacy level of the women and serve as a precursor to evaluate the cultural appropriateness. For instance, \textit{``Condom Kya hota hai?''} translated as \textit{`What is condom?'} has been categorized as a basic informational question where as \textit{``15 saal ki ladki kaise karegi family planning?''} translated as \textit{``How will a 15 year old girl do the family planning?''} was categorized as a question on norms and ethics. Examples of how user queries were categorized by topic and the type of question are presented in Table \ref{tab:examples} in the Appendix.}
% However, due to space constraints, we have not included examples for every possible combination of topic and question type.
Most users asked knowledge-seeking questions (over half of all user queries), suggesting a strong desire for basic educational content on the topic. However, our analysis of the types of questions asked indicated that a significant number of questions focused on concerns about their specific situation with their partner or in-laws, safety concerns around contraceptives, social norms around sex and family planning, misconceptions, and more (Table \ref{tab:type}). 

Our analysis of users' prompts showed a variability in length. While some users provided very short queries, others provided more elaborate ones. The length of user prompts ranged from a minimum of 3 words to a maximum of 70 words, with an average length of 11 words. We found that the chatbot could handle user prompts of varying lengths.
We also analyzed the timestamps from the user interactions log data to determine the times of day when users were most active. We identified that interactions began as early as 2 AM. Interaction activity peaked between 6 AM and noon, with nearly 1,900 of 2118 queries being asked during this time frame. After noon the interaction activity gradually declined for the rest of the day. This may be because of other responsibilities that they were expected to manage at home, with them having most personal quiet time before their family members wake up and after they leave home for work and school.

% We observed the need for providing information that enhances their understanding of the topics they are inquiring about. 

% 

\begin{figure}[h]
  \centering
  \includegraphics[width=0.5\textwidth]{timeslot.png} 
  \caption{\textbf{Timing of questions asked by users.} The graph summarizes the number of queries received at different times of the day. The highest activity was between 5AM and 12PM.}
  \label{fig:timeslot}
 \Description{The bar graph plots the number of user messages against the time of day, with the X-axis representing the time in hourly intervals from 2:00 AM to 10:00 PM and the Y-axis representing the number of user messages ranging from 0 to 400. The highest message activity is observed between 6:00 AM and 12:00 PM, peaking at around 7:00 AM and 12:00 PM, with over 300 messages during these hours. There is minimal activity after 1:00 PM, with a significant drop in user messages throughout the afternoon and evening. Very few messages are sent between 2:00 AM and 5:00 AM, as well as after 7:00 PM.}
\end{figure}

\begin{table*}[h]
\centering
\small
  \begin{tabular}{>{\raggedright\arraybackslash}p{3cm} >{\raggedright\arraybackslash}p{9cm} >{\raggedleft\arraybackslash}p{2.5cm}}
\toprule
\textbf{Topic} &  \textbf{Description} & \textbf{Number of queries} \\
\toprule
Contraceptive methods & Information about \chirm{various ways to prevent pregnancy} \chiadd{various non-permanent contraceptive methods} & 657 \\
Family planning & Guidance on planning pregnancies &541 \\
Sexual Health &Information on safe sexual practices\chiadd{, including protection from HIV and STDs} & 258 \\
Pregnancy & Information on conception, and health during pregnancy&227 \\
Sterilization &  Permanent methods of contraception & 182 \\
Reproductive anatomy & Structure and function of the male and female reproductive systems. & 99 \\
Menstrual health & Concerns about menstrual cycles&80 \\
Abortion &  Information on the medical termination of pregnancy & 34 \\
\chiadd{Fertility support} \chirm{IVF} & Information about assisted reproductive technology,  \chiadd{such as IVF} & 17 \\
Miscarriage & Information on natural loss of pregnancy & 04 \\
Follow-up & Generic followup questions &09 \\
\bottomrule
\textbf{Total} &  &\textbf{2118} \\
  \end{tabular}
    % \vspace{10pt}
 \caption{\textbf{Categorization of SRH Topics Covered by Users.} Categorization of topics with the description of how they were grouped. \chiadd{The Follow-up category includes prompts asked by users in direct response to a previous interaction or by clicking on a follow-up question suggested by the chatbot. Examples for categorization are presented in the Table \ref{tab:examples}.} }
  \label{tab:topic}
  \Description{Categorization of sexual and reproductive health topics with the description of how they were grouped and the number of queries asked for each topic. Contraceptive methods is the most asked topic with 657 questions and Miscarriage was the lowest with 4 questions.}
  \end{table*}
% 

%   
\begin{table*}[h]
\centering
\small
  \begin{tabular}{>{\raggedright\arraybackslash}p{3.5cm} >{\raggedright\arraybackslash}p{8.5cm} >{\raggedleft\arraybackslash}p{2.5cm}}
\toprule
\textbf{\chirm{Topic} \chiadd{Type of Questions}} &\textbf{Description}& \textbf{Number of queries} \\
\toprule
\chiadd{Basic Conceptual Inquiries} & \chiadd{Inquiries focused on basic information about SRH topics, such as definitions of SRH terms} & \chiadd{577} \\     
\chiadd{Complex Queries} & \chiadd{Inquiries that go beyond basic SRH knowledge, such as use of a specific contraceptive} & \chiadd{438} \\  
Advice/Opinion & Seeking personal recommendations, \chiadd{and guidance about making informed choices in personal or health-related issues} &\chiadd{504}\chirm{220} \\
Health and Safety concerns & Safety issues and health risks \chiadd{, including concerns around hygiene and safe sexual practices such as preventing HIV and STDs} & \chiadd{365}\\ 
Misconception &  Clarifying misunderstandings or beliefs &137 \\
Norms and Ethics & Concerns about social norms and ethics, \chiadd{including expectations on gender roles} &\chirm{18}\chiadd{68}\\
Healthcare Access & Challenges in accessing healthcare due to financial or social constraints &20 \\
Follow-up & Generic followup questions & 09 \\
\bottomrule
\textbf{Total} &  &\textbf{2118} \\
  \end{tabular}
    % \vspace{10pt}
 \caption{\textbf{Types of Questions Asked by Users.} Types of questions with description based intent of the user prompt. \chiadd{The Follow-up category includes prompts asked by users in direct response to a previous interaction or response provided by the chatbot. } }
  \label{tab:type}
  \Description{Categorization of types of questions with descriptions based on intent of the user prompt. Knowledge-seeking questions were the most asked questions with 924 and decision-making questions were the least asked with 17. Follow-up questions were generic questions that did not fit into any category. Examples for categorization are presented in the Table \ref{tab:examples}.  }
  \end{table*}
% 



\subsubsection{Communication style of users and the chatbot}
We paid close attention to the grammar, sentence construction, choice of both health and non-medical terms, and dialects used by the users and the chatbot. 
These attributes play an important role in interpreting the user prompt and generating a response that is easily understandable by the user.
We noticed that initially, the chatbot responses were designed to provide information to user queries directly without acknowledging the local social norm. After a system design update, a predefined response template was used to improve the interaction by incorporating cultural expression, as shown in Figure \ref{fig:interface}. The addition of ``Namaste'' gave the interaction a more culturally respectful tone, acknowledging the user before giving the information. 
We also learned that the tone of the chatbot had to be updated. For instance, early in the chatbot development process, Dr. Preeti shared that the chatbot should be using \textit{``aap, aapka, aapko''} [seen as a respectful and formal way of addressing someone in the second-person] instead of \textit{``tu, tujhe''} (informal ways of addressing someone).

In our focus groups with workers, they also expressed frustration when the chatbot's responses were overly generic or dismissive, particularly when it directed them to consult a doctor without providing sufficient information. For instance, users seeking guidance on family planning may expect more than a referral to a healthcare provider. They might want actionable insights or advice that considers societal expectations, financial constraints, and community norms. 
Though suggesting talking to a medical professional seems a reasonable way to ensure no over-reliance on chatbot responses, in practice, it resulted in users feeling dissatisfied with the response. This was because they were using the chatbot as they did not have another channel available to them.
Balancing medical recommendations with culturally sensitive, actionable information is thus key to maintaining user trust and engagement. To address this, Myna integrated the option of doing a teleconsultation, rather than just suggesting that the user talk to a doctor.

We also noted that few follow-up questions were asked by participants, even when encouraged by the chatbot. For instance, we noted that one user asked a question on the possible reasons for not getting pregnant (see Table \ref{tab:reasons} in the Appendix). The chatbot asked follow-up questions to gather specific details about the user’s situation, such as changes in menstrual cycles, weight fluctuations, or any existing health issues like thyroid problems or PCOS. However, the user responded by saying that they did not have any information and requested an explanation of the possible reasons.
Despite the lack of context, the chatbot effectively responded to the query by providing a generic list of potential reasons for not getting pregnant. 
This shows the system's capability to give relevant information even when a user was not willing to provide further details. 
% This variability in the context-seeking approach across interactions shows the chatbot’s limitation.

\subsubsection{Adapting to local language patterns}
We noted that user queries frequently included grammatically incorrect sentence structures (in both Hindi and English), spelling errors, and the use of terms that may be local to the context rather than a medical term.
By design, the chatbot did not account for regional dialects or script variations in the user's prompt. Before the generation of the response with the LLM, another prompt was fed to the LLM to ``correct'' the spelling and grammatical errors in the query.
This approach resulted in the chatbot being able to generate medically accurate answers in most cases, but it could occasionally result in misunderstandings of user queries if context was lost through this approach. For instance, the chatbot struggled to understand words with slight differences in spelling or meaning. Dr. Sheela pointed out one example in a WhatsApp conversation with the Myna team: \textit{``English response to Hindi question.the bot failed to distinguish the words `rokna' and `rukna' (miscarriage and abortion)''} showing that the chatbot's language processing capabilities are not efficient in handling nuanced context-specific meanings. %As a result, chatbot misinterpreted the user query, and provided inaccurate and irrelevant responses. 
%We observed that there are instances where the chatbot used English words in its responses even when the user’s query is entirely in Hindi. We noted that though the response is accuarte in terms of content, contextual and cultural references were not captured when response is provided in English.

Our analysis also indicated that the chatbot tended to interpret questions literally, without considering the user's background or context.
For instance, Amar (a chatbot developer at Myna) shared his observation over a WhatsApp conversation---\textit{``baccha kyu nahi rukta'' has been interpreted as `why doesn't baby stop crying' when the actual intention of the user was to ask `why Can't I get pregnant' ''}. This gap between the user's intentions and how they framed their query could result in an incorrect response by the chatbot. Our focus groups with Myna revealed that in an early effort to address such issues, they tried fine-tuning using the chatbot by providing a developed question-answer dataset as training data. However, the poor performance of the model and the relative financial and computational cost led the team to continue to use the RAG approach and focus on improving intent detection and the knowledge base.

We also noticed that the chatbot did not always correctly understand that a question relating to socioeconomic status was being asked, misinterpreting the user's intent and providing an inaccurate response. For example, Table \ref{tab:emp} shows that the user query on \textit{``How important is} my \textit{job for family planning''} resulted in the chatbot responding how it can play an important role in family planning. 
This issue stemmed from an error in the translation module of the system, which translated the user query to \textit{``How important is }your\textit{ job for family planning''}. 
While the multi-stage system design that involved first improving grammar and then translating and generating text in English was effective in most cases, it also occasionally introduced other errors in the process.

To improve the system design, the software development team at Myna introduced a localization module in the system, which involved maintaining a dictionary where medical terms generated by the LLM were replaced with local terms instead. 
Beyond the use of medical terms, we also found that the word choice of the chatbot at times was not always appropriate, even if the chatbot had interpreted the user question accurately and the response as a whole was medically accurate.
This concern came up with respect to a chatbot response to a question on contraceptive sponges. 
The chatbot responded with: {\textit{``Garbhnirodhak sponge ek soft, disk-shaped, polyurethane foam ka tukda hota hai jo spermicide se bhara hota hai. Yeh bhi yoni mein daala jata hai aur yeh shukranuon ko rokne ke saath-saath unko marne ka kaam bhi karta hai.''}} (which translates to \textit{``A contraceptive sponge is a soft, disk-shaped piece of polyurethane foam that is filled with spermicide. It is also inserted into the vagina and works both by blocking sperm and by killing them.''}. 
Dr. Preeti shared that the chatbot should have used \textit{``Istemal''} (which translates to ``use'' in English) instead of \textit{``daala jata hai''} (which translates to ``is put''). The latter could be perceived as being too explicit.
% despite conveying similar meanings.

% % While the latter may be used occasionally 
% In general, we observed that the chatbot had not been designed to respond to social norms like greeting the user before starting a conversation. It would immediately start to respond to the user's prompt without distinguishing terms that are considered respectful or disrespectful in various cultures.
% \textbf{

% For example, when a user asked a question about family planning, the chatbot responded directly with `\textit{Nahi, family planning ke nirnay lene wale sirf education wale log nahi hote. Family planning har vyakti ke liye mahatvapurna hai, chahe ve shahar mein rahein ya gaon mein.}''. This response had relevant information but missed culturally appropriate greeting, which is essential in communication in many contexts.
%\subsection{Improving Accessibility for a Broad Audience}
\begin{table*}[!h]
  \centering
   \fontsize{7}{8}\selectfont
  \begin{tabular}{>{\raggedright\arraybackslash}p{1.5cm} p{2cm} p{2.5cm} p{8cm}}
    \toprule
   \textbf{Topic} & \textbf{Language} & \textbf{User Query} & \textbf{Chatbot Response} \\
        \midrule
 \multirow{2}{*}\chiadd{Health Literacy} &English (Translated) & Does Saheli tablet change the date of periods? &
Saheli, also known as Centchroman, is a non-steroidal oral contraceptive pill. Sometimes this can lead to changes in the menstrual cycle, due to which the dates of periods can change. This is because it can affect the hormones that regulate the menstrual cycle. If you notice any significant changes in your cycle or have any concerns, it is important to discuss this with your healthcare provider. Remember, every woman's body reacts differently to contraceptives, so it is beneficial to keep track of your cycle and note any changes. If you feel uncomfortable or have any questions, please consult a Myna's Telehealth doctor.\\
\cmidrule{2-4}
 &Hinglish & Saheli tablet se periods ka date badal jata hai kya? & Saheli, jo Centchroman ke naam se bhi jaani jaati hai, ek non-steroidal oral contraceptive pill hai.
Kabhi-kabhi yeh menstrual cycle mein parivartan laa sakta hai, jiske karan period ki dates mein
badlav ho sakta hai. Iska yeh wajah hai ki yeh hormones par asar daal sakta hai jo menstrual cycle ko
niyantrit karte hain. Agar aapko apne cycle mein koi bhi bada parivartan mehsoos ho ya koi bhi chinta
ho, toh isko healthcare provider ke saath discuss karna mahatvapurn hai. Yaad rakhein, har mahila ke
sharir ka prativarsh contraceptives ke prati alag prakar se react karta hai, isliye apne cycle ka dhyan
rakhna aur kisi bhi parivartan ko note karna faydemand hai. Agar aapko asamanata mehsoos ho rahi
hai ya aur koi sawaal ho, toh kripya Myna’s Telehealth ke doctor se paramarsh karein.\\

\midrule
 \multirow{2}{*} \chiadd{Sterilization}&English (Translated) & Can vasectomy fail after 3 years? & The chances of a vasectomy failing are very low, but technically, it is possible that a vasectomy may
fail even after several years. Its main function is recanalization, in which the separated parts of the
vasa deferens start reattaching themselves and thankfully beginning to meet the semen. However,
this is very rare, especially if the initial tests after vasectomy are not found successfully.
If you are concerned about conceiving after a vasectomy, it would be appropriate to have a semen
analysis done. This test will help you confirm whether the vasectomy is still effective.
Do you need any further information about the long-term effectiveness of vasectomy?\\
\cmidrule{2-4}
&Hinglish &Kya 3 sal bad purush nasbandi fail ho sakti hai?&Vasectomy ke fail hone ke chances bahut hi kam hote hain, lekin technically, yeh sambhav hai ki
vasectomy ke kai saal baad bhi fail ho sakta hai. Iska mukhya karan hota hai recanalization, jisme vasa
deferens ke kata hua hisse apne aap phir se jud jaate hain aur shukranu veerya mein milne lagte hain.
Lekin, aisa hona bahut hi durlabh hai, khaas taur par agar vasectomy ke baad ke shuruaati veerya
parikshan mein shukranu nahi paaye gaye ho.
Yadi aapko vasectomy ke baad garbh dharan ki chinta hai, to veerya parikshan (semen analysis)
karwana uchit hoga. Yeh jaanch aapko yeh pushti karne mein madad karegi ki vasectomy abhi bhi
prabhavi hai ya nahi.
Kya aapko vasectomy ke long-term prabhavi hone ya isse judi aur koi jaankari chahiye?\\
\bottomrule
  \end{tabular}
%    \vspace{10pt}
  \caption{\textbf{Understanding Information-seeking behaviors around SRH}. An instance of the chatbot's use of several complex medical terms and chatbot response to a question on the vasectomy.}
  \label{tab:literacy}
  \Description{Table shows an instance of the chatbot’s use of several complex medical terms and chatbot response to a question on vasectomy with both original Hinglish and English text. }
\end{table*} 

\subsubsection{Being responsive to a range of health and digital literacy levels}
From our interviews and by analyzing log data, we found that health and digital literacy played an important role in how the users understood the responses and interacted with the chatbot. 
Healthcare professionals at Myna expressed a preference for audio responses over textual responses considering users' digital literacy. However, several issues were observed with the audio output of the bot. Words that are commonly not found in English, such as the name of the NGO itself (which was in Hindi), were pronounced incorrectly. 
Some complex medical terms were omitted entirely in the audio responses such as ``Etonogestrel''.
We also observed that chatbot responses frequently consisted of complex medical terms.
The user interaction in Table \ref{tab:literacy}, \chiadd{under the health literacy topic}, shows how the chatbot used complex medical terms like ``non-steroidal'', ``menstrual cycle'', ``hormones'', ``contraceptives''.
Dr. Preeti also shared this concern, and observed that the chatbot sometimes used very formal words in Hindi that are not commonly used and understood by all the users:
% There are instances where the chatbot used English words in its responses even when the user’s query is entirely in Hindi
 % and a lower level of comprehension among users. 

\begin{quote}
``like doctor is called `swaasthya seva pradaata', expert advice is called `peshevar salah', medically accurate is called `chikitsakiya tareeke'. these would not be understandable by people''. 

—Dr. Preeti (Health Care Professional)
\end{quote} 

Another example in Table \ref{tab:literacy} \chiadd{on sterilization} shows how the chatbot has used a different set of medical terms in its response, instead of using the same local terms that the user employed in their question. 
This can impact the user's understanding of the complete response, as they may not recognize the terms used. 
% Initially, chatbot was not able to simplify complex health information for users who may have limited education or understanding of medical concepts. 
Over time, with the introduction of local words for medical terms in the dictionary and the expansion of the knowledge base by the Myna team, the chatbot was able to replace complex terms with simple terms improving comprehensibility.

% The chatbot also used medical terms in the responses without any explanation, assuming the user had a high level of health literacy. 




%\subsubsection{\chiadd{Providing actionable information} \chirm{Considering role of socioeconomic status}}
% From the user interaction data, class can be defined by various Cultural factors such as education, income, and employment status. 
% \chiadd{Actionable information is essential for improving health outcomes, as it connects awareness of risks with the ability to take meaningful actions to address them. Specifically for underserved populations, access to actionable information must address unique challenges, such as limited access to resources like affordable healthcare options and health education programs.}

% Our findings showed that the chatbot did not always take into account the socioeconomic status of participants, which could shape access to healthcare services. 
% Some of the services or practices  (e.g. IVF) suggested by the chatbot could be quite expensive and impractical for participants.
% % Despite providing a factual response, the chatbot did not address the underlying concern of access to medical services based on income. 
% % Access to medical services is highly dependent on income and employment status. 
% One user specifically asked a question on whether family planning can be done if there is no money. The chatbot responded by saying---\textit{``Of course, it does not matter whether you have money or not for family planning. Many measures can be beneficial for you.''} However, it did not provide tailored information on free or low-cost family planning resources available through government or non-profit initiatives. % \cite{article}



% \subsubsection{Presenting affordable and locally-relevant food choices}
% Despite recommending food and diet choices based on general Indian preferences. 


\subsection{Taking Social and Cultural Norms into Account}
% \chiadd{There is also the notion of what constitutes a misconception versus a belief. Misconceptions, often rooted in factual inaccuracies, and beliefs, shaped by personal, cultural, or religious values, may require either correction or alignment, depending on the context. Our findings highlight the complexity of navigating these dynamics.} 
We now focus on how the chatbot engaged with users' social and cultural contexts when responding to queries. We discuss the need for balancing medical accuracy with respect for cultural beliefs, when providing information on sensitive topics like family planning and sexual health. 

\subsubsection{Creating a space to discuss cultural taboos}
Our focus groups with the \chiadd{women from the community}\chirm{workers} revealed that the chatbot offered a channel to ask questions that they might not have asked at all otherwise. One worker shared that, \textit{``Before this work, I had not thought about these questions. But because of this job, I started to try to come up with questions and ask questions from my sister, in-laws, and neighbors, and look online for ideas of questions to ask. It was very difficult to come up with them.''}
% In many conservative cultures, women prefer to see female doctors for sensitive health issues because they feel more comfortable discussing SRH topics with them. 
% This preference is often due to cultural taboo around modesty and potential belief that it is inappropriate to discuss SRH issues with someone of the opposite gender. The comfort associated with preserving modesty and adhering to cultural expectations of privacy significantly influences the choice of a female doctor over a male doctor, a sentiment that was echoed by 
Dr. Pooja also reflected on her years of experience working with marginalized women in the community---\textit{``People like to talk to female doctors because they can't talk to male doctors more openly but can talk to these apps''}. She continued to add, \textit{``Young married women aged 18 to 30 said they want to speak up about SRH, but it's very difficult for them to get those words. They feel uncomfortable. Few of them mentioned that they would only ask their husbands about such matters, as talking to neighbors was difficult due to concerns about privacy, shame, and stigma around it''.} The analysis of the questions asked presented in Table \ref{tab:type} earlier, also indicates that the design of the chatbot was largely effective in eliciting questions on deeply personal, sensitive, and sometimes controversial topics.

%significant challenges faced by users, particularly young married women aged 18 to 30, when discussing sensitive topics like family planning and reproductive health. Many users experience shame, discomfort, and fear of judgment, leading them to avoid conversations about these issues, even with close family members including mothers and sisters. This cultural taboo and desire for privacy often result in users limiting their discussions to only their husbands, in some cases, not seeking advice at all. 
Our analysis showed that the chatbot maintained a neutral tone when addressing queries associated with cultural taboo (see Table \ref{tab:Baby Sex} in Appendix), which helped in creating a comfortable environment for users, encouraging them to discuss sensitive topics that might otherwise be avoided in real-world interactions. This also allowed users to seek information or advice from the chatbot without fear of judgment. For example, one of the user questions was---\textit{``I don’t know how to have sex. I have just recently gotten married''}. This indicates the users' comfort level with the chatbot to seek information on sensitive topics. 

\subsubsection{Correcting misconceptions while respecting cultural beliefs} 
Our findings indicated that religious and community beliefs could shape health behaviors. 
We identified queries around beliefs that stemmed from misconceptions, and both harmful and non-harmful community practices. For instance, a user asked a question on preferential sex selection, as seen in Table \ref{tab:Baby Sex}. Prior work by Neogi et al. has examined how Indian women use certain medical and non-medical techniques with the belief that these methods could influence the sex of their child \cite{Neogi2015ConsumptionOI}. The chatbot only provided a response that was medically accurate and scientific. It did not recommend any unsafe and unproven techniques to further boost the user's beliefs.

\begin{table*}[h]
  \centering
     \fontsize{7}{8}\selectfont
  \begin{tabular}{>{\raggedright\arraybackslash}p{1.5cm}p{2.5cm} p{10cm}}
    \toprule
  \textbf{Topic} & \textbf{User Query} & \textbf{Chatbot Response} \\
        \midrule
   \chiadd{Misconceptions and Gender Norms} & In family planning, I only want a boy, Are there any methods for that??
 &
 There is no guaranteed way to choose the gender of a baby. The main objective of family planning is to control pregnancy, not to select a child's gender. Some people try different methods, such as timing and specific techniques, to predict whether they will have a boy or a girl, but these methods are not scientifically proven and have no reliable basis. If you are considering family planning options, it is important to focus on methods that are best suited for you and your family. If you would like more information or need any specific advice, please consult a doctor with Myna's Telehealth. \\
  \midrule
  \chiadd{Cultural taboo} & Why shouldn't the word 'sex' be used in front of family when discussing family planning? &
In every family and community, communication styles vary. In some families, the term 'sex' is not used openly because it may be uncomfortable for them. This can be due to social or cultural beliefs. When it comes to family planning, people often discuss the topic in a more considerate manner, using terms like 'making relations', 'family planning', or 'planning for children'. It is important to respect the feelings of your family members when discussing this topic and to choose words that are comfortable for everyone. This helps in having open and meaningful conversations without making anyone feel uncomfortable.\\
\midrule
 \chiadd{Dietary recommendations} & What to eat to heal family planning sterilization wounds &
   Eating a nutritious diet is essential for proper wound healing after sterilization. Here are some dietary recommendations that can help in recovery: 
1. Protein rich diet: Protein is important for wound healing. Eat protein-rich foods like dal, chana, soybean, paneer, curd, eggs, chicken, and fish.
2. Vitamin C and Zinc: Both these dietary elements help in wound healing. Oranges, lemons, kiwi, plums, bell peppers, walnuts, and seeds should be consumed in the diet.
3. Green Vegetables: Spinach, fenugreek, sarson, and other green vegetables are rich in vitamins and minerals that help in tissue repair.
4. Fruits: Fresh fruits like apple, banana, papaya, and berries are rich in antioxidants which keep the body healthy.
5. Adequate water: Drink adequate amount of water because hydration is necessary for wound healing.
6. Whole Grains: Whole grains like wheat, barley, and oats are rich in fiber which keeps the digestive system healthy.
In addition to this diet, diets containing alcohol, caffeine, and other sugars should be avoided as they can slow down the wound healing process. Do you need any more suggestions or information? \\
\midrule
\chiadd{Religion} & What to do if surgery is prohibited in religion? &
    If your religion or beliefs are against operations or surgical procedures, you should consider some non-surgical birth control methods. There are some non-surgical methods:
1. Condoms: This is an effective method that protects against pregnancy and sexually transmitted infections (STIs).
2. Oral contraceptive pills: Taken daily by women to prevent pregnancy.
3. Vaginal rings: These are inserted into the vagina for a month at a time.
4. Contraceptive patches: These are patches applied on the skin which have to be changed weekly.
5. Natural family planning methods: Like ovulation tracking and withdrawal method.
Any of these options may be right for you. But, before using them, it is important to understand that some methods, such as natural family planning, may be less effective. 
You should seek guidance from your religious leader or community as well as a doctor. You can also contact ``Myna's Telehealth'' where you can consult a doctor who can give you the right guidance taking into account your religious beliefs.\\
    \bottomrule
  \end{tabular}
%%    \vspace{10pt}
\caption{\textbf{Taking Social and Cultural Norms into Account.} Chatbot response to questions on cultural beliefs, taboo, food recommendations and role of religion. Find the original
Hinglish text in Table \ref{tab:stigma} (in the Appendix)}
    \label{tab:Baby Sex}
    \Description{Chatbot response to user questions on cultural beliefs, taboos,
food recommendations and the role of religion are presented in the table with English text. The original Hinglish text is referenced to Table \ref{tab:stigma}}
\end{table*} 

In one instance, a user asked if condoms cause HIV. The chatbot responded: \textit{``Yes, it is important to use a condom every time you have sex to protect against HIV and other sexually transmitted diseases.''}. Though this was scientifically accurate, it did not address the underlying misconception of the user that condoms do not cause HIV and are, in fact, a preventive measure. This may have been a missed opportunity to correct misinformation. Identifying when a user's question reveals a misconception is critical to address any harmful myths that may otherwise lead to uninformed health decisions.

We also found cultural beliefs come up in other ways, such as with dietary practices. However, the chatbot did not consider the specific dietary preferences of individuals when responding to questions on this topic. The user interaction on food recommendations listed in Table \ref{tab:Baby Sex} shows that while the recommended food choices were Indian, they were quite broad and did not consider the user's specific dietary needs or preferences and affordability. Given India's diversity, dietary choices are often influenced by factors such as religion, region, festivals, and socioeconomic status. 

\subsubsection{\chiadd{Providing relevant healthcare service suggestions} \chirm{Considering the role of religion}}
We also found that the sociocultural context could shape what healthcare services were considered acceptable by users. We present two examples below where this was apparent---in relation to religious beliefs and socioeconomic status. 

In the example presented in Table \ref{tab:Baby Sex}, the user asks a question about what to do if surgery (sterilization) is prohibited in one's religion. Rather than rebutting this claim, the chatbot offers alternative non-surgical methods and also suggests seeking guidance from a religious leader as well as a doctor.
% we observed that the chatbot placed religion and the benefit of the user at the same level, which reflects a culturally sensitive approach. However, it does not establish a hierarchy that prioritizes safety over religious considerations. 
In another user query on the acceptability of family planning in the Muslim community, the chatbot responded by saying:{\textit{``Absolutely, family planning is necessary in every social community, be it Muslim community or any
other. This is something every family should think about, because it affects not only your health,
but also your children’s care and financial status. There are some easy and permissible methods for
family planning in Islam, such as temporary methods such as condoms or permanent methods such as
tubectomy or vasectomy. What do you think, do you need more information about family planning?.''}}
% (which translates to \textit{``Yes, it is important to use a condom every time you have sex to protect against HIV and other sexually transmitted diseases.''}.
While the response did not get into the specifics of which religious scriptures indicate whether or not family planning is permissible, it advocates more generically for family planning in every community. 
% nstead, it provided a more generic response.
Once again the response suggests talking to a religious or community leader.
While the balance between sharing medical information and religious beliefs was handled diplomatically in the two examples, there are situations where conflicts could emerge.
% this could present . 
For instance, in the case that a religious practice or belief is harmful or medically inaccurate, then the safety of the user is critical, and medical advice should be emphasized.

Our findings showed that the chatbot did not always take into account the socioeconomic status of participants, which could shape access to healthcare services. 
Some of the services or practices  (e.g. IVF) suggested by the chatbot could be quite expensive and impractical for participants.
% Despite providing a factual response, the chatbot did not address the underlying concern of access to medical services based on income. 
% Access to medical services is highly dependent on income and employment status. 
One user specifically asked a question on whether family planning can be done if there is no money. The chatbot responded by saying---\textit{``Of course, it does not matter whether you have money or not for family planning. Many measures can be beneficial for you.''} However, it did not provide tailored information on free or low-cost family planning resources available through government or non-profit initiatives. % \cite{article}


\subsection{Responding to Social Dynamics}
We now address how the chatbot responded to queries involving social dynamics. We highlight the challenges in interpreting the social and emotional contexts of user queries, such as family pressure around pregnancy or the complexities of marital relationships.

\subsubsection{The role of legal implications}
The chatbot was inconsistent in responding to legal aspects of user queries, which mostly related to the appropriate age for sex and marriage. 
It generally advised users to follow legal procedures, but did not always offer specific information on legalities applicable to that context. 
For instance, Table \ref{tab:planning} presents a response to the user question---\textit{``What is the proper age for having sex for the first time.''} The chatbot responds by saying that the legal age varies in every country and is between 16 and 18 years, rather than providing the legal age specific to the user context (the age of consent is 18 years in India). 
In another example, however, the chatbot did not address the legal implications related to sex at all.
Table \ref{tab:planning} presents an interaction about whether a 15-year-old can think about family planning, but does not discuss the legal implications of having sex at that age at all.
This was a critical gap that the Myna team attempted to address by improving intent detection and updating the knowledge base with more legal information specific to the study context.
% The inconsistency in the chatbot's ability to recognize and apply context specific legal information should be emphasized in order to maintain system's reliability and trust among users.


\begin{table*}[h]
\centering
      \fontsize{7}{8}\selectfont
  \begin{tabular}{>{\raggedright\arraybackslash}p{1.5cm} p{2.5cm} p{10cm}}
    \toprule
      \textbf{Topic} & \textbf{User Query} & \textbf{Chatbot Response} \\
      \midrule
     \chiadd{Age of sexual maturity} & What is the proper age to have sex for the first time? &
The concept of `proper age' for sex depends on legal and emotional maturity. In every country, the legal age of having sex (also known as the `age of consent') varies, and is usually between 16 and 18 years. This legal age is required to ensure that a person is emotionally and physically mature enough to understand his/her decisions and handle their consequences. But, merely being of legal age is not enough. It is also important that the person is emotionally ready, has the right information, and is ready to have a healthy and balanced relationship with his/her partner. It is also important to understand the importance of safe sex and consent. Therefore, the `right time' to have sex for the first time depends on the person's own thoughts, feelings, and circumstances. If someone needs more help or information in this matter, it would be wise to talk to a trusted doctor or healthcare provider.\\
\midrule 
    \chiadd{Family planning} & Can a 15 year old girl do family planning?
 &
Yes, a 15 year old girl can think about family planning, but in this age, mostly the focus will be on education and personal development. If family planning is needed, you should first consider non-invasive methods such as barrier methods (condom use) or oral contraceptives (pills). Both of these methods are safe and reversible, meaning you can stop using them whenever you want. But, before taking any decision, it is important to consult a specialist or doctor of Myna's Telehealth. His guidance will guide you in the right direction.\\
  \midrule
   \chiadd{Divorce} & If I don’t stay with my husband after marriage, is it right to get a
divorce? &
  If you are no longer able to get along with your husband after marriage and you feel that there are problems between you, then it is important to first understand what the problem is. Every relationship is different and sometimes it can be difficult to understand or find a solution. But, it is also important that both of you talk to each other openly and try to find a solution to the problem. If you feel that you have tried everything and still cannot find a solution, then you should listen to your heart. This is a very big decision, so take time to think about it and also consult a specialist or counselor if necessary. Your happiness and health come first. \\
   \midrule
   \chiadd{Gender roles} & Why don't men do vasectomy? &
Male sterilization, or vasectomy, is a permanent contraceptive procedure that is very safe and effective. But, sometimes men do not undergo this procedure due to many reasons:

1. Lack of information: Many men do not have the correct knowledge about this procedure.
2. Misconceptions: Some men believe that sterilization will affect their sexual strength or masculinity, which is incorrect.
3. Social Pressure: In some communities, there is a belief that sterilization is only meant for women.
4. Fear: The fear of surgery or medical procedures can also be a factor.

Vasectomy does not reduce a man's sexual ability nor does it cause any changes in male hormones. This is a safe and less painful procedure. If you or someone you know is interested and would like more information, please contact your local health center or doctor. Would you like to ask anything else about this?\\
  \bottomrule
  \end{tabular}
%    \vspace{10pt}
    \caption{\textbf{Responding to Social Dynamics.} Chatbot responses to questions on legal implications, and gender roles. Find the original Hinglish text in Table \ref{tab:mar} (in the Appendix)}
      \label{tab:planning}
      \Description{Chatbot response to user questions on legal implications, and gender roles is presented in the table with English text. The original Hinglish text is referenced to Table \ref{tab:mar}}
\end{table*}

\subsubsection{Considering family and partner dynamics} 
We observed that the system currently is not able to understand the familial context in the user prompt. It focuses solely on the medical terms present in the prompt without heeding attention to the family context. From a WhatsApp chat conversation, we noted that there was a user query---\textit{``I am newly married, I don't want to get pregnant yet but my family is pressuring me to have children, so what should I do?''} The chatbot responded with information on methods of preventing pregnancy rather than addressing the user’s concern about handling family pressure regarding pregnancy. 
% The response should have been more aligned with providing guidance on managing family expectations and addressing emotional aspects related to the situation. 

%\subsubsection{Navigating relationship with the partner}
Our analysis also showed a potential limitation of the chatbot in providing personalized advice on important life decisions. In the case of a user question about divorce (Table \ref{tab:planning}), the chatbot ignored the context provided by the user in the query that the user's husband is not living with her and provided a generic response recommending improved communication or counseling. Counseling might not be accessible, affordable, or culturally acceptable in this context.
By not asking follow-up questions, the chatbot missed the opportunity to get more information on the dynamic with the partner and potentially provide more emotional support. 
We also note that this question went beyond the focus on SRH and focused on well-being in a relationship more broadly, pointing to other potential gaps in information access and socioemotional support among participants.
% Evaluating the specifics of the user context should be emphasized, considering users might make informed decisions based on the response. 
\subsubsection{Dealing with gender bias} 
Our analysis uncovered that the chatbot did not reinforce traditional gender roles in the home and around family planning.
% Our analysis uncovered a few inconsistencies in how the chatbot when responded to gender-related queries.
% First, we observed that the chatbot gave generic responses to topics like family planning without accounting for cultural variations in gender roles. 
For example, in one of the user's queries about the women's role in family planning, the response of the chatbot was: \textit{``No, the responsibility of family planning does not rest solely on women. It depends on the understanding and cooperation of both partners. Men can also contribute to family planning, such as by using
condoms, opting for vasectomy (male sterilization), or supporting their partner’s family planning
decisions. Family planning should be a shared decision, in which both partners participate and agree.
Do you need any further help on this topic?''}. 
% It gave medically accurate information but overlooked the cultural context in which gender roles may have an impact on family planning decisions.
This approach was by design as a result of Myna's focus on supporting women's agency and empowerment in healthcare decisions, as an organization. 

% \textbf{Further analysis showed that it maintained gender neutrality in responding to gender-related queries. It avoided making any assumptions and responded with consideration for all genders. Despite lacking gender specificity in some responses, 
This extended to also being able to respond to questions about why gendered expectations exist around reproductive health in the first place.
For example, a user asked a query about why men do not undergo vasectomy. The chatbot was able to acknowledge societal pressures and misconceptions around gender roles, particularly when discussing topics like vasectomy. It acknowledged social pressure as a possible reason why some men might hesitate to undergo surgery (see Table \ref{tab:planning}), and emphasized that ``manliness'' does not reduce as a result of undergoing a vasectomy. 
 % These findings indicate a gap in the system's ability to respond based on culturally specific gender norms.



