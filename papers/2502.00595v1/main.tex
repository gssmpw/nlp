%%%%%%%% ICML 2025 EXAMPLE LATEX SUBMISSION FILE %%%%%%%%%%%%%%%%%

\documentclass{article}

% Recommended, but optional, packages for figures and better typesetting:
\usepackage{microtype}
\usepackage{graphicx}
\usepackage{subfigure}
\usepackage{booktabs} % for professional tables

% hyperref makes hyperlinks in the resulting PDF.
% If your build breaks (sometimes temporarily if a hyperlink spans a page)
% please comment out the following usepackage line and replace
% \usepackage{icml2025} with \usepackage[nohyperref]{icml2025} above.
\usepackage{hyperref}
\usepackage{enumitem}

% Attempt to make hyperref and algorithmic work together better:
\newcommand{\theHalgorithm}{\arabic{algorithm}}
% \usepackage{algorithm}

% Use the following line for the initial blind version submitted for review:
% \usepackage{icml2025}
% If accepted, instead use the following line for the camera-ready submission:
\usepackage[accepted]{icml2025}

% For theorems and such
\usepackage{amsmath}
\usepackage{amssymb}
\usepackage{mathtools}
\usepackage{amsthm}

% if you use cleveref..
\usepackage[capitalize,noabbrev]{cleveref}

%%%%%%%%%%%%%%%%%%%
% Custom packages
%%%%%%%%%%%%%%%%%%%%
\usepackage[T1]{fontenc}
\usepackage[utf8]{inputenc}
\usepackage{listings}
\usepackage{xcolor}
\usepackage{multirow}

\newcommand{\benchmark}{\textsc{RPGBench}}

% Define a custom style for JSON
\lstdefinelanguage{json}{
    showstringspaces=false,
    breaklines=true,
    frame=single,
    basicstyle=\ttfamily\footnotesize,
    stringstyle=\color{brown},
    numberstyle=\tiny\color{gray},
    keywordstyle=\color{blue}\bfseries,
    commentstyle=\color{gray},
    morestring=[b]"
}
\lstdefinelanguage{plaintext}{
    showstringspaces=false,
    breaklines=true,
    frame=single,
    basicstyle=\ttfamily\footnotesize,
    % morestring=[b]"
}

\lstset{
  tabsize=2,
  numbers=left,
  stepnumber=1,
  firstnumber=1,
  numbersep=6pt
}

%%%%%%%%%%%%%%%%%%%%%%%%%%%%%%%%
% THEOREMS
%%%%%%%%%%%%%%%%%%%%%%%%%%%%%%%%
\theoremstyle{plain}
\newtheorem{theorem}{Theorem}[section]
\newtheorem{proposition}[theorem]{Proposition}
\newtheorem{lemma}[theorem]{Lemma}
\newtheorem{corollary}[theorem]{Corollary}
\theoremstyle{definition}
\newtheorem{definition}[theorem]{Definition}
\newtheorem{assumption}[theorem]{Assumption}
\theoremstyle{remark}
\newtheorem{remark}[theorem]{Remark}

% Todonotes is useful during development; simply uncomment the next line
%    and comment out the line below the next line to turn off comments
%\usepackage[disable,textsize=tiny]{todonotes}
\usepackage[textsize=tiny]{todonotes}

% The \icmltitle you define below is probably too long as a header.
% Therefore, a short form for the running title is supplied here:
\icmltitlerunning{RPGBench}

\begin{document}

\twocolumn[
\icmltitle{\benchmark{}: 
Evaluating Large Language Models as Role-Playing Game Engines}

% It is OKAY to include author information, even for blind
% submissions: the style file will automatically remove it for you
% unless you've provided the [accepted] option to the icml2025
% package.

% List of affiliations: The first argument should be a (short)
% identifier you will use later to specify author affiliations
% Academic affiliations should list Department, University, City, Region, Country
% Industry affiliations should list Company, City, Region, Country

% You can specify symbols, otherwise they are numbered in order.
% Ideally, you should not use this facility. Affiliations will be numbered
% in order of appearance and this is the preferred way.
\icmlsetsymbol{equal}{*}

\begin{icmlauthorlist}
\icmlauthor{Pengfei Yu}{comp}
\icmlauthor{Dongming Shen}{comp}
\icmlauthor{Silin Meng}{comp}
\icmlauthor{Jaewon Lee}{comp}
\icmlauthor{Weisu Yin}{comp}
\icmlauthor{Andrea Yaoyun Cui}{sch}
\icmlauthor{Zhenlin Xu}{comp}
%\icmlauthor{}{sch}
\icmlauthor{Yi Zhu}{comp}
\icmlauthor{Xingjian Shi}{comp}
\icmlauthor{Mu Li}{comp}
\icmlauthor{Alex Smola}{comp}
%\icmlauthor{}{sch}
%\icmlauthor{}{sch}
\end{icmlauthorlist}

% \icmlaffiliation{yyy}{Department of XXX, University of YYY, Location, Country}
\icmlaffiliation{comp}{Boson AI}
\icmlaffiliation{sch}{University of Illinois Urbana Champaign}

\icmlcorrespondingauthor{Pengfei Yu}{pengfei@boson.ai}

% You may provide any keywords that you
% find helpful for describing your paper; these are used to populate
% the "keywords" metadata in the PDF but will not be shown in the document
\icmlkeywords{Role-Playing, Evaluation, Benchmark}

\vskip 0.3in
]

% this must go after the closing bracket ] following \twocolumn[ ...

% This command actually creates the footnote in the first column
% listing the affiliations and the copyright notice.
% The command takes one argument, which is text to display at the start of the footnote.
% The \icmlEqualContribution command is standard text for equal contribution.
% Remove it (just {}) if you do not need this facility.

\printAffiliationsAndNotice{}  % leave blank if no need to mention equal contribution
% \printAffiliationsAndNotice{\icmlEqualContribution} % otherwise use the standard text.

\begin{abstract}
We present an image blending pipeline, \textit{IBURD}, that creates realistic synthetic images to assist in the training of deep detectors for use on underwater autonomous vehicles (AUVs) for marine debris detection tasks. 
Specifically, IBURD generates both images of underwater debris and their pixel-level annotations, using source images of debris objects, their annotations, and target background images of marine environments. 
With Poisson editing and style transfer techniques, IBURD is even able to robustly blend transparent objects into arbitrary backgrounds and automatically adjust the style of blended images using the blurriness metric of target background images. 
These generated images of marine debris in actual underwater backgrounds address the data scarcity and data variety problems faced by deep-learned vision algorithms in challenging underwater conditions, and can enable the use of AUVs for environmental cleanup missions. 
Both quantitative and robotic evaluations of IBURD demonstrate the efficacy of the proposed approach for robotic detection of marine debris. 
\end{abstract}



\section{Introduction}

In today’s rapidly evolving digital landscape, the transformative power of web technologies has redefined not only how services are delivered but also how complex tasks are approached. Web-based systems have become increasingly prevalent in risk control across various domains. This widespread adoption is due their accessibility, scalability, and ability to remotely connect various types of users. For example, these systems are used for process safety management in industry~\cite{kannan2016web}, safety risk early warning in urban construction~\cite{ding2013development}, and safe monitoring of infrastructural systems~\cite{repetto2018web}. Within these web-based risk management systems, the source search problem presents a huge challenge. Source search refers to the task of identifying the origin of a risky event, such as a gas leak and the emission point of toxic substances. This source search capability is crucial for effective risk management and decision-making.

Traditional approaches to implementing source search capabilities into the web systems often rely on solely algorithmic solutions~\cite{ristic2016study}. These methods, while relatively straightforward to implement, often struggle to achieve acceptable performances due to algorithmic local optima and complex unknown environments~\cite{zhao2020searching}. More recently, web crowdsourcing has emerged as a promising alternative for tackling the source search problem by incorporating human efforts in these web systems on-the-fly~\cite{zhao2024user}. This approach outsources the task of addressing issues encountered during the source search process to human workers, leveraging their capabilities to enhance system performance.

These solutions often employ a human-AI collaborative way~\cite{zhao2023leveraging} where algorithms handle exploration-exploitation and report the encountered problems while human workers resolve complex decision-making bottlenecks to help the algorithms getting rid of local deadlocks~\cite{zhao2022crowd}. Although effective, this paradigm suffers from two inherent limitations: increased operational costs from continuous human intervention, and slow response times of human workers due to sequential decision-making. These challenges motivate our investigation into developing autonomous systems that preserve human-like reasoning capabilities while reducing dependency on massive crowdsourced labor.

Furthermore, recent advancements in large language models (LLMs)~\cite{chang2024survey} and multi-modal LLMs (MLLMs)~\cite{huang2023chatgpt} have unveiled promising avenues for addressing these challenges. One clear opportunity involves the seamless integration of visual understanding and linguistic reasoning for robust decision-making in search tasks. However, whether large models-assisted source search is really effective and efficient for improving the current source search algorithms~\cite{ji2022source} remains unknown. \textit{To address the research gap, we are particularly interested in answering the following two research questions in this work:}

\textbf{\textit{RQ1: }}How can source search capabilities be integrated into web-based systems to support decision-making in time-sensitive risk management scenarios? 
% \sq{I mention ``time-sensitive'' here because I feel like we shall say something about the response time -- LLM has to be faster than humans}

\textbf{\textit{RQ2: }}How can MLLMs and LLMs enhance the effectiveness and efficiency of existing source search algorithms? 

% \textit{\textbf{RQ2:}} To what extent does the performance of large models-assisted search align with or approach the effectiveness of human-AI collaborative search? 

To answer the research questions, we propose a novel framework called Auto-\
S$^2$earch (\textbf{Auto}nomous \textbf{S}ource \textbf{Search}) and implement a prototype system that leverages advanced web technologies to simulate real-world conditions for zero-shot source search. Unlike traditional methods that rely on pre-defined heuristics or extensive human intervention, AutoS$^2$earch employs a carefully designed prompt that encapsulates human rationales, thereby guiding the MLLM to generate coherent and accurate scene descriptions from visual inputs about four directional choices. Based on these language-based descriptions, the LLM is enabled to determine the optimal directional choice through chain-of-thought (CoT) reasoning. Comprehensive empirical validation demonstrates that AutoS$^2$-\ 
earch achieves a success rate of 95–98\%, closely approaching the performance of human-AI collaborative search across 20 benchmark scenarios~\cite{zhao2023leveraging}. 

Our work indicates that the role of humans in future web crowdsourcing tasks may evolve from executors to validators or supervisors. Furthermore, incorporating explanations of LLM decisions into web-based system interfaces has the potential to help humans enhance task performance in risk control.






\section{Related Work}
\label{sec:relatedworks}

% \begin{table*}[t]
% \centering 
% \renewcommand\arraystretch{0.98}
% \fontsize{8}{10}\selectfont \setlength{\tabcolsep}{0.4em}
% \begin{tabular}{@{}lc|cc|cc|cc@{}}
% \toprule
% \textbf{Methods}           & \begin{tabular}[c]{@{}c@{}}\textbf{Training}\\ \textbf{Paradigm}\end{tabular} & \begin{tabular}[c]{@{}c@{}}\textbf{$\#$ PT Data}\\ \textbf{(Tokens)}\end{tabular} & \begin{tabular}[c]{@{}c@{}}\textbf{$\#$ IFT Data}\\ \textbf{(Samples)}\end{tabular} & \textbf{Code}  & \begin{tabular}[c]{@{}c@{}}\textbf{Natural}\\ \textbf{Language}\end{tabular} & \begin{tabular}[c]{@{}c@{}}\textbf{Action}\\ \textbf{Trajectories}\end{tabular} & \begin{tabular}[c]{@{}c@{}}\textbf{API}\\ \textbf{Documentation}\end{tabular}\\ \midrule 
% NexusRaven~\citep{srinivasan2023nexusraven} & IFT & - & - & \textcolor{green}{\CheckmarkBold} & \textcolor{green}{\CheckmarkBold} &\textcolor{red}{\XSolidBrush}&\textcolor{red}{\XSolidBrush}\\
% AgentInstruct~\citep{zeng2023agenttuning} & IFT & - & 2k & \textcolor{green}{\CheckmarkBold} & \textcolor{green}{\CheckmarkBold} &\textcolor{red}{\XSolidBrush}&\textcolor{red}{\XSolidBrush} \\
% AgentEvol~\citep{xi2024agentgym} & IFT & - & 14.5k & \textcolor{green}{\CheckmarkBold} & \textcolor{green}{\CheckmarkBold} &\textcolor{green}{\CheckmarkBold}&\textcolor{red}{\XSolidBrush} \\
% Gorilla~\citep{patil2023gorilla}& IFT & - & 16k & \textcolor{green}{\CheckmarkBold} & \textcolor{green}{\CheckmarkBold} &\textcolor{red}{\XSolidBrush}&\textcolor{green}{\CheckmarkBold}\\
% OpenFunctions-v2~\citep{patil2023gorilla} & IFT & - & 65k & \textcolor{green}{\CheckmarkBold} & \textcolor{green}{\CheckmarkBold} &\textcolor{red}{\XSolidBrush}&\textcolor{green}{\CheckmarkBold}\\
% LAM~\citep{zhang2024agentohana} & IFT & - & 42.6k & \textcolor{green}{\CheckmarkBold} & \textcolor{green}{\CheckmarkBold} &\textcolor{green}{\CheckmarkBold}&\textcolor{red}{\XSolidBrush} \\
% xLAM~\citep{liu2024apigen} & IFT & - & 60k & \textcolor{green}{\CheckmarkBold} & \textcolor{green}{\CheckmarkBold} &\textcolor{green}{\CheckmarkBold}&\textcolor{red}{\XSolidBrush} \\\midrule
% LEMUR~\citep{xu2024lemur} & PT & 90B & 300k & \textcolor{green}{\CheckmarkBold} & \textcolor{green}{\CheckmarkBold} &\textcolor{green}{\CheckmarkBold}&\textcolor{red}{\XSolidBrush}\\
% \rowcolor{teal!12} \method & PT & 103B & 95k & \textcolor{green}{\CheckmarkBold} & \textcolor{green}{\CheckmarkBold} & \textcolor{green}{\CheckmarkBold} & \textcolor{green}{\CheckmarkBold} \\
% \bottomrule
% \end{tabular}
% \caption{Summary of existing tuning- and pretraining-based LLM agents with their training sample sizes. "PT" and "IFT" denote "Pre-Training" and "Instruction Fine-Tuning", respectively. }
% \label{tab:related}
% \end{table*}

\begin{table*}[ht]
\begin{threeparttable}
\centering 
\renewcommand\arraystretch{0.98}
\fontsize{7}{9}\selectfont \setlength{\tabcolsep}{0.2em}
\begin{tabular}{@{}l|c|c|ccc|cc|cc|cccc@{}}
\toprule
\textbf{Methods} & \textbf{Datasets}           & \begin{tabular}[c]{@{}c@{}}\textbf{Training}\\ \textbf{Paradigm}\end{tabular} & \begin{tabular}[c]{@{}c@{}}\textbf{\# PT Data}\\ \textbf{(Tokens)}\end{tabular} & \begin{tabular}[c]{@{}c@{}}\textbf{\# IFT Data}\\ \textbf{(Samples)}\end{tabular} & \textbf{\# APIs} & \textbf{Code}  & \begin{tabular}[c]{@{}c@{}}\textbf{Nat.}\\ \textbf{Lang.}\end{tabular} & \begin{tabular}[c]{@{}c@{}}\textbf{Action}\\ \textbf{Traj.}\end{tabular} & \begin{tabular}[c]{@{}c@{}}\textbf{API}\\ \textbf{Doc.}\end{tabular} & \begin{tabular}[c]{@{}c@{}}\textbf{Func.}\\ \textbf{Call}\end{tabular} & \begin{tabular}[c]{@{}c@{}}\textbf{Multi.}\\ \textbf{Step}\end{tabular}  & \begin{tabular}[c]{@{}c@{}}\textbf{Plan}\\ \textbf{Refine}\end{tabular}  & \begin{tabular}[c]{@{}c@{}}\textbf{Multi.}\\ \textbf{Turn}\end{tabular}\\ \midrule 
\multicolumn{13}{l}{\emph{Instruction Finetuning-based LLM Agents for Intrinsic Reasoning}}  \\ \midrule
FireAct~\cite{chen2023fireact} & FireAct & IFT & - & 2.1K & 10 & \textcolor{red}{\XSolidBrush} &\textcolor{green}{\CheckmarkBold} &\textcolor{green}{\CheckmarkBold}  & \textcolor{red}{\XSolidBrush} &\textcolor{green}{\CheckmarkBold} & \textcolor{red}{\XSolidBrush} &\textcolor{green}{\CheckmarkBold} & \textcolor{red}{\XSolidBrush} \\
ToolAlpaca~\cite{tang2023toolalpaca} & ToolAlpaca & IFT & - & 4.0K & 400 & \textcolor{red}{\XSolidBrush} &\textcolor{green}{\CheckmarkBold} &\textcolor{green}{\CheckmarkBold} & \textcolor{red}{\XSolidBrush} &\textcolor{green}{\CheckmarkBold} & \textcolor{red}{\XSolidBrush}  &\textcolor{green}{\CheckmarkBold} & \textcolor{red}{\XSolidBrush}  \\
ToolLLaMA~\cite{qin2023toolllm} & ToolBench & IFT & - & 12.7K & 16,464 & \textcolor{red}{\XSolidBrush} &\textcolor{green}{\CheckmarkBold} &\textcolor{green}{\CheckmarkBold} &\textcolor{red}{\XSolidBrush} &\textcolor{green}{\CheckmarkBold}&\textcolor{green}{\CheckmarkBold}&\textcolor{green}{\CheckmarkBold} &\textcolor{green}{\CheckmarkBold}\\
AgentEvol~\citep{xi2024agentgym} & AgentTraj-L & IFT & - & 14.5K & 24 &\textcolor{red}{\XSolidBrush} & \textcolor{green}{\CheckmarkBold} &\textcolor{green}{\CheckmarkBold}&\textcolor{red}{\XSolidBrush} &\textcolor{green}{\CheckmarkBold}&\textcolor{red}{\XSolidBrush} &\textcolor{red}{\XSolidBrush} &\textcolor{green}{\CheckmarkBold}\\
Lumos~\cite{yin2024agent} & Lumos & IFT  & - & 20.0K & 16 &\textcolor{red}{\XSolidBrush} & \textcolor{green}{\CheckmarkBold} & \textcolor{green}{\CheckmarkBold} &\textcolor{red}{\XSolidBrush} & \textcolor{green}{\CheckmarkBold} & \textcolor{green}{\CheckmarkBold} &\textcolor{red}{\XSolidBrush} & \textcolor{green}{\CheckmarkBold}\\
Agent-FLAN~\cite{chen2024agent} & Agent-FLAN & IFT & - & 24.7K & 20 &\textcolor{red}{\XSolidBrush} & \textcolor{green}{\CheckmarkBold} & \textcolor{green}{\CheckmarkBold} &\textcolor{red}{\XSolidBrush} & \textcolor{green}{\CheckmarkBold}& \textcolor{green}{\CheckmarkBold}&\textcolor{red}{\XSolidBrush} & \textcolor{green}{\CheckmarkBold}\\
AgentTuning~\citep{zeng2023agenttuning} & AgentInstruct & IFT & - & 35.0K & - &\textcolor{red}{\XSolidBrush} & \textcolor{green}{\CheckmarkBold} & \textcolor{green}{\CheckmarkBold} &\textcolor{red}{\XSolidBrush} & \textcolor{green}{\CheckmarkBold} &\textcolor{red}{\XSolidBrush} &\textcolor{red}{\XSolidBrush} & \textcolor{green}{\CheckmarkBold}\\\midrule
\multicolumn{13}{l}{\emph{Instruction Finetuning-based LLM Agents for Function Calling}} \\\midrule
NexusRaven~\citep{srinivasan2023nexusraven} & NexusRaven & IFT & - & - & 116 & \textcolor{green}{\CheckmarkBold} & \textcolor{green}{\CheckmarkBold}  & \textcolor{green}{\CheckmarkBold} &\textcolor{red}{\XSolidBrush} & \textcolor{green}{\CheckmarkBold} &\textcolor{red}{\XSolidBrush} &\textcolor{red}{\XSolidBrush}&\textcolor{red}{\XSolidBrush}\\
Gorilla~\citep{patil2023gorilla} & Gorilla & IFT & - & 16.0K & 1,645 & \textcolor{green}{\CheckmarkBold} &\textcolor{red}{\XSolidBrush} &\textcolor{red}{\XSolidBrush}&\textcolor{green}{\CheckmarkBold} &\textcolor{green}{\CheckmarkBold} &\textcolor{red}{\XSolidBrush} &\textcolor{red}{\XSolidBrush} &\textcolor{red}{\XSolidBrush}\\
OpenFunctions-v2~\citep{patil2023gorilla} & OpenFunctions-v2 & IFT & - & 65.0K & - & \textcolor{green}{\CheckmarkBold} & \textcolor{green}{\CheckmarkBold} &\textcolor{red}{\XSolidBrush} &\textcolor{green}{\CheckmarkBold} &\textcolor{green}{\CheckmarkBold} &\textcolor{red}{\XSolidBrush} &\textcolor{red}{\XSolidBrush} &\textcolor{red}{\XSolidBrush}\\
API Pack~\cite{guo2024api} & API Pack & IFT & - & 1.1M & 11,213 &\textcolor{green}{\CheckmarkBold} &\textcolor{red}{\XSolidBrush} &\textcolor{green}{\CheckmarkBold} &\textcolor{red}{\XSolidBrush} &\textcolor{green}{\CheckmarkBold} &\textcolor{red}{\XSolidBrush}&\textcolor{red}{\XSolidBrush}&\textcolor{red}{\XSolidBrush}\\ 
LAM~\citep{zhang2024agentohana} & AgentOhana & IFT & - & 42.6K & - & \textcolor{green}{\CheckmarkBold} & \textcolor{green}{\CheckmarkBold} &\textcolor{green}{\CheckmarkBold}&\textcolor{red}{\XSolidBrush} &\textcolor{green}{\CheckmarkBold}&\textcolor{red}{\XSolidBrush}&\textcolor{green}{\CheckmarkBold}&\textcolor{green}{\CheckmarkBold}\\
xLAM~\citep{liu2024apigen} & APIGen & IFT & - & 60.0K & 3,673 & \textcolor{green}{\CheckmarkBold} & \textcolor{green}{\CheckmarkBold} &\textcolor{green}{\CheckmarkBold}&\textcolor{red}{\XSolidBrush} &\textcolor{green}{\CheckmarkBold}&\textcolor{red}{\XSolidBrush}&\textcolor{green}{\CheckmarkBold}&\textcolor{green}{\CheckmarkBold}\\\midrule
\multicolumn{13}{l}{\emph{Pretraining-based LLM Agents}}  \\\midrule
% LEMUR~\citep{xu2024lemur} & PT & 90B & 300.0K & - & \textcolor{green}{\CheckmarkBold} & \textcolor{green}{\CheckmarkBold} &\textcolor{green}{\CheckmarkBold}&\textcolor{red}{\XSolidBrush} & \textcolor{red}{\XSolidBrush} &\textcolor{green}{\CheckmarkBold} &\textcolor{red}{\XSolidBrush}&\textcolor{red}{\XSolidBrush}\\
\rowcolor{teal!12} \method & \dataset & PT & 103B & 95.0K  & 76,537  & \textcolor{green}{\CheckmarkBold} & \textcolor{green}{\CheckmarkBold} & \textcolor{green}{\CheckmarkBold} & \textcolor{green}{\CheckmarkBold} & \textcolor{green}{\CheckmarkBold} & \textcolor{green}{\CheckmarkBold} & \textcolor{green}{\CheckmarkBold} & \textcolor{green}{\CheckmarkBold}\\
\bottomrule
\end{tabular}
% \begin{tablenotes}
%     \item $^*$ In addition, the StarCoder-API can offer 4.77M more APIs.
% \end{tablenotes}
\caption{Summary of existing instruction finetuning-based LLM agents for intrinsic reasoning and function calling, along with their training resources and sample sizes. "PT" and "IFT" denote "Pre-Training" and "Instruction Fine-Tuning", respectively.}
\vspace{-2ex}
\label{tab:related}
\end{threeparttable}
\end{table*}

\noindent \textbf{Prompting-based LLM Agents.} Due to the lack of agent-specific pre-training corpus, existing LLM agents rely on either prompt engineering~\cite{hsieh2023tool,lu2024chameleon,yao2022react,wang2023voyager} or instruction fine-tuning~\cite{chen2023fireact,zeng2023agenttuning} to understand human instructions, decompose high-level tasks, generate grounded plans, and execute multi-step actions. 
However, prompting-based methods mainly depend on the capabilities of backbone LLMs (usually commercial LLMs), failing to introduce new knowledge and struggling to generalize to unseen tasks~\cite{sun2024adaplanner,zhuang2023toolchain}. 

\noindent \textbf{Instruction Finetuning-based LLM Agents.} Considering the extensive diversity of APIs and the complexity of multi-tool instructions, tool learning inherently presents greater challenges than natural language tasks, such as text generation~\cite{qin2023toolllm}.
Post-training techniques focus more on instruction following and aligning output with specific formats~\cite{patil2023gorilla,hao2024toolkengpt,qin2023toolllm,schick2024toolformer}, rather than fundamentally improving model knowledge or capabilities. 
Moreover, heavy fine-tuning can hinder generalization or even degrade performance in non-agent use cases, potentially suppressing the original base model capabilities~\cite{ghosh2024a}.

\noindent \textbf{Pretraining-based LLM Agents.} While pre-training serves as an essential alternative, prior works~\cite{nijkamp2023codegen,roziere2023code,xu2024lemur,patil2023gorilla} have primarily focused on improving task-specific capabilities (\eg, code generation) instead of general-domain LLM agents, due to single-source, uni-type, small-scale, and poor-quality pre-training data. 
Existing tool documentation data for agent training either lacks diverse real-world APIs~\cite{patil2023gorilla, tang2023toolalpaca} or is constrained to single-tool or single-round tool execution. 
Furthermore, trajectory data mostly imitate expert behavior or follow function-calling rules with inferior planning and reasoning, failing to fully elicit LLMs' capabilities and handle complex instructions~\cite{qin2023toolllm}. 
Given a wide range of candidate API functions, each comprising various function names and parameters available at every planning step, identifying globally optimal solutions and generalizing across tasks remains highly challenging.



\section{Dataset Collection}

In this section, we first introduce our game design, including the representations of game setup and mechanics. We then describe a two-stage data collection process for the Game Creation (GC) and Game Simulation (GS) tasks. In the first stage, we build a non-player character (NPC) pool from fictional character Wikipedia pages, and prompt various LLMs to create one game per NPC. An automatic game validity checker applies for selecting valid games. In the second stage, we assemble a test set of valid games for GS.

\subsection{Game Design}\label{sec:gd}

% A game is represented as a JSON dictionary, as illustrated in Figure~\ref{fig:rpebench-overview}, containing information on \textbf{Game World}, \textbf{Player Character Name}, \textbf{Player Character Description}, \textbf{Main NPC Name}, \textbf{Main NPC Description}, \textbf{Main NPC Personality} (using the Big Five traits), \textbf{Main NPC Facts}, \textbf{Game Objective}, \textbf{Game Scenes}, \textbf{State Variables}, \textbf{Game Events} and \textbf{Termination Conditions}.


The games in \benchmark{}, as illustrated in Figure~\ref{fig:rpebench-overview}, are structured around several core components that create a text-based role-playing game (RPG) experience. This design ensures sufficient flexibility for diverse storytelling while maintaining support for objective mechanic evaluation:

\begin{itemize} 
\item \textbf{Game World}: The overarching setting where the story unfolds (e.g., "Gotham City"). 
\item \textbf{Player Character}: The protagonist controlled by the player, including a name and description (e.g., "Ann," a detective and ally of Batman). 
\item \textbf{Main NPC}: A key non-player character controlled by the game engine, characterized by a name, description, Big Five personality traits, and relevant facts (e.g., "Bruce Wayne (Batman)"). 
\item \textbf{Game Objective}: The primary goal to accomplish.
% (e.g., "Uncover the mastermind behind a conspiracy"). 
\item \textbf{Game Scenes}: Distinct locations where events occur.
% (e.g., "Wayne Manor"). 
\end{itemize}


% \begin{itemize}
%     \item \textbf{Game World}: A textual description of the game world.
%     \item \textbf{Player Character Name}: Name of the player character.
%     \item \textbf{Player Character Description}: A brief textual description of the player character.
%     \item \textbf{Main NPC Name}: Name of the main NPC.
%     \item \textbf{Main NPC Description}: A brief textual description of the main NPC.
%     \item \textbf{Main NPC Personality}: The Big Five personality traits for the main NPC, each scored from 1--5 (Openness, Conscientiousness, Extraversion, Agreeableness, and Neuroticism).
%     \item \textbf{Main NPC Facts}: A list of simple facts related to the main NPC.
%     \item \textbf{Game Objective}: A short textual description of the game objective.
%     \item \textbf{Game Scenes}: A list of game scenes.
%     \item \textbf{State Variables}: A list of finite discrete state variables. Each has an initial value, ensuring the total number of possible states is finite.
%     \item \textbf{Game Events}: A list of game events.
%     \item \textbf{Termination Conditions}: A set of conditions that, when met, lead to either success or failure.
% \end{itemize}
% The full JSON structure is provided in the Appendix~\ref{app:game_json}.

% \noindent\textbf{Game Mechanics.} 
% We present mechanics via interactions between state variables and game events, specified by four properties of each event: \textbf{Entering Condition}, \textbf{Success Condition}, \textbf{Success Effect} and \textbf{Fail Effect}. All the conditions are boolean expressions of state variables, and all effects are value updates of state variables.
% \begin{itemize}[leftmargin=*]
%     \item : Conditions for the event to occur.
%     \item : Conditions for the event to succeed.
%     \item : Effects on state variables if succeeds.
%     \item : Effects on state variables if fails.
% \end{itemize}
The core \textbf{game mechanics} in \benchmark{} are structured around \textbf{event-state interactions}, which define how game events modify the game state:

\begin{itemize} 
    \item \textbf{State Variables} represent numerical values that track the game's dynamic elements, such as character skills and trust levels. These variables always have an initial value along with minimum and maximum bounds.
    \item \textbf{Game Events} drives game progression and modifies \textbf{state variables} upon execution. Each event has an \textbf{entering condition} (whether it can occur) and a \textbf{success condition} (whether it succeeds). Upon execution, an event applies either a \textbf{success effect} or, if applicable, a \textbf{fail effect}, updating the state variables accordingly.
    \item \textbf{Termination Conditions} specify when the game ends by evaluating specific state variable expressions. These conditions, checked before processing game events, determine whether the game ends with success or failure.
\end{itemize}

This design creates an interactive experience where player actions and game events influence the game state. An LLM serves as the game engine, creating the game, simulating the game world based on user actions. Further details, including the exact game JSON schema, are provided in Appendix~\ref{app:game_json}.



\subsection{Game Data Collection}\label{sec:gdc}

We select 100 fictional characters from Wikipedia to serve as the test set for GC. For each character, we prompt an LLM to create a JSON-formatted game (as specified above) that treats this character as the main NPC. We employ a 5-shot prompting approach, where the examples are generated by initially prompting \emph{GPT 4o} using a manually crafted game. The full prompt is provided in the Appendix~\ref{app:gc_prompt}.

We parse LLM outputs to ensure they conform to the JSON format. Any game that passes this format check is then tested for validity using a BFS-based checker (see Section~\ref{sec:evaluation_gc}), which confirms whether a game can end in both success and failure, and whether all events can be reached. All valid games from multiple models are collected for the GS task (Table~\ref{tab:data_stats} shows the distribution).

\begin{table}[!ht]
    \centering
    \begin{tabular}{lr}
    \toprule
    Game Source & \# of Valid Games (Out of 100)\\
    \midrule
    Claude 3.5 Sonnet & 1\\
    DeepSeek V3       & 38\\
    Gemini 1.5 Pro    & 4\\
    Gemini 2.0 Flash Exp & 33\\
    GPT 4o            & 49\\
    \midrule
    Total             & 125 \\
    \bottomrule
    \end{tabular}
    \caption{Generated Game Statistics}
    \label{tab:data_stats}
\end{table}

\section{Evaluation}
\label{sec:evaluation}

%\subsection{Implementation}
%\label{sec:implementation}

We implemented \sys on top of Block-STM~\cite{blockstm} and \basesys~\cite{chiron} in Rust to evaluate its performance impact on both an optimistic execution engine and a guided execution engine, covering two of the most widely adopted approaches to parallel execution in the blockchain space.
The implementation is publicly available on Github~\footnote{https://github.com/ISTA-SPiDerS/Anthemius}.
As \basesys is built on top of Block-STM, this simplifies the implementation and allows for an easier comparison of the results.
Furthermore, we use the parallel execution benchmarks proposed in \basesys~\cite{chiron}. 

%We picked \basesys to represent guided execution engines as it is built on top of BlockSTM which makes comparing the two approaches easier.
%Therefore, aside from evaluating how BlockSTM performs with our block scheduling approach, we also leverage the implementation of \basesys~\cite{chiron} to evaluate the performance of guided execution engines which can be found in blockchains such as Solana~\cite{solana} or Sui~\cite{sui}. 

%Due to this, we can evaluate how \sys affects the performance of two popular approaches to parallel smart contract execution.

Finally, we implemented the batch handler ($\sim70$ lines of code) and the batch scheduler ($\sim120$ lines of code) to assemble blocks and then forward these blocks to the respective execution engines.


%% LIMIT is 1000 (10%)
% MAX relax num = 2, max relax rate = 100
% TARGETINCRATE = At least double of numtx/c
% MAXLEN = 10k
% hot read limit = 4

%The constructed blocks are then passed to the execution engine.

%Depending on the approach to consensus, the block construction in \sys can be optimized.
%For instance, if the same leader proposes multiple consecutive blocks, as in Narwahl~\cite{narwahl}, Kauri~\cite{kauri}, or PBFT~\cite{pbft}, the leader can optimize scheduling by tracking whether a batch primarily consisted of contended transactions at a given point and then skipping that batch in subsequent rounds. The leader then waits until the batch becomes the first batch, to start including the transactions again.

\subsection{Benchmark}

The experiments were executed on a Debian GNU/Linux 12 server with two AMD EPYC 7763 64-Core Processors and 1024 GB of RAM. We generated batches of transactions with different distributions of read/write-accesses and different user distributions with the help of \basesys~\cite{chiron} for all five proposed workloads. Namely, one peer-to-peer workload (P2PTX), two Decentralized Exchange Workloads (DEXAVG and DEXBURSTY), one NFT workload (NFT), and one mixed workload (MIXED). 
These workloads are derived from real-world data from Ethereum and Solana and are designed to evaluate parallel transaction execution engines under realistic levels of contention. 
Each workload has a unique and realistic resource access pattern, along with a varying count of read and write operations per transaction.

Each experiment was executed a total of 10 times and the results we outline in this section present the average of all 10 runs. Furthermore, in each workload, we vary the number of worker threads from 4 to 32 in increments of 4.
Finally, we are interested in two key metrics: throughput, to assess the performance improvement introduced by \sys, and latency, to determine the average delay introduced by \sys.


We set the following parameters for the batch handler and batch scheduler:
First, we evaluate the execution engines using blocks of up to $\textsc{maxlen} = 10{,}000$ transactions, as this block size represents a sweet spot for both engines, where the execution setup overhead (e.g., virtual machine initialization) becomes negligible. Accordingly, we configured the batch size to match the target block size, as smaller batch sizes increase block construction overhead, while larger batch sizes reduce the batch handler's flexibility to adapt to the workload's characteristics.

Next, to minimize tail latency for transactions accessing hot resources, we allow the first and last $\textsc{lim} = 1{,}000$ transactions to be included freely without restrictions.  
Furthermore, we permit up to $\textsc{maxrelaxnum} = 2$ relaxations of the inclusion rate as we observed diminishing returns from additional relaxations and large scheduling costs beyond this point.  
We set the relaxation rate to a maximum of $\textsc{maxrelaxrate} = 100$, targeting an inclusion rate of $\textsc{targetincrate} = 2\frac{\text{maxlen}}{c}$. This accounts for the higher returns from a more aggressive target inclusion rate as the concurrency potential increases.  
Finally, we configure $\textsc{maxhotr} = 4$ to avoid uniting too many critical paths of transactions, ensuring manageable contention levels.




%As previously stated, we evaluate the performance of \sys with Optimistic Parallel Execution using Block-STM~\cite{blockstm} and Guided Parallel Execution using \basesys~\cite{chiron}. 


%Given the number of worker threads, we created an equal number of batches for the batch handler.

%The batch handler first tries to include transactions from the first batch and then attempts to include transactions from the following batches. Following that, the execution finishes once the first batch is fully exhausted. 
%Thus, each run of \sys presents the average run overall for several blocks with varying block sizes up to 10.000 transactions. For vanilla BlockSTM and \basesys, there is a stable block size of 10.000 transactions.
%To measure the average latency, we only evaluated the latency for the transactions in the first batch, adding up the execution time and scheduling time of all blocks until the given transaction was successfully included in a block and executed.

\subsection{Throughput}



As \sys delays the inclusion of some transactions in favor of others to enhance system performance, we provide the batch handler with several batches of $10{,}000$ transactions to saturate the system and measure the maximum throughput. Each batch is generated with the same distribution of resource accesses, both within and across batches. We then evaluate \sys by passing all batches to the batch handler and run \sys until all transactions from the first batch are successfully executed. Consequently, the evaluation for \sys spans multiple blocks, where the reported throughput represents the average throughput over the entire runtime and accounts for scheduling and execution time.
For the baseline versions of Block-STM and \basesys, we use a single block containing $10{,}000$ transactions that also fully saturates the system, with runtime variations dependent solely on the specific workload.  


As blockchains such as Aptos or Sui decouple consensus from execution, block scheduling could be moved outside of the critical path of consensus. This can significantly reduce the overhead, as scheduling requires only a single thread and only has to be done at the proposer node. Due to this, we display two lines for \sys. First, one that serves as a ceiling on performance, where we assume that there is an idle thread that can be used for scheduling outside of the critical path of consensus, denoted \textit{Decoupled \sys}. Second, one that serves as a floor on performance where we count the full scheduling overhead on the critical path of consensus, referred to as \textit{\sys}.

\begin{figure*}[t]
\begin{subfigure}{0.5\textwidth}
\includegraphics[width=1\linewidth]{figures/tputpygoodblock.pdf} 
\caption{Throughput per Second - \basesys}
\label{fig:tputpythia}
\end{subfigure}
\begin{subfigure}{0.5\textwidth}
\includegraphics[width=1\linewidth]{figures/tputstmgoodblock.pdf}
\caption{Throughput per Second - Block-STM}
\label{fig:tputblockstm}
\end{subfigure}
\caption{Throughput per Second}
\label{fig:image2}
\end{figure*}


The results for \sys with \basesys are shown in Figure~\ref{fig:tputpythia}, with the throughput in transactions per second on the y-axis and the number of worker threads on the x-axis. With the NFT workload, we only see a small speedup from creating good blocks. This is due to the account distribution in this workload, where transactions from users appear very frequently in several batches. Due to this, once a transaction of a given user is skipped, the following transactions also have to be skipped, resulting in long scheduling times and leaving very few transactions behind that can be included in the block. In comparison, in the peer-to-peer workload there is already a significant improvement, where with an increasing number of worker threads, we can reach almost twice the initial throughput. Following that, with increasing contention and less repetitive users, the decentralized exchange workloads reach over 240\% performance boost compared to vanilla \basesys. While in the average DEX workload, the scheduling overhead is very small, with increasing contention and increasing number of worker threads we can also see an increased scheduling overhead.
Finally, in the mixed workload, we also see a large performance advantage. This is also due to the much higher overall execution complexity compared to the scheduling overhead. Due to the complexity of the workload, the overhead is constant after 12 cores, but \sys under this workload shows over 200\% performance advantage compared to vanilla \basesys.


The throughput results for \sys with Block-STM are shown in Figure~\ref{fig:tputblockstm}, with the throughput in transactions per second on the y-axis and the number of worker threads on the x-axis. 
Compared to the results with \basesys, the results for Block-STM vary more as the high contention within each block results in a large re-execution overhead. 
As such, even when we build better blocks with \sys, the contention in the block is still so high, that Block-STM struggles to take advantage of that.
We can still see the largest disadvantage in the NFT workload, due to the user distribution preventing us from building better blocks. Furthermore, we can see that in the peer-to-peer workload, once we reach 20 threads, \sys is starting to be able to compensate for the re-execution overhead of Block-STM and reach a speed-up of up to 25\%. Similarly, for the DEX workloads, there is an initial performance drop due to the re-execution overhead, which is only compensated with more worker threads later.
Finally, in the MIXED workload, \sys shows a constant speed up compared to vanilla Block-STM up to 200\% the original performance.

%For both approaches the total scheduling overhead was only between 5m and 10m per batch

\subsection{Latency}

\begin{figure}[t]
\includegraphics[width=1\linewidth]{figures/anthemiuslatency.pdf} 
\caption{Tail Latency for \basesys and Block-STM}
\label{fig:latency}
\end{figure}


As we are delaying the inclusion of some transactions that access hot resources, we expect a latency overhead increase at the tail. 
Similarly to the throughput evaluation, we send several batches of transactions to the batch handler. To fully assess the effect of \sys, we evaluate how the tail latency develops when awaiting the finished execution of up to five batches for all workloads with a fixed number of $16$ cores.
The results of this evaluation are shown in Figure~\ref{fig:latency}, where the yellow line indicates the 50th percentile (median), the box represents the 25th and 75th percentiles (interquartile range), and the whiskers denote the 10th and 90th percentiles.

The results mirror what we saw in the throughput evaluation where in almost all workloads and configurations where \sys shows a significant speedup the average transaction latency is significantly lower. Furthermore, thanks to the large throughput advantage in these settings, especially when paired with \textit{Chiron}, \sys has a latency advantage for up to the 90\% percentile of transactions.

On the other hand, as expected, \sys shows a growing tail latency with an increasing number of batches. This
is expected since the congestion caused by the highly contended workloads results in different scheduling decisions. Nevertheless, we can see that the growing tail latency affects not only \sys but also the reference systems, although for certain workloads the effects of \sys are more prominent at the p90 percentile. 

This is a tradeoff the blockchain needs to take into account based on their expected workload and tune \sys parameters to better match the chracteristics of the transactions expected.




%We evaluate the tail latency of \sys compared to the vanilla single block latency in Block-STM and \basesys. We split transactions into three categories: p50, p75, and p90, relative to the latency percentile they were in. As such, the presented graphs show the latency for the 3 given percentiles.

%The results for \basesys are displayed in Figure~\ref{fig:latpythia}, with the latency in seconds on the y-axis and the different latency percentiles on the x-axis. We can see that for all workloads excluding the mixed workload, 50\% of all transactions were executed faster than in vanilla \basesys. This is the case as they were paired with less contended transactions such that one or even multiple blocks can be executed in the same time frame as a single block in vanilla \basesys.
%In the mixed workload, the latency is mostly on par with a slight latency overhead. 

%In p75 the results vary more. In the P2P and DEX workloads, the latency shows an advantage or is on par with vanilla \basesys. While in the MIXED and NFT workloads, there is a slight latency overhead.

%Finally, in the p90 group, there is a latency overhead in almost all settings. This is expected as we are delaying some transactions to improve the overall throughput and to improve the latency of transactions that are not accessing hot resources.
%However, we would like to point out that even in the p90, none of the transactions took more than twice as long in any setting. This indicates that for \basesys, \sys only presents a small tradeoff from system throughput to tail latency overhead.



%The results for Block-STM are displayed in Figure~\ref{fig:latpythia} and are very similar to the ones of \basesys. For the p50 group, we see an advantage compared to vanilla execution, and, analogous to \basesys, in the p75 group we see some workloads with a small advantage while others show a small overhead.
%However, finally, in the p90 group, for workloads with very high contention, we can see a more significant latency overhead. Nonetheless, in no setting, this is significantly larger than three times the vanilla latency.

\subsection{Summary}

In this section, we evaluated the throughput improvement \sys can provide across different execution engines. Our findings demonstrate that while \sys improves throughput for both types of execution engines under several of the workloads, its impact is significantly larger when combined with guided execution engines. In this case, \sys provides a large throughput improvement across all but one of the workloads. The only exception is the NFT workload, where many high-frequency users appear across multiple blocks, preventing \sys from effectively rescheduling their transactions.

When it comes to latency, we analyzed the tail latency percentiles of delayed transactions. Our results show that for most workloads the majority of transactions (over 75\%) have lower or similar latency compared to the vanilla execution, while only the slowest 25\% of transactions sustain a latency overhead.
This indicates that \sys can be a valuable addition to any blockchain with a parallel execution engine where the workload does not primarily stem from a very small set of users.

\section{Experiments and Discussions}
\subsection{Experiments}
We chose seven different buildings to test our framework. These include well-known landmarks, commercial, residential, and institutional buildings. We extract 31 multi-view images in a 360$\degree$ view pose around the building of interest, which we then use in conjunction with our GBM module to create the 3D colored mesh of the building. Then we subsample six images, one every 70$\degree$, as inputs to the Multi-Agent LLM module. We also use the Google Map Platform integration to retrieve two aerial/satellite image(s), one at Google Maps zoom level 18, and one at Google Maps zoom level 19 as inputs to the Multi-Agent LLM module.

In preliminary experiments, we noticed a relatively large variation in final CLIP-score across different attempts even while using the same model and prompt, since the LLMs' outputs are not deterministic (even when using \textit{LLM Temperature} = 0). As such, we perform two experiments. We want to understand the performance of the module when using different models as LLM agents, and we want to understand the distribution of scores across different attempts for both the keyword extraction step, and the captioning step. 

\begin{figure}[htpb]
\centering
\includegraphics[width = 0.4\textwidth]{Figures/keyword_Perplexity.png}
\caption{Box plot of image-to-keyword perplexity distribution per model and level of image detail (2240 samples total).}\label{fig:perplexity}
\end{figure}
\subsubsection{Keyword Extraction}
For multi-agent image-to-keyword extraction, both \textit{chatgpt-4o-latest, and gpt-4o-mini} are suitable. Additionally, both high-resolution and low-resolution image analysis are available. We test all 4 combinations for all 7 scenes, for 10 iterations, with 8 images and LLM agents, resulting in a total of 2240 API calls. We record the LLM responses and the perplexity scores. The results are shown in Fig.\ref{fig:perplexity}.

%%%%%%%%%%%%%%%%%%%%%%

\subsubsection{Captioning}
For the keyword-to-caption step, we fix the per-image keywords for each scene using the results from one of the \textit{gpt-4o high image resolution} API calls. For each of the 7 buildings, we test 5 iterations of keyword-aggregation-caption for each of the four models: \textit{gpt-4o-mini, chatgpt-4o-latest, deepseek-chat, deepseek-reasoner}. Each test requires two API calls, totaling 448 API calls. We additionally calculate CLIP scores for every single one of the input images. Resulting in $7\cdot5\cdot4\cdot8 = 1120$ CLIP scores. The per-model clip score distributions are visualized in Fig \ref{fig:4ModelsCaptionCLIp}.

\subsubsection{Visualization}
We present a visualization of the extracted 3D model, caption, keywords, and Google Maps Platform-based information for the Perimeter Institute (PI) building scene in Fig. \ref{fig:visualizationFig}. The Perimeter Institute for Theoretical Physics is an independent research centre located at 31 Caroline St. N, Waterloo, Ontario, Canada. We show the 3D mesh and depth maps extracted from the scene, the 2D map, and the aerial image with the building's polygon at Google Maps zoom level 18, retrieved via the Google Maps Platform Static Maps API. We also plot the keywords extracted from a single view, as well as the caption generated by the Multi-Agent LLM module.
\begin{figure}[htpb]
\centering
\includegraphics[width = 0.4\textwidth]{Figures/Captioning_CLIP.png}
\caption{Box plot of CLIP score distribution of building captions generated from aggregating multiple iterations of multi-view keywords and captioning (1280 samples total).}\label{fig:4ModelsCaptionCLIp}
\end{figure}

\begin{figure*}[htbp]
    \centering
    % First row of subfigures
    \begin{subfigure}{0.4\textwidth}
        \centering
        \includegraphics[trim=200pt 200pt 200pt 100pt, clip, width=\textwidth]{Figures/Mesh.png}  % Replace with your image path% Caption for image (a)
        \label{fig:subfig1}
    \end{subfigure}
    \begin{subfigure}{0.42\textwidth}
        \centering
        \includegraphics[trim=0pt 0pt 50 0pt, clip,width=\textwidth]{Figures/sat_img.png}  % Replace with your image path % Caption for image (b)
        \label{fig:subfig2}
    \end{subfigure}
    \\
    \begin{subfigure}[b]{0.4\textwidth}
        \centering
        \includegraphics[trim=200pt 100pt 200pt 100pt, clip,width=\textwidth]{Figures/depth.png} 
 % Caption for image (c)
        \label{fig:subfig3}
    \end{subfigure}
    \begin{subfigure}[b]{0.43\textwidth}
        \centering
        \includegraphics[trim=0pt 0pt 50 0, clip,width=\textwidth]{Figures/map.png}  % Replace with your image  % Caption for image (d)
        \label{fig:subfig4}
    \end{subfigure}
    \caption{Visualization of results. Top Left: colored 3D mesh; Bottom Left: depth map; Top Right: aerial image with keywords and captions and retrieved polygon mask. Bottom Right: retrieved map with map information (entrance is labelled with a red place marker).}
    \label{fig:visualizationFig}
\end{figure*}


\subsection{Discussion}
Fig. \ref{fig:perplexity} shows the level of detail does not significantly affect the LLM agents' confidence in their own predictions. Perhaps surprisingly, the much smaller \textit{gpt-4o-mini} is more confident in its own responses on average. We note this does not necessarily denote keyword accuracy since it is possible the larger model considers many equivalent alternate visual descriptions, lowering its confidence in any individual description. Visual inspection of captions and images shows that across all four configurations, caption-to-image agreement is high. Although it is possible that the smaller model is enough for the visual captioning task, we nonetheless used the larger model with high visual quality for multi-view/multi-scale keyword extraction to test the keyword-to-caption step.

\textit{Deepseek-reasoner}, the Deepseek-R1 model has the poorest captioning score. Additionally, this model sometimes failed at the task, shown as outliers in Fig. \ref{fig:4ModelsCaptionCLIp}. This is perhaps because image captioning is an autoregressive text generation task and not a reasoning task. Consequently, we decided not to test OpenAI's reasoning model, the GPT-o1. Its performance on reasoning tasks is comparable to Deepseek-R1 according to \cite{deepseekr1}, yet it is more expensive by a factor of 30-100. \textit{Deepseek-chat}. The Deepseek-V3 model has on average the best captioning performance, offering a very good best price-performance ratio with performance comparable to GPT-4o, but prices similar to GPT-4o Mini (see Table \ref{Tab:models}).



Our future research aims to leverage the multi-agent LLM tool for geospatial data analysis, integrating various data sources from Google Cloud Platform services, including Google Maps Platform APIs and Google Earth Engine. Additionally, benchmarking the reasoning capabilities of large language models, such as GPT-o1/o3 and Deepseek-R1, for remote sensing and GIS tasks could yield valuable insights.








\section{Conclusion}
In this paper, we propose ChineseEcomQA, a scalable question-answering benchmark designed to rigorously assess LLMs on fundamental e-commerce concepts. ChineseEcomQA is characterized by three core features: Focus on Fundamental Concept, E-Commerce Generalizability, and Domain-Specific Expertise, which collectively enable systematic evaluation of LLMs' e-commerce knowledge. Leveraging ChineseEcomQA, we conduct extensive evaluations on mainstream LLMs, yielding critical insights into their capabilities and limitations. Our findings not only highlight performance disparities across models but also delineate actionable directions for advancing LLM applications in the e-commerce domain.
\section*{Impact Statement}

This work aims to advance the field of Machine Learning by introducing \benchmark{}, a benchmark specifically designed to evaluate large language models (LLMs) in the context of text-based role-playing games. The development of \benchmark{} has potential societal implications related to the deployment of LLMs in interactive and narrative-driven applications, including fostering more immersive and engaging gaming experiences.

Ethical considerations include ensuring that LLMs evaluated and fine-tuned using \benchmark{} adhere to principles of fairness and inclusivity, particularly in the portrayal of characters and narratives. Misuse of the benchmark to develop systems that propagate harmful biases or enforce stereotypical characterizations is a concern that developers should address when applying this work. Additionally, the use of LLMs as evaluative judges raises questions about transparency, reliability, and the potential for unintended bias in automated assessments.

By encouraging further research on hybrid evaluation methods that combine subjective LLM-based judgments with objective scoring mechanisms, this work contributes to ongoing discussions about improving the accountability and robustness of machine learning systems in creative and interactive domains.
% Acknowledgements should only appear in the accepted version.
% \section*{Acknowledgements}





\bibliography{main}
\bibliographystyle{icml2025}


%%%%%%%%%%%%%%%%%%%%%%%%%%%%%%%%%%%%%%%%%%%%%%%%%%%%%%%%%%%%%%%%%%%%%%%%%%%%%%%
%%%%%%%%%%%%%%%%%%%%%%%%%%%%%%%%%%%%%%%%%%%%%%%%%%%%%%%%%%%%%%%%%%%%%%%%%%%%%%%
% APPENDIX
%%%%%%%%%%%%%%%%%%%%%%%%%%%%%%%%%%%%%%%%%%%%%%%%%%%%%%%%%%%%%%%%%%%%%%%%%%%%%%%
%%%%%%%%%%%%%%%%%%%%%%%%%%%%%%%%%%%%%%%%%%%%%%%%%%%%%%%%%%%%%%%%%%%%%%%%%%%%%%%
\newpage
\appendix
\onecolumn
\section{Game JSON Structure in \benchmark{}}\label{app:game_json}

As introduced in Section~\ref{sec:gd}, each game in \benchmark{} is represented by a JSON dictionary. Figures~\ref{lst:json-schema} and \ref{fig:trait-schema}--\ref{fig:pre-event-schema} provide the complete schema and its referenced object definitions. Below, we clarify naming discrepancies between this JSON specification and the terminology used in the main article, and also highlight a few design details omitted for brevity.

\paragraph{Naming Discrepancies.}
The JSON schema in Figure~\ref{lst:json-schema} has property names slightly different from those in Figure~\ref{fig:rpebench-overview} from the main article. For clarity, we list them side by side as ``JSON schema name --- main article name'':

\begin{enumerate}
    \item \texttt{player\_name} --- Player Character / Name
    \item \texttt{player\_description} --- Player Character / Description
    \item \texttt{main\_npc\_description / text} --- Main NPC /Description
    \item \texttt{main\_npc\_description / big5\_personality\_traits} --- Main NPC / Personality
    \item \texttt{main\_npc\_description / additional\_facts} --- Main NPC / Facts
    \item \texttt{state\_variables} + \texttt{hidden\_variables} --- State Variables
    \item \texttt{pre\_event\_checks} --- Termination Conditions
\end{enumerate}

For consistency, the appendices continue to use the names from the main article unless otherwise specified. Although \texttt{state\_variables} and \texttt{hidden\_variables} are separate fields in the JSON schema, they collectively represent the State Variables described in the main text. In our design, \texttt{hidden\_variables} (unlike \texttt{state\_variables}) are not displayed to players; however, this distinction does not impact the benchmark evaluations and is thus not emphasized in the main article.

We also require \texttt{hidden\_variables} to include at least two special Boolean flags, \texttt{has\_succeeded} and \texttt{has\_failed}, which interact with \texttt{pre\_event\_checks} (a list of two check objects \texttt{If Succeeded} and \texttt{If Failed}). Each check object includes a \texttt{condition} (a Boolean expression over the state variables) and an \texttt{effect} that sets \texttt{has\_succeeded=1} or \texttt{has\_failed=1}, if not already set\footnote{Some games directly set \texttt{has\_succeeded} or \texttt{has\_failed} in other event effects, leaving effects of \texttt{pre\_event\_checks} empty.}. Conceptually, these properties mirror the Termination Conditions in the main article.

\paragraph{Explanatory Content.}
Several text fields in the JSON schema contain descriptive or explanatory information that we omit from the main article, such as:
\begin{enumerate}
    \item \texttt{\$def/trait/description}: Describes the personality trait score in natural language.
    \item \texttt{\$def/scene\_object/background\_description}: Describes the scene.
    \item \texttt{\$def/variable\_object/description}: Describes a particular state variable.
    \item \texttt{\$def/event\_object/explanations}: Explains event effects.
    \item \texttt{\$def/pre\_event\_check\_object/explanation}: Explains the termination condition check.
\end{enumerate}
Although these fields do not affect our validity checks, they provide additional context for LLMs and are included in prompts given to LLMs during game simulation.

\paragraph{Game Scenes in the BFS Validity Check.}
Because each event references exactly one scene (Figure~\ref{fig:event-schema}), we also verify that all declared scenes are referenced by at least one event. This check is straightforward and independent of the BFS procedure, so it is omitted from the main article for simplicity.

\begin{figure*}[!htp]
\centering
\small
\begin{minipage}{0.95\textwidth}
\begin{lstlisting}[language=json]
{
  "title": "Game Configuration",
  "type": "object",
  "required": [
    "game_world",
    "player_name",
    "player_description",
    "main_npc_name",
    "main_npc_description",
    "game_objectives",
    "scenes",
    "state_variables",
    "hidden_variables",
    "events",
    "pre_event_checks"
  ],
  "properties": {
    "game_world": { "type": "string" },
    "player_name": { "type": "string" },
    "player_description": { "type": "string" },
    "main_npc_name": { "type": "string" },
    "main_npc_description": {
      "type": "object",
      "required": [ "text", "big5_personality_traits", "additional_facts" ],
      "properties": {
        "text": { "type": "string" },
        "big5_personality_traits": { "$ref": "#/$defs/big5_traits" },
        "additional_facts": { "type": "array", "items": { "type": "string" } }
      },
      "additionalProperties": false
    },
    "game_objectives": { "type": "string" },
    "scenes": { "type": "array", "items": { "$ref": "#/$defs/scene_object" }
    },
    "state_variables": { "type": "array", "items": { "$ref": "#/$defs/variable_object" } },
    "hidden_variables": {
      "type": "array",
      "minItems": 2,
      "items": { "$ref": "#/$defs/variable_object" },
      "contains": { "properties": { "value_name": { "enum": [ "has_succeeded", "has_failed" ] } } }
    },
    "events": { "type": "array", "items": { "$ref": "#/$defs/event_object" } },
    "pre_event_checks": { "type": "array", "items": { "$ref": "#/$defs/pre_event_check_object" } },
    "source": { "type": "string" }
  },
  "additionalProperties": false,
}
\end{lstlisting}
\caption{JSON Schema for Game Configuration}
\label{lst:json-schema}
\end{minipage}
\end{figure*}

\begin{figure}[!ht]
\centering
\small
\begin{minipage}{0.95\textwidth}
\begin{lstlisting}[language=json]
{
  "$defs": {
    "trait": {
      "type": "object",
      "required": ["rate", "description"],
      "properties": {
        "rate": { "type": "number" },
        "description": { "type": "string" }
      },
      "additionalProperties": false
    }
  }
}
\end{lstlisting}
\caption{\texttt{trait} object schema}
\label{fig:trait-schema}
\end{minipage}
% \end{figure}

% \begin{figure}[!ht]
% \centering
% \small
\begin{minipage}{0.95\textwidth}
\begin{lstlisting}[language=json]
{
  "$defs": {
    "big5_traits": {
      "type": "object",
      "required": [
        "openness",
        "conscientiousness",
        "extraversion",
        "agreeableness",
        "neuroticism"
      ],
      "properties": {
        "openness": { "$ref": "#/$defs/trait" },
        "conscientiousness": { "$ref": "#/$defs/trait" },
        "extraversion": { "$ref": "#/$defs/trait" },
        "agreeableness": { "$ref": "#/$defs/trait" },
        "neuroticism": { "$ref": "#/$defs/trait" }
      },
      "additionalProperties": false
    }
  }
}
\end{lstlisting}
\caption{\texttt{big5\_traits} object schema}
\label{fig:big5-schema}
\end{minipage}
% \end{figure}

% \begin{figure}[!ht]
% \centering
% \small
\begin{minipage}{0.95\textwidth}
\begin{lstlisting}[language=json]
{
  "$defs": {
    "scene_object": {
      "type": "object",
      "required": [ "scene_name", "unique_id", "background_description", "scene_type" ],
      "properties": {
        "scene_name": { "type": "string" },
        "unique_id": { "type": "string" },
        "background_description": { "type": "string" },
        "scene_type": { "type": "string" }
      },
      "additionalProperties": false
    }
  }
}
\end{lstlisting}
\caption{\texttt{scene\_object} schema}
\label{fig:scene-schema}
\end{minipage}
\end{figure}

\begin{figure}[!ht]
\centering
\small
\begin{minipage}{0.95\textwidth}
\begin{lstlisting}[language=json]
{
  "$defs": {
    "variable_object": {
      "type": "object",
      "required": [ "value_name", "unique_id", "description", "min_value", "max_value" ],
      "properties": {
        "value_name": { "type": "string" },
        "unique_id": { "type": "string" },
        "description": { "type": "string" },
        "initial_value": { "type": "string" },
        "min_value": { "type": "string" },
        "max_value": { "type": "string" }
      }, "additionalProperties": false
    }
  }
}
\end{lstlisting}
\caption{\texttt{variable\_object} schema}
\label{fig:variable-schema}
\end{minipage}
% \end{figure}

% \begin{figure}[!ht]
% \centering
% \small
\begin{minipage}{0.95\textwidth}
\begin{lstlisting}[language=json]
{
  "$defs": {
    "event_object": {
      "type": "object",
      "required": [ "event_name", "unique_id", "scene", "entering_condition", "succeed_condition", "succeed_effect", "fail_effect" ],
      "properties": {
        "event_name": { "type": "string" },
        "unique_id": { "type": "string" },
        "scene": { "type": "array", "items": { "type": "string" } },
        "entering_condition": { "type": "array", "items": { "type": "string" } },
        "succeed_condition": { "type": "array", "items": { "type": "string" } },
        "succeed_effect": { "type": "array", "items": { "type": "string" } },
        "fail_effect": { "type": "array", "items": { "type": "string" } },
        "explanations": { "type": "string" }
      }, "additionalProperties": false
    }
  }
}
\end{lstlisting}
\caption{\texttt{event\_object} schema}
\label{fig:event-schema}
\end{minipage}
% \end{figure}

% \begin{figure}[!t]
% \centering
% \small
\begin{minipage}{0.95\textwidth}
\begin{lstlisting}[language=json]
{
  "$defs": {
    "pre_event_check_object": {
      "type": "object",
      "required": [ "check_name", "unique_id", "description", "condition", "effect" ],
      "properties": {
        "check_name":  { "type": "string" },
        "unique_id":   { "type": "string" },
        "description": { "type": "string" },
        "condition": { "type": "array", "items": { "type": "string" } },
        "effect": { "type": "array", "items": { "type": "string" } },
        "explanation": { "type": "string" }
      }, "additionalProperties": false
    }
  }
}
\end{lstlisting}
\caption{\texttt{pre\_event\_check\_object} schema}
\label{fig:pre-event-schema}
\end{minipage}
\end{figure}

\section{Game Creation Prompt}\label{app:gc_prompt}
For the Game Creation (GC) task, we use the prompt shown below. It references the Wikipedia content of the chosen main NPC (\texttt{\{wikicontent\}}) and the JSON schema defined in Appendix~\ref{app:game_json} (\texttt{\{schema\}}). The full text of this schema is provided to the model so it can generate a well-structured JSON output.
\begin{center}
\begin{minipage}{0.95\textwidth}
\begin{lstlisting}[language=plaintext, frame=none, numbers=none]
Here is a character description:
{wikicontent}

Based on this character, create a detailed game scenario exactly following JSON structure of previous examples and the following schema:
{schema}

## Guidelines
- All numerical values should use consistent ranges (e.g., 0-100)
- Events should have clear cause-and-effect relationships
- Scene progression should depend on variable thresholds
- Include both mandatory and optional events
- Create meaningful connections between variables
- Balance difficulty and achievability
- Ensure all IDs follow consistent formatting (P### for checks, S### for scenes, V### for state variables, H### for hidden variables, E### for events)
- Include proper fail states and success conditions
- Make sure all scenes are specific locations
- Create logical progression paths through the game

Format the response as a single JSON object with all fields properly nested. Must ensure all arrays and objects are properly closed and formatted.
\end{lstlisting}
\end{minipage}
\end{center}

\paragraph{5-Shot Prompt} To guide LLMs more effectively, we supply five example JSON games prior to the main creation prompt. Because each game JSON can be quite lengthy, stacking them directly after the prompt may cause the model to overlook important details in the instruction. Instead, we present the five-shot examples as sequential conversation entries, followed by the actual creation prompt. The resulting conversation structure is illustrated below.
\begin{center}
\begin{minipage}{0.95\textwidth}
\begin{lstlisting}[language=plaintext, frame=none, numbers=none]
    USER: Give me an example game JSON.
    ASSISTANT: {EXAMPLE_1}
    USER: Give me an example game JSON.
    ASSISTANT: {EXAMPLE_2\}
    USER: Give me an example game JSON.
    ASSISTANT: {EXAMPLE_3\}
    USER: Give me an example game JSON.
    ASSISTANT: {EXAMPLE_4\}
    USER: Give me an example game JSON.
    ASSISTANT: {EXAMPLE_5}
    USER: {Prompt for Game Creation}
\end{lstlisting}
\end{minipage}
\end{center}

\section{Evaluation Prompts and Detailed Score Calculations}\label{app:eval_prompt}
We employ a consistent three-part format for most evaluation prompts: an instruction section, a JSON schema specifying the output format, and an example response. To keep this appendix concise, we omit the JSON schemas and example responses when the instruction text alone clearly explains the expected output structure. Below, we detail the prompts and score calculations for four metrics: Main NPC Factual Consistency (\textbf{FAC}), Main NPC Personality Consistency (\textbf{PER}), Interestingness (\textbf{INT}), and Action Choice Quality (\textbf{ACT}).
\subsection{Main NPC Factual Consistency (FAC)}
The prompt below assesses how closely the generated game content aligns with each fact about the main NPC. We concatenate all LLM-generated game narration across the multi-round trajectory into \texttt{game\_content}\footnote{Event Plan and State Variables are omitted because they are not visible to players.}.
\begin{center}
\begin{minipage}{0.95\textwidth}
\begin{lstlisting}[language=plaintext, frame=none, numbers=none]
You are given a piece of narrative game content and a set of facts about a specific non-player character (NPC). Your task is to analyze whether each fact is supported, contradicted, or not addressed by the provided game content. For each fact, determine one of the following judgements based solely on the given game content:
- "align": The game content supports or is consistent with the fact.
- "contradict": The game content directly conflicts with or contradicts the fact.
- "neutral": The game content is unrelated or does not provide enough information to judge the fact.
Please disregard prior knowledge and analyze the NPC purely based on the game content and the facts.

**NPC**: {main_npc_name}

**Game Content**:

{game_content}

**Facts**

{main_npc_facts}

**Output Format**:  
Return the results as a JSON array, where each element is an object with:
- fact_id: the corresponding fact's ID.
- judgement: one of "align", "contradict", or "neutral".
- explanation: a brief explanation for your judgment, referencing specific parts of the game content if applicable.
The return json array should follow this json schema:
{schema}

**Example Response**:
{example}
\end{lstlisting}
\end{minipage}
\end{center}
The judge assigns one of three labels for each fact: ``align,'' ``contradict,'' or ``neutral.'' The final trajectory-level FAC score is computed as
\begin{equation}
    \textbf{FAC}_\text{traj} = \frac{\# \text{align}}{\# \text{align} + \# \text{contradict}},
\end{equation}
and we then average over all trajectories:
\begin{equation}
    \textbf{FAC} = \frac{\sum_{\text{traj}} \textbf{FAC}_\text{traj}}{\# \text{trajectories}}.
\end{equation}

\subsection{Main NPC Personality Consistency (PER)}\label{app:per_eval}
\paragraph{TIPI PER Score} As described in the main article, we derive the PER score using a Ten-Item Personality Inventory (TIPI) approach~\cite{gosling2003very,cao2024large}, prompting the LLM judge to rate each of ten statements and then converting the ratings into Big Five trait scores.
\begin{center}
\begin{minipage}{0.95\textwidth}
\begin{lstlisting}[language=plaintext, frame=none, numbers=none]
You will be given information about a character. Here are a number of personality traits that may or may not apply to the character. Please write a number to each statement to indicate the extent to which you agree or disagree with that statement. You should rate the extent to which the pair of traits applies to the character, even if one characteristic applies more strongly than the other.

For the ratings:
- 1: Disagree strongly
- 2: Disagree moderately
- 3: Disagree a little
- 4: Neither agree nor disagree
- 5: Agree a little
- 6: Agree moderately
- 7: Agree strongly

Please give your ratings for the following 10 statements.

I see the character as:
A. Extraverted, enthusiastic.
B. Critical, quarrelsome.
C. Dependable, self-disciplined.
D. Anxious, easily upset.
E. Open to new experiences, complex.
F. Reserved, quiet.
G. Sympathetic, warm.
H. Disorganized, careless.
I. Calm, emotionally stable.
J. Conventional, uncreative

Please return ratings for all 10 traits in a dictionary following this schema:
{schema}

Please give your ratings for the following character.
{character}
\end{lstlisting}
\end{minipage}
\end{center}
Here, \texttt{character} consists of the main NPC name and the concatenated LLM-generated game narration sections. According to \citet{gosling2003very}, we use the following formulas to calculate personality trait scores,
\begin{equation}
    \begin{aligned}
        &\text{Openness: }&o_{tipi} =& E + 8 - J\\
        &\text{Conscientiousness: }&c_{tipi} =& C + 8 - H\\
        &\text{Extroversion: }&e_{tipi} =& A + 8 - F\\
        &\text{Agreeableness: }&a_{tipi} =& G + 8 - B\\
        &\text{Neuroticism: }&n_{tipi} =& I + 8 - D\\
    \end{aligned}
\end{equation}

To compute the personality consistency, we compare the above scores, after being scaled to $[1,5]$, with the main NPC personality specifications in the game JSON,
\begin{equation}
    d_{\{o,c,e,a,n\}} = \left|\frac{\{o,c,e,a,n\}_{tipi} + 1}{3} - \{o,c,e,a,n\}_{game}\right|.
\end{equation}
The PER score is the squared sum of these differences, normalized to $[0, 1]$,
\begin{equation}
\begin{aligned}
    \textbf{PER}_{\text{traj}} &= 1 - \frac{\sqrt{\sum_{x\in\{o,c,e,a,n\}} d_x^2}}{4\sqrt{5}}\\
    \textbf{PER} &= \frac{\sum_{\text{traj}} \textbf{PER}_\text{traj}}{\# trajectories}
\end{aligned}.
\end{equation}

\paragraph{Direct Evaluation of Personality Consistency}
We also experiment with a direct evaluation approach (referred to as \textbf{PER}$^d$), which instructs the LLM judge to provide a 1--5 alignment rating for each of the five personality traits. 
\begin{center}
\begin{minipage}{0.95\textwidth}
\begin{lstlisting}[language=plaintext, frame=none, numbers=none]
Assign a score from 1 to 5 to indicate how well the game narrative aligns with the main NPC's personality traits:
- Many Conflicts (1): The narrative frequently contradicts the NPC's personality.
- Some Conflicts (2): The narrative shows noticeable inconsistencies with the NPC's personality.
- Neutral (3): The narrative is only partially aligned or does not strongly reflect the NPC's personality.
- Strong Alignment (4): The narrative closely matches the NPC's personality, with only minor deviations or uncertainties.
- Perfect Alignment (5): The narrative flawlessly reflects the NPC's personality in every aspect, with no contradictions.

Please give one score for each personality trait, and provide a brief explanation for each score.

Game narrative:
{game_content}

NPC personality:
{npc_personality}

Please return a score as a json object following this schema:
{schema}
\end{lstlisting}
\end{minipage}
\end{center}
Here, \texttt{npc\_personality} consists of the Big Five personality traits in the game JSON. We compute the final score by averaging the normalized scores across all traits and, subsequently, across all trajectories. We deter discussions of results from this approach to Appendix~\ref{app:human_eval}, where we compare both TIPI estimations and direct evaluation results from LLM judges and human annotators. We refer this score as \textbf{PER$^d$} for the remaining of this article. 

\subsection{Interestingness (INT)}
We prompt an LLM judge to rate the interestingness of the generated content on a 1--5 scale.
\begin{center}
\begin{minipage}{0.95\textwidth}
\begin{lstlisting}[language=plaintext, frame=none, numbers=none]
Your task is to evaluate the **interestingness** of the following game content. Please give a score from 1 (least interesting) to 5 (most interesting), with a brief explanation of your rationale.


[[start of game content]]
{game_content}
[[end of game content]]

Please return your evaluation score in a json dictionary with the following format:
{schema}

Example output:
{example}
\end{lstlisting}
\end{minipage}
\end{center}
We normalize the final score to \([0, 1]\), sum over rounds within a trajectory and then average:
\begin{equation}
\begin{aligned}
    \textbf{INT}_{\text{traj}} &= \frac{int-1}{4}\\
    \textbf{INT} &= \frac{\sum_{\text{traj}} \textbf{INT}_\text{traj}}{\# trajectories}
\end{aligned}.
\end{equation}
\subsection{Action Choice Quality (ACT)}
At each round, the LLM judge scores three candidate actions on three rubrics: Diversity, Relevance, and Understandability, each on a 1--5 scale with the following prompt. 
\begin{center}
\begin{minipage}{0.95\textwidth}
\begin{lstlisting}[language=plaintext, frame=none, numbers=none]
Please act as an experienced RPG game player and evaluate the choices provided by the game engine, given the user-AI interaction history and the general game instruction outlining the basic game settings. You will be given the general game instruction, the interaction history, and the current choices offered by the game engine. Evaluate the choices based on the following rubric:

Please assess the choices provided by the game engine based on this rubric:

[[start of rubric]]
{rubric}
[[end of rubric]]

[[start of general game instruction]]
{game}
[[end of general game instruction]]

[[start of history]]
{history}
[[end of history]]

Please assess the choices provided by the game engine:

[[start of choices]]
{choices}
[[end of choices]]

Your output should be a JSON object structured as follows:
{
    "reason": <your reasoning here>
    "score": <score from 1 to 5 based on the rubric provided>
}
You must NOT output anything else other than this JSON object.
\end{lstlisting}
\end{minipage}
\end{center}
We provide in the prompt above the game JSON~(\texttt{game}), game history up to the current round~(\texttt{history}), and three candidate choices~(\texttt{choices}). We present the following paragraphs in \texttt{rubric} for Diversity, Relevance and Understandability respectively.
\begin{center}
\begin{minipage}{0.95\textwidth}
\begin{lstlisting}[language=plaintext, frame=none, numbers=none]
Diversity: Does the set of choices provide distinct and varied options for the player?
1: The choices are nearly identical, offering no meaningful differences between them.
2: The choices have slight variations but are mostly redundant, leading to a limited sense of variety.
3: The choices exhibit some diversity but may still overlap in intent or outcome.
4: The choices are mostly distinct and provide meaningful differences that allow the player to explore different paths.
5: The choices are highly diverse, with each option offering unique and creative directions for the player.
\end{lstlisting}
\end{minipage}
\end{center}
\begin{center}
\begin{minipage}{0.95\textwidth}
\begin{lstlisting}[language=plaintext, frame=none, numbers=none]
Relevance: Are the choices appropriate and contextually aligned with the story and scene?
1: The choices are entirely irrelevant, disconnected from the scene or story, and break immersion.
2: The choices have limited relevance, with some alignment to the story but containing jarring or out-of-place elements.
3: The choices are moderately relevant, generally aligning with the story but occasionally introducing inconsistencies.
4: The choices are mostly relevant, fitting well within the context and contributing meaningfully to the story.
5: The choices are fully relevant, seamlessly integrated into the story and enhancing the narrative experience.
\end{lstlisting}
\end{minipage}
\end{center}
\begin{center}
\begin{minipage}{0.95\textwidth}
\begin{lstlisting}[language=plaintext, frame=none, numbers=none]
Understandability:  Are the choices clear, concise, and easy to understand for the player?
1: The choices are confusing, overly complex, or poorly worded, making them difficult to interpret.
2: The choices are somewhat understandable but may include ambiguous language or unnecessary complexity.
3: The choices are moderately clear, with minor ambiguities that require some interpretation.
4: The choices are clear and concise, easy to read, and free of significant ambiguity.
5: The choices are exceptionally clear and well-written, making them effortless to understand and act upon
\end{lstlisting}
\end{minipage}
\end{center}

We average these three rubric scores to obtain $act$, then normalize via $(act - 1)/4$. Trajectories are evaluated by averaging per-round scores, and we then take the mean across all trajectories:
\begin{equation}
\begin{aligned}
    \textbf{ACT}_{\text{round}} &= \frac{act-1}{4}\\
    \textbf{ACT}_{\text{traj}} &= \frac{\sum_{\text{round}} \textbf{ACT}_\text{round}}{\# rounds}\\
    \textbf{ACT} &= \frac{\sum_{\text{traj}} \textbf{ACT}_\text{traj}}{\# trajectories}
\end{aligned}.
\end{equation}




\section{Human Evaluation Details}\label{app:human_eval}
\subsection{Interface Layout}
Figure~\ref{fig:human-eval} shows a screenshot of our human evaluation interface. Although it is cut off due to display size, the four main components are visible: \textbf{Text RPG Information}, \textbf{NPC Information}, \textbf{Dialog History}, and \textbf{Responses}. As discussed in Appendix~\ref{app:per_eval}, we use two interfaces: one for TIPI-based personality estimation and one for direct personality-consistency evaluation. These interfaces only differ in how the \textbf{NPC Information} and \textbf{Responses} sections are presented. To help annotators remain focused when assessing a multi-round trajectory, each round in a trajectory is annotated separately by the same annotator.

\noindent\textbf{Text RPG Information:}  
Annotators see the Game World description, the player character’s name and description, and the overall game objective. This information persists throughout the trajectory.

\noindent\textbf{Dialog History:}  
We show the game trajectory up to the current round, including the model’s narration and three candidate actions (boldfaced). One of these actions, selected at random, is displayed on the right side. This component updates every round to reflect the new content.

\noindent\textbf{NPC Information:}  
For TIPI-based personality estimation, we present only the NPC’s name and facts (omitting personality traits so they can be inferred through the TIPI questions). In the direct-evaluation interface, the main NPC personality traits are included here.

\noindent\textbf{Responses:}  
This section poses natural-language questions to gather human judgments on subjective dimensions. It differs slightly between TIPI-based and direct-evaluation interfaces, as detailed below.

\begin{figure}[!ht]
    \centering
    \includegraphics[width=0.95\linewidth]{figs/ScreenShot.png}
    \caption{Screenshot of the human evaluation interface.}
    \label{fig:human-eval}
\end{figure}

\subsection{Evaluation Questions}
\subsection{Evaluation Questions}
Our human evaluation asks annotators to rate various subjective aspects. Questions A--D appear every round in both the TIPI and direct-evaluation interfaces:
\begin{center}
\begin{minipage}{0.95\textwidth}
\begin{lstlisting}[language=plaintext, frame=none, numbers=none]
A. Please give a score (1-5) to indicate how interesting the game narrative is.

B. Do you think all the candidate actions are valid based on the game narrative? - 0 (no) - 1 (yes)
    
C. Are candidate choices different enough from each other, or are they essentially the same? - 0 (same) - 1 (different)
    
D. Please give a score (1-5) to measure whether the game narrative is consistent with the given facts about the main NPC? 
    - 1 has many conflicts
    - 2 has some conflicts
    - 3 neutral
    - 4 matches the description
    - 5 perfectly matches the description
\end{lstlisting}
\end{minipage}
\end{center}
These ratings inform the INT, ACT, and FAC metrics as follows:
\begin{equation}
    \begin{aligned}
        \textbf{INT}_\text{round} &= \frac{A - 1}{4},\\
        \textbf{ACT}_\text{round} &= \frac{B + C}{2},\\
        \textbf{FAC}_\text{round} &= \frac{D - 1}{4}.
    \end{aligned}
\end{equation}
Here, Question B corresponds to Relevance and Understandability in the ACT automatic evaluation, while Question C corresponds to Diversity. We average these round-level scores to obtain a trajectory-level score, then average across all trajectories.

\paragraph{Personality Consistency Questions (E1 and E2).}
We measure PER using two different question sets:
\begin{itemize}
    \item \textbf{E1: TIPI Estimation.}  
    Shown only once per trajectory (at the final round of the TIPI interface), requiring annotators to assess the entire trajectory.  
    \item \textbf{E2: Direct Evaluation.}  
    Appears at every round in the direct-evaluation interface.
\end{itemize}

Both methods yield PER scores analogous to the automatic evaluations in Appendix~\ref{app:per_eval}.

\begin{center}
\begin{minipage}{0.95\textwidth}
\begin{lstlisting}[language=plaintext, frame=none, numbers=none]
E1. Here are a number of personality traits that may or may not apply to the character. Please write a number to each statement to indicate the extent to which you agree or disagree with that statement, based ONLY on the game narratives. You should rate the extent to which the pair of traits applies to the character, even if one characteristic applies more strongly than the other. Use a score range of 1-7:
    - 1: Disagree strongly
    - 2: Disagree moderately
    - 3: Disagree a little
    - 4: Neither agree nor disagree
    - 5: Agree a little
    - 6: Agree moderately
    - 7: Agree strongly

    I see the main NPC as
    A. Extraverted, enthusiastic.
    B. Critical, quarrelsome.
    C. Dependable, self-disciplined.
    D. Anxious, easily upset.
    E. Open to new experiences, complex.
    F. Reserved, quiet.
    G. Sympathetic, warm.
    H. Disorganized, careless.
    I. Calm, emotionally stable.
    J. Conventional, uncreative

E2. Please give a score (1-5) to measure whether the game narrative is consistent with the given facts about the main NPC?
    - 1 has many conflicts
    - 2 has some conflicts
    - 3 neutral
    - 4 matches the description
    - 5 perfectly matches the description
\end{lstlisting}
\end{minipage}
\end{center}
\subsection{Annotation Setup}
We recruited 15 human annotators. Each trajectory is annotated at the round level, resulting in two annotations per interface type and therefore four total annotations per trajectory. We ensure that each annotator encounters any given trajectory only once, regardless of interface type. Consequently, each trajectory ends up with four sets of INT, ACT, and FAC scores, and two sets of PER and \textbf{PER}$^d$ scores. We take the mean over all trials to produce the final reported values. For inter-annotator agreement (Table~\ref{tab:gs_corr}), we randomly divide the collected annotations into two groups and compare their scores.

\subsection{PER vs. PER$^d$ Evaluation Results}
\begin{table*}[!ht]
\centering
\begin{tabular}{lrrrrrr}
\toprule
\multirow{2}{*}{Model} & \multicolumn{2}{c}{PER (Subset)} & \multicolumn{2}{c}{PER$^d$ (Subset)} & \multicolumn{1}{c}{PER (Full)} & \multicolumn{1}{c}{PER$^d$ (Full)}\\ 
& \multicolumn{1}{c}{Auto} & \multicolumn{1}{c}{Human} & \multicolumn{1}{c}{Auto} & \multicolumn{1}{c}{Human} & \multicolumn{1}{c}{Auto} & \multicolumn{1}{c}{Auto}\\
\midrule
Claude 3.5 Sonnet & 0.729 & 0.648 & 0.768 & \textbf{0.832} & 0.589 & 0.738 \\
Deepseek V3 & 0.742 & 0.645 & 0.750 & \underline{0.826} & 0.583 & \textbf{0.778}\\
Gemini 1.5 Pro & 0.740 & 0.648 & \textbf{0.800} & 0.769 & \underline{0.596} & \underline{0.777}\\
Gemini 2.0 Flash Exp & 0.737& \underline{0.651} & 0.707 & 0.769 &\textbf{0.598} & 0.750\\
GPT 4o & 0.711 & \textbf{0.667} & 0.780 & 0.724 & 0.585 & 0.768 \\
GPT 4o mini & \textbf{0.753}  & 0.648 & \underline{0.788} & 0.735 & 0.588 & 0.763\\
Llama 3.1 70B & \underline{0.744}  & 0.627 & 0.768 & 0.752 & 0.586 & 0.765\\
Llama 3.3 70B & 0.739 & 0.640 & 0.739 & 0.755 & 0.585 &0.774\\
\bottomrule
\end{tabular}
\caption{PER and PER$^d$ results from automatic and human evaluation on a subset of 20 games, and automatic evaluation on the full set of games.}
\label{tab:per_evaluation_results}
\end{table*}


\begin{table}[!ht]
    \centering
    \begin{tabular}{cc|rrr}
    \toprule
    \multicolumn{2}{c|}{Comparisons} & Pearson & Kendall & MAD \\
    \midrule
    \multirow{1}{*}{Auto-Auto Agreement} & PER Auto - PER$^d$ Auto & 0.013 & 0.109 & 0.037 \\
    \midrule
    \multirow{3}{*}{Auto-Human Agreement} & PER Auto - PER Human     & -0.691 & -0.429 & 0.090\\
    & PER$^d$ Auto - PER$^d$ Human &-0.297 & -0.255 & 0.047\\
    \midrule
    \multirow{3}{*}{Human-Human Agreement} &PER Human - PER Human & -0.310 & -0.286 & 0.023\\
    &PER$^d$ Human - PER$^d$ Human & 0.649 & 0.143 & 0.035\\
    &PER Human - PER$^d$ Human &-0.175&-0.143 & 0.124\\
    \bottomrule
    \end{tabular}
    \caption{Agreement analysis for PER and PER$^d$ scores. We present Pearson correlation coefficient (Pearson), Kendall rank correlation coefficient (Kendall), and Mean Absolute Difference (MAD)}
    \label{tab:per_agreement}
\end{table}

In our main article, we adopt the PER score for evaluating NPC personality consistency. Here, we further analyze both PER and PER$^d$ scores from automatic and human evaluations on a subset of 20 games in Table~\ref{tab:per_evaluation_results}, with additionally automatic evaluation results on the full dataset. We also report agreement metrics in Table~\ref{tab:per_agreement} Our analysis reveals several key observations:

\begin{enumerate}
    \item \textbf{PER$^d$ tends to be higher than PER in both automatic and human evaluations.} Across models, we observe that PER$^d$ scores are consistently higher than PER scores, indicating that direct evaluation of personality consistency is generally more lenient than the TIPI-based method. This trend holds for both automatic and human evaluators.

    \item \textbf{LLMs achieve similar PER scores across the dataset.} The automatic PER and PER$^d$ scores on the full set of games show little variation across models, with all models achieving scores around 0.58–0.60 for PER and around 0.74–0.78 for PER$^d$. This suggests that models perform comparably in terms of maintaining personality consistency in text-based role-playing.

    \item \textbf{Human evaluators rate PER$^d$ higher than PER, but with noticeable variation.} While automatic evaluations show a clear gap between PER and PER$^d$, human annotations exhibit a similar pattern but with greater variability. Notably, human evaluators assign significantly higher PER$^d$ scores to some models, such as Claude 3.5 Sonnet and DeepSeek V3, compared to their automatic scores.

    \item \textbf{Human and automatic PER scores exhibit poor correlation.} Table~\ref{tab:per_agreement} shows that the Pearson correlation between PER Auto and PER Human is negative (-0.691), with Kendall correlation also negative (-0.429). This suggests a fundamental mismatch between how LLM-based and human evaluators assess personality consistency through TIPI.

    \item \textbf{Better human agreement for PER$^d$, but still unstable.} While inter-human correlation for PER is negative (-0.310 Pearson, -0.286 Kendall), PER$^d$ exhibits a stronger but still weak agreement (0.649 Pearson). This suggests that directly rating personality alignment may be more intuitive for human evaluators than using TIPI scores but remains somewhat unstable. However, there is still concern over whether human annotators are capable of accurately understanding Big Five traits in the direct evaluation scenario.

    \item \textbf{Low agreement between PER and PER$^d$.} The Pearson correlation between PER and PER$^d$ scores (both automatic and human) is low (0.013 for Auto-Auto and -0.175 for Human-Human), indicating that these two evaluation methods capture different aspects of personality consistency. While PER$^d$ measures direct alignment with given traits, PER (TIPI) estimates personality traits implicitly, which may introduce more variance in judgments.
\end{enumerate}

\paragraph{Justification for Choosing TIPI (PER) in the Main Article.}  
We adopt TIPI-based personality consistency (\textbf{PER}) rather than direct evaluation (\textbf{PER}$^d$) in the main study for several reasons. First, TIPI does not require evaluators to have prior knowledge of the Big Five personality traits, making it a structured and interpretable method for assessing personality consistency. Additionally, the high variance in PER$^d$ human scores (as seen in Table~\ref{tab:per_agreement}) suggests that direct personality evaluation is more susceptible to subjective biases. The negative correlation between automatic and human PER scores further emphasizes the challenge of aligning LLM-based and human-based assessments, reinforcing the need for a more systematic approach like TIPI.

Overall, these results highlight the complexity of evaluating personality consistency, where different evaluation paradigms yield divergent results. The instability in human-human agreement for both PER and PER$^d$ suggests that subjective evaluation remains a challenging aspect of LLM benchmarking, warranting further research into more reliable personality evaluation methodologies.






%%%%%%%%%%%%%%%%%%%%%%%%%%%%%%%%%%%%%%%%%%%%%%%%%%%%%%%%%%%%%%%%%%%%%%%%%%%%%%%
%%%%%%%%%%%%%%%%%%%%%%%%%%%%%%%%%%%%%%%%%%%%%%%%%%%%%%%%%%%%%%%%%%%%%%%%%%%%%%%


\end{document}


% This document was modified from the file originally made available by
% Pat Langley and Andrea Danyluk for ICML-2K. This version was created
% by Iain Murray in 2018, and modified by Alexandre Bouchard in
% 2019 and 2021 and by Csaba Szepesvari, Gang Niu and Sivan Sabato in 2022.
% Modified again in 2023 and 2024 by Sivan Sabato and Jonathan Scarlett.
% Previous contributors include Dan Roy, Lise Getoor and Tobias
% Scheffer, which was slightly modified from the 2010 version by
% Thorsten Joachims & Johannes Fuernkranz, slightly modified from the
% 2009 version by Kiri Wagstaff and Sam Roweis's 2008 version, which is
% slightly modified from Prasad Tadepalli's 2007 version which is a
% lightly changed version of the previous year's version by Andrew
% Moore, which was in turn edited from those of Kristian Kersting and
% Codrina Lauth. Alex Smola contributed to the algorithmic style files.
