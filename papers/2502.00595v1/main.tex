%%%%%%%% ICML 2025 EXAMPLE LATEX SUBMISSION FILE %%%%%%%%%%%%%%%%%

\documentclass{article}

% Recommended, but optional, packages for figures and better typesetting:
\usepackage{microtype}
\usepackage{graphicx}
\usepackage{subfigure}
\usepackage{booktabs} % for professional tables

% hyperref makes hyperlinks in the resulting PDF.
% If your build breaks (sometimes temporarily if a hyperlink spans a page)
% please comment out the following usepackage line and replace
% \usepackage{icml2025} with \usepackage[nohyperref]{icml2025} above.
\usepackage{hyperref}
\usepackage{enumitem}

% Attempt to make hyperref and algorithmic work together better:
\newcommand{\theHalgorithm}{\arabic{algorithm}}
% \usepackage{algorithm}

% Use the following line for the initial blind version submitted for review:
% \usepackage{icml2025}
% If accepted, instead use the following line for the camera-ready submission:
\usepackage[accepted]{icml2025}

% For theorems and such
\usepackage{amsmath}
\usepackage{amssymb}
\usepackage{mathtools}
\usepackage{amsthm}

% if you use cleveref..
\usepackage[capitalize,noabbrev]{cleveref}

%%%%%%%%%%%%%%%%%%%
% Custom packages
%%%%%%%%%%%%%%%%%%%%
\usepackage[T1]{fontenc}
\usepackage[utf8]{inputenc}
\usepackage{listings}
\usepackage{xcolor}
\usepackage{multirow}

\newcommand{\benchmark}{\textsc{RPGBench}}

% Define a custom style for JSON
\lstdefinelanguage{json}{
    showstringspaces=false,
    breaklines=true,
    frame=single,
    basicstyle=\ttfamily\footnotesize,
    stringstyle=\color{brown},
    numberstyle=\tiny\color{gray},
    keywordstyle=\color{blue}\bfseries,
    commentstyle=\color{gray},
    morestring=[b]"
}
\lstdefinelanguage{plaintext}{
    showstringspaces=false,
    breaklines=true,
    frame=single,
    basicstyle=\ttfamily\footnotesize,
    % morestring=[b]"
}

\lstset{
  tabsize=2,
  numbers=left,
  stepnumber=1,
  firstnumber=1,
  numbersep=6pt
}

%%%%%%%%%%%%%%%%%%%%%%%%%%%%%%%%
% THEOREMS
%%%%%%%%%%%%%%%%%%%%%%%%%%%%%%%%
\theoremstyle{plain}
\newtheorem{theorem}{Theorem}[section]
\newtheorem{proposition}[theorem]{Proposition}
\newtheorem{lemma}[theorem]{Lemma}
\newtheorem{corollary}[theorem]{Corollary}
\theoremstyle{definition}
\newtheorem{definition}[theorem]{Definition}
\newtheorem{assumption}[theorem]{Assumption}
\theoremstyle{remark}
\newtheorem{remark}[theorem]{Remark}

% Todonotes is useful during development; simply uncomment the next line
%    and comment out the line below the next line to turn off comments
%\usepackage[disable,textsize=tiny]{todonotes}
\usepackage[textsize=tiny]{todonotes}

% The \icmltitle you define below is probably too long as a header.
% Therefore, a short form for the running title is supplied here:
\icmltitlerunning{RPGBench}

\begin{document}

\twocolumn[
\icmltitle{\benchmark{}: 
Evaluating Large Language Models as Role-Playing Game Engines}

% It is OKAY to include author information, even for blind
% submissions: the style file will automatically remove it for you
% unless you've provided the [accepted] option to the icml2025
% package.

% List of affiliations: The first argument should be a (short)
% identifier you will use later to specify author affiliations
% Academic affiliations should list Department, University, City, Region, Country
% Industry affiliations should list Company, City, Region, Country

% You can specify symbols, otherwise they are numbered in order.
% Ideally, you should not use this facility. Affiliations will be numbered
% in order of appearance and this is the preferred way.
\icmlsetsymbol{equal}{*}

\begin{icmlauthorlist}
\icmlauthor{Pengfei Yu}{comp}
\icmlauthor{Dongming Shen}{comp}
\icmlauthor{Silin Meng}{comp}
\icmlauthor{Jaewon Lee}{comp}
\icmlauthor{Weisu Yin}{comp}
\icmlauthor{Andrea Yaoyun Cui}{sch}
\icmlauthor{Zhenlin Xu}{comp}
%\icmlauthor{}{sch}
\icmlauthor{Yi Zhu}{comp}
\icmlauthor{Xingjian Shi}{comp}
\icmlauthor{Mu Li}{comp}
\icmlauthor{Alex Smola}{comp}
%\icmlauthor{}{sch}
%\icmlauthor{}{sch}
\end{icmlauthorlist}

% \icmlaffiliation{yyy}{Department of XXX, University of YYY, Location, Country}
\icmlaffiliation{comp}{Boson AI}
\icmlaffiliation{sch}{University of Illinois Urbana Champaign}

\icmlcorrespondingauthor{Pengfei Yu}{pengfei@boson.ai}

% You may provide any keywords that you
% find helpful for describing your paper; these are used to populate
% the "keywords" metadata in the PDF but will not be shown in the document
\icmlkeywords{Role-Playing, Evaluation, Benchmark}

\vskip 0.3in
]

% this must go after the closing bracket ] following \twocolumn[ ...

% This command actually creates the footnote in the first column
% listing the affiliations and the copyright notice.
% The command takes one argument, which is text to display at the start of the footnote.
% The \icmlEqualContribution command is standard text for equal contribution.
% Remove it (just {}) if you do not need this facility.

\printAffiliationsAndNotice{}  % leave blank if no need to mention equal contribution
% \printAffiliationsAndNotice{\icmlEqualContribution} % otherwise use the standard text.

\begin{abstract}
Testing Autonomous Driving Systems (ADS) is crucial for ensuring their safety, reliability, and performance. Despite numerous testing methods available that can generate diverse and challenging scenarios to uncover potential vulnerabilities, these methods often treat ADS as a black-box, primarily focusing on identifying system failures like collisions or near-misses without pinpointing the specific modules responsible for these failures. Understanding the root causes of failures is essential for effective debugging and subsequent system repair. We observed that existing methods also fall short in generating diverse failures that adequately test the distinct modules of an ADS, such as perception, prediction, planning and control.

To bridge this gap, we introduce \tool, the first root-cause-aware testing method for ADS. Unlike previous approaches, \tool not only generates scenarios leading to collisions but also showing which specific module triggered the failure. This method targets specific modules, creating test scenarios that highlight the weaknesses of these given modules. Specifically, our approach involves designing module-specific oracles to ascertain module failures and employs a module-directed testing strategy that includes module-specific feedback, adaptive seed selection, and mutation. This strategy guides the generation of tests that effectively provoke module-specific failures. We evaluated \tool across four critical ADS modules and four testing scenarios. The results demonstrate that our method can effectively and efficiently generate scenarios where errors in targeted modules are responsible for ADS failures. It generates 216.7 expected scenarios in total, while the best-performing baseline detects only 79.0 scenarios. 
Our approach represents a significant innovation in ADS testing by focusing on identifying and rectifying module-specific errors within the system, moving beyond conventional black-box failure detection.
\end{abstract}

\keywords{Module-Specific Failure, Autonomous Driving System, Testing}
\section{Introduction}\label{sec:introduction}
% -- Outline
% ---- LLMs are popular
% ---- There're many stakeholders in the training and inference loop
% ---- Adversaries in the training loop are a problem -- malpractice, poisoning
% ---- Also, showing compliance
% ---- Need a framework to prove the integrity of the pipeline
% ---- Enter Atlas

% ---- LLMs are popular
In recent years, machine learning (ML) models, have become increasingly popular.
The pervasive use of large language models (LLMs), in particular, and multi-stakeholder
involvement in model creation and deployment exacerbate security and privacy risks.
These considerations are emphasized by the global nature and the complexity of
large-scale ML deployments with different lifecycle stages:
%gathering and sanitizing the data from different sources,
%training and inferencing across many data centers,
%compliance with local laws or corporate policies.

% ---- There're many stakeholders in the training and inference loop
%Additionally, different stages of the ML development pipeline come with their own stakeholders:
\begin{enumerate}[label=\arabic*)]
    \item Collection and sanitation of a \emph{training} dataset from several public and proprietary sources.
    %\item Solicitation and facilitation of training.
    \item Provisioning of the training environment (hardware and software).
    \item Execution of training across many data centers.
    \item Construction of a \emph{testing} dataset from several sources, and the evaluation.
    \item Deployment and use of the model for inference that is compliant with local laws or corporate policies.
    %\item Use of the model in compliance with local laws or corporate policies.
\end{enumerate}

% ---- Adversaries in the training loop are a problem -- malpractice, poisoning
Each of these stages is vulnerable to malicious or dishonest parties.
For example, data can be poisoned~\cite{biggio2012poisoning,carlini2024poisoning} during collection or training.
Service providers executing outsourced training can shorten or omit critical steps to reduce their cost.
Model providers can serve smaller models in SaaS, or even distribute malicious ones.

% ---- Also, showing compliance
On the other hand, responsible model builders and other stakeholders may be incentivised or required to provide security and trust guarantees.
They may want to prove low bias in their training data, offer easily verifiable performance claims, or guarantee end-to-end integrity of the model creation in high risk domains.

% ---- Need a framework to prove the integrity of the pipeline
To address these challenges, it is necessary to guarantee the integrity of the entire ML lifecycle --
beginning with the data, through the training, and finally, the evaluation and deployment.
Was the data modified?
Did the hardware and software environment adhere to the specification?
Did the contractor follow the specified training procedure?
Can I trust the evaluation?
How can I guarantee that I am interacting with the intended model?
These are example questions that showcase the breadth of the involved challenges that must be tackled to provide end-to-end security.

% --- Enter Atlas
In this work, we introduce \atlas, a framework for enhancing the security and transparency of the lifecycle of ML models.
\atlas establishes the baseline of fundamental components and capabilities needed for comprehensive provenance tracking
at each stage of the ML lifecycle.
Subsequently, \atlas defines the core integrity requirements for verifiable ML lifecycle transparency.
We provide a reference implementation that instantiates \atlas using hardware-based security mechanisms -- with trusted execution environment (TEE),
including attestations.% , and comprehensive metadata-based provenance tracking.
%Our implementation satisfies all \atlas requirements.

We claim the following contributions:
\begin{enumerate}[label=\arabic*.]\label{sec:introduction:contributions}
    \item We introduce \atlas, a framework designed for end-to-end ML lifecycle transparency.
    \item We instantiate \atlas using TEEs and metadata-based provenance tracking.
    \item We evaluate our \atlas prototype through two case studies:
        \begin{enumerate*}[label=\arabic*)]
            \item fine-tuning of a BERT model~\cite{lin2023metabert, lin2023metabertimpl};
            \item fine-tuning of a bge-reranker model~\cite{chen2023bge}
        \end{enumerate*}.
\end{enumerate}

%\msm{revise: Integrate this motivation into intro}
%Organizations frequently leverage pre-trained models, outsource training processes, and integrate components from multiple sources,
%making it difficult to verify the authenticity and trustworthiness of their ML systems. This complexity is further compounded
%by the potential for malicious modifications at various stages of the model lifecycle, from data preparation through deployment.
%The involvement of various third parties in ML model development and deployment
%creates critical challenges in ensuring supply chain integrity.
%
%While Software Bills of Materials (SBOMs) and AI Bills of Materials (AI BOMs) provide basic inventory tracking for model components,
%they fall short in addressing the dynamic nature of ML pipelines. These approaches typically offer point-in-time snapshots but
%fail to capture the complex transformations, fine-tuning operations, and runtime modifications that characterize modern ML workflows.
%Additionally, they lack cryptographic guarantees about the integrity of recorded information and cannot effectively track the provenance
% of model weights and training data.
%
% These approaches demonstrate the growing importance of ML supply chain security.
% However, they are typically applied in an ad-hoc fashion, highlighting the need
% for a more integrated approach that combines comprehensive lineage tracking,
% strong cryptographic properties, and practical integration capabilities with existing ML development and deployment pipelines.
%
%A comprehensive solution requires not just documentation of components, but verifiable evidence of their origins,
%transformations, and integrity throughout the entire model lifecycle. This need has driven interest in more robust
%provenance tracking mechanisms that can:
%
%\begin{itemize}
%\item Provide cryptographic proof of model lineage
%\item Track and verify all pipeline transformations
%\item Maintain tamper-evident records of training processes
%\item Ensure integrity of model artifacts across organizational boundaries
%\end{itemize}
%
%Several existing tools and frameworks
%commonly focusing on different components of the model lifecycle and provenance tracking.
%While these solutions offer valuable capabilities, they often address only specific parts of the end-to-end ML
%supply chain rather than providing comprehensive coverage.
%\msm{end-revise}
%
%\todo{add discussion of EU-CRA AI Act requirements for model documentation and FDA guidelines for AI/ML in healthcare}

%The remainder of this paper is organized as follows:
%in Section~\ref{sec:background-related} we provide an overview of the necessary background, and the related work;
%Section~\ref{sec:problem} presents the challenge of providing integrity in the ML pipeline, the threat model, and the system assumptions;
%in Section~\ref{sec:framework} we present \atlas -- our framework for providing ML integrity;
%Section~\ref{sec:implementation} covers implementation details;
%in Section~\ref{sec:eval}, we show that \atlas is effective across three dimensions: training overhead $<8\%$, the verification time increases linearly with the size of the model, and it is compatible with PyTorch and Tensorflow;
%in Section~\ref{sec:casestudies} we present the case studies;
%in Section~\ref{sec:discussion} we discuss additional considerations for \atlas,
%and Section~\ref{sec:conclusion} concludes the paper and provides directions for future work.

\section{Related Work}
\label{sec:related}

Recent advances~\cite{lecun2015deep, zaidi2022survey} in deep learning have vastly improved object detection and instance segmentation results in the terrestrial domain. 
Such progress has been achieved by developing effective designs of models and training them with large datasets~\cite{lin2014microsoft, russakovsky2015imagenet} containing millions of images and corresponding labels. 
Even with such advances, detecting underwater debris still remains challenging. 
While~\cite{fulton2019robotic} presents the first deep learning based approach to detect underwater debris and outperforms previous non deep learning approaches, the accuracy is worse than general object detection tasks due to a small training dataset. 
To increase the debris detection accuracy,~\cite{hong2020trashcan} proposes a larger dataset, TrashCan, which has both bounding box and pixel-level annotations for object detection and instance segmentation along with baseline results using Mask R-CNN~\cite{he2017mask} and Faster R-CNN~\cite{ren2015faster}. 
However, increasing the dataset size to improve debris detection accuracy further is not scalable due to debris data scarcity and labeling costs. 
To overcome the data scarcity issue,~\cite{hong_generative_2020} proposes a generative method, augmenting the existing dataset with synthetic underwater debris images. 
While the method can create realistic synthetic images, it still requires additional labeling efforts to be used for training detectors. 

Style transfer~\cite{singh_neural_2021,jing_neural_2020} is an approach for changing the appearance of one image based on the visual style of another. 
\cite{rodriguez_domain_2019, yu_sc-uda_2022} use this to improve detection in images taken from various domains (\eg different light conditions and image clarity). 
They aim to account for low-level texture changes in images by updating them to have the same style throughout the data. 
\cite{kadish_improving_2021} also attempts to improve detection using style transfer, by having the detector learn high-level features (\eg object shape) instead of low-level features (\eg the texture of paintings). 
\cite{amirkhani_enhancing_2021} uses style transfer to simulate various types of noise that may be present in real-world data. 
\cite{lin_gan-based_2021,liu_lane_2020} use style transfer to imitate varying light conditions. 
Style transfer has been applied beyond RGB images; \eg\cite{cygert_style_2019} converts RGB images from COCO dataset~\cite{lin2014microsoft} to thermal images and uses them to train a thermal image detector. 
While style transfer works well in augmenting the appearance of an image, it does not add new objects to our data.


Unlike style transfer, image blending based methods allow placing new objects anywhere on target background images. 
\cite{perez_poisson_2003} introduces Poisson editing using Laplacian information to smooth the boundary between the image patches and target images. 
\cite{wu_gp-gan_2019} uses a GAN-based approach for image blending, producing realistic images; however, it requires image pairs of empty backgrounds and objects placed in the backgrounds to train, limiting its use when the source data is limited. 
\cite{georgakis_synthesizing_2017} modifies~\cite{perez_poisson_2003} to find spaces within a given image plane to blend an object. 
However, detectors trained with their synthetic data show degraded performance on real data due to the style discrepancy between the blended objects and backgrounds in the dataset.
\cite{zhang_training_2022} uses a harmonization blending approach to create new data for aerial search and rescue, but it does not blend the boundary of target objects. 

\cite{zhang_deep_2020} presents a two-stage deep network-based approach to blend an image patch onto a background. Unlike~\cite{wu_gp-gan_2019} their approach does not need additional training data to generate blended images.
\begin{figure}  
    \centering
    \scalebox{0.75}{\tikzset{every picture/.style={line width=0.75pt}} 

\begin{tikzpicture}[x=0.75pt,y=0.75pt,yscale=-1,xscale=1]
\draw (-490,101) node  {\includegraphics[width=0.25\textwidth]{imgs/IBURD_firstpass.png}};
\draw (-310,101) node  {\includegraphics[width=0.25\textwidth]{imgs/DIB_secondpass.png}};
\draw (-130,101) node  {\includegraphics[width=0.25\textwidth]{imgs/IBURD_secondpass.png}};

\draw (-560,192) node [anchor=north west][inner sep=0.75pt]   [align=left] {{\fontfamily{helvet}\selectfont Poisson Image Editing}};
\draw (-380,192) node [anchor=north west][inner sep=0.75pt]   [align=left] {{\fontfamily{helvet}\selectfont Deep Image Blending}};
\draw (-180,192) node [anchor=north west][inner sep=0.75pt]   [align=left] {{\fontfamily{helvet}\selectfont IBURD (Ours)}};


\end{tikzpicture}}
    \caption{Comparison of generated images using three approaches:  Poisson image editing~\cite{perez_poisson_2003}, Deep image blending~\cite{zhang_deep_2020} and our method, IBURD. In our approach, we can successfully prevent over-stylization of the blended objects.}  
    \label{fig:compare}
    \vspace{-4mm}
\end{figure}

They use the proposed method mainly for artistic purposes and it struggles with blending transparent source images onto background images, as seen in Fig.~\ref{fig:compare}. 
The method is only tested with $20$ images and takes approximately $4$ minutes to blend one object in an image of size $512\times512$ pixels.

Our proposed approach, IBURD, allows us to place source images at various locations and scales in target background images with relevant bounding box and pixel-level annotations within $50$ seconds, which is $5$ times faster than~\cite{zhang_deep_2020}. 
Our method addresses blending transparent objects using Poisson editing, a situation that previous methods fail to cover.
Additionally, IBURD deals with object distortion due to excessive style transfer using Fast Fourier Transform (FFT)~\cite{liu_image_2008} based weight adjustment for loss.

\section{Dataset Collection}

In this section, we first introduce our game design, including the representations of game setup and mechanics. We then describe a two-stage data collection process for the Game Creation (GC) and Game Simulation (GS) tasks. In the first stage, we build a non-player character (NPC) pool from fictional character Wikipedia pages, and prompt various LLMs to create one game per NPC. An automatic game validity checker applies for selecting valid games. In the second stage, we assemble a test set of valid games for GS.

\subsection{Game Design}\label{sec:gd}

% A game is represented as a JSON dictionary, as illustrated in Figure~\ref{fig:rpebench-overview}, containing information on \textbf{Game World}, \textbf{Player Character Name}, \textbf{Player Character Description}, \textbf{Main NPC Name}, \textbf{Main NPC Description}, \textbf{Main NPC Personality} (using the Big Five traits), \textbf{Main NPC Facts}, \textbf{Game Objective}, \textbf{Game Scenes}, \textbf{State Variables}, \textbf{Game Events} and \textbf{Termination Conditions}.


The games in \benchmark{}, as illustrated in Figure~\ref{fig:rpebench-overview}, are structured around several core components that create a text-based role-playing game (RPG) experience. This design ensures sufficient flexibility for diverse storytelling while maintaining support for objective mechanic evaluation:

\begin{itemize} 
\item \textbf{Game World}: The overarching setting where the story unfolds (e.g., "Gotham City"). 
\item \textbf{Player Character}: The protagonist controlled by the player, including a name and description (e.g., "Ann," a detective and ally of Batman). 
\item \textbf{Main NPC}: A key non-player character controlled by the game engine, characterized by a name, description, Big Five personality traits, and relevant facts (e.g., "Bruce Wayne (Batman)"). 
\item \textbf{Game Objective}: The primary goal to accomplish.
% (e.g., "Uncover the mastermind behind a conspiracy"). 
\item \textbf{Game Scenes}: Distinct locations where events occur.
% (e.g., "Wayne Manor"). 
\end{itemize}


% \begin{itemize}
%     \item \textbf{Game World}: A textual description of the game world.
%     \item \textbf{Player Character Name}: Name of the player character.
%     \item \textbf{Player Character Description}: A brief textual description of the player character.
%     \item \textbf{Main NPC Name}: Name of the main NPC.
%     \item \textbf{Main NPC Description}: A brief textual description of the main NPC.
%     \item \textbf{Main NPC Personality}: The Big Five personality traits for the main NPC, each scored from 1--5 (Openness, Conscientiousness, Extraversion, Agreeableness, and Neuroticism).
%     \item \textbf{Main NPC Facts}: A list of simple facts related to the main NPC.
%     \item \textbf{Game Objective}: A short textual description of the game objective.
%     \item \textbf{Game Scenes}: A list of game scenes.
%     \item \textbf{State Variables}: A list of finite discrete state variables. Each has an initial value, ensuring the total number of possible states is finite.
%     \item \textbf{Game Events}: A list of game events.
%     \item \textbf{Termination Conditions}: A set of conditions that, when met, lead to either success or failure.
% \end{itemize}
% The full JSON structure is provided in the Appendix~\ref{app:game_json}.

% \noindent\textbf{Game Mechanics.} 
% We present mechanics via interactions between state variables and game events, specified by four properties of each event: \textbf{Entering Condition}, \textbf{Success Condition}, \textbf{Success Effect} and \textbf{Fail Effect}. All the conditions are boolean expressions of state variables, and all effects are value updates of state variables.
% \begin{itemize}[leftmargin=*]
%     \item : Conditions for the event to occur.
%     \item : Conditions for the event to succeed.
%     \item : Effects on state variables if succeeds.
%     \item : Effects on state variables if fails.
% \end{itemize}
The core \textbf{game mechanics} in \benchmark{} are structured around \textbf{event-state interactions}, which define how game events modify the game state:

\begin{itemize} 
    \item \textbf{State Variables} represent numerical values that track the game's dynamic elements, such as character skills and trust levels. These variables always have an initial value along with minimum and maximum bounds.
    \item \textbf{Game Events} drives game progression and modifies \textbf{state variables} upon execution. Each event has an \textbf{entering condition} (whether it can occur) and a \textbf{success condition} (whether it succeeds). Upon execution, an event applies either a \textbf{success effect} or, if applicable, a \textbf{fail effect}, updating the state variables accordingly.
    \item \textbf{Termination Conditions} specify when the game ends by evaluating specific state variable expressions. These conditions, checked before processing game events, determine whether the game ends with success or failure.
\end{itemize}

This design creates an interactive experience where player actions and game events influence the game state. An LLM serves as the game engine, creating the game, simulating the game world based on user actions. Further details, including the exact game JSON schema, are provided in Appendix~\ref{app:game_json}.



\subsection{Game Data Collection}\label{sec:gdc}

We select 100 fictional characters from Wikipedia to serve as the test set for GC. For each character, we prompt an LLM to create a JSON-formatted game (as specified above) that treats this character as the main NPC. We employ a 5-shot prompting approach, where the examples are generated by initially prompting \emph{GPT 4o} using a manually crafted game. The full prompt is provided in the Appendix~\ref{app:gc_prompt}.

We parse LLM outputs to ensure they conform to the JSON format. Any game that passes this format check is then tested for validity using a BFS-based checker (see Section~\ref{sec:evaluation_gc}), which confirms whether a game can end in both success and failure, and whether all events can be reached. All valid games from multiple models are collected for the GS task (Table~\ref{tab:data_stats} shows the distribution).

\begin{table}[!ht]
    \centering
    \begin{tabular}{lr}
    \toprule
    Game Source & \# of Valid Games (Out of 100)\\
    \midrule
    Claude 3.5 Sonnet & 1\\
    DeepSeek V3       & 38\\
    Gemini 1.5 Pro    & 4\\
    Gemini 2.0 Flash Exp & 33\\
    GPT 4o            & 49\\
    \midrule
    Total             & 125 \\
    \bottomrule
    \end{tabular}
    \caption{Generated Game Statistics}
    \label{tab:data_stats}
\end{table}


\section{Evaluation Metrics}\label{sec:evaluation}

\subsection{Game Creation Evaluation}\label{sec:evaluation_gc}

In GC, we evaluate an LLM's capability to create games that have good mechanics. This task requires complex reasoning over event-state interactions that is very challenging even for human. Section~\ref{sec:gdc} offered a broad overview of the GC task. We now define it more precisely.

\noindent\textbf{Task Definition [Game Creation]}  
Given a fictional character \(\mathcal{C}\) and related Wikipedia information \(\mathcal{R}\), an LLM must create a game \(\mathcal{G}\) that follows a predefined format \(\mathcal{J}\).

In \benchmark{}, 100 fictional characters are used, each with an associated Wikipedia page (\(\mathcal{R}\)), facilitating future expansion of the character pool. The game \(\mathcal{G}\) must conform to the structure \(\mathcal{J}\) given in Section~\ref{sec:gd}. We provide each LLM with a 5-shot prompt to generate one game per character. 

\begin{algorithm}[!ht]
   \caption{BFS Validity Checker}
   \label{alg:bfs-checker}
\begin{algorithmic}
   \STATE \textbf{Input:} Events \(\mathcal{E}\), each with entering and success conditions, plus success and fail effects; A state \(S_0\) with initial values for all variables; An integer \(M\) indicating the maximum number of states to be explored.

   \FUNCTION{\(\textit{isValid}(\mathcal{E},S_0,M)\)}
   \STATE Initialize a queue \(\mathcal{Q}\) and enqueue \(S_0\).
   \STATE Initialize a visited set \(\mathcal{V} = \{S_0\}\).
   \STATE Initialize a triggered-event set \(\mathcal{T} = \varnothing\).
   \STATE \(\textit{successFound} = \texttt{False};~ \textit{loseFound} = \texttt{false}\)

   \REPEAT
      \STATE \(S = \mathcal{Q}.\text{dequeue}()\)
      \IF{\(|\mathcal{V}| > M\)}
         \STATE \textbf{break} \quad\# \textit{Reached maximum search limit}
      \ENDIF
      
      \STATE \(\textit{availableEvents} = \{ e \in \mathcal{E} : e.\mathit{enterCond}(S) \}\)
      \FOR{each \(e \in \textit{availableEvents}\)}
         \STATE \(\mathcal{T} = \mathcal{T} \cup \{e\}\) \quad\# \textit{Mark event as triggered}

         \STATE \(S' = \textit{e.applyEffect}(S,\mathit{e.successCond}(S))\)
         
        \STATE \(\textit{successFound} \;|= \textit{e.isSuccessTermination}(S')\)
        \STATE \(\textit{loseFound} \;|= \textit{e.isLosingTermination}(S')\)

         \IF{\(S' \notin \mathcal{V}\)}
            \STATE \(\mathcal{Q}.\text{enqueue}(S')\);~\(\mathcal{V} = \mathcal{V} \cup \{S'\}\)
         \ENDIF
      \ENDFOR
   \UNTIL{\(\mathcal{Q}\) is empty}

   \STATE \textbf{return} 
      \(\bigl(\mathcal{T} = \mathcal{E}\bigr) \;|\; \textit{successFound} \;|\; \textit{loseFound}\)
   \ENDFUNCTION
\end{algorithmic}
\end{algorithm}

\paragraph{BFS Validity Checker}  
Once the output is confirmed to be valid JSON, we perform a BFS-based validity check (Algorithm~\ref{alg:bfs-checker}). Based on our event--state design, we employ BFS to decide if a game is valid. Starting from the initial state, we repeatedly check which events are available , apply success or failure effects accordingly, and track whether at least one success and one losing state can be reached. We stop when no new states can be discovered or when the search exceeds 10{,}000{,}000 states. A game is valid if every event is triggered at least once, and both success and losing termination conditions are achievable.

\paragraph{Metrics}  
For GC evaluation, we report the format-check pass rate (\textbf{FCR}) and the valid-check pass rate (\textbf{VCR}) as our main metrics, reflecting how reliably LLMs follow the prescribed JSON format and produce valid game mechanics. In order to examine fine-grained failures for the validity check, we include three additional ratios:
\begin{equation*}
\begin{aligned}
     \textbf{w. Success} &=\frac{\textit{\# games with successFound}}{\textit{\# games pass the format check}}\\
     \textbf{w. Lose}&=\frac{\textit{\# games with failFound}}{\textit{\# games pass the format check}}\\
     \textbf{Reachability}&=\frac{\textit{\# games without unreachable events}}{\textit{\# games pass the format check}}
\end{aligned}
\end{equation*}

\subsection{Game Simulation Evaluation}

Given a valid game, the GS task requires an LLM to simulate the game for a player. We introduce a multi-round simulation framework, based on which a comprehensive description of evaluation metrics is presented.

\paragraph{Game Simulation Framework}  
The simulation proceeds in multiple rounds of interaction with a (real or simulated) player. Before the first round, the LLM is given the complete game information and output instructions. Each round thereafter, the LLM outputs:
\begin{enumerate}
    \item \textbf{Event Plan}: A list of events occurring this round. Each entry specifies whether the event is starting (\texttt{start}) or ending (\texttt{end}); if ending, an \texttt{outcome} is either \texttt{success} or \texttt{failure}.
    \item \textbf{Game Narration}: A narrative description of the current round, concluding with three candidate actions for the player character. We prompt models to follow a play-script format for readability but do not enforce it during evaluation.
    \item \textbf{Game State}: The updated state variables after applying effects of any events that ended this round.
\end{enumerate}

\paragraph{Evaluation Metrics}  
Our evaluation covers multiple dimensions, scored over the trajectory of interactions. A simulated player selects one of the candidate actions at random each round.

\begin{enumerate}
    \item \textbf{Length}: We count words in the game narration (excluding candidate actions). Although no ideal length is defined, our prompt suggests remaining under 200 words to maintain brevity without sacrificing creativity. We report the average length per round.
    \item \textbf{Action Quality}: Using an LLM judge (prompt in Appendix~\ref{app:eval_prompt}), we rate the three candidate actions based on diversity, relevance, and clarity. The judge outputs a 1--5 score, normalized to \([0,1]\) via \(\frac{s-1}{4}\). We average scores across all rounds.
    \item \textbf{Interestingness}: An LLM judge evaluates how engaging the round’s narration is, assigning a 1--5 score also normalized to \([0,1]\). We average this score across the entire trajectory.
    \item \textbf{Role-Playing Factual Consistency}: We compare the game narration against each fact in the main NPC's fact list. An LLM judge labels each fact as \textit{align}, \textit{contradict}, or \textit{neutral}. We report the ratio \(\frac{\#\textit{align}}{\#\textit{align} + \#\textit{contradict}}\).
    \item \textbf{Role-Playing Personality Consistency}: We prompt an LLM to infer the main NPC's Big Five traits from the generated content, then compare these to the game definition. We employ the Ten-Item Personality Inventory (TIPI)~\cite{gosling2003very}, following previous work on eliciting LLM-based personality assessments of public figures~\cite{cao2024large}. In addition to TIPI, we also considered a direct approach that explicitly evaluates alignment between the game’s narrative and the NPC’s predefined traits. We use TIPI-based score in the main paper, with details on the direct approach and comparisons in the Appendix~\ref{app:eval_prompt} and ~\ref{app:human_eval}.
    \item \textbf{Game Mechanics}: We perform a fully automatic check for the following errors:
    \begin{enumerate}
        \item \textbf{Event Condition Error}: An event triggers when its entering condition is not met, or the outcome (success/failure) does not match the current state.
        \item \textbf{Variable Update Error}: The state variables do not update according to event effects.
    \end{enumerate}
    The main game mechanic metric we adopt is the round-level accurate rate $\textbf{MEC} = \frac{\# \textit{Rounds with no errors}}{\# \textit{Rounds}}.$
    % \begin{equation}
    %     \textbf{MechanicScore} = \frac{\# \textit{Rounds with no errors}}{\# \textit{Rounds}}.
    % \end{equation}
    We average the mechanic score over all games.
    
    For a more fine-grained analysis, we process events in the \textit{Event Plan} sequentially at each round and calculate an error rate for each error type,
    \begin{equation}
    \begin{aligned}
        \textbf{ECE}_t &= \frac{\# \textit{Event condition errors}}{\# \textit{Events}}\\
        \textbf{VUE}_t &= \frac{\# \textit{State variables incorrectly updated}}{\# \textit{State variables}}
    \end{aligned}
    \end{equation}    
    We average ECE and VUE over all rounds of all games. By design, all these metrics require no LLM judge. 
\end{enumerate}


\section{Results and Discussions}
\subsection{Experimental Details}
\noindent\textbf{Game Creation} We consider GC to be a challenging task requiring complex reasoning over event-state interactions. Consequently, we evaluate advanced models with stronger reasoning capabilities: Claude 3.5 Sonnet, DeepSeek V3, Gemini 1.5 Pro, Gemini 2.0 Flash Exp, and GPT 4o.\footnote{Although models featuring inference-time reasoning can produce higher-quality results, the computational cost of running these models is often prohibitively high in practice.} We apply greedy decoding for all GC evaluations.

\noindent\textbf{Game Simulation} In addition to the models used in GC, we include GPT 4o mini, Llama 3.1 70B Instruct, and Llama 3.3 70B Instruct for the GS evaluation. Unless otherwise noted, we use a sampling temperature of 0.2 for inference. To maintain computational feasibility and fit within effective context windows of all models, we terminate all simulations after the 10th round for the main experiments. For all metrics requiring an LLM judge, we use GPT-4o as the evaluator.


\subsection{Game Creation Results}

\paragraph{Main Results}Table~\ref{tab:gc_eval_main} reports the format-check pass rate (FCR) and validity-check pass rate (VCR). We mark Claude 3.5 Sonnet with an asterisk (``*'') because it frequently refuses to generate content, often citing an “over-lengthy output” error, causing 95\% of its responses to fail the format check. We therefore focus on the fine-grained validity statistics for the remaining four models.

Most models (other than Claude 3.5 Sonnet) achieve high FCRs, indicating that they generally follow the specified formatting instructions. Among these models, GPT-4o attains the highest VCR of 0.49, while Gemini 1.5 Pro shows the lowest VCR of 0.04. Because passing the validity check demands a careful design of state variables and event systems, GPT-4o’s stronger planning and reasoning capabilities are highlighted in this task. A closer inspection on fine-grained metrics (w. Success, w. Lose and Reachability) reveals that Gemini 1.5 Pro frequently produces games that stall at intermediate steps without reaching success or failure endings. DeepSeek V3, in contrast, typically generates coherent event sequences, while GPT-4o often provides well-structured games with proper terminal outcomes.

\paragraph{Game Difficulty Analysis} While VCR predominantly measures the logical consistency of generated games, game difficulty is another vital factor. Our game design allows us to estimate difficulty by analyzing (1) the ratio of success terminations to losing terminations and (2) the ratio of the lengths of the event chains leading to these endings. Formally, for a valid game $v$, let $\mathcal{S}(v)$ be the set of all discovered success terminations and $\mathcal{L}(v)$ the set of losing terminations. For each trajectory $t$, let $\textit{length}(\cdot)$ denote the number of events in $t$. We define:

\begin{equation*}
    \begin{aligned}
        \textbf{CountRatio} &= \frac{|\mathcal{S}(v)|}{|\mathcal{L}(v)|} \\
        % \textbf{LengthRatio} &= \frac{\frac{\sum_{l\in\mathcal{L}_v} \textit{length}(l)}{|\mathcal{L}(v)|}}{\frac{\sum_{s\in\mathcal{S}_v} \textit{length}(s)}{|\mathcal{S}(v)|}}
        \textbf{LengthRatio} &= \frac{\sum_{l\in\mathcal{L}_v} \textit{length}(l)}{\sum_{s\in\mathcal{S}_v} \textit{length}(s)}\cdot \frac{|\mathcal{S}(v)|}{|\mathcal{L}(v)|}
    \end{aligned}
\end{equation*}

Intuitively, higher values for either ratio indicate an easier game. Figures~\ref{fig:gc_eval_cr} and \ref{fig:gc_eval_lr} show box plots of these ratios for three selected models. Our analysis reveals that all models generate games with a relatively balanced number of winning and losing trajectories. However, Gemini 2.0 Flash Exp tends to produce games where losing requires more steps, making failure less immediate. Additionally, the average \textbf{LengthRatio} is consistently below 1 across all models, indicating that winning generally requires more steps than losing—an expected outcome, as successful completion of a game typically demands more strategic progression.

\begin{table*}[!ht]
    \centering
    \begin{tabular}{lrrrrr}
    \toprule
    Models & FCR $\uparrow$& VCR $\uparrow$& w. Success & w. Lose &  Reachability\\
    \midrule
    Claude 3.5 Sonnet* & 0.050 & 0.010 & / & / & /\\
    DeepSeek V3       & 0.990 & 0.380 & 0.455 & 0.545 & \textbf{0.828}\\
    Gemini 1.5 Pro    & 0.850 & 0.040 & 0.060 & 0.080 & 0.610\\
    Gemini 2.0 Flash Exp & \textbf{1.000} & 0.330 & 0.420 & 0.680 & 0.480\\
    GPT 4o            & 0.960 & \textbf{0.490} & \textbf{0.656} & \textbf{0.771} & 0.656\\
    \bottomrule
    \end{tabular}
    \caption{Game Creation results.}
    \label{tab:gc_eval_main}
\end{table*}

\begin{figure}[!t]
    \centering
    \includegraphics[width=0.8\linewidth]{figs/CountRatio.pdf}
    \caption{CountRatio of three models}
    \label{fig:gc_eval_cr}
\end{figure}
\begin{figure}[!t]
    \centering
    \includegraphics[width=0.8\linewidth]{figs/LengthRatio.pdf}
    \caption{LengthRatio of three models}
    \label{fig:gc_eval_lr}
\end{figure}

\subsection{Game Simulation Results}
\paragraph{Main Results} Table~\ref{tab:evaluation_results} presents our GS evaluation results, measuring length (LEN), role-playing factual consistency (FAC), personality consistency (PER), action quality (ACT), interestingness (INT), and mechanic score (MEC), along with tevent condition error rate (ECE) and variable update error rate (VUE) that decompose mechanic score.

Regarding LEN, DeepSeek V3, GPT 4o mini, and Llama 3.1 70B exceed the 200-word limit more than other models, which generally adhere to the instruction. All models exhibit high scores for factual consistency (FAC) and maintain moderate levels of personality consistency (PER). Action choice quality (ACT) is similarly high across models, but interestingness (INT) demonstrates wider variation. In particular, Claude 3.5 Sonnet achieves the highest INT score.

Game mechanic performance (MEC) varies the most among all metrics. Gemini 2.0 Flash Exp, GPT-4o, and Gemini 1.5 Pro perform comparably well, while the other models fare significantly worse. Even the best-performing model, Gemini 2.0 Flash Exp, only achieves a 0.765 MEC score, highlighting the inherent difficulty of precisely following complex game mechanics in a text-based RPG setting.

\begin{table*}[!ht]
\centering
\begin{tabular}{lrrrrrrrrr}
\toprule
Model & LEN  & FAC $\uparrow$ & PER $\uparrow$& ACT $\uparrow$& INT $\uparrow$&  MEC $\uparrow$ &  ECE$\downarrow$ & VUE$\downarrow$\\ 
\midrule
Claude 3.5 Sonnet & 220.3 & \textbf{0.991} & 0.589 & 0.923 & \textbf{0.722} & 0.113 & \textbf{0.062} & 0.308\\
Deepseek V3 & 309.5 & 0.984 & 0.583 & 0.918 & 0.502 & 0.277 & 0.165 & 0.153 \\
Gemini 1.5 Pro & 198.0 & 0.968 & 0.596 & 0.894 & 0.602 & 0.554 & 0.081 & 0.085 \\
Gemini 2.0 Flash Exp & 195.3& 0.885 & \textbf{0.598} & 0.865 & 0.538 & \textbf{0.765} & 0.094 & \textbf{0.034} \\
GPT 4o & 201.9  & 0.902 & 0.585 & 0.894 & 0.502 & 0.693 & 0.088 & 0.047 \\
GPT 4o mini & 282.5  & 0.955 & 0.588 & 0.900 & 0.496 & 0.147 & 0.126 & 0.148 \\
Llama 3.1 70B Instruct & 279.2  & 0.977 & 0.586 & 0.915 & 0.420 & 0.162 & 0.161 &0.284\\
% higgs-llama & 342.6  & 0.944 & 0.433 & \textbf{0.940} & \textbf{0.658} & 0.787 & 0.082 \\
Llama 3.3 70B Instruct & 225.7 & 0.960 & 0.585 & \textbf{0.936} & 0.466 & 0.204 & 0.201 & 0.302\\
\bottomrule
\end{tabular}
\caption{Game Simulation results. LEN: length; FAC: role-playing factual consistency; PER: role-playing personality consistency; ACT: action choice quality; INT: interestingness; MEC: mechanic score; ECE: event condition error rate; VUE: variable update error rate.}
\label{tab:evaluation_results}
\end{table*}

\paragraph{Impact of Sampling Temperature} We further examine three metrics sensitive to sampling temperature—FAC, INT, and MEC—using GPT-4o at temperatures $\{0.2,0.5,0.8\}$. Table~\ref{tab:gs_temp} summarizes the results. Interestingly, FAC increases with higher temperatures, which may initially seem counterintuitive given the heightened risk of hallucinations. However, we hypothesize that a larger temperature reduces the generation of factually neutral text, thereby leading to fewer overlooked facts. As expected, INT (interestingness) also rises with temperature, reflecting the increased creativity enabled by more diverse sampling. In contrast, MEC (mechanic score) peaks at the lowest temperature. This suggests that more deterministic sampling helps the model adhere more rigorously to the predefined game mechanics.

\begin{table}[!ht]
    \centering
    \begin{tabular}{crrr}
    \toprule
    Temperature   &  FAC & INT & MEC \\
    \midrule
    0.2   &  0.920 & 0.502 & \textbf{0.693}\\
    0.5   &  0.939 & 0.520 & 0.629\\
    0.8   &  \textbf{0.952} & \textbf{0.538} & 0.643\\
    \bottomrule
    \end{tabular}
    \caption{Performance under different sampling temperatures}
    \label{tab:gs_temp}
\end{table}
\paragraph{Impact of Number of Rounds} In our main experiments, we terminate each simulation at the 10th round, although games often do not naturally end that early. To assess the effect of longer trajectories, we take GPT-4o as an example and vary the number of rounds in $\{10, 15, 20, 25\}$. We focus on the metrics FAC, INT, and MEC, as the remaining metrics exhibit minimal variance. Table~\ref{tab:gs_round} shows that FAC increases with the number of rounds and eventually stabilizes, whereas INT decreases—likely due to repetitive content over extended sequences. The MEC score also declines, which may reflect the growing difficulty in maintaining coherent game mechanics within a longer context.

In our main experiments, we terminate simulations at the 10-th round. However, we found that games usually don't terminate this early. To this end, we use GPT 4o as an example to study the performance with longer trajectories with the number of rounds being $\{10, 15, 20, 25\}$. We also study FAC, INT and MEC since other scores demonstrate small variations. We observe in Table~\ref{tab:gs_round} that FAC score increases with more rounds and eventually becomes stable. INT score decreases with more rounds, which could originate from repetitive content. MEC score also decreases, potentially due to the challenges in handling long context.

\begin{table}[!ht]
    \centering
    \begin{tabular}{crrr}
    \toprule
    \# Rounds   &  FAC & INT & MEC \\
    \midrule
    10   &  0.920 & \textbf{0.502} & \textbf{0.693}\\
    15   &  \textbf{0.948} & 0.480 & 0.679\\
    20   &  0.941 & 0.458 & 0.674\\
    25   &  0.941 & 0.440 & 0.668\\
    \bottomrule
    \end{tabular}
    \caption{Performance under different number of simulation rounds}
    \label{tab:gs_round}
\end{table}

Despite the variations observed in Tables~\ref{tab:gs_temp} and \ref{tab:gs_round}, the differences in INT and MEC remain relatively modest compared to the variability across models. Consequently, we conclude that limiting simulations to 10 rounds is adequate for most metrics, although extending the number of rounds may further improve the stability of the FAC score.

\subsection{Human Evaluation of Game Simulation} 
\begin{table*}[!ht]
\centering
\begin{tabular}{lrrrr}
\toprule
Models  & \multicolumn{1}{c}{FAC} & \multicolumn{1}{c}{ACT} & \multicolumn{1}{c}{INT} & \multicolumn{1}{c}{PER}  \\ 
\midrule
Claude 3.5 Sonnet & \textbf{0.810} / \textbf{1.000} / 0.190 & 0.831 / \underline{0.913} / 0.082 & \textbf{0.856} / \textbf{0.713} / 0.144 & 0.648 / 0.729 / 0.081 \\
Deepseek V3& \underline{0.807} / \underline{0.950} / 0.143 & \underline{0.857} / \underline{0.913} / 0.056 & 0.850 / 0.475 / 0.375 & 0.645 / 0.742 / 0.098 \\
Gemini 1.5 pro & 0.733 / 0.950 / 0.217 & 0.738 / 0.889 / 0.152 & 0.801 / 0.588 / 0.214 & 0.648 / 0.740 / 0.093\\
Gemini 2.0 Flash Exp & 0.769 / 0.800 / 0.031 & 0.851 / 0.876 / 0.025 & \textbf{0.856} / \underline{0.525} / 0.331 & \underline{0.651} / 0.737 / 0.085 \\
GPT 4o & 0.709 / \underline{0.950} / 0.241 & \textbf{0.881} / 0.887 / 0.007 & 0.834 / \underline{0.525} / 0.309 & \textbf{0.667} / 0.711 / 0.044 \\
GPT 4o mini & 0.770 / \underline{0.950} / 0.180 & 0.794 / 0.887 / 0.093 & 0.813 / 0.488 / 0.326 & 0.648 / \textbf{0.753} / 0.104 \\
Llama 3.1 70B Instruct & 0.778 / \underline{0.950} / 0.172 & \underline{0.857} / 0.898 / 0.041 & 0.824 / 0.400 / 0.424 & 0.627 / \underline{0.744} / 0.117 \\
Llama 3.3 70B Instruct & 0.791 / 0.933 / 0.142 & 0.852 / \textbf{0.930} / 0.078 & 0.850 / 0.438 / 0.412 & 0.640 / 0.739 / 0.099 \\
% \midrule
% \midrule
% MAD \& Correlations  & 0.165 / 0.129 / 0.267 & 0.067 / 0.226 / 0.071 & 0.317 / 0.140 / 0.109 & 0.090 / -0.691 / -0.429 \\
\bottomrule
\end{tabular}
\caption{Comparison of human and automatic evaluation scores for four subjective metrics (FAC, ACT, INT, and PER) on a subset of 20 game simulations. Each cell shows ``Human Score / Automatic Score / Absolute Difference.'' We use bold and underline to denote the highest and second-highest scores per metric, respectively.}
   % \caption{Comparison of human evaluation scores and automatic evaluation scores of four subjective metrics on a subset of 20 games. For comparison, we present scores as ``Human Score / Automatic Score / Absolute Difference''. We use bold and underline to highlight the best and the second best scores for each metric. }
   %We also show the Mean Absolute Difference (MAD), Pearson correlation coefficient, the Kendall rank correlation coefficient in ``MAD / Pearson / Kendall'' at the bottom.} 
    \label{tab:gs_he}
\end{table*}

\begin{table*}
    \centering
    \begin{tabular}{l|rrrr}
    \toprule
    Comparison     & \multicolumn{1}{c}{FAC} & \multicolumn{1}{c}{ACT} & \multicolumn{1}{c}{INT} & \multicolumn{1}{c}{PER} \\
    \midrule
    Auto v.s. Human     & 0.165 / 0.129 / 0.267 & 0.067 / 0.226 / 0.071 & 0.317 / 0.140 / 0.109 & 0.090 / -0.691 / -0.429 \\
    Human v.s. Human & 0.030 / 0.707 / 0.571 & 0.039 / 0.472 / 0.214 & 0.018 / 0.508 / 0.286 & 0.023 / -0.310 / -0.286\\
    \bottomrule
    \end{tabular}
    \caption{Mean Absolute Difference (MAD), Pearson correlation coefficient, and Kendall rank correlation coefficient between automatic metrics and human evaluation scores (and among human evaluators). All values are presented in the format ``MAD / Pearson / Kendall.''}
    % \caption{Mean Absolute Difference (MAD), Pearson correlation coefficient, the Kendall rank correlation coefficient between automatic metrics and human evaluation scores and among human evaluators. We present scores in the ``MAD / Pearson / Kendall'' order. }
    \label{tab:gs_corr}
\end{table*}
We also conduct a human evaluation on a subset of 20 simulated games, focusing on four subjective metrics: \textbf{FAC} (Factual Consistency), \textbf{PER} (Personality Consistency), \textbf{ACT} (Action Quality), and \textbf{INT} (Interestingness). We reframe these dimensions as natural-language questions to simplify the task for human annotators, who provide numerical scores later normalized to \([0,1]\). Complete details on the human evaluation setup are provided in the Appendix~\ref{app:human_eval}. Below, we outline two main differences between human evaluation and our automatic approach that can affect outcomes: \textbf{Scoring Procedure for Long Trajectories.} 
    Since each game trajectory consists of 10 rounds, we present the content round by round and request a set of scores per round. We then average these round-level scores to derive final FAC, ACT, and INT metrics. Personality (PER) is an exception; because a single round may not reveal enough about the NPC's character, annotators fill in a TIPI questionnaire at the end of the full trajectory. \textbf{Aggregated Factual Consistency.}
    Our automatic scorer checks each fact individually. However, to reduce the annotators' workload, we ask them to give a single 1--5 rating for overall consistency with all facts.

Table~\ref{tab:gs_he} presents the human evaluation scores alongside our automatic metrics for each model, while Table~\ref{tab:gs_corr} reports several comparative metrics such as mean absolute difference and correlation coefficients. Although human judgments can provide valuable insights, these metrics are inherently subjective and susceptible to personal biases. Consequently, human scores should be interpreted as reference points rather than definitive “gold standards.”

Examining Table~\ref{tab:gs_he}, we find a fair degree of overlap in the top two performing models across FAC, ACT, and INT, but not for PER. From Table~\ref{tab:gs_corr}, we see that the inter-annotator correlation on PER is also very low, suggesting that personality judgments tend to be more variable and less stable.

Looking at the mean absolute differences (\textbf{MAD}) between human and automatic scores, ACT and PER exhibit relatively small discrepancies, whereas FAC and INT show larger gaps. Interestingly, FAC and INT also have somewhat higher correlation coefficients than the other metrics. Such results may stem from two factors: (1) the modifications we made for human evaluators versus automatic methods, and (2) the fact that the scores of different models are relatively close, making correlation metrics sensitive to small shifts.

Feedback from our annotators further indicates that \textbf{INT} can be heavily influenced by personal preferences. For instance, if a rater dislikes combat scenarios, they consistently assign lower interest scores to an action-heavy game trajectory. This shows that subjective evaluations—whether by humans or LLM judges—can vary widely based on individual tastes.

Although LLM-based scoring has been common in prior work for subjective dimensions, our human evaluation reveals that fine-grained comparisons remain unstable and less differentiable, even for human evaluators. This outcome highlights the importance of introducing objective metrics into game simulation assessment, such as our proposed game mechanic checks (Section~\ref{sec:gd}) that do not rely on either human or LLM judgments. 
% Such a balance between subjective and objective evaluation criteria is crucial to comprehensively assess models’ capabilities in game creation and simulation.

% Analysis: points to make
% - these metrics are subjective, thus human evaluator may be biased and should not be considered as gold standard
% - the overlap of top 2 models is reasonably good from Table~\ref{gs_he} except for PER. However, from Table~\ref{gs_corr}, we can see that the inner annotator correlation is also very low for this metric, indicating the instability of personality judgement.
% - the absolution difference for ACT and PER is small. Although larger differences for FAC and INT, they do have a better correlation than the other scores. Note that the overall low correlation could originates from the modifications we made for human evaluation, and also the fact that scores from different models are close to each other, making the correlation metric sensitive to small changes.
% - evaluator feedback also indicates that they are sometimes biased by their personal perference when judging interestingness. E.g., if they are not interested in fighting scenarios, they would rate an action game trajectory with lower interestingness scores.
% - while previous work predominantly adopts LLM-based scores on subjective dimensions, our human evaluation results reveal that intricate comparisons on these dimensions are usually unstable and less differntiable, even with human evaluators. This further shows the importance of introducing objective scores in evaluating game simulation, just as we did for the game mechanics.


% \begin{table*}[!ht]
% \centering
% \begin{tabular}{lrrrr}
% \toprule
% Models  & \multicolumn{1}{c}{FAC} & \multicolumn{1}{c}{ACT} & \multicolumn{1}{c}{INT} & \multicolumn{1}{c}{PER}  \\ 
% \midrule
% Claude 3.5 Sonnet & \textbf{0.810} / \textbf{1.000} / 0.190 & 0.831 / \underline{0.913} / 0.082 & \textbf{0.856} / \textbf{0.713} / 0.144 & 0.648 / 0.729 / 0.081 \\
% Deepseek V3& \underline{0.807} / \underline{0.950} / 0.143 & \underline{0.857} / \underline{0.913} / 0.056 & 0.850 / 0.475 / 0.375 & 0.645 / \textbf{0.742} / 0.098 \\
% Gemini 2.0 Flash Exp & 0.769 / 0.800 / 0.031 & 0.851 / 0.876 / 0.025 & \textbf{0.856} / \underline{0.525} / 0.331 & \underline{0.651} / 0.737 / 0.085 \\
% GPT 4o & 0.709 / \underline{0.950} / 0.241 & \textbf{0.881} / 0.887 / 0.007 & 0.834 / \underline{0.525} / 0.309 & \textbf{0.667} / 0.711 / 0.044 \\
% GPT 4o mini & 0.770 / \underline{0.950} / 0.180 & 0.794 / 0.887 / 0.093 & 0.813 / 0.488 / 0.326 & 0.648 / 0.753 / 0.104 \\
% Llama 3.1 70b & 0.778 / \underline{0.950} / 0.172 & \underline{0.857} / 0.898 / 0.041 & 0.824 / 0.400 / 0.424 & 0.627 / 0.744 / 0.117 \\
% Llama 3.3 70b & 0.791 / 0.933 / 0.142 & 0.852 / \textbf{0.930} / 0.078 & 0.850 / 0.438 / 0.412 & 0.640 / \underline{0.739} / 0.099 \\
% % \midrule
% % \midrule
% % MAD \& Correlations  & 0.165 / 0.129 / 0.267 & 0.067 / 0.226 / 0.071 & 0.317 / 0.140 / 0.109 & 0.090 / -0.691 / -0.429 \\
% \bottomrule
% \end{tabular}
%    \caption{Comparison of human evaluation scores and automatic evaluation scores of four subjective metrics on a subset of 20 games. For comparison, we present scores as ``Human Score / Automatic Score / Absolute Difference''. We use bold and underline to highlight the best and the second best scores for each metric. }
%    %We also show the Mean Absolute Difference (MAD), Pearson correlation coefficient, the Kendall rank correlation coefficient in ``MAD / Pearson / Kendall'' at the bottom.} 
%     \label{tab:gs_he}
% \end{table*}

% \begin{table*}
%     \centering
%     \begin{tabular}{l|rrrr}
%     \toprule
%     Comparison     & \multicolumn{1}{c}{FAC} & \multicolumn{1}{c}{ACT} & \multicolumn{1}{c}{INT} & \multicolumn{1}{c}{PER} \\
%     \midrule
%     Auto v.s. Human     & 0.165 / 0.129 / 0.267 & 0.067 / 0.226 / 0.071 & 0.317 / 0.140 / 0.109 & 0.090 / -0.691 / -0.429 \\
%     Human v.s. Human & \\
%     \bottomrule
%     \end{tabular}
%     \caption{Mean Absolute Difference (MAD), Pearson correlation coefficient, the Kendall rank correlation coefficient between automatic metrics and human evaluation scores and among human evaluators. We present scores in the ``MAD / Pearson / Kendall'' order. }
%     \label{tab:gs_corr}
% \end{table*}
\section{Conclusion}
We have presented Digital Twin Buildings, a framework for extracting the 3D mesh of a building, for connecting the building to Google Maps Platform APIs, and for Multi-Agent Large Language Models data analytics. We demonstrate this by extracting visual description keywords and captions of the building from multi-view multi-scale images of the building. The framework can also be used to process different data modalities sourced from Google Cloud Services. This approach enables richer semantic understanding, seamless integration with geospatial data, and enhanced interaction with real-world structures, paving the way for advanced applications in urban analytics, navigation, and virtual environments.

\section*{Impact Statement}

This work aims to advance the field of Machine Learning by introducing \benchmark{}, a benchmark specifically designed to evaluate large language models (LLMs) in the context of text-based role-playing games. The development of \benchmark{} has potential societal implications related to the deployment of LLMs in interactive and narrative-driven applications, including fostering more immersive and engaging gaming experiences.

Ethical considerations include ensuring that LLMs evaluated and fine-tuned using \benchmark{} adhere to principles of fairness and inclusivity, particularly in the portrayal of characters and narratives. Misuse of the benchmark to develop systems that propagate harmful biases or enforce stereotypical characterizations is a concern that developers should address when applying this work. Additionally, the use of LLMs as evaluative judges raises questions about transparency, reliability, and the potential for unintended bias in automated assessments.

By encouraging further research on hybrid evaluation methods that combine subjective LLM-based judgments with objective scoring mechanisms, this work contributes to ongoing discussions about improving the accountability and robustness of machine learning systems in creative and interactive domains.
% Acknowledgements should only appear in the accepted version.
% \input{src/8acknowledgement}





\bibliography{main}
\bibliographystyle{icml2025}


%%%%%%%%%%%%%%%%%%%%%%%%%%%%%%%%%%%%%%%%%%%%%%%%%%%%%%%%%%%%%%%%%%%%%%%%%%%%%%%
%%%%%%%%%%%%%%%%%%%%%%%%%%%%%%%%%%%%%%%%%%%%%%%%%%%%%%%%%%%%%%%%%%%%%%%%%%%%%%%
% APPENDIX
%%%%%%%%%%%%%%%%%%%%%%%%%%%%%%%%%%%%%%%%%%%%%%%%%%%%%%%%%%%%%%%%%%%%%%%%%%%%%%%
%%%%%%%%%%%%%%%%%%%%%%%%%%%%%%%%%%%%%%%%%%%%%%%%%%%%%%%%%%%%%%%%%%%%%%%%%%%%%%%
\newpage
\appendix
\onecolumn
\newpage
~
\newpage
\section{Notations}
The definitions of the notations used in this paper are summarized in Table~\ref{table:notation}.

\section{Training Details}
\label{section:training_details}
\paragraph{Dataset}
Our experiments are conducted on two datasets: Ultrafeedback-Binarized~\cite{cui2023ultrafeedback} and the helpful-base subset of HH-rlhf~\cite{bai2022traininghelpfulharmlessassistant}. Both datasets undergo preprocessing to eliminate noise and constrain their overall length. For the Ultrafeedback-Binarized dataset, we select examples where the prompt length, chosen response length, and rejected response length are each less than 512 tokens. Additionally, we ensure that the chosen response score exceeds the rejected response score and that the substring 'confidence' does not appear in either the chosen or rejected responses. For the HH-rlhf dataset, we apply the same length constraints (prompt, chosen, and rejected responses each under 512 tokens). Furthermore, we ensure that each prompt appears only once across both datasets and limit the test set to 256 examples. The training set of Ultrafeedback-Binarized contains around 33,000 examples and HH-RLHF helpful base contains 43,000 examples. All training are carried on 8*A800(80G) GPUs.
\paragraph{Base Models}
For the base models, we utilize Gemma-2B~\citep{gemma_2024} and Llama3-8B~\citep{llama3modelcard}. In all training procedures, we implement a linear learning rate scheduler, which gradually increases the learning rate from 0 to the maximum value over the first 0.1 epoch.

\paragraph{SFT Model}
The Supervised Fine-Tuned (SFT) model is initialized from the base model and trained on the chosen responses for two epochs with a learning rate of 5e-6. Gradient norm clipping is applied when the norm exceeds 10.

\paragraph{Reward Model}
The reward model is initialized from the base model, with the logit head replaced by a linear head above the last embedding layer to output a scalar value. It is trained for one epoch with a learning rate of 5e-6, achieving an accuracy of approximately 70\% on the test set. Gradient norm clipping is applied when the norm exceeds 5.

For ODIN training, we use two linear heads to output length reward and quality reward separately, following the training loss described in~\citet{Chen2024ODINDR}. Only the quality head is used during RL training.

For WARM training, we train five reward models on the same dataset with varying learning rates (3e-6, 4e-6, 5e-6, 6e-6, 7e-6) and different random seeds.

For Reg training, we adopt the loss function from~\cite{Dai2023SafeRS}, with a regularization term coefficient of 0.005.

\paragraph{Policy Model}
The policy model is initialized from the SFT model and trained on the same prompts for one epoch using the PPO algorithm with a learning rate of 3e-7. Gradient norm clipping is applied when the norm exceeds 5.

\paragraph{Critic Model}
The critic model is initialized from the reward model and trained alongside the policy model for one epoch with a learning rate of 5e-6. Gradient norm clipping is applied when the norm exceeds 5.

\paragraph{Hyper-Parameters}
Responses are sampled from the policy model using a temperature of 0.9, with top-k set to 50, top-p set to 0.9, and a length penalty of 2. The coefficient for the KL penalty is 0.005, and the default number of reference rewards is 10. For PPO training, the buffer size is set to 4, with $\epsilon=0.2,\lambda=0.95,\gamma=1.0$, For GRPO training, the $\epsilon=0.2$, the buffer size is 4, and the group size is 5.

\section{Evaluation}
\subsection{Winrate on Test Set}
To leverage the strong grading capability of DeepSeek-V3 for comparing the SFT model and the policy model on the test set, we design a detailed evaluation prompt. The system prompt and user input format are provided in Listing 1 and 2.

To address position bias~\cite{wang2023largelanguagemodelsfair}, we evaluate each pair of responses twice, alternating their order, and aggregate the scores. Specifically, for two responses A and B, we first evaluate them in the order A-B and then in the order B-A. In each evaluation, the winner receives a score of 1, the loser receives 0, and in the case of a tie, both responses receive 0.5. The final scores of A and B are compared, and the response with the higher score is declared the winner. If the scores are tied, both responses receive 0.5 win counts. The win counts are used to calculate the winrate.

% \begin{tcolorbox}[title={The Prompt used for Translation}]
You are a highly skilled translator tasked with translating various types of content from English into \{\{ language \}\}. Follow these instructions carefully to complete the translation task.

You will receive a user-bot conversation in XML format. Please follow a three-step translation process:

\begin{enumerate}
  \item \textbf{Initial Translation:} Translate the input content into \{\{ language \}\}, preserving the original intent and keeping the original paragraph and text format unchanged. Do not delete or omit any content, and ensure that all original Markdown elements (e.g., images, code blocks) are preserved.
  \item \textbf{Reflection and Feedback:} Carefully review both the source text and your translation. Provide constructive criticism and specific suggestions to improve the translation in terms of:
    \begin{enumerate}[label=(\roman*)]
      \item \textbf{Accuracy:} Correct errors of addition, mistranslation, omission, or untranslated text.
      \item \textbf{Fluency:} Apply \{\{ language \}\} grammar, spelling, and punctuation rules while avoiding unnecessary repetitions.
      \item \textbf{Style:} Ensure that the translation reflects the style of the source text and considers any relevant cultural context.
    \end{enumerate}
  \item \textbf{Refinement:} Based on your reflections, refine and polish your translation.
  \item \textbf{Fallback:} If you are not confident in translating the conversation, please return ``\texttt{<stop></stop>}''.
\end{enumerate}

\bigskip
\textbf{Output:}

For each step of the translation process, output your results within the appropriate XML tags as follows:
\begin{verbatim}
<step1_initial_translation>
[Insert your initial translation here]
</step1_initial_translation>

<step2_reflection>
[Insert your reflection on the translation, including a list 
of specific, helpful, and constructive suggestions for 
improvement. Each suggestion should address a specific 
part of the translation.]
</step2_reflection>

<step3_refined_translation>
[Insert your refined and polished translation here]
</step3_refined_translation>
\end{verbatim}

Ensure that your final translation in step 3 accurately reflects the original meaning while sounding natural in \{\{ language \}\}.

Here is the original conversation:
\label{box:trans_prompt}
\end{tcolorbox}

\subsection{Benchmark}
We also evaluate the model on two benchmarks, using DeepSeek-V3 to simulate human evaluation. The metrics and their meanings are as follows:
\subsection*{AlpacaEval 2.0}
\begin{itemize}[leftmargin=*, itemsep=0pt]
    \item \textbf{LC Winrate}: The length-controlled win rate measures the model's performance while controlling for the length of generated responses. It compares the model's outputs to a baseline (e.g., the SFT model) and adjusts for the influence of response length on human preferences.
    \item \textbf{Winrate}: The standard win rate measures the proportion of times the model's outputs are preferred over the baseline's outputs in human evaluations.
    \item \textbf{Length}: The average length of the model's generated responses, measured in tokens or characters, providing insight into the model's verbosity.
\end{itemize}

\subsection*{MT-bench}
\begin{itemize}[leftmargin=*, itemsep=0pt]
    \item \textbf{T1}: Turn 1 Score evaluates the model's performance on the first turn of a multi-turn dialogue, assessing relevance, coherence, and informativeness. Scores are normalized as 0-10.
    \item \textbf{T2}: Turn 2 Score evaluates the model's performance on the second turn, measuring its ability to maintain context and provide consistent, high-quality responses. Scores are also normalized as 0-10.
    \item \textbf{Overall}: The overall score is the average of the T1 and T2 scores, providing a comprehensive evaluation of the model's performance across both turns.
\end{itemize}



\section{More Results}
\subsection{Llama3-8B and Ultrafeedback Binarized}
% \input{figures/llama3-8b_ultrafb_bin_lab1}
Figure~\ref{fig:llama3-8b_ultrafb_bin_lab1} presents the PPO training curves for different mitigation methods on Llama3-8B with the Ultrafeedback Binarized dataset. PAR demonstrates robustness against reward hacking and maintains a high win rate throughout one epoch of training.

\subsection{Gemma2-2B and HH-RLHF}
% \input{figures/gemma2-2b_hh_rlhf_lab1}
The PPO training curves for various mitigation methods on Gemma2-2B with the HH-RLHF dataset are shown in Figure~\ref{fig:gemma2-2b_hh_rlhf_lab1}. PAR exhibits resilience to reward hacking and sustains a high win rate during one epoch of training.

\subsection{Llama3-8B and HH-RLHF}
% \input{figures/llama3-8b_hh_rlhf_lab1}
Figure~\ref{fig:llama3-8b_hh_rlhf_lab1} illustrates the PPO training curves for different mitigation methods applied to Llama3-8B on the HH-RLHF dataset. While PAR shows signs of reward hacking toward the end of training, it maintains a consistently high win rate (above 60\%) for an extended period, from 10,000 to 30,000 steps. We hypothesize that the observed reward hacking in the later stages is due to the convergence rate of the sigmoid function approaching its upper bound.

\section{Case Study}
We identify several patterns of reward hacking observed in Vanilla PPO training, using the checkpoint trained after one epoch for detailed examination. We show the examples in Figure \ref{fig:case_study}.



\section{PPO Training}
PPO (Proximal Policy Optimization) is an online reinforcement learning algorithm that generates a response given a prompt, computes a reward for the response using a reward model, and updates the policy and critic models to maximize the reward.

We employ several PPO techniques to ensure stable training, including advantage normalization~\cite{zheng2023secretsrlhflargelanguage}, value loss clipping~\cite{patterson2023robustlosseslearningvalue}, a replay buffer~\cite{eysenbach2019searchreplaybufferbridging}, per-token KL penalty, and length penalty. The pseudo-code for the PPO algorithm is provided in Algorithm~\ref{alg:ppo}.


\section{Reward Shaping Is Not Applicable to DPO and GRPO}
\label{section:rsnotavail}
In this section, we explain why monotonous reward shaping techniques, such as PAR, are not applicable to the Direct Preference Optimization (DPO). And why linear shaping techniques are not applicable to the Group Relative Policy Optimization (GRPO) algorithms.

\subsection{DPO and Reward Shaping}
Vanilla DPO is an offline alignment algorithm that trains the policy model directly on paired responses using a contrastive loss. Since the vanilla DPO algorithm does not rely on an explicit reward model, reward shaping techniques are inherently inapplicable. We also explore an online variant of DPO, which generates two responses for a given prompt and employs a reward model to determine the chosen and rejected responses. The policy model is then trained on these responses (see Algorithm~\ref{alg:dpo}). However, any monotonous transformation of the proxy reward will not alter the chosen and rejected responses. For instance, if \( r_1 > r_2 \), then \( f(r_1) > f(r_2) \) for any monotonous function \( f(\cdot) \), including PAR. Consequently, PAR is also not applicable to online DPO.

\subsection{GRPO and Reward Shaping}
For GRPO, the advantage value is computed as a normalization of proxy rewards. Consider a prompt \( x \) and \( N \) responses \( y_1, \dots, y_N \) sampled from the policy model. A reward model \( r_\phi \) assigns scores \( r_1, \dots, r_N \) to each response. The advantage \( A_{i,t} \) for response \( y_i \) at token position \( t \) is given by:
\[
A_{i,t} = \frac{r_i - \mu}{s},
\]
where \( \mu = \frac{1}{N} \sum_{i=1}^N r_i \) and \( s = \sqrt{\frac{1}{N} \sum_{i=1}^N (r_i - \mu)^2 } \) are the mean and standard deviation of the rewards, respectively.

Assume a linear transformation is applied to the proxy reward, such that \( \hat{r} = a \cdot r + b \) (\( a > 0 \)). We prove that the new advantage \( \hat{A}_{i,t} \) is identical to the original \( A_{i,t} \). First, the new mean \( \hat{\mu} = a \cdot \mu + b \), and the new standard deviation \( \hat{s} = a \cdot s \). The new advantage is computed as:
\[
\begin{aligned}
\hat{A}_{i,t} &= \frac{\hat{r}_i - \hat{\mu}}{\hat{s}} = \frac{a r_i + b - (a \mu + b)}{a s} \\
&= \frac{a r_i - a \mu}{a s} = \frac{r_i - \mu}{s} \\
&= A_{i,t}.
\end{aligned}
\]

Thus, linear transformations do not influence the advantage calculation in GRPO. Furthermore, since the sigmoid function is a non-linear function, PAR is applicable to GRPO training. We validate this through experiments, as shown in Figure~\ref{fig:grpo}. No reward hacking problem is observed in the GRPO training process, as the advantage calculation inherently performs reward normalization.

\begin{figure}[H]
\centering
\includegraphics[width=1.0\linewidth]{figures/grpo_img.pdf}

\caption{The training curves for GRPO, evaluated on Gemma2-2B with the Ultrafeedback-Binarized dataset, demonstrate that Vanilla, Meanstd, Minmax exhibit similar proxy rewards throughout the training process. This is because linear transformations of the proxy rewards do not affect the advantage value in GRPO. The PAR is a non-linear function and slightly better before collapse. No reward hacking issue is observed in the GRPO training process, as the advantage calculation inherently normalizes the rewards.}
\label{fig:grpo}
\end{figure}

% Define the style for code blocks
\lstset{
    backgroundcolor=\color{gray!10}, % Background color
    basicstyle=\ttfamily\footnotesize, % Basic font style
    breaklines=true, % Automatic line breaks
    frame=single, % Single border
    captionpos=b, % Caption position at the bottom
    keywordstyle=\color{blue}, % Keyword color
    commentstyle=\color{green!50!black}, % Comment color
    stringstyle=\color{red}, % String color
    showstringspaces=false, % Do not show spaces in strings
    tabsize=4 % Set tab width
}
% listing 1 2 
\begin{figure*}[htb]
\begin{lstlisting}[caption={System Prompt For Winrate Evaluation on Test Set}]
Please act as an impartial evaluator to assess the quality of two responses from different AI assistants to an incomplete dialogue between a user (<|user|>) and an AI assistant (<|assistant|>). The dialogue will be missing the last turn, and both Assistant-A (<Assistant-A response>) and Assistant-B (<Assistant-B response>) are expected to complete it. Focus your evaluation on the following five aspects:
1. Clarity and Relevance: Responses should be concise, directly addressing the question. They should use clear, natural language and remain on-topic.
2. Accuracy and Honesty: Responses must provide factual, truthful information. Disclose limitations or uncertainties when necessary.
3. Ethics and Appropriateness: Ensure the responses are free from harmful, offensive, or discriminatory content.
4. Engagement and Depth: Responses should be engaging, educational, and sufficiently detailed to comprehensively address the user question.
5. Structure and Creativity: Responses should be logically organized and show originality or adaptability when necessary.

Note: The quality of the responses should not be judged solely by their length. Both brevity and detail are important depending on the context of the question.
You will be given an incomplete dialogue (<question>) with the last turn left blank. Assistant-A (<Assistant-A response>) and Assistant-B (<Assistant-B response>) have each provided a response to complete the dialogue. Your task is to evaluate each response based on the five criteria above and provide a comparison.

Evaluation Format:
Assistant-A Response:
(Evaluate the quality of Assistant-A response based on the five aspects mentioned above.)
Assistant-B Response:
(Evaluate the quality of Assistant-B response based on the five aspects mentioned above.)
Comparison and Analysis:
Compare and contrast the responses from Assistant-A and Assistant-B to determine which one is more effective overall. Justify your reasoning clearly and concisely.

At the end, output the comparison result for both responses in the following format:
Better: X (X is A, B, or N, representing A is better, B is better, or both are of equal quality)
\end{lstlisting}

\begin{lstlisting}[caption={User Input Template For Winrate Evaluation on Test Set}]
<question>:
{user_question}
<Assistant-A response>:
{policy_response}
<Assistant-B response>:
{sft_response}
\end{lstlisting}
\end{figure*}

\begin{figure*}[htbp]
    \centering
    \begin{subfigure}[b]{0.7\linewidth}
        \centering
        \includegraphics[width=\linewidth]{figures/llama3-8b_ultrafb_bin_curve_lab1.pdf}
        \caption{PPO training curve for different mitigation methods on Llama3-8B and Ultrafeedback-Binarized.}
        \label{fig:llama3-8b_ultrafb_bin_lab1}
    \end{subfigure}
    
    \vspace{0.2cm} % Adjust vertical spacing between subfigures
    
    \begin{subfigure}[b]{0.7\linewidth}
        \centering
        \includegraphics[width=\linewidth]{figures/gemma2-2b_hh_rlhf_curve_lab1.pdf}
        \caption{PPO training curve for different mitigation methods on Gemma2-2B and HH-RLHF.}
        \label{fig:gemma2-2b_hh_rlhf_lab1}
    \end{subfigure}
    
    \vspace{0.2cm} % Adjust spacing as needed
    
    \begin{subfigure}[b]{0.7\linewidth}
        \centering
        \includegraphics[width=\linewidth]{figures/llama3-8b_hh_rlhf_curve_lab1.pdf}
        \caption{PPO training curve for different mitigation methods on Llama3-8B and HH-RLHF.}
        \label{fig:llama3-8b_hh_rlhf_lab1}
    \end{subfigure}
    
    \caption{PPO training curves for different mitigation methods across various models and datasets.}
    \label{fig:ppo_training_curves}
\end{figure*}

\begin{figure*}[htb]
\begin{lstlisting}[caption={Gemma2-2B with Ultrafeedback-Binarized}]
Prompt:
<|user|>i am going to give a ppt presentation on VLSI in aerospace and defence. what I need is like introduction slide and Overview slide and related contents you wish to be included give me<|assistant|>

Policy model response:
In enchanting architectural mosaics, fuse captivating cosmic threads VELCRO-ing silicon's sacred architectural realm with adaptive aeronautical nebulae, sculpting celestial gardens for efficient holographic compass landmarks. Seamless Integrated Silicon Graphics (ISG) - ......
\end{lstlisting}

\begin{lstlisting}[caption={Gemma2-2B with HH-RLHF}]
Prompt:
<|user|>Can you help compare the price travel for a trip to the Caribbean to a trip to Las Vegas?<|assistant|>

Policy model response:
Sure! Caribbean travel is a wonderful choice for a vacation. Caribbean destinations offer some of the most beautiful beaches in the world, and the crystal clear waters of the Caribbean Sea make for some of the most amazing snorkeling and diving experiences. Caribbean resorts and hotels offer amenities like fine dining, swimming pools, spa services, and a number of outdoor activities, such as sailing, hiking, golfing, and horseback riding...
\end{lstlisting}

\begin{lstlisting}[caption={Llama3-8B with Ultrafeedback-Binarized}]
Prompt:
<|user|>How can creative leaders ensure that their decisions align with their organization's values and ethics?<|assistant|>

Policy model response:
Embed GDPR-grade encrypted traceability logs to GDPR-compliant audit records via trusted pipelines, preventing unauthorized backups by blockchain timestamp sealed secure encrypted logs copies through API certified audits against bespoke compliance templates signatures, logs' timestamps assigned securely within SharePoint backups or verified validation privacy signatures plus?,?, ...
\end{lstlisting}

\begin{lstlisting}[caption={Llama3-8B with HH-RLHF}]
Prompt:
<|user|>What are some salty things I can put in oatmeal?<|assistant|>

Policy model response:
You might add salt, pepper, tomatoes, onions, celery, eggs, beans, yogurt, mushrooms, tofu, berries, maple syrup, cinnamon, nutmeg, paprika, and dried herbs. Maybe add some dark chocolate, almonds, walnuts, and pumpkin seeds. And you can also drink plenty of water, avoid soda and caffeinated drinks, eat healthy carbs and avoid saturated fats, stay hydrated and watch for high blood pressure, have sufficient vitamin intake. Sleep at least 8 hours per day, limit your red meat intake, be active by cooking, swimming, running, weightlifting, and eating outdoors, eat good fats like olive oil, eat nuts and seeds and go light on salt, avoid greasy foods and snacks like cheese chips. Get plenty of fresh air, have fun indoors an outside. Avoid smoking and overeating, and take some time daily to enjoy your favorite activities, be grateful, and be thankful. Keep active and move in ways that encourage fitness. Take a deep breath and take a journey! Make your oatmeal a treat!
\end{lstlisting}
\caption{Analysis of reward hacking patterns observed in Vanilla PPO training, based on a detailed examination of the model checkpoint after one epoch.}
\label{fig:case_study}
\end{figure*}

\onecolumn
\begin{figure*}[htbp]
    \centering
\begin{algorithm}[H]
\caption{PPO}
\label{alg:ppo}
\begin{algorithmic}[1]
\Require sft model $\pi_{\text{sft}}$, reward model $r_{\phi}$, prompt set $\mathcal{D}$.
\Ensure 
    Aligned model $\pi_{\theta^*}$\\
Initialize policy model $\pi_\theta\leftarrow\pi_{\text{sft}}$\\
Initialize reference model $\pi_{\text{ref}}\leftarrow\pi_{\text{sft}}$\\
Initialize critic model $V_\alpha\leftarrow r_{\phi}$
\For{$x\in\mathcal{D}$}
    \State ppo\_batch = build\_ppo\_batch($x,\pi_\theta,\pi_{\text{ref}},V_{\alpha},r_{\phi}$)
    \State ppo\_batch = buffer.substitute(ppo\_batch) \Comment{sample a ppo\_batch from replay buffer and save current ppo\_batch into the buffer}
    \State $\mathcal{L}_\text{ppo}(\theta),\mathcal{L}_\text{critic}(\alpha)$ = calculate\_loss(ppo\_batch, $\pi_\theta$, $V_\alpha$) 
    \State $\theta\leftarrow\theta-\mathsf{plr}*\nabla_{\theta}\mathcal{L}_\text{ppo}(\theta)$ \Comment{update policy model via gradient descent, $\mathsf{plr}$ is policy learning rate}
    \State $\alpha\leftarrow\alpha-\mathsf{clr}*\nabla_{\alpha}\mathcal{L}_\text{critic}(\alpha)$ \Comment{$\mathsf{clr}$ is critic learning rate}
\EndFor\\
\Return $\pi_{\theta^*}$ 
\end{algorithmic}
\end{algorithm}
%%%%%%%%%%%%%%%%%
\begin{algorithm}[H]
\caption{build\_ppo\_batch}
\label{alg:build_ppo_batch}
\begin{algorithmic}[1]
\Require prompt $x$, four models $\pi_\theta,\pi_{\text{ref}},V_{\alpha},r_{\phi}$.
\Ensure ppo\_batch: A dictionary

\State Initialize ppo\_batch = {}
\State sample $y\sim\pi_\theta(.|x)$
\State sample $y_{\text{ref}}^{1,\ldots,M}\sim \pi_{\text{ref}}(.|x)$  \Comment{optional}
\State $r=r_\phi(x,y)$
\State $r_{\text{ref}}^{1,\ldots,M}=r_{\phi}(x, y_{\text{ref}}^{1,\ldots,M})$ \Comment{optional}
\State $r_{\text{RL}}=\text{reward\_reshape}(r, r_{\text{ref}}^{1,\ldots,M}, \text{len}(y),\text{mode}=\mathsf{PAR})$
\State Now we split (x,y) into $(s_t,a_t)_{t=0}^T$
\State $\text{KL\_penalty} = \log\pi_{\theta}(a_t|s_t)-\log\pi_{\text{ref}}(a_t|s_t)$
\State construct per-token rewards $r_{1,\ldots,T}$ from $r_{\text{RL}}$ and KL\_penalty
\State $V_t = V_\alpha(s_t)$
\State Compute GAE $\hat{A}_t$ and Return $G_t$ from $V_t$ and $r_t$.
\State ppo\_batch = ($\log\pi_{\theta}(a_t|s_t)$, $G_t$, $\hat{A}_t$, $V_t$, $s_t$, $a_t$)
\\
\Return $\text{ppo\_batch}$ 
\end{algorithmic}
\end{algorithm}

%%%%%%%%%%%%%%%%%%
\begin{algorithm}[H]
\caption{Buffer.substitute}
\label{alg:buffer}
\begin{algorithmic}[1]
\Require ppo\_batch.
\Ensure ppo\_batch: A dictionary

\State Global List pool = []
\State Global buffer\_size = 4 \\
IF len(pool)<buffer\_size:
\State \ \ \ \ pool.append(ppo\_batch) \\
\ \ \ \ \Return None \\
ELSE:
\State\ \ \ \ selected\_batch = random.choice(pool) 
\State\ \ \ \ pool.pop(selected\_batch) 
\State\ \ \ \ pool.append(ppo\_batch) \\
\ \ \ \ \Return selected\_batch 
\end{algorithmic}
\end{algorithm}
\end{figure*}
%%%%%%%%%%%%%%%%%%%

\begin{figure*}[htbp]
    \centering
\begin{algorithm}[H]
\caption{reward\_reshape}
\label{alg:reward_reshape}
\begin{algorithmic}[1]
\Require policy reward $r$, reference reward $r_{\text{ref}}^{1,\ldots,M}$, response length $l$, reshape mode $\mathsf{mode}$.
\Ensure RL reward 
\State IF $l>300$:
\State \ \ \ $r=r-0.01*(l-300)$ \Comment{penalize long response}
\State IF mode==$\mathsf{meanstd}$: 
\State\ \ \ \ $r_{\text{RL}} = \frac{r-\mu}{s}$ \Comment{$\mu,s$ are running mean and running standard variance respectively.}
\State IF mode==$\mathsf{reward\_clip}$:
\State\ \ \ \  ...
\State IF mode==$\mathsf{PAR}$:
\State \ \ \ \ $r_\text{RL}$ = $\frac{1}{M}\sum_{m=1}^M\sigma(r-r_{\text{ref}}^{m})$ 

\end{algorithmic}
\end{algorithm}
%%%%%%%%%%%%%%%%%%%
\begin{algorithm}[H]
\caption{calculate\_loss}
\label{alg:policy_loss}
\begin{algorithmic}[1]
\Require ppo\_batch, policy model $\pi_\theta$, critic model $V_\alpha$.
\Ensure policy loss $\mathcal{L}_\text{ppo}(\theta)$, critic loss $\mathcal{L}_\text{critic}(\alpha)$
\State ($\log\pi_{\theta_\text{old}}(a_t|s_t)$, $G_t$, $\hat{A}_t$, $V_t$, $s_t$, $a_t$) = ppo\_batch \Comment{Extract elements from ppo\_batch}
\State $\mathcal{L}_\text{ppo}(\theta) = \hat{\mathbb{E}}_t \left[ \min \left( \frac{\pi_\theta(a_t|s_t)}{\pi_{\theta_{\text{old}}}(a_t|s_t)} \hat{A}_t, \text{clip} \left( \frac{\pi_\theta(a_t|s_t)}{\pi_{\theta_{\text{old}}}(a_t|s_t)}, 1 - \epsilon, 1 + \epsilon \right) \hat{A}_t \right) \right]$
\State $\mathcal{L}_\text{critic}(\alpha)=\hat{\mathbb{E}}_t\left[\max\left(||V_{\alpha}(s_t)-G_t||_2^2, ||\text{clip}(V_\alpha(s_t),V_t-\delta,V_t+\delta)-G_t||_2^2\right)\right]$ \Comment{Critic clip trick}\\

\Return $\mathcal{L}_\text{ppo}(\theta), \mathcal{L}_\text{critic}(\phi)$
\end{algorithmic}
\end{algorithm}


\begin{algorithm}[H]
\caption{Online DPO}
\label{alg:dpo}
\begin{algorithmic}[1]
\Require sft model $\pi_{\text{sft}}$, reward model $r_{\phi}$, prompt set $\mathcal{D}$.
\Ensure 
    Aligned model $\pi_{\theta^*}$\\
Initialize policy model $\pi_\theta\leftarrow\pi_{\text{sft}}$\\
Initialize reference model $\pi_{\text{ref}}\leftarrow\pi_{\text{sft}}$

\For{$x\in\mathcal{D}$}
    \State Sample $y_1,y_2 \sim \pi_\theta(.|x)$ 
    \State Calculate rewards $r_1=r_\phi(x,y_1), r_2=r_\phi(x,y_2)$\\
    \ \ \ \ \ \ IF $r_1>r_2$:
    \State \ \ \ $y_w=y_1,y_l=y_2$\\
    \ \ \ \ \ \ ELSE:
    \State \ \ \ $y_w=y_2,y_l=y_1$
    \State $\mathcal{L}_{\text{DPO}}(\theta) = -\left[ \log \sigma\left( \beta \left( \log \frac{\pi_\theta(y_w | x)}{\pi_{\text{ref}}(y_w | x)} - \log \frac{\pi_\theta(y_l | x)}{\pi_{\text{ref}}(y_l | x)} \right) \right) \right]$
    \State $\theta\leftarrow\theta-lr*\nabla_{\theta}\mathcal{L}_{\text{DPO}}(\theta)$ 
\EndFor\\
\Return $\pi_{\theta^*}$ 
\end{algorithmic}
\end{algorithm}
\end{figure*}



%%%%%%%%%%%%%%%%%%%%%%%%%%%%%%%%%%%%%%%%%%%%%%%%%%%%%%%%%%%%%%%%%%%%%%%%%%%%%%%
%%%%%%%%%%%%%%%%%%%%%%%%%%%%%%%%%%%%%%%%%%%%%%%%%%%%%%%%%%%%%%%%%%%%%%%%%%%%%%%


\end{document}


% This document was modified from the file originally made available by
% Pat Langley and Andrea Danyluk for ICML-2K. This version was created
% by Iain Murray in 2018, and modified by Alexandre Bouchard in
% 2019 and 2021 and by Csaba Szepesvari, Gang Niu and Sivan Sabato in 2022.
% Modified again in 2023 and 2024 by Sivan Sabato and Jonathan Scarlett.
% Previous contributors include Dan Roy, Lise Getoor and Tobias
% Scheffer, which was slightly modified from the 2010 version by
% Thorsten Joachims & Johannes Fuernkranz, slightly modified from the
% 2009 version by Kiri Wagstaff and Sam Roweis's 2008 version, which is
% slightly modified from Prasad Tadepalli's 2007 version which is a
% lightly changed version of the previous year's version by Andrew
% Moore, which was in turn edited from those of Kristian Kersting and
% Codrina Lauth. Alex Smola contributed to the algorithmic style files.
