
% \label{sec:evaluation}
% \begin{table*}[!ht]
% \centering
% \begin{tabular}{lrrrrrrr}
% \toprule
% Model & Length  & Fact  & Personality & Action & Interestingness &  Mechanics  &  Mechanics  round \\ 
% \midrule
% gpt-4o & 201.9  & 0.892 & 0.468 & 0.894 & 0.502 & 0.921 & 0.648 \\
% deepseek-chat & 309.5 & 0.964 & 0.465 & 0.918 & 0.502 & 0.827 & 0.258 \\
% gemini-1.5-pro & 198.0 & 0.968 & 0.478 & 0.894 & 0.602 & 0.907 & 0.534 \\
% gemini-2.0-flash-exp & 195.3& 0.885 & \textbf{0.479} & 0.865 & 0.538 & \textbf{0.936} & \textbf{0.745} \\
% gpt-4o-mini & 282.5  & 0.955 & 0.468 & 0.900 & 0.496 & 0.613 & 0.141 \\
% claude-3-5-sonnet & 563.0 & \textbf{0.991} & 0.471 & 0.923 & 0.722 & 0.838 & 0.099 \\
% llama-3-1-70b & 279.2  & 0.977 & 0.468 & 0.915 & 0.420 & 0.828 & 0.157 \\
% higgs-llama & 342.6  & 0.944 & 0.433 & \textbf{0.940} & \textbf{0.658} & 0.787 & 0.082 \\
% llama-3-3-70b & 225.7 & 0.960 & 0.467 & 0.936 & 0.466 & 0.810 & 0.195 \\
% \bottomrule
% \end{tabular}
% \caption{Evaluation Results}
% \label{tab:evaluation_results}
% \end{table*}
% \begin{table*}[!ht]
% \centering
% \begin{tabular}{lrrrrrrrr}
% \toprule
% Model & Length & Rounds & Fact  & Personality & choice & Interestingness &  Mechanics  &  Msechanics  round \\ 
% \midrule
% gpt-4o & 201.9 & 9.736 & 0.892 & 0.468 & 0.894 & 0.502 & 0.921 & 0.648 \\
% deepseek-chat & 309.5 & 8.984 & 0.964 & 0.465 & 0.918 & 0.502 & 0.827 & 0.258 \\
% gemini-1.5-pro & 198.0 & 8.784 & 0.968 & 0.478 & 0.894 & 0.602 & 0.907 & 0.534 \\
% gemini-2.0-flash-exp & 195.3 & 8.744 & 0.885 & \textbf{0.479} & 0.865 & 0.538 & \textbf{0.936} & \textbf{0.745} \\
% gpt-4o-mini & 282.5 & 8.976 & 0.955 & 0.468 & 0.900 & 0.496 & 0.613 & 0.141 \\
% claude-3-5-sonnet & 563.0 & 9.552 & \textbf{0.991} & 0.471 & 0.923 & 0.722 & 0.838 & 0.099 \\
% llama-3-1-70b & 279.2 & 8.440 & 0.977 & 0.468 & 0.915 & 0.420 & 0.828 & 0.157 \\
% higgs-llama & 342.6 & 9.044 & 0.944 & 0.433 & \textbf{0.940} & \textbf{0.658} & 0.787 & 0.082 \\
% llama-3-3-70b & 225.7 & 8.320 & 0.960 & 0.467 & 0.936 & 0.466 & 0.810 & 0.195 \\
% \bottomrule
% \end{tabular}
% \caption{Evaluation Results}
% \label{tab:evaluation_results}
% \end{table*}
\section{Electronic Submission}
\label{submission}

Submission to ICML 2025 will be entirely electronic, via a web site
(not email). Information about the submission process and \LaTeX\ templates
are available on the conference web site at:
\begin{center}
\textbf{\texttt{http://icml.cc/}}
\end{center}

The guidelines below will be enforced for initial submissions and
camera-ready copies. Here is a brief summary:
\begin{itemize}
\item Submissions must be in PDF\@. 
\item If your paper has appendices, submit the appendix together with the main body and the references \textbf{as a single file}. Reviewers will not look for appendices as a separate PDF file. So if you submit such an extra file, reviewers will very likely miss it.
\item Page limit: The main body of the paper has to be fitted to 8 pages, excluding references and appendices; the space for the latter two is not limited in pages, but the total file size may not exceed 10MB. For the final version of the paper, authors can add one extra page to the main body.
\item \textbf{Do not include author information or acknowledgements} in your
    initial submission.
\item Your paper should be in \textbf{10 point Times font}.
\item Make sure your PDF file only uses Type-1 fonts.
\item Place figure captions \emph{under} the figure (and omit titles from inside
    the graphic file itself). Place table captions \emph{over} the table.
\item References must include page numbers whenever possible and be as complete
    as possible. Place multiple citations in chronological order.
\item Do not alter the style template; in particular, do not compress the paper
    format by reducing the vertical spaces.
\item Keep your abstract brief and self-contained, one paragraph and roughly
    4--6 sentences. Gross violations will require correction at the
    camera-ready phase. The title should have content words capitalized.
\end{itemize}



% \begin{algorithm}[tb]
%    \caption{Bubble Sort}
%    \label{alg:example}
% \begin{algorithmic}
%    \STATE {\bfseries Input:} data $x_i$, size $m$
%    \REPEAT
%    \STATE Initialize $noChange = true$.
%    \FOR{$i=1$ {\bfseries to} $m-1$}
%    \IF{$x_i > x_{i+1}$}
%    \STATE Swap $x_i$ and $x_{i+1}$
%    \STATE $noChange = false$
%    \ENDIF
%    \ENDFOR
%    \UNTIL{$noChange$ is $true$}
% \end{algorithmic}
% \end{algorithm}


% \begin{table}[t]
% \caption{Classification accuracies for naive Bayes and flexible
% Bayes on various data sets.}
% \label{sample-table}
% \vskip 0.15in
% \begin{center}
% \begin{small}
% \begin{sc}
% \begin{tabular}{lcccr}
% \toprule
% Data set & Naive & Flexible & Better? \\
% \midrule
% Breast    & 95.9$\pm$ 0.2& 96.7$\pm$ 0.2& $\surd$ \\
% Cleveland & 83.3$\pm$ 0.6& 80.0$\pm$ 0.6& $\times$\\
% Glass2    & 61.9$\pm$ 1.4& 83.8$\pm$ 0.7& $\surd$ \\
% Credit    & 74.8$\pm$ 0.5& 78.3$\pm$ 0.6&         \\
% Horse     & 73.3$\pm$ 0.9& 69.7$\pm$ 1.0& $\times$\\
% Meta      & 67.1$\pm$ 0.6& 76.5$\pm$ 0.5& $\surd$ \\
% Pima      & 75.1$\pm$ 0.6& 73.9$\pm$ 0.5&         \\
% Vehicle   & 44.9$\pm$ 0.6& 61.5$\pm$ 0.4& $\surd$ \\
% \bottomrule
% \end{tabular}
% \end{sc}
% \end{small}
% \end{center}
% \vskip -0.1in
% \end{table}

% \begin{definition}
% \label{def:inj}
% A function $f:X \to Y$ is injective if for any $x,y\in X$ different, $f(x)\ne f(y)$.
% \end{definition}
% \begin{proposition}
% If $f$ is injective mapping a set $X$ to another set $Y$, 
% the cardinality of $Y$ is at least as large as that of $X$
% \end{proposition}
% \begin{proof} 
% Left as an exercise to the reader. 
% \end{proof}
% \begin{lemma}
% \label{lem:usefullemma}
% For any $f:X \to Y$ and $g:Y\to Z$ injective functions, $f \circ g$ is injective.
% \end{lemma}
% \begin{theorem}
% \label{thm:bigtheorem}
% If $f:X\to Y$ is bijective, the cardinality of $X$ and $Y$ are the same.
% \end{theorem}
% \begin{corollary}
% If $f:X\to Y$ is bijective, 
% the cardinality of $X$ is at least as large as that of $Y$.
% \end{corollary}
% \begin{assumption}
% The set $X$ is finite.
% \label{ass:xfinite}
% \end{assumption}
% \begin{remark}
% According to some, it is only the finite case (cf. \cref{ass:xfinite}) that is interesting.
% \end{remark}
%restatable


% \textbf{Do not} include acknowledgements in the initial version of
% the paper submitted for blind review.

% If a paper is accepted, the final camera-ready version can (and
% usually should) include acknowledgements.  Such acknowledgements
% should be placed at the end of the section, in an unnumbered section
% that does not count towards the paper page limit. Typically, this will 
% include thanks to reviewers who gave useful comments, to colleagues 
% who contributed to the ideas, and to funding agencies and corporate 
% sponsors that provided financial support.

You can have as much text here as you want. The main body must be at most $8$ pages long.
For the final version, one more page can be added.
If you want, you can use an appendix like this one.  

The $\mathtt{\backslash onecolumn}$ command above can be kept in place if you prefer a one-column appendix, or can be removed if you prefer a two-column appendix.  Apart from this possible change, the style (font size, spacing, margins, page numbering, etc.) should be kept the same as the main body.


% In the unusual situation where you want a paper to appear in the
% references without citing it in the main text, use \nocite
\nocite{langley00}