\section{Related Work}
\subsection{Self-supervised Learning for ECG Signals}
In recent years, ECG self-supervised learning (eSSL) has demonstrated its ability to learn general representations from unlabeled ECG signals, significantly improving the performance of downstream tasks ____. eSSL methods can be broadly categorized into two types: contrastive-based methods and reconstruction-based methods. For contrastive-based approaches, CLOCS ____ enhances contrastive learning by leveraging cross-space, time, and patient-level relationships in ECG signals, while ASTCL ____ employs adversarial learning to capture spatio-temporal invariances in ECG signals. ISL ____ enhances cross-subject generalization ability through inter-subject and intra-subject contrastive learning, while BTFS ____ enhances ECG signal classification performance by combining time-domain and frequency-domain contrastive learning. On the other hand, reconstruction-based methods like MaeFE ____ and ST-MEM ____ adopt a spatio-temporal approach, learning general ECG representations by masking and reconstructing temporal or spatial content. CRT ____ obtains general representations in ECG signals by mutually reconstructing the time-domain and frequency-domain data.  However, existing eSSL methods predominantly focus on spatio-temporal or time-frequency domain representation learning of ECG signals, treating them as ordinary time-series data. This perspective often neglects the morphologically rich semantic information embedded in individual heartbeats.






\subsection{ECG Language Processing}

ECG language processing (ELP) is an emerging paradigm for handling ECG signals, first proposed by ____. Since ECG signals inherently possess significant and clear semantic information in heartbeats, they can be processed using methods similar to natural language processing (NLP). Both ____ and ____ employ approaches that segment different waves within heartbeats to construct vocabularies for modeling. However, when dealing with ECG signals of varying quality, existing methods struggle to accurately segment fine-grained waveforms. Moreover, current ELP methods have relatively small vocabularies (no more than 70 clusters), which limits the richness of the semantic information. In addition, research on ELP remains sparse, highlighting it as a field in urgent need of further exploration. To address these limitations, we propose a new perspective that directly treats heartbeats as words for modeling and have built the largest ECG vocabulary to date, consisting of 5,394 words, which will significantly advance the development of the ELP research field.