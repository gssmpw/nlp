\newcommand{\apmb}[2]{#1\textcolor{black!60}{$\scriptstyle\pm$\scriptsize #2}}
\begin{table}[t]
    \vskip -0.1in
    \caption{
    \textbf{Comparison with generic diffusion models on Kinetics-600.} \xred{``\ding{55}''}, \xorange{``\ding{115}''}, and \xgreen{``\ding{52}''} indicate whether a model can condition on a ``single predefined,'' ``arbitrary under approximation,'' or ``arbitrary'' history. DFoT, both trained from \emph{scratch} and \emph{fine-tuned}, outperforms all generic diffusion baselines under the same architecture and is on par with industry models trained with more compute resources (see \cref{app:exp_details_benchmarks}).
    }
    \label{tab:comparison_quantitative}
    \vskip 0.05in
    \centering
    \begin{adjustbox}{max width=\linewidth}
    \begin{tabular}{l c l l }
    \toprule
    & Flexible? & Method & FVD $\downarrow$\\
    \midrule
    \multirow{6}{*}{\rotatebox{90}{\shortstack[c]{Industry size\\and compute}}} & \multirow{3}{*}{\red{\ding{55}}} & MAGVIT-v2~\cite{yu2023language} & \apmb{4.3}{0.1}\\
    & & W.A.L.T~\cite{gupta2023photorealistic} & \apmb{\textbf{3.3}}{0.1}\\
    & & Rolling Diffusion~\cite{ruhe2024rolling} & 5.2\\

    \cmidrule{2-4}
    & \xorange{\ding{115}} & Video Diffusion~\cite{ho2022video} & \apmb{16.2}{0.3}\\
    \cmidrule{2-4}
    & \xgreen{\ding{52}} & MAGVIT~\cite{yu2023magvit} & \apmb{9.9}{0.3}\\
    \midrule
    \multirow{6}{*}{\rotatebox{90}{\shortstack[c]{Same\\ Architecture}}} & \xred{\ding{55}} & SD & \apmb{4.8}{0.0} \\
    \cmidrule{2-4}
    & \xorange{\ding{115}} & FS & \apmb{95.5}{0.4} \\
    \cmidrule{2-4}
    & \multirow{3}{*}{\xgreen{\ding{52}}} & BD & \apmb{6.4}{0.1} \\
    & & \textbf{\mtd} (\emph{scratch}) & \apmb{\textbf{4.3}}{0.1} \\
    & & \textbf{\mtd} (\emph{fine-tuned from FS}) & \apmb{\underline{4.7}}{0.0}\\
    \bottomrule
    \end{tabular}
    \end{adjustbox}
    \vskip -0.2in
\end{table}






\section{History Guidance}
\label{sec:our_history_guidance}

Leveraging the flexibility of \method (\mtd), we introduce \emph{History Guidance} (\HG), a family of techniques for history-conditioned video generation. These methods enhance generation quality, improve motion dynamics, enable robustness to out-of-distribution (OOD) histories, and unlock novel capabilities such as compositional video generation. Please refer to \cref{fig:sampling} for an overview.

\textbf{Simplest \HG: Vanilla History Guidance.} The simplest form of \HG, referred to as \emph{Vanilla History Guidance} (\HGv), directly performs classifier-free guidance (CFG) with a chosen history length, following Equation~\ref{eq:history_guidance}. The conditional score for any history $\cH$ can be computed as described in the previous section. To perform CFG, we need to estimate the \emph{unconditional} score $\score p_k(\xGk)$. Notably, the unconditional score is a special case of the conditional score with $\cH \tighteq \varnothing$ and can be estimated by masking history frames $\xH$ with \emph{complete} noise. Even this simple form of \HG significantly improves generation quality and consistency.

\textbf{History Guidance Across Time and Frequency.} While history guidance has been presented as a special case of CFG so far, its full potential extends far beyond CFG.
Consider the following generalization of Equation~\ref{eq:history_guidance}:
\vspace{-5pt}
\begin{equation}
\scalebox{0.85}{$
\score p_k(\xGk) + \sum_i \omega_i \big[\score p_k(\xGk|\bx_{\cH_i}^{k_{\cH_i}}) - \score p_k(\xGk)\big]
$},
\label{eq:history_guidance_across_tf}
\vspace{-5pt}
\end{equation}
where the total score is a weighted sum of conditional scores, each conditioned on possibly \emph{different segments of history} $\{\cH_i\}$, and each masked with a possibly \emph{different noise level} $k_{\cH_i}$. This formulation enables better generalization than a single score function conditioned on a full long history. By composing scores, each individual score component operates on a restricted conditional context, reducing the likelihood of being out-of-distribution~\cite{du2024compositional}. Appendix~\ref{appendix:add_score} provides informal mathematical intuition on why summing conditional scores is permissible.

Equation~\ref{eq:history_guidance_across_tf} effectively allows us to compose the scores conditioned on 1) different history subsequences, and 2) history frames that are partially noisy. We refer to these two principal axes as \emph{time} and \emph{frequency}, which together form a 2D plane of options that we refer to as \emph{History Guidance across Time and Frequency}. For simplicity, we introduce composition along these two axes separately.


\begin{table*}[ht!]
\centering
\footnotesize
\renewcommand{\arraystretch}{1.2}
\begin{tabular}{p{2.15cm} p{14.05cm}}
\toprule
 & \includegraphics[height=2.6cm]{figure/qualitative_example.pdf}\\
\midrule
Original \phantom{00000} Description & Bromazepam binds to the GABA receptor GABA-A, causing a conformational change and \textbf{increasing inhibitory effects of GABA}. Other neurotransmitters are not influenced. One of the \textbf{benzodiazepines} that is \textbf{used in the treatment of anxiety disorders.} \\
\midrule
User & Explain the features of the given molecule that are most distinct from those of other molecules. \\
\midrule
Mol-LLaMA (Ours) & The given molecule is a member of the \textcolor{tabblue}{benzodiazepine class}, which is characterized by a fusion of a benzene ring and a seven-membered diazepine ring. This structure is crucial for its biological activity, particularly its role as \textcolor{tabblue}{a sedative and hypnotic agent}. ...$<$omitted$>$...\\

& The given molecule is also characterized by a ketone group within the diazepine ring, which is a common feature in benzodiazepines. \textcolor{tabblue}{This functional group is important for the interaction with the GABA-A receptor, enhancing the inhibitory effects of GABA}. ...$<$omitted$>$...\\

\midrule
GPT-4o & 1. **Quinazoline Core**: The molecule contains a \textcolor{tabred}{quinazoline core}, ...$<$omitted$>$... \\
 & Overall, the combination of a \textcolor{tabred}{quinazoline} structure with unique substitution patterns, including bromine halogenation and the presence of a pyridinyl group, ...$<$omitted$>$... . \\
\midrule
LLaMo & The molecule has a \textcolor{tabblue}{benzodiazepine} structure with a bromo substituent at the 7-position and \textcolor{tabred}{a methyl group at the 1-position}. It is distinct from other molecules in that it contains a bromine atom, a nitrogen atom, and a methyl group, which are not present in the other molecules. \\
\midrule
3D-MoLM & ...$<$omitted$>$... It belongs to the class of compounds called \textcolor{tabred}{quinazolines}, which are characterized by a bicyclic structure consisting of a benzene ring fused to a pyrimidine ring. ...$<$omitted$>$...\\
& The molecule's structure suggests potential applications in medicinal chemistry, as quinazolines have been found to possess various biological activities, including \textcolor{tabred}{antitumor, antimicrobial, and anti-inflammatory properties.} \\
\midrule
Mol-Instructions & The molecule is a \textcolor{tabred}{quinoxaline derivative}. \\
\bottomrule
\end{tabular}
\vspace{-0.1in}
\caption{\small Case study to compare molecular understanding and reasoning ability. Mol-LLaMA accurately understands the molecular features, answering a correct molecular taxonomy and providing its distinct properties that are relevant to the given molecule.}
\label{tab:qualitative}
\vspace{-0.1in}
\end{table*}
\begin{figure}[t]
    \begin{subfigure}[t]{0.49\textwidth}
        \centering
        \includegraphics[width=\textwidth]{figures/binary_guidance/quantitative}
        \vskip -0.02in
        \caption{
            FVD as a function of guidance scale $\boldsymbol{\omega}$ for \mtd and BD using \HG. Both with \HGv, \mtd yields better FVD-$\omega$ curves than BD and thus achieves a lower best FVD score. Applying \HGf, which is specific to \mtd, enlarges the performance gap.
        }
        \label{fig:binary_guidance_quantitative}
    \end{subfigure}
    \hfill
    \begin{subfigure}[t]{0.49\textwidth}
        \raisebox{0.09in}{
            \includegraphics[width=\textwidth]{figures/binary_guidance/qualitative}
        }
        \vskip -0.02in
        \caption{
            Qualitative comparison of \mtd and BD using \HGv with optimal guidance scales $\omega = 1.5$. While \mtd generates consistent, high-quality samples, BD struggles to remain consistent with the history frames and produces artifacts. \textcolor{red}{\setlength{\fboxsep}{1.5pt}\textcolor{red}{\fbox{\textcolor{black}{Red box}}}} = history frames.
        }
        \label{fig:binary_guidance_qualitative}
    \end{subfigure}
    \vskip -0.05in
    \caption{
        \textbf{History Guidance works better with \mtd than with Binary-Dropout Diffusion (BD).}
    }
    \label{fig:binary_guidance}
    \vskip -0.1in
\end{figure}


\textbf{Time Axis: Temporal History Guidance.}
Due to the curse of dimensionality, the amount of data that we require to guarantee constant data support grows exponentially with the length of history we wish to condition on. As a result, history conditioned models are particularly prone to out-of-distribution (OOD) history without an inductive bias of sparse dependency. Common symptoms include blowing up or overfitting to irrelevant features. To address this, we propose \emph{Temporal History Guidance} (\HGt), which composes scores conditioned on different subsequences of history by setting $k_{\cH_i} = 0$ in Equation \ref{eq:history_guidance_across_tf}.
This composition can be performed with either: 1) long and short history $\{\cH_{\text{long}}, \cH_{\text{short}}\}$, aiming to trade-off between the two imperfect predictive models, reducing the likelihood of OOD while preserving both long and short-term dependencies, or 2) multiple short, overlapping in-distribution histories $\{\cH_{\text{short}_1}, \cH_{\text{short}_2}, \cdots\}$, to simulate the conditional distribution of the full history.

\begin{figure}[t]
    \begin{subfigure}[t]{0.49\textwidth}
        \centering
        \includegraphics[width=\textwidth]{figures/binary_guidance/quantitative}
        \vskip -0.02in
        \caption{
            FVD as a function of guidance scale $\boldsymbol{\omega}$ for \mtd and BD using \HG. Both with \HGv, \mtd yields better FVD-$\omega$ curves than BD and thus achieves a lower best FVD score. Applying \HGf, which is specific to \mtd, enlarges the performance gap.
        }
        \label{fig:binary_guidance_quantitative}
    \end{subfigure}
    \hfill
    \begin{subfigure}[t]{0.49\textwidth}
        \raisebox{0.09in}{
            \includegraphics[width=\textwidth]{figures/binary_guidance/qualitative}
        }
        \vskip -0.02in
        \caption{
            Qualitative comparison of \mtd and BD using \HGv with optimal guidance scales $\omega = 1.5$. While \mtd generates consistent, high-quality samples, BD struggles to remain consistent with the history frames and produces artifacts. \textcolor{red}{\setlength{\fboxsep}{1.5pt}\textcolor{red}{\fbox{\textcolor{black}{Red box}}}} = history frames.
        }
        \label{fig:binary_guidance_qualitative}
    \end{subfigure}
    \vskip -0.05in
    \caption{
        \textbf{History Guidance works better with \mtd than with Binary-Dropout Diffusion (BD).}
    }
    \label{fig:binary_guidance}
    \vskip -0.1in
\end{figure}


\textbf{Frequency Axis: Fractional History Guidance.} 
We observe that a major failure mode of \HGv under high guidance scales is the generation of overly static videos with minimal motion. This occurs because \HGv encourages consistency with history, leading to a trivial solution of simply copying the most recent history frame. To address this, we propose \emph{Fractional History Guidance} (\HGf), which guides the sampling process using \emph{fractionally masked history}. Fractionally masking history retains only low-frequency information~(\citet{dieleman2024spectral}, \cref{app:frequency_guidance}), allowing high-frequency details (e.g., fine textures and fast motions) to remain unconstrained by guidance. This approach makes videos more dynamic while maintaining consistency, which is mainly associated with low-frequency details.
Specifically, the \HGf score is given by:
\vspace{-4pt}
\begin{equation} 
\label{eq:fractional_guidance}
\scalebox{0.85}{$
\score p_k(\xGk | \xH) + \omega \big[\score p_k\big(\xGk | \bx_{\cH}^{k_{\cH}}\big) - \score p_k(\xGk)\big]
$},
\vspace{-2pt}
\end{equation}
where $\kH \in (0, 1)$ controls the degree of masking to focus on lower-frequency details, and $\omega$ is the guidance scale for the partially masked history $\bx_{\cH}^{k_{\cH}}$.
In principle, different history frames could contribute information at different frequency bands, such as high-frequency details from recent frames and low-frequency motion from earlier frames. While a detailed exploration of sophisticated sampling strategies is left to future work, our experiments show that even simple implementations of \HGf significantly improve motion dynamics without sacrificing consistency.















