\begin{figure}[t]
    \begin{subfigure}[t]{0.49\textwidth}
        \centering
        \includegraphics[width=\textwidth]{figures/fine_tune/training_curve_full}
        \vskip -0.02in
        \caption{
            A comprehensive view of the training loss curves. \xgreen{DFoT \emph{(fine-tuned)}} achieves a low training loss early in the iterations and converges significantly faster than \xblue{DFoT \emph{(scratch)}}.
        }
        \label{fig:training_curve_full}
    \end{subfigure}
    \hfill
    \begin{subfigure}[t]{0.49\textwidth}
        \includegraphics[width=\textwidth]{figures/fine_tune/training_curve_zoomed}
        \vskip -0.02in
        \caption{
            A zoomed-in view of the training loss curves. Only after 80k iterations, \xgreen{DFoT \emph{(fine-tuned)}} displays a lower training loss than \xblue{DFoT \emph{(scratch)}} trained for 640k iterations.
        }
        \label{fig:training_curve_zoomed}
    \end{subfigure}
    \vskip -0.05in
    \caption{
        \textbf{Training loss curves for \mtd, \xblue{trained from \emph{scratch}} and \xgreen{\emph{fine-tuned} from the pre-trained FS model}, on Kinetics-600.}
    }
    \label{fig:training_curve}
\end{figure}
