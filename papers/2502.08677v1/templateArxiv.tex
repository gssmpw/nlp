\documentclass{article}


\usepackage{PRIMEarxiv}

\usepackage[utf8]{inputenc} % allow utf-8 input
\usepackage[T1]{fontenc}    % use 8-bit T1 fonts
\usepackage{hyperref}       % hyperlinks
\usepackage{url}            % simple URL typesetting
\usepackage{booktabs}       % professional-quality tables
\usepackage{amsfonts}       % blackboard math symbols
\usepackage{nicefrac}       % compact symbols for 1/2, etc.
\usepackage{microtype}      % microtypography
\usepackage{lipsum}
\usepackage{fancyhdr}       % header
\usepackage{graphicx}       % graphics
\graphicspath{{media/}}     % organize your images and other figures under media/ folder
%% For including figures, graphicx.sty has been loaded in elsarticle.cls.
%% The amssymb package provides various useful mathematical symbols
\usepackage{amssymb,amsmath}
\usepackage{graphicx}
\usepackage{hyperref}
\usepackage{listings}
\usepackage{xcolor}
\usepackage{longtable} % Include in your preamble
\usepackage{graphicx} % Include this in your preamble
\usepackage[backend=bibtex,style=numeric]{biblatex}
\addbibresource{references.bib}
\usepackage{geometry}
\usepackage{tabularx} 
\usepackage{tabu}
\usepackage{adjustbox}


%% Define custom colors for listings
\definecolor{codegreen}{rgb}{0,0.6,0}
\definecolor{codegray}{rgb}{0.5,0.5,0.5}
\definecolor{codepurple}{rgb}{0.58,0,0.82}
\definecolor{backcolour}{rgb}{0.95,0.95,0.92}

%% Set up the listings environment
\lstdefinestyle{mystyle}{
    backgroundcolor=\color{backcolour},   
    commentstyle=\color{codegreen},
    keywordstyle=\color{magenta},
    numberstyle=\tiny\color{codegray},
    stringstyle=\color{codepurple},
    basicstyle=\ttfamily\footnotesize,
    breakatwhitespace=false,         
    breaklines=true,                 
    captionpos=b,                    
    keepspaces=true,                 
    numbers=left,                    
    numbersep=5pt,                  
    showspaces=false,                
    showstringspaces=false,
    showtabs=false,                  
    tabsize=2
}
\lstset{style=mystyle}

%% Journal name
%Header
\pagestyle{fancy}
\thispagestyle{empty}
\rhead{ \textit{ }} 

% Update your Headers here
\fancyhead[LO]{RMCDA, a universal R package for MCDA}
% \fancyhead[RE]{Firstauthor and Secondauthor} % Firstauthor et al. if more than 2 - must use \documentclass[twoside]{article}



  
%% Title
\title{RMCDA: The Comprehensive R Library for Applying Multi-Criteria Decision Analysis Methods
%%%% Cite as
%%%% Update your official citation here when published 


}

\author{
  Annice Najafi$^{1}$, Shokoufeh Mirzaei$^{1,*}$ \\
  $^{1}$California Polytechnic Institute of Technology, Pomona \\
  Pomona, CA\\
  $^{*}$Corresponding author \\
  \texttt{smirzaei@cpp.edu}
}


\begin{document}
\maketitle


\begin{abstract}
Multi-Criteria Decision Making (MCDM) is a branch of operations research used in a variety of domains from health care to engineering to facilitate decision-making among multiple options based on specific criteria. Several \texttt{R} packages have been developed for the application of traditional MCDM approaches. However, as the discipline has advanced, many new approaches have emerged, necessitating the development of innovative and comprehensive tools to enhance the accessibility of these methodologies. Here, we introduce \texttt{RMCDA}, a comprehensive and universal \texttt{R} package that offers access to a variety of established MCDM approaches (e.g., \texttt{AHP}, \texttt{TOPSIS}, \texttt{PROMETHEE}, and \texttt{VIKOR}), along with newer techniques such as Stratified MCDM (\texttt{SMCDM}) and the Stratified Best-Worst Method (\texttt{SBWM}). Our open source software intends to broaden the practical use of these methods through supplementary visualization tools and straightforward installation. 
\end{abstract}


% keywords can be removed
\keywords{MCDA \and MCDM \and \texttt{R} package \and Universal Library \and Decision Making}

\section*{Metadata}

\begin{table}[!h]
\caption{Code metadata}
\vspace{5mm}
\begin{tabular}{|l|p{6.5cm}|p{6.5cm}|}
\hline
\textbf{Nr.} & \textbf{Code metadata description} & \textbf{Metadata} \\
\hline
C1 & Current code version & 0.3 \\
\hline
C2 & Permanent link to code/repository & \url{https://github.com/AnniceNajafi/RMCDA} \\
\hline
C3 & Permanent link to Reproducible Capsule & None\\
\hline
C4 & Legal Code License   & MIT License \\
\hline
C5 & Code versioning system used & git \\
\hline
C6 & Software code languages, tools and services used & \texttt{R} \\
\hline
C7 & Compilation requirements, operating environments \& dependencies & \texttt{R} package: \texttt{dplyr},
    \texttt{stats},
    \texttt{igraph},
    \texttt{fmsb},
    \texttt{lpSolve},
    \texttt{MASS},
    \texttt{matlib},
    \texttt{nloptr},
    \texttt{matrixStats},
    \texttt{pracma},.\\
\hline
C8 & Link to developer documentation/manual & \url{https://github.com/AnniceNajafi/RMCDA/blob/main/RMCDA_0.3.pdf} \\
\hline
C9 & Support email for questions & \texttt{annicenajafi27@gmail.com}\\
\hline
\end{tabular}
\label{codeMetadata} 
\end{table}

\section{Motivation and significance}
Multi-Criteria Decision Analysis (MCDA) methods have been historically used to compare and rank options based on multiple criteria \cite{mardani2015multiple, zionts1979mcdm}. Over the years, various approaches, such as the Analytic Hierarchy Process (\texttt{AHP}) \cite{saaty1980analytic} and the Technique for Order of Preference by Similarity to Ideal Solution (\texttt{TOPSIS}) \cite{tzeng2011multiple}, have been extensively developed. Bibliographic analyses reveal that \texttt{TOPSIS}, \texttt{AHP}, \texttt{VIKOR} (VIsekriterijumsko KOmpromisno Rangiranje), \texttt{PROMETHEE} (Preference Ranking Organization Method for Enrichment Evaluation), and \texttt{ANP} (Analytic Network Process) are among the most popular MCDA methods \cite{rocha2024review}.

Numerous \texttt{R} packages (e.g., \texttt{topsis} \cite{topsisR}, \texttt{ahpsurvey} \cite{ahpsurveyR}, \texttt{MCDA} \cite{MCDAR, bigaret2017supporting}, \texttt{AHPtools} \cite{AHPtools}, \texttt{PROMETHEE} \cite{ishizaka2018visual, greco2019methodological}, and others) have simplified the application of these established techniques. However, the scope of their functionality is limited to a few methods. Nonetheless, newer methods are not widely available in \texttt{R}, including the Stratified MCDM (\texttt{SMCDM}) \cite{asadabadi2018stratified}, and Stratified Best-Worst Method (\texttt{SBWM}) \cite{torkayesh2021sustainable}. Despite the recent introduction of these methods, they have been classified as having above-average Category Normalized Citation Impact by Web Of Science, highlighting their growing significance in MCDA research. These methods handle more complex and “stratified” problems, such as decisions that must account for changing probabilities or uncertain future states, such as policy planning under shifting economic or climate conditions.

While implementations of traditional MCDA methods exist in other programming environments (e.g., \texttt{pyDecision} \cite{pereira2024enhancing}, \texttt{pymcdm} \cite{kizielewicz2023pymcdm}, \texttt{pyrepo} \cite{wkatrobski2022pyrepo}, \texttt{JMcDM} \cite{satman2021jmcdm}, and even standalone tools such as DIVIZ \cite{meyer2012diviz}), there is currently no equivalent implementation of stratified MCDM approaches in these languages. Moreover, even though the \texttt{R} ecosystem has a large user base among researchers and data analysts, no package provides the same extent of MCDA methods available in other programming languages.



Our goal is therefore to offer a comprehensive \texttt{R} package (\texttt{RMCDA}) encompassing both new and established methods. By filling this gap, \texttt{RMCDA} can potentially broaden the accessibility of these methods, enabling both deterministic and probabilistic forms of decision-making. 

\section{Comparison with existing \texttt{R} libraries and other tools}
We used the R package `pkgsearch'\cite{pkgsearch} to find R packages with the keywords `MCDA' and `MCDM'. A total of 15 packages were found. The ones associated with the keyword `MCDA' were `cocosoR', `MCDA', `smaa', `PROMETHEE', `RXMCDA', `brisk', and `ConsRank'. Packages associated with the `MCDM' keyword were `ahptopsis2n', `topsis', `FuzzyMCDM', `sapevom', `IFMCDM', `fucom', `rafsi', `SDEFSR'. Of these packages, the `MCDA' and `MCDM' packages were the most general and covered the most MCDA methods (we use the term MCDA and MCDM interchangeably throughout this manuscript). 

Next, we used the `cranlogs' package in R \cite{cranlogs} to find the number of downloads of the `MCDA' and `MCDM' packages over time and through Locally Estimated Scatterplot Smoothing (LOES) showed that both packages followed similar temporal patterns in terms of their number of downloads. As shown in Figure \ref{cranlogs}A-B, the temporal patterns of the two packages can be divided into three phases. First, the packages saw a decline in popularity between years 2016 and 2018, followed by an increase in the number of downloads from 2018 to 2021 eventually accompanied by a sudden decline in popularity. To understand whether this pattern is specific to the two MCDA packages in R or indirectly caused by the changing popularity of R as a programming language, we applied the same methodology to two of the most used R packages, `ggplot2', and `dplyr' and compared the trajectories. The results showed a monotonic increase in the number of downloads of both packages followed by a decline after 2022 (Figure \ref{cranlogs}C-D). This was inconsistent with the temporal patterns of the `MCDA' and `MCDM' packages. We also note that the number of downloads of these two packages over time is inconsistent with the number of publications related to MCDA \cite{srivastava2024multi} suggesting a decline in the popularity of these R packages among users despite the increase in the popularity of MCDA methods in general. 


\begin{figure}
    \centering
    \includegraphics[width=0.8\linewidth]{cranlogs.jpg}
    \caption{\textbf{A-E show the number of downloads over time by package. F compares the number of supported methods by package.}}
    \label{cranlogs}
\end{figure}

To understand if this pattern is specific to MCDA-related packages in R or consistent across packages in other programming languages, we used the `pypinfo' package in Python \cite{levwilk} to retrieve the number of downloads for one of the comprehensive Python packages for MCDA, the \texttt{pyDecision} package \cite{pereira2024enhancing}. The results showed a linear increase in the number of downloads of this package since its introduction, suggesting that the decline in the popularity of the most comprehensive R packages may be due to their lack of coverage of the newer MCDA methods (Figure \ref{cranlogs}E). To put this into perspective, we found the total number of MCDA methods included in each package and have plotted the results in Figure \ref{cranlogs}F. As demonstrated in the figure, the MCDA-related packages in Python offer significantly more methods than the previously introduced R packages. The figure also demonstrates that the MCDA package in R provides a significant number of unique methods that are not covered in other MCDA-related libraries, indicating that the package may include methods that are unpopular. 
Compared to other options, our RMCDA package stands out by offering widely used methods among researchers. For example, it includes COPRAS \cite{zavadskas2007multi}, SMCDM \cite{asadabadi2018stratified}, and SBWM \cite{torkayesh2021sustainable} that were previously unavailable in any R package. These methodologies have demonstrated strong academic impact, with above-average Category Normalized Citation Impact ratings on the Web of Science.

The RMCDA R package covers more than 50 popular MCDA and weighting methods; Analytical Hierarchy Process (AHP) \cite{saaty2004decision}, Analytical Network Process (ANP) \cite{saaty2006decision}, Additive Ratio ASsessment (ARAS) \cite{zavadskas2010new}, Borda \cite{borda1781m}, Best Worst Method (BWM) \cite{rezaei2015best}, Criterion Impact LOSs (CILOS) \cite{zavadskas2016integrated}, Combined Compromise Solution (CoCoSo) \cite{yazdani2019combined}, Combinative Distance-based ASsessment (CODAS) \cite{keshavarz2016new}, Copeland \cite{saari1996copeland}, Complex Proportional ASsessment (COPRAS) \cite{zavadskas2007multi}, Compromise RAnking and Distance from Ideal Solution (CRADIS) \cite{puvska2022evaluation}, CRiteria Importance Through Intercriteria Correlation (CRITIC) \cite{diakoulaki1995determining}, DEcision-MAking Trial and Evaluation Laboratory (DEMATEL) \cite{si2018dematel}, Evaluation based on Distance from Average Solution (EDAS) \cite{keshavarz2015multi}, ELimination Et Choix Traduisant la REalité (ELECTRE) \cite{roy1968classement}, Entropy \cite{shannon1948mathematical}, Grey Relational Analysis (GRA) \cite{kuo2008use}, Integrated Determination of Objective CRIteria Weights (IDOCRIW) \cite{zavadskas2016integrated}, Multi-Attributive Border Approximation Area Comparison (MABAC) \cite{pamuvcar2015selection}, Measuring Attractiveness by a Categorical Based Evaluation TecHnique (MACBETH) \cite{marcelino2019development}, Multi-Attribute Ideal Real Comparative Analysis (MAIRCA) \cite{hadian2022multi}, Multi-Attribute Ranking Approach (MARA) \cite{gligoric2022novel}, Measurement of Alternatives and Ranking based on COmpromise Solution (MARCOS) \cite{stevic2020sustainable}, Multi-Attribute Utility Theory \cite{keeney1993decisions}, Multi-Objective Optimization by Ratio Analysis (MOORA) \cite{brauers2006moora}, Multi-Objective Optimization on the basis of Simple Ratio Analysis (MOOSRA) \cite{jagadish2014green}, Multi-Objective Optimization on the basis of a Ratio Analysis plus the full MULTIplicative form (Multi-MOORA) \cite{brauers2006moora}, Outranking Compromise Ranking Approach (OCRA) \cite{madic2015selection}, Ordered Performance Analysis (OPA) \cite{ataei2020ordinal}, Organisation, Rangement Et Synthèse De Données Relatives À L’évaluation (ORESTE) \cite{roubens1982preference}, Position Index Value (PIV) \cite{mufazzal2018new}, Preference Ranking Organization Method for Enrichment Evaluation (PROMETHEE) \cite{brans2016promethee}, Preference Selection Index (PSI) \cite{maniya2010selection}, Ranking Alternatives using Full Subset Inference (RAFSI) \cite{vzivzovic2020eliminating}, Regime \cite{hinloopen1990qualitative}, Ranking Index Method (RIM) \cite{cables2016rim}, Range of Value (ROV) \cite{madic2016application}, Simple Additive Weighting (SAW) \cite{panjaitan2019simple}, Stratified Best Worst Method (SBWM) \cite{torkayesh2021sustainable}, Simple Evaluation of Complex Alternatives (SECA) \cite{keshavarz2018simultaneous}, Simple Multi-Attribute Rating Technique (SMART) \cite{olson1997decision}, Stratified Multi-Criteria Decision Making (SMCDM) \cite{asadabadi2018stratified}, Stable Preference Ordering Towards Ideal Solution (SPOTIS) \cite{dezert2020spotis}, Simple Ranking Method using reference Profiles (SRMP) \cite{khannoussi2022simple}, TOmada de Decisão Interativa e Multicritério (TODIM) \cite{gomes2009application}, Technique for Order of Preference by Similarity to Ideal Solution (TOPSIS) \cite{hwang1981methods}, VIšekriterijumsko KOmpromisno Rangiranje (VIKOR) \cite{opricovic2004compromise}, Weighted Aggregated Sum Product ASsessment (WASPAS) \cite{zavadskas2012optimization}, Weighted Product Method (WPM), Weighted Sum Method (WSM) \cite{san2012weighted}, Weighted Influence Nonlinear Gauge System (WINGS) \cite{michnik2013weighted}, and Weighted Influence Score Preference (WISP) \cite{stanujkic2021integrated}. 
Among the four MCDA packages that we investigated, only four methods are not included in the RCMDA package; Simulated Uncertainty Range Evaluations (SURE) \cite{hodgett2019sure}, Multi-Attribute Range Evaluations \cite{hodgett2014handling}, Utilité Additive (UTA) \cite{jacquet1982assessing}, and Characteristic Objects METhod (COMET) \cite{salabun2015characteristic}.
Table~\ref{comparisonTable} illustrates a short comparison of the key MCDA-related packages in different programming languages. By introducing RMCDA, we aim to provide a universal and comprehensive R library that enhances the accessibility to MCDA and criteria-weighing methods. We note that in the table we have not included the implementation of fuzzy variations of the methods (RMCDA only offers the fuzzy variation of the AHP method \cite{pehlivan2017comparison}). 


\clearpage % Ensures the table starts on a new page

\begin{longtabu}[c]{|c|c|c|c|c|c|}
\caption{Comparison of select MCDA packages with \texttt{RMCDA}.}
\label{comparisonTable} \\
\hline
\textbf{Feature} & \texttt{MCDA(R)} & \texttt{MCDM(R)} & \texttt{PyMCDM(Python)} & \texttt{pyDecision(Python)} & \texttt{RMCDA(R)} \\
\hline
\endfirsthead

\hline
\textbf{Feature} & \texttt{MCDA(R)} & \texttt{MCDM(R)} & \texttt{PyMCDM(Python)} & \texttt{pyDecision(Python)} & \texttt{RMCDA(R)} \\
\hline
\endhead

\hline
\multicolumn{6}{r}{\textit{Continued on next page}} \\
\hline
\endfoot

\hline
\endlastfoot

\texttt{AHP}      & \checkmark & -          & -          & \checkmark & \checkmark \\
\texttt{ANP}      & - & -          & -          & - & \checkmark \\
\texttt{ARAS}      & - & -          & \checkmark          & \checkmark & \checkmark \\
\texttt{BORDA}      & - & -          & -          & \checkmark & \checkmark \\
\texttt{BWM}       & - & -          & - & \checkmark          & \checkmark \\
\texttt{CILOS}    & -          & - & \checkmark          & \checkmark          & \checkmark \\
\texttt{COCOSO}    & -          & - & \checkmark          & \checkmark          & \checkmark \\
\texttt{CODAS}          & -          & -          & \checkmark          & \checkmark          & \checkmark \\
\texttt{COMET}   & -          & -          & \checkmark          & -          & - \\
\texttt{COPELAND}   & -          & -          & -          & \checkmark          & \checkmark \\
\texttt{COPRAS}   & -          & -          & \checkmark          & \checkmark          & \checkmark \\
\texttt{CRADIS}   & -          & -          & -          & \checkmark          & \checkmark \\
\texttt{CRITIC}   & -          & -          & \checkmark          & \checkmark          & \checkmark \\
\texttt{DEMATEL}   & -          & -          & -          & \checkmark          & \checkmark \\
\texttt{EDAS}   & -          & -          & \checkmark          & \checkmark          & \checkmark \\
\texttt{ELECTRE}   & \checkmark          & -          & -          & \checkmark          & \checkmark \\
\texttt{ENTROPY}   & -          & -          & \checkmark          & \checkmark          & \checkmark \\
\texttt{GRA}   & -          & -          & -          & \checkmark          & \checkmark \\
\texttt{IDOCRIW}   & -          & -          & \checkmark          & \checkmark          & \checkmark \\
\texttt{MABAC}   & -          & -          & \checkmark          & \checkmark          & \checkmark \\
\texttt{MACBETH}   & -          & -          & -          & \checkmark          & \checkmark \\
\texttt{MAICRA}   & -          & -          & \checkmark          & \checkmark          & \checkmark \\
\texttt{MARCOS}   & -          & -          & \checkmark          & \checkmark          & \checkmark \\
\texttt{MARA}   & -          & -          & -          & \checkmark          & \checkmark \\
\texttt{MARE}   & \checkmark          & -          & -          & -          & - \\
\texttt{MAUT}   & -          & -          & -          & \checkmark          & \checkmark \\
\texttt{MEREC}   & -          & -          & \checkmark          & \checkmark          & \checkmark \\
\texttt{MOORA}   & -          & \checkmark          & \checkmark          & \checkmark          & \checkmark \\
\texttt{MOOSRA}   & -          & -          & -          & \checkmark          & \checkmark \\
\texttt{MRSORT}   & \checkmark          & -          & -          & -          & \checkmark \\
\texttt{MULTIMOORA}   & -          & \checkmark          & -          & \checkmark          & \checkmark \\
\texttt{OCRA}   & -          & -          & \checkmark          & \checkmark          & \checkmark \\
\texttt{ORESTE}   & -          & -          & -          & \checkmark          & \checkmark \\
\texttt{PIV}   & -          & -          & -          & \checkmark          & \checkmark \\
\texttt{PROMETHEE}   & \checkmark          & -          & \checkmark          & \checkmark          & \checkmark \\
\texttt{PSI}   & -          & -          & -          & \checkmark          & \checkmark \\
\texttt{REGIME}   & -          & -          & -          & \checkmark          & \checkmark \\
\texttt{RIM}   & -          & \checkmark          & -          & \checkmark          & \checkmark \\
\texttt{ROV}   & -          & -          & -          & \checkmark          & \checkmark \\
\texttt{SAW}   & -          & -          & -          & \checkmark          & \checkmark \\
\texttt{SBWM}   & -          & -          & -          & -          & \checkmark \\
\texttt{SECA}   & -          & -          & -          & \checkmark          & \checkmark \\
\texttt{SMART}   & -          & -          & -          & \checkmark          & \checkmark \\
\texttt{SMCDM}   & -          & -          & -          & -          & \checkmark \\
\texttt{SPOTIS}   & -          & -          & \checkmark          & \checkmark          & \checkmark \\
\texttt{SRMP}   & \checkmark          & -          & -          & -          & \checkmark \\
\texttt{SURE}   & \checkmark          & -          & -          & -          & - \\
\texttt{TODIM}   & -          & -          & -          & \checkmark          & \checkmark \\
\texttt{TOPSIS}   & \checkmark          & \checkmark          & \checkmark          & \checkmark          & \checkmark \\
\texttt{UTA}   & \checkmark          & -          & -          & -          & - \\
\texttt{VIKOR}   & \checkmark          & \checkmark          & \checkmark          & \checkmark          & \checkmark \\
\texttt{WASPAS}   & -          & \checkmark          & -          & \checkmark          & \checkmark \\
\texttt{WPM}   & -          & \checkmark          & -          & \checkmark          & \checkmark \\
\texttt{WSM}   & -          & \checkmark          & -          & \checkmark          & \checkmark \\
\texttt{WINGS}   & -          & -          & -          & \checkmark          & \checkmark \\
\texttt{WISP}   & -          & -          & -          & \checkmark          & \checkmark \\
\end{longtabu}




\section{Software description and architecture}
The \texttt{RMCDA} package can be divided into three main segments: (i) functions for reading input data from CSV or data frames, (ii) functions to apply a variety of MCDM methods (both classical and advanced), and (iii) visualization and reporting utilities. 

\subsection{Overview of key functions}
Table \ref{functionTab} lists and describes major functions of the RMCDA package.

\begin{table}[!h]
\centering
\caption{Key Functions in \texttt{RMCDA}}
\vspace{5mm}
\resizebox{\linewidth}{!}{%
\begin{tabular}{|c|c|c|}
\hline
\textbf{Function Name} & \textbf{Type} & \textbf{Description}  \\
\hline
\texttt{apply.AHP}         & MCDA method & Runs AHP on pairwise comparison data \\
\texttt{apply.ANP}         & MCDA method & Extends AHP to network structures \\
\texttt{apply.ARAS}         & MCDA method & Applies ARAS to data \\
\texttt{apply.BORDA}         & MCDA method & Applies BORDA to data \\
\texttt{apply.BWM}         & MCDA method & Applies Best-Worst Method \\
\texttt{apply.CILOS}         & Weighting & Finds criteria weights using CILOS\\
\texttt{apply.COCOSO}         & MCDA method & Applies COCOSO to data \\
\texttt{apply.CODAS}         & MCDA method & Applies CODAS to data \\
\texttt{apply.COPELAND}         & MCDA method & Applies COPELAND to data \\
\texttt{apply.COPRAS}         & MCDA method & Applies COPRAS to data \\
\texttt{apply.CRADIS}         & MCDA method & Applies CRADIS to data \\
\texttt{apply.CRITIC}      & Weighting    & Finds criteria weights using CRITIC \\
\texttt{apply.DEMATEL}         & Weighting & Finds criteria weights using DEMATEL \\
\texttt{apply.EDAS}         & MCDA method & Applies EDAS to data \\
\texttt{apply.ELECTRE1}     & MCDA method    & Applies ELECTRE I method to data \\
\texttt{apply.Entropy}     & Weighting    & Finds criteria weights using Entropy \\
\texttt{apply.GRA}      & MCDA method & Applies GRA to data \\
\texttt{apply.IDOCRIW}      & Weighting & Finds criteria weights using IDOCRIW \\
\texttt{apply.MABAC}      & MCDA method & Applies MABAC to data \\
\texttt{apply.MACBETH}      & MCDA method & Applies MACBETH to data \\
\texttt{apply.MAIRCA}      & MCDA method & Applies MAIRCA to data \\
\texttt{apply.MARA}      & MCDA method & Applies MARA to data \\
\texttt{apply.MAUT}      & MCDA method & Applies MAUT to data \\
\texttt{apply.MARCOS}      & MCDA method & Applies MARCOS to data \\
\texttt{apply.MOORA}      & MCDA method & Applies MOORA to data \\
\texttt{apply.MOOSRA}      & MCDA method & Applies MOOSRA to data \\
\texttt{apply.MULTIMOORA}      & MCDA method & Applies Multi-MOORA to data \\
\texttt{apply.OCRA}      & MCDA method & Applies OCRA to data \\
\texttt{apply.OPA}      & MCDA method & Applies OPA to data \\
\texttt{apply.ORESTE}      & MCDA method & Applies ORESTE to data \\
\texttt{apply.PIV}      & MCDA method & Applies PIV to data \\
\texttt{apply.PROMETHEE}      & MCDA method & Applies PROMETHEE to data \\
\texttt{apply.PSI}      & MCDA method & Applies PSI to data \\
\texttt{apply.RAFSI}      & MCDA method & Applies RAFSI to data \\
\texttt{apply.REGIME}      & MCDA method & Applies REGIME to data \\
\texttt{apply.RIM}      & MCDA method & Applies RIM to data \\
\texttt{apply.ROV}      & MCDA method & Applies ROV to data \\
\texttt{apply.SAW}      & MCDA method & Applies SAW to data \\
\texttt{apply.SBWM}        & MCDA method & Applies Stratified Best-Worst Method to data \\
\texttt{apply.SECA}      & Weighting & Finds criteria weights using SECA \\
\texttt{apply.SMART}      & MCDA method& Applies SMART to data \\
\texttt{apply.SMCDM}       & MCDA method & Stratified MCDM solution \\
\texttt{apply.SPOTIS}      & MCDA method& Applies SPOTIS to data \\
\texttt{apply.SRMP}       & MCDA method & Applies SRMP to data  \\
\texttt{apply.TODIM}       & MCDA method & Applies TODIM to data  \\
\texttt{apply.TOPSIS}       & MCDA method & Applies TOPSIS to data  \\
\texttt{apply.VIKOR}       & MCDA method & Applies VIKOR to data\\
\texttt{apply.WASPAS}       & MCDA method & Applies WASPAS to data \\
\texttt{apply.WINGS}       & Weighting & Finds criteria weights using WINGS \\
\texttt{apply.WISP}       & MCDA method & Applies WISP to data \\
\texttt{apply.WSM}       & MCDA method & Applies WSM to data \\
\texttt{apply.WPM}       & MCDA method & Applies WPM to data \\




\hline
\texttt{read.csv.AHP.matrices}, \texttt{read.csv.SBWM.matrices}, \ldots & I/O & Import specialized CSV formats \\
\hline
\texttt{plot.AHP.decision.tree}, \texttt{plot.spider}, etc. & Visualization & Decision tree and radar/spider charts \\
\hline \label{functionTab}
\end{tabular}
}
\end{table}

\subsection{Software functionalities}
RMCDA offers a range of MCDM methods, each implemented with user-friendly functionality and optional visualizations to enhance interpretability of data. Below we provide an overview of three key methods that were not previously included in any comprehensive MCDA packages:


\subsection{Analytical Network Process (ANP)}
The Analytic Network Process (ANP) extends the Analytic Hierarchy Process (AHP) introduced by Saaty in 1980 by incorporating network structures rather than simple hierarchies \cite{saaty1980analytic}. ANP evaluates dependencies and feedback loops between criteria and alternatives. The methodology involves performing pairwise comparisons and computing priority weights using eigenvector analysis \cite{saaty2006decision}. ANP first extracts the criteria weight vector and the weighted matrix for alternatives. Then, it constructs a supermatrix that integrates these values, organizing the structure into a larger matrix where decision elements interact. This includes placing the criteria weights, the alternatives' weighted values, and an identity matrix to account for dependencies between alternatives. Finally, the function raises the supermatrix to a specified power, amplifying the influence of network relationships, and returns the processed matrix. This refined structure captures both direct and indirect dependencies in decision-making, making ANP a more robust method for complex evaluations compared to AHP alone.

\subsection{Stratified MCDM (SMCDM)}
The Stratified MCDM (SMCDM) method was developed by Asadabadi in 2018 \cite{asadabadi2018stratified} and is useful for situations involving uncertainty such as a buying a house where the importance of criteria (price, number of rooms, or proximity to schools) may change depending on the customer's situation. For example, the customer may earn a promotion with some probability which affects the importance of the price of the house. We consider the initial situation of the customer as state 0 located in stratum I and create a second stratum (stratum II) which contains states with the occurrence of only one possible event. In Figure \ref{SMCDM-stratums} we show three possible events that may happen; event $A$, event $B$, and event $C$. We create a third stratum (stratum III) which contains states with the concurrent occurrence of two states together and lastly we create a fourth stratum with one state in which all three states happen simultaneously. If the probabilities of occurrence of each state across strati are given then we calculate the weights of each criterion then the scores of alternatives. Otherwise, if then events happen independently of each other and the probability of occurrence of states in strati I and II are given, then we calculate the probability of occurrence of each state with respect to the first probability of staying in the baseline state (state 0). We know that the sum of the probabilities of the system being in either of the states should equate 1. Therefore:

\begin{equation}
    \sum_{i =0}^{8}p_i = p_0 + (\frac{w_1}{w_0}+\frac{w_2}{w_0}+\frac{w_3}{w_0})p_0 + (\frac{w_1}{w_0}\frac{w_2}{w_0} + \frac{w_1}{w_0}\frac{w_3}{w_0} + \frac{w_2}{w_0}\frac{w_3}{w_0})p_0^2 + (\frac{w_1}{w_0}\frac{w_2}{w_0}\frac{w_3}{w_0})p_0^3 = 1
\end{equation}
$p_i$ and $w_j$ in the above equation represent the probability of occurrence of state $i$ and the associated weight of the event $j$ respectively.
Our R package provides an \texttt{apply.SMCDM} function which automatically sets up this problem, solves this equation and finds the optimal option based on the non-imaginary root of this polynomial equation in the case where we only know the probability of occurrence of single independent events. We provide detailed instructions regarding the use of this function in the Example II section. 
\begin{figure}[ht]
    \centering
    \includegraphics[width=1\linewidth]{stratified_MCDM.jpg}
    \caption{\textbf{Flowchart demonstrating how the strata and states are structured in the SMCDM method with three events which occur independent of each other.}}
    \label{SMCDM-stratums}
\end{figure}
\subsection{Stratified Best Worst Method (SBWM)}
The SBWM method calculates optimal decision values for various alternatives across different states based on specified criteria. The method requires several inputs: a comparison matrix of alternatives and criteria, matrices for comparing criteria to the `worst' and `best' criteria in each state, lists specifying the worst and best criteria per state, and a vector of likelihoods indicating the probability of each state. The methodology involves iterating over each state and calculating the weight vector for each state's criteria. The resulting state-specific weights are compiled into a matrix and processed along with the likelihood vector following the same steps as the SMCDM method. The output is a numerical result representing the optimal decision values for the alternatives based on the weighted criteria across states \cite{torkayesh2021sustainable}.
\subsection{ShinyRMCDA}
To provide easier accessibility to these methodologies to users with no programming experience, we have developed a web-based application which receives a CSV file as input and provides tools for visualization and data analysis. To use the application please use the following link: \href{https://najafiannice.shinyapps.io/AHP_app/}{shinyRMCDA}. Upon visiting the web-based application, you will be guided through an instructions page where you can select a methodology from the drop-down menu. The instructions page would get automatically updated upon selection of a methodology. After running the analysis, the application outputs a set of plots and tables which can be easily downloaded from the web page (Figure \ref{shinyRMCDA-overview} provides an overview of the ShinyRMCDA web-based application). 
\begin{figure}
    \centering
    \includegraphics[width=1\linewidth]{shinyRMCDA_overview.jpg}
    \caption{\textbf{Overview of the ShinyRMCDA application.}}
    \label{shinyRMCDA-overview}
\end{figure}
\section{Illustrative examples}
\subsection{Installation}
\begin{enumerate}
\item Ensure you have \texttt{R} version $>$= 4.0 installed.
\item Install the required dependencies, e.g.:
\begin{lstlisting}
install.packages(c("devtools", "igraph", "lpSolve", "fmsb"))
\end{lstlisting}
\item Install \texttt{RMCDA} from GitHub:
\begin{lstlisting}
devtools::install_github("AnniceNajafi/RMCDA")
\end{lstlisting}
\end{enumerate}
\textbf{Note:} If you encounter an error that \texttt{devtools} or \texttt{igraph} is not available, please install them manually via \texttt{install.packages("devtools")} or \texttt{install.packages("igraph")}.
\subsection{Example I.} 
We are deciding to purchase a computer and we have compared the computers in a pairwise fashion based on three criteria; cost, user friendliness, and software availability. In addition, we have also compared the importance of the criteria to each other in a pairwise fashion. We have stored the results in a CSV file as shown in Figure \ref{CSV-input-AHP}. The CSV file should contain the inputs for the AHP method in a sequential order starting with the pairwise comparison of criteria followed by matrices related to pairwise comparison of alternatives for each criterion. The number of inputs in the CSV file would always be $n+1$ with $n$ being the number of criteria.
\begin{figure}
    \centering
    \includegraphics[width=0.9\linewidth]{pic_3.png}
    \caption{\textbf{Format of the input CSV file for AHP.}}
    \label{CSV-input-AHP}
\end{figure}
Our goal is to find out which computer we should buy. Below we demonstrate how we can utilize the RMCDA package to load the parameters stored in a CSV file and apply AHP and ANP to the data. 
\begin{lstlisting}
>library(RMCDA)
>data <- read.csv("AHP_example_computers.csv", header=FALSE)
>data.lst <- read.csv.AHP.matrices(data)
>A <- data.lst[[1]]
>comparing.competitors <- data.lst[[2]]
>AHP.result <- apply.AHP(A, comparing.competitors)
\end{lstlisting}
The function returns four outputs. The first output is the ratio of the consistency index to the random index where a value below $0.1$ indicates no significant inconsistencies found in data. The second and third outputs return the unweighted and weighted scores matrix respectively. Lastly, the fourth output returns the final AHP scores for the three alternatives. To visualize the decision tree corresponding to the AHP process, we can use the following code:
\begin{lstlisting}
>plot.AHP.decision.tree(A, comparing.competitors)
\end{lstlisting}
We can also use the \texttt{plot.spider} function to show the scores or weights of criteria as a radar plot. Figure \ref{AHPDecisionTree} A and B depict the AHP decision tree and the radar plot corresponding to the weighted matrix for criteria related to this example respectively (The same figure can be generated if the input CSV file is uploaded in ShinyRMCDA). 

To apply ANP on the data, we can utilize the \texttt{apply.ANP} function which will output the supermatrix of the ANP method. 

\begin{lstlisting}
>power = 3 #exponential for the super matrix. 
>apply.ANP(A, comparing.competitors, power)
\end{lstlisting}

\begin{figure}
    \centering
    \includegraphics[width=1.1\linewidth]{AHP_outputs.jpg}
    \caption{\textbf{Visualization outputs from the RMCDA package for AHP.}}
    \label{AHPDecisionTree}
\end{figure}
\subsection{Example II.}
Suppose we are buying a house and we have specific criteria based on which we weigh our options. At our current situation we have ranked our criteria in Table \ref{AlternativesSMCDM} below.
\begin{table}[h!]
\centering
\caption{\textbf{Scores of alternatives based on criteria.} }
\vspace{5mm}
\begin{tabular}{|c|c|c|c|}
\hline
\textbf{Alternatives} & \textbf{Quality} & \textbf{Price} & \textbf{Delivery} \\
\hline
A & 0.23 & 0.49 & 0.3 \\
\hline
B & 0.36 & 0.14 & 0.45 \\
\hline
C & 0.41 & 0.37 & 0.25 \\
\hline
\end{tabular}
\label{AlternativesSMCDM}
\end{table}
With probability $0.17$ we will stay in our current state and with probabilities $0.42, 0.17, 0.08$ events $A, B$, and $C$ may happen and we would switch from our current state (state 0) to one of seven states with states 1 to 3 representing the occurrence of only one event, states 4 to 6 representing the concurrent occurrence of two events (happening with probabilities $0.08, 0.05, 0.02$ respectively and state 7 representing the occurrence of all events together. We are aware that the importance of criteria in states are as shown in Table \ref{TableIII}. The goal is to apply SMCDM to our data and score each alternative based in this particular example involving uncertainty. 
\begin{table}[h!]
\centering
\caption{\textbf{Scores of alternatives based on criteria}}
\vspace{5mm}
\begin{tabular}{|c|c|c|c|c|c|c|c|c|}
\hline
\textbf{Criteria} & \textbf{State 0} & \textbf{State I} & \textbf{State II} & \textbf{State III} & \textbf{State IV} & \textbf{State V} & \textbf{State VI} & \textbf{State VII}\\
\hline
Quality & 0.2 & 0.21 & 0.52 & 0.23 & 0.32 & 0.1 & 0.29 & 0.12\\
\hline
Price & 0.4 & 0.21 & 0.11 & 0.38 & 0.05 & 0.22 & 0.03 & 0.02\\
\hline
Delivery & 0.4 & 0.58 & 0.37 & 0.39 & 0.63 & 0.68 & 0.68 & 0.86\\
\hline
\end{tabular}
\label{TableIII}
\end{table}
We can either define the variables manually or store them in a CSV file as shown in Figure \ref{SMCDM-CSV} and read and process them using the following code:
\begin{lstlisting}
>data <- read.csv("SMCDM_input.csv", header=FALSE)
>data.lst <- read.csv.SMCDM.matrices(data)
>comparison.mat <- data.lst[[1]] 
>state.criteria.probs<- data.lst[[2]] 
>likelihood.vector <- data.lst[[3]]
>apply.SMCDM(comparison.mat, state.criteria.probs, likelihood.vector, independent = FALSE)
\end{lstlisting}
The CSV file should contain the comparison matrix for the alternatives based on different criteria followed by the importance of different criteria in each state and the likelihood of staying in the baseline state and occurrence of each state. 
\begin{figure}
    \centering
    \includegraphics[width=0.5\linewidth]{pic_1.png}
    \caption{\textbf{The input CSV file for the SMCDM methodology.}}
    \label{SMCDM-CSV}
\end{figure}
In Figure \ref{fig:SMCDM-shiny-output} we show the output of the shinyRMCDA app for the same CSV file. Figure \ref{fig:SMCDM-shiny-output} A  shows the barplot corresponding to the scores of options based found through SMCDM, and B illustrates the heatmap for the importance of each criterion related to each alternative. C shows the decision tree with edge widths corresponding to the likelihood of transitioning to each state and the node size demonstrating the importance of the criteria in each state.
\begin{figure}
    \centering
    \includegraphics[width=1\linewidth]{SMCDM_suppliers_example.jpg}
    \caption{\textbf{Visualization outputs from the RMCDA package for SMCDM.}}
    \label{fig:SMCDM-shiny-output}
\end{figure}
\subsection{Example III.}

For this example we use the first case study data from Torkayesh et al \cite{torkayesh2021sustainable}. We have stored the inputs of the case study in a CSV file in the format shown in Figure \ref{Torkayesh-csv}. The first input is a table containing the importance of each criteria corresponding to each alternative. The table is then followed by two tables containing information related to the rankings of the worst criteria to others and the best criteria to others for every state respectively. We have then included two vectors indicating the worst and best criteria in each state respectively. Lastly, we have stored the likelihood of the occurrence of each event in the CSV file.
\begin{figure}
    \centering
    \includegraphics[width=0.9\linewidth]{pic_2.png}
    \caption{\textbf{Format of the CSV file containing data for the application of SBWM.}}
    \label{Torkayesh-csv}
\end{figure}

To apply SBWM we utilize the following code:
\begin{lstlisting}
>df.path <- read.csv(input$file$datapath, header=FALSE)
>df.lst <- read.csv.SBWM.matrices(df.path)
>comparison.mat <- df.lst[[1]]
>others.to.worst <-  df.lst[[2]]
>others.to.best <- df.lst[[3]]
>state.worst.lst <- df.lst[[4]]
>state.best.lst <- df.lst[[5]]
>likelihood.vector <- df.lst[[6]]
>SBWM.results <- SBWM(comparison.mat, others.to.worst, others.to.best, >state.worst.lst, state.best.lst, likelihood.vector)[,1]
\end{lstlisting}
%\textit{This is the main section of the article and reviewers will weight it appropriately.
%Please indicate:}
%\begin{itemize}
%    \item \textit{Any new research questions that can be pursued as a result of your software.}
%    \item \textit{In what way, and to what extent, your software improves the pursuit of existing research questions.}
%    \item \textit{Any ways in which your software has changed the daily practice of its users.}
%    \item \textit{How widespread the use of the software is within and outside the intended user group (downloads, number of users if your software is a service, citable publications, etc.).}
%    \item \textit{How the software is being used in commercial settings and/or how it has led to the creation of spin-off companies.}
%    \end{itemize}

\section{Algorithm Run Time}
We generated random inputs for the BWM, SMCDM, and SBWM functions, each time increasing the number of criteria and recorded the time it takes to run the functions. We then fitted polynomial equations to the curves which demonstrated that the time complexity of the BWM and SMCDM are both linear ($O(n)$) while the time complexity of the SBWM method is quadratic ($O(n^2)$) as shown in Figure \ref{figAlgorithmRunTimes} B. For the purpose of comparison, we have plotted the time complexity of running the AHP algorithm on data and have shown the results in Figure \ref{figAlgorithmRunTimes} A demonstrating that the time complexity of the SBWM algorithm with 3 independent events is similar to the AHP method.
\begin{figure}
    \centering
    \includegraphics[width=0.9\linewidth]{algorithm_run_times_RMCDA.png}
    \caption{\textbf{Time complexity of the BWM, SBWM, and SMCDM methods compared to the AHP method.}}
    \label{figAlgorithmRunTimes}
\end{figure}
\section{Impact}

Despite the increasing interest in Multi-Criteria Decision Analysis (MCDA), R packages offer only a limited number of MCDA methods, and their usage in R has been declining. We established RMCDA to fulfill the need for MCDA methods inside the R ecosystem by providing comprehensive coverage of popular MCDA methodologies. RMCDA integrates a wide array of classical, probabilistic, and stratified approaches into a single package, thereby exceeding the functionalities of current R packages and reviving R's status as a platform for MCDA applications. Additionally, by using the user-friendly online application ShinyRMCDA, we extend the accessibility of these methodologies for academics and practitioners with minimal programming expertise, hence promoting wider acceptance and implementation in decision-making scenarios. This program strengthens computational decision analysis and fits with the growing multidisciplinary dependence on MCDA approaches across many fields, including healthcare, finance, engineering, and environmental planning.
%{\textit{Please note that points 1 and 2 are best demonstrated by
%  references to citable publications.
%The RMCDA software significantly facilitates decision-making processes by providing researchers, analysts, and non-programming users access to sophisticated decision-making tools. In addition to providing a user-friendly web-based application, RMCDA offers previously unavailable methods in R thereby facilitating complex and multi-dimensional decision analyses important in domains characterized by significant uncertainty or intricate criteria, including finance, engineering, and healthcare.
\section{Conclusions}
Despite the growing demand for MCDA methodologies, current R packages offer limited implementations, resulting in a decrease in their utilization. On the other hand, Python has experienced an increase in MCDA-related libraries and their use, suggesting a demand for a strong alternative in R. Here, we introduced a comprehensive R package, RMCDA that offers a user-friendly MCDA framework. RMCDA encompasses a wide range of traditional, probabilistic, and stratified decision-making techniques, while also improving accessibility via its ShinyRMCDA web application, which is accessible to users with limited programming skills.

Although RMCDA offers comprehensive capabilities, some limitations remain. At present, it excludes fuzzy MCDA methods, which are essential for decision-making in conditions of uncertainty or imprecision. Future versions of RMCDA will integrate fuzzy logic techniques. Moreover, computational efficiency will be enhanced for large-scale applications, especially concerning high-dimensional datasets. Despite these constraints, RMCDA serves as decision-support tool for researchers and practitioners in various sectors, including healthcare, finance, engineering, and environmental planning who utilize the R programming language.



%Bibliography

%\bibliography{references}  
\printbibliography

\end{document}
