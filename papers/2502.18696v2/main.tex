%%%%%%%%%%%%%%%%%%%% author.tex %%%%%%%%%%%%%%%%%%%%%%%%%%%%%%%%%%%
%
% sample root file for your "contribution" to a proceedings volume
%
% Use this file as a template for your own input.
%
%%%%%%%%%%%%%%%% Springer %%%%%%%%%%%%%%%%%%%%%%%%%%%%%%%%%%


\documentclass{styles/svproc}
%
% RECOMMENDED %%%%%%%%%%%%%%%%%%%%%%%%%%%%%%%%%%%%%%%%%%%%%%%%%%%
%

% to typeset URLs, URIs, and DOIs
\usepackage{url}
\usepackage{hyperref}

\usepackage{graphicx}
\usepackage{amsmath}
\usepackage{mathtools}
\usepackage{multicol}
\usepackage{amssymb}
\usepackage{colortbl} % Include in the preamble for color support
\usepackage{xcolor}
\usepackage{cleveref}
\usepackage{tcolorbox}
\usepackage{tikz}     % For drawing curly brackets

% \newcommand\inlineeqno{\stepcounter{equation}\ (\theequation)}


\def\UrlFont{\rmfamily}

\begin{document}
\mainmatter              % start of a contribution
%
\title{Interpretable Data-Driven Ship Dynamics Model: Enhancing Physics-Based Motion Prediction with Parameter Optimization}
%
\titlerunning{Interpretable Data-Driven Ship Dynamics Model}  % abbreviated title (for running head)
%                                     also used for the TOC unless
%                                     \toctitle is used
%


\author{ Christos Papandreou \and  Michail Mathioudakis \and
 Theodoros Stouraitis \and  Petros Iatropoulos \and  Antonios Nikitakis \and
Stavros Paschalakis \and Konstantinos Kyriakopoulos}
%
\authorrunning{Papandreou et al.} % abbreviated author list (for running head)
%
%%%% list of authors for the TOC (use if author list has to be modified)
\tocauthor{Papandreou Christos, Mathioudakis Michail, Stouraitis Theodoros, Iatropoulos Petros, Nikitakis Antonios,
Stavros Paschalakis and Konstantinos Kyriakopoulos}
%
\institute{DeepSea Technologies, Greece\\
\email{c.papandreou@deepsea.ai},\\ 
%WWW 
home page:
\texttt{https://www.deepsea.ai/}
}

\maketitle              % typeset the title of the contribution

\begin{abstract} 
The deployment of autonomous navigation systems on ships necessitates accurate motion prediction models tailored to individual vessels. Traditional physics-based models, while grounded in hydrodynamic principles, often fail to account for ship-specific behaviors in real-world conditions. Conversely, purely data-driven models offer specificity but lack interpretability and robustness in edge cases. This study proposes a data-driven physics-based model that integrates physics-based equations with data-driven parameter optimization, leveraging the strengths of both approaches to ensure interpretability and adaptability.
The model incorporates physics-based components such as 3-DoF dynamics, rudder, and propeller forces, while parameters such as resistance curve and rudder coefficients are optimized using synthetic data. By embedding domain knowledge into the parameter optimization process, the fitted model maintains physical consistency.  
Validation of the approach is realized with two container ships by comparing, both qualitatively and quantitatively, predictions against ground-truth trajectories. The results demonstrate significant improvements, in predictive accuracy and reliability, of the data-driven physics-based models over baseline physics-based models tuned with traditional marine engineering practices. The fitted models capture ship-specific behaviors in diverse conditions with their predictions being, 51.6\% (ship A) and 57.8\% (ship B) more accurate, 72.36\% (ship A) and 89.67\% (ship B) more consistent.





% The deployment of autonomous navigation systems on ships requires tailored maneuvering and seakeeping models to accurately predict the motion of each particular ship. Motion models based on physical laws often fail to capture the specifics of how ships behave under actual operating conditions~(\cite{spyrou1996dynamic}). Purely data-driven models are tailored to specific ships, yet these black-box models lack interpretability and may exhibit unexpected behaviour in edge cases. Hence, a promising alternative is to develop a hybrid model that aims to leverage the advantages of both approaches. 

% The foundation of such models lies in physics-based equations governing the motion of a ship, while data are used to tailor the open parameters of the model which ensures its adaptability to various operating conditions~(\cite{suyama2024parameter}). While adapting the parameters of a model, a key challenge is to ensure the consistency of the fitted hybrid model, which is usually referred as  safeguarding. To this end, knowledge from similar ships has been utilized~(\cite{kanazawa2023bridging}). In this study we focus on obtaining interpretable data-driven models, which in turn can be safeguarded by prescribing marine engineers' domain knowledge into the fitting process.  

% Specifically, this work presents a hybrid maneuvering and seakeeping model that is composed out of physics-based components, such as 3-DoF dynamics, rudder and propeller, while seamlessly allowing the optimization of open parameters, such as resistance curve and rudder coefficients, based on synthetic and/or real-world data of a particular ship. A key contribution of our proposal is the use of open parameters that have been traditionally employed to tune maneuvering and seakeeping models according to naval principles. This allows us to obtain interpretable insights from the tuned hybrid models and safeguard them.

% We train our hybrid model using various datasets, which demonstrates the capabilities of the proposed model to capture the behaviour of different ships or conditions. We validate the tailored hybrid models by comparing their predictions' accuracy visually, numerically and statistically against ground-truth trajectories.
% The results demonstrate that our proposed approach enhances the accuracy of physics-based models, which in turn allows our models to closely replicate ship-specific behavior. 
\end{abstract}

\section{Introduction}
Motion prediction of marine vessels is essential for safe and efficient navigation of autonomous ships. Thus, it needs to be (i) reliable to minimize risks and enable precise maneuvers, \textit{e.g.} in-port and collision avoidance, and (ii) accurate to facilitate informed decision-making, route optimization, and economical propulsion under the influence of dynamic environmental conditions such as currents, waves, and wind. 

Motion prediction is typically built upon physics-based models that leverage well-established hydrodynamic principles to be reliable. Yet, these models often struggle to capture ship-specific behaviors influenced by hull design, propulsion dynamics, and operational conditions. Physics-based models may lack the specificity needed for unique vessel conditions, leading to inaccurate predictions. Conversely, purely data-driven models can be tailored to individual ships to predict its motion accurately, but are often treated as black-boxs that are challenging to validate in unforeseen conditions, hence they lack interpretability and reliability.

Combining these two approaches into a data-driven physics-based model (grey-box) can mitigate these challenges, as a physics-based structure with data-driven parameter optimization can maintain interpretability while ensuring specificity. The objective of this paper is to present a data-driven physics-based model that integrates physics principles with data-driven adaptation methods. Specific parameters of the physics-based model that are key to its accuracy and reliability are identified and are marked as unknown. Using a constraint non-linear least squares (cNLS) optimization method~\cite{stephen2022convex} the values of these parameters are fitted to improve the prediction accuracy of the motion prediction.
The resulting model achieves superior predictive accuracy, and improved reliability while preserving the interpretability of ship's modeling aspects across diverse operating conditions. The key contributions of this work are: (i) development of a data-driven physics-based model that maintains physical consistency while incorporating vessel-specific behavioral patterns, (ii) implementation of an optimization framework that successfully adapts physics parameters to vessel-specific characteristics utilizing vessel data, and (iii) validation with two vessels across diverse operational conditions, with spatial coverage.


%This research was implemented as a part of the development work with Nabtesco corporation for the "Nippon Foundation MEGURI2040 Fully Autonomous Ship Program" and is organized as follows: \cref{sec:related_Work} reviews and related work in ship motion modeling. \cref{sec:framework} describes our proposed framework, detailing its components and optimization. \cref{sec:results} presents the evaluation setup and results. Finally, \cref{sec:conclusion} concludes the study and, discusses future work.


The rest of the paper is organized as follows: \cref{sec:related_Work} reviews  related work in ship motion modeling. \cref{sec:framework} describes our proposed framework, detailing its components and optimization. \cref{sec:results} presents the evaluation setup and results. Finally, \cref{sec:conclusion} concludes the study and, discusses future work.  \vspace{-2mm}
% This paper is organized as follows: \cref{sec:related_Work} reviews related work in ship motion modeling. 
% % , from traditional physics-based approaches to recent data-driven and hybrid methods. 
% \cref{sec:framework} describes our proposed data-driven physics-based framework, detailing the physics-based components, open parameters, and optimization process. \cref{sec:results} presents the evaluation setup, including datasets, evaluation metrics, and testing conditions, followed by analysis of results. Finally, \cref{sec:conclusion} discusses the implications of our findings, concludes the paper, and outlines directions for future work.

\section{Related Work}
\label{sec:related_Work}

\textbf{Physics-based (white-box) models}:
Research on maneuvering models began with~\cite{davidson1946turning}. Later,\cite{motora1959measurement},\cite{motora1955course}, and~\cite{nomoto1957steering} contributed to hydrodynamic aspects and stability principles. \cite{abkowitz1964ship} and~\cite{aastrom1976identification} used Taylor expansion for maneuvering models. The MMG model that is widely used in autonomous navigation was introduced by~\cite{ogawa1977mmg}, was improved by~\cite{yasukawa2015introduction} and has been applied by~\cite{li2013active} and~\cite{zhang2017ship}. Other models for low-speed hydrodynamic forces include the HD model~\cite{yasukawa2015introduction}, the CD model~\cite{yoshimura2012hydrodynamic}, and the TBL model~\cite{sutulo2015development}.

\textbf{Black box models}: Recently machine Learning approaches such as   artificial neural networks~\cite{moreira2003dynamic,rajesh2008system,zhang2013black,oskin2013neural,luo2016modeling,woo2018dynamic} and Gaussian processes~\cite{arizaramirez2018nonparametric,xue2020system} have been explored. \cite{SILVA2022103222} uses deep neural networks (DNN) for 6-DoF motion prediction in waves to demonstrate how DNNs can capture nonlinear dynamics between the ship state and wave conditions while maintaining good generalization.

\textbf{Grey box models}:
An other research avenue has focused on developing parameter estimation methods to combine traditional physics-based models with data-driven adaptation.
\cite{suyama2024parameter,MiyauchiCMAES} employed a gradient-free optimization method (namely CMA-ES) to tune parameters of an MMG model~\cite{yasukawa2015introduction} using sea-trial data of a real-scale ship. 
\cite{kanazawa2023bridging} proposed a framework that 
fits open parameters of an Abkowitz model~\cite{abkowitz1964ship} to data of the target ship while using a well-validated model of a similar ship to regularize the fitting process. In this way, the resulting predictions are on par with the validated model while they are tailored to the target ship's behavioral patterns. 
Although, improved prediction accuracy has been demonstrated, depending on the type of fitted parameters reliability to the model can be lost, if the parameters are not interpretable.
Unlike these approaches, the data-driven physics-based model introduced in this paper, which is based on the white-box model~\cite{mathioudakis2025realworldvalidationphysicsbasedship}, focuses on optimizing interpretable parameters, making the results more accessible and understandable to domain experts.   \vspace{-4mm}


% This enables a direct relation between the model's output and the underlying physical and operational characteristics of the ship. While focusing parameter estimation on interpretable parameters might seem advantageous, it can lead to reduced reliability of the resulting model if the constraints imposed by the physics-based framework are not properly maintained.
% \textcolor{red}{Here, we need to describe how our approach differs to the above, for example depending on the open parameters interpretability can be lost and the fact that our physics-based model is Spyrou-based which implies .... also, here describe that if parameter estimation focuses on non interpretable parameters then the resulting model is not reliable. }



% bridges the dynamics of similar ships by using parameters from a well-validated model as a knowledge foundation for developing models of new ships. 
% In parallel, work has been done on parameter fine-tuning methods for existing models.

% to simultaneously adjust multiple parameters while maintaining realistic values, 
% ~\cite{MiyauchiCMAES}
\begin{figure}[t]
    \includegraphics[width=1\linewidth]{simulation_flow.pdf} 
    \vspace{-8mm}
    \caption{Computational flow of a physics-based vessel motion prediction model, with the following key blocks: (a) ``Control": input commands of the rudder and the propeller, (b) ``Environment": environmental effects from the wind, waves, and sea currents, (c) ``Force calculation": computation of forced using the outputs of (a) and (b) along with hydrodynamic forces and the current ship state, and (d) ``Equations of motion": dynamics equations and integration over time to produce the next ship state.}
    \label{fig:simulation_system}
    \vspace{-3mm}
\end{figure}


\section{Prediction Model and Method}
\label{sec:framework}
In this section, first we briefly describe the motion prediction process, the primary aspects of the physics-based model, and the respective notation. Second, we introduce the open parameters of the physics-based model that are selected for fitting, and third we provide the details on the cNLS fitting method.  \vspace{-1mm}


\subsection{Motion Prediction and Model}
\label{subsec:model}

The state of the vessel has 3-DoF and it is denoted as $\mathbf{s} = [x, y, \psi, u, v, r, n, \delta]^T$, where $x$ and $y$ are the vessel cartesian coordinates, $\psi$ is the heading, $u$, $v$, $r$ are the surge, sway and angular velocities, $n$ is the propeller rpm and $\delta$ is the rudder angle. The $q$ inputs to the system are denoted with $\mathbf{u} \in \mathbb{R}^q$ and they comprise of: (i) control commands for the propeller $c_n$ and the rudder $c_\delta$, and (ii) environmental factors denoted with $e$, such as wind, waves, and currents. 

As shown in~\cref{fig:simulation_system}, given the current state of the ship $\mathbf{s}_k$, and a sequence of inputs $\mathbf{u}_k, \forall k \in \{0,..,K-1\}$, the motion prediction iteratively generates a trajectory $\xi = \{\mathbf{s}_0, ..., \mathbf{s}_{K}\}$ that is sequence of $K$ (number of knots) vessel states. The trajectory generation process is recursive and it involves at each timestep the calculation of: (a) the actuation forces from propeller and rudder, and (b) the environmental forces from wind, waves, and currents, and (c) the numerical integration (explicit Euler) of the motion equations from $\mathbf{s}_k$ to $\mathbf{s}_{k+1}$, as per~\cite{mathioudakis2025realworldvalidationphysicsbasedship}.  

 
For the equations of motion (dynamics), a 3-DoF nonlinear ODE from~\cite{spyrou1996dynamic} is used. This is an HD model (see~\cref{sec:related_Work}) described by~\cref{eq:dynamics}.
\begin{equation}
    \label{eq:dynamics}
    \begin{split}
        (m - X_{\dot{u}})  \dot{u} +  (Y_{\dot{v}} -X_{vr} -m) vr + (Y_{\dot{r}} - m x_G) r^2 \\
        (m - Y_{\dot{v}}) \dot{v} + (m x_G -  Y_{\dot{r}})\dot{r} + (-Y_{v}U v + (mu - Y_rU)r - \hspace{1cm}\\ Y_{vv} v |v| - Y_{vr} v |r| - Y_{rr} r |r|) \\
        (m x_G - N_{\dot{v}}) \dot{v} + (I_z - N_{\dot{r}})\dot{r} - N_v U v 
         + (m x_G u - N_rU) r - \hspace{1cm}\\ 
         N_{rr} r |r| - N_{rrv} \frac{rrv}{U} - N_{vvr} \frac{vvr}{U} 
    \end{split}
    \begin{split}
        = \Sigma X \\
        \\
        = \Sigma Y  \\
        \\
        = \Sigma N 
    \end{split}
\end{equation} 
In~\cref{eq:dynamics}, m is the mass of the ship, $I_{zz}$ is the yaw moment of inertia and $x_G$ is the longitudinal distance of ship’s centre of gravity from the moving axes’ origin, $O_s$. $X$, $Y$ and $N$ terms with velocities $u$, $v$, $r$ and accelerations $\dot{u},\ \dot{v},\ \dot{r}$ subscripts are the maneuvering coefficients of the ship per dimension, which capture information regarding the hydrodynamic effects based on its geometric characteristics. In general, the left-hand-side (LHS) of~\eqref{eq:dynamics} incorporates terms that express the added mass, damping and restoring forces phenomena. Parameters on the LHS are assumed to be accurate as the hydrodynamic coefficients are calculated for each ship according to~\cite{Clarke1982TheAO} and~\cite{inoue1981hydrodynamic}. 

The right-hand-side (RHS) of~\cref{eq:dynamics} collects all the external forces and moments acting on the ship, which are expanded in~\cref{eq:forces}. 
\begin{equation}
\label{eq:forces}
    \begin{split}
        \Sigma X &= X_{n} + X_\delta + X_{e} + R(u) \\
        \Sigma Y &= Y_{n} + Y_\delta + Y_{e} \\
        \Sigma N &= N_{n} + N_\delta + N_{e}
    \end{split} 
\end{equation}
For each force dimension, the subscripts denote its source and $R(u) \in \mathbb{R}$ denotes the the resistance of the vessel in Newton. In this work, we study the effect of fitting key parameters on the RHS of~\cref{eq:dynamics}, hence parameters of~\cref{eq:forces}. 

% will be considered adaptable and freed  following section 

% the propeller thrust $X_{thr}$, 

% the wind $[X_{wind},  Y_{wind},  N_{wind}]^T$ 
% and the waves forces $[X_{wave},  Y_{wave},  N_{wave}]^T$. 

% In addition, the resistance of the vessel $R(u$) is also included in the RHS as part of $\Sigma X$. 

% All the parameters from the LHS of the equation and some on the RHS are considered correct, while 



\subsection{Key Parameters}
\label{subsec:open_parameters}
Considering that the hybrid model's adaptability and interpretability stems from carefully selecting the parameters to be fitted (see grey box models in~\cref{sec:related_Work}), we identified eleven key parameters. These parameters are chosen to enhance three pivotal aspects of ship behavior while maintaining physical significance. The three aspects of ship behavior follow.

\textbf{Propeller thrust}: For the propeller's thrust $X_n$ we use a fully defined propulsion model~\cite{mathioudakis2025realworldvalidationphysicsbasedship}, and we estimate the lateral force $Y_n$ generated by the propeller according to $ Y_{n} = p_{0} X_{n} $.
% ~\cref{eq:lateral_thrust}. 
% \begin{equation}
% \label{eq:lateral_thrust}
%    $ Y_{n} = p_{0} X_{n}  \inlineeqno$
% \end{equation}
In this way, a single parameter relates the two dimensions of the thrust linearly, while the moment $N_n$ is calculated geometrically based on the propeller's position and $Y_n$.
Parameter $p_{0}$ models course instabilities and captures the lateral force produced by the propeller.
 
\textbf{Vessel resistance}: The resistance is a $3_{rd}$ order polynomial curve, described as $ R(u) = p_1 u + p_2 u^2 +p_3 u^3 $,
% ~\cref{eq:resistance_curve}, 
characterizes the ship's hull resistance through water. Obtaining a resistance curve %systematically, which is also 
that is appropriately balanced with the trust $X_n$, enhances the model's calculation accuracy of the ship's surge speed $u$. 
% \begin{equation}
% \label{eq:resistance_curve}
%     R(u) = p_1 u + p_2 u^2 +p_3 u^3
% \end{equation}
The $0_{th}$ order coefficient is set to zero, because the water resistance to a non-moving vessel is zero. 
Therefore, three of the parameters $\{p_1, p_2, p_{3}\}$ are the coefficients of the resistance curve. 
Typically the resistance curve is $2_{nd}$ degree polynomial. Here, a $3_{rd}$ order was used for improved expressivity.

\textbf{Rudder forces}: The rudder forces shown in~\cref{eq:forces} are associated with the rudder lift $c_L$ and drag $c_D$ coefficients.
These are modeled as 
% $3_{rd}$ and $2_{nd}$ degree polynomials, 
$ c_L(a_r) = p_4 + p_5 a_r + p_6 a_r^2 + p_7 a_r^3 $ and $ c_D(a_r) = p_8 + p_9 a_r + p_{10} a_r^2 $, respectively. 
% as shown in~\cref{eq:rudder_coefficients}. 
The polynomials degrees were selected to be as close as possible to their experimental measurements against rudder inflow angles $a_r$.
    % \begin{equation}
    % \label{eq:rudder_coefficients}
    % \begin{split}
    %     c_L(a_r) &= p_4 + p_5 a_r + p_6 a_r^2 + p_7 a_r^3 \\
    %     c_D(a_r) &= p_8 + p_9 a_r + p_{10} a_r^2
    % \end{split}
    % \end{equation}
Lift and drag coefficients are the key factors involved in ship's turning abilities, hence seven $\{p_4, ..., p_{10}\}$ of the parameters are the polynomials coefficients of $c_L(\cdot)$ and $c_D(\cdot)$. 


By targeting these specific elements, the model can be fine-tuned to capture ship-specific characteristics and operating conditions, enhancing its predictive accuracy while preserving the interpretability of the underlying physics-based model. The selected parameters affect the motion in all 3-DoF and capture typical sea trials, such as speed, turning and course instability tests. \vspace{-1mm}

\subsection{Parameter Optimization}
\label{subsec:param_opt}
To fit the parameters mentioned above to data of a specific vessel, we solve the following optimization problem, described by~\cref{eq:fit_opt}, where $M$ is the number of trajectories in the dataset. \vspace{-3mm}
\begin{equation}
\label{eq:fit_opt}
\begin{aligned}
& \underset{\mathbf{p}}{\text{min}}
& & \frac{1}{M} \sum_{i=0}^{M} \varepsilon(\xi_i, \bar{\xi}_i) \\
& ~ \text{s. t.}
& & \mathbf{b_l} \leq \mathbf{p} \leq \mathbf{b_u}, \\
& & & g(\mathbf{p}) \leq b_j, \; j = 1, \ldots, J
\end{aligned}
\end{equation}
This is a constraint nonlinear least squares (cNLS) problem, where $\varepsilon(\xi_i, \bar{\xi}_i)$ denotes the squared deviation between the predicted trajectory $\xi_i$ and the measured trajectory $\bar{\xi}_i$ over all timesteps. This is defined as \vspace{-1mm}
\begin{equation}
\label{eq:fit_cost}
    \varepsilon(\xi_i, \bar{\xi}_i) =  %\frac{1}{K}\sum_{k=1}^{K}
    (\bar{\xi}_i - \xi_i)^T \mathbf{W} (\bar{\xi}_i - \xi_i) \vspace{-1mm}
\end{equation}
\noindent with $\mathbf{W} \in \mathbb{R}^{8\times8}$ being the diagonal weight matrix that balances the contribution of each dimension of the state.
$\mathbf{p} = \left[p_0, p_1, p_2, p_3, p_4, p_5, p_6, p_7, p_8, p_9, p_{10} \right]^T$ is the optimization variable, which is a vector of the key parameters described in~\cref{subsec:open_parameters}. $\mathbf{b_l}$ and $\mathbf{b_u}$ are the lower and upper bound vectors of parameters $\mathbf{p}$ that limit the range of possible values. $g(\mathbf{p})$ are non-linear functions of $\mathbf{p}$, which are constrained by $b_j$ to incorporate domain knowledge in the parameter optimization. The optimization problem is solved numerically using an interior-point gradient-based optimization method. This approach ensures that the estimated parameters minimize the weighted sum of squared deviations between measured and predicted trajectories, while adhering to the specified parameter bounds and additional constraints that prescribe the motion of the vessel.  \vspace{-1mm}



\section{Evaluation \& Results}
\label{sec:results}
\vspace{-1mm}

In this section, we validate the proposed parameter optimization approach and evaluate the resulting models in terms of their accuracy and reliability. 
The validation of our approach, involves fitting the data-driven physics-based model into data from two container ships (target ships), \textit{i.e.} given a dataset from one of the container ships the key parameters (see ~\cref{subsec:open_parameters}) of the data-driven physics-based model (see ~\cref{subsec:model}) are fitted by solving the cNLS optimization problem (see ~\cref{subsec:param_opt}). The evaluation of the fitted models involves comparing their accuracy and reliability against respective baseline physics-based models both numerically and visually. We consider two container ships of different sizes, containership A with 1750 DWT (approximately 85 meters in length) and the containership B with 8000 DWT (around 130 meters in length), to assess whether the parameter optimization process can: (a) converge to parameter values that are interpretable, physically meaningful, and (b) enable the fitted models predict motions across different ship configurations. Next, we describe the validation setup, including the details on: (i) the datasets used, (ii) the parameter optimization, (iii) the evaluation protocol, and (iv) the distance measures used. 
% to compare the resulting trajectories.   

% \subsection{Validation Setup}
% \vspace{-1mm}

\textbf{Datasets}:
In our validation setup, we use synthetic datasets for both container ships. The maneuvers are based on realistic routes and speeds, focusing on port approaches and departures, with fewer open-sea scenarios. 
In-port maneuvers involve large drift angles which challenge most physics-based models. Fitting $\mathbf{p}$ to these maneuvers, we expect strong performance in simpler scenarios. Each dataset is generated using a distinct MMG-based simulator~\cite{yasukawa2015introduction,okuda2023maneuvering}, enabling evaluation of the optimization process's robustness to simulator-specific biases. These synthetic datasets, covering diverse vessel routes, allow comprehensive analysis under various conditions and support flexible maneuver testing.

The statistical summary of the datasets, shown in~\cref{table:errors}, reveals important characteristics that support the diversity of scenarios, \textit{e.g.} extensive range of $u$ and $n$. All trajectories have 120 knots and in terms of the datasets size (number of trajectories), for ship A, $M=165$ and for ship B, $M=100$, while the train-test split is conducted randomly to avoid any potential bias towards maneuvering or sea-keeping. The test scenarios for each vessel represent approximately the 12.5 \% of the trajectories. Next, we describe prominent aspects of the datasets.
\begin{table}[t]
\caption{Statistics of the datasets. The physical units of each dimension are as follows: $u$ and $v$ in $m/s$, $r$ in $rad/s$, $n$ is a real number, and $\delta$ in degrees. }
\centering
~~~~~~~~~~~~~~~~~~~\:\,
\begin{tabular}{|c||c|}
\hline
~~~~~~~~~~~~~~~~Train~~~~~~~~~~~~~~ & ~~~~~~~~~~~~~~Test~~~~~~~~~~~~~~~~\,\\
\end{tabular}\\
\parbox[t][4em][t]{4em}{\centering \vspace{-0.8cm} \textbf{Ship A \\ \vspace{1.2cm} Ship B}} % Parbox placed next to the table
\hspace{0.1cm} % Horizontal space between table and the parbox
\begin{tabular}{|c|c|c|c|c|c||c|c|c|c|c|c|}
\hline
 & $u$ & $v$  & $r$  & $n$ & $\delta$  & $u$  & $v$  & $r$  & $n$ & $\delta$ \\ \hline
Min & -4.47 & -2.05 & -0.03 & -249.1 & -32.5 & 1.47 & -1.74 & -0.03 & -129.4 & -26.9 \\ \hline
Max & 8.23 & 1.76 & 0.03 & 249.4 &  33.5 & 7.38 & 1.76 & 0.03 & 245.9 & 27.9 \\ \hline
Mean & 3.27 & -0.02 & 0 & 95.0 &  1.2 & 4 & 0.09 & 0 & 129.1 & -0.04 \\ \hline
\multicolumn{11}{|c|}{\cellcolor{gray!10}{}} \\ 
\hline
Min & -0.47 & -1.04 & -0.02 & -158.8 & -40.0 & 0.1 & -0.57 & -0.01 & -98.9 & -39.5 \\ \hline
Max & 6.17 & 1.03 & 0.02 & 169.7 & 40.0 & 5.45 & 0.54 & 0.01 & 168.8 & 32.6 \\ \hline
Mean & 3.83 & 0.04 & 0 & 93.3 & -1.0 & 3.71 & 0.11 & 0 & 88.8 & -0.8  \\ \hline
\end{tabular}
\label{table:errors}
\vspace{-4mm}
\end{table}

% In our validation setup, we utilize synthetic datasets for both container ships. The maneuvers considered in this study are designed based on realistic vessel routes and speeds, an in particular those executed when approaching or leaving from ports. Open sea navigation maneuvers are also present in the considered datasets, yet they occur less frequently in comparison to in-port maneuvers. In-port maneuvers are often more complex than standard  open-sea scenarios, as they involve large drift angles in which most marine physics-based models fall short. Consequently, we posit that by fitting our models to effectively capture to these demanding maneuvers, they will also perform satisfactory in less complex scenarios. 
 
% The two datasets (one for each target ship) are generated using two similar but distinct  MMG-based simulators~\cite{yasukawa2015introduction,okuda2023maneuvering}. 
% By employing different simulators for each ship, allows us to evaluate the generalization  and robustness of the parameter optimization process to potential biases or peculiarities inherent in individual simulation tools. This approach not only provides diverse data sources but also allows us to assess the fitted models' flexibility to capture diverse simulation environments. In addition, the datasets are generated while the vessels navigate diverse routes. This is a benefit of synthetic datasets (simulation-based) as they allow performing any type of maneuvers which can be used to facilitates a comprehensive analysis of the model's performance under different conditions.



\noindent
\underline{Speed Characteristics}: 
As we can see in~\cref{table:errors} the datasets cover a large portion of the operational conditions the vessels may encounter during their lifetime. 

\noindent
\underline{Maneuvering Characteristics}: 
Sway speeds, and yaw rate of turn indicates inclusion of broad maneuvering behaviors. 

\noindent
\underline{Propulsion Characteristics}: The range $n$ differs between vessels, yet it is consistent between the respective training and test sets. Although the vessels have different propulsion system characteristics (different MCR) the range of $n$ indicates that almost all operational conditions are present in the datasets.
% These statistical differences between vessels, while maintaining physical consistency within each vessel's operational envelope, are crucial for our hybrid model's development.


% simply memorizing specific patterns. 

% This diversity in the training and test datasets aims to supports our goal of developing 
% % a generalizable models that can handle various vessel types and 
% models able to handle various operating conditions while maintaining physical consistency.


% In summary, the slight differences between train and test sets for each ship, combined with the broader differences between vessels, provide ideal validation setup, as this allows us to investigate the ability of the proposed approach to capture vessel-specific behaviors that generalize across ships. 
In summary, the differences within each ship's data and across vessels create an ideal validation setup to assess the approach's ability to capture vessel-specific behaviors that generalize.

\textbf{Parameter Optimization}:
For the cost function of~\cref{eq:fit_cost}, we select the hyperparameter $\mathbf{W} = diag(\frac{1}{\bar{L}_i}, \frac{1}{\bar{L}_i}, \frac{1}{\pi}, \frac{1}{\bar{U}_i}, \frac{1}{\bar{U}_i}, \frac{1}{max(r_i)}, 0, 0)$ as we  aim to minimize the discrepancies of the predicted trajectory $\xi_i$ and the ground-truth trajectory $\bar{\xi}_i$ in terms of pose and velocities dimensions. $n$ and $\delta$ are assumed to be identical in $\xi_i$ and $\bar{\xi}_i$. The denominators in $\mathbf{W}$ rescale and nondimensionalize the different dimensions. $\bar{L}_i$ is the Cartesian length of $\bar{\xi}_i$, $\bar{U}_i$ is the average speed of $\bar{\xi}_i$ and $max(r_i)$ the maximum angular velocity set to be 0.0314 rad/s and the 
initial guess of parameter vector for ship A is $[34187.03, -12569.98, 1586.29, 0.00, $ $ -0.02, 2.70, 0.42, -3.23, 0.06, -0.11, 1.97]^T$ and for ship B is $[5504.65, -4218.06, $ $ 1063.42, -0.05, -0.01, 2.73, 0.35, -3.10, 0.06, -0.11, 1.97]^T$.

One could solve~\cref{eq:fit_opt} without using any constraints (non-linear least squares), yet parameters' physical consistency may be compromised, making the fitted model less interpretable.
% the physical consistency of the parameters could be jeopardized and the resulting fitted model main not be interpretable. 
Thus, we incorporate constraints $g(\cdot)$ that are functions of $\mathbf{p}$ and bounds $b_j$ both derived from marine engineering domain expertise. 
These are: (i) $0 \leq R(u) \leq 80000 $, (ii) $ \dot{R}(u) > 0 $, (iii) $ -0.05 X_{\delta} \leq Y_{\delta} \leq 0.05 X_{\delta}$, (iv) $ -1.0 \leq c_L(a_r) \leq 0, \forall a_r < 0$ and $ 0 \leq c_L(a_r) \leq 1.0, \forall a_r \geq 0$, (v) $ 0 \leq c_D(a_r) \leq 1.5$ and $c_D(0) \leq 0.05$,
while all elements of $\mathbf{b_l}$ are set to \texttt{-inf} and for $\mathbf{b_u}$ to \texttt{inf}.  

\textbf{Evaluation Protocol}:
We compare three distinct trajectories for each ship: stemming from ground truth $\bar{\xi}_i$, baseline physics-based $\tilde{\xi}_i$, and our fitted model $\xi_i^*$. To do that, for each ship we quantify the distance between
(i) the ground truth and the baseline trajectories, (ii) the ground truth and fitted model trajectory, and report the relative improvement from (i) to (ii). 


\textbf{Distance Measures}:
To quantify the distance between two trajectories, we use a Manhattan Distance (MD) per dimension,
and a custom Vessel Distance Measure (cVDM) that rescales and nondimensionalizes the different dimensions, both given in~\cref{eqref:MD,eqref:cVDM}. Notation-wise see~\cref{sec:framework}. For a study on different measures and their utility see~\cite{mathioudakis2025realworldvalidationphysicsbasedship}, where we show the aptness of cVDM (over other measures, such as MD) to quantify deviations between trajectories. Also, note that given the selected $\mathbf{W}$ discussed above, \cref{eq:fit_cost} is the differentiable equivalent to cVDM.  To evaluate~\cref{eqref:MD,eqref:cVDM} for $\tilde{\xi}_i$ and $\xi_i^*$, we replace the variables accordingly, \textit{e.g.} $\tilde{\xi}_i$, $\mathbf{s}_{d,k}$ with $\tilde{\mathbf{s}}_{d,k}$, and $\xi_i^*$, $\mathbf{s}_{d,k}$ with $\mathbf{s}^*_{d,k}$.
\vspace{-2mm}
% \begin{tcolorbox}[float*=t, colback=white, colframe=black, width=\textwidth, boxrule=0.25mm]
\begin{align}
\label{eqref:MD}
     & \hspace{3cm} MD = \frac{1}{K} \sum_{k=1}^{K} \left|\bar{\mathbf{s}}_{d,k} - \mathbf{s}_{d,k} \right|
\end{align}
\vspace{-3mm}
\begin{align}
\label{eqref:cVDM}
    cVDM &= \frac{100}{K} \sum_{k=1}^{K} \bigg(
        \frac{\left| \bar{x}_{k} - x_{k} \right|}{\bar{L}}
        + \frac{\left| \bar{y}_{k} - y_{k} \right|}{\bar{L}} + \frac{\left| \bar{\psi}_{k} - \psi_{k} \right|}{\pi} + \notag \\
        &\quad \quad \quad \quad \quad ~
        \frac{\left| \bar{u}_{k} - u_{k} \right|}{\bar{U}}
        + \frac{\left| \bar{v}_{k} - v_{k} \right|}{\bar{U}}
        + \frac{\left| \bar{r}_{k} - r_{k} \right|}{r_{max}} \bigg)
\end{align}
\vspace{-5mm}
% \end{tcolorbox}





%%%%%%%%% CONTINUE FROM HERE


\subsection*{Results:}
\vspace{-1mm}

% Petros graph

Here, we report the fitted parameters for both target ships based on the respective datasets and discuss their interpretation. Next, we present the overall performance following the evaluation protocol and inspect qualitatively and quantitatively individual scenarios.     
 
\textbf{Fitted parameters}: 
\Cref{table:coeffs} includes the parameters used for the baseline models of each ship and the ones of the fitted models, denoted with $\mathbf{\tilde{p}}$ and $\mathbf{p^*}$, respectively. $\mathbf{\tilde{p}}$ are calculated based on captive test data. To inspect the different sets of parameters, we plot (see~\cref{fig:fitted_curves}) six curves for each ship, two for $R(u)$, two for $c_L$ and two for $c_D$. All curves follow marine engineers expectations, yet they are significantly different to the ones of the baselines. $c_L$ demonstrates a S-shape and $c_D$ a parabolic pattern, while both maintain the asymmetry which is also observed in the baseline curves. The curves suggest that the fitted models still obey the underlying physics, hence the models remain interpretable. 

\textbf{Overall performance}: 
We obtain a relative improvement based on MD, referred to as $mARI$\footnote{For 6 dimensions and $M$ trajectories $MD \in \mathrm{R}^{6 \times M}$. Average relative improvement for each trajectory 
is the set $\{ \alpha_0, ..., \alpha_M \}$ and $mARI$ is the median of the set.} and based on $cVDM$. For ship A, $mARI$ is 32.1\% and $cVDM$ is 51.6\%, while for ship B $mARI$ is 42.9\% and $cVDM$ is 57.8\%. These percentages indicate how much closer to the groundtruth are the trajectories of the fitted models compared to the baselines, while the predictions are 72.36\% 
\begin{table}[!h]
\vspace{-2mm}
\caption{Parameters for both vessels. $\mathbf{\tilde{p}}$ denotes parameters of the baseline models while $\mathbf{p^*}$ denotes parameters of the fitted models.}
\vspace{-1mm}
\centering
~~~~~~~~\:
\begin{tabular}{|c|c|c|c|}
\hline
 ~~$Y_n$~~\, & ~~~~~~~~$R(u)$~~~~~~~~~\: & ~~~~~~~~~~~$c_L(a_r)$~~~~~~~~~~~~~~ & ~~~~~~~~$c_D(a_r)$~~~~~~~~~\\
\end{tabular}\\
\parbox[t][4em][t]{0em}{\centering \textbf{\vspace{-1.4cm} \footnotesize{Ship} \\ \vspace{0.2cm} ~~ A \\ \vspace{0.9cm} ~~ B}} % Parbox placed next to the table
\hspace{0.6cm} % Horizontal space between table and the parbox
\begin{tabular}{|c|c|c|c|c|c|c|c|c|c|c|c|}
\hline
  & $p_0$ & $p_1$ & $p_2$ & $p_3$ & $p_4$ & $p_5$ & $p_6$ & $p_7$ & $p_8$ & $p_9$ & $p_{10}$ \\ \hline
$\mathbf{\tilde{p}}$ & 0.000 & 10500 & -1900  & 346 & -0.012 & 0.864 & 0.182 & -1.191 & 0.005 & -0.1230 & 0.779 \\ \hline
$\mathbf{p^*}$  & -0.017 & 34187 & -12568 & 1594 & -0.039 & 3.193 & 0.205 & -4.882 & 0.048 & 0.0746 & 1.370 \\ \hline
\multicolumn{12}{|c|}{\cellcolor{gray!10}{}} \\ \hline
$\mathbf{\tilde{p}}$  & 0.000 & 5669 & -1127 & 538 & -0.012 & 0.864 & 0.182 & -1.191 & 0.005 & -0.1230 & 0.779 \\ \hline
$\mathbf{p^*}$ & -0.027 & 5505 & -4215 & 1077 & 0.012 & 2.420 & -0.019 & -3.195 & 0.047 & 0.0001 & 1.359 \\ \hline
\end{tabular}
\label{table:coeffs}
\vspace{-1mm}
\end{table}
and 89.67\% more consistent\footnote{The variance of $cVDM$ of the fitted model is x\% lower than the one of the baseline.}, for ships A and B, respectively.

\begin{figure}[t]
    \centering
    \includegraphics[width=0.99\linewidth]{figures/Coefs.pdf}
    \vspace{-2mm}
    \caption{Fitted ($\mathbf{p^*}$) vs baseline ($\mathbf{\tilde{p}}$) model curves for resistance, lift ($c_L$), and drag ($c_D$) coefficients for both vessels. Note that both ships have the same rudder type.} 
    \label{fig:fitted_curves}
    \vspace{-4mm}
\end{figure}  

\textbf{Distinct maneuver scenarios}:
Here, we discuss four distinct maneuver scenarios (two from ship A and two from ship B). Please note that the chosen test trajectories presented here are not just the most optimal but the most instructive ones. First category (see~\cref{fig:Ship_A_good,fig:Ship_B_good}) of scenarios show remarkable improvement of cVDM = 76.0\% for ship A and cVDM = 69.2 \% for ship B, which is also verified by visual inspection. Second category (see~\cref{fig:Ship_A_moderate,fig:Ship_B_moderate}) of scenarios where the fitted models shows moderate improvement over the baseline. Ship A has cVDM = 28.0\% and ship B has cVDM = 22.5\%. The latter, indicate existence of unmodelled phenomena that we plan to study in future work. 

\textbf{Cross-vessel takeaways}:
The fitted models' performance across both vessels reveals several key insights: (i) maintain physical consistency and interpretability, 
\begin{figure}[h]
    \centering
    \includegraphics[width=0.99\linewidth]{figures/ship_A_good_JPTSU-JM02C-ae3_3.rtz_Waypoints_W-0_T-0_A-0.pdf}
     \vspace{-2mm}
    \caption{Trajectory comparison for ship A with excellent accuracy improvement. Generated trajectory: $\bar{\xi}$ from the groundtruth data, $\tilde{\xi}$ from the baseline model and $\xi^*$ from the fitted model.   MD$(\%)$ per dimension is: $x:90.6$, $y:56.4$, $\psi:86.9$, $u:13.0$, $v:13.1$, $r:52.2$, and cVDM$(\%)$ is 76.0.}
    \label{fig:Ship_A_good}
    \vspace{-3mm}
\end{figure}  
(ii) consistent improvement in terms of prediction accuracy and consistency over the baselines regardless of the ship, and (iii) reliable performance across different operating conditions and speed ranges. \vspace{-2mm}

% While the overall trajectory prediction is better than the baseline, there is over-sensitivity, which forces the vessel to turn more rapidly than expected. These are highly instructive examples,



\begin{figure}[t]
    \centering
    \includegraphics[width=0.99\linewidth]{figures/ship_B_good_output_1_380.pdf}    \vspace{-2mm}
    \caption{Trajectory comparison for ship B with remarkable  accuracy improvement. Generated trajectories same as in~\cref{fig:Ship_A_good}.   MD$(\%)$ per dimension is: $x:75.7$, $y:-80.3$, $\psi:77.3$, $u:-25.9$, $v:61.2$, $r:80.9$, and cVDM$(\%)$ is 69.2.} 
    \label{fig:Ship_B_good}
    \vspace{-2mm}
\end{figure}  



\begin{figure}[t]
    \centering
    \includegraphics[width=0.99\linewidth]{figures/ship_A_moderate_JPTSU-JM02C-ab3_2.rtz_Waypoints_W-0_T-0_A-0.pdf}     \vspace{-2mm}
    \caption{Trajectory comparison for ship A with increased accuracy and minor overshoot. Generated trajectories same as in~\cref{fig:Ship_A_good}.  MD$(\%)$ per dimension is: $x:36.3$, $y:22.1$, $\psi:43.9$, $u:34.5$, $v:23.5$, $r:31.9$, and cVDM$(\%)$ is 28.0.} 
    \label{fig:Ship_A_moderate}
    \vspace{-2mm}
\end{figure}  

\begin{figure}[t]
    \centering
    \includegraphics[width=0.99\linewidth]{figures/ship_B_moderate_output_2_300.pdf}     \vspace{-2mm}
    \caption{Improved trajectory prediction for ship B with overreaction (i.e. in $x$ and $u$). Generated trajectories same as in~\cref{fig:Ship_A_good}. MD$(\%)$ per dimension is: $x:-6.7$, $y:36.1$, $\psi:58.8$, $u:-9.4$, $v:22.3$, $r:41.7$, and cVDM$(\%)$ is 22.5.} 
    \label{fig:Ship_B_moderate}
    \vspace{-3mm}
\end{figure}  



    % \item Better handling of given maneuvers 

% The results demonstrate that our hybrid approach successfully generalizes across different vessel types while maintaining physical interpretability. The model shows particular strength in improving in all state variables. The metrics demonstrate that even in cases where visual results show some overshooting, the hybrid model consistently outperforms the baseline physics-based approach. On a side note, this performance advantage, which is always positive, varies depending on the complexity and intensity of the maneuver being executed.


%\textcolor{blue}{
\section{Conclusion and future work}
\label{sec:conclusion}


%Implications of Hybrid Model
%Benefits of combining physics-based and data-driven models in terms of accuracy and interpretability.
%Limitations and Challenges
%Discussion of any observed limitations, such as dependency on specific datasets or complexity of parameter tuning.
%Future Work
%Suggestions for further refinement of the hybrid model, such as improving safeguarding techniques, extending to more complex ship models, or testing in real-world scenarios.}
 

In this work we introduce a novel data-driven physics-based modeling approach that successfully combines physics-based models with data-driven techniques for marine vessel trajectory prediction. We present the core physics-based model, its key parameters and a cNLS parameter optimization method. The  validation across two distinct vessels, ship A and ship B, demonstrates: (i) substantial and consistent improvement in prediction accuracy across different operational regimes, and (ii) successful adaptation to vessel-specific characteristics while maintaining physical interpretability  of the fitted models. By utilizing the same data-driven adaptation process for both ships, we showcase the versatility and applicability of the approach in handling different vessel characteristics. In terms of future work, we plan to investigate additional key parameters relevant to environmental factors (wind, waves, currents) to enhance the adaptability of the data-driven models in broader range of conditions. \vspace{-2mm}


\section{Acknowledgments}
\vspace{-2mm}

This research is part of the developments with Nabtesco corporation for the Nippon Foundation \texttt{MEGURI2040} Fully Autonomous Ship Program. \vspace{-2mm}

% \subsubsection{Significant Findings}
% \begin{itemize}
%     \item Substantial improvements in prediction accuracy, utilizing our custom metrics cVDM, with improvement of 44.44\% for ship A and 61.48\% for ship B
%     \item Consistent performance across different operational regimes, from low-speed maneuvering to higher speed operations
%     \item Successful adaptation to vessel-specific characteristics while maintaining physical interpretability, as evidenced by the optimized hydrodynamic coefficients
%     \item Robust handling of complex maneuvers, demonstrated by the wide ranges of yaw rates and rudder angles
% \end{itemize}

% \subsubsection{Future Research Directions}
% \begin{enumerate}
%     \item Extension to Different Vessel Types with different operational profiles
%     \begin{itemize}
%         \item Validation on a broader range of vessel classes and sizes
%         \item Investigation of scaling effects and parameter transferability
%     \end{itemize}
%     \item Model Enhancement
%     \begin{itemize}
%         \item Incorporation of additional environmental factors (swell, wind waves, currents)
%         \item Development of adaptive optimization techniques for real-time parameter adjustment
%         \item Integration of uncertainty quantification methods
%         \item As already discussed there are studies that promote accuracy instead of interpretability \cite{suyama2024parameter}. Further experimentation with different open parameters and their thresholds may enhance model's adaptability. 
%     \end{itemize}
%     \item Practical Applications
%     \begin{itemize}
%         \item Implementation in real-time navigation systems
%     \end{itemize}
% \end{enumerate}






% \bibliographystyle{styles/bibtex/splncs03_unsrt}
\bibliographystyle{splncs04}
\bibliography{references}

\end{document}
