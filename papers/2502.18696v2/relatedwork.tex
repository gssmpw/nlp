\section{Related Work}
\label{sec:related_Work}

\textbf{Physics-based (white-box) models}:
Research on maneuvering models began with~\cite{davidson1946turning}. Later,\cite{motora1959measurement},\cite{motora1955course}, and~\cite{nomoto1957steering} contributed to hydrodynamic aspects and stability principles. \cite{abkowitz1964ship} and~\cite{aastrom1976identification} used Taylor expansion for maneuvering models. The MMG model that is widely used in autonomous navigation was introduced by~\cite{ogawa1977mmg}, was improved by~\cite{yasukawa2015introduction} and has been applied by~\cite{li2013active} and~\cite{zhang2017ship}. Other models for low-speed hydrodynamic forces include the HD model~\cite{yasukawa2015introduction}, the CD model~\cite{yoshimura2012hydrodynamic}, and the TBL model~\cite{sutulo2015development}.

\textbf{Black box models}: Recently machine Learning approaches such as   artificial neural networks~\cite{moreira2003dynamic,rajesh2008system,zhang2013black,oskin2013neural,luo2016modeling,woo2018dynamic} and Gaussian processes~\cite{arizaramirez2018nonparametric,xue2020system} have been explored. \cite{SILVA2022103222} uses deep neural networks (DNN) for 6-DoF motion prediction in waves to demonstrate how DNNs can capture nonlinear dynamics between the ship state and wave conditions while maintaining good generalization.

\textbf{Grey box models}:
An other research avenue has focused on developing parameter estimation methods to combine traditional physics-based models with data-driven adaptation.
\cite{suyama2024parameter,MiyauchiCMAES} employed a gradient-free optimization method (namely CMA-ES) to tune parameters of an MMG model~\cite{yasukawa2015introduction} using sea-trial data of a real-scale ship. 
\cite{kanazawa2023bridging} proposed a framework that 
fits open parameters of an Abkowitz model~\cite{abkowitz1964ship} to data of the target ship while using a well-validated model of a similar ship to regularize the fitting process. In this way, the resulting predictions are on par with the validated model while they are tailored to the target ship's behavioral patterns. 
Although, improved prediction accuracy has been demonstrated, depending on the type of fitted parameters reliability to the model can be lost, if the parameters are not interpretable.
Unlike these approaches, the data-driven physics-based model introduced in this paper, which is based on the white-box model~\cite{mathioudakis2025realworldvalidationphysicsbasedship}, focuses on optimizing interpretable parameters, making the results more accessible and understandable to domain experts.   \vspace{-4mm}


% This enables a direct relation between the model's output and the underlying physical and operational characteristics of the ship. While focusing parameter estimation on interpretable parameters might seem advantageous, it can lead to reduced reliability of the resulting model if the constraints imposed by the physics-based framework are not properly maintained.
% \textcolor{red}{Here, we need to describe how our approach differs to the above, for example depending on the open parameters interpretability can be lost and the fact that our physics-based model is Spyrou-based which implies .... also, here describe that if parameter estimation focuses on non interpretable parameters then the resulting model is not reliable. }



% bridges the dynamics of similar ships by using parameters from a well-validated model as a knowledge foundation for developing models of new ships. 
% In parallel, work has been done on parameter fine-tuning methods for existing models.

% to simultaneously adjust multiple parameters while maintaining realistic values, 
% ~\cite{MiyauchiCMAES}
\begin{figure}[t]
    \includegraphics[width=1\linewidth]{simulation_flow.pdf} 
    \vspace{-8mm}
    \caption{Computational flow of a physics-based vessel motion prediction model, with the following key blocks: (a) ``Control": input commands of the rudder and the propeller, (b) ``Environment": environmental effects from the wind, waves, and sea currents, (c) ``Force calculation": computation of forced using the outputs of (a) and (b) along with hydrodynamic forces and the current ship state, and (d) ``Equations of motion": dynamics equations and integration over time to produce the next ship state.}
    \label{fig:simulation_system}
    \vspace{-3mm}
\end{figure}