\section{Introduction}
\label{sec:intro}

The concept of public goods is not only a significant area of interest in economic research but also closely related to the web era.
Public goods encompass a wide range of web resources, in forms like open-source software (\eg, GitHub), public databases (\eg, MNIST), scientific technologies (\eg, papers in The WebConf), and widely accessible scientific knowledge (\eg, Wikipedia, Stack overflow).
The defining characteristics of public goods are their non-excludability, meaning all community members can freely use these resources without excluding anyone, and non-rivalry, where one person’s use does not diminish the availability for others. 
Such characteristics are particularly notable on the internet. 
Web and internet research delves into how to effectively provide and manage digital public goods to maximize social welfare. This exploration is not just theoretical but also has practical implications for policy and the development of website content, attracting an increasing number of researchers to this burgeoning field.

% in Wikipedia, users can upload and edit dictionary to the Wikipedia, and dictionary would be benefited by user; in GitHub, users can share their codes and collaborate the open-source code, which can be benefited by users; in stack overflow, users can ask questions and answer questions proposed by others; in the volunteering review of the paper in conference, the reviews have a direct effect (good/bad) to the authors of the paper and a indirect effect to the conference. All these goods can be seen as public goods.

However, from a societal perspective, digital public goods often face challenges due to insufficient provision, a problem frequently attributed to the issue of free-riding. Consequently, each participant must decide how much effort to contribute when investing in digital public goods, aware that their efforts will also benefit others. This strategic decision-making process embodies what is known as a {\it public goods game}. This game can reveal complex interactions between cooperation and competition, as individuals shall balance their contributions against the collective benefits. Much of the prior research~\cite{public-network:bramoulle2007public} has focused on idealized models where participants are assumed to be homogeneous. However, in reality, especially in the case of digital public goods, users exhibit significant heterogeneity. For example, a specialized dictionary on Wikipedia is more beneficial to those within the relevant field. On the other hand, in the context of paper reviews, the efforts of one reviewer benefit the entire conference but may disadvantage the author of a low-quality submission. This demonstrates that the impact of a public good (or bad) can be either positive or negative, and varies across different participants.

In this paper, we are more interested in the networked public goods games, which effectively capture the social connections 
among individuals. Specifically, all participants are positioned at the vertices of the network, and the links—each weighted differently—represent the relationships and influence between any two participants~\cite{jichen:li2023altruism}.
\citet{public-network:bramoulle2007public} pioneered the study of public goods games within a network. 
In their homogeneous model, the utility functions of all players are consistently formulated as $u_i(\bmx) = f(x_i + \sum_{j \in N_i} x_j) - cx_i$, where $x_i$ is the effort level of player $i$, $f(\cdot)$ is a homogeneous benefit function applicable to all players, and the cost function is linear, characterized by a uniform unit cost $c$ for all players. 
Furthermore, this model is unweighted, as each player exhibits the same preference for both their efforts and those of others when computing the benefit.
Based on this simplified and idealistic setting, \citet{public-network:bramoulle2007public} demonstrated the existence of an equilibrium where some players exert the same maximum effort while all others engage in free riding. Moreover, they showed that those contributing positive effort form an independent set within the network. 
While later studies have explored the public goods games with heterogeneous utility functions \cite{public-network-direct-complexity:papadimitriou2021public,public-network-direct-BRD:bayer2023best}, their focus remained on the linear cost scenarios.

However, practical scenarios often feature non-linear cost functions, particularly evident in digital public goods.
For instance, the initial setup of a Wikipedia article involves adding basic facts and general information—tasks that are relatively low in cost. Yet, as the article develops, ensuring accuracy and providing in-depth analysis demand increasingly specialized knowledge, research, and citations, raising the marginal cost of contributions. Unfortunately, the predominant body of research on public goods games focuses on linear cost functions \cite{public-network:bramoulle2007public,public-network-direct:lopez2013public,public-network-direct-complexity:papadimitriou2021public,public-network-direct-BRD:bayer2023best},  and very few studies delve into the implications of non-linear cost functions.

This paper presents a novel model of networked public goods games that incorporates convex cost functions, aiming at understanding the equilibrium and dynamics in the field of digital public goods. 
Specifically, given an effort profile $\bfx=(x_1,x_2,\cdots,x_n)$, each player's payoff is determined by the net gain, which is the difference between a benefit function $f_i(k_i)$ and a cost function $c_i(x_i)$. The benefit function $f_i$ for player $i$ is both concave and strictly increasing, and it is derived from the gain $k_i$. This gain $k_i$ is computed as a weighted linear combination of the efforts of both the player and her neighbors. The cost function $c_i$, which is convex and strictly increases, depends exclusively on the player's effort $x_i$.


%Let $G=(V,E)$ represent the network, where $V$ is the set of players and $E$ is the set of edges. We use $w_{ij}\in [0,1]$ to denote the weight of the edge $(i,j)\in E$, which represents the marginal gain of player $i$ from player $j$'s effort. Consequently, the matrix of marginal gains is represented by $W=\{w_{ij}\}_{(i,j)\in E}$, which is symmetric and where $w_{ii}=1$ for any $i\in V$. In our networked public goods games with convex costs, given an effort profile $\bfx=(x_1,x_2,\cdots,x_n)$, each player $i$ achieves a gain, $k_i=\sum_{j\in \Gamma_i}w_{ij}x_j$. Additionally, each player's is characterized by the difference between a benefit function $f_i(k_i)$ and a cost function $c_i(x_i)$, where $f_i$ is concave and strictly increasing, and $c_i$ is convex and strictly increasing. 

% \noindent {\bf Results and Techniques.}
\subsection{Results and Techniques}
Our work is the first one to study the networked public good games with convex costs. The heterogeneity of benefit functions and cost functions lends greater generality to the networked public goods game studied in this paper.
We start by exploring the concept of welfare solutions, focusing on the maximization of social welfare and the investigation of pseudo-gradient ascent dynamics, which shows insight into the following analysis. We carefully analyze the existence and uniqueness of Nash Equilibrium (NE) across various settings, providing deep insights into the NE's structures in public goods games. Our examination extends to cases in which distinct characteristics of cost and benefit functions play a crucial role in ensuring the NE's uniqueness. Building on these foundations, we delve into comparative statics to assess the effects of subtle shifts in the model's parameters, which we regard as money redistribution, on the utilities of the players involved. Comparative statics is a crucial analytical method in economics. This element of our study illuminates how minor adjustments can significantly influence economic outcomes and player behaviors within the game.
We also study a special case, in which the game structure is pre-defined and show how these theorems can be applied to this case.

The proof of the existence of a Nash Equilibrium (NE) primarily relies on the application of the Brouwer fixed-point theorem.
% , which assesses the Lipschitz continuity of the argmax function. 
Brouwer's fixed-point theorem states that any continuous function mapping a compact, convex set to itself must have a fixed point \cite{brouwer:brouwer1911abbildung}. 
% Additionally, \cref{lem:argmax:Lipschitz} establishes that under specific conditions, the argmax function is Lipschitz continuous.
It's important to note that the best-response function is continuous when the utility functions are strictly concave. The proof then proceeds through a strategic modification of the utility functions, ensuring they meet the criteria stipulated by Brouwer's fixed-point theorem.

To carve out the uniqueness of NE, we bring out the concept of near-potential game and show that under certain conditions, the NE of near-potential game is unique, and pseudo-gradient ascent dynamic will converge to this point with exponential rate. The proof constructs the discrete version of pseudo-gradient ascent dynamic and shows that it is compressive mapping, which is guaranteed to have a unique fixed point by Banach's theorem \citep{banach:banach1922operations}.
We then bridge the gap between near-potential games and public good games, showing three conditions under which we can transform the public good games into a specifically designed near-potential game while holding the NEs invariant, therefore guaranteeing the uniqueness of NE. 
We also propose the concept of game equivalence, which ensures the one-to-one relationships between the NEs of corresponding games, which can also broaden the class of games possessing unique NE.

To study the comparative statics on money redistribution, we mainly use the high-dimensional implicit function theorem.
We rewrite the conditions of Nash equilibrium $\bmx^*$ as an implicit function of infinitesimal model change $t\bmdd$ and corresponding NE $\bmx^*(t)$.
By differentiating this implicit function on $t$, we can derive the relation between $\bmx^*(t)$, $\bmdd$ and $t$.

\subsection{Related Works}
%\subsubsection{Networked Public Goods Games}

\emph{\bf Public Goods in the Web Era.}
In the web era, public goods play a crucial role in fostering collaborative contributions and maintaining online platforms. 
\citet{public_good-web-1:gallus2017fostering} demonstrates the impact of symbolic awards on volunteer retention in a public good setting like Wikipedia, where recognition and community engagement can encourage sustained contributions without direct financial incentives. 
Similarly, the challenges of knowledge-sharing in Web 2.0 communities have been framed as a public goods problem, where social dilemmas like free-riding are mitigated through enhanced group identity and pro-social behavior \citep{public_good-web-2:allen2010knowledge}.
Experimental research on cooperation in web-based public goods games further examines how network structures influence contribution behavior, with findings suggesting that contagion effects in cooperative behavior are limited to direct network neighbors \citep{public_good-web-3:suri2011cooperation}.
Moreover, the broader economic dynamics of the web are analyzed through the concept of "web goods," where users contribute content, exchange information, and interact in a socio-economic system that requires balancing open access with incentive structures for content production and infrastructure development \citep{public_good-web-4:vafopoulos2012web}.
These works collectively highlight the unique challenges and opportunities of managing public goods in the digital age, emphasizing the importance of community-driven incentives and network effects in fostering web-based cooperation.

\emph{\bf Networked Public Good Games.}
\citet{public-network:bramoulle2007public} initiated the study of public goods in a network. They studied the public good games on an unweighted, undirected network with linear cost functions and homogeneous players. Under their models, there is a unique level $k^*$ such that it's optimal for each player to make the sum efforts within her neighborhoods to be $k^*$, which greatly simplifies the analysis of the model. The authors showed that the NE of the game corresponds to the maximal independent set, where the player in the maximal independent set asserts full effort $k^*$, and the players outside free-ride. 

There are many other works following this literature, see \citep{public-network-follow:bramoulle2014strategic, public-network-follow:boncinelli2012stochastic,public-network:allouch2015private,public-network:elliott2021network}.
\citet{public-network-follow:bramoulle2014strategic} extended the model to the imperfectly substituted
public goods case, and proved the existence and uniqueness of Nash equilibrium, under the condition of sufficiently small lowest eigenvalue of the graph matrix.
\citet{public-network:allouch2015private} differentiated the provision of public goods and private goods, and their results of existence and uniqueness of Nash equilibrium also rely on the lowest eigenvalue of the graph matrix.
\citet{public-network-direct:lopez2013public} began with the studies of public good games in directed networks, by discussing both the static model and the dynamic model. To be specific, in the static model, all players are situated within a fixed network where they choose their actions simultaneously. \citet{public-network-direct:lopez2013public} demonstrated that the structure of Nash equilibria correlates with the maximal independent set. In contrast, the dynamic model is characterized by a dynamic sampling process, where agents periodically sample a subset of other agents and base their decisions on a myopic-best response. The author established the existence of a unique globally stable proportion of public good providers in this model. \citet{public-network-direct-BRD:bayer2023best} studied the convergence of best response dynamic on the public good games in directed networks.

% \paragraph{Binary Networked Public Good Games}
A significant networked public goods game variant considers indivisible goods, where players can only make binary decisions\cite{galeotti2010network}. 
Building upon this binary networked public goods (BNPG) game model, \citet{yu2020computing} introduced the algorithmic inquiry of determining the existence of pure-strategy Nash equilibrium (PSNE). 
Specifically, they investigated the existence of PSNE in the BNPG game and proved that it is NP-hard in both homogeneous and heterogeneous settings.
The computational complexity of public goods games with a network structure, such as tree or clique~\cite{yang2020refined, maiti2024parameterized}, and regular graph~\cite{feldman2013pricing} has also been extensively studied.
Papadimitriou and Peng\cite{public-network-direct-complexity:papadimitriou2021public} proved that finding an approximate NE of the public good games in directed networks is PPAD-hard, even if the utility is in a summation form.
Subsequently, \citet{public-uncertainty:gilboa2022complexity} modeled players as different patterns and showed that the existence of PSNE on some non-trivial patterns is NP-complete, while a polynomial time algorithm exists for some specific patterns.
In addition, \citet{klimm2023complexity} further demonstrated the complexity results of the BNPG game on undirected graphs with different utility patterns to be NP-hard. 
They also showed that computing equilibrium in games with integer weight edges is PLS-complete.
% \footnote{Due to space limit, we leave related works on binary networked public goods games to \cref{app:related}.}

\emph{\bf Continuous-time Public Good Games.}
A branch of the literature on public goods focuses on studying the dynamic provision of public goods in continuous time. \citet{public-dynamic:fershtman1991dynamic} was the first to explore this problem. They proposed two equilibrium concepts: the open-loop equilibrium and the feedback equilibrium, showing that in the feedback equilibrium, the players' utilities are lower than in the open-loop equilibrium. This result is derived under the linear strategy assumption of the feedback equilibrium, as the feedback equilibrium is not generally unique. Later, \citet{public-dynamic-nonlinear:wirl1996dynamic} discovered that if non-linear strategies are allowed in the feedback equilibrium, players' utilities can be higher in some feedback equilibria than in the open-loop equilibrium. \citet{public-dynamic-CES:fujiwara2009dynamic} generalized these findings to more general utility functions and confirmed that the results still hold. \citet{public-dynamic-uncertain:wang2010dynamic} extended this work by considering environments with uncertainty.

Although these studies present findings in dynamic scenarios, they generally assume homogeneity among players in terms of utility functions (both gains and costs) and interpersonal relationships and thus do not consider network effects. To the best of our knowledge, no previous research has simultaneously explored the dynamic provision of public goods with heterogeneous players.

\emph{\bf Concave Games.}
% \myx{need polish}
\citet{concave_game-initial:rosen1965existence}
firstly introduced the concept of concave games, in which the utility function of each player is concave to her strategy. In this paper, \citet{concave_game-initial:rosen1965existence} provided a sufficient condition for such games to have a unique equilibrium and introduced a differential equation that converges to this equilibrium. 
Because of the foundational results of \citet{concave_game-initial:rosen1965existence}, several works have extended the study of concave games in various settings, such as learning perspective of equilibrium in concave games \citep{concave_game-learning:nesterov2009primal, concave_game-learning:mertikopoulos2019learning, concave_game-learning:bravo2018bandit}, equilibrium concept in concave games \citep{concave_game-concept:forgo1994existence, concave_game-concept:ui2008correlated, concave_game-concept:goktas2021convex}.
However, limited research has applied the concave game framework to public goods scenarios. Our work is pioneering in applying the convex game framework to public goods games. We demonstrate that public goods games can be treated as a specific type of concave game, called a near-potential game, where the potential function is meticulously designed for diverse scenarios. The uniqueness of equilibrium in near-potential games, therefore, directly supports the uniqueness of equilibrium in public goods games.
%However, seldom existing work applied the framework of concave games to the scenarios of public goods. Therefore, our work is the first one to study the public goods games by applying the convex game framework. For this purpose, we demonstrate that a public goods game can be considered a specific type of near-potential game, where the potential function is carefully designed for different cases. Consequently, the uniqueness of equilibrium in near-potential games directly implies the uniqueness of equilibrium in public goods games.



%Notably, a public goods game is, by definition, a concave game. \citet{concave_game-initial:rosen1965existence} proposed a sufficient condition under which a concave game has a unique equilibrium point. They also introduced a differential equation that converges to the equilibrium point. These results are analogous to the findings on ``near-potential games'' and the ``pseudo-gradient ascent dynamic'' presented in this work. 

%However, it is important to note that our concept of ``near-potential games'' is developed specifically from the perspective of public goods games, whereas \citet{concave_game-initial:rosen1965existence}'s results are not directly related to public goods games. From a higher-level viewpoint, our results demonstrate that a public goods game can be considered a specific type of near-potential game, where the potential function is carefully designed for different cases. Consequently, the uniqueness of equilibrium in near-potential games directly implies the uniqueness of equilibrium in public goods games.

%Since the foundational results of \citet{concave_game-initial:rosen1965existence}, several works have extended the study of concave games in various settings, such as learning perspective of equilibrium in concave games \citep{concave_game-learning:nesterov2009primal, concave_game-learning:mertikopoulos2019learning, concave_game-learning:bravo2018bandit}, equilibrium concept in concave games \citep{concave_game-concept:forgo1994existence, concave_game-concept:ui2008correlated, concave_game-concept:goktas2021convex}.
%To the best of our knowledge, no previous work has applied the framework of concave games to the public goods game scenario.


%\paragraph{Binary Networked Public Goods Games} 

%A significant networked public goods game variant considers indivisible goods, where players can only make binary decisions\cite{galeotti2010network}. Building upon this binary networked public goods (BNPG) game model, \citet{yu2020computing} introduced the algorithmic inquiry of determining the existence of pure-strategy Nash equilibrium (PSNE). Specifically, they investigated the existence of PSNE in the BNPG game and proved that it is NP-hard in both homogeneous and heterogeneous settings. The computational complexity of public goods games with a network structure, such as tree or clique~\cite{yang2020refined, maiti2024parameterized}, and regular graph~\cite{feldman2013pricing} has also been extensively studied. Papadimitriou and Peng\cite{public-network-direct-complexity:papadimitriou2021public} proved that finding an approximate NE of the public good games in directed networks is PPAD-hard, even the utility is in a summation form. Subsequently, \citet{public-uncertainty:gilboa2022complexity} modeled players as different patterns and showed that the existence of PSNE on some non-trivial patterns is NP-complete, while a polynomial time algorithm exists for some specific patterns. In addition, \citet{klimm2023complexity} further demonstrated the complexity results of the BNPG game on undirected graphs with different utility patterns to be NP-hard. They also showed that computing equilibrium in games with integer weight edges is PLS-complete.


% \myx{Technique}
% The model is introduced in \cref{sec:model}, followed with some basic definitions and a brief analysis of social welfare and social optimal solutions.
% We establish the existence and uniqueness of \emph{pure} Nash Equilibrium (NE) of public good games in \cref{sec:equilibrium}. We show that pure NE always exists in our public good games model. The proof’s crux hinges on applying the Brouwer fixed-point theorem, as well as a lemma that measures the Lipschitz property over the argmax function. 

%Not surprisingly, the pure NE might not be unique. To carve out the uniqueness of NE, we bring out the concept of near-potential game, and show that under certain condition, the NE of near-potential game is unique, and best-response dynamic will converges to this point with exponential rate. The proof constructs the discrete version of best-response dynamic and show that it is compressive mapping, which is guaranteed to have unique fixed point by Banach's theorem.
%We then bridge the gap between near-potential game and public good games, showing three conditions under which corresponding near-potential game can be constructed and thus the uniqueness of NE is guaranteed. We also proposes the concept of game equivalence, that ensures the one-to-one relationships between the NEs of corresponding games, which can also broaden the class of games possessing unique NE.

%Finally, we carry out some case study. We firstly studies the comparative statics, which studies the change on utilities profile under infinitesimal change of money redistribution. We also give an specific example, showing that how the results in \cref{sec:equilibrium} can be used into applications.

% \myx{Contribution}
% Above all, the contributions of this paper lie in three folds:
% \begin{itemize}
%     \item 
% \end{itemize}

% \noindent{\bf Related Works}



% There are many works study the complexity of public good games\citep{jichen:li2023altruism, public-network-direct-complexity:papadimitriou2021public,public-uncertainty:gilboa2022complexity}. Among which, \citet{public-network-direct-complexity:papadimitriou2021public} proves that finding an approximate NE of the public good games in directed networks is PPAD-hard, even the utility is in a summation form.
% However, to the best of our knowledge, we extend the model assumption of linear cost functions to the convex cost functions in the first time.
% there is no works that extend the assumption of linear costs function, which restricts the model generality. 


% \subsubsection{Netwoked Public Good Games with $0-1$ Action Space}

% \subsubsection{Dynamic Public Goods Games}

