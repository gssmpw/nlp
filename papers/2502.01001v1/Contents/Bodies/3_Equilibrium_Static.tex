\section{Equilibrium Solutions of Public Good Games}
\label{sec:equilibrium}

In this section, we establish the existence and uniqueness results of equilibrium solutions in public good games. An effort profile is $\bmx$ as an (pure strategy) NE, if no player can unilaterally increase her utility by changing her effort. Formally, $\bmx$ is an NE if, for any player $i$ and any alternative effort $x_i'$, we have
\begin{align}
\label{eq:static:NE}
    u_i(x_i', \bmx_{-i}) \le u_i(x_i, \bmx_{-i}) \quad \forall x'_i \in [\underline{x}_i, \bar{x}_i].
\end{align}

% \myx{can delete}
% The first order condition states that an effort profile $\bmx$ is a NE iff for all $i$,
% \begin{align*}
%     f'(k_i) =& c'(x_i) \quad \text{if $\xlow_i < x_i < \xhigh_i$}
%     \\
%     f'(k_i) \le& c'(x_i) \quad \text{if $x_i = \xlow_i$}
%     \\
%     f'(k_i) \ge& c'(x_i) \quad \text{if $x_i = \xhigh_i$}
% \end{align*}


\subsection{Existence of Nash Equilibrium}

Generally, an (pure strategy) NE may not exist in normal-form games. Fortunately, the following theorem states that an NE always exists in the networked public good games studied in this paper. 

\begin{restatable}{theorem}{thmNEExist}
\label{thm:NE:exist}
    In the public good game $G = (\{f_i\}_\iinn, \{c_i\}_\iinn,$ $\{X_i\}_\iinn, W)$, an (pure strategy) NE always exists.
\end{restatable}
%The proof's crux hinges on applying the Brouwer's fixed-point theorem to demonstrate that a NE is invariably achievable \citep{brouwer:brouwer1911abbildung}. 
% To construct the basis of our argument, we first present an essential lemma, whose proof is shown in \cref{app:omitted-prf}.
% \begin{restatable}{lemma}{lemArgmaxLipschitz}
\label{lem:argmax:Lipschitz}
Let function $u(c,x) = g(c,x) + f(x)$ be defined on a bounded, closed, and convex set $X$. Suppose that $g(c,x)$ is concave on $x$, twice differentiable almost everywhere with 
$$\sigma_{max}( \frac{\partial g}{\partial c\partial x} ) \coloneqq \sup_{||r_c|| = ||r_x|| = 1} r_c^T \frac{\partial g}{\partial c\partial x} r_x \le L$$ for some positive real number $L$ and $f(x)$ is an $\alpha$-concave function. We denote the maximum solution of $u(c, x)$ by $x(c) = \arg\max_{x\in X} u(c,x)$. Thus, $x(c)$ not only exists and is unique, but it is also $\frac{2L}{\alpha}$-Lipschitz continuous with respect to $c$.
\end{restatable}
\begin{proof}[Proof Sketch of \cref{thm:NE:exist}]
Let us construct the best-response function $\BR(\bmx)$ for all players. If it is continuous, then by Brouwer's fixed point theorem \citep{brouwer:brouwer1911abbildung}, there is a fixed point, which is also the NE of the game.

However, this is not always the case, since the best response may be discontinuous and even not a singleton set. To address this issue, we modify the cost function to be an $\alpha$-convex function. This results in an $\alpha$-modified game, whose best-response function is continuous.
% by \cref{lem:argmax:Lipschitz}, the best-response function on 
% the $\alpha$-modified game is $\frac{2L}{\alpha}$-Lipschitz, thus continuous.

As long as the $\alpha$-modified game has an NE $\bmx^*_\alpha$, we first let $\alpha \to 0$, then by compactness of $X$, we claim that there is an accumulation point $\bmx^*$ that is the limitation of $\bmx^*_{\alpha_k}$ for a sequence $\alpha_k\to 0$. 
To establish the desirable result in this theorem, we continue proving that $\bmx^*$ is the NE of the original game, given $\bmx^*_{\alpha_k}$ is the NE of $\alpha_k$-modified game for all $k$.
This involves carefully taking limits through several steps.
\end{proof}

Similar to normal-form games, NE in public goods games may not be unique. This is elaborated upon in the example below.

\begin{figure}[t]
\centering
\begin{subfigure}{0.3\textwidth}
    \includegraphics[width=\textwidth]{Contents/Figures/1.pdf}
    \label{fig:NE:nonunique-1}
    \caption{}
\end{subfigure}
\begin{subfigure}{0.3\textwidth}
    \includegraphics[width=\textwidth]{Contents/Figures/2.pdf}
    \label{fig:NE:nonunique-2}
    \caption{}
\end{subfigure}
\caption{Examples of non-unique NE in public good games. (a): There are four players on two sides, with two players in each side. (b) There are $n_1\times n_2$ players in $n_1$ groups, with $n_2$ players in each group.}
\label{fig:NE:nonunique}
\end{figure}

\begin{example}
\label{eg:NE:nonunique}
Consider a public good game containing four players, see \cref{fig:NE:nonunique}-(a) . The marginal gain is $1$ between one player from the left side and the other from the right side, and $0$ otherwise. We specify homogeneous utility functions and action spaces for all players. The action space is specified as $[0,1]$, while the only two constraints for utility functions are: 
\begin{align*}
    f'(1) \ge c'(1)~~\mbox{and}~~
    f'(2) \le c'(0).
\end{align*}
It's straightforward to verify that players on one side exert full effort, i.e., $x_i = 1$,
while players on the opposite side free ride, i.e., $x_i = 0$, constitutes a Nash Equilibrium (NE). Thus, there are at least two NEs in this game. This example can be readily extended to a scenario involving $n_1\times n_2$ players, distributed into $n_1$ groups with $n_2$ players in each group. All pairs of players from different groups are connected, see \cref{fig:NE:nonunique}-(b). The second condition then becomes $f'(n_2) \le c'(0)$.
%It's easy to verify that players in one side assert full efforts, \ie, $x_i = 1$, while players in the other side free ride, \ie, $x_i = 0$, constitutes an NE. Therefore there are at least two NEs in this game.
%This example can be easily extended to the case with $n_1 n_2$ players, with $n_1$ groups and $n_2$ players in each groups. The edges are connected for all pairs of players from different groups. The second condition becomes $f'(n_2) \le c'(0)$.
\end{example}

Example \ref{eg:NE:nonunique} motivates us to investigate the scenarios in which the NE is unique.

\subsection{Uniqueness of Nash Equilibrium}
\label{subsec:NE:unique}

In this section, we explore the conditions under which NE of the public good game is unique.
We begin by introducing a necessary lemma along with some definitions that will be frequently utilized in the subsequent theorems.


\begin{definition}[$(\gamma, \sigma)$-closeness]
\label{def:closeness}
    A function $g(x): X \to \bbR$ is $(\gamma, \sigma)$-close ($\gamma, \sigma \in \bbR_{+}$) to a function $f(x): X \to \bbR$, where $X \subseteq \bbR^k$, if $\gamma \nabla_x g(x) - \nabla_x f(x)$ is $\sigma$-Lipschitz on $x$.
\end{definition}

\begin{definition}[Near-potential Game]
\label{def:near-potential}
    Consider a game containing $n$ players. Denote $x_i \in X_i \subset \bbR$ as the action of player $i$ and $u_i(\bmx)$ as the utility function of player $i$ given the joint action $\bmx$. 

    We say that a game is a $(\bmgg,\Sigma)$-near-potential ($\bmgg \in \bbR_{++}^n, \Sigma\in\bbR_+^{n\times n}$) game \wrt~  a potential function $u(\bmx)$, if for two players $\ijinn$ (it could be $i = j$), we have that $u_i(\bmx)$ is $(\gamma_i, \sigma_{ij})$-close to $u(\bmx)$ on the domain $X_j$, assuming that $\bmx_{-j}$ is fixed, \ie, 
    \begin{align*}
        \gamma_i \frac{\pp u_i}{\pp x_i}(x_j,\bmx_{-j}) - \frac{\pp u}{\pp x_i}(x_j,\bmx_{-j})
    \end{align*}
    is $\sigma_{ij}$-Lipschitz on $x_j$ for all fixed $\bmx_{-j}\in X_{-j}$, where $\Sigma = \{\sigma_{ij}\}_\ijinn$.
\end{definition}


\begin{proof}
\label{prf:lem:near-potential}

\emph{Step 1: The uniqueness of NE.}

We prove the uniqueness of NE mainly by constructing a contraction mapping and using Banach fixed-point theorem.
The contraction mapping is constructed by a discretized version of the best-response dynamic:
\begin{align*}
    x'_i = g_i(\bmx) = x_i + \varepsilon \gamma_i \frac{\pp u_i}{\pp x_i}(x_i, \bmx_{-i}),\quad \forall \iinn
\end{align*}
for some $\varepsilon > 0$ small enough. Now we prove that $g: X \to X$ is a contraction mapping.

Consider two effort profiles $\bmx, \bmy \in X$, we have that
\begin{align*}
    \| g(\bmx) - g(\bmy) \|^2 =& \sum_\iinn \left( g_i(\bmx) - g_i(\bmy) \right)^2
    \\
    =& \sum_\iinn \left( x_i + \varepsilon\gamma_i \frac{\pp u_i}{\pp x_i}(x_i, \bmx_{-i}) - y_i - \varepsilon\gamma_i \frac{\pp u_i}{\pp x_i}(y_i, \bmy_{-i}) \right)^2
    \\
    =& \| \bmx - \bmy \|^2 + \varepsilon \langle \bmx - \bmy, \gamma_i\left(\frac{\pp u_i}{\pp x_i}(x_i,\bmx_{-i}) - \frac{\pp u_i}{\pp x_i}(y_i,\bmy_{-i})\right)_\iinn\rangle
    \\
    +& \varepsilon^2 \sum_\iinn\gamma_i^2 \left( \frac{\pp u_i}{\pp x_i}(x_i,\bmx_{-i}) - \frac{\pp u_i}{\pp x_i}(y_i,\bmy_{-i}) \right)^2
\end{align*}

We focus on the $\varepsilon$ term:
\begin{align*}
    & \langle \bmx - \bmy, \gamma_i \left( \frac{\pp u_i}{\pp x_i}(x_i,\bmx_{-i}) - \frac{\pp u_i}{\pp x_i}(y_i,\bmy_{-i}) \right)_\iinn \rangle
    \\
    =& \langle \bmx - \bmy, \frac{\pp u}{\pp \bmx}(\bmx) - \frac{\pp u}{\pp \bmx}(\bmy) \rangle
    + \langle \bmx - \bmy, \left( \frac{\pp (\gamma_i u_i - u)}{\pp x_i}(x_i,\bmx_{-i}) - \frac{\pp (\gamma_i u_i - u)}{\pp x_i}(y_i,\bmy_{-i}) \right)_\iinn \rangle
\end{align*}
Since the $c$-concavity of $u(\bmx)$, we have
\begin{align*}
    \langle \bmx - \bmy, \frac{\pp u}{\pp \bmx}(\bmx) - \frac{\pp u}{\pp \bmx}(\bmy) \rangle \le -c \| \bmx - \bmy \|^2
\end{align*}
By $\sigma_{ij}$-Lipschitzness of $\frac{\pp (\gamma_i u_i - u)}{\pp x_i}(\bmx)$ on $x_j$, we have
\begin{align*}
    & \langle \bmx - \bmy, \left( \frac{\pp (\gamma_i u_i - u)}{\pp x_i}(x_i,\bmx_{-i}) - \frac{\pp (\gamma_i u_i - u)}{\pp x_i}(y_i,\bmy_{-i}) \right)_\iinn \rangle 
    \\
    \le& \sum_\iinn \left( \frac{\pp (\gamma_i u_i - u)}{\pp x_i}(x_i,\bmx_{-i}) - \frac{\pp (\gamma_i u_i - u)}{\pp x_i}(y_i,\bmy_{-i}) \right) |x_i - y_i|
    \\
    \le& \sum_\iinn \sum_\jinn \sigma_{ij} |x_j - y_j| |x_i-y_i|
    \\
    =& \| \bmx - \bmy \|^2 \sum_\iinn \sum_\jinn \sigma_{ij} z_i z_j
    \\
    =& \| \bmx - \bmy \|^2 \bmz^T \Sigma \bmz
    \\
    \le& \sigma_{max}(\Sigma) \| \bmx - \bmy \|^2
\end{align*} 

where in the third equality, $z_i \coloneqq \frac{|x_i - y_i|}{\| \bmx - \bmy\|}$, we have $\| z_i \| = 1$. Also notice that $\sigma_{max}(\Sigma) = \max_{\|\bmz\| = \|\bmz'\| = 1} \bmz^T \Sigma \bmz'$.

Therefore, the $\varepsilon$ term:
\begin{align*}
    & \langle \bmx - \bmy, \gamma_i\left( \frac{\pp u_i}{\pp x_i}(x_i,\bmx_{-i}) - \frac{\pp u_i}{\pp x_i}(y_i,\bmy_{-i}) \right)_\iinn \rangle
    \\
    \le& (\sigma_{max} - c) \| \bmx - \bmy \|^2
\end{align*}

Therefore, we can choose $\varepsilon$ so small such that $\| g(\bmx) - g(\bmy) \|^2 \le (1 + \frac{\varepsilon}{2}(\sigma_{max} - c)) \| \bmx - \bmy \|^2 < \| \bmx - \bmy \|^2$ for all $\bmx,\bmy$, which indicates that $g$ is a contraction mapping. By Banach fixed-point theorem, we know that there exists a unique fixed point $\bmx^*$ of $g$, which means that there is a unique $\bmx^*$ such that $\frac{\pp u_i}{\pp x_i}(\bmx^*) = 0$ for all $i$, indicating that $\bmx^*$ is the unique NE.

\emph{Step 2: The exponential convergence rate.}

To do this, we aim at constructing an energy function $E(\bmx)$ such that it holds the following properties:
\begin{itemize}
    \item $E(\bmx) \ge 0$ for all $\bmx$, the equality holds if $\bmx = \bmx^*$.
    \item $\frac{\dd E(\bmx(t))}{\dd t} \le -p_0 E(\bmx(t))$ for some $c_0 > 0$.
\end{itemize}

As long as these properties hold, we immediately get that $E(\bmx(t)) \le E(\bmx(0)) \exp(-p_0 t)$. Then taking $c_0 = p_0$ completes the proof.

We first define the energy function $E(\bmx) = u(\bmx^*) - u(\bmx) + \langle \bmx - \bmx^* , \nabla u(\bmx^*) \rangle$. Since $u(\bmx)$ is a concave function, we know that,
\begin{align*}
    u(y) - u(x) \le \langle y-x, \nabla u(x)\rangle
\end{align*}
Take $x = \bmx^*$ and $y = \bmx$, we derive that $E(\bmx) \ge 0$ for all $\bmx$. When $\bmx = \bmx^*$, we have $E(\bmx^*) = 0$. We also know that $E(\bmx)$ is $c$-strongly convex function.

By little computation and define $v_i(\bmx) = \gamma_i u_i(\bmx) - u(\bmx)$, we have
\begin{align*}
    \frac{\pp E}{\pp \bmx}(\bmx) =& \left(- \frac{\pp u}{\pp \bmx}(\bmx) + \frac{\pp u}{\pp \bmx}(\bmx^*)\right)
    \\
    \frac{\dd x_i}{\dd t}(t) =& \gamma_i \frac{\pp u_i}{\pp x_i}(\bmx(t))
    = \frac{\pp v_i}{\pp x_i}(\bmx(t)) + \frac{\pp u}{\pp x_i}(\bmx(t))
    = \frac{\pp v_i}{\pp x_i}(\bmx(t)) + \frac{\pp u}{\pp x_i}(\bmx^*) - \frac{\pp E}{\pp x_i}(\bmx(t))
\end{align*}

Next, we compute the derivative of $E(\bmx(t))$:
\begin{align*}
    \frac{\dd E}{\dd t}(\bmx(t))
   & =\langle \frac{\pp E}{\pp \bmx}(\bmx(t)), \frac{\dd \bmx}{\dd t}(t)\rangle
    \\
    =& - \| \frac{\pp E}{\pp \bmx}(\bmx(t)) \|^2 \cdots\text{first term}
    \\
    +& \langle \frac{\pp E}{\pp \bmx}(\bmx(t)), \frac{\pp u}{\pp \bmx}(\bmx^*) \rangle
    + \sum_\iinn \frac{\pp v_i}{\pp x_i}(\bmx(t)) \frac{\pp E}{\pp x_i}(\bmx(t)) \cdots\text{second term}
    \\
\end{align*}

% The first term satisfies,
% \begin{align*}
%     - \| \frac{\pp E}{\pp \bmx}(\bmx(t)) \|^2 \le -2c E(\bmx(t))
% \end{align*}


We also know $\frac{\pp u_i}{\pp x_i}(\bmx^*) = 0$ by definition of NE.
Combining them in the second term, we achieve,
\begin{align*}
    \text{second term} =& \sum_\iinn \frac{\pp E}{\pp x_i}(\bmx(t))\cdot \frac{\pp u}{\pp x_i}(\bmx^*)
    + \sum_\iinn \frac{\pp v_i}{\pp x_i}(\bmx(t)) \cdot \frac{\pp E}{\pp x_i}(\bmx(t))
    \\
    =& \sum_\iinn \frac{\pp E}{\pp x_i}(\bmx(t))\cdot \frac{\pp (u - \gamma_i u_i)}{\pp x_i}(\bmx^*)
    + \sum_\iinn \frac{\pp v_i}{\pp x_i}(\bmx(t)) \cdot \frac{\pp E}{\pp x_i}(\bmx(t))
    \\
    =& - \sum_\iinn \frac{\pp E}{\pp x_i}(\bmx(t))\cdot \frac{\pp v_i}{\pp x_i}(\bmx^*)
    + \sum_\iinn \frac{\pp v_i}{\pp x_i}(\bmx(t)) \cdot \frac{\pp E}{\pp x_i}(\bmx(t))
    \\
    =& \sum_\iinn \frac{\pp E}{\pp x_i}(\bmx(t))\cdot (\frac{\pp v_i}{\pp x_i}(\bmx(t)) - \frac{\pp v_i}{\pp x_i}(\bmx^*))
    \\
\end{align*}

Denote $\Delta v_i = \frac{\pp v_i}{\pp x_i}(\bmx(t)) - \frac{\pp v_i}{\pp x_i}(\bmx^*)$ and $\Delta \bmv = (\Delta v_1,...,\Delta v_n)$, we have,
\begin{align*}
    \text{second term} =& \langle \frac{\pp E}{\pp \bmx}(\bmx(t)), \Delta \bmv \rangle
    \\
    \le & \|\frac{\pp E}{\pp \bmx}(\bmx(t))\| \|\Delta \bmv\|
\end{align*}

We also know that $\frac{\pp v_i}{\pp x_i}$ is $\sigma_{ij}$-Lipschitz on $x_j$, now we consider $\| \Delta v_i\|$,
\begin{align*}
    \|\Delta v_i \| =& \max_{\|z \|=1} \sum_\iinn z_i \Delta v_i
    \\
    \le& \max_{\|z \|=1} \sum_\iinn\sum_\jinn z_i \sigma_{ij} |x_j(t) - x_j^*|
    \\
    \le& \| \bmx(t) - \bmx^* \| \max_{\|z \|=1, \| y\| =1} \sum_\iinn\sum_\jinn \sigma_{ij} z_i y_j
    \\
    =& \| \bmx(t) - \bmx^* \| \sigma_{max}(\Sigma)
    \\
    \le& \frac{\sigma_{max}(\Sigma)}{c} \| \frac{\pp E}{\pp \bmx}(\bmx(t))\|
\end{align*}

Combining these, we have,
\begin{align*}
    \frac{\dd E}{\dd t}(\bmx(t)) \le& -(1-\frac{\sigma_{max}(\Sigma)}{c}) \| \frac{\pp E}{\pp \bmx}(\bmx(t))\|^2
    \\
    \le& -2(c - \sigma_{max}(\Sigma)) E(\bmx(t))
\end{align*}

Take $p_0 = 2(c - \sigma_{max}(\Sigma))$, we complete the proof.


\end{proof}

Lemma \ref{lem:near-potential} can be deduced from the results in \citet{concave_game-initial:rosen1965existence} by the utilization of \emph{concave games} and \emph{diagonal strict concavity}.
This deduction relies on the technical assumption of the second-order differentiability of $u_i(\bmx)$'s and $u(\bmx)$. 
The proof for \cref{lem:near-potential} can be done by verifying whether the conditions in \citet{concave_game-initial:rosen1965existence} are satisfied, given the conditions in this lemma. We provide a proof sketch below, with the full derivation available in arXiv ???.
% The proof is not too technical and then the corresponding derivation is moved to appendix.
%For completeness, we also present a self-contained proof without the results of \citet{concave_game-initial:rosen1965existence}. The complete proof is also proved in appendix, only showing the proof sketch.

\begin{proof}[Proof Sketch of \cref{lem:near-potential}]
\citet{concave_game-initial:rosen1965existence} proved that diagonal strict concavity indicates the uniqueness of NE. He also proposed a sufficient condition for diagonal strict concavity is that $G(\bmx,\bmgg) + G^T(\bmx, \bmgg)$ is negative definite. Here, $G(\bmx, \bmgg)$ is the Jacobian of $g(\bmx, \bmgg)$ \wrt\ $\bmx$, $G^T$ is the transpose of matrix $G$, and $g(\bmx, \bmgg)$ is the vector $(\gamma_i \frac{\pp u_i}{\pp x_i}(\bmx))_\iinn$, representing the pseudo-gradient of game $(u_i(\bmx))_\iinn$.

By careful computation, we can express $G(\bmx, \bmgg)$ as
\begin{align*}
    G(\bmx, \bmgg) =& H(\bmx) + \begin{bmatrix}
    \frac{\pp^2 (\gamma_1 u_1 - u)}{\pp x_1^2} (\bmx) & \cdots & \frac{\pp^2 (\gamma_1 u_1 - u)}{\pp x_1 \pp x_n}(\bmx)
    \\
    \vdots & \ddots & \vdots 
    \\
    \frac{\pp^2 (\gamma u_n - u)}{\pp x_n x_1} (\bmx) & \cdots & \frac{\pp^2 (\gamma u_n - u)}{\pp x_n \pp x_n}(\bmx)
    \end{bmatrix}
    \\
    \triangleq& H(\bmx) + I(\bmx,\bmgg) 
\end{align*}
where $H(\bmx)$ is the Hessian matrix of $u(\bmx)$ \wrt\ $\bmx$, thus is $c$-negative definite. 
By the near-potential property of the game $(u_i(\bmx))_\iinn$, we can bound the $I(\bmx, \bmgg)$ by $\Sigma$, with the largest eigenvalue of $\Sigma + \Sigma^T$ less than $2c$. Therefore, $G(\bmx, \bmgg) +  G^T(\bmx, \bmgg)$ is negative definite, which completes the proof.
\end{proof}

% \begin{proof}[Proof Sketch of \cref{lem:near-potential}]\\

% \emph{Step 1: The uniqueness of NE.}
%  Consider the $\bmgg$-scaled pseudo-gradient ascent dynamic $\bmx(t)$, which satisfies the differential equations in \cref{def:BRD}. We aim at showing the dynamic is compressive. To prove this, we decompose $\frac{\dd \bmx(t)}{\dd t}$ into two parts: the first part portraits the dynamic on maximizing $u(\bmx)$, while the second part maximizes $\gamma_i u_i(\bmx) - u(\bmx)$. 

%  For the first part, by the technique in \cref{prf:thm:SW:BRD} we can show that the compressive factor is at least $c$, for $c$-concavity of $u(\bmx)$. 
%  For the second part, due to the near-potential property of the game $\{u_i(\bmx)\}_\iinn$, we could derive that the expansion factor is at most $\sigma_{max}(\Sigma)$.
%  Combining them, the $\bmgg$-scaled pseudo-gradient ascent dynamic is compressive overall. We then completes the proof by Banach's theorem.

%  \emph{Step 2: The exponential convergence rate.}
% We intend to construct an energy function $E(\bmx)$ such that: (1) $E(\bmx) \ge 0$, the equality holds iff $\bmx$ is a NE; (2) $\frac{\dd E(\bmx(t))}{\dd t}\le -c_0 E(\bmx(t))$ always holds.

% Combining them, we have $E(\bmx(t)) \le \exp(-c_0 t)E(\bmx(0))$ by standard differential equation analysis, which completes the proof.
%  \end{proof}

% We first show that, if $W$ is so small that the game is close to an individual-interest game, then the NE is unique.

Next, we will present three results of the uniqueness of NE under different conditions.

\begin{proof}
\label{prf:thm:NE:unique:near-individual}
We shall apply \cref{lem:near-potential} to prove this theorem.

Firstly, we construct a near-potential game by  specifying the potential function $u(\bmx)$ and the utilities $u_i(\bmx)$ for all players.

To do this, we let the potential function $u(\bmx) = \sum_\iinn \gamma_i \left( f_i(k_i) - c_i(x_i) \right)$, and specify the utilities $u_i(\bmx)$ in the near-potential game identical to the utilities in the public good game. With some straightforward calculations, we derive that
\begin{align*}
    \frac{\pp u_i}{\pp x_i}(\bmx) =& f'_i(k_i) - c'_i(x_i);
    \\
    \frac{\pp u}{\pp x_i}(\bmx) =& \sum_{i' \ne i} \gamma_{i'} f'_{i'}(k_{i'})w_{i'i} + \gamma_i \left( f'_i(k_i) - c'_i(x_i) \right).
\end{align*}

Since $f'_{i'}(k_{i'})$ is $L_0$-Lipschitz on $k_{i'}$, and $k_{i'}$ is $|w_{i'j}|$-Lipschitz on $x_j$, we have $\gamma_{i'} f'_{i'}(k_{i'})$ is $L_0 \gamma_{i'} |w_{i'j}|$-Lipschitz on $x_j$ and $\sum_{i'\ne i} \gamma_{i'} f'_{i'}(k_{i'}) w_{i'i}$ is $ L_0 \sum_{i'\ne i} \gamma_{i'} |w_{i'j} w_{i'i}|$-Lipschitz on $x_j$.

By constructing matrix $\Sigma = \{\sigma_{ij}\}_\ijinn$ with $\sigma_{ij} = L_0 \sum_\knei \gamma_k |w_{ki} w_{kj}|$, we can prove that $u_i(\bmx)$ is $\Sigma$-near-potential respect to $u(\bmx)$.
By \cref{lem:near-potential}, we obtain the result and thus complete the proof.
\end{proof}

\begin{proof}
\label{prf:thm:NE:unique:near-individual}
We shall apply \cref{lem:near-potential} to prove this theorem.

Firstly, we construct a near-potential game by  specifying the potential function $u(\bmx)$ and the utilities $u_i(\bmx)$ for all players.

To do this, we let the potential function $u(\bmx) = \sum_\iinn \gamma_i \left( f_i(k_i) - c_i(x_i) \right)$, and specify the utilities $u_i(\bmx)$ in the near-potential game identical to the utilities in the public good game. With some straightforward calculations, we derive that
\begin{align*}
    \frac{\pp u_i}{\pp x_i}(\bmx) =& f'_i(k_i) - c'_i(x_i);
    \\
    \frac{\pp u}{\pp x_i}(\bmx) =& \sum_{i' \ne i} \gamma_{i'} f'_{i'}(k_{i'})w_{i'i} + \gamma_i \left( f'_i(k_i) - c'_i(x_i) \right).
\end{align*}

Since $f'_{i'}(k_{i'})$ is $L_0$-Lipschitz on $k_{i'}$, and $k_{i'}$ is $|w_{i'j}|$-Lipschitz on $x_j$, we have $\gamma_{i'} f'_{i'}(k_{i'})$ is $L_0 \gamma_{i'} |w_{i'j}|$-Lipschitz on $x_j$ and $\sum_{i'\ne i} \gamma_{i'} f'_{i'}(k_{i'}) w_{i'i}$ is $ L_0 \sum_{i'\ne i} \gamma_{i'} |w_{i'j} w_{i'i}|$-Lipschitz on $x_j$.

By constructing matrix $\Sigma = \{\sigma_{ij}\}_\ijinn$ with $\sigma_{ij} = L_0 \sum_\knei \gamma_k |w_{ki} w_{kj}|$, we can prove that $u_i(\bmx)$ is $\Sigma$-near-potential respect to $u(\bmx)$.
By \cref{lem:near-potential}, we obtain the result and thus complete the proof.
\end{proof}

\begin{remark}
\label{rmk:NE:unique:near-individual}
The conditions in \cref{thm:NE:unique:near-individual} intuitively means that the players are close to playing an individual-interest game, \ie, the non-diagonal elements of $W$---those describe the interactions among different players---are small enough. In fact, from the expression of potential $u(\bmx) = \sum_\iinn \gamma_i u_i(\bmx)$, we know that the NE solution is close to the (weighted) social optimal solution.
\end{remark}

\begin{restatable}{theorem}{thmNENearPotential}
\label{thm:NE:unique:near-potential}
Given a public goods game $G = \langle \{f_i(k)\}_\iinn,$ $\{c_i(x)\}_\iinn, \{X_i\}_\iinn, W \rangle$. If the following conditions hold, 
%If the public good game $G = \langle \{f_i(k)\}_\iinn, \{c_i(x)\}_\iinn, \{X_i\}_\iinn, W \rangle$ holds following conditions, 
\myx{W is near potential}
\begin{itemize}
    \item[(1)] $f_i(k)$ is $(\gamma_i,\sigma_i)$-close to $f(k)$ for all $\iinn$;
    \item[(2)] $f(x + d) - \gamma_i c_i(x)$ is $c$-strongly concave on $x$ for all $\iinn$ and all $d \in [\dlow_i, \dhigh_i]$, and $f'(k)$ is $c^1$-Lipschitz on $k$, $f''(k)$ is $c^2$-Lipschitz on $k$, $c,c^1,c^2\in 
    \bbR_+$;
    \item[(3)] $c > \sigma_{max}(B)$, where $B = \{\beta_{ij}\}_\ijinn$ and $\beta_{ij} = \sigma_i |w_{ij}| + c^1 |w_{ij}-1| + c^2 \sum_\jinn |w_{ij}-1| \max\{ -\xlow_j, \xhigh_j \}$,
\end{itemize}
then the NE is unique.
\end{restatable}

\begin{remark}
\label{rmk:NE:unique:near-potential}
It's important to note that the conditions specified in \cref{thm:NE:unique:near-potential} intuitively suggest that each element of $W$ closely approximates $1$, and the values derived from the gains ${f_i(k)}_\iinn$ are nearly identical (when scaled). Consequently, the game approaches the characteristics of an identical-interest game, where the players' actions nearly maximize the potential function $u(\bmx) = f(\| \bmx \|_1) - \sum_\iinn \gamma_i c_i(x_i)$.
However, the social welfare is close to $n f(\| \bmx \|_1) - \sum_\iinn \gamma_i c_i(x_i)$, the $\frac{1}{n}$ coefficients on values means that in this case, the free-ride phenomenon can occur.
\end{remark}

\begin{restatable}{theorem}{thmNENearSymmetric}
\label{thm:NE:unique:near-symmetric}
Given a public goods game $G = \langle \{f_i(k)\}_\iinn,$ $\{c_i(x)\}_\iinn, \{X_i\}_\iinn, W \rangle$. If the following conditions hold, 
%If the public good game $G = \langle \{f_i(k)\}_\iinn, \{c_i(x)\}_\iinn, \{X_i\}_\iinn, W \rangle$ holds following conditions, 
\myx{W is near positive definite}
\begin{itemize}
    \item[(1)] $W^0$ is positive definite and $\sigma_{min}(W^0) = \sigma_0 > 0$. We also restrict $w^0_{ii} = 1,\ \forall \iinn$ where $W^0 = \{w^0_{ij}\}_\ijinn$;
    \item[(2)] $c'_i(x)$ is $L_i$-Lipschitz on $x$ for all $i$;
    \item[(3)] $f_i(k)$ is $C_i$-concave on $k$ for all $i$;
    \item[(4)] $\sigma_0 > \sigma_{max}(\Sigma)$, where $\Sigma = \{\sigma_{ij}\}_\ijinn$ and $\sigma_{ii} = 0$ and $\sigma_{ij} = \frac{2L_i |w_{ij}|}{C_i} + |w^0_{ij} - w_{ij}|$,
\end{itemize}
where $\sigma_{min}(W)$ represents the minimal eigenvalue of a symmetric matrix $W$, then the NE is unique.
\end{restatable}

\begin{proof}[Proof Sketch of \cref{thm:NE:unique:near-symmetric}]
\citet{public-network-direct-BRD:bayer2023best} proved that, when $W$ is symmetric and the cost functions $c_i(x)$s are linear, then the best-response dynamic converges. The insight is that when $c_i(x)$s are linear, each player $i$ has its own marginal cost $c_i$, and the ideal $k_i$ such that $f'_i(k_i) = c_i$. Therefore, each player $i$ plays the best response to her ideal gain $k_i$, and $\phi(\bmx) = \bmk^T \bmx - \frac{1}{2} \bmx^T W \bmx$ becomes a potential function. Moreover, the NE must be unique if $W$ is positive semi-definite.

Our proof follows this insight. We construct the potential function $\phi(\bmx) = \bmk^{*T}\bmx - \frac{1}{2}\bmx^T W^0 \bmx$. Similarly define $y_i(\bmx_{-i})$ as the optimal gain level of player $i$, when the strategy profile of other players is $\bmx_{-i}$. Then the utilities in the near-potential game are, 
\begin{align*}
    \phi_i(\bmx) = y_i(\bmx_{-i}) x_i - \frac{x_i^2}{2} - \sum_\jnei w_{ij} x_i x_j.
\end{align*}
We then prove that: (1) the NE of the near-potential game corresponds to the NE of the original public good game; and (2) the constructed game $\{\phi_i(\bmx)\}_\iinn$ is indeed a near-potential game.
Following these results, the proof can be completed by 
\cref{lem:near-potential}.
\end{proof}

\begin{remark}
Theorem \ref{thm:NE:unique:near-symmetric} intuitively suggests that, if $W$ is close to a positive definite matrix $W_0$, as well as that the profit functions $f_i(k)$s are more concave than cost functions $c_i(x)$s, then the NE is unique.
\end{remark}

In addition to these three theorems that establish the uniqueness of the NE, we introduce a concept, called \emph{game equivalence}, which can expand the applicability of these theorems.

\begin{definition}[Game Equivalence]
\label{def:equivalence}
Given two public goods games $G^1, G^2$ with $n$ players, where 
\begin{align*}
    G^j = (\{f^j_i\}_\iinn, \{c^j_i\}_\iinn, \{X^j_i = [\xlow^j_i, \xhigh^j_i]\}_\iinn, W^j), j \in \{1,2\}.
\end{align*}
$G^1$ is equivalent to $G^2$, if there is a diagonal matrix $D = \diag(d_1,...,d_n)$, $d_i\in \bbR_{++}$ and an offset vector $\bm{b} \in \bbR^n$, satisfying that, 
\begin{align*}
    W^2 =& D W^1 D^{-1};
    \\
    \xlow^2_i =& d_i \xlow^1_i + b_i;
    \\
    \xhigh^2_i =& d_i \xhigh^1_i + b_i;
    \\
    c^1_i(x) =& c^2_i(d_i x + b_i)\quad\forall x \in X^1_i;
    \\
    f^1_i(k) =& f^2_i(d_i k + m_i)\quad\forall k \in K^1_i,
\end{align*}
where $m_1,...,m_n$ are constants such that $m_i = d_i \sum_\jinn \frac{w^1_{ij} b_j}{d_j}$.

\end{definition}

Intuitively, Definition \ref{def:equivalence} states that if $G^1$ is equivalent to $G^2$, then $G^1$ and $G^2$ are intrinsically the same in terms of linear transformation. Through this insight, we have the following theorem.

\begin{restatable}{theorem}{thmNEEquivalence}
\label{thm:NE:unique:equivalence}
If two games, $G_1$ and $G_2$, are equivalent, then there exists a one-to-one mapping between NEs of $G_1$ and the NEs of $G_2$.
\end{restatable}

From \cref{thm:NE:unique:equivalence},it is evident that the uniqueness property of the NE is preserved within the equivalent class. Therefore, we present the following corollary, which further broadens the class of public goods games that have a unique NE.

\begin{corollary}
\label{cor:NE:unique:equivalence}
For a public goods game $G^1$, if $G^1$ is equivalent to the  game $G^2$, and $G^2$ satisfies the conditions in \cref{thm:NE:unique:near-individual}, \cref{thm:NE:unique:near-potential} or \cref{thm:NE:unique:near-symmetric}, then $G^1$ has a unique NE.
\end{corollary}


