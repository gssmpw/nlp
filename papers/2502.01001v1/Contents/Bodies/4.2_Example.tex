\subsection{Some Applications of Results}
\label{subsec:eg}

In this section, we propose a specific example to illustrate how the results regarding the uniqueness of NE can be applied in practice. This example is inspired by \citet{public-dynamic:fershtman1991dynamic}, where the cost functions are modeled as quadratic functions.

Specifically, we assume the homogeneity of players in the public good game $G$, i.e., the values of gains, costs of efforts, and action spaces are identical among players, with differences only in the network structure $W$. Therefore, we use $f(k)$ and $c(x)$ instead of $f_i(k_i)$ and $c_i(x_i)$ to represents values and costs, when the context allows. 

Assume $f(k)$ and $c(x)$ has following expression:
\begin{align*}
    f(k) =& 
    \begin{cases}
    ak - b k^2 \quad &\text{if $0\le k \le \frac{a}{2b}$}
    \\
    \frac{a^2}{4b} \quad &\text{if $k > \frac{a}{2b}$}
    \end{cases}
    \\
    c(x) =& \frac{c_0}{2}x^2\quad\qquad\quad \text{for $c_0>0$}
\end{align*}
and $X = [0,\xhigh]$ for a sufficiently large 
$\xhigh$ such that choosing $\xhigh$ is a dominated strategy for all players, due to extremely high costs and bounded values for gains. 
The values and costs are quadratic functions in their domains, with a clipping on the value function at the maximum point.
We also restrict $w_{ij}$ to be either $0$ or $1$.


%and $X = [0,\xhigh]$ for sufficiently large $\xhigh$ such that acting $\xhigh$ is a dominated strategy for all players (for the extremely large costs and bounded values for gains). We restrict $w_{ij}$ to be either $0$ or $1$.

From the expressions of $f(k)$ and $c(x)$, we know that $c(x)$ is $c_0$-strongly convex, $c'(x)$ is $c_0$-Lipschitz, $f(k)$ is $2b$-strongly concave in the domain $[0,\frac{a}{2b}]$ , and $f'(k)$ is $2b$-Lipschitz on the full domain.

\subsubsection{The Application of \cref{thm:NE:unique:near-individual}}
In this part, we assume that the non-diagonal elements of $W$ are $\iid$ generated with probability $p = \frac{p_0}{n}$ equals to $1$ and $0$ otherwise, where $p_0>0$ is a constant.
We have the following theorem,
\begin{proof}
\label{prf:thm:case:1}

We firstly substitute the model into the conditions of \cref{thm:NE:unique:near-individual}.
\begin{itemize}
    \item For the first condition, we just specify $\gamma_i \equiv 1$, then $f_i(x+d) - c_i(x)$ is $c$-concave.
    \item For the second condition, $f'_i(k)$ is $2b$-Lipschitz on $k$ for all $i$.
    \item For the third condition, we need that $\sigma_{max}(\Sigma) < \frac{c}{2b}$, where $\sigma_{ij} = \sum_\knei w_{ki}w_{kj}$.
\end{itemize}

It's well-known that $\sigma_{max}(\Sigma) \le \min\{ \| \Sigma \|_\infty , \| \Sigma \|_1 \}$, where $\| \Sigma \|_\infty$ ($\| \Sigma \|_1$) represents the $\infty$-norm (one-norm) of $\Sigma$, \ie, maximum row sum (column sum) of $\Sigma$, respectively. Specifically,
\begin{align*}
    \| \Sigma \|_\infty =& \max_{i} \sum_\jinn \sum_\knei w_{ki}w_{kj}
    \\
    =& \max_i \sum_\knei w_{ki} + \sum_\jnei \sum_\knei w_{ki} w_{kj}
    \\
    =& \max_i \sum_\knei w_{ki} + \sum_\jnei w_{ji} + \sum_\jnei \sum_{k\ne i,j} w_{ki} w_{kj}
    \\
    =& \max_i \quad 2\sum_\jnei w_{ji} + \sum_\jnei \sum_{k\ne i,j} w_{ki} w_{kj}
\end{align*}

Denote $\delta_i = 2\sum_\jnei w_{ij} + \sum_\jnei \sum_{k\ne i,j} w_{ki} w_{kj}$, then,
\begin{align*}
    \bbE_W [\delta_i] = 2(n-1)p + (n-1)(n-2)p^2 \le 2p_0 + p_0^2
\end{align*}
When the context is clear we denote $\bbE_W[\delta] = \bbE_W[\delta_i]$, since this term is constant.

We also compute the second-order moment of $\delta_i$, \ie,
\begin{align*}
    \delta_i^2
    =& \left( 2\sum_{j_1\ne i} w_{j_1,i} + \sum_{j_1\ne i, k_1\ne j_1, i} w_{j_1,i} w_{j_1,k_1} \right)^2
    \\
    =& 4 \sum_{j_1\ne i, j_2\ne i} w_{j_1,i}w_{j_2,i}
    + 4 \sum_{j_2\ne i}\sum_{j_1\ne i, k_1\ne j_1,i} w_{j_1,i}w_{j_1,k_1}w_{j_2,i}
    + \sum_{j_1\ne i, k_1\ne j_1, i}\sum_{j_2\ne i, k_2\ne j_2, i} w_{j_1,i}w_{j_2,i}w_{j_1,k_1}w_{j_2,k_2}
\end{align*}

By simple counting, we have,
\begin{align*}
    \bbE_W[\delta_i^2] = 4(n-1)p + 9(n-1)(n-2)p^2 + 6(n-1)(n-2)^2 p^3 + (n-1)(n-2)(n^2-5n+5) p^4
\end{align*}
and the variance of $\delta_i$,
\begin{align*}
    & \Var_W [\delta_i] = \bbE_W [\delta_i^2] - \bbE_W^2[\delta_i]
    \\
    =& 4(n-1)p + (n-1)(5n-14)p^2 + (n-1)(n-2)(2n-8)p^3 + (n-1)(n-2)(-2n+3)p^4
    \\
    \le& 4p_0 + 5p_0^2 + 2 p_0^3
\end{align*}

Combining them by using Chebyshev inequalities, 
\begin{align*}
    \Pr[\delta_i - \bbE[\delta] \ge k]\le \frac{\Var[\delta_i]}{k^2}
\end{align*}
and
\begin{align*}
    \Pr[\| \Sigma \|_\infty - \bbE[\delta] \ge k]\le \frac{n\Var[\delta_i]}{k^2}
\end{align*}
since $\| \Sigma \|_\infty = \max_\iinn \delta_i$.

To make $\mathrm{RHS} = \frac{1}{2}$, we take $k = \sqrt{n(8p_0 + 10 p_0^2 + 4 p_0^3)}$, therefore, with probability at least $\frac{1}{2}$, we have that $\sigma_{max}(\Sigma) \le \| \Sigma \|_\infty \le 2p_0 + p_0^2 + \sqrt{n(8p_0 + 10 p_0^2 + 4 p_0^3)< \frac{c}{2b}}$, and the game has unique NE by \cref{thm:NE:unique:near-individual}.
    
\end{proof}


\begin{remark*}
\label{rmk:thm:case:eg}

This proof is done by substituting \cref{thm:NE:unique:near-individual} and using Chebyshev's inequalities. Notice that $\sigma_{max}(\Sigma)$ can be bounded by the $\infty$-norm $\| \Sigma \|_\infty$, which is the maximum row sum of $\Sigma$.
We extract the sum of each row $i$ by $\gamma_i$, using Chebyshev's inequalities to bound the tail of $\gamma_i$ and union bound to control $\| \Sigma \|_\infty = \max_i \gamma_i$.

Notice that the result inevitably has a dependency on the square root of $n$ by Chebshev's inequality. 
Due to dependence between $\sigma_{ij}$ and $\sigma_{ij'}$, we can not directly use concentration inequalities, such as Chernoff's inequality \citep{chernoff:chernoff1952measure}, which can help decrease the dependency to $\log n$. 
However, we believe that the $\mathrm{poly} \log(n)$ dependency can be established, by the intrinsic independence on $\{w_{ij}\}_\ijinn$, which allows for further studies.

\end{remark*}

\subsubsection{The Application of \cref{thm:NE:unique:near-symmetric} and \cref{thm:NE:unique:equivalence}}

In this part, we assume that $W$ has a specific up-triangular structure, \ie, $w_{ij} = 0$ if $i>j$. Next, we will show that under this assumption, the NE of public good game is unique.

\begin{restatable}{theorem}{thmCaseTwo}
\label{thm:case:2}
If $W$ is an up-triangular matrix, \ie, $w_{ij}=0$ for $i>j$, then the public good game $G$ has a unique NE.
\end{restatable}



\begin{remark*}
In such scenarios, the conditions specified in \cref{thm:NE:unique:near-individual,thm:NE:unique:near-potential,thm:NE:unique:near-symmetric} may no longer be satisfied. However, we can employ the technique described in \cref{thm:NE:unique:equivalence} to transform the original game $G$ into another game $G'$
that meets the conditions outlined in \cref{thm:NE:unique:near-symmetric}.

Notice that this game must have a unique NE. It is because the following insight: since $w_{ij} $ for $i>j$ means that the efforts of players with lower identifiers $j$ have no externalities on players with higher identifiers $i$. Therefore, player $n$ is playing an individual-interest game, and thus has an optimal strategy $x_n^*$. Given $x_n^*$ fixed, player $n-1$ can also determine an optimal strategy $x_{n-1}^*$. Overall, each player can determine an optimal strategy in turn, which forms an equilibrium. 
However, our proof can give a stronger result that, if $w_{ij} = O(\varepsilon^{i+1-j})$\footnote{here $\varepsilon$ is a constant used in the proof} for $i>j$, we can also guarantee the uniqueness of NE.
\footnote{It is because after the transformation in the proof, the lower-triangular elements hold to be $O(\varepsilon)$.}

\end{remark*}


