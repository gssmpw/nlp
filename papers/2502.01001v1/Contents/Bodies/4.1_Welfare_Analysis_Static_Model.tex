\subsection{Comparative Statics: Money Redistribution for Welfare Analysis}
\label{subsec:comparative_money}

% In this section, we assume that $f''_i(k^*_i) > 0$ for model regularity. 

In this section, we study comparative statics, \ie, how the players' utilities will change if the model parameters are modified by an infinitesimal amount. 
We characterize the infinitesimal modification by money redistribution, \ie, replace $\{f_i(k_i)\}_\iinn$ by $\{f_i(k_i + \delta_i t)\}_\iinn$, where $\bmdd = (\delta_1,\dots, \delta_n)\in\bbR^n$ is called the direction of money redistribution and $t\in \bbR$ is called the change magnitude. 
Overall, there is a $\bmdd t$ shift in the gain level of players.
The goal of infinitesimal change drives us to study the case $t\to 0$.

% Let $\bmdd$ be an infinitesimal change of money redistribution, \ie, the direction of money change. We model the change magnitude by $t > 0$.
In this way, the utility of player $i$ becomes
\begin{eqnarray*}
  u_i(\bmx;t) = f_i(k_i + \delta_i t) - c_i(x_i).   
\end{eqnarray*}

Denote $\bmx^*(t)$ as the NE when the change magnitude is $t$. We do not assume the uniqueness of NE anymore, and $\bmx^*(t)$ might be not unique. However, we assume the first-order differentiability of $\bmx^*(t)$ with respect to $t$, as well as that $\bmx^*(t)$ is an inner point of $X$. 
These assumptions are quite natural. 
For the first assumption, if the game changes with an infinitesimal magnitude and players always achieve the rational outcome, \ie, NE, then it is imaginable and intuitive that the outcome of players should also change minimally.
The second assumption is only technical.
We denote $u_i(t) = u_i(\bmx^*(t);t)$ for a little abuse of notation when the context is clear. We are mainly concerned about $u'_i(0)$, which means that what the marginal change of $\delta$ would affect the players' utilities. Thus, we have the following result.

%We build this result as follows.

\begin{proof}
\label{prf:thm:comparative:money}
$\bmx^*(t)$ should satisfy,
\begin{equation}
\label{eq:comparative:money:NE}
    x^*_i(t) = \argmax_{x_i} f_i(x_i + \sum_\jnei w_{ij} x^*_j(t) + \delta_i t) - c_i(x_i),
\end{equation}

By \cref{eq:comparative:money:NE}, we have
\begin{align*}
    c'_i(x^*_i(t)) - f'_i(k^*_i(t) + \delta_i t) = 0,
\end{align*}
which carves out an implicit function
\begin{align*}
    F_i(\bmx^*(t), t) = 0,\quad \forall \iinn
\end{align*}
with $F_i(\bmx,t) = c'_i(x_i) - f'_i(k_i + \delta_i t)$. Take $E_i(t) = F_i(\bmx^*(t), t)$, by implicit function theorem, we have

\begin{equation}
\label{eq:comparative:money:implicit}
    \frac{\dd E_i}{\dd t}(t) = \frac{\pp F_i}{\pp \bmx}(\bmx^*(t),t) \frac{\dd \bmx^*}{\dd t}(t) + \frac{\pp F_i}{\pp t}(\bmx^*(t),t) = 0
\end{equation}

Together \cref{eq:comparative:money:implicit} with all $i$, we have 
\begin{align*}
    \frac{\dd \bmx^*}{\dd t}(t) = - \left( \frac{\pp F}{\pp \bmx} \right)^{-1} \frac{\pp F}{\pp t}(\bmx^*(t),t)
\end{align*}
where $F(\bmx^*(t),t) = (F_1(\bmx^*(t),t), F_2(\bmx^*(t),t), \dots, F_n(\bmx^*(t),t))$, by computation we have,
\begin{align*}
    & \frac{\pp F}{\pp t}(\bmx^*(t),t) = -\diag(\bmf''(\bmk^*(t) + t\bmdd)) \bmdd
    \\
    & \frac{\pp F}{\pp \bmx}(\bmx^*(t),t) = \diag(\bmc''(\bmx^*(t))) - \diag(\bmf''(\bmk^*(t) + t\bmdd))W
\end{align*}

By $\bmu(t) = \bmu(x^*(t);t)$, we derive that,
\begin{align*}
    \bmu'(0) = \frac{\pp \bmu}{\pp t}(\bmx^*;0) + \frac{\pp \bmu}{\pp \bmx}(\bmx^*;0) \frac{\dd \bmx^*}{\dd t}(0)
\end{align*}
where
\begin{align*}
    & \frac{\pp \bmu}{\pp t}(\bmx^*;0) = \diag(\bmf'(\bmk^*)) \bmdd
    \\
    & \frac{\pp \bmu}{\pp \bmx}(\bmx^*;0) = \diag(\bmf'(\bmk^*)) W - \diag(\bmc'(\bmx^*))
    \\
    & \frac{\dd \bmx^*}{\dd t}(0) = \left( \diag(\bmc''(\bmx^*)) - \diag(\bmf''(\bmk^*))W \right)^{-1} \diag(\bmf''(\bmk^*)) \bmdd
\end{align*}

By equilibrium condition, we have
\begin{align*}
    \diag(\bmc'(\bmx^*)) = \diag(\bmf'(\bmk^*))
\end{align*}
and thus
\begin{align*}
    \frac{\pp \bmu}{\pp \bmx}(\bmx^*;0) = \diag(\bmf'(\bmk^*))(W - I)
\end{align*}

Above all,
\begin{align*}
    \bmu'(0) =& \diag(\bmf'(\bmk^*)) \left( \bmdd + (W-I) \left( \diag(\bmc''(\bmx^*)) - \diag(\bmf''(\bmk^*))W \right)^{-1} \diag(\bmf''(\bmk^*)) \bmdd \right)
    \\
    =& \diag(\bmf'(\bmk^*)) \left( I + (W-I) \left( \diag(\bmc''(\bmx^*)) - \diag(\bmf''(\bmk^*))W \right)^{-1} \diag(\bmf''(\bmk^*)) \right) \bmdd
    \\
    =& \diag(\bmf'(\bmk^*)) \left( I + (W-I) \left( \diag(\bmc''(\bmx^*)/\bmf''(\bmk^*)) - W \right)^{-1} \right) \bmdd
    \\
    =& \diag(\bmf'(\bmk^*)) \left( \left( \diag(\bmc''(\bmx^*)/\bmf''(\bmk^*)) - W \right)\cdot \left( \diag(\bmc''(\bmx^*)/\bmf''(\bmk^*)) - W \right)^{-1}
    \right.
    \\
    +& \left. (W-I) \left( \diag(\bmc''(\bmx^*)/\bmf''(\bmk^*)) - W \right)^{-1} \right) \bmdd
    \\
    =& \diag(\bmf'(\bmk^*)) \left( (\diag(\bmc''(\bmx^*)/\bmf''(\bmk^*))-I) \left( \diag(\bmc''(\bmx^*)/\bmf''(\bmk^*)) - W \right)^{-1} \right) \bmdd
    \\
    =& \diag(\bmf'(\bmk^*)) \cdot \diag(\bmc''(\bmx^*) - \bmf''(\bmk^*)) \cdot \left[ \diag(\bmc''(\bmx^*)) - W \diag(\bmf''(\bmk^*)) \right]^{-1} \bmdd
\end{align*}
which completes the proof.

\end{proof}

We also demonstrate some examples to illustrate the implications of the result in \cref{thm:comparative:money}.
\begin{example}
\label{eg:comparative:money:simple}
Here are some simple cases of \cref{thm:comparative:money}.
\begin{enumerate}
\item If the value function is linear on gain, \ie, $f''_i(k) \equiv 0$, then, it becomes that
\begin{align*}
    \bmu'(0) = \diag(\bmf'(\bmk^*)) \bmdd.
\end{align*}
This result is intuitive, because a linear value function indicates that the NE is unique and is constant since the marginal values for players' efforts are constants and marginal costs only depend on players' strategies. Therefore, the change of money redistribution has a direct change on the utilities.

\item If the cost function is linear on effort, \ie, $c''_i(x) \equiv 0$, then, it becomes that
\begin{align*}
    \bmu'(0) = \diag(\bmf'(\bmk^*)) W^{-1} \bmdd.
\end{align*}

In this case, NE might be not constant and not unique (see \cref{eg:NE:nonunique}). Therefore, the redistribution of money will affect the interactions of players, and thus have an indirect effect on the utilities. Specifically, the indirect effect imposes the inverse of $W$---the matrix that portrays the interactions of players---to the money redistribution $\bmdd$.

\item If we want the money redistribution to be Pareto dominant, \ie, $u'_i(0) \ge 0$ for all players, since the first two diagonal matrices are positive diagonal matrices, the only requirement of $\bmdd$ is:
\begin{align*}
    \left[ \diag(\bmc''(\bmx^*)) - W \diag(\bmf''(\bmk^*)) \right]^{-1} \bmdd \ge \zeros.
\end{align*}

Besides, a linear cost would reduce the requirements to,
\begin{align*}
    W^{-1}\bmdd \ge \zeros.
\end{align*}

\end{enumerate}

\end{example}

% \subsection{Prosociality}

% In this section, we study the effect of prosociality on the players' utilities. We denote $V = \{v_{ij}\}_\ijinn$ as the infinitesimal change of prosociality, \ie, the direction of prosociality change. $v_{ij},\ \inej$ represnets the altruistic degree of player $i$ on player $j$. We normalize that $v_{ii} = 0$ and we admit the case that $v_{ij}<0$. We also model the change magnitude by $t > 0$. In this way, the utility of player $i$ becomes,
% \begin{align*}
%     u_i(\bmx;t) = f_i(k_i) - c_i(x_i) + \sum_\jnei (v_{ij} t) f_j(k_j) ,
% \end{align*}

% The $u_i(t)$ and $\bmx^*(t)$ is similarly defined. We have following results,

% \begin{theorem}
% \label{thm:comparative:prosociality}
% \begin{align*}
%     \bmu'(0) = \myx{TBA}
% \end{align*}
% \end{theorem}
