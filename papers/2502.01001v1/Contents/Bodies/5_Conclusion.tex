\section{Conclusion}
\label{sec:conclusion}
%In this paper, we have presented a novel approach to understanding networked public goods games featuring heterogeneous players and convex cost functions. Through rigorous analysis and theoretical explorations, we have expanded the conventional understanding of strategic interactions in public goods provision within networked environments. Our model, which integrates heterogeneous benefits and convex costs, provides a more realistic portrayal of individual contributions and the resultant dynamics compared to traditional models with linear and homogeneous cost structures.

% The introduction of convex cost functions into the networked public good games has allowed us to capture the increasing marginal costs associated with higher levels of effort, which is a common scenario in real-world settings. This enhancement not only makes the model more practical but also introduces a variety of strategic considerations for the players involved. Our findings on the existence and uniqueness of Nash Equilibrium (NE) contribute significantly to the literature, offering new insights into how individual strategies stabilize in the presence of complex and realistic cost structures. The concept of game equivalence, introduced within this paper, extends the utility of our results by enabling the application of these findings to a broader class of public goods games, thereby enhancing their relevance and applicability.

In this paper, we have presented a novel approach to understanding networked public goods games featuring heterogeneous players and convex cost functions. Through rigorous analysis and theoretical explorations, we have expanded the conventional understanding of strategic interactions in public goods provision within networked environments. Our model, which integrates heterogeneous benefits and convex costs, provides a more realistic portrayal of individual contributions and the resultant dynamics compared to traditional models with linear and homogeneous cost structures.

The theoretical insights and methodological contributions of our study on networked public goods games with heterogeneous players and convex costs fill a significant gap in the economic theory and also provide new perspectives for policymakers. Specifically, by understanding the conditions under which Nash Equilibrium can be achieved and sustained, policymakers can better design interventions and incentives in the context of the Internet economy and social networks, that encourage optimal contribution levels to public goods. In future research, it would be valuable to extend this model to consider dynamic environments, where players can adjust their strategies over time. Additionally, incorporating stochastic elements to model uncertainty in player interactions could provide further insights into the robustness of the model in more complex and realistic scenarios.


 


%In summary, our exploration into networked public goods games with heterogeneous players and convex costs not only fills a significant gap in the economic theory of public goods but also sets the stage for future research in this area. In future research, it would be valuable to extend this model to consider dynamic environments, where players can adjust their strategies over time. Additionally, incorporating stochastic elements to model uncertainty in player interactions could provide further insights into the robustness of the model in more complex and realistic scenarios.


