\begin{definition}[$c$-concavity]
\label{def:concave}
A differentiable function $g(x): X \to \bbR$ is $c$-(strongly) concave ($c \in \bbR_+$) on $x \in X \subseteq \bbR^d$, if $X$ is a convex set and,
% \begin{align*}
%     f(\lambda x + (1-\lambda)y) \ge \lambda f(x) + (1-\lambda) f(y) - \frac{c}{2} \lambda (1-\lambda) \|x-y\|^2
% \end{align*}
% for all $x,y \in X$ and $\lambda \in [0,1]$.
\begin{equation}
\label{eq:concave}
    g(y) \le g(x) + \langle y-x, \nabla g(x)\rangle - \frac{c}{2}\| y - x \|^2, \quad \forall x,y\in X.
\end{equation}
% \begin{align*}
%     g''(s) \le -c
% \end{align*}
% always holds, where $g: \bbR \to \bbR$ is a function defined as $g(s) = f(x + sv)$ for any $x\in X$ and any unit vector $v\in \bbR^d$ with $\| v \| = 1$.
\end{definition}