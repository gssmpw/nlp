\def\comment#1{}
\begin{abstract}
%The abstract should briefly summarize the contents of the paper in
%150--250 words.



In the digital age, resources such as open-source software and publicly accessible databases form a crucial category of digital public goods, providing extensive benefits for Internet. 
However, these public goods' inherent non-exclusivity and non-competitiveness frequently result in under-provision, a dilemma exacerbated by individuals' tendency to free-ride. This scenario fosters both cooperation and competition among users, leading to the public goods games.

This paper investigates networked public goods games involving heterogeneous players and convex costs, focusing on the characterization of Nash Equilibrium (NE). In these games, each player can choose her effort level, representing her contributions to public goods.
% —not only impacts their utility but also affects others, reflecting the communal nature of the efforts.
Network structures  are employed to model the interactions among participants. Each player's utility consists of a concave value component, influenced by the collective efforts of all players, and a convex cost component, determined solely by the individual's own effort. 
To the best of our knowledge, this study is the first to explore the networked public goods game with convex costs.

% We establish the presence of NE in this model and provide an in-depth analysis of the conditions under which NE is unique. 

Our research begins by examining welfare solutions aimed at maximizing social welfare and ensuring the convergence of pseudo-gradient ascent dynamics. We establish the presence of NE in this model and provide an in-depth analysis of the conditions under which NE is unique. 
% Additionally, we introduce the concept of game equivalence, which expands the range of public goods games that can support a unique NE.
We also delve into \emph{comparative statics}, an essential tool in economics, to evaluate how slight modifications in the model—interpreted as monetary redistribution—affect player utilities. In addition, we analyze a particular scenario with a predefined game structure, illustrating the practical relevance of our theoretical insights.
% Additionally, we explore a fully heterogeneous version of the game, in which each player's utility is characterized by distinct value and cost functions.
Overall, our research enhances the broader understanding of strategic interactions and structural dynamics in networked public goods games, with significant implications for policy design in internet economic and social networks.




% We start at investigating welfare solutions, focusing on social welfare maximization and the convergence of best-response dynamics. 
% We establish the existence of Nash Equilibrium (NE) for this model, and carefully analyze the uniqueness of Nash Equilibrium (NE) under different conditions.
% Additionally, we propose the concept of game equivalence, which further broaden the class of public good games with unique NE.
% We also explore comparative statics to understand the impact of infinitesimal shift of the model, which we consider as money redistribution, on the utilities of players involved. 
% Furthermore, we studies a specific example, in which the game structure is predefined, showing how these results can be applied into this example.
% Our findings contribute to the broader understanding of strategic interactions and structures in networked public goods games, with implications for policy design in economic and social networks.
% \keywords{Public Good Games \and Networks \and Nash Equilibrium \and Social Welfare \and Best-Response Dynamics \and Comparative Statics.}
\end{abstract}

\keywords{Public Good Games; Networks; Nash Equilibrium; Social Welfare; Pseudo-Gradient Dynamics; Comparative Statics.}