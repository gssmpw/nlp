\begin{example}
\label{eg:NE:nonunique}
Consider a public good game containing four players, see \cref{fig:NE:nonunique}-(a) . The marginal gain is $1$ between one player from the left side and the other from the right side, and $0$ otherwise. We specify homogeneous utility functions and action spaces for all players. The action space is specified as $[0,1]$, while the only two constraints for utility functions are: 
\begin{align*}
    f'(1) \ge c'(1)~~\mbox{and}~~
    f'(2) \le c'(0).
\end{align*}
It's straightforward to verify that players on one side exert full effort, i.e., $x_i = 1$,
while players on the opposite side free ride, i.e., $x_i = 0$, constitutes a Nash Equilibrium (NE). Thus, there are at least two NEs in this game. This example can be readily extended to a scenario involving $n_1\times n_2$ players, distributed into $n_1$ groups with $n_2$ players in each group. All pairs of players from different groups are connected, see \cref{fig:NE:nonunique}-(b). The second condition then becomes $f'(n_2) \le c'(0)$.
%It's easy to verify that players in one side assert full efforts, \ie, $x_i = 1$, while players in the other side free ride, \ie, $x_i = 0$, constitutes an NE. Therefore there are at least two NEs in this game.
%This example can be easily extended to the case with $n_1 n_2$ players, with $n_1$ groups and $n_2$ players in each groups. The edges are connected for all pairs of players from different groups. The second condition becomes $f'(n_2) \le c'(0)$.
\end{example}