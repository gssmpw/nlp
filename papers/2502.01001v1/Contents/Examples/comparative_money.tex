\begin{example}
\label{eg:comparative:money:simple}
Here are some simple cases of \cref{thm:comparative:money}.
\begin{enumerate}
\item If the value function is linear on gain, \ie, $f''_i(k) \equiv 0$, then, it becomes that
\begin{align*}
    \bmu'(0) = \diag(\bmf'(\bmk^*)) \bmdd.
\end{align*}
This result is intuitive, because a linear value function indicates that the NE is unique and is constant since the marginal values for players' efforts are constants and marginal costs only depend on players' strategies. Therefore, the change of money redistribution has a direct change on the utilities.

\item If the cost function is linear on effort, \ie, $c''_i(x) \equiv 0$, then, it becomes that
\begin{align*}
    \bmu'(0) = \diag(\bmf'(\bmk^*)) W^{-1} \bmdd.
\end{align*}

In this case, NE might be not constant and not unique (see \cref{eg:NE:nonunique}). Therefore, the redistribution of money will affect the interactions of players, and thus have an indirect effect on the utilities. Specifically, the indirect effect imposes the inverse of $W$---the matrix that portrays the interactions of players---to the money redistribution $\bmdd$.

\item If we want the money redistribution to be Pareto dominant, \ie, $u'_i(0) \ge 0$ for all players, since the first two diagonal matrices are positive diagonal matrices, the only requirement of $\bmdd$ is:
\begin{align*}
    \left[ \diag(\bmc''(\bmx^*)) - W \diag(\bmf''(\bmk^*)) \right]^{-1} \bmdd \ge \zeros.
\end{align*}

Besides, a linear cost would reduce the requirements to,
\begin{align*}
    W^{-1}\bmdd \ge \zeros.
\end{align*}

\end{enumerate}

\end{example}