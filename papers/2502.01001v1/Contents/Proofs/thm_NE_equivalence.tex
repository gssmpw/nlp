\begin{proof}
\label{prf:thm:NE:unique:equivalence}

We prove this theorem by reduction, \ie, there is an injective function $g: X^1 \to X^2$ such that if $\bmx^1$ is an NE in $G^1$, then $\bmx^2$ is an NE in $G^2$.

We construct $\bmx^2$ by following,
\begin{align*}
    x^2_i = d_i x^1_i + b_i
\end{align*}
The construction is injective.
Therefore, we have 
\begin{align*}
    k^2_i =& \sum_\jinn w^2_{ij} x^2_j = \sum_\jinn \frac{d_i}{d_j}w^1_{ij} (d_j x^1_j + b_j)
    \\
    =& d_i \sum_\jinn w^1_{ij} x^1_j + d_1 \sum_\jinn \frac{w^1_{ij} b_j}{d_j}
    \\
    =& d_i k^1_i + m_i
\end{align*}

Now fix $\bmx^2_{-i}$, consider the case that player $i$ choose action $x^2_i$:
\begin{align*}
    u^2_i(x^2_i, \bmx^2_{-i}) =& f^2_i(k^2_i) - c^2_i(x^2_i)
    \\
    =& f^2_i(d_i k^1_i + m_i) - c^2_i(d_i x^1_i + b_i)
    \\
    =& f^1_i(k^1_i) - c^1_i(x^1_i)
\end{align*}
which is the maximum utility player $i$ can achieve, since $\bmx^1$ is an NE of $G^1$.
Therefore, $\bmx^2$ is an NE of $G^2$.

We also need to prove that the inverse direction also holds, to clarify this statements, we show that the equivalence relation is symmetric, \ie, if $G^1$ is equivalent to $G^2$, then $G^2$ is also equivalent to $G^1$.

To show this, we let $d'_i = 1/d_i$ and $b'_i = -b_i/d_i$, we have $d'_i \in \bbR_{++}$ and $b'_i \in \bbR$. 
Denote $D' = D^{-1} = \diag(d'_1,...,d'_n)$, then we have
\begin{align*}
    W^1 =& D' W^2 D'^{-1}
    \\
    \xlow^1_i =& d'_i \xlow^2_i + b'_i
    \\
    \xhigh^1_i =& d'_i \xhigh^2_i + b'_i
    \\
    c^2_i(x) =& c^1_i(d'_i x + b'_i)\quad\forall x \in X^2_i
    \\
    f^2_i(k) =& f^1_i(d'_i k + m'_i)\quad\forall k \in K^2_i
\end{align*}
for some constants $\{m'_i\}_\iinn$.
This indicates that an NE of $G^2$ also corresponds to an NE of $G^1$, which completes the proof.

\end{proof}