\begin{proof}
\label{prf:lem:concave}
The case  $c_0 = 0$ is trivial. 
Consider the case for $c_0 > 0$, if we take maximum on both sides of \cref{eq:concave}, we have
\begin{align*}
    g(x^*) =& \max_{y\in X}\quad g(y)
    \\
    \le& \max_{y\in \bbR^d}\quad g(x) + \langle y-x, \nabla g(x)\rangle - \frac{c_0}{2}\| y - x\|^2
\end{align*}

The maximum of RHS is achieved at $y^* = x + \frac{1}{c_0}\nabla g(x)$, and therefore,
\begin{align*}
    g(x^*) \le& g(x) + \langle y^* - x, \nabla g(x)\rangle - \frac{c_0}{2}\| y^* - x \|^2
    \\
    =& g(x) + \frac{1}{2c_0} \| \nabla g(x) \|^2
\end{align*}
which completes the proof.
\end{proof}