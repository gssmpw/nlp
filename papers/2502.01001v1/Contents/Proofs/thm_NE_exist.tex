\begin{proof}
\label{prf:thm:NE:exist}

We first consider the case where all $c_i(x_i)$s are $c$-strongly concave for some $c>0$, in which case best responses of players are always unique and continuous. We will generalize the result to the general case in the second step.

\paragraph{Case 1: Strongly Convex Cost.}
Consider the best response function of players: $\BR:
\times_\iinn X_i \to \times_\iinn X_i$, where $\BR_i(\bmx)$ represents the best response of player $i$ given the effort profile $\bmx_{-i}$, omitting the dummy variable $x_i$. 

Consider the utility function of player $i$:
\begin{align*}
    u_i(\bmx) = f_i\left(\sum_\jinn w_{ij}x_j\right) - c_i(x_i)
    \\
    \BR_i(\bmx) = \arg\max_{x'_i} u_i(x'_i, \bmx_{-i})
\end{align*}

Now we will show the continuity of $\BR_i(\bmx)$. We assume the negation holds. It indicates that there are two sequences $\{\bmx^j_k\}_{k\in\bbN_+}, j\in\{1,2\}$ such that $\lim \bmx^1_k = \lim \bmx^2_k$ but $\lim \BR_i(\bmx^1_k) \ne \lim \BR_i(\bmx^2_k)$ or one of the limitations do not exist. If the latter holds, since the compactness of $X$ we can choose a sub-sequence of $\{\bmx^j_k\}$ such that the limitation of $\BR_i(\bmx^j_k)$ exists for $j=1,2$. Therefore we only consider the former case.

Let $d = \| \lim \BR_i(\bmx^1_k) - \lim \BR_i(\bmx^2_k) \|$. 
By optimality of $\BR_i(\bmx^j_k)$, we have
\begin{align*}
    u_i(\BR_i(\bmx^j_k), \bmx^j_{k,-i}) \ge u_i(\BR_i(\bmx^{-j}_k, \bmx^j_{k,-i}) + \frac{c}{2} \| \BR_i(\bmx^j_k) - \BR_i(\bmx^{-j}_k)\|^2,\quad j = 1,2
\end{align*}

Summing up with $j=1,2$, we have
\begin{align*}
    \sum_{j=1}^2 \left[ u_i(\BR_i(\bmx^j_k), \bmx^j_{k,-i}) - u_i(\BR_i(\bmx^{j}_k), \bmx^{-j}_{k,-i}) \right] \ge c \| \BR_i(\bmx^1_k) - \BR_i(\bmx^2_k)\|^2
\end{align*}

Then taking limits of $k\to \infty$, we know that LHS becomes $0$ and RHS becomes $c d^2 > 0$, which leads to a contradiction.


% By little computation,
% \begin{align*}
%     & \frac{\pp^2 f_i}{\pp x_i \pp x_j} = f''(k_i)w_{ij}
%     \\
%     & \frac{\pp^2 f_i}{\pp x_i \pp \bmx_{-i}} = f''(k_i) \bmw_{i,-i}
%     \\
%     & \sup_{|r_i| = 1, \| \bmr_{-i} \| = 1} r_i \frac{\pp^2 f_i}{\pp x_i \pp \bmx_{-i}} \bmr_{-i} = |f''(k_i)| \cdot \| \bmw_{i,-i} \|
% \end{align*}

% Denote $\khigh_i = \sum_\jinn \max\{ w_{ij} \xhigh_j, -w_{ij} \xlow_j\}$, where $\max\{ w_{ij} \xhigh_j, -w_{ij} \xlow_j\}$ is the maximum externalities of player $j$'s effort on player $i$'s gain. 
% Similarly, denote $\klow_i = \sum_\jinn \min\{ w_{ij} \xlow_j, -w_{ij} \xhigh_j\}$, where $\min\{ w_{ij} \xlow_j, -w_{ij} \xhigh_j\}$ is the minimum externalities of player $j$'s effort on player $i$'s gain. 
% We know that $k_i$ is bounded in $[\klow_i, \khigh_i]$ for all $i$ and all profile $\bmx$, and $|f''(k_i)|$ is also bounded (and continuous). Therefore, there is a universal $L_i>0$ such that $\| \frac{\pp^2 f_i}{\pp x_i \pp \bmx_{-i}}(\bmx) \| \le L_i$.

% By utilizing the results of \cref{lem:argmax:Lipschitz}, we know that $\BR_i(\bmx)$ is $\frac{2L_i}{c}$-Lipschitz continuous with respect to $\bmx$, and thus $\BR(\bmx)$ is $\frac{2\sqrt{\sum_\iinn L_i^2}}{c}$-Lipschitz continuous with respect to $\bmx$, thus continuous on $\bmx$. 

Since $\times_\iinn X_i$ is a bounded convex set, by Brouwer's fixed-point theorem, we know that there exists $\bmx \in X$ such that $\BR(\bmx) = \bmx$, which indicates that $\bmx$ is an NE.

\paragraph{Case 2: General Convex Cost.}

To deal with the case that $c_i(x_i)$ might be not strongly concave, we use the technique of utility reshaping. Specifically, we define another public good game $G^\beta = (\{f_i\}_\iinn, \{c^\beta_i\}_\iinn \{X_i\}_\iinn, W)$ where $\beta > 0$ and $c^\beta(x_i) = c(x_i) + \beta x_i^2$. It's obvious that in a public good game $G^\beta$, the cost functions are $\beta$-strongly concave, and the NE must exist.

We denote an NE of $G^\beta$ as $\bmx^\beta$. The next step is to construct a strategy profile $\bmx$, from $\bmx^\beta$ for $\beta >0$, such that $\bmx$ is an NE of $G$. Notice that a simple limit may not work, since there is no guarantee that $\bmx^\beta$ is continuous with $\beta$, even that $\bmx^\beta$ might be unmeasurable in the usual sense.

To resolve this issue, we notice that $\bmx^\beta \in X$ where $X$ is a compact set. By the Bolzano-Weierstrass theorem, we know that there exists a convergent subsequence $\bmx^{\beta_k} \to \bmx$ for some $\bmx \in X$ and  $ \beta_k \overset{k \to \infty}{\longrightarrow} 0$, $k\in\bbZ_+$.

Finally, we verify the NE property of $\bmx$. Notice that $X$ is compact again, we know that $c^{\beta_k}(x_i)$ converges to $c(x_i)$ consistently, therefore, $u^\beta_i(\bmx) = f_i(k_i) - c^{\beta_k}_i(x_i)$ also converges to $u_i(\bmx)$ consistently. Taking limits on both sides of the following equality,
\begin{align*}
    u^{\beta_k}_i(\bmx^{\beta_k}) =& \max_{x'_i\in X_i} u^{\beta_k}_i(x'_i,\bmx^{\beta_k}_{-i}),
\end{align*}
we achieve that, 
\begin{align*}
    u_i(\bmx) =& \max_{x'_i\in X_i} u_i(x'_i,\bmx_{-i}),
\end{align*}
which indicates that $\bmx$ is an NE of $G$.

\end{proof}
