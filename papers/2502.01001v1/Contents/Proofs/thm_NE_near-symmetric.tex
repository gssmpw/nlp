\begin{proof}
\label{prf:thm:NE:unique:near-symmetric}

\citet{public-network-direct-BRD:bayer2023best} shows that, when $W$ is symmetric and the cost functions $c_i(x)$s are linear, then the best-response dynamic converges. The insight is that when $c_i(x)$s are linear, each player $i$ has its own marginal cost $c_i$, and the ideal $k_i$ such that $f'_i(k_i) = c_i$. Therefore, every player $i$'s best-response to her ideal gain $k_i$, and $\phi(\bmx) = \bmk^T \bmx - \frac{1}{2} \bmx^T W \bmx$ becomes a potential function. Moreover, the NE must be unique if $W$ is positive semi-definite.

Our proof follows this insight and tries to utilize the conclusion of \cref{lem:near-potential}. By \cref{thm:NE:exist} we know that there is an NE $\bmx^*$, let $\bmk^* = W \bmx^*$ be the gain profile in the equilibrium level. We construct the potential following,
\begin{align*}
    u(\bmx) = \bmk^{*T} \bmx - \frac{1}{2} \bmx^T W_0 \bmx
\end{align*}
It's easy to observe that $u(\bmx)$ is $\sigma_0$-strongly concave.

Now consider $y_i(\bmx_{-i})$ as the best response function of player $i$, \ie, $y_i(\bmx_{-i})$ is the gain level $k_i$ such that, it's optimal for player $i$ to choose the effort level $x_i$ such that her gain level becomes $k_i$. 
If we define $u^0_i(k_i,\bmx_{-i})$ is the utility of player $i$ when other players play $\bmx_{-i}$ and the gain level of player $i$ is $k_i$, we can write that,
\begin{align*}
    u^0_i(k_i,\bmx_{-i}) =& f_i(k_i) - c_i(k_i - \sum_\jnei w_{ij} x_j)
    \\
    =& (- c^0_i(k_i, \bmx_{-i})) + f_i(k_i)
\end{align*}
where $c^0_i(k_i, \bmx_{-i}) = c_i(k_i - \sum_\jnei w_{ij} x_j)$ is the cost function of player $i$ (in another form).

% Notice that
% \begin{align*}
%     u_i(\bmx) =& f_i(k_i) - c_i(x_i)
%     \\
%     =& ( - c^0_i(k_i, \bmx_{-i})) + f_i(k_i)
% \end{align*}


We have $c'_i(k_i)$ is $L_i$-Lipschitz on $k_i$ by assumption, therefore, we could easily find that $\frac{\pp^2 c^0_i}{\pp k_i \pp x_j}(k_i,\bmx_{-i})$ is upper bounded by $|w_{ij}| L_i$. 
% Then we have that,
% \begin{align*}
%     \| \frac{\pp^2 c^0_i}{\pp k_i \pp \bmx_{-j}}(k_i, \bmx_{-i}) \| \le L \| \bmw_{i,-i} \|
% \end{align*}
% where $\bmw_{i,-i}$ is the vector $(w_{i1},w_{i2},...,w_{i,i-1},w_{i,i+1},...,w_{in})\in \bbR^{n-1}$.

Together with $f_i(k_i)$ is $C_i$-concave on $k_i$, by \cref{lem:argmax:Lipschitz}, we have that
\begin{align*}
    y_i(\bmx_{-i}) = \argmax_{k_i}\quad u^0_i(k_i,\bmx_{-i}) = ( - c^0_i(k_i, \bmx_{-i})) + f_i(k_i)
\end{align*}
is $\frac{2L_i|w_{ij}|}{C_i}$-Lipschitz on $x_j$.


Now we define the utility function for player $i$ in the near-potential game,
\begin{align*}
    u_i(\bmx) = y_i(\bmx_{-i}) x_i - \frac{x_i^2}{2} - \sum_\jnei w_{ij} x_i x_j
\end{align*}

If $\bmx$ is an NE for the game $\{u_i(\bmx)\}_\iinn$ constructed above, we have that
\begin{align}
    0 =& \frac{\pp u_i}{\pp x_i}(\bmx) = y_i(\bmx_{-i}) - x_i - \sum_\jnei w_{ij} x_j
    \\
    \Rightarrow y_i(\bmx_{-i}) =& x_i + \sum_\jnei w_{ij} x_j
\end{align}
\ie, the choice of $x_i$ will make her gain level to $y_i(\bmx_{-i})$, which is also the optimal gain level of player $i$ given $\bmx_{-i}$ by definition of $y(\bmx_{-i})$, therefore, $\bmx$ is also an NE of the original public good game $G$.

The last step is to show that the NE for the near-potential game $\{u_i(\bmx)\}_\iinn$ is unique. We derive that,
\begin{align*}
    \frac{\pp u}{\pp x_i}(\bmx) =& k^*_i - \sum_\jinn w^0_{ij} x_j
    \\
    \frac{\pp u_i}{\pp x_i}(\bmx) =& y_i(\bmx_{-i}) - \sum_\jinn w_{ij} x_j
\end{align*}

It's obvious that $\frac{\pp (u_i - u)}{\pp x_i}(\bmx)$ is $\sigma_{ii} = |w^0_{ii} - 1| = 0$-Lipschitz on $x_i$ (constant on $x_i$) and $\sigma_{ij} = \frac{2L_i|w_{ij}|}{C_i} + |w^0_{ij} - w_{ij}|$-Lipschitz on $x_j$. Thus the constructed utilities, $\{u_i(\bmx)\}_\iinn$, are $(\ones,\Sigma)$-near potential to $u(\bmx)$. By \cref{lem:near-potential}, we have that the NE of the game $\{u_i(\bmx)\}_\iinn$ is unique, which completes the proof.

\end{proof}
