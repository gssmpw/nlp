\begin{proof}
\label{prf:lem:argmax:Lipschitz}
The existence and uniqueness of the maximum solution are readily established, as it is a well-known property that a strongly concave function always possesses a unique maximum within a bounded, closed, and convex set. 

The proof of property of Lipschitzness will be presented in the subsequent discussion.

Let $x_i = x(c_i)$ for $i=1,2$ and denote $y = \frac{x_1 + x_2}{2}$. We know $y\in X$ by convexity of $X$. By the optimality of $x_i$ we have,
\begin{align}
    g(c_1,x_1) + f(x_1) \ge& g(c_1,y) + f(y);\label{lemma4.6-ineq1}
    \\
    g(c_2,x_2) + f(x_2) \ge& g(c_2,y) + f(y).\label{lemma4.6-ineq2}
\end{align}
Combining (\ref{lemma4.6-ineq1}) and (\ref{lemma4.6-ineq2}), we know that
\begin{align}
    g(c_1,x_1) + g(c_2,x_2) - \left[ g(c_1,y) + g(c_2,y)\right] \ge& 2f(\frac{x_1+x_2}{2}) - (f(x_1) + f(x_2)) \ge \frac{\alpha}{4}||x_1 - x_2||^2,\label{lemma4.6-ineq3}
\end{align}
where the last inequality is satisfied by the $\alpha$-concavity of $f(x)$.

By concavity of $g(c,x)$ on $x$, we also have
\begin{align}
    &g(c_1,x_1) + g(c_2,x_2) - \left[ g(c_1,y) + g(c_2,y)\right] \nonumber\\
    \le& g(c_1,x_1) + g(c_2,x_2) - \frac{1}{2}\left[ g(c_1,x_1) + g(c_2,x_2) + g(c_1,x_2) + g(c_2,x_1) \right]
    \nonumber\\
    =& \frac{1}{2}(g(c_1,x_1) + g(c_2,x_2) - g(c_1,x_2) - g(c_2,x_1))
    \nonumber\\
    =& \frac{1}{2} \int_{L_c} \int_{L_x} \ \bm{r}_c^T \frac{\partial g}{\partial c \partial x}(c,x) \bm{r}_x \ dc dx
    \nonumber\\
    \le& \frac{L}{2} ||c_1 - c_2||\cdot ||x_1 - x_2||\label{lemma4.6-ineq4}
\end{align}
where $L_c, L_x$ represent the straight line between $c_1,c_2$ and $x_1,x_2$, respectively, and $\bm{r}_c$ and $\bm{r}_x$  denote the unit vector with direction $c_2 - c_1$ and $x_2 - x_1$.
Combining (\ref{lemma4.6-ineq3}) and (\ref{lemma4.6-ineq4}), we derive that
\begin{align*}
    ||x_1 - x_2|| \le \frac{2L}{\alpha} ||c_1 - c_2||,
\end{align*}
which completes the proof for $\frac{2L}{\alpha}$-Lipschitzness of $x(c)$.
\end{proof}