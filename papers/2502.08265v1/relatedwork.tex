\section{Related Work}
\paragraph{Big Five Personality Model} 

The Big Five personality model is considered to be a valid approach to evaluate one's personality traits due to its reliability and practicality~\cite{john2008paradigm}. Compared to MBTI personality measurement, the Big Five model has proven its validity across different populations~\cite{schweiger1985measuring}. According to the Big Five model, an individual's personality can be represented along five dimensions, namely Extraversion, Neuroticism, Agreeableness, Conscientiousness, and Openness to Experience. Each dimension is a dichotomy with low and high trait manifestations. Recent research adds a new dimension, Honesty-Humility, to the original model~\cite{costa2008revised}.

Initially, the Big Five model is based on the linguistic hypothesis, claiming that the natural language descriptors of person's traits, moods, and characteristics can be generalized and represented as a personality model. Subsequently, there appear also psychological questionnaires, namely those as TDA \cite{goldberg1992development}, NEO \cite{costa1978objective}, and BFI \cite{john1991big}, that help to determine personality traits by answering a predefined set of questions. All the questionnaires show respectable validity and reliability, but BFI questionnaire stands out for question formulation~\cite{john2008paradigm}. The Big Five model is critiqued for lacking detailed personality classification, as all traits seem to be abstract~\cite{john2008paradigm}; moreover, the model only includes traits linked to natural language analysis, neglecting non-social personal traits~\cite{trofimova2014observer}. Nevertheless, these characteristics can be advantageous in the context of LLMs, as they can be utilized to generate a personality in natural language modeling using the Big Five model.

% Это основной для понимания статьи абзац, но из него вообще ничего не ясно. Нужно больше подробностей. Если успеется, то нарисовать бы схему. (Лида)

\paragraph{Personality of Large Language Models}
%вводная часть про LLM (следуют инструкциям, могут принимать роли и т.д.)
%LLM + генерация персональности (это обзор статей)

The research devoted to personality in LLMs focuses on two main aspects, including internal LLM personality and LLM-prompted personality. LLM personality is usually analyzed by leveraging the Big Five questionnaire results, LLM-generated text analysis (both linguistic and psychological), or automated text classification for the Big Five traits.

Studies on LLM personalities, e.g., \cite{sorokovikova2024llms}, provide statistics on how LLMs answer the different the Big Five questionnaires (IPIP-NEO, BFI-44). This should reveal the LLM hidden personality. The findings seem to be quiet impressive, as they portray GPT-4 and Mixtral as open, agreeable, conscientious, extroverted and emotionally stable agents capable of engaging conversations. However, other studies disagree with this conclusion, as the questionnaire results do not seem to reflect real model personality and do not help to predict model actions~\cite{ai2024cognition}.

Studies on how we can induce personality in LLMs demonstrate different ways a personality can be prompted to an LLM. Some approaches suggest describing personality based on the personal facts (e.g., providing the LLM with a personality description from PersonaChat dataset) or based on descriptions of each the Big Five characteristic~\cite{serapiogarcía2023personality, liu2024dynamic}. The other research also tests personality simulation through text generation~\cite{jiang2024evaluating} or sentence completion~\cite{hilliard2024eliciting}. Personality modeling is usually affected by the model size, as it is stated in all above-described studies. 
% 
% Этот параграф лучше, но формулировки предложений очень сложные (много доп конструкций внутри предложений). Я бы посоветовала упростить текст, чтобы его было проще читать, и немного разнообразить предложения (Лида)