\section{Conclusion and Future Work}\label{sec_conclusion}



This paper introduces \ourSystem for scalable RF signal propagation modeling.  
We customize 3D Gaussian attributes and employ gradient-guided learning, integrating an RF-customized CUDA-based ray tracing module for efficient 3D scene representation and received signal computation.
Extensive experiments validate \ourSystem's efficiency.
Future work explores key challenges and potential enhancements.


\textbf{Multiple Receivers in a Scene:}
Our system currently supports a single fixed receiver for predicting signals at any transmitter location in a scene. 
However, real-world scenarios may involve multiple receivers.  
Since they share the same physical space, a potential solution is a coordinate-based transformation function that models RF propagation variations with receiver location, allowing for interpolation and extrapolation of RF signal characteristics.



\textbf{Generalizability Across Scenes:} 
Our system enables quick training when data is available for each scene.  
However, it does not support zero-shot inference for new scenes.  
A potential solution is pre-training on large multi-scene datasets to enhance generalization, leveraging transfer learning for adaptation with minimal or no additional training.



\textbf{Temporal Dynamics:}  
Our system is designed for static environments.  
If the scene layout changes, retraining or fine-tuning is required.  
A potential solution is to incorporate recurrent learning mechanisms to capture temporal dependencies.  
For example, each Gaussian’s attributes can be modeled as a function of discrete time steps, enabling the system to adapt to environmental changes.



\section*{Acknowledgments}
This research was funded in part by the Air Force Office of Scientific Research under awards \#~FA95502210193 and FA95502310559, and the DEVCOM Army Research Laboratory under award \#~W911NF-17-2-0196.
Wan Du was partially supported by NSF Grant \#~2239458, a UC Merced Fall 2023 Climate Action Seed Competition grant, and a UC Merced Spring 2023 Climate Action Seed Competition grant. 
Kang Yang was partially supported at UC Merced by a financial assistance award approved by the Economic Development Administration’s Farms Food Future program.
Sijie Ji's research is supported through a  Schmidt Science Fellowship.





