
\documentclass[sigconf,10pt]{acmart}

\usepackage{tikz}
\usepackage{amsmath}

\usepackage[linesnumbered, ruled]{algorithm2e}

\usepackage{colortbl}
\usepackage{eucal}
\usepackage{bm}
\usepackage{subfigure}
\usepackage{booktabs}
\usepackage{multirow}
\usepackage{enumitem}
\usepackage{url}
\usepackage{xspace}
\usepackage{amsfonts}
\usepackage{booktabs}


\usepackage[english]{babel}
\usepackage{blindtext}

\usepackage{pifont} 
\usepackage{stackengine}

\usepackage{setspace}

\setlength{\textfloatsep}{5pt}
\setlength{\abovecaptionskip}{1pt}
\setlength{\belowcaptionskip}{2pt}



\newcolumntype{C}[1]{>{\centering\arraybackslash}p{#1}}
\newcolumntype{L}[1]{>{\raggedright\arraybackslash}p{#1}}



\renewcommand\footnotetextcopyrightpermission[1]{}

\settopmatter{printacmref=false, printccs=false, printfolios=true}

\newcommand{\authoraff}{\fontsize{10.3pt}{11pt} \selectfont}

\newcommand{\authorname}{\fontsize{14pt}{11pt} \selectfont}


\def\ourSystem{\textit{RFSPM}\xspace}

\def\ie{\textit{i.e.},\xspace}
\def\eg{\textit{e.g.},\xspace}
\def\etl{\textit{et al.}\xspace}
\def\vs{\textit{vs.}\xspace}
\def\nerft{{NeRF$^2$}\xspace}

\setlength{\pdfpagewidth}{8.5in}
\setlength{\pdfpageheight}{11in}


\begin{document}


\title{Scalable 3D Gaussian Splatting-Based RF Signal Spatial Propagation Modeling}


    \author{\authorname Kang Yang}
    \affiliation{ \authoraff
    \institution{University of California, Los Angeles}
    \city{Los Angeles}
    \state{California}
    \country{USA}
    }
    \email{kyang73@g.ucla.edu}
    \authornote{This work was partially done when Kang Yang was a PhD student in Dr. Wan Du’s group at UC Merced.}
    
    \author{\authorname Gaofeng Dong}
    \affiliation{ \authoraff
    \institution{University of California, Los Angeles}
    \city{Los Angeles}
    \state{California}
    \country{USA}
    }
    \email{gfdong@g.ucla.edu}

    \author{\authorname Sijie Ji}
    \affiliation{\authoraff
    \institution{University of California, Los Angeles}
    \city{Los Angeles}
    \state{California}
    \country{USA}
    }
    \affiliation{\authoraff
    \institution{California Institute of Technology}
    \city{Pasadena}
    \state{California}
    \country{USA}
    }
    \email{sijieji@caltech.edu}

    \author{\authorname Wan Du}
    \affiliation{
    \institution{\authoraff University of California, Merced}
    \city{Merced}
    \state{California}
    \country{USA}
    }
    \email{wdu3@ucmerced.edu}
    \authornote{Wan Du is the corresponding author.}
    
    \author{\authorname Mani Srivastava}
    \affiliation{
    \institution{\authoraff University of California, Los Angeles \\ and Amazon}
    \city{Los Angeles}
    \state{California}
    \country{USA}
    }
    \email{mbs@ucla.edu}
    \authornote{Mani Srivastava holds concurrent appointments as a Professor of ECE and CS~(joint) at the University of California, Los Angeles and as an Amazon Scholar. This paper describes work performed at the University of California, Los Angeles and is not associated with Amazon.}
% }



\begin{abstract}



Effective network planning and sensing in wireless networks require resource-intensive site surveys for data collection.  
An alternative is Radio-Frequency~(RF) signal spatial propagation modeling, which computes received signals given transceiver positions in a scene~(\eg a conference room).  
We identify a fundamental trade-off between scalability and fidelity in the state-of-the-art method. To address this issue, we explore leveraging 3D Gaussian Splatting~(3DGS), an advanced technique for the image synthesis of 3D scenes in real-time from arbitrary camera poses.  
By integrating domain-specific insights, we design three components for adapting 3DGS to the RF domain, including Gaussian-based RF scene representation, gradient-guided RF attribute learning, and RF-customized CUDA for ray tracing.
Building on them, we develop \ourSystem, an end-to-end framework for scalable \textbf{RF} signal \textbf{S}patial \textbf{P}ropagation \textbf{M}odeling.  
We evaluate \ourSystem in four field studies and two applications across RFID, BLE, LoRa, and 5G, covering diverse frequencies, antennas, signals, and scenes.  
The results show that \ourSystem matches the fidelity of the state-of-the-art method while reducing data requirements, training GPU-hours, and inference latency by up to 9.8\,$\times$, 18.6\,$\times$, and 84.4\,$\times$, respectively.



\end{abstract}

\keywords{RF Signal Spatial Propagation Modeling, Scalability}


\maketitle
\pagestyle{plain}

\section{Introduction}\label{sec:intro}

In computational finance, Monte Carlo simulations are used extensively to estimate the expected value of financial payoffs based on the solution of stochastic differential equations (SDEs) which model the evolution of stock prices, interest rates, exchange rates and other quantities \cite{glasserman04}.  Monte Carlo methods are very general and flexible, but for high accuracy it requires generating a large number of costly SDE path approximations, which has motivated research into a number of variance reduction or, equivalently, cost reduction techniques. One such method is
Multilevel Monte Carlo (MLMC), which was proposed in \cite{GILES2008} and was adapted for various applications that are summarised in \cite{Giles_overview17} and successfully combined with other methods such as quasi-Monte Carlo methods. The main idea of MLMC is to approximate the payoff using different time stepping resolutions when numerically solving the underlying SDE and to generate an optimal number of samples on each level, such that the overall computational cost is minimised subject to the desired bound on the variance. %, such that the total computational cost is minimised. 
The computational savings come from the fact that most samples are computed on the coarser levels and hence are less expensive while only a few samples from the finest levels are required \cite{GILES2008}.


Among the directions in which the computational cost 
of MLMC methods could further be reduced, an important avenue is the use of lower precision calculations, especially for the first Monte Carlo levels where the targeted accuracy is relatively low. 
 An overview of the research on mixed precision for the standard Monte Carlo (MC) framework is provided in \cite{ChowMixedPrecisionStandardMC} but only a few references study the potential of low precision computation in the MLMC framework \cite{Rounding_error_oliver}. To the best of our knowledge, the only MLMC framework with customised precision in the literature is \cite{brugger2014mixed}, but they use a uniform precision for all operations on each Monte Carlo level instead of optimising 
 the precision of each intermediary variable to reduce as much as possible the cost of path generation.
 
An important motivation for an MLMC framework with variable precision would be performing the low precision computations on reconfigurable hardware devices such as Field Programmable Gate Arrays (FPGAs). FPGAs contain customizable logic blocks and connectors that make it easy to adapt the digital circuit architecture for a specific application, leading to a highly parallel and optimised implementation. Therefore they are successfully exploited in applications that require high speed and have high computational workload, such as signal processing \cite{woods2008fpga}, and real time applications like high frequency trading \cite{HFT1,HFT2}. That is why a number of previous works in hardware architecture design implemented the MLMC algorithm to price financial options using FPGAs as accelerators, which resulted in improved speed and power efficiency compared to full CPU architectures \cite{Schryver2013AMM}. The paper \cite{lindsey2016domain} also proposed 
a Domain Specific Language to automate the configuration of FPGAs for this specific application. However, only \cite{brugger2014mixed} proposed a heuristic to reduce the precision in calculations.

In addition, all aforementioned works considered that the random number generation (RNG) is performed in single or double precision. Yet in most cases an important portion of the workload in the overall MLMC simulation comes from the RNG and in \cite{brugger2014mixed} this limited the total computational savings.
To reduce the cost of MLMC simulations in particular those based on the Geometric Brownian Motion (GBM), \cite{approximateICDF_Oliver, NestedOliver} have proposed to use approximate random numbers that are generated by applying an approximation of the inverse CDF to uniform random numbers. In \cite{NestedOliver}, the authors proposed a way to integrate these lower precision random variables into a \textit{nested} MLMC framework and completed a numerical analysis to bound the resulting error at each MC level by a product of the time step and the error in the random number approximation. The same authors show in \cite{approximateICDF_Oliver} that using approximate random variables reduces the cost of path generation by a factor 7.


In this paper we propose a nested MLMC framework that combines the use of approximate random normal variables and lower precision calculations to reduce the computational cost of MLMC even further than \cite{brugger2014mixed,NestedOliver}. We illustrate the efficiency of our framework in Matlab, after making several assumptions on the cost of operations and size of the errors that we carefully justify. We focus on the case of GBM and use the approximate RNG methods presented in \cite{approximateICDF_Oliver} as well as a new slightly modified method that combines CDF inversion and the central limit theorem. To choose the precision of the variables in the low precision path generation, we introduce a novel method to optimise the bit-widths. This optimisation is performed before the main path generation loop is executed and is based on a linear model of the payoff error  
due to rounding when computing in low precision. The error model relies on algorithmic differentiation in a similar manner to \cite{unifying-bwoptim,bitwidth-AD,ADAPT}. The bit-width optimisation procedure can be performed off-line, so this stage can be excluded from the on-line time complexity of our framework. The user specified desired accuracy is then enforced by calculating on-line the number of samples that need to be generated.

In terms of hardware design, we suggest implementing the low precision path generation on FPGAs and the full-precision ones on a CPU or GPU. 
The FPGA offers enough flexibility to define a separate bit-width for every variable in the low precision path generation, and can be reconfigured periodically to update the bit-widths when the market parameters have changed considerably. 


The paper is organized as follows : \Cref{sec:MLMC} introduces MLMC and nested MLMC to make clear the estimator that is implemented in our framework. Then in \Cref{sec:RNG} we detail the methods that could be used to obtain approximate random normally distributed numbers very cheaply for the low precision path generation. In \Cref{sec:error_model} and \Cref{sec:costModel} we propose an error model and a cost model (resp.) that we then use to formulate the optimisation problem that is solved to obtain the optimal bit-widths of fixed point variables in \Cref{sec:optimisation}. Finally we summarise our results and future directions in \Cref{sec:conclusion}.








\section{Scenario Overview}


We consider a scenario where a receiver is fixed at a position~(\eg 5G base station or LoRa gateway), while a transmitter~(\eg smartphone or LoRa node) can be at any location in 3D space.
Given a dataset of some transmitter locations and their corresponding received signals, the goal is to predict the received signal from a transmitter at a new position.


Alternatively, the roles can be reversed: the transmitter is fixed~(\eg WiFi router), while the receiver is placed at different locations~(\eg smartphone).
According to reciprocity between the transmitter and receiver~\cite{liu2021fire}, these two scenarios are essentially equivalent.  
Consequently, this work focuses solely on the first scenario.



Received signal type depends on the receiver's antenna setup.  
\textbf{i)}~For an antenna array, the received signal is a \((360, 90)\) spatial spectrum matrix, where each entry represents the signal power in a specific direction at one-degree resolution.  
\textbf{ii)}~For a single antenna, the received signal is either a single complex-valued measurement or a single signal power.





\section{Related Work}\label{sec_relatedWork}


\textbf{i) NeRF-Based:}
\nerft~\cite{zhao2023nerf} develops voxel-based scene representations and ray tracing algorithms to achieve state-of-the-art modeling accuracy, outperforming conventional methods such as simulations~\cite{wirelessinsite_web, orekondy2022winert, RayTracingToolbox}, empirical models~\cite{rappaport1996wireless, parsons2012mobile, hata1980empirical}, and ML models~\cite{parralejo2021comparative, liu2021fire, malmirchegini2012spatial}.  
However, it faces a scalability-fidelity tradeoff.  
\ourSystem addresses this tradeoff through novel designs.
\textit{First,} \ourSystem employs 3D Gaussians with varying positions, shapes, complex-valued emission and attenuation to construct a more flexible and expressive scene representation.
This enables \ourSystem to focus on object regions while avoiding large empty spaces.
\textit{Second,} each Gaussian explicitly stores its attributes, eliminating extensive large MLP queries.
\textit{Third,} the irregular and discrete placement of Gaussians renders voxel-based ray tracing ineffective. 
We address this with orthographic projection-based splatting to identify ray-Gaussian intersections, followed by a point-based complex-valued blending algorithm to aggregate emission, attenuation, and Gaussian shape contributions per ray.


NeWRF~\cite{lunewrf} shares the same methodology as \nerft but differs in that it restricts rays to directions identified by Direction-of-Arrival (DoA) data, rather than considering all \(360*90\) directions.  
However, collecting DoA data requires deploying an additional antenna array at each position, increasing the complexity and burden of data collection.
This violates our three scalable principles.  
Consequently, we treat \nerft and NeWRF as equivalent methods.  







\begin{figure}[t]
\vskip -0.1in
    \centering
    \includegraphics[width=.47\linewidth]{Figures/test_Growth_Comparison_First_3_Iter_X_of_onehot_Variance_BLOCK_OUTPUT_1B_post_no_broken.png} \includegraphics[width=.47\linewidth]{Figures/test_Growth_Comparison_First_3_Iter_X_of_onehot_Variance_BLOCK_OUTPUT_pre_1B.png} 
    \vskip -0.05in
    \caption{Illustration of hidden-state variance across different model depths and training iterations. From the initialization stage up to the point where 6.3 billion tokens were trained, we observed the variance growth of hidden states for Pre-LN and Post-LN architectures. The analysis was conducted using a $1.5$B-parameter model, and consistent trends were observed across models of different sizes. Detailed settings and more results are in Section~\ref{subsec:growth of hidden state}.}
    \label{fig:3iter}
    \vskip -0.1in
\end{figure}

\section{Background and Motivation}
The analysis of activation variance at model initialization has long been central to understanding normalization layers and enhancing stability in convolutional neural networks (CNNs) \citep{cnnvariance, identity, BrockDSS21}. \citet{cnnvariance} showed that batch normalization in residual blocks can bias networks toward the identity function, thereby stabilizing gradients and improving overall training dynamics. 

Similar investigations have emerged for Transformer architectures, examining how variance propagates and how gradients behave in both post-layer normalization (Post-LN) \citep{attentionisallyouneed} and pre-layer normalization (Pre-LN) \citep{llama3} configurations \citep{onlayer, transformersgetstable, smallproxies, mixln}. Early work comparing Post- and Pre-LN primarily focused on gradient scales and loss behavior. \citet{onlayer} observed that Pre-LN architectures tend to exhibit more stable gradients, but can still encounter issues such as gradient spikes and divergence, especially in deeper models or large-scale pre-training scenarios \citep{attentioncollapse, smallproxies, mlpswiglu, embeddingln}. 

Among these challenges, the phenomenon of ``massive activations'' has attracted particular attention \citep{llm.int8,yu2024super,mlpswiglu}. \citet{massiveactivation} identified that in Pre-LN architectures, large spikes in activation magnitude can persist across layers due to residual connections. These massive activations act as fixed biases, potentially narrowing the model’s focus to certain tokens and may influence generalization. However, the underlying mechanisms behind these large values—and their exact impact on the training process—remain not yet well understood.

Analytical work has provided theoretical frameworks to explain phenomena like gradient explosion and vanishing in Transformers. For instance, \citet{transformersgetstable} introduced a signal propagation theory that details how activation variance and gradient instability can evolve with depth, identifying critical factors that impair stability and performance. Recent studies have discussed how Pre-LN architectures can allow large values from Attention or MLP modules to flow unimpeded through residual connections \citep{moeut, mlpswiglu, attentioncollapse, smallproxies}, but the precise impact of this behavior on large-scale training remains insufficiently explored.

These observations underscore the ongoing need to clarify how activation dynamics, normalization strategies, and architectural choices interact, especially in large-scale models. In response, this work aims to deepen our understanding of activation evolution during Transformer training under different normalization architectures, focusing on the role of massive activations and their effects on overall stability and performance.

We defer an extended discussion of the related literature to Appendix~\ref{appendix:relatedwork}, owing to space limitations.




\section{Design of \ourSystem}\label{sec_design}

\begin{figure*}[t]
\centering
{\includegraphics[width=.95\textwidth]{figs/workflow.pdf}}
\caption{Illustration of the \ourSystem architecture, trained end-to-end, consisting of three main building blocks.}
  \Description[]{}
	\label{fig_workflow}
  \vspace{-10pt}
\end{figure*}



Figure~\ref{fig_workflow} illustrates \ourSystem, which synthesizes the received signal at a specific receiver for a transmitter positioned at any location in the scene. 
\textit{i)~Gaussian-based RF scene representation} models the scene with 3D Gaussians, each storing geometric and RF-related attributes.  
\textit{ii)~Gradient-guided attribute learning} optimizes these attributes via gradients while dynamically adjusting the number of Gaussians.  
\textit{iii)~RF-customized CUDA for ray tracing} computes the received signal by emitting rays from the receiver, identifying ray-Gaussian intersections via orthographic projection-based splatting, and aggregating RF attributes along each ray.


\subsection{Gaussian-Based Scene Representation}

Each customized 3D Gaussian distribution carries four attributes:  
i)~mean \(\boldsymbol{\mu}\) and ii)~covariance matrix \(\boldsymbol{\Sigma}\), which define its geometric properties, including position, size, shape, and orientation;
iii)~emission \(\boldsymbol{\psi}\) and iv)~attenuation \(\boldsymbol{\rho}\), which characterize the Gaussian's influence on RF signal propagation.


\textbf{i)~Mean~\(\boldsymbol{\mu}\) and ii)~Covariance Matrix~\(\boldsymbol{\Sigma}\):}  
A 3D Gaussian  distribution resembles an ellipsoid, representing a probability distribution in 3D space, as defined by the probability density function~(PDF) in Equation~(\ref{eqn_gaussian}).
The center of the distribution is a 3D position (mean \(\boldsymbol{\mu}\)), indicating the peak location, while the spread and orientation in space are determined by a full~\(3 \times 3\) covariance matrix \(\boldsymbol{\Sigma}\):
\begin{equation}
\label{eqn_gaussian}
    P\left(\mathbf{x}\right) = e^{-\frac{1}{2} \left(\mathbf{x} - \boldsymbol{\mu}\right)^\text{T} \boldsymbol{\Sigma}^{-1} \left(\mathbf{x} - \boldsymbol{\mu}\right)}
\end{equation}






\textbf{iii)~Emission~\(\boldsymbol{\psi}\):}
Each point on a wavefront serves as a source of wavelets, according to the Huygens–Fresnel principle~\cite{born2013principles}.
Analogously, when an RF signal from a transmitter encounters a 3D Gaussian, the Gaussian acts as a scattering point, re-emitting an RF signal termed emission \(\boldsymbol{\psi}\). 
This emission is characterized by \(\boldsymbol{\psi} = \left|\boldsymbol{\psi}\right| e^{j \angle \boldsymbol{\psi}}\), where \(\left|\boldsymbol{\psi}\right|\) represents the amplitude and \(\angle \boldsymbol{\psi}\) denotes the phase.



Emission \(\boldsymbol{\psi}\) varies due to factors such as surface orientation relative to the incident signal and the material properties at the Gaussian's position.  
Moreover, the transmitter position influences \(\boldsymbol{\psi}\) by altering the incident angle.  
Thus, \(\boldsymbol{\psi}\) depends on both the direction and the transmitter position.
To this end, we employ a small neural network, \(f_\Theta\), within each Gaussian to encode the directional emission signal:
\begin{equation}
\label{eqn_radiance_mlp}
\boldsymbol{\psi} = f_\Theta \left(x_{\text{tx}}, y_{\text{tx}}, z_{\text{tx}}, \alpha, \beta\right)
\end{equation}
where \(f_\Theta\) takes the transmitter position \(\left\{x_{\text{tx}}, y_{\text{tx}}, z_{\text{tx}}\right\}\) and the direction \(\left\{\alpha, \beta\right\}\) as inputs and outputs the corresponding emission \(\boldsymbol{\psi}\). 
Here, \(\left\{\alpha, \beta\right\}\) represents the direction from the 3D Gaussian's position to the receiver.  
Since \(f_\Theta\) models only its own Gaussian's emission, it has a small number of parameters,~\ie~two fully connected layers with ReLU activation.



\textbf{iv) Attenuation~\(\boldsymbol{\rho}\):}
An RF signal passing through a 3D Gaussian undergoes attenuation \(\boldsymbol{\rho}\), resulting in an amplitude reduction \( \left|\boldsymbol{\rho}\right| \) and a phase shift \( \angle \boldsymbol{\rho} \).  
According to Maxwell's equations~\cite{maxwell1873treatise}, attenuation depends on material properties.  
Thus, the attenuation \(\boldsymbol{\rho}\) of a 3D Gaussian is primarily determined by the material properties at its location.



\subsection{Gradient-Guided Attribute Learning}  
We initialize the number of Gaussians and their attributes, then optimize both with gradient-based strategies. 
Gradients are calculated after computing the loss in §\ref{sec_training_loss}.  
Finally, we discuss the efficiency of using 3D Gaussians.


\subsubsection{Cube-Based Initialization}
We partition the scene into equal-sized cubes, each with a side length \( L_{\text{cube}} \), empirically set to six times the wavelength.  
The center of each cube is assigned as a Gaussian's mean.  
The covariance matrix is initialized based on the average distance to the \( N_{\text{cube}} \) nearest cube centers, where \( N_{\text{cube}} \) is set to three, while attenuation and emission are assigned randomly.
Compared to random initialization, this cube-based strategy ensures that the Gaussians cover the entire scene, leading to faster convergence.



\subsubsection{Gradient-Based Updating}
The following two strategies are employed to update the number of Gaussians and their attributes for flexible and efficient scene representation.


\textbf{i)~Attribute Updating:}
Each 3D Gaussian explicitly stores its own attributes and updates them using SGD~\cite{amari1993backpropagation}:
\begin{equation}
\label{eqn_updating}
w^{\left(j+1\right)} = w^{\left(j\right)} - \eta_{w} \cdot \nabla_{w} \mathcal{L}\left(w^{\left(j\right)}\right)
\end{equation}
where \(w\) represents any attribute of a Gaussian, each with its own learning rate \(\eta_{w}\).  
The term \(\nabla_{w} \mathcal{L}\left(w^{(j)}\right)\) denotes the gradient of the loss function \(\mathcal{L}\), defined in Equation~(\ref{eqn_loss}), with respect to \(w\) at iteration \(j\).  
For emission \(\boldsymbol{\psi}\), the updated parameters belong to the network \(f_\Theta\).

The covariance matrix \(\boldsymbol{\Sigma}\) is physically meaningful only when positive semi-definite~\cite{de2011strict}, but the update equation above does not guarantee this property.  
To address this, we adopt the solution proposed in \cite{kerbl20233d}, which represents \(\boldsymbol{\Sigma}\) as \(\Sigma = R S S^{T} R^{T}\), where \(R\) is a rotation matrix and \(S\) is a scaling matrix.  
Updates are applied independently to \(R\) and \(S\), ensuring that \(\boldsymbol{\Sigma}\) remains positive semi-definite.



\textbf{ii)~Number of Gaussian Updating:}
The initial number of Gaussians is set by cube-based initialization.  
However, this number is suboptimal, as some areas require more Gaussians~(\eg object regions), while others need fewer~(\eg free space) to model RF signal propagation effectively.
We observe that such cases lead to large gradients for the Gaussian's mean~\(\boldsymbol{\mu}\), as the existing 3D Gaussians do not adequately capture the area's effect on RF signal propagation.  
The mean~\(\boldsymbol{\mu}\) exhibits larger gradients than other attributes because it represents the position with the highest probability, making it crucial for modeling RF signal behavior.



To this end, we employ a gradient-threshold-based strategy: 
\textit{First}, every \(N_{\boldsymbol{\mu}}\) iterations, we compute the average gradient of the mean \(\boldsymbol{\mu}\) for all Gaussians and select those with a mean gradient exceeding a threshold \(\epsilon_{\boldsymbol{\mu}}\).  
\textit{Second}, we determine the radius of each selected Gaussian, approximated as the average of the diagonal values of its covariance matrix.  
A radius threshold \(\epsilon_{r}\) classifies them as small or large Gaussians.
\textit{Third}, small Gaussians are cloned by duplicating them and shifting the copies in the direction of the gradient.
Large Gaussians are split into two new Gaussians, reducing their scaling matrix \(R\) by a factor of \(\phi\) and initializing their positions by sampling from the original Gaussian's PDF.

Additionally, every \(N_{\boldsymbol{\rho}}\) iterations, we remove Gaussians with attenuation~\(\boldsymbol{\rho}\) below a threshold \(\epsilon_{\boldsymbol{\rho}}\), as they minimally impact signal propagation, \eg in free space.  
A single 3D Gaussian distribution can represent a large free space.




\subsubsection{Efficiency of 3D Gaussian}\label{sec_design_gaussian}


Unlike fixed voxel grids, which require numerous voxels to capture the entire scene's effects on signal propagation, 3D Gaussians adjust their position, shape, size, and orientation to represent these effects.  
This adaptability enables 3D Gaussians to achieve similar representation quality with far fewer voxels, improving computational efficiency and reducing training data requirements.
For example, a typical conference room~\cite{matlab_conference_room} requires learning the attributes of 31,257,628 voxels~(Section~\ref{sec_theoretical_ana}).  
In contrast, only 393,920 Gaussians may be needed, an \(\mathord{\sim}80\)-fold reduction, assuming the optimal count matches the number of points in the conference room's point cloud data. 





\subsection{RF-Customized CUDA for Ray Tracing}

The emitted rays from the receiver are formalized, followed by an orthographic projection-based splatting module, which identifies Gaussians intersected by each ray.  
Next, the complex-valued blending algorithm computes the received signal based on these intersections.  
Finally, computation is parallelized using customized CUDA kernels.


\subsubsection{Definition of Rays}
Rays extend from the receiver in various directions, \eg ray \(\gamma\) in Figure~\ref{fig_blending}:
\begin{equation}
\label{eqn_ray_define}
\gamma(d) = \mathbf{l}_{\text{rx}} + d \hat{\mathbf{v}}, \quad \text{where} \quad d \geq r_{\text{rx}}
\end{equation}
where \(d\) is the distance from the receiver to a point \(r(d)\) on the ray, \(\mathbf{l}_{\text{rx}} = \left(x_{\text{rx}}, y_{\text{rx}}, z_{\text{rx}}\right)\) denotes the receiver position, and the unit vector \(\hat{\mathbf{v}} = \left(\cos\alpha \cos\beta, \sin\alpha \cos\beta, \sin\beta\right)^\top\) defines the ray direction, with \(\alpha\) and \(\beta\) as the azimuthal and elevation angles, respectively.  
The condition \(d \geq r_{\text{rx}}\) indicates that the ray starts at a distance \(r_{\text{rx}}\) from the receiver.
Thus, \(360 * 90\) rays are emitted from a spherical surface centered at the receiver with radius \(r_{\text{rx}}\).
We refer to this surface as the Ray Emitting Spherical Surface~(RESS).



\subsubsection{Orthographic Projection-Based Splatting}\label{sec_ortho}
The uniform voxel grid structure allows straightforward identification of the voxels a ray passes through.  
However, the irregular and discrete placement of 3D Gaussians complicates determining which Gaussians a ray intersects.  
Intuitively, each ray must be checked against all Gaussians, resulting in a computational complexity of \(O(M \times N)\), where \(M\) is the total number of rays and \(N\) is the total number of Gaussians.


In 3DGS, to accelerate the determination of which Gaussians affect each ray~(pixel), 3D Gaussians are projected~(or "splatted") onto a 2D image plane.  
This splatting process utilizes the View Matrix, Projection Matrix, and Jacobian Matrix~\cite{takikawa2021neural} to form 2D Gaussians on the image plane.  
Each projected 2D Gaussian is represented as a circle centered at its mean, with a radius determined by the 2D covariance matrix.  
Pixels~(rays) within this circle are considered affected by the original 3D Gaussian.  
This splatting reduces computational complexity to \(O(N)\) by localizing each Gaussian’s influence to a specific region on the image plane.

 
\begin{figure}[!tp]	{\includegraphics[width=.47\textwidth]{figs/bleding.pdf}}
    \caption{Illustration of the complex-valued blending algorithm, which calculate the received signal in direction \(\gamma\). Four 3D Gaussians are shown, where \(\boldsymbol{\psi}\) and \(\boldsymbol{\rho}\) denote the emission and attenuation of each Gaussian. Each emission \(\boldsymbol{\psi}_i\) is attenuated by \(\boldsymbol{\rho}_m\) from Gaussians \(m\) (from \(1\) to \(i - 1\)). The final received signal in direction \(\gamma\) is the sum of these attenuated emissions.}
\label{fig_blending}
 \Description[]{}
\end{figure}


\textbf{2D RF Plane:}  
The above splatting is not applicable in the RF domain without an image plane.   
Instead, the received signal is measured on the RESS.
To enable splatting, the RESS is transformed into a 2D RF plane.  
Specifically, the Cartesian coordinates \((x, y, z)\) of each point on the RESS are converted to spherical coordinates \((\zeta, \alpha, \beta)\), with \((\alpha, \beta)\) rounded to the nearest integers to achieve one-degree resolution:
\begin{equation}
\label{eqn_projection}
\begin{aligned}
\begin{pmatrix}
\zeta \\
\alpha \\
\beta
\end{pmatrix}
&=
\begin{pmatrix}
\sqrt{x^2 + y^2 + z^2} \\
\arctan2\left(y, x\right) \\
\arccos\left(\frac{z}{\sqrt{x^2 + y^2 + z^2}}\right)
\end{pmatrix} \\
x' &= \lfloor \alpha \rfloor, \quad y' = \lfloor \beta \rfloor
\end{aligned}
\end{equation}
where \(\zeta\) is the radial distance, \(\alpha\) the azimuthal angle, \(\beta\) the elevation angle, and \(\lfloor \cdot \rfloor\) the floor function.  
\(\left(x', y'\right)\) represents the projected coordinates in the 2D RF plane.


\textbf{Splatting:}  
Each 3D Gaussian is splatted onto the RF plane, forming a 2D Gaussian represented as a circle.  
Gaussian mean \(\boldsymbol{\mu}\) is projected onto the 2D plane using Equation~(\ref{eqn_projection}), defining the circle's center.  
Jacobian matrix~\cite{takikawa2021neural} maps the \(3 \times 3\) covariance matrix into a \(2 \times 2\) covariance matrix, whose eigenvalues determine the circle's radius.  
Rays within this circle are considered influenced by the original 3D Gaussian.


\subsubsection{Complex-Valued Blending Algorithm}\label{sec_complex_blending}  
We introduce a complex-valued blending algorithm to process a given ray and its identified intersected Gaussians.  
First, the Gaussians are sorted by their distance to the receiver.  
Then, the received signal for the ray is computed by aggregating their RF attributes, incorporating both amplitude and phase channels:
\begin{equation}
\label{eqn_blending}
S = \sum_{i=1}^{N_{\text{intr}}}  \left|\boldsymbol{\psi_i}\right| e^{j \angle \boldsymbol{\psi_i}} \cdot \prod_{m=1}^{i-1} \left( 1 - \left|\boldsymbol{\rho_m}\right| e^{j \angle \boldsymbol{\rho_m}} \right)
\end{equation}
where \(S\) is the received signal for a ray, \(N_{\text{intr}}\) denotes the number of Gaussians intersecting the ray, and \(\boldsymbol{\psi}_i\) and \(\boldsymbol{\rho}_m\) represent the emission and attenuation of the \(i\)-th and \(m\)-th Gaussians, respectively.  
Emission \(\boldsymbol{\psi}_i\) is attenuated by \(\boldsymbol{\rho}_m\) from preceding Gaussians.
The received signal is the sum of these attenuated emissions.
Figure~\ref{fig_blending} illustrates a ray intersecting four Gaussians, represented as follows:
\begin{equation}
\begin{aligned}
S = & \ \boldsymbol{\psi_1} + 
\boldsymbol{\psi_2} \cdot (1 - \boldsymbol{\rho_1}) + \boldsymbol{\psi_3} \cdot (1 - \boldsymbol{\rho_2}) \cdot (1 - \boldsymbol{\rho_1}) \\
& + \boldsymbol{\psi_4} \cdot (1 - \boldsymbol{\rho_3}) \cdot (1 - \boldsymbol{\rho_2}) \cdot (1 - \boldsymbol{\rho_1})
\end{aligned}
\end{equation}


\textbf{Impact of Gaussian Geometry:}  
In a voxel-based ray tracing algorithm, a ray is assumed to pass through the center of each voxel.  
However, this assumption does not hold in Gaussian-based scene representation.  
For example, in Figure~\ref{fig_blending}, both G2 and G3 intersect the ray, but the intersection point on G2 is closer to its mean than that on G3.  
Even if G2 and G3 share the same emission and attenuation attributes, their contributions to the final received signal differ due to the varying distances of their intersection points from their means.  
These distances affect the probability of each Gaussian influencing the ray.  
Therefore, the blending process in Equation~(\ref{eqn_blending}) should account for Gaussian geometry.




To achieve this, the intersection point is first determined by solving the ray equation~(Equation~\ref{eqn_ray_define}) and the ellipsoid equation~(Equation~\ref{eqn_gaussian}).  
If two solutions exist, their midpoint is taken as the intersection point.  
Next, the distance between the intersection point and the Gaussian’s mean is calculated.  
The intersection probability \(p_{\text{intr}}\) is then determined by evaluating the Gaussian’s PDF at this distance.
Finally, the emission is adjusted by multiplying it by \(p_{\text{intr}}\): \(\boldsymbol{\psi} = p_{\text{intr}} \cdot \boldsymbol{\psi}\).




\subsubsection{CUDA Kernel}
We develop two CUDA kernels for the forward and backward computations in ray tracing.


\textbf{Forward Kernel:}  
Algorithm~\ref{alg_cuda} outlines the forward kernel.  
The inputs include the number of rays in azimuth and elevation, the means, covariance matrices, emissions, and attenuations of all 3D Gaussians, as well as the positions of the receiver and transmitter.  
The output is the received signal computed for all \(360 * 90\) rays.



Specifically, Line 2 projects 3D Gaussians onto the 2D RF plane.  
Line 3 partitions all rays into multiple grids, each containing \(N_{\text{rays}}\) rays in the azimuth and elevation directions, to accelerate processing.  
Line 4 applies the splatting process to identify which Gaussians influence each grid.  
Line 5 records the sorted Gaussians within each grid.  
Finally, Lines 7–12 compute the received signal for each ray in parallel using the complex-valued blending algorithm.



\textbf{Backward Kernel:}
Since the Forward Kernel is invoked for ray tracing forward computation, PyTorch cannot automatically compute the corresponding computation graph gradients.
After computing the received signal \(S\) and the loss \(\mathcal{L}\), PyTorch calculates the gradient \(\frac{\partial \mathcal{L}}{\partial S}\), which is then passed to the Backward Kernel.  
This kernel reverses the computations of the Forward Kernel to compute the gradients for each Gaussian attribute.




\subsection{Training Loss}\label{sec_training_loss}
After calculating the received signal for each ray, the choice of loss function depends on the type of receiver antenna.


\begin{algorithm}[!tp]
\caption{Forward CUDA Kernel for Ray Tracing}
\label{alg_cuda}
\KwIn{$w, h$: numbers of rays in azimuth and elevation}
\KwIn{$M, C$: means \& covariances of all Gaussians}
\KwIn{$E, A$: emissions \& attenuations of all Gaussians}
\KwIn{$L$: positions of receiver and transmitter}
\KwOut{$O$: received signals for all rays}
\SetKwFunction{FMain}{RayTracing}
\SetKwProg{Fn}{Function}{:}{}

\Fn{\FMain{w, h, M, C, E, A, L}}{
    $M'$, $C'$ $\gets$ \text{sphericalGaussian}($M$, $C$, $L$)  \\
    \text{Grids} $\gets$ \text{buildGrid}($w$, $h$) \\
    \text{Idx}, \text{Kys} $\gets$ \text{sphericalSplatting}($M'$, \text{Grids})  \\
    \text{Ranges} $\gets$ \text{computeGridRange}(\text{Idx}, \text{Kys}) \\
    $O$ $\gets$ 0  \\
    \ForAll{grid G in Grids}{
        \ForAll{ray i in G}{
            ra $\gets$ \text{getGridRange}(\text{Ranges}, $g$) \\
            $O$[$i$] $\gets$ \text{Blend}($i$, \text{Idx}, \text{ra} \text{Kys}, $M$', $C'$, $E$, $A$) \\
        }
    }
    \Return $O$
}
\end{algorithm}


\textbf{i) Antenna Array:}
When the receiver is an antenna array, it captures signal power from all directions, represented as a~\((360, 90)\) ground-truth matrix, where 360 and 90 correspond to azimuth and elevation angles, respectively, each with one-degree resolution~\cite{zhao2023nerf}.  
Each matrix entry represents a single real-valued signal power for its corresponding direction.  
The loss function \(\mathcal{L}\) combines the \(\mathcal{L}_{1}\) loss and the Structural Similarity Index Measure (\(\mathcal{L}_{\text{SSIM}}\)) loss: 
\begin{equation}
\label{eqn_loss}
\mathcal{L} = (1 - \lambda) \mathcal{L}_{1} + \lambda \mathcal{L}_{\text{SSIM}}
\end{equation}
where \(\mathcal{L}_{1}\) measures the average difference between the actual and predicted signal power across all rays.  
Since the~\((360, 90)\) matrix can be viewed as an image~(Figure~\ref{fig_vis_d1}), \(\mathcal{L}_{\text{SSIM}}\) evaluates the structural similarity between the predicted and ground-truth "images," helping \ourSystem learn spatial patterns across rays.  
The parameter \(\lambda\) balances these two losses.



\textbf{ii) Single Antenna:}  
For a single antenna, the ground-truth received signal is either a single real-valued signal power or a complex-valued number containing both amplitude and phase, assumed to be the sum of signals from all directions~\cite{zhao2023nerf}.  
Thus, \ourSystem computes the predicted signal by summing the signals from all rays.  
If the received signal is real-valued, the \(\mathcal{L}_{1}\) loss is applied.  
For a complex-valued signal, the \(\mathcal{L}_{1}\) loss is computed separately for amplitude and phase, then averaged to obtain the final loss.
 


\section{Training details}
\label{sec:HFGD:training_settings}

Unless specified otherwise, the training settings for our proposed VPNeXt are similar to existing works that use ViT mask decoders~\cite{cSETR,cSegViT,cMask2Former}.
This includes the AdamW optimizer, a batch size of 16, and the use of clipnorm along with a mask loss that combines focal and dice losses.

Given that this work focuses exclusively on the plain ViT backbone, all the experiments we conducted are based on the plain ViT without pyramid modifications. 
%
Following common practices, the weights of the ViT are initialized through modern pre-training~\cite{cAugReg,cEVA}.

To accommodate new readers in the field, we utilize the commonly used Mean Intersection over Union (mIOU) metric to evaluate the prediction accuracy of our model.

\section{Experimental Protocol}
\label{sec:evaluation}
\subsection{Model and Dataset}
\begin{figure}[t]
    \centering
\includegraphics[width=\linewidth]{fig/instruction.png}
    \caption{Instructions given to the LLM for the bias detecrtion.}
    \label{fig:instruction}
\end{figure}
We utilized Stable Diffusion 3.5-large~\cite{sd3} as our text-to-image (T2I) model and employed GPT-4o~\cite{gpt4} for bias detection as a blackbox model, and DeepSeek-V3~\cite{liu2024deepseek} as an open-sourced model. The LLM receives prompts as illustrated in Figure \ref{fig:instruction}. Through in-context learning techniques, we enhance model performance by exposing it to an exemplar task~\cite{brown2020language}. To evaluate the debiasing performance for occupations, we used the occupation dataset from Stable Bias~\cite{Luccioni_2023} (hereafter referred to as the stable bias profession dataset), which contains 131 occupations sourced from the U.S. Bureau of Labor Statistics (BLS). The dataset composition is detailed in the Appendix A of~\cite{Luccioni_2023}. All input prompts were formatted as ``A portrait photo of [profession]'' to ensure that the T2I model interprets them specifically as occupations rather than other potential meanings. To assess the performance in removing implicit social biases present in prompts beyond occupations, we used the Parti Prompt dataset~\cite{yu2022scaling}, which consists of over 1,600 diverse English prompts designed to comprehensively evaluate text-to-image generation models and test their limitations. For attribute rebalancing, we employed the uniform distribution, as our primary goal was to verify the debiasing capability of our latent variable guidance.

% For experiments involving bias adjustment using employment statistics log-probabilities, we conducted experiments to mitigate gender bias using BLS2022 statistical data for five occupation prompts mentioned in \cite{naik2023social}: ``CEO'', ``doctor'', ``computer programmer'', ``house keeper'', and ``nurse''.

\subsection{Human Evaluation}
For each prompt, nine images are generated using three methods: a baseline method without debiasing, and two LLM-assisted debiasing methods employing GPT-4o and DeepSeek-V3. These images are arranged in a 3 $\times$ 3 grid, and evaluators assess pairs of images based on image quality, prompt reflection, and diversity of generations. Image quality refers to the aesthetic appeal, high resolution, natural appearance, and detailed refinement of the images. Prompt adherence measures the degree to which the generated images reflect the input text. Diversity of generations evaluates the variety of generated results, particularly whether the images avoid stereotypes and fixed patterns. For each criterion, evaluators rate the results on a 5-point scale, ranging from 1 (very poor) to 5 (very good). To facilitate relative comparisons, images generated by different models for the same input prompt are presented in consecutive questions. This comparative evaluation across the three criteria enables a detailed assessment of the proposed methods' relative strengths and limitations. We randomly selected 50 prompts from Stable Bias profession dataset and Parti Prompt dataset. The subset used for the human evaluation is detailed in Table\ref{tab:sd_subset} and Table\ref{tab:pp_subset} in the supplementary materials. Responses were collected from 20 evaluators, ensuring a diverse range of perspectives. 
\subsection{Non-parametric Evaluation}
Quantitative evaluation of generation diversity presents significant challenges. To address this, we adopt the clustering-based evaluation methodology proposed in Stable Bias~\cite{Luccioni_2023}, implementing a nonparametric diversity assessment using k-Nearest Neighbors (kNN)~\cite{fix1985discriminatory}. Specifically, we generate anchor images based on prompts structured as ``a portrait of a [ethnicity] [gender] at work,'' creating nine images for each combination of ethnicity and gender. This analysis employs 18 ethnic labels from Stable Bias and three gender categories: ``male'', ``female'', and ``non-binary'' (detailed ethnic labels are provided in the Appendix A of~\cite{Luccioni_2023}).

For image embeddings, we utilize Google's VertexAI multimodal embedding model\footnote{https://cloud.google.com/vertex-ai/docs/generative-ai/embeddings/get-multimodal-embeddings}, which converts 512 $\times$ 512 images into 1048-dimensional vector representations. For each prompt in the identity dataset, 30 unique images are generated, yielding a total of 54 $\times$ 30 $=$ 1620 images that serve as anchor points for classification. To examine local trends linked to specific professions, we follow the methodology outlined in \cite{naik2023social}, generating 210 images per method for five professions: ``CEO'', ``computer programmer'', ``doctor'', ``nurse'', and ``housekeeper''. The classification results are visualized to uncover potential biases or distinct patterns specific to each profession.

% In addition, to capture global trends across the entire profession dataset, we generate nine images per profession prompt for each method. These classification results provide an overarching perspective on diversity and potential biases in the generated outputs.



\section{Conclusion and future directions} \label{sec:conclusion}

In this paper we proposed a nested MLMC framework that offers important computational savings by performing most calculations in low precision and exploiting approximate random normal variables for the low precision path calculations. The low precision calculations could be performed in fixed precision on an FPGA for greater efficiency, and we suggested a procedure to optimise the bit-widths of every variable at each Monte Carlo level. This is an important improvement over previous mixed precision MLMC frameworks which held the lower precision fixed \cite{Rounding_error_oliver} or defined uniform bit-width at every level heuristically \cite{brugger2014mixed}. Our numerical results suggest that for the first levels our procedure reduces the cost at these levels by a factor 5 or 7. Hence the overall savings are significant since most paths are calculated on the first levels. Our approach would be even more efficient for the Milstein scheme because its higher order strong convergence leads to a greater proportion of the computational costs being on the coarsest levels.

The next stage of the research project will be to implement the RNG methods and the nested framework on FPGAs to determine the hardware requirements and confirm the extent of the computational savings. It would also be good to compare the performance benefits to using half-precision floating point arithmetic on GPUs or CPUs for the low-accuracy computations.





\bibliographystyle{unsrt}
% Template for ISBI paper; to be used with:
%          spconf.sty  - ICASSP/ICIP LaTeX style file, and
%          IEEEbib.bst - IEEE bibliography style file.
% --------------------------------------------------------------------------
\documentclass{article}
\usepackage{spconf,amsmath,graphicx}

% It's fine to compress itemized lists if you used them in the
% manuscript
\usepackage{enumitem}
\usepackage{multirow}
\setlist{nosep, leftmargin=14pt}
\usepackage{booktabs}
\usepackage{caption}
\usepackage{subcaption}
\usepackage{mwe} % to get dummy images
\usepackage{url}

% Example definitions.
% --------------------
\def\x{{\mathbf x}}
\def\L{{\cal L}}

% Title.  
% ------
\title{Anatomical Grounding Pre-training for Medical Phrase Grounding}
%
% Single address.
% ---------------
% \name{Wenjun Zhang, Shakes Chandra, Aaron Nicolson}
% \address{University of Queensland}
%
% For example:
% ------------
%\address{School\\
%	Department\\
%	Address}
%
% Two addresses (uncomment and modify for two-address case).
% ----------------------------------------------------------
%\twoauthors
%  {A. Author-one, B. Author-two\sthanks{Some author footnote.}}
%	{School A-B\\
%	Department A-B\\
%	Address A-B}
%  {C. Author-three, D. Author-four\sthanks{The fourth author performed the work
%	while at ...}}
%	{School C-D\\
%	Department C-D\\
%	Address C-D}
%
% More than two addresses
% -----------------------
\name{Wenjun Zhang$^{\star}$ \qquad Shekhar S. Chandra$^{\star}$ \qquad Aaron Nicolson$^{\dagger}$}

\address{$^{\star}$University of Queensland\\$^{\dagger}$Australian e-Health Research Centre, CSIRO Health and Biosecurity, Brisbane, Australia}
% \address{$^{\dagger}$}Australian e-Health Research Centre, CSIRO Health and Biosecurity, Brisbane, Australia \\}

\begin{document}
%\ninept
%
\maketitle

\begin{abstract}

Medical Phrase Grounding (MPG) maps radiological findings described in medical reports to specific regions in medical images. The primary obstacle hindering progress in MPG is the scarcity of annotated data available for training and validation. We propose anatomical grounding as an in-domain pre-training task that aligns anatomical terms with corresponding regions in medical images, leveraging large-scale datasets such as Chest ImaGenome. Our empirical evaluation on MS-CXR demonstrates that anatomical grounding pre-training significantly improves performance in both a zero-shot learning and fine-tuning setting, outperforming state-of-the-art MPG models. Our fine-tuned model achieved state-of-the-art performance on MS-CXR with an mIoU of 61.2, demonstrating the effectiveness of anatomical grounding pre-training for MPG.

% Phrase grounding models maps phrases to specific regions in an image, while for medical phrase grounding, the phrase 

\end{abstract}

\section{Introduction}
MPG involves mapping a descriptive phrase containing a radiological finding to a specific region in a medical image \cite{10.1007/978-3-031-43990-2_35}. An MPG model could be used to visually connect findings in a radiologist report---whether produced by radiologist or by automatic report generation model---to the corresponding regions in the images. Findings accompanied by their associated bounding boxes are easier to verify, enhancing the reliability of reported information \cite{bernstein_can_2023, 10204026, doi:10.1148/ryai.2020190043}.

MPG is a specialised application within the broader field of phrase grounding. State-of-the-art general-domain phrase grounding models are pre-trained on large-scale phrase-to-region datasets and demonstrate strong zero-shot learning and few-shot transferability on downstream localisation tasks \cite{9879567, 9710994, 10.1007/978-3-031-72970-6_3}. However, despite their success in general-domain tasks, these models struggle to generalise to MPG, especially in a zero-shot learning setting. One possible reason is the significant domain shift from general-domain to medical-domain data \cite{zhao-titov-2023-transferability}. Furthermore, large-scale pre-training is challenging in the medical domain due to the scarcity of annotated MPG datasets, with only a small public benchmark dataset available \cite{10.1007/978-3-031-20059-5_1}. 

To overcome the challenges of limited MPG training data and the large domain gap between MPG and the general phrase grounding data, we propose to leverage anatomical grounding as an in-domain pre-training task for MPG, as demonstrated in Figure \ref{fig:concept} (middle). Anatomical grounding involves aligning text describing an anatomical region with the corresponding region within a medical image. This approach leverages the extensive anatomical text-to-region data available in datasets such as Chest ImaGenome \cite{wu2021chestimagenomedatasetclinical}, enabling effective fine-tuning or zero-shot learning for MPG tasks, where data is more limited \cite{ 10.1007/978-3-031-20059-5_1}. This pre-training step equips the model to recognise common anatomical landmarks, which radiologists frequently reference when describing findings in radiological reports. For instance, by learning to localise the \textit{right apical zone} with the Chest ImaGenome dataset, the model is more capable of localising findings such as a \textit{small right apical pneumothorax}.


\begin{figure}
    \centering
    \includegraphics[width=1\linewidth]{concept.png}
    \caption{Anatomical grounding as an in-domain pre-training task for Medical Phrase Grounding (MPG).}
    \label{fig:concept}
\end{figure}

We evaluated the effectiveness of anatomical grounding pre-training on MS-CXR, a MPG dataset, using two pre-trained general-domain phrase grounding models, TransVG \cite{9710016} and MDETR \cite{9710994}. We also evaluate it in both a zero-shot learning and a fine-tuning setting. Figure \ref{fig:concept} describes the training process; TransVG or MDETR is first pre-trained on anatomical grounding. They are then fine-tuned on MPG (if they are not evaluated in a zero-shot learning setting). Our empirical evaluation demonstrates that anatomical grounding pre-training significantly improves performance in a zero-shot learning setting, and significantly improves the performance of MDETR in a fine-tuning setting. We compare our anatomically grounded pre-trained models to state-of-the-art MPG models from the literature, and demonstrate that our models achieve an improvement in performance. The pre-trained models, and demo for this work are available at: \url{https://github.com/Claire1217/AGPT}.


\section{Related Work}
\subsection{General-domain Phrase Grounding}
Vision-language models pre-trained on large-scale image-text datasets, such as CLIP, have shown strong zero-shot learning and few-shot learning capabilities on global image understanding tasks \cite{pmlr-v139-radford21a}. GLIP extends this by pre-training on large-scale phrase grounding data \cite{9879567}. The learned representations demonstrate strong transferability to various local-level recognition tasks. Current pre-trained general-domain phrase grounding models are typically applied to two primary tasks: phrase localisation and referring expression comprehension. Phrase localisation focuses on identifying and locating multiple objects mentioned in a sentence. MDETR is a phrase localisation model, associating sub-phrases within a sentence with multiple object queries \cite{9710994}. In contrast, TransVG is a referring expression comprehension model---it detects a single object or region in an image for a whole sentence \cite{9710016}.

\subsection{Medical Phrase Grounding}
Due to the scarcity of annotated data, MPG has received limited attention in the literature. Boecking \textit{et al.} introduced MS-CXR, a phrase grounding chest X-ray benchmark dataset \cite{10.1007/978-3-031-20059-5_1}. Their objective with the dataset was to evaluate the grounding performance of their self-supervised biomedical vision-language model (BioViL). BioViL demonstrates strong zero-shot learning capabilities, given that it is not trained for MPG. Recently, Chen \textit{et al.} directly fine-tuned TransVG on a split of MS-CXR in order to directly learn MPG, forming MedRPG \cite{10.1007/978-3-031-43990-2_35}. Here, a bounding box supervised loss and a specific contrastive loss were leveraged. Unlike these models, we pre-train on large-scale anatomical grounding data using Chest ImaGenome, in order to provide in-domain pre-training.

\subsection{Anatomical Information in Medical Imaging}
Anatomical information has been effectively used in tasks like pathology detection and classification to improve accuracy and localisation. For example, the Anatomy-Driven Pathology Detection (ADPD) model \cite{muller_anatomy-driven_2023} used easy-to-annotate anatomical regions as proxies for pathologies, helping to locate disease locations without detailed pathology-specific bounding boxes. AnaXNet \cite{agu_anaxnet_2021} used anatomical relationships to improve classification by identifying the exact regions where findings occur. Despite these successes, no work has applied anatomical information to medical phrase grounding. 

\section{Methodology}\label{sec:methodology}
Our work addresses \textbf{medical phrase grounding} (MPG),  which involves mapping a descriptive phrase containing radiological finding to a specific
region in a medical image. This can be defined as learning a function  \( f: P \times I \rightarrow B \), where \( P \) represents the set of medical phrases, \( I \) represents the set of medical images, and \( B \) represents the set of bounding boxes. Given a phrase \( p \in P \) and an image \( i \in I \), the model predicts a bounding box \( b \in B \) such that $b = f(p, i)$. Our approach introduces a novel training framework for MPG, which involves extending the pre-training of general phrase grounding models with an anatomical grounding pre-training. 

Anatomical grounding involves predicting bounding boxes for anatomical structures using textual descriptions of their locations. The task can be formulated as 
\( f_{\text{anat}}: A \times I \rightarrow B \). Specifically, for each anatomical term \( a \in A \) and image \( i \in I \), the model predicts a bounding box \( b \in B \) such that $b = f_{\text{anat}}(a, i; \theta_{\text{gen}})$, 
where \( \theta_{\text{gen}} \) are the initial general-domain pre-trained weights. Through anatomical grounding pre-training, we refine the weights to create anatomy-specific parameters  \( \theta_{\text{anat}} \). 

To enhance generalisation and robustness, we leverage GPT-4 to generate four additional synonymous variations for each anatomical location in the Chest ImaGenome dataset. This aligns with clinical practice, where radiologists frequently use interchangeable terms to describe the same region. For example, ``left lung base” might also be referred to as ``left basal lung” or ``left lower lung base”. The detailed augmentation of anatomical regions is included in the aforementioned code repository. 

\section{Datasets} \label{sec:dataset}

\paragraph*{Chest ImaGenome \cite{wu2021chestimagenomedatasetclinical}}
We use the Chest ImaGenome dataset for anatomical grounding pre-training. Chest ImaGenome is a scene graph-structured dataset that includes $242\,072$ images. It contains $1\,256$ combinations of relational annotations between 29 anatomical structures in chest X-rays, with bounding box coordinates and additional attributes organised as a scene graph per image. In this study, we use the names and bounding box coordinates of these 29 anatomical structures, focusing specifically on frontal images. Examples of anatomical structures include ``left lung base", ``left lung apical zone", and ``right hilar structures".

\paragraph*{MS-CXR \cite{10.1007/978-3-031-20059-5_1}} 
We use the MS-CXR dataset for the MPG task. It contains $1\,162$ medical phrase-bounding box pairs across eight pathologies, such as \textit{cardiomegaly} and \textit{pleural effusion}. The findings are manually annotated and described by radiologists, ensuring precise alignment between medical phrases and bounding boxes. Example phrases include ``Large right-sided pneumothorax", and ``Small bilateral pleural effusions". The whole dataset was used for testing for the zero-shot learning setting with the general-domain pre-trained and anatomical pre-trained phrase grounding models, while the train-test-val split from \cite{10.1007/978-3-031-43990-2_35} was used for the fine-tuning setting. 

\section{Experiment Setup}
\paragraph*{Model}
Experiments were conducted with two models, TransVG and MDETR. For TransVG, ResNet-50 and ClinicalBERT were used as the visual and text encoders, respectively, whereas ResNet-101 and RoBERTa-base were used for MDETR. Here, MDETR functions on a sentence-level, mapping a medical phrase to one region in an image. This differs from its standard function, where it maps multiple sub-phrases from a sentence to multiple regions in the image. Full-model anatomical grounding pre-training of MDETR resulted in an unstable training process, likely due to its multi-object detection task. To address this, we applied Low-Rank Adaptation (LoRA) \cite{Hu2021LoRA:Models} during anatomical grounding pre-training. This likely stabilised training by limiting trainable parameters to low-rank layers, preventing drastic weight updates and reducing instability during adaptation.

\paragraph*{Pre-training and Fine-Tuning}
For anatomical grounding pre-training, we process mini-batches of eight images, each paired with five anatomical regions chosen from five synonymous terms, creating 40 anatomical text-region pairs per mini-batch. For MPG fine-tuning, both models were trained on the MS-CXR training set with a mini-batch size of 12. During fine-tuning, all of the weights of MDETR were trainable, including the LoRA weights. The AdamW optimiser with a learning rate of 1e-4 and 1e-5 was used for pre-training and fine-tuning, respectively \cite{DBLP:conf/iclr/LoshchilovH19}. Each model was trained for 1 epoch during pre-training and 90 epochs during fine-tuning. Images were resized and padded to a size of 640$\times$640. During training, the images were augmented with colour jitter and Gaussian noise.

% When fine-tuning the anatomical pre-trained models on the training set of MS-CXR. The task is formulated as \( f_{\text{MPG}}: P \times I \rightarrow B \), where given a medical phrase \( p \in P \) and an image \( i \in I \), the task is to produce a bounding box \( b \in B \) as follows: $b = f_{\text{MPG}}(p, i; \theta_{\text{anat}})$. With fine-tuning, the weights are updated to \( \theta_{\text{MPG}} \). 

\paragraph*{Evaluation}
We used mIoU and accuracy (Acc) as metrics. For accuracy, a predicted bounding box was considered true if the mIoU with the ground truth bounding box was larger than 0.5. We evaluate the anatomical grounding pre-trained MDETR and TransVG models on the MS-CXR dataset in both zero-shot learning and fine-tuning settings. The self-supervised pre-trained models GLoRIA \cite{9710099} and BioViL \cite{10.1007/978-3-031-20059-5_1} were used for comparison. In the fine-tuning setting, we further fine-tuned the anatomical grounding pre-trained MDETR and TransVG models on the training split of MS-CXR (described in Section \ref{sec:dataset}). These were compared to MDETR and TransVG without anatomical grounding pre-training and MedRPG \cite{10.1007/978-3-031-43990-2_35}. For zero-shot learning and fine-tuning, the epoch with the highest validation mIoU was selected for testing.

\section{Results \& Discussion}
\subsection{Effectiveness of Anatomical Grounding Pre-training}
The performance of anatomical grounding pre-training is demonstrated in Table \ref{tab:anat_comparison}. Applying MDETR and TransVG to MPG in a zero-shot learning setting produced low scores on both metrics, underscoring the limitations of general-domain phrase grounding models for MPG. However, pre-training with anatomical grounding led to a statistically significant improvement in both models’ performance across both metrics for zero-shot learning of MPG. These results demonstrate that anatomical grounding pre-training improves the models’ ability to generalise to MPG.

\begin{table}[ht]
\small
\centering
\caption{Performance of \textbf{anatomical grounding pre-training (AGPT)} on MS-CXR. Underlined indicates a stat. sig. difference to the model without anatomical grounding pre-training ($p < 0.05)$.}
\renewcommand{\arraystretch}{0.85}
\begin{tabular}{lcccc}
\toprule
\multirow{2}{*}{\textbf{Model}} & \multicolumn{2}{c}{\textbf{Zero-shot}} & \multicolumn{2}{c}{\textbf{Fine-tuning}} \\
\cmidrule(lr){2-3} \cmidrule(lr){4-5}
                                & \textbf{Acc}   & \textbf{mIoU}   & \textbf{Acc}    & \textbf{mIoU}   \\
\midrule
TransVG           & 1.2         & 10.3         & 68.9          & \textbf{59.4}          \\
\quad+ AGPT   & \textbf{\underline{39.8}}         & \textbf{\underline{40.7}}         & \textbf{70.7}          & 59.2 \\
\addlinespace % Adds space between the two groups
MDETR              & 3.0         & 14.7         & 66.9          & 57.8         \\
\quad+ AGPT   &\textbf{ \underline{34.7}}         & \textbf{\underline{32.6}}         & \textbf{\underline{70.7} }         & \textbf{\underline{61.2}} \\
\bottomrule
\end{tabular}
\label{tab:anat_comparison}
\end{table}


When fine-tuning on the MS-CXR training set, anatomical grounding pre-training produced a statistically significant improvement across all metrics for MDETR. It also demonstrated an improvement with TransVG for Acc. This indicates that anatomical grounding pre-training is effective for MPG fine-tuning, particularly for certain types of models.

In Figure \ref{fig:viz}, we illustrate the models performing MPG in zero-shot learning settings on two examples: ``right apical pneumothorax" and ``mild cardiomegaly". Without anatomical grounding pre-training, both TransVG and MDETR fail to ground the phrases accurately. However, with anatomy pre-training, both models are able to ground the text to the correct anatomical region---the right apical zone for pneumothorax and the heart for cardiomegaly. Fine-tuning offers a further improvement in the grounding accuracy.

\begin{figure}[t]
    \centering
    \includegraphics[width=1\linewidth]{vizz.png}
    \caption{MPG with and without \textbf{anatomical grounding pre-training (AGPT)}. The top example contains the anatomical region within the text, whereas the bottom example does not. Blue and red boxes indicate the ground-truth and predicted bounding boxes, respectively.}
    \label{fig:viz}
\end{figure}



\begin{table*}[t]
\small
\centering
\caption{A comparison of \textbf{anatomical grounding pre-training (AGPT)} with other models in the literature in both zero-shot learning and fine-tuning settings with \textbf{mIoU} as the metric. $\dagger$ indicates scores sourced from the BioViL paper \cite{10.1007/978-3-031-20059-5_1}.}
\label{table:combined_iou_scores}
\renewcommand{\arraystretch}{0.85}
\begin{tabular}{l c c c c c c c c c c}
\toprule
\textbf{Model} & \textbf{Supervision} & \textbf{Cardio.} & \textbf{Opacity} & \textbf{Edema} & \textbf{Consol.} & \textbf{Pneu.} & \textbf{Atelect.} & \textbf{Pneumo.} & \textbf{Pl. Eff.} & \textbf{Avg} \\
\midrule
\multicolumn{11}{c}{\textbf{Zero-shot learning}} \\ 
\midrule
\textbf{GLoRIA \cite{9710099}}$^\dagger$ & Self-super. & 27.3 & 19.8 & 25.1 & 32.4 & 24.6 & 26.1 & 10.0 & 25.4 & 24.6 \\
\textbf{BioViL \cite{10.1007/978-3-031-20059-5_1}}$^\dagger$ & Self-super. & 37.5 & 20.9 & \textbf{27.5} & \textbf{34.6} & 31.5 & 30.2 & 13.5 & \textbf{31.5} & 28.4 \\
\cmidrule(lr){1-11}
\textbf{MDETR + AGPT} & Box-super. & 61.3 & 6.0 & 8.7 & 18.5 & 18.8 & 8.2 & 16.1 & 14.6 & 32.6 \\
\textbf{TransVG + AGPT} & Box-super. & \textbf{61.5} & \textbf{23.0} & 14.5 & 33.0 & \textbf{31.9} & \textbf{39.3} & \textbf{26.9} & 21.1 & \textbf{40.7} \\
\midrule
\multicolumn{11}{c}{\textbf{Fine-tuning}} \\ 
\midrule
\textbf{MedRPG \cite{10.1007/978-3-031-43990-2_35}} & Box-super. & 80.5 & 39.3 & \textbf{51.7} & 49.1 & 46.4 & \textbf{48.8} & 38.5 & 52.8 & 59.6 \\
\textbf{MDETR \cite{9710994}} & Box-super. & 79.6 & 43.1 & 45.5 & 45.8 & 40.1 & 36.0 & 39.1 & 50.5 & 57.8 \\
\textbf{TransVG \cite{9710016}} & Box-super. & 80.6 & \textbf{46.8} & 35.6 & 42.7 & \textbf{48.5} & 42.8 & 38.3 & 49.5 & 59.4 \\
\cmidrule(lr){1-11}
\textbf{MDETR + AGPT} & Box-super. & \textbf{81.2} & 45.1 & 25.2 & \textbf{56.3} & 38.9 & 47.4 & \textbf{43.1} & \textbf{57.2} & \textbf{61.2} \\
\textbf{TransVG + AGPT} & Box-super. & 79.1 & 37.6 & 43.0 & 45.4 & 45.9 & 47.7 & 41.9 & 54.1 & 59.2 \\
\bottomrule
\end{tabular}
\end{table*}
 
\subsection{Comparison to other MPG models}
First, we compare our anatomical grounding pre-trained MDETR and TransVG models (MDETR + AGPT and TransVG + AGPT, respectively) in a zero-shot learning setting, as shown in Table \ref{table:combined_iou_scores}. We compare these to two self-supervised models, GLoRIA \cite{9710099} and BioViL \cite{10.1007/978-3-031-20059-5_1}. Both MDETR + AGPT and TransVG + AGPT outperformed GLoRIA and BioViL. This indicates that anatomical grounding pre-training is more effective for zero-shot learning MPG than the self-supervised learning strategies of GLoRIA and BioViL. Furthermore, our fine-tuned MDETR + AGPT model attained an mIoU improvement of 1.6 over the current state-of-the-art model, MedRPG \cite{10.1007/978-3-031-43990-2_35}. 

% The baseline performance for the two models is from the BioViL paper.

\subsection{Effectiveness of Synonymous Anatomical Term Augmentation}
We conducted ablation studies to evaluate the impact of adding synonymous variations of the anatomical locations, as described in Section \ref{sec:methodology}. The results show that synonymous augmentation improved the scores for both TransVG and MDETR, with a stronger effect observed in TransVG. Notably, anatomical grounding pre-training with synonymous augmentation led to a 15.6\% improvement in zero-shot learning accuracy. This provides the model with a broader range of terms for the same anatomical location. This allows the model to better generalise to new phrases in a zero-shot learning setting.
\begin{table}[h!]
\small
\centering
\caption{Improvement in performance with when using synonymous variations of the anatomical locations. Underlined indicates a stat. sig. difference to the model without synonymous variations ($p < 0.05)$.}
\label{table:syn_augmentation_effect}
\renewcommand{\arraystretch}{0.85}
\begin{tabular}{lcccc}
\toprule
\multirow{2}{*}{\textbf{Model}}  & \multicolumn{2}{c}{\textbf{Zero-shot}} & \multicolumn{2}{c}{\textbf{Fine-tuning}} \\
\cmidrule(lr){2-3} \cmidrule(lr){4-5}
 & \textbf{Acc} & \textbf{mIoU} & \textbf{Acc} & \textbf{mIoU} \\
\midrule
TransVG   & \underline{+15.6} & \underline{+13.9} & \underline{+5.4} & \underline{+2.6} \\
MDETR     & \underline{+1.8}  & \underline{+1.2}   & +2.4 & +0.9 \\
\bottomrule
\end{tabular}
\end{table}

\vspace{-10pt}
\section{Conclusion}
In this paper, we introduced anatomical grounding pre-training to address the challenges of MPG, a task constrained by limited in-domain data and significant domain shifts from general-domain pre-trained models. Our methodology involved pre-training phrase grounding models on anatomical text-region pairs using the Chest ImaGenome dataset, followed by MPG-specific fine-tuning on the MS-CXR dataset. Empirical results demonstrated that anatomical grounding pre-training significantly improved zero-shot learning and fine-tuning performance on MPG, surpassing existing self-supervised and state-of-the-art MPG models. Additionally, our augmentation with synonymous anatomical terms further enhanced generalisability. This work demonstrates that leveraging anatomical grounding pre-training is an effective solution to the challenge of limited MPG data.

\section{Compliance with Ethical Standards}  
\vspace{-2mm} % Reduce space above section title
\noindent This study used public data from MIMIC-CXR (under PhysioNet’s credentialed license). Ethical approval was not required as confirmed by the license attached with the open access data.
\vspace{-2mm}

\section*{Acknowledgments}
\vspace{-2mm} % Reduce space before acknowledgments
\noindent No funding was received. The authors declare no competing interests.  

% Tighten bibliography spacing
\vspace{-5mm} % Reduce space before references


\bibliographystyle{IEEEtran}
\bibliography{isbi/ISBI_2024_template-master/references}
\end{document}





\end{document}

