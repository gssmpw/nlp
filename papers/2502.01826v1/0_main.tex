
\documentclass[sigconf,10pt]{acmart}

\usepackage{tikz}
\usepackage{amsmath}

\usepackage[linesnumbered, ruled]{algorithm2e}

\usepackage{colortbl}
\usepackage{eucal}
\usepackage{bm}
\usepackage{subfigure}
\usepackage{booktabs}
\usepackage{multirow}
\usepackage{enumitem}
\usepackage{url}
\usepackage{xspace}
\usepackage{amsfonts}
\usepackage{booktabs}


\usepackage[english]{babel}
\usepackage{blindtext}

\usepackage{pifont} 
\usepackage{stackengine}

\usepackage{setspace}

\setlength{\textfloatsep}{5pt}
\setlength{\abovecaptionskip}{1pt}
\setlength{\belowcaptionskip}{2pt}



\newcolumntype{C}[1]{>{\centering\arraybackslash}p{#1}}
\newcolumntype{L}[1]{>{\raggedright\arraybackslash}p{#1}}



\renewcommand\footnotetextcopyrightpermission[1]{}

\settopmatter{printacmref=false, printccs=false, printfolios=true}

\newcommand{\authoraff}{\fontsize{10.3pt}{11pt} \selectfont}

\newcommand{\authorname}{\fontsize{14pt}{11pt} \selectfont}


\def\ourSystem{\textit{RFSPM}\xspace}

\def\ie{\textit{i.e.},\xspace}
\def\eg{\textit{e.g.},\xspace}
\def\etl{\textit{et al.}\xspace}
\def\vs{\textit{vs.}\xspace}
\def\nerft{{NeRF$^2$}\xspace}

\setlength{\pdfpagewidth}{8.5in}
\setlength{\pdfpageheight}{11in}


\begin{document}


\title{Scalable 3D Gaussian Splatting-Based RF Signal Spatial Propagation Modeling}


    \author{\authorname Kang Yang}
    \affiliation{ \authoraff
    \institution{University of California, Los Angeles}
    \city{Los Angeles}
    \state{California}
    \country{USA}
    }
    \email{kyang73@g.ucla.edu}
    \authornote{This work was partially done when Kang Yang was a PhD student in Dr. Wan Du’s group at UC Merced.}
    
    \author{\authorname Gaofeng Dong}
    \affiliation{ \authoraff
    \institution{University of California, Los Angeles}
    \city{Los Angeles}
    \state{California}
    \country{USA}
    }
    \email{gfdong@g.ucla.edu}

    \author{\authorname Sijie Ji}
    \affiliation{\authoraff
    \institution{University of California, Los Angeles}
    \city{Los Angeles}
    \state{California}
    \country{USA}
    }
    \affiliation{\authoraff
    \institution{California Institute of Technology}
    \city{Pasadena}
    \state{California}
    \country{USA}
    }
    \email{sijieji@caltech.edu}

    \author{\authorname Wan Du}
    \affiliation{
    \institution{\authoraff University of California, Merced}
    \city{Merced}
    \state{California}
    \country{USA}
    }
    \email{wdu3@ucmerced.edu}
    \authornote{Wan Du is the corresponding author.}
    
    \author{\authorname Mani Srivastava}
    \affiliation{
    \institution{\authoraff University of California, Los Angeles \\ and Amazon}
    \city{Los Angeles}
    \state{California}
    \country{USA}
    }
    \email{mbs@ucla.edu}
    \authornote{Mani Srivastava holds concurrent appointments as a Professor of ECE and CS~(joint) at the University of California, Los Angeles and as an Amazon Scholar. This paper describes work performed at the University of California, Los Angeles and is not associated with Amazon.}
% }



\begin{abstract}



Effective network planning and sensing in wireless networks require resource-intensive site surveys for data collection.  
An alternative is Radio-Frequency~(RF) signal spatial propagation modeling, which computes received signals given transceiver positions in a scene~(\eg a conference room).  
We identify a fundamental trade-off between scalability and fidelity in the state-of-the-art method. To address this issue, we explore leveraging 3D Gaussian Splatting~(3DGS), an advanced technique for the image synthesis of 3D scenes in real-time from arbitrary camera poses.  
By integrating domain-specific insights, we design three components for adapting 3DGS to the RF domain, including Gaussian-based RF scene representation, gradient-guided RF attribute learning, and RF-customized CUDA for ray tracing.
Building on them, we develop \ourSystem, an end-to-end framework for scalable \textbf{RF} signal \textbf{S}patial \textbf{P}ropagation \textbf{M}odeling.  
We evaluate \ourSystem in four field studies and two applications across RFID, BLE, LoRa, and 5G, covering diverse frequencies, antennas, signals, and scenes.  
The results show that \ourSystem matches the fidelity of the state-of-the-art method while reducing data requirements, training GPU-hours, and inference latency by up to 9.8\,$\times$, 18.6\,$\times$, and 84.4\,$\times$, respectively.



\end{abstract}

\keywords{RF Signal Spatial Propagation Modeling, Scalability}


\maketitle
\pagestyle{plain}

\section{Introduction}

Video generation has garnered significant attention owing to its transformative potential across a wide range of applications, such media content creation~\citep{polyak2024movie}, advertising~\citep{zhang2024virbo,bacher2021advert}, video games~\citep{yang2024playable,valevski2024diffusion, oasis2024}, and world model simulators~\citep{ha2018world, videoworldsimulators2024, agarwal2025cosmos}. Benefiting from advanced generative algorithms~\citep{goodfellow2014generative, ho2020denoising, liu2023flow, lipman2023flow}, scalable model architectures~\citep{vaswani2017attention, peebles2023scalable}, vast amounts of internet-sourced data~\citep{chen2024panda, nan2024openvid, ju2024miradata}, and ongoing expansion of computing capabilities~\citep{nvidia2022h100, nvidia2023dgxgh200, nvidia2024h200nvl}, remarkable advancements have been achieved in the field of video generation~\citep{ho2022video, ho2022imagen, singer2023makeavideo, blattmann2023align, videoworldsimulators2024, kuaishou2024klingai, yang2024cogvideox, jin2024pyramidal, polyak2024movie, kong2024hunyuanvideo, ji2024prompt}.


In this work, we present \textbf{\ours}, a family of rectified flow~\citep{lipman2023flow, liu2023flow} transformer models designed for joint image and video generation, establishing a pathway toward industry-grade performance. This report centers on four key components: data curation, model architecture design, flow formulation, and training infrastructure optimization—each rigorously refined to meet the demands of high-quality, large-scale video generation.


\begin{figure}[ht]
    \centering
    \begin{subfigure}[b]{0.82\linewidth}
        \centering
        \includegraphics[width=\linewidth]{figures/t2i_1024.pdf}
        \caption{Text-to-Image Samples}\label{fig:main-demo-t2i}
    \end{subfigure}
    \vfill
    \begin{subfigure}[b]{0.82\linewidth}
        \centering
        \includegraphics[width=\linewidth]{figures/t2v_samples.pdf}
        \caption{Text-to-Video Samples}\label{fig:main-demo-t2v}
    \end{subfigure}
\caption{\textbf{Generated samples from \ours.} Key components are highlighted in \textcolor{red}{\textbf{RED}}.}\label{fig:main-demo}
\end{figure}


First, we present a comprehensive data processing pipeline designed to construct large-scale, high-quality image and video-text datasets. The pipeline integrates multiple advanced techniques, including video and image filtering based on aesthetic scores, OCR-driven content analysis, and subjective evaluations, to ensure exceptional visual and contextual quality. Furthermore, we employ multimodal large language models~(MLLMs)~\citep{yuan2025tarsier2} to generate dense and contextually aligned captions, which are subsequently refined using an additional large language model~(LLM)~\citep{yang2024qwen2} to enhance their accuracy, fluency, and descriptive richness. As a result, we have curated a robust training dataset comprising approximately 36M video-text pairs and 160M image-text pairs, which are proven sufficient for training industry-level generative models.

Secondly, we take a pioneering step by applying rectified flow formulation~\citep{lipman2023flow} for joint image and video generation, implemented through the \ours model family, which comprises Transformer architectures with 2B and 8B parameters. At its core, the \ours framework employs a 3D joint image-video variational autoencoder (VAE) to compress image and video inputs into a shared latent space, facilitating unified representation. This shared latent space is coupled with a full-attention~\citep{vaswani2017attention} mechanism, enabling seamless joint training of image and video. This architecture delivers high-quality, coherent outputs across both images and videos, establishing a unified framework for visual generation tasks.


Furthermore, to support the training of \ours at scale, we have developed a robust infrastructure tailored for large-scale model training. Our approach incorporates advanced parallelism strategies~\citep{jacobs2023deepspeed, pytorch_fsdp} to manage memory efficiently during long-context training. Additionally, we employ ByteCheckpoint~\citep{wan2024bytecheckpoint} for high-performance checkpointing and integrate fault-tolerant mechanisms from MegaScale~\citep{jiang2024megascale} to ensure stability and scalability across large GPU clusters. These optimizations enable \ours to handle the computational and data challenges of generative modeling with exceptional efficiency and reliability.


We evaluate \ours on both text-to-image and text-to-video benchmarks to highlight its competitive advantages. For text-to-image generation, \ours-T2I demonstrates strong performance across multiple benchmarks, including T2I-CompBench~\citep{huang2023t2i-compbench}, GenEval~\citep{ghosh2024geneval}, and DPG-Bench~\citep{hu2024ella_dbgbench}, excelling in both visual quality and text-image alignment. In text-to-video benchmarks, \ours-T2V achieves state-of-the-art performance on the UCF-101~\citep{ucf101} zero-shot generation task. Additionally, \ours-T2V attains an impressive score of \textbf{84.85} on VBench~\citep{huang2024vbench}, securing the top position on the leaderboard (as of 2025-01-25) and surpassing several leading commercial text-to-video models. Qualitative results, illustrated in \Cref{fig:main-demo}, further demonstrate the superior quality of the generated media samples. These findings underscore \ours's effectiveness in multi-modal generation and its potential as a high-performing solution for both research and commercial applications.


\section{Related Work}\label{sec_relatedWork}

\textbf{Generalization on Optical NeRF.}  
Optical NeRF has revolutionized novel view synthesis by learning 3D scene representations from 2D images. 
However, traditional NeRF requires extensive per-scene training, limiting its generalization to new scenes. Generalizable NeRF approaches overcome this limitation by adapting to unseen scenes without retraining~\cite{trevithick2020grf, chen2021mvsnerf, 10377995, chou2024gsnerf, tian2023mononerf, liu2022neural}. 
For example, MVSNeRF integrates multi-view stereo with neural rendering to reconstruct radiance fields from a few input views~\cite{chen2021mvsnerf}, and WaveNeRF employs wavelet-based representations for generalization~\cite{10377995}. 
GSNeRF incorporates semantics to generate novel views and semantic maps for unseen scenes~\cite{chou2024gsnerf}, while NeuRay aggregates image features from multiple views to predict voxel properties~\cite{liu2022neural}. 
These methods, designed for visible light, cannot be directly applied to wireless signals due to fundamental differences in wavelength and propagation behaviors.
This work extends generalizable NeRF to the wireless domain for spatial wireless signal propagation modeling.


\textbf{NeRF for Wireless Signal Modeling.}
There are pioneering efforts that have explored NeRF for wireless signal modeling~\cite{zhao2023nerf, orekondy2022winert}.
For instance, \nerft~trains an MLP to compute voxel attributes and proposes Equation~(\ref{eqn_tracing_rf}) as the ray tracing algorithm to compute the received signal for each ray. 
However, it requires scene-specific training, which limits scalability and generalization to new environments.
NeWRF~\cite{lunewrf} leverages direction-of-arrivals~(DoA) as priors to reduce the required rays, thereby enhancing the efficiency of signal reception computation. 
However, this efficiency comes at the cost of requiring additional antenna arrays for DoA data collection, which are infeasible in complex, obstructed environments.
\ourSystem avoids this additional data requirement.
WiNeRT~\cite{orekondy2022winert} employs differentiable ray tracing with scene CAD models, which is impractical due to the difficulty of obtaining accurate CAD models. 
In contrast, \ourSystem eliminates this dependency, offering a more practical and flexible solution.
More importantly, these models lack generalization. 
\ourSystem introduces generalizable neural radiance fields for wireless signal propagation, adapting to unseen scenes without scene-specific training.





\section{Motivation} \label{sec_motivation}


While SSMs provide notable efficiency advantages, deploying them on NPUs presents unique challenges due to their computational patterns and hardware requirements. Unlike traditional deep learning models, SSMs exhibit characteristics that deviate from standard kernel operations, necessitating specialized optimizations. Existing NPUs are designed primarily for data-parallel operations like matrix multiplications, which dominate workloads in transformers and CNNs. SSMs, however, involve sequential computations and specialized operators, such as activation functions (e.g., Swish and Softplus) and cumulative summations (CumSum). These operations do not align with the highly parallelized architecture of NPUs, leading to inefficient execution when mapped directly. Fig.~\ref{fig:motivation_exec_lat_brkdwn} highlights execution bottlenecks for Mamba and Mamba-2 models on the Intel\textregistered\ Core\texttrademark\ Ultra Series 2~\cite{lnl} NPU. For Mamba, the majority of execution time is consumed by activation functions, such as Swish (SiLU) and Softplus, which are executed sequentially on DSPs. These DSPs are less optimized for such operations, resulting in prolonged execution times and underutilization of the data-parallel units. In Mamba-2, CumSum and ReduceSum emerge as primary bottlenecks, as these operations also rely on DSPs for sequential processing. This sequential nature hinders efficient reuse of local SRAM, increasing memory traffic and access latency. Both models further face challenges with elementwise multiplication (Multiply), which similarly runs on DSPs and contributes to inefficiencies. Handling long sequences in SSMs requires careful memory optimization. Limited on-chip memory must be utilized effectively to avoid frequent off-chip memory accesses, which incur significant latency and energy costs. The lack of optimized dataflow alignment for SSM computations exacerbates this issue, leading to poor performance. Blind, out-of-the-box mapping of SSMs on NPUs results in suboptimal performance, leaving much of their potential benefits untapped. Addressing these challenges is essential to fully leverage the advantages of SSMs in resource-constrained environments.










\section{Problem Formulation}

In this section, we introduce the problem formulation of \system{}. 
First, based on the retrieve-then-adapt idea~\cite{qian2020retrieve}, we discuss our approach for generating data reports by breaking it down into retrieving a related report, deducing, and reproducing a sequence of data analysis segments, each comprising analytical objectives, data processing, and insights.
Based on the formulation, we further conduct a preliminary study to understand the detailed design requirements of what can be reused from previous reports and how to rectify the new data. 
Based on the findings, we propose \system{}'s design considerations.

\subsection{Definition of Data Reports}
\label{subsec:problem_formulation}

To create a data report, data scientists need to explore and analyze the data, obtain data insights, and organize them into coherent narratives and charts~\cite{li2023wherearewesofar}. 
A shortcut for this process might be referring to an existing report and attempting to adapt it with new datasets.
To begin, it is required to retrieve a report that shares a similar topic with the current dataset.
In this section, we assume the retrieval is performed with a reliable search engine or by the users and focus on the parsing and reconstruction of the report.

Given a report, the key to its reconstruction is to decompose the report into logically coherent segments and validate whether the new data can fulfill the needs of different segments or provide similar insights supporting the argument. 
If not, how can adequate transformation of the original segments or new data be employed to make the whole process successful? 

Previous research~\cite{bar2020automatically, li2023edassistant, batch2017interactive} has outlined the analysis workflow as the three steps: 
1) Given a dataset to analyze, a data scientist usually begins by viewing the data and setting an analytical objective. 
2) Then, the data scientist would perform various data transformation steps, usually writing and executing code, and potentially encode the transformed data into a chart~\cite{wang2023dataFormulator}. 
3) Finally, the scientist inspects the results of the analysis and obtains insights into the data. 
We define an ``analysis segment''  as a triplet:
$$ segment := (objective, transformation, insight).$$

After completing an analysis segment, data scientists continue the analysis by creating new analytical objectives from previous ones, hustling to gain new insights, and generating a subsequent analysis segment. 
By repeating this process, the data scientists create a report with a complete analysis workflow and data insights. 
In this study, the segment is considered a basic building block of a data report.

Based on the definitions, we formulate the analysis workflow as a sequence of interconnected analysis segments $S = \{s_0, s_1, \cdots, s_N\}$, where $N$ is the number of segments in a data report.
Each segment $s_j$ is defined by three key components $(o_j, t_j, i_j)$, where $o_j$ is the analytical objective that guides the inquiry, $t_j$ is the data transformations that process the data, and $i_j$ is the insights that emerge from this exploration. 
Moreover, these analysis segments have interdependent dependencies $D = \{d_0, d_1, \cdots\}$, as the objective for one segment might stem from either insight of a previous segment or the data.
Each dependency $d\in D$ denotes a directed link of a tuple of segments $(s_i, s_j)$, representing that the segment $s_j$ stems from $s_i$.
In this light, the analysis workflow can be organized into a tree~\autoref{fig:formulation}a1, where each node denotes an analysis segment, and each edge denotes a dependency. 
Specifically, the initial analysis segment depends on the data. 
Based on this formulation, a data report is essentially a structural form of insights $S = \{s_0, s_1, ..., s_N\}$ distilled from these segments of analysis. 

Therefore, our method revolves around deducing and reproducing the sequence of analysis segments $S$ from the reference report, including analytical objectives, data processing, and insights. 
Finally, we organize the newly gleaned insights into a coherent and informative report, thereby reviving the original report with fresh data insights.

\begin{figure}[!htb] 
  \centering
  \includegraphics[width=0.5\linewidth]{figs/formulation.png}
  \caption{
  Producing a data report (c1) involves analyzing the data (a) and summarizing the analyzed insights into a data report (b). 
  Specifically, the data analysis workflow (a) includes a series of interdependent analysis segments (a1), each corresponding to an analytical objective, data transformation steps, and data insights (a2). 
  To reuse an existing report on a new dataset, we first deduce the data analysis workflow and reproduce it on the new data (c2). 
  }
  \label{fig:formulation}
\end{figure}

\subsection{Preliminary Study}
\label{subsec:preliminary_study_settings}

Based on the definitions, we aim to decompose a data report into segments and apply them to new data. 
However, the organization of data reports is flexible, highly depending on user preference and experiences. 
Besides, the aspects of a report that can be inherited and need adjustment remain unclear. 
To bridge this gap, we conducted a preliminary study with two main objectives:
(1) Investigate the relationship between the narrative structure of data reports and the corresponding analysis segments. 
(2) Identify the similarities and differences in analysis segments among data reports on the same topic and analyze how the differences stem from the data.

We collected data reports of different topics from well-known organizations that publish data reports, such as ONS~\cite{ons}, YouGov~\cite{yougov}, Pew Research Center~\cite{pewResearchCenter} and PPIC~\cite{ppic}. 
% \todo{based on the collected report, make a report repository. }
For the first objective, we analyzed the narrative structures of these reports to assess their alignment with our analysis segments. 

For the second objective, we observed that these organizations often publish reports on similar datasets, such as epidemic data collected at different times, typically issuing one report per dataset. 
These reports usually look similar but vary subtly in their analysis and content, which can be evidence to inspect which features are inherited and which require adjustment. 
Therefore, we further constructed 39 pairs of data reports that share the same topics and conducted pair-wise analysis on them. 
We identified the similarities and differences in each report pair and analyzed how the differences were sourced from the data. 
Treating a data report containing a series of analysis segments, including \textbf{analytical objectives}, \textbf{data transformations}, and \textbf{report content}, we summarized the patterns of similarities and differences regarding these elements.

\subsubsection{Narrative Structure of Reports}

To extract analysis segments from a report, we identified how the report's content aligns with these segments, defined as \textbf{analytical objectives}, \textbf{data transformations}, and \textbf{report content}. 
We explored whether the narrative structure could be segmented so that each part corresponds to a distinct analysis segment.

Our analysis of 35 reports showed that in most cases (32/35), the analytical content was presented as distinct segments, each focused on a single objective rather than interspersed with multiple topics, with related text and possibly a chart grouped together.
Additionally, 11 out of 35 reports included non-analytical content, such as background information, which could either supplement a specific analysis or the entire report and appeared flexibly throughout. 
Some reports (23/35) also included a summary of key insights at the beginning or end, which we excluded to concentrate on the main analytical content.

\subsubsection{Similarities and Differences in Pair Reports}

We summarize the patterns of similarities and differences between the data reports on similar topics. 
These findings lay the foundation for designing and implementing a method to reuse data reports with new data. 

\paragraph{Analytical Objectives}
Analytical objectives are guidance for the exploration of insights and findings from the data.
For example, the report of internet users~\footnote{https://www.ons.gov.uk/businessindustryandtrade/itandinternetindustry/\\bulletins/internetusers/2018} holds an analytical objective to explore the Internet use among each different age group. 
Therefore, we identified the analytical objectives in the collected reports by inspecting the aspects of the data findings that were discussed. 
After identification, we compared the analytical objectives between each pair of reports and analyzed their alignment and variations. 

As a result, all of our collected pair reports reflect similar analytical objectives.
Specifically, 34/39 of them involve analytical objectives that are exactly the same, while 35/39 of them involve slight differences.
Most slight differences source from different \textbf{data contexts and scopes}. 
For example, the analytical objectives of two data with different time ranges will also focus on distinct time frames. 
Others stem from the \textbf{dependencies to previous data insights}. 
Analytical objectives may be formed based on previous insights through logical dependencies, e.g., exploring the reason for an increasing trend. 
Therefore, the changes in previous insights may also cause adjustments in the latter objectives. 
Additionally, some pair reports involve completely different analytical objectives (18/39), which mainly source from \textbf{different data fields}. 
For example, newer data may introduce additional data fields, thus triggering new analysis objectives.
Reports may also incorporate insights from external data sources (4/39), spanning different contexts, scopes, and fields. 
These insights typically maintain a logical dependency on previously established insights, such as generalizing from local to national trends, which also introduces varied analytical objectives. 


\paragraph{Transformation Operations}
Data transformation operations are not explicitly outlined within the data reports. 
Moreover, most of the source data provided by our collected reports have been processed (34/39), making it harder to infer the specific data processing conducted. 
Nonetheless, two facets of data processing can still be discerned from the reports. 
Firstly, the reports mirror the output of the data processing, as the charts and narratives presenting data insights directly originate from these outputs.
Secondly, they also reflect detailed data processing choices, particularly regarding charts, which involve decisions on chart type, encoding, binning, etc. 

Considering these factors, we compared the content referring to similar analytical objectives between each pair of reports and analyzed their similarities and differences in analysis outputs and data processing choices. 
Consequently, we observed that similar analytical objectives always yield similar \textbf{analysis output forms}.
For example, the objective of analyzing trends always results in a chart with the temporal field on the x-axis, indicating a transformed dataset measuring variables across time periods.
However, the \textbf{detailed data processing choices} may vary to accommodate the data difference (13/39). 
Varied formats and scopes of data fields could potentially result in different chart types or levels of binning granularity to better align with visualization rules.

\paragraph{Report Content}
The report content, comprising both textual information and charts, is directly derived from the result of data processing. 
Since analysis results from different datasets naturally differ, resulting in varying values in the report content, our primary focus lies in the similarities and differences beyond mere numerical distinctions. 

As a result, regarding the similarity of report content, we observe that each pair of reports shares a similar \textbf{narrative and visual style}, such as the formality degree in tone and infographic design. 
As our main objective is to reuse the analysis workflow, the inheritance of content style is not our primary focus and is therefore not considered in this study.
The differences primarily manifest in the textual descriptions of data insights. 
Different reports may describe varying types of data insights. 
For example, one report might focus on detailing an outlier, while another might only describe overall trends. 
This difference stems essentially from \textbf{distinct analysis results}, which not only result in numerical disparities but also lead to variations in the reflected data insights.
Under the same data processing steps, one dataset may exhibit a highly significant outlier in the results, while another dataset may not.

\subsection{Expert Interview}
The findings of the preliminary study revealed that various aspects of existing data reports can be leveraged to generate new reports, but these elements require adjustments to align with the new data. 
The study also provides theoretical guidance on how to identify incompatible aspects and the directions in which adjustments should be made. 

To further understand the user requirements in reusing existing reports to analyze data, we conducted an expert interview with two experienced data analysts, EA and EB. 
EA has over two years of experience working as an actuary at an insurance company, where they frequently analyze data, organize results, and present them to clients to guide future decision-making. 
EB is a seasoned researcher in data storytelling and an experienced data journalist. 

We conducted 45-minute interviews with each expert, during which we discussed the following questions: (1) What formats do they typically use to present data analysis results, such as data reports, data stories, or others? (2) Do they encounter scenarios in which they reuse or refer to existing data analysis materials for analyzing new data and proposing new presentation materials? (3) If so, what is their workflow?

For the first two questions, EA noted that he use different formats depending on the case, including Excel files, slides, and data reports. 
He noted that for all three formats, he often refers to existing analysis materials. 
For example, to analyze and present data for a new insurance product,  he may refer to past reports or related reports from other products. 
In these cases, both the data and analysis goals are usually quite similar, making existing materials particularly useful.
EB, on the other hand, discussed the distinction between data reports and data stories. 
She pointed out that data reports are highly structured and commonly used in formal settings such as official documents, presentations, and communications. 
These reports often follow stable templates, making it easy to refer to existing materials when generating new reports. 
In contrast, EB sees data stories as a more creative format and tends not to reference previous materials. 
She prefers to avoid replicating others' approaches and focuses on originality in crafting data stories.

For the third question, Although EA acknowledged using existing materials in multiple formats, he emphasized that the aspects reused and the workflow vary depending on the format. 
For Excel files, these existing files are often with previous SQL scripts and formulas, which can be seen as preserved analysis code. 
In these cases, small modifications (e.g., updating SQL query conditions for new data) are typically sufficient. 
For data reports and slides, however, the process is different. 
These formats often lack original analysis files, and while existing materials can inspire analysis goals, chart creation, and textual descriptions, the data analysis still needs to be conducted manually.
EA further elaborated on the differences between slides and data reports. 
While slides typically feature relevant charts, they can lack detailed narrative descriptions, as slides are often presented orally. 
In contrast, data reports are expected to contain more comprehensive written content, including detailed explanations.


EA and EB further elaborated on the workflow of reusing and referring to previous data reports. 
For cases where the data fields are similar, or the analysis goals remain consistent, both EA and EB noted that these cases allow simply replacing charts with new charts and replacing the numerical conclusions to align with the new data. 
However, when the fields differ, EA mentioned that adjustments are necessary, either by modifying the analysis goals or removing irrelevant sections of the report. 
For new fields, EA would develop new analysis goals and rewrite the descriptive analysis accordingly. In cases where the insights change, the analysis may need to be entirely redone to accommodate the new findings.

\subsection{Expert Interview}

The findings of the preliminary study reveal that various aspects of existing data reports can be leveraged to generate new reports, but these elements require adjustments to align with the new data. 
The study also provides guidance on how to identify incompatible aspects and the directions in which adjustments should be made. However, it is unclear how users reuse past analysis reports for new scenarios.

Therefore, we conducted an expert interview targeted at data analysts who frequently explore new data and compose reports to communicate insights to leaders or clients. Moreover, we expected the interviewees to have certain experience in using LLMs in their analysis and prototyping process.
We interviewed two experienced data analysts, EA and EB. 
EA has over two years of experience working as an actuary at an insurance company, where they frequently analyze a variety of data, including claims history, policyholder demographics, risk factors, and market trends.
Findings are presented to a range of clients, including internal stakeholders (such as underwriters, product managers, and senior executives) and external clients (such as brokers, corporate policyholders, or regulatory bodies), to guide strategic decision-making and ensure compliance with industry standards. 
EB is an assistant professor in the Department of Journalism of a top-tier university. She also holds a Ph.D. in data science and frequently writes data journalism for news media.


We conducted 45-minute interviews with each expert, during which we raised the following questions: (1) What formats do they typically use to present data analysis results, such as slides, reports, dashboards, and spreadsheets? (2) Do they encounter scenarios in which they reuse or refer to existing data analysis materials for analyzing new data and proposing new presentation materials? (3) If so, what is their workflow?

For the first two questions, EA mentioned that he uses different formats depending on the specific case, including Excel files, slides, and data reports. He emphasized that for all three formats, he frequently refers to existing analysis materials. For instance, when analyzing and presenting data for a new insurance product, he often consults past reports or related analyses from similar products. In such cases, both the data and the analysis objectives tend to align closely, making existing materials particularly valuable for efficiency and consistency. Additionally, EA noted that reports are his primary format for conveying data insights, particularly in formal settings.
On the other hand, EB explained that her approach to data presentation is more scenario-dependent, and her use of previous materials varies accordingly. She highlighted that data reports, which are highly structured and commonly employed in formal contexts such as official documents, presentations, and communications, often follow standardized templates. This structure makes it straightforward to adapt or reference existing materials when creating new reports.
However, EB also expressed her interest in crafting ``data stories,'' which, while similar to data reports in format, are more creative and exploratory. Unlike reports, she tends to avoid relying on specific pre-existing materials for data stories, as doing so might constrain her thinking. Instead, she prioritizes originality and creativity, focusing on developing unique narratives that reflect her individual insights and perspectives.


For the third question, Although EA acknowledged using existing materials in multiple formats, he emphasized that the aspects reused and the workflow vary depending on the format. 
For Excel files, these existing files are often with previous SQL scripts and formulas, which can be seen as preserved analysis code. 
In these cases, small modifications (e.g., updating SQL query conditions for new data) are typically sufficient. 
For data reports and slides, however, the process is different. 
These formats often lack original analysis files, and while existing materials can inspire analysis goals, chart creation, and textual descriptions, the data analysis still needs to be conducted manually.
EA further elaborated on the differences between slides and data reports. 
While slides typically feature relevant charts, they can lack detailed narrative descriptions, as slides are often presented orally. 
In contrast, data reports are expected to contain more comprehensive written content, including detailed explanations.


EA and EB further elaborated on the workflow of reusing and referring to previous data reports. 
For cases where the data fields are similar, or the analysis goals remain consistent, both EA and EB noted that these cases allow simply replacing charts with new charts and replacing the numerical conclusions to align with the new data. 
However, when the fields differ, EA mentioned that adjustments are necessary, either by modifying the analysis goals or removing irrelevant sections of the report. 
For new fields, EA would develop new analysis goals and rewrite the descriptive analysis accordingly. In cases where the insights change, the analysis may need to be entirely redone to accommodate the new findings.

\subsection{Design Considerations}

Based on the problem formulation and the findings of the preliminary study, we summarize five design considerations (\textbf{C1}-\textbf{C5}) for an automatic method of reusing data reports with new data. 

\begin{enumerate}[label=\textbf{C\arabic*}]
\item \textbf{Support analytical objectives extraction, correction, and addition. } 
The key of the method is to extract and re-execute the analysis workflow of the existing report, which corresponds to a series of interdependent analytical objectives. 
To achieve this, the method should first split the report to identify distinct analysis segments and then extract the analytical objectives and their dependencies. 
Additionally, it should support smart objective correction and addition based on the dependencies and data features.

\item \textbf{Generate appropriate data processing steps automatically. } 
Based on the preliminary study, the data processing steps are not reflected directly in the existing report. 
Therefore, the method necessitates autonomous reasoning about appropriate data processing steps that yield outputs similar to those in the existing report while also making informed data processing choices adaptable to the new dataset. 

\item \textbf{Produce insightful report content derived from analysis. } 
The textual and visual content that presents data insights constitutes the primary component of a data report. 
The method should produce content that effectively presents new data insights derived from analysis. 
Our preliminary study also revealed that reports often include non-analytical content, such as background information. 
As LLMs are pre-trained on extensive background knowledge, we allow them to generate this content. 
However, any generated non-analytical content will be highlighted, as it may not always be reliable.

\item \textbf{Enable real-time output observation and report modification. } 
Given the complexity of the method, which involves analytical objectives, data processing, report content, and structure, uncertainties naturally arise, potentially leading to deviations from user expectations. 
To address this, the method should provide an interactive interface, enabling users to observe generated outputs in real-time and make necessary adjustments. This ensures that the final report aligns closely with user expectations. 

\item \textbf{Facilitate report structure organization and modification.}
Since the report is generated based on the reference report's workflow, it will naturally exhibit a similar narrative structure. 
The method should allow for re-organizing this inherited structure, including adding and generating (sub-)titles, to ensure the report is well-structured and tailored to the new content.
\end{enumerate}

Based on the design considerations, we develop an intelligent method, \system{}, to deduce and reproduce the authoring workflow of existing data reports on the new data. 
The pipeline of \system{} consists of three stages. 
\textbf{In the pre-processing stage,} \system{} recommends the most relevant reports from a built repository, ranked by similarity to the user's dataset. 
It then dissects the existing report into interconnected segments, each corresponding to the data insights of an analysis segment (\textbf{C1}). 
Based on the segmentation, it extracts the analytical objectives of each segment and deduces their dependencies (\textbf{C1}). 
\textbf{In the analysis stage,} \system{} executes each segment by reusing the information from the original report, identifying the inconsistencies, and customizing the analytical objectives, approaches, and report contents based on the new data (\textbf{C1-C3}). 
\textbf{In the organization stage,} \system{} inherits the original report structure and enables title re-generation (\textbf{C5}). 
Moreover, to enhance usability, we integrate an interactive interface for \system{}, allowing users to inspect real-time outputs, add new analytical objectives, and modify report content as needed (\textbf{C2, C5}). 




%%
%%---------------------------------------------
\section{Implementation Experience}
\label{sec:implementation}
%%
\noindent
%
In this section we describe our approach to implement a MEC Customer Orchestrator and to support MEC application slicing using Kubernetes and Helm technologies. Although our implementation is still at a work-in-progress state, our proof-of-concept prototype (hereafter simply PoC), shows the feasibility of our design approach and allowed us to collect a preliminary insight on the efficiency of application slicing using Kubernetes resource management capabilities. In the following, we first briefly overview Kubernetes and Helm features. Then, we detail the implementation of the new APIs and functional components  of our PoC,  namely the MECO and the ACF Image Repository. Finally, we describe our practical approach to support slicing of Kubernetes resources. For the sake of presentation clarity, Figure~\ref{fig:poc} overviews the architecture of the PoC and its internal components.
%
\begin{figure*}[ht]
    \centering
    \includegraphics[clip,trim= 1cm 4.5cm 0cm 2.5cm,width=0.7\textwidth]{figures/fig_poc.pdf}
   \caption{PoC architecture and internal components.}
    \label{fig:poc}
    \vspace{-0.2cm}
\end{figure*}
%
%%
%%---------------------------------------------
\subsection{PoC enabling technologies}
\label{sec:poc_tech}
%%
\noindent
%
Kubernetes is an open-source container life-cycle manager and orchestrator, which is the de-facto standard for running container-based cloud native applications on a cluster of (physical or virtual) machines (called \textit{nodes}). It is out of the scope of this paper to describe the complete Kubernetes architecture and high-level abstractions, but we focus on the components that are most relevant for our PoC. 

A Kubernetes cluster is composed of $(i)$ a set of \textit{worker} nodes that run containerized applications, also called \textit{workloads}; and $(ii)$ (at least) a \textit{master} node that runs the services of the \textit{control plane}, and it is responsible to enforce the desired state of the cluster. Kubernetes provides several built-in workload resources to support various application behaviours (e.g., stateless tasks) and management functions (e.g., creating or deleting application replicas). Each workload must run into a \textit{Pod}, a Kubernetes object that represents a collection of containers running in the same execution environment, which share the same storage, networking, and lifecycle. Pods runs into worker nodes, which host an agent, called \texttt{kubelet}, that is responsible for managing worker's local containers and for synchronising the status with the master node. As better explained later, another main component of the worker node is the \texttt{kube-proxy}, which is responsible for implementing Kubernetes networking services and to enable communication to Pods from inside and outside the cluster. The master node is composed of different components including:$i)$ \textit{etcd}, a distributed key-value store that holds and manages all cluster critical data; $ii)$ \texttt{kube-apiserver}, a component providing a REST-based frontend to the control plane through which all other components interact; $iii)$ \texttt{kube-controller}, a component that monitors the shared state of the cluster using apiserver and runs the controller processes; and $iv)$ \texttt{kube-scheduler}, a component that assigns newly created Pods to nodes.  Note that Kubernetes also support the \textit{namespace} abstraction, namely virtual clusters that share the same IP Address and port space, facilitating the grouping and organisation of objects.    

Networking is a central part of the Kubernetes design and a fundamental capability for application slicing. Specifically, the Kubernetes network model demands certain network features, such as every Pod should have a unique, routable IP inside the cluster, and inter-pod communications should happen without using NATs, regardless of wherever the Pods reside or not on the same worker nodes (i.e. network segmentation is not allowed). Since IP addresses of Pods are ephemeral and change whenever the Pods are restarted or migrated, Kubernetes also defines \textit{Service} resources, namely REST objects that are used to group identical Pods together to provide a consistent means of accessing them, e.g by bounding them to a virtual IP address (called cluster IP) that never changes. It is important to point out that 
Kubernetes does not directly handle the networking aspects, but it rather allows the use of third-party networking plugins that adhere to the Container Network Interface (CNI) specification\footnote{\url{https://github.com/containernetworking/cni}} to manage the containers' data plane. Of particular relevance for our PoC, is Kube-OVN\footnote{\url{https://www.kube-ovn.io/}}, a CNI plugin that integrates network virtualisation into Kubernets  by leveraging the OVN (Open Virtual Network)\footnote{\url{https://www.ovn.org/en/}} technology. Kube-OVN supports advanced features, such as unique subnets per namespace, network policies, namespaced gateways, subnet isolation and dynamcic QoS. We extensively leverage some of those features to support application slicing in our PoC. 

Another technology we use as a basis for our PoC is Helm\footnote{\url{https://helm.sh/}}, an application packaging manager for Kubernetes. Specifically, Helm defines a data format, called \textit{Helm Chart}, to bundle a set of Kubernetes object definition files (i.e. files describing properties of Kubernetes objects) into a single package. This permits to manage the instantiation, upgrade and deletion of the corresponding Kubernetes objects as they were a single entity. The Helm charts are stored into a separate repository, and every time a new instance of the same chart is installed into Kubernetes, a new chart \emph{release} is created. Furthermore, Helm allows to augment Kubernetes object definition files with Helm template commands. By providing Helm a list of arguments for these template commands at chart instantiation time, it is possible to dynamically customise the chart before it is actually deployed.  Examples of these customisation span from overriding object default values with the one passed as command arguments (e.g. a port exposed by a container, the namespace name, etc..) to dynamically enabling disabling chart sections. As explained later, we extensively leveraged Helm features to implement different components of our PoC, such as the KDPManager and the MAPSS Chart Registry (see Figure~\ref{fig:poc}).

We conclude this section by noting that the ETSI MEC standard has recently started analysing how MEC features should be adjusted when deploying a MEC system using a container-based virtualisation technique~\cite{MEC027}. Furthermore, a few initial implementations exist of specific MEC components and interfaces using Kubernets as VIM, such as Akraino\footnote{\url{https://www.lfedge.org/projects/akraino/}} and LightEdge\footnote{\url{https://lightedge.io/}}. A recent work~\cite{2020_broadnets_mec_k8s} analyses how to use Kubernetes not only as a VIM, but also as the core of the MEPM, also leveraging Helm technology for the life-cycle management of MEC applications. 
%
%%
%%---------------------------------------------
\subsection{PoC design and development}
\label{sec:poc_design}
%%
\noindent
%
One of the main objectives of our proof-of-concept implementation is to demonstrate how Kubernetes can natively support multiple, isolated instances of MEC application slice subnets (MAPSSs for brevity) as defined in Section~\ref{sec:pre_concepts}. The key building block of a MAPSS is the ACF. For the sake of simplicity, in our PoC we ignore VNFs and we assume that ACF Suppliers can provide ACFs in the form of Docker container images coupled with an ACF user manual or descriptor (called ACFD). Then, an ASP leverages the ACFDs to select the set of ACFs that are needed to build the AS requested by its end customers, as well as to derive the proper run-time configuration of the graph of ACFs composing the AS. Therefore, a key component of our PoC is the \textit{ACF registry}, where authorised ACFDs are published and stored. The ACF registry is implemented using the open-source Docker Registry 2.0 application\footnote{\url{https://hub.docker.com/_/registry}}, a storage and distribution system for named Docker images. A generic ACFD is structured into two separate sections. The first one details the RESTful APIs exposed by the ACF (OpenAPI\footnote{\url{https://www.openapis.org/}} is used to specify these APIs in a standard, language-agnostic format). The second section details how to properly configure the ACF parameters in order to control how the ACF will behave at run-time. The ACF parameter customisation is a crucial aspect to consider, especially in the context of ACF composition and automated orchestration. An example of ACF customisation is the selection of buffer sizes, which may influence the run-time behaviours and performance of ACFs, directly affecting the fulfilment of SLA requirements. For the sake of simplicity, in our PoC we use \textit{environmental variables} to pass configuration paramters to the running container images of ACF. Finally, it is important to point out that the ACF registry must be accessed also by the \texttt{kube-scheduler} agent to fetch the container images of ACFs to be deployed.

The other key component of our PoC is the MECO that we have implements as an additional component of the master node, using Go as programming language. Internally, the MECO component is composed of three different modules: $i)$ the \textit{MECO API Server} to enable communications with the MEC APSSMF; $ii)$ the \textit{MAPSS Chart Repository} to store the templates of Kubernetes deployment plans of MAPSS; and $iii)$ the \textit{KDPManager}, to manage the Kubernetes deployment plans of run-time instances of activated MAPSS. In the following, we elaborate on the purpose and operations of each module more in detail. 

The main role of the MECO API Server is to act as a lifecycle management proxy for MAPSSs. Specifically, it receives commands for the instantiation, update and termination of MAPSS from the MEC APSSMF, and translates these commands into suitable Kubernetes actions. To this end, the MECO API Server exposes a RESTful management interface, called \texttt{mapss\_mm} API, defined using the OpenAPI description language (see Figure~\ref{fig:poc}). For the sake of the experimentation, the \texttt{mapss\_mm} API currently implements a \texttt{POST} method, which allows the MEC APSSMF to request the instantiation of a new MAPSS instance. The payload of this \texttt{POST} method carries a descriptor, called MAPSSD, which contains all the necessary information to allow the MECO to instantiate at run-time a specific MAPSS instance. The \texttt{mapss\_mm} API also implements a DELETE method, which allows to delete a running instance of a MAPSS. It is clear from this discussion, that the MAPSSD plays a key role in our PoC. We envision a MAPSSD organised into four different sections, as also illustrated in the example in Figure~\ref{fig:poc_massd_example}. The first parts includes the unique identifier of the MAPSS instance (\texttt{mappsiId}), a human readable description of the slice subnet features, and the identifier of an implementation template to be used for the deployment of the MAPSS instance (\texttt{mapssImplTemplateId}) -- see later this section for more details on how to use the \texttt{mapssImplTemplateId}. The second section includes a set of ``\textit{slice-subnet-wise}'' (computational, storage and networking) resource requirements. For example, the MAPSSD shown in Figure~\ref{fig:poc_massd_example} requires two dedicated CPU cores, 8 GB of memory and 100 GB of permanent storage, which will be shared among its constituent ACFs. The third part includes the list of ACFs that compose the MAPSS instance. Each ACF is associated to a unique identifier (\texttt{acfId}), and ``\textit{acf-wise}'' resource requirements can also be specified. For example, the acf1 in Figure \ref{fig:poc_massd_example} requires a dedicated CPU core out of the two dedicated to the whole slice subnet. Moreover, the field \texttt{customParams} can be leveraged to pass arguments (e.g. buffer size in figure) to configure specific ACF behaviours. Finally, the last section includes a list of virtual links among pairs of ACFs and their networking requirements (e.g. maximum usable bandwidth). It is worth noting that the data format of the MAPSSD is agnostic from the underlying virtualisation technology. In other words, the MEC APSSMF could leverage the same data model to interact with a MECO that relies on a different VIM than Kubernetes. Another advantage of the proposed MAPSSD is that it allows the MEC APSSMF to seamlessly integrate SLS requirements that address needs of different architectural levels (i.e. specific to the entire slice subnet, individual ACFs and virtual links between ACFs) in the same data object. 
%
\begin{figure}[th]
    \centering
    \includegraphics[clip,trim= 0cm 16cm 10cm 0cm,width=0.5\textwidth]{figures/fig_poc_mappsd.pdf}
   \caption{Illustrative example of a MEC application slice subnet descriptor (MAPSSD)}
    \label{fig:poc_massd_example}
\end{figure}
%

Clearly, the translation from a VI-agnostic MAPSSD into a Kubernetes deployment plan (\textit{KDP} for short) of the MAPSS instance, namely a package of properly configured Kubernetes objects implementing the requested MAPSS instance, is a critical functionality of the MECO. Following the approach proposed in~\cite{NGMN028} for supporting cost-efficient customisation of network slices, we assume that the MECO hosts a pre-loaded set of MAPSS templates/blueprints that can be used to speed up the creation of a MAPSS instance. Specifically, we implemented each MAPSS KDP template as an Helm chart that includes a set of pre-configured Kubernetes objects. These objects define: $i)$ ACFs Docker containers to run (e.g. via Kubernetes Pods objects), $ii)$ ACFs behavioural parameters (e.g. via environmental variables in Kubernetes ConfigMaps objects), $iii)$ ACFs connection points (i.e. exposed ports), $iv)$ custom scheduling policies (e.g., number of replicas, failure behaviour, etc.), and any other Kubernetes feature that is necessary for the correct deployment of the MAPSS instance. Then, the \textit{mapssImplTemplateId} field of the MAPSSD is used to retrieve the correct MAPSS KDP template. It is important to point out that the MECO should be able to dynamically customise at run-time the MAPSS KDP template using information derived from the MAPSSD (e.g., container resource requirements). To this end we leverage built-in objects and control structures of Helm template that provide access to values passed into an Helm chart, and the ability to include conditions in the template's generation. In the current implementation, we limit such customisation to the selection of: $i)$ the name of the namespace to which objects will belong; $ii)$ the computational, storage and networking requirements for the namespace; $iii)$ the computational and storage requirements per ACF, and the networking requirements per ACF pairs. In our Poc the MAPSS KDP templates are stored in the MAPSS Chart Registry (see Figure~\ref{fig:poc}). According to the operational and management roles defined in Section~\ref{subsec:roles}, the MEC Customer plays the roles of both the MEC operator and the ASP. Thus, the MEC Customer has the necessary expertise not only to properly select, compose and configure ACFs to provide an AS, but also to select and properly configure the subset of Kubernetes objects that allows to implement the MAPSS instance of the designed AS. Finally, the KDPManager module is simply a wrapper of the Helm library, which allows the MECO to embed the Helm functionalities.   

%
\begin{figure*}[ht]
    \centering
    \includegraphics[clip,trim= 0cm 2cm 0cm 2cm,width=0.9\textwidth]{figures/fig_poc_seq_diag.pdf}
   \caption{Sequence of operations to instantiate a new MAPSS in the PoC architecture.}
    \label{fig:poc_seq_diag}
\end{figure*}
%
We can now discuss the sequence of operations and request/response exchanges that are executed to deploy a new MAPPS instance in our Poc, which are also graphically illustrated in Figure~\ref{fig:poc_seq_diag}. First of all, the MEC APSSMF initiates the deployment process by sending a \texttt{POST} request to the MEC API Server over the \texttt{mapss\_mm} interface (see Figure~\ref{fig:poc}), which contains the MAPSSD of the requested MAPSS. In the figure, the requested MAPSS is identified as \textit{demoSlice}, while its Helm chart template is identified as \textit{demoTpl}. In \textit{step2}, the MECO API Server performs a preliminary analysis of the MAPSSD to discover the set of parameters that can be modified to customise the template. Furthermore, the MECO API Server retrieves from the \textit{mapssImplTemplateId} field of the MAPSSD the identified of the associated Helm chart template (\textit{dempoTpl} in this example). Finally, \textit{step2} is concluded with the API Server that instructs the KDPManager to deploy the \textit{demoTpl} Helm chart with the correct set of chart arguments. Subsequently, in \textit{step3}, the KDPManager fetches the \textit{demoTpl} Helm chart from the MAPSS Chart Repository, and it starts the deployment process. First, in \textit{step4}, the KDPManager contacts the \texttt{kube-apiserver} to create a new Kubernetes namespace with name \textit{demoSlice}. Then, the KDPManager applies (\textit{step5}) the customised arguments (e.g., number of cores to be assigned to an ACF container), and starts (\textit{step6}) chart release process, using \textit{dempoSlice} as release name. This process involves the generation of the proper set of Kubernetes objects definition files (i.e. the Kubernetes Deployment Plan). Once the KDP is complete, the KDPManager instructs the \texttt{kube-apiserver} to create the Kubernetes objects in the etcd database (\textit{step7}). Finally, in \textit{step8}, the \texttt{kube-controller} starts performing control actions according to the received Kubernetes objects. The latter includes contacting the ACF Image Repository to fetch ACFs container images for scheduling. For the sake of completeness, we remind that a termination of a run-time instance of a MAPSS is initiated by a \texttt{DELETE} request sent by the MEC APSSMF to the MECO via the \texttt{mapss\_mm} interface. This \texttt{DELETE} contains the \textit{mapssId}) of the MAPSS instance to delete. In this case, the MECO API Server requests the KDPManager to uninstall the chart release associated to \textit{mapssId}. Finally, the KDPManager removes all the Kubernetes objects of the release from Kubernetes cluster, and deletes the \textit{mapssId} namespace.

We complete the presentation of our PoC by explaining how application slice isolation is enforced using the Kubernetes resource control objects. First of all, we create a new Kubernetes namespace for each MAPSS instance, with a name equal to the \textit{mapssId}. All Kubernetes objects within a MAPSS instance are deployed using the same MAPSS namespace. We leverage a combination of Kubernetes \textit{ResourceQuota} objects, \textit{QoS classes} for Pods, and Kubernetes \textit{VolumeClaim} requests to limit the amount of computational and storage resources that could be consumed by both the whole namespace (to enforce requirements for invidual slice subnets) and individual Pods (to enforce requirements for individual ACFs). Network isolation between different instances of MAPSSs is implemented by exploiting Kube-OVN network policies, so that that the traffic arriving from Pods belonging to other namespaces is blocked (except for the system namespace). Finally, we implement per Pod ingress/egress rate limitation via Kube-OVN QoS policies. The implementation of more advanced QoS-aware network control policies (e.g. latency assurance, QoS tagging, etc.), and fine-grained network isolation policies (e.g. tunable network isolation degree with exception handling, etc) is planned as future work.


We conclude this section by  observing that an ASC could discover available ASes and their features by querying a catalogue that is exposed by a web portal (see Figure~\ref{fig:poc}), on which ASPs publish the descriptors of their ASes. We can foresee that AS descriptors include information such as $i)$ high level description of the offered AS; $ii)$ a pointer to the ASP offering the AS; $iii)$ a set of achievable SLAs (e.g. maximum resolution of a video processing service); and, possibly, $iv)$ billing information. The definition of a data model for AS descriptors is out of the scope of our present work. 
%
%%
%%---------------------------------------------
\subsection{Open implementation gaps}
\label{sec:poc_limitations}
%%
\noindent
%
During the implementation of our PoC we also faced several difficulties due to the limitations of the technologies and standards we have used. In the following, we summarise the main technological gaps we have observed to highlight areas of future investigations.

\begin{itemize}[noitemsep,topsep=2pt]
    \item ETSI MEC specification has defined the methods and the data formats for the \texttt{Mm1} reference interface between the OSS and the MEO, which is used to trigger the instantiation and the termination of MEC applications in the MEC system. However, the \texttt{Mm1} implicitly consider a MEC application as a single application package. For instance, in the MEC-in-NFV architecture, the \texttt{Mm1} allows the MEAO to  deploy a single VNF onboarded as a VNF descriptor. In our use case, a MAPSS instance represents a set of ACFs, and it could be conveniently modelled as a graph. To some extent the \texttt{Mm1} interface should be expanded to resemble the capabilities of the \texttt{Os-Ma-nfvo} interface between the NSSMF and the NFVO~\cite{NFV-SOL005} , which allows the NSSMF to request a network service (i.e. a collection of VNFs) to the NFVO.
    %
    \item Kubernetes ResourceQuota objects only permit to limit the amount of CPU and memory resources that Pods in a namespace could use. Pods can get assigned to a ``Guaranteed'' (highest priority) QoS class to receive reserved CPU and memory resources. VolumeClaim requests allow to reserve storage resources to a scheduled Pod. However, Kubernetes does not provide a straightforward mechanism to allocate resources at the namespace level but only at Pod level. This limitation complicates the implementation of resource over-provisioning strategies in dynamic slicing context (e.g. when a slice subnet is assigned more resources than needed to accommodate future demand changes).
    %dynamic slicing (e.g. when an application slice changes the set of deployed ACFs) 
    %
    \item Default scheduling mechanisms available in Kubernetes take into account only CPU and RAM usage rates when scheduling Pods, while network-related metrics (e.g. latency or bandwidth usage rates) are often ignored. However, a network-aware resource allocation and scheduling is crucial for our application slicing model, and initial proposals can be found in~\cite{2019_netsoft_netaware_kube} and~\cite{2020_noms_delayaware_kube}.
    %
    \item Kube-OVN allows to limit the transmission rate on both ingress and egress traffic at the Pod level. This is obtained by using a QoS-aware queue and traffic policing at the vswitch port to which the Pod is connected. However, Kube-OVN does not support to set up rate limits on individual traffic flows, which is an useful feature if a Pod needs to communicate with several other Pods (e.g., for inter-slice communications). A possible workaround is to leverage Multus CNI\footnote{\url{https://github.com/k8snetworkplumbingwg/multus-cni}}, a CNI plugin for Kubernetes that enables attaching multiple network interfaces to pods, to allow a Pod to have a dedicated virtual interface (i.e. network port) for each destination Pod. Then, separated instances of Kube-OVN could be installed on each virtual interface to enforce different QoS policies at the port level. Furthermore, bandwidth reservation mechanisms similar to the ones proposed for SDN-based networks~\cite{2016_bwguar_openflow} should be included in Kube-OVN. 
    %
    \item Service chaining allows to link together VNFs to compose service function chains (SFCs). The implementation of SFCs usually requires support from the network (e.g. via SDN) to route a packet from one VNF to the next in the chain. However, service chaining (which is a crucial feature for integrating VNFs with ACFs in a MAPSS is missing in Kubernetes. Recently, a few projects, such as OVN4NFV K8s Plugin\footnote{\url{https://github.com/opnfv/ovn4nfv-k8s-plugin}} and Service Meshes\footnote{\url{https://istio.io/}} have been initiated to provide support for service chaining in Kubernetes environments . 
    
    %and Network Service Mesh\footnote{\url{https://networkservicemesh.io/}} have been initiated to provide support for service chaining in Kubernetes environments . 
    %
    \item The MAPSSD provides the blueprint for building an application slice subnet within a MEC environments. For the sake of our PoC, we have defined a custom data model for specifying a MAPPSD. On the other hand, standard modelling languages exist, such as TOSCA (Topology and Orchestration Specification for Cloud Applications) for describing the components of a cloud application and their relationships, which facilitate the interoperability, portability and orchestration in a multi-cloud environment~\cite{2018_MCC_TOSCA}. ETIS MANO already advocates the use of TOSCA to specify NFV descriptors~\cite{NFV-SOL001}. In principle, TOSCA could also be leveraged to specify the MAPSS descriptors. However, TOSCA is specifically designed to model classical cloud applications and it needs some adaptations to natively support also contenarised applications. Several approaches have been recently proposed to either $(i)$ extend the TOSCA normative types for support of container-based orchestration platforms (e.g. Cloudify\footnote{\url{https://cloudify.co/}}); or $(ii)$ to decouple the application modelling from the application provisioning by developing ad hoc software connectors between TOSCA workflow and cloud provider’s API (e.g. TORCH~\cite{2021_jgc_torch}). However, no standard specifications have been released yet.  
    %
\end{itemize}




\section{Evaluation}

\subsection{Experimental Setup}\label{subsec:exp_setup}

In our evaluation, we adopt \textit{Code-Llama-7B} as the pre-trained model for fine-tuning, employing the low-rank-adaption (QLoRA)~\cite{hu2021lora, dettmers2024qlora} technique for faster training and lower memory consumption. 
Key configurations include loading the model in 8-bit, a sequence length of 4096, sample packing, and padding to sequence length. We set the warmup steps to 100, with a gradient accumulation of 4 steps, a micro-batch size of 4, and an inference batch size of 2.
For both syntax and functionality checks, we measure pass@3 accuracy as metrics. In the ablation study from~\secref{subsec:exp_finetune} to~\secref{subsec:complexity}, we adopt \textit{MachineGen} for evaluation.

Experiments are conducted on a server with four NVIDIA L20 GPUs (48 GB each), an 80 vCPU Intel® Xeon® Platinum 8457C, and 100GB of RAM. This setup ensures sufficient computational power and memory to handle the intensive demands of fine-tuning and inference efficiently, especially for long data sequences in the feedback loop experiment. 

\subsection{Effect of Supervised Finetuning}\label{subsec:exp_finetune}
Our first ablation study investigates the effect of the model fine-tuning.
We evaluated the performance based on both syntax and functionality checks. 
As shown in~\figref{fig:finetune_cot}(a), the results demonstrate that the finetuning dramatically increases syntax correctness from $54.85$\% to $88.44$\%. 
More importantly, the impact of finetuning is even more pronounced in the functionality evaluation, where the non-finetuned model failed to achieve any correct functionality test, but the accuracy is improved to $53.20$\% in the finetuned model. These enhancements highlight the critical role of finetuning in producing not only syntactically correct but also functionally viable codes, which demonstrates the benefits of finetuning LLMs for hardware design in the HLS code generation task.




\subsection{Effect of Chain-of-Thought Prompting}\label{subsec:exp_cot}
To assess the effect of the chain-of-thought (CoT) technique,
we perform both syntax and functionality evaluation on the fine-tuned model with and without the use of CoT.
As indicated in~\figref{fig:finetune_cot}(b), incorporating CoT leads to a noticeable improvement in both metrics. 
Specifically, syntax correctness increases from $88.44$\% to $94.33$\%, and functionality score rises from $53.20$\% to $61.45$\%. 
The result demonstrates the effectiveness of CoT in enhancing the reasoning capability, thereby improving its overall performance.


\subsection{Effect of Feedback Loops}\label{subsec:exp_feedback}
Our two-step feedback loop provides both syntax and functionality feedback. We evaluate the impact of these feedback loops with different numbers of iterations, ranging from 0 to 2.The results, shown in Figure ~\figref{fig:syntax_feedback} and ~\figref{fig:func_feedback}, indicate that both syntax and functionality feedback loops significantly improve model performance, especially when combined with COT prompting. The initial feedback loop yields substantial accuracy improvements in both syntax correctness and functionality evaluation, though the second loop shows diminishing returns.Syntax feedback loops enhance both syntax correctness and functionality performance, suggesting that iterative refinement is particularly effective for complex tasks. Similarly, functionality feedback loops not only improve functionality checks but also boost syntax accuracy, indicating that enhancements in functional understanding contribute to better syntactic performance.

 \begin{figure}[t]
    \centering
    \includegraphics[width=1\linewidth]{./figures/merged_finetune_cot.pdf}
    \vspace{-5mm}
    \caption{Effect of fine-tuning and chain-of-thought.}
    \label{fig:finetune_cot}
\end{figure}

\begin{figure}[t]
    \centering
    \includegraphics[width=0.95\linewidth]{./figures/Effect_of_Syntax_Feedback_Loop.pdf}
    \vspace{-2mm}
    \caption{Effect of syntax feedback loop.}
    \vspace{-2mm}
    \label{fig:syntax_feedback}
\end{figure}
\begin{figure}[t]
    \centering
    \includegraphics[width=0.95\linewidth]{./figures/Effect_of_Functionality_Feedback_Loop.pdf}
    \caption{Effect of functionality feedback loop.}
    \vspace{-2mm}
    \label{fig:func_feedback}
\end{figure}


\subsection{Time Cost and Hardware Performance}\label{subsec:exp_timecost}

\figref{fig:time_cost} shows the time cost for generating 120 data entries under different conditions, measuring the impact of CoT and feedback loops. Without a feedback loop, CoT significantly reduces the time. Adding a syntax feedback loop increases the time, but CoT continues to notably decrease the duration. The functionality feedback loop is the most time-consuming, though CoT still provides a notable reduction, albeit less dramatic. This demonstrates CoT's effectiveness in reducing operational times across varying complexities.

For the test set,
we evaluate the latency and resource consumption of the generated \textit{HLS} designs using a Xilinx VCU118 as our target FPGA, with a clock frequency of $200$MHz and Xilinx Vivado 2020.1 for synthesis.
As shown in~\tabref{tb:perf_resource}, all \textit{HLS} designs demonstrate reasonable performance, with BRAM usage consistently remained at zero due to the design scale.

\begin{figure}
    \centering
    \includegraphics[width=0.95\linewidth]{./figures/Time_Cost_Analysis.pdf}
    \vspace{-2mm}
    \caption{Time cost of code generation.}
    \label{fig:time_cost}
\end{figure}


\begin{table}[htb]
\centering
\caption{Latency and resource usage of LLM-generated designs synthesized on a VCU118 FPGA.}
\label{tb:perf_resource}
\setlength\tabcolsep{1pt} 
\scalebox{0.8}{
% \begin{tabular}{L{2cm}ccC{1.5cm}C{2.5cm}}
\begin{tabular}{C{2.5cm}|C{1.9cm}|C{1.5cm}|C{1.5cm}|C{1.3cm}|C{1.3cm}}
\toprule
{}& \textbf{Latency} (ms)& \textbf{LUTs} & \textbf{Registers} & \textbf{DSP48s} & \textbf{BRAMs} \\ \midrule
{\textbf{Available}} & - & 1182240 & 2364480 & 6840 & 4320 \\ \midrule
{\textit{ellpack}} & 0.304 & 1011 & 1079 & 11 & 0 \\
{\textit{syrk}} & 21.537 & 1371 & 1621 & 19 & 0 \\
{\textit{syr2k}} & 40.626 & 1572 & 1771 & 19 & 0 \\
{\textit{stencil2d}} & 1.368 & 287 & 123 & 3 & 0 \\
{\textit{trmm-opt}} & 15.889 & 1262 & 1239 & 11 & 0 \\
{\textit{stencil3d}} & 21.537 & 1173 & 1271 & 20 & 0 \\
{\textit{symm}} & 24.601 & 1495 & 1777 & 19 & 0 \\
{\textit{symm-opt}} & 16.153 & 1361 & 1608 & 19 & 0 \\
{\textit{symm-opt-medium}} & 579.0 & 2223 & 2245 & 22 & 0 \\
\bottomrule
\end{tabular}}
\end{table}

\subsection{Effect of Task Complexity}\label{subsec:complexity}
We analyze the effects of code complexity on the performance of fine-tuning our language model with CoT prompting and tested without the use of any feedback loops during inference. 
We categorize \textit{MachineGen} into three classes according to their code complexity: easy, medium, and difficult.
The results shown in the \tabref{tab:model_performance} indicates a clear trend: as the complexity of the generated code increases, both syntax and functionality correctness rates decline. This outcome could be attributed to several factors. First, more complex code inherently presents more challenges in maintaining syntactic integrity and functional accuracy. Second, the absence of feedback loops in the inference phase may have limited the model's ability to self-correct emerging errors in more complicated code generations.

\begin{table}[h]
\centering
\caption{Performance across different complexity levels.}
\scalebox{0.8}{
\begin{tabular}{c|c|c}
\hline
\textbf{Test Set} & \textbf{Syntax Check} & \textbf{Functionality} \\
\hline
Easy & 96.67\% & 63.33\% \\
Medium & 96.67\% & 53.33\% \\
Difficult & 90\% & 53.33\% \\
\hline
\end{tabular}}
\label{tab:model_performance}
\end{table}

\subsection{Analysis of \textit{MachineGen} and \textit{HumanRefine}}
\begin{table}[h]
\centering
\caption{Performance on \textit{MachineGen} and \textit{HumanRefine}.}
\scalebox{0.9}{
\begin{tabular}{c|c|c}
\hline
\textbf{Test Set} & \textbf{Syntax Check} & \textbf{Functionality Check} \\
\hline
\textit{MachineGen} & 93.83\% & 62.24\% \\
\hline
\textit{HumanRefine} & 47.29\% & 21.36\% \\
\hline
\end{tabular}}
\vspace{-3mm}
\label{table:eval_comparison}
\end{table}

As shown in~\tabref{table:eval_comparison}, this section compares the performance of our model on \textit{MachineGen} and \textit{HumanRefine} test sets.
Our findings reveal that the performance on the \textit{HumanRefine} is significantly lower than on the \textit{MachineGen}. This disparity suggests that the model is more adept at handling machine-generated prompts. The primary reasons for this are: the model's training data bias towards machine-generated prompts, the increased complexity and nuanced nature of human-generated prompts, and the conciseness and clarity of human-generated prompts that often omit repetitive or explicit details found in machine-generated prompts, making it harder for the model to generate syntactically and functionally correct code.

\subsection{Thoughts, Insights, and Limitations}

\noindent \textbf{1. \textit{HLS} versus \textit{HDL} for AI-assisted code generation:} The selection of programming language for hardware code generation should mainly depend on two  factors:
\begin{itemize}[leftmargin=*]
    \item \textit{Quality of Generated Hardware Design}: The evaluation of hardware design's quality includes syntax correctness, functionality, and hardware performance.
    Since \textit{HLS} shares similar semantics and syntax with programming languages commonly used during LLM pre-training, this work demonstrates that the LLM-assisted code generation for \textit{HLS} has the potential to achieve high syntax and functional correctness in hardware designs. While this work does not leverage hardware performance as feedback for design generation, it identifies this aspect as a key direction for future research and enhancements.
    \item \textit{Runtime Cost of Hardware Generation}: Although \textit{HLS}-based designs typically require fewer tokens compared to \textit{HDL} during the code generation phase—suggesting potentially lower costs—the overall runtime costs associated with HLS synthesis must also be considered. A more comprehensive quantitative comparison of these runtime costs is planned for our future work. 
\end{itemize}

\noindent \textbf{2. Input instructions and datasets are crucial}: The fine-tuning of pre-trained LLMs on \textit{HLS} dataset can bring a significant improvement in the design quality, echoing findings from previous studies on \textit{Verilog} code generation~\cite{thakur2023verigen}. 
Additionally, during our evaluation, we found that employing simple CoT prompting largely improves hardware design quality. 
This result contrasts with the application of CoT in general-purpose programming languages, where a specialized form of CoT is necessary~\cite{li2023structured}.
Therefore, future efforts for further enhancement can focus on collecting high-quality datasets and exploring better refinement of input prompts.

\noindent \textbf{3. Limitations}: At the time of this research, more advanced reasoning models, such as DeepSeek-R1~\cite{guo2025deepseek}, were not available for evaluation. Additionally, test-time scaling approaches~\cite{welleck2024decoding} could be incorporated to further enhance performance in the future.
Moreover, we observe that the diversity of hardware designs in the benchmark is limited, which may impact the generalizability of our findings.
We intend to address these limitations in our future work.



\section{Discussion}\label{sec:discussion}



\subsection{From Interactive Prompting to Interactive Multi-modal Prompting}
The rapid advancements of large pre-trained generative models including large language models and text-to-image generation models, have inspired many HCI researchers to develop interactive tools to support users in crafting appropriate prompts.
% Studies on this topic in last two years' HCI conferences are predominantly focused on helping users refine single-modality textual prompts.
Many previous studies are focused on helping users refine single-modality textual prompts.
However, for many real-world applications concerning data beyond text modality, such as multi-modal AI and embodied intelligence, information from other modalities is essential in constructing sophisticated multi-modal prompts that fully convey users' instruction.
This demand inspires some researchers to develop multimodal prompting interactions to facilitate generation tasks ranging from visual modality image generation~\cite{wang2024promptcharm, promptpaint} to textual modality story generation~\cite{chung2022tale}.
% Some previous studies contributed relevant findings on this topic. 
Specifically, for the image generation task, recent studies have contributed some relevant findings on multi-modal prompting.
For example, PromptCharm~\cite{wang2024promptcharm} discovers the importance of multimodal feedback in refining initial text-based prompting in diffusion models.
However, the multi-modal interactions in PromptCharm are mainly focused on the feedback empowered the inpainting function, instead of supporting initial multimodal sketch-prompt control. 

\begin{figure*}[t]
    \centering
    \includegraphics[width=0.9\textwidth]{src/img/novice_expert.pdf}
    \vspace{-2mm}
    \caption{The comparison between novice and expert participants in painting reveals that experts produce more accurate and fine-grained sketches, resulting in closer alignment with reference images in close-ended tasks. Conversely, in open-ended tasks, expert fine-grained strokes fail to generate precise results due to \tool's lack of control at the thin stroke level.}
    \Description{The comparison between novice and expert participants in painting reveals that experts produce more accurate and fine-grained sketches, resulting in closer alignment with reference images in close-ended tasks. Novice users create rougher sketches with less accuracy in shape. Conversely, in open-ended tasks, expert fine-grained strokes fail to generate precise results due to \tool's lack of control at the thin stroke level, while novice users' broader strokes yield results more aligned with their sketches.}
    \label{fig:novice_expert}
    % \vspace{-3mm}
\end{figure*}


% In particular, in the initial control input, users are unable to explicitly specify multi-modal generation intents.
In another example, PromptPaint~\cite{promptpaint} stresses the importance of paint-medium-like interactions and introduces Prompt stencil functions that allow users to perform fine-grained controls with localized image generation. 
However, insufficient spatial control (\eg, PromptPaint only allows for single-object prompt stencil at a time) and unstable models can still leave some users feeling the uncertainty of AI and a varying degree of ownership of the generated artwork~\cite{promptpaint}.
% As a result, the gap between intuitive multi-modal or paint-medium-like control and the current prompting interface still exists, which requires further research on multi-modal prompting interactions.
From this perspective, our work seeks to further enhance multi-object spatial-semantic prompting control by users' natural sketching.
However, there are still some challenges to be resolved, such as consistent multi-object generation in multiple rounds to increase stability and improved understanding of user sketches.   


% \new{
% From this perspective, our work is a step forward in this direction by allowing multi-object spatial-semantic prompting control by users' natural sketching, which considers the interplay between multiple sketch regions.
% % To further advance the multi-modal prompting experience, there are some aspects we identify to be important.
% % One of the important aspects is enhancing the consistency and stability of multiple rounds of generation to reduce the uncertainty and loss of control on users' part.
% % For this purpose, we need to develop techniques to incorporate consistent generation~\cite{tewel2024training} into multi-modal prompting framework.}
% % Another important aspect is improving generative models' understanding of the implicit user intents \new{implied by the paint-medium-like or sketch-based input (\eg, sketch of two people with their hands slightly overlapping indicates holding hand without needing explicit prompt).
% % This can facilitate more natural control and alleviate users' effort in tuning the textual prompt.
% % In addition, it can increase users' sense of ownership as the generated results can be more aligned with their sketching intents.
% }
% For example, when users draw sketches of two people with their hands slightly overlapping, current region-based models cannot automatically infer users' implicit intention that the two people are holding hands.
% Instead, they still require users to explicitly specify in the prompt such relationship.
% \tool addresses this through sketch-aware prompt recommendation to fill in the necessary semantic information, alleviating users' workload.
% However, some users want the generative AI in the future to be able to directly infer this natural implicit intentions from the sketches without additional prompting since prompt recommendation can still be unstable sometimes.


% \new{
% Besides visual generation, 
% }
% For example, one of the important aspect is referring~\cite{he2024multi}, linking specific text semantics with specific spatial object, which is partly what we do in our sketch-aware prompt recommendation.
% Analogously, in natural communication between humans, text or audio alone often cannot suffice in expressing the speakers' intentions, and speakers often need to refer to an existing spatial object or draw out an illustration of her ideas for better explanation.
% Philosophically, we HCI researchers are mostly concerned about the human-end experience in human-AI communications.
% However, studies on prompting is unique in that we should not just care about the human-end interaction, but also make sure that AI can really get what the human means and produce intention-aligned output.
% Such consideration can drastically impact the design of prompting interactions in human-AI collaboration applications.
% On this note, although studies on multi-modal interactions is a well-established topic in HCI community, it remains a challenging problem what kind of multi-modal information is really effective in helping humans convey their ideas to current and next generation large AI models.




\subsection{Novice Performance vs. Expert Performance}\label{sec:nVe}
In this section we discuss the performance difference between novice and expert regarding experience in painting and prompting.
First, regarding painting skills, some participants with experience (4/12) preferred to draw accurate and fine-grained shapes at the beginning. 
All novice users (5/12) draw rough and less accurate shapes, while some participants with basic painting skills (3/12) also favored sketching rough areas of objects, as exemplified in Figure~\ref{fig:novice_expert}.
The experienced participants using fine-grained strokes (4/12, none of whom were experienced in prompting) achieved higher IoU scores (0.557) in the close-ended task (0.535) when using \tool. 
This is because their sketches were closer in shape and location to the reference, making the single object decomposition result more accurate.
Also, experienced participants are better at arranging spatial location and size of objects than novice participants.
However, some experienced participants (3/12) have mentioned that the fine-grained stroke sometimes makes them frustrated.
As P1's comment for his result in open-ended task: "\emph{It seems it cannot understand thin strokes; even if the shape is accurate, it can only generate content roughly around the area, especially when there is overlapping.}" 
This suggests that while \tool\ provides rough control to produce reasonably fine results from less accurate sketches for novice users, it may disappoint experienced users seeking more precise control through finer strokes. 
As shown in the last column in Figure~\ref{fig:novice_expert}, the dragon hovering in the sky was wrongly turned into a standing large dragon by \tool.

Second, regarding prompting skills, 3 out of 12 participants had one or more years of experience in T2I prompting. These participants used more modifiers than others during both T2I and R2I tasks.
Their performance in the T2I (0.335) and R2I (0.469) tasks showed higher scores than the average T2I (0.314) and R2I (0.418), but there was no performance improvement with \tool\ between their results (0.508) and the overall average score (0.528). 
This indicates that \tool\ can assist novice users in prompting, enabling them to produce satisfactory images similar to those created by users with prompting expertise.



\subsection{Applicability of \tool}
The feedback from user study highlighted several potential applications for our system. 
Three participants (P2, P6, P8) mentioned its possible use in commercial advertising design, emphasizing the importance of controllability for such work. 
They noted that the system's flexibility allows designers to quickly experiment with different settings.
Some participants (N = 3) also mentioned its potential for digital asset creation, particularly for game asset design. 
P7, a game mod developer, found the system highly useful for mod development. 
He explained: "\emph{Mods often require a series of images with a consistent theme and specific spatial requirements. 
For example, in a sacrifice scene, how the objects are arranged is closely tied to the mod's background. It would be difficult for a developer without professional skills, but with this system, it is possible to quickly construct such images}."
A few participants expressed similar thoughts regarding its use in scene construction, such as in film production. 
An interesting suggestion came from participant P4, who proposed its application in crime scene description. 
She pointed out that witnesses are often not skilled artists, and typically describe crime scenes verbally while someone else illustrates their account. 
With this system, witnesses could more easily express what they saw themselves, potentially producing depictions closer to the real events. "\emph{Details like object locations and distances from buildings can be easily conveyed using the system}," she added.

% \subsection{Model Understanding of Users' Implicit Intents}
% In region-sketch-based control of generative models, a significant gap between interaction design and actual implementation is the model's failure in understanding users' naturally expressed intentions.
% For example, when users draw sketches of two people with their hands slightly overlapping, current region-based models cannot automatically infer users' implicit intention that the two people are holding hands.
% Instead, they still require users to explicitly specify in the prompt such relationship.
% \tool addresses this through sketch-aware prompt recommendation to fill in the necessary semantic information, alleviating users' workload.
% However, some users want the generative AI in the future to be able to directly infer this natural implicit intentions from the sketches without additional prompting since prompt recommendation can still be unstable sometimes.
% This problem reflects a more general dilemma, which ubiquitously exists in all forms of conditioned control for generative models such as canny or scribble control.
% This is because all the control models are trained on pairs of explicit control signal and target image, which is lacking further interpretation or customization of the user intentions behind the seemingly straightforward input.
% For another example, the generative models cannot understand what abstraction level the user has in mind for her personal scribbles.
% Such problems leave more challenges to be addressed by future human-AI co-creation research.
% One possible direction is fine-tuning the conditioned models on individual user's conditioned control data to provide more customized interpretation. 

% \subsection{Balance between recommendation and autonomy}
% AIGC tools are a typical example of 
\subsection{Progressive Sketching}
Currently \tool is mainly aimed at novice users who are only capable of creating very rough sketches by themselves.
However, more accomplished painters or even professional artists typically have a coarse-to-fine creative process. 
Such a process is most evident in painting styles like traditional oil painting or digital impasto painting, where artists first quickly lay down large color patches to outline the most primitive proportion and structure of visual elements.
After that, the artists will progressively add layers of finer color strokes to the canvas to gradually refine the painting to an exquisite piece of artwork.
One participant in our user study (P1) , as a professional painter, has mentioned a similar point "\emph{
I think it is useful for laying out the big picture, give some inspirations for the initial drawing stage}."
Therefore, rough sketch also plays a part in the professional artists' creation process, yet it is more challenging to integrate AI into this more complex coarse-to-fine procedure.
Particularly, artists would like to preserve some of their finer strokes in later progression, not just the shape of the initial sketch.
In addition, instead of requiring the tool to generate a finished piece of artwork, some artists may prefer a model that can generate another more accurate sketch based on the initial one, and leave the final coloring and refining to the artists themselves.
To accommodate these diverse progressive sketching requirements, a more advanced sketch-based AI-assisted creation tool should be developed that can seamlessly enable artist intervention at any stage of the sketch and maximally preserve their creative intents to the finest level. 

\subsection{Ethical Issues}
Intellectual property and unethical misuse are two potential ethical concerns of AI-assisted creative tools, particularly those targeting novice users.
In terms of intellectual property, \tool hands over to novice users more control, giving them a higher sense of ownership of the creation.
However, the question still remains: how much contribution from the user's part constitutes full authorship of the artwork?
As \tool still relies on backbone generative models which may be trained on uncopyrighted data largely responsible for turning the sketch into finished artwork, we should design some mechanisms to circumvent this risk.
For example, we can allow artists to upload backbone models trained on their own artworks to integrate with our sketch control.
Regarding unethical misuse, \tool makes fine-grained spatial control more accessible to novice users, who may maliciously generate inappropriate content such as more realistic deepfake with specific postures they want or other explicit content.
To address this issue, we plan to incorporate a more sophisticated filtering mechanism that can detect and screen unethical content with more complex spatial-semantic conditions. 
% In the future, we plan to enable artists to upload their own style model

% \subsection{From interactive prompting to interactive spatial prompting}


\subsection{Limitations and Future work}

    \textbf{User Study Design}. Our open-ended task assesses the usability of \tool's system features in general use cases. To further examine aspects such as creativity and controllability across different methods, the open-ended task could be improved by incorporating baselines to provide more insightful comparative analysis. 
    Besides, in close-ended tasks, while the fixing order of tool usage prevents prior knowledge leakage, it might introduce learning effects. In our study, we include practice sessions for the three systems before the formal task to mitigate these effects. In the future, utilizing parallel tests (\textit{e.g.} different content with the same difficulty) or adding a control group could further reduce the learning effects.

    \textbf{Failure Cases}. There are certain failure cases with \tool that can limit its usability. 
    Firstly, when there are three or more objects with similar semantics, objects may still be missing despite prompt recommendations. 
    Secondly, if an object's stroke is thin, \tool may incorrectly interpret it as a full area, as demonstrated in the expert results of the open-ended task in Figure~\ref{fig:novice_expert}. 
    Finally, sometimes inclusion relationships (\textit{e.g.} inside) between objects cannot be generated correctly, partially due to biases in the base model that lack training samples with such relationship. 

    \textbf{More support for single object adjustment}.
    Participants (N=4) suggested that additional control features should be introduced, beyond just adjusting size and location. They noted that when objects overlap, they cannot freely control which object appears on top or which should be covered, and overlapping areas are currently not allowed.
    They proposed adding features such as layer control and depth control within the single-object mask manipulation. Currently, the system assigns layers based on color order, but future versions should allow users to adjust the layer of each object freely, while considering weighted prompts for overlapping areas.

    \textbf{More customized generation ability}.
    Our current system is built around a single model $ColorfulXL-Lightning$, which limits its ability to fully support the diverse creative needs of users. Feedback from participants has indicated a strong desire for more flexibility in style and personalization, such as integrating fine-tuned models that cater to specific artistic styles or individual preferences. 
    This limitation restricts the ability to adapt to varied creative intents across different users and contexts.
    In future iterations, we plan to address this by embedding a model selection feature, allowing users to choose from a variety of pre-trained or custom fine-tuned models that better align with their stylistic preferences. 
    
    \textbf{Integrate other model functions}.
    Our current system is compatible with many existing tools, such as Promptist~\cite{hao2024optimizing} and Magic Prompt, allowing users to iteratively generate prompts for single objects. However, the integration of these functions is somewhat limited in scope, and users may benefit from a broader range of interactive options, especially for more complex generation tasks. Additionally, for multimodal large models, users can currently explore using affordable or open-source models like Qwen2-VL~\cite{qwen} and InternVL2-Llama3~\cite{llama}, which have demonstrated solid inference performance in our tests. While GPT-4o remains a leading choice, alternative models also offer competitive results.
    Moving forward, we aim to integrate more multimodal large models into the system, giving users the flexibility to choose the models that best fit their needs. 
    


\section{Conclusion}\label{sec:conclusion}
In this paper, we present \tool, an interactive system designed to help novice users create high-quality, fine-grained images that align with their intentions based on rough sketches. 
The system first refines the user's initial prompt into a complete and coherent one that matches the rough sketch, ensuring the generated results are both stable, coherent and high quality.
To further support users in achieving fine-grained alignment between the generated image and their creative intent without requiring professional skills, we introduce a decompose-and-recompose strategy. 
This allows users to select desired, refined object shapes for individual decomposed objects and then recombine them, providing flexible mask manipulation for precise spatial control.
The framework operates through a coarse-to-fine process, enabling iterative and fine-grained control that is not possible with traditional end-to-end generation methods. 
Our user study demonstrates that \tool offers novice users enhanced flexibility in control and fine-grained alignment between their intentions and the generated images.



\bibliographystyle{unsrt}
% \section*{References}
\begin{thebibliography}{00}

\bibitem{EnvironmentalPollution} Liu W, Xu Y, Fan D, et al. Alleviating corporate environmental pollution threats toward public health and safety: the role of smart city and artificial intelligence[J]. Safety Science, 2021, 143: 105433.

\bibitem{PervasiveandMobileComputing} Nagy A M, Simon V. Survey on traffic prediction in smart cities[J]. Pervasive and Mobile Computing, 2018, 50: 148-163.

\bibitem{ExpertSystems} Pramanik M I, Lau R Y K, Demirkan H, et al. Smart health: Big data enabled health paradigm within smart cities[J]. Expert Systems with Applications, 2017, 87: 370-383.

\bibitem{ITJ19} Kong X, Liu X, Jedari B, et al. Mobile crowdsourcing in smart cities: Technologies, applications, and future challenges[J]. IEEE Internet of Things Journal, 2019, 6(5): 8095-8113.

\bibitem{BigData} Zhai S, Li R, Yang Y. Crowdsensing big data: sensing, data selection, and understanding[C]//Journal of Physics: Conference Series. IOP Publishing, 2021, 1848(1): 012045.

\bibitem{TMC19} Liu C H, Dai Z, Zhao Y, et al. Distributed and energy-efficient mobile crowdsensing with charging stations by deep reinforcement learning[J]. IEEE Transactions on Mobile Computing, 2019, 20(1): 130-146.

% \bibitem{INFO23} Sun J, Jin H, Ding R, et al. Multi-Objective Order Dispatch for Urban Crowd Sensing with For-Hire Vehicles[C]//IEEE INFOCOM 2023-IEEE Conference on Computer Communications. IEEE, 2023: 1-10.

\bibitem{INFO21} Fan G, Zhao Y, Guo Z, et al. Towards fine-grained spatio-temporal coverage for vehicular urban sensing systems[C]//IEEE INFOCOM 2021-IEEE Conference on Computer Communications. IEEE, 2021: 1-10.


\bibitem{INFO23Privacy}Sun H, Xiao M, Xu Y, et al. Privacy-preserving Stable Crowdsensing Data Trading for Unknown Market[C]//IEEE INFOCOM 2023-IEEE Conference on Computer Communications. IEEE, 2023: 1-10.


\bibitem{INFO22Privacy} Wang H, Wang E, Yang Y, et al. Privacy-preserving online task assignment in spatial crowdsourcing: A graph-based approach[C]//IEEE INFOCOM 2022-IEEE Conference on Computer Communications. IEEE, 2022: 570-579.

\bibitem{TITS2020Privacy} Zhang J, Yang F, Ma Z, et al. A decentralized location privacy-preserving spatial crowdsourcing for internet of vehicles[J]. IEEE Transactions on Intelligent Transportation Systems, 2020, 22(4): 2299-2313.

\bibitem{vehiclecrowdsensingdeepapproach} Zhu X, Luo Y, Liu A, et al. A deep learning-based mobile crowdsensing scheme by predicting vehicle mobility[J]. IEEE Transactions on Intelligent Transportation Systems, 2020, 22(7): 4648-4659.

\bibitem{ICDE23} Ye Y, Liu C H, Dai Z, et al. Exploring both individuality and cooperation for air-ground spatial crowdsourcing by multi-agent deep reinforcement learning[C]//2023 IEEE 39th International Conference on Data Engineering (ICDE). IEEE, 2023: 205-217.

\bibitem{KDD17} Zhang L, Hu T, Min Y, et al. A taxi order dispatch model based on combinatorial optimization[C]//Proceedings of the 23rd ACM SIGKDD international conference on knowledge discovery and data mining. 2017: 2151-2159.

\bibitem{WWW19} Yuen C F, Singh A P, Goyal S, et al. Beyond shortest paths: Route recommendations for ride-sharing[C]//The World Wide Web Conference. 2019: 2258-2269.

\bibitem{Ubiquitous18} Xie X, Zhang F, Zhang D. PrivateHunt: Multi-source data-driven dispatching in for-hire vehicle systems[J]. Proceedings of the ACM on Interactive, Mobile, Wearable and Ubiquitous Technologies, 2018, 2(1): 1-26.

\bibitem{vehicle_dispatch_modle_based} Zhang R, Rossi F, Pavone M. Model predictive control of autonomous mobility-on-demand systems[C]//2016 IEEE international conference on robotics and automation (ICRA). IEEE, 2016: 1382-1389.

\bibitem{introduce_drl} Arulkumaran K, Deisenroth M P, Brundage M, et al. Deep reinforcement learning: A brief survey[J]. IEEE Signal Processing Magazine, 2017, 34(6): 26-38.

\bibitem{TITS22} Haydari A, Yılmaz Y. Deep reinforcement learning for intelligent transportation systems: A survey[J]. IEEE Transactions on Intelligent Transportation Systems, 2020, 23(1): 11-32.

\bibitem{INFO20} Liu C H, Dai Z, Yang H, et al. Multi-task-oriented vehicular crowdsensing: A deep learning approach[C]//IEEE INFOCOM 2020-IEEE Conference on Computer Communications. IEEE, 2020: 1123-1132.

\bibitem{INFO21VEHICLES} Ding R, Yang Z, Wei Y, et al. Multi-agent reinforcement learning for urban crowd sensing with for-hire vehicles[C]//IEEE INFOCOM 2021-IEEE Conference on Computer Communications. IEEE, 2021: 1-10.

\bibitem{INFO18} Oda T, Joe-Wong C. MOVI: A model-free approach to dynamic fleet management[C]//IEEE INFOCOM 2018-IEEE Conference on Computer Communications. IEEE, 2018: 2708-2716.

\bibitem{TITS20} Liu Z, Li J, Wu K. Context-aware taxi dispatching at city-scale using deep reinforcement learning[J]. IEEE Transactions on Intelligent Transportation Systems, 2020, 23(3): 1996-2009.

\bibitem{KDD18} Lin K, Zhao R, Xu Z, et al. Efficient large-scale fleet management via multi-agent deep reinforcement learning[C]//Proceedings of the 24th ACM SIGKDD international conference on Knowledge Discovery and Data Mining. 2018: 1774-1783.

\bibitem{KDD22} Sun J, Jin H, Yang Z, et al. Optimizing long-term efficiency and fairness in ride-hailing via joint order dispatching and driver repositioning[C]//Proceedings of the 28th ACM SIGKDD Conference on Knowledge Discovery and Data Mining. 2022: 3950-3960.

\bibitem{introduce_AoI1} Zhang S, Zhang H, Han Z, et al. Age of information in a cellular internet of UAVs: Sensing and communication trade-off design[J]. IEEE Transactions on Wireless Communications, 2020, 19(10): 6578-6592.

\bibitem{introduce_AoI2} Chen H, Gu Y, Liew S C. Age-of-information dependent random access for massive IoT networks[C]//IEEE INFOCOM 2020-IEEE Conference on Computer Communications Workshops (INFOCOM WKSHPS). IEEE, 2020: 930-935.

\bibitem{the_reason_why_data_freshness_is_important} Abd-Elmagid M A, Pappas N, Dhillon H S. On the role of age of information in the Internet of Things[J]. IEEE Communications Magazine, 2019, 57(12): 72-77.

\bibitem{a_example_for_the_reason_why_data_freshness_is_important} Cheng Y, Wang X, Zhou P, et al. Freshness-Aware Incentive Mechanism for Mobile Crowdsensing with Budget Constraint[J]. IEEE Transactions on Services Computing, 2023.

\bibitem{why_use_MARL}Zhang K, Yang Z, Başar T. Multi-agent reinforcement learning: A selective overview of theories and algorithms[J]. Handbook of reinforcement learning and control, 2021: 321-384.

\bibitem{R-GCN} Schlichtkrull M, Kipf T N, Bloem P, et al. Modeling relational data with graph convolutional networks[C]//The Semantic Web: 15th International Conference, ESWC 2018, Heraklion, Crete, Greece, June 3–7, 2018, Proceedings 15. Springer International Publishing, 2018: 593-607.

\bibitem{liyihong} Li Y, Zeng T, Zhang X, et al. TapFinger: Task Placement and Fine-Grained Resource Allocation for Edge Machine Learning[C]//IEEE INFOCOM. 2023.

\bibitem{neurips22} Yu C, Velu A, Vinitsky E, et al. The surprising effectiveness of ppo in cooperative multi-agent games[J]. Advances in Neural Information Processing Systems, 2022, 35: 24611-24624.

\bibitem{PPO} Schulman J, Wolski F, Dhariwal P, et al. Proximal policy optimization algorithms[J]. arXiv preprint arXiv:1707.06347, 2017.


\bibitem{newyork_taxi} https://www1.nyc.gov/site/tlc/about/tlc-trip-record-data.page.

\bibitem{assume_cost_one_time_slot_to_reach_dispatched_destination} Jin J, Zhou M, Zhang W, et al. Coride: joint order dispatching and fleet management for multi-scale ride-hailing platforms[C]//Proceedings of the 28th ACM international conference on information and knowledge management. 2019: 1983-1992.

\bibitem{UCB} Kuleshov V, Precup D. Algorithms for multi-armed bandit problems[J]. arXiv preprint arXiv:1402.6028, 2014.

\bibitem{espeholt2018impala} Espeholt L, Soyer H, Munos R, et al. Impala: Scalable distributed deep-rl with importance weighted actor-learner architectures[C]//International conference on machine learning. PMLR, 2018: 1407-1416.

\bibitem{li2019privacy} Li M, Zhu L, Lin X. Privacy-preserving traffic monitoring with false report filtering via fog-assisted vehicular crowdsensing[J]. IEEE Transactions on Services Computing, 2019, 14(6): 1902-1913.
% \bibitem{StreetView} Anguelov D, Dulong C, Filip D, et al. Google street view: Capturing the world at street level[J]. Computer, 2010, 43(6): 32-38.
\bibitem{zhang2021data}Zhang X, Wang J, Zhang H, et al. Data-driven transportation network company vehicle scheduling with users’ location differential privacy preservation[J]. IEEE Transactions on Mobile Computing, 2021, 22(2): 813-823.
\bibitem{liu2024multi}Liu L, Huang Z, Xu J. Multi-Agent Deep Reinforcement Learning Based Scheduling Approach for Mobile Charging in Internet of Electric Vehicles[J]. IEEE Transactions on Mobile Computing, 2024.
% \bibitem{INFO22} Xiang C, Li Y, Zhou Y, et al. A comparative approach to resurrecting the market of mod vehicular crowdsensing[C]//IEEE INFOCOM 2022-IEEE Conference on Computer Communications. IEEE, 2022: 1479-1488.

% \bibitem{ICDE23UAV} Wang Y, Wu J, Hua X, et al. Air-ground spatial crowdsourcing with uav carriers by geometric graph convolutional multi-agent deep reinforcement learning[C]//2023 IEEE 39th International Conference on Data Engineering (ICDE). IEEE, 2023: 1790-1802.


\end{thebibliography}
\vspace{12pt}
\color{red}





\end{document}

