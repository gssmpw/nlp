
\section{Evaluation}\label{sec_evaluation}

We evaluate \ourSystem in three studies and two applications:
\textbf{\textbullet~RFID}~spatial spectrum synthesis;
\textbf{\textbullet~BLE} received signal strength indicator~(RSSI) prediction, applied to fingerprinting-based localization application;
\textbf{\textbullet~LoRa} RSSI prediction, applied to LoRa gateway coverage estimation application.




\begin{figure}[t]
\centering
{\includegraphics[width=.4\textwidth]{figs/vis_four_locations.pdf}}
	\vspace{5pt}
\caption{\nerft and \ourSystem visualizations of spatial spectrum at four transmitter positions.}
    \Description[]{}
	\label{fig_vis_d1}
\end{figure}


\subsection{Field Study - RFID Spectrum Synthesis}\label{sec_overall_rfid}


\subsubsection{Experimental Setup}
\underline{\textsc{i) Task:}}
Given a transmitter sending signals at location~$\left(x_{\text{tx}}, y_{\text{tx}}, z_{\text{tx}}\right)$, the task is to synthesize the spatial spectrum received by the receiver~(antenna array) at a fixed location.  
The spatial spectrum, represented as a (360, 90) matrix, captures received signal power from all directions, covering azimuth and elevation angles at one-degree resolution.
The elevation angle is limited to 90$^\circ$ as only the front hemisphere of the antenna array is considered~\cite{zhao2023nerf}.  


\underline{\textsc{ii) Dataset:}}
The public RFID dataset from \nerft~\cite{zhao2023nerf}, collected in real-world environments, is employed.  
It contains 6,123 transmitter~(RFID tag) locations within a 3D space of 275.8\,$\text{ft}^3$ and their corresponding spatial spectra, received by a receiver with a $4 \times 4$ antenna array operating at 915\,MHz.
The dataset is randomly split 70\% training and 30\% testing.  


\underline{\textsc{iii) Metrics:}}
We employ the following two metrics:

\textbullet~
Mean Squared Error (MSE)$\downarrow$: This metric calculates the average of the squared differences between the synthesized spectrum and the ground truth.

\textbullet~
Peak Signal-to-Noise Ratio~(PSNR, in dB)$\uparrow$: Treating the spatial spectrum as an image, PSNR measures structural similarity, with higher values indicating better quality.



\underline{\textsc{iv) Baselines:}}
We compare \ourSystem with \nerft~\cite{zhao2023nerf}.  
Other simulation-based or ML-based methods, such as simulation via MATLAB~\cite{RayTracingToolbox}, DCGAN~\cite{radford2015unsupervised}, and VAE~\cite{liu2021fire}, perform worse on the RFID dataset~\cite{zhao2023nerf} compared to \nerft.  





\subsubsection{Overall Performance}\label{sec_spectrum_syn}
\textbf{Setting:}
To evaluate \ourSystem's performance in scenarios with insufficient data, we randomly select 220 instances from the training dataset instead of using the full training data.  
This creates a sparse dataset with a measurement density of 0.8\,$\text{measurements/ft}^3$.  



\textbf{Result:}  
Figure~\ref{fig_vis_d1} presents the real-collected spatial spectra for four randomly selected transmitter positions~(first row), alongside those generated by \ourSystem~(second row) and~\nerft (third row).  
Visually, the spectra synthesized by \ourSystem more closely match the ground truth compared to those by \nerft.  
Figure~\ref{fig_quanti_d1} then shows the Cumulative Distribution Function~(CDF) of the two metric scores for both models on the testing data. 
\ourSystem outperforms \nerft by 21.2\% and 56.4\% on average in PSNR and MSE, respectively.  
This superior performance is attributed to the Gaussian-based scene representation module, which enables more efficient learning of scene representation from sparse training data.  







\begin{figure}[tp]
\centering
	\subfigure[MSE score$\downarrow$]{
\includegraphics[width=.23\textwidth]{figs/mse_gs_rfid.pdf}}
	\subfigure[PSNR score (dB)$\uparrow$]{
	\includegraphics[width=.23\textwidth]{figs/psnr_gs.pdf}}
\caption{Comparison between \nerft and \ourSystem for sparse measurement density of 0.8\,\(\text{measurements}/\text{ft}^3\).}
\Description[]{}
	\label{fig_quanti_d1}
\end{figure}


\subsubsection{Training and Inference Time}  
\textbf{Setting:}
Both methods are trained for 100,000 iterations.  
Training time is measured by running each method 10 times on two computers equipped with NVIDIA GeForce RTX 3080 Ti or 4090 GPUs.  
Inference time for each model is also recorded.



\textbf{Result:}
Figure~\ref{fig_time_rfid_training} shows that \ourSystem trains in 0.27 hours, achieving an 18.56-fold speedup over \nerft's 5.01 hours.
Similarly, Figure~\ref{fig_time_rfid_inference} demonstrates that \ourSystem achieves an 84.39-fold speedup, reducing inference time from 352.73 ms with \nerft to 4.18 ms with \ourSystem.
This improvement is due to the flexible Gaussian-based scene representation, which avoids computations in empty space and focuses only on areas containing objects. 
Moreover, \ourSystem's ray tracing algorithm avoids querying large MLP networks, a significant time-consuming process in \nerft.


\subsubsection{Measurement Density}\label{sec_rfid_density}  
\textbf{Setting:} 
We compare the performance of \ourSystem~(trained on a dataset with a density of 0.8\,\(\text{measurements}/\text{ft}^3\)) to \nerft~(trained on varying densities: 0.8, 1.6, 3.1, 4.7, 7.8, 12.4, and 15.5\,\(\text{measurements}/\text{ft}^3\)).  


\textbf{Result:}  
Figure~\ref{fig_density_3dgs_nerft_mse} presents the MSE scores on the testing dataset.  
\ourSystem achieves a comparable MSE to \nerft trained on a dataset with a density of 7.8\,\(\text{measurements}/\text{ft}^3\).  
Specifically, \ourSystem records an average MSE of 0.0312, compared to 0.0349 for \nerft.  
This indicates that \ourSystem requires 9.8\,$\times$ less training data to achieve similar spectrum quality to \nerft.  
The improvement arises from \ourSystem's 3D Gaussian-based scene representation, which focuses on object features rather than empty space, making it more efficient than \nerft's voxel-based radiance fields.  



\subsubsection{Practical Benefits}\label{sec_rfid_loc}
An angular artificial neural network~(AANN) identifies the Angle of Arrival~(AoA) of line-of-sight path from received spatial spectra, enabling spectrum-based localization~\cite{an2020general}.  
The AANN is trained on pairs of spectra and their corresponding AoA labels.
Both \ourSystem and \nerft can synthesize spectra for AANN training.  


Compared to \nerft, adopting \ourSystem can significantly reduce real-world resource consumption.  
For example, in a conference room~(\(26.2\,\textit{ft} \times 16.4\,\textit{ft} \times 9.8\,\textit{ft}\))~\cite{matlab_conference_room} with a measurement time of one minute per measurement, reducing the measurement density from 7.8 to 0.8\,\(\text{measurements}/\text{ft}^3\) saves approximately 200\,hours of data collection time.  
Additionally, \ourSystem reduces computing time by 5.71\,hours, including 4.74\,hours for training and 0.97\,hours for inference, both of which greatly save computational resources.  
Localization results with an AANN trained on synthesized spectra are not presented in this section.  
However, as shown in Section~\ref{sec_spectrum_syn}, \ourSystem's superior spectrum quality suggests improved localization performance.
Section~\ref{sec_overall_ble} further quantifies the localization-related performance gain.



\subsection{Field Study - BLE RSSI Prediction}~\label{sec_overall_ble}


\subsubsection{Experimental Setup}


\underline{\textsc{i) Task:}}
This task verifies that \ourSystem supports single-antenna setups capturing a single-valued RSSI.
Given a transmitter~(BLE node) sending signals from location~$\left(x_{\text{tx}}, y_{\text{tx}}, z_{\text{tx}}\right)$, the goal is to predict the RSSI~(in dBm) received by a receiver~(BLE gateway with a single antenna) at a fixed location.  
The measured RSSI represents the aggregate signal power from all directions~\cite{zhao2023nerf}.  


Additionally, we conduct a fingerprint-based localization application to showcase \ourSystem's sensing advantages.  



\begin{figure}[t]
	\begin{minipage}[t]{0.48\linewidth} 
    	\includegraphics[width=\textwidth]{figs/training_time_3dgs.pdf}
                \caption{Training times for spectrum synthesis.}
        \label{fig_time_rfid_training}
	\end{minipage}
 \hspace{0.05in}
 	\begin{minipage}[t]{0.48\linewidth} 
    	\includegraphics[width=\textwidth]{figs/inference_time_3dgs.pdf}
                \caption{Inference times for spectrum synthesis.}
        \label{fig_time_rfid_inference}
	\end{minipage}
 \Description[]{}
\end{figure}



\underline{\textsc{ii) Dataset:}}
The public BLE dataset~\cite{zhao2023nerf}, collected in an elderly nursing home, is employed.  
Twenty-one receivers operating at 2.4\,GHz capture RSSI readings.  
The dataset contains 6,000 transmitter positions, each paired with a 21-dimensional tuple of RSSI readings from the 21 receivers.  
If a receiver misses a signal, the RSSI is recorded as -100\,dBm.  
The dataset is randomly split 70\% training and 30\% testing.  


\underline{\textsc{iii) Metrics:}}
RSSI prediction error is the absolute difference between the model's predictions and the ground truth.


\underline{\textsc{iv) Baselines:}}
We compare \ourSystem with \nerft. 
Other empirical and ML-based methods, including MRI~\cite{shin2014mri} and CGAN~\cite{parralejo2021comparative}, are excluded because they perform worse than \nerft on the same testing dataset, as demonstrated in~\cite{zhao2023nerf}.  




\subsubsection{Overall Performance}\label{sec_ble_overall}

\textbf{Setting:}
To evaluate the performance of \ourSystem in scenarios with sufficient data, both models are trained on the full training dataset.  



\textbf{Result:}  
Figure~\ref{fig_amplitude_phase} indicates that \ourSystem achieves an average RSSI prediction error of 4.09\,dBm, compared to \nerft's 6.09\,dBm.
This represents a 32.79\% improvement, highlighting \ourSystem's effectiveness even with a single-antenna receiver.
The performance gain stems from \ourSystem's flexible 3D Gaussian-based scene representation, which efficiently utilizes training data by focusing on objects rather than empty space and aligning with object geometry.  




\subsubsection{Localization Application}  
\textbf{Setting:}
In fingerprinting-based localization, the RSSI value from an unknown transmitter queries a fingerprint database containing pairs of transmitter positions and corresponding RSSI values.  
The K Nearest Neighbors~(KNN) identifies the K nearest matches and estimates the unknown transmitter position as the average of these K positions~\cite{parralejo2021comparative}. We generate synthetic datasets using \ourSystem and \nerft, trained as described in subsection~\ref{sec_ble_overall}, to build the fingerprint database for comparison.  



\textbf{Result:}  
Figure~\ref{fig_amplitude_2s} shows that \ourSystem outperforms \nerft by 31.40\% on average.  
The slightly smaller localization gain compared to the RSSI prediction gain is due to the resilience of the KNN algorithm, which mitigates prediction errors by averaging across K nearest neighbors.  
This demonstrates that high-fidelity data benefits localization applications.  


\begin{figure}[!tp]
	{\includegraphics[width=.45\textwidth]{figs/mse_gs_ratios.pdf}}
\caption{MSE of \ourSystem~(RM) at 0.8\,\(\text{measurements}/\text{ft}^3\) and \nerft~(NF) across different densities.}
\label{fig_density_3dgs_nerft_mse}
\Description[]{}
\end{figure}


\subsubsection{Practical Benefits}\label{sec_ble_loc}  
Similar to the previous field study, \ourSystem eliminates the need for site surveys, significantly reducing data collection time.  
Its fast training GPU-hours and low inference latency save server computation resources, accelerating the construction of the fingerprint database.  



\subsection{Field Study - LoRa RSSI Prediction}\label{sec_lora_gateway}



\subsubsection{Experimental Setup}

\underline{\textsc{i) Task:}}
Given a transmitter~(LoRa node) at any location~$\left(x_{\text{tx}}, y_{\text{tx}}, z_{\text{tx}}\right)$, the task is to predict the single-valued RSSI~(in dBm) received by a fixed receiver~(LoRa gateway with a single antenna).




Additionally, we conduct the LoRa gateway coverage estimation application to demonstrate the practical advantages of \ourSystem for network planning task.  




\underline{\textsc{ii) Dataset:}} 
We use LoRa network~\cite{shahid2021concurrent} to collect spatial dataset in a pistachio orchard~\cite{ding2024optimizing}, where trees have an average height of 6.0\,m, are spaced 4.9\,m apart in columns, and 6.6\,m apart between rows.  
The receiver has a single antenna and reports a single-valued signal strength, RSSI, in dBm for each transmitter location.  
The transmitter is positioned at the right side of 64 trees in an $8 \times 8$ grid, with the average RSSI from 10 measurements at each location taken as the ground truth.  
Since this data alone is insufficient to train models, we augment it by adding Gaussian noise with a mean of zero and a standard deviation of 0.1\,m to each transmitter location.  
The augmented locations are assigned the same RSSI values as the original transmitter locations, resulting in a total of 6,400 transmitter location measurements.  
The dataset is randomly split 70\% training and 30\% testing.  



\underline{\textsc{iii) Metrics:}}
RSSI prediction error is the absolute difference between the model's predictions and the ground truth.


\begin{figure}[t]
	\begin{minipage}[t]{0.48\linewidth} 
    \includegraphics[width=\textwidth]{figs/mse_gs_ble.pdf}
\caption{RSSI prediction errors on the BLE dataset.}
        \label{fig_amplitude_phase}
	\end{minipage}
 \hspace{0.02in}
 	\begin{minipage}[t]{0.48\linewidth} 
    \includegraphics[width=\textwidth]{figs/localization.pdf}
        \caption{Fingerprinting-based localization error.}
        \label{fig_amplitude_2s}
	\end{minipage}
 \Description[]{}
\end{figure}


\begin{figure*}[tp]
	\subfigure[Ground truth]{
\includegraphics[width=.24\textwidth]{figs/rssi_gt.pdf}}
	\subfigure[\ourSystem]{
\includegraphics[width=.24\textwidth]{figs/rssi_gs.pdf}}
	\subfigure[\nerft]{
\includegraphics[width=.24\textwidth]{figs/rssi_nerf.pdf}}
	\subfigure[RSSI prediction error]{
\includegraphics[width=.24\textwidth]{figs/rssi_lora.pdf}}
\caption{RSSI prediction from three methods (\ourSystem uses only 30\% of the training data; others use 100\%).}
	\label{fig_case_study_esp}
  \Description[]{}
  \vspace{-10pt}
\end{figure*}

\underline{\textsc{iv) Baselines:}}
We compare \ourSystem with \nerft~\cite{zhao2023nerf}.  
We also include FLog~\cite{yang2023link}, a state-of-the-art empirical model for orchards built on the Log-Normal~\cite{rappaport1996wireless} model.  




\subsubsection{Overall Performance}  
\textbf{Setting:}
To highlight \ourSystem's data efficiency, it is trained on only 30\% of the full training data, \ie $6400 * 70\% * 30\% = 1344$.  
In contrast, baselines use the full training data, \ie $6400 * 70\% = 4480$.  
All methods are evaluated on the same testing data.






\textbf{Result:}  
Figure~\ref{fig_case_study_esp} presents heatmaps of the ground truth and the generated RSSI from \nerft and \ourSystem, along with the CDF of RSSI prediction errors for all models.  
From Figures~\ref{fig_case_study_esp}(a-c), \ourSystem and \nerft achieve similar generation accuracy, as verified in Figure~\ref{fig_case_study_esp}(d).  
However, \ourSystem uses at least $3\times$ less training data compared to \nerft, demonstrating its practical data efficiency.  
Figure~\ref{fig_case_study_esp}(d) also shows that both \ourSystem and \nerft outperform the FLog model.  
This is because FLog oversimplifies complex RF signal propagation by relying on a single path loss exponent (PLE)~\cite{rappaport1996wireless}, whereas \ourSystem and \nerft compute more fine-grained effects across the entire space through their scene representation modules.  




\subsubsection{Gateway Coverage Estimation Application}  

\textbf{Setting:}
Generated RSSI data can estimate a receiver's coverage by calculating the Bit Error Rate~(BER), using RSSI values and the Spreading Factor~(SF) as inputs to the existing BER calculation equation~\cite{elshabrawy2018closed}.  
A BER below $10^{-4}$ indicates sufficient reliability, with a packet delivery ratio exceeding 90\%~\cite{kasi2020towards}. 


\textbf{Result:}  
Both theoretically calculated BER values, derived from \ourSystem's generated data, and measured BER values are effectively zero across all ranges, confirming reliable coverage.  
Leveraging \ourSystem's high fidelity, detailed coverage maps can be generated.  
Given the orchard's uniform layout, similar coverage maps can be assumed for all receivers, enabling strategic placement to ensure full coverage, minimize blind spots, and optimize resource allocation.  
These findings provide confidence in transceiver deployment, ensuring reliable data transmission for applications such as environmental monitoring in precision agriculture.  



\subsubsection{Practical Benefits}  
\ourSystem provides several practical advantages, including cost savings by reducing the number of gateways~(up to $20,000$ per gateway~\cite{cell_tower_cost}), high-fidelity coverage maps to eliminate blind spots, and optimized resource allocation for maximum coverage with minimal hardware.  
It enables scalable deployment through customizable maps for vast orchards.
Reliable connectivity supports precision agriculture technologies like soil monitoring and autonomous machinery, enhancing efficiency and sustainability while lowering energy consumption and environmental impact.




