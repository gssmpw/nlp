\section{Related Work}
\label{sec_relatedWork}


\textbf{i) NeRF-Based:}
\nerft~\cite{zhao2023nerf} develops voxel-based scene representations and ray tracing algorithms to achieve state-of-the-art modeling accuracy, outperforming conventional methods such as simulations~\cite{wirelessinsite_web, orekondy2022winert, RayTracingToolbox}, empirical models~\cite{rappaport1996wireless, parsons2012mobile, hata1980empirical}, and ML models~\cite{parralejo2021comparative, liu2021fire, malmirchegini2012spatial}.  
However, it faces a scalability-fidelity tradeoff.  
\ourSystem addresses this tradeoff through novel designs.
\textit{First,} \ourSystem employs 3D Gaussians with varying positions, shapes, complex-valued emission and attenuation to construct a more flexible and expressive scene representation.
This enables \ourSystem to focus on object regions while avoiding large empty spaces.
\textit{Second,} each Gaussian explicitly stores its attributes, eliminating extensive large MLP queries.
\textit{Third,} the irregular and discrete placement of Gaussians renders voxel-based ray tracing ineffective. 
We address this with orthographic projection-based splatting to identify ray-Gaussian intersections, followed by a point-based complex-valued blending algorithm to aggregate emission, attenuation, and Gaussian shape contributions per ray.


NeWRF~\cite{lunewrf} shares the same methodology as \nerft but differs in that it restricts rays to directions identified by Direction-of-Arrival (DoA) data, rather than considering all \(360*90\) directions.  
However, collecting DoA data requires deploying an additional antenna array at each position, increasing the complexity and burden of data collection.
This violates our three scalable principles.  
Consequently, we treat \nerft and NeWRF as equivalent methods.