\begin{abstract}
Trajectory prediction and planning are fundamental components for autonomous vehicles to navigate safely and efficiently in dynamic environments.
Traditionally, these components have often been treated as separate modules, limiting the ability to perform interactive planning and leading to computational inefficiency in multi-agent scenarios.
In this paper, we present a novel unified and data-driven framework that integrates prediction and planning with a single consistency model.
Trained on real-world human driving datasets, 
our consistency model generates samples from high-dimensional, multimodal joint trajectory distributions of the ego and multiple surrounding agents, enabling end-to-end predictive planning.
It effectively produces interactive behaviors, such as proactive nudging and yielding to ensure both safe and efficient interactions with other road users.
To incorporate additional planning constraints on the ego vehicle, we propose an alternating direction method for multi-objective guidance in online guided sampling.
Compared to diffusion models, our consistency model achieves better performance with fewer sampling steps, making it more suitable for real-time deployment.
Experimental results on Waymo Open Motion Dataset (WOMD) demonstrate our method's superiority in trajectory quality, constraint satisfaction, and interactive behavior compared to various existing approaches.
% \di{Trajectory quality is shown with respect to ground truth, dynamic minimization, and constraint satisfaction: this is good. Computational Efficiency is shown, but we appear to do worse that most. It's good to include the analysis, but we shouldn't be claiming superiority here if we do worse or similar. For interactivity we have Fig 4 as an example, are there quantitative ways to show interactivity? }

% \ajnote{
% % major contribution:
% % \begin{itemize}
% %     \item a joint prediction/planning framework, model joint distribution, facilitate interactive planning
% %     \item a consistency model for trajectory generation, good performance with high efficiency (a few step)
% %     \item Multi-objective guided sampling, alternating direction, planning constraints
% % \end{itemize}
% page limit: RAL: 6 pages, RSS: no limits, typically around 8 pages (not clear about reference)
% }

\end{abstract}