\section{Problem Formulation} \label{sec: problem formulation}

We aim to plan the trajectory of the ego vehicle with a planning goal, simultaneously predicting the interactive behaviors of others. We assume 
the trajectory histories of traffic agents and map information are given and known.
% \replaced{We aim to plan the trajectory of the ego vehicle with a planning goal, simultaneously predicting the interactive behaviors of others. We assume 
% the trajectory histories of traffic agents and map information are given and known.}{In this paper, we aim to simultaneously perform trajectory planning for the ego vehicles and trajectory prediction for the surrounding agents, given their trajectory history, the ego agent's goal, and the map information.} 
% \bae{Perfect observation may imply zero noise, which I guess is not the case -- since the data itself may include noises. If that's the case, you may rephrase, e.g., ``we assume the perception history and map information are given and known...''.}

% \deleted{We assume perfect observation of the environment.}
At the current timestep $t_0 \in \mathbb{R}$, the ego agent's state history over the past $H_1 \in \mathbb{N}$ timesteps including the current state 
% \deleted{and the states over the past $H_1$ timesteps}
is given as $S_e^{\text{hist}} = (s_e^{t_0}, s_e^{t_0-1}, ..., s_e^{t_0-H_1}) \in \mathbb{R}^{(H_1 + 1) \times d_s}$, where $d_s$ is the dimension of the states, e.g. position, velocity, etc.
Let $\mathcal{O}$ denote the set of 
other agents of all types such as
% \replaced{every type of}{all other} agents \deleted{in the scenario} \replaced{such as}{including} 
vehicles, pedestrians, and cyclists.
Their history states are given for the current timestep $t_0$ and the past $H_1$ timesteps, denoted by $S_{\mathcal{O}}^{\text{hist}} = (s_{o}^{t_0}, s_{o}^{t_0-1}, ..., s_{o}^{t_0-H_1})_{o \in \mathcal{O}} \in \mathbb{R}^{N_{\mathcal{O}} \times (H_1 + 1) \times d_s}$, where $N_{\mathcal{O}} = |\mathcal{O}|$.
% \di{The ego agent's goal position is notated $s_e^g \in \mathbb{R}^{d_s}$.} 
The map information is denoted by $M \in \mathbb{R}^{N_m \times c \times d_m}$, which includes $N_m \in \mathbb{N}$ polylines in the scenario with $c$ points on each polyline and $d_m$ attributes per point.
We use fixed dimensions $N_{\mathcal{O}}$ and $N_m$ for agents and map polylines respectively, which are chosen large enough to accommodate all scenarios with padding and masks applied to invalid entries.
We assume the ego agent's goal position is given by some high-level planner and is notated $s_e^g \in \mathbb{R}^{d_s}$.
% \bae{Just curiosity, would separating sets for each type of agents make more sense, instead of merging all types of agents into one history set?}

Let $\mathcal{O}_s \subset \mathcal{O}$ denote the subset of surrounding agents that may interact with the ego agents.
Given the information $(S_e^{\text{hist}}, S_{\mathcal{O}}^{\text{hist}}, M, s_e^g)$, for the future $H_2 \in \mathbb{N}$ timesteps, we aim to plan the ego agent's trajectory $S_e^{\text{future}} = (s_e^{t_0+1}, s_e^{t_0+2}, ..., s_e^{t_0+H_2}) \in \mathbb{R}^{H_2 \times d_s}$, and predict the trajectories of surrounding agents $S_{\mathcal{O}_s}^{\text{future}} = (s_{o}^{t_0+1}, s_{o}^{t_0+2}, ..., s_{o}^{t_0+H_2})_{o \in \mathcal{O}_s} \in \mathbb{R}^{N_{\mathcal{O}_s} \times (H_2) \times d_s}$, where $N_{\mathcal{O}_s} = |\mathcal{O}_s|$.
This joint prediction and planning naturally captures interactive behaviors between agents.
% and expect interactive behaviors are considered in both prediction and planning.
Additionally, the planned trajectory for the ego vehicle should satisfy $N_c \in \mathbb{N}$ constraints $c_i( S_e^{\text{future}}) \leq 0, i=1,2,...,N_c$, which may include goal reaching, control limits, or other requirements.

