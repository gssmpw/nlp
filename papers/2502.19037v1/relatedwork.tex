\section{Related Work}
%\noindent\textbf{Polyp Segmentation.}
Early polyp segmentation methods used feature-based machine learning models \cite{mamonov2014automated,tischendorf2010computer,zhou2021review}, extracted features from data and classified polyp using algorithms such as linear classifiers, k-Nearest Neighbors, and Support Vector Machines. 
However, these models can hardly capture the global context information and are not robust to complex scenarios. 
With the development of deep learning, several CNN-based polyp detection and segmentation models have been proposed \cite{tajbakhsh2015automated,zhang2018polyp}, these models have higher sensitivity compared to previous polyp detection models.
To locate the polyp boundaries with more precision, Fully Convolutional Networks (FCN) were applied to polyp segmentation \cite{akbari2018polyp,brandao2017fully}. 
U-Net is a classic medical image segmentation network, whose core architecture adopts a symmetric encoder-decoder structure and fuses low-level details with high-level semantic information through skip connections, achieving high-precision segmentation under limited data conditions \cite{ronneberger2015u}.
U-Net++ \cite{zhou2018unet++} and ResUNet++ \cite{jha2019resunet++} are improvements built upon U-Net and have been used for polyp segmentation, achieving good performance. 
However, these methods focus on segmenting the entire region of the polyp and neglect the boundary constraints of the region \cite{fan2020pranet}.
%Thus, we propose PolypFLow, which enhances the model's ability to perceive polyp boundaries and improves its interpretability.

%\noindent\textbf{Flow Matching.}
Flow Matching is a recent framework in generative modeling that has achieved state-of-the-art performance across various domains, including image, video, audio, and biological structures~\cite{lipman2024flow,yun2025flowhigh,lipman2022flow}. Weinzaepfel et al.~\cite{weinzaepfel2013deepflow} proposed a large-displacement optical flow computation method named DeepFlow, which addresses the performance bottlenecks of traditional approaches in scenarios with rapid motion and large displacements by integrating the deep matching algorithm into a variational optical flow framework. Shi et al.~\cite{shi2023flowformer++} introduced flow matching into the pre-training process for optical flow estimation, offering inspiration for the methodology presented in this work. Itai et al.~\cite{gat2025discrete} extended the applicability of flow matching by adapting its principles to discrete sequence data. Based on the advantages of the interpretability and efficient calculation of flow matching, we apply it to the polyp segmentation task. 

\begin{figure}[t]
    \centering
    \includegraphics[width=1\linewidth]{fig/model.png}
    \caption{Overview of PolypFlow. (a) The overall training and inference process of PolypFlow based on flow matching. (b) The ordinary differential equation trajectory begins from a data-dependent prior distribution. (c) Vector field based on Self-attention and DCT. It is worth noting that our vector field is a core step in feature extraction, focusing on local features (convolution), global features (Self-attention), and the frequency domain (DCT).}
    \label{fig:model}
\end{figure}