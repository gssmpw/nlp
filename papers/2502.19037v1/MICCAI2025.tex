% This is a modified version of Springer's LNCS template suitable for anonymized MICCAI 2025 main conference submissions. 
% Original file: samplepaper.tex, a sample chapter demonstrating the LLNCS macro package for Springer Computer Science proceedings; Version 2.21 of 2022/01/12

\documentclass[runningheads]{llncs}
%
\usepackage[T1]{fontenc}
\usepackage{amsmath}
\usepackage{booktabs}
\usepackage{multirow}
\usepackage{amsmath}
\usepackage{graphicx}
\usepackage[table]{xcolor}
% T1 fonts will be used to generate the final print and online PDFs,
% so please use T1 fonts in your manuscript whenever possible.
% Other font encodings may result in incorrect characters.
%
\usepackage{graphicx,verbatim}
\usepackage{amssymb}
\usepackage[pagebackref=true,breaklinks=true,letterpaper=true,colorlinks,citecolor=blue,linkcolor=blue,bookmarks=false]{hyperref}
% Used for displaying a sample figure. If possible, figure files should
% be included in EPS format.
%
% If you use the hyperref package, please uncomment the following two lines
% to display URLs in blue roman font according to Springer's eBook style:
%\usepackage{color}
%\renewcommand\UrlFont{\color{blue}\rmfamily}
%\urlstyle{rm}
%
\begin{document}
%
\title{PolypFlow: Reinforcing Polyp Segmentation with Flow-Driven Dynamics}
%
\begin{comment}  %% Removed for anonymized MICCAI 2025 submission
\author{First Author\inst{1}\orcidID{0000-1111-2222-3333} \and
Second Author\inst{2,3}\orcidID{1111-2222-3333-4444} \and
Third Author\inst{3}\orcidID{2222--3333-4444-5555}}
%
\authorrunning{F. Author et al.}
% First names are abbreviated in the running head.
% If there are more than two authors, 'et al.' is used.
%
\institute{Princeton University, Princeton NJ 08544, USA \and
Springer Heidelberg, Tiergartenstr. 17, 69121 Heidelberg, Germany
\email{lncs@springer.com}\\
\url{http://www.springer.com/gp/computer-science/lncs} \and
ABC Institute, Rupert-Karls-University Heidelberg, Heidelberg, Germany\\
\email{\{abc,lncs\}@uni-heidelberg.de}}

\end{comment}

\author{Pu Wang\textsuperscript{1}, Huaizhi Ma\textsuperscript{2}, Zhihua Zhang\textsuperscript{1}, Zhuoran Zheng\textsuperscript{3}\thanks{Corresponding author: Zhuoran Zheng (zhengzr@njust.edu.cn)}}
\institute{
  $^1$\small Shandong University, School of mathematics, Jinnan 250100 , China \\
  $^2$\small  School of Artificial Intelligence Zaozhuang University, Zaozhuang 277100, China\\
  $^3$\small Sun Yat-sen University, School of cyber science and technology, 518000, China \\
  \email{202411943@mail.sdu.edu.cn, mahuaizhi@uzz.edu.cn, zhangzhihua@sdu.edu.cn, zhengzr@njust.edu.cn} }


% \author{Anonymized Authors}  %% Added for anonymized MICCAI 2025 submission
% \authorrunning{Anonymized Author et al.}
% \institute{Anonymized Affiliations \\
%     \email{email@anonymized.com}}

    
\maketitle              % typeset the header of the contribution
%
\begin{abstract}%医学
Accurate polyp segmentation remains challenging due to irregular lesion morphologies, ambiguous boundaries, and heterogeneous imaging conditions. While U-Net variants excel at local feature fusion, they often lack explicit mechanisms to model the dynamic evolution of segmentation confidence under uncertainty. Inspired by the interpretable nature of flow-based models, we present \textbf{PolypFLow}, a flow-matching enhanced architecture that injects physics-inspired optimization dynamics into segmentation refinement. Unlike conventional cascaded networks, our framework solves an ordinary differential equation (ODE) to progressively align coarse initial predictions with ground truth masks through learned velocity fields. This trajectory-based refinement offers two key advantages: 1) Interpretable Optimization: Intermediate flow steps visualize how the model corrects under-segmented regions and sharpens boundaries at each ODE-solver iteration, demystifying the ``black-box" refinement process; 2) Boundary-Aware Robustness: The flow dynamics explicitly model gradient directions along polyp edges, enhancing resilience to low-contrast regions and motion artifacts. %Evaluated on Kvasir-SEG, PolypFLow achieves state-of-the-art Dice of 0.923 (3.2\% gain over U-Net) while maintaining stable performance across varying illumination and occlusion scenarios.
Numerous experimental results show that PolypFLow achieves a state-of-the-art  while maintaining consistent performance in different lighting scenarios.%有一个比较
%Ablation studies verify that the flow-matching module contributes 68\% of boundary-related error reduction compared to skip-connection baselines. 
%By unifying deep learning with differential equation-driven refinement, our work provides a new paradigm for trustworthy medical image analysis—one where accuracy and interpretability evolve synergistically.

\keywords{Polyp Segmentation  \and Flow Matching.}
% Authors must provide keywords and are not allowed to remove this Keyword section.

\end{abstract}
%
%
%

\section{Introduction}
%第一段的背景介绍太长浓缩到4-5行句子,直扑主题,息肉分割不是一个冷门课题
The precise segmentation of the polyp region is considered reliable measures to reduce CRC incidence and mortality~\cite{haggar2009colorectal,winawer1993prevention}. However, polyps exhibit high heterogeneity in size, color, texture, and boundary clarity, leading to a high rate of missed diagnoses~\cite{misawa2018artificial}. Developing high-precision polyp segmentation algorithms is of significant clinical importance for improving early CRC screening efficiency.
% \textcolor{red}{Colorectal cancer (CRC) is the third most common cancer globally, and early screening and intervention are crucial for reducing mortality rates. Studies have shown that approximately 90\% of CRC cases originate from the malignant transformation of colorectal polyps, especially adenomatous polyps \cite{silva2014toward}. 
% %
% Colonoscopy and complete removal of tumor lesions (such as adenomas) are considered reliable measures to reduce CRC incidence and mortality \cite{haggar2009colorectal,winawer1993prevention}. A large study demonstrated that CRC mortality decreased by approximately 70\% after colonoscopy screening \cite{nishihara2013long}. As the primary screening method, one of the core tasks of colonoscopy is the precise segmentation of the polyp region, which provides diagnostic support for doctors and guides surgical resection. However, polyps exhibit high heterogeneity in size, color, texture, and boundary clarity, and they often have low contrast with the surrounding mucosa, leading to a high rate of missed diagnoses (approximately 20\%-30\%) and subjectivity in manual detection \cite{misawa2018artificial}. Therefore, developing automatic, high-precision polyp segmentation algorithms is of significant clinical importance for improving early CRC screening efficiency and reducing unnecessary polyp removal risks.}

Currently, polyp segmentation algorithms can generally be classified into two categories. One class is low-parameter models based on convolutional~\cite{akbari2018polyp,ronneberger2015u} or Transformer networks~\cite{9845389}, they can be trained from scratch on a single 3090RTX GPU. These methods only use the inherent parameters for one or non-progressive continuous execution of inference, resulting in the problem of imprecise extraction of polyp contour information and blurred edge judgment.
%For example, U-Net$++$ and ResUNet$++$~\cite{akbari2018polyp,ronneberger2015u} demonstrate effective handling of polyp body regions, yet they fail to adequately model the interactions between polyp regions and boundaries, resulting in incomplete predictions or over-smoothed segmentation outcomes. Colonformer~\cite{9845389} employs a hybrid Transformer encoder as its backbone architecture, effectively balancing local and global feature extraction in polyp regions. However, the image patch method is non-adaptive, and the edge information capture of polyps may be inaccurate. 
Another approach is a polyp segmentation method based on the Segment Anything Model (SAM) for fine-tuning large models, such as Polyp-sam~\cite{li2024polyp} and Ployp-sam++~\cite{biswas2023polyp}. 
Since the pre-trained SAM has prior knowledge of the instance objects, it can be easily transferred to the polyp segmentation task and is more robust.
%Recently, SAM was introduced as a novel solution tailored for natural image segmentation tasks~\cite{kirillov2023segment}. Since it comes with rich prior knowledge of image understanding, a large number of image understanding tasks, such as~\cite{zhou2023can}. Notably, in the context of the polyp segmentation task, particularly when dealing with tiny lesion regions, the approach put forward by Ma et al. involves conducting a comprehensive fine-tuning of the original SAM (Segment Anything Model) on medical data~\cite{ma2024segment}. 
However, the method based on SAM has two drawbacks: 1) lack of interpretability, despite having a large number of instance level priors, it is difficult to understand its principle; 2) The number of parameters is large and struggle to deployed on edge devices. In addition, fine-tuning based on SAM also introduces uncertainty.

% Current polyp segmentation algorithms can generally be classified into two categories, one is based on convolution and Transformer small models, such as U-Net$++$~\cite{akbari2018polyp}, ResUNet$++$~\cite{ronneberger2015u}, Colonformer~\cite{9845389}.  However, these models often inadequately model the interplay between polyp regions and boundaries, leading to fragmented or over-smooth predictions. The attention mechanism in Transformer has a high computational cost and requires a large number of parameters and computing resources. Another approach is a polyp segmentation method based on SAM for fine-tuning large models, such as Polyp-sam~\cite{wei2021shallow}, \textcolor{red}{Shallow Attention Network for Polyp Segmentation}~\cite{wei2021shallow}. However, such algorithms are limited by SAM's reliance on user prompts and are difficult to become fully end-to-end polyp segmentation models using SAM.

% Early approaches relied on handcrafted features, such as vascular patterns or texture descriptors, combined with machine learning classifiers \cite{tischendorf2010computer}. While these methods achieved moderate accuracy, their reliance on manual feature engineering limited generalizability \cite{sikkandar2024utilizing}. The advent of fully convolutional networks (FCNs) and U-Net architectures revolutionized the field by enabling end-to-end pixel-level predictions \cite{akbari2018polyp,ronneberger2015u}. Subsequent improvements, including U-Net++ and ResUNet++, introduced nested skip connections and advanced backbones to enhance multi-scale feature fusion \cite{jha2019resunet++,zhou2018unet++}. However, these models often inadequately modeled the interplay between polyp regions and boundaries, leading to fragmented or oversmoothed predictions.

% Recent innovations focus on attention mechanisms and feature refinement. For instance, PraNet [4] proposed a parallel reverse attention module to iteratively refine boundaries using high-level semantic guidance, achieving a Dice score of 0.898 on the Kvasir dataset. Similarly, MSNet [11] replaced traditional feature concatenation with subtraction units to reduce redundancy, while SANet [5] leveraged shallow attention and color augmentation to improve small polyp segmentation. Despite progress, balancing accuracy, computational efficiency, and robustness remains an open challenge. 

In order to solve the obstacles existing in the above problems, we propose the PolypFLow framework, which integrates U-Net with Flow Matching Equations (FME)~\cite{lipman2022flow}. Flow matching technology has prominent advantages in medical image analysis, its solution process can intuitively show how the model corrects under-segmented regions, featuring good interpretability. Further, PolypFLow introduces the discrete cosine transform (DCT) and the self-attention mechanism. DCT provides precise features for flow matching, and the self-attention mechanism focuses on global information, avoids local limitations, and significantly improves the segmentation accuracy. 
%On one hand, flow matching is interpretable, as intermediate flow steps can visualize how the model corrects under-segmented regions, demystifying the model's optimization process. On the other hand, compared with traditional cascaded networks, it calculates more efficiently by solving ordinary differential equations and using learned velocity fields to adjust prediction results. 
Our contributions are summarized as follows: 
\begin{enumerate}
    \item To the best of our knowledge, this paper is the first to introduce flow matching in the task of polyp image segmentation. It has good interpretability and effectively avoids the problems of inaccurate single inference of fixed-parameter models and the difficulty of deploying large-parameter-number models on edge devices.
    \item We develop a vector field that introduces self-attention and DCT, which can effectively capture the frequency domain and global information of the image. Extensive experiments on five benchmark datasets reveal that Polyp outperforms state-of-the-art methods.
\end{enumerate}
%This approach enhances segmentation accuracy and improves edge detection capabilities by incorporating physics-inspired optimized dynamics.
% Therefore, we propose the FME-UNet model, which combines the U-Net with Flow Matching Equation(FME)~\cite{lipman2022flow},  enhances segmentation accuracy and interpretability by introducing physics-inspired optimization dynamics.
%\textcolor{red}{As shown in Figure 1}, unlike progressive polyp segmentation approaches, PolypFLow demonstrates enhanced interpretability with smooth and reversible feature evolution paths.
% The contributions of this research are as follows.
% \begin{enumerate}
%     \item Interpretable Optimization: The proposed FME-UNet uses flow-based generative models to enhance segmentation refinement by solving an ordinary differential equation (ODE). This allows for intermediate flow steps that visualize how the model progressively corrects under-segmented regions and sharpens boundaries, offering interpretability to the "black-box" segmentation process.
%     \item Boundary-Aware Robustness: The model introduces flow dynamics that explicitly model gradient directions along polyp edges. This enhances the network's robustness, particularly in handling low-contrast regions and motion artifacts, which are typically challenging for traditional segmentation models.
%     \item ODE-Driven Segmentation Refinement: Unlike conventional cascaded networks, FME-UNet employs ODE-based trajectory refinement to align coarse initial predictions with ground truth. This unique approach significantly improves segmentation accuracy, achieving a state-of-the-art Dice score of 0.923 and showing stable performance across diverse imaging conditions like illumination and occlusion.
% \end{enumerate}

\section{Related Work}
%\noindent\textbf{Polyp Segmentation.}
Early polyp segmentation methods used feature-based machine learning models \cite{mamonov2014automated,tischendorf2010computer,zhou2021review}, extracted features from data and classified polyp using algorithms such as linear classifiers, k-Nearest Neighbors, and Support Vector Machines. 
However, these models can hardly capture the global context information and are not robust to complex scenarios. 
With the development of deep learning, several CNN-based polyp detection and segmentation models have been proposed \cite{tajbakhsh2015automated,zhang2018polyp}, these models have higher sensitivity compared to previous polyp detection models.
To locate the polyp boundaries with more precision, Fully Convolutional Networks (FCN) were applied to polyp segmentation \cite{akbari2018polyp,brandao2017fully}. 
U-Net is a classic medical image segmentation network, whose core architecture adopts a symmetric encoder-decoder structure and fuses low-level details with high-level semantic information through skip connections, achieving high-precision segmentation under limited data conditions \cite{ronneberger2015u}.
U-Net++ \cite{zhou2018unet++} and ResUNet++ \cite{jha2019resunet++} are improvements built upon U-Net and have been used for polyp segmentation, achieving good performance. 
However, these methods focus on segmenting the entire region of the polyp and neglect the boundary constraints of the region \cite{fan2020pranet}.
%Thus, we propose PolypFLow, which enhances the model's ability to perceive polyp boundaries and improves its interpretability.

%\noindent\textbf{Flow Matching.}
Flow Matching is a recent framework in generative modeling that has achieved state-of-the-art performance across various domains, including image, video, audio, and biological structures~\cite{lipman2024flow,yun2025flowhigh,lipman2022flow}. Weinzaepfel et al.~\cite{weinzaepfel2013deepflow} proposed a large-displacement optical flow computation method named DeepFlow, which addresses the performance bottlenecks of traditional approaches in scenarios with rapid motion and large displacements by integrating the deep matching algorithm into a variational optical flow framework. Shi et al.~\cite{shi2023flowformer++} introduced flow matching into the pre-training process for optical flow estimation, offering inspiration for the methodology presented in this work. Itai et al.~\cite{gat2025discrete} extended the applicability of flow matching by adapting its principles to discrete sequence data. Based on the advantages of the interpretability and efficient calculation of flow matching, we apply it to the polyp segmentation task. 

\begin{figure}[t]
    \centering
    \includegraphics[width=1\linewidth]{fig/model.png}
    \caption{Overview of PolypFlow. (a) The overall training and inference process of PolypFlow based on flow matching. (b) The ordinary differential equation trajectory begins from a data-dependent prior distribution. (c) Vector field based on Self-attention and DCT. It is worth noting that our vector field is a core step in feature extraction, focusing on local features (convolution), global features (Self-attention), and the frequency domain (DCT).}
    \label{fig:model}
\end{figure}
\section{Method}
We construct PolypFLow by combining U-Net's  feature extraction capability with the  optimization dynamics of flow matching in Fig~\ref{fig:model}, more details are presented in the following subsections.
\subsection{Preliminaries}
Given the availability of empirical observations of the data distribution \(x_0\sim p_0\) and the noise distribution \(x_1\sim p_1\) (where \(p_1\) is typically Gaussian), the objective of the flow matching framework is to infer a coupling \(\pi(p_0,p_1)\) that characterizes the transformation between these two distributions. Continuous Normalizing Flows (CNFs)~\cite{chen2018neural} define a time-dependent probability density function, known as the probability density path \(p_t:[0,1]\times\mathbb{R}^d\rightarrow\mathbb{R}_{>0}\), with \(\int p_t(x)dx = 1\) and \(t\in[0,1]\). The flow, a time-dependent diffeomorphic mapping \(\phi_t:[0,1]\times\mathbb{R}^d\rightarrow\mathbb{R}^d\), which gives rise to the path \(p_t\), is propelled by a time-dependent vector field \(v_t:[0,1]\times\mathbb{R}^d\rightarrow\mathbb{R}^d\). This vector field is defined through the following ordinary differential equation (ODE):
\begin{equation}
   \frac{d}{dt}\phi_t(x)=v_t(\phi_t(x)),\quad\phi_0(x)=x\label{eq:ODE}
\end{equation}
Flow Matching seeks to optimize the simple regression objective:
\begin{equation}
  \mathcal{L}_{FM}(\theta) = \mathbb{E}_{t,p_t(x)}\left\|v_t(x;\theta)-u_t(x)\right\|^2\label{eq1}  
\end{equation}
In the above, $v_t(x;\theta)$ represents the parametric vector field for the CNFs, $u_t^\theta$ is the velocity with learnable parameter $\theta$, and \(u_t(x)\) is a vector field that generates a probability path \(p_t\). Euler's method is a commonly used ODE solution approach:
\begin{equation}
\mathbf{x}_{n + 1} = \mathbf{x}_n + v_t(\mathbf{x}_n)\Delta t\label{eq:Euler}
\end{equation}

%Although Eq.~\ref{eq1} is the ideal objective function for optimization, the lack of knowledge about \((p_t, u_t)\) renders this computation intractable. \cite{lipman2022flow} proposed a tractable way to optimize. First, they defined conditional probability paths and vector fields. Marginalizing over \(q_0(x_0)\) and \(q_1(x_1)\) yields \(p_t(x)\) and \(u_t(x)\). For generative modeling, \(q_0(x_0)\) is a simple noise distribution, enabling a one-sided construction:
% \begin{equation}
% p_t(x)=\int p_t(x|x_1)q_1(x_1)dx_1 
% \end{equation}

% \begin{equation}
% u_t(x)=\int u_t(x|x_1)\frac{p_t(x|x_1)q_1(x_1)}{p_t(x)}dx_1
% \end{equation}
% The conditional probability path is chosen such that \(p_{t = 0}(x|x_1)=q_0(x)\) and \(p_{t = 1}(x|x_1)=\delta(x - x_1)\), where \(\delta(x - a)\) is a Dirac mass at \(a\in\mathbb{R}^d\). Thus, \(p_t(x|x_1)\) satisfies both marginal constraints. \cite{tong2023improving} further generalized Eq.~\ref{eq1} as:
% \begin{equation}
% \mathcal{L}_{CFM}(\theta)=\mathbb{E}_{t\sim U[0,1],q(z),p_t(x|z)}\left\|u_t(x|z)-v_t(x;\theta)\right\|_2^2 
% \end{equation}
% where \(p_t\) and \(u_t\) depend on \(z\) sampled from \(q\). It was also shown that \(\mathcal{L}_{FM}(\theta)\) and \(\mathcal{L}_{CFM}(\theta)\) have the same gradients for \(\theta\) during model optimization. 
\subsection{U-Net: Structure and Output}
U-Net and its variations have demonstrated remarkable efficiency in medical image segmentation tasks. In the encoder part, for a given image $\mathbf{I} \in \mathbb{R}^{H \times W \times 3}$, semantic information of images is gradually extracted through convolution to obtain a feature map:
\begin{equation}
\mathbf{E_1} = \mathrm{ReLU}(\mathrm{BN}(\mathrm{Conv}(\mathbf{I})))
\end{equation}
\begin{equation}
\mathbf{E_l} = \mathrm{ReLU}(\mathrm{BN}(\mathrm{Conv}(\mathbf{E_{l - 1}})))
\end{equation}
 where $\mathbf{E_l}$ is the $l-th \in \{1, 2, 3, 4\}$ layer encoder output, $\mathrm{ReLU}(\cdot)$ represents $\mathrm{ReLU}$ function, $\mathrm{BN}(\cdot)$ stands for batch normalization, $\mathrm{Conv}$ is a convolution. 
 In the decoder part, by transposing convolution and skipping join, encoder features $\mathbf{E_l}$ and upsampling features $\mathbf{D_{l + 1}}$ are spliced:
 \begin{equation}
    \mathbf{D_l} = \mathrm{ConvB}(\mathrm{Cat}(\mathrm{UpConv}(\mathbf{D_{l + 1}}), \mathbf{E_{4 - l}}))
 \end{equation}
 where $\mathrm{ConvB}(\cdot)$ denotes the convolution block and $\mathrm{Cat}(\cdot)$, $\mathrm{UpConv}(\cdot)$ are the feature concatenation operation and upsampling.
Here, we use a $1\times1$ convolution to get the model output.
So that low-level details and high-level semantic information are fused together to produce accurate segmentation results. 
%On this basis, we made key optimizations for the polyp segmentation task: 1) an extra convolution block with a residual connection to process skip connection features. 2) deep supervision: add auxiliary losses at each layer of the decoder to enhance boundary learning. 
% \begin{equation}
%     \mathcal{L}_{\text{seg}} = \sum_{l = 1}^{4} \lambda_l \cdot \text{Dice}(\mathbf{D_l}, Y)\label{eq:loss}
% \end{equation}
% where $\mathbf{D_l}$ is the output of the $l$-layer decoder, and $Y$, $\lambda_l$ are ground truth and weight coefficient.

\subsection{Flow Matching for Polyp Segmentation}
In the polyp segmentation task, to boost the model's ability to learn polyp features and segmentation accuracy, we propose a flow trajectories design method (vector field) that combines the mask and feature extractor.

%\noindent\textbf{Design of Flow Trajectories.}
%To guide the model to learn polyp features along effective paths, based on traditional flow trajectories in Eq.~\ref{eq:ODE} and combined with mask $m$ information~\cite{liu2022flow}, we have designed:
%\begin{equation}
%    z_t = (1 - t)\mathbf{X_0} + t \cdot \text{G}(\mathbf{X_0}, m) \epsilon
%\end{equation}
%\begin{equation}
%G(\mathbf{X}_0, m)=\mathbf{W}\cdot(\mathbf{E_i}\odot m)+b, i =1, 2, 3, 4
% \end{equation}
%where the weight matrix of the fully-connected layer is $\mathbf{W} \in \mathbb{R}^{1\times H \times W\times 3}$ and the bias is $b \in \mathbb{R}$, $\mathbf{X_0}$ is the initial prediction of the U-Net output, $t$ is the time step, and $\epsilon$ is the noise. $G(\mathbf{X_0}, m)$ is a generation function. It adjusts how noise $\epsilon$ is added according to the polyp's position in the mask, helping the model focus on polyp region feature evolution during the flow.   

\noindent\textbf{Design Vector Field.} The  vector field aims to capture the key features and relationships of polyps and their surrounding tissues:
\begin{equation}
\small
v_{\theta}(t, z_t, \mathbf{X}_{i}) = \underbrace{\text{Softmax}\left(\frac{QK^{T}}{\sqrt{d}}\right)V}_{\text{Self-attention weights}} \odot \left(\text{UNet}(z_t \oplus \mathbf{X}_{i} \oplus m) + \text{DCT}(\mathbf{X}_{i})\right)
\end{equation}
where $v_{\theta}$ is the vector field, representing the direction and degree of the model's attention to image features in different states. $t$ represents the time step, reflecting the stage changes of the model during training. $z_t$ is a latent variable that stores the information of the model's intermediate process. $\mathbf{X}$ is the input medical image, containing the visual information of polyps and surrounding tissues, $m$ is the mask of the polyp. %As a binary image, it marks the polyp region as 1 and the rest as 0, highlighting the polyp part and guiding the model to focus on it. $Q$ and $K$ are the query and key matrices after linear transformation, which are used to measure the correlation between different positions in the self-attention mechanism. $d$ is the dimension, and $\sqrt{d}$ is used to normalize the attention weights to ensure stable calculation.  
$\text{DCT}(\mathbf{X})$ is the discrete cosine transform of the image $\mathbf{X}$, introducing frequency-domain prior knowledge to assist the model in exploring the potential structure of the image. $\odot$ represents element-wise multiplication. Through this operation, the self-attention weights are integrated with the feature information to generate the final vector field.
%
It is worth noting that the output of U-Net is encoded into $Q$, $K$, and $V$ by a large-scale convolutional kernel for self-attention, and $m$ is a learnable parameter matrix.
%
% In the polyp segmentation task, the progressive segmentation algorithm relies on the black-box network to implicitly learn the boundary features, and requires a fixed number of iterations. However, flow matching dynamically adjusts computing resources through the ODE solver, models the optimization process as a continuous transformation from the initial prediction $P_{0}$ to the true mask $Y$. This transformation is governed by a velocity field $v_{\theta}$,  which is learned by the network. This allows the model to adaptively refine the prediction based on the local and global context. The evolution of the prediction $P(t)$ over time $t$ is described by the following ODE:
% \begin{equation}
%     \frac{\partial P(t)}{\partial t} = v_{\theta}(P(t), \mathbf{E_{\mathrm{1, 2, 3, 4}}})
% \end{equation}
% where $t \in [0, T]$, $v_{\theta}$ is the velocity field parameterized by the network, whose input includes the current segmentation prediction $P_{t}$ and the encoder multi-scale features $\{\mathbf{E_1},\mathbf{E_2}, \mathbf{E_3}, \mathbf{E_4}\}$.

% In this study, we design a flow matching module to learn the vector field $v_{\theta}$. This module consists of a series of convolutional layers, aiming to extract features from the input data and generate the corresponding vector field. For a given polyp images $\mathbf{X} \in \mathbb{R}^{H \times W \times 3}$, the velocity field $v_{\theta}$ is implemented as a three-layer convolutional network. The first convolutional layer use a convolutional layer with a kernel size of $3\times3$ and padding of $1$ to map the input $C=3$ channels to $H_c$ hidden channels, the second convolutional layer further process the hidden features $\mathbf{h}_1$ and obtain new hidden features $\mathbf{h}_2$, and output convolutional layer map the hidden features $\mathbf{h}_2$ back to $C=3$ channels to obtain the vector field $v(\theta)$. This architecture ensures that the velocity field can capture both local and global patterns in the segmentation prediction, enabling precise refinement of the initial output. This procedure can
% be written as:
% \begin{equation}
%     \mathbf{h}_1 = \mathrm{ReLU}(\mathbf{W}_1 * \mathbf{X} + \mathbf{b}_1)
% \end{equation}
% \begin{equation}
%     \mathbf{h}_2 = \mathrm{ReLU}(\mathbf{W}_2 * \mathbf{h_1} + \mathbf{b}_2)
% \end{equation}
% \begin{equation}
%     v(\theta) = \mathbf{W}_3 * \mathbf{h}_2 + \mathbf{b}_3
% \end{equation}
% where $\mathbf{W}_1 \in \mathbb{R}^{H_c\times C\times 3\times 3}$, $\mathbf{W}_2 \in \mathbb{R}^{H_c\times H_c\times 3\times 3}$, $\mathbf{W}_1 \in \mathbb{R}^{C\times H_c\times 3\times 3}$ are the convolutional kernel weight. $\mathbf{b}_1 \in \mathbb{R}^{H_c}$, $\mathbf{b}_2 \in \mathbb{R}^{H_c}$, $\mathbf{b}_3 \in \mathbb{R}^{C}$ are the bias term.

After obtaining the vector field $v_\theta$, we use Eq.~\ref{eq:Euler} to update the input polyp images $\mathbf{X}$. %见初始化
Through multiple iterations, we can gradually transform the input polyp images $\mathbf{X_0}$ to $\mathbf{X_N}$ that is closer to the target distribution. The number of integration steps $N$ is set to 10, s shown in Figure~\ref{fig:step}. Our loss function can be formulated as:
\begin{equation}
    \mathcal{L} = \mathcal{L}_{IoU}^{w} + \mathcal{L}_{BCE}^{w}
\end{equation}
where $\mathcal{L}_{IoU}^{w}$ and $\mathcal{L}_{BCE}^{w}$ represent the weighted intersection over union loss and weighted binary cross entropy loss~\cite{zheng2024polyp}. The loss function enables the model to focus on boundary regions while optimizing the overlap between predictions and ground truth labels in the segmentation task. 

\begin{figure}[h]
    \centering
    \includegraphics[width=1\textwidth]{fig/step_10.png}
    \caption{Visualize the results at each step.}
    \label{fig:step}
\end{figure}


% \subsection{Loss Function}
% The U-Net generates an initial segmentation prediction $P_0$, which is then refined by the Flow Matching (FM) module. The final output is obtained by applying a sigmoid activation to the refined prediction:
% \begin{equation}
%     P_{\text{final}} = \text{Sigmoid}(P(T))
% \end{equation}
% where $P(T)$ is the prediction after $T$ integration steps. This integration allows the model to leverage the strengths of both U-Net and FM, achieving state-of-the-art performance on polyp segmentation tasks.

% The overall loss function for the FM module combines the segmentation loss and a regularization term to ensure smoothness of the velocity field:
% \begin{equation}
%     \mathcal{L}_{\text{total}} = \mathcal{L}_{\text{seg}} + \beta \mathcal{L}_{\text{reg}}
% \end{equation}
% where $\mathcal{L}_{\text{reg}}$ penalizes large fluctuations in the velocity field to ensure smooth trajectories, $\mathcal{L}_{\text{seg}}$ is the binary cross-entropy loss in Eq.~\ref{eq:loss}.%between the final prediction $P(T)$ and the ground truth $Y$
\section{Experiment}
\begin{table}[t]
\centering
\caption{Quantitative results on Kvasir and ClinicDB datasets.}
\label{table:1}
\fontsize{15pt}{7pt}\selectfont
\setlength{\tabcolsep}{8pt}
\resizebox{\textwidth}{!}{
\begin{tabular}{ll|c|cccccc}
\toprule
\multicolumn{2}{c|}{Methods} & year & mDice & mIoU & $F_\beta^\omega$ & $S_\alpha$ & $E_\phi^{\text{max}}$ & MAE \\
\midrule
\multirow{6}{*}{\rotatebox{90}{Kvasir-SEG}} & U-Net~\cite{ronneberger2015u} & 2015 & 0.818 & 0.746 & 0.794 & 0.858 & 0.893 & 0.055 \\
\multirow{6}{*}{} & U-Net++~\cite{zhou2018unet++} & 2018 & 0.821 & 0.744 & 0.808 & 0.862 & 0.909 & 0.048 \\
\multirow{6}{*}{} & PraNet~\cite{fan2020pranet} & 2020 & 0.898 & 0.841 & 0.885 & 0.915 & 0.948 & 0.030 \\
\multirow{6}{*}{} & SANet~\cite{wei2021shallow} & 2021 & 0.904 & 0.847 & 0.892 & 0.915 & 0.953 & 0.032 \\
\multirow{6}{*}{} & MedSAM~\cite{ma2024segment} & 2024 & 0.862 & 0.795 & - & - & - & - \\
\multirow{6}{*}{} & $\text{M}^2$ixNet~\cite{zheng2024polyp} & 2024 & 0.929 & 0.881 & 0.929 & 0.935 & 0.971 & 0.014 \\
\multirow{6}{*}{} & CAFE-Net~\cite{liu2024cafe} & 2024 & 0.933 & 0.889 & 0.927 & 0.939 & 0.971 & 0.019 \\
\multirow{6}{*}{} & \textbf{PolypFlow(Ours)} & 2025 & \textbf{0.971} & \textbf{0.948} & \textbf{0.967} & \textbf{0.964} & \textbf{0.993} & \textbf{0.008} \\
\midrule
\midrule
\multirow{6}{*}{\rotatebox{90}{ClinicDB}} & U-Net~\cite{ronneberger2015u} & 2015 & 0.823 & 0.755 & 0.811 & 0.890 & 0.953 & 0.019\\
\multirow{6}{*}{} & U-Net++~\cite{zhou2018unet++} & 2018 & 0.749 & 0.729 & 0.785 & 0.873 & 0.931 & 0.022\\
\multirow{6}{*}{} & PraNet~\cite{fan2020pranet} & 2020 & 0.899 & 0.849 & 0.896 & 0.937 & 0.979 & 0.009\\
\multirow{6}{*}{} & SANet~\cite{wei2021shallow} & 2021 & 0.916 & 0.859 & 0.909 & 0.940 & 0.976 & 0.012 \\
\multirow{6}{*}{} & MedSAM~\cite{ma2024segment} & 2024 & 0.867 & 0.803 & - & - & - & - \\
\multirow{6}{*}{} & $\text{M}^2$ixNet~\cite{zheng2024polyp} & 2024 & 0.941 & 0.891 & 0.934 & 0.955 & 0.986 & 0.005 \\
\multirow{6}{*}{} & CAFE-Net~\cite{liu2024cafe} & 2024 & 0.943 & 0.899 & \textbf{0.941} & 0.957 & 0.986 & 0.009 \\
\multirow{6}{*}{} & \textbf{PolypFlow(Ours)} & 2025 & \textbf{0.951} & \textbf{0.900} & 0.929 & \textbf{0.961} & \textbf{0.987} & \textbf{0.004} \\
\bottomrule
\end{tabular}
}
\end{table}
%In this section, we conduct a comparison between our FME-UNet and existing methods, focusing on aspects such as learning ability, generalization capacity, complexity, and qualitative outcomes. 
\subsection{Datasets and Training Settings.}
We compare the proposed PolypFlow with previous state-of-the-art methods, such as U-Net~\cite{ronneberger2015u}, U-Net++~\cite{zhou2018unet++}, PraNet~\cite{fan2020pranet}, SANet~\cite{wei2021shallow}, MedSAM~\cite{ma2024segment}, CAFE-Net~\cite{liu2024cafe}, and $\text{M}^2$ixNet~\cite{zheng2024polyp}. To assess the performance of the proposed PolypFLow. To this end, we adhere to the prevalent experimental setups in~\cite{zheng2024polyp}. Five publicly-accessible benchmarks are adopted: Kvasir-SEG~\cite{jha2020kvasir}, ClinicDB~\cite{bernal2015wm}, ColonDB~\cite{tajbakhsh2015automated}, Endoscene~\cite{vazquez2017benchmark}, and ETIS~\cite{silva2014toward}. In line with previous work~\cite{zheng2024polyp}, we utilize six widely-used metrics mDice, mIoU, MAE, $F_{\beta}^{w}$, $S_{\alpha}$, and $E_{\xi}^{max}$ for quantitative evaluation. %These include the Mean Dice Similarity Coefficient (mDice), mean Intersection over Union (mIoU), Mean Absolute Error (MAE), weighted F-measure ($F_{\beta}^{w}$), S-measure ($S_{\alpha}$), and max E-measure ($E_{\xi}^{max}$). mDice and mIoU are regional-level similarity measures, while MAE and $F_{\beta}^{w}$ are used to assess pixel-level accuracy. $S_{\alpha}$ and $E_{\xi}^{max}$ are applied to analyze global-level similarity. For MAE, a lower value is preferable, whereas for the other metrics, a higher value is better. 


%\noindent\textbf{Training Settings.} 
The training set is composed of 900 images from Kvasir-SEG and 550 images from CVC-ClinicDB. 100 images from Kvasir-SEG and 62 images from CVC-ClinicDB are set aside as test sets. Since these images have been encountered during training, they are employed to evaluate the learning capacity of our model. We use ColonDB, Endoscene, and ETIS serve to measure the generalization ability of our model. Futher, PolypFlow is implemented using the PyTorch 2.0 framework. The GPU used is an NVIDIA TITAN RTX 4090 with 32G of RAM. During the training process, all input images are uniformly resized to 352×352. The network is trained for 100 epochs with a batch size of 8. For the optimizer, we choose AdamW to update the parameters. The learning rate is set to 1e-4. 

\begin{table}[h]
\centering
\caption{Quantitative results on ColonDB, Endoscene and ETIS datasets.}
\label{table:2}

\fontsize{15pt}{7pt}\selectfont
\setlength{\tabcolsep}{8pt}
\resizebox{\textwidth}{!}{
\begin{tabular}{ll|c|cccccc}
\toprule
\multicolumn{2}{c|}{Methods} & year & mDice & mIoU & $F_\beta^\omega$ & $S_\alpha$ & $E_\phi^{\text{max}}$ & MAE \\
\midrule
\multirow{6}{*}{\rotatebox{90}{ColonDB}} & U-Net~\cite{ronneberger2015u} & 2015 & 0.504 & 0.436 & 0.491 & 0.710 & 0.781 & 0.059 \\
\multirow{6}{*}{} & U-Net++~\cite{zhou2018unet++} & 2018 & 0.481 & 0.408 & 0.467 & 0.693 & 0.763 & 0.061 \\
\multirow{6}{*}{} & PraNet~\cite{fan2020pranet} & 2020 & 0.712 & 0.640 & 0.699 & 0.820 & 0.872 & 0.043 \\
\multirow{6}{*}{} & SANet~\cite{wei2021shallow} & 2021 & 0.752 & 0.669 & 0.725 & 0.837 & 0.875 & 0.043 \\
\multirow{6}{*}{} & MedSAM~\cite{ma2024segment} & 2024 & 0.734 & 0.651 & - & - & - & - \\
\multirow{6}{*}{} & $\text{M}^2$ixNet~\cite{zheng2024polyp} & 2024 & 0.820 & \textbf{0.855} & 0.799 & 0.869 & 0.935 & \textbf{0.021} \\
\multirow{6}{*}{} & CAFE-Net~\cite{liu2024cafe} & 2024 & 0.820 & 0.740 & 0.803 & 0.874 & 0.918 & 0.026 \\
\multirow{6}{*}{} & \textbf{PolypFlow(Ours)} & 2025 & \textbf{0.822} & 0.741 & \textbf{0.810} & \textbf{0.877} & \textbf{0.952} & 0.022 \\
\midrule
\midrule
\multirow{6}{*}{\rotatebox{90}{Endoscene}} & U-Net~\cite{ronneberger2015u} & 2015 & 0.710 & 0.627 & 0.684 & 0.843 & 0.875 & 0.022\\
\multirow{6}{*}{} & U-Net++~\cite{zhou2018unet++} & 2018 & 0.707 & 0.624 & 0.687 & 0.839 & 0.898 & 0.018\\
\multirow{6}{*}{} & PraNet~\cite{fan2020pranet} & 2020 & 0.871 & 0.797 & 0.843 & 0.925 & 0.972 & 0.010\\
\multirow{6}{*}{} & SANet~\cite{wei2021shallow} & 2021 & 0.888 & 0.815 & 0.859 & 0.928 & 0.972 & 0.008 \\
\multirow{6}{*}{} & MedSAM~\cite{ma2024segment} & 2024 & 0.870 & 0.798 & - & - & - & - \\
\multirow{6}{*}{} & $\text{M}^2$ixNet~\cite{zheng2024polyp} & 2024 & 0.895 & \textbf{0.861} & 0.879 & 0.941 & 0.978 & \textbf{0.005} \\
\multirow{6}{*}{} & CAFE-Net~\cite{liu2024cafe} & 2024 & \textbf{0.901} & 0.834 & 0.882 & 0.939 & 0.981 & 0.006 \\
\multirow{6}{*}{} & \textbf{PolypFlow(Ours)} & 2025 & \textbf{0.901} & 0.852 & \textbf{0.885} & \textbf{0.953} & \textbf{0.986} & \textbf{0.005} \\
\midrule
\midrule
\multirow{6}{*}{\rotatebox{90}{ETIS}} & U-Net~\cite{ronneberger2015u} & 2015 & 0.398 & 0.335 & 0.366 & 0.684 & 0.740 & 0.036\\
\multirow{6}{*}{} & U-Net++~\cite{zhou2018unet++} & 2018 & 0.401 & 0.343 & 0.390 & 0.683 & 0.776 & 0.035\\
\multirow{6}{*}{} & PraNet~\cite{fan2020pranet} & 2020 & 0.628 & 0.567 & 0.600 & 0.794 & 0.841 & 0.031\\
\multirow{6}{*}{} & SANet~\cite{wei2021shallow} & 2021 & 0.750 & 0.654 & 0.685 & 0.849 & 0.897 & 0.015 \\
\multirow{6}{*}{} & MedSAM~\cite{ma2024segment} & 2024 & 0.687 & 0.604 & - & - & - & - \\
\multirow{6}{*}{} & $\text{M}^2$ixNet~\cite{zheng2024polyp} & 2024 & 0.891 & 0.866 & 0.753 & 0.882 & 0.933 & \textbf{0.009} \\
\multirow{6}{*}{} & CAFE-Net~\cite{liu2024cafe} & 2024 & 0.822 & 0.738 & 0.775 & \textbf{0.898} & 0.940 & 0.014 \\
\multirow{6}{*}{} & \textbf{PolypFlow(Ours)} & 2025 & \textbf{0.895} & \textbf{0.872} & \textbf{0.815} & 0.895 & \textbf{0.972} & 0.010 \\

\bottomrule
\end{tabular}
}
\end{table}

\subsection{Performance Comparisons}%包括消融分析
% We compare the proposed PolypFLow with previous state-of-the-art methods, such as U-Net~\cite{ronneberger2015u}, U-Net++~\cite{zhou2018unet++}, PraNet~\cite{fan2020pranet}, SANet~\cite{wei2021shallow} and $\text{M}^2$ixNet~\cite{zheng2024polyp}. To ensure a fair comparison, we use their open-source codes to evaluate on the same datasets or rely on the predictions they provided.
%\noindent\textbf{Quantitative Comparison:} 
%As shown in Tab.~\ref{table:1} and Tab.~\ref{table:2}, PolypFlow achieves the best scores on almost all metrics across five datasets.
we conduct two experiments to validate our model’s learning ability on two seen datasets, i.e., Kvasir and ClinicDB. As shown in Tab.~\ref{table:1}, PolypFlow outperforms all baselines by a large margin (mDice > 4\%) across both two datasets, in all metrics. This suggests that our model has a strong learning ability to effectively segment polyps.

We conduct three experiments to test the model’s generalizability. PolypFlow outperforms existing classical medical segmentation baselines (U-Net, U-Net++), with significant improvements on all three unseen datasets in Tab.~\ref{table:2}.
%\noindent\textbf{Visual Comparison:} 
Fig.~\ref{fig:vision} presents the visualization results of our model and the compared models. Our proposed model outperforms other methods significantly, with segmentation results closest to the ground truth. %It has several advantages. For example, it can adapt to different lighting conditions, like the overexposure in the third and sixth images of Fig.~\ref{fig:vision}. This is due to the role of the image-level prior introduced by DAM. 
\begin{figure}[h]
    \centering
    \includegraphics[width=1\linewidth]{fig/vision.png}
    
    \caption{visual results of different methods.}
    \label{fig:vision}
\end{figure}
\begin{table}[h]
\centering
\caption{Ablation study.}
\label{table:3}

\fontsize{8pt}{6.5pt}\selectfont
\setlength{\tabcolsep}{8pt}
%\fontsize{15pt}{7pt}\selectfont
%\setlength{\tabcolsep}{8pt}
\resizebox{\textwidth}{!}{
\begin{tabular}{r|cc|cc}
\toprule
Settings & \multicolumn{2}{c|}{ClinicDB (seen)} & \multicolumn{2}{c}{Endoscene (unseen)}\\

& mDice & mIoU
% & \( S_{\alpha} \) 
& mDice & mIoU
% & \( S_{\alpha} \)
\\
\midrule 
Backbone & 0.736 & 0.568 
% & 0.735
& 0.721 & 0.607
% & 0.670 
\\
SA + Backbone & 0.835 & 0.743 
% & 0.912 
& 0.863 & 0.786
% & 0.888 
\\
DCT($\cdot$)+ Backbone & 0.875 & 0.847
% & \textbf{0.936} 
&0.881 & \textbf{0.859}
% & \textbf{0.925} 
\\
SA +DCT($\cdot$) + Backbone & \textbf{0.901} & \textbf{0.885} 
& \textbf{0.891} & 0.797
\\

\bottomrule
\end{tabular}
}
\end{table}
\subsection{Ablation Study and Discussion}
We conduct tests on every component of our PolypFlow that the model has encountered during seen datasets and unseen datasets to offer a more profound understanding of our model's performance and characteristics. 
The experimental results demonstrate that the self-attention module (SA) significantly improves model performance. From Tab.~\ref{table:3}, on the ClinicDB (seen dataset), after adding SA, the mDice and mIoU increase from 0.736 and 0.568 to 0.835 and 0.743, respectively. On the Endosceme (unseen dataset), the mDice and mIoU increase from 0.721 and 0.607 to 0.863 and 0.786, respectively. This indicates that SA effectively captures global contextual information, thereby enhancing segmentation accuracy.
The DCT($\cdot$) also plays a crucial role in improving model performance.  From Tab.~\ref{table:3}, on the ClinicDB (seen dataset), after adding DCT, the mDice and mIoU increase to 0.875 and 0.847, respectively. On the Endosceme (unseen dataset), the mDice and mIoU increase to 0.881 and 0.859, respectively. By extracting frequency-domain features, DCT enhances the model's ability to capture detailed information, thereby further optimizing segmentation results.

Here, we have conducted two additional experiments to verify the effectiveness of our method. 1: We selected iterative vector fields with 1, 5, 8, and 15 steps. We found that increasing the number of steps can improve the model's performance, but the difference between 15 steps and 10 steps is minimal. In addition, we replaced U-Net with U-Net++, which improved the performance of polyp segmentation by 2.3\%.


\section{Conclusion}
In this work, we develop PolypFlow, for automatically segmenting polyps from colonoscopy images, which a vector field based on self-attention and DCT. Extensive experimental evaluations have shown that, on five challenging datasets, PolypFlow consistently surpasses all existing state-of-the-art methods with a significant margin, achieving an improvement of more than 4.5\%. Another advantage is that PolypFlow visualizes the process of polyp segmentation.


%\section{Acknowledgement}
% \section{Important Messages to MICCAI Authors}

% This is a modified version of Springer's LNCS template suitable for anonymized MICCAI 2025 main conference submissions. The author section has already been anonymized for you.  You cannot remove the author section to gain extra writing space.  You must not remove the abstract and the keyword sections in your submission.

% To avoid an accidental anonymity breach, you are not required to include the acknowledgments and the Disclosure of Interest sections for the initial submission phase.  If your paper is accepted, you must reinsert these sections to your final paper. 

% To avoid desk rejection of your paper, you are encouraged to read "Avoiding Desk Rejection" by the MICCAI submission platform manager.  A few important rules are summarized below for your reference:  
% \begin{enumerate}
%     \item Do not modify or remove the provided anonymized author section.
%     \item Do not remove the abstract and keyword sections to gain extra writing space. Authors must provide at least one keyword.
%     \item Inline figures (wrapping text around a figure) are not allowed.
%     \item Authors are not allowed to change the default margins, font size, font type, and document style.  If you are using any additional packages, it is the author's responsibility that any additional packages do not inherently change the document's layout.
%     \item It is prohibited to use commands such as \verb|\vspace| and \verb|\hspace| to reduce the pre-set white space in the document.
%     \item Please make sure your figures and tables do not span into the margins.
%     \item If you have tables, the smallest font size allowed is 8pt.
%     \item Please ensure your paper is properly anonymized.  Refer to your previously published paper in the third person.  Data collection location and private dataset should be masked.
%     \item Your paper must not exceed the page limit.  You can have 8 pages of the main content (text (including paper title, author section, abstract, keyword section), figures and tables, and conclusion) plus up to 2 pages of references.  The reference section does not need to start on a new page.  If you include the acknowledgments and the Disclosure of Interest sections, these two sections are considered part of the main content.
%     \item If you submit supplementary material, please note that reviewers are encouraged but not required to read it.  The main paper should be self-contained.  *NEW* Supplementary materials are limited to multimedia content (e.g., videos) as warranted by the technical application (e.g. robotics, surgery, ….). These files should not display any proofs, analysis, or additional results, and should not show any identification markers either. Violation of this guideline will lead to desk rejection. PDF files may NOT be submitted as supplementary materials in 2025 unless authors are citing a paper that has not yet been published.  In such a case, authors are required to submit an anonymized version of the cited paper.
% \end{enumerate}


% \section{Section}
% \subsection{A Subsection Sample}
% Please note that the first paragraph of a section or subsection is
% not indented. The first paragraph that follows a table, figure,
% equation etc. does not need an indent, either.

% Subsequent paragraphs, however, are indented.

% \subsubsection{Sample Heading (Third Level)} Only two levels of
% headings should be numbered. Lower level headings remain unnumbered;
% they are formatted as run-in headings.

% \paragraph{Sample Heading (Fourth Level)}
% The contribution should contain no more than four levels of
% headings. Table~\ref{tab1} gives a summary of all heading levels.

% \begin{table}
% \caption{Table captions should be placed above the
% tables.}\label{tab1}
% \begin{tabular}{|l|l|l|}
% \hline
% Heading level &  Example & Font size and style\\
% \hline
% Title (centered) &  {\Large\bfseries Lecture Notes} & 14 point, bold\\
% 1st-level heading &  {\large\bfseries 1 Introduction} & 12 point, bold\\
% 2nd-level heading & {\bfseries 2.1 Printing Area} & 10 point, bold\\
% 3rd-level heading & {\bfseries Run-in Heading in Bold.} Text follows & 10 point, bold\\
% 4th-level heading & {\itshape Lowest Level Heading.} Text follows & 10 point, italic\\
% \hline
% \end{tabular}
% \end{table}


% \noindent Displayed equations are centered and set on a separate
% line.
% \begin{equation}
% x + y = z
% \end{equation}
% Please try to avoid rasterized images for line-art diagrams and
% schemas. Whenever possible, use vector graphics instead (see
% Fig.~\ref{fig1}).

% \begin{figure}
% \includegraphics[width=\textwidth]{fig1.eps}
% \caption{A figure caption is always placed below the illustration.
% Please note that short captions are centered, while long ones are
% justified by the macro package automatically.} \label{fig1}
% \end{figure}

% \begin{theorem}
% This is a sample theorem. The run-in heading is set in bold, while
% the following text appears in italics. Definitions, lemmas,
% propositions, and corollaries are styled the same way.
% \end{theorem}
% %
% % the environments 'definition', 'lemma', 'proposition', 'corollary',
% % 'remark', and 'example' are defined in the LLNCS documentclass as well.
% %
% \begin{proof}
% Proofs, examples, and remarks have the initial word in italics,
% while the following text appears in normal font.
% \end{proof}
% For citations of references, we prefer the use of square brackets
% and consecutive numbers. Citations using labels or the author/year
% convention are also acceptable. The following bibliography provides
% a sample reference list with entries for journal
% articles~\cite{ref_article1}, an LNCS chapter~\cite{ref_lncs1}, a
% book~\cite{ref_book1}, proceedings without editors~\cite{ref_proc1},
% and a homepage~\cite{ref_url1}. Multiple citations are grouped
% \cite{ref_article1,ref_lncs1,ref_book1},
% \cite{ref_article1,ref_book1,ref_proc1,ref_url1}.

% \begin{comment}  %% removed for anonymized MICCAI 2025 submission.
    
%     % The following acknowledgement and disclaimer sections should be removed for the double-blind review process.  
%     % If and when your paper is accepted, reinsert the acknowledgement and the disclaimer clause in your final camera-ready version.

% \begin{credits}
% \subsubsection{\ackname} A bold run-in heading in small font size at the end of the paper is
% used for general acknowledgments, for example: This study was funded
% by X (grant number Y).

% \subsubsection{\discintname}
% It is now necessary to declare any competing interests or to specifically
% state that the authors have no competing interests. Please place the
% statement with a bold run-in heading in small font size beneath the
% (optional) acknowledgments\footnote{If EquinOCS, our proceedings submission
% system, is used, then the disclaimer can be provided directly in the system.},
% for example: The authors have no competing interests to declare that are
% relevant to the content of this article. Or: Author A has received research
% grants from Company W. Author B has received a speaker honorarium from
% Company X and owns stock in Company Y. Author C is a member of committee Z.
% \end{credits}

% \end{comment}
% %
% ---- Bibliography ----
%
% BibTeX users should specify bibliography style 'splncs04'.
% References will then be sorted and formatted in the correct style.
%
\bibliographystyle{splncs04}
\bibliography{ref}


\end{document}
