%%%% ijcai24.tex

\typeout{IJCAI--24 Instructions for Authors}

% These are the instructions for authors for IJCAI-24.

\documentclass{article}
\pdfpagewidth=8.5in
\pdfpageheight=11in
% The file ijcai24.sty is a copy from ijcai22.sty
% The file ijcai22.sty is NOT the same as previous years'
\usepackage{ijcai24}
% Use the postscript times font!
\usepackage{times}
\usepackage{soul}
\usepackage{url}
\usepackage[hidelinks]{hyperref}
\usepackage[utf8]{inputenc}
\usepackage[small]{caption}
\usepackage{graphicx}
\usepackage{amsmath}
\usepackage{amsthm}
\usepackage{booktabs}
\usepackage{algorithm}
\usepackage{algorithmic}
\usepackage[switch]{lineno}
\usepackage{multirow}
\usepackage{amsmath}
\usepackage{xcolor}
\usepackage{colortbl}
\usepackage{amssymb}
\usepackage{hhline}
% Comment out this line in the camera-ready submission
% \linenumbers
% \urlstyle{same}

% the following package is optional:
%\usepackage{latexsym}

% See https://www.overleaf.com/learn/latex/theorems_and_proofs
% for a nice explanation of how to define new theorems, but keep
% in mind that the amsthm package is already included in this
% template and that you must *not* alter the styling.
\newtheorem{example}{Example}
\newtheorem{theorem}{Theorem}

% Following comment is from ijcai97-submit.tex:
% The preparation of these files was supported by Schlumberger Palo Alto
% Research, AT\&T Bell Laboratories, and Morgan Kaufmann Publishers.
% Shirley Jowell, of Morgan Kaufmann Publishers, and Peter F.
% Patel-Schneider, of AT\&T Bell Laboratories collaborated on their
% preparation.

% These instructions can be modified and used in other conferences as long
% as credit to the authors and supporting agencies is retained, this notice
% is not changed, and further modification or reuse is not restricted.
% Neither Shirley Jowell nor Peter F. Patel-Schneider can be listed as
% contacts for providing assistance without their prior permission.

% To use for other conferences, change references to files and the
% conference appropriate and use other authors, contacts, publishers, and
% organizations.
% Also change the deadline and address for returning papers and the length and
% page charge instructions.
% Put where the files are available in the appropriate places.


% PDF Info Is REQUIRED.

% Please leave this \pdfinfo block untouched both for the submission and
% Camera Ready Copy. Do not include Title and Author information in the pdfinfo section
\pdfinfo{
/TemplateVersion (IJCAI.2024.0)
}

\newif\ifshowcomments
\showcommentstrue % Set to true to show comments, set to false to hide
\ifshowcomments
    \newcommand{\yj}[1]{{\color{blue}[YJ: #1]}}
\else
    \newcommand{\yj}[1]{}
\fi

\title{
% DistriMark: Robust Multi-Bit Semantic Watermarking for Latent Diffusion
% Models in Model Distribution Scenarios 
Secure and Efficient Watermarking for Latent Diffusion Models in Model Distribution Scenarios}
% \yj{Secure and Efficient Watermarking for Latent Diffusion Models in Model Distribution Scenarios}}

% \author{
%     Anonymous
% }
% % Single author syntax
% % \author{
% %     Author Name
% %     \affiliations
% %     Affiliation
% %     \emails
% %     email@example.com
% % }

% % Multiple author syntax (remove the single-author syntax above and the \iffalse ... \fi here)
% \iffalse

\author{
Liangqi Lei$^1$
\and
Keke Gai$^1$\and
Jing Yu$^{2}$\And
Liehuang Zhu$^1$\\
Qi Wu$^3$\\
\affiliations
$^1$Beijing Institute of Technology\\
$^2$School of Information Engineering, Minzu University of China.\\
$^3$School of Computer Science, The University of Adelaide. \\
\emails
3120245873@bit.edu.cn, gaikeke@bit.edu.cn,
jing.yu@muc.edu.cn,
liehuangz@bit.edu.cn,
qi.wu01@adelaide.edu.au
}


\begin{document}

\maketitle

\begin{abstract}

Latent diffusion models have exhibited considerable potential in generative tasks. Watermarking is considered to be an alternative to safeguard the copyright of generative models and prevent their misuse.
However, in the context of model distribution scenarios, the accessibility of models to large  scale of model users brings new challenges to the security, efficiency and robustness of existing watermark solutions. To address these issues, we propose a secure and efficient watermarking solution. A new security mechanism is designed to prevent watermark leakage and watermark escape, which considers  watermark randomness and watermark-model association as two constraints for  mandatory watermark injection. %To increase the security of watermark distribution, we propose watermark generation module to transform the watermark into a watermark-specific distribution and obtain a watermarked latent variable through sampling strategy. 
To reduce the time cost of training the security module, watermark injection and the security mechanism are decoupled, ensuring that fine-tuning VAE only accomplishes the security mechanism without the burden of learning watermark patterns. A watermark distribution-based verification strategy is proposed to enhance the robustness against diverse attacks in the model distribution scenarios.
% we decouple the watermark injection and the security mechanism, ensuring that fine-tuning VAE only accomplishes the security mechanism without the burden of learning watermark patterns.
% Specifically, the robustness of watermark detection is ensured by verifying whether the watermarked latent variables fall within the predefined watermark distribution. 
% We develop a watermark-network controller that is a security mechanism integrated into the latent diffusion model component, ensuring the integrity and security of watermark embeddings. 
Experimental results prove that our watermarking  consistently outperforms existing six baselines on effectiveness and robustness against ten image processing attacks and adversarial attacks, while enhancing security in the distribution scenarios.
%Latent diffusion models have demonstrated immense potential in generative tasks but have also raised concerns regarding illicit usage. Watermarking has been proved to be an effective solution for copyright protection and provenance tracking of generated content. 
% However, in model distribution scenarios, post-hoc watermarks can be easily circumvented, and existing model watermarking methods designed for latent diffusion models fail to withstand reconstruction attacks and adaptive adversarial sample attacks. 
%However, in model distribution scenarios, since model users can obtain the parameters of the model, it poses new threats to the security and robustness of watermarks.
% \yj{What's the core challenge in the distribution scenario, robustness, security, or both?} 
%Therefore, we propose a semantic watermarking method for latent diffusion model in model distribution scenarios. 
%We propose a Secure Latent Watermark Distribution method. 
%The robustness of watermark detection is ensured by detecting whether the watermarked latent variables are within the watermark distribution. 
%We introduce Watermark-Network Controller, a security mechanism integrated into latent diffusion model component to guarantee the security of watermark embedding.
% We propose a novel watermark distribution method : Watermarked Sampling from Distribution \yj{Introduce the mechanism to guarantee robustness} and first introduce Latent-VAE Skip-Binder, a security mechanism integrated into latent diffusion model component to guarantee the security of watermark embedding.
%Experimental results demonstrate that our watermarking  is consistently effective and robust compared with existing six baselines over ten image processing attacks and challenging adversarial attacks while is more secure in the distribution scenarios. 
The code is available at https://anonymous.4open.science/r/DistriMark-F11F/.
%that compared to recent methods such as FSWatermark and Stable Signature, our method not only surpasses them in robustness but also exhibits significant advantages in terms of image quality.
\end{abstract}



% 
% 
The widespread integration of communication networks and smart devices in modern control systems has increased the vulnerability of industrial systems to online cyber-attacks, e.g., Industroyer, Blackenergy, etc \citep{osti_1505628}.
% Modern control systems have seen a large push to include communication networks and smart devices to increase performance, made possible by improvements in communication device cost and energy consumption. This trend has been coupled with the usage of open-standard communication protocols among industrial control systems, making them vulnerable to online cyber-attacks such as Industroyer, Blackenergy, etc \citep{osti_1505628}. 
To counter this, methods have been developed to improve security by achieving attack detection, mitigation, and monitoring, among others \citep{sandberg2022secure}. This paper focuses on active attack diagnosis to mitigate stealthy attacks. 
%
%\subsection{Literature review}

Active diagnosis techniques rely on the inclusion of additional moduli to control systems
% inclusion within the control system of additional moduli 
to alter the behavior of the system compared to information known by the attacker. 
For instance, the concept of additive watermarking was introduced in \cite{mo2015physical}, where noise signals of known mean and variance are added at the plant and compensated for it at the controller. 
This compensation, however, is not exact, causing some performance degradation. Thus, trade-offs between performance and detectability  are necessary \citep{zhu2023detection}.
% A later work \citep{zhu2023detection} designs the watermark signal by trading performance for detection. Thus, although additive watermarking serves as a good detection scheme, they endure performance losses even in the nominal case. 

In encrypted control \citep{darup2021encrypted}, the sensor data is encrypted, sent to the controller, and then operated on directly. Encrypted input signals are sent back to the plant for decryption. Although encryption is widespread in IT security, in control systems it presents some concerns, such as the introduction of time delays \citep{stabile2024verifiable}, while it may present inherent weaknesses \citep{alisic2023model}.
% they are not preferred as they introduce time delays \citep{stabile2024verifiable} which can cause instability, and some encryption schemes can be very weak  \citep{alisic2023model}. 

In moving target defense \citep{griffioen2020moving}, the plant is augmented with fictitious dynamics, known to the controller. The plant output is transmitted to the controller along with the fictitious states over a network under attack. 
The additional measurements then aide in the detection of attacks. 
This comes at the cost of higher communication bandwidth needs, which increases rapidly with the dimension of the augmented systems.
% Since the dynamics of the fictitious dynamics are exactly known to the controller, the attack is detected easily. However, when the scale of the system increases, the communication bandwidth used by moving the target defense approach increases rapidly. 

Other recently proposed works include two-way coding \citep{fang2019two}, a weak encryuption technique, and dynamic masking \citep{abdalmoaty2023privacy}, which enhances privacy as well as security, have been shown to be effective against zero-dynamics attacks.
% Two-way coding \citep{fang2019two} and dynamic masking \citep{abdalmoaty2023privacy} are other recently proposed approaches. Two-way coding is another form of weak encryption technique whilst dynamic masking proposes an architecture that enhances both privacy and security. These schemes are shown to be effective against zero dynamics attacks but remain to be studied for other classes of attacks. 
% Recent extensions include \citep{mukherjee2021secure,ramos2024privacy}.
% Some other works which are related are \citep{mukherjee2021secure}, an extension of \cite{fang2019two}. The work \citep{ramos2024privacy} is an extension of moving target defense for multi-agent systems. 
Furthermore, filtering techniques for attack detection are proposed by \cite{murguia2020security,hashemi2022codesign,escudero2023safety}, while not focusing on stealthy attacks.
% The works \citep{murguia2020security,hashemi2022codesign,escudero2023safety} develop filtering techniques to guarantee safety, without being focused on stealthy covert attacks.

Multiplicative watermarking (mWM) has been proposed by the authors as a diagnosis technique \citep{ferrari2020switching}. mWM consists of a pair of filters on each communication channel between the plant and its controller; the scheme is affine to weak encryption, whereby ``encoding'' and ``decoding'' are done by changing signals' dynamic characteristics through inverse pairs of filters. This enables original signals to be recovered exactly, and thus does not lead to performance degradation.
% A multiplicative watermark is an affine to a weak encryption technique, through which the signal is ``encoded'' by a filter, changing its dynamic behavior. The use of inverse pairs means that the original signal can be recovered, through ``decoding'' via an inverse filter. As such, differently to techniques based on additive watermarking, no performance is lost due to the injection of noise, and there are no bandwidth limitations.

%\subsection{Contributions}
One of the critical features of multiplicative watermarking is that to detect stealthy attacks, the mWM filter parameters must be switched over time. In this paper, an algorithm to optimally design the mWM parameters after a switching event is presented, enhancing detection performance, without changing the switching time.
% This is done without changing the switching time, which is taken as given.

\textcolor{black}{
To formalize the filter design problem, we suppose the defender is interested in optimal performance against adversaries injecting covert attacks with matched system parameters \citep{smith2015covert}, including the mWM parameters prior to the switch. This scenario represents a worst case where malicious agents can take full control of the system while remaining undetected.
Thus, the attack strategy is explicitly included within the formulation of the closed-loop system, and the mWM filters are chosen by solving an optimization problem minimizing the attack-energy-constrained output-to-output gain (AEC-OOG) \citep{anand2023risk}, a variation of the output-to-output gain proposed in  \cite{teixeira2015strategic}.
}
The main contributions of this paper are:
% We consider an adversary injecting a covert attack with matched system parameters \citep{smith2015covert}, i.e., an attacker with full knowledge of the control system parameters, including those of the mWM filters before the switch. This scenario is taken as a worst case, as it has been shown that this class of attacks can be made stealthy. To quantitatively define a cost, the output-to-output gain (OOG) \citep{teixeira2015strategic} is leveraged,
% a metric introduced to evaluate the impact of an additive attack in a control system. %Specifically, OOG evaluates the worst-case performance loss that an attacker injecting an undetectable attack can obtain. 
% Here, the maximum performance loss caused by a stealthy adversary with limited energy is taken, the attack-energy-constrained OOG (AEC-OOG) \citep{anand2023risk}. The main contributions of this paper are:
\begin{enumerate}
%[label=\alph*.]
\item The problem of optimally designing the switching mWM filters is formulated as an optimization problem, with the AEC-OOG is taken as the objective;%where the AEC-OOG is taken as the impact metric; 
\item The worst-case scenario of a covert attack with exact knowledge of plant and mWM filter parameters is embedded within the design problem;
% The optimization problem is defined to incorporate the worst-case scenario of a covert attack with exact knowledge of plant and mWM filter parameters;
\item The feasibility of the optimization problem is shown to be dependent only on stability conditions; 
\item A solution scheme is proposed to promote randomization of the mWM filter parameters such that an eavesdropping adversary cannot remain stealthy.
\end{enumerate} 

This builds on the results of \cite{ferrari2020switching}, where the focus was on the design of the switching protocols, rather than the parameters themselves.
Compared to previous work \citep{gallo2021design}, this paper introduces an optimization problem which is always feasible (thanks to the use of AEC-OOG in the objective), while also considering a more sophisticated class of covert attacks, where the presence of watermark is known to the adversary. 
Moreover, this paper poses a different objective than \citep{zhang2023hybrid}; indeed, while \citep{zhang2023hybrid} provided a design strategy to ensure certain privacy properties, in this paper we address the problem of optimal parameter design following a switching event.


%\subsection{Organization}
The rest of the paper is organized as follows. 
After formulating the problem in Section~\ref{sec:PF}, we propose our design algorithm in Section~\ref{sec:main}, and analyze its properties. It is then evaluated through a numerical example in Section~\ref{sec:NE}, and concluding remarks are given Section~\ref{sec:Con}.
% We provide the problem background in Section~\ref{sec:PF}. We formulate the design problem in Section~\ref{sec:main}, together with an analysis of its properties. The proposed algorithm is evaluated through a numerical example in Section \ref{sec:NE}. Concluding remarks are offered in Section \ref{sec:Con}.
\section{Related Work}
\subsection{Multimodal Large Language Models}
% Building on the success of large language models (LLMs) \citep{yao2024tree, glm2024chatglm, achiam2023gpt, touvron2023llama, brown2020language}, multimodal large language models (MLLMs) \citep{liu2024improved, li2023blip, zhu2023minigpt, wang2023cogvlm, liu2024visual} extend these capabilities by integrating vision and text processing, achieving remarkable performance in tasks involving images, videos, and multimodal reasoning. However, handling visual data poses computational challenges due to the redundancy and low information density of high-resolution tokens \citep{liang2022evit} and the quadratic scaling of attention mechanisms \citep{vaswani2017attention}.
% For instance, models like LLaVA \citep{liu2023improvedllava} and mini-Gemini-HD \citep{li2024mini} encode high-resolution images into thousands of tokens, while video-based models such as VideoLLaVA \citep{lin2023video} and VideoPoet \citep{kondratyuk2023videopoet} allocate even more tokens to process multiple frames. These challenges highlight the need for more efficient token representations and longer context lengths to enable scalability. Recent advancements, such as Gemini \citep{geminiteam2023gemini} and LWM \citep{liu2024world}, have focused on addressing these issues by optimizing token efficiency and extending the context length, paving the way for more scalable and effective MLLMs.

The remarkable success of large language models (LLMs) \citep{radford2019language, brown2020language} has spurred a growing trend of extending their advanced reasoning capabilities to multi-modal tasks, leading to the development of vision-language models (VLMs) \citep{huang2023languageneedaligningperception, driess2023palmeembodiedmultimodallanguage, liu2024visual, Qwen-VL}. These VLMs typically consist of a visual encoder \citep{radford2021learning} that serializes input image representations and an LLM responsible for text generation. To enable the LLM to process visual inputs, an alignment module is employed to bridge the gap between visual and textual modalities. This module can take various forms, such as a simple linear layer, an MLP projector, or a more complex query-based network. While this integration allows the LLM to gain visual perception, it also introduces significant computational challenges due to the long sequences of visual tokens.

Moreover, existing VLMs often exhibit limitations, such as visual shortcomings or hallucinations, which hinder their performance. Efforts to enhance VLM capabilities by increasing input image resolution have further exacerbated computational demands. For instance, encoding higher-resolution images results in a substantial increase in the number of visual tokens. A model like LLaVA-1.5 \citep{liu2024improved} generates 576 visual tokens for a single image, while its successor, LLaVA-NeXT \citep{liu2024llavanext}, produces up to 2880 tokens at double the resolution, far exceeding the length of typical textual prompts.
Optimizing the inference efficiency of VLMs is thus a critical task to facilitate their deployment in real-world scenarios with limited computational resources.

\subsection{Visual Token Compression}
% Visual tokens often exceed text tokens by tens to hundreds of times, with visual signals being more spatially redundant compared to information dense text \citep{marr2010vision}.
% Various methods have been proposed to address this issue. For instance, LLaMA-VID \citep{li2023llama} uses a Q-Former with context tokens, and DeCo \citep{yao2024deco} applies adaptive pooling to downsample visual tokens at the patch level.
% However, these approaches require modifying model components and additional training, increasing computational and training costs.
% ToMe~\citep{bolya2022tome} reduces tokens without training by adding a token merge module to ViTs, but this disrupts early cross-modal interactions in language models~\citep{xing2024PyramidDrop}. FastV~\citep{chen2024image} selects important visual tokens using attention scores, while SparseVLM~\citep{zhang2024sparsevlm} incorporates text guidance via cross-modal attention.
% However, these methods forgo flash-attention~\citep{dao2022flashattention, dao2023flashattention2} and primarily focus on token importance, overlooking the impact of token duplication.
% In our work, we preserve hardware acceleration compatibility, including flash attention, while considering both token importance and duplication for token reduction.

Visual tokens are often significantly more numerous than text tokens, with higher spatial redundancy and lower information density. To address this issue, various methods have been proposed for reducing visual token counts in vision language models. For instance, some approaches modify model components, such as using context tokens in Q-Former \citep{li2023llama} or applying adaptive pooling at the patch level, but these typically require additional training and increase computational costs. Other techniques, like Token Merging (ToMe) \citep{bolya2022tome} and FastV \citep{chen2024image}, focus on reducing tokens without retraining by merging tokens or selecting important ones based on attention scores. SparseVLM \cite{zhang2024sparsevlm} incorporates text guidance through cross-modal attention to refine token selection. However, these methods often overlook hardware acceleration compatibility and fail to account for token duplication alongside token importance. Furthermore, while token pruning has been extensively explored in natural language processing and computer vision to improve inference efficiency, its application to VLMs remains under-explored. Existing pruning strategies, such as those in FastV and SparseVLM, rely on text-visual attention within large language models (LLMs) to evaluate token importance, which may not align well with actual visual token relevance.


\section{Methodology}
In this section, we outline the key research questions driving this study, followed by a detailed description of the methodology used to design and conduct the survey.
\subsection{Research Questions}
\begin{enumerate}
    \item[\textbf{RQ1:}] How do developers allocate their time during a typical workweek, and how does this compare to their perception of an \textbf{ideal workweek?}
    \item[\textbf{RQ2:}] How are developer's satisfaction and productivity affected by \textbf{deviations} from their ideal workweek?
     \item[\textbf{RQ3:}] For which tasks do developers prefer using \textbf{AI tools}, and how does the frequency of AI tool usage \textbf{influence} their satisfaction and productivity?
\end{enumerate}

\subsection{Survey Design}
% Describe how the survey was conducted, survey structure, sample size, which activities were selected and how, incentives, etc. 

To gain insights into the types of activities developers engage in during a typical work week, we conducted a series of exploratory interviews with 12 randomly selected participants. These semi-structured interviews provided a qualitative foundation, allowing us to iteratively develop a comprehensive list of higher-level activities that reflect both ideal and actual workweek allocations. The findings from these interviews were instrumental in refining our survey questions and design.

% - When was it distributed
% - How many people were invited
% - how was the survey advertised
% - incentive provided to participants
% - how many responses received (with response rates)
% - Board of ethics description \& instruments
% - Describe the main questions asked in the survey

The survey was distributed in \textcolor{blue}{May 2024} to software engineers working in Microsoft teams across India and the United States. A total of 6000 developers were invited to participate via email. Framed as a study aimed at boosting developer productivity by understanding how they allocate their time in a workday, the survey received 510 complete responses (responses rate of 8.5\%). After finishing the survey, the participants could enter a sweepstake to win one out of ten \$50 Amazon.com Gift Cards.
\textcolor{blue}{description of ethics}.

The main questions in the survey were as follows:
\begin{enumerate}
    \item Their roles and years of experience in the industry/team
    \item The hours spent on various activities in their typical workweek
    \item Ideally, the percentage of time they would want to allocate to each activity in a workweek
    \item How productive and satisfied were they by their past workweek
    \item Activities they find most cognitively challenging
    \item How often do they use AI tools to assist in their daily activities
    \item Two open-ended questions about the activities they would want to automate using AI tools, and advice for new hires to boost their productivity and satisfaction levels 
\end{enumerate}



\subsection{Data Analysis \& Exploration}
% Here, we could start with discussing the survey group:
% - demographic observations
% - distribution of participants (based on the years experience in the industry/team), 

From the exploratory interviews, we identified sixteen key activities, which were subsequently used to quantify the developers' time allocation across their work week. 

\subsection{Limitations}

\newpage
\section{Experiment}\label{sec-experiment}
\subsection{Experimental Setup}
We briefly introduce experimental settings to verify our proposed MoR, including Datasets \& Baselines, Implementation Details, and Evaluation Metrics. More details are in Appendix~\ref{app-expr-setting}.

\textbf{Datasets \& Baselines:} We use three TG-KBs from STaRK~\cite{wu2024stark} covering three knowledge domains, including E-commerce Products (Amazon), Academic Papers (MAG), and Biomedicine (Prime). We compare our MoR with baselines established by~\citet{wu2024stark} and categorize them into textual/structural/hybrid-based ones. More recent state-of-the-art hybird retrieval approaches fro TG-KBs such as KAR~\cite{xia2024knowledge} and MFAR$^{*}$~\cite{li2024multi} are also compared.


\textbf{Implementation Details:} 
To enhance the planning capability of our planning module, we fine-tune the Llama 3.2 (3B) on 1000 sampled queries with their corresponding ground-truth planning graphs, serving as the textual graph generator. In the absence of ground-truths, we synthesize them using LLMs. For the Prime dataset, we empirically find that directly prompting LLMs can hardly generate accurate planning graphs due to the lack of biomedical domain knowledge~\cite{Shen2024TagLLMRG}. Therefore, we adopt an alternative approach. First, we instruct LLMs to extract triplets from each query and then construct the planning graphs by merging triplets with shared entities. 
During mixed traversal, textual matching can be implemented using any lexical or semantic methods. For this study, we employ BM25 for Amazon and MAG and fine-tune a contriever to complement the biomedical knowledge for Prime.
To initialize the structural traversal, we employ textual matching to locate the top 5 nodes that are most relevant to the query as seeds. Additionally, at each layer, we incorporate the top 10 nodes retrieved via textual matching and append them to the current candidate set for the next round of traversal. Notably, due to the uncertainty of LLMs, the generated planning graphs can be invalid. In this case, we will directly conduct textual matching to retrieve candidates. For our ablations without reranker, we employ Ada-002~\cite{wu2024stark} with cosine similarity as the scorer to rank candidates for evaluating performance.

\textbf{Evaluation Metrics:}
We follow~\citet{wu2024stark} for evaluation by reporting Hit@1 (H@1), Hit@5 (H@5), Recall@20 (R@20), and mean reciprocal rank MRR to evaluate in the full spectrum. 


 

\newpage
\subsection{Overall Retrieval Performance}
We compare MoR with other baselines on three TG-KBs in Table~\ref{tab-merged}. Generally, hybrid methods, AvaTAR, KAR, MFAR$^{*}$, and our MoR, achieve better performance than purely textual or structural methods owing to their ability to integrate both structural and textual knowledge. 
Among all baselines, our proposed MoR achieves the overall best performance with a substantial margin on average, with the first ranking on MAG and the second ranking on Amazon/Prime datasets. This demonstrates the effectiveness of our proposed mixture of structural and textual knowledge retrieval. 
Textual retrieval performs better on Amazon than on MAG, suggesting that Amazon queries rely more on textual knowledge. In contrast, its weaker performance on MAG is due to MAG's lower textual richness and stronger structural signals. This disparity aligns with the distribution analysis presented by~\citet{wu2024stark} and supports our hypothesis that queries in different TG-KB datasets require varying desires for textual and structural knowledge. Meanwhile, structural retrieval methods such as conventional knowledge graph-based ones perform poorly because they are designed for graphs with minimal textual information compared to TG-KBs.
Different from Amazon and MAG, all existing methods without supervised tuning (e.g., Ada-002) exhibit significantly lower performance on Prime. This is due to the extreme domain expertise required in biology, where word-count-based, pre-trained textual similarity-based, and even more powerful LLMs are all poorly applicable here. Through fine-tuning, MFAR$^{*}$ and our proposed MoR generally achieve better performance, demonstrating the necessity of domain-specific knowledge for answering queries in knowledge-intensive domains. 




\newpage
\subsection{Ablation Study}
After verifying the superiority of MoR, we conduct ablation studies to assess its different components, including module and feature ablation.

\subsubsection{Module Ablation}


To assess the contribution of each module in MoR, namely, Text Matching-based Retrieval, Neighborhood-Fetching-based Structural Retrieval, and Reranker, we conduct a series of ablation experiments. First, we remove the Reranker, resulting in the variant MoR$_{\text{w/o R}}$. On top of that, we further separately eliminate Text Retrieval and Structural Retrieval, yielding MoR$_{\text{w/o RT}}$ and MoR$_{\text{w/o RS}}$, respectively.
As shown in Table~\ref{tab-merged}, the complete MoR framework consistently achieves the highest performance across all datasets, demonstrating the synergistic effect of the Textual Retriever, Structural Retriever, and Reranker.
After removing Reranker, MoR$_{\text{w/o R}}$ exhibits a consistent performance drop across all datasets and evaluation metrics. This underscores the importance of the Reranker in refining retrieval by filtering noisy candidates from the intermediate reasoning stage. 
Eliminating Text Retrieval, i.e., MoR$_{\text{w/o RT}}$, leads to a notable performance drop on Amazon but an unexpected improvement on MAG. This suggests that while textual knowledge benefits Amazon, it introduces misleading hard negatives that compromise the ranking method (e.g., Ada-002) for MAG. Conversely, removing Structural Retrieval, MoR$_{\text{w/o RS}}$, results in a slight performance decrease further on MAG, reinforcing the importance of structural knowledge in MAG-related queries.
%
These results underscore the Reranker's crucial role in adaptively harmonizing, balancing, and selecting knowledge from both structural and textual retrieval experts.






\begin{table}[t!]
\small
\setlength\tabcolsep{4.5pt}
\centering
\begin{tabular}{l|ccc|cccc}
\toprule
\textbf{Dataset} &\textbf{TF} & \textbf{SF} & \textbf{TI} & \textbf{H@1} & \textbf{H@5} & \textbf{R@20} & \textbf{MRR} \\ \midrule
\multirow{7}{*}{\textbf{MAG}} 
& \cmark & \xmark & \xmark & 48.96 & 73.02 & 72.44 & 59.79 \\
&      \xmark            & \cmark       &         \xmark         & 18.79 & 41.91 & 52.85 & 29.84 \\
&        \xmark          &         \xmark         & \cmark       & 18.16 & 41.53 & 52.78 & 29.31 \\
\cline{2-8}
& \cmark       & \cmark       &    \xmark              & 58.04 & 77.14 & 74.42 & 66.75 \\
& \cmark       &        \xmark          & \cmark       & \underline{58.16} & \underline{77.59} & \underline{74.96} & \underline{66.85} \\
&          \xmark        & \cmark       & \cmark       & 17.93 & 38.01 & 46.79 & 27.48 \\
\cline{2-8}
& \cmark       & \cmark       & \cmark       & \textbf{58.19} & \textbf{78.34} & \textbf{75.01} & \textbf{67.14} \\ \midrule
\multirow{7}{*}{\textbf{Amazon}}    
& \cmark       &      \xmark            &       \xmark           & \underline{51.21} & \underline{74.05} & \underline{59.79} & \underline{61.27} \\
&        \xmark          & \cmark       &      \xmark            & 8.09  & 24.48 & 25.62 & 16.94 \\
&         \xmark         &      \xmark            & \cmark       & 5.84  & 16.62 & 12.94 & 11.57 \\
\cline{2-8}
& \cmark       & \cmark       &      \xmark            & 50.91 & 73.38 & 59.58 & 61.15 \\
& \cmark       &         \xmark         & \cmark       & 51.09 & 73.56 & 59.61 & 61.14 \\
&            \xmark      & \cmark       & \cmark       & 8.09  & 24.48 & 25.62 & 16.94 \\
\cline{2-8}
& \cmark       & \cmark       & \cmark       & \textbf{52.19} & \textbf{74.65} & \textbf{59.92} & \textbf{62.24} \\ \bottomrule
\end{tabular}
\caption{Ablation study investigating the importance of three features, Textual Fingerprint (\textbf{TF}), Structural Fingerprint (\textbf{SF}), and Traversal Identifier (\textbf{TI}), of the traversal trajectories used in our Structure-aware Reranker.}
\label{tab-feature-ablation}
\vspace{-2ex}
\end{table}



\subsubsection{Feature Ablation}
The above ablation study highlights the crucial role of Structure-aware Reranker in adaptively integrating structural and textual knowledge. To further analyze the contributions of its three key features, \textbf{Textual Fingerprint (TF)}, \textbf{Structural Fingerprint (SF)}, and \textbf{Traversal Identifier (TI)} defined in Section~\ref{sec-organizing}, we conduct a feature ablation analysis and report retrieval performance across different feature configurations in Table~\ref{tab-feature-ablation}.
%Overall and individual performance
Overall, using three features together yields the best performance on both MAG and Amazon, highlighting their synergistic effect. Individually, TF contributes the most and outperforms SF and TI on both datasets. 
The reason is that based on the definition in Section~\ref{sec-organizing}, TF directly captures the relevance between the query and the retrieved nodes along the trajectory, whereas SF and TI primarily characterize the structural patterns and retrieval types, serving more as complementary factors. Therefore, equipping TF with these complementary factors (i.e., SF or TI) yields around 10\% additional gains on MAG. This is because SF and TI help the reranker selectively emphasize the relevance scores given by TF for certain nodes along the path. However, this boost is not observed on Amazon. We hypothesize that the textual knowledge needed there is predominantly derived from the final node on each path, making the structural cues provided by SF and TI less beneficial and even prone to overfitting. A deeper analysis to further justify this hypothesis is in Section~\ref{sec-further}. Overall, these findings underscore the varying importance of structural features in ranking across datasets.



\begin{table}[t!]
\small
\setlength\tabcolsep{4pt}
\centering
\begin{tabular}{l|ccc|ccc}
\toprule
\multirow{2}{*}{\textbf{Feature}} & \multicolumn{3}{c|}{\textbf{MAG}} & \multicolumn{3}{c}{\textbf{Amazon}} \\

 & H@1 & R@20 & MRR & H@1 & R@20 & MRR \\
\midrule
Last Node & 49.91 & 73.49 & 59.92 & 50.36 & 59.62 & 61.05   \\
Full Path & \textbf{58.19} & \textbf{75.01} & \textbf{67.14} & \textbf{52.19} & \textbf{59.92} & \textbf{62.24}   \\
\bottomrule
\end{tabular}
\caption{Comparing reranking performance using last node in the retrieved trajectory and the whole trajectory.}
\label{tab-Reranker-ablation}
\vspace{-2ex}
\end{table}

\begin{figure}[t!]
    \centering
    \includegraphics[width=0.49\textwidth, height = 0.22\textwidth]{figures/query-pattern-20250215.png}
    \vspace{-4.5ex}
    \caption{Imbalance number of queries and performance of different retrievers across different logical structures.}
    \label{fig-analysis}
    \vspace{-3ex}
\end{figure}





\subsection{Further Analysis}\label{sec-further}
This section understands MoR’s behavior by examining three questions, each of which enriches our insight into MoR’s functionality and offers novel perspectives inspiring future query retrieval research.

\textbf{Do structure signals affect reranking?}
To assess the impact of trajectory information on the Reranker's decision-making, we introduce a node-based Reranker that constructs trajectory features using only TF/SF/TI of the last node. In Table~\ref{tab-Reranker-ablation}, the path-based Reranker outperforms the node-based variant, especially on MAG. This highlights the critical role of trajectory features/structural knowledge in reranking. The minor performance boost on Amazon after switching to the full path trajectory indicates its textual knowledge preference over the last node rather than the whole trajectory.


\textbf{How does MoR perform on different logical structures?}
Figure~\ref{fig-analysis} shows the average performance of MoR on each query group categorized by their logical structures, where "Others" refer to queries with undefined logical structures in~\citet{wu2024stark} MoR consistently outperforms structural and textual retrievers across different logical structures. Among all queries, MoR performs the worst on "P → P" queries due to the ambiguity, although well-known, uniquely caused by repeated product entities from multi-step traversal.
The average-performing ``Others" group underscores the utility of diverse planning strategies for the same query.
Lastly, the skewed query distribution and retrieval performance across planning patterns reflect the varying nature of real-world planning needs. We hope these insights inspire research on data-centric reasoning designs and error control of planning.


\begin{figure}[t!]
    \centering
    \includegraphics[width=0.5\textwidth]{figures/heatmap-20250215.pdf}
    \vspace{-3ex}
    \caption{Saliency map visualization of query attention over three entities along the retrieved paths}
    \label{fig-map}
    \vspace{-2ex}
\end{figure}

\textbf{Does MoR indeed adaptively leverage the trajectory knowledge?} To understand how our proposed reranker prioritizes candidates in the Top-K results, we visualize the saliency map by computing the gradient of ranking scores with respect to the textual fingerprint (TF) of three nodes along the traversed path, which quantifies their importance for answering a given query. Figure~\ref{fig-map} illustrates this by analyzing trajectories for 100 ground-truth candidates across 100 queries on the Amazon and MAG datasets. Each dimension corresponds to a traversed node, with the final one representing the candidate itself. 
While the saliency score is concentrated in the last dimension for Amazon, 
MAG exhibits a more evenly distributed saliency pattern, where multiple nodes along the path contribute significantly to ranking score computation. This suggests that structural knowledge is more critical for answering queries in MAG, aligning with the previously observed lower performance of purely textual retrieval on MAG in Table~\ref{tab-merged}. Further case studies explain why the reranker attends different nodes for different queries. In Figure~\ref{fig-map}(a), the reranker favors the last two dimensions as the rich textual restriction (i.e., "Northwest Company..." and "NFL Seattle...") aids in identifying the correct node at the corresponding reasoning step, as discussed in Section~\ref{sec-reasoning}. The correct nodes, having higher similarity scores with the query, help guide the retrieval process toward the ground truth.
Conversely, in Figure~\ref{fig-map}(b),
since the first node ("University of Lausanne") helps narrow the search space and the last node ("frameless...") further filter candidates, both nodes have high saliency scores. Overall, our findings demonstrate that the reranker dynamically adapts its reliance on structural and textual knowledge depending on the dataset and query. 

Software development is increasingly conceived as a collaboration activity between developers and AIs. Indeed, IDEs already implement features to enable interactive development, with AI suggesting implementations that are reused by developers.

Although multiple studies show this interaction can be successful, there is still limited understanding of how the models must be configured and used in the context of code generation tasks. This study addresses this gap, systematically investigating the impact of several key parameters, including the repeated submission of a prompt to accommodate for the non-deterministic nature of the models.

Our study reveals several key findings about the usage of ChatGPT. In particular, we discovered how creativity, although up to a limited extent, is useful to increase the range of methods whose code can be generated correctly. A major role is played by parameter top-p, which is commonly underrated, and instead has a major impact on the correctness of the results, with lower values producing better results. Finally, prompts should be submitted multiple times, with $5$ repetitions combined with a temperature of $1.2$ resulting in an effective configuration in our experiments.  

Future work concerns two main research directions. One is about replicating this experiment with other AI assistants, to validate our findings in multiple contexts. The second research direction concerns finding strategies to deal with the need to submit the same prompt multiple times to obtain a useful result, and thus developing approaches able to select or merge multiple responses automatically. 


%% The file named.bst is a bibliography style file for BibTeX 0.99c
\bibliographystyle{named}
\bibliography{ijcai24}

\end{document}

