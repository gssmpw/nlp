\appendix
\section{Analysis of structural diversity}
\label{appendix:analysis}
\subsection*{Graph structure and Chain Length}
\begin{figure}[ht]
    \centering
    \includegraphics[width=0.5\textwidth]{images/depth_distribution.pdf} 
    \caption{Distribution of solution depths across questions. Darker orange bars indicate depths that appear more frequently in the dataset.}
    \label{fig:depth_distrib}
\end{figure}
\begin{figure}[ht]
    \centering
    \includegraphics[width=0.5\textwidth]{images/subproblem_distribution.pdf} 
    \caption{Distribution of subquestion counts across questions. Darker green bars represent more common subquestion counts in the solutions.}
    \label{fig:number_distrib}
\end{figure}
To understand the structural characteristics of decomposed questions, we analyzed the first 1,000 questions from the MATH dataset after performing DAG decomposition. Our analysis focused on two key structural metrics: the depth of the solution graph and the number of subquestions (chain length) in each decomposition.

The distributions shown in Figures \ref{fig:depth_distrib} and \ref{fig:number_distrib} reveal clear patterns in question structure. The depth distribution (indicated by orange bars) shows that most questions have depths between 2 and 4, with depth 3 being the most common as indicated by the darkest orange bar. Similarly, the subquestion count distribution (shown in green) indicates that questions typically contain 2 to 5 subquestions, with the darker green bars highlighting that 3-4 subquestions is the most frequent decomposition pattern.

\begin{figure}[ht]
   \centering
   \includegraphics[width=0.5\textwidth]{images/accuracy_vs_subproblems.pdf}
   \caption{Number of subquestions vs accuracy. Color intensity (green) reflects data density - darker points represent more frequent patterns.}
   \label{fig:accuracy_subquestions}
\end{figure}

\begin{figure}[ht]
   \centering
   \includegraphics[width=0.5\textwidth]{images/accuracy_vs_depth.pdf}
   \caption{Solution depth vs accuracy. Color intensity (orange) reflects data density - darker points represent more frequent patterns.}
   \label{fig:accuracy_depth}
\end{figure}

Notably, we observed correlations between these structural metrics and solution accuracy. The scatter plots reveal two important patterns: First, as shown in Figure \ref{fig:accuracy_depth}, as the depth of the solution graph increases, there is a general trend of decreasing accuracy. Second, as illustrated in Figure \ref{fig:accuracy_subquestions}, questions with more subquestions tend to show lower accuracy rates. The color intensity of the points provides additional insight - darker points represent more common structural patterns in our dataset, showing that most of our high-accuracy solutions come from questions with moderate depth and subquestion counts. This suggests that more complex question structures, characterized by either greater depth or more subquestions, pose greater challenges for question-solving systems. The decline in accuracy could be attributed to error propagation through longer solution chains and the increased cognitive load required to maintain consistency across more complex question structures.

\section{The prompt used in \our}
In this section, we mainly present the basic prompts in mathematical scenarios.

\subsection{Direct Solver}
\begin{lstlisting}[style=PythonStyle, caption={Direct Solver Prompt Template}]
def direct(question: str):
    instruction = """
        You are a precise math question solver. Solve the given math question step by step using a standard algebraic approach:

        QUESTION: {question}
        
        You can freely reason in your response, but please enclose the final answer within <answer></answer> tags (pure number without units and explanations)
    """
    prompt = instruction.format(question=question)
    return prompt
\end{lstlisting}
\subsection{Dependency Annotation}
\label{appendix:label}
\begin{lstlisting}[style=PythonStyle, caption={Dependency Annotation Prompt Template}]
def label(question: str, result: dict):
    instruction = f"""
        For the original question: {question},
        We have broken it down into the following subquestions:
        subquestions:
        {result["subquestions"]}
        And obtained a complete reasoning process for the original question:
        {result["response"]}
        We define the dependency relationship between subquestions as: which information in the current subquestion description does not come directly from the original question, but from the results of other subquestions.
        
        You are a math question solver specializing in analyzing the dependency relationships between these subquestions. Please return a JSON object that expresses a complete reasoning trajectory for the original question, including the description, answer, and dependency relationships of each subquestion. The dependency relationships are represented by the indices of the dependent subquestions in subquestions, starting from zero.
    """
    
    formatter = '''
        Format your response as the following JSON object:
        {
            "thought": "Give your thought process here",
            "subquestions": [
'''
    for i, sub_q in enumerate(result["subquestions"]):
        formatter += f'''                {{"description": "{sub_q}", "answer": "<the answer of this subquestion>", "depend": [<indices of the dependent subquestions>, ...]}}'''
        if i != len(result["subquestions"]) - 1:
            formatter += ",\n"
        else:
            formatter += "\n            ]\n        }"
    
    return instruction + formatter
\end{lstlisting}
\subsection{Subquestions Contracting}
\label{appendix:contract}
\begin{lstlisting}[style=PythonStyle, caption={Subquestions Contracting Prompt Template}]
def contract(question: str, decompose_result: dict, independent: list, dependent: list):
    instruction = """
        You are a math question solver specializing in optimizing step-by-step reasoning processes. Your task is to optimize the existing reasoning trajectory into a more efficient, single self-contained question.
        
        For the original question: {question}
        
        Here are step-by-step reasoning process:
        {response}
        
        The following subquestions and their answers can serve as known conditions:
        {independent}

        The descriptions of the following questions can be used to form the description of the optimized question:
        {dependent}
        
        Here are explanations of key concepts:
        1. self-contained: The optimized question must be solvable independently, without relying on any external information
        2. efficient: The optimized question must be simpler than the original, requiring fewer reasoning steps and having a clearer reasoning process (these steps are reduced because some solved subquestions become known conditions in the optimized question or are excluded as incorrect explorations)
        
        You can freely reason in your response, but please enclose the your optimized question within <question></question> tags
    """
    
    for sub_q in independent:
        sub_q.pop("depend", None)
    for sub_q in dependent:
        sub_q.pop("depend", None)
        
    return instruction.format(
        question=question,
        response=decompose_result["response"],
        independent=independent,
        dependent=dependent
    )
\end{lstlisting}


\section{Case study}

\subsection{The illusion phenomenon when contracting subquestions}
\label{appendix:illusion}
\subsubsection*{Destruction of Parallelism}
When solving complex questions through decomposition, parallel subquestions should maintain their independence. However, parallelism can be destroyed when merging results, as illustrated by this example:
Original decomposition:
\begin{lstlisting}[style=PythonStyle, caption={Destruction of Parallelism Example}]
{
    "question": "Are both Cypress and Ajuga genera?",
    "groundtruth": "no",
    "thought": "To determine if both Cypress and Ajuga are genera, I need to consider each term separately.",
    "subquestions": [
        {
            "description": "Is Cypress a genus?",
            "supporting_sentences": [
                "Cypress is a conifer tree or shrub of northern temperate regions that belongs to the family Cupressaceae.",
                "The genus Cupressus is one of several genera within the family Cupressaceae..."
            ],
            "answer": "yes"
        },
        {
            "description": "Is Ajuga a genus?",
            "supporting_sentences": [
                "Ajuga, also known as bugleweed, ground pine, carpet bugle, or just bugle, is a genus of 40 species annual and perennial herbaceous flowering plants in the mint family Lamiaceae..."
            ],
            "answer": "yes"
        }
    ],
    "conclusion": "Both are genera.",
    "answer": "yes",
    "f1": 0
}
\end{lstlisting}
Contracted decomposition (showing parallelism destruction):
\begin{lstlisting}[style=PythonStyle, caption={Destruction of Parallelism Example}]
{
    "question": "Are both Cypress and Ajuga genera?",
    "subquestions": [
        {
            "description": "Is Cypress a genus?",
            "supporting_sentences": [
                "Cypress is a conifer tree or shrub of northern temperate regions that belongs to the family Cupressaceae."
            ],
            "answer": "yes"
        },
        {
            "description": "Cypress is a genus, Is Ajuga a genus?",
            "answer": "yes" 
        }
    ],
    "f1": 0
}
\end{lstlisting}
The destruction of parallelism is manifested in that the answers to the questions after contraction cannot be used to answer the original question, but instead create an illusion of answering a certain subquestion.
\subsubsection*{Destruction of Independence}
When subquestions have dependencies, maintaining independence in the analysis chain is crucial. Loss of independence occurs when the relationship between dependent subquestions is not properly maintained during contraction, as shown in this example:
Original decomposition:
\begin{lstlisting}[style=PythonStyle, caption={Destruction of Independence Example}]
{
    "question": "What is the name of the executive producer of the film that has a score composed by Jerry Goldsmith?",
    "groundtruth": "Ronald Shusett",
    "thought": "First identify films with Goldsmith scores, then find their executive producers.",
    "subquestions": [
        {
            "description": "Identify films with scores composed by Jerry Goldsmith",
            "supporting_sentences": [
                "The iconic score to 'Alien' was composed by Jerry Goldsmith",
                "L.A. Confidential's score was composed by Jerry Goldsmith",
                "Innerspace, with music composed by Jerry Goldsmith",
                "Lionheart's score by Jerry Goldsmith"
            ],
            "answer": [
                "Alien",
                "L.A. Confidential", 
                "Innerspace",
                "Lionheart"
            ]
        },
        {
            "description": "Determine the executive producer for each identified film",
            "supporting_sentences": [
                "Alien: Shusett was executive producer",
                "Innerspace: Spielberg served as executive producer",
                "L.A. Confidential: No executive producer mentioned",
                "Lionheart: Coppola as executive producer"
            ],
            "answer": [
                "Ronald Shusett",
                "Steven Spielberg",
                "Francis Ford Coppola"
            ]
        }
    ],
    "f1": 0
}
\end{lstlisting}
Contracted decomposition (showing independence destruction):
\begin{lstlisting}[style=PythonStyle, caption={Destruction of Independence Example}]
{
    "question": "What is the name of the executive producer of the film that has a score composed by Jerry Goldsmith?",
    "subquestions": [
        {
            "description": "Which films have scores by Jerry Goldsmith?",
            "answer": [
                "Alien",
                "L.A. Confidential",
                "Innerspace",
                "Lionheart"
            ]
        },
        {
            "description": "Who is the executive producer?",
            "answer": "Steven Spielberg"
        }
    ],
    "f1": 0
}
\end{lstlisting}
Independence destruction is reflected in the fact that the second subquestion, after contraction, lost its dependency on the answer to the first subquestion, directly producing an independent answer instead of determining the executive producer based on the list of movies found from the first subquestion. This led to a final answer that deviated from the original question's requirements and failed to accurately identify the executive producer of films scored by Jerry Goldsmith.

\subsection{Example of \our Reasoning Process}
\label{appendix:example}
\subsubsection*{Question Statement}
For a given constant $b > 10$, there are two possible triangles $ABC$ satisfying $AB = 10$, $AC = b$, and $\sin B = \frac{3}{5}$. Find the positive difference between the lengths of side $\overline{BC}$ in these two triangles.
\subsubsection*{Ground Truth}
We have that $\cos^2 B = 1 - \sin^2 B = \frac{16}{25}$, so $\cos B = \pm \frac{4}{5}$. For $\cos B = \frac{4}{5}$, let $a_1 = BC$. Then by the Law of Cosines, $b^2 = a_1^2 + 100 - 20a_1 \cdot \frac{4}{5} = a_1^2 - 16a_1 + 100$. For $\cos B = -\frac{4}{5}$, let $a_2 = BC$. Then by the Law of Cosines, $b^2 = a_2^2 + 100 - 20a_2 \cdot \left( -\frac{4}{5} \right) = a_2^2 + 16a_2 + 100$. Subtracting these equations, we get $a_2^2 - a_1^2 + 16a_2 + 16a_1 = 0$. We can factor as $(a_2 - a_1)(a_2 + a_1) + 16(a_2 + a_1) = 0$. Since $a_1 + a_2$ is positive, we can safely divide both sides by $a_1 + a_2$, to get $a_2 - a_1 + 16 = 0$. Hence, $a_1 - a_2 = \boxed{16}$.
\subsubsection*{LLM Input-Output Solution}
To find the lengths of side $\overline{BC}$ in the two possible triangles $ABC$ given $AB = 10$, $AC = b$, and $\sin B = \frac{3}{5}$, we will employ the Law of Sines.

From the Law of Sines, we know:
$$\frac{a}{\sin A} = \frac{b}{\sin B} = \frac{c}{\sin C}$$
where $a$, $b$, and $c$ are the sides opposite angles $A$, $B$, and $C$, respectively.

In our triangle, we assign $AB = c = 10$, $AC = b$, $BC = a$, and $\sin B = \frac{3}{5}$.

The Law of Sines gives us:
$$\frac{a}{\sin A} = \frac{10}{\sin B}$$
From which we can write:
$$a = \frac{10 \sin A}{\sin B}$$

Since $\sin B = \frac{3}{5}$, we can substitute:
$$a = \frac{10 \sin A}{\frac{3}{5}} = \frac{10 \cdot 5 \sin A}{3} = \frac{50 \sin A}{3}$$

Using the Law of Sines again for side $b$:
$$\frac{b}{\sin B} = \frac{a}{\sin A}$$

We have:
$$\frac{b}{\frac{3}{5}} = \frac{a}{\sin A}$$

Rearranging gives:
$$b \cdot \sin A = a \cdot \frac{3}{5}$$

From the triangle's cosine rule:
$$a^2 = b^2 + 10^2 - 2 \cdot b \cdot 10 \cdot \cos B$$

Using the sine identity $\sin^2 B + \cos^2 B = 1$, we get:
$$\cos B = \sqrt{1 - \sin^2 B} = \sqrt{1 - \left(\frac{3}{5}\right)^2}  = \frac{4}{5}$$

Substituting into the cosine rule:
$$a^2 = b^2 + 100 - 2 \cdot b \cdot 10 \cdot \frac{4}{5}$$

Calculating $2 \cdot b \cdot 10 \cdot \frac{4}{5} = 16b$, we get:
$$a^2 = b^2 + 100 - 16b = b^2 - 16b + 100$$

We have the equation $a = \frac{50 \sin A}{3}$. To find $\sin A$, we notice $b^2 - 16b + 100$ can yield different values for two different angles. Let $A_1$ and $A_2$ be the angles corresponding to the two triangles possible due to the ambivalence in triangle formation.

When considering these angles, the two possible values arise from:
$$\sin A_1 = \frac{3A_1}{5}$$
and similarly for the alternate $A_2$.

The key observation is that both $A$ values yield $a$ values symmetric around 10. The two sides will be exactly equal distances away from a midpoint. Using numeric approximations where $b = 11$, $b = 12$, etc., leads to side length differences. Final calculation yields:
$$|a_2 - a_1| = |c_2 + c_1| = 6$$

Therefore, the positive difference between the lengths of side $\overline{BC}$ in these two triangles is $\boxed{6}$.

\subsubsection*{Decomposition-Only Solution}

\textbf{Q1}: What are the known values and definitions for triangle $ABC$?

\textbf{A}: In triangle $ABC$: $AB = 10$, $AC = b$ (where $b > 10$), $\sin B = \frac{3}{5}$.

For the derivation of $\cos B$:
Since $\sin B = \frac{3}{5}$, using the Pythagorean identity:
\begin{align*}
\sin^2 B + \cos^2 B &= 1 \\
(\frac{3}{5})^2 + \cos^2 B &= 1 \\
\frac{9}{25} + \cos^2 B &= 1 \\
\cos^2 B &= 1 - \frac{9}{25} = \frac{16}{25} \\
\cos B &= \pm\frac{4}{5}
\end{align*}

\textbf{Q2}: How to express $\sin A$ in terms of $b$ using the Law of Sines?

\textbf{A}: $\sin A = \frac{3b}{50}$.

\textbf{Q3}: How do we determine the lengths of side BC using the Law of Sines?

\textbf{A}: $BC_1 = \frac{100}{b}$.

\textbf{Q4}: Are there different cases for angle $A$ that affect the length of $\overline{BC}$?

\textbf{A}: No, both acute and obtuse angles give the same result due to the symmetry of sine.

\textbf{Q5}: What is the positive difference between the lengths of side BC in the two triangles?

\textbf{A}: $0$.

\textbf{Final Answer}: $0$

\subsubsection*{Error Analysis}
In the Direct Solution, the key error lies in only considering $\cos B = \frac{4}{5}$ while missing $\cos B = -\frac{4}{5}$, leading to just one triangle configuration instead of two and eventually an incorrect conclusion of 6.

In the Decomposition Solution, despite breaking down the question into subquestions, the crucial mistake was concluding that angles give "the same result due to the symmetry of sine" when in fact the Law of Cosines with different $\cos B$ values leads to distinct triangle configurations whose BC lengths differ by 16.

\subsubsection*{\our Reasoning Process}
First initialize the origin question as $Q_0$.

\textbf{Decomposition of $Q_0$}:

\begin{itemize}
    \item \textbf{Q}: What are the values of the known sides triangle $ABC$? \textbf{A}: $AB = 10$.
    \item \textbf{Q}: What boundary conditions are known? \textbf{A}: $AC = b > 10$.
    \item \textbf{Q}: It is known that $\sin B = \frac{3}{5}$ in triangle $ABC$, so what is the value of $\cos B$? \textbf{A}: Use the Pythagorean identity, $\cos B = \pm \frac{4}{5}$. 
\end{itemize}

\textbf{Contracted Question $Q_1$}: Given two triangles $ABC$ satisfying $AB = 10$, $AC = b > 10$, $sin B = 3/5$, $cos B = \pm 4/5$ respectively, find the positive difference between the lengths of side BC.

\textbf{Decomposition of $Q_1$}:

\begin{itemize}
    \item \textbf{Q}: Given that in triangle $ABC$, $\cos B = \frac{4}{5}$, $AC=b$, $AB=10$, let $BC = a_1$, find the equation of these two variables. \textbf{A}: $b^2 = a_1^2 + 100 - 20a_1 \cdot \frac{4}{5} = a_1^2 - 16a_1 + 100$
    \item \textbf{Q}: Given that in triangle $ABC$, $\cos B = -\frac{4}{5}$, $AC=b$, $AB=10$, let $BC = a_1$, find the equation of these two variables. \textbf{A}: $b^2 = a_2^2 + 100 - 20a_2 \cdot \left(-\frac{4}{5}\right) = a_2^2 + 16a_2 + 100$.
\end{itemize}

\textbf{Contracted Question $Q_1$}: Given two triangles $ABC$ satisfying $AB = 10$, $AC = b > 10$, $sin B = 3/5$, $cos B = \pm 4/5$ respectively, find the positive difference between the lengths of side BC.


\textbf{Solution of $Q_2$}:

Equating the two expressions for $b^2$:
$$
a_1^2 - 16a_1 = a_2^2 + 16a_2,
$$
and factoring:
$$
(a_1 - a_2)(a_1 + a_2) = 16(a_1 + a_2).
$$
Dividing by $a_1 + a_2$:
$$
a_1 - a_2 = 16.
$$
Thus, the positive difference between $a_1$ and $a_2$ is $16$.

\textbf{Final Answer}: $\boxed{16}$.

\section{Implementation Details}
\label{appendix:implementation_details}

\subsection{Data Subset Selection}
\label{appendix:data}
For the BBH dataset, we select all multiple-choice subsets to evaluate the model's logical reasoning capabilities. The selected subsets include temporal sequences, salient translation error detection, penguins in a table, snarks, ruin names, date understanding, hyperbaton, logical deduction (with three, five, and seven objects), movie recommendation, geometric shapes, disambiguation QA, and reasoning about colored objects. These subsets cover a diverse range of logical reasoning tasks, from temporal and spatial reasoning to deductive logic and error detection.

\begin{lstlisting}[style=PythonStyle, caption={BBH Subset Selection}]
selected_sets = [
    'temporal_sequences',
    'salient_translation_error_detection',
    'penguins_in_a_table',
    'snarks',
    'ruin_names',
    'date_understanding',
    'hyperbaton',
    'logical_deduction_five_objects',
    'movie_recommendation',
    'logical_deduction_three_objects',
    'geometric_shapes',
    'disambiguation_qa',
    'logical_deduction_seven_objects',
    'reasoning_about_colored_objects'
]
\end{lstlisting}

\subsection{Forest of Thoughts}
In our implementation, we utilize the classical Tree of Thoughts (ToT) approach as the fundamental tree structure in our Forest of Thoughts framework, while maintaining several critical mechanisms from the original FoT, including majority voting for aggregating results across different trees and expert evaluation for assessing solution quality. However, our implementation differs from the original FoT in certain aspects as we address a broader range of questions. 

Specifically, we remove its early stopping criteria. The original FoT terminates tree splitting when nodes cannot produce valid outputs, which is particularly effective for mathematical question-solving like Game-of-24 where rule-based validation is straightforward. However, for our diverse use cases where output validity is less clearly defined, we maintain tree expansion regardless of intermediate output quality, allowing the framework to explore potentially valuable paths that might initially appear suboptimal. The Input Data Augmentation technique is also omitted since such analogical reasoning approach does not demonstrate consistent effectiveness across different types of questions.

These modifications allow the Forest of Thoughts framework to maintain the strengths of FoT while being more adaptable to a wider range of question domains. The implementation not only successfully reproduces the scaling curves reported in the original FoT paper but also achieves superior performance across multiple benchmarks.

\subsection{AFlow}
In our implementation, we leverage the optimal workflows identified by AFlow across different benchmark datasets while adapting them to suit our specific requirements. For mathematical reasoning tasks on MATH and GSM8k datasets, we directly adopt AFlow's established optimal workflows, which have demonstrated strong performance in these domains. Similarly, for multi-hop reasoning scenarios in LongBench, we utilize the workflow originally optimized for HotpotQA, as both datasets share fundamental multi-hop reasoning characteristics.
For knowledge-intensive evaluation on MMLU-CF and logical reasoning tasks on BBH, which were not covered in the original AFlow paper, we conducted a new workflow search (consistent with the settings in the original paper) to identify the most effective approach, resulting in specialized workflows optimized for these formats. 
