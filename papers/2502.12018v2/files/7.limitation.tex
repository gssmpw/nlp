\section{Limitations}
A key limitation of \our lies in its Markov state transition process without a well-designed reflection mechanism. When the initial DAG decomposition fails to properly model parallel relationships between subquestions or captures unnecessary dependencies, it can negatively impact subsequent contraction and reasoning process, a scenario that occurs frequently in practice. The framework currently lacks the ability to detect and rectify such poor decompositions, potentially leading to compounded errors in the atomic state transitions. This limitation suggests the need for future research into incorporating effective reflection and adjustment mechanisms to improve the robustness of DAG-based decomposition.

\section{Ethics Statement}
While this work advances the computational efficiency and test-time scaling capabilities of language models through the \our framework, we acknowledge that these models process information and conduct reasoning in ways fundamentally different from human cognition. Making direct comparisons between our Markov reasoning process and human thought patterns could be misleading and potentially harmful. The atomic state representation and dependency-based decomposition proposed in this research are computational constructs designed to optimize machine reasoning, rather than models of human cognitive processes. Our work merely aims to explore more efficient ways of structuring machine reasoning through reduced computational resources and simplified state transitions, while recognizing the distinct nature of artificial and human intelligence. We encourage users of this technology to be mindful of these limitations and to implement appropriate safeguards when deploying systems based on our framework.