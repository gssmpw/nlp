\clearpage
\begin{figure*}[h!]
  \centering
    \includegraphics[width=\linewidth]{figures/src/tao_fail.pdf}
  \caption{\textbf{Getting-up from prone pose result visualization of \citet{Learning2GetUp22}}.
  The motion generated by method~\cite{Learning2GetUp22} is highly unstable and unsafe, and it keeps jittering and jumping during the getting-up phase.
  }\label{fig:tao_case}
\end{figure*}
\appendix
\section{Material Statements}
In our supplementary material, we provide a video demo presenting \ours, and this PDF for supplementary information including our reward designs, training details, and additional qualitative analysis.

\section*{Rewards}

\subsubsection{Rewards components in Stage I} 

Detailed reward components used in Stage I are summarized in \cref{tab:reward_stage1}.

\begin{table}[t]
\setlength{\abovecaptionskip}{0.cm}
\setlength{\belowcaptionskip}{-0.cm}
\caption{\small \textbf{Reward definitions used in \our.}}
\resizebox{\linewidth}{!}{
\centering
\label{tab:rewards}
\begin{tabular}{l|l|l}
\toprule
\multicolumn{1}{c|}{Term} & \multicolumn{1}{c|}{Definition}                                                                                                                                                               & \multicolumn{1}{c}{Weight} \\ \midrule
\multicolumn{3}{c}{Task Reward}                                                                                                                                                                                                                    \\ \midrule
Linear Velocity Tracking    & $\exp\left ( -\left \| v_{\text{xy}}^\text{target}-v_{\text{xy}} \right \|^2 / 0.2 \right) $                                                                                                                         & 2                        \\
Angular Velocity Tracking    & $\exp\left ( -\left \| \omega_{\text{z}}^\text{target}-\omega_{\text{z}} \right \|^2 / 0.2 \right) $                                                                                                                 & 2                        \\ \midrule
\multicolumn{3}{c}{Behavior Reward}                                                                                                                                                                                                                \\ \midrule
Body Height Tracking     & $\left \| h^\text{target}-h \right \|^2$                                                                                                                                                     & -40                        \\
Body Pitch Tracking      & $\left \| p^\text{target}-p \right \|^2$                                                                                                                                                     & -10                        \\
Waist Yaw Tracking       & $\left \| w^\text{target}-w \right \|^2$                                                                                                                                                     & -2                         \\
Foot Swing Tracking & $\sum_i [1-C(\phi_i)]\left \| l^{\text{target},i} - l^{\text{foot},i} \right \|^2$                                                                                                                       & -30                        \\
Contact-Swing Tracking   & \makecell[l]{$
-\sum_i [1-C(\phi_i)]\left[1-\exp(\left \|  f^{\text{foot}, i} \right \|^2 / 50)\right]$ \\
$~~~~~~~~~~~ -C(\phi_i)\left[1-\exp\left(\| v^{\text{foot}, i}_{xy} \|^2 / 5\right)\right]$} & -2                         \\ \midrule
\multicolumn{3}{c}{Regularization Reward}                                                                                                                                                                                                                 \\ \midrule
R-P Angular Velocity      & $\left \| \omega_{\text{xy}} \right \|^2$                                                                                                                                                       & -0.5                       \\
Vertical Body Movement            & $\left \| v_{\text{z}} \right \|^2$                                                                                                                                                           & -0.1                       \\
Feet Slip                & $1- \sum_i \exp\left(-\left \| v_{\text{xy}}^\text{foot,i} \right \|^2\right)$                                                                                                                                           & -0.2                       \\
Action Rate              & $\left \| a_t - a_{t-1} \right \|^2$                                                                                                                                                       & -0.01                      \\
Action Smoothness        & $\left \| a_{t-2}-2a_{t-1}+a_t \right \|^2$                                                                                                                                                & -0.01                      \\
Joint Torque             & $\left \| \tau \right \|^2$                                                                                                                                                            & -5e-6                      \\
Joint Acceleration       & $\left \| \ddot{q} \right \|^2$                                                                                                                                                            & -2.5e-7                    \\
Upper Joint Deviation    & $\left \| q_\text{upper} - q^\text{nominal}_{\text{upper}} \right \|^2$                                                                                                                                             & -0.5                       \\
Hip Joint Deviation      & $\left \| q_{\text{hip},\text{xz}} - q^\text{nominal}_{\text{hip}, \text{xz}} \right \|^2$             
& -2                         \\
Feet Symmetry         & $\mathds{1} [\bar \phi_1 = \bar \phi_2] \left \| p_\text{foot,0}^{\text{xz}}-p_\text{foot,1}^{\text{xz}} \right \|^2$                                                                                                        & -5                        \\
Termination & $\mathds{1} [\text{Early Terminate}]$ & -200 \\ \bottomrule
\end{tabular}}
\vspace{-6pt}
\end{table}

\subsubsection{Rewards components in Stage II} 
Detailed reward components used in Stage II are summarized in \cref{tab:reward_stage2}.

\begin{table}[ht!]
\caption{\textbf{Reward components and weights in Stage II.} Penalty rewards prevent undesired behaviors for sim-to-real transfer, regularization refines motion, and task rewards ensure successful whole-body tracking in real time.}
\label{tab:reward_stage2}
\centering
\renewcommand{\arraystretch}{1.3} % Adjust row spacing
\resizebox{\linewidth}{!}{
\begin{tabular}{l c c}
\toprule[0.95pt]
{\scshape Term} & {\scshape Expression} & {\scshape Weight} \\
\midrule[0.6pt]
\multicolumn{3}{l}{\textit{\textbf{Penalty:}}} \\
\midrule[0.6pt]
Torque limits & $ \mathds{1}({\torque \notin [\bs{\tau}_{\min}, \bs{\tau}_{\max} ]}) $ & -5 \\
Ankle torque limits & $ \mathds{1}({\torque^{\text{ankle}} \notin [\bs{\tau}^{\text{ankle}}_{\min}, \bs{\tau}^{\text{ankle}}_{\max} ]}) $ & -0.01 \\
Upper torque limits & $ \mathds{1}({\torque^{\text{upper}} \notin [\bs{\tau}^{\text{upper}}_{\min}, \bs{\tau}^{\text{upper}}_{\max} ]}) $ & -0.01 \\
DoF position limits & $ \mathds{1}({\dofpos \notin [\bs{q}_{\min}, \bs{q}_{\max} ]}) $ & -5 \\
Ankle DoF position limits & $ \mathds{1}({\dofpos^{\text{ankle}} \notin [\bs{q}^{\text{ankle}}_{\min}, \bs{q}^{\text{ankle}}_{\max} ]}) $ & -5 \\
Upper DoF position limits & $ \mathds{1}({\dofpos^{\text{upper}} \notin [\bs{q}^{\text{upper}}_{\min}, \bs{q}^{\text{upper}}_{\max} ]}) $ & -5 \\
Energy & $ \lVert\ \boldsymbol{\tau} \odot \dot{\mathbf{q}} \rVert $ & -1e-4 \\
Termination & $\mathds{1}_\text{termination}$ & -50 \\
\midrule[0.6pt]
\multicolumn{3}{l}{\textit{\textbf{Regularization:}}} \\
\midrule[0.6pt]
DoF acceleration & $\lVert \dofacc \rVert_2$ & -1e-7 \\
DoF velocity & $\lVert \dofvel \rVert_2^2$ & -1e-3 \\
Action rate & $ \lVert \bs{a}_t \rVert_2^2 $ & -0.1 \\
Torque & $\lVert\torque\rVert$ & -0.003 \\
Ankle torque & $\lVert\torque^{\text{ankle}}\rVert$ & -6e-7 \\
Upper torque & $\lVert\torque^{\text{upper}}\rVert$ & -6e-7 \\
Angular velocity & $\lVert\omega^2\rVert$ & -0.1 \\
Base velocity & $\lVert \bs{v}^2\rVert$ & -0.1 \\
\midrule[0.6pt]
\multicolumn{3}{l}{\textit{\textbf{Tracking Rewards:}}} \\
\midrule[0.6pt]
Tracking DoF position & $ \exp \left( - \frac{(\dofpos - \dofpos^{\text{target}})^2}{4} \right) $ & 8 \\
Feet distance reward & \makecell{
$\frac{1}{2} \Big( \exp(-100 \left| \max(\bs{d}_{\text{feet}} - \bs{d}_{\min}, -0.5) \right|)$ \\ 
$+ \exp(-100 \left| \max(\bs{d}_{feet} - \bs{d}_{\max}, 0) \right|) \Big)$
} & 2 \\
Foot orientation & $ \sqrt{\lVert \mathbf{g}_{xy}^{\text{feet}} \rVert} $ & -0.5 \\
\bottomrule[0.95pt]
\end{tabular}
}
\end{table}



\section*{Training Details}
\ours contains two-stage learning, which are both trained in simulation.
In Stage I, we train the discovery policy $f$ for overall 5B simulation steps in total, and 20K simulation steps for Stage II deployable policy $\pi$.
All training is conducted on IsaacGym~\cite{IsaacGym21}, and we train our policies using 4,096 paralleled environments on a single NVIDIA RTX 4090 or L40S GPU.
For the getting-up task, we slow down the discovered trajectory to 8 seconds.
For the rolling-over task, the trajectory is slowed down to 4 seconds.
We use flat terrains in Stage I and varied terrains during Stage II training, involving flat, rough, and slopes.
We follow previous works~\cite{OmniH2O24,ExtremeParkour24,HumanoidParkour24} to apply varied dynamics randomization such as friction, base CoM offset, and control delay.



\section*{Additional Qualitative Simulation Baseline Results}
\cref{fig:tao_case} showcases a visualization of getting up from a prone pose generated by baseline method~\cite{Learning2GetUp22}.
It can be observed that this method generates motion that is highly unstable and unsafe to deploy in the real world.
For example, its joints continuously jitter, the feet are stumbling and the body is keep jumping up.
This indicates that this baseline~\cite{Learning2GetUp22} cannot be Sim2Real.
