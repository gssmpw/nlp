\section{Related Work}
We review related works on humanoid control, learning for humanoid control, and work specifically targeted toward fall recovery for legged robots.

\subsection{Humanoid Control}
Controlling a high degree of freedom humanoid robots have fascinated researchers for the last several decades. 
Model-based techniques, such as those based on the Zero Moment Point (ZMP) principle~\cite{ZeroMoment04,HondaHumanoid98,ASIMO02,WalkMan17}, optimization~\cite{OptimizationBasedLocomotion16,bookwalkingrunning2023,BipedalRunning23}, and Model Predictive Control (MPC)~\cite{MITHumanoid21,galliker2022planar,DynamicLocomotionMITConvexMPC18,FullOrderSamplingBasedMPC24}, have demonstrated remarkable success in fundamental locomotion tasks like walking, running and jumping.
However, these approaches often struggle to generalize or adapt to novel environments.
In contrast, learning-based approaches have recently made significant strides, continuously expanding the generalization capabilities of humanoid locomotion controllers.
\section{Overview}

\revision{In this section, we first explain the foundational concept of Hausdorff distance-based penetration depth algorithms, which are essential for understanding our method (Sec.~\ref{sec:preliminary}).
We then provide a brief overview of our proposed RT-based penetration depth algorithm (Sec.~\ref{subsec:algo_overview}).}



\section{Preliminaries }
\label{sec:Preliminaries}

% Before we introduce our method, we first overview the important basics of 3D dynamic human modeling with Gaussian splatting. Then, we discuss the diffusion-based 3d generation techniques, and how they can be applied to human modeling.
% \ZY{I stopp here. TBC.}
% \subsection{Dynamic human modeling with Gaussian splatting}
\subsection{3D Gaussian Splatting}
3D Gaussian splatting~\cite{kerbl3Dgaussians} is an explicit scene representation that allows high-quality real-time rendering. The given scene is represented by a set of static 3D Gaussians, which are parameterized as follows: Gaussian center $x\in {\mathbb{R}^3}$, color $c\in {\mathbb{R}^3}$, opacity $\alpha\in {\mathbb{R}}$, spatial rotation in the form of quaternion $q\in {\mathbb{R}^4}$, and scaling factor $s\in {\mathbb{R}^3}$. Given these properties, the rendering process is represented as:
\begin{equation}
  I = Splatting(x, c, s, \alpha, q, r),
  \label{eq:splattingGA}
\end{equation}
where $I$ is the rendered image, $r$ is a set of query rays crossing the scene, and $Splatting(\cdot)$ is a differentiable rendering process. We refer readers to Kerbl et al.'s paper~\cite{kerbl3Dgaussians} for the details of Gaussian splatting. 



% \ZY{I would suggest move this part to the method part.}
% GaissianAvatar is a dynamic human generation model based on Gaussian splitting. Given a sequence of RGB images, this method utilizes fitted SMPLs and sampled points on its surface to obtain a pose-dependent feature map by a pose encoder. The pose-dependent features and a geometry feature are fed in a Gaussian decoder, which is employed to establish a functional mapping from the underlying geometry of the human form to diverse attributes of 3D Gaussians on the canonical surfaces. The parameter prediction process is articulated as follows:
% \begin{equation}
%   (\Delta x,c,s)=G_{\theta}(S+P),
%   \label{eq:gaussiandecoder}
% \end{equation}
%  where $G_{\theta}$ represents the Gaussian decoder, and $(S+P)$ is the multiplication of geometry feature S and pose feature P. Instead of optimizing all attributes of Gaussian, this decoder predicts 3D positional offset $\Delta{x} \in {\mathbb{R}^3}$, color $c\in\mathbb{R}^3$, and 3D scaling factor $ s\in\mathbb{R}^3$. To enhance geometry reconstruction accuracy, the opacity $\alpha$ and 3D rotation $q$ are set to fixed values of $1$ and $(1,0,0,0)$ respectively.
 
%  To render the canonical avatar in observation space, we seamlessly combine the Linear Blend Skinning function with the Gaussian Splatting~\cite{kerbl3Dgaussians} rendering process: 
% \begin{equation}
%   I_{\theta}=Splatting(x_o,Q,d),
%   \label{eq:splatting}
% \end{equation}
% \begin{equation}
%   x_o = T_{lbs}(x_c,p,w),
%   \label{eq:LBS}
% \end{equation}
% where $I_{\theta}$ represents the final rendered image, and the canonical Gaussian position $x_c$ is the sum of the initial position $x$ and the predicted offset $\Delta x$. The LBS function $T_{lbs}$ applies the SMPL skeleton pose $p$ and blending weights $w$ to deform $x_c$ into observation space as $x_o$. $Q$ denotes the remaining attributes of the Gaussians. With the rendering process, they can now reposition these canonical 3D Gaussians into the observation space.



\subsection{Score Distillation Sampling}
Score Distillation Sampling (SDS)~\cite{poole2022dreamfusion} builds a bridge between diffusion models and 3D representations. In SDS, the noised input is denoised in one time-step, and the difference between added noise and predicted noise is considered SDS loss, expressed as:

% \begin{equation}
%   \mathcal{L}_{SDS}(I_{\Phi}) \triangleq E_{t,\epsilon}[w(t)(\epsilon_{\phi}(z_t,y,t)-\epsilon)\frac{\partial I_{\Phi}}{\partial\Phi}],
%   \label{eq:SDSObserv}
% \end{equation}
\begin{equation}
    \mathcal{L}_{\text{SDS}}(I_{\Phi}) \triangleq \mathbb{E}_{t,\epsilon} \left[ w(t) \left( \epsilon_{\phi}(z_t, y, t) - \epsilon \right) \frac{\partial I_{\Phi}}{\partial \Phi} \right],
  \label{eq:SDSObservGA}
\end{equation}
where the input $I_{\Phi}$ represents a rendered image from a 3D representation, such as 3D Gaussians, with optimizable parameters $\Phi$. $\epsilon_{\phi}$ corresponds to the predicted noise of diffusion networks, which is produced by incorporating the noise image $z_t$ as input and conditioning it with a text or image $y$ at timestep $t$. The noise image $z_t$ is derived by introducing noise $\epsilon$ into $I_{\Phi}$ at timestep $t$. The loss is weighted by the diffusion scheduler $w(t)$. 
% \vspace{-3mm}

\subsection{Overview of the RTPD Algorithm}\label{subsec:algo_overview}
Fig.~\ref{fig:Overview} presents an overview of our RTPD algorithm.
It is grounded in the Hausdorff distance-based penetration depth calculation method (Sec.~\ref{sec:preliminary}).
%, similar to that of Tang et al.~\shortcite{SIG09HIST}.
The process consists of two primary phases: penetration surface extraction and Hausdorff distance calculation.
We leverage the RTX platform's capabilities to accelerate both of these steps.

\begin{figure*}[t]
    \centering
    \includegraphics[width=0.8\textwidth]{Image/overview.pdf}
    \caption{The overview of RT-based penetration depth calculation algorithm overview}
    \label{fig:Overview}
\end{figure*}

The penetration surface extraction phase focuses on identifying the overlapped region between two objects.
\revision{The penetration surface is defined as a set of polygons from one object, where at least one of its vertices lies within the other object. 
Note that in our work, we focus on triangles rather than general polygons, as they are processed most efficiently on the RTX platform.}
To facilitate this extraction, we introduce a ray-tracing-based \revision{Point-in-Polyhedron} test (RT-PIP), significantly accelerated through the use of RT cores (Sec.~\ref{sec:RT-PIP}).
This test capitalizes on the ray-surface intersection capabilities of the RTX platform.
%
Initially, a Geometry Acceleration Structure (GAS) is generated for each object, as required by the RTX platform.
The RT-PIP module takes the GAS of one object (e.g., $GAS_{A}$) and the point set of the other object (e.g., $P_{B}$).
It outputs a set of points (e.g., $P_{\partial B}$) representing the penetration region, indicating their location inside the opposing object.
Subsequently, a penetration surface (e.g., $\partial B$) is constructed using this point set (e.g., $P_{\partial B}$) (Sec.~\ref{subsec:surfaceGen}).
%
The generated penetration surfaces (e.g., $\partial A$ and $\partial B$) are then forwarded to the next step. 

The Hausdorff distance calculation phase utilizes the ray-surface intersection test of the RTX platform (Sec.~\ref{sec:RT-Hausdorff}) to compute the Hausdorff distance between two objects.
We introduce a novel Ray-Tracing-based Hausdorff DISTance algorithm, RT-HDIST.
It begins by generating GAS for the two penetration surfaces, $P_{\partial A}$ and $P_{\partial B}$, derived from the preceding step.
RT-HDIST processes the GAS of a penetration surface (e.g., $GAS_{\partial A}$) alongside the point set of the other penetration surface (e.g., $P_{\partial B}$) to compute the penetration depth between them.
The algorithm operates bidirectionally, considering both directions ($\partial A \to \partial B$ and $\partial B \to \partial A$).
The final penetration depth between the two objects, A and B, is determined by selecting the larger value from these two directional computations.

%In the Hausdorff distance calculation step, we compute the Hausdorff distance between given two objects using a ray-surface-intersection test. (Sec.~\ref{sec:RT-Hausdorff}) Initially, we construct the GAS for both $\partial A$ and $\partial B$ to utilize the RT-core effectively. The RT-based Hausdorff distance algorithms then determine the Hausdorff distance by processing the GAS of one object (e.g. $GAS_{\partial A}$) and set of the vertices of the other (e.g. $P_{\partial B}$). Following the Hausdorff distance definition (Eq.~\ref{equation:hausdorff_definition}), we compute the Hausdorff distance to both directions ($\partial A \to \partial B$) and ($\partial B \to \partial A$). As a result, the bigger one is the final Hausdorff distance, and also it is the penetration depth between input object $A$ and $B$.


%the proposed RT-based penetration depth calculation pipeline.
%Our proposed methods adopt Tang's Hausdorff-based penetration depth methods~\cite{SIG09HIST}. The pipeline is divided into the penetration surface extraction step and the Hausdorff distance calculation between the penetration surface steps. However, since Tang's approach is not suitable for the RT platform in detail, we modified and applied it with appropriate methods.

%The penetration surface extraction step is extracting overlapped surfaces on other objects. To utilize the RT core, we use the ray-intersection-based PIP(Point-In-Polygon) algorithms instead of collision detection between two objects which Tang et al.~\cite{SIG09HIST} used. (Sec.~\ref{sec:RT-PIP})
%RT core-based PIP test uses a ray-surface intersection test. For purpose this, we generate the GAS(Geometry Acceleration Structure) for each object. RT core-based PIP test takes the GAS of one object (e.g. $GAS_{A}$) and a set of vertex of another one (e.g. $P_{B}$). Then this computes the penetrated vertex set of another one (e.g. $P_{\partial B}$). To calculate the Hausdorff distance, these vertex sets change to objects constructed by penetrated surface (e.g. $\partial B$). Finally, the two generated overlapped surface objects $\partial A$ and $\partial B$ are used in the Hausdorff distance calculation step.

\subsubsection{Learning for humanoid control}
Learning in simulation via reinforcement followed by a sim-to-real transfer has led to many successful locomotion results for quadrupeds~\cite{AgileDynamicMotorSkills19, RMA21} and humanoids~\cite{RealWorldHumanoidLocomotionScienceRobotics24, HumanoidLocomotionNextTokenPrediction24, HumanoidLocomotionChallengingTerrain24, advancinglocomotion2024, LCP24, KinodynamicFabrics23}. This has enabled locomotion on challenging in-the-wild terrain~\cite{HumanoidLocomotionChallengingTerrain24,DenoisingWorldModel24}, agile motions like jumping~\cite{BipedalJumpingControl23,WoCoCo24}, and even locomotion driven by visual inputs~\cite{HumanoidParkour24,long2024learning}. Researchers have also expanded the repertoire of humanoid motions to skillful movements like dancing and naturalistic walking gaits through use of human mocap or video data~\cite{Exbody2_24, Exbody24, UH1_24, Hover24}. Some works address locomotion and manipulation problems for humanoids simultaneously to enable loco-manipulation controllers in an end-to-end fashion facilitated by teleportation~\cite{OmniH2O24,HumanPlus24,MobileTelevision24}. 
Notably, these tasks mostly involve contact between the feet and the environment, thus requiring only limited contact reasoning. How to effectively develop controllers for more \textit{contact-rich} tasks like crawling, tumbling, and getting up that require numerous, dynamic, and unpredictable contacts between the whole body and the environment remains under-explored.

\subsection{Legged robots fall recovery}
Humanoid robots are vulnerable to falls due to under-actuated control dynamics, high-dimensional states, and unstructured environments~\cite{WABOT1_73,HondaHumanoid98,MABEL09,krotkov2018darpa,Humanoid35DoF96,gu2025humanoid}, making the ability to recover from falling of great significance. 
Over the years, this problem has been tackled in the following ways.

\subsubsection{Getting up via motion planning}
Early work from \citet{Learning2StandUp98} solved the getting-up problem for a two-joint, three-link walking robot in 2D, and several discrete states are used as subgoals to transit via hierarchical RL. 
This line of work can be viewed as an application of motion planning by \textit{configuration graph transition} learning~\cite{StateTransitionGraph96}, where stored robot states between lying and standing are used as graph nodes to transit~\cite{UKEMI02,FirstHumanoidGetUp03,GettingUpMotionPlanning07,HumanoidBalancing16}.
More recently, some progress has been made to enable toy-sized humanoid robots to get up.
For example, \citet{HumanoidStandingUpLfDMultimodalReward13} explores getting up from a canonical sitting posture with motion planning by imitating human demonstration with ZMP ceriterion.
To address the high-dimensionality of humanoid configurations, \citet{StandUpSymmetry16} leverage bilateral symmetry to reduce the control DoFs by half and a clustering technique is used for further reducing the complexity of configuration space, thereby improving getting-up learning efficiency.
However, such state machine learning using predefined configuration graphs may not be sufficient for generalizing to unpredictable initial and intermediate states, which happens when the robot operates on challenging terrains.

\subsubsection{Hand-designed getting-up trajectories} 
Another solution, often adopted by commercial products, is to replay a manually designed motion trajectory. For example, Unitree~\cite{UnitreeG124} has a getting-up controller built into G1's default controllers. Booster Robotics~\cite{Booster} designed a specific recovery controller for their robots that can help the robot recover from fallen states. The main drawback of such pre-defined trajectory getting-up controllers is that they can only handle a limited number of fallen states and lack generalization, as our experimental comparisons will show.

\subsubsection{Learned getting-up policies for real robots} 
RL followed by sim-to-real has also been successfully applied for quadruped~\cite{AgileDynamicMotorSkills19,dribblebot2023,guardiansasyoufall2024,ma2023learning} fall recovery. 
For example, \citet{AgileDynamicMotorSkills19} explore sim2real RL to achieve real-world quadruped fall recovery from complex configurations.
\citet{dribblebot2023} train a recovery policy that enables the quadruped to dribble in snowy and rough terrains continuously.
\citet{guardiansasyoufall2024} develop a quadruped recovery policy in highly dynamic scenarios..

\subsubsection{Learned getting-up policies for character animation} A parallel research effort in character animation, also explores the design of RL-based motion imitation algorithms: DeepMimic~\cite{DeepMimic18}, AMP~\cite{AMP2021}, PHC~\cite{PHC23}, among others~\cite{PhysicsBasedMocapImitationDRL18,PULSE24,MaskedMimic24,HierachicalWorldModel25,CooHOI24,HOIHumanLevelInstruction24}. These have also demonstrated successful getting-up controllers in simulation. 
By tracking user-specified getting-up curves, \citet{frezzato2022synthesizing} enable humanoid characters to get up by synthesizing physically plausible motion.
Without recourse to mocap data, such naturalistic getting-up controllers for simulated humanoid characters can also be developed with careful curriculum designs~\cite{Learning2GetUp22}.
Some works explore sampling-based methods for addressing contact-rich character locomotion including getting up~\cite{SAMCONSamplingBasedContactRichMotion10,OnlineMotionSynthesis14,SampleEfficientCE21}, while some works have demonstrated success in humanoid getting up with online model-predictive control~\cite{OnlineTrajectoryOptimization12}.
It is worth noticing, however, that these works use humanoid characters with larger DoFs compared to humanoid robots (\eg, 69 DoFs in SMPL~\cite{SMPL15}) and use simplified dynamics.
As a result, learned policies operate body parts at high velocities and in infeasible ways, leading to behavior that cannot be transferred into the real world directly.
Hence, developing generalizable recovery controllers for humanoid robots remains an open problem. 


