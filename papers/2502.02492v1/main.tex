\documentclass{article}

% Recommended, but optional, packages for figures and better typesetting:
\usepackage{microtype}
\usepackage{graphicx}
\usepackage{wrapfig}
% \usepackage{subfigure}
\usepackage{subcaption}
\usepackage{booktabs} % for professional tables
 \usepackage{array,multirow,graphicx}

% hyperref makes hyperlinks in the resulting PDF.
% If your build breaks (sometimes temporarily if a hyperlink spans a page)
% please comment out the following usepackage line and replace
% \usepackage{icml2025} with \usepackage[nohyperref]{icml2025} above.
\usepackage{hyperref}
\usepackage{arydshln}


% Attempt to make hyperref and algorithmic work together better:
\newcommand{\theHalgorithm}{\arabic{algorithm}}

% Use the following line for the initial blind version submitted for review:
% \usepackage{icml2025}

% If accepted, instead use the following line for the camera-ready submission:
\usepackage[accepted]{icml2025}
% \usepackage{icml2025}

% For theorems and such
\usepackage{amsmath}
\usepackage{amssymb}
\usepackage{mathtools}
\usepackage{amsthm}
% if you use cleveref..
\usepackage[capitalize,noabbrev]{cleveref}

%%%%%%%%%%%%%%%%%%%%%%%%%%%%%%%%
% THEOREMS
%%%%%%%%%%%%%%%%%%%%%%%%%%%%%%%%
\theoremstyle{plain}
\newtheorem{theorem}{Theorem}[section]
\newtheorem{proposition}[theorem]{Proposition}
\newtheorem{lemma}[theorem]{Lemma}
\newtheorem{corollary}[theorem]{Corollary}
\theoremstyle{definition}
\newtheorem{definition}[theorem]{Definition}
\newtheorem{assumption}[theorem]{Assumption}
\theoremstyle{remark}
\newtheorem{remark}[theorem]{Remark}

% Todonotes is useful during development; simply uncomment the next line
%    and comment out the line below the next line to turn off comments
%\usepackage[disable,textsize=tiny]{todonotes}
\usepackage[textsize=tiny]{todonotes}


% The \icmltitle you define below is probably too long as a header.
% Therefore, a short form for the running title is supplied here:
% \icmltitlerunning{Pixels in Motion: Enhancing Motion Understanding in Video Models via Joint Appearance-Motion Representations}
% \icmltitlerunning{VideoJAM: Joint Appearance-Motion Representations for Enhanced Motion Understanding in Video Generators}
% \icmltitlerunning{VideoJAM: Joint Appearance-Motion Representations for Enhanced Motion Generation in Video Models}
\icmltitlerunning{VideoJAM: Joint Appearance-Motion Representations for Enhanced Motion Generation in Video Models}
\newif\ifcomments
% uncomment to bring back comments
\commentstrue 
\ifcomments
    \providecommand\hila[1]{[\textcolor{blue}{Hila: {#1}}]}
    \providecommand\shelly[1]{[\textcolor{purple}{Shelly: {#1}}]}
    \providecommand\lior[1]{[\textcolor{red}{Lior: {#1}}]}
    \providecommand\adam[1]{[\textcolor{magenta}{Adam: {#1}}]}
    \providecommand\amit[1]{[\textcolor{violet}{Amit: {#1}}]}
    \providecommand\uriel[1]{[\textcolor{green}{Uriel: {#1}}]}
    \providecommand\yk[1]{[\textcolor{blue}{YK: {#1}}]}
    \providecommand\todo[1]{[\textcolor{pink}{TODO: {#1}}]}
\else
    \providecommand{\hila}[1]{}
    \providecommand{\shelly}[1]{}
    \providecommand{\lior}[1]{}
    \providecommand{\adam}[1]{}
    \providecommand{\amit}[1]{}
    \providecommand{\uriel}[1]{}
    \providecommand{\yk}[1]{}
    \providecommand{\todo}[1]{}
\fi

\begin{document}

\twocolumn[
% \icmltitle{Pixels in Motion: Enhancing Motion Understanding in Video Models via Joint Appearance-Motion Representations}
% \icmltitle{VideoJAM: Joint Appearance-Motion Representations for Enhanced Motion Understanding in Video Generators}
\icmltitle{VideoJAM: Joint Appearance-Motion Representations for Enhanced Motion Generation in Video Models}

% It is OKAY to include author information, even for blind
% submissions: the style file will automatically remove it for you
% unless you've provided the [accepted] option to the icml2025
% package.

% List of affiliations: The first argument should be a (short)
% identifier you will use later to specify author affiliations
% Academic affiliations should list Department, University, City, Region, Country
% Industry affiliations should list Company, City, Region, Country

% You can specify symbols, otherwise they are numbered in order.
% Ideally, you should not use this facility. Affiliations will be numbered
% in order of appearance and this is the preferred way.
\icmlsetsymbol{equal}{*}
\begin{icmlauthorlist}
\icmlauthor{Hila Chefer}{equal,yyy,comp}
\icmlauthor{Uriel Singer}{yyy}
\icmlauthor{Amit Zohar}{yyy}
\icmlauthor{Yuval Kirstain}{yyy}
\\
\icmlauthor{Adam Polyak}{yyy}
\icmlauthor{Yaniv Taigman}{yyy}
\icmlauthor{Lior Wolf}{comp}
\icmlauthor{Shelly Sheynin}{yyy}
% \\
% \url{https://videojam-paper.github.io/}
\end{icmlauthorlist}

\icmlaffiliation{yyy}{GenAI, Meta}
\icmlaffiliation{comp}{Tel Aviv University}

\icmlcorrespondingauthor{Hila Chefer}{hilach70@gmail.com}
% \icmlcorrespondingauthor{Firstname2 Lastname2}{first2.last2@www.uk}

% You may provide any keywords that you
% find helpful for describing your paper; these are used to populate
% the "keywords" metadata in the PDF but will not be shown in the document
\icmlkeywords{Machine Learning, ICML}

\vskip 0.1in

{

\begin{center}
    \centering
    \captionsetup{type=figure}
    \includegraphics[width=\textwidth]
    {figures/teaser.pdf}
    \vspace{-26px}
    \captionof{figure}{\protect{\textbf{Text-to-video samples generated by VideoJAM.}} We present VideoJAM, a framework that explicitly instills a strong motion prior to any video generation model. Our framework significantly enhances motion coherence across a wide variety of motion types. }
    \label{fig:teaser}
\end{center}%
\vskip 0.1in

}] 

% this must go after the closing bracket ] following \twocolumn[ ...

% This command actually creates the footnote in the first column
% listing the affiliations and the copyright notice.
% The command takes one argument, which is text to display at the start of the footnote.
% The \icmlEqualContribution command is standard text for equal contribution.
% Remove it (just {}) if you do not need this facility.

% \printAffiliationsAndNotice{}  % leave blank if no need to mention equal contribution
% TODO
\printAffiliationsAndNotice{\icmlEqualContribution} % otherwise use the standard text.

\begin{abstract}

% Recent works to jointly reconstruct 3D human and object from a single RGB image, are mostly model-based, that fail to capture the fine details of the clothed human body and object surface. In this paper, we introduce ReCHOR, a novel, model-free, first-method to produce realistic clothed human-object reconstructions from a monocular view. This is extremely challenging due to human-object occlusions, diverse interactions and depth ambiguity, as it needs to infer both 3D spatial awareness and high resolution details. Our core idea is based on estimating neural implicit representations for human and object respectively by an attention-based neural implicit model that attends to pixel-aligned features from both the global human-object image for spatial awareness and  the local separate view of human and object images for high quality details. Additionally, the network is conditioned on semantic features from an initial estimated human-object pose prior and a generative diffusion model that inpaints occluded regions, thus enabling the retrieval of details from them.
% We also propose a synthetic dataset with rendered scenes of diverse, inter-occluded 3D human and object scans, to train our network. We evaluate our method on the synthetic and real world BEHAVE dataset. Our experiments show that our method outperforms the SOTA in achieving realistic clothed human-object reconstructions.
Recent approaches to jointly reconstruct 3D humans and objects from a single RGB image represent 3D shapes with template-based or coarse models, which fail to capture details of loose clothing on human bodies. In this paper, we introduce a novel implicit approach for jointly reconstructing realistic 3D clothed humans and objects from a monocular view. For the first time, we model both the human and the object with an implicit representation, allowing to capture more realistic details such as clothing. This task is extremely challenging due to human-object occlusions and the lack of 3D information in 2D images, often leading to poor detail reconstruction and depth ambiguity. To address these problems, we propose a novel attention-based neural implicit model that leverages image pixel alignment from both the input human-object image for a global understanding of the human-object scene and from local separate views of the human and object images to improve realism with, for example, clothing details. Additionally, the network is conditioned on semantic features derived from an estimated human-object pose prior, which provides 3D spatial information about the shared space of humans and objects. To handle human occlusion caused by objects, we use a generative diffusion model that inpaints the occluded regions, recovering otherwise lost details. For training and evaluation, we introduce a synthetic dataset featuring rendered scenes of inter-occluded 3D human scans and diverse objects. Extensive evaluation on both synthetic and real-world datasets demonstrates the superior quality of the proposed human-object reconstructions over competitive methods.
\end{abstract}    
\section{Introduction}
\label{sec:intro}
% Image editing methods in diffusion models depend on user-defined control directions - users can unlock their creativity using these methods by specifying the desired manipulation through prompts~\cite{gandikota2023concept}, reference images~\cite{ruiz2022dreambooth, kumari2022customdiffusion, gal2022image, chen2024trainingfreeregionalpromptingdiffusion}, or attribute vectors~\cite{parmar2023zero,hertz2022prompt}. In this work, we ask a fundamentally different question: \emph{Can we automatically discover the underlying visual structure of a concept within diffusion model's knowledge?} %Rather than requiring user-specified controls, we aim to decompose the model's internal knowledge into meaningful directions.

% This question touches on a fundamental limitation in how we interact with diffusion models. Current control methods ~\cite{zhang2023addingconditionalcontroltexttoimage, gandikota2023concept, ye2023ipadaptertextcompatibleimage,ye2023ipadaptertextcompatibleimage, hertz2024stylealignedimagegeneration, li2023photomaker, shi2024instantbooth, chen2024trainingfreeregionalpromptingdiffusion} require users to specify their desired manipulations in advance, limiting interactive creativity. This contrasts with natural human artistic workflows, where creators dynamically explore creative ideas while jointly refining them toward meaningful artistic outcomes~\cite{hoffmann2016modeling}. This synergy between specification and exploration is not new to generative models. Early GAN architectures naturally developed disentangled latent spaces that enabled continuous\cite{harkonen2020ganspace,radford2015unsupervised, wu2021stylespace, shen2020interfacegan}, compositional control over generated images. Users could explore these spaces to discover interesting variations that would be difficult to describe in words~\cite{wu2021stylespace}, then combine them to achieve their creative goals~\cite{grabe2022towards}. 


% While diffusion models have largely superseded GANs in conditional image synthesis~\cite{dhariwal2021diffusion},  their underlying structure remains less understood. Diffusion models achieve remarkable diversity through high-dimensional latents, unlike GANs' compact latent spaces.  With a single prompt, diffusion models can generate radically different variations through different random initializations of input noise. We ask - Is it possible to discover interpretable structure within this vast space of variations?

Text-to-image diffusion models are capable of generating remarkable visual variations from a single prompt through different random initializations. However, this vast creative potential remains largely opaque to users---while we can generate diverse images, we lack understanding of the underlying structure of these variations. This presents a fundamental challenge: how can we discover and expose the latent visual capabilities encoded within these models?

\let\thefootnote\relax \footnote{$^{*}$Correspondence to \texttt{gandikota.ro@northeastern.edu}}

The challenge touches on a key limitation in how we interact with diffusion models today. Current control methods require users to explicitly specify their desired edits in advance through prompts~\cite{gandikota2023concept}, reference images~\cite{zhang2023addingconditionalcontroltexttoimage, chen2024trainingfreeregionalpromptingdiffusion, ruiz2022dreambooth,kumari2022customdiffusion, Ryu_lora, hu2021lora}, or attribute vectors~\cite{ye2023ipadaptertextcompatibleimage, hertz2024stylealignedimagegeneration, li2023photomaker, shi2024instantbooth,parmar2023zero,hertz2022prompt}. That contrasts sharply with natural human creative workflows, where artists dynamically explore creative ideas and jointly refine them toward meaningful artistic outcomes~\cite{hoffmann2016modeling}. The need for pre-specified controls creates a barrier between users and the full creative potential of these models.

Interestingly, earlier generative models like GANs~\cite{gans,karras2019style,brock2018large} naturally developed more interpretable internal structures. Their compact latent spaces often exhibited emergent disentanglement~\cite{harkonen2020ganspace,radford2015unsupervised, wu2021stylespace, shen2020interfacegan}, enabling continuous and compositional control over generated images. Users could explore these spaces to discover interesting variations that would be difficult to describe in words~\cite{wu2021stylespace}, then combine them to achieve their creative goals~\cite{grabe2022towards}.

Diffusion models have largely superseded GANs in conditional image synthesis~\cite{dhariwal2021diffusion}, achieving greater diversity through much higher-dimensional latents. And yet an understanding of the underlying structure of these larger latent spaces has remained elusive. In this work, we ask a fundamental question: \emph{Can we automatically discover the visual structure within a diffusion model's knowledge of a concept?} Rather than requiring user-specified controls, we aim to decompose the model's internal representations into expressive directions that users can explore and combine.

To address these needs, we present \textbf{SliderSpace}, a framework that brings systematic explorability to diffusion models. Given just a text prompt, SliderSpace discovers a canonical set of meaningful, diverse, and controllable directions within the model's knowledge of that concept. Each direction is implemented as a low-rank adapter~\cite{hu2021lora} that can be scaled and composed with others, allowing users to explore and smoothly combine different aspects of variation, as shown in Figure~\ref{fig:intro}.

We ground SliderSpace discovery in three key requirements for meaningful decomposition of a diffusion model's visual manifold: 
\begin{enumerate}
    \item \textbf{Unsupervised Discovery:} The decomposition process should emerge from the intrinsic structure of the model's learned representation, rather than being guided by predefined attributes. This ensures we capture the true topology of the model's knowledge space rather than projecting our assumptions onto it.
    
    \item \textbf{Semantic Orthogonality:} Each discovered control must represent a distinct semantic direction. This is enforced in a semantic feature space, like CLIP, where every slider has an orthogonal effect in embeddings. This prevents discovering multiple controls that create similar semantic effects, making the system more efficient and easier.
    
    \item \textbf{Distribution Consistency:} Directions must induce consistent transformations across both random seeds and prompt variations. 
\end{enumerate}

These requirements naturally lead to our proposed framework, which we formalize in Section~\ref{sec:method}. As we show in our experiments, SliderSpace is architecture-agnostic, working with both conventional U-Net based models like Stable Diffusion~\cite{rombach2022high, rombach2022sd20, podell2023sdxl, turbo, dmd} and recent transformer-based architectures like Flux~\cite{flux}.

We demonstrate the expressiveness of SliderSpace through three applications: First, we show how SliderSpace can decompose high-level concepts into diverse and expressive components, revealing the natural axes of variation in the model's understanding. Second, we explore artistic style variation, where SliderSpace discovers directions that match or exceed the diversity of manually curated artist lists while being judged more useful by human evaluators. Finally, we show how SliderSpace can help reverse the mode collapse commonly observed in distilled diffusion models, restoring diversity while maintaining generation speed.

Beyond providing practical creative control, SliderSpace opens new avenues for understanding and utilizing the latent capabilities of diffusion models. By mapping these models' visual potential into intuitive, composable directions, we take a step toward making their creative possibilities more accessible and interpretable to users.

% Image editing methods in diffusion models unlock the creativity of users. In this work we ask an alternate question: \emph{Can we organize and expose what of the diffusion model is already capable of?}.
% Existing methods for controlling image generation typically require users to manually specify edit directions for desired changes. This process is time-consuming, requires technical expertise, and limits the spontaneity of the creative process. For instance, if a user wants to adjust the smile of a generated person, they must explicitly request this edit, often through imprecise prompt engineering or model fine-tuning. This approach of predefined controls or manual specifications restricts users from fully exploring the latent capabilities of the model. There may be interesting stylistic variations or attributes that the model can generate, but users have no easy way to discover or utilize these.

% Natural visual disentanglement was an emergent property in the latent space of Generative Adversarial Models (GANs) \cite{harkonen2020ganspace,radford2015unsupervised, wu2021stylespace, shen2020interfacegan}. In particular, it has been observed that StyleGAN~\cite{karras2019style} stylespace neurons offer detailed control over many meaningful aspects of images that would be difficult to describe in words~\cite{wu2021stylespace}. However, diffusion models do not share such a compact latent space~\cite{park2023unsupervised}; and efforts to uncover such a space in the semantic embeddings of the text conditioning have met with limited success \nik{Nick - is there a specific citation you were thinking about?}.

% In this work we introduce \textbf{SliderSpace}, which takes a step towards uncovering an analogous low dimensional representation of diffusion models' visual breadth; in essence treating the diffusion model as many generators sharing parameters, where a particular generator is defined by a specific prompt. For a given prompt we sample many random seeds (and optionally prompt expansions using an LLM), generate the corresponding images, and apply an off the shelf feature extractor (in this work CLIP, but our method can be applied to any differentiable feature extractor). We use PCA to analyze these features, and for each of the leading $k$ principal components we train a LoRA \cite{} which causes the diffusion model to produces images which increase the feature magnitude along that component when passed back through the same feature extractor. This leads to a 'Slider' for each principal component, because each LoRA can be scaled and applied to the original diffusion model, continuously varying those visual features in the generated results (as measured, in our case, by CLIP).

% There are many other works that enhance the controllability of diffusion models. One common approach is enabling users to add spatial constraints to a generation either manually, or via a reference image \cite{zhang2023addingconditionalcontroltexttoimage, chen2024trainingfreeregionalpromptingdiffusion}, a second is leveraging more abstract embeddings (e.g. identity, style) extracted from a reference image \cite{ye2023ipadaptertextcompatibleimage, hertz2024stylealignedimagegeneration, li2023photomaker, shi2024instantbooth}, a third is finetuning a foundation model to better generate a concept important to the user \cite{ruiz2022dreambooth, kumari2022customdiffusion, Ryu_lora, hu2021lora}, and a fourth (most relevant to this work) is finding low-rank adaptors of the model based on a prompt or small training set which can be scaled to provide continous control over one aspect of generated image (e.g. night vs day, basic vs luxury, etc.) \cite{gandikota2023concept}. SliderSpace is complementary to all of these methods and offers something distinct. All of the other methods we are aware require the user (and / or model designer) to know in advance what type of control they want. In contrast SliderSpace assists users in discovering and controlling hidden capabilities present in the diffusion model's distribution of possible generations.

%We propose that truly intuitive creative control in a text-to-image model should meet three key criteria: \emph{discoverability}, \emph{intuitiveness}, and \emph{specificity}. The model should reveal controllable attributes that may not be immediately obvious, offer controls that are easy to understand and manipulate, and ensure each control affects a distinct attribute of the generated image.

% We demonstrate the utility and power of SliderSpace using three applications built on top of SDXL-DMD \cite{dmd}, because its fast generation speed lends itself well to the continuous control offered by SliderSpace.

% First, we study concept decomposition (Section \ref{sec:concept_exp}), where we learn sliders for a specific concept (e.g. 'monster', 'waterfall', 'car'). Through quantitative metrics of diversity and text alignment we demonstrate that the learned sliders dramatically boost the diversity of generations when randomly applied without harming text alignment; we also ask humans to qualitatively judge these results in a user study where they find the SliderSpace results to be more 'Diverse', 'Useful', and 'Creative' than our baselines.

% Second, we attempt to compare the automatic discoveries of SliderSpace to a large scale manual study of artistic styles (Section \ref{sec:art_exp}), open-sourced by ParrotZone \cite{parrotzone}. In this study SDXL was prompted with over 4300 artist names,  and based on visual inspection the cases of successful stylistic mimicry recorded. Quantitatively SliderSpace more closely matches the distribution of artistic variation discovered by ParrotZone than other baselines, and in our user studies was judged to be significantly more 'Diverse' and 'Useful' than the baselines. To our surprise humans even judged SliderSpace results to be slightly more 'Diverse' than the results generated by the manually discovered artist names of \cite{parrotzone}.

% Third, we attempt to use SliderSpace to reverse the mode collapse commonly observed in distilled few-step diffusion models relative to the original teacher model (Section \ref{sec:diverse_exp}). We quantitatively demonstrate that applying SliderSpace to SDXL-DMD leads to more closely matching the distribution of images by the original teacher, SDXL.

%Through extensive experiments on various state-of-the-art text-to-image models, we demonstrate that SliderSpace significantly enhances user control and creative expression in AI-assisted image generation tasks. Our method enables a range of applications, including concept decomposition and control, diversity improvement in generated images, customization dissection and edits, and the exploration of artistic styles inherent in the model.

% SliderSpace goes beyond providing a practical tool for enhanced creative control. By mapping the visual potential of diffusion models it can open new avenues for generative creativity and deepens our understanding of each model's hidden potential.
\section{Related Work}

\paragraph{LLMs for Agent tasks.}

Our research is related to deploying large language models (LLMs) as agents for decision-making tasks in interactive environments~\citep{liu2023agentbench,zhou2023webarena,shridhar2020alfred,toyama2021androidenv}. Earlier works, such as~\citep{yao2023webshopscalablerealworldweb}, fine-tuned models like BERT~\citep{devlin2019bertpretrainingdeepbidirectional} for decision-making in simplified environments, such as online shopping or mobile phone manipulation. With the advent of large language models~\citep{brown2020languagemodelsfewshotlearners,openai2024gpt4technicalreport}, it became feasible to perform decision-making tasks through zero-shot or few-shot in-context learning. To better assess the capabilities of LLMs as agents, several models have been developed~\citep{deng2024mind2web,xiong2024watch,hong2023cogagent,yan2023gpt}. Most approaches~\citep{zheng2024seeact,deng2024mind2web} provide the agent with observation and action history, and the language model predicts the next action via in-context learning. Additionally, some methods~\citep{zhang2023building,li2023camel,song2024trial} attempt to distill trajectories from state-of-the-art language models to train more effective policy models. In contrast, our paper introduces a novel framework that automatically learns a reward model from LLM agent navigation, using it to guide the agents in making more effective plans.

\textbf{LLM Planning.} Our paper is also related to planning with large language models. Early researchers~\citep{brown2020languagemodelsfewshotlearners} often prompted large language models to directly perform agent tasks. Later, \citet{yao2022react} proposed ReAct, which combined LLMs for action prediction with chain-of-thought prompting~\citep{wei2022chain}. Several other works~\citep{yao2023treethoughtsdeliberateproblem,hao2023reasoning,zhao2023large,qiao2024agentplanningworldknowledge} have focused on enhancing multi-step reasoning capabilities by integrating LLMs with tree search methods. Our model differs from these previous studies in several significant ways. First, rather than solely focusing on text generation tasks, our pipeline addresses multi-step action planning tasks in interactive environments, where we must consider not only historical input but also multimodal feedback from the environment. Additionally, our pipeline involves automatic learning of the reward model from the environment without relying on human-annotated data, whereas previous works rely on prompting-based frameworks that require large commercial LLMs like GPT-4~\citep{openai2024gpt4technicalreport} to learn action prediction. Furthermore, \Model supports a variety of planning algorithms beyond tree search.

\textbf{Learning from AI Feedback.} In contrast to prior work on LLM planning, our approach also draws on recent advances in learning from AI feedback~\citep{bai2022constitutional,lee2023rlaif,yuan2024self,sharma2024critical,pan2024autonomous,koh2024tree}. These studies initially prompt state-of-the-art large language models to generate text responses that adhere to predefined principles and then potentially fine-tune the LLMs with reinforcement learning. Like previous studies, we also prompt large language models to generate synthetic data. However, unlike them, we focus not on fine-tuning a better generative model but on developing a classification model that evaluates how well action trajectories fulfill the intended instructions. This approach is simpler, requires no reliance on state-of-the-art LLMs, and is more efficient. We also demonstrate that our learned reward model can integrate with various LLMs and planning algorithms, consistently improving their performance.

\textbf{Inference-Time Scaling.} ~\citet{snell2024scaling} validates the efficacy of inference-time scaling for language models. Based on inference-time scaling, various methods have been proposed, such as random sampling~\citep{wang2022self} and tree-search methods~\citep{hao2023reasoning, zhang2024accessing, guan2025rstar}. Concurrently, several works have also leveraged inference-time scaling to improve the performance of agentic tasks. ~\citet{koh2024tree} adopts a training-free approach, employing MCTS to enhance policy model performance during inference and prompting the LLM to return the reward. ~\citet{gu2024your} introduces a novel speculative reasoning approach to bypass irreversible actions by leveraging LLMs or VLMs. It also employs tree search to improve performance and prompts an LLM to output rewards. ~\citet{yu2024exact} proposes Reflective-MCTS to perform tree search and fine-tune the GPT model, leading to improvements in ~\citet{koh2024visualwebarena}. ~\citet{putta2024agent} also utilizes MCTS to enhance performance on web-based tasks such as ~\citet{yao2023webshopscalablerealworldweb} and real-world booking environments. ~\cite{lin2025qlass} utilizes the stepwise reward to give effective intermediate guidance across different agentic tasks. Our work differs from previous efforts in two key aspects: (1) Broader Application Domain. Unlike prior studies that primarily focus on tasks from a single domain, our method demonstrates strong generalizability across web agents, mathematical reasoning, and scientific discovery domains, further proving its effectiveness. (2) Flexible and Effective Reward Modeling. Instead of simply prompting an LLM as a reward model, we finetune a small scale VLM~\citep{lin2023vila} to evaluate input trajectories. %Our reward scores range continuously between 0 and 1, in contrast to existing methods that rely on discrete scoring (e.g., 0 and 1, or 0, 0.5, and 1) through direct LLM prompting.

% Concurrently, several works have also leveraged inference-time scaling to improve the performance of agentic tasks. ~\citet{pan2024autonomous} demonstrates that LLMs and VLMs, such as the GPT series, can function as evaluators or reward models to provide guidance for fine-tuning or reflection, thereby enhancing digital agents. This lays the groundwork for subsequent studies that directly prompt LLMs as reward models. ~\citet{koh2024tree} adopts a training-free approach, employing MCTS to enhance policy model performance during inference. However, it is limited to web environments~\citep{koh2024visualwebarena}. Moreover, its value function relies on prompting an LLM, which is less effective than our proposed method. We validate our approach through ablation studies, demonstrating that our fine-tuned reward model is more effective. ~\citet{gu2024your} introduces a novel speculative reasoning approach to bypass irreversible actions, such as purchasing a product, by leveraging LLMs or VLMs. It also employs tree search to improve performance, but it remains restricted to the web domain~\citep{koh2024visualwebarena, deng2024mind2web}. Additionally, it lacks reward modeling and instead prompts an LLM to output rewards. ~\citet{yu2024exact} proposes Reflective-MCTS to perform tree search and fine-tune the GPT model, leading to improvements in ~\citep{koh2024visualwebarena}. However, this work focuses solely on a single web agent task, and its reward modeling is derived from multi-agent debate, differing from our more effective and efficient reward modeling approach. ~\citet{putta2024agent} also utilizes MCTS to enhance performance, but it is limited to web-based tasks such as ~\citep{yao2023webshopscalablerealworldweb} and real-world booking environments.
\begin{table}[t]
    \tiny
    % \scriptsize
    % \small
    % \vspace{-5pt}
    \caption{
        Camera bias and accuracy of various state-of-the-art ReID models based on clustering results. 
        ``SL'' and ``CA'' denote the supervised learning and camera-aware method, respectively.
        ``Bias'' and ``Accuracy'' denote the Normalized Mutual Information (NMI) scores between cluster labels and camera labels, and between cluster labels and identity labels, respectively, in $\times$100 scale.
        ISR is trained on external videos and the other models are trained on MSMT17-Train.
    }
    \vspace{3pt}
    \label{tab:motivation}
    % \setlength{\tabcolsep}{0.79em}  % 0.75em
    \setlength{\tabcolsep}{0.61em}  % 0.75em
    \centering
    {
    \renewcommand{\arraystretch}{1.2}
    \begin{tabular}{lccc|cc|cc|cc|cc|cc}
        \hline
        \multicolumn{1}{l}{\multirow{2}{*}{Method}} & 
        \multicolumn{1}{c}{\multirow{2}{*}{SL}} & 
        \multicolumn{1}{c}{\multirow{2}{*}{CA}} & 
        \multicolumn{1}{c|}{\multirow{2}{*}{Backbone}} & 
        \multicolumn{2}{c|}{MSMT17-Train} & 
        \multicolumn{2}{c|}{MSMT17-Test} & 
        \multicolumn{2}{c|}{Market-1501} &
        \multicolumn{2}{c|}{CUHK03-NP} &
        \multicolumn{2}{c}{PersonX} \\
        
        \cline{5-14}
        \multicolumn{1}{c}{} &
        \multicolumn{1}{c}{} &
        \multicolumn{1}{c}{} &  
        \multicolumn{1}{c|}{} & 
        Bias & Accuracy & Bias & Accuracy & Bias & Accuracy & Bias & Accuracy  & Bias & Accuracy \\ 
        \hline \hline
        
        % CC~\citep{dai2022cluster} & \xmark & \xmark & R50 & 34.4 & 89.4 & 32.2 & 87.8	& 15.9 & 78.5 & 14.6 & 76.4 \\
        CC~\citep{dai2022cluster} & \xmark & \xmark & R50 & 34.7 & 89.3 & 32.5 & 88.0 & 17.1 & 81.0 & 17.6 & 74.6 & 20.6 & 78.9 \\
        PPLR~\citep{cho2022part} & \xmark & \xmark & R50 & 31.8 & 90.3 & 30.2 & 89.0	& 15.6 & 81.7 & 15.9 & 77.4 & 15.3 & 82.0 \\
        TransReID-SSL~\citep{luo2021self} & \xmark & \xmark & ViT & 29.3 &	93.1 &	27.1 &	92.8	& 9.7	& 92.2	& 7.0	& 84.2 & 12.5 & 88.8 \\
        ISR~\citep{dou2023identity} & \xmark & \xmark & ViT & 31.8 & 90.5 & 30.3 & 89.4 & 9.7 & 95.8 & 5.4 & 87.7 & 6.1 & 94.9 \\
        PPLR-CAM~\citep{cho2022part} & \xmark & \cmark & R50 & 29.3 & 92.8 & 26.7 & 92.4	& 14.3 & 84.1 & 13.7 & 78.4 & 14.6 & 81.8 \\
        TransReID~\citep{he2021transreid} & \cmark & \cmark & ViT & 24.4 & 98.3 & 23.6	& 94.5 & 13.6 & 89.8	& 3.9 & 84.7 & 6.6 & 92.7 \\
        SOLIDER~\citep{chen2023beyond} & \cmark & \xmark & ViT & 23.2 & 98.7 & 21.3 & 96.9	& 7.3 & 96.5 & 1.6 & 90.8 & 2.8 & 93.8 \\
        \hdashline
        Ground Truth & - & - & - & 21.1 & - & 19.2 & -	& 6.4 & - & 0.1 & - & 0.0 & - \\
        % \bottomrule
        \hline
    \end{tabular}
    }
    % \vspace{-2mm}
\end{table}




\section{Quantitative analysis on camera bias}
\label{sec:motivation}

In this section, we quantitatively investigate the camera bias in existing ReID models.
The camera bias is the phenomenon where the feature distribution is biased towards the camera labels of the samples, which degrades ReID performance.
Many camera-aware methods have been proposed to address this problem.
However, the scope of the discussion has been primarily limited to training domain and the camera bias on unseen domains has not been thoroughly explored.
We focus on the camera bias of ReID models on unseen domains, examining various types of models including camera-aware/agnostic, supervised/unsupervised, and domain generalizable approaches, with the widely used backbones such as ResNet~\citep{he2016deep} and ViT~\citep{dosovitskiy2020vit}.

To measure the bias, we utilize Normalized Mutual Information (NMI) which quantifies the shared information between two clustering results.
We extract the features of samples and perform clustering to them using InfoMAP~\citep{infomap}.
Then, the camera bias is computed by NMI between cluster labels and camera labels of the samples.
The accuracy of the clusters are measured by NMI between the cluster labels and the identity labels.
The results on MSMT17, Market-1501, CUHK03-NP~\citep{cuhk_np}, and PersonX~\citep{personx} are shown in Table~\ref{tab:motivation}, where the bias of the ground truth (\ie, NMI between the identity labels and the camera labels) indicates the inherent imbalance in a dataset.
All models except ISR~\citep{dou2023identity} are trained on MSMT17, hence the other datasets are unseen domains for them.
For ISR, all datasets are unseen domains.

We make two notable observations from the results.
First, the existing ReID models have a large camera bias on the unseen domains, regardless of their training setups or backbones.
% The bias of PPLR~\citep{cho2022part} is mitigated by the camera-aware learning of PPLR-CAM~\citep{cho2022part}, but its bias is still significantly higher than the ground truth.
Second, the unsupervised models have a large camera bias on the seen domain, 
even on their training data.
These imply that debiasing methods for unseen domains are needed in general, and there is room for performance improvement of unsupervised methods by reducing the camera bias during training.
Relatively, the recent supervised models exhibit less debiased results on the training domain.
\section{Causal IL as CMRs}\label{sec:method}

In this section, we demonstrate that performing causal IL in our framework is possible using trajectory histories as instruments. In the next step, we show that the problem can be described as CMRs and propose an effective algorithm to solve it.

The typical target for IL would be the expert policy $\pi_E$ itself. However, since the expert has access to information, namely $u^o_t$, which the imitator does not, the best thing an imitator can do is to learn a history-dependent policy $\pi_h$ that is the closest to the expert. A natural choice is the conditional expectation of $\pi_E(s_t,u^o_t)$ on the history $h_t$:
\begin{align}
\pi_h(h_t)\coloneqq \expectE_{\probP(u^o_t\mid h_t)}[\pi_E(s_t,u^o_t)]=\expectE[\pi_E(s_t,u^o_t)\mid h_t],\nonumber
\end{align}
% where $p(u^o_t\mid h_t)$ is a distribution over expert-observable confounders and captures the information about $u^o_t$ can be inferred from the trajectory history. 
because the conditional expectation minimizes the least squares criterion~\citep{hastie01statisticallearning} and $\pi_h$ is the best predictor of $\pi_E$ given $h_t$. In $\pi_h$, the distribution $\probP(u^o_t\mid h_t)$ captures the information about $u^o_t$ that can be inferred from trajectory histories.
\begin{remark}
\emph{Learning $\pi_h$ is not trivial. Policies learnt naively using behaviour cloning (i.e., $\expectE[a_t\mid h_t]$) fail to match $\pi_E$. In view of~\cref{eq:action}, we have that
\begin{align} 
\expectE[a_t\mid h_t]&=\expectE[\pi_E(s_t,u^o_t) \mid h_{t}]+\expectE[u^\epsilon_t\mid h_{t}]\nonumber\\
&=\pi_h(h_t)+\expectE[u^\epsilon_t\mid h_{t}],\label{eq:history_policy}
\end{align}
where $\expectE[u^\epsilon_t\mid h_{t}]\neq 0$ due to the spurious correlation between $u^\epsilon_t$ and the trajectory history $h_t$. As a result, $\expectE[a_t\mid h_t]$ becomes biased, which can lead to arbitrarily worse performance compared to $\pi_E$.   }
\end{remark}

\vspace{-5pt}
\paragraph{Derivation of CMRs.} 
Leveraging the confounding horizon from Assumption~\ref{assump:horizon}, it becomes possible to break the spurious correlation using the independence of $u^\epsilon_t$ and $u^\epsilon_{t-k}$. We propose to use the $k$-step trajectory history $h_{t-k}=(s_{1},a_{1},...,s_{t-k})$ as an instrument for the current state $s_t$. Taking the expectation conditional on $h_{t-k}$ in~\cref{eq:history_policy} yields
\begin{align*}
    \expectE[a_t\mid h_{t-k}] & = \expectE\left[\expectE[a_t\mid h_{t}]\mid h_{t-k}\right] \\ & = \expectE[\pi_h(h_t)\mid h_{t-k}]+\expectE[\expectE[u^\epsilon_t\mid h_{t}]\mid h_{t-k}] \\
    & = \expectE[\pi_h(h_t) \mid h_{t-k}]+\expectE[u^\epsilon_t\mid h_{t-k}]
\end{align*}
where we use the fact that $h_{t-k}$ is $\sigma(h_t)$-measurable because $h_{t-k}\subseteq h_t$. Next, recall that $u^\epsilon_t\indep u^\epsilon_{t-k}$ by Assumption~\ref{assump:horizon}, which implies $u^\epsilon_t\indep h_{t-k}$, so that % Hence, since $\expectE[u^\epsilon_t] = 0$, we obtain
\begin{align}
    \expectE[a_t\mid h_{t-k}] &= \expectE[\pi_h(h_t) \mid h_{t-k}]+\expectE[u^\epsilon_t]\nonumber\\
    &=\expectE[\pi_h(h_t) \mid h_{t-k}].
\end{align}

As a result, the problem of learning $\pi_h$ reduces to solving for $\pi_h$ that satisfies the following identity
\begin{align}
    \expectE[a_t-\pi_h(h_t)\mid h_{t-k}]=0,\label{eq:CMR}
\end{align}
which is a CMR problem as defined in~\cref{sec:cmr}. In this case, both $a_t$ and $h_t$ are observed in the confounded expert demonstrations, and $h_{t-k}$ acts as the instrument. 

To make sure the instrument $h_{t-k}$ is valid, we check that it satisfies the conditions of~\cref{assump:iv}. Firstly, we have checked that $u^\epsilon_t\indep h_{t-k}$. Secondly, the environment and the expert policy are non-trivial, which means $\probP(h_t\mid h_{t-k})$ is not constant in $h_{t-k}$. Finally, $h_{t-k}$ indeed only affects $a_t$ through $s_t$ by the Markovian property. However, the strength of the instrument, which informally represents the correlation between the instrument $h_{t-k}$ and $h_t$, plays an important role in how well we can identify $\pi_h(h_t)$ by solving the CMRs in~\cref{eq:CMR}. In particular, we see that, as the confounding horizon $k$ increases, the correlation between $h_{t-k}$ and $h_t$ weakens and $h_{t-k}$ becomes a weaker instrument. This means that it is less able to identify $\pi_h$ via the CMR in~\cref{eq:CMR} and the final learnt imitator will have poorer performance. This is confirmed theoretically in Proposition~\ref{prop:ill-posed} and experimentally in~\cref{sec:exps}, and we will formalise this notion of instrument strength in~\cref{sec:theory}.


% Note this problem is equivalent to solving an IV regression on~\cref{eq:history_policy}, where $Y=\expectE[a_t\lvert h_t]$, $f(x)=\pi_h(h_t)$, $\epsilon=\expectE[u^\epsilon_t$ and the instrument $Z=h_{t-k}$.




\subsection{Practical Algorithms for Solving the CMRs}

\begin{algorithm}[tb]
   \caption{DML-IL}
   \label{alg:DML-IL}
\begin{algorithmic}[1]
   \STATE {\bfseries input} Dataset $\dataset_E$ of expert demonstrations, Confounding noise horizon $k$
   \STATE Initialize the roll-out model $\hat{M}$ as a Gaussian mixture model\label{algo:roll_out_1}
    \REPEAT
   \STATE Sample $(h_{t},a_t)$ from data $\dataset_E$
   \STATE Fit the roll-out model $(h_t,a_t)\sim\hat{M}(h_{t-k})$ to maximize the log likelihood 
\UNTIL{convergence}\label{algo:roll_out_2}
   \STATE Initialize the expert model $\hat \pi_h$ as a neural network
   \REPEAT
   % \FOR{$k=1$ {\bfseries to} $K$}
   \STATE Sample $h_{t-k}$ from $\dataset_E$
   \STATE Generate $\hat{h}_t$ and $\hat{a}_t$ using the roll-out model $\hat{M}$
   \STATE Update $\hat \pi_h$ to minimise the loss $\ell:= \norm{\hat{a}_t - \hat{\pi}_h (\hat h_t)}_2$
   % \ENDFOR
    \UNTIL{convergence}
    \STATE {\bfseries return} A history-dependent imitator policy $\hat{\pi}_h$
\end{algorithmic}
\end{algorithm}

There are various techniques~\citep{Shao2024,Bennett2019,Xu2020,Dikkala2020} for solving the CMRs $\expectE[a_t\lvert h_{t-k}]=\expectE[\pi_h(h_t) \lvert h_{t-k}]$. Here, the \textit{CMR error} that we aim to minimise is given by 
\begin{align*}
\sqrt{\expectE\big[\expectE[a_t-\hat{\pi}_h(h_t)\lvert h_{t-k}]^2\big]}=\norm{\expectE[a_t-\hat{\pi}_h(h_t)\lvert h_{t-k}]}_{2}.    
\end{align*}
In~\cref{alg:DML-IL}, we introduce DML-IL, an algorithm adapted from the IV regression algorithm DML-IV~\citep{Shao2024}\footnote{DML stands for double machine learning~\citep{Chernozhukov2018Double}, which is a statistical technique to ensure fast convergence rate for two-step regression, as is the case in~\cref{alg:DML-IL}.}, which solves our CMRs by minimising the CMR error. The first part of the algorithm (line 3-7) learns a roll-out model $\hat{M}$ that generates a trajectory $k$ steps ahead given $h_{t-k}$. Then, the roll-out model $\hat{M}$ is used to train the policy model $\hat{\pi}_h$ (line 8-13). $\hat{\pi}_h$ takes the generated trajectory $\hat{h}_t$ from $\hat{M}(h_{t-k})$ as inputs, and minimises the mean squared error to the next action. Using generated trajectories is crucial in breaking the spurious correlation caused by $u^\epsilon_t$ between past states and actions, and using the trajectory history before $h_{t-k}$ allows the imitator to infer information about $u^o_t$.

DML-IL can also be implemented with $K$-fold cross-fitting, where the dataset is partitioned into $K$ folds, with each fold alternately used to train $\hat{\pi}_h$ and the remaining folds to train $\hat{M}$. This ensures unbiased estimation and improves the stability of training. The base IV algorithm DML-IV with $K$-fold cross-fitting is theoretically shown to converge at the rate of $O(N^{-1/2})$~\citep{Shao2024}, where $N$ is the sample size, under regularity conditions. DML-IL with $K$-fold cross-fitting (see~\cref{appendix:dmlil} for details) will thus inherit this convergence rate guarantee. 

Note that~\cref{alg:DML-IL} requires the confounding noise horizon $k$ as input. While the exact value of $k$ can be difficult to obtain in reality, any upper bound $\bar{k}$ of $k$ is sufficient to guarantee the correctness of ~\cref{alg:DML-IL}, since $h_{t-\bar{k}}$ is also a valid instrument. Ideally, we would like a data-driven approach to determine $k$. Unfortunately, it is generally intractable to empirically verify whether $h_{t-k}$ is a valid instrument from a static dataset, especially the unconfounded instrument condition (i.e., $h_{t-k}\indep u^\epsilon_t$). Therefore, we rely on the user to provide a sensible choice of $\bar{k}$ based on the environment that does not substantially overestimate $k$.


\subsection{Theoretical Analysis}\label{sec:theory}

% \begin{align}
% p(u_t\lvert do(a_{t-k+1}),...,do(a_{t-1}),s_{t-k+1},...,s_{t-1})&\propto p(h_t)p_{\mu_0}(s_{t-k+1})\prod_{i=t-k+1}^{t-1} \transitions(s_{i+1}\lvert s_i,a_i,u_i)
% \end{align}

% since $$(u_t\indep a_{(t-k+1)...(t-1)} \lvert s_{(t-k+1)...(t_1)})_{\mathcal{G}_{\underline{a{(t-k+1)...(t-1)}}}}$$
% on the causal graph $\mathcal{G}_{\underline{a{(t-k+1)...(t-1)}}}$ where the arrows going into $a_{(t-k+1)...(t-1)}$ are removed.



In this section, we derive theoretical guarantees for our algorithm, focusing on the imitation gap and its relationship with existing work.


On a high level, in order to bound the imitation gap of the learnt policy $\hat{\pi}_h$, i.e., $J(\pi_E)-J(\hat{\pi}_h)$, we need to control:
\begin{enumerate}
    \item[($i$)] The amount of information about the hidden confounders that can be inferred from trajectory histories;
    \item[($ii$)] The ill-posedness (or identifiability) of the set of CMRs, which intuitively measures the strength of the instrument $h_{t-k}$;
    \item[($iii$)] The disturbance of the confounding noise to the states and actions at test time.
\end{enumerate}
These factors are all determined by the environment and the expert policy. To control ($i$), we measure how much information about $u^o_t$ is captured by the trajectory history $h_t$ by analysing the Total Variation (TV) distance between the distribution of $u^o_t$ and $\expectE[u^o_t\lvert h_t]$ along the trajectories of $\pi_E$. To control ($ii$) and ($iii$), we need to introduce the following two key concepts.

\begin{definition}[The ill-posedness of CMRs~\citep{Dikkala2020,Chen2012}]

Given the derived CMRs in~\cref{eq:CMR}, for a policy $\pi\in\Pi$, $\norm{\pi_E-\pi}_2$ is the root mean squared error to the expert and $\norm{\expectE[a_t-\pi(s_t)\lvert s_{t-k}]}_2$ is the CMR error we aim to minimise. Then, the \emph{ill-posedness} $\ill(\Pi,k)$ of the policy space with confounding noise horizon $k$ is given by
\begin{align*}
    \ill(\Pi,k)=\sup_{\pi\in\Pi} \frac{\norm{\pi_E-\pi}_{2}}{\norm{\expectE[a_t-\pi(h_t)\lvert h_{t-k}]}_{2}}.
\end{align*}
\end{definition}
The ill-posedness $\ill(\Pi,k)$ measures the strength of the instrument where a higher $\ill(\Pi,k)$ indicates a weaker instrument. It bounds the ratio between the learning error of the imitator following our CMR objective and its $L_2$ error to the expert policy. 

As discussed previously, intuitively, the strength of the instrument would decrease as the confounding horizon $k$ increases. This is in fact true and is confirmed by the following proposition. The proof is deferred to~\cref{appendix:prop}. 
\begin{proposition}\label{prop:ill-posed}
The ill-posedness $\ill(\Pi,k)$ is monotonically increasing as the confounded horizon $k$ increases.
\end{proposition}

Next, we introduce the notion of c-TV stability.
\begin{definition}[c-total variation stability~\citep{Bassily2021,Swamy2022_temporal}]
Let $P(X)$ be the distribution of a random variable $X:\Omega\rightarrow \mathcal{X}$. $P(X)$ is c-TV stable if for $a_1,a_2\in \mathcal{X}$ and $\Delta>0$,
\begin{align*}
\norm{a_1-a_2}\leq\Delta \implies \delta_{TV}(a_1+X,a_2+X)\leq c\Delta.
\end{align*}
where $\norm{\cdot}$ is some norm defined on $\mathcal{X}$ and $\delta_{TV}$ is the total variation distance.
\end{definition}
A wide range of distributions are c-TV stable. For example, standard normal distributions are $\frac{1}{2}$-TV stable. We apply this notion to the distribution over $u^\epsilon_t$ to bound the disturbance it induces in the trajectory and the expected return.

With the notion of ill-posedness and c-TV stability, we can now analyse and upper bound the imitation gap $J(\pi_E)-J(\hat{\pi}_h)$ by controlling the three components $(i)-(iii)$ discussed above. 
% We present the main result for this paper, where t
The full proof is deferred to~\cref{appendix:gap}.

\begin{theorem}[Imitation Gap Bound]\label{thm:gap}
Let $\hat{\pi}_h$ be the learnt policy with CMR error $\epsilon$ and let $\ill(\Pi,k)$ be the ill-posedness of the problem. Assume that $\delta_{TV}(u^o_t,\expectE_{\pi_E}[u^o_t\lvert h_t])\leq\delta$ for $\delta\in\realNumber^+$, $P(u^\epsilon_t)$ is c-TV stable and $\pi_E$ is deterministic. Then, the imitation gap is upper bounded by 
\begin{align*}
    J(\pi_E)-J(\hat{\pi}_h)\leq T^2\big(c\epsilon\ill(\Pi,k)+2\delta\big)=\mathcal{O}\big(T^2(\delta+\epsilon)\big).
\end{align*}
\end{theorem}
This upper bound scales at the rate of $T^2$, which aligns with the expected behaviour of imitation learning without an interactive expert~\citep{Ross2010}.
Next, we show that the upper bounds on the imitation gap from prior work~\citep{Swamy2022_temporal, Swamy2022} are special cases of
% of  subsumed by the unifying causal IL framework introduced in Section~\ref{sec:setting} are special cases of 
Theorem~\ref{thm:gap}. The proofs are deferred to~\cref{appendix:corollaries}.
\begin{corollary}\label{corollary:noUo}
In the special case that $u^o_t = 0$, i.e., there are no expert-observable confounders, or $u^o_t=\expectE_{\pi_E}[u^o_t\lvert h_t]$, i.e., $u^o_t$ is $\sigma(h_t)$ measurable (all information about $u^o_t$ is contained in the history), the imitation gap is upper bounded by
\begin{align*}
    J(\pi_E)-J(\hat{\pi}_h)\leq T^2\big(c\epsilon\ill(\Pi,k)\big)=\mathcal{O}\big(T^2\epsilon\big),
\end{align*}
which coincides with Theorem 5.1 of~\citet{Swamy2022_temporal}.
\end{corollary}

When there are no hidden confounders, i.e, $u^\epsilon_t=0$, our framework is reduced to that of~\citet{Swamy2022}. However, \citet{Swamy2022} provided an abstract bound that directly uses the supremum of key components in the imitation gap over all possible Q functions to bound the imitation gap. We further extend and concretise the bound using the learning error $\epsilon$ and the TV distance bound $\delta$ instead of relying on the suprema.


\begin{corollary}\label{corollary:unconfounded}
In the special case that $u^\epsilon_t=0$, if the learnt policy has optimisation error $\epsilon$,  the imitation gap is upper bounded by
\begin{align*}
    J(\pi_E)-J(\hat{\pi}_h)\leq T^2\left(\frac{2}{\sqrt{\dim(A)}}\epsilon+2\delta \right),
\end{align*}
which is a concrete bound that extends the abstract bound in Theorem 5.4 of~\cite{Swamy2022}.
\end{corollary}

\begin{remark}
\emph{If both $u^\epsilon_t$ and $u^o_t$ are zero, we then recover the classic setting of IL without confounders~\citep{Ross2010}, and the imitation gap bound is $T^2\epsilon$, where $\epsilon$ is the optimisation error of the algorithm.}
\end{remark}


\section{Experimental Results}
\begin{table*}[t]
\centering
\caption{Total Variation Distance on CIFAR-10-LT ($N_l = 500$, $M_l = 4000$) with different class imbalance ratios $\gamma_l$ and $\gamma_u$ under five different unlabeled class distributions.}
\label{tab:cifar10-tv}
\resizebox{\textwidth}{!}{
\begin{tabular}{lccccccccccc}
\toprule
& & \multicolumn{2}{c}{consistent} & \multicolumn{2}{c}{uniform} & \multicolumn{2}{c}{reversed} & \multicolumn{2}{c}{middle} & \multicolumn{2}{c}{head-tail} \\
\cmidrule(lr){3-4} \cmidrule(lr){5-6} \cmidrule(lr){7-8} \cmidrule(lr){9-10} \cmidrule(lr){11-12}
& & $\gamma_l = 150$ & $\gamma_l = 100$ & $\gamma_l = 150$ & $\gamma_l = 100$ & $\gamma_l = 150$ & $\gamma_l = 100$ & $\gamma_l = 150$ & $\gamma_l = 100$ & $\gamma_l = 150$ & $\gamma_l = 100$ \\
Model & Estimator & $\gamma_u = 150$ & $\gamma_u = 100$ & $\gamma_u = 1$ & $\gamma_u = 1$ & $\gamma_u = 1/150$ & $\gamma_u = 1/100$ & $\gamma_u = 150$ & $\gamma_u = 100$ & $\gamma_u = 150$ & $\gamma_u = 100$ \\
\midrule
Supervised & MLLS & 0.269 ± 0.252 & 0.038 ± 0.006 & 0.251 ± 0.046 & 0.255 ± 0.060 & 0.429 ± 0.028 & 0.493 ± 0.050 & 0.333 ± 0.042 & 0.320 ± 0.009 & 0.457 ± 0.034 & 0.444 ± 0.043 \\
Supervised & RLLS & 0.043 ± 0.001 & 0.044 ± 0.010 & 0.348 ± 0.034 & 0.305 ± 0.068 & 0.769 ± 0.016 & 0.678 ± 0.028 & 0.430 ± 0.008 & 0.368 ± 0.013 & 0.539 ± 0.018 & 0.503 ± 0.020 \\
\midrule
MLE & IPW & 0.027 ± 0.001 & 0.027 ± 0.000 & 0.319 ± 0.072 & 0.243 ± 0.010 & 0.674 ± 0.020 & 0.646 ± 0.041 & 0.438 ± 0.020 & 0.454 ± 0.026 & 0.547 ± 0.049 & 0.491 ± 0.059 \\
MLE & OR & 0.045 ± 0.004 & 0.042 ± 0.000 & 0.215 ± 0.026 & 0.203 ± 0.032 & 0.433 ± 0.017 & 0.395 ± 0.033 & 0.193 ± 0.006 & 0.209 ± 0.037 & 0.307 ± 0.147 & 0.249 ± 0.130 \\
MLE & DR & 0.090 ± 0.002 & 0.079 ± 0.000 & 0.407 ± 0.027 & 0.360 ± 0.007 & 0.425 ± 0.007 & 0.421 ± 0.029 & 0.256 ± 0.001 & 0.286 ± 0.031 & 0.435 ± 0.136 & 0.362 ± 0.122 \\
\midrule
EM & IPW & 0.035 ± 0.002 & 0.040 ± 0.001 & 0.021 ± 0.001 & 0.029 ± 0.015 & 0.303 ± 0.187 & 0.091 ± 0.010 & 0.119 ± 0.011 & 0.105 ± 0.022 & 0.104 ± 0.026 & 0.104 ± 0.051 \\
EM & OR & 0.037 ± 0.003 & 0.042 ± 0.002 & 0.016 ± 0.001 & 0.024 ± 0.012 & 0.269 ± 0.183 & 0.090 ± 0.008 & 0.122 ± 0.012 & 0.103 ± 0.022 & 0.072 ± 0.012 & 0.073 ± 0.024 \\
EM & DR & 0.034 ± 0.004 & 0.037 ± 0.001 & 0.014 ± 0.001 & 0.027 ± 0.020 & 0.264 ± 0.191 & 0.092 ± 0.005 & 0.111 ± 0.019 & 0.097 ± 0.026 & 0.077 ± 0.016 & 0.073 ± 0.028 \\
\midrule
SimPro & IPW & 0.070 ± 0.011 & 0.058 ± 0.000 & 0.046 ± 0.001 & 0.049 ± 0.005 & 0.254 ± 0.074 & 0.223 ± 0.098 & 0.097 ± 0.025 & 0.067 ± 0.002 & 0.105 ± 0.066 & 0.110 ± 0.079 \\
SimPro & OR & 0.071 ± 0.012 & 0.058 ± 0.000 & 0.045 ± 0.001 & 0.049 ± 0.006 & 0.040 ± 0.003 & 0.059 ± 0.017 & 0.074 ± 0.006 & 0.075 ± 0.002 & 0.033 ± 0.003 & 0.033 ± 0.003 \\
SimPro & DR & 0.017 ± 0.004 & 0.026 ± 0.001 & 0.019 ± 0.002 & 0.018 ± 0.003 & 0.039 ± 0.003 & 0.058 ± 0.025 & 0.091 ± 0.007 & 0.031 ± 0.001 & 0.015 ± 0.003 & 0.019 ± 0.007 \\
\bottomrule
\end{tabular}
}
\end{table*}


\begin{table*}[t]
\centering
\caption{Total Variation Distance on CIFAR-100-LT ($N_l = 50$, $M_l = 400$) with different class imbalance ratios $\gamma_l$ and $\gamma_u$ under five different unlabeled class distributions.}
\label{tab:cifar100-tv}
\resizebox{\textwidth}{!}{
\begin{tabular}{lccccccccccc}
\toprule
& & \multicolumn{2}{c}{consistent} & \multicolumn{2}{c}{uniform} & \multicolumn{2}{c}{reversed} & \multicolumn{2}{c}{middle} & \multicolumn{2}{c}{head-tail} \\
\cmidrule(lr){3-4} \cmidrule(lr){5-6} \cmidrule(lr){7-8} \cmidrule(lr){9-10} \cmidrule(lr){11-12}
& & $\gamma_l = 20$ & $\gamma_l = 10$ & $\gamma_l = 20$ & $\gamma_l = 10$ & $\gamma_l = 20$ & $\gamma_l = 10$ & $\gamma_l = 20$ & $\gamma_l = 10$ & $\gamma_l = 20$ & $\gamma_l = 10$ \\
Model & Estimator & $\gamma_u = 20$ & $\gamma_u = 10$ & $\gamma_u = 1$ & $\gamma_u = 1$ & $\gamma_u = 1/20$ & $\gamma_u = 1/10$ & $\gamma_u = 20$ & $\gamma_u = 10$ & $\gamma_u = 20$ & $\gamma_u = 10$ \\
\midrule
Supervised & MLLS & 0.707 ± 0.016 & 0.313 ± 0.100 & 0.445 ± 0.172 & 0.309 ± 0.119 & 0.383 ± 0.075 & 0.397 ± 0.006 & 0.570 ± 0.001 & 0.373 ± 0.107 & 0.543 ± 0.009 & 0.231 ± 0.057 \\
Supervised & RLLS & 0.520 ± 0.007 & 0.133 ± 0.003 & 0.337 ± 0.125 & 0.253 ± 0.082 & 0.424 ± 0.060 & 0.463 ± 0.003 & 0.454 ± 0.021 & 0.306 ± 0.074 & 0.460 ± 0.028 & 0.241 ± 0.040 \\
\midrule
MLE & IPW & 0.075 ± 0.000 & 0.071 ± 0.001 & 0.229 ± 0.001 & 0.167 ± 0.002 & 0.565 ± 0.005 & 0.443 ± 0.007 & 0.415 ± 0.000 & 0.311 ± 0.005 & 0.343 ± 0.000 & 0.280 ± 0.001 \\
MLE & OR & 0.065 ± 0.002 & 0.061 ± 0.001 & 0.200 ± 0.007 & 0.143 ± 0.001 & 0.526 ± 0.011 & 0.399 ± 0.023 & 0.360 ± 0.003 & 0.256 ± 0.012 & 0.328 ± 0.003 & 0.266 ± 0.005 \\
MLE & DR & 0.149 ± 0.019 & 0.145 ± 0.010 & 0.243 ± 0.004 & 0.214 ± 0.019 & 0.568 ± 0.005 & 0.464 ± 0.014 & 0.403 ± 0.014 & 0.309 ± 0.012 & 0.365 ± 0.007 & 0.320 ± 0.004 \\
\midrule
EM & IPW & 0.097 ± 0.008 & 0.092 ± 0.004 & 0.239 ± 0.007 & 0.179 ± 0.003 & 0.478 ± 0.012 & 0.329 ± 0.020 & 0.262 ± 0.016 & 0.202 ± 0.003 & 0.312 ± 0.002 & 0.227 ± 0.001 \\
EM & OR & 0.121 ± 0.007 & 0.108 ± 0.005 & 0.261 ± 0.007 & 0.189 ± 0.004 & 0.489 ± 0.013 & 0.335 ± 0.020 & 0.274 ± 0.016 & 0.211 ± 0.004 & 0.336 ± 0.003 & 0.235 ± 0.001 \\
EM & DR & 0.125 ± 0.005 & 0.111 ± 0.004 & 0.269 ± 0.007 & 0.194 ± 0.005 & 0.497 ± 0.010 & 0.336 ± 0.024 & 0.281 ± 0.019 & 0.219 ± 0.008 & 0.336 ± 0.007 & 0.233 ± 0.004 \\
\midrule
SimPro & IPW & 0.125 ± 0.001 & 0.100 ± 0.005 & 0.166 ± 0.007 & 0.141 ± 0.009 & 0.353 ± 0.023 & 0.261 ± 0.008 & 0.202 ± 0.003 & 0.158 ± 0.005 & 0.277 ± 0.009 & 0.197 ± 0.003 \\
SimPro & OR & 0.133 ± 0.005 & 0.100 ± 0.004 & 0.160 ± 0.007 & 0.138 ± 0.010 & 0.322 ± 0.014 & 0.253 ± 0.008 & 0.202 ± 0.003 & 0.156 ± 0.005 & 0.269 ± 0.006 & 0.191 ± 0.004 \\
SimPro & DR & 0.122 ± 0.003 & 0.106 ± 0.006 & 0.188 ± 0.009 & 0.149 ± 0.006 & 0.343 ± 0.023 & 0.257 ± 0.007 & 0.219 ± 0.010 & 0.172 ± 0.002 & 0.279 ± 0.007 & 0.198 ± 0.004 \\
\bottomrule
\end{tabular}
}
\end{table*}
\begin{table*}[t]
\centering
\caption{Top-1 accuracy (\%) on CIFAR-10-LT ($N_l = 500$, $M_l = 4000$) with different class imbalance ratios $\gamma_l$ and $\gamma_u$ under five different unlabeled class distributions. In most settings, our two stage algorithm improves SimPro (9 / 10) and BOAT (8 / 10). We use {\green green} to indicate when our plug-in improves and {\red red} when it degrades the base model.}
\label{tab:cifar10-acc}
\resizebox{\textwidth}{!}{
\begin{tabular}{lcccccccccc}
\toprule

& \multicolumn{2}{c}{consistent} & \multicolumn{2}{c}{uniform} & \multicolumn{2}{c}{reversed} & \multicolumn{2}{c}{middle} & \multicolumn{2}{c}{head-tail} \\
\cmidrule(lr){2-3} \cmidrule(lr){4-5} \cmidrule(lr){6-7} \cmidrule(lr){8-9} \cmidrule(lr){10-11}

& $\gamma_l = 150$ & $\gamma_l = 100$ & $\gamma_l = 150$ & $\gamma_l = 100$ & $\gamma_l = 150$ & $\gamma_l = 100$ & $\gamma_l = 150$ & $\gamma_l = 100$ & $\gamma_l = 150$ & $\gamma_l = 100$ \\
& $\gamma_u = 150$ & $\gamma_u = 100$ & $\gamma_u = 1$ & $\gamma_u = 1$ & $\gamma_u = 1/150$ & $\gamma_u = 1/100$ & $\gamma_u = 150$ & $\gamma_u = 100$ & $\gamma_u = 150$ & $\gamma_u = 100$ \\

\midrule

FixMatch & 62.9 $\pm$ 0.36 & 67.8 $\pm$ 1.13 & 67.6 $\pm$ 2.56 & 73.0 $\pm$ 3.81 & 59.9 $\pm$ 0.82 & 62.5 $\pm$ 0.94 & 64.3 $\pm$ 0.63 & 71.7 $\pm$ 0.46 & 58.3 $\pm$ 1.46 & 66.6 $\pm$ 0.87 \\
CReST+ & 67.5 $\pm$ 0.45 & 76.3 $\pm$ 0.86 & 74.9 $\pm$ 0.90 & 82.2 $\pm$ 1.53 & 62.0 $\pm$ 1.18 & 62.9 $\pm$ 1.39 & 58.5 $\pm$ 0.68 & 71.4 $\pm$ 0.60 & 59.3 $\pm$ 0.72 & 67.2 $\pm$ 0.48 \\
DASO & 70.1 $\pm$ 1.81 & 76.0 $\pm$ 0.37 & 83.1 $\pm$ 0.47 & 86.6 $\pm$ 0.84 & 64.0 $\pm$ 0.11 & 71.0 $\pm$ 0.95 & 69.0 $\pm$ 0.31 & 73.1 $\pm$ 0.68 & 70.5 $\pm$ 0.59 & 71.1 $\pm$ 0.32 \\
% w/ ACR$\dagger$ (Wei \& Gan, 2023) & 70.9 $\pm$ 0.37 & 76.1 $\pm$ 0.42 & 91.9 $\pm$ 0.02 & 92.5 $\pm$ 0.19 & 83.2 $\pm$ 0.39 & 85.2 $\pm$ 0.12 & 77.6 $\pm$ 0.20 & 79.3 $\pm$ 0.30 & 73.8 $\pm$ 0.83 & 79.3 $\pm$ 0.48 \\
% w/ SimPro & 74.2 $\pm$ 0.90 & 80.7 $\pm$ 0.30 & 93.6 $\pm$ 0.08 & 93.8 $\pm$ 0.10 & 83.5 $\pm$ 0.95 & 85.8 $\pm$ 0.48 & 82.6 $\pm$ 0.38 & 84.8 $\pm$ 0.54 & 81.0 $\pm$ 0.27 & 83.0 $\pm$ 0.36 \\
Supervised & 63.2 $\pm$ 0.14 & 66.0 $\pm$ 0.27 & 63.3 $\pm$ 0.28 & 65.8 $\pm$ 0.19 & 63.1 $\pm$ 0.19 & 65.9 $\pm$ 0.51 & 63.5 $\pm$ 0.22 & 65.8 $\pm$ 0.03 & 63.0 $\pm$ 0.18 & 66.4 $\pm$ 0.07 \\
\midrule
EM & 69.1 $\pm$ 1.29 & 73.8 $\pm$ 0.71 & 94.0 $\pm$ 0.08 & 93.2 $\pm$ 0.94 & 76.6 $\pm$ 2.72 & 82.2 $\pm$ 0.24 & 79.5 $\pm$ 0.35 & 81.6 $\pm$ 0.58 & 79.2 $\pm$ 0.50 & 79.8 $\pm$ 0.17 \\
\midrule
SimPro & 74.4 $\pm$ 0.71 & 79.7 $\pm$ 0.45 & 93.3 $\pm$ 0.10 & 93.3 $\pm$ 0.47 & 83.8 $\pm$ 0.80 & 84.1 $\pm$ 0.24 & 78.7 $\pm$ 0.30 & 84.2 $\pm$ 0.26 & 81.2 $\pm$ 0.20 & 82.0 $\pm$ 1.07 \\
% \midrule
SimPro+ & \green 77.8 $\pm$ 1.50 & \green 81.2 $\pm$ 0.39 & \green 93.7 $\pm$ 0.07 & \green 93.7 $\pm$ 0.24 & \red 83.3 $\pm$ 0.38 & \green 84.7 $\pm$ 0.78 & \green 79.2 $\pm$ 0.70 & \green 85.4 $\pm$ 0.66 & \green 81.3 $\pm$ 0.27 & \green 82.5 $\pm$ 0.56 \\
\midrule
BOAT & 80.5 $\pm$ 0.39 & 83.3 $\pm$ 0.27 & 93.9 $\pm$ 0.03 & 94.1 $\pm$ 0.10 & 79.7 $\pm$ 0.25 & 81.1 $\pm$ 0.15 & 79.7 $\pm$ 1.15 & 81.6 $\pm$ 0.09 & 79.4 $\pm$ 0.44 & 80.9 $\pm$ 0.16 \\
% \midrule
BOAT+ & \green 81.6 $\pm$ 0.15 & \green 83.8 $\pm$ 0.04 & \red 93.7 $\pm$ 0.23 & 94.1 $\pm$ 0.17 & \green 80.4 $\pm$ 0.71 & \green 81.7 $\pm$ 0.38 & \green 80.3 $\pm$ 0.28 & \green 83.1 $\pm$ 0.45 & \green 79.7 $\pm$ 0.29 & \green 81.0 $\pm$ 0.36 \\
\bottomrule
\end{tabular}
}
\end{table*}

\begin{table*}[t]
\centering
\caption{Top-1 accuracy (\%) on CIFAR-100-LT ($N_l = 50$, $M_l = 400$) with different class imbalance ratios $\gamma_l$ and $\gamma_u$ under five different unlabeled class distributions. Despite poor estimation in stage 1, our approach does not degrade the accuracy for most of the settings. We use {\green green} to indicate when our plug-in improves and {\red red} when it degrades the base method.}
\label{tab:cifar100-acc}
\resizebox{\textwidth}{!}{
\begin{tabular}{lccccccccccc}
\toprule

& \multicolumn{2}{c}{consistent} & \multicolumn{2}{c}{uniform} & \multicolumn{2}{c}{reversed} & \multicolumn{2}{c}{middle} & \multicolumn{2}{c}{head-tail} \\
\cmidrule(lr){2-3} \cmidrule(lr){4-5} \cmidrule(lr){6-7} \cmidrule(lr){8-9} \cmidrule(lr){10-11}

& $\gamma_l = 20$ & $\gamma_l = 10$ & $\gamma_l = 20$ & $\gamma_l = 10$ & $\gamma_l = 20$ & $\gamma_l = 10$ & $\gamma_l = 20$ & $\gamma_l = 10$ & $\gamma_l = 20$ & $\gamma_l = 10$ \\
& $\gamma_u = 20$ & $\gamma_u = 10$ & $\gamma_u = 1$ & $\gamma_u = 1$ & $\gamma_u = 1/20$ & $\gamma_u = 1/10$ & $\gamma_u = 20$ & $\gamma_u = 10$ & $\gamma_u = 20$ & $\gamma_u = 10$ \\

\midrule
% FixMatch & 40.0 $\pm$ 0.96 & 45.2 $\pm$ 0.55 & 39.6 $\pm$ 1.16 & \\
% CReST+ & 40.1 $\pm$ 1.28 & 44.5 $\pm$ 0.94 & 37.6 $\pm$ 0.88 & \\
% DASO & 43.0 $\pm$ 0.15 & 49.8 $\pm$ 0.24 & 49.4 $\pm$ 0.93 & \\
Supervised & 32.4 $\pm$ 0.40 & 38.4 $\pm$ 0.18 & 32.7 $\pm$ 0.25 & 38.0 $\pm$ 0.22 & 32.5 $\pm$ 0.51 & 38.4 $\pm$ 0.43 & 32.3 $\pm$ 0.08 & 37.9 $\pm$ 0.43 & 32.1 $\pm$ 0.33 & 38.2 $\pm$ 0.38 \\
% \midrule
EM & 42.4 $\pm$ 0.43 & 49.6 $\pm$ 0.30 & 50.9 $\pm$ 0.27 & 58.0 $\pm$ 0.35 & 42.1 $\pm$ 0.16 & 49.8 $\pm$ 0.47 & 42.8 $\pm$ 0.41 & 49.6 $\pm$ 0.36 & 41.5 $\pm$ 1.26 & 49.5 $\pm$ 0.18 \\
\midrule
SimPro & 42.5 $\pm$ 0.58 & 49.6 $\pm$ 0.22 & 51.7 $\pm$ 0.22 & 58.1 $\pm$ 0.53 & 44.9 $\pm$ 0.21 & 51.8 $\pm$ 0.42 & 42.7 $\pm$ 0.06 & 49.8 $\pm$ 0.45 & 43.3 $\pm$ 0.76 & 50.9 $\pm$ 0.19 \\
% \midrule
SimPro+ & \green 42.8 $\pm$ 0.49 & \green 50.1 $\pm$ 0.33 & \red 51.6 $\pm$ 0.63 & \red 57.8 $\pm$ 0.38 & \red 44.7 $\pm$ 0.51 & \red 51.4 $\pm$ 0.88 & \green 43.4 $\pm$ 0.58 & \green 50.4 $\pm$ 0.28 & \green 43.8 $\pm$ 0.50 & \red 50.7 $\pm$ 0.76 \\
\midrule
BOAT & 43.7 $\pm$ 0.16 & 51.4 $\pm$ 0.32 & 55.1 $\pm$ 0.95 & 60.5 $\pm$ 0.15 & 43.1 $\pm$ 0.49 & 52.7 $\pm$ 0.23 & 43.6 $\pm$ 0.19 & 51.4 $\pm$ 0.39 & 43.9 $\pm$ 0.42 & 51.4 $\pm$ 0.14 \\
% \midrule
BOAT+ & \green 44.8 $\pm$ 0.13 & 51.4 $\pm$ 0.51 & \red 53.8 $\pm$ 0.32 & 60.5 $\pm$ 0.69 & \green 43.4 $\pm$ 0.56 & \red 52.4 $\pm$ 0.36 & \green 43.9 $\pm$ 0.59 & \red 50.8 $\pm$ 0.09 & \red 43.6 $\pm$ 0.50 & \green 51.9 $\pm$ 0.49 \\
\bottomrule
\end{tabular}
}
\end{table*}

We perform experiments for each stage of our algorithm. In the first stage, we compare among various methods to estimate the unlabeled class distribution $P(Y|A=0)$, showing that SimPro + DR performs well. In the second stage, we freeze the unlabeled class distribution, using our best estimator  SimPro + DR, and plug it into 2 SOTA semi-supervised learning algorithms, SimPro and BOAT~\cite{boat}. We show that this simple procedure improves the existing methods, and is even capable of improving the original SimPro when used for both stages.


% \textbf{Datasets} We adopt 4 standard benchmarks used frequently in other semi-supervised learning work: CIFAR-10, CIFAR-100~\cite{cifar}, STL-10~\cite{stl10} and Imagenet-127~\cite{cossl}. To match our RTSSL setting, we create long-tailed labeled and unlabeled sets from CIFAR-10 and CIFAR-100. Specifically, we use $\gamma_l$ and $n_1$ to denote the imbalance ratio and the head class's number of samples of the labeled data, the remaining class's size is computed as $n_c = n_1 \times \gamma_l^{-\frac{c-1}{C-1}}$ and likewise, $\gamma_u$ and $m_1$ of the unlabeled data. For CIFAR-10, we fix $n_1=500$ and $m_1=4000$. We test 2 different configurations $\gamma_l=\gamma_c=150$ and $\gamma_l=\gamma_c=100$. We further permute classes the unlabeled sets in 5 ways: consistent, uniform, reversed, middle and headtail, similar to \cite{simpro} and visualized in figure~\ref{fig:distribution}, which results in 10 different datasets in total. Similarly for CIFAR-100, we fix $n_1=500$ and $m_1=4000$, use 2 configurations $\gamma_l=\gamma_c=20$ and $\gamma_l=\gamma_c=10$, and permute the classes in 5 ways, resulting in 10 datasets as well. For STL-10, the unlabeled set has no ground truth labels, therefore we use all samples in the head class and set the imbalance ratio $\gamma_l$ to $10$ or $20$. Imagenet-127 is a naturally long-tailed dataset with 127 classes. We train on 32x32 and 64x64 image resolutions following ~\cite{cossl}.


\textbf{Datasets} We evaluate our method on four standard semi-supervised learning benchmarks: CIFAR-10, CIFAR-100~\cite{cifar}, STL-10~\cite{stl10}, and Imagenet-127~\cite{cossl}. To simulate RTSSL, we construct long-tailed labeled and unlabeled sets for CIFAR-10 and CIFAR-100. The labeled data follows an imbalance ratio $\gamma_l$ with head class size $n_1$, while the remaining class sizes are computed as $n_c = n_1 \times \gamma_l^{-\frac{c-1}{C-1}}$. The unlabeled data follows a similar setup with $\gamma_u$ and $m_1$.  

For CIFAR-10, we set $n_1 = 500$, $m_1 = 4000$, and test two configurations: $\gamma_l = \gamma_u = 150$ and $\gamma_l = \gamma_u = 100$. We generate 10 datasets by permuting the unlabeled class distributions in five ways: \textit{consistent, uniform, reversed, middle}, and \textit{head-tail}, as in~\cite{simpro}. CIFAR-100 follows the same setup with $n_1 = 50$, $m_1 = 400$, and $\gamma_l, \gamma_u$ values of 20 and 10.  

For STL-10, where unlabeled data lacks ground-truth labels, we use all head-class samples and set $\gamma_l$ to 10 or 20. Imagenet-127 is naturally long-tailed with 127 classes, and we train on 32$\times$32 and 64$\times$64 resolutions as in~\cite{cossl}.


\paragraph{Training.} We follow the implementation and hyperparameter settings of \cite{simpro}. We defer these details in \cref{subsec:training-setting}. One important exception is that for Imagenet-127, we use the smaller Wide ResNet-28-2 in stage 1 and the larger ResNet-50 for stage 2, to demonstrate that a smaller model is sufficient for distribution estimation.


\begin{table}[t]
\small
\centering
\caption{Top-1 Accuracy (\%) on STL-10. Our two-stage algorithms improves both SimPro and BOAT for both settings.}
\label{tab:stl10-acc}
% \resizebox{\linewidth}{!}{
\begin{tabular}{lcc}
\toprule
Method & $\gamma_l=10$ & $\gamma_l=20$ \\ \hline
Supervised & 73.9 $\pm$ 0.57 & 70.4 $\pm$ 0.95 \\
\midrule
MLE & 67.6 $\pm$ 0.57 & 58.9 $\pm$ 4.05 \\
\midrule
EM & 84.9 $\pm$ 0.14 & 83.6 $\pm$ 0.25 \\
\midrule
SimPro & 82.4 $\pm$ 1.57 & 80.5 $\pm$ 0.96 \\
SimPro+ & \green 83.9 $\pm$ 0.76 & \green 82.7 $\pm$ 0.86 \\
\midrule
BOAT & 83.8 $\pm$ 0.20 & 82.0 $\pm$ 0.34 \\
BOAT+ & \green 84.1 $\pm$ 0.38 & \green 82.4 $\pm$ 0.10 \\
\bottomrule
\end{tabular}
\end{table}















\begin{table}[t]
% \setlength{\tabcolsep}{3.5pt}
\small
\centering
\caption{Top-1 Accuracy (\%) on Imagenet-127. Our two-stage approach improves both SimPro and BOAT for both resolutions.}
\label{tab:imagenet-127-acc}
% \resizebox{\linewidth}{!}{
\begin{tabular}{lcc}
\toprule
Method & $32 \times 32$ & $64 \times 64$ \\ \hline
SimPro & 54.8 & 63.7 \\
SimPro+ & \green 55.1 & \green 64.2 \\
\midrule
BOAT & 51.6 & 58.7 \\
BOAT+ & \green 52.0 & \green 59.2 \\

\bottomrule
\end{tabular}
% }
\end{table}


\begin{table}[t]
% \setlength{\tabcolsep}{3.5pt}
\small\centering
\caption{Total Variation Distance on Imagenet-127}
\label{tab:imagenet-127-tv}
% \resizebox{\linewidth}{!}{
\begin{tabular}{cccc}
\toprule
Method & Estimator & $32 \times 32$ & $64 \times 64$ \\ \hline
MLE & IPW  & 0.103 $\pm$ 0.034 & 0.051 $\pm$ 0.000 \\
MLE & OR  & 0.153 $\pm$ 0.052 & 0.041 $\pm$ 0.000 \\
MLE & DR  & \green 0.100 $\pm$ 0.029 & \green 0.075 $\pm$ 0.003 \\
\midrule
EM & IPW  & 0.141 $\pm$ 0.006 & 0.163 $\pm$ 0.010 \\
EM & OR  & 0.205 $\pm$ 0.006 & 0.236 $\pm$ 0.011 \\
EM & DR  & \green 0.024 $\pm$ 0.001 & \green 0.042 $\pm$ 0.004 \\
\midrule
SimPro & IPW  & 0.041 $\pm$ 0.012 & 0.224 $\pm$ 0.040 \\
SimPro & OR  & 0.036 $\pm$ 0.014 & 0.291 $\pm$ 0.079 \\
SimPro & DR  & \green 0.017 $\pm$ 0.000 & \green 0.037 $\pm$ 0.004 \\
\bottomrule
\end{tabular}
% }
\end{table}

\subsection{Better results on label distribution} 
\label{subsec:label}
We have mentioned various ways throughout the papers to estimate the unlabeled class distribution. In what follows, method consists of a model, which is how the learning is done, and an estimator, which is how the final distribution is estimated using parameters learned from the model.

%\begin{enumerate}
%\item 
\noindent
\textbf{Supervised}. The model is trained on the labeled set only and used to estimate the unlabeled class distribution \cite{unifiedlabelshift}. 2 successful estimators for this setting are \textbf{RLLS} \cite{rlls} and \textbf{MLLS} \cite{mlls}. 

%\item 
\noindent\textbf{MLE}. The model is trained by directly maximizing the likelihood \cref{eq:likelihood}. We also use the decomposition $P(Y|X)$ and $P(A|Y)$, and write the unlabeled term as $P(A=0, X) = \sum_{c} P(Y=c|X) P(A=0|Y=c)$, which enables gradient descent training on these parameters. This is also the MLE method to estimate $P(A|Y)$ in \cite{arelabelsinformative}.

%\item 
\noindent\textbf{EM}. We further test the EM algorithm in \cref{subsec:em}. In particular we also use strong and weak augmentations similar to FixMatch, but not the pseudo labeling operator. Confidence thresholding removes the soft predictions of the non-dominant classes, which may be better to keep since our target of the first stage is the global class statistics. We also try 3 estimators on this model.

%\item 
\noindent\textbf{SimPro} \cite{simpro} can be seen as our previous EM but also with FixMatch's confidence thresholding and logit adjustment loss in \cref{subsec:simpro}. Confidence thresholding is a powerful regularization technique that encodes the assumption that classes are well separated \cite{entropyminimization}, but can introduce bias to the estimation, which justifies the use of DR.
%\end{enumerate}

% For semi-supervised methods MLE, EM and SimPro, as we now have additional information on the missingness mechanism, we can use 3 estimators OR, IPW and DR presented in \cref{subsec:2-stage}


Results on \cref{tab:cifar10-tv} presents the performance of various models and estimators on CIFAR-10. We can see that SimPro + DR performs best. In contrast, SimPro + OR, SimPro's original way of estimating $P(Y|A=0)$, and SimPro + IPW tend to underperform EM on the consistent and uniform datasets. The consistent setting is worth noting, since it arises when data is sampled uniformly at random for labeling,  representative of a large number of real world situations. EM is competitive to SimPro as well even without pseudo labeling, but overall we found this regularization to offer significant gains in the reversed, middle and head-tail settings. Finally, Supervised with either MLLS or RLLS estimators performs much worse than the semi-supervise methods.

\cref{tab:imagenet-127-tv} aligns with the observations  made in \cref{tab:cifar10-tv}. In particular, SimPro + DR is the best method for class distribution estimation of the much larger Imagenet-127. We also found that a small neural network and a small image resolution is sufficient for the distribution estimation of the much larger dataset Imagenet-127. This matches our intuition that the finite-dimensional parameter is easier to learn.

\cref{tab:cifar100-tv} shows that most methods understandably struggle to estimate the class distributions in CIFAR-100. This is because there are few samples in each class, the head class has 10 times less samples while the number of classes multiplies 10 times compared to CIFAR-10. We see here that SimPro + DR is not the best method, but the relative gap between estimators are small.

% Among the models, the supervised baseline do not perform well even in the consistent setting, showing that when unlabeled data is available during training, learning from them can be valuable for class distribution estimation, especially in the cases with little labeled data like ours. Both the MLE and supervised models perform badly on the reversed, middle and head-tail settings

% Among the estimators, we see that DR boosts the performance of SimPro and EM in CIFAR-10, and of all semi-supervised models in Imagenet-127. It does not improve MLE on CIFAR-10, and it does not improve on CIFAR-100. However, for most of the time, the decrease is not much. In constrast, IPW estimators can be significantly worse, for example in the reversed setting of CIFAR-10, where the distance is $0.254$ for $\gamma_l=150$ and $0.233$ for $\gamma_l=100$, compared to OR's 0.040 and 0.059. 

% Both the MLE and supervised models perform badly on the reversed, middle and head-tail settings. EM does a decent job, though not as well as SimPro, on all 5 distribution settings of CIFAR-10. However, on Imagenet-127, EM without DR performs worse than MLE.

% We note that the performance on DR is similar to OR in these cases, showing that DR has a double robustness property. While IPW only relies on the finite-dimensional $P(A|Y)$, which intuitively is easy to estimate, we found that the inverse probability weight can nevertheless be unstable when some probabilities are small, and this is where DR shows its strength by combining both IPW and OR.



\subsection{Two-stage algorithm improves accuracy}

In the second stage of our algorithm, we freeze our estimation and plug it in SimPro and BOAT. We denote SimPro+ and BOAT+ for algorithms that use our first stage estimate.



\cref{tab:cifar10-acc} shows that for CIFAR-10 SimPro+ and BOAT+ improve over their original versions across most settings, leading to large improvements in both the consistent and middle class distribution settings. In particular, our two-stage approach improves SimPro in 9 / 10 settings and BOAT in 8 / 10 settings.
We also observe consistent improvements ove both base algorithms, SimPro and BOAT, for several other datasets. \cref{tab:stl10-acc} demonstrates improvements for 2 / 2 class imbalance ratios in STL-10 and \cref{tab:imagenet-127-acc} for 2 / 2  different resolutions of ImageNet-127. 


We also evaluate on CIFAR-100 for multiple unlabeled  class distribution settings and with mediocre class label distribution estimates in stage 1, demonstrate no degradation in accuracy in stage 2. As shown in \cref{tab:cifar100-acc}, the two stage algorithm with a mediocre stage 1 estimation leads to parity with the baseline. Stage 2 provides small improvements in 5 / 10 settings for SimPro and in 4 / 10 (with 2 ties) for BOAT.


\subsection{Ablation Study: Alternative implementations.}
\label{subsec:ablation-1}
In this section, we ablate on our 2-stage choice. Specifically, we consider 2 alternative implementations:
\paragraph{\textbf{Doubly-robust risk}}  
This approach is \cite{arelabelsinformative, onnonrandommissinglabels}, as discussed in \cref{sec:background}. we consider the doubly-robust risk as our training loss. We use the missingness mechanism estimation from stage-1 of SimPro+ for fair comparison. \cref{eq:dr-risk} is used for training in which the pseudo-labeling operators can be applied straightforwardly. More detail in \cref{subsec:dr-risk}
\paragraph{\textbf{Batch-update doubly-robust $P(Y|A)$}} Different from SimPro+, here we remove the first stage and instead update our doubly robust estimation of the unlabeled class distribution using a moving average of the batch statistics.

\cref{tab:cifar10-ablation-1} shows that the batch-update version of SimPro+ is significantly worse on the consistent and uniform settings, while the doubly-robust risk is worst overall, especially in the reversed setting where $P(A|Y)$ is very small for the labeled tail classes, causing instability issues during training. In conclusion, our 2-stage approach is the best.


\begin{table}[t]
\small
\centering
\caption{Top-1 Accuracy (\%) on CIFAR-10. We compare our 2-stage SimPro+ with 1) an 1-stage alternative that updates and uses the doubly-robust estimation on-the-fly and 2) SimPro with doubly-robust risk. We use $\gamma_l=150$. {\green green} color indicates that our method performs best.}
\label{tab:cifar10-ablation-1}
\resizebox{\linewidth}{!}{
\begin{tabular}{lccccc}
\toprule
Method & consistent & uniform & reversed & middle & headtail\\ \hline
SimPro+ & \green 77.8 & \green 93.7 & \green 83.3 & \green 79.2 & \green 81.3 \\
batch-update & 71.9 & 91.4 & 82.6 & 78.6 & 81.2 \\
DR-risk & 72.1 & 89.8 & 67.1 & 75.6 & 79.5 \\
\bottomrule
\end{tabular}
}
\end{table}
\section{Conclusions}

Video generation poses a unique challenge, requiring the modeling of both spatial interactions and temporal dynamics. Despite impressive advancements, video models continue to struggle with temporal coherence, even for basic motions well-represented in training datasets (Fig.~\ref{fig:failures}). In this work, we identify the training objective as a key factor that prioritizes appearance fidelity over motion coherence.

To address this, we propose VideoJAM, a framework that equips video models with an explicit motion prior. The core idea is intuitive and natural: a single latent representation captures both appearance and motion jointly. Using only two additional linear layers and no additional training data, VideoJAM significantly improves motion coherence, achieving state-of-the-art results even against powerful proprietary models.
Our approach is generic, offering numerous opportunities for future enhancement of video models with real-world priors such as complex physics, paving the way for holistic modeling of real-world interactions.

\section*{Impact Statements}
The primary goal of this work is to advance motion modeling in video generation, empowering models to understand and represent the world more faithfully. As with any technology in the content generation field, video generation carries the potential for misuse, a concern that is widely discussed within the research community. However, our work does not introduce any specific risks that were not already present in previous advancements. We strongly believe in the importance of developing and applying tools to detect biases and mitigate malicious use cases, ensuring the safe and fair use of generative tools, including ours.

\bibliography{main}
\bibliographystyle{icml2025}


%%%%%%%%%%%%%%%%%%%%%%%%%%%%%%%%%%%%%%%%%%%%%%%%%%%%%%%%%%%%%%%%%%%%%%%%%%%%%%%
%%%%%%%%%%%%%%%%%%%%%%%%%%%%%%%%%%%%%%%%%%%%%%%%%%%%%%%%%%%%%%%%%%%%%%%%%%%%%%%
% APPENDIX
%%%%%%%%%%%%%%%%%%%%%%%%%%%%%%%%%%%%%%%%%%%%%%%%%%%%%%%%%%%%%%%%%%%%%%%%%%%%%%%
%%%%%%%%%%%%%%%%%%%%%%%%%%%%%%%%%%%%%%%%%%%%%%%%%%%%%%%%%%%%%%%%%%%%%%%%%%%%%%%
\newpage
\appendix
\onecolumn
%%%%%%%%%%%%%%%%%%%%%%%%%%%%%%%%%%%%%%%%%%%%%%%%%%%%%%%%%%%%%%%%%%%

\section{Compositional Guidance vs. Inner-Guidance}
\label{sec:IP2P}
\citet{Liu2022CompositionalVG} proposed \emph{Composable Diffusion Models} where a diffusion model can be conditioned on several signals $c_1, \dots, c_n$. The model's conditional sampling distribution is, therefore,
\begin{align}
    p_\theta(x | c_1, \dots, c_n) = \frac{p_\theta(x,c_1, \dots, c_n)} {p_\theta (c_1,\dots, c_n)} \propto p_\theta(x, c_1, \dots, c_n) \propto p_\theta(x) \prod_{i=1}^n p_\theta(c_i | x).
\end{align}
where $\theta$ represents the model weights, and $p$ is the sampling distribution. Importantly, this formulation assumes that $c_1,\dots, c_n$ are \emph{independent of each other and the weights of the model $\theta$}, allowing to drop the denominator $p_\theta (c_1,\dots, c_n)$. Notice that this assumption does not hold in our setting, where the motion condition $d_t$ is noisy and strictly dependent on the neural network, as one of its outputs, as well as the text conditioning, as it serves as another input to the model. 

Inspired by \citet{Liu2022CompositionalVG}, InstructPix2Pix (IP2P)~\cite{brooks2022instructpix2pix} used a similar compositional formulation to extend Classifier-Free Guidance~\cite{ho2022classifier} to two conditioning signals. Formally, given two conditions $c_1, c_2$,
\begin{align}
    p_\theta(x | c_1, c_2) = \frac{p_\theta(x, c_1, c_2)} {p_\theta (c_1, c_2)} =\frac{p_\theta (c_1 | c_2,x) p_\theta (c_2 | x) p_\theta (x)}{p_\theta (c_1, c_2)},
\end{align}
taking the log derivative this gives us,
\begin{align}
    \nabla \log p_\theta(x | c_1, c_2) =\nabla \log p_\theta (c_1 | c_2,x) + \nabla \log p_\theta (c_2 | x) p_\theta (x) - \nabla \log p_\theta (c_1, c_2),
    \label{eq:ip2p}
\end{align}
next, the IP2P formulation assumes (similar to \citet{Liu2022CompositionalVG}) that we can omit the term $p_\theta (c_1, c_2)$ since it is independent of $\theta$, which is again incorrect in our case.

For completeness, our ablations in Sec.~\ref{sec:experiments} compare our Inner-Guidance formulation with that of IP2P, and find that this theoretical gap causes significant degradation in the performance. The direct interpretation of Eq.~\ref{eq:ip2p} to VideoJAM employed in our experiments is as follows,
\begin{align*}
    \bold{\Tilde{u}^+}([x_t, d_t], y, t; \theta') = \bold{u^+}([x_t, \emptyset]), \emptyset, t; \theta') + \quad \quad \\
     w_1\cdot \left( \bold{u^+}([x_t, d_t], \emptyset, t; \theta') - \bold{u^+}([x_t, \emptyset]), \emptyset, t; \theta') \right) + \\
     w_2\cdot \left( \bold{u^+}([x_t, d_t], y, t; \theta') - \bold{u^+}([x_t, d_t], \emptyset, t; \theta') \right) \quad
\end{align*}   
where the notations follow Sec.~\ref{sec:inner_guidance}, and we employ the same guidance scales as we do for Inner-Guidance, i.e. $w_1=3, w_2=5$. Note that the notations for $w_1, w_2$ are reversed with respect to Eq.~\ref{eq:guidance} since IP2P condition on the visual signal first and the textual signal second and order matters for IP2P, while our Inner-Guidance formulation is order invariant.


\section{Motivation Experiments} 
\label{sec:motivation_supp}
\begin{wrapfigure}{R}{0.5\textwidth}
\vspace{-26px}
\centering
\noindent
\includegraphics[width=1.01\linewidth, clip]{figures/motivation_fig_supp.pdf}
\vspace{-22px}
\caption{\textbf{Qualitative motivation.} We noise input videos to different timesteps ($20, 60,80$) and continue the generation. By step $60$, the video's coarse motion and structure are mostly determined.
\label{fig:motivation_supp}}
\vspace{-14px}
\end{wrapfigure}
To exemplify that steps $t\leq 60$ of the generation are indeed meaningful to determine the motion, we conduct an SDEdit~\cite{meng2021sdedit} experiment, in which we noise videos to different timesteps ($20, 60, 80$), and continue the generation given the noised videos. In Fig.~\ref{fig:motivation_supp}, we show a representative appearance frame and two motion frames for each video, using RAFT~\cite{raft} to estimate optical flow. We observe that the coarse motion and structure of the generated videos are determined between steps $20$ and $60$, since the generation from step $20$ changes the entire video while starting from step $60$ maintains the coarse motion and structure of the input video, suggesting that they are already determined by the input noisy video.
Note that the appearance may still change between steps $60$ and $80$ (right), whereas from step $80$, both appearance and motion seem to be determined.

\begin{figure*}[t!]
\centering
\includegraphics[width=0.99\textwidth]{figures/qualitative_supp.pdf}
\vspace{-10px}
\caption{\textbf{Additional text-to-video results using VideoJAM-30B.} }
\label{fig:qualitative_supp}
\vspace{-4px}
\end{figure*}

\section{Implementation Details}
\label{sec:implementation_details}
VideoJAM-4B was fine-tuned using $32$ A100 GPUs with a batch size of $32$ for $50,000$ iterations on a spatial resolution of $256\times 256$. It has a latent dimension of $3072$ and $32$ attention blocks (same as the base model). 
VideoJAM-30B was fine-tuned using $256$ A100 GPUs with a batch size of $256$ for $35,000$ iterations on a spatial resolution of $256\times 256$. It has a latent dimension of $6144$ and $48$ attention blocks (same as the base model). Each attention block is constructed of a self-attention layer that performs spatiotemporal attention between all the video tokens, and a cross-attention layer that integrates the text.
Both models were trained with a fixed learning rate of $5e-6$, using the Flow Matching paradigm~\cite{flow-matching} (see Sec.~\ref{sec:preliminaries}). 

During inference, we perform $100$ denoising steps with a linear quadratic t-schedule using a text guidance scale of $w_1=5$ and a motion guidance scale of $w_2=3$ (see Eq.~\ref{eq:guidance}), other than the ablations that test these components. Additionally, we only employ the motion guidance for the first half of the generation steps ($50$ steps) following the conclusions from our motivational experiments (Sec.~\ref{sec:motivation}), as these are the steps that determine the coarse motion in the video, and display less sensitivity to temporal incoherence before applying VideoJAM. In practice, Inner-Guidance is performed similarly to Classifier-Free Guidance~\cite{ho2022classifier}, where all results are generated in a batch 
$\bold{u^+}([x_t, d_t]), y, t; \theta'), \bold{u^+}([x_t, d_t], \emptyset, t; \theta'), \bold{u^+}([x_t, \emptyset], y, t;)$ and the final prediction is calculated following Eq.~\ref{eq:guidance}. The models are trained to generate $128$ frame videos at 24 frames per second, resulting in 5-second video generations.

The models operate in the latent space of a TAE, as specified in Sec.~\ref{sec:preliminaries}. The TAE structure follows that of~\citet{moviegen}, with a temporal compression rate of $\times8$ and a spatial compression rate of
$8\times8$. The Transformer patch size is $1\times 2\times2$.  The text prompt conditioning is processed by three different text encoders: UL2~\cite{tay2022ul2}, ByT5~\cite{xue2022byt5}, and MetaCLIP~\cite{xu2023demystifying}.

Both DiT models were pre-trained using the framework in Sec.~\ref{sec:preliminaries} on a dataset of $\mathcal{O}(100 \text{ M})$ videos. We then fine-tune the models using VideoJAM on under $3$ million random samples from the model's original training set, which constitute less than $3\%$ of the training videos. This allows our fine-tuning phase to be light and efficient. During this fine-tuning, we employ RAFT~\cite{raft} to obtain optical flow per training video.

 
Since each of the baselines generates videos in different resolutions, we resize the baseline results to a $256$ resolution to facilitate a fair and unbiased comparison. No cherry-picking is involved in the evaluation of any of the models, and the first result obtained by each model is taken. All baselines produce the same length of videos ($5$ seconds), therefore we only resize the videos spatially.
For the qualitative results in the website, we train an additional super-resolution model to spatially upsample the $256\times256$ videos to $512\times512$ videos. The training regime follows that of VideoJAM-30B. Note that all our experiments (besides the visualizations on the website) are in the lower $256$ resolution due to resource limitations.

\subsection{VBench Metrics} 
\label{sec:vbench}
We employ all metrics supported by VBench on both VideoJAM-bench and the Movie Gen benchmark. Inspired by the protocol in the VBench paper, we split the metrics into a motion category and an appearance category.
For the appearance category, we include the aesthetic quality and image quality metrics, which assess the per-frame quality of the generated videos, as well as subject consistency and background consistency, which assess the model's ability to maintain a consistent appearance. 
For motion comprehension, we include the motion smoothness score, which aims to assess the realism of the motion, and the dynamic degree score which estimates the amount of motion in the generated videos. In other words, the motion score measures the model's ability to generate meaningful motion (i.e., non-static videos) that is also coherent and plausible.

All scores are normalized and a weighted score is calculated according to the weights suggested in the VBench paper. 
The full results of all VBench metrics for each benchmark are reported in App.~\ref{sec:motion_benchmark},~\ref{sec:moviegen_benchmark}.


\section{VideoJAM-bench: Automatic Metrics Breakdown and Prompts}
\label{sec:motion_benchmark}
In the following, we provide a breakdown of the automatic metrics calculated on our motion benchmark using VBench~\cite{huang2023vbench} for the 4B model (Tab.~\ref{tab:4b_metrics}) and the 30B model (Tab.~\ref{tab:30b_metrics}). As mentioned in App.~\ref{sec:vbench}, the motion metrics measure the amount of motion in the video and the coherence of the motion. In the smaller model category, CogVideo2B scores the highest dynamic degree and the lowest motion smoothness. This indicates that while there is abundant motion in the generated videos, it is incoherent. The DiT-4B base model obtains the best smoothness score, and the worst dynamic degree, indicating that it produces videos with very subtle movements. As can be observed, VideoJAM strikes the best balance, where plenty of motion is generated while maintaining strong coherence.  

For the larger DiT-30B model, we observe, again, that there is a trade-off between the dynamic degree and the motion smoothness, where CogVideo5B produces the most motion, yet it is incoherent. Among the competitive proprietary baselines, notice that Runway Gen 3 obtains a very high dynamic degree, yet it has the lowest motion smoothness among all the proprietary baselines (Runway Gen 3, Sora, Kling 1.5). In Fig.~\ref{fig:qualitative}, we show comparisons to Sora and Kling since these are the most competitive with VideoJAM according to the human evaluation, which is generally considered to be a more reliable evaluation form~\cite{Lumiere,moviegen,Wang2024MotiF}. However, in the website, we include a comparison to Runway Gen 3 in addition to Sora and Kling for completeness. Furthermore, Kling shows the best motion smoothness, with the lowest dynamic degree. Observe that VideoJAM, again, strikes the best balance between motion coherence and the amount of generated motion. Additionally, it outperforms the base model (DiT-30B) across all motion metrics, and nearly all appearance metrics, indicating that our method improves all aspects of the generation.

A full list of the prompts considered in our motion benchmark is provided in App.~\ref{sec:motion_benchmark_prompts}.

\begin{table}[h!]
    \caption{\textbf{Breakdown of the automatic metrics} from VBench comparing our 4B model and previous work on VideoJAM-bench. Our method strikes the best balance between the dynamic degree (higher implies more motion) and the motion smoothness (higher implies smooth motion).}
  \label{tab:4b_metrics}
  \centering
    \setlength{\tabcolsep}{3.5pt}
  \scalebox{0.95}{%
  \begin{tabular}{@{}lcccccc@{}}
    \toprule
      & \multicolumn{4}{c}{Appearance Metrics} & \multicolumn{2}{c}{\textbf{Motion Metrics}}  \\
      \cmidrule(r){2-5}
      \cmidrule(r){6-7}
    \multirow{2}{*}{Method}     &      \small{Aesthetic}        &  \small{Image}     &        \small{Subject}  & \small{Background}        & \textbf{\small{Motion}}& \textbf{\small{Dynamic}}\\
    & \small{Quality}        &  \small{Quality}     &        \small{Consistency}  & \small{Consistency}        & \textbf{\small{Smoothness}}& \textbf{\small{Degree}}\\
    \midrule
    \small{CogVideo2B}   &  46.9 &	48.9 &	87.8 &	93.9 & {97.1} &	\textbf{88.6}   \\
    \small{CogVideo5B}   &  51.1 &	52.9 &	91.3 &	\underline{95.3}  & {97.3}	&{87.5}   \\
    \midrule
    \small{DiT-4B} &   \textbf{51.8} &	\textbf{61.4} &	\underline{93.0} &	\textbf{96.7} & \textbf{99.3} &	{38.3}  \\
    \textbf{+VideoJAM-4B}  & \underline{51.6} &	\underline{61.1} &	\textbf{93.5} &	\textbf{96.7} & \underline{98.8} &	\underline{87.5}   \\
    \bottomrule
    \end{tabular}}
\end{table}

\begin{table}[h!]
    \caption{\textbf{Breakdown of the automatic metrics} from VBench comparing our 30B model and previous work on VideoJAM-bench. Our method strikes the best balance between the dynamic degree (higher implies more motion) and the motion smoothness (higher implies smooth motion).}
  \label{tab:30b_metrics}
  \centering
    \setlength{\tabcolsep}{3.5pt}
  \scalebox{0.95}{%
  \begin{tabular}{@{}lcccccc@{}}
    \toprule
      & \multicolumn{4}{c}{Appearance Metrics} & \multicolumn{2}{c}{\textbf{Motion Metrics}}  \\
      \cmidrule(r){2-5}
      \cmidrule(r){6-7}
    \multirow{2}{*}{Method}     &      \small{Aesthetic}        &  \small{Image}     &        \small{Subject}  & \small{Background}        & \textbf{\small{Motion}}& \textbf{\small{Dynamic}}\\
    & \small{Quality}        &  \small{Quality}     &        \small{Consistency}  & \small{Consistency}        & \textbf{\small{Smoothness}}& \textbf{\small{Degree}}\\
    \midrule
    \small{CogVideo5B}   &  51.1 &	52.9 &	91.3 &	{95.3}  & {97.3}	&\textbf{87.5}   \\
    \small{RunWay Gen3} &    55.1 &	55.1&	90.7	& 95.2 & {98.4} &	\underline{84.4}  \\
    \small{Mochi} &   49.5	& 48.8 &	89.7 &	95.2 & {98.4} &	{78.1}  \\
    \small{Sora} &  \underline{56.8} &	\underline{57.7} &	\underline{93.0} &	\underline{96.4} & {98.7}	& {82.0}  \\
    \small{Kling 1.5} &  \textbf{58.5} &	\textbf{60.4} &	\textbf{93.9} &	\textbf{96.5} & \textbf{99.2}	& {64.8}  \\
    \midrule
    \small{DiT-30B} &   49.2 &	56.8 &	91.3 &	95.5 & {98.8} &	{71.1}  \\
    \textbf{+VideoJAM-30B}  & 51.2 &	55.9	 & \underline{93.0} &	96.1 & \underline{99.0} &	{82.3}   \\
    \bottomrule
    \end{tabular}}
\end{table}


\section{Movie Gen Benchmark}
\label{sec:moviegen_benchmark}

We employ the prompts from the official benchmark labeled as containing ``high'' motion since our primary objective is to estimate motion coherence. Additionally, since the Movie Gen benchmark is significantly larger than VideoJAM-bench, and mostly contains less relevant prompts (Sec.~\ref{sec:experiments}), we consider the baselines that provide open-source code and can run automatically. Importantly, note that the apples-to-apples comparison to the pre-trained model, DiT-30B is presented for this benchmark as well, allowing us to assess the direct impact of VideoJAM on a large video generation model.

The results are reported in Tab.~\ref{tab:moviegen}, with a breakdown of the automatic metrics in Tab.~\ref{tab:moviegen_metrics}.
Similarly to the results on our motion benchmark, VideoJAM strikes the best balance between the amount of motion and the coherence of the generated motion. While  CogVideo5B consistently produces the most motion, it is also consistently the least coherent baseline. Mochi, on the other hand, suffers from the complementary problem where less motion is generated. Notably, VideoJAM outperforms all baselines, by a significant margin across all metrics, both human-based and automatic (other than the dynamic degree, where CogVideo5B scores the highest, as mentioned). Importantly, we observe a consistent improvement over the base model used by VideoJAM, DiT-30B in both the appearance and motion metrics across all evaluations, which further substantiates our method's ability to improve all aspects of video generation.

\begin{table}[h!]
    \caption{\textbf{Comparison of VideoJAM-30B with prior work on the Movie Gen benchmark.} Human evaluation shows \emph{percentage of votes favoring VideoJAM}; automatic metrics use VBench.}
  \label{tab:moviegen}
  \centering
    \setlength{\tabcolsep}{3.5pt}
  \scalebox{0.95}{%
  \begin{tabular}{@{}lccccc@{}}
    \toprule
      & \multicolumn{3}{c}{\textbf{Human Eval}} & \multicolumn{2}{c}{\textbf{Auto. Metrics}}  \\
      \cmidrule(r){2-4}
      \cmidrule(r){5-6}
    Method     &      \small{Text Faith.}        &  \small{Quality}     &        \textbf{\small{Motion}}  & \small{Appearance}        & \textbf{\small{Motion}}\\
    \midrule
    \small{CogVideo5B}   &    61.4      &   77.0     &   {78.7} & \underline{70.8}  & \underline{88.8}   \\
    \small{Mochi} &    53.5     &   59.4     &    {69.1}  &  70.4 & {85.1}  \\
    \midrule
    \small{DiT-30B} &    60.3     &   64.6     &    {66.1}  & 70.5  & {87.3}  \\
    \textbf{+VideoJAM-30B}  & -  & -    & - &  \textbf{73.7} &  \textbf{90.8}   \\
    \bottomrule
    \end{tabular}}
\end{table}


\begin{table}[h!]
    \caption{\textbf{Breakdown of the automatic metrics} from VBench comparing our 30B model and previous work on the Movie Gen benchmark. Our method strikes the best balance between the dynamic degree (higher implies more motion) and the motion smoothness (higher implies smooth motion).}
  \label{tab:moviegen_metrics}
  \centering
    \setlength{\tabcolsep}{3.5pt}
  \scalebox{0.95}{%
  \begin{tabular}{@{}lcccccc@{}}
    \toprule
      & \multicolumn{4}{c}{Appearance Metrics} & \multicolumn{2}{c}{\textbf{Motion Metrics}}  \\
      \cmidrule(r){2-5}
      \cmidrule(r){6-7}
    \multirow{2}{*}{Method}     &      \small{Aesthetic}        &  \small{Image}     &        \small{Subject}  & \small{Background}        & \textbf{\small{Motion}}& \textbf{\small{Dynamic}}\\
    & \small{Quality}        &  \small{Quality}     &        \small{Consistency}  & \small{Consistency}        & \textbf{\small{Smoothness}}& \textbf{\small{Degree}}\\
    \midrule
    \small{CogVideo5B}   &  \underline{50.9} &	\underline{51.9} &	89.5 &	94.7 & {97.5} &	\textbf{81.6}  \\
    \small{Mochi} &   50.4 &	50.1 &	89.0 &	\underline{95.4} & \underline{98.9} &	{60.7}  \\
    \midrule
    \small{DiT-30B} &   48.7 &	50.6 &	\underline{90.8} &	95.3 & \underline{98.9} &	{67.8}\\
    \textbf{+VideoJAM-30B}  & \textbf{51.5}	& \textbf{56.4} &	\textbf{93.3} &	\textbf{96.2} &\textbf{99.1}	 & \underline{76.9}  \\
    \bottomrule
    \end{tabular}}
\end{table}



\section{VideoJAM-bench Prompts}
\label{sec:motion_benchmark_prompts}
Below, we present the full set of $128$ prompts used in our motion benchmark, VideoJAM-bench. The benchmark is designed to be diverse, encompassing simple motions (e.g., walking), complex human movements (e.g., gymnastics), rotational motions (e.g., spinning balls), and physics-based actions (e.g., a woman hula hooping). To ensure clarity, the prompts were refined using an LLM to focus on specific motion types, enabling a precise evaluation of the model’s ability to generate coherent movement. Additionally, the prompts vary in detail and include camera instructions to test the model’s performance across a wide range of scenarios.

\begin{enumerate}
    \item \emph{``A woman performing an intricate dance on stage, illuminated by a single spotlight in the first frame. She is dressed in a long black dress and a wide-brimmed hat, with her arms raised above her head. The woman dance Argentine flamenco dance.''}
    \item \emph{``A woman doing a headstand on a beach.''}
    \item \emph{``A woman engaging in a challenging workout routine, performing pull-ups on green bars.''}

    \item \emph{``Two ibexes navigating a rocky hillside. They are walking down a steep slope covered in small rocks and dirt. In the background, there are more rocks and some greenery visible through an opening in the rocks.''}

     \item \emph{``A close-up of a runner's legs as they sprint through a crowded city street, dodging pedestrians and street vendors, with the sounds of the city all around.''}
    
    \item \emph{``Athletic man doing gymnastics elements on horizontal bar in city park. Male sportsmen perform strength exercises outdoors.''}
    \item \emph{``A small dog playing with a red ball on a hardwood floor.''}
    \item \emph{``A woman engaging in a lively trampoline workout. The woman jumps and exercises on the trampoline. The background is a room with white walls and a white ceiling, and there are two large windows on the left side of the wall, and a mirror on the right side reflecting the womans image.''}
    \item \emph{``A man performing a handstand on a wooden deck overlooking a green lake surrounded by trees.''}
    \item \emph{``Young adult female performs an air gymnastic show on circus arena, holding ring in hand, making twine exercise, spin around''}
    \item \emph{``A woman enjoying the fun of hula hooping.''}
    \item \emph{``A man juggling with three red balls in a city street.''}

    \item \emph{``A white kitten playing with a ball.''}
    \item \emph{``A slow-motion shot captures a runner's legs as they dash through a busy intersection, dodging cars and pedestrians, the city life bustling around them.''}
    \item \emph{``A young girl playing basketball in a red brick wall background. The girl, with fair skin and long blonde hair, is wearing a green jacket and has her left arm up to throw the ball. In the mid-frame, the girl is still playing basketball, with her right hand holding the ball in front of her face. The ground is dark gray cement with some patches of grass growing through it. As the video progresses, the girl is seen playing near some grassy areas on the ground.''}
    \item \emph{``A basketball game in progress, with two players reaching up to grab the ball as it spills out of the net. The player on the left has his hand outstretched, while the player on the right has both hands raised high. The ball is just above their fingertips, indicating that they are both trying to grab it simultaneously. The background of the image is blurred, but it appears to be a gymnasium or sports arena, with fluorescent lights illuminating the scene. As the video progresses, the players continue to jump and stretch to gain possession of the ball, their movements becoming more urgent and intense. The ball flies back and forth between them, with neither player able to secure it. In the final frame, the ball is still in mid-air, the players hands reaching up to grab it as the video ends.''}
    \item \emph{``A group of basketballs floating in mid-air in slow motion, with a larger ball on the left and two smaller balls on either side in the initial frame. Overall, the video captures the dynamic and energetic movement of basketballs as they float and bounce through space.''}
    
    \item \emph{``A dog playing with an orange ball with blue stripes. The dog picks up the ball and holds it in its mouth, conveying a sense of playfulness and energy. Throughout the video, the dog is seen playing with the ball, capturing the joy and excitement of the moment.''}


    \item \emph{``A woman doing acrobatic exercises on a pole in the gym.''}
    \item \emph{``A young man performing a cartwheel on a gray surface. He is dressed in orange pants, a black t-shirt, and white sneakers. As he executes the cartwheel, his right arm is extended upward, and his left arm is bent at the elbow, reaching down to the ground. His right leg is extended behind him, while his left leg is bent at the knee, pointing towards the camera. The background is a featureless gray wall. The mans energy and focus are evident as he completes the cartwheel, showcasing his athleticism and coordination.''}
    
    
    \item \emph{``A golden retriever playing fetch on a grassy field. The dog is running with a frisbee in its mouth, its fur waving in the wind.''}
    \item \emph{``A brightly colored ball spins rapidly on a flat surface, its patterns blurring as it twirls in place.''}
    \item \emph{``A basketball spins on a player's fingertip, maintaining balance while gradually slowing down.''}
    \item \emph{``A person jogs along a forest trail at dawn, their feet kicking up dirt with every stride, the sunlight filtering through the trees casting long shadows on the path.''}

   \item \emph{``A child jumps up and down in place, their feet leaving the ground briefly before landing again.''}

    \item \emph{``A person lifts one knee high in a marching motion, then places their foot back down and repeats with the other leg.''}

    \item \emph{``Professional cyclist training indoors on a stationary bike trainer.''}

    \item \emph{``Young Adult Male Doing Handstand on the beach.''}

    \item \emph{``A young woman practicing boxing in a gym.''}

    \item \emph{``A man jumping in a pool.''}

    \item \emph{``A man doing push-ups on a ledge overlooking a body of water. The man appears to be doing a push-up, with his head down.''}

    \item \emph{``A man enjoying a leisurely bike ride along a road next to a body of water during a sunset. As he pedals, he looks down at his front wheel, seemingly focused on his ride. The background features a large body of water, with a gray wall along the left side of the road in the mid-frame caption.''}

    \item \emph{``close up shot of the feet of a woman exercising on a cardio fitness machine in a fitness club. As the video progresses, the legs continue to pedal the bike in a smooth, consistent motion.''}

    \item \emph{``A woman engaging in an intense workout on a stationary bike while monitoring her progress on a screen.''}

    \item \emph{``A woman running along a river with a city skyline in the background.''}

    \item \emph{``A skier walking up a snowy hill with their skis on their back and ski poles in hand.''}

    \item \emph{``A woman running through a grassy area, wearing a black tank top, gray and white leggings, and white sneakers. She is initially running on a dirt path, surrounded by trees with green leaves. As she continues to run, the scenery changes to a park, and her leggings change to a blue and white pattern. She is still running on a dirt path, surrounded by trees and green grass. The video captures her journey as she runs through the grassy area, enjoying the outdoors and the beauty of nature.''}

    \item \emph{``A young girl coloring at her desk.''}

    \item \emph{``A close-up of a runner's legs as they dash through a rainstorm, their shoes splashing through puddles as they push forward with determination.''}

    \item \emph{``Tracking camera shot. A kangaroo hops swiftly across an open grassy plain.''}

    \item \emph{``A close-up view of a spiral object with a glowing center. The object appears to be made of metal and has a shiny, reflective surface. . This light creates a series of concentric circles around the objects circumference, which are visible due to the reflection of the light off the metal surface.''}

    \item \emph{``A roulette wheel in a dimly lit room or casino floor. In the center of the wheel, there's a small white ball that appears to be spinning rapidly as it moves around the track. The ball spins around the wheel, and the wheel rotates counterclockwise.''}

    \item \emph{``A close-up of a jogger's feet as they run along a rocky coastal path, their shoes gripping the uneven surface, with the ocean waves crashing below.''}

    \item \emph{``A person's hands as they shape and mold clay on a pottery wheel. The hands are covered in brown clay and are visible from the elbows down, with the forearms resting on top of a large yellow pottery wheel.''}

    \item \emph{``A conveyor belt pouring out a large amount of small, brown objects into a pile on the ground. The objects being poured are falling from the conveyor belt in a steady stream, forming a large pile on the ground below. In the background, the sky is bright blue and cloudless, providing a stark contrast to the darker colors of the conveyor belt and the pile of objects.''}

    \item \emph{``A 3d rendering of coins and small objects floating against a black background. The coins are gold, silver, bronze, and copper, with various denominations and sizes. Some have a shiny finish, while others are matte or tarnished. The scene is chaotic and dynamic, with the objects seemingly flying around in all directions. As the video progresses, the coins and objects tumble and spin, creating a sense of movement and energy. By the end, the screen is filled with white objects of various shapes and sizes, suggesting that something exciting is happening.''}

    \item \emph{``A puppy runs through a grassy field.''}

    \item \emph{``A cinematic shot of a person walking along a quiet country road, their feet crunching on the gravel with every step, fields of wheat swaying in the breeze on either side.''}


    \item \emph{``A washing machine undergoing a full cycle. It begins with a top-down view of the machine filled with water and white soap suds, with two black rubber seals on either side of the stainless steel drum. The video progresses to show the drum spinning, with the suds becoming more agitated and the seals moving along with the drums motion.''}

    \item \emph{``Sweet Cherries on Stems Colliding and Splashing Water Droplets''}

    \item \emph{``A series of colorful balloons floating in mid-air, creating a festive and celebratory atmosphere.''}

    \item \emph{``A cinematic shot of a person jogging along a riverside path, their feet rhythmically tapping against the ground, the river flowing gently beside them.''}

    \item \emph{``A green helicopter taking off from an airport runway.''}

    \item \emph{``A hand holding a yellow fidget spinner. The hand is fair-skinned and holds the bright yellow fidget spinner with silver bearings. The background is blurred and appears to be trees against a blue sky. The video captures the subtle movements of the hand as it spins the fidget spinner, creating a soothing and mesmerizing visual effect. As the video progresses, the hand continues to hold the fidget spinner, showcasing its smooth and satisfying motion. The background remains blurred, adding a sense of tranquility to the scene. Overall, the video is a calming and enjoyable display of the simple pleasure of fidget spinning.''}

    \item \emph{``A windmill spinning in a green field.''}

    \item \emph{``A bicycle wheel spins forward, moving in a circular motion while keeping balance.''}

    \item \emph{``A waterwheel turns as water flows over it, the paddles rotating consistently.''}

    \item \emph{``A close-up of a person's legs as they walk through a sun-dappled forest, the light playing off their shoes as they navigate the uneven terrain.''}

    \item \emph{``A man riding a mountain bike on a dirt trail.''}

    \item \emph{``A child’s toy top spins on a smooth surface, rotating without stopping.''}

    \item \emph{``A basketball spins on a player's fingertip, showcasing balance and skill.''}

    \item \emph{``A jellyfish swimming in shallow water. The jellyfish has a translucent body with a distinctive pattern of white circles and lines. It appears to be swimming just below the surface of the water, which is dark and murky due to the presence of algae or other aquatic plants.''}

    \item \emph{``A cinematic shot of a person walking along a cobblestone street in a historic town, their feet making a rhythmic tap on the stones as they move.''} 

    \item \emph{``A group of horses grazing in a grassy field behind a black wooden fence''}
    
    \item \emph{``A fish swims forward in a steady line, its tail swaying side to side as it propels itself.''}

    \item \emph{``A penguin waddles in a straight line, shifting from one foot to the other.''}

    \item \emph{``A man is jumping rope on the sandy beach, with waves crashing in the background.''}

    \item \emph{``A man enjoying water skiing on a brown river with a green shore and lily pads in the background. Water sprays up from underneath him as he skis across the surface of the lake.''}

    \item \emph{``A man is swimming in a clear blue pool, enjoying the cool water and the freedom of movement in the pool. As he continues to swim, he glides gracefully through the water, his arms and legs moving in a smooth and coordinated rhythm.''}

    \item \emph{``A kid running in the mountains of Campo Imperatore, Italy, at the sunset. He is wearing a red polo shirt, blue jeans, and brown shoes. As he runs, he passes by some white rocks on the ground.''}


    \item \emph{``A woman doing push-up exercise on a beach at sunset.''}

    \item \emph{``A woman is shown running through a field, with tall grass and wildflowers all around her. She is a fair-skinned woman with long, red hair, wearing a black t-shirt and leggings, and listening to music on her phone. In the background, there are trees and more fields of greenery.''}

    \item \emph{``A man exercising with battle ropes at a gym.''}

    \item \emph{``A person engaging in a boxing workout at a gym.''}

    \item \emph{``A dark gray horse running in an enclosed corral. It is running towards the camera.''}

    \item \emph{``A close-up of a runner's legs as they dash up a flight of stairs in a city park, their feet hitting each step with precision and power.''}


    \item \emph{``A man is swimming  in the ocean. In the background, the sky is hazy and overexposed, with the sun shining brightly above the horizon. As the video progresses, the man continues to swim, his arms moving rhythmically through the water.''}

    \item \emph{``A herd of white cows walking down a dirt path. The cows are all facing forward and walking towards the right side of the image. The background is blurry but appears to be a field or pasture.''}

    \item \emph{``A person jogs along a trail in a dense forest, their legs pumping as they navigate the roots and rocks that dot the path.''}

    \item \emph{``A young woman dances in the night bustle against the backdrop of a glowing fanfare.''}

    \item \emph{``A man is walking down the street while pushing a trash can. The man, wearing a red t-shirt, blue jeans, and brown sandals, pushes the black trash can on wheels.''}

    
    \item \emph{``A man enjoying a mountain biking adventure through a forest. He is seen riding a black and white mountain bike down a dirt path, with his back to the camera.''}

    \item \emph{``Women's legs walk into the sea with waves.''}

    \item \emph{``A young man walking on a treadmill. He is wearing a white tank top and red shorts, and has his hands on the sides of the machine as he runs.''}

    \item \emph{``Closeup of feet of a professional soccer player training with ball on stadium field with artificial turf.''}

    \item \emph{``A helicopter flying over a forest. The helicopter is black and has two large rotor blades on top. It is flying low to the ground, with its nose pointing slightly upwards.''}

        
   \item \emph{``A close-up of a person's feet as they walk through a field of wildflowers, their shoes brushing against the blooms with each step.''}


    \item \emph{``A man is playing basketball, dribbling the ball and making shots.''}

    \item \emph{``A giraffe running through an open field. The background is a bright blue sky with fluffy white clouds.''}

       \item \emph{``A person jogs along a city waterfront, their legs moving steadily as the sun sets, casting a warm glow over the water and the buildings behind them.''}

    \item \emph{``A woman is doing push-ups on a mat in the studio.''}

    \item \emph{``Two dancers perform on a stage. The man stands behind the woman with his left arm is lifted over his head and the other is stretched to the right. The woman lets go of the man's right hand, swinging her leg to the left and performing a pirouette. She spins four times and ends up facing the man.''}

    \item \emph{A woman drinks from a water bottle in a forest. The woman has fair skin and brown hair. She is wearing a black jacket and black and white gloves.}

    \item \emph{``Tracking camera shot. A polar bear walks across a snowy landscape. It looks curiously around as it plods through the snow. The background is a snowy landscape with footprints visible in the snow. Sunlight shines from overhead and casts the bear's shadow on the snow.''}

       \item \emph{``A cinematic shot of a person walking through a desert at midday, their legs moving slowly but steadily across the sand dunes, with heat waves distorting the distant horizon.''}

    \item \emph{``A man jumping rope on a dark stage. His movements are fluid and energetic. Two spotlights shine down from above him.''}
    
    \item \emph{``A woman twirls a hula hoop around her waist in a park during sunset. The woman, with medium-length curly black hair and a yellow tank top, stands on a grassy field surrounded by trees. As the hoop revolves around her waist, she shifts her hips rhythmically to keep it moving. The golden sunlight casts a long shadow behind her.''}

    \item \emph{``A man exercises on a leg press machine at a gym.''}

    \item \emph{``A young woman enjoys a cup of coffee on a balcony.''}

    \item \emph{``A man energetically bangs on a drum kit. He holds drumsticks in both hands and bashes on the drum kit with the drumsticks.''}

    \item \emph{``A woman performs high knees on a beach.''}

    \item \emph{``Aerial tracking camera shot. A white semi-truck drives on a highway.''}

    \item \emph{``A woman is holding a clear wine glass partly filled with a burgundy-colored wine. Facing forward, the woman smiles, she raises the glass with her left hand and takes a small sip.''}

    
    \item \emph{``A man works on a piece of wood in a workroom. He holds a shiny silver chisel with a wooden handle in his right hand.''}

    \item \emph{``Sliced green apples are tossed in a brown liquid. The apples are cut into thick slices and have shiny green skins with some light-colored speckling. They begin to rotate clockwise, flying out in every direction as the light amber liquid splashes and swirls behind them.''}

    \item \emph{``A baboon eats a mango.''}

    \item \emph{``A young woman vapes in the living room. The woman exhales the thick, billowing smoke.''}

    \item \emph{``A woman performing an aerial hoop trick. The woman hangs from a black aerial hoop attached to the ceiling by a rope. In the initial frame, she has her legs wrapped around the hoop and her arms extended outward, holding onto the hoop with both hands. Her body is twisted, looking up towards the ceiling, with her shadow cast on the white wall behind her. As the video progresses, she continues to hang from the hoop, her body twisted in various positions, her arms and legs wrapped around the hoop as she performs the aerial trick. The background remains the same, with shadows from the aerial hoop and the woman's body on the white wall.''}

    \item \emph{``Modern urban street ballet dancer performing acrobatics and jumps.''}

    \item \emph{``A woman doing a pirouette in an empty dance studio.''}

    \item \emph{``A woman dancing hip hop, street dancing in the studio. Slow motion.''}

    \item \emph{``A brunette woman doing some acrobatic elements on aerial hoop outdoors.''}

    \item \emph{``A woman, with long brown hair and wearing a black top and gray bottoms, climbs on a pole with her right leg wrapped around it and her left arm extended upward. The background is a white wall with a mirror reflecting the woman's images.''}

    \item \emph{``A man performing a backflip. Slow motion.''}

    \item \emph{``A woman dancing in a gym. The woman is spinning around repeatedly.''}

    \item \emph{``A group of duck are walking in a row, one after the other. The background is a Japanese temple.''}

    \item \emph{``Arc camera shot. A young woman doing stretches on a beach.''}

    \item \emph{``A woman walking through a field of beautiful sunflowers. She spins counterclockwise and laughs. A field of shoulder-length sunflowers grow in the background, with trees on the horizon stretching up towards a cloudy sky.''}

    \item \emph{``Arc camera shot. A man playing the guitar.''}

    \item \emph{``A boy blowing out candles on a birthday cake.''}

    \item \emph{``A cheetah running in the Savannah.''}

    \item \emph{``Tracking shot. A golden retriever runs through a grassy park. The dog’s ears flop up and down with each bounding step, and its tongue hangs out to one side. A frisbee flies into view from the left, and the dog leaps into the air to catch it. A group of people in the background claps and cheers.''}

    \item \emph{``A young girl skips down a quiet suburban street lined with trees. She has light brown skin and long, wavy black hair tied back with a red ribbon. The girl wears a white t-shirt, a denim skirt, and bright yellow sneakers. Her arms swing loosely as she skips''}

    \item \emph{``A woman doing sit-ups at a gym.''}

    \item \emph{``A child riding his bicycle on a dirt path. The background is a dirt path lined with trees on either side.''}

    \item \emph{``A runner moves at full speed along a suburban sidewalk. The background is rows of houses and trees passing by in a blur.''}

    \item \emph{``A young woman engaging in a boxing workout. She is wearing red boxing gloves and a white t-shirt, and has long blonde hair. In the first frame, she is standing in front of a black punching bag, with her right arm extended and her left arm bent, ready to punch the bag. She appears focused and determined. In the second frame, she has moved to the left of the bag and is looking towards the right side of the image. She continues to punch the bag with her right arm extended and her left arm bent. In the final frame, she is still standing to the left of the bag and is looking towards the right side of the image. She is still wearing her red boxing gloves and white t-shirt, and her long blonde hair is visible. The background of  a blue wall with a window on the left and a doorway on the right, as well as two black objects hanging from the ceiling. Throughout the video, the woman is intensely focused on her workout, punching the bag with precision and skill.''}


    \item \emph{``A brown bear walks in a grassy field.''}

    
    
\end{enumerate}


%%%%%%%%%%%%%%%%%%%%%%%%%%%%%%%%%%%%%%%%%%%%%%%%%%%%%%%%%%%%%%%%%%%%%%%%%%%%%%%
%%%%%%%%%%%%

\end{document}