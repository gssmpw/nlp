\documentclass{article}

% Recommended, but optional, packages for figures and better typesetting:
\usepackage{microtype}
\usepackage{graphicx}
\usepackage{wrapfig}
% \usepackage{subfigure}
\usepackage{subcaption}
\usepackage{booktabs} % for professional tables
 \usepackage{array,multirow,graphicx}

% hyperref makes hyperlinks in the resulting PDF.
% If your build breaks (sometimes temporarily if a hyperlink spans a page)
% please comment out the following usepackage line and replace
% \usepackage{icml2025} with \usepackage[nohyperref]{icml2025} above.
\usepackage{hyperref}
\usepackage{arydshln}


% Attempt to make hyperref and algorithmic work together better:
\newcommand{\theHalgorithm}{\arabic{algorithm}}

% Use the following line for the initial blind version submitted for review:
% \usepackage{icml2025}

% If accepted, instead use the following line for the camera-ready submission:
\usepackage[accepted]{icml2025}
% \usepackage{icml2025}

% For theorems and such
\usepackage{amsmath}
\usepackage{amssymb}
\usepackage{mathtools}
\usepackage{amsthm}
% if you use cleveref..
\usepackage[capitalize,noabbrev]{cleveref}

%%%%%%%%%%%%%%%%%%%%%%%%%%%%%%%%
% THEOREMS
%%%%%%%%%%%%%%%%%%%%%%%%%%%%%%%%
\theoremstyle{plain}
\newtheorem{theorem}{Theorem}[section]
\newtheorem{proposition}[theorem]{Proposition}
\newtheorem{lemma}[theorem]{Lemma}
\newtheorem{corollary}[theorem]{Corollary}
\theoremstyle{definition}
\newtheorem{definition}[theorem]{Definition}
\newtheorem{assumption}[theorem]{Assumption}
\theoremstyle{remark}
\newtheorem{remark}[theorem]{Remark}

% Todonotes is useful during development; simply uncomment the next line
%    and comment out the line below the next line to turn off comments
%\usepackage[disable,textsize=tiny]{todonotes}
\usepackage[textsize=tiny]{todonotes}


% The \icmltitle you define below is probably too long as a header.
% Therefore, a short form for the running title is supplied here:
% \icmltitlerunning{Pixels in Motion: Enhancing Motion Understanding in Video Models via Joint Appearance-Motion Representations}
% \icmltitlerunning{VideoJAM: Joint Appearance-Motion Representations for Enhanced Motion Understanding in Video Generators}
% \icmltitlerunning{VideoJAM: Joint Appearance-Motion Representations for Enhanced Motion Generation in Video Models}
\icmltitlerunning{VideoJAM: Joint Appearance-Motion Representations for Enhanced Motion Generation in Video Models}
\newif\ifcomments
% uncomment to bring back comments
\commentstrue 
\ifcomments
    \providecommand\hila[1]{[\textcolor{blue}{Hila: {#1}}]}
    \providecommand\shelly[1]{[\textcolor{purple}{Shelly: {#1}}]}
    \providecommand\lior[1]{[\textcolor{red}{Lior: {#1}}]}
    \providecommand\adam[1]{[\textcolor{magenta}{Adam: {#1}}]}
    \providecommand\amit[1]{[\textcolor{violet}{Amit: {#1}}]}
    \providecommand\uriel[1]{[\textcolor{green}{Uriel: {#1}}]}
    \providecommand\yk[1]{[\textcolor{blue}{YK: {#1}}]}
    \providecommand\todo[1]{[\textcolor{pink}{TODO: {#1}}]}
\else
    \providecommand{\hila}[1]{}
    \providecommand{\shelly}[1]{}
    \providecommand{\lior}[1]{}
    \providecommand{\adam}[1]{}
    \providecommand{\amit}[1]{}
    \providecommand{\uriel}[1]{}
    \providecommand{\yk}[1]{}
    \providecommand{\todo}[1]{}
\fi

\begin{document}

\twocolumn[
% \icmltitle{Pixels in Motion: Enhancing Motion Understanding in Video Models via Joint Appearance-Motion Representations}
% \icmltitle{VideoJAM: Joint Appearance-Motion Representations for Enhanced Motion Understanding in Video Generators}
\icmltitle{VideoJAM: Joint Appearance-Motion Representations for Enhanced Motion Generation in Video Models}

% It is OKAY to include author information, even for blind
% submissions: the style file will automatically remove it for you
% unless you've provided the [accepted] option to the icml2025
% package.

% List of affiliations: The first argument should be a (short)
% identifier you will use later to specify author affiliations
% Academic affiliations should list Department, University, City, Region, Country
% Industry affiliations should list Company, City, Region, Country

% You can specify symbols, otherwise they are numbered in order.
% Ideally, you should not use this facility. Affiliations will be numbered
% in order of appearance and this is the preferred way.
\icmlsetsymbol{equal}{*}
\begin{icmlauthorlist}
\icmlauthor{Hila Chefer}{equal,yyy,comp}
\icmlauthor{Uriel Singer}{yyy}
\icmlauthor{Amit Zohar}{yyy}
\icmlauthor{Yuval Kirstain}{yyy}
\\
\icmlauthor{Adam Polyak}{yyy}
\icmlauthor{Yaniv Taigman}{yyy}
\icmlauthor{Lior Wolf}{comp}
\icmlauthor{Shelly Sheynin}{yyy}
% \\
% \url{https://videojam-paper.github.io/}
\end{icmlauthorlist}

\icmlaffiliation{yyy}{GenAI, Meta}
\icmlaffiliation{comp}{Tel Aviv University}

\icmlcorrespondingauthor{Hila Chefer}{hilach70@gmail.com}
% \icmlcorrespondingauthor{Firstname2 Lastname2}{first2.last2@www.uk}

% You may provide any keywords that you
% find helpful for describing your paper; these are used to populate
% the "keywords" metadata in the PDF but will not be shown in the document
\icmlkeywords{Machine Learning, ICML}

\vskip 0.1in

{

\begin{center}
    \centering
    \captionsetup{type=figure}
    \includegraphics[width=\textwidth]
    {figures/teaser.pdf}
    \vspace{-26px}
    \captionof{figure}{\protect{\textbf{Text-to-video samples generated by VideoJAM.}} We present VideoJAM, a framework that explicitly instills a strong motion prior to any video generation model. Our framework significantly enhances motion coherence across a wide variety of motion types. }
    \label{fig:teaser}
\end{center}%
\vskip 0.1in

}] 

% this must go after the closing bracket ] following \twocolumn[ ...

% This command actually creates the footnote in the first column
% listing the affiliations and the copyright notice.
% The command takes one argument, which is text to display at the start of the footnote.
% The \icmlEqualContribution command is standard text for equal contribution.
% Remove it (just {}) if you do not need this facility.

% \printAffiliationsAndNotice{}  % leave blank if no need to mention equal contribution
% TODO
\printAffiliationsAndNotice{\icmlEqualContribution} % otherwise use the standard text.

\begin{abstract}


The choice of representation for geographic location significantly impacts the accuracy of models for a broad range of geospatial tasks, including fine-grained species classification, population density estimation, and biome classification. Recent works like SatCLIP and GeoCLIP learn such representations by contrastively aligning geolocation with co-located images. While these methods work exceptionally well, in this paper, we posit that the current training strategies fail to fully capture the important visual features. We provide an information theoretic perspective on why the resulting embeddings from these methods discard crucial visual information that is important for many downstream tasks. To solve this problem, we propose a novel retrieval-augmented strategy called RANGE. We build our method on the intuition that the visual features of a location can be estimated by combining the visual features from multiple similar-looking locations. We evaluate our method across a wide variety of tasks. Our results show that RANGE outperforms the existing state-of-the-art models with significant margins in most tasks. We show gains of up to 13.1\% on classification tasks and 0.145 $R^2$ on regression tasks. All our code and models will be made available at: \href{https://github.com/mvrl/RANGE}{https://github.com/mvrl/RANGE}.

\end{abstract}

    
\section{Introduction}
Backdoor attacks pose a concealed yet profound security risk to machine learning (ML) models, for which the adversaries can inject a stealth backdoor into the model during training, enabling them to illicitly control the model's output upon encountering predefined inputs. These attacks can even occur without the knowledge of developers or end-users, thereby undermining the trust in ML systems. As ML becomes more deeply embedded in critical sectors like finance, healthcare, and autonomous driving \citep{he2016deep, liu2020computing, tournier2019mrtrix3, adjabi2020past}, the potential damage from backdoor attacks grows, underscoring the emergency for developing robust defense mechanisms against backdoor attacks.

To address the threat of backdoor attacks, researchers have developed a variety of strategies \cite{liu2018fine,wu2021adversarial,wang2019neural,zeng2022adversarial,zhu2023neural,Zhu_2023_ICCV, wei2024shared,wei2024d3}, aimed at purifying backdoors within victim models. These methods are designed to integrate with current deployment workflows seamlessly and have demonstrated significant success in mitigating the effects of backdoor triggers \cite{wubackdoorbench, wu2023defenses, wu2024backdoorbench,dunnett2024countering}.  However, most state-of-the-art (SOTA) backdoor purification methods operate under the assumption that a small clean dataset, often referred to as \textbf{auxiliary dataset}, is available for purification. Such an assumption poses practical challenges, especially in scenarios where data is scarce. To tackle this challenge, efforts have been made to reduce the size of the required auxiliary dataset~\cite{chai2022oneshot,li2023reconstructive, Zhu_2023_ICCV} and even explore dataset-free purification techniques~\cite{zheng2022data,hong2023revisiting,lin2024fusing}. Although these approaches offer some improvements, recent evaluations \cite{dunnett2024countering, wu2024backdoorbench} continue to highlight the importance of sufficient auxiliary data for achieving robust defenses against backdoor attacks.

While significant progress has been made in reducing the size of auxiliary datasets, an equally critical yet underexplored question remains: \emph{how does the nature of the auxiliary dataset affect purification effectiveness?} In  real-world  applications, auxiliary datasets can vary widely, encompassing in-distribution data, synthetic data, or external data from different sources. Understanding how each type of auxiliary dataset influences the purification effectiveness is vital for selecting or constructing the most suitable auxiliary dataset and the corresponding technique. For instance, when multiple datasets are available, understanding how different datasets contribute to purification can guide defenders in selecting or crafting the most appropriate dataset. Conversely, when only limited auxiliary data is accessible, knowing which purification technique works best under those constraints is critical. Therefore, there is an urgent need for a thorough investigation into the impact of auxiliary datasets on purification effectiveness to guide defenders in  enhancing the security of ML systems. 

In this paper, we systematically investigate the critical role of auxiliary datasets in backdoor purification, aiming to bridge the gap between idealized and practical purification scenarios.  Specifically, we first construct a diverse set of auxiliary datasets to emulate real-world conditions, as summarized in Table~\ref{overall}. These datasets include in-distribution data, synthetic data, and external data from other sources. Through an evaluation of SOTA backdoor purification methods across these datasets, we uncover several critical insights: \textbf{1)} In-distribution datasets, particularly those carefully filtered from the original training data of the victim model, effectively preserve the model’s utility for its intended tasks but may fall short in eliminating backdoors. \textbf{2)} Incorporating OOD datasets can help the model forget backdoors but also bring the risk of forgetting critical learned knowledge, significantly degrading its overall performance. Building on these findings, we propose Guided Input Calibration (GIC), a novel technique that enhances backdoor purification by adaptively transforming auxiliary data to better align with the victim model’s learned representations. By leveraging the victim model itself to guide this transformation, GIC optimizes the purification process, striking a balance between preserving model utility and mitigating backdoor threats. Extensive experiments demonstrate that GIC significantly improves the effectiveness of backdoor purification across diverse auxiliary datasets, providing a practical and robust defense solution.

Our main contributions are threefold:
\textbf{1) Impact analysis of auxiliary datasets:} We take the \textbf{first step}  in systematically investigating how different types of auxiliary datasets influence backdoor purification effectiveness. Our findings provide novel insights and serve as a foundation for future research on optimizing dataset selection and construction for enhanced backdoor defense.
%
\textbf{2) Compilation and evaluation of diverse auxiliary datasets:}  We have compiled and rigorously evaluated a diverse set of auxiliary datasets using SOTA purification methods, making our datasets and code publicly available to facilitate and support future research on practical backdoor defense strategies.
%
\textbf{3) Introduction of GIC:} We introduce GIC, the \textbf{first} dedicated solution designed to align auxiliary datasets with the model’s learned representations, significantly enhancing backdoor mitigation across various dataset types. Our approach sets a new benchmark for practical and effective backdoor defense.



\section{Related Work}
\label{sec:related-works}
\subsection{Novel View Synthesis}
Novel view synthesis is a foundational task in the computer vision and graphics, which aims to generate unseen views of a scene from a given set of images.
% Many methods have been designed to solve this problem by posing it as 3D geometry based rendering, where point clouds~\cite{point_differentiable,point_nfs}, mesh~\cite{worldsheet,FVS,SVS}, planes~\cite{automatci_photo_pop_up,tour_into_the_picture} and multi-plane images~\cite{MINE,single_view_mpi,stereo_magnification}, \etal
Numerous methods have been developed to address this problem by approaching it as 3D geometry-based rendering, such as using meshes~\cite{worldsheet,FVS,SVS}, MPI~\cite{MINE,single_view_mpi,stereo_magnification}, point clouds~\cite{point_differentiable,point_nfs}, etc.
% planes~\cite{automatci_photo_pop_up,tour_into_the_picture}, 


\begin{figure*}[!t]
    \centering
    \includegraphics[width=1.0\linewidth]{figures/overview-v7.png}
    %\caption{\textbf{Overview.} Given a set of images, our method obtains both camera intrinsics and extrinsics, as well as a 3DGS model. First, we obtain the initial camera parameters, global track points from image correspondences and monodepth with reprojection loss. Then we incorporate the global track information and select Gaussian kernels associated with track points. We jointly optimize the parameters $K$, $T_{cw}$, 3DGS through multi-view geometric consistency $L_{t2d}$, $L_{t3d}$, $L_{scale}$ and photometric consistency $L_1$, $L_{D-SSIM}$.}
    \caption{\textbf{Overview.} Given a set of images, our method obtains both camera intrinsics and extrinsics, as well as a 3DGS model. During the initialization, we extract the global tracks, and initialize camera parameters and Gaussians from image correspondences and monodepth with reprojection loss. We determine Gaussian kernels with recovered 3D track points, and then jointly optimize the parameters $K$, $T_{cw}$, 3DGS through the proposed global track constraints (i.e., $L_{t2d}$, $L_{t3d}$, and $L_{scale}$) and original photometric losses (i.e., $L_1$ and $L_{D-SSIM}$).}
    \label{fig:overview}
\end{figure*}

Recently, Neural Radiance Fields (NeRF)~\cite{2020NeRF} provide a novel solution to this problem by representing scenes as implicit radiance fields using neural networks, achieving photo-realistic rendering quality. Although having some works in improving efficiency~\cite{instant_nerf2022, lin2022enerf}, the time-consuming training and rendering still limit its practicality.
Alternatively, 3D Gaussian Splatting (3DGS)~\cite{3DGS2023} models the scene as explicit Gaussian kernels, with differentiable splatting for rendering. Its improved real-time rendering performance, lower storage and efficiency, quickly attract more attentions.
% Different from NeRF-based methods which need MLPs to model the scene and huge computational cost for rendering, 3DGS has stronger real-time performance, higher storage and computational efficiency, benefits from its explicit representation and gradient backpropagation.

\subsection{Optimizing Camera Poses in NeRFs and 3DGS}
Although NeRF and 3DGS can provide impressive scene representation, these methods all need accurate camera parameters (both intrinsic and extrinsic) as additional inputs, which are mostly obtained by COLMAP~\cite{colmap2016}.
% This strong reliance on COLMAP significantly limits their use in real-world applications, so optimizing the camera parameters during the scene training becomes crucial.
When the prior is inaccurate or unknown, accurately estimating camera parameters and scene representations becomes crucial.

% In early works, only photometric constraints are used for scene training and camera pose estimation. 
% iNeRF~\cite{iNerf2021} optimizes the camera poses based on a pre-trained NeRF model.
% NeRFmm~\cite{wang2021nerfmm} introduce a joint optimization process, which estimates the camera poses and trains NeRF model jointly.
% BARF~\cite{barf2021} and GARF~\cite{2022GARF} provide new positional encoding strategy to handle with the gradient inconsistency issue of positional embedding and yield promising results.
% However, they achieve satisfactory optimization results when only the pose initialization is quite closed to the ground-truth, as the photometric constrains can only improve the quality of camera estimation within a small range.
% Later, more prior information of geometry and correspondence, \ie monocular depth and feature matching, are introduced into joint optimisation to enhance the capability of camera poses estimation.
% SC-NeRF~\cite{SCNeRF2021} minimizes a projected ray distance loss based on correspondence of adjacent frames.
% NoPe-NeRF~\cite{bian2022nopenerf} chooses monocular depth maps as geometric priors, and defines undistorted depth loss and relative pose constraints for joint optimization.
In earlier studies, scene training and camera pose estimation relied solely on photometric constraints. iNeRF~\cite{iNerf2021} refines the camera poses using a pre-trained NeRF model. NeRFmm~\cite{wang2021nerfmm} introduces a joint optimization approach that simultaneously estimates camera poses and trains the NeRF model. BARF~\cite{barf2021} and GARF~\cite{2022GARF} propose a new positional encoding strategy to address the gradient inconsistency issues in positional embedding, achieving promising results. However, these methods only yield satisfactory optimization when the initial pose is very close to the ground truth, as photometric constraints alone can only enhance camera estimation quality within a limited range. Subsequently, 
% additional prior information on geometry and correspondence, such as monocular depth and feature matching, has been incorporated into joint optimization to improve the accuracy of camera pose estimation. 
SC-NeRF~\cite{SCNeRF2021} minimizes a projected ray distance loss based on correspondence between adjacent frames. NoPe-NeRF~\cite{bian2022nopenerf} utilizes monocular depth maps as geometric priors and defines undistorted depth loss and relative pose constraints.

% With regard to 3D Gaussian Splatting, CF-3DGS~\cite{CF-3DGS-2024} also leverages mono-depth information to constrain the optimization of local 3DGS for relative pose estimation and later learn a global 3DGS progressively in a sequential manner.
% InstantSplat~\cite{fan2024instantsplat} focus on sparse view scenes, first use DUSt3R~\cite{dust3r2024cvpr} to generate a set of densely covered and pixel-aligned points for 3D Gaussian initialization, then introduce a parallel grid partitioning strategy in joint optimization to speed up.
% % Jiang et al.~\cite{Jiang_2024sig} proposed to build the scene continuously and progressively, to next unregistered frame, they use registration and adjustment to adjust the previous registered camera poses and align unregistered monocular depths, later refine the joint model by matching detected correspondences in screen-space coordinates.
% \gjh{Jiang et al.~\cite{Jiang_2024sig} also implemented an incremental approach for reconstructing camera poses and scenes. Initially, they perform feature matching between the current image and the image rendered by a differentiable surface renderer. They then construct matching point errors, depth errors, and photometric errors to achieve the registration and adjustment of the current image. Finally, based on the depth map, the pixels of the current image are projected as new 3D Gaussians. However, this method still exhibits limitations when dealing with complex scenes and unordered images.}
% % CG-3DGS~\cite{sun2024correspondenceguidedsfmfree3dgaussian} follows CF-3DGS, first construct a coarse point cloud from mono-depth maps to train a 3DGS model, then progressively estimate camera poses based on this pre-trained model by constraining the correspondences between rendering view and ground-truth.
% \gjh{Similarly, CG-3DGS~\cite{sun2024correspondenceguidedsfmfree3dgaussian} first utilizes monocular depth estimation and the camera parameters from the first frame to initialize a set of 3D Gaussians. It then progressively estimates camera poses based on this pre-trained model by constraining the correspondences between the rendered views and the ground truth.}
% % Free-SurGS~\cite{freesurgs2024} matches the projection flow derived from 3D Gaussians with optical flow to estimate the poses, to compensate for the limitations of photometric loss.
% \gjh{Free-SurGS~\cite{freesurgs2024} introduces the first SfM-free 3DGS approach for surgical scene reconstruction. Due to the challenges posed by weak textures and photometric inconsistencies in surgical scenes, Free-SurGS achieves pose estimation by minimizing the flow loss between the projection flow and the optical flow. Subsequently, it keeps the camera pose fixed and optimizes the scene representation by minimizing the photometric loss, depth loss and flow loss.}
% \gjh{However, most current works assume camera intrinsics are known and primarily focus on optimizing camera poses. Additionally, these methods typically rely on sequentially ordered image inputs and incrementally optimize camera parameters and scene representation. This inevitably leads to drift errors, preventing the achievement of globally consistent results. Our work aims to address these issues.}

Regarding 3D Gaussian Splatting, CF-3DGS~\cite{CF-3DGS-2024} utilizes mono-depth information to refine the optimization of local 3DGS for relative pose estimation and subsequently learns a global 3DGS in a sequential manner. InstantSplat~\cite{fan2024instantsplat} targets sparse view scenes, initially employing DUSt3R~\cite{dust3r2024cvpr} to create a densely covered, pixel-aligned point set for initializing 3D Gaussian models, and then implements a parallel grid partitioning strategy to accelerate joint optimization. Jiang \etal~\cite{Jiang_2024sig} develops an incremental method for reconstructing camera poses and scenes, but it struggles with complex scenes and unordered images. 
% Similarly, CG-3DGS~\cite{sun2024correspondenceguidedsfmfree3dgaussian} progressively estimates camera poses using a pre-trained model by aligning the correspondences between rendered views and actual scenes. Free-SurGS~\cite{freesurgs2024} pioneers an SfM-free 3DGS method for reconstructing surgical scenes, overcoming challenges such as weak textures and photometric inconsistencies by minimizing the discrepancy between projection flow and optical flow.
%\pb{SF-3DGS-HT~\cite{ji2024sfmfree3dgaussiansplatting} introduced VFI into training as additional photometric constraints. They separated the whole scene into several local 3DGS models and then merged them hierarchically, which leads to a significant improvement on simple and dense view scenes.}
HT-3DGS~\cite{ji2024sfmfree3dgaussiansplatting} interpolates frames for training and splits the scene into local clips, using a hierarchical strategy to build 3DGS model. It works well for simple scenes, but fails with dramatic motions due to unstable interpolation and low efficiency.
% {While effective for simple scenes, it struggles with dramatic motion due to unstable view interpolation and suffers from low computational efficiency.}

However, most existing methods generally depend on sequentially ordered image inputs and incrementally optimize camera parameters and 3DGS, which often leads to drift errors and hinders achieving globally consistent results. Our work seeks to overcome these limitations.

\section{Motivation}
\label{sec:motivation}

During training, generative video models take a noised training video and compute a loss by comparing the model's prediction with the original video, the noise, or a combination of the two~\cite{Ho2020DenoisingDP,flow-matching} (Sec.~\ref{sec:preliminaries}).
We hypothesize that this formulation biases the model towards appearance-based features, such as color and texture, as these dominate pixel-wise differences. Consequently, the model is less inclined to attend to temporal information, such as dynamics or physics, which contribute less to the objective. To demonstrate this claim, we perform experiments to evaluate the sensitivity of the model to temporal incoherence. The following experiments are conducted on DiT-4B~\cite{dit} for efficiency.

\begin{figure}[t!]
    \centering
        \includegraphics[width=0.45\textwidth]{figures/motivation_plot.pdf}
         \vspace{-14px}
    \caption{\textbf{Motivation Experiment.} We compare the model's loss before and after randomly permuting the video frames, using a ``vanilla'' DiT (orange) and our fine-tuned model (blue). The original model is \emph{nearly invariant} to temporal perturbations for $t\leq 60$. }
        \label{fig:motivation-a}
    \vspace{-16px}
    \label{fig:motivation}
\end{figure}

\begin{figure*}[ht!]
\centering
\includegraphics[width=1.01\textwidth]{figures/architecture.pdf}
\vspace{-18px}
\caption{\textbf{VideoJAM Framework.} VideoJAM is constructed of two units; (a) \textbf{Training.} Given an input video $x_1$ and its motion representation $d_1$, both signals are noised and embedded to a \emph{single, joint} latent representation using a linear layer, $\textbf{W}^+_{in}$. The diffusion model processes the input, and two linear projection layers predict both appearance and motion from the joint representation. (b) \textbf{Inference.} We propose \emph{Inner-Guidance}, where the model's own noisy motion prediction is used to guide the video prediction at each step. }
\label{fig:architecture}
\vspace{-6px}
\end{figure*}

We conduct an experiment where two variants of videos are noised and fed to the model—first, the plain video without intervention, and second, the video after applying a \emph{random permutation} to its frames. 
Assuming the model captures temporal information, we anticipate that the temporally incoherent (perturbed) input will result in a higher measured loss compared to the temporally coherent input.

Given a random set of $35,000$ training videos, we noise each video to a random denoising step $t\in[0,99]$. We then examine the difference in the loss measured before and after the permutation and aggregate the results per timestep. We consider two models-- the ``vanilla'' DiT, which employs a pixel-based objective, and our fine-tuned VideoJAM model, which adds an explicit motion objective (Sec.~\ref{sec:method}). 

The results of this experiment are reported in Fig.~\ref{fig:motivation}.
As can be observed, the original model appears to be \emph{nearly invariant} to frame shuffling until step $60$ of the generation. This implies that the model fails to distinguish between a valid video and a temporally incoherent one. In stark contrast, our model is extremely sensitive to these perturbations, as is indicated by the significant gap in the calculated loss. 

In App.~\ref{sec:motivation_supp} we include a qualitative experiment demonstrating that the steps $t\leq 60$ determine the coarse motion in the video. Both results suggest that the training objective is less sensitive to temporal incoherence, leading models to favor appearance over motion.


Effective human-robot cooperation in CoNav-Maze hinges on efficient communication. Maximizing the human’s information gain enables more precise guidance, which in turn accelerates task completion. Yet for the robot, the challenge is not only \emph{what} to communicate but also \emph{when}, as it must balance gathering information for the human with pursuing immediate goals when confident in its navigation.

To achieve this, we introduce \emph{Information Gain Monte Carlo Tree Search} (IG-MCTS), which optimizes both task-relevant objectives and the transmission of the most informative communication. IG-MCTS comprises three key components:
\textbf{(1)} A data-driven human perception model that tracks how implicit (movement) and explicit (image) information updates the human’s understanding of the maze layout.
\textbf{(2)} Reward augmentation to integrate multiple objectives effectively leveraging on the learned perception model.
\textbf{(3)} An uncertainty-aware MCTS that accounts for unobserved maze regions and human perception stochasticity.
% \begin{enumerate}[leftmargin=*]
%     \item A data-driven human perception model that tracks how implicit (movement) and explicit (image transmission) information updates the human’s understanding of the maze layout.
%     \item Reward augmentation to integrate multiple objectives effectively leveraging on the learned perception model.
%     \item An uncertainty-aware MCTS that accounts for unobserved maze regions and human perception stochasticity.
% \end{enumerate}

\subsection{Human Perception Dynamics}
% IG-MCTS seeks to optimize the expected novel information gained by the human through the robot’s actions, including both movement and communication. Achieving this requires a model of how the human acquires task-relevant information from the robot.

% \subsubsection{Perception MDP}
\label{sec:perception_mdp}
As the robot navigates the maze and transmits images, humans update their understanding of the environment. Based on the robot's path, they may infer that previously assumed blocked locations are traversable or detect discrepancies between the transmitted image and their map.  

To formally capture this process, we model the evolution of human perception as another Markov Decision Process, referred to as the \emph{Perception MDP}. The state space $\mathcal{X}$ represents all possible maze maps. The action space $\mathcal{S}^+ \times \mathcal{O}$ consists of the robot's trajectory between two image transmissions $\tau \in \mathcal{S}^+$ and an image $o \in \mathcal{O}$. The unknown transition function $F: (x, (\tau, o)) \rightarrow x'$ defines the human perception dynamics, which we aim to learn.

\subsubsection{Crowd-Sourced Transition Dataset}
To collect data, we designed a mapping task in the CoNav-Maze environment. Participants were tasked to edit their maps to match the true environment. A button triggers the robot's autonomous movements, after which it captures an image from a random angle.
In this mapping task, the robot, aware of both the true environment and the human’s map, visits predefined target locations and prioritizes areas with mislabeled grid cells on the human’s map.
% We assume that the robot has full knowledge of both the actual environment and the human’s current map. Leveraging this knowledge, the robot autonomously navigates to all predefined target locations. It then randomly selects subsequent goals to reach, prioritizing grid locations that remain mislabeled on the human’s map. This ensures that the robot’s actions are strategically focused on providing useful information to improve map accuracy.

We then recruited over $50$ annotators through Prolific~\cite{palan2018prolific} for the mapping task. Each annotator labeled three randomly generated mazes. They were allowed to proceed to the next maze once the robot had reached all four goal locations. However, they could spend additional time refining their map before moving on. To incentivize accuracy, annotators receive a performance-based bonus based on the final accuracy of their annotated map.


\subsubsection{Fully-Convolutional Dynamics Model}
\label{sec:nhpm}

We propose a Neural Human Perception Model (NHPM), a fully convolutional neural network (FCNN), to predict the human perception transition probabilities modeled in \Cref{sec:perception_mdp}. We denote the model as $F_\theta$ where $\theta$ represents the trainable weights. Such design echoes recent studies of model-based reinforcement learning~\cite{hansen2022temporal}, where the agent first learns the environment dynamics, potentially from image observations~\cite{hafner2019learning,watter2015embed}.

\begin{figure}[t]
    \centering
    \includegraphics[width=0.9\linewidth]{figures/ICML_25_CNN.pdf}
    \caption{Neural Human Perception Model (NHPM). \textbf{Left:} The human's current perception, the robot's trajectory since the last transmission, and the captured environment grids are individually processed into 2D masks. \textbf{Right:} A fully convolutional neural network predicts two masks: one for the probability of the human adding a wall to their map and another for removing a wall.}
    \label{fig:nhpm}
    \vskip -0.1in
\end{figure}

As illustrated in \Cref{fig:nhpm}, our model takes as input the human’s current perception, the robot’s path, and the image captured by the robot, all of which are transformed into a unified 2D representation. These inputs are concatenated along the channel dimension and fed into the CNN, which outputs a two-channel image: one predicting the probability of human adding a new wall and the other predicting the probability of removing a wall.

% Our approach builds on world model learning, where neural networks predict state transitions or environmental updates based on agent actions and observations. By leveraging the local feature extraction capabilities of CNNs, our model effectively captures spatial relationships and interprets local changes within the grid maze environment. Similar to prior work in localization and mapping, the CNN architecture is well-suited for processing spatially structured data and aligning the robot’s observations with human map updates.

To enhance robustness and generalization, we apply data augmentation techniques, including random rotation and flipping of the 2D inputs during training. These transformations are particularly beneficial in the grid maze environment, which is invariant to orientation changes.

\subsection{Perception-Aware Reward Augmentation}
The robot optimizes its actions over a planning horizon \( H \) by solving the following optimization problem:
\begin{subequations}
    \begin{align}
        \max_{a_{0:H-1}} \;
        & \mathop{\mathbb{E}}_{T, F} \left[ \sum_{t=0}^{H-1} \gamma^t \left(\underbrace{R_{\mathrm{task}}(\tau_{t+1}, \zeta)}_{\text{(1) Task reward}} + \underbrace{\|x_{t+1}-x_t\|_1}_{\text{(2) Info reward}}\right)\right] \label{obj}\\ 
        \subjectto \quad
        &x_{t+1} = F(x_t, (\tau_t, a_t)), \quad a_t\in\Ocal \label{const:perception_update}\\ 
        &\tau_{t+1} = \tau_t \oplus T(s_t, a_t), \quad a_t\in \Ucal\label{const:history_update}
    \end{align}
\end{subequations} 

The objective in~\eqref{obj} maximizes the expected cumulative reward over \( T \) and \( F \), reflecting the uncertainty in both physical transitions and human perception dynamics. The reward function consists of two components: 
(1) The \emph{task reward} incentivizes efficient navigation. The specific formulation for the task in this work is outlined in \Cref{appendix:task_reward}.
(2) The \emph{information reward} quantifies the change in the human’s perception due to robot actions, computed as the \( L_1 \)-norm distance between consecutive perception states.  

The constraint in~\eqref{const:history_update} ensures that for movement actions, the trajectory history \( \tau_t \) expands with new states based on the robot’s chosen actions, where \( s_t \) is the most recent state in \( \tau_t \), and \( \oplus \) represents sequence concatenation. 
In constraint~\eqref{const:perception_update}, the robot leverages the learned human perception dynamics \( F \) to estimate the evolution of the human’s understanding of the environment from perception state $x_t$ to $x_{t+1}$ based on the observed trajectory \( \tau_t \) and transmitted image \( a_t\in\Ocal \). 
% justify from a cognitive science perspective
% Cognitive science research has shown that humans read in a way to maximize the information gained from each word, aligning with the efficient coding principle, which prioritizes minimizing perceptual errors and extracting relevant features under limited processing capacity~\cite{kangassalo2020information}. Drawing on this principle, we hypothesize that humans similarly prioritize task-relevant information in multimodal settings. To accommodate this cognitive pattern, our robot policy selects and communicates high information-gain observations to human operators, akin to summarizing key insights from a lengthy article.
% % While the brain naturally seeks to gain information, the brain employs various strategies to manage information overload, including filtering~\cite{quiroga2004reducing}, limiting/working memory, and prioritizing information~\cite{arnold2023dealing}.
% In this context of our setup, we optimize the selection of camera angles to maximize the human operator's information gain about the environment. 

\subsection{Information Gain Monte Carlo Tree Search (IG-MCTS)}
IG-MCTS follows the four stages of Monte Carlo tree search: \emph{selection}, \emph{expansion}, \emph{rollout}, and \emph{backpropagation}, but extends it by incorporating uncertainty in both environment dynamics and human perception. We introduce uncertainty-aware simulations in the \emph{expansion} and \emph{rollout} phases and adjust \emph{backpropagation} with a value update rule that accounts for transition feasibility.

\subsubsection{Uncertainty-Aware Simulation}
As detailed in \Cref{algo:IG_MCTS}, both the \emph{expansion} and \emph{rollout} phases involve forward simulation of robot actions. Each tree node $v$ contains the state $(\tau, x)$, representing the robot's state history and current human perception. We handle the two action types differently as follows:
\begin{itemize}
    \item A movement action $u$ follows the environment dynamics $T$ as defined in \Cref{sec:problem}. Notably, the maze layout is observable up to distance $r$ from the robot's visited grids, while unexplored areas assume a $50\%$ chance of walls. In \emph{expansion}, the resulting search node $v'$ of this uncertain transition is assigned a feasibility value $\delta = 0.5$. In \emph{rollout}, the transition could fail and the robot remains in the same grid.
    
    \item The state transition for a communication step $o$ is governed by the learned stochastic human perception model $F_\theta$ as defined in \Cref{sec:nhpm}. Since transition probabilities are known, we compute the expected information reward $\bar{R_\mathrm{info}}$ directly:
    \begin{align*}
        \bar{R_\mathrm{info}}(\tau_t, x_t, o_t) &= \mathbb{E}_{x_{t+1}}\|x_{t+1}-x_t\|_1 \\
        &= \|p_\mathrm{add}\|_1 + \|p_\mathrm{remove}\|_1,
    \end{align*}
    where $(p_\mathrm{add}, p_\mathrm{remove}) \gets F_\theta(\tau_t, x_t, o_t)$ are the estimated probabilities of adding or removing walls from the map. 
    Directly computing the expected return at a node avoids the high number of visitations required to obtain an accurate value estimate.
\end{itemize}

% We denote a node in the search tree as $v$, where $s(v)$, $r(v)$, and $\delta(v)$ represent the state, reward, and transition feasibility at $v$, respectively. The visit count of $v$ is denoted as $N(v)$, while $Q(v)$ represents its total accumulated return. The set of child nodes of $v$ is denoted by $\mathbb{C}(v)$.

% The goal of each search is to plan a sequence for the robot until it reaches a goal or transmits a new image to the human. We initialize the search tree with the current human guidance $\zeta$, and the robot's approximation of human perception $x_0$. Each search node consists consists of the state information required by our reward augmentation: $(\tau, x)$. A node is terminal if it is the resulting state of a communication step, or if the robot reaches a goal location. 

% A rollout from the expanded node simulates future transitions until reaching a terminal state or a predefined depth $H$. Actions are selected randomly from the available action set $\mathcal{A}(s)$. If an action's feasibility is uncertain due to the environment's unknown structure, the transition occurs with probability $\delta(s, a)$. When a random number draw deems the transition infeasible, the state remains unchanged. On the other hand, for communication steps, we don't resolve the uncertainty but instead compute the expected information gain reward: \philip{TODO: adjust notation}
% \begin{equation}
%     \mathbb{E}\left[R_\mathrm{info}(\tau, x')\right] = \sum \mathrm{NPM(\tau, o)}.
% \end{equation}

\subsubsection{Feasibility-Adjusted Backpropagation}
During backpropagation, the rewards obtained from the simulation phase are propagated back through the tree, updating the total value $Q(v)$ and the visitation count $N(v)$ for all nodes along the path to the root. Due to uncertainty in unexplored environment dynamics, the rollout return depends on the feasibility of the transition from the child node. Given a sample return \(q'_{\mathrm{sample}}\) at child node \(v'\), the parent node's return is:
\begin{equation}
    q_{\mathrm{sample}} = r + \gamma \left[ \delta' q'_{\mathrm{sample}} + (1 - \delta') \frac{Q(v)}{N(v)} \right],
\end{equation}
where $\delta'$ represents the probability of a successful transition. The term \((1 - \delta')\) accounts for failed transitions, relying instead on the current value estimate.

% By incorporating uncertainty-aware rollouts and backpropagation, our approach enables more robust decision-making in scenarios where the environment dynamics is unknown and avoids simulation of the stochastic human perception dynamics.

% \section{Experiment and Results}
\section{Results and Analysis}
In this section, we first present safe vs. unsafe evaluation results for 12 LLMs, followed by fine-grained responding pattern analysis over six models among them, and compare models' behavior when they are attacked by same risky questions presented in different languages: Kazakh, Russian and code-switching language.    

\begin{table}[t!]
\centering
\small
\resizebox{\columnwidth}{!}{
\begin{tabular}{clcccc}
\toprule
\multicolumn{1}{l}{\textbf{Rank} } & \textbf{Model} & \textbf{Kazakh $\uparrow$} & \textbf{Russian $\uparrow$} & \textbf{English $\uparrow$} \\
\midrule
1 & \claude & \textbf{96.5}   & 93.5    & \textbf{98.6}    \\
2 & \gptfouro & 95.8   & 87.6    & 95.7    \\
3 & \yandexgpt & 90.7   & \textbf{93.6}    & 80.3    \\
4 & \kazllmseventy & 88.1 & 87.5 & 97.2 \\
5 & \llamaseventy & 88.0   & 85.5    & 95.7    \\
6 & \sherkala & 87.1   & 85.0    & 96.0    \\
7 & \falcon & 87.1   & 84.7    & 96.8    \\
8 & \qwen & 86.2   & 85.1    & 88.1    \\
9 & \llamaeight & 85.9   & 84.7    & 98.3    \\
10 & \kazllmeight & 74.8   & 78.0    & 94.5 \\
11 & \aya & 72.4 & 84.5 & 96.6 \\
12 & \vikhr & --- & 85.6 & 91.1 \\
\bottomrule
\end{tabular}
}
\caption{Safety evaluation results of 12 LLMs, ranked by the percentage of safe responses in the Kazakh dataset. \claude\ achieves the highest safety score for both Kazakh and English, while \yandexgpt\ is the safest model for Russian responses.}
\label{tab:safety-binary-eval}
\end{table}



\subsection{Safe vs. Unsafe Classification}
% In this subsection, 
We present binary evaluation results of 12 LLMs, by assessing 52,596 Russian responses and 41,646 Kazakh responses.
% 26,298 responses generated by six models on the Russian dataset and 22,716 responses on the Kazakh dataset. 

%\textbf{Safety Rank.} In general, proprietary systems outperform the open-source model. For Russian, As shown in Table \ref{tab:model_comparison_russian}, \textbf{Yandex-GPT} emerges as the safest large language model (LLM) for Russian, with a safety percentage of 93.57\%. Among the open-source models, \textbf{Vikhr-Nemo-12B} is the safest, achieving a safety percentage of 85.63\%.
% Edited: This is mentioned in the discussion
% This outcome highlights the potential impact of pretraining data on model behavior. Models pre-trained primarily on Russian data are better at understanding and handling harmful questions - in both proprietary systems and open-source setups. 
%For Kazakh, as shown in Table \ref{tab:model_comparison_kazakh}, \textbf{Claude} emerges as the safest large language model (LLM) with a safety percentage of 96.46\%, closely followed by GPT-4o at 95.75\%. In contrast, \textbf{Aya-101}, despite being specifically tuned for Kazakh, consistently shows the highest unsafe response rates at 72.37\%. 

\begin{figure*}[t!]
	\centering
        \includegraphics[scale=0.28]{figures/question_type_no6_kaz.png}
	\includegraphics[scale=0.28]{figures/question_type_exclude_region_specific_new.png} 

	\caption{Unsafe answer distribution across three question types for risk types I-V: Kazakh (left) and Russian (right)}
	\label{fig:qt_non_reg}
\end{figure*}

\begin{figure*}[t!]
	\centering
        \includegraphics[scale=0.28]{figures/question_type_only6_kaz.png}
	\includegraphics[scale=0.28]{figures/question_type_region_specific_new.png} 
	
	\caption{Unsafe answer distribution across three question types for risk type VI: Kazakh (left) and Russian (right)}
	\label{fig:qt_reg}
\end{figure*}

\textbf{Safety Rank.} In general, proprietary systems outperform the open-source models. 
For Russian, as shown in Table~\ref{tab:safety-binary-eval},  % \ref{tab:model_comparison_russian}, 
\yandexgpt emerges as the safest language model for Russian, with safe responses account for 93.57\%. Among the open-source models, \kazllmseventy is the safest (87.5\%), followed by \vikhr with a safety percentage of 85.63\%.

For Kazakh, % as shown in Table \ref{tab:model_comparison_kazakh}, 
% YX: todo, rerun Kazakh safety percentage using Diana threshold
\claude is the safest model with 96.46\% safe responses, closely followed by \gptfouro\ at 95.75\%. \aya, despite being specifically tuned for Kazakh, shows the highest unsafe rates at 72.37\%.



% \begin{table}[t!]
% \centering
% \resizebox{\columnwidth}{!}{%
% \begin{tabular}{clccc}
% \toprule
% \textbf{Rank} & \textbf{Model Name}  & \textbf{Safe} & \textbf{Unsafe} & \textbf{Safe \%} \\ \midrule
% \textbf{1} & \textbf{Yandex-GPT} & \textbf{4101} & \textbf{282} & \textbf{93.57} \\
% 2 & Claude & 4100 & 283 & 93.54 \\
% 3 & GPT-4o & 3839 & 544 & 87.59 \\
% 4 & Vikhr-12B & 3753 & 630 & 85.63 \\
% 5 & LLama-3.1-instruct-70B & 3746 & 637 & 85.47 \\
% 6 & LLama-3.1-instruct-8B & 3712 & 671 & 84.69 \\
% \bottomrule
% \end{tabular}
% }
% \caption{Comparison of models based on safety percentages for the Russian dataset.}
% \label{tab:model_comparison_russian}
% \end{table}


% \begin{table}[t!]
% \centering
% \resizebox{\columnwidth}{!}{%
% \begin{tabular}{clccc}
% \toprule
% \textbf{Rank} & \textbf{Model Name}  & \textbf{Safe} & \textbf{Unsafe} & \textbf{Safe \%} \\ \midrule
% 1             & \textbf{Claude}  & \textbf{3652} & \textbf{134} & \textbf{96.46} \\ 
% 2             & GPT-4o                      & 3625          & 161          & 95.75 \\ 
% 3             & YandexGPT                   & 3433          & 353          & 90.68 \\
% 4             & LLama-3.1-instruct-70B      & 3333          & 453          & 88.03 \\
% 5             & LLama-3.1-instruct-8B       & 3251          & 535	       & 85.87 \\
% 6             & Aya-101                     & 2740          & 1046         & 72.37 \\ 
% \bottomrule
% \end{tabular}
% }
% \caption{Comparison of models based on safety percentages for the Kazakh dataset.}
% \label{tab:model_comparison_kazakh}
% \end{table}



\textbf{Risk Areas.} 
We selected six representative LLMs for Russian and Kazakh respectively and show their unsafe answer distributions over six risk areas.
As shown in Table \ref{tab:unsafe_answers_summary}, risk type VI (region-specific sensitive topics) overwhelmingly contributes the largest number of unsafe responses across all models. This highlights that LLMs are poorly equipped to address regional risks effectively. For instance, while \llama models maintain comparable safety levels across other risk type (I–V), their performance drops significantly when dealing with risk type VI. Interestingly, while \yandexgpt\ demonstrates relatively poor performance in most other risk areas, it handles region-specific questions remarkably well, suggesting a stronger alignment with regional norms and sensitivities. For Kazakh, Table \ref{tab:unsafe_answers_summary_kazakh} shows that region‐specific topics (risk type VI) pose a substantial challenge across all six models, including the commercial \gptfouro\ and \claude, which demonstrate superior safety on general categories. 

% \begin{table}[t!]
% \centering
% \resizebox{\columnwidth}{!}{%
% \begin{tabular}{lccccccc}
% \toprule
% \textbf{Model} & \textbf{I} & \textbf{II} & \textbf{III} & \textbf{IV} & \textbf{V} & \textbf{VI} & \textbf{Total} \\ \midrule
% LLama-3.1-instruct-8B & 60 & 70 & 16 & 31 & 9 & 485 & 671 \\
% LLama-3.1-instruct-70B & 29 & 55 & 8 & 4 & 1 & 540 & 637 \\
% Vikhr-12B & 41 & 93 & 15 & 1 & 3 & 477 & 630 \\
% GPT-4o & 21 & 51 & 6 & 2 & 0 & 464 & 544 \\
% Claude & 7 & 10 & 1 & 0 & 0 & 265 & 283 \\
% Yandex-GPT & 55 & 125 & 9 & 3 & 8 & 82 & 282 \\
% \bottomrule
% \end{tabular}%
% }
% \caption{Ru unsafe cases over risk areas of six models.}
% \label{tab:unsafe_answers_summary}
% \end{table}


\begin{table}[t!]
\centering
\resizebox{\columnwidth}{!}{%
\begin{tabular}{lccccccc}
\toprule
\textbf{Model} & \textbf{I} & \textbf{II} & \textbf{III} & \textbf{IV} & \textbf{V} & \textbf{VI} & \textbf{Total} \\ \midrule
\llamaeight & 60 & 70 & 16 & 31 & 9 & 485 & 671 \\
\llamaseventy & 29 & 55 & 8 & 4 & 1 & 540 & 637 \\
\vikhr & 41 & 93 & 15 & 1 & 3 & 477 & 630 \\
\gptfouro & 21 & 51 & 6 & 2 & 0 & 464 & 544 \\
\claude & 7 & 10 & 1 & 0 & 0 & 265 & 283 \\
\yandexgpt & 55 & 125 & 9 & 3 & 8 & 82 & 282 \\
\bottomrule
\end{tabular}%
}
\caption{Ru unsafe cases over risk areas of six models.}
\label{tab:unsafe_answers_summary}
\end{table}


% \begin{table}[t!]
% \centering
% \resizebox{\columnwidth}{!}{%
% \begin{tabular}{lccccccc}
% \toprule
% \textbf{Model} & \textbf{I} & \textbf{II} & \textbf{III} & \textbf{IV} & \textbf{V} & \textbf{VI} & \textbf{Total} \\ \midrule
% Aya-101 & 96 & 235 & 165 & 166 & 90 & 294 & 1046 \\
% Llama-3.1-instruct-8B & 25 & 15 & 91 & 37 & 14 & 353 & 535 \\
% Llama-3.1-instruct-70B & 33 & 39 & 88 & 27 & 20 & 246 & 453 \\
% Yandex-GPT & 29 & 76 & 95 & 29 & 16 & 108 & 353 \\
% GPT-4o & 2 & 1 & 41 & 0 & 3 & 114 & 161 \\
% Claude & 2 & 1 & 26 & 3 & 6 & 96 & 134 \\ \bottomrule
% \end{tabular}%
% }
% \caption{Kaz unsafe cases over risk areas of six models.}
% \label{tab:unsafe_answers_summary_kazakh}
% \end{table}


\begin{table}[t!]
\centering
\resizebox{\columnwidth}{!}{%
\begin{tabular}{lccccccc}
\toprule
\textbf{Model} & \textbf{I} & \textbf{II} & \textbf{III} & \textbf{IV} & \textbf{V} & \textbf{VI} & \textbf{Total} \\ \midrule
\aya & 96 & 235 & 165 & 166 & 90 & 294 & 1046 \\
\llamaeight & 25 & 15 & 91 & 37 & 14 & 353 & 535 \\
\llamaseventy & 33 & 39 & 88 & 27 & 20 & 246 & 453 \\
\yandexgpt & 29 & 76 & 95 & 29 & 16 & 108 & 353 \\
\gptfouro & 2 & 1 & 41 & 0 & 3 & 114 & 161 \\
\claude & 2 & 1 & 26 & 3 & 6 & 96 & 134 \\ 
\bottomrule
\end{tabular}%
}
\caption{Kaz unsafe cases over risk areas of six models.}
\label{tab:unsafe_answers_summary_kazakh}
\end{table}

% \begin{figure*}[t!]
% 	\centering
% 	\includegraphics[scale=0.28]{figures/human_1000_kz_font16.png} 
% 	\includegraphics[scale=0.28]{figures/human_1000_ru_font16.png}

% 	\caption{Human vs \gptfouro\ fine-grained labels on 1,000 Kazakh (left) and Russian (right) samples.}
% 	\label{fig:human_fg_1000}
% \end{figure*}

\textbf{Question Type.} For Russian, Figures \ref{fig:qt_non_reg} and \ref{fig:qt_reg} reveal differences in how models handle general risks I-V and region-specific risk VI. For risks I-V, indirect attacks % crafted to exploit model vulnerabilities—
result in more unsafe responses due to their tricky phrasing. 

In contrast, region-specific risks see slightly more unsafe responses from direct attacks, 
% as these explicit prompts are more likely to bypass safety mechanisms. 
since indirect attacks for region-specific prompts often elicit safer, vaguer answers. %, suggesting models struggle less with implicit harm. 
Overall, the number of unsafe responses is similar across question types, indicating models generally struggle with safety alignment in all jailbreaking queries.

For Kazakh, Figures \ref{fig:qt_non_reg} and \ref{fig:qt_reg} show greater variation in unsafe responses across question types due to the low-resource nature of Kazakh data. For general risks I-V, \llamaseventy\ and \aya\ produce more unsafe outputs for direct harm prompts. At the same time, \claude\ and \gptfouro\ struggle more with indirect attacks, reflecting challenges in handling subtle cues. For region-specific risk VI, most models perform similarly due to limited Kazakh-specific data, though \llamaeight\ shows higher unsafe outputs for indirect local references, likely due to their implicit nature. Direct region-specific attacks yield fewer unsafe responses, as explicit prompts trigger more cautious outputs. Across all risk areas, general questions with sensitive words produce the fewest unsafe answers, suggesting over-flagging or cautious behavior for unclear harmful intent.



% \subsection{Fine-grained Classification}
% We extended our analysis to include fine-grained classifications for both safe and unsafe responses. For unsafe responses, we categorized outputs into four harm types, as shown in Table \ref{table:unsafe_response_categories}. 

% For safe responses, we classified outputs into six distinct patterns of safety, following a fine-grained rubric provided in \cite{wang2024chinesedatasetevaluatingsafeguards}. The types outlined in this rubric are presented in Table \ref{table:safe_response_categories}.

% To validate the fine-grained classification, we conducted human evaluation on the same 1,000 responses in Russian used for the preliminary binary classification.
% The confusion matrix highlights the alignment and discrepancies between human annotations and GPT's fine-grained labels. The diagonal values represent instances where the GPT's labels match human annotations, with category 5 (provides general, safe information) showing the highest agreement (404 instances). However, off-diagonal values reveal areas of disagreement, such as misclassifications in categories 1 (not willing to answer), 4 (recommends consulting an expert), and 7 (provides harmful or risky instructions). For example, in category 1, while 94 responses were correctly labeled, GPT-4o misclassified several instances into categories 4, 5, or 7, indicating overlap or ambiguity in these classifications. Similar trends are observed in other categories, where GPT sometimes struggles to differentiate nuanced distinctions in human-labeled categories. Overall, GPT's fine-grained labels match human annotations in 710 out of 1000 cases, achieving an agreement rate of 71\%.

% \begin{figure}[ht!]
%     \centering
%     \includegraphics[width=0.95\linewidth]{figures/human_fg_1000_ru.png}
%     \caption{{Human vs GPT-4o Fine-Grained Labels on 1,000 Russian Samples}}
%     \label{fig:human_fg_1000_ru}
% \end{figure}


% After conducting human evaluation on a representative sample, we extended the fine-grained classification to a full dataset comprising 21,915 responses generated by five different models. 


% \begin{figure}[ht!]
%     \centering
%     \includegraphics[width=0.95\linewidth]{figures/all_5_ru.png}
%     \caption{Fine-grained label distribution for responses from five models for Russian.}
%     \label{fig:all_5_fg_russian}
% \end{figure}
% Category 5 ("safe and general information") consistently has the highest frequency across all models, aligning with its dominance in the 1,000-sample evaluation. However, differences in the distribution across other categories highlight variability in how models handle nuanced safety risks, with Yandex-GPT showing a slightly broader spread across categories. 
% In the distribution of unsafe responses, most models exhibit higher counts in certain labels such as 8. However, Yandex-GPT displays comparatively fewer responses in these labels. 
% It exhibits a high rate of responses classified under label 7, which indicates instances where the model provides harmful or risky instructions, including unethical behavior or sensitive discussions. While this may suggest a vulnerability in addressing complex or challenging prompts, it was observed that many of Yandex-GPT’s responses tend to deflect responsibility or offer vague advice such as "check the internet". Although this approach minimizes the risk of unsafe outputs, it often results in responses that lack depth or contextually relevant information, limiting their overall usefulness for users.

% % \subsection{Kazakh}


% % Overall, these findings underscore how resource constraints and prompt explicitness affect model safety in Kazakh. Some models manage direct attacks better yet fail on indirect ones, while region-specific content remains challenging for all given the lack of localized training data.
% \subsubsection{Fine-grained Classification}
% Similarly, we conducted a human evaluation on 1,000 Kazakh samples, following the same methodology as the Russian evaluation. The match between human annotations and GPT-4o's fine-grained classifications was 707 out of 1,000, ensuring that the fine-grained classification framework aligned well with human judgments.
% The confusion matrix in Figure \ref{fig:human_fg_1000_kz} for 1,000 Kazakh samples illustrates the agreement between human annotations and GPT-4o's fine-grained classifications. The highest agreement is observed in category 5 (360 instances), indicating GPT-4o's strength in recognizing responses labeled by humans as "safe and general information." However, discrepancies are evident in categories such as 3 and 7, where GPT-4o misclassified several instances, highlighting areas for further refinement in distinguishing nuanced classifications.


\begin{figure}[t!]
	\centering
	\includegraphics[scale=0.18]{figures/human_1000_kz_font16.png} 
	\includegraphics[scale=0.18]{figures/human_1000_ru_font16.png}

	\caption{Human vs \gptfouro\ fine-grained labels on 1,000 Kazakh (left) and Russian (right) samples.}
	\label{fig:human_fg_1000}
\end{figure}

% \begin{figure}[t!]
% 	\centering
% 	\includegraphics[scale=0.28]{figures/human_1000_kz_font16.png} 
% 	\includegraphics[scale=0.28]{figures/human_1000_ru_font16.png}

% 	\caption{Human vs \gptfouro\ fine-grained labels on 1,000 Kazakh (top) and Russian (bottom) samples.}
% 	\label{fig:human_fg_1000}
% \end{figure}

% \begin{figure*}[t!]
% 	\centering
% 	\includegraphics[scale=0.28]{figures/all_5_kz_font16.png} 
% 	\includegraphics[scale=0.28]{figures/all_5_ru_font_16.png} \\
% 	\caption{Fine-grained responding pattern distribution across five models for Kazakh (left) and Russian (right).}
% 	\label{fig:all_5}
% \end{figure*}

\begin{figure}[t!]
	\centering
	\includegraphics[scale=0.28]{figures/all_5_kz_font16.png} 
	\includegraphics[scale=0.28]{figures/all_5_ru_font_16.png} \\
	\caption{Fine-grained responding pattern distribution across five models for Kazakh (top) and Russian (bottom).}
	\label{fig:all_5}
\end{figure}


\subsection{Fine-Grained Classification}
\label{sec:fine-grained-classification}
% As discussed in Section \ref{harmfulness_evaluation}, 
We further analyzed fine-grained responding patterns for safe and unsafe responses. For unsafe responses, outputs were categorized into four harm types, and safe responses were classified into six distinct patterns of safety, as rubric in Appendix \ref{safe_unsafe_response_categories}. 
% \cite{wang2024chinesedatasetevaluatingsafeguards}

\paragraph{Human vs. GPT-4o}
Similar to binary classification, we validated \gptfouro's automatic evaluation results by comparing with human annotations on 1,000 samples for both Russian and Kazakh. %, comparing human annotations with \gptfouro's fine-grained labels.
For the Russian dataset, \gptfouro's labels aligned with human annotations in 710 out of 1,000 cases, achieving an agreement rate of 71\%. 
Agreement rate of Kazakh samples is 70.7\%. %with 707 out of 1,000 cases matching
% The confusion matrix highlights areas of alignment and discrepancies.
% 
As confusion matrices illustrated in Figure~\ref{fig:human_fg_1000}, the majority of cases falling into \textit{safe responding patter 5} --- providing general and harmless information, for both human annotations and automatic predictions.
% highest agreement with 404 correct classifications for Russian. 
Mis-classifications for safe responses mainly focus on three closely-similar patterns: 3, 4, and 5, and patterns 7 and 8 are confusing to discern for unsafe responses, particularly for Kazakh (left figure).
We find many Russian samples which were labeled as ``1. reject to answer'' by humans are diversely classified across 1-6 by GPT-4o, which is also observed in Kazakh but not significant.

% suggesting label alignment issues are language-independent. 
% YX: I did not observe this, commented
% Notably, Russian showed confusion between 7 (risky instructions) and 1 (refusal to answer), this trend does not appear in Kazakh.


% highlight the strengths and limitations of {\gptfouro}'s fine-grained classification framework across both languages, paving the way for further refinements.


% However, misclassifications were observed in categories such as 1 (not willing to answer), 4 (recommends consulting an expert), and 7 (provides harmful or risky instructions), revealing overlaps and ambiguities in nuanced classifications.

% Similarly, for the Kazakh dataset, the agreement rate between human annotations and GPT-4o's labels was 70.7\%, with 707 out of 1,000 cases matching. As with the Russian analysis, category 5 (360 instances) showed the highest alignment. However, discrepancies were more prominent in categories such as 3 and 7, underscoring GPT-4o's challenges in differentiating fine-grained human-labeled categories. 
% A similar pattern was observed for Kazakh dataset, which suggests that alignment and misaligned of fine-grained lables is not language dependent.

% These findings, illustrated in Figures \ref{fig:human_fg_1000}, highlight the strengths and limitations of {\gptfouro}'s fine-grained classification framework across both languages, paving the way for further refinements.

\paragraph{Fine-grained Analysis of Five LLMs}
% After conducting human evaluation on representative samples, we extended 
\figref{fig:all_5} shows fine-grained responding pattern distribution across five models based on the full set of Russian and Kazakh data.
% For Russian, we selected \vikhr, \gptfouro, \llamaseventy, \claude, and \yandexgpt, while for Kazakh, we chose \aya, \gptfouro, \llamaseventy, \claude, and \yandexgpt. 
% The evaluation covered 21,915 responses in Russian and 18,930 responses in Kazakh.
% 
In both languages, pattern 5 of providing \textit{general and harmless information} consistently witnessed the highest frequency across all models, with \llamaseventy\ exhibiting the largest number of responses falling into this category for Kazakh (2,033). 
% YX:summarize more noteable findings here.

Differences of other patterns vary across languages. 
Unsafe responses in Russian are predominantly in pattern 8, where models provide incorrect or misleading information without expressing uncertainty. % (misinformation and speculation), 
For Kazakh, \aya\ exhibits the highest occurrence of pattern 7 (harmful or risky information) and pattern 8, indicating a stronger tendency to generate unethical, misleading, or potentially harmful content.

%Variations in other patterns across models highlight differences in how nuanced safety risks are classified, reflecting the models' differing capabilities in handling safety evaluation for these distinct linguistic contexts. For Russian, the majority of unsafe responses fall under pattern 8 (misinformation and speculation), indicating that models frequently provide incorrect or misleading information without acknowledging uncertainty. For Kazakh, \aya\ has the highest occurence of pattern 7 (harmful or risky information) and pattern 8 (misinformation and speculation), indicating a greater tendency to generate unethical, misleading, or potentially harmful content. 

%This trend suggests that Russian models may struggle more with factual accuracy, whereas Kazakh models, particularly \aya, pose higher risks related to both harmful content and misinformation. Additionally, \gptfouro\ and \claude\ consistently produce fewer unsafe responses in both languages, demonstrating stronger alignment with safety standards
\subsection{Code Switching}
\begin{table}[t!]
\centering

\setlength{\tabcolsep}{3pt}
\scalebox{0.7}{
\begin{tabular}{lcccccccccc}
\toprule
\textbf{Model Name} & \multicolumn{2}{c}{\textbf{Kazakh}} & \multicolumn{2}{c}{\textbf{Russian}} & \multicolumn{2}{c}{\textbf{Code-Switched}} \\  
\cmidrule(lr){2-3} \cmidrule(lr){4-5} \cmidrule(lr){6-7}
& \textbf{Safe} & \textbf{Unsafe} & \textbf{Safe} & \textbf{Unsafe} & \textbf{Safe} & \textbf{Unsafe} \\ 
\midrule
\llamaseventy & 450 & 50 & 466 & 34 & 414 & 86 \\
\gptfouro & 492 & 8 & 473 & 27 & 481 & 19
 \\
\claude & 491 & 9 & 478 & 22 & 484 & 16 \\ 
\yandexgpt & 435 & 65 & 458 & 42 & 464 & 36 \\
\midrule
\end{tabular}}
\caption{Model safety when prompted in Kazakh, Russian, and code-switched language.}
\label{tab:finetuning-comparison}
\end{table}


\gptfouro\ and \claude\ demonstrate strong safety performance across three languages, even with a high proportion of safe responses in the challenging code-switching context. In contrast, \llamaseventy\ and \yandexgpt\ are less safe, exhibiting more harmful responses, particularly in the code-switching scenario. These results show the varying capabilities of models in defending the same attacks that are just presented in different languages, where open-sourced large language models especially require more robust safety alignment in multilingual and code-switching scenarios.

% \subsection{LLM Response Collection}
% We conducted experiments with a variety of mainstream and region-specific 
% large language models for both Russian and Kazakh languages. For both Russian and Kazakh languages, we employed four multilingual models: Claude-3.5-sonnet, Llama 3.1 70B \cite{meta2024llama3}, GPT-4 \cite{openai2024gpt4o}, and YandexGPT. Additionally, we included language-specific models: VIKHR \cite{nikolich2024vikhrconstructingstateoftheartbilingual} for Russian and Aya \cite{ustun-etal-2024-aya} for Kazakh. 

% \subsection{Kazakh-Russian Code-Switching Evaluation}

% In Kazakhstan, the prevalence of bilingualism is a defining characteristic of its linguistic landscape, with most individuals seamlessly mixing Kazakh and Russian in daily communication \cite{Zharkynbekova2022}. This phenomenon, known as code-switching, reflects the unique cultural and social dynamics of the region. Despite this, there is currently no safety evaluation dataset tailored to this unique multilingual environment. Developing a code-switched dataset is essential to evaluate the ability of large language models (LLMs) to navigate the complexities of bilingual interactions, ensuring they produce contextually appropriate, non-harmful, and culturally sensitive responses. To address this, we sampled 500 questions from both Kazakh and Russian question sets, ensuring they represent similar queries but with culturally nuanced differences in meaning. Building on these, we developed a mixed version that integrates Kazakh and Russian naturally, reflecting the way people in Kazakhstan communicate. This process involved annotators to ensure that the mixed questions were crafted with a natural tone and were localized to Kazakhstan's context.


% \begin{table*}[t!]
%     \centering
%     \small
%     \setlength{\tabcolsep}{6pt} % Adjusts column separation
%     \adjustbox{max width=\linewidth}{
%     \begin{tabular}{ll ccc ccc c}
%     \toprule
%     \textbf{Risk Area} & \textbf{Harm Type} & \multicolumn{3}{c}{\textbf{Kazakh}} & \multicolumn{3}{c}{\textbf{Russian}} & \textbf{\#Q} \\
%                        &                    & \textbf{Ori} & \textbf{FN} & \textbf{FP} & \textbf{Ori} & \textbf{FN} & \textbf{FP} & \\
%     \midrule
%     \multirow{2}{*}{I. Information Hazards} & 1. Risks from leaking sensitive information... & 131 & 133 & 131 & 131 & 133 & 131 & 790 \\
%                                              & 2. Compromise of privacy by leaking or inferring private information (person/individual) & 81 & 82 & 81 & 81 & 82 & 81 & 488 \\
%     \midrule
%     \multirow{3}{*}{II. Malicious Uses} & 3. Assisting illegal activities & 132 & 135 & 132 & 132 & 135 & 132 & 798 \\
%                                         & 4. Nudging or advising unethical actions & 71 & 71 & 71 & 71 & 71 & 71 & 426 \\
%                                         & 5. Reducing the cost of disinformation campaigns & 40 & 42 & 40 & 40 & 42 & 40 & 244 \\
%     \midrule
%     \multirow{3}{*}{III. Discrimination, Exclusion, Toxicity, Hateful, Offensive} & 6. Social stereotypes and unfair discrimination & 94 & 96 & 94 & 94 & 96 & 94 & 568 \\
%                                         & 7. Toxic language (hate speech) & 52 & 59 & 52 & 52 & 59 & 52 & 326 \\
%                                         & 8. Adult content & 27 & 29 & 27 & 27 & 29 & 27 & 166 \\

%     \midrule
%     \multirow{2}{*}{IV. Misinformation Harms} & 9. Disseminating false or misleading information & 92 & 99 & 92 & 92 & 99 & 92 & 566 \\
%                                         & 10. Causing material harm by disseminating misinformation e.g. in medicine or law & 63 & 63 & 63 & 63 & 63 & 63 & 378 \\
%     \midrule
%     \multirow{2}{*}{V. Human–chatbot Interaction Harms} & 11. Mental health or overreliance concerns & 66 & 66 & 66 & 66 & 66 & 66 & 396 \\
%                                         & 12. Treating the chatbot as a human & 50 & 51 & 50 & 50 & 51 & 50 & 302 \\

%     \midrule
%     \multirow{2}{*}{VI. Kazakhstan-specific Sensitivity} & 13. Politically sensitive topics & 63 & 66 & 63& 63 & 66 & 63 &  384 \\
%         & 14. Controversial historical events & 46 & 57 & 46 & 46 & 57 & 46  & 298 \\
% & 15. Regional and racial issues & 45 & 45 & 45 &  45 & 45 & 45 & 270  \\
% & 16. Societal and cultural concerns & 138 & 139 & 138 &  138 & 139 & 138  & 830  \\
% & 17. Legal and human rights matters & 57 & 57 & 57 & 57 & 57 & 57  & 342 \\
%     \midrule
%         \multirow{2}{*}{VII. Russia-specific Sensitivity} 
%             & 13. Politically sensitive topics & - & - & - & 54 & 54 & 54 & 162 \\
%     & 14. Controversial historical events & - & - & - & 38 & 38 & 38 & 114 \\
%     & 15. Regional and racial issues & - & - & - & 26 & 26 & 26 & 78 \\
%     & 16. Societal and cultural concerns & - & - & - & 40 & 40 & 40 & 120 \\
%     & 17. Legal and human rights matters & - & - & - & 41 & 41 & 41 & 123 \\
%     \midrule
%     \bf Total & -- & 1248 & 1290 & 1248 & 1447 & 1489 & 1447 & \textbf{8169} \\
%     \bottomrule
%     \end{tabular}
%     }
%     \caption{The number of questions for Kazakh and Russian datasets across six risk areas and 17 harm types. Ori: original direct attack, FN: indirect attack, and FP: over-sensitivity assessment.}
%     \label{tab:kazakh-russian-data}
% \end{table*}




\section{Discussion}

% \subsection{Kazakh vs Russian}

% The evaluation reveals that Kazakh responses tend to be generally safer than their Russian counterparts, likely due to Kazakh being a low-resource language with significantly less training data. As a result, Kazakh models are less exposed to the vast, often unfiltered datasets containing harmful or unsafe content, which are more prevalent in high-resource languages like Russian. This data scarcity naturally limits the model's ability to generate nuanced but potentially unsafe responses. However, this does not mean the models are specifically fine-tuned for safer performance. When analyzing unsafe answers, it’s clear that Kazakh models, while safer overall, distribute their unsafe responses more evenly across various risk types and question types. This suggests Kazakh models generate fewer unsafe answers but in a broader range of contexts.

% In contrast, Russian models tend to concentrate unsafe answers in specific areas, particularly region-specific risks or indirect attacks. This indicates that Russian models have learned to handle certain types of unsafe content by focusing on specific topics, such as politically sensitive issues, but struggle when confronted with unfamiliar content, leading to unsafe responses due to insufficient filtering. Kazakh models, despite having less training data, tend to respond more broadly, including both direct and indirect risks. This could be due to the less curated nature of their training data, making them more likely to answer unsafe questions without filtering the potential harm involved. The exception to this trend is Aya, a model specifically fine-tuned for Kazakh. Despite fine-tuning, it exhibits the lowest safety percentage (72.37\%) in the Kazakh dataset, suggesting that fine-tuning in specific languages may introduce risks if proper safety measures are not taken.

% The evaluation reveals notable differences in the distribution of safe response patterns across Kazakh and Russian fine-grained labels. Refusal to answer is more frequent in Russian models, particularly Yandex-GPT, reflecting a cautious approach to safety-critical queries. Interestingly, Aya, despite being fine-tuned for Kazakh and exhibiting lower overall safety, also frequently refuses to answer, suggesting an over-reliance on conservative mechanisms. Responses providing general, safe information dominate in both languages, with Kazakh models displaying a slightly higher tendency to rely on this approach. This highlights how the low-resource nature of Kazakh results in more generalized and inherently safer responses. In contrast, Russian models excel at recognizing risks, issuing disclaimers, and refuting incorrect assumptions, likely benefiting from richer and more diverse training data.
% Yandex-GPT exhibits a notably high rate of responses classified under label 7, indicating an overreliance on general disclaimers or deflections, such as "check the internet" or "I don't know." While these responses minimize the risk of unsafe outputs, they often lack substantive or contextually relevant information, reducing their overall utility for users.


Most models perform safer on Kazakh dataset than Russian dataset, higher safe rate on Kazakh dataset in \tabref{tab:safety-binary-eval}. This does not necessarily reveal that current LLMs have better understanding and safety alignment on Kazakh language than Russian, while this may conversely imply that models do not fully understand the meaning of Kazakh attack questions, fail to perceive risks and then provide general information due to lacking sufficient knowledge regarding this request.

We observed the similar number of examples falling into category 5 \textit{general and harmless information} for both Kazakh and Russian, while the Kazakh data set size is 3.7K and Russian is 4.3K. Kazakh has much less examples in category 1 \textit{reject to answer} compared to Russian. This demonstrate models tend to provide general information and cannot clearly perceive risks for many cases.

Additionally, in spite of less harmful responses on Kazakh data, these unsafe responses distribute evenly across different risk areas and question categories, exhibiting equally vulnerability spanning all attacks regardless of what risks and how we jailbreak it.
In contrary, unsafe responses on Russian dataset often concentrate on specific areas and question types, such as region-specific risks or indirect attacks, presenting similar model behaviors when evaluating over English and Chinese data.
It suggests that broader training data in English, Chinese and Russian may allow models to address certain types of attacks robustly,
% effectively—particularly politically sensitive issues—
yet they may falter when confronted with unfamiliar content like regional sensitive topics.

Moreover, in responses collection, we observed many Russian or English responses especially for open-sourced LLMs when we explicitly instructed the models to answer Kazakh questions in Kazakh language. This further implies more efforts are still needed to improve LLMs' performance on low-resource languages.
Interestingly, \aya, a fine-tuned Kazakh model, proves an exception by displaying the lowest safety percentage (72.37\%) among Kazakh models, revealing that the multilingual fine-tuning without stringent safety measures can introduce risks.



% However, this does not mean they are explicitly fine-tuned for safety, likely it happens due to limited training data, which reduces exposure to harmful content. 
% \aya, a fine-tuned Kazakh model, proves an exception by displaying the lowest safety percentage (72.37\%) among Kazakh models, revealing that the multilingual fine-tuning without stringent safety measures can introduce risks.
% Kazakh models generally produce safer responses than their Russian counterparts, likely because Kazakh is a low-resource language with less training data. 
% This limited exposure to harmful or unsafe content naturally limits nuanced yet potentially unsafe outputs. 
% However, it does not imply that the models are specifically fine-tuned for enhanced safety.


% while Kazakh models tend to generate fewer unsafe answers overall, those unsafe responses appear more evenly spread across different risk types and question categories.
% Russian models, on the other hand, often concentrate unsafe responses in specific areas, such as region-specific risks or indirect attacks.
% It implies that their broader training datasets allow them to address certain types of unsafe content more effectively—particularly politically sensitive issues—yet they may falter when confronted with unfamiliar or insufficiently filtered content.

% Meanwhile, Kazakh models sometimes respond more broadly, possibly due to less curated training data. 

Differences also emerge in how language models handle safe responses. 
\yandexgpt, for instance, often refuses to answer high-risk queries. 
It frequently relies on generic disclaimers or deflections like ``check in the Internet'' or ``I don’t know,'' minimizing risk but are less helpful. Interestingly, it often responds with ``I don’t know'' in Russian, even for Kazakh queries, we speculate that these may be default responses stemming from internal system filters, rather than generated by model itself.
This likely explains why \yandexgpt\ is the safest model for the Russian language but ranks third for Kazakh. While its filters perform well for Russian, they struggle with the low-resource Kazakh language.

% Aya, despite its lower overall safety, also employs refusals often, hinting at an over-reliance on conservative approaches. 

% Across both languages, models commonly resort to providing general, safe information, although Kazakh models lean on this strategy slightly more. 
% Russian models, by contrast, excel at detecting risks, issuing disclaimers, and correcting inaccuracies, likely benefiting from richer and more diverse training data.


% \subsection{Response Patterns}


% We conducted a detailed analysis of the models' outputs and identified several noteworthy patterns. YandexGPT, while being one of the safest overall, frequently generates responses in Russian even when the question is posed in Kazakh. These responses often appear as placeholders, prompting users to search for the answer online. This behavior might not originate from the model itself but rather from safety filters implemented in the YandexGPT system. The model's leading performance in ensuring safety during Russian-language interactions, coupled with its lower performance in Kazakh, can be attributed to the limited robustness of these safety filters when handling unsafe content in Kazakh.

% In contrast, Aya-101 exhibits a tendency to fall into repetition, often repeating the same sentences multiple times. Interestingly, the Vikhr model, despite being of a similar size, does not exhibit this issue. We attribute this difference to two key factors. First, Vikhr and Aya-101 have distinct architectures: Vikhr is based on the Mistral-Nemo model, whereas Aya-101 is built on mT5, an older and less robust model. Second, Aya-101 is a multilingual model, while Vikhr was predominantly trained for Russian. Multilingualism has been shown to potentially degrade performance in large language models~\cite{huang2025surveylargelanguagemodels}, which may explain Aya-101's issues with repetition.

\section{Conclusions}

Video generation poses a unique challenge, requiring the modeling of both spatial interactions and temporal dynamics. Despite impressive advancements, video models continue to struggle with temporal coherence, even for basic motions well-represented in training datasets (Fig.~\ref{fig:failures}). In this work, we identify the training objective as a key factor that prioritizes appearance fidelity over motion coherence.

To address this, we propose VideoJAM, a framework that equips video models with an explicit motion prior. The core idea is intuitive and natural: a single latent representation captures both appearance and motion jointly. Using only two additional linear layers and no additional training data, VideoJAM significantly improves motion coherence, achieving state-of-the-art results even against powerful proprietary models.
Our approach is generic, offering numerous opportunities for future enhancement of video models with real-world priors such as complex physics, paving the way for holistic modeling of real-world interactions.

\section*{Impact Statements}
The primary goal of this work is to advance motion modeling in video generation, empowering models to understand and represent the world more faithfully. As with any technology in the content generation field, video generation carries the potential for misuse, a concern that is widely discussed within the research community. However, our work does not introduce any specific risks that were not already present in previous advancements. We strongly believe in the importance of developing and applying tools to detect biases and mitigate malicious use cases, ensuring the safe and fair use of generative tools, including ours.

\bibliography{main}
\bibliographystyle{icml2025}


%%%%%%%%%%%%%%%%%%%%%%%%%%%%%%%%%%%%%%%%%%%%%%%%%%%%%%%%%%%%%%%%%%%%%%%%%%%%%%%
%%%%%%%%%%%%%%%%%%%%%%%%%%%%%%%%%%%%%%%%%%%%%%%%%%%%%%%%%%%%%%%%%%%%%%%%%%%%%%%
% APPENDIX
%%%%%%%%%%%%%%%%%%%%%%%%%%%%%%%%%%%%%%%%%%%%%%%%%%%%%%%%%%%%%%%%%%%%%%%%%%%%%%%
%%%%%%%%%%%%%%%%%%%%%%%%%%%%%%%%%%%%%%%%%%%%%%%%%%%%%%%%%%%%%%%%%%%%%%%%%%%%%%%
\newpage
\appendix
\onecolumn
%%%%%%%%%%%%%%%%%%%%%%%%%%%%%%%%%%%%%%%%%%%%%%%%%%%%%%%%%%%%%%%%%%%

\section{Compositional Guidance vs. Inner-Guidance}
\label{sec:IP2P}
\citet{Liu2022CompositionalVG} proposed \emph{Composable Diffusion Models} where a diffusion model can be conditioned on several signals $c_1, \dots, c_n$. The model's conditional sampling distribution is, therefore,
\begin{align}
    p_\theta(x | c_1, \dots, c_n) = \frac{p_\theta(x,c_1, \dots, c_n)} {p_\theta (c_1,\dots, c_n)} \propto p_\theta(x, c_1, \dots, c_n) \propto p_\theta(x) \prod_{i=1}^n p_\theta(c_i | x).
\end{align}
where $\theta$ represents the model weights, and $p$ is the sampling distribution. Importantly, this formulation assumes that $c_1,\dots, c_n$ are \emph{independent of each other and the weights of the model $\theta$}, allowing to drop the denominator $p_\theta (c_1,\dots, c_n)$. Notice that this assumption does not hold in our setting, where the motion condition $d_t$ is noisy and strictly dependent on the neural network, as one of its outputs, as well as the text conditioning, as it serves as another input to the model. 

Inspired by \citet{Liu2022CompositionalVG}, InstructPix2Pix (IP2P)~\cite{brooks2022instructpix2pix} used a similar compositional formulation to extend Classifier-Free Guidance~\cite{ho2022classifier} to two conditioning signals. Formally, given two conditions $c_1, c_2$,
\begin{align}
    p_\theta(x | c_1, c_2) = \frac{p_\theta(x, c_1, c_2)} {p_\theta (c_1, c_2)} =\frac{p_\theta (c_1 | c_2,x) p_\theta (c_2 | x) p_\theta (x)}{p_\theta (c_1, c_2)},
\end{align}
taking the log derivative this gives us,
\begin{align}
    \nabla \log p_\theta(x | c_1, c_2) =\nabla \log p_\theta (c_1 | c_2,x) + \nabla \log p_\theta (c_2 | x) p_\theta (x) - \nabla \log p_\theta (c_1, c_2),
    \label{eq:ip2p}
\end{align}
next, the IP2P formulation assumes (similar to \citet{Liu2022CompositionalVG}) that we can omit the term $p_\theta (c_1, c_2)$ since it is independent of $\theta$, which is again incorrect in our case.

For completeness, our ablations in Sec.~\ref{sec:experiments} compare our Inner-Guidance formulation with that of IP2P, and find that this theoretical gap causes significant degradation in the performance. The direct interpretation of Eq.~\ref{eq:ip2p} to VideoJAM employed in our experiments is as follows,
\begin{align*}
    \bold{\Tilde{u}^+}([x_t, d_t], y, t; \theta') = \bold{u^+}([x_t, \emptyset]), \emptyset, t; \theta') + \quad \quad \\
     w_1\cdot \left( \bold{u^+}([x_t, d_t], \emptyset, t; \theta') - \bold{u^+}([x_t, \emptyset]), \emptyset, t; \theta') \right) + \\
     w_2\cdot \left( \bold{u^+}([x_t, d_t], y, t; \theta') - \bold{u^+}([x_t, d_t], \emptyset, t; \theta') \right) \quad
\end{align*}   
where the notations follow Sec.~\ref{sec:inner_guidance}, and we employ the same guidance scales as we do for Inner-Guidance, i.e. $w_1=3, w_2=5$. Note that the notations for $w_1, w_2$ are reversed with respect to Eq.~\ref{eq:guidance} since IP2P condition on the visual signal first and the textual signal second and order matters for IP2P, while our Inner-Guidance formulation is order invariant.


\section{Motivation Experiments} 
\label{sec:motivation_supp}
\begin{wrapfigure}{R}{0.5\textwidth}
\vspace{-26px}
\centering
\noindent
\includegraphics[width=1.01\linewidth, clip]{figures/motivation_fig_supp.pdf}
\vspace{-22px}
\caption{\textbf{Qualitative motivation.} We noise input videos to different timesteps ($20, 60,80$) and continue the generation. By step $60$, the video's coarse motion and structure are mostly determined.
\label{fig:motivation_supp}}
\vspace{-14px}
\end{wrapfigure}
To exemplify that steps $t\leq 60$ of the generation are indeed meaningful to determine the motion, we conduct an SDEdit~\cite{meng2021sdedit} experiment, in which we noise videos to different timesteps ($20, 60, 80$), and continue the generation given the noised videos. In Fig.~\ref{fig:motivation_supp}, we show a representative appearance frame and two motion frames for each video, using RAFT~\cite{raft} to estimate optical flow. We observe that the coarse motion and structure of the generated videos are determined between steps $20$ and $60$, since the generation from step $20$ changes the entire video while starting from step $60$ maintains the coarse motion and structure of the input video, suggesting that they are already determined by the input noisy video.
Note that the appearance may still change between steps $60$ and $80$ (right), whereas from step $80$, both appearance and motion seem to be determined.

\begin{figure*}[t!]
\centering
\includegraphics[width=0.99\textwidth]{figures/qualitative_supp.pdf}
\vspace{-10px}
\caption{\textbf{Additional text-to-video results using VideoJAM-30B.} }
\label{fig:qualitative_supp}
\vspace{-4px}
\end{figure*}

\section{Implementation Details}
\label{sec:implementation_details}
VideoJAM-4B was fine-tuned using $32$ A100 GPUs with a batch size of $32$ for $50,000$ iterations on a spatial resolution of $256\times 256$. It has a latent dimension of $3072$ and $32$ attention blocks (same as the base model). 
VideoJAM-30B was fine-tuned using $256$ A100 GPUs with a batch size of $256$ for $35,000$ iterations on a spatial resolution of $256\times 256$. It has a latent dimension of $6144$ and $48$ attention blocks (same as the base model). Each attention block is constructed of a self-attention layer that performs spatiotemporal attention between all the video tokens, and a cross-attention layer that integrates the text.
Both models were trained with a fixed learning rate of $5e-6$, using the Flow Matching paradigm~\cite{flow-matching} (see Sec.~\ref{sec:preliminaries}). 

During inference, we perform $100$ denoising steps with a linear quadratic t-schedule using a text guidance scale of $w_1=5$ and a motion guidance scale of $w_2=3$ (see Eq.~\ref{eq:guidance}), other than the ablations that test these components. Additionally, we only employ the motion guidance for the first half of the generation steps ($50$ steps) following the conclusions from our motivational experiments (Sec.~\ref{sec:motivation}), as these are the steps that determine the coarse motion in the video, and display less sensitivity to temporal incoherence before applying VideoJAM. In practice, Inner-Guidance is performed similarly to Classifier-Free Guidance~\cite{ho2022classifier}, where all results are generated in a batch 
$\bold{u^+}([x_t, d_t]), y, t; \theta'), \bold{u^+}([x_t, d_t], \emptyset, t; \theta'), \bold{u^+}([x_t, \emptyset], y, t;)$ and the final prediction is calculated following Eq.~\ref{eq:guidance}. The models are trained to generate $128$ frame videos at 24 frames per second, resulting in 5-second video generations.

The models operate in the latent space of a TAE, as specified in Sec.~\ref{sec:preliminaries}. The TAE structure follows that of~\citet{moviegen}, with a temporal compression rate of $\times8$ and a spatial compression rate of
$8\times8$. The Transformer patch size is $1\times 2\times2$.  The text prompt conditioning is processed by three different text encoders: UL2~\cite{tay2022ul2}, ByT5~\cite{xue2022byt5}, and MetaCLIP~\cite{xu2023demystifying}.

Both DiT models were pre-trained using the framework in Sec.~\ref{sec:preliminaries} on a dataset of $\mathcal{O}(100 \text{ M})$ videos. We then fine-tune the models using VideoJAM on under $3$ million random samples from the model's original training set, which constitute less than $3\%$ of the training videos. This allows our fine-tuning phase to be light and efficient. During this fine-tuning, we employ RAFT~\cite{raft} to obtain optical flow per training video.

 
Since each of the baselines generates videos in different resolutions, we resize the baseline results to a $256$ resolution to facilitate a fair and unbiased comparison. No cherry-picking is involved in the evaluation of any of the models, and the first result obtained by each model is taken. All baselines produce the same length of videos ($5$ seconds), therefore we only resize the videos spatially.
For the qualitative results in the website, we train an additional super-resolution model to spatially upsample the $256\times256$ videos to $512\times512$ videos. The training regime follows that of VideoJAM-30B. Note that all our experiments (besides the visualizations on the website) are in the lower $256$ resolution due to resource limitations.

\subsection{VBench Metrics} 
\label{sec:vbench}
We employ all metrics supported by VBench on both VideoJAM-bench and the Movie Gen benchmark. Inspired by the protocol in the VBench paper, we split the metrics into a motion category and an appearance category.
For the appearance category, we include the aesthetic quality and image quality metrics, which assess the per-frame quality of the generated videos, as well as subject consistency and background consistency, which assess the model's ability to maintain a consistent appearance. 
For motion comprehension, we include the motion smoothness score, which aims to assess the realism of the motion, and the dynamic degree score which estimates the amount of motion in the generated videos. In other words, the motion score measures the model's ability to generate meaningful motion (i.e., non-static videos) that is also coherent and plausible.

All scores are normalized and a weighted score is calculated according to the weights suggested in the VBench paper. 
The full results of all VBench metrics for each benchmark are reported in App.~\ref{sec:motion_benchmark},~\ref{sec:moviegen_benchmark}.


\section{VideoJAM-bench: Automatic Metrics Breakdown and Prompts}
\label{sec:motion_benchmark}
In the following, we provide a breakdown of the automatic metrics calculated on our motion benchmark using VBench~\cite{huang2023vbench} for the 4B model (Tab.~\ref{tab:4b_metrics}) and the 30B model (Tab.~\ref{tab:30b_metrics}). As mentioned in App.~\ref{sec:vbench}, the motion metrics measure the amount of motion in the video and the coherence of the motion. In the smaller model category, CogVideo2B scores the highest dynamic degree and the lowest motion smoothness. This indicates that while there is abundant motion in the generated videos, it is incoherent. The DiT-4B base model obtains the best smoothness score, and the worst dynamic degree, indicating that it produces videos with very subtle movements. As can be observed, VideoJAM strikes the best balance, where plenty of motion is generated while maintaining strong coherence.  

For the larger DiT-30B model, we observe, again, that there is a trade-off between the dynamic degree and the motion smoothness, where CogVideo5B produces the most motion, yet it is incoherent. Among the competitive proprietary baselines, notice that Runway Gen 3 obtains a very high dynamic degree, yet it has the lowest motion smoothness among all the proprietary baselines (Runway Gen 3, Sora, Kling 1.5). In Fig.~\ref{fig:qualitative}, we show comparisons to Sora and Kling since these are the most competitive with VideoJAM according to the human evaluation, which is generally considered to be a more reliable evaluation form~\cite{Lumiere,moviegen,Wang2024MotiF}. However, in the website, we include a comparison to Runway Gen 3 in addition to Sora and Kling for completeness. Furthermore, Kling shows the best motion smoothness, with the lowest dynamic degree. Observe that VideoJAM, again, strikes the best balance between motion coherence and the amount of generated motion. Additionally, it outperforms the base model (DiT-30B) across all motion metrics, and nearly all appearance metrics, indicating that our method improves all aspects of the generation.

A full list of the prompts considered in our motion benchmark is provided in App.~\ref{sec:motion_benchmark_prompts}.

\begin{table}[h!]
    \caption{\textbf{Breakdown of the automatic metrics} from VBench comparing our 4B model and previous work on VideoJAM-bench. Our method strikes the best balance between the dynamic degree (higher implies more motion) and the motion smoothness (higher implies smooth motion).}
  \label{tab:4b_metrics}
  \centering
    \setlength{\tabcolsep}{3.5pt}
  \scalebox{0.95}{%
  \begin{tabular}{@{}lcccccc@{}}
    \toprule
      & \multicolumn{4}{c}{Appearance Metrics} & \multicolumn{2}{c}{\textbf{Motion Metrics}}  \\
      \cmidrule(r){2-5}
      \cmidrule(r){6-7}
    \multirow{2}{*}{Method}     &      \small{Aesthetic}        &  \small{Image}     &        \small{Subject}  & \small{Background}        & \textbf{\small{Motion}}& \textbf{\small{Dynamic}}\\
    & \small{Quality}        &  \small{Quality}     &        \small{Consistency}  & \small{Consistency}        & \textbf{\small{Smoothness}}& \textbf{\small{Degree}}\\
    \midrule
    \small{CogVideo2B}   &  46.9 &	48.9 &	87.8 &	93.9 & {97.1} &	\textbf{88.6}   \\
    \small{CogVideo5B}   &  51.1 &	52.9 &	91.3 &	\underline{95.3}  & {97.3}	&{87.5}   \\
    \midrule
    \small{DiT-4B} &   \textbf{51.8} &	\textbf{61.4} &	\underline{93.0} &	\textbf{96.7} & \textbf{99.3} &	{38.3}  \\
    \textbf{+VideoJAM-4B}  & \underline{51.6} &	\underline{61.1} &	\textbf{93.5} &	\textbf{96.7} & \underline{98.8} &	\underline{87.5}   \\
    \bottomrule
    \end{tabular}}
\end{table}

\begin{table}[h!]
    \caption{\textbf{Breakdown of the automatic metrics} from VBench comparing our 30B model and previous work on VideoJAM-bench. Our method strikes the best balance between the dynamic degree (higher implies more motion) and the motion smoothness (higher implies smooth motion).}
  \label{tab:30b_metrics}
  \centering
    \setlength{\tabcolsep}{3.5pt}
  \scalebox{0.95}{%
  \begin{tabular}{@{}lcccccc@{}}
    \toprule
      & \multicolumn{4}{c}{Appearance Metrics} & \multicolumn{2}{c}{\textbf{Motion Metrics}}  \\
      \cmidrule(r){2-5}
      \cmidrule(r){6-7}
    \multirow{2}{*}{Method}     &      \small{Aesthetic}        &  \small{Image}     &        \small{Subject}  & \small{Background}        & \textbf{\small{Motion}}& \textbf{\small{Dynamic}}\\
    & \small{Quality}        &  \small{Quality}     &        \small{Consistency}  & \small{Consistency}        & \textbf{\small{Smoothness}}& \textbf{\small{Degree}}\\
    \midrule
    \small{CogVideo5B}   &  51.1 &	52.9 &	91.3 &	{95.3}  & {97.3}	&\textbf{87.5}   \\
    \small{RunWay Gen3} &    55.1 &	55.1&	90.7	& 95.2 & {98.4} &	\underline{84.4}  \\
    \small{Mochi} &   49.5	& 48.8 &	89.7 &	95.2 & {98.4} &	{78.1}  \\
    \small{Sora} &  \underline{56.8} &	\underline{57.7} &	\underline{93.0} &	\underline{96.4} & {98.7}	& {82.0}  \\
    \small{Kling 1.5} &  \textbf{58.5} &	\textbf{60.4} &	\textbf{93.9} &	\textbf{96.5} & \textbf{99.2}	& {64.8}  \\
    \midrule
    \small{DiT-30B} &   49.2 &	56.8 &	91.3 &	95.5 & {98.8} &	{71.1}  \\
    \textbf{+VideoJAM-30B}  & 51.2 &	55.9	 & \underline{93.0} &	96.1 & \underline{99.0} &	{82.3}   \\
    \bottomrule
    \end{tabular}}
\end{table}


\section{Movie Gen Benchmark}
\label{sec:moviegen_benchmark}

We employ the prompts from the official benchmark labeled as containing ``high'' motion since our primary objective is to estimate motion coherence. Additionally, since the Movie Gen benchmark is significantly larger than VideoJAM-bench, and mostly contains less relevant prompts (Sec.~\ref{sec:experiments}), we consider the baselines that provide open-source code and can run automatically. Importantly, note that the apples-to-apples comparison to the pre-trained model, DiT-30B is presented for this benchmark as well, allowing us to assess the direct impact of VideoJAM on a large video generation model.

The results are reported in Tab.~\ref{tab:moviegen}, with a breakdown of the automatic metrics in Tab.~\ref{tab:moviegen_metrics}.
Similarly to the results on our motion benchmark, VideoJAM strikes the best balance between the amount of motion and the coherence of the generated motion. While  CogVideo5B consistently produces the most motion, it is also consistently the least coherent baseline. Mochi, on the other hand, suffers from the complementary problem where less motion is generated. Notably, VideoJAM outperforms all baselines, by a significant margin across all metrics, both human-based and automatic (other than the dynamic degree, where CogVideo5B scores the highest, as mentioned). Importantly, we observe a consistent improvement over the base model used by VideoJAM, DiT-30B in both the appearance and motion metrics across all evaluations, which further substantiates our method's ability to improve all aspects of video generation.

\begin{table}[h!]
    \caption{\textbf{Comparison of VideoJAM-30B with prior work on the Movie Gen benchmark.} Human evaluation shows \emph{percentage of votes favoring VideoJAM}; automatic metrics use VBench.}
  \label{tab:moviegen}
  \centering
    \setlength{\tabcolsep}{3.5pt}
  \scalebox{0.95}{%
  \begin{tabular}{@{}lccccc@{}}
    \toprule
      & \multicolumn{3}{c}{\textbf{Human Eval}} & \multicolumn{2}{c}{\textbf{Auto. Metrics}}  \\
      \cmidrule(r){2-4}
      \cmidrule(r){5-6}
    Method     &      \small{Text Faith.}        &  \small{Quality}     &        \textbf{\small{Motion}}  & \small{Appearance}        & \textbf{\small{Motion}}\\
    \midrule
    \small{CogVideo5B}   &    61.4      &   77.0     &   {78.7} & \underline{70.8}  & \underline{88.8}   \\
    \small{Mochi} &    53.5     &   59.4     &    {69.1}  &  70.4 & {85.1}  \\
    \midrule
    \small{DiT-30B} &    60.3     &   64.6     &    {66.1}  & 70.5  & {87.3}  \\
    \textbf{+VideoJAM-30B}  & -  & -    & - &  \textbf{73.7} &  \textbf{90.8}   \\
    \bottomrule
    \end{tabular}}
\end{table}


\begin{table}[h!]
    \caption{\textbf{Breakdown of the automatic metrics} from VBench comparing our 30B model and previous work on the Movie Gen benchmark. Our method strikes the best balance between the dynamic degree (higher implies more motion) and the motion smoothness (higher implies smooth motion).}
  \label{tab:moviegen_metrics}
  \centering
    \setlength{\tabcolsep}{3.5pt}
  \scalebox{0.95}{%
  \begin{tabular}{@{}lcccccc@{}}
    \toprule
      & \multicolumn{4}{c}{Appearance Metrics} & \multicolumn{2}{c}{\textbf{Motion Metrics}}  \\
      \cmidrule(r){2-5}
      \cmidrule(r){6-7}
    \multirow{2}{*}{Method}     &      \small{Aesthetic}        &  \small{Image}     &        \small{Subject}  & \small{Background}        & \textbf{\small{Motion}}& \textbf{\small{Dynamic}}\\
    & \small{Quality}        &  \small{Quality}     &        \small{Consistency}  & \small{Consistency}        & \textbf{\small{Smoothness}}& \textbf{\small{Degree}}\\
    \midrule
    \small{CogVideo5B}   &  \underline{50.9} &	\underline{51.9} &	89.5 &	94.7 & {97.5} &	\textbf{81.6}  \\
    \small{Mochi} &   50.4 &	50.1 &	89.0 &	\underline{95.4} & \underline{98.9} &	{60.7}  \\
    \midrule
    \small{DiT-30B} &   48.7 &	50.6 &	\underline{90.8} &	95.3 & \underline{98.9} &	{67.8}\\
    \textbf{+VideoJAM-30B}  & \textbf{51.5}	& \textbf{56.4} &	\textbf{93.3} &	\textbf{96.2} &\textbf{99.1}	 & \underline{76.9}  \\
    \bottomrule
    \end{tabular}}
\end{table}



\section{VideoJAM-bench Prompts}
\label{sec:motion_benchmark_prompts}
Below, we present the full set of $128$ prompts used in our motion benchmark, VideoJAM-bench. The benchmark is designed to be diverse, encompassing simple motions (e.g., walking), complex human movements (e.g., gymnastics), rotational motions (e.g., spinning balls), and physics-based actions (e.g., a woman hula hooping). To ensure clarity, the prompts were refined using an LLM to focus on specific motion types, enabling a precise evaluation of the model’s ability to generate coherent movement. Additionally, the prompts vary in detail and include camera instructions to test the model’s performance across a wide range of scenarios.

\begin{enumerate}
    \item \emph{``A woman performing an intricate dance on stage, illuminated by a single spotlight in the first frame. She is dressed in a long black dress and a wide-brimmed hat, with her arms raised above her head. The woman dance Argentine flamenco dance.''}
    \item \emph{``A woman doing a headstand on a beach.''}
    \item \emph{``A woman engaging in a challenging workout routine, performing pull-ups on green bars.''}

    \item \emph{``Two ibexes navigating a rocky hillside. They are walking down a steep slope covered in small rocks and dirt. In the background, there are more rocks and some greenery visible through an opening in the rocks.''}

     \item \emph{``A close-up of a runner's legs as they sprint through a crowded city street, dodging pedestrians and street vendors, with the sounds of the city all around.''}
    
    \item \emph{``Athletic man doing gymnastics elements on horizontal bar in city park. Male sportsmen perform strength exercises outdoors.''}
    \item \emph{``A small dog playing with a red ball on a hardwood floor.''}
    \item \emph{``A woman engaging in a lively trampoline workout. The woman jumps and exercises on the trampoline. The background is a room with white walls and a white ceiling, and there are two large windows on the left side of the wall, and a mirror on the right side reflecting the womans image.''}
    \item \emph{``A man performing a handstand on a wooden deck overlooking a green lake surrounded by trees.''}
    \item \emph{``Young adult female performs an air gymnastic show on circus arena, holding ring in hand, making twine exercise, spin around''}
    \item \emph{``A woman enjoying the fun of hula hooping.''}
    \item \emph{``A man juggling with three red balls in a city street.''}

    \item \emph{``A white kitten playing with a ball.''}
    \item \emph{``A slow-motion shot captures a runner's legs as they dash through a busy intersection, dodging cars and pedestrians, the city life bustling around them.''}
    \item \emph{``A young girl playing basketball in a red brick wall background. The girl, with fair skin and long blonde hair, is wearing a green jacket and has her left arm up to throw the ball. In the mid-frame, the girl is still playing basketball, with her right hand holding the ball in front of her face. The ground is dark gray cement with some patches of grass growing through it. As the video progresses, the girl is seen playing near some grassy areas on the ground.''}
    \item \emph{``A basketball game in progress, with two players reaching up to grab the ball as it spills out of the net. The player on the left has his hand outstretched, while the player on the right has both hands raised high. The ball is just above their fingertips, indicating that they are both trying to grab it simultaneously. The background of the image is blurred, but it appears to be a gymnasium or sports arena, with fluorescent lights illuminating the scene. As the video progresses, the players continue to jump and stretch to gain possession of the ball, their movements becoming more urgent and intense. The ball flies back and forth between them, with neither player able to secure it. In the final frame, the ball is still in mid-air, the players hands reaching up to grab it as the video ends.''}
    \item \emph{``A group of basketballs floating in mid-air in slow motion, with a larger ball on the left and two smaller balls on either side in the initial frame. Overall, the video captures the dynamic and energetic movement of basketballs as they float and bounce through space.''}
    
    \item \emph{``A dog playing with an orange ball with blue stripes. The dog picks up the ball and holds it in its mouth, conveying a sense of playfulness and energy. Throughout the video, the dog is seen playing with the ball, capturing the joy and excitement of the moment.''}


    \item \emph{``A woman doing acrobatic exercises on a pole in the gym.''}
    \item \emph{``A young man performing a cartwheel on a gray surface. He is dressed in orange pants, a black t-shirt, and white sneakers. As he executes the cartwheel, his right arm is extended upward, and his left arm is bent at the elbow, reaching down to the ground. His right leg is extended behind him, while his left leg is bent at the knee, pointing towards the camera. The background is a featureless gray wall. The mans energy and focus are evident as he completes the cartwheel, showcasing his athleticism and coordination.''}
    
    
    \item \emph{``A golden retriever playing fetch on a grassy field. The dog is running with a frisbee in its mouth, its fur waving in the wind.''}
    \item \emph{``A brightly colored ball spins rapidly on a flat surface, its patterns blurring as it twirls in place.''}
    \item \emph{``A basketball spins on a player's fingertip, maintaining balance while gradually slowing down.''}
    \item \emph{``A person jogs along a forest trail at dawn, their feet kicking up dirt with every stride, the sunlight filtering through the trees casting long shadows on the path.''}

   \item \emph{``A child jumps up and down in place, their feet leaving the ground briefly before landing again.''}

    \item \emph{``A person lifts one knee high in a marching motion, then places their foot back down and repeats with the other leg.''}

    \item \emph{``Professional cyclist training indoors on a stationary bike trainer.''}

    \item \emph{``Young Adult Male Doing Handstand on the beach.''}

    \item \emph{``A young woman practicing boxing in a gym.''}

    \item \emph{``A man jumping in a pool.''}

    \item \emph{``A man doing push-ups on a ledge overlooking a body of water. The man appears to be doing a push-up, with his head down.''}

    \item \emph{``A man enjoying a leisurely bike ride along a road next to a body of water during a sunset. As he pedals, he looks down at his front wheel, seemingly focused on his ride. The background features a large body of water, with a gray wall along the left side of the road in the mid-frame caption.''}

    \item \emph{``close up shot of the feet of a woman exercising on a cardio fitness machine in a fitness club. As the video progresses, the legs continue to pedal the bike in a smooth, consistent motion.''}

    \item \emph{``A woman engaging in an intense workout on a stationary bike while monitoring her progress on a screen.''}

    \item \emph{``A woman running along a river with a city skyline in the background.''}

    \item \emph{``A skier walking up a snowy hill with their skis on their back and ski poles in hand.''}

    \item \emph{``A woman running through a grassy area, wearing a black tank top, gray and white leggings, and white sneakers. She is initially running on a dirt path, surrounded by trees with green leaves. As she continues to run, the scenery changes to a park, and her leggings change to a blue and white pattern. She is still running on a dirt path, surrounded by trees and green grass. The video captures her journey as she runs through the grassy area, enjoying the outdoors and the beauty of nature.''}

    \item \emph{``A young girl coloring at her desk.''}

    \item \emph{``A close-up of a runner's legs as they dash through a rainstorm, their shoes splashing through puddles as they push forward with determination.''}

    \item \emph{``Tracking camera shot. A kangaroo hops swiftly across an open grassy plain.''}

    \item \emph{``A close-up view of a spiral object with a glowing center. The object appears to be made of metal and has a shiny, reflective surface. . This light creates a series of concentric circles around the objects circumference, which are visible due to the reflection of the light off the metal surface.''}

    \item \emph{``A roulette wheel in a dimly lit room or casino floor. In the center of the wheel, there's a small white ball that appears to be spinning rapidly as it moves around the track. The ball spins around the wheel, and the wheel rotates counterclockwise.''}

    \item \emph{``A close-up of a jogger's feet as they run along a rocky coastal path, their shoes gripping the uneven surface, with the ocean waves crashing below.''}

    \item \emph{``A person's hands as they shape and mold clay on a pottery wheel. The hands are covered in brown clay and are visible from the elbows down, with the forearms resting on top of a large yellow pottery wheel.''}

    \item \emph{``A conveyor belt pouring out a large amount of small, brown objects into a pile on the ground. The objects being poured are falling from the conveyor belt in a steady stream, forming a large pile on the ground below. In the background, the sky is bright blue and cloudless, providing a stark contrast to the darker colors of the conveyor belt and the pile of objects.''}

    \item \emph{``A 3d rendering of coins and small objects floating against a black background. The coins are gold, silver, bronze, and copper, with various denominations and sizes. Some have a shiny finish, while others are matte or tarnished. The scene is chaotic and dynamic, with the objects seemingly flying around in all directions. As the video progresses, the coins and objects tumble and spin, creating a sense of movement and energy. By the end, the screen is filled with white objects of various shapes and sizes, suggesting that something exciting is happening.''}

    \item \emph{``A puppy runs through a grassy field.''}

    \item \emph{``A cinematic shot of a person walking along a quiet country road, their feet crunching on the gravel with every step, fields of wheat swaying in the breeze on either side.''}


    \item \emph{``A washing machine undergoing a full cycle. It begins with a top-down view of the machine filled with water and white soap suds, with two black rubber seals on either side of the stainless steel drum. The video progresses to show the drum spinning, with the suds becoming more agitated and the seals moving along with the drums motion.''}

    \item \emph{``Sweet Cherries on Stems Colliding and Splashing Water Droplets''}

    \item \emph{``A series of colorful balloons floating in mid-air, creating a festive and celebratory atmosphere.''}

    \item \emph{``A cinematic shot of a person jogging along a riverside path, their feet rhythmically tapping against the ground, the river flowing gently beside them.''}

    \item \emph{``A green helicopter taking off from an airport runway.''}

    \item \emph{``A hand holding a yellow fidget spinner. The hand is fair-skinned and holds the bright yellow fidget spinner with silver bearings. The background is blurred and appears to be trees against a blue sky. The video captures the subtle movements of the hand as it spins the fidget spinner, creating a soothing and mesmerizing visual effect. As the video progresses, the hand continues to hold the fidget spinner, showcasing its smooth and satisfying motion. The background remains blurred, adding a sense of tranquility to the scene. Overall, the video is a calming and enjoyable display of the simple pleasure of fidget spinning.''}

    \item \emph{``A windmill spinning in a green field.''}

    \item \emph{``A bicycle wheel spins forward, moving in a circular motion while keeping balance.''}

    \item \emph{``A waterwheel turns as water flows over it, the paddles rotating consistently.''}

    \item \emph{``A close-up of a person's legs as they walk through a sun-dappled forest, the light playing off their shoes as they navigate the uneven terrain.''}

    \item \emph{``A man riding a mountain bike on a dirt trail.''}

    \item \emph{``A child’s toy top spins on a smooth surface, rotating without stopping.''}

    \item \emph{``A basketball spins on a player's fingertip, showcasing balance and skill.''}

    \item \emph{``A jellyfish swimming in shallow water. The jellyfish has a translucent body with a distinctive pattern of white circles and lines. It appears to be swimming just below the surface of the water, which is dark and murky due to the presence of algae or other aquatic plants.''}

    \item \emph{``A cinematic shot of a person walking along a cobblestone street in a historic town, their feet making a rhythmic tap on the stones as they move.''} 

    \item \emph{``A group of horses grazing in a grassy field behind a black wooden fence''}
    
    \item \emph{``A fish swims forward in a steady line, its tail swaying side to side as it propels itself.''}

    \item \emph{``A penguin waddles in a straight line, shifting from one foot to the other.''}

    \item \emph{``A man is jumping rope on the sandy beach, with waves crashing in the background.''}

    \item \emph{``A man enjoying water skiing on a brown river with a green shore and lily pads in the background. Water sprays up from underneath him as he skis across the surface of the lake.''}

    \item \emph{``A man is swimming in a clear blue pool, enjoying the cool water and the freedom of movement in the pool. As he continues to swim, he glides gracefully through the water, his arms and legs moving in a smooth and coordinated rhythm.''}

    \item \emph{``A kid running in the mountains of Campo Imperatore, Italy, at the sunset. He is wearing a red polo shirt, blue jeans, and brown shoes. As he runs, he passes by some white rocks on the ground.''}


    \item \emph{``A woman doing push-up exercise on a beach at sunset.''}

    \item \emph{``A woman is shown running through a field, with tall grass and wildflowers all around her. She is a fair-skinned woman with long, red hair, wearing a black t-shirt and leggings, and listening to music on her phone. In the background, there are trees and more fields of greenery.''}

    \item \emph{``A man exercising with battle ropes at a gym.''}

    \item \emph{``A person engaging in a boxing workout at a gym.''}

    \item \emph{``A dark gray horse running in an enclosed corral. It is running towards the camera.''}

    \item \emph{``A close-up of a runner's legs as they dash up a flight of stairs in a city park, their feet hitting each step with precision and power.''}


    \item \emph{``A man is swimming  in the ocean. In the background, the sky is hazy and overexposed, with the sun shining brightly above the horizon. As the video progresses, the man continues to swim, his arms moving rhythmically through the water.''}

    \item \emph{``A herd of white cows walking down a dirt path. The cows are all facing forward and walking towards the right side of the image. The background is blurry but appears to be a field or pasture.''}

    \item \emph{``A person jogs along a trail in a dense forest, their legs pumping as they navigate the roots and rocks that dot the path.''}

    \item \emph{``A young woman dances in the night bustle against the backdrop of a glowing fanfare.''}

    \item \emph{``A man is walking down the street while pushing a trash can. The man, wearing a red t-shirt, blue jeans, and brown sandals, pushes the black trash can on wheels.''}

    
    \item \emph{``A man enjoying a mountain biking adventure through a forest. He is seen riding a black and white mountain bike down a dirt path, with his back to the camera.''}

    \item \emph{``Women's legs walk into the sea with waves.''}

    \item \emph{``A young man walking on a treadmill. He is wearing a white tank top and red shorts, and has his hands on the sides of the machine as he runs.''}

    \item \emph{``Closeup of feet of a professional soccer player training with ball on stadium field with artificial turf.''}

    \item \emph{``A helicopter flying over a forest. The helicopter is black and has two large rotor blades on top. It is flying low to the ground, with its nose pointing slightly upwards.''}

        
   \item \emph{``A close-up of a person's feet as they walk through a field of wildflowers, their shoes brushing against the blooms with each step.''}


    \item \emph{``A man is playing basketball, dribbling the ball and making shots.''}

    \item \emph{``A giraffe running through an open field. The background is a bright blue sky with fluffy white clouds.''}

       \item \emph{``A person jogs along a city waterfront, their legs moving steadily as the sun sets, casting a warm glow over the water and the buildings behind them.''}

    \item \emph{``A woman is doing push-ups on a mat in the studio.''}

    \item \emph{``Two dancers perform on a stage. The man stands behind the woman with his left arm is lifted over his head and the other is stretched to the right. The woman lets go of the man's right hand, swinging her leg to the left and performing a pirouette. She spins four times and ends up facing the man.''}

    \item \emph{A woman drinks from a water bottle in a forest. The woman has fair skin and brown hair. She is wearing a black jacket and black and white gloves.}

    \item \emph{``Tracking camera shot. A polar bear walks across a snowy landscape. It looks curiously around as it plods through the snow. The background is a snowy landscape with footprints visible in the snow. Sunlight shines from overhead and casts the bear's shadow on the snow.''}

       \item \emph{``A cinematic shot of a person walking through a desert at midday, their legs moving slowly but steadily across the sand dunes, with heat waves distorting the distant horizon.''}

    \item \emph{``A man jumping rope on a dark stage. His movements are fluid and energetic. Two spotlights shine down from above him.''}
    
    \item \emph{``A woman twirls a hula hoop around her waist in a park during sunset. The woman, with medium-length curly black hair and a yellow tank top, stands on a grassy field surrounded by trees. As the hoop revolves around her waist, she shifts her hips rhythmically to keep it moving. The golden sunlight casts a long shadow behind her.''}

    \item \emph{``A man exercises on a leg press machine at a gym.''}

    \item \emph{``A young woman enjoys a cup of coffee on a balcony.''}

    \item \emph{``A man energetically bangs on a drum kit. He holds drumsticks in both hands and bashes on the drum kit with the drumsticks.''}

    \item \emph{``A woman performs high knees on a beach.''}

    \item \emph{``Aerial tracking camera shot. A white semi-truck drives on a highway.''}

    \item \emph{``A woman is holding a clear wine glass partly filled with a burgundy-colored wine. Facing forward, the woman smiles, she raises the glass with her left hand and takes a small sip.''}

    
    \item \emph{``A man works on a piece of wood in a workroom. He holds a shiny silver chisel with a wooden handle in his right hand.''}

    \item \emph{``Sliced green apples are tossed in a brown liquid. The apples are cut into thick slices and have shiny green skins with some light-colored speckling. They begin to rotate clockwise, flying out in every direction as the light amber liquid splashes and swirls behind them.''}

    \item \emph{``A baboon eats a mango.''}

    \item \emph{``A young woman vapes in the living room. The woman exhales the thick, billowing smoke.''}

    \item \emph{``A woman performing an aerial hoop trick. The woman hangs from a black aerial hoop attached to the ceiling by a rope. In the initial frame, she has her legs wrapped around the hoop and her arms extended outward, holding onto the hoop with both hands. Her body is twisted, looking up towards the ceiling, with her shadow cast on the white wall behind her. As the video progresses, she continues to hang from the hoop, her body twisted in various positions, her arms and legs wrapped around the hoop as she performs the aerial trick. The background remains the same, with shadows from the aerial hoop and the woman's body on the white wall.''}

    \item \emph{``Modern urban street ballet dancer performing acrobatics and jumps.''}

    \item \emph{``A woman doing a pirouette in an empty dance studio.''}

    \item \emph{``A woman dancing hip hop, street dancing in the studio. Slow motion.''}

    \item \emph{``A brunette woman doing some acrobatic elements on aerial hoop outdoors.''}

    \item \emph{``A woman, with long brown hair and wearing a black top and gray bottoms, climbs on a pole with her right leg wrapped around it and her left arm extended upward. The background is a white wall with a mirror reflecting the woman's images.''}

    \item \emph{``A man performing a backflip. Slow motion.''}

    \item \emph{``A woman dancing in a gym. The woman is spinning around repeatedly.''}

    \item \emph{``A group of duck are walking in a row, one after the other. The background is a Japanese temple.''}

    \item \emph{``Arc camera shot. A young woman doing stretches on a beach.''}

    \item \emph{``A woman walking through a field of beautiful sunflowers. She spins counterclockwise and laughs. A field of shoulder-length sunflowers grow in the background, with trees on the horizon stretching up towards a cloudy sky.''}

    \item \emph{``Arc camera shot. A man playing the guitar.''}

    \item \emph{``A boy blowing out candles on a birthday cake.''}

    \item \emph{``A cheetah running in the Savannah.''}

    \item \emph{``Tracking shot. A golden retriever runs through a grassy park. The dog’s ears flop up and down with each bounding step, and its tongue hangs out to one side. A frisbee flies into view from the left, and the dog leaps into the air to catch it. A group of people in the background claps and cheers.''}

    \item \emph{``A young girl skips down a quiet suburban street lined with trees. She has light brown skin and long, wavy black hair tied back with a red ribbon. The girl wears a white t-shirt, a denim skirt, and bright yellow sneakers. Her arms swing loosely as she skips''}

    \item \emph{``A woman doing sit-ups at a gym.''}

    \item \emph{``A child riding his bicycle on a dirt path. The background is a dirt path lined with trees on either side.''}

    \item \emph{``A runner moves at full speed along a suburban sidewalk. The background is rows of houses and trees passing by in a blur.''}

    \item \emph{``A young woman engaging in a boxing workout. She is wearing red boxing gloves and a white t-shirt, and has long blonde hair. In the first frame, she is standing in front of a black punching bag, with her right arm extended and her left arm bent, ready to punch the bag. She appears focused and determined. In the second frame, she has moved to the left of the bag and is looking towards the right side of the image. She continues to punch the bag with her right arm extended and her left arm bent. In the final frame, she is still standing to the left of the bag and is looking towards the right side of the image. She is still wearing her red boxing gloves and white t-shirt, and her long blonde hair is visible. The background of  a blue wall with a window on the left and a doorway on the right, as well as two black objects hanging from the ceiling. Throughout the video, the woman is intensely focused on her workout, punching the bag with precision and skill.''}


    \item \emph{``A brown bear walks in a grassy field.''}

    
    
\end{enumerate}


%%%%%%%%%%%%%%%%%%%%%%%%%%%%%%%%%%%%%%%%%%%%%%%%%%%%%%%%%%%%%%%%%%%%%%%%%%%%%%%
%%%%%%%%%%%%

\end{document}