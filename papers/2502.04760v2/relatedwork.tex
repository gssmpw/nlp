\section{Related Work}
Given the limited cache storage capacity, optimizing the placement of contents that are most likely to be requested by users within the local cache is of paramount importance in the context of edge computing. Traditional caching strategies \cite{chen2013shortest}, \cite{chen2013shortest} often rely on static rules such as FIFO, LRU, and LFU to update cache contents. However, these approaches fall short in adapting to the dynamically shifting patterns of content popularity. Recent advancements in research have concentrated on developing dynamic cache strategies that consider the popularity dynamics of content. Broadly, these strategies can be categorized into two groups: cache algorithms that leverage a priori knowledge of content popularity distribution and those that operate without such knowledge.

In this study, we commence by providing a concise overview of pertinent research that assumes a prior understanding of content popularity. In certain scenarios, user content requests follow a Zipf distribution pattern. Taking advantage of this insight, Maddah-Ali et al. \cite{maddah2014fundamental} exploit the broadcast nature of wireless mediums through coded caching techniques to enhance cache efficiency. Similarly, \cite{gungor2015proactive},\cite{chen2018proactive},\cite{wang2022energy} addresses the goal of augmenting downlink energy efficiency for proactive content caching by assuming predictable user request patterns. Furthermore, there exists a set of content caching strategies that do not rely on prior knowledge of content popularity. Leveraging machine learning techniques within caching algorithms to estimate file popularity has gained prominence, encompassing methods such as reinforcement learning and collaborative filtering. For instance, Bastug et al. \cite{bastug2014living} present a caching algorithm designed for small cell networks based on collaborative filtering (CF). This algorithm provides content popularity estimates through a training phase utilizing sparse training data. On the other hand, the multi-armed bandit (MAB) caching algorithm learns file popularity online by observing cached content demands and subsequently updating cache contents at fixed intervals. Sengupta et al. \cite{sengupta2014learning} introduce a coded caching scheme where the base station leverages demand history to estimate file popularity, employing a combinatorial multi-armed bandit formulation. This approach seamlessly integrates popularity estimation and content placement strategies. Moreover, recognizing that diverse users contribute to content popularity, a contextual MAB algorithm is employed to learn content popularity while considering individual user characteristics. This builds upon prior work \cite{xu2020collaborative},\cite{maghsudi2020non} which amalgamates contextual information like user density and file request times.

Nonetheless, the methods outlined above are primarily tailored for centralized environments where servers aggregate all data. This centralized setup raises apprehensions regarding user privacy, as users often harbor concerns about entrusting their private data to servers. Consequently, we propose an innovative proactive content caching methodology that leverages a joint framework of collaborative stacked auto-encoders and federated learning. This dual approach not only predicts content popularity but also prioritizes user privacy, making it suitable for edge computing environments.