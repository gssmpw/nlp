
\documentclass[10pt,twocolumn,letterpaper]{article}

\usepackage[pagenumbers]{cvpr} %
\usepackage{minitoc}
\renewcommand \thepart{}
\renewcommand \partname{}

\usepackage[dvipsnames]{xcolor}


\definecolor{cvprblue}{rgb}{0.21,0.49,0.74}
\usepackage[pagebackref,breaklinks,colorlinks,citecolor=cvprblue]{hyperref}
\usepackage{tikz,lipsum}
\usepackage[most]{tcolorbox}
\usepackage[capitalize]{cleveref}
\usepackage{multirow}
\usepackage{listings}
\usepackage{xcolor}
\usepackage{tabularx}
\usepackage{makecell}


\def\confName{CVPR}
\def\confYear{2025}

\title{\vspace{-0.8cm}SwiftSketch: A Diffusion Model for Image-to-Vector Sketch Generation\vspace{-0.3cm}}



\author{\hspace{-1cm} Ellie Arar$^{1}$ \hspace{-0.8cm} 
\and Yarden Frenkel$^{1}$ \hspace{-0.8cm}
\and Daniel Cohen-Or$^{1}$ \hspace{-0.8cm} 
\and Ariel Shamir$^{2}$ \hspace{-0.8cm}
\and Yael Vinker$^{1,3}$ \hspace{-1cm}
\and \hspace{0.9\linewidth} \and
$^{1}$Tel Aviv University \\ {\tt\small \{elliearar, Yf2, dcor\}@mail.tau.ac.il}\and $^{2}$Reichman University  \\ {\tt\small arik@runi.ac.il} \and $^{3}$MIT \\ {\tt\small yaelvink@mit.edu} \vspace{-0.5cm}\\ \and {\tt\small \href{https://swiftsketch.github.io/}{https://swiftsketch.github.io/} } \vspace{-0.6cm}
}


\newcommand{\methodname}{{SwiftSketch\xspace}}


\begin{document}
\doparttoc %
\faketableofcontents %

\twocolumn[{%
\vspace{-0.65cm}
\maketitle
\renewcommand\twocolumn[1][]{#1}%
\vspace{-0.8cm}
\begin{center}
    \centering
    \includegraphics[width=0.95\linewidth]{figs/swift_teaser2.pdf}
    \captionsetup{type=figure}
    \vspace{-0.3cm}
    \caption{SwiftSketch is a diffusion model that generates vector sketches by denoising a Gaussian in stroke coordinate space (top). It generalizes effectively across diverse classes and takes under a second to produce a single high-quality sketch (bottom).}
    \label{fig:teaser}
\end{center}
}]
\begin{abstract}
\vspace{-0.4cm}
Recent advancements in large vision-language models have enabled highly expressive and diverse vector sketch generation. However, state-of-the-art methods rely on a time-consuming optimization process involving repeated feedback from a pretrained model to determine stroke placement. Consequently, despite producing impressive sketches, these methods are limited in practical applications.
In this work, we introduce \emph{SwiftSketch}, a diffusion model for image-conditioned vector sketch generation that can produce high-quality sketches in less than a second.
SwiftSketch operates by progressively denoising stroke control points sampled from a Gaussian distribution. 
Its transformer-decoder architecture is designed to effectively handle the discrete nature of vector representation and capture the inherent global dependencies between strokes.
To train SwiftSketch, we construct a \emph{synthetic} dataset of image-sketch pairs, addressing the limitations of existing sketch datasets, which are often created by non-artists and lack professional quality. For generating these synthetic sketches, we introduce ControlSketch, a method that enhances SDS-based techniques by incorporating precise spatial control through a depth-aware ControlNet.
We demonstrate that SwiftSketch generalizes across diverse concepts, efficiently producing sketches that combine high fidelity with a natural and visually appealing style.
\end{abstract}

   
\section{Introduction}
\label{sec:intro}
% Image editing methods in diffusion models depend on user-defined control directions - users can unlock their creativity using these methods by specifying the desired manipulation through prompts~\cite{gandikota2023concept}, reference images~\cite{ruiz2022dreambooth, kumari2022customdiffusion, gal2022image, chen2024trainingfreeregionalpromptingdiffusion}, or attribute vectors~\cite{parmar2023zero,hertz2022prompt}. In this work, we ask a fundamentally different question: \emph{Can we automatically discover the underlying visual structure of a concept within diffusion model's knowledge?} %Rather than requiring user-specified controls, we aim to decompose the model's internal knowledge into meaningful directions.

% This question touches on a fundamental limitation in how we interact with diffusion models. Current control methods ~\cite{zhang2023addingconditionalcontroltexttoimage, gandikota2023concept, ye2023ipadaptertextcompatibleimage,ye2023ipadaptertextcompatibleimage, hertz2024stylealignedimagegeneration, li2023photomaker, shi2024instantbooth, chen2024trainingfreeregionalpromptingdiffusion} require users to specify their desired manipulations in advance, limiting interactive creativity. This contrasts with natural human artistic workflows, where creators dynamically explore creative ideas while jointly refining them toward meaningful artistic outcomes~\cite{hoffmann2016modeling}. This synergy between specification and exploration is not new to generative models. Early GAN architectures naturally developed disentangled latent spaces that enabled continuous\cite{harkonen2020ganspace,radford2015unsupervised, wu2021stylespace, shen2020interfacegan}, compositional control over generated images. Users could explore these spaces to discover interesting variations that would be difficult to describe in words~\cite{wu2021stylespace}, then combine them to achieve their creative goals~\cite{grabe2022towards}. 


% While diffusion models have largely superseded GANs in conditional image synthesis~\cite{dhariwal2021diffusion},  their underlying structure remains less understood. Diffusion models achieve remarkable diversity through high-dimensional latents, unlike GANs' compact latent spaces.  With a single prompt, diffusion models can generate radically different variations through different random initializations of input noise. We ask - Is it possible to discover interpretable structure within this vast space of variations?

Text-to-image diffusion models are capable of generating remarkable visual variations from a single prompt through different random initializations. However, this vast creative potential remains largely opaque to users---while we can generate diverse images, we lack understanding of the underlying structure of these variations. This presents a fundamental challenge: how can we discover and expose the latent visual capabilities encoded within these models?

\let\thefootnote\relax \footnote{$^{*}$Correspondence to \texttt{gandikota.ro@northeastern.edu}}

The challenge touches on a key limitation in how we interact with diffusion models today. Current control methods require users to explicitly specify their desired edits in advance through prompts~\cite{gandikota2023concept}, reference images~\cite{zhang2023addingconditionalcontroltexttoimage, chen2024trainingfreeregionalpromptingdiffusion, ruiz2022dreambooth,kumari2022customdiffusion, Ryu_lora, hu2021lora}, or attribute vectors~\cite{ye2023ipadaptertextcompatibleimage, hertz2024stylealignedimagegeneration, li2023photomaker, shi2024instantbooth,parmar2023zero,hertz2022prompt}. That contrasts sharply with natural human creative workflows, where artists dynamically explore creative ideas and jointly refine them toward meaningful artistic outcomes~\cite{hoffmann2016modeling}. The need for pre-specified controls creates a barrier between users and the full creative potential of these models.

Interestingly, earlier generative models like GANs~\cite{gans,karras2019style,brock2018large} naturally developed more interpretable internal structures. Their compact latent spaces often exhibited emergent disentanglement~\cite{harkonen2020ganspace,radford2015unsupervised, wu2021stylespace, shen2020interfacegan}, enabling continuous and compositional control over generated images. Users could explore these spaces to discover interesting variations that would be difficult to describe in words~\cite{wu2021stylespace}, then combine them to achieve their creative goals~\cite{grabe2022towards}.

Diffusion models have largely superseded GANs in conditional image synthesis~\cite{dhariwal2021diffusion}, achieving greater diversity through much higher-dimensional latents. And yet an understanding of the underlying structure of these larger latent spaces has remained elusive. In this work, we ask a fundamental question: \emph{Can we automatically discover the visual structure within a diffusion model's knowledge of a concept?} Rather than requiring user-specified controls, we aim to decompose the model's internal representations into expressive directions that users can explore and combine.

To address these needs, we present \textbf{SliderSpace}, a framework that brings systematic explorability to diffusion models. Given just a text prompt, SliderSpace discovers a canonical set of meaningful, diverse, and controllable directions within the model's knowledge of that concept. Each direction is implemented as a low-rank adapter~\cite{hu2021lora} that can be scaled and composed with others, allowing users to explore and smoothly combine different aspects of variation, as shown in Figure~\ref{fig:intro}.

We ground SliderSpace discovery in three key requirements for meaningful decomposition of a diffusion model's visual manifold: 
\begin{enumerate}
    \item \textbf{Unsupervised Discovery:} The decomposition process should emerge from the intrinsic structure of the model's learned representation, rather than being guided by predefined attributes. This ensures we capture the true topology of the model's knowledge space rather than projecting our assumptions onto it.
    
    \item \textbf{Semantic Orthogonality:} Each discovered control must represent a distinct semantic direction. This is enforced in a semantic feature space, like CLIP, where every slider has an orthogonal effect in embeddings. This prevents discovering multiple controls that create similar semantic effects, making the system more efficient and easier.
    
    \item \textbf{Distribution Consistency:} Directions must induce consistent transformations across both random seeds and prompt variations. 
\end{enumerate}

These requirements naturally lead to our proposed framework, which we formalize in Section~\ref{sec:method}. As we show in our experiments, SliderSpace is architecture-agnostic, working with both conventional U-Net based models like Stable Diffusion~\cite{rombach2022high, rombach2022sd20, podell2023sdxl, turbo, dmd} and recent transformer-based architectures like Flux~\cite{flux}.

We demonstrate the expressiveness of SliderSpace through three applications: First, we show how SliderSpace can decompose high-level concepts into diverse and expressive components, revealing the natural axes of variation in the model's understanding. Second, we explore artistic style variation, where SliderSpace discovers directions that match or exceed the diversity of manually curated artist lists while being judged more useful by human evaluators. Finally, we show how SliderSpace can help reverse the mode collapse commonly observed in distilled diffusion models, restoring diversity while maintaining generation speed.

Beyond providing practical creative control, SliderSpace opens new avenues for understanding and utilizing the latent capabilities of diffusion models. By mapping these models' visual potential into intuitive, composable directions, we take a step toward making their creative possibilities more accessible and interpretable to users.

% Image editing methods in diffusion models unlock the creativity of users. In this work we ask an alternate question: \emph{Can we organize and expose what of the diffusion model is already capable of?}.
% Existing methods for controlling image generation typically require users to manually specify edit directions for desired changes. This process is time-consuming, requires technical expertise, and limits the spontaneity of the creative process. For instance, if a user wants to adjust the smile of a generated person, they must explicitly request this edit, often through imprecise prompt engineering or model fine-tuning. This approach of predefined controls or manual specifications restricts users from fully exploring the latent capabilities of the model. There may be interesting stylistic variations or attributes that the model can generate, but users have no easy way to discover or utilize these.

% Natural visual disentanglement was an emergent property in the latent space of Generative Adversarial Models (GANs) \cite{harkonen2020ganspace,radford2015unsupervised, wu2021stylespace, shen2020interfacegan}. In particular, it has been observed that StyleGAN~\cite{karras2019style} stylespace neurons offer detailed control over many meaningful aspects of images that would be difficult to describe in words~\cite{wu2021stylespace}. However, diffusion models do not share such a compact latent space~\cite{park2023unsupervised}; and efforts to uncover such a space in the semantic embeddings of the text conditioning have met with limited success \nik{Nick - is there a specific citation you were thinking about?}.

% In this work we introduce \textbf{SliderSpace}, which takes a step towards uncovering an analogous low dimensional representation of diffusion models' visual breadth; in essence treating the diffusion model as many generators sharing parameters, where a particular generator is defined by a specific prompt. For a given prompt we sample many random seeds (and optionally prompt expansions using an LLM), generate the corresponding images, and apply an off the shelf feature extractor (in this work CLIP, but our method can be applied to any differentiable feature extractor). We use PCA to analyze these features, and for each of the leading $k$ principal components we train a LoRA \cite{} which causes the diffusion model to produces images which increase the feature magnitude along that component when passed back through the same feature extractor. This leads to a 'Slider' for each principal component, because each LoRA can be scaled and applied to the original diffusion model, continuously varying those visual features in the generated results (as measured, in our case, by CLIP).

% There are many other works that enhance the controllability of diffusion models. One common approach is enabling users to add spatial constraints to a generation either manually, or via a reference image \cite{zhang2023addingconditionalcontroltexttoimage, chen2024trainingfreeregionalpromptingdiffusion}, a second is leveraging more abstract embeddings (e.g. identity, style) extracted from a reference image \cite{ye2023ipadaptertextcompatibleimage, hertz2024stylealignedimagegeneration, li2023photomaker, shi2024instantbooth}, a third is finetuning a foundation model to better generate a concept important to the user \cite{ruiz2022dreambooth, kumari2022customdiffusion, Ryu_lora, hu2021lora}, and a fourth (most relevant to this work) is finding low-rank adaptors of the model based on a prompt or small training set which can be scaled to provide continous control over one aspect of generated image (e.g. night vs day, basic vs luxury, etc.) \cite{gandikota2023concept}. SliderSpace is complementary to all of these methods and offers something distinct. All of the other methods we are aware require the user (and / or model designer) to know in advance what type of control they want. In contrast SliderSpace assists users in discovering and controlling hidden capabilities present in the diffusion model's distribution of possible generations.

%We propose that truly intuitive creative control in a text-to-image model should meet three key criteria: \emph{discoverability}, \emph{intuitiveness}, and \emph{specificity}. The model should reveal controllable attributes that may not be immediately obvious, offer controls that are easy to understand and manipulate, and ensure each control affects a distinct attribute of the generated image.

% We demonstrate the utility and power of SliderSpace using three applications built on top of SDXL-DMD \cite{dmd}, because its fast generation speed lends itself well to the continuous control offered by SliderSpace.

% First, we study concept decomposition (Section \ref{sec:concept_exp}), where we learn sliders for a specific concept (e.g. 'monster', 'waterfall', 'car'). Through quantitative metrics of diversity and text alignment we demonstrate that the learned sliders dramatically boost the diversity of generations when randomly applied without harming text alignment; we also ask humans to qualitatively judge these results in a user study where they find the SliderSpace results to be more 'Diverse', 'Useful', and 'Creative' than our baselines.

% Second, we attempt to compare the automatic discoveries of SliderSpace to a large scale manual study of artistic styles (Section \ref{sec:art_exp}), open-sourced by ParrotZone \cite{parrotzone}. In this study SDXL was prompted with over 4300 artist names,  and based on visual inspection the cases of successful stylistic mimicry recorded. Quantitatively SliderSpace more closely matches the distribution of artistic variation discovered by ParrotZone than other baselines, and in our user studies was judged to be significantly more 'Diverse' and 'Useful' than the baselines. To our surprise humans even judged SliderSpace results to be slightly more 'Diverse' than the results generated by the manually discovered artist names of \cite{parrotzone}.

% Third, we attempt to use SliderSpace to reverse the mode collapse commonly observed in distilled few-step diffusion models relative to the original teacher model (Section \ref{sec:diverse_exp}). We quantitatively demonstrate that applying SliderSpace to SDXL-DMD leads to more closely matching the distribution of images by the original teacher, SDXL.

%Through extensive experiments on various state-of-the-art text-to-image models, we demonstrate that SliderSpace significantly enhances user control and creative expression in AI-assisted image generation tasks. Our method enables a range of applications, including concept decomposition and control, diversity improvement in generated images, customization dissection and edits, and the exploration of artistic styles inherent in the model.

% SliderSpace goes beyond providing a practical tool for enhanced creative control. By mapping the visual potential of diffusion models it can open new avenues for generative creativity and deepens our understanding of each model's hidden potential.
\begin{figure*}
  \centering
  \includegraphics[width=1\linewidth]{figs/seg_pipe1.pdf}
  \vspace{-0.7cm}
  \caption{Overview of the sketch segmentation pipeline. Given an input sketch image, our framework first detects bounding boxes using a customized Grounding DINO to obtain region proposals, and then perform segmentation with SAM models. The localization and segmentation are refined by incorporating the depth features. The result segmentation can be viewed as a layered decomposition of object components in the original sketch.}  
  \label{fig:methods}
\end{figure*}

\section{Related Work}
\subsection{Part-Level Sketch Segmentation} 
The majority of work in the sketch segmentation domain focuses on part-level semantic segmentation, in which the goal is to assign labels to object parts (\eg, the body, wings, and head of a bird). These methods often rely on curated part-level sketch segmentation datasets
% The majority of work in the sketch segmentation domain focuses on part-level semantic segmentation. These methods label object parts (\eg, identify the body, wings, and head of a bird, \etc). They often rely on curated part-level sketch segmentation datasets
\cite{ge2020creative,li2018universalperceptualgrouping,Wu2018SketchsegnetAR,eitz2012hdhso,dataDrivenSeg2014} to train a segmentation model, and use various network architectures, including CNNs~\cite{Zhu2018PartLevelSS, wang2020multicolumnseg}, RNNs~\cite{sketchSegNet2019, Wu2018SketchsegnetAR, symbolReconSeg2020}, Graph Neural Networks~\cite{ENDE-GNN2022, sketchGNN2021}, Transformers~\cite{wang2024contextseg, segFormer2023}, and more specific techniques such as deformation networks~\cite{oneshot2021} and CRFs~\cite{CRFseg2016}. These approaches typically operate on a fixed set of object classes, and recognize a predefined set of object parts within them. Other work focuses on perceptual grouping~\cite{ li2018universalperceptualgrouping, Li2019TowardDU} to achieve class-agnostic segmentation. However all of these methods are designed to tackle part-level segmentation and are not suitable for scene sketches.

% which computes stroke affinity matrices in vector sketches using either hand-crafted features or implicit modeling of Gestalt principles to achieve class-agnostic grouping. 
% \vspace{-0.2cm}
\subsection{Scene-Level Sketch Segmentation}
Scene-level sketch segmentation remains largely under-explored. Qi \etal~\shortcite{edgePerceptual2015} extend the perceptual grouping approach to scene-level images, forming semantically meaningful groupings of edges, though with limited accuracy on complicated scenes.
Zou \etal\shortcite{Zou18SketchyScene} construct the SketchyScene dataset, providing annotated scene sketches with meaningful layouts of object interactions, and use it to train an instance segmentation model based on the Mask R-CNN architecture~\cite{he2018maskrcnn}. However, their method is limited to the predefined categories included in the dataset, and the proposed dataset contains sketches with clipart-like appearance which challenges the model's ability to generalize to other artistic styles. 
Building on SketchyScene, Ge \etal \shortcite{GeLocalDetailPerception} introduced SKY-Scene and TUB-Scene by replacing its object components with sketches from the Sketchy \shortcite{sketchydataset} and TU-Berlin \shortcite{eitz2012hdhso} datasets. However, their proposed fusion network is fundamentally limited to the fixed set of classes it was trained on, and the trained network weights are not publicly available.
% , as its architecture is based on DeepLabV2, a closed-vocabulary segmentation model, which restricts its ability to generalize to unseen categories.
SFSD \cite{zhang2023strokeSeg} develops a dataset featuring more complex scene sketches, and utilizes a bidirectional LSTM to produce stroke-level segmentation. Unfortunately, the dataset and model are not publicly available.
SketchSeger \cite{yang2023sceneHierTransformer} proposes a hierarchical Transformer-based model for semantic sketch segmentation. However their model is inherently restricted to the predefined set of classes used during training. Bourouis \etal \shortcite{bourouis2024open} finetune the CLIP image encoder \cite{radford2021clip} on the FS-COCO \cite{fscoco} dataset, leveraging the model's vision-language prior to enable open-vocabulary scene segmentation. However their method is designed for semantic segmentation, and it struggles to generalize to more challenging sketch styles and scene layouts.
% \yael{decide if we keep:}Kutuk et al's method ~\shortcite{Kutuk2024ClassAgnosticVS} targets vector scene sketches, leveraging a class-agnostic object detector with temporal stroke information and a pre-trained sketch classifier. They demonstrate the ability to distinguish object instances; however, their visual examples are confined to scenes with minimal or no complex overlaps between object instances, which fail to closely resemble real-world sketches. 
% The most recent scene sketch dataset, FrISS, introduced by Kutuk \etal \cite{Kutuk2024ClassAgnosticVS}, was curated by recruiting participants to draw quick sketches based on text descriptions from the MS COCO dataset. While this dataset is not publicly available, the examples shown in their paper reveal predominantly simple sketches with minimal detail.

% Other scene sketches datasets exists
% \yael{decide if to keep:}

% \yael{this one we keep and use the data for comparison, but the paper is not about segmentation}
% CBSC \cite{zhang2018CBSC}, the first scene sketch dataset to include indoor scenes, is a small-scale dataset of 332 human-drawn sketches characterized by quick, freehand designs.
% While these extensions emphasize simplified, novice-style object representations, they only expand the style variation in a single direction—towards symbolic sketches.
% \vspace{-0.2cm}
\subsection{Image Segmentation}
The task of image segmentation have been widely explored~\cite{he2018maskrcnn,  Bolya_2019_yolact, cheng2021mask2former, wang2021solo}. 
The advent of vision-language models \cite{radford2021clip, liu2023llava,xiao2023florence} has led to numerous object detection and segmentation methods with
impressive generalization capabilities \cite{zhang2022glipv2, ren2024grounded, kirillov2023segany, minderer2022owlvit}.
% Grounded SAM \cite{ren2024grounded} is among the leading approaches in this domain. It combines two state-of-the-art models, Grounding DINO \cite{liu2023grounding} and Segment Anything (SAM) \cite{kirillov2023segany}, for open-vocabulary image segmentation, achieving robust performance across diverse object categories. 
Grounding DINO \cite{liu2023grounding} is a state-of-the-art object detection model trained on over 10 million images. It builds on top of DINO \cite{DINOcaron2021emerging}, a strong vision encoder, with effective grounding module that fuses visual and textual information, enabling open-vocabulary detection of unseen objects. Segment Anything (SAM) \cite{kirillov2023segany} is an image segmentation model trained on over 11 million images and 1.1 billion masks, capable of producing high-quality object masks based on various forms of conditioning such as bounding boxes. Grounded SAM \cite{ren2024grounded}, which our method builds upon, combines Grounding DINO and SAM for open-vocabulary image segmentation, achieving robust performance across diverse object categories. 
Yet, despite demonstrating impressive capabilities on natural images, we show that these models struggle with segmenting sketches.  

% is among the leading approaches in this domain.



\section{Methodology}
\paragraph{Preliminaries.}
We primarily focus on the homologous model merging, in which $\boldsymbol{\theta}_i$ all come from the same base model $\boldsymbol{\theta}_{\rm{base}}$. Given $K$ tasks $\{T_1,T_2,\cdots,T_K\}$ and $K$ corresponding fine-tuned models with parameters $\{\boldsymbol{\theta}_1,\boldsymbol{\theta}_2,\cdots,\boldsymbol{\theta}_K\}$, model merging aims to combine $K$ fine-tuned models into one single model simultaneously performing on $\{T_1,T_2,\cdots,T_K\}$ without post-training~\cite{method_p1_1,method_p1_2}.
Task vector~\cite{ilharco2023editing,yang2024adamerging} is a key element in merging method which could enhances the base model‘s ability or enable the model to handle other tasks. Specifically, for task $T_i$, the task vector $\boldsymbol\tau_i\in \mathbb{R}^D$ is defined as the vector obtained by subtracting the SFT weights $\boldsymbol{\theta}_i$ from the base model weight
$\boldsymbol{\theta}_{\rm{base}}$, \emph{i.e.}, $\boldsymbol\tau_i=\boldsymbol{\theta}_i-\boldsymbol{\theta}_{\rm{base}}$. The merged model could be denoted as $\boldsymbol{\theta}_m=\boldsymbol{\theta}_{\rm{base}}+\sum_i \lambda_i\boldsymbol{\tau}_i$, which $\lambda_i$ is the scaling factor measuring the importance of task vector. For clarification, we also denote the neuron set in $\boldsymbol{\theta}_i$ as $\mathcal{N}_i$, the neuron set in $\boldsymbol{\tau}_i$ as $\mathcal{T}_i$.



\begin{algorithm}[!ht]
    \caption{LED-Merging}
    \label{alg1}
    \begin{algorithmic}[1]
        \REQUIRE  base model $\boldsymbol{\theta}_{\rm{base}}$, SFT models $\{\boldsymbol{\theta}_{i}\mid i\in [K]\}$, mask ratios \{$r_{i} \mid i\in [K]\}$, scaling factors $\{\lambda_i\mid i\in[K]\}$, location datasets $\{\mathcal{X}_{i}\mid i\in[K]\}$
        \ENSURE merged parameter $\boldsymbol{\theta}_{m}$
        \STATE $\mathcal{M}\leftarrow\phi$
        \STATE $\boldsymbol{\theta}_{m}\leftarrow \boldsymbol{\theta}_{\rm{base}}$
        \FOR{$i\in [K]$}
        \STATE $I(\boldsymbol{\theta}_i)=\mathbb{E}_{x\sim \mathcal{X}_i}|\boldsymbol{\theta}_{i}\odot \nabla_{\boldsymbol{\theta}_i}\mathcal{L}(x)|$
        \STATE $I(\boldsymbol{\theta}_{\rm{base}})=\mathbb{E}_{x\sim \mathcal{X}_i}|\boldsymbol{\theta}_{\rm{base}}\odot \nabla_{\boldsymbol{\theta}_{\rm{base}}}\mathcal{L}(x)|$
        
        \STATE calculate $\mathcal{T}^{r_i}_{i}$ following Equation \ref{vote}
        \STATE  $\mathcal{M}\leftarrow \mathcal{M}\cup\{\mathcal{T}^{r_i}_i\}$
       
        
   
        
        
        \ENDFOR  
        \FOR{$i\in [K]$}
        
        \STATE calculate $\text{Disjoint}(\mathcal{T}_i^{r_i})$ use Equation~\ref{disjoint_safety}
        \STATE $\boldsymbol{m}_i \leftarrow \boldsymbol{0}$
        \FOR{$d\in \mathcal{T}_i^{r_i}$}
        \STATE $\boldsymbol{m}_{i,d}=1$
        \ENDFOR
        \STATE $\boldsymbol{\theta}_{m}\leftarrow \boldsymbol{\theta}_{m}+\lambda_i \boldsymbol{\tau}_i\odot \boldsymbol{m}_{i}$
        \ENDFOR
    \end{algorithmic}
\end{algorithm}
    %\vspace{-5pt}
\begin{figure*}[h!]
    \centering
    \includegraphics[width=\linewidth]{figs/pipeline_v2.pdf}
    \vspace{-40mm}
    \caption{Overview of our two-stage training pipeline {\ours}.}
    \label{fig:pipeline}
\end{figure*}


\paragraph{LED-Merging: Location, Election, and Disjoint Merging}
To address the neuron misidentification and interference issues in existing model merging methods, we propose LED-Merging (Location, Election, and Disjoint Merging). Specifically, previous studies \cite{modelstock, ilharco2023editing, tiesmerging} fail to accurately identify safety-related neurons in task vectors with a single magnitude score, namely \textit{neuron misidentification}. Meanwhile, there exists an interference between safety-related and utility-related task vector neurons during the merging process, namely \textit{neuron interference}. To address neuron misidentification, we first locate important neurons both in the base and fine-tuned models and then elect neurons from the task vector considering these two scores together. Subsequently, to mitigate the interference, we introduce a disjoint step, isolating these important neurons so that they influence different base neurons. The whole process is illustrated in Figure~\ref{fig:method}. 




In the location and election step, we consider the importance score from base and fine-tuned models simultaneously to locate task-specific neurons. In this way, it is more accurate than relying on the magnitude score alone because task-specific neurons with high importance score in the fine-tuned model may not necessarily score high in the base model, and vice versa.

{\textbf{Location}}.  We first calculate importance scores for each neuron in a base/fine-tuned model. Given a location dataset $\mathcal{X}_i=\{(x,y)_k\}$, where $x$ is the question and $y$ is the answer, we calculate the importance scores for the weight $\boldsymbol{\theta}_i\in\mathbb{R}^D$ in any  layer as follows~\cite{snip,spareseGPT,sun2024a}:
\begin{equation}
    I(\boldsymbol{\theta}_i)=\mathbb{E}_{x\sim \mathcal{X}_i}[\boldsymbol{\theta}_i\odot \nabla _{\boldsymbol{\theta}_i}\mathcal{L}(x)],
    \label{location}
\end{equation}
which $\mathcal{L}(x)=-\log p(y\mid x)$ is the conditional negative log-likelihood loss. We choose the SNIP score~\cite{snip} because it balances computational efficiency and performance~\cite{cq}. Please refer to Sec.~\ref{sec:ablation} for the comparison between different location methods. After computing importance scores, we choose top-$r_i$ neurons as the important neuron subset $\mathcal{N}_{i}^{r_i}$ from $I(\boldsymbol{\theta}_i)$.
 
 % After computing locating scores, we select the neurons scoring both high in base and fine-tuned models as important neurons in task vectors. Then in the disjoint step,  with preventing  polysemantic neurons  from receiving gradient updates towards different directions,
 % we use set difference to isolate the safety   and utility-related neurons  and construct corresponding masks for merging process,

{\textbf{Election}}. A natural question is how to select important neurons in the task vector $\boldsymbol{\tau}_i$ based on $I(\boldsymbol{\theta}_{\rm{base}})$ and $I(\boldsymbol{\theta}_{i})$. The important neurons in the base model may be different from neurons in the fine-tuned model. Therefore, we introduce the following election strategy to select neurons with high scores in both base and fine-tuned models:
\begin{equation}
    \mathcal{T}_i^{r_i}=\mathcal{N}_i^{r_i}\cap \mathcal{N}_{\rm{base}}^{r_i}.
    \label{vote}
\end{equation}
\emph{Remark}. We compare different choosing methods, including scoring low or high in base or fine-tuned model in Section~\ref{sec:ablation} and find that Equation \ref{vote} achieves the best performance.





{\textbf{Disjoint}}. As important neurons from different task vectors may conflict with each other at the same position, we use the set difference to disjoint the neurons from others to prevent interference:
\begin{equation}
    \text{Disjoint}(\mathcal{T}^{r_i}_{i})=\mathcal{T}^{r_i}_{i}-\mathop{\cup}\limits_{{J}\subsetneqq [K],|J|\geq 2}\mathop{\cap}\limits_{j\in {J}}\mathcal{T}^{r_j}_{j}.
    \label{disjoint_safety}
\end{equation}

Next, we construct a mask $\boldsymbol{m}_i\in\mathbb{R}^D$ to implement disjoint in the merging process. Specifically, this mask $\boldsymbol{m}_i$ is used to select neurons from $\mathcal{T}_i$. The mask ratio is $r_i$, where $r\in(0,1]$. The mask $\boldsymbol{m}_i$ can be derived from:
\begin{equation}
    \boldsymbol{m}_{i,d}=\begin{aligned} &\left\{ \begin{array}{ll} 1, & \text{if } d\in \text{Disjoint}(\mathcal{T}_{i}^{r_i}), \\ 0, & \text{otherwise}. \end{array} \right. \end{aligned}
    \label{mask_safety}
\end{equation}


% \subsection{Merging Models with Masks}
{\textbf{Merging}}. The final
merged task vector $\boldsymbol{\tau}_m$ is as follows:
\begin{equation}
    \boldsymbol{\tau}_m= \sum_i \lambda_i\boldsymbol{\tau}_{i}\odot\boldsymbol{m}_i.
    \label{merged_task_vector}
\end{equation}
We summarize the workflow in Algorithm \ref{alg1}.



\section{Results}
\label{sec:results}
Following \nksr, we evaluate our method using metrics including the standard Chamfer-$L_1$ Distance~(CD-$L_1 \times 10^{-2}$, $\downarrow$) and F-score~($\uparrow$) with a threshold~($\delta{=}0.010$). 
We also report additional metrics proposed in \nksr~including Chamfer-$L_1$ Distance by Completeness (Comp.~$\times 10^{-2}$, $\downarrow$) and Accuracy (Acc.~$\times 10^{-2}$, $\downarrow$) in the \texttt{Supplementary Material}. 
We evaluate our method on multiple datasets, under two settings including in-domain evaluation for accuracy estimation -- training set and test set are from same dataset, and cross-domain evaluation for generalization ability estimation where training set and test set are from different datasets. 
Additionally, for cross-domain evaluation we use the following datasets prepared by the leading voxel-based baseline, \nksr, and one additional dataset from RangeUDF~\cite{wang2022rangeudf}:

\begin{itemize}
    \item \synthetic{}  is a synthetic dataset created from ShapeNet objects~\cite{chang2015shapenet}. Each scene contains 2-3 objects. 
    Following prior works~\cite{wang2022rangeudf,chibane2020ndf}, we re-scale the synthetic rooms to roughly match real-world scale.
    There are 3750 scenes as training set and \ws{995 scenes} as the test set. 
    \item \scannet{} is a real-world indoor scene dataset. We use the setting from previous work~\cite{wang2022rangeudf, tang2021SACon, peng2020convoccnet, boulch2022poco} where we train on 1201 rooms and test on 312 rooms. 
    \item \carla is a large-scale outdoor driving scene prepared by NKSR~\cite{huang2023neural} using the CARLA simulator~\cite{dosovitskiy2017carla}. 
    \ws{Following NSKR~\cite{huang2023neural}, we test on two subsets including the 'Original' subset (10 random drives simulated on 3 towns) and the 'Novel' subset (3 drives from an additional town only for testing).}
    To avoid exploding GPU memory during training, we follow NKSR~\cite{huang2023neural} to divide a large scene into patches. The resultant training set has {3757} patches. 
    \item \scenenn{}  is a real-world indoor dataset prepared by RangeUDF~\cite{wang2022rangeudf} which we used for cross-domain evaluation. We only use its test set which consists of 20 scenes.
\end{itemize}



\begin{table*}
\centering
\resizebox{\linewidth}{!}{
\setlength{\tabcolsep}{3pt}
\begin{tabular}{LccccccccccccC}
\toprule
Methods & & \multicolumn{3}{c}{\ws{{\bf \synthetic}}}  &  \multicolumn{3}{c}{{\bf \scannet}} & \multicolumn{3}{c}{\ws{{\bf \carla(Original)}}} & \multicolumn{3}{c}{\ws{{\bf \carla(Novel)}}} \\ 
 \cmidrule(lr){3-5} \cmidrule(lr){6-8} \cmidrule(lr){9-11} \cmidrule(lr){12-14} 
&Primitive& CD ($10^{-2}$) $\downarrow$ & F-Score  $\uparrow$ & Latency (s) $\downarrow$  & CD ($10^{-2}$) $\downarrow$ & F-Score  $\uparrow$ & Latency (s) $\downarrow$  & CD (cm) $\downarrow$ & F-Score  $\uparrow$ & Latency (s) $\downarrow$ & CD (cm) $\downarrow$ & F-Score  $\uparrow$ & Latency (s) $\downarrow$ \\        
\midrule
SA-CONet~\cite{tang2021SACon} & Voxels & {0.496} & {93.60} & - & - & - & - & - & - & - & - & - & -\\
ConvOcc~\cite{peng2020convoccnet} & Voxels & {0.420} & {96.40} & - & - & - & - & - & - & - & - & - & -\\
NDF~\cite{chibane2020ndf} & Voxels & {0.408} & {95.20} & - & 0.385  & 96.40  & -  & - & - & - & - & - & -\\
RangeUDF~\cite{wang2022rangeudf} & Voxels & {0.348} & {97.80} & {-} & 0.286 & 98.80 & - & - & - & - & - & - & -\\
\ws{TSDF-Fusion~\cite{zeng20163dmatch}} & -  & - & - & - & - & - & - & 8.1 & 80.2 & - & 7.6 & 80.7 & - \\
\ws{POCO~\cite{boulch2022poco}} & - & - & - & - & - & - & - & 7.0 & 90.1 & - & 12.0 & 92.4 & - \\
\ws{SPSR~\cite{kazhdan2013screened}} & - & - & - & - & - & - & - & 13.3 & 86.5 & - & 11.3 & 88.3 & - \\
\nksr & Voxels &  \underline{0.346} &  \underline{97.41} & \underline{0.40} & \underline{0.246} & \underline{99.51} & \underline{1.54} &  \underline{3.9} &  \underline{93.9} &  \underline{2.0} &  \underline{2.9} &  \underline{96.0} &  \underline{1.8} \\
\nksr (more data) & Voxels & - & - & - & - & - & - & {3.6} & {94.0} & {2.0} & {3.0} & {96.0} & {1.8}\\
Ours~(Minkowski)~\cite{choy20194d} \scriptsize{(w/ KNN)} & Voxels & - & \todo{} & \todo{} & 0.254 & 99.41 & 0.46 & 3.4 & 97.2 &1.9 & 2.7 & 98.1 & 2.0 \\
Ours~(Minkowski)~\cite{choy20194d} & Voxels & - & \todo{} & \todo{} & 0.301 & 98.48 & 0.31 & 3.8 & 96.2 & 1.5 & 3.0 & 97.4 & 1.5\\
\rowcolor{1st} Ours \scriptsize{(w/ KNN)} & Points &{0.321} & {98.34} & {0.13} & {0.243} & {99.61} & {0.48} &{3.2} & {97.5} & {3.2} &{2.6} & {98.3} & {3.4}\\
\rowcolor{1st}Ours & Points & {0.360} & {96.32} & 0.14 & 0.257 & 99.33 & 0.49 & {3.3} & {97.4} & 1.7 & {2.7} & {98.2} & 1.7 \\

\bottomrule
\end{tabular}
}
\caption{\textbf{In-domain evaluation} -- We show that our method achieves the best accuracy (CD and F-score) with significantly improved time efficiency~(inference latency).
Note we retrain \nksr (numbers are underlined) for fairer comparison, \ws{as the training data for \nksr is different from ours -- i.e., they reported some models trained on a ``mix'' of datasets, which is impossible to reproduce.
}
}
\label{tab:indomain}
\end{table*}


\paragraph{Evaluation pipeline}
To evaluate our method, we first extract the mesh with Dual Marching Cubes~\cite{schaefer2004dual} on the predicted SDF, and then compute the CD and F-score between 100k points sampled on the mesh, and 100k points sampled from the ground-truth dense point cloud.
We use the same approach as \nksr to prepare the input point clouds for training and evaluation from the ground-truth dense point clouds through downsampling.
Specifically, for indoor datasets (i.e., \synthetic, 
\scannet and \scenenn), we uniformly sample 10K points sampled from the ground truth dense point cloud. 
For outdoor driving scenes~(i.e., \carla), we follow the evaluation pipeline from \nksr.
We sample sparse input point clouds with a sparse 32-beam LiDAR with a ray distance noise of 0-5 cm and pose noise of $0-3^\circ$, and obtain the ground truth from a noise-free dense 256-beam LiDAR.

\begin{figure*}
\centering
\includegraphics[width=\linewidth]{visualizations/test_set_results.pdf}
\caption{
{\textbf{Qualitative results on \carla and \synthetic}} -- our method achieves high quality surface reconstructions which preserve more details than \nksr~which loses information due to quantization for large and non-uniformly sampled datasets like Carla.
}
\label{fig:qual_results_carla_syn}
\end{figure*}
 
\begin{figure*}
\centering
\vspace{-1em}
\includegraphics[width=.95\linewidth]{visualizations/scannet_results_0.pdf}
\caption{
Qualitative results on \scannet: We compare our method with prior SOTA~\cite{huang2023neural} and Ours~(Minkowski)~\cite{choy20194d} that is more comparable as it only differs from ours in the backbone. Our method achieves reconstruction of similar quality to the SOTA. It also \textit{significantly} outperforms Ours~(Minkowski), highlighting the importance of point-based methods. 
% \TODO{callouts too small? almost no zoom? why?}
}
\vspace{-1em}
\label{fig:scannet_results}
\end{figure*}
  

\paragraph{Implementation details}
We base our feature backbone on PointTransformerV3~\cite{wu2024point} with 4-levels.
The PointNet-style network is a 2-layered residual connection MLP, with hidden dimension of $32$ and output feature dimension of $32$.    
The grid size used in neighborhood function is $0.01$ meters.
Following \nksr, we use the similar coefficients for loss terms -- i.e., $\lambda_{\text{SDF}}$ is $300$ and $\lambda_{\text{mask}}$ is $150$.
However, we empirically set $\lambda_{\text{Eikonal}}$ to $10$~(\nksr does not need this regularizer thanks to its specialized surface solver).
We train our model with a batch size of $4$ on either a single \texttt{NVIDIA RTX A6000 ADA} or an \texttt{NVIDIA L40S}, and a learning rate of $10^{-3}$.
We adopt the Adam optimizer with default parameters.
We set the maximum number of epochs to 200 and employ a cosine learning rate decay starting from epoch 120.


\begin{table*}
\centering
\resizebox{\linewidth}{!}{
\setlength{\tabcolsep}{2pt}
\begin{tabular}{LccccccccccC}
\toprule
Methods & & \multicolumn{3}{c}{{\bf \synthetic $\rightarrow$ \scannet}}  &  \multicolumn{3}{c}{{{\bf \scannet $\rightarrow$ \synthetic}}} & \multicolumn{3}{c}{{{\bf \scannet $\rightarrow$ \scenenn}}} \\ 
 \cmidrule(lr){3-5} \cmidrule(lr){6-8} \cmidrule(lr){9-11}
&Primitive& CD ($10^{-2}$) $\downarrow$ & F-Score  $\uparrow$ & {Latency (s) $\downarrow$ } & CD ($10^{-2}$) $\downarrow$ & F-Score  $\uparrow$ & {Latency (s) $\downarrow$ } & CD ($10^{-2}$) $\downarrow$ & F-Score  $\uparrow$ & {Latency (s) }$\downarrow$ \\       
\midrule
SA-CONet~\cite{tang2021SACon} & Voxels & 0.845 & 77.80 & - & - & - & - & - & - & - \\
ConvOcc~\cite{peng2020convoccnet} & Voxels & 0.776 & 83.30  & - & - & - & - & - & - & - \\
NDF~\cite{chibane2020ndf} & Voxels & 0.452 & 96.00 & - & {0.568} & {88.10} & - & 0.425 & 94.80 & - \\
RangeUDF~\cite{wang2022rangeudf} & Voxels & {0.303} & {98.60} & {-} & 0.481& 91.50 & - & 0.324 & 97.80 & - \\
\nksr & Voxels & {0.329} & {97.37} & {2.02} & {0.351} & {97.41} & {0.46} & {0.268} & {99.18} & {1.95} \\
\rowcolor{1st} Ours (w/ KNN) & Points & {0.284} & {98.65} & {0.54} & {0.327} &{98.37} & {0.13} & {0.277} & {99.00} & {0.50} \\
\bottomrule
\end{tabular}
}
\caption{\textbf{Cross-domain evaluation} -- we achieve the best generalization ability in two cases with much better time efficiency. In the other case where we generalize from \scannet to \scenenn, we achieve accuracy on par with the SOTA baseline~\cite{huang2023neural} with less than a half of their latency.  
}
\vspace{-1.4em}
\label{tab:across_domain}
\end{table*}


\paragraph{Reconstruction latency}
For both our models and NKSR, we record the reconstruction latency for all indoor scenes on a single \texttt{NVIDIA RTX 3090}, and for large outdoor scenes on a single \texttt{NVIDIA L40s} given that more GPU memory is required.
We omit data loading time, and only record the average forward pass time. 

\subsection{In-domain evaluation}
We compare against \nksr~(the current state-of-the-art), RangeUDF~\cite{wang2022rangeudf},  SPSR~\cite{kazhdan2013screened}, NDF~\cite{chibane2020ndf}, ConvOcc~\cite{peng2020convoccnet} and SA-CONet~\cite{tang2021SACon}.     
We further include a baseline that replaces our backbone with MinkowskiNet~\cite{choy20194d} (i.e., Ours~(Minkowski)) to show the degraded performance due to the information loss caused by voxelization.

\paragraph{Quantitative results -- \Cref{tab:indomain}}
Across indoor and outdoor datasets, our method outperforms baselines in terms of accuracy and time efficiency. Especially in outdoor datasets, our method achieves the best surface reconstruction with the smallest latency -- nearly \textit{half} of the second best's latency.
In indoor datasets, which have relatively uniform sampling patterns, we achieve accuracy on par with the previous state-of-the-art, but with significantly improved time efficiency.
Note that we achieve this advantage even with KNN because, in smaller indoor point clouds, the highly engineered KNN implementation has similar time efficiency to that of our neighborhood function.
We further detail our analysis on this matter in the \texttt{Supplementary Material}. 
We also note that our approximate neighborhood function is still effective, as it outperforms the directly comparable baseline MinkowskiNet~\cite{choy20194d}, which shares the same structure except for the backbone and neighborhood function.


\paragraph{Qualitative results -- \Cref{fig:qual_results_carla_syn,fig:scannet_results}}
We show that our method tends to reconstruct surfaces of the best quality among the compared methods.
Especially, on the non-uniform large scale \carla, our method tends to preserve more details than the previous state-of-the-art~\cite{huang2023neural}, which voxelizes the point cloud.   

\subsection{Cross-domain evaluation -- \Cref{tab:across_domain}}
We further test the generalization ability of our method with a cross-domain evaluation.
We evaluate models trained with dataset A on other a different dataset B; we denote this as~A $\rightarrow$ B. 
As shown in \Cref{tab:across_domain}, there are three cases in total.
In two cases (i.e., \synthetic $\rightarrow$ \scannet and \scannet $\rightarrow$ \synthetic), our method achieves the best accuracy with the best time efficiency. 
In another case (\scannet $\rightarrow$ \scenenn), we achieve accuracy on par with SOTA~\cite{huang2023neural} with a much better time efficiency, i.e., less than a half of the latency required by the SOTA~\cite{huang2023neural}.

\subsection{Ablation studies}
Our ablations are executed on \scannet, as it is a real-world dataset, and is equipped with precise ground truth surface meshes.

\begin{table}
\centering
\resizebox{.9\columnwidth}{!}{
\begin{tabular}{LccccccC}
\toprule
{\bf Neighbor Num.} & {CD (10\textsuperscript{-2})} $\downarrow$ & {F-score} $\uparrow$ & Latency (s) $\downarrow$ \\ \midrule
 2 & 0.246 & 99.56 & 109 \\
 4 & 0.244 & 99.59 & 127 \\
 \rowcolor{1st} 
8 & {0.243} & 99.61 & 151 \\
16 & 0.256 & 99.28 & 187 \\
\bottomrule
\end{tabular}
}
\caption{{\bf The impact of neighborhood size} -- larger neighborhoods lead to increased computational cost, and we find that 8 neighbors gives the best balance of cost and quality.}
\label{tab:numpts_neighbor}
\vspace{-1em}
\end{table}

\paragraph{Impact of neighborhood size -- \Cref{tab:numpts_neighbor}}
We analyze the impact of neighborhood size on performance. Larger neighborhood size leads to increased computation overhead. 
We show that the 8-nearest neighboring points gives the best trade-off between accuracy and time efficiency.
Considering a large number (e.g., 16) of neighboring points degrades performance as the the aggregation module has limited capacity to predict the precise SDF from a large local point cloud.

\begin{table}
\centering
\resizebox{.95\columnwidth}{!}{
\begin{tabular}{@{}lcccccc@{}}
\toprule
\makecell{\bf Num. of hidden\\\bf layers in $\aggregation$} & CD (10\textsuperscript{-2}) $\downarrow$ & F-score $\uparrow$ & Latency (s) $\downarrow$ \\ \midrule
 2 & 0.257 & 99.33 & 152 \\
 4 & 0.256 & 99.32 & 166 \\
\bottomrule
\end{tabular}
}
\caption{{\bf Impact of capacity of $\aggregation$} -- we find that increasing the number of layers in $\aggregation$ beyond 2 decreases time efficiency without substantially improving the reconstruction quality.}
\label{tab:agg_capacity}
\vspace{-1em}
\end{table}

\paragraph{Impact of capacity of $\aggregation$ -- \Cref{tab:agg_capacity}} 
We report how the capacity of the aggregation module $\aggregation$ (i.e., different number of hidden layers) impacts the performance.
We observe that aggregation modules of higher capacity give better performance but degraded time efficiency. However, as shown in~\Cref{tab:agg_capacity}, a very large capacity (4 layers) for $\aggregation$ does not help.
We show that we we use 2 layers to have a good trade-off between accuracy and time efficiency. 
We supplement~\Cref{tab:agg_capacity} with an analysis across even more levels in the \texttt{Supplementary Material}.

\begin{table}
\centering
\resizebox{.9\columnwidth}{!}{
\begin{tabular}{@{}lcccc@{}}
\toprule
\textbf{Num. of scales} &KNN & Minkowski & Z-order & Hilbert  \\ \midrule
0 & 1.00 & 0.17 & 0.44  & \cellcolor{1st}0.46  \\
1 & 1.00 & 0.29 & 0.48  & \cellcolor{1st}0.50  \\
2 & 1.00 & 0.38 & 0.49  & \cellcolor{1st}0.52  \\
3 & 1.00 & 0.44 & 0.49  & \cellcolor{1st}0.53  \\ %
\bottomrule
\end{tabular}
}
\caption{\textbf{Recall rate of our Hilbert-curve based $\neighbor$} -- we find that the Hilbert curve consistently outperforms both the Z-order curve~\cite{morton1966computer} and the one-ring neighborhood from Minkowski relative to the exact k-nearest neighbors.
}
\vspace{-1em}
\label{tab:locality_neighbor}
\end{table}

\paragraph{Analysis of neighbors retrieved by~$\neighbor$ -- \Cref{tab:locality_neighbor}}
\at{We now investigate the quality of the point neighborhoods retrieved by various possible implementations for $\neighbor$.
In particular, we are interested to experimentally study whether our serialization indeed preserves locality.
To quantify this, we treat the neighborhood retrieved with KNN as the ground-truth.}
We report the recall rate of a local neighborhood by comparing it with this ground truth~(we ignore the precision rate because we remove false positives with a distance threshold).
We also report the recall rate of the one-ring neighborhood retrieved in Minkowski~\cite{choy20194d}.
We show that the recall rate of our Hilbert $\neighbor$ is the best across variants, and across all scales.

\begin{table}[t]
\centering
\resizebox{\columnwidth}{!}{
\begin{tabular}{L rr rR}
\toprule
Methods & \multicolumn{2}{c}{Uniform} & \multicolumn{2}{c}{Non-Uniform}   \\ 
\cmidrule(r){1-1}
\cmidrule(lr){2-3}
\cmidrule(l){4-5}
\nksr & 0.246 & 480s & 0.273 & 668s  \\
Ours~(Minkowski)~\cite{choy20194d}  & 0.301 & 97s & 0.349 & 94s \\
Ours~(Minkowski)~\cite{choy20194d} {(w/ KNN)} & 0.254 & 145s & 0.294 & 155s \\
\rowcolor{1st} Ours~(w/ serialization) & {0.257} & {152s} & {0.296} & {145s} \\
\rowcolor{1st} Ours~(w/ KNN) & \textbf{0.243} & \textbf{151s} & \textbf{0.273} & \textbf{142s}  \\
\bottomrule
\end{tabular}
}
\caption{
\textbf{The impact of sampling} -- we evaluate uniform vs non-uniform sampling on ScanNet. We find that our method achieves the best accuracy (in terms of CD ($10^{-2}$)) and good time efficiency compared to \nksr~for both sampling types.
}
\vspace{-1em}
\label{tab:nonuniform_scannet}
\end{table}

\paragraph{The impact of sampling pattern --~\Cref{tab:nonuniform_scannet}} 
We report the impact of sampling pattern on performance by evaluating models on ScanNet point clouds that are uniformly or non-uniformly sampled. 
{To non-uniformly sample the ScanNet point clouds, we first partitioned the scene into eight blocks and randomly sampled a different number of points from each block. The number of samples followed an arithmetic sequence with a common difference of 200. Finally, we padded the last block to ensure that the total number of points remained 10K.}
 
We show that our method achieves better robustness to non-uniform sampling than the baselines, highlighting the importance of avoiding quantization of the point cloud for high quality surface reconstruction. 


\section{Limitations} 

In this work, we compared the effectiveness and interplay of SFT and RL-based methods, under fixed data constraints. In particular, we chose offline methods like DPO and KTO as the baseline implementation of the RL method because it eliminates the need for reward modeling or iterative finetuning. This means that the process of development is limited to collecting an offline dataset and fientuning it - making it the most fair comparable to SFT in terms of implementation effort, compute costs and annotation efforts. Since this baseline RL method shows optimal performance over SFT, we hope that this motivates future work to study more complex RL-based methods and their interplay with SFT. In addition, we used GPT4o annotation for synthetic data generation, and also for evaluating Summarization and Helpfulness, which could include potential biases inherited from the model. 

In addition, we limited the size of the model to under 10 Billion parameters, to keep the finetuning cost low enough to ignore as compared to the data annotation costs. In addition, it would be extremely compute resource intensive to run thousands of finetuning runs with larger model sizes like 70B parameters. We hope that future work would study the scaling trends of RL-based methods against different model sizes, and also study the compute-data trade-off in-depth.


\section{Limitations}

While our approach is typically more general than current training-based approaches, it still has limitations. One limitation arises from surprising entanglements in the CLIP and diffusion feature spaces. For example, when attempting to combine a zebra's body with a leopard fur pattern (\cref{fig:limitations} (top)), the diffusion model tends to produce animals with the head of a giraffe, even though no giraffe appears in either input image. We hypothesize that this may be related to the tendency of diffusion models to represent some concepts as a composition of more basic visual components~\citep{chefer2023hidden}, but leave further investigation to future work.

On the other hand, some concepts may be \textit{more} disentangled in CLIP-space than intuitively expected. For example, outfit types and colors are disentangled in CLIP-space, hence, an ``outfit'' subspace spanned with descriptions of different types of outfits (``dress'', ``tuxedo''...) will not preserve outfit colors (\cref{fig:limitations} (bottom)). However, this can be easily amended by also specifying colors in the spanning texts (``\textit{red} dress'', ``\textit{blue} tuxedo''...).



Finally, we note that IP-Adapter itself is limited in the level of detail captured from the input image. Hence, our approach will not be sufficient for capturing delicate details such as exact identities. Stronger encoders may achieve higher fidelity, but it is not clear that our embedding-space projections would generalize to more complex feature spaces.

\section{Conclusions}

We presented IP-Composer, a training-free method that allows a user to compose novel images from visual concepts derived through a set of input images. To do so, our approach uses a CLIP-based IP-Adapter, leveraging their joint disentangled subspace structure. Through this approach, we achieve comparable or better performance compared with existing training-based methods, and can more easily generalize to novel concepts derived solely from textual descriptions. 

We hope that our work can serve as an additional component of the creative toolbox, and open the way to additional composable-concept discovery methods. 

\section{Acknowledgment}
We would like to thank Ron Mokady and Yoad Tewel for providing feedback and helpful suggestions.



\section{Acknowledgements}
We thank Guy Tevet and Oren Katzir for their valuable insights and engaging discussions. We also thank Yuval Alaluf, Elad Richardson, and Sagi Polaczek for providing feedback on early versions of our manuscript. 
This work was partially supported by Joint NSFC-ISF Research Grant no. 3077/23 and Isf 3441/21.

\begin{figure*}
    \centering
    \includegraphics[width=0.87\linewidth]{figs/results/swiftsketch_seen.pdf}
    \vspace{-0.2cm}
    \caption{Sketches generated by SwiftSketch for seen categories, using input images not included in the training data.}
    \label{fig:qualitative-swift-seen}
\end{figure*}



\begin{figure*}
    \centering
    \includegraphics[width=0.87\linewidth]{figs/results/swiftsketch_unseen.pdf}
    \vspace{-0.2cm}
    \caption{Sketches generated by SwiftSketch for unseen categories.}
    \label{fig:qualitative-swift-unseen}
\end{figure*}


\begin{figure*}[t]
    \centering
    \setlength{\tabcolsep}{1pt}
    {\small
    \begin{tabular}{c c c |c c c |c c c}
         Input & ControlSk. &  CLIPasso  & Input  & ControlSk. &  CLIPasso & Input  & ControlSk. &  CLIPasso  \\
        \includegraphics[width=0.1\linewidth]{figs/figure_comparison_opt/dolphin_331_input.png} &
        \includegraphics[width=0.1\linewidth]{figs/figure_comparison_opt/dolphin_331_control_sketch.png} &
       \includegraphics[width=0.1\linewidth]{figs/figure_comparison_opt/dolphin_331_CLIPasso.png} &
       \includegraphics[width=0.1\linewidth]{figs/figure_comparison_opt/bear_3120_input.png} &
       \includegraphics[width=0.1\linewidth]{figs/figure_comparison_opt/bear_3120_control_sketch.png} &
       \includegraphics[width=0.1\linewidth]{figs/figure_comparison_opt/bear_3120_CLIPasso.png} &

        \includegraphics[width=0.1\linewidth]{figs/figure_comparison_opt/dog_2184_input.png} &
        \includegraphics[width=0.1\linewidth]{figs/figure_comparison_opt/dog_2184_control_sketch.png} &
       \includegraphics[width=0.1\linewidth]{figs/figure_comparison_opt/dog_2184_CLIPasso.png}  \\
       Input & ControlSk. &  CLIPasso  & Input  & ControlSk. &  CLIPasso & Input  & ControlSk. &  CLIPasso  \\
       \includegraphics[width=0.1\linewidth]{figs/figure_comparison_opt/parrot_221_input.png} &
        \includegraphics[width=0.1\linewidth]{figs/figure_comparison_opt/parrot_221_control_sketch.png} &
       \includegraphics[width=0.1\linewidth]{figs/figure_comparison_opt/parrot_221_CLIPasso.png} &
        \includegraphics[width=0.1\linewidth]{figs/figure_comparison_opt/helicopter_957_input.png} &
        \includegraphics[width=0.1\linewidth]{figs/figure_comparison_opt/helicopter_957_control_sketch.png} &
       \includegraphics[width=0.1\linewidth]{figs/figure_comparison_opt/helicopter_957_CLIPasso.png}  &
       \includegraphics[width=0.1\linewidth]{figs/figure_comparison_opt/broccoli_695_input.png} &
        \includegraphics[width=0.1\linewidth]{figs/figure_comparison_opt/broccoli_695_control_sketch.png} &
       \includegraphics[width=0.1\linewidth]{figs/figure_comparison_opt/broccoli_695_CLIPasso.png} \\
        Input & ControlSk. &  CLIPasso  & Input  & ControlSk. &  CLIPasso & Input  & ControlSk. &  CLIPasso  \\
      

         \includegraphics[width=0.1\linewidth]{figs/figure_comparison_opt/camel_176_input.png} &
        \includegraphics[width=0.1\linewidth]{figs/figure_comparison_opt/camel_176_control_sketch.png} &
       \includegraphics[width=0.1\linewidth]{figs/figure_comparison_opt/camel_176_CLIPasso.png}  &
        \includegraphics[width=0.1\linewidth]{figs/figure_comparison_opt/backpack_45_input.png} &
        \includegraphics[width=0.1\linewidth]{figs/figure_comparison_opt/backpack_45_control_sketch.png} &
       \includegraphics[width=0.1\linewidth]{figs/figure_comparison_opt/backpack_45_CLIPasso.png}&  
       \includegraphics[width=0.1\linewidth]{figs/figure_comparison_opt/duck_45_input.png} &
        \includegraphics[width=0.1\linewidth]{figs/figure_comparison_opt/duck_45_control_sketch.png} &
       \includegraphics[width=0.1\linewidth]{figs/figure_comparison_opt/duck_45_CLIPasso.png} \\
        Input & ControlSk. &  CLIPasso  & Input  & ControlSk. &  CLIPasso & Input  & ControlSk. &  CLIPasso  \\

        \includegraphics[height=0.09\linewidth]{figs/figure_comparison_opt/406124.png} &
        \includegraphics[width=0.1\linewidth]{figs/figure_comparison_opt/406124_control_sketch.png} &
       \includegraphics[width=0.1\linewidth]{figs/figure_comparison_opt/406124_CLIPasso.png}  &
       \raisebox{0.23\height}{\includegraphics[width=0.1\linewidth]{figs/figure_comparison_opt/584196.png}} &
        \includegraphics[width=0.1\linewidth]{figs/figure_comparison_opt/584196_control_sketch.png} &
       \includegraphics[width=0.1\linewidth]{figs/figure_comparison_opt/584196_CLIPasso.png} &
        \raisebox{0.15\height}{\includegraphics[height=0.08\linewidth]{figs/figure_comparison_opt/sheep_31.png}} &
        \includegraphics[width=0.1\linewidth]{figs/figure_comparison_opt/sheep_31_control_sketch.png} &
       \includegraphics[width=0.1\linewidth]{figs/figure_comparison_opt/sheep_31_CLIPasso.png}  \\

       
    
    \end{tabular}
    }
    \vspace{-0.3cm}
    \caption{Comparison of ControlSketch with CLIPasso \cite{vinker2022clipasso}. ControlSketch captures fine details (e.g., camel and bear), avoids artifacts in small objects (e.g., dog and duck), and handles challenging inputs effectively (last row).}
    \label{fig:comparison_opt}
\end{figure*}

\begin{figure*}
    \centering
    \includegraphics[width=0.95\linewidth]{figs/reifnment_swift.pdf}
    \vspace{-0.2cm}
    \caption{Effect of the refinement network. The output sketches from the diffusion model may contain slight noise, which the refinement network addresses by performing an additional cleaning step. However, this process can sometimes reduce details in the sketch (see Limitations).}
    \label{fig:refine_ablation}
\end{figure*}


{
    \small
    \bibliographystyle{ieee_fullname}
    \bibliography{main}
}

\clearpage
\appendix
\setcounter{page}{1}

\newpage
\twocolumn[
\centering
\Large
\textbf{\thetitle}\\
\vspace{2em}Supplementary Material \\
\vspace{1.0em}
] %

   

\part{}
\vspace{-20pt}
\parttoc





\vspace{-0.2cm}









\begin{figure}
    \centering
    \includegraphics[width=1\linewidth]{figs_sup/data_sample_cat.pdf}
    \caption{An example from the ControlSketch dataset, which includes the input image, object mask, attention map, and the corresponding sketch generated using ControlSketch.}
    \label{fig:data_sample}
\end{figure}

\section{ControlSketch Dataset}
The data generation process begins with generating images, followed by creating corresponding sketches using the ControlSketch framework, as described in section 4.2 in the main paper.
We generate images using the SDXL model \cite{podell2023sdxlimprovinglatentdiffusion} with the following prompt:
\textit{``A highly detailed wide-shot image of one $<c>$, set against a plain mesmerizing background. Center.''}, where $c$ is the class label.
Additionally, a negative prompt, \textit{``close up, few, multiple,''} is applied to ensure images depict a single object in a clear and high-quality pose. The generated images are of size $1024 \times 1024$. An example output image for the class ``cat'' is shown in \Cref{fig:data_sample}.

During the image generation process, we retain cross attention maps of the class label token extracted from internal layers of the model for future use. To isolate the object, we employ the BRIA Background Removal v1.4 Model  \cite{briaRMBG} to extract an object mask. After generating the image, we use BLIP2 \cite{li2023blip2bootstrappinglanguageimagepretraining} to extract the image caption that provides context beyond the object’s class. For example, for the image in \cref{fig:data_sample}, the caption  describe the cat as sitting, offering richer semantic information.
The sketches are generated using the ControlSketch method with 32 strokes. These strokes are subsequently arranged according to our stroke-sorting schema. The final SVG files contains the sorted strokes.
We use the Hugging Face implementation of SDXL version 1.0 \cite{podell2023sdxlimprovinglatentdiffusion} with its default parameters. Generating a single image with SDXL takes approximately 10 seconds, while sketch generation using the ControlSketch method on an NVIDIA RTX3090 GPU requires about 10 minutes.

Our dataset comprises 35,000 pairs of images and their corresponding sketches in SVG format, spanning 100 object categories. These categories are derived by combining common ones from existing sketch datasets {\cite{Mukherjee2023SEVALS,SketchRNN,SketchyCOCO2020,Eitz2012HowDH} with additional, unique categories such as astronaut, robot and sculpture. These unique categories are not present in prior datasets, highlighting the advantages of a synthetic data approach. The full list of categories is available in Table~\ref{tab:ControlSketch_dataset}.
All the sketches in our data were manually verified, we filtered very few generated images with artifacts that caused artifacts in the generated sketches.
The 15 categories used in training are: angel, bear, car, chair, crab, fish, rabbit, sculpture, astronaut, bicycle, cat, dog, horse, robot, woman. For each of these categories we generated 1200 image-sketch pairs, where 1000 samples are used for training and the rest for testing.
For the rest of 85 categories we created 200 samples per class. 
We show 78 random samples from each class of the training data in \Cref{fig:dog,fig:angle,fig:astronaut,fig:bicycle,fig:chair,fig:bear,fig:car,fig:cat,fig:horse,fig:chair,fig:crab,fig:fish,fig:rabbit,fig:Sculpture,fig:robot,fig:woman}, and 100 random samples from the entire dataset (one of each class) in \Cref{fig:100ControlSketch}.
Since the entire data creation pipeline is fully automated, we continuously extend the dataset and plan to release the code to enable future work in this area.

\begin{table}[h!]
\centering
\setlength{\tabcolsep}{2pt}
\begin{tabular}{|c|c|c|c|c|}
\midrule
\small
airplane & alarm clock & angel & astronaut & backpack \\ \midrule
bear & bed & bee & beer & bicycle \\ \midrule
boat & broccoli & burger & bus & butterfly \\ \midrule
cabin & cake & camel & camera & candle \\ \midrule
car & carrot & castle & cat & cell phone \\ \midrule
chair & chicken & child & cow & crab \\ \midrule
cup & deer & doctor & dog & dolphin \\ \midrule
dragon & drill & duck & elephant & fish \\ \midrule
flamingo & floor lamp & flower & fork & giraffe \\ \midrule
goat & hammer & hat & helicopter & horse \\ \midrule
house & ice cream & jacket & kangaroo & kimono \\ \midrule
laptop & lion & lobster & margarita & mermaid \\ \midrule
motorcycle & mountain & octopus & parrot & pen \\ \midrule
\begin{tabular}[c]{@{}c@{}}pickup \\ truck \end{tabular}  & pig & purse & quiche & rabbit \\ \midrule
robot & sandwich & scissors & sculpture & shark \\ \midrule
sheep & spider & squirrel & strawberry & sword \\ \midrule
t-shirt & table & teapot & television & tiger \\ \midrule
tomato & train & tree & truck & umbrella \\ \midrule
vase & waffle & watch & whale & wine bottle \\ \midrule
woman & yoga & zebra & \begin{tabular}[c]{@{}c@{}}The Eiffel \\ Tower \end{tabular} & book \\ \midrule
\end{tabular}
\caption{The 100 categories of the ControlSketch dataset.}
\label{tab:ControlSketch_dataset}
\end{table}



\begin{figure*}
    \centering
    \includegraphics[width=1\linewidth]{figs_sup/data/all1.pdf}
    \caption{100 random samples of sketches generated with ControlSketch.}
    \label{fig:100ControlSketch}
\end{figure*}




\section{ControlSketch Method}
\paragraph{Technical details}
In the ControlSketch optimization, we leverage the pretrained depth ControlNet model \cite{controlnet2023} to compute the SDS loss. The Adam optimizer is employed with a learning rate of 0.8. The optimization process runs for 2000 iterations, taking approximately 10 minutes to generate a single sketch on an RTX 3090 GPU.  However, after 700 iterations most images already yield a clearly identifiable sketch.

\paragraph{Strokes initialization}
The number of areas, $k$, is defined as the rounded square root of the total number of strokes $n$ (for our default number of strokes, 32, $k$ is set to 6). 
Our initialization technique combines between saliency and full coverage of the sketch, which we find to be important when the SDS loss is applied with our spatial control. In \Cref{fig:initalization_vis} we demonstrate how the final sketches will look like when applied with and without our enhances initialization, where the default case is defined based on the attention map as was proposed in CLIPasso \cite{vinker2022clipasso}. As seen, our approach ensures comprehensive object coverage while emphasizing critical areas, resulting in visually effective and recognizable sketches without omitting essential elements. For example, in the lion image, initializing strokes based solely on saliency results in almost all strokes focusing on the lion's head. Consequently, the final sketch omits significant portions of the lion's body.


\paragraph{Spatial control}
The ControlNet model receives two inputs as conditions: the text prompt and the depth condition. The balance between these conditions which is determined  by the conditioning scale parameter influences the final sketch attributes. We found that a conditioning scale of 1.5 provides the best results, effectively maintaining both semantic and geometric attributes of the subject. 

The depth ControlNet model used in the control SDS loss can be replaced with any other ControlNet model, along with the extraction of the appropriate condition from the input image. Different ControlNet models influence the style and attributes of the final sketch. Examples of different sketches generated with different ControlNet models and conditions are shown in Figure~\ref{fig:deifferent_conditions}.


 


\begin{figure}[t]
    \centering
    \setlength{\tabcolsep}{0pt}
    {\small
        \begin{tabular}{c |c c c}
         \midrule
        Input &  Depth & Scribble & Segmentation  \\
         \midrule
        \includegraphics[width=0.25\linewidth]{figs_sup/cond_inputs/masked_bull.png} &
        \includegraphics[width=0.25\linewidth]{figs_sup/cond_inputs/b_depth2.png} & 
        \includegraphics[width=0.25\linewidth]{figs_sup/cond_inputs/b_edge2.png} & 
        \includegraphics[width=0.25\linewidth]{figs_sup/cond_inputs/b_seg.png} \\

        &
        \includegraphics[width=0.25\linewidth]{figs_sup/different_conditions/bull_depth.png} &
        \includegraphics[width=0.25\linewidth]{figs_sup/different_conditions/bull_scribble.png} & \includegraphics[width=0.25\linewidth]{figs_sup/different_conditions/bull_segmentation.png} \\

         \midrule
        Input &  Depth & Scribble & Segmentation  \\
         \midrule

         \includegraphics[width=0.25\linewidth]{figs_sup/different_conditions/flamingo} &
        \includegraphics[width=0.25\linewidth]{figs_sup/cond_inputs/f_depth.png} &
        \includegraphics[width=0.25\linewidth]{figs_sup/cond_inputs/f_edge.png}  & 
        \includegraphics[width=0.25\linewidth]{figs_sup/cond_inputs/f_seg.png} \\

         &
        \includegraphics[width=0.25\linewidth]{figs_sup/different_conditions/flamingo_depth.png} &
        \includegraphics[width=0.25\linewidth]{figs_sup/different_conditions/flamingo_scribble.png}  & 
        \includegraphics[width=0.25\linewidth]{figs_sup/different_conditions/flamingo_segmentation.png} \\
       
    \end{tabular}
    }
    \caption{Examples of sketches generated by ControlSketch using different ControlNet models, alongside their respective conditions which influence the style and attributes of the final sketches.}
    \label{fig:deifferent_conditions}
\end{figure}




\begin{figure}
    \centering
    \includegraphics[width=0.6\linewidth]{figs_sup/initalization_vis.pdf}
    \caption{ Strokes initialization in the ControlSketch method. The "Saliency" column demonstrates the result when strokes are initialized based solely on the attention map (following common practive \cite{vinker2022clipasso}), often leading to an overemphasis on critical regions, such as the lion's head, at the expense of other important parts like the body. The "Saliency + coverage" column showcases our enhanced initialization method, which combines saliency with full object coverage, ensuring both essential details and global object representation are maintained, resulting in complete and recognizable sketches.}
    \label{fig:initalization_vis}
\end{figure}

\section{SwiftSketch}
Our implementation is built on the MDM codebase \cite{tevet2023human}.
Our model consists of 8 self- and cross-attention layers. It was trained with a batch size of 32, a learning rate of $5 \times 10^{-5}$, for 400,000 steps. The refinement network shares the same architecture as our diffusion model and is initialized with its final weights. The timestep condition is fixed at 0. We train the refinement network on the diffusion output sketches from the training dataset, using only the LPIPS loss between the network’s rendered output sketch and the target rendered sketch, as we found it resulting in more polished and visually improved final sketches. The refinement network was trained for 30,000 steps with a learning rate of $5 \times 10^{-6}$.

For training, We scaled the ground truth $(x, y)$ coordinates to the range [-2, 2]. Our experiments revealed that a scaling factor of 2 outperformed the standard value of 1.0 which is used in image generation tasks.
To extract input image features for our model, the image is processed using a pretrained CLIP ResNet model \cite{Radfordclip}, with features extracted from its fourth layer. These features are subsequently refined through three convolutional layers to capture additional spatial details. Each patch embedding is further refined using three linear layers, enhancing feature learning and aligning dimensions for compatibility with the model. The resulting feature representation is seamlessly integrated into the generation process via a cross-attention mechanism.

To encourage the diffusion model to focus on fine details, we adjust the noise scheduler to perturb the signal more subtly for small timesteps, by reducing the exponent in the standard cosine noise schedule proposed in \cite{Nichol2021ImprovedDD} from 2 to 0.4.
Our model $M_{\theta}$ was trained using classifier-free guidance so during inference, we enhance fidelity to the input image by extrapolating the following variants using s= 2.5:
\begin{equation}
    M_{\theta_s}(s^t, t, I) = M_{\theta}(s^t, t, \emptyset) + s \cdot \big( M_{\theta}(s^t, t, I) - M_{\theta}(s^t, t, \emptyset) \big)
\end{equation}.



Figure~\ref{fig:swiftsketch_random} showcases 100 random SwiftSketch samples across all categories in the ControlSketch dataset. The last three rows correspond to classes our model was trained on, while the remaining rows are unseen classes. Each sketch is generated in under a second. The results demonstrate that our model generalizes well to unseen categories, producing sketches with high fidelity to the input images. However, in some cases, high-level details are absent, and the sketch's category label can be difficult to identify. More examples for unseen classes are shown in Figure~\ref{fig:swiftskwtch_unseen1}, Figure~\ref{fig:swiftskwtch_unseen2} and Figure~\ref{fig:swiftskwtch_unseen3}


\section{Qualitative Comparison}
Figure~\ref{fig:comparison_train} and Figure~\ref{fig:comparison_test}  show more examples of qualitative comparison of seen and unseen categories. Input images are shown on the left. From left to right, the sketches are generated using  PhotoSketching \cite{Li2019PhotoSketchingIC}, Chan et al. \cite{Chan2022LearningTG} (in anime style), InstantStyle \cite{Wang2024InstantStyleFL}, and CLIPasso \cite{vinker2022clipasso}. On the right are the resulting sketches from our proposed methods, ControlSketch and SwiftSketch.


\begin{figure}
    \centering
    \includegraphics[width=1\linewidth]{figs_sup/strokes_number_sort.jpeg}
    \caption{Stroke Order Visualization. SwiftSketch generated sketches are visualized progressively, with the stroke count shown on top. The first row for each example is with our sorting technique (w), while the second row omits it (w/o)}
    \label{fig:strokes_number_sort}
\end{figure}



\section{Quantitative Evaluation}
In this section, we present the details of the user study conducted to compare our new optimization method, ControlSketch, with the state-of-the-art optimization method for object sketching, CLIPasso.
We selected 24 distinct categories for the user study: 16 categories from our ControlSketch dataset, and 8 categories from the SketchyCOCO dataset. For each category, we randomly sampled one image. Participants were presented with the input image alongside two sketches—one generated by CLIPasso and the other by ControlSketch—displayed in random order. We asked participants two questions for each pair of sketches: 1. Which sketch more effectively depicts the input image? 2. Which sketch is of higher quality? Participants were required to choose one sketch for each question. A total of 40 individuals participated in the survey. The results are as follows: For the ControlSketch dataset, 87\% of participants chose ControlSketch for the first question, and 88\% for the second question. For the SketchyCOCO dataset—which is more challenging due to its low-resolution images and difficult lighting conditions—90\% chose ControlSketch for the first question, and 93\% for the second question.
These results highlight the significant advantages of ControlSketch over CLIPasso across diverse categories and datasets.


\section{Ablation}
Figure~\ref{fig:comparison_refine} presents a comparison of results with and without the refinement step in the SwiftSketch pipeline. As can be seen, the final output sketches generated by the denoising process of our diffusion model may still retain slight noise. Incorporating the refinement stage significantly enhances the quality and cleanliness of the sketches
Figure~\ref{fig:strokes_number_sort} illustrates the impact of the stroke sorting technique used for training. Early strokes effectively capture the object’s contour and key features, while later strokes refine the details. With sorting, the object is significantly more recognizable with fewer strokes compared to the case without sorting.



\begin{figure}
    \centering
    \includegraphics[width=1\linewidth]{figs_sup/limitation_supp_new.jpg}
    \caption{Limitations of SwiftSketch. (a) When trained solely on masked object images, SwiftSketch struggles to generate accurate sketches for complex scenes. As shown, it incorrectly assigns strokes to the image frame instead of capturing the scene's key elements. (b) During the refinement stage, fine details particularly facial features are often lost, resulting in oversimplified representations. (c) Sketches may appear unrecognizable. }
    \label{fig:limitations1}
\end{figure}


\section{Limitations}
SwiftSketch, which was trained only on masked object images, faces challenges in handling complex scenes. When provided with a scene image, as illustrated in Figure~\ref{fig:limitations1}(a), SwiftSketch struggles to generate accurate sketches, often misplacing strokes onto the image frame instead of capturing key elements of the scene. Another significant limitation is its tendency to omit fine details, particularly facial features, leading to oversimplified representations, as shown in Figure~\ref{fig:limitations1}(b). In some cases, sketches may appear unrecognizable, as shown in Figure~\ref{fig:limitations1}(c).





\begin{figure*}[t]
    \centering
    \setlength{\tabcolsep}{1pt}
    {\small
    \begin{tabular}{c}
         Seen Classes  \\
        \includegraphics[width=0.6\linewidth]{figs_sup/refine_vs_without_new/train4.pdf}   \\
         Unseen Classes  \\
        \includegraphics[width=0.6\linewidth]{figs_sup/refine_vs_without_new/test2.pdf}   \\
    
    \end{tabular}
    }
    \caption{Comparison of SwiftSketch sketches with (right) and without (left) the refinement step. This highlights the critical role of the refinement network in significantly improving the quality of the generated sketches and reducing noise}
    \label{fig:comparison_refine}
\end{figure*}



\begin{figure*}
    \centering
    \includegraphics[width=0.8\linewidth]{figs_sup/ramdom_refine/test.pdf}
    \caption{100 random sapmels of SwiftSketch sketches. The last three rows are seen classes, while the remaining rows are unseen classes}
    \label{fig:swiftsketch_random}
\end{figure*}

\begin{figure*}
    \centering
    \includegraphics[width=1\linewidth]{figs_sup/swiftsketch_unseen_6A.pdf}
    \caption{ Sketches generated by SwiftSketch for unseen categories.}
    \label{fig:swiftskwtch_unseen1}
\end{figure*}

\begin{figure*}
    \centering
    \includegraphics[width=1\linewidth]{figs_sup/swiftsketch_unseen_6b.pdf}
    \caption{ Sketches generated by SwiftSketch for unseen categories.}
    \label{fig:swiftskwtch_unseen2}
\end{figure*}

\begin{figure*}
    \centering
    \includegraphics[width=1\linewidth]{figs_sup/swiftsketch_unseen_6c.pdf}
    \caption{ Sketches generated by SwiftSketch for unseen categories.}
    \label{fig:swiftskwtch_unseen3}
\end{figure*}




\begin{figure*}
    \centering
    \includegraphics[trim=0cm 0.2cm 0 0cm,clip,width=0.8\linewidth]{figs_sup/comparison_images/titles.pdf}
    \includegraphics[trim=0cm 0cm 0 3.2cm,clip,width=0.8\linewidth]{figs_sup/comparison_images/train2.pdf}
    \caption{Qualitative comparison, seen categories}
    \label{fig:comparison_train}
\end{figure*}

\begin{figure*}
    \centering
    \includegraphics[trim=0cm 0.2cm 0 0cm,clip,width=0.8\linewidth]{figs_sup/comparison_images/titles.pdf}
    \includegraphics[trim=0cm 0cm 0 3.2cm,clip,width=0.8\linewidth]{figs_sup/comparison_images/test.pdf}
    \caption{Qualitative comparison, unseen categories}
    \label{fig:comparison_test}
\end{figure*}



 
\begin{figure*}
    \centering
    \includegraphics[width=1\linewidth]{figs_sup/data/dog.pdf}
    \caption{Dog - SwiftSketch training data examples }
    \label{fig:dog}
\end{figure*}

\begin{figure*}
    \centering
    \includegraphics[width=1\linewidth]{figs_sup/data/horse.pdf}
    \caption{Horse - SwiftSketch training data examples }
    \label{fig:horse}
\end{figure*}

\begin{figure*}
    \centering
    \includegraphics[width=1\linewidth]{figs_sup/data/cat.pdf}
    \caption{Cat - SwiftSketch training data examples}
    \label{fig:cat}
\end{figure*}

\begin{figure*}
    \centering
    \includegraphics[width=1\linewidth]{figs_sup/data/angel.pdf}
    \caption{Angel - SwiftSketch training data examples}
    \label{fig:angle}
\end{figure*}

\begin{figure*}
    \centering
    \includegraphics[width=1\linewidth]{figs_sup/data/astronaut.pdf}
    \caption{Astronaut - SwiftSketch training data examples}
    \label{fig:astronaut}
\end{figure*}

\begin{figure*}
    \centering
    \includegraphics[width=1\linewidth]{figs_sup/data/bear.pdf}
    \caption{Bear - SwiftSketch training data examples}
    \label{fig:bear}
\end{figure*}

\begin{figure*}
    \centering
    \includegraphics[width=1\linewidth]{figs_sup/data/bicycle.pdf}
    \caption{Bicycle - SwiftSketch training data examples}
    \label{fig:bicycle}
\end{figure*}

\begin{figure*}
    \centering
    \includegraphics[width=1\linewidth]{figs_sup/data/car.pdf}
    \caption{Car - SwiftSketch training data examples}
    \label{fig:car}
\end{figure*}

\begin{figure*}
    \centering
    \includegraphics[width=1\linewidth]{figs_sup/data/chair.pdf}
    \caption{Chair - SwiftSketch training data examples }
    \label{fig:chair}
\end{figure*}

\begin{figure*}
    \centering
    \includegraphics[width=1\linewidth]{figs_sup/data/crab.pdf}
    \caption{Crab - SwiftSketch training data examples}
    \label{fig:crab}
\end{figure*}

\begin{figure*}
    \centering
    \includegraphics[width=1\linewidth]{figs_sup/data/fish.pdf}
    \caption{Fish - SwiftSketch training data examples}
    \label{fig:fish}
\end{figure*}

\begin{figure*}
    \centering
    \includegraphics[width=1\linewidth]{figs_sup/data/rabbit.pdf}
    \caption{Rabbit - SwiftSketch training data examples}
    \label{fig:rabbit}
\end{figure*}

\begin{figure*}
    \centering
    \includegraphics[width=1\linewidth]{figs_sup/data/sculpture.pdf}
    \caption{:Sculpture - SwiftSketch training data examples }
    \label{fig:Sculpture}
\end{figure*}

\begin{figure*}
    \centering
    \includegraphics[width=1\linewidth]{figs_sup/data/robot.pdf}
    \caption{Robot - SwiftSketch training data examples }
    \label{fig:robot}
\end{figure*}


\begin{figure*}
    \centering
    \includegraphics[width=1\linewidth]{figs_sup/data/woman.pdf}
    \caption{Woman - SwiftSketch training data examples }
    \label{fig:woman}
\end{figure*}





\end{document}
