
\documentclass[10pt,twocolumn,letterpaper]{article}

\usepackage[pagenumbers]{cvpr} %
\usepackage{minitoc}
\renewcommand \thepart{}
\renewcommand \partname{}

\usepackage[dvipsnames]{xcolor}


\definecolor{cvprblue}{rgb}{0.21,0.49,0.74}
\usepackage[pagebackref,breaklinks,colorlinks,citecolor=cvprblue]{hyperref}
\usepackage{tikz,lipsum}
\usepackage[most]{tcolorbox}
\usepackage[capitalize]{cleveref}
\usepackage{multirow}
\usepackage{listings}
\usepackage{xcolor}
\usepackage{tabularx}
\usepackage{makecell}


\def\confName{CVPR}
\def\confYear{2025}

\title{\vspace{-0.8cm}SwiftSketch: A Diffusion Model for Image-to-Vector Sketch Generation\vspace{-0.3cm}}



\author{\hspace{-1cm} Ellie Arar$^{1}$ \hspace{-0.8cm} 
\and Yarden Frenkel$^{1}$ \hspace{-0.8cm}
\and Daniel Cohen-Or$^{1}$ \hspace{-0.8cm} 
\and Ariel Shamir$^{2}$ \hspace{-0.8cm}
\and Yael Vinker$^{1,3}$ \hspace{-1cm}
\and \hspace{0.9\linewidth} \and
$^{1}$Tel Aviv University \\ {\tt\small \{elliearar, Yf2, dcor\}@mail.tau.ac.il}\and $^{2}$Reichman University  \\ {\tt\small arik@runi.ac.il} \and $^{3}$MIT \\ {\tt\small yaelvink@mit.edu} \vspace{-0.5cm}\\ \and {\tt\small \href{https://swiftsketch.github.io/}{https://swiftsketch.github.io/} } \vspace{-0.6cm}
}


\newcommand{\methodname}{{SwiftSketch\xspace}}


\begin{document}
\doparttoc %
\faketableofcontents %

\twocolumn[{%
\vspace{-0.65cm}
\maketitle
\renewcommand\twocolumn[1][]{#1}%
\vspace{-0.8cm}
\begin{center}
    \centering
    \includegraphics[width=0.95\linewidth]{figs/swift_teaser2.pdf}
    \captionsetup{type=figure}
    \vspace{-0.3cm}
    \caption{SwiftSketch is a diffusion model that generates vector sketches by denoising a Gaussian in stroke coordinate space (top). It generalizes effectively across diverse classes and takes under a second to produce a single high-quality sketch (bottom).}
    \label{fig:teaser}
\end{center}
}]
\begin{abstract}
\vspace{-0.4cm}
Recent advancements in large vision-language models have enabled highly expressive and diverse vector sketch generation. However, state-of-the-art methods rely on a time-consuming optimization process involving repeated feedback from a pretrained model to determine stroke placement. Consequently, despite producing impressive sketches, these methods are limited in practical applications.
In this work, we introduce \emph{SwiftSketch}, a diffusion model for image-conditioned vector sketch generation that can produce high-quality sketches in less than a second.
SwiftSketch operates by progressively denoising stroke control points sampled from a Gaussian distribution. 
Its transformer-decoder architecture is designed to effectively handle the discrete nature of vector representation and capture the inherent global dependencies between strokes.
To train SwiftSketch, we construct a \emph{synthetic} dataset of image-sketch pairs, addressing the limitations of existing sketch datasets, which are often created by non-artists and lack professional quality. For generating these synthetic sketches, we introduce ControlSketch, a method that enhances SDS-based techniques by incorporating precise spatial control through a depth-aware ControlNet.
We demonstrate that SwiftSketch generalizes across diverse concepts, efficiently producing sketches that combine high fidelity with a natural and visually appealing style.
\end{abstract}

   
\section{Introduction}
Backdoor attacks pose a concealed yet profound security risk to machine learning (ML) models, for which the adversaries can inject a stealth backdoor into the model during training, enabling them to illicitly control the model's output upon encountering predefined inputs. These attacks can even occur without the knowledge of developers or end-users, thereby undermining the trust in ML systems. As ML becomes more deeply embedded in critical sectors like finance, healthcare, and autonomous driving \citep{he2016deep, liu2020computing, tournier2019mrtrix3, adjabi2020past}, the potential damage from backdoor attacks grows, underscoring the emergency for developing robust defense mechanisms against backdoor attacks.

To address the threat of backdoor attacks, researchers have developed a variety of strategies \cite{liu2018fine,wu2021adversarial,wang2019neural,zeng2022adversarial,zhu2023neural,Zhu_2023_ICCV, wei2024shared,wei2024d3}, aimed at purifying backdoors within victim models. These methods are designed to integrate with current deployment workflows seamlessly and have demonstrated significant success in mitigating the effects of backdoor triggers \cite{wubackdoorbench, wu2023defenses, wu2024backdoorbench,dunnett2024countering}.  However, most state-of-the-art (SOTA) backdoor purification methods operate under the assumption that a small clean dataset, often referred to as \textbf{auxiliary dataset}, is available for purification. Such an assumption poses practical challenges, especially in scenarios where data is scarce. To tackle this challenge, efforts have been made to reduce the size of the required auxiliary dataset~\cite{chai2022oneshot,li2023reconstructive, Zhu_2023_ICCV} and even explore dataset-free purification techniques~\cite{zheng2022data,hong2023revisiting,lin2024fusing}. Although these approaches offer some improvements, recent evaluations \cite{dunnett2024countering, wu2024backdoorbench} continue to highlight the importance of sufficient auxiliary data for achieving robust defenses against backdoor attacks.

While significant progress has been made in reducing the size of auxiliary datasets, an equally critical yet underexplored question remains: \emph{how does the nature of the auxiliary dataset affect purification effectiveness?} In  real-world  applications, auxiliary datasets can vary widely, encompassing in-distribution data, synthetic data, or external data from different sources. Understanding how each type of auxiliary dataset influences the purification effectiveness is vital for selecting or constructing the most suitable auxiliary dataset and the corresponding technique. For instance, when multiple datasets are available, understanding how different datasets contribute to purification can guide defenders in selecting or crafting the most appropriate dataset. Conversely, when only limited auxiliary data is accessible, knowing which purification technique works best under those constraints is critical. Therefore, there is an urgent need for a thorough investigation into the impact of auxiliary datasets on purification effectiveness to guide defenders in  enhancing the security of ML systems. 

In this paper, we systematically investigate the critical role of auxiliary datasets in backdoor purification, aiming to bridge the gap between idealized and practical purification scenarios.  Specifically, we first construct a diverse set of auxiliary datasets to emulate real-world conditions, as summarized in Table~\ref{overall}. These datasets include in-distribution data, synthetic data, and external data from other sources. Through an evaluation of SOTA backdoor purification methods across these datasets, we uncover several critical insights: \textbf{1)} In-distribution datasets, particularly those carefully filtered from the original training data of the victim model, effectively preserve the model’s utility for its intended tasks but may fall short in eliminating backdoors. \textbf{2)} Incorporating OOD datasets can help the model forget backdoors but also bring the risk of forgetting critical learned knowledge, significantly degrading its overall performance. Building on these findings, we propose Guided Input Calibration (GIC), a novel technique that enhances backdoor purification by adaptively transforming auxiliary data to better align with the victim model’s learned representations. By leveraging the victim model itself to guide this transformation, GIC optimizes the purification process, striking a balance between preserving model utility and mitigating backdoor threats. Extensive experiments demonstrate that GIC significantly improves the effectiveness of backdoor purification across diverse auxiliary datasets, providing a practical and robust defense solution.

Our main contributions are threefold:
\textbf{1) Impact analysis of auxiliary datasets:} We take the \textbf{first step}  in systematically investigating how different types of auxiliary datasets influence backdoor purification effectiveness. Our findings provide novel insights and serve as a foundation for future research on optimizing dataset selection and construction for enhanced backdoor defense.
%
\textbf{2) Compilation and evaluation of diverse auxiliary datasets:}  We have compiled and rigorously evaluated a diverse set of auxiliary datasets using SOTA purification methods, making our datasets and code publicly available to facilitate and support future research on practical backdoor defense strategies.
%
\textbf{3) Introduction of GIC:} We introduce GIC, the \textbf{first} dedicated solution designed to align auxiliary datasets with the model’s learned representations, significantly enhancing backdoor mitigation across various dataset types. Our approach sets a new benchmark for practical and effective backdoor defense.



\section{Related Work}
\paragraph{\textbf{Sketch Datasets}}
Existing sketch datasets are primarily composed of human-drawn sketches, and are designed to accomplish different sketching tasks. Class-conditioned datasets \cite{Eitz2012HowDH,SketchRNN} are particularly common, with the largest being the QuickDraw dataset \cite{SketchRNN}, containing 50 million sketches spanning 345 categories. Datasets of image-referenced sketches cover a spectrum of styles, including image trace and contours \cite{Wang2021Tracing,Li2019PhotoSketchingIC,ArbelaezBSDS500,Eitz2012HowDH}, or more abstract but still fine-grained depictions \cite{Sangkloy2016TheSD,SketchyCOCO2020}, and very abstract sketches \cite{Mukherjee2023SEVALS}. These large-scale datasets are often created by non-artists.
Efforts have been made to collect sketches from professionals \cite{Berger2013StyleAA, Gryaditskaya2019OpenSketch, Han2023AGF, Xiao2022DifferSketching}, but these datasets are often smaller in scale, and are limited to specific domains like portraits \cite{Berger2013StyleAA} or household items \cite{Gryaditskaya2019OpenSketch}. These constraints make them unsuitable for training generative models that can generalize broadly to diverse concepts.

\paragraph{\textbf{Data-Driven Sketch Generation}}
These datasets have facilitated data-driven approaches for various sketch-related tasks \cite{SurveySketchxu2020deep}. Multiple generative frameworks and architectures have been explored for vector sketch generation, including RNNs \cite{SketchRNN}, BERT \cite{Lin2020SketchBERTLS}, Transformers \cite{Bhunia2020EdinburghRE, Ribeiro2020SketchformerTR}, CNNs \cite{Kampelmhler2020SynthesizingHS,Chen2017Sketchpix2seqAM,Song2018LearningTS}, LSTMs \cite{Qi2021SketchLatticeLR, Song2018LearningTS}, GANs \cite{V2019TeachingGT}, reinforcement learning \cite{Zhou2018LearningTD,Muhammad2018LearningDS}, and diffusion models \cite{wang2023sketchknitter}. However, these methods are fundamentally designed to operate in a class-conditioned manner, restricting their ability to generate sketches to only the classes included in the training data. Additionally, they rely on crowd-sourced datasets which contain non-professional sketches, restricting their ability to handle more complex or artistic styles.
On the other hand, existing works for generating more professionally looking sketches are either restricted to specific domains \cite{Liu_2021_ICCV} or can only generate sketches in pixel space \cite{Li2019PhotoSketchingIC,Chan2022LearningTG}.
Note that image-to-sketch generating can be formulated as a style transfer task, with recent works that employ the text-to-image diffusion priors achieving highly artistic results with high fidelity \cite{Wang2024InstantStyleFL,frenkel2024implicit,hertz2023StyleAligned}, however, all of these works also operate only in pixel space.
In contrast, we focus on vector sketches due to their resolution independence, smooth and clean appearance, control over abstraction, and editable nature.



\paragraph{\textbf{VLMs for Vector Sketches}}
To reduce reliance on existing vector datasets, recent research leverages the rich priors of large pre-trained vision-language models (VLMs) in a zero-shot manner. Early methods \cite{vinker2022clipasso,Frans2021CLIPDrawET,Vinker2022CLIPasceneSS} utilize CLIP \cite{Radfordclip} as the backbone for image- and text-conditioned generation. These approaches iteratively optimize a randomly initialized set of strokes using a differentiable renderer \cite{Li2020DifferentiableVG} to bridge the gap between vector and pixel representations. More recently, text-to-image diffusion models \cite{rombach2022highresolution} have been employed as backbones, with the SDS loss \cite{Poole2022DreamFusionTU} used to guide the optimization process, achieving superior results \cite{jain2022vectorfusion,svgdreamer_xing_2023,Xing2023DiffSketcherTG}. However, the use of the SDS loss has so far been limited to text-conditioned generation.
While these approaches yield highly artistic results across diverse concepts, they are computationally expensive, relying on iterative backpropagation.


\paragraph{\textbf{Diffusion Models for Non-Pixel Data}}
Diffusion models have emerged as a powerful generative framework, extending their impact beyond traditional pixel-based data. Recent research demonstrates their versatility across diverse domains, including tasks such as human motion synthesis \cite{tevet2023human}, 3D point cloud generation \cite{Luo2021DiffusionPM, Huang2025SPAR3DSP}, and object detection reframed as a generative process \cite{Chen2022DiffusionDetDM}.
Some prior works have explored diffusion models for vector graphics synthesis. 
VecFusion \cite{Thamizharasan_2024_CVPR} uses a two-stage diffusion process for vector font generation but its architecture and vector representation are highly complex and specialized for fonts, limiting adaptability to other vector tasks. SketchKnitter \cite{wang2023sketchknitter} and Ashcroft~\etal~\cite{ashcroft2024modelling} generate vector sketches using a diffusion-based model trained on the QuickDraw and Anime-Vec10k dataset, but without conditioning on images or text inputs.



















\section{Preliminaries}
\label{sec:preliminaries}
\paragraph{\textbf{Diffusion Models}}
\label{sec:diffusion}
Diffusion models \cite{ddpm2020Ho,song2021ddim} are a class of generative models that learn a distribution by gradually denoising a Gaussian. 
Diffusion models consist of a forward process $q(x_t|x_{t-1})$ that progressively noises data samples $x_0 \sim p_{data}$ at different timesteps $t\in [1,T]$, and a backward or reverse process $p(x_{t-1}|x_t)$ that progressively cleans the noised signal. 
The reverse process is the generative process and is approximate with a neural network $\epsilon_{\theta}(x_t,t)$.
During training, a noised signal at differnet timesteps is derived from a sample $x_0$ as follows:
\begin{equation}
\label{eq:xt}
x_t = \sqrt{\bar{\alpha}_t} x_0 + \sqrt{1 - \bar{\alpha}_t} \epsilon,
\end{equation}
where $\epsilon \sim \mathcal{N}(0, \mathbf{I})$, and $\bar{\alpha}_t = \prod_{s=1}^t \alpha_s$ is called the noise scheduler.
The common approach for training the model is with the following simplified objective:
\begin{equation}
\label{eq:lsimple}
L_\text{simple} = \mathbb{E}_{x_0\sim q(x_0), \epsilon \sim \mathcal{N}(0, \mathbf{I}), t \sim \mathcal{U}(1, T)} \big\| \epsilon - \epsilon_\theta(x_t, t) \big\|^2.
\end{equation}
At inference, to generate a new sample, the process starts with a Gaussian noise $x_T \sim \mathcal{N}(0, \mathbf{I})$ and the denoising network is applied iteratively for $T$ steps, yielding a final sample $x_0$.













\paragraph{\textbf{SDS Loss}}
The Score Distillation Sampling (SDS) loss \cite{Poole2022DreamFusionTU} is used to extract signals from a pretrained text-to-image diffusion model to optimize a parametric representation. For vector graphics, the parameters $\phi$ defining an SVG can be optimized using the SDS loss to represent a desired textual concept. A differentiable rasterizer \cite{Li2020DifferentiableVG} rasterize $\phi$ into a pixel image $x$, which is then noised to produce $x_t$ at a sampled timestep $t$. This noised image, conditioned on a text prompt $c$, is passed through the pretrained diffusion model, $\epsilon_\theta(x_t, t, c)$.
The deviation of the diffusion loss in \cref{eq:lsimple} is used to approximate the gradients of the initial image synthesis model's parameters, $\phi$, to better align its outputs with the conditioning prompt. Specifically, the gradient of the SDS loss is defined as:
\begin{equation}\label{eq:sds_loss}
    \nabla_\phi \mathcal{L}_{SDS} = \left[ w(t)(\epsilon_\theta(x_t,t,y) - \epsilon) \frac{\partial x}{\partial \phi} \right] ,
\end{equation}
where $w(t)$ is a constant that depends on $\alpha_t$. This optimization process iteratively adjusts the parametric model.




\section{Study Design}
% robot: aliengo 
% We used the Unitree AlienGo quadruped robot. 
% See Appendix 1 in AlienGo Software Guide PDF
% Weight = 25kg, size (L,W,H) = (0.55, 0.35, 06) m when standing, (0.55, 0.35, 0.31) m when walking
% Handle is 0.4 m or 0.5 m. I'll need to check it to see which type it is.
We gathered input from primary stakeholders of the robot dog guide, divided into three subgroups: BVI individuals who have owned a dog guide, BVI individuals who were not dog guide owners, and sighted individuals with generally low degrees of familiarity with dog guides. While the main focus of this study was on the BVI participants, we elected to include survey responses from sighted participants given the importance of social acceptance of the robot by the general public, which could reflect upon the BVI users themselves and affect their interactions with the general population \cite{kayukawa2022perceive}. 

The need-finding processes consisted of two stages. During Stage 1, we conducted in-depth interviews with BVI participants, querying their experiences in using conventional assistive technologies and dog guides. During Stage 2, a large-scale survey was distributed to both BVI and sighted participants. 

This study was approved by the University’s Institutional Review Board (IRB), and all processes were conducted after obtaining the participants' consent.

\subsection{Stage 1: Interviews}
We recruited nine BVI participants (\textbf{Table}~\ref{tab:bvi-info}) for in-depth interviews, which lasted 45-90 minutes for current or former dog guide owners (DO) and 30-60 minutes for participants without dog guides (NDO). Group DO consisted of five participants, while Group NDO consisted of four participants.
% The interview participants were divided into two groups. Group DO (Dog guide Owner) consisted of five participants who were current or former dog guide owners and Group NDO (Non Dog guide Owner) consisted of three participants who were not dog guide owners. 
All participants were familiar with using white canes as a mobility aid. 

We recruited participants in both groups, DO and NDO, to gather data from those with substantial experience with dog guides, offering potentially more practical insights, and from those without prior experience, providing a perspective that may be less constrained and more open to novel approaches. 

We asked about the participants' overall impressions of a robot dog guide, expectations regarding its potential benefits and challenges compared to a conventional dog guide, their desired methods of giving commands and communicating with the robot dog guide, essential functionalities that the robot dog guide should offer, and their preferences for various aspects of the robot dog guide's form factors. 
For Group DO, we also included questions that asked about the participants' experiences with conventional dog guides. 

% We obtained permission to record the conversations for our records while simultaneously taking notes during the interviews. The interviews lasted 30-60 minutes for NDO participants and 45-90 minutes for DO participants. 

\subsection{Stage 2: Large-Scale Surveys} 
After gathering sufficient initial results from the interviews, we created an online survey for distributing to a larger pool of participants. The survey platform used was Qualtrics. 

\subsubsection{Survey Participants}
The survey had 100 participants divided into two primary groups. Group BVI consisted of 42 blind or visually impaired participants, and Group ST consisted of 58 sighted participants. \textbf{Table}~\ref{tab:survey-demographics} shows the demographic information of the survey participants. 

\subsubsection{Question Differentiation} 
Based on their responses to initial qualifying questions, survey participants were sorted into three subgroups: DO, NDO, and ST. Each participant was assigned one of three different versions of the survey. The surveys for BVI participants mirrored the interview categories (overall impressions, communication methods, functionalities, and form factors), but with a more quantitative approach rather than the open-ended questions used in interviews. The DO version included additional questions pertaining to their prior experience with dog guides. The ST version revolved around the participants' prior interactions with and feelings toward dog guides and dogs in general, their thoughts on a robot dog guide, and broad opinions on the aesthetic component of the robot's design. 

\section{Experimental Results}
In this section, we present the main results in~\secref{sec:main}, followed by ablation studies on key design choices in~\secref{sec:ablation}.

\begin{table*}[t]
\renewcommand\arraystretch{1.05}
\centering
\setlength{\tabcolsep}{2.5mm}{}
\begin{tabular}{l|l|c|cc|cc}
type & model     & \#params      & FID$\downarrow$ & IS$\uparrow$ & Precision$\uparrow$ & Recall$\uparrow$ \\
\shline
GAN& BigGAN~\cite{biggan} & 112M & 6.95  & 224.5       & 0.89 & 0.38     \\
GAN& GigaGAN~\cite{gigagan}  & 569M      & 3.45  & 225.5       & 0.84 & 0.61\\  
GAN& StyleGan-XL~\cite{stylegan-xl} & 166M & 2.30  & 265.1       & 0.78 & 0.53  \\
\hline
Diffusion& ADM~\cite{adm}    & 554M      & 10.94 & 101.0        & 0.69 & 0.63\\
Diffusion& LDM-4-G~\cite{ldm}   & 400M  & 3.60  & 247.7       & -  & -     \\
Diffusion & Simple-Diffusion~\cite{diff1} & 2B & 2.44 & 256.3 & - & - \\
Diffusion& DiT-XL/2~\cite{dit} & 675M     & 2.27  & 278.2       & 0.83 & 0.57     \\
Diffusion&L-DiT-3B~\cite{dit-github}  & 3.0B    & 2.10  & 304.4       & 0.82 & 0.60    \\
Diffusion&DiMR-G/2R~\cite{liu2024alleviating} &1.1B& 1.63& 292.5& 0.79 &0.63 \\
Diffusion & MDTv2-XL/2~\cite{gao2023mdtv2} & 676M & 1.58 & 314.7 & 0.79 & 0.65\\
Diffusion & CausalFusion-H$^\dag$~\cite{deng2024causal} & 1B & 1.57 & - & - & - \\
\hline
Flow-Matching & SiT-XL/2~\cite{sit} & 675M & 2.06 & 277.5 & 0.83 & 0.59 \\
Flow-Matching&REPA~\cite{yu2024representation} &675M& 1.80 & 284.0 &0.81 &0.61\\    
Flow-Matching&REPA$^\dag$~\cite{yu2024representation}& 675M& 1.42&  305.7& 0.80& 0.65 \\
\hline
Mask.& MaskGIT~\cite{maskgit}  & 227M   & 6.18  & 182.1        & 0.80 & 0.51 \\
Mask. & TiTok-S-128~\cite{yu2024image} & 287M & 1.97 & 281.8 & - & - \\
Mask. & MAGVIT-v2~\cite{yu2024language} & 307M & 1.78 & 319.4 & - & - \\ 
Mask. & MaskBit~\cite{weber2024maskbit} & 305M & 1.52 & 328.6 & - & - \\
\hline
AR& VQVAE-2~\cite{vqvae2} & 13.5B    & 31.11           & $\sim$45     & 0.36           & 0.57          \\
AR& VQGAN~\cite{vqgan}& 227M  & 18.65 & 80.4         & 0.78 & 0.26   \\
AR& VQGAN~\cite{vqgan}   & 1.4B     & 15.78 & 74.3   & -  & -     \\
AR&RQTran.~\cite{rq}     & 3.8B    & 7.55  & 134.0  & -  & -    \\
AR& ViTVQ~\cite{vit-vqgan} & 1.7B  & 4.17  & 175.1  & -  & -    \\
AR & DART-AR~\cite{gu2025dart} & 812M & 3.98 & 256.8 & - & - \\
AR & MonoFormer~\cite{zhao2024monoformer} & 1.1B & 2.57 & 272.6 & 0.84 & 0.56\\
AR & Open-MAGVIT2-XL~\cite{luo2024open} & 1.5B & 2.33 & 271.8 & 0.84 & 0.54\\
AR&LlamaGen-3B~\cite{llamagen}  &3.1B& 2.18& 263.3 &0.81& 0.58\\
AR & FlowAR-H~\cite{flowar} & 1.9B & 1.65 & 296.5 & 0.83 & 0.60\\
AR & RAR-XXL~\cite{yu2024randomized} & 1.5B & 1.48 & 326.0 & 0.80 & 0.63 \\
\hline
MAR & MAR-B~\cite{mar} & 208M & 2.31 &281.7 &0.82 &0.57 \\
MAR & MAR-L~\cite{mar} &479M& 1.78 &296.0& 0.81& 0.60 \\
MAR & MAR-H~\cite{mar} & 943M&1.55& 303.7& 0.81 &0.62 \\
\hline
VAR&VAR-$d16$~\cite{var}   & 310M  & 3.30& 274.4& 0.84& 0.51    \\
VAR&VAR-$d20$~\cite{var}   &600M & 2.57& 302.6& 0.83& 0.56     \\
VAR&VAR-$d30$~\cite{var}   & 2.0B      & 1.97  & 323.1 & 0.82 & 0.59      \\
\hline
\modelname& \modelname-B    &172M   &1.72&280.4&0.82&0.59 \\
\modelname& \modelname-L   & 608M   & 1.28& 292.5&0.82&0.62\\
\modelname& \modelname-H    & 1.1B    & 1.24 &301.6&0.83&0.64\\
\end{tabular}
\caption{
\textbf{Generation Results on ImageNet-256.}
Metrics include Fréchet Inception Distance (FID), Inception Score (IS), Precision, and Recall. $^\dag$ denotes the use of guidance interval sampling~\cite{guidance}. The proposed \modelname-H achieves a state-of-the-art 1.24 FID on the ImageNet-256 benchmark without relying on vision foundation models (\eg, DINOv2~\cite{dinov2}) or guidance interval sampling~\cite{guidance}, as used in REPA~\cite{yu2024representation}.
}\label{tab:256}
\end{table*}

\subsection{Main Results}
\label{sec:main}
We conduct experiments on ImageNet~\cite{deng2009imagenet} at 256$\times$256 and 512$\times$512 resolutions. Following prior works~\cite{dit,mar}, we evaluate model performance using FID~\cite{fid}, Inception Score (IS)~\cite{is}, Precision, and Recall. \modelname is trained with the same hyper-parameters as~\cite{mar,dit} (\eg, 800 training epochs), with model sizes ranging from 172M to 1.1B parameters. See Appendix~\secref{sec:sup_hyper} for hyper-parameter details.





\begin{table}[t]
    \centering
    \begin{tabular}{c|c|c|c}
      model    &  \#params & FID$\downarrow$ & IS$\uparrow$ \\
      \shline
      VQGAN~\cite{vqgan}&227M &26.52& 66.8\\
      BigGAN~\cite{biggan}& 158M&8.43 &177.9\\
      MaskGiT~\cite{maskgit}& 227M&7.32& 156.0\\
      DiT-XL/2~\cite{dit} &675M &3.04& 240.8 \\
     DiMR-XL/3R~\cite{liu2024alleviating}& 525M&2.89 &289.8 \\
     VAR-d36~\cite{var}  & 2.3B& 2.63 & 303.2\\
     REPA$^\ddagger$~\cite{yu2024representation}&675M &2.08& 274.6 \\
     \hline
     \modelname-L & 608M&1.70& 281.5 \\
    \end{tabular}
    \caption{
    \textbf{Generation Results on ImageNet-512.} $^\ddagger$ denotes the use of DINOv2~\cite{dinov2}.
    }
    \label{tab:512}
\end{table}

\noindent\textbf{ImageNet-256.}
In~\tabref{tab:256}, we compare \modelname with previous state-of-the-art generative models.
Out best variant, \modelname-H, achieves a new state-of-the-art-performance of 1.24 FID, outperforming the GAN-based StyleGAN-XL~\cite{stylegan-xl} by 1.06 FID, masked-prediction-based MaskBit~\cite{maskgit} by 0.28 FID, AR-based RAR~\cite{yu2024randomized} by 0.24 FID, VAR~\cite{var} by 0.73 FID, MAR~\cite{mar} by 0.31 FID, and flow-matching-based REPA~\cite{yu2024representation} by 0.18 FID.
Notably, \modelname does not rely on vision foundation models~\cite{dinov2} or guidance interval sampling~\cite{guidance}, both of which were used in REPA~\cite{yu2024representation}, the previous best-performing model.
Additionally, our lightweight \modelname-B (172M), surpasses DiT-XL (675M)~\cite{dit} by 0.55 FID while achieving an inference speed of 9.8 images per second—20$\times$ faster than DiT-XL (0.5 images per second). Detailed speed comparison can be found in Appendix \ref{sec:speed}.



\noindent\textbf{ImageNet-512.}
In~\tabref{tab:512}, we report the performance of \modelname on ImageNet-512.
Similarly, \modelname-L sets a new state-of-the-art FID of 1.70, outperforming the diffusion based DiT-XL/2~\cite{dit} and DiMR-XL/3R~\cite{liu2024alleviating} by a large margin of 1.34 and 1.19 FID, respectively.
Additionally, \modelname-L also surpasses the previous best autoregressive model VAR-d36~\cite{var} and flow-matching-based REPA~\cite{yu2024representation} by 0.93 and 0.38 FID, respectively.




\noindent\textbf{Qualitative Results.}
\figref{fig:qualitative} presents samples generated by \modelname (trained on ImageNet) at 512$\times$512 and 256$\times$256 resolutions. These results highlight \modelname's ability to produce high-fidelity images with exceptional visual quality.

\begin{figure*}
    \centering
    \vspace{-6pt}
    \includegraphics[width=1\linewidth]{figures/qualitative.pdf}
    \caption{\textbf{Generated Samples.} \modelname generates high-quality images at resolutions of 512$\times$512 (1st row) and 256$\times$256 (2nd and 3rd row).
    }
    \label{fig:qualitative}
\end{figure*}

\subsection{Ablation Studies}
\label{sec:ablation}
In this section, we conduct ablation studies using \modelname-B, trained for 400 epochs to efficiently iterate on model design.

\noindent\textbf{Prediction Entity X.}
The proposed \modelname extends next-token prediction to next-X prediction. In~\tabref{tab:X}, we evaluate different designs for the prediction entity X, including an individual patch token, a cell (a group of surrounding tokens), a subsample (a non-local grouping), a scale (coarse-to-fine resolution), and an entire image.

Among these variants, cell-based \modelname achieves the best performance, with an FID of 2.48, outperforming the token-based \modelname by 1.03 FID and surpassing the second best design (scale-based \modelname) by 0.42 FID. Furthermore, even when using standard prediction entities such as tokens, subsamples, images, or scales, \modelname consistently outperforms existing methods while requiring significantly fewer parameters. These results highlight the efficiency and effectiveness of \modelname across diverse prediction entities.






\begin{table}[]
    \centering
    \scalebox{0.92}{
    \begin{tabular}{c|c|c|c|c}
        model & \makecell[c]{prediction\\entity} & \#params & FID$\downarrow$ & IS$\uparrow$\\
        \shline
        LlamaGen-L~\cite{llamagen} & \multirow{2}{*}{token} & 343M & 3.80 &248.3\\
        \modelname-B& & 172M&3.51&251.4\\
        \hline
        PAR-L~\cite{par} & \multirow{2}{*}{subsample}& 343M & 3.76 & 218.9\\
        \modelname-B&  &172M& 3.58&231.5\\
        \hline
        DiT-L/2~\cite{dit}& \multirow{2}{*}{image}& 458M&5.02&167.2 \\
         \modelname-B& & 172M&3.13&253.4 \\
        \hline
        VAR-$d16$~\cite{var} & \multirow{2}{*}{scale} & 310M&3.30 &274.4\\
        \modelname-B& &172M&2.90&262.8\\
        \hline
        \baseline{\modelname-B}& \baseline{cell} & \baseline{172M}&\baseline{2.48}&\baseline{269.2} \\
    \end{tabular}
    }
    \caption{\textbf{Ablation on Prediction Entity X.} Using cells as the prediction entity outperforms alternatives such as tokens or entire images. Additionally, under the same prediction entity, \modelname surpasses previous methods, demonstrating its effectiveness across different prediction granularities. }%
    \label{tab:X}
\end{table}

\noindent\textbf{Cell Size.}
A prediction entity cell is formed by grouping spatially adjacent $k\times k$ tokens, where a larger cell size incorporates more tokens and thus captures a broader context within a single prediction step.
For a $256\times256$ input image, the encoded continuous latent representation has a spatial resolution of $16\times16$. Given this, the image can be partitioned into an $m\times m$ grid, where each cell consists of $k\times k$ neighboring tokens. As shown in~\tabref{tab:cell}, we evaluate different cell sizes with $k \in \{1,2,4,8,16\}$, where $k=1$ represents a single token and $k=16$ corresponds to the entire image as a single entity. We observe that performance improves as $k$ increases, peaking at an FID of 2.48 when using cell size $8\times8$ (\ie, $k=8$). Beyond this, performance declines, reaching an FID of 3.13 when the entire image is treated as a single entity.
These results suggest that using cells rather than the entire image as the prediction unit allows the model to condition on previously generated context, improving confidence in predictions while maintaining both rich semantics and local details.





\begin{table}[t]
    \centering
    \scalebox{0.98}{
    \begin{tabular}{c|c|c|c}
    cell size ($k\times k$ tokens) & $m\times m$ grid & FID$\downarrow$ & IS$\uparrow$ \\
       \shline
       $1\times1$ & $16\times16$ &3.51&251.4 \\
       $2\times2$ & $8\times8$ & 3.04& 253.5\\
       $4\times4$ & $4\times4$ & 2.61&258.2 \\
       \baseline{$8\times8$} & \baseline{$2\times2$} & \baseline{2.48} & \baseline{269.2}\\
       $16\times16$ & $1\times1$ & 3.13&253.4  \\
    \end{tabular}
    }
    \caption{\textbf{Ablation on the cell size.}
    In this study, a $16\times16$ continuous latent representation is partitioned into an $m\times m$ grid, where each cell consits of $k\times k$ neighboring tokens.
    A cell size of $8\times8$ achieves the best performance, striking an optimal balance between local structure and global context.
    }
    \label{tab:cell}
\end{table}



\begin{table}[t]
    \centering
    \scalebox{0.95}{
    \begin{tabular}{c|c|c|c}
      previous cell & noise time step &  FID$\downarrow$ & IS$\uparrow$ \\
       \shline
       clean & $t_i=0, \forall i<n$& 3.45& 243.5\\
       increasing noise & $t_1<t_2<\cdots<t_{n-1}$& 2.95&258.8 \\
       decreasing noise & $t_1>t_2>\cdots>t_{n-1}$&2.78 &262.1 \\
      \baseline{random noise}  & \baseline{no constraint} &\baseline{2.48} & \baseline{269.2}\\
    \end{tabular}
    }
    \caption{
    \textbf{Ablation on Noisy Context Learning.}
    This study examines the impact of noise time steps ($t_1, \cdots, t_{n-1} \subset [0, 1]$) in previous entities ($t=0$ represents pure Gaussian noise).
    Conditioning on all clean entities (the ``clean'' variant) results in suboptimal performance.
    Imposing an order on noise time steps, either ``increasing noise'' or ``decreasing noise'', also leads to inferior results. The best performance is achieved with the "random noise" setting, where no constraints are imposed on noise time steps.
    }
    \label{tab:ncl}
\end{table}


\noindent\textbf{Noisy Context Learning.}
During training, \modelname employs Noisy Context Learning (NCL), predicting $X_n$ by conditioning on all previous noisy entities, unlike Teacher Forcing.
The noise intensity of previous entities is contorlled by noise time steps $\{t_1, \dots, t_{n-1}\} \subset [0, 1]$, where $t=0$ corresponds to pure Gaussian noise.
We analyze the impact of NCL in~\tabref{tab:ncl}.
When conditioning on all clean entities (\ie, the ``clean'' variant, where $t_i=0, \forall i<n$), which is equivalent to vanilla AR (\ie, Teacher Forcing), the suboptimal performance is obtained.
We also evaluate two constrained noise schedules: the ``increasing noise'' variant, where noise time steps increase over AR steps ($t_1<t_2< \cdots < t_{n-1}$), and the `` decreasing noise'' variant, where noise time steps decrease ($t_1>t_2> \cdots > t_{n-1}$).
While both settings improve over the ``clean'' variant, they remain inferior to our final ``random noise'' setting, where no constraints are imposed on noise time steps, leading to the best performance.




        

\section{Conclusions}
\label{sec:conclusions}

In this paper, we introduce a novel sketch-to-image translation technique that uses a learnable lightweight mapping network (LCTN) for latent code translation from sketch to image domain, followed by $k$ forward diffusion and $T$ backward denoising steps through a pre-trained text-to-image LDM. We show that by selecting an optimal value for $k \sim [1, T]$ near the upper threshold ($k \approx T$, $k < T$), it is possible to generate highly detailed photorealistic images that closely resemble the intended structures in the given sketches. Our experiments demonstrate that the proposed technique outperforms the existing methods in most visual and analytical comparisons across multiple datasets. Additionally, we show that the proposed method retains structural consistency across different visual styles, allowing photorealistic style manipulation in the generated images.


\section{Acknowledgements}
We thank Guy Tevet and Oren Katzir for their valuable insights and engaging discussions. We also thank Yuval Alaluf, Elad Richardson, and Sagi Polaczek for providing feedback on early versions of our manuscript. 
This work was partially supported by Joint NSFC-ISF Research Grant no. 3077/23 and Isf 3441/21.

\begin{figure*}
    \centering
    \includegraphics[width=0.87\linewidth]{figs/results/swiftsketch_seen.pdf}
    \vspace{-0.2cm}
    \caption{Sketches generated by SwiftSketch for seen categories, using input images not included in the training data.}
    \label{fig:qualitative-swift-seen}
\end{figure*}



\begin{figure*}
    \centering
    \includegraphics[width=0.87\linewidth]{figs/results/swiftsketch_unseen.pdf}
    \vspace{-0.2cm}
    \caption{Sketches generated by SwiftSketch for unseen categories.}
    \label{fig:qualitative-swift-unseen}
\end{figure*}


\begin{figure*}[t]
    \centering
    \setlength{\tabcolsep}{1pt}
    {\small
    \begin{tabular}{c c c |c c c |c c c}
         Input & ControlSk. &  CLIPasso  & Input  & ControlSk. &  CLIPasso & Input  & ControlSk. &  CLIPasso  \\
        \includegraphics[width=0.1\linewidth]{figs/figure_comparison_opt/dolphin_331_input.png} &
        \includegraphics[width=0.1\linewidth]{figs/figure_comparison_opt/dolphin_331_control_sketch.png} &
       \includegraphics[width=0.1\linewidth]{figs/figure_comparison_opt/dolphin_331_CLIPasso.png} &
       \includegraphics[width=0.1\linewidth]{figs/figure_comparison_opt/bear_3120_input.png} &
       \includegraphics[width=0.1\linewidth]{figs/figure_comparison_opt/bear_3120_control_sketch.png} &
       \includegraphics[width=0.1\linewidth]{figs/figure_comparison_opt/bear_3120_CLIPasso.png} &

        \includegraphics[width=0.1\linewidth]{figs/figure_comparison_opt/dog_2184_input.png} &
        \includegraphics[width=0.1\linewidth]{figs/figure_comparison_opt/dog_2184_control_sketch.png} &
       \includegraphics[width=0.1\linewidth]{figs/figure_comparison_opt/dog_2184_CLIPasso.png}  \\
       Input & ControlSk. &  CLIPasso  & Input  & ControlSk. &  CLIPasso & Input  & ControlSk. &  CLIPasso  \\
       \includegraphics[width=0.1\linewidth]{figs/figure_comparison_opt/parrot_221_input.png} &
        \includegraphics[width=0.1\linewidth]{figs/figure_comparison_opt/parrot_221_control_sketch.png} &
       \includegraphics[width=0.1\linewidth]{figs/figure_comparison_opt/parrot_221_CLIPasso.png} &
        \includegraphics[width=0.1\linewidth]{figs/figure_comparison_opt/helicopter_957_input.png} &
        \includegraphics[width=0.1\linewidth]{figs/figure_comparison_opt/helicopter_957_control_sketch.png} &
       \includegraphics[width=0.1\linewidth]{figs/figure_comparison_opt/helicopter_957_CLIPasso.png}  &
       \includegraphics[width=0.1\linewidth]{figs/figure_comparison_opt/broccoli_695_input.png} &
        \includegraphics[width=0.1\linewidth]{figs/figure_comparison_opt/broccoli_695_control_sketch.png} &
       \includegraphics[width=0.1\linewidth]{figs/figure_comparison_opt/broccoli_695_CLIPasso.png} \\
        Input & ControlSk. &  CLIPasso  & Input  & ControlSk. &  CLIPasso & Input  & ControlSk. &  CLIPasso  \\
      

         \includegraphics[width=0.1\linewidth]{figs/figure_comparison_opt/camel_176_input.png} &
        \includegraphics[width=0.1\linewidth]{figs/figure_comparison_opt/camel_176_control_sketch.png} &
       \includegraphics[width=0.1\linewidth]{figs/figure_comparison_opt/camel_176_CLIPasso.png}  &
        \includegraphics[width=0.1\linewidth]{figs/figure_comparison_opt/backpack_45_input.png} &
        \includegraphics[width=0.1\linewidth]{figs/figure_comparison_opt/backpack_45_control_sketch.png} &
       \includegraphics[width=0.1\linewidth]{figs/figure_comparison_opt/backpack_45_CLIPasso.png}&  
       \includegraphics[width=0.1\linewidth]{figs/figure_comparison_opt/duck_45_input.png} &
        \includegraphics[width=0.1\linewidth]{figs/figure_comparison_opt/duck_45_control_sketch.png} &
       \includegraphics[width=0.1\linewidth]{figs/figure_comparison_opt/duck_45_CLIPasso.png} \\
        Input & ControlSk. &  CLIPasso  & Input  & ControlSk. &  CLIPasso & Input  & ControlSk. &  CLIPasso  \\

        \includegraphics[height=0.09\linewidth]{figs/figure_comparison_opt/406124.png} &
        \includegraphics[width=0.1\linewidth]{figs/figure_comparison_opt/406124_control_sketch.png} &
       \includegraphics[width=0.1\linewidth]{figs/figure_comparison_opt/406124_CLIPasso.png}  &
       \raisebox{0.23\height}{\includegraphics[width=0.1\linewidth]{figs/figure_comparison_opt/584196.png}} &
        \includegraphics[width=0.1\linewidth]{figs/figure_comparison_opt/584196_control_sketch.png} &
       \includegraphics[width=0.1\linewidth]{figs/figure_comparison_opt/584196_CLIPasso.png} &
        \raisebox{0.15\height}{\includegraphics[height=0.08\linewidth]{figs/figure_comparison_opt/sheep_31.png}} &
        \includegraphics[width=0.1\linewidth]{figs/figure_comparison_opt/sheep_31_control_sketch.png} &
       \includegraphics[width=0.1\linewidth]{figs/figure_comparison_opt/sheep_31_CLIPasso.png}  \\

       
    
    \end{tabular}
    }
    \vspace{-0.3cm}
    \caption{Comparison of ControlSketch with CLIPasso \cite{vinker2022clipasso}. ControlSketch captures fine details (e.g., camel and bear), avoids artifacts in small objects (e.g., dog and duck), and handles challenging inputs effectively (last row).}
    \label{fig:comparison_opt}
\end{figure*}

\begin{figure*}
    \centering
    \includegraphics[width=0.95\linewidth]{figs/reifnment_swift.pdf}
    \vspace{-0.2cm}
    \caption{Effect of the refinement network. The output sketches from the diffusion model may contain slight noise, which the refinement network addresses by performing an additional cleaning step. However, this process can sometimes reduce details in the sketch (see Limitations).}
    \label{fig:refine_ablation}
\end{figure*}


{
    \small
    \bibliographystyle{ieee_fullname}
    \bibliography{main}
}

\clearpage
\appendix
\setcounter{page}{1}

\newpage
\twocolumn[
\centering
\Large
\textbf{\thetitle}\\
\vspace{2em}Supplementary Material \\
\vspace{1.0em}
] %

   

\part{}
\vspace{-20pt}
\parttoc





\vspace{-0.2cm}









\begin{figure}
    \centering
    \includegraphics[width=1\linewidth]{figs_sup/data_sample_cat.pdf}
    \caption{An example from the ControlSketch dataset, which includes the input image, object mask, attention map, and the corresponding sketch generated using ControlSketch.}
    \label{fig:data_sample}
\end{figure}

\section{ControlSketch Dataset}
The data generation process begins with generating images, followed by creating corresponding sketches using the ControlSketch framework, as described in section 4.2 in the main paper.
We generate images using the SDXL model \cite{podell2023sdxlimprovinglatentdiffusion} with the following prompt:
\textit{``A highly detailed wide-shot image of one $<c>$, set against a plain mesmerizing background. Center.''}, where $c$ is the class label.
Additionally, a negative prompt, \textit{``close up, few, multiple,''} is applied to ensure images depict a single object in a clear and high-quality pose. The generated images are of size $1024 \times 1024$. An example output image for the class ``cat'' is shown in \Cref{fig:data_sample}.

During the image generation process, we retain cross attention maps of the class label token extracted from internal layers of the model for future use. To isolate the object, we employ the BRIA Background Removal v1.4 Model  \cite{briaRMBG} to extract an object mask. After generating the image, we use BLIP2 \cite{li2023blip2bootstrappinglanguageimagepretraining} to extract the image caption that provides context beyond the object’s class. For example, for the image in \cref{fig:data_sample}, the caption  describe the cat as sitting, offering richer semantic information.
The sketches are generated using the ControlSketch method with 32 strokes. These strokes are subsequently arranged according to our stroke-sorting schema. The final SVG files contains the sorted strokes.
We use the Hugging Face implementation of SDXL version 1.0 \cite{podell2023sdxlimprovinglatentdiffusion} with its default parameters. Generating a single image with SDXL takes approximately 10 seconds, while sketch generation using the ControlSketch method on an NVIDIA RTX3090 GPU requires about 10 minutes.

Our dataset comprises 35,000 pairs of images and their corresponding sketches in SVG format, spanning 100 object categories. These categories are derived by combining common ones from existing sketch datasets {\cite{Mukherjee2023SEVALS,SketchRNN,SketchyCOCO2020,Eitz2012HowDH} with additional, unique categories such as astronaut, robot and sculpture. These unique categories are not present in prior datasets, highlighting the advantages of a synthetic data approach. The full list of categories is available in Table~\ref{tab:ControlSketch_dataset}.
All the sketches in our data were manually verified, we filtered very few generated images with artifacts that caused artifacts in the generated sketches.
The 15 categories used in training are: angel, bear, car, chair, crab, fish, rabbit, sculpture, astronaut, bicycle, cat, dog, horse, robot, woman. For each of these categories we generated 1200 image-sketch pairs, where 1000 samples are used for training and the rest for testing.
For the rest of 85 categories we created 200 samples per class. 
We show 78 random samples from each class of the training data in \Cref{fig:dog,fig:angle,fig:astronaut,fig:bicycle,fig:chair,fig:bear,fig:car,fig:cat,fig:horse,fig:chair,fig:crab,fig:fish,fig:rabbit,fig:Sculpture,fig:robot,fig:woman}, and 100 random samples from the entire dataset (one of each class) in \Cref{fig:100ControlSketch}.
Since the entire data creation pipeline is fully automated, we continuously extend the dataset and plan to release the code to enable future work in this area.

\begin{table}[h!]
\centering
\setlength{\tabcolsep}{2pt}
\begin{tabular}{|c|c|c|c|c|}
\midrule
\small
airplane & alarm clock & angel & astronaut & backpack \\ \midrule
bear & bed & bee & beer & bicycle \\ \midrule
boat & broccoli & burger & bus & butterfly \\ \midrule
cabin & cake & camel & camera & candle \\ \midrule
car & carrot & castle & cat & cell phone \\ \midrule
chair & chicken & child & cow & crab \\ \midrule
cup & deer & doctor & dog & dolphin \\ \midrule
dragon & drill & duck & elephant & fish \\ \midrule
flamingo & floor lamp & flower & fork & giraffe \\ \midrule
goat & hammer & hat & helicopter & horse \\ \midrule
house & ice cream & jacket & kangaroo & kimono \\ \midrule
laptop & lion & lobster & margarita & mermaid \\ \midrule
motorcycle & mountain & octopus & parrot & pen \\ \midrule
\begin{tabular}[c]{@{}c@{}}pickup \\ truck \end{tabular}  & pig & purse & quiche & rabbit \\ \midrule
robot & sandwich & scissors & sculpture & shark \\ \midrule
sheep & spider & squirrel & strawberry & sword \\ \midrule
t-shirt & table & teapot & television & tiger \\ \midrule
tomato & train & tree & truck & umbrella \\ \midrule
vase & waffle & watch & whale & wine bottle \\ \midrule
woman & yoga & zebra & \begin{tabular}[c]{@{}c@{}}The Eiffel \\ Tower \end{tabular} & book \\ \midrule
\end{tabular}
\caption{The 100 categories of the ControlSketch dataset.}
\label{tab:ControlSketch_dataset}
\end{table}



\begin{figure*}
    \centering
    \includegraphics[width=1\linewidth]{figs_sup/data/all1.pdf}
    \caption{100 random samples of sketches generated with ControlSketch.}
    \label{fig:100ControlSketch}
\end{figure*}




\section{ControlSketch Method}
\paragraph{Technical details}
In the ControlSketch optimization, we leverage the pretrained depth ControlNet model \cite{controlnet2023} to compute the SDS loss. The Adam optimizer is employed with a learning rate of 0.8. The optimization process runs for 2000 iterations, taking approximately 10 minutes to generate a single sketch on an RTX 3090 GPU.  However, after 700 iterations most images already yield a clearly identifiable sketch.

\paragraph{Strokes initialization}
The number of areas, $k$, is defined as the rounded square root of the total number of strokes $n$ (for our default number of strokes, 32, $k$ is set to 6). 
Our initialization technique combines between saliency and full coverage of the sketch, which we find to be important when the SDS loss is applied with our spatial control. In \Cref{fig:initalization_vis} we demonstrate how the final sketches will look like when applied with and without our enhances initialization, where the default case is defined based on the attention map as was proposed in CLIPasso \cite{vinker2022clipasso}. As seen, our approach ensures comprehensive object coverage while emphasizing critical areas, resulting in visually effective and recognizable sketches without omitting essential elements. For example, in the lion image, initializing strokes based solely on saliency results in almost all strokes focusing on the lion's head. Consequently, the final sketch omits significant portions of the lion's body.


\paragraph{Spatial control}
The ControlNet model receives two inputs as conditions: the text prompt and the depth condition. The balance between these conditions which is determined  by the conditioning scale parameter influences the final sketch attributes. We found that a conditioning scale of 1.5 provides the best results, effectively maintaining both semantic and geometric attributes of the subject. 

The depth ControlNet model used in the control SDS loss can be replaced with any other ControlNet model, along with the extraction of the appropriate condition from the input image. Different ControlNet models influence the style and attributes of the final sketch. Examples of different sketches generated with different ControlNet models and conditions are shown in Figure~\ref{fig:deifferent_conditions}.


 


\begin{figure}[t]
    \centering
    \setlength{\tabcolsep}{0pt}
    {\small
        \begin{tabular}{c |c c c}
         \midrule
        Input &  Depth & Scribble & Segmentation  \\
         \midrule
        \includegraphics[width=0.25\linewidth]{figs_sup/cond_inputs/masked_bull.png} &
        \includegraphics[width=0.25\linewidth]{figs_sup/cond_inputs/b_depth2.png} & 
        \includegraphics[width=0.25\linewidth]{figs_sup/cond_inputs/b_edge2.png} & 
        \includegraphics[width=0.25\linewidth]{figs_sup/cond_inputs/b_seg.png} \\

        &
        \includegraphics[width=0.25\linewidth]{figs_sup/different_conditions/bull_depth.png} &
        \includegraphics[width=0.25\linewidth]{figs_sup/different_conditions/bull_scribble.png} & \includegraphics[width=0.25\linewidth]{figs_sup/different_conditions/bull_segmentation.png} \\

         \midrule
        Input &  Depth & Scribble & Segmentation  \\
         \midrule

         \includegraphics[width=0.25\linewidth]{figs_sup/different_conditions/flamingo} &
        \includegraphics[width=0.25\linewidth]{figs_sup/cond_inputs/f_depth.png} &
        \includegraphics[width=0.25\linewidth]{figs_sup/cond_inputs/f_edge.png}  & 
        \includegraphics[width=0.25\linewidth]{figs_sup/cond_inputs/f_seg.png} \\

         &
        \includegraphics[width=0.25\linewidth]{figs_sup/different_conditions/flamingo_depth.png} &
        \includegraphics[width=0.25\linewidth]{figs_sup/different_conditions/flamingo_scribble.png}  & 
        \includegraphics[width=0.25\linewidth]{figs_sup/different_conditions/flamingo_segmentation.png} \\
       
    \end{tabular}
    }
    \caption{Examples of sketches generated by ControlSketch using different ControlNet models, alongside their respective conditions which influence the style and attributes of the final sketches.}
    \label{fig:deifferent_conditions}
\end{figure}




\begin{figure}
    \centering
    \includegraphics[width=0.6\linewidth]{figs_sup/initalization_vis.pdf}
    \caption{ Strokes initialization in the ControlSketch method. The "Saliency" column demonstrates the result when strokes are initialized based solely on the attention map (following common practive \cite{vinker2022clipasso}), often leading to an overemphasis on critical regions, such as the lion's head, at the expense of other important parts like the body. The "Saliency + coverage" column showcases our enhanced initialization method, which combines saliency with full object coverage, ensuring both essential details and global object representation are maintained, resulting in complete and recognizable sketches.}
    \label{fig:initalization_vis}
\end{figure}

\section{SwiftSketch}
Our implementation is built on the MDM codebase \cite{tevet2023human}.
Our model consists of 8 self- and cross-attention layers. It was trained with a batch size of 32, a learning rate of $5 \times 10^{-5}$, for 400,000 steps. The refinement network shares the same architecture as our diffusion model and is initialized with its final weights. The timestep condition is fixed at 0. We train the refinement network on the diffusion output sketches from the training dataset, using only the LPIPS loss between the network’s rendered output sketch and the target rendered sketch, as we found it resulting in more polished and visually improved final sketches. The refinement network was trained for 30,000 steps with a learning rate of $5 \times 10^{-6}$.

For training, We scaled the ground truth $(x, y)$ coordinates to the range [-2, 2]. Our experiments revealed that a scaling factor of 2 outperformed the standard value of 1.0 which is used in image generation tasks.
To extract input image features for our model, the image is processed using a pretrained CLIP ResNet model \cite{Radfordclip}, with features extracted from its fourth layer. These features are subsequently refined through three convolutional layers to capture additional spatial details. Each patch embedding is further refined using three linear layers, enhancing feature learning and aligning dimensions for compatibility with the model. The resulting feature representation is seamlessly integrated into the generation process via a cross-attention mechanism.

To encourage the diffusion model to focus on fine details, we adjust the noise scheduler to perturb the signal more subtly for small timesteps, by reducing the exponent in the standard cosine noise schedule proposed in \cite{Nichol2021ImprovedDD} from 2 to 0.4.
Our model $M_{\theta}$ was trained using classifier-free guidance so during inference, we enhance fidelity to the input image by extrapolating the following variants using s= 2.5:
\begin{equation}
    M_{\theta_s}(s^t, t, I) = M_{\theta}(s^t, t, \emptyset) + s \cdot \big( M_{\theta}(s^t, t, I) - M_{\theta}(s^t, t, \emptyset) \big)
\end{equation}.



Figure~\ref{fig:swiftsketch_random} showcases 100 random SwiftSketch samples across all categories in the ControlSketch dataset. The last three rows correspond to classes our model was trained on, while the remaining rows are unseen classes. Each sketch is generated in under a second. The results demonstrate that our model generalizes well to unseen categories, producing sketches with high fidelity to the input images. However, in some cases, high-level details are absent, and the sketch's category label can be difficult to identify. More examples for unseen classes are shown in Figure~\ref{fig:swiftskwtch_unseen1}, Figure~\ref{fig:swiftskwtch_unseen2} and Figure~\ref{fig:swiftskwtch_unseen3}


\section{Qualitative Comparison}
Figure~\ref{fig:comparison_train} and Figure~\ref{fig:comparison_test}  show more examples of qualitative comparison of seen and unseen categories. Input images are shown on the left. From left to right, the sketches are generated using  PhotoSketching \cite{Li2019PhotoSketchingIC}, Chan et al. \cite{Chan2022LearningTG} (in anime style), InstantStyle \cite{Wang2024InstantStyleFL}, and CLIPasso \cite{vinker2022clipasso}. On the right are the resulting sketches from our proposed methods, ControlSketch and SwiftSketch.


\begin{figure}
    \centering
    \includegraphics[width=1\linewidth]{figs_sup/strokes_number_sort.jpeg}
    \caption{Stroke Order Visualization. SwiftSketch generated sketches are visualized progressively, with the stroke count shown on top. The first row for each example is with our sorting technique (w), while the second row omits it (w/o)}
    \label{fig:strokes_number_sort}
\end{figure}



\section{Quantitative Evaluation}
In this section, we present the details of the user study conducted to compare our new optimization method, ControlSketch, with the state-of-the-art optimization method for object sketching, CLIPasso.
We selected 24 distinct categories for the user study: 16 categories from our ControlSketch dataset, and 8 categories from the SketchyCOCO dataset. For each category, we randomly sampled one image. Participants were presented with the input image alongside two sketches—one generated by CLIPasso and the other by ControlSketch—displayed in random order. We asked participants two questions for each pair of sketches: 1. Which sketch more effectively depicts the input image? 2. Which sketch is of higher quality? Participants were required to choose one sketch for each question. A total of 40 individuals participated in the survey. The results are as follows: For the ControlSketch dataset, 87\% of participants chose ControlSketch for the first question, and 88\% for the second question. For the SketchyCOCO dataset—which is more challenging due to its low-resolution images and difficult lighting conditions—90\% chose ControlSketch for the first question, and 93\% for the second question.
These results highlight the significant advantages of ControlSketch over CLIPasso across diverse categories and datasets.


\section{Ablation}
Figure~\ref{fig:comparison_refine} presents a comparison of results with and without the refinement step in the SwiftSketch pipeline. As can be seen, the final output sketches generated by the denoising process of our diffusion model may still retain slight noise. Incorporating the refinement stage significantly enhances the quality and cleanliness of the sketches
Figure~\ref{fig:strokes_number_sort} illustrates the impact of the stroke sorting technique used for training. Early strokes effectively capture the object’s contour and key features, while later strokes refine the details. With sorting, the object is significantly more recognizable with fewer strokes compared to the case without sorting.



\begin{figure}
    \centering
    \includegraphics[width=1\linewidth]{figs_sup/limitation_supp_new.jpg}
    \caption{Limitations of SwiftSketch. (a) When trained solely on masked object images, SwiftSketch struggles to generate accurate sketches for complex scenes. As shown, it incorrectly assigns strokes to the image frame instead of capturing the scene's key elements. (b) During the refinement stage, fine details particularly facial features are often lost, resulting in oversimplified representations. (c) Sketches may appear unrecognizable. }
    \label{fig:limitations1}
\end{figure}


\section{Limitations}
SwiftSketch, which was trained only on masked object images, faces challenges in handling complex scenes. When provided with a scene image, as illustrated in Figure~\ref{fig:limitations1}(a), SwiftSketch struggles to generate accurate sketches, often misplacing strokes onto the image frame instead of capturing key elements of the scene. Another significant limitation is its tendency to omit fine details, particularly facial features, leading to oversimplified representations, as shown in Figure~\ref{fig:limitations1}(b). In some cases, sketches may appear unrecognizable, as shown in Figure~\ref{fig:limitations1}(c).





\begin{figure*}[t]
    \centering
    \setlength{\tabcolsep}{1pt}
    {\small
    \begin{tabular}{c}
         Seen Classes  \\
        \includegraphics[width=0.6\linewidth]{figs_sup/refine_vs_without_new/train4.pdf}   \\
         Unseen Classes  \\
        \includegraphics[width=0.6\linewidth]{figs_sup/refine_vs_without_new/test2.pdf}   \\
    
    \end{tabular}
    }
    \caption{Comparison of SwiftSketch sketches with (right) and without (left) the refinement step. This highlights the critical role of the refinement network in significantly improving the quality of the generated sketches and reducing noise}
    \label{fig:comparison_refine}
\end{figure*}



\begin{figure*}
    \centering
    \includegraphics[width=0.8\linewidth]{figs_sup/ramdom_refine/test.pdf}
    \caption{100 random sapmels of SwiftSketch sketches. The last three rows are seen classes, while the remaining rows are unseen classes}
    \label{fig:swiftsketch_random}
\end{figure*}

\begin{figure*}
    \centering
    \includegraphics[width=1\linewidth]{figs_sup/swiftsketch_unseen_6A.pdf}
    \caption{ Sketches generated by SwiftSketch for unseen categories.}
    \label{fig:swiftskwtch_unseen1}
\end{figure*}

\begin{figure*}
    \centering
    \includegraphics[width=1\linewidth]{figs_sup/swiftsketch_unseen_6b.pdf}
    \caption{ Sketches generated by SwiftSketch for unseen categories.}
    \label{fig:swiftskwtch_unseen2}
\end{figure*}

\begin{figure*}
    \centering
    \includegraphics[width=1\linewidth]{figs_sup/swiftsketch_unseen_6c.pdf}
    \caption{ Sketches generated by SwiftSketch for unseen categories.}
    \label{fig:swiftskwtch_unseen3}
\end{figure*}




\begin{figure*}
    \centering
    \includegraphics[trim=0cm 0.2cm 0 0cm,clip,width=0.8\linewidth]{figs_sup/comparison_images/titles.pdf}
    \includegraphics[trim=0cm 0cm 0 3.2cm,clip,width=0.8\linewidth]{figs_sup/comparison_images/train2.pdf}
    \caption{Qualitative comparison, seen categories}
    \label{fig:comparison_train}
\end{figure*}

\begin{figure*}
    \centering
    \includegraphics[trim=0cm 0.2cm 0 0cm,clip,width=0.8\linewidth]{figs_sup/comparison_images/titles.pdf}
    \includegraphics[trim=0cm 0cm 0 3.2cm,clip,width=0.8\linewidth]{figs_sup/comparison_images/test.pdf}
    \caption{Qualitative comparison, unseen categories}
    \label{fig:comparison_test}
\end{figure*}



 
\begin{figure*}
    \centering
    \includegraphics[width=1\linewidth]{figs_sup/data/dog.pdf}
    \caption{Dog - SwiftSketch training data examples }
    \label{fig:dog}
\end{figure*}

\begin{figure*}
    \centering
    \includegraphics[width=1\linewidth]{figs_sup/data/horse.pdf}
    \caption{Horse - SwiftSketch training data examples }
    \label{fig:horse}
\end{figure*}

\begin{figure*}
    \centering
    \includegraphics[width=1\linewidth]{figs_sup/data/cat.pdf}
    \caption{Cat - SwiftSketch training data examples}
    \label{fig:cat}
\end{figure*}

\begin{figure*}
    \centering
    \includegraphics[width=1\linewidth]{figs_sup/data/angel.pdf}
    \caption{Angel - SwiftSketch training data examples}
    \label{fig:angle}
\end{figure*}

\begin{figure*}
    \centering
    \includegraphics[width=1\linewidth]{figs_sup/data/astronaut.pdf}
    \caption{Astronaut - SwiftSketch training data examples}
    \label{fig:astronaut}
\end{figure*}

\begin{figure*}
    \centering
    \includegraphics[width=1\linewidth]{figs_sup/data/bear.pdf}
    \caption{Bear - SwiftSketch training data examples}
    \label{fig:bear}
\end{figure*}

\begin{figure*}
    \centering
    \includegraphics[width=1\linewidth]{figs_sup/data/bicycle.pdf}
    \caption{Bicycle - SwiftSketch training data examples}
    \label{fig:bicycle}
\end{figure*}

\begin{figure*}
    \centering
    \includegraphics[width=1\linewidth]{figs_sup/data/car.pdf}
    \caption{Car - SwiftSketch training data examples}
    \label{fig:car}
\end{figure*}

\begin{figure*}
    \centering
    \includegraphics[width=1\linewidth]{figs_sup/data/chair.pdf}
    \caption{Chair - SwiftSketch training data examples }
    \label{fig:chair}
\end{figure*}

\begin{figure*}
    \centering
    \includegraphics[width=1\linewidth]{figs_sup/data/crab.pdf}
    \caption{Crab - SwiftSketch training data examples}
    \label{fig:crab}
\end{figure*}

\begin{figure*}
    \centering
    \includegraphics[width=1\linewidth]{figs_sup/data/fish.pdf}
    \caption{Fish - SwiftSketch training data examples}
    \label{fig:fish}
\end{figure*}

\begin{figure*}
    \centering
    \includegraphics[width=1\linewidth]{figs_sup/data/rabbit.pdf}
    \caption{Rabbit - SwiftSketch training data examples}
    \label{fig:rabbit}
\end{figure*}

\begin{figure*}
    \centering
    \includegraphics[width=1\linewidth]{figs_sup/data/sculpture.pdf}
    \caption{:Sculpture - SwiftSketch training data examples }
    \label{fig:Sculpture}
\end{figure*}

\begin{figure*}
    \centering
    \includegraphics[width=1\linewidth]{figs_sup/data/robot.pdf}
    \caption{Robot - SwiftSketch training data examples }
    \label{fig:robot}
\end{figure*}


\begin{figure*}
    \centering
    \includegraphics[width=1\linewidth]{figs_sup/data/woman.pdf}
    \caption{Woman - SwiftSketch training data examples }
    \label{fig:woman}
\end{figure*}





\end{document}
