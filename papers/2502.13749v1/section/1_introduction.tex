\section{Introduction}
\label{xhw_study::sec::introduction}

At the core of the digital revolution are microchips, tiny electronic devices that store and process digital data.
A microchip contains numerous nanometer-sized electronic components (\eg, transistors) on a single piece of semiconductor material (typically silicon). 
These components work together to perform digital processing tasks such as executing computations (\acsp{CPU}, \acsp{GPU}), storing data (\acsp{SSD}, \acs{RAM}), or cryptographic and \acs{AI} acceleration.
Microchips serve as the basic building blocks in the electronic devices we use every day, including smartphones, vehicles, and medical equipment.

As microchips have become ubiquitous and are increasingly being used in critical areas, their geopolitical importance is growing. 
However, due to their rising complexity~\cite{Burrell2016How, DBLP:conf/re/MannCKSS23,apple2022m1ultra}, a globally distributed supply chain~\cite{weste2015cmos,lienig2020fundamentals}, and intentional concealment to protect trade secrets, microchips are often regarded as highly opaque.
This opacity can make it challenging to identify potential safety and security issues, thereby complicating efforts to build trust in these technologies.
Consequently, several concerns regarding microchips have not yet been resolved.
For instance, microchips are susceptible to attacks from a diverse range of adversaries. 
They can be manipulated through hardware Trojans~\cite{Adee2008Hunt}, particularly when employed in safety- and security-critical tasks such as encryption~\cite{DBLP:conf/crypto/KocherJJ99, DBLP:conf/host/DaRoltNFR11}. 
Similarly, previous studies have demonstrated how security issues in the hardware~\cite{Adee2008Hunt,Becker2013Stealthy,Lipp2018Meltdown} can impact the security of end-user devices~\cite{CVE-2023-38606}. 

In response to these concerns, numerous countries have introduced subsidies and regulations to bolster domestic microchip industries~\cite{uschips2022, euchips2022}.
These measures aim to secure production, promote innovation, and foster talent while, at the same time, addressing global supply chain vulnerabilities and security threats.
However, their primary goals are tied to geopolitical strategy and achieving or maintaining technological leadership, underscoring the high stakes in the global microchip race.

Despite the focus on industry and geopolitics, one crucial stakeholder often overlooked in these regulatory discussions is the end user. 
The question arises: should users be considered, and perhaps studied, as integral stakeholders in the microchip ecosystem? 
We think the question is worth exploring because end users are already constantly interacting with and relying on the proper functioning of microchips in their daily lives, albeit often unknowingly and indirectly. 
While end users may be familiar with the fact that the \acsp{CPU} within their computers are microchips, the application of microchips to other technologies and devices might be more hidden. 
Modern cars are built from hundreds of microchips, and smartphones and laptops contain dozens.
Microchips are also increasingly found in medical equipment like insulin pumps, pacemakers, and ventilators---technologies on which someone's life may depend.

We see the potential that improving end-users' understanding of microchips can lead to numerous benefits, such as making more informed product choices, which often start with functionality.
For instance, many people compare CPUs before purchasing a computer.
Some vendors even make microchips the centerpiece of their marketing, such as Apple with its A- and M-series microchips.
However, product choices can also be influenced by factors such as security, trustworthiness, and sustainability.
In this context, the (country of the) manufacturer, materials used, and power consumed during microchip production~\cite{DBLP:conf/hpca/GuptaKLTLW0W21,WILLIAMS20042,VILLARD201598} could become key considerations for product choice~\cite{LIN201211}.

Research in other contexts shows that limited understanding of technologies like the Internet~\cite{kang2015my}, Wi-Fi~\cite{klasnja2009wi}, or home computer security~\cite{wash2010folk} can lead to a false sense of security and inadequate protective practices. 
However, to the best of our knowledge, the academic community has yet to study end-user understanding of, information needs concerning, and trust in microchips.
In light of this gap, we seek to answer the following \acp{RQ}:
\begin{itemize}
    \item \textit{RQ1 [Understanding]} 
    How do end users currently understand microchips?
    \item \textit{RQ2 [Desiderata]} 
    What do end users value concerning and what do they desire to know about microchips?
    \item \textit{RQ3 [Information Needs]}  
    What factors shape end users' information needs when it comes to microchips?
\end{itemize}

To answer these \acp{RQ}, we conducted and evaluated an online survey with 250 end-user participants. 
Our key findings include:
\begin{itemize}
    \item \textbf{End-User Understanding of Microchips.} 
    Participants had a basic understanding of what microchips are and where they are used. 
    However, we also found several misconceptions, and participants mentioned little about the %broader societal impacts of microchips or 
    security and privacy implications of microchips.
    
    \item \textbf{Desirable Properties of Microchips.} 
    When prompted, participants rated cyber security and trustworthiness as their most valued objectives for microchips. 
    At the same time, participants rated safety, accountability, and ethical standards still as \enquote{very important} on average.
    
    \item \textbf{Factors Shaping End Users' Information Needs.} 
    Our participants indicated that they want to know more about microchips and are willing to invest time to that end. 
    The exact type of information they wished for depends on the microchip's specific application environment as well as the participant's affinity for technology interaction.
\end{itemize}

Finally, we discuss interesting patterns from our findings that call for further investigation. 
For example, based on our results, we find that the goals of ongoing political initiatives around microchips might not serve the needs of end users. 
Our study lays the foundation for future research to more thoroughly look into end users' mental models of microchips and design mechanisms that effectively convey information about microchips to end users.