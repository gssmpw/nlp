\section{Discussion}
\label{xhw_study::sec::discussion}

Below, we reflect on the fundamental questions of why end users need to understand more about microchips and the role of end users in the microchip ecosystem (\autoref{sec:why:understanding}). 
We then discuss our findings' implications for future research that promotes user understanding of microchips and microchip transparency (\autoref{sec:hci:implications}). 
Finally, we reflect on our work's policy implications considering regulatory efforts around microchips (\autoref{sec:policy:implications}).


\subsection{Do End Users Need to Understand More About Microchips?}
\label{sec:why:understanding}
Our study is motivated by the fact that microchips run the electronics of the world and are featured prominently in regulatory efforts, yet microchips remain largely opaque from the general public view.
Nonetheless, they play an increasingly vital role in security as they often form the root of trust in a system, \eg, as a cryptographic accelerator, \ac{HSM}, or \ac{TEE}.
In other domains and application areas, such as \acs{AI} and \ac{IoT} devices, we have seen concrete evidence that a lack of transparency causes security and trust issues~\cite{von2021transparency}. 
In contrast, end users are empowered to make more informed decisions with a better understanding of the system's inner structure and potential risks~\cite{emami2020ask,johnson2020impact}. 
Thus, we see the value of at least envisioning the integration of end users into the hardware ecosystem since their role is largely overlooked at the moment.
Our findings further underscore the necessity of helping end users understand more about microchips in a few ways. 

First, the need is supported by our participants' own preferences---76\% of participants indicated they would personally like to understand more about microchips, recognizing microchips' omnipresence in their daily lives and expressing particular interest in knowing more about microchip's functionality and interactions with the underlying system. 

Second, while our participants exhibited a basic understanding of what microchips are and where they are used, we found a lack of awareness regarding potential security and privacy concerns, the critical societal aspects of microchips, and misconceptions such as linking microchips to animal implants and conspiracy theories. 
Beliefs in such conspiracies can lead to hesitations in adopting new technologies and mistrust in government bodies.
In fact, during the COVID-19 pandemic, conspiracy theorists falsely claimed vaccines were used to implant microchips into people, which led to lower vaccination rates~\cite{ROMER2020113356,ULLAH202193}.

Finally, helping end users better understand microchips has numerous practical impacts. 
For instance, with a better understanding, end users would be more equipped to participate in discussions around legislative efforts such as the US CHIPS and Science Act~\cite{uschips2022} and the European Chips Act~\cite{euchips2022}, and---as citizens in democratic societies---to inform and hold their governments accountable for the significant investment through such programs. 


%Another related avenue is the \enquote{right to repair} directive recently adopted by the European Parliament to ensure that manufacturers provide timely and cost-effective repair services and inform consumers about their rights to repair~\cite{EuroParl24}. 
%With more knowledge of how microchips work, end users can more meaningfully take advantage of the right to repair, such as by identifying suitable replacement parts and finding the right documentation, when devices and infrastructure involving microchips break down~\cite{svensson2018emerging,DBLP:conf/cscw/RosnerA14}. 

With all the reasons summarized, we believe that the remaining question is not \textit{whether} we need to help end users understand more about microchips, but rather \textit{when} and \textit{how} to achieve this objective. 
Regarding the \enquote{when} aspect, it is important to acknowledge that end users have varying degrees of decision-making across the different scenarios in which microchips are deployed. 
For instance, when the device in question is a computer or tablet provided with dozens of microchips, we can reasonably expect that end users may adjust their level of trust in the device to purchase based on information about the microchip's performance (\eg., about its functionality), security (\eg, based on certifications), and ethical considerations (such as fair wages and working conditions) for workers involved in the manufacturing processes. 

However, this is less likely the case when deciding which airplane to take, as microchips are deployed en masse in planes and are generally inaccessible to end users.
Here, other factors such as the ticket's price and availability come as priorities~\cite{anwar2021effect}, and end users can only rarely choose the airplane type. 
Interestingly, this stands in contrast to our finding that our participants rated the airplane and pacemaker scenarios as the most critical (and more critical scenarios drive higher information needs).
In the pacemaker scenario, the deployment of microchips is less complicated.
Beyond medical reasons~\cite{doi:10.1161/01.CIR.91.4.1063}, end users have a fair degree of decision-making agency between individual devices and vendors.

The key to finding the right \enquote{when} moment is to identify other application settings that are not only important and relevant to end users, but also offer space for end users to make meaningful and informed decisions.
Going beyond the scenarios presented in our survey, we could imagine smartwatches and smartglasses as well as \acs{IoT} and smart home devices to fall into this category.
However, other complex applications, such as industrial machines and (digital) infrastructure components, are likely out of scope.


\subsection{Towards Microchip Transparency for End Users}
\label{sec:hci:implications}
We believe that future interdisciplinary research is required and the usable security community is uniquely positioned to tackle the \enquote{how} aspect of helping end users better understand microchips.
Below, we outline a few possible directions informed by our findings and speculate potential ideas to explore based on our own knowledge. 

\subsubsection{Building Mental Models of Microchips}
As our study is first-of-its-kind for the topic and exploratory in nature, we gauged participants' understanding of microchips in a simple open-ended question.
Our initial results pave the way for more thorough analyses of end users' mental models of microchips, which serve as foundational knowledge for any tools, resources, and educational interventions that seek to teach users about microchips.
For instance, future work can elicit end users' mental models in qualitative methods such as interviews, focus groups, co-design sessions, and drawing activities that enable deeper insights into users' reasoning processes and why misconceptions occur~\cite{jones2011mental}. 

Future work can also replicate prior studies on the mental models of computer security~\cite{wash2010folk,camp2009mental} and privacy~\cite{DBLP:journals/popets/OatesAMSZBC18} in the microchip setting to see to what extent users' existing models and metaphors still apply. 
Moreover, as prior work has consistently demonstrated the gaps between experts and laypeople regarding mental models~\cite{DBLP:journals/popets/OatesAMSZBC18,camp2007experimental,DBLP:conf/uss/BinkhorstFKPL22}, and microchips remain opaque even to experts~\cite{DBLP:conf/re/SpeithSBZBP24}, it is crucial to compare the mental models held by non-expert end users with those held by other stakeholders in the hardware ecosystem (such as designers, manufacturers, system integrators, and policymakers)~\cite{DBLP:conf/re/SpeithSBZBP24} in order to identify and close the gaps.


\subsubsection{Deciding Specific Information to Provide to End Users}
\label{subsubsec:info-to-provide}
Our study hints at the types of information that end users prioritize for understanding more about microchips.
However, the categories we presented in our study were quite broad. 
Future work is needed to empirically compare the effectiveness and downstream impacts on users (\eg, in terms of comprehension, trust in the system, and purchase behaviors) across the different information types, ideally with vignettes that feature the specific information adapted for the application setting.
Inspirations can also be drawn from the nudging literature for the framing of the presented information~\cite{acquisti2017nudges}. 

For instance, since our findings demonstrate that end users may lack awareness of the broader societal, economic, and security implications of microchips regarding risks and harms, future work can explore the effectiveness of presenting information that saliently features concrete harms.
Examples of harm can include hardware security issues, critical malfunctions in pacemakers, and environmental harms in communities involved in the mining of resources required for microchip manufacturing.
By making more informed purchase decisions, collective actions from end users could help improve working conditions and reduce environmental impact.


\subsubsection{Designing and Evaluating Transparency Mechanisms for Microchips}
Once the specific information to be provided has been determined, the follow-up question is how to effectively convey the information to laypeople through transparency mechanisms specifically applicable to microchips.
For instance, hardware datasheets have existed for a while. 
They contain information on the functionality and connectivity of a microchip as well as on its ideal operating conditions. 
However, they often contain technical jargon that makes them inaccessible to end users.
Drawing from standardized labels for \acs{IoT} devices~\cite{emami2020ask} and mobile apps~\cite{Cranor2022Mobile}, model cards for ML models~\cite{DBLP:conf/fat/MitchellWZBVHSR19}, and datasheets for datasets~\cite{DBLP:journals/cacm/GebruMVVWDC21}, we see the promise of creating \enquote{microchip labels} that enhance existing hardware datasheets beyond providing the typical technical documentations to make them more accessible and useful to end users. 
Taking our findings into account, the label can cover the microchip's functionality, interaction with the system, supply chain actors, involved certification bodies, and more. 
Such a label for a tablet computer could, for example, provide a score related to all microchips in the device based on manufacturing location and conditions, sustainability, and security. 
A QR code as part of this label could then lead to a list of all contained microchips as well as details on properties such as their functionality, manufacturer, and interoperability.
Similar to the IoT label development pipeline~\cite{DBLP:journals/cacm/CranorAN24}, much more work is needed after the initial proposal to reach a consensus on details surrounding the label (\eg, having minimal vs. more complicated labels, the presence of a QR code, the label's size, and how the label is encouraged or mandated in regulations). 



\subsection{Involving End Users in Regulatory Initiatives Around Microchips}
\label{sec:policy:implications}
Microchips represent a subject with natural policy implications. 
Against the background of public discourse about the use of Huawei equipment in network infrastructure~\cite{Webster2019} and the political efforts to promote domestic chip production in the United States and the European Union~\cite{euchips2022,uschips2022}, one of our key findings stands out---our participants were less interested in information about how and by whom a microchip was manufactured compared to the other types of information, whereas this aspect has been featured front and center in these regulatory initiatives.

Our study suggests that there is a potential gap between what legislators prioritize to address versus what end users desire to know. 
This may be due to the fact that microchip manufacturing is an intricate process that end users are mostly unaware of.
Given the level of knowledge required to comprehend microchip manufacturing, we argue that it would be best to leave technical manufacturing details to the regulators and instead focus on ethical aspects of manufacturing as well as microchip functionality and interaction within a system when designing explanations for end users.

One thing is known for sure: we cannot assume that the current multi-billion dollar investments from regulators will guarantee end-user trust in microchips. 
Therefore, similar to existing research on user perceptions of rights prescribed in the \ac{GDPR}~\cite{DBLP:conf/chi/KyiMSRB24,DBLP:conf/IEEEares/ManginiTM20,strycharz2020data,DBLP:conf/soups/KaushikYD021}, more work is needed to understand end users' perceptions of ongoing regulatory initiatives around microchips in order to capture and embed laypeople's opinions about microchips into policymaking. 
