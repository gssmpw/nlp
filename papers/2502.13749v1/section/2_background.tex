\section{Related Work}
\label{xhw_study::sec::background}

\subsection{User Understanding and Transparency}
Within the usable security and privacy community, past research has studied end users' understanding of \ac{E2EE}~\cite{DBLP:conf/eurousec/SchaewitzLSR21,DBLP:conf/soups/WuZ18}, HTTPS~\cite{DBLP:conf/sp/KrombholzBP0Z19}, home computer security~\cite{wash2010folk}, the Internet~\cite{kang2015my}, online behavioral advertising~\cite{DBLP:conf/cscw/YaoR017}, \acp{VPN}~\cite{DBLP:conf/uss/BinkhorstFKPL22,DBLP:conf/uss/RameshVE23}, and more. 
Some studies further draw the line between non-expert end users and experts such as system administrators and developers~\cite{DBLP:conf/uss/BinkhorstFKPL22,DBLP:conf/sp/KrombholzBP0Z19}. 
Misconceptions are common and often have downstream effects on users' behaviors. 
For example, Renaud~\etal~\cite{DBLP:conf/pet/RenaudVR14} found that incomplete threat models and a general lack of understanding of the email architecture are possible explanations for the low adoption of \ac{E2EE} for emails.
Importantly, there is no perfectly correct understanding~\cite{wash2010folk}, and even experts (with a deeper technical understanding of the technology) can still hold false beliefs~\cite{DBLP:conf/uss/BinkhorstFKPL22, DBLP:conf/sp/KrombholzBP0Z19}.

Studies on end-user understanding contribute insights into theirmisconceptions~\cite{rader2020have,DBLP:conf/soups/WuZ18,DBLP:conf/cscw/YaoR017,kang2015my} and reasoning processes behind threat models~\cite{DBLP:conf/eurousec/SchaewitzLSR21}, which then inform recommendations for how to encourage a secure use of the technology (\eg, through training, better communication, or system design changes)~\cite{DBLP:conf/uss/BinkhorstFKPL22}. 
The mental model approach is often used to describe the model in one's mind about how things work~\cite{wash2010folk}, usually with metaphors from already known domains~\cite{DBLP:journals/ijmms/StaggersN93}. 
For example, Stransky~\etal~\cite{DBLP:conf/soups/StranskyWSHAFWU21} compared six visualizations of security mechanisms for messaging apps based on users' mental models of \ac{E2EE}, finding that simple text disclosures were sufficient, yet user perceptions were more fundamentally shaped by preconceived expectations. 
Other work has sought to build visualization dashboards~\cite{DBLP:journals/popets/ReitingerWMU24,DBLP:journals/popets/FarkeBGA24} and design probes~\cite{DBLP:journals/imwut/BarbosaWUW21} to improve users' understanding of online tracking and inferences. 
Researchers have also explored using labels to convey the data practices of \ac{IoT} devices~\cite{DBLP:journals/ieeesp/NaeiniDAC22} and mobile apps~\cite{DBLP:conf/soups/ZhangKN0C24} to help consumers make purchase decisions, and such initiatives have received buy-ins from industry players and regulators~\cite{DBLP:journals/cacm/CranorAN24}.

Parallel efforts exist in the \acs{XAI} community, where the focus is to unpack the black box of \acs{AI}-based systems to end users, making the decision-making more understandable and transparent~\cite{DBLP:conf/fat/Speith22}. 
An individual's understanding of an \acs{AI}-based system can be increased by \enquote{white-box} explanations (\ie, that show the inner workings of an algorithm)~\cite{DBLP:conf/chi/ChengWZOGHZ19}, contextualizing general terminologies~\cite{DBLP:journals/pacmhci/ShenJCPZH20}, showing each feature's contribution to the model's prediction~\cite{DBLP:conf/iui/WangY21}, among other techniques.
The understanding can also be affected by the individual's domain expertise in the decision-making task~\cite{DBLP:conf/iui/WangY21} as well as the explanation's modality (\eg, textual, visual, or interactive)~\cite{DBLP:conf/fat/SchmudeKMT23}. 
Speith~\etal~\cite{DBLP:conf/re/SpeithSBZBP24} connect explainability to hardware in the context of requirements engineering, with a particular focus on microchips.
Among their future research directions, they explicitly propose to explore end-users' mental models of microchips.

Against these backgrounds, we see the potential that a better understanding of microchips can benefit end users. Our study provides novel knowledge of end users' current understanding of microchips and their informational wants, laying the foundation for future work on transparency mechanisms and educational efforts.


\subsection{Studies on Microchip (Security)}
To the best of our knowledge, there has been no prior work on end-user understanding of and interactions with microchips. That being said, prior research has examined the relationship between users and various microchip-based technologies, including autonomous vehicles~\cite{DBLP:conf/chi/TranPHWT24,DBLP:conf/chi/ChangCDCYCZZG24,DBLP:conf/chi/ChuZSGLGDZ23}, drones~\cite{DBLP:conf/chi/DongZCCL24}, robots~\cite{DBLP:conf/chi/SchneidersBCMCN24,DBLP:conf/chi/LuoDK24}, smart home devices~\cite{DBLP:conf/chi/ChiangKBC24}, and sensors in smart cities~\cite{DBLP:conf/chi/CorbettD24,DBLP:conf/chi/WindlW0M23}. 
These studies collectively contribute to our understanding of how users interact with and perceive emerging microchip technologies.

Research has also focused on improving the design and sustainability of \acfp{PCB}.\footnote{A \ac{PCB} is a flat surface that electrically connects electronic devices such as microchips.} 
Lin~\etal~\cite{DBLP:conf/chi/LinRPTDHM24} highlighted design space exploration as a promising alternative to fully automated or manual \ac{PCB} design approaches.
Yan~\etal~\cite{DBLP:conf/chi/YanL0P24} proposed SolderlessPCB to enhance the reusability of electronic components by eliminating the need for soldering components onto the \ac{PCB}.
Similarly, Arroyos \etal~\cite{DBLP:conf/chi/ArroyosVKOSSIN22} presented a functional computer mouse made from biodegradable \ac{PCB} materials, demonstrating that these components can dissolve in water, which allows for the reuse of mounted microchips.
Strasnick~\etal~\cite{DBLP:conf/chi/StrasnickAF21} introduced a \ac{PCB} debugging tool that aids in analog circuit debugging by facilitating the comparison between the physical circuit and a simulated model.

Focusing on security research, a few usable security papers have touched upon the role of hardware, although the findings were often discussed in passing as a small part of the main insights.
For example, Schmüser~\etal~\cite{DBLP:conf/chi/SchmuserRWSB0SW24} conducted a study on online security advice during the Ukraine war and found that the Twitter community regarded hardware as a medium-level concern, which was discussed primarily in the context of locking devices, disabling biometrics, and turning off location services.
Similarly, Gallardo~\etal~\cite{DBLP:conf/chi/GallardoEBBC24} discovered that security experts and energy system operators tend to underestimate the risks associated with hardware-based attacks.
Yu~\etal~\cite{DBLP:conf/chi/YuSDW24} found that while cryptocurrency users prefer hardware wallets for security reasons, they often refrain from using them due to usability challenges.
Reynolds \etal~\cite{DBLP:conf/sp/ReynoldsSRDRS18} highlighted usabilty issues in setting up YubiKeys (\ie hardware security tokens for two-factor authentication) with Google, Facebook, and Windows accounts.
Pfeffer \etal~\cite{DBLP:conf/uss/PfefferMDGSWFK21} later surveyed the effectiveness and usability of authenticity checks for such tokens, finding that users often neglect these essential checks, thereby undermining the security guarantees of the tokens.

Previous research has also explored the role of users in the security assurance of microchips~\cite{DBLP:conf/soups/0001WARP20,DBLP:journals/tochi/WiesenBWPR23,DBLP:conf/chi/WalendyWL0WE0FR24}. 
These studies examine the cognitive processes~\cite{DBLP:conf/soups/0001WARP20} and strategies~\cite{DBLP:journals/tochi/WiesenBWPR23} involved in hardware reverse engineering, employing methods such as eye tracking and think-aloud protocols~\cite{DBLP:conf/chi/WalendyWL0WE0FR24} to gain insights on how users interact with and analyze microchips.

While these works offer valuable insights into user interactions with hardware, our study goes beyond the technical aspects of hardware and broadens this inquiry by focusing on end-users' understanding of microchips, perceptions of their broader societal and security implications, and end-user information needs.