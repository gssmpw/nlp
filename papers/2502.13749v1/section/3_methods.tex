\section{Methods}
\label{xhw_study::sec::methods}
We conducted an online survey with 250 participants recruited via Prolific. 
A core part of the survey is a vignette setup: to make the concept of microchips less abstract and more accessible, we presented the participants with five scenarios based on real-world applications of microchips.
Each vignette consisted of a setting that describes a particular use case of a microchip and a desideratum (\ie, a property that might be desirable to end users).
Following the vignette description, we asked participants to rate the importance of the desideratum as well as the importance of receiving specific types of information about the microchip in the respective scenario.
Below, we outline our rationale for selecting vignette components and information types, present details of questionnaire design and study procedures, and address ethical considerations and data analysis techniques.

\subsection{Topic Selection and Item Generation}
\label{xhw_study::subsec::topic}
As the first step of scoping the survey, we identified five concrete settings in which microchips may be used, five desiderata that end users may want satisfied for microchips, and five kinds of information presented to end users.
The settings, desiderata, and information types were derived
from a literature review and discussions among experts, and refined in pilot studies (see \autoref{xhw_study::subsec::implementation}).

\paragraph{Derivation of settings involving microchips}
To help end users relate to microchips, we selected five settings in which microchips are employed.
We deliberately selected settings across a diverse range of applications, touching on aspects that end users may encounter in their everyday lives.
Specifically, we consider microchips (i)~\textit{controlling the entertainment system in a car}, (ii)~\textit{enabling wireless communication in a cell tower}, (iii)~\textit{controlling a pacemaker to maintain an adequate heart rate}, (iv)~\textit{enabling fingerprint unlocking of a smartphone}, and (v)~\textit{controlling the steering of an airplane}.

\paragraph{Derivation of end-user desiderata}
In our survey, we consider different goals that are desirable for end users.
We borrow an initial set of desiderata from literature on other technical systems~\cite{Chazette2021Exploring,Langer2021What,speith2023building}.
Through pilot testing (see \autoref{xhw_study::subsec::implementation}), we narrowed down the selection to five desiderata that are relevant for microchips and at the same time relatable to end users:
(a) \textit{accountability}, (b)~\textit{safety}, (c)~\textit{cyber security}, (d)~\textit{trustworthiness}, and (e)~\textit{ethical standards}.

\paragraph{Derivation of information facilitating microchip understanding}
\label{xhw_study::par::information}
The five different kinds of information offered to the end user are derived from different stages of the microchip design and manufacturing process~\cite{weste2015cmos,lienig2020fundamentals}.
Microchips are designed using a high-level language similar to regular programming languages.
The design descriptions are then implemented as an electronic circuit using automated software tools.
Next, the  design is handed to the manufacturer who produces the microchip in one of their production facilities, also known as \emph{fabs}~\cite{geiger1990vlsi}.
Derived from this process, we list the following as information to provide about microchips: (1)~\textit{who designed and manufactured the microchips} and (2)~\textit{how the microchips were designed and manufactured}.

Especially safety- and security-critical microchips must be certified by independent government bodies or dedicated testing service providers before use. 
As such, another useful piece of information could be (3)~\textit{how the microchips have been approved for use}.

The fabricated chip is finally integrated into a device such as a smartphone, a pacemaker, or a car.
Therefore, further relevant information could be (4)~\textit{how the microchips interact with the system} and (5)~\textit{which functionality the microchips provide}.

\subsection{Questionnaire Design}
\label{xhw_study::subsec::questionnaire}
We drew from our team members' expertise in usable security and embedded systems when designing the questionnaire. 
We took care to make our questionnaire understandable to end users through several rounds of piloting.
In the following, we briefly describe the flow of our questionnaire (see \autoref{xhw_study::app::survey} for a full version).

\subsubsection{Introduction}
At the beginning, we stated the purpose of our study, the expected duration of 25 minutes, and provided information on data handling and data protection.
Before participants could proceed, we asked them to give informed consent and to confirm that they were residents of the United States and at least 18 years of age~(\autoref{xhw_study::question::consent}).
Next, we asked participants nine questions on a six-point Likert scale to assess their tendency to actively engage in technology interaction using a validated psychometric scale~(\autoref{xhw_study::question::ati})~\cite{franke2019personal}.


\subsubsection{General questions on microchips}
In an open question, we asked participants what comes to mind when thinking of microchips (\autoref{xhw_study::question::perception}).
Further, we asked them whether or not they would like to understand more about microchips and invited them to give reasons for their choice~(\autoref{xhw_study::question::understand_more}).
We also queried participants regarding the time they would be willing to invest to better understand microchips~(\autoref{xhw_study::question::time_willing_before}) and on the time they currently invest for the same purpose~(\autoref{xhw_study::question::time_actual}), both on a five-point scale.
Next, we provided some background on microchips to align participants' basic understanding~(\autoref{xhw_study::question::background}).
To conclude this block, we presented five settings in randomized order involving microchips and asked participants to rate their criticality as the impact that a microchip malfunction would have on the participant themselves~(\autoref{xhw_study::question::criticality}).

\subsubsection{Vignettes}
From the 25 possible combinations of settings and desiderata (see \autoref{xhw_study::subsec::topic}), we formed five sets of five vignettes each, in which each setting and each desideratum occurs only exactly once.
At the core of our questionnaire, we showed participants one of these sets.
An example vignette is shown in \autoref{xhw_study::question::v0}.
For each vignette, we first asked participants to rate the importance of having a high level of the respective desideratum in the setting at hand on a five-point Likert scale~(\autoref{xhw_study::question::v0_desideratum}).
We then invited participants to explain their choices in an open-ended response~(\autoref{xhw_study::question::v0_desideratum_open}).
Second, we asked participants to rate the importance of receiving each of the five types of information (see \autoref{xhw_study::subsec::topic}) to assess the given desideratum in the specified setting on a five-point Likert scale~(\autoref{xhw_study::question::v0_information}).
Subsequently, we requested them to briefly explain their choice for one of the information types in an open-ended response~(\autoref{xhw_study::question::v0_information_open}).
Throughout the vignettes, we provided tooltips for some phrases (see \textbf{\textcolor[HTML]{B6321C}{red}} parts in \autoref{xhw_study::question::v0}) that, once hovered over with the cursor, would explain desiderata and types of information in simple language so that participants' mental models of these items are aligned to our understanding and they could get clarifications as needed as they completed the survey.

\subsubsection{Comprehension check}
To determine whether participants actually understood the desiderata, we presented them with an assignment exercise that asked them to match five randomly ordered sentences indicating the meaning of a desideratum to the desideratum in question~(\autoref{xhw_study::question::properties}).
The content of the sentences was based on the tooltips for the desiderata from the vignettes.
We again asked participants about their willingness to invest time in understanding microchips~(\autoref{xhw_study::question::time_willing_after})
to see whether it has changed compared to before~(\autoref{xhw_study::question::time_willing_before}).

\subsubsection{Demographics}
We asked for participants to indicate their gender~(\autoref{xhw_study::question::gender}), age range~(\autoref{xhw_study::question::age}), highest level of education~(\autoref{xhw_study::question::education}), and whether they had any prior practical experience with microchips~(\autoref{xhw_study::question::experience}).
Finally, we inquired if our participants had any feedback or anything they would like to share with us~(\autoref{xhw_study::question::feedback}).

\subsection{Survey Implementation}
\label{xhw_study::subsec::implementation}
We implemented our questionnaire using Qualtrics and recruited US-based English-speaking participants via Prolific.

\subsubsection{Pilot Testing} 
We conducted several pilot studies with a total of 79 participants to ensure end-user comprehension---specifically of the desiderata---by analyzing the open-ended questions~\autoref{xhw_study::question::v0_desideratum_open} and \autoref{xhw_study::question::v0_information_open} on participants' assessment of~\autoref{xhw_study::question::v0_desideratum} and \autoref{xhw_study::question::v0_information} with respect to misunderstandings.
Through these pilots, we aimed to determine whether our desiderata are indeed relevant to and comprehensible for end users, and if we had missed any desiderata that were important to them.
The extensive piloting led to several iterations of the questionnaire, particularly in terms of wording, sharpening of information types, and exclusion of unclear or irrelevant desiderata.

\subsubsection{Data Collection} 
We rolled out the main study with a gender-balanced sample of 250 participants over 10 days by releasing slots to batches of 25 participants, each at different times of the day.
The sample size of 250 was determined using a power analysis for multiple regression models.
We aimed for the detection of a small effect size $f^2$$=$$0.15$, power$=$$0.95$, and a significance level of $\alpha$$=$$0.05$.
For our power analysis, we indicated a total of 27 predictors, which is the sum of the number of vignettes, one's \ac{ATI} score~(see \autoref{xhw_study::question::ati})~\cite{franke2019personal} and whether or not participants would like to understand more about microchips (see \autoref{xhw_study::question::understand_more}).
%As we qualitatively analyzed participants' open-ended responses, we identified and excluded five submissions where the responses were obviously generated by \acs{AI}.
Participants took a median time of 24:19 minutes to complete our questionnaire and were compensated with 7.50~GBP, thus an hourly wage of 18.51~GBP.

\subsection{Ethics and Data Protection}
\label{xhw_study::subsec::ethics}
We could not have our planned study fully reviewed by an ethics committee because our department did not operate an \ac{IRB} at the time.
However, we reviewed our study in line with the application form for ethical approval of human studies from another department and reached the conclusion that our study would be \ac{IRB}-exempt in their case.
In addition, by limiting the survey to a few demographic questions, notably not asking about region of residence, we ensured the anonymity of our participants from the beginning. 
All data collected were stored on our institution's own servers, to which only the researchers involved in the project have access.

\subsection{Data Analysis}
\label{xhw_study::subsec::analysis}

\subsubsection{Qualitative Analysis}
To obtain insights into end-user perceptions of microchips~(\autoref{xhw_study::question::perception}) and their willingness to understand more about microchips~(\autoref{xhw_study::question::understand_more}), we conducted qualitative analysis~\cite{mayring_qualitative_2014} of the open-ended responses.
To this end, we used inductive thematic analysis.
The coding was executed by two coders, one with a background in hardware security and the other in computer science and \acs{AI} ethics.
Both coders first independently coded 50 responses (20\%). 
Each response could be assigned one or more codes.
Both coders then discussed their results and agreed on a common codebook for each of the two open questions.
In the process, the codebooks were refined through discussions among the coders by deleting, merging, and adding codes.
In the end, the final codebook for \autoref{xhw_study::question::perception} contained 57 codes while the one for \autoref{xhw_study::question::understand_more} comprised 24 codes.
They then both applied these codebooks to the remaining 200 responses (80\%).
To measure inter-coder reliability, Krippendorff's alpha~\cite{krippendorff2018content} was computed over all codes based on the MASI distance~\cite{passonneau2006measuring} between codes assigned by both coders.
This resulted in $\alpha$$=$$0.71$ for \autoref{xhw_study::question::perception} and $\alpha$$=$$0.76$ for \autoref{xhw_study::question::understand_more}, indicating \textit{substantial agreement} between the coders~\cite{landis1977measurement}.
Finally, both coders discussed discrepancies in their code assignments and fully agreed on a common coding.

\subsubsection{Statistical Analysis}
We applied descriptive statistics to describe the sample and overall trends regarding participants' perception of the importance of different scenarios, desiderata, as well as their affinity for technology interaction~\cite{franke2019personal}.
To explain participants' perceived importance of desiderata in different scenarios and information that might facilitate microchip understanding, we utilized inferential statistics.

For the perceived importance of desiderata in different scenarios, we used multiple linear regression models with dummy variables.
It is reasonable to assume that the ratings given by an individual participant are more similar than those given between participants. 
Therefore, we used multilevel modeling for this analysis.
We calculated \ac{ICC}~\cite{Koch2006}, or in this case, intra-individual correlation with intercept-only models. 
As \ac{ICC} accounts for $36\% - 45\%$ of the overall variance, we decided to use random-intercept models for further linear regression analysis.
For the random-intercept models, we calculated \textit{marginal} $R^2$ as well as \textit{conditional} $R^2$~\cite{Nakagawa2012}.
Marginal $R^2$ considers only the variance of the fixed effects, while the conditional $R^2$ takes both the fixed and random effects---in this case, participant ID---into account.
By subtracting marginal $R^2$ from conditional $R^2$, the contribution of the random effects can be calculated.
For model comparisons, we have also considered \ac{AIC}, \ac{BIC}, and deviance.

For all regression models, we applied Bonferroni corrections to take into account the probability of observing a false positive (\ie, a type I error).
In other words, we considered regression coefficients statistically significant only when $p$$<$$.001$ ($p$$=$$\alpha/m$ for the Bonferroni correction where $m$ is the number of comparisons, and $m$$=$$55$ when we had five regression models with 11 predictors each). 

\subsection{Limitations}
\label{xhw_study::subsec::limitations}
To the best of our knowledge, we are the first to explore end-user perspectives on microchips.
Accordingly, we had to develop our questionnaire from scratch.
To make the topic more accessible to end users, we decided to present our participants with vignettes.
However, despite careful selection for diversity, our vignette settings can only represent a small sample of the actual applications of microchips.
We also had to make a pre-selection for the desiderata and the types of information we investigated. 
We mitigated the self-selection bias by iteratively checking for missing items from participants' open-ended responses during pilot testing.

It is possible that comprehension issues may arise from the desiderata we provided (especially for similar ones like security and safety): participants may not clearly differentiate between the desiderata, or their understanding of the desiderata might differ from our definitions. 
We included tooltips as well as comprehension checks to address this issue, and our results show that participants correctly matched the descriptions to the respective desideratum in 90\% of all cases.
Participants had more issues comprehending trustworthiness (84\%) than ethical standards (94\%). 
For the other desiderata, comprehension is between 88\% and 92\%.

Further, we conducted our study only with residents of the United States, and our results may not be generalizable to other countries or societies where there may be specific sociocultural and political factors that shape discussions about microchips.
Last, in our survey, we only collected self-reported data about participants' willingness and time spent learning about microchips, which might not accurately reflect their actual behaviors.
