\section{Conclusion}
\label{xhw_study::sec::conclusion}
Microchips have become ubiquitous in people's daily lives, whether in the cars we drive, the phones we use, or even in our household appliances.
This observation highlights their indispensable role within socio-technical systems.
To better understand end-user perceptions of microchips, we conducted a survey with 250 participants.

While our participants appear to have a fundamental understanding of what microchips are and what they are used for, their knowledge of the consequences of microchip malfunction and their impact on society, in general, seems limited.
In particular, few participants had issues like cyber security, trustworthiness, or safety in mind, yet they considered them very important when explicitly asked about them.
Furthermore, our participants' information needs depend on their general affinity for technology, their willingness to understand more about microchips, and the considered desideratum and use case.
Based on our findings, future work could further explore end users' mental models of microchips and how to determine and convey information about them, so that end users can make more informed decisions about the purchase and use of electronic devices in the future.