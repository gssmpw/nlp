%%%%%%%% ICML 2023 EXAMPLE LATEX SUBMISSION FILE %%%%%%%%%%%%%%%%%

\documentclass[10pt]{article} % For LaTeX2e
%%%%% NEW MATH DEFINITIONS %%%%%

% \usepackage{amsmath,amsfonts,bm}
\usepackage{amsmath,amsfonts}

\usepackage{pifont}


\newcommand{\R}{\mathbb{R}}


\def\va{{\mathbf{a}}}
\def\vg{{\mathbf{g}}}

% Sets
\def\sR{\mathbb{R}}
\def\sC{\mathbb{C}}
\def\sZ{\mathbb{Z}}
\def\sN{\mathbb{N}}
\def\sQ{\mathbb{Q}}

\def\sS{\mathcal{S}}



% Vectors
\def\vzero{{\mathbf{0}}}
\def\vone{{\mathbf{1}}}
\def\vmu{{\mathbf{\mu}}}
\def\vtheta{{\mathbf{\theta}}}
\def\va{{\mathbf{a}}}
\def\vb{{\mathbf{b}}}
\def\vc{{\mathbf{c}}}
\def\vd{{\mathbf{d}}}
\def\ve{{\mathbf{e}}}
\def\vf{{\mathbf{f}}}
\def\vg{{\mathbf{g}}}
\def\vh{{\mathbf{h}}}
\def\vi{{\mathbf{i}}}
\def\vj{{\mathbf{j}}}
\def\vk{{\mathbf{k}}}
\def\vl{{\mathbf{l}}}
\def\vm{{\mathbf{m}}}
\def\vn{{\mathbf{n}}}
\def\vo{{\mathbf{o}}}
\def\vp{{\mathbf{p}}}
\def\vq{{\mathbf{q}}}
\def\vr{{\mathbf{r}}}
\def\vs{{\mathbf{s}}}
\def\vt{{\mathbf{t}}}
\def\vu{{\mathbf{u}}}
\def\vv{{\mathbf{v}}}
\def\vw{{\mathbf{w}}}
\def\vx{{\mathbf{x}}}
\def\vy{{\mathbf{y}}}
\def\vz{{\mathbf{z}}}
\def\vzeta{{\mathbf{\zeta}}}

% Matrix
\def\mA{{\mathbf{A}}}
\def\mB{{\mathbf{B}}}
\def\mC{{\mathbf{C}}}
\def\mD{{\mathbf{D}}}
\def\mE{{\mathbf{E}}}
\def\mF{{\mathbf{F}}}
\def\mG{{\mathbf{G}}}
\def\mH{{\mathbf{H}}}
\def\mI{{\mathbf{I}}}
\def\mJ{{\mathbf{J}}}
\def\mK{{\mathbf{K}}}
\def\mL{{\mathbf{L}}}
\def\mM{{\mathbf{M}}}
\def\mN{{\mathbf{N}}}
\def\mO{{\mathbf{O}}}
\def\mP{{\mathbf{P}}}
\def\mQ{{\mathbf{Q}}}
\def\mR{{\mathbf{R}}}
\def\mS{{\mathbf{S}}}
\def\mT{{\mathbf{T}}}
\def\mU{{\mathbf{U}}}
\def\mV{{\mathbf{V}}}
\def\mW{{\mathbf{W}}}
\def\mX{{\mathbf{X}}}
\def\mY{{\mathbf{Y}}}
\def\mZ{{\mathbf{Z}}}
\def\mBeta{{\mathbf{\beta}}}
\def\mPhi{{\mathbf{\Phi}}}
\def\mLambda{{\mathbf{\Lambda}}}
\def\mSigma{{\mathbf{\Sigma}}}


% Expectation
% \def\eE{\mathop{\mathbb{E}}\limits}
\def\eE{\mathbb{E}}

% Probability
\def\pP{\mathbb{P}}

% Tilde
\def\tf{\tilde{f}}
\def\tS{\tilde{S}}
\def\wtF{\widetilde{\mathcal{F}}}
\def\whR{\widehat{R}}
\def\tvx{\tilde{\mathbf{x}}}
\def\ty{\tilde{y}}


\def\defeq{\overset{\textup{def}}{=}}
% \def\defeq{\overset{.}{=}}
\def\defone{\overset{\text{\ding{172}}}{=}}
\def\deftwo{\overset{\text{\ding{173}}}{=}}
\def\leqone{\overset{\text{\ding{172}}}{\leq}}
\def\leqtwo{\overset{\text{\ding{173}}}{\leq}}
\def\leqthree{\overset{\text{\ding{174}}}{\leq}}
\def\leqfour{\overset{\text{\ding{175}}}{\leq}}
\def\eqone{\overset{\text{\ding{172}}}{=}}
\def\eqtwo{\overset{\text{\ding{173}}}{=}}
\def\eqthree{\overset{\text{\ding{174}}}{=}}
\def\eqfour{\overset{\text{\ding{175}}}{=}}
\def\geqfive{\overset{\text{\ding{176}}}{\geq}}

\usepackage{paper}
% \usepackage[preprint]{paper}

% Recommended, but optional, packages for figures and better typesetting:
\usepackage{microtype}
\usepackage{graphicx}
\usepackage{xcolor}
\usepackage{booktabs} % for professional tables
\usepackage{multicol}
\usepackage{multirow}
\usepackage{enumitem}
\usepackage{subcaption}
\usepackage{hyperref}
\usepackage{wrapfig}

% For theorems and such
\usepackage{amsmath}
\usepackage{amssymb}
\usepackage{mathtools}
\usepackage{amsthm}

% if you use cleveref..
\usepackage[capitalize,noabbrev]{cleveref}
\usepackage[textsize=tiny]{todonotes} 

\newcommand{\sstd}[1]{\textcolor{black}{\tiny{$\pm #1$}}}
\newcommand{\std}[1]{}
\newcommand{\highlight}[1]{\colorbox{blue!10}{#1}}
\newcommand{\theHalgorithm}{\arabic{algorithm}}

% hyperref makes hyperlinks in the resulting PDF.
% If your build breaks (sometimes temporarily if a hyperlink spans a page)
% please comment out the following usepackage line and replace
% \usepackage{icml2023} with \usepackage[nohyperref]{icml2023} above.
\definecolor{mydarkblue}{rgb}{0,0.08,0.45}
\hypersetup{ %
    colorlinks=true,
    linkcolor=mydarkblue,
    citecolor=mydarkblue,
    filecolor=mydarkblue,
    urlcolor=mydarkblue,
}

%%%%%%%%%%%%%%%%%%%%%%%%%%%%%%%%
% THEOREMS
%%%%%%%%%%%%%%%%%%%%%%%%%%%%%%%%
\theoremstyle{plain}
\newtheorem{theorem}{Theorem}[section]
\newtheorem{proposition}[theorem]{Proposition}
\newtheorem{lemma}[theorem]{Lemma}
\newtheorem{corollary}[theorem]{Corollary}
\theoremstyle{definition}
\newtheorem{definition}[theorem]{Definition}
\newtheorem{assumption}[theorem]{Assumption}
\theoremstyle{remark}
\newtheorem{remark}[theorem]{Remark}

\title{Amortizing Bayesian Posterior Inference in \\Tractable Likelihood Models}

\author{\name Sarthak Mittal \email sarthmit@gmail.com\\
\addr Mila, Universit\'e de Montr\'eal
\AND
\name Niels Leif Bracher \email nbracher@yorku.ca \\
\addr York University, Vector Institute
\AND
\name Guillaume Lajoie \email g.lajoie@umontreal.ca \\
\addr Mila, Universit\'e de Montr\'eal
\AND
\name Priyank Jaini \email pjaini@google.com \\
\addr Google DeepMind
\AND
\name Marcus A Brubaker \email mab@eecs.yorku.ca \\
\addr York University, Vector Institute 
% \name Kyunghyun Cho \email kyunghyun.cho@nyu.edu \\
%       \addr Department of Computer Science\\
%       University of New York
%       \AND
%       \name Raia Hadsell \email raia@google.com \\
%       \addr DeepMind
%       \AND
%       \name Hugo Larochelle \email hugolarochelle@google.com\\
%       \addr Mila, Universit\'e de Montr\'eal \\
%       Google Research\\
%       CIFAR Fellow
}

% The \author macro works with any number of authors. Use \AND 
% to separate the names and addresses of multiple authors.

\newcommand{\fix}{\marginpar{FIX}}
\newcommand{\new}{\marginpar{NEW}}

\def\month{MM}  % Insert correct month for camera-ready version
\def\year{YYYY} % Insert correct year for camera-ready version
\def\openreview{\url{https://openreview.net/forum?id=XXXX}} % Insert correct link to OpenReview for camera-ready version

\begin{document}

\maketitle
\begin{abstract}
\looseness=-1
Bayesian inference provides a natural way of incorporating prior beliefs and assigning a probability measure to the space of hypotheses. However, it is often infeasible in practice as it requires expensive iterative routines like MCMC to approximate the posterior distribution. Not only are these methods computationally expensive, but they must also be re-run whenever new observations are available, making them impractical or of limited use. To alleviate such difficulties, we amortize the posterior parameter inference for probabilistic models through permutation invariant architectures. While this paradigm is briefly explored in Simulation Based Inference (SBI), Neural Processes (NPs) and Gaussian Process (GP) kernel estimation, a more general treatment of amortized Bayesian inference in known likelihood models has been largely unexplored. We additionally utilize a simple but strong approach to further amortize on the dimensionality of observations, allowing a single system to infer variable dimensional parameters. In particular, we rely on the reverse-KL based amortized Variational Inference (VI) approach to train inference systems and compare them with forward-KL based SBI approaches across different architectural setups. We conduct thorough experiments to demonstrate the effectiveness of our proposed approach, especially in real-world and model misspecification settings.
\end{abstract}

\section{Introduction}
\label{sec:intro}

\begin{figure*}[tb]
    \centering
    \includegraphics[width=0.848\linewidth]{figs/circuitnn.pdf} 
    \caption{Illustration of differentiable CircuitNN. CircuitNN is designed based on differentiable NAND gates. After DAS is guided by PI and PO pairs of the truth table, CircuitNN can get the precise circuit architecture logic equivalent to the truth table.}
    \label{fig:circuitnn}
\end{figure*}

% 1. Describe the importance of logic synthesis
% 2. Existing Problems
% (a) Neural Architecture Search: Unstable, Predefined Setting, etc.
% (b) Circuit Generation: Probabilistic Model, Logic Equivalence

With the rapid advancement of technology, the scale of integrated circuits (ICs) has expanded exponentially. 
This expansion has introduced significant challenges in chip manufacturing, particularly concerning power and area metrics.
A primary objective in IC design is achieving the same circuit function with fewer transistors, thereby reducing power usage and area occupancy.

Logic synthesis~\cite{hachtel2005logicsynth}, a critical step in electronic design automation (EDA), transforms behavioral-level circuit designs into optimized gate-level circuits, ultimately yielding the final IC layout. 
The primary goal of logic synthesis is to identify the physical implementation with the fewest gates for a given circuit function. 
This task constitutes a challenging NP-hard combinatorial optimization problem. 
Current logic synthesis tools~\cite{brayton2010abc, wolf2013yosys} rely on human-designed heuristics, often leading to sub-optimal outcomes.

Differentiable architecture search (DAS) techniques~\cite{liu2018darts, chu2020darts} offer novel perspectives on addressing challenges in this problem.
Circuit functions can be represented through truth tables, which map binary inputs to their corresponding outputs. 
Truth tables provide a precise representation of input-output relationships, ensuring the design of functionally equivalent circuits.
Inspired by this, researchers~\cite{deepmind2024ai4sys, wang2024tnet} have begun exploring the application of DAS to synthesize circuits directly from truth tables.
Specifically, \citet{deepmind2024ai4sys} proposed CircuitNN, a framework that learns differentiable connection structures with logic gates, enabling the automatic generation of logic circuits from truth tables.
This approach significantly reduces the complexity of traditional circuit generation. 
Building on this, \citet{wang2024tnet} introduced T-Net, a triangle-shaped variant of CircuitNN, incorporating regularization techniques to enhance the efficiency of DAS.

Despite these advancements, several challenges remain. 
The computational complexity of DAS grows quadratically with the number of gates, posing scalability issues.
Although triangle-shaped architecture~\cite{wang2024tnet} partially mitigates this problem, redundancy persists. 
%Additionally, DAS is susceptible to converging to local optima, limiting the ability to search architectures that satisfy the given truth tables~\cite{liu2018darts}. 
%Furthermore, hyperparameters (network depth and layer width) require extensive searches, introducing complexity and prolonging the synthesis process. 
Additionally, DAS is susceptible to converging to local optima~\cite{liu2018darts} and hyperparameters (network depth and layer width) require extensive searches. 
The challenges arise from the vast search space in DAS. 
% Even with predefined settings for CircuitNN, finding a configuration that meets the truth table requires extensive trial and error during the DAS process. 
Intuitively, limiting the search space through predefined parameters (network depth, gates per layer, and connection probabilities) can significantly reduce the complexity.

Recent advances~\cite{openai2023gpt4, abramson2024alphafold3, esser2024sd3, li2024mar} in conditional generative models have demonstrated remarkable performance across language, vision, and graph generation tasks. 
Motivated by these developments, we propose a novel approach to circuit generation that generates preliminary circuit structures to guide DAS in generating refined circuits matching specified truth tables. 
Firstly, we introduce CircuitVQ, a tokenizer with a discrete codebook for circuit tokenization. 
Built upon our Circuit AutoEncoder framework~\cite{hou2022graphmae,li2023maskgae,wu2025mgvga}, CircuitVQ is trained through a circuit reconstruction task. 
Specifically, the CircuitVQ encoder encodes input circuits into discrete tokens using a learnable codebook, while the decoder reconstructs the circuit adjacency matrix based on these tokens.
Subsequently, the CircuitVQ encoder serves as a circuit tokenizer for CircuitAR pretraining, which employs a masked autoregressive modeling paradigm~\cite{chang2022maskgit, li2023mage}. 
In this process, the discrete codes function as supervision signals. 
After training, CircuitAR can generate discrete tokens progressively, which can be decoded into initial circuit structures by the decoder of the CircuitVQ. 
These prior insights can guide DAS in producing refined circuits that match the target truth tables precisely.

Our key contributions can be summarized as follows:
\begin{itemize}
\item We introduce CircuitVQ, a circuit tokenizer that facilitates graph autoregressive modeling for circuit generation, based on our Circuit AutoEncoder framework;
\item Develop CircuitAR, a model trained using masked autoregressive modeling, which generates initial circuit structures conditioned on given truth tables;
\item Propose a refinement framework that integrates differentiable architecture search to produce functionally equivalent circuits guided by target truth tables;
\item Comprehensive experiments demonstrating the scalability and capability emergence of our CircuitAR and the superior performance of the proposed circuit generation approach.
\end{itemize}

% Motivation
% (a) Diffusion (Vision, Graph), Autoregressive (Language, Vision)
% (b) Circuit Generation for Predefined Setting
% (c) Neural Architecture Search for Strict Logic Equivalence

% Contribution
% (a) Circuit Tokenizer (new transformer arch, training strategy)
% (b) CircuitAR (train and gen strategies, post-ar strategy)
% (c) Extensive Evaluation including BitD (Bit Distance) for Scalability

\section{Basic Background: Supervised Learning and the PAC Model}
\label{sec:background}

At this point almost everyone has heard of machine learning (ML). Anyone likely to stumble upon this article will have also heard of its most influential special case, supervised learning, and those theoretically inclined will also be familiar with the PAC model. Nonetheless, I will set the stage by  recapping the basics.

\subsection{Basics of Supervised Learning}%Let's set the stage in any case

\emph{Supervised Learning} is the task of ``coming up'' with a function $f: \X \to \Y$ to ``explain'' or ``fit'' a sequence of input/output examples   $(x_1,y_1), \ldots, (x_n,y_n)$, with $x_i \in \X$ and $y_i \in \Y$.  Here $\X$ is a \emph{data domain} consisting of \emph{datapoints} $x \in \X$, $\Y$ is a \emph{label set} consisting of \emph{labels} $y \in \Y$, and the sequence $(x_1,y_1),\ldots,(x_n,y_n)$ is the \emph{training data} consisting of \emph{labeled examples (a.k.a. samples)}~$(x_i,y_i)$.  I~will refer to the chosen function $f$ as a \emph{predictor}, and to $n$ as the \emph{sample size}. A \emph{learning algorithm} takes as input training data, and outputs (some representation of) a predictor $f \in \Y^\X$.\footnote{Note that this describes the usual \emph{batch}, a.k.a.~\emph{offline}, setting of supervised learning. I do not discuss other paradigms such as online or active learning in this article.} 



Success in supervised learning is defined as \emph{generalization} to  future examples: For a typical \emph{test example}  $(x_{\tst},y_{\tst})$, the predicted label $y'_{\tst}=f(x_{\tst})$ should ``equal'' $y_{\tst}$, perhaps approximately. We usually assume the test example is drawn from the same  ``source'' as the training data  --- commonly, i.i.d.~from the same distribution. The quality of the prediction is quantified by $\ell(y'_{\tst},y_{\tst})$, where $\ell:~\Y~\times~\Y \to \RR_{\geq 0}$ is a \emph{loss function} chosen as part of the problem definition. Common loss functions include the 0-1 loss $\ell_{0-1}(y',y) = [y' \neq y]$ for \emph{classification} problems,\footnote{The notation $[P]$ denotes $1$ when predicate $P$ is true, and denotes $0$ when $P$ is false.} as well as the absolute loss $|y'-y|$ or squared loss $(y'-y)^2$ for \emph{regression problems} featuring $\Y  \sse \RR$.

Nontrivial generalization properties are typically only possible if one assumes something about the data.\footnote{The need for such an assumption is formalized by the  \emph{no free lunch theorems} of supervised learning \cite{wolpert_connection_1992,wolpert_lack_1996,schaffer_conservation_1994}.} The Bayesian approach to  machine learning, common in many applications, assumes some parametric form for the distribution generating the data, and postulates a prior on the parameters. This is not the approach I will take in this article. Instead, I will focus on the frequentist --- and some would say ``worst-case'' or ``adversarial'' ---  approach that is common in the computational learning theory community, embodied by the PAC model. Here we assume that the (training and test) data can be explained, perhaps approximately, by a function in some ``simple enough to learn'' class of functions $\H \sse \Y^\X$, often called the \emph{hypotheses}. Equivalently, we  seek a predictor which explains the unseen data roughly  as well as the best hypothesis $h^* \in \H$, whether or not we assume that $h^*$ itself provides a perfect explanation.



 \paragraph{Common Algorithmic Templates.} Perhaps the best known general-purpose supervised learning algorithm is \emph{empirical risk minimization (ERM)}, which chooses as its predictor a hypothesis $f \in \H$ minimizing $\frac{1}{n} \sum_{i=1}^n \ell(f(x_i),y_i)$ --- a quantity called the \emph{training error}, \emph{empirical error}, or \emph{empirical risk} of $f$. %\footnote{When multiple hypotheses minimize the empirical risk, we assume ERM breaks ties arbitrarily.}
A common template for generalizing ERM involves adding a \emph{regularization term} $\psi(f)$ to the  objective function, typically chosen to measure some notion of ``hypothesis complexity.'' An algorithm instantiating this template is known as a \emph{structural risk minimizer (SRM)}, and chooses as its predictor the hypothesis $f \in \H$ minimizing the \emph{structural risk} $\frac{1}{n} \sum_{i=1}^n \ell(f(x_i),y_i) + \psi(f)$. Other well-known algorithms, such as gradient descent and its variations,  can frequently be interpreted as approximate implementations of ERM or SRM.


\paragraph{Proper vs Improper Learning.} A learning algorithm is said to be \emph{proper} if its predictor $f$ is always chosen from the hypothesis class, i.e., $f \in \H$, otherwise it is said to be \emph{improper}. ERM  is an example of a proper learning algorithm, as are SRM algorithms of the form described above.  In the \emph{proper regime} of learning, algorithms are required to be proper. This article will be concerned with the more flexible \emph{improper regime} (a.k.a \emph{representation-independent learning}), where no such constraint is placed on the learner. In other words, all we care about is predictive power at test time, rather than any insights derived from the functional form or representation of the predictor~itself.


\subsection{The PAC Model}
A standard mathematical setup for evaluation of supervised learning algorithms, at least in the theoretical computer science community, is Valiant's \emph{Probably Approximately Correct (PAC) model} of learning (see e.g.~\cite{kearns_introduction_1994,mohri_foundations_2018}). Here, we assume there is an unknown distribution $\D$ on $\X \times \Y$ from which training and test data are  drawn.  Specifically, the labeled datapoints of the training set  $(x_1,y_1), \ldots, (x_n,y_n)$, as well as the test data  $(x_\tst,y_\tst)$, are i.i.d.~from $\D$. Often it is assumed that $\D$ lies in some class of distributions of interest. The \emph{true expected loss}, or simply \emph{loss}, of a predictor $f: \X \to \Y$ is the expected loss it incurs on draws from $\D$, written $L_\D(f) = \Ex_{(x,y) \sim \D} \ell(f(x),y)$.


There are two main ``settings'' in PAC learning. The  \emph{realizable setting} only requires that the data be perfectly explained by some hypothesis in $\H$. More generally, the \emph{agnostic setting} makes no assumption relating the data to the hypotheses, but shifts the goalposts as necessary to allow nontrivial guarantees: the expected loss at test time is evaluated only ``relative'' to that of the best hypothesis $h^* \in \H$. There are other settings which make more nuanced assumptions, such as $\D$ being of a particular parametric form or its support living in some (unknown) lower-dimensional space, etc. I will mostly discuss the realizable and agnostic settings in this article, those being the simplest and most studied from a theoretical perspective. %TODO:We will briefly discuss other settings in Section ??

The PAC model demands high probability guarantees of learners, in the worst case over distributions of interest. Consider first the realizable setting, where $\D$ is such that $\min_{h \in \H} L_{\D}(h) = 0$. A PAC learner has \emph{error} $\epsilon=\epsilon(n)$ and \emph{confidence} $\delta=\delta(n)$ if, when training data consists of $n$ i.i.d~samples from a realizable distribution $\D$, it produces a predictor $f$  satisfying $L_\D(f) \leq \epsilon$ with probability at least $1-\delta$. In the agnostic setting, where $\D$ can be arbitrary, we require $L_\D(f) - \min_{h \in \H} L_\D(h) \leq \epsilon$ with probability $1-\delta$.

In both the realizable and agnostic settings, we look for PAC learners with small $\epsilon$ and $\delta$ as a function of the sample size $n$. An equivalent perspective looks at the sample complexity $m(\epsilon,\delta)$, which is the minimum sample size which guarantees error  at most $\epsilon$ with probability at least $1-\delta$. We say a problem is \emph{PAC learnable} if its PAC sample complexity is finite whenever $\epsilon,\delta > 0$.

For most PAC learning problems, learnability and sample complexity are characterized in terms of a  ``dimension'' of the hypothesis class. Most prominently this is the \emph{VC dimension} for binary classification, the \emph{fat shattering dimension} for agnostic regression, and the \emph{DS dimension} for multiclass classification (see \cite{anthony_neural_1999,daniely_optimal_2014,brukhim_characterization_2022}). Treatment of these is beyond the scope of this article. The unfamiliar reader need not worry, however,  as dimensions will feature only tangentially in our~discussion.




%\paragraph{Learning settings: Realizable, Agnostic, etc.} In learning theory, evaluating a supervised learning algorithm requires specifying a data model and an objective. We will leave the details of the data model flexible for now, to allow for both the PAC model and the adversarial transductive model. Nonetheless we will describe two variations, which we call ``settings'', which cut across different models. The  \emph{realizable setting}  requires only that the data be perfectly explained by some hypothesis $h \in \H$ --- i.e., there exists a hypothesis which is guaranteed to suffer a loss of $0$ on training and test data. The performance of the learning algorithm is its expected loss at test time for some ``worst case'' realizable instance. More generally, the \emph{agnostic setting} makes no assumption relating the data to the hypotheses, but shifts the goalposts as necessary to allow nontrivial guarantees: the expected loss at test time is evaluated only ``relative'' to that of the best hypothesis $h^* \in \H$, again for some ``worst case'' instance. There are other settings which make more nuanced assumptions about the data, such as it is drawn from a distribution of a particular parametric form, or that it lives in some (unknown) lower-dimensional space, etc. We will mostly discuss the realizable and agnostic settings, those being the simplest and most studied from a theoretical perspective.




%%% Local Variables:
%%% mode: latex
%%% TeX-master: "learning_matching"
%%% End:

% \begin{figure}
%     \centering
%     \includegraphics[width=0.5\linewidth]{Move_teaser.pdf}
%     \caption{Comparison of different dynamic compute approaches. length of arrow indicates residual transformation per token while width indicates velocity of transformation.}
%     \label{fig:enter-label}
% \end{figure}

\section{Method}
\label{sec:method}
Residual connections play a crucial role in shaping token representations, yet their dynamics remain underexplored in the context of efficient decoding. In this work, we delve deeper into transformer residual dynamics and investigate how modulating residual transformation velocity can improve inference efficiency in token-level processing, optimizing both dense and sparse MoE transformers.


\subsection{Residual Dynamics and Motivation for Multi-rate Residuals} \label{sec:motivation}

To analyze how hidden representations evolve across different layers of a transformer architecture, it's crucial to consider the effect of residual connections. Each transformer decoder layer typically has residual connections across attention and MLP submodules. As the residual stream $h_i$ traverses from interval $E_j$ to $E_{j+1}$, it undergoes a residual transformation given by:  
% \begin{equation}
% \label{eq:slow_residual_transformation}
% H_{E_{j+1}} = H_{E_j} \prod_{i=E_j}^{E_{j+1}} \left( I + \mathcal{A}_i \right) \left( I + \mathcal{M}_i \right) \quad \text{where} \quad \mathcal{A}_i = f(c_i, h_{i}), \mathcal{M}_i = g(h_i)
% \end{equation}

\begin{equation} \label{eq:slow_residual_transformation}
h_{E_{j+1}} = h_{E_j} + \sum_{i=E_j}^{E_{j+1}-1} \left( \mathcal{A}_i(h_i) + \mathcal{M}_i(h_i + \mathcal{A}_i(h_i)) \right) \quad \text{where} \quad \mathcal{A}_i = f(c_i, h_{i}), \mathcal{M}_i = g(h_i). 
\end{equation}

Here, \( \mathcal{A}_i \) denotes the non-linear transformation introduced by the multi-head attention mechanism at layer \( i \), while \( \mathcal{M}_i \) corresponds to the non-linear transformation of the MLP block at the same layer. These transformations depend on the input residual stream \( h_i \) and, in the case of \( \mathcal{A}_i \), the previous contextual representation \( c_i \).\footnote{Normalization layers are typically applied in practice but are omitted here for simplicity of the argument.}


% For easy tokens, the magnitude and direction of this delta transformation become progressively smaller with each successive layer as shown in \cref{fig:delta_transformation}. Consequently, it is feasible to predict these tokens after only a few residual connections, whereas harder tokens necessitate more extensive processing through additional layers.

\begin{figure}[ht]
    \centering
    \begin{subfigure}{0.48\textwidth}
        \centering
        \includegraphics[width=\textwidth]{sections/figures/residual_change.pdf}
        \caption{}
        \label{fig:residual_change}
    \end{subfigure}%
    \hfill
    \begin{subfigure}{0.48\textwidth}
        \centering
        \includegraphics[width=\textwidth]{sections/figures/alignment_wrt_dedicated_model.pdf}
        \caption{}
    \label{fig:alignment_wrt_dedicated_model}
    \end{subfigure}
    \caption{(a) As residual streams propagate through the model, the directional shifts in the residuals become progressively smaller. (b) A dedicated model with $k$ layers achieves a faster rate of change in residual streams and higher alignment than base model leveraging early exit mechanisms at layer $k$.}
    \label{fig}
\end{figure}


To examine whether residual transformations can be accelerated across layers, we conducted experiments using a diverse set of prompts on a pre-trained Phi3 model~\cite{phi3_report}. As illustrated in \cref{fig:residual_change}, we measured the directional shift in residual states as \( 1 - \mathcal{C}(h_{i-1}, h_i) \), where \(\mathcal{C}\) denotes normalized cosine similarity. This shift is notably higher in the initial layers, gradually decreasing in subsequent layers. This behavior allows traditional early exit approaches to effectively accelerate decoding by enabling earlier exits for simpler tokens. However, these approaches typically rely on a distance-based approximation, where the full residual transformation of the model is approximated by the residual transformations of the initial layers. To gain deeper insights into the distance versus velocity aspects of residual transformation, we conducted a comparative study. Specifically, we trained an early exit head at layer $k$ of the Phi3 model, which consists of 32 layers, restricting the distance traveled by each token. To accelerate the residual transformation relative to number of layers, we trained a smaller model consisting of only $k$ layers, while keeping all other hyperparameters consistent. We then compared the next-token prediction accuracy of the early exit head of the base model with that of the smaller model. To ensure an equal number of trainable parameters, we inserted low-rank adapters into the smaller model and trained only these adapters, whereas, in the distance-based approach, we trained solely the early exit head. In addition, to accelerate the residual transformation in smaller model, we distilled the residual streams from the larger model by incorporating a distillation loss ~\cite{sanh2019distilbert} between the residual state at layer \(i\) of the smaller model and the residual state at layer \(4 \times i\) of the larger model. As shown in ~\cref{fig:alignment_wrt_dedicated_model} the smaller model demonstrates a significantly faster rate of change in residual streams, leading to higher next token prediction accuracy after $k$ layers compared to the base model that employs traditional early exit mechanisms after $k$ layers \cite{schuster2022confident, chen2023eellm, varshney-etal-2024-investigating}. This experimental setup, which modifies only the rate of change in residual streams while keeping other factors constant, suggests that dense transformers, trained with a fixed number of layers, may inherently possess a slow residual transformation bias.

This observation raises an intriguing question: if the rate of change in residual streams could be accelerated relative to the number of layers, is it possible to facilitate earlier alignment for a greater proportion of tokens? Earlier alignment would be beneficial to not only facilitate dynamic computation but also for generating speculative tokens efficiently with high acceptance rates in speculative decoding setups ~\cite{leviathan2023fast, chen2023accelerating}. 

%thereby enhancing the efficiency of early exiting? 
 % This bias likely constrains the effectiveness of early exiting, particularly for easier tokens. By addressing this limitation through accelerated residual transformations, we hypothesize that it is possible to substantially improve the efficiency and accuracy of early exit strategies in transformer models.

\subsection{Multi-Rate Residual Transformation} \label{m2r2_method}

To address the slow residual transformation bias described in ~\cref{sec:motivation}, we introduce \textit{accelerated residual streams} that operate at rate $R$ relative to original slow residual stream. We pair slow residual stream, $h$ with an accelerated residual stream, $p$, which has an intrinsic bias towards earlier alignment. Relative to ~\cref{eq:slow_residual_transformation}, accelerated residual transformation from interval $E_j$ to $E_{j+1}$ can be represented as: 

% \begin{equation}
% \label{eq:fast_residual_transformation}
% P_{E_{j+1}} = P_{E_j} \prod_{i=E_j}^{E_{j+1}} \left( I + \hat{\mathcal{A}_i} \right) \left( I + \hat{\mathcal{M}_i} \right) \quad \text{where} \quad \hat{\mathcal{A}_i} = \hat{f}(c_i, P_{i}), \hat{\mathcal{M}_i} = \hat{g}(P_{i})
% \end{equation}


\begin{equation} \label{eq:fast_residual_transformation}
p_{E_{j+1}} = p_{E_j} + \sum_{i=E_j}^{E_{j+1}-1} \left( \hat{\mathcal{A}_i}(p_i) + \hat{\mathcal{M}_i}(p_i + \hat{\mathcal{A}_i}(p_i)) \right) \quad \text{where} \quad \hat{\mathcal{A}_i} = \hat{f}(c_i, p_{i}), \hat{\mathcal{M}_i} = \hat{g}(h_i), 
\end{equation}



where $\hat{\mathcal{A}_i}$ and $\hat{\mathcal{M}_i}$ denote non-linear transformation added by layer $i$ to previous accelerated residual $p_{i}$. Similar to $\mathcal{A}_i$, non-linear transformation $\hat{\mathcal{A}_i}$ attends to same context $c_i$ but uses a different transformation $\hat{f}$ for accelerating $p_{E_j}$ relative to $h_{E_j}$. 

We integrate accelerated residual transformation directly into the base network using parallel accelerator adapters such that rank of accelerator adapters $R_p << d$ where $d$ denotes base model hidden dimension. This setup allows the slow residual stream $h_{E_j}$ to pass through the base model layers while the accelerated residual stream $p_{E_j}$ utilizes these parallel adapters as shown in ~\cref{fig:m2r2_main}. Both slow and accelerated residuals are processed in same forward pass via attention masking and incur negligible additional inference latency in memory bound decoding setups, while in compute bound decoding setups where FLOPs optimization is essential, accelerated residual stream utilizes a fraction of attention heads that of slow residual (see ~\cref{sec:flops_optimization}). Additionally, to maximize the utility of accelerated residual transformations without introducing dedicated KV caches, we propose a shared caching mechanism between the slow and accelerated streams which minimally impact alignment benefits of our approach while offering substantial memory savings (see ~\cref{fig:koala_alignment}). Specifically, the attention operation on the slow residuals \( \text{MHA}(h_t, h_{\leq t}, h_{\leq t}) \) is redefined for accelerated residuals as 
\[
\hat{\mathcal{A}} = MHA(p_t, h_{<t} \oplus p_t, h_{<t} \oplus p_t),
\]
where the accelerated residual at time-step $t$, \( p_t \) attends to the slow residual’s KV cache, facilitating the reuse of contextual information across both residual streams without incurring additional caching costs. Here, \(MHA(q, k, v) \) represents multi-head attention between query \( q \), key \( k \), and value \( v \).

\begin{figure}
    \centering
    \includegraphics[width=0.8\linewidth]{sections//figures/m2r2_main2.pdf}
    \caption{Multi-rate Residuals Framework: Slow residual stream of base model is accompanied by a faster stream that operates at a $2-(J+1)\times$ rate relative to the slow stream, undergoing transformations via accelerator adapters as detailed in \cref{m2r2_method}, where J denotes number of early exit intervals. Colors within the slow and fast residual streams indicate similarity, with matching colors representing the most closely aligned residual states. At the beginning of the forward pass and at each exit point, the accelerated residual state is initialized from the corresponding slow residual state to avoid gradient conflict during training (see ~\cref{sec:grad_conflict}). Early exiting decisions are informed by the Accelerated Residual Latent Attention (ARLA) mechanism, described in \cref{method_arla}, which evaluates residual dynamics across consecutive exit gates.}
    \label{fig:m2r2_main}
\end{figure}

% Furthermore. to maximize the benefits of fast residual transformations without using dedicated KV caches, we propose sharing the fast network’s cache with the slow network. Formally speaking, We modify attention operation on slow residuals $MHA(H_t, H_{<=t}, H_{<=t})$ as $MHA(P_{t}, H_{<t} \oplus P_t, H_{<t}  \oplus P_t)$ such that accelerated residuals attend to previous slow context KV cache, where $MHA(q,k,v)$ denotes multi head attention between query, $q$, key $k$ and value $v$.


\subsection{Enhanced Early Residual Alignment}
Early residual alignment is instrumental in optimizing early exiting, speculative decoding, and Mixture-of-Experts (MoE) inference mechanisms. In this section, we provide a detailed analysis of how accelerated residuals enhance these inference setups.

% By aligning the residual states of intermediate layers with the final output representations, the model can maintain high prediction accuracy even when computations are truncated at earlier layers. This enables more reliable early exiting, reducing the overall computational cost while preserving performance. Additionally, in speculative decoding, early residual alignment allows the model to make confident predictions using faster, partial computations, thereby accelerating inference without sacrificing output quality.


\subsubsection{Early Exiting} \label{method_early_exiting}

A prevalent strategy for enabling early exiting at an intermediate layer $E_{j}$ involves approximating the residual transformation between $E_{j}$ and the final layer $N-1$ using a linear, context independent mapping, $\mathcal{T}$, such that $H_{N-1} \approx \mathcal{T}(H_{E_{j}})$. This approximation has been extensively employed in conventional approaches ~\cite{schuster2022confident, chen2023eellm, varshney-etal-2024-investigating}, providing a computationally efficient means to project the output of deeper layers from intermediate states. Specifically, residual state of layer $N-1$ with this approximation can be expressed as:


% \begin{equation}
% \label{eq: vanila_ea_assumption}
% \Phi(H_{E_{j}}) \sim H_{E_{j}} \prod_{i=E_{j}}^{N}\left( I + \mathcal{A}_i \right) \left( I + \mathcal{M}_i \right) \quad \text{where} \quad \Phi \perp C
% \end{equation}

\begin{equation} \label{eq:early_exiting}
h_{E_j} + \sum_{i=E_j}^{N-1} \left( \mathcal{A}_i(h_i) + \mathcal{M}_i(h_i + \mathcal{A}_i(h_i)) \right) \sim \mathcal{T}(h_{E_{j}})  \quad \text{where} \quad \mathcal{T} \perp c. 
\end{equation}


Here, $\mathcal{A}_i$ and $\mathcal{M}_i$ represent the residual contributions of the multi-head attention and MLP layers, respectively, while $\mathcal{T}$ remains independent of $c$, the preceding context.

This approach is inherently limited by two major factors: first, the assumption of linearity between $h_{E_{j}}$ and $h_{N-1}$ may not hold uniformly for all tokens, particularly when $E_j \ll N$. Second, the linear transformation $\mathcal{T}$ disregards the influence of the context $c$ and fails to account for the latent representations of previous contextual states. In contrast, M2R2 accelerated residual states mitigate both of these challenges by approximating the slow residual transformation of all layers via a faster residual transformation of fewer layers as:
% \begin{equation}
% H_{E_j} \prod_{i=E_j}^{N}\left( I + \mathcal{A}_i \right) \left( I + \mathcal{M}_i \right) \sim P_{E_j} \prod_{i=E_j}^{E_j+1}\left( I + \hat{\mathcal{A}_i} \right) \left( I + \hat{\mathcal{M}_i} \right)
% \end{equation}


\begin{equation} \label{eq:m2r2_approximating_ea}
h_{E_j} + \sum_{i=E_j}^{N-1} \left( \mathcal{A}_i(h_i) + \mathcal{M}_i(h_i + \mathcal{A}_i(h_i)) \right) \sim p_{E_j} + \sum_{i=E_j}^{E_{j+1}-1} \left( \hat{\mathcal{A}_i}(p_i) + \hat{\mathcal{M}_i}(p_i + \hat{\mathcal{A}_i}(p_i)) \right), 
\end{equation}

% \begin{equation} \label{eq:fast_residual_transformation}
% p_{E_{j+1}} = p_{E_j} + \sum_{i=E_j}^{E_{j+1}-1} \left( \hat{\mathcal{A}_i}(p_i) + \hat{\mathcal{M}_i}(p_i + \hat{\mathcal{A}_i}(p_i)) \right) \quad \text{where} \quad \hat{\mathcal{A}_i} = \hat{f}(c_i, p_{i}), \hat{\mathcal{M}_i} = \hat{g}(h_i) 
% \end{equation}






where $p_{E_j}$ is initialized from the slow residual state $h_{E_j}$ at each early exit interval $E_j$ using an identity transformation (see ~\cref{fig:m2r2_main}). As shown in ~\cref{fig:m2r2_residual_sim}, accelerated residuals offer a smoother, more consistent shift in residual direction across layers, in contrast to the abrupt changes typically seen at early exit points in standard early exit methods. Moreover, the normalized cosine similarity between accelerated states at early exit intervals and final residual states is substantially higher compared to traditional early exit techniques, highlighting improved alignment with final layer representations. Traditional adaptive compute methods are constrained by two principal factors: the number of tokens eligible for early exit at intermediate layers and the precision of early exit decision. If residual streams fail to saturate early, the majority of tokens remain ineligible for exit, thereby diminishing potential speedups. Additionally, imprecise delineations between tokens suitable for early exit can lead to underthinking (premature exits that adversely affect accuracy) or overthinking (unnecessary processing that compromises efficiency) ~\cite{zhou2020self, dai2020dynamic}. Enhanced early alignment using ~\cref{eq:m2r2_approximating_ea} helps to address  first issue. To address the second issue we introduce Accelerated Residual Latent Attention, which dynamically assesses the saturation of the residual stream, allowing for a more precise differentiation between tokens that can exit early and those requiring further processing.

% This results in uniform change in residual direction    
% % We keep $\mathcal{A} = \hat{\mathcal{A}}$, while $\hat{\mathcal{M}}$ is accelerated by a factor of $2 - (N_{E}+1)X$ relative to the slower residual transformation $\mathcal{M}$, where $N_E$ represents number of early exiting intervals.
% Figure~\cref{fig:rate_change_comparison} illustrates the comparative rate of change between these transformation streams.



% fig:rate_change_comparison
% - grid plot x axis -> layer id (0, 8) , y axis -> layer id -> dark color cell for max similarity , lighter for lower 
% 
-------------------------------------------------------
Let's consider residual stream $h_i$ traverses through interval $E_j$ to $E_{j+1}$ and undergoes residual transformation given by 
\begin{equation}
h_{E_{j+1}} = h_{E_j} \prod_{i=E_j}^{E_{j+1}} \left( 1 + \delta_i \right)    
\end{equation}

where $\delta_i$ denotes non-linear transformation added by layer $i$. Each non-linear transformation of layer $i$ is a function of previous contextual representation, $c_i$ and input residual stream $h_i-1$ as
$\delta_i = f(c_i, h_{i-1})$ 

One way to exit early at exit $E_j+1$ is to assume that residual transformation from $E_j+1$ to final layer $N-1$ can be approximated by a linear function $\phi$ as $h_{N-1} \sim \Phi(h_{E_j+1})$ and most conventional approaches such as \todo{cite EA papers} use this approach. In other words, 

\begin{equation}
\Phi(h_{E_j+1} \sim h_{E_j+1} \prod_{i=E_j+1}^{N} \left( 1 + \delta_i \right)   
\end{equation}

This approach suffers from two primary issues, linearity assumption from $h_E_j+1$ to $H_N-1$ if often incorrect, particularly when $E_j << N$. More importantly, linear transformation $\Phi$ doesn't consider effect of context $C_i$. M2R2  effectively addresses these issues as accelerated residual stream at interval $E_j+1$ can be represented as 

\begin{equation}
r_{E_{j+1}} = r_{E_j} \prod_{i=E_j}^{E_{j+1}} \left( 1 + \gamma_i \right)    
\end{equation}

where $\gamma_i$ denotes non-linear transformation added by layer $i$ to previous accelerated residual $r_i-1$. Similar to $\delta_i$, non-linear transformation $\gamma_i$ considers context $C_i$ as 
$\gamma_i = g(c_i, r_{i-1})$. So in summary, slow residual transformation is approximated by accelerated residual as: 

\begin{equation}
h_{E_j} \prod_{i=E_j}^{N} \left( 1 + \delta_i \right) \sim h_{E_j} \prod_{i=E_j}^{E_j+1} \left( 1 + \gamma_i \right)
\end{equation}

It's worth noting that accelerated residual $r_i$ and slow residual $h_i$ are processed concurrently at layer $i$ by constructing proper attention mask such as attention of slow residual is represented as 

$MHA(H_it, H_{i<=t}, H_{i<=t}$ while attention of fast residual is computed as 

$MHA(r_it, H_{i<=t}, H_{i<=t}$ where $MHA(q,k,v$ denotes multi head attention between query, $q$, key $k$ and value $v$.


------------------------------------------------------------------

Vertical latent attention on accelerated residual is computed as 
$MHA(S_mt, S(Ej<=i<=m)t, S(Ej<=i<=m)t)$ where $Smt$ denotes query/key/value projection in latent domain at layer $m$ at time $t$. 
------------------------------------------------------------------

Gradient conflict Avoidance: 

Let's consider $w_j$ is a trainable parameter that belongs to a layer between $E_j$ and $E_j+1$. Consider early exit loss at gate $E_j+1$, $L_j+1$, gradient propagation of $w_j$ at another trainable parameter $w_j-n$ can be gives as 

$\sum_{k=E_j-n}^{E_j} \beta_k \frac{\partial L_{E_k}}{\partial w_k}$

where $\beta_j$ denotes backward transformation coefficient for weight $w_j$ to reach gate $E_j$. 
 
On the other hand, gradient propagation in proposed approach can be represented as 

\[
\frac{\partial L_{E_j}}{\partial w_j} = 
\begin{cases} 
\beta_j \frac{\partial L_{E_j}}{\partial w_j} & \text{if } E_j \leq w_j \leq E_{j+1} \\
0 & \text{otherwise}
\end{cases}
\]







% \begin{figure}[ht]
%     \centering
%     \includegraphics[width=0.8\textwidth, height=5cm]{rate_change_comparison.png}
%     \caption{Rate of change comparison between fast and slow residual streams.}
%     \label{fig:rate_change_comparison}
% \end{figure}

%vary k and and plot EA accuracy for larger and smaller models. 

% \begin{figure}[ht]
%     \centering
%     \includegraphics[width=0.5\textwidth,height=5cm]{sections/figures/alignment_comparison_dialogsum.pdf}
%     \caption{Alignment of exited tokens for different early exit layers using traditional early exiting heads, dedicated faster networks, and faster residuals.}
%     \label{fig:small_model_early_exiting}
% \end{figure}


\textbf{Accelerated Residual Latent Attention} \label{method_arla}

In the context of residual streams, we observe that the decision to exit at a given layer can be more effectively informed by analyzing the dynamics of residual stream transformations, instead of solely relying on a classification head applied at the early exit interval $E_j$. To capture the subtle dynamics of residual acceleration, we propose a \textit{Accelerated Residual Latent Attention} (ARLA) mechanism. This approach involves making the exit decision at gate $E_j$ by attending to the residuals spanning from gate $E_{j-1}$ to $E_j$, rather than considering only the residual at gate $E_j$. To minimize the computational overhead associated with exit decision-making, the attention mechanism operates within the latent domain as depicted in ~\cref{fig:arla_arch}. Formally, for each interval $[E_j, E_{j+1}]$, the accelerated residuals are projected into Query ($Q^s_{E_j}, \ldots, Q^s_{E_{j+1}}$), Key ($K^s_{E_j}, \ldots, K^s_{E_{j+1}}$), and Value ($V^s_{E_j}, \ldots, V^s_{E_{j+1}}$) vectors, with latent dimension $d^s$ for $Q^s$, $K^s$, and $V^s$ being significantly smaller than hidden dimension of $p$.\footnote{We use $d^s = 64$ for experiments described in ~\cref{sec:experiments}.} Notably, when the router is allowed to make exit decisions at gate $E_j$ based on residual change dynamics, we observe that the attention is not confined to the residual state at $E_j$ but is distributed across residual states from $E_{j-1}$ to $E_j$, %as illustrated in Figure~\ref{fig:vertical_latent_attention_dynamics}. 
This broader focus on residual dynamics significantly reduces decision ambiguity in early exits, as demonstrated in Figure~\ref{fig:roc_arla}, which contrasts routers based on the last hidden state, and the proposed ARLA router.

%show R -> S transformation. 
%show parameter and flop overhead as compared to adapter on last hidden state.

% \begin{figure}[ht]
%     \centering
%     \includegraphics[width=0.5\textwidth,height=5cm]{sections/figures/roc_arla.pdf}
%     \caption{ROC curves of early exit decision strategies: confidence-based methods (CALM/LITE), routers based on the accelerated hidden state, and latent attention routers.}
%     \label{fig:decision_making_comparison}
% \end{figure}

% \begin{figure}[ht]
%     \centering
%     \includegraphics[width=0.5\textwidth,height=5cm]{vertical_latent_attention.png}
%     \caption{Vertical latent attention mechanism for optimizing early exit decisions by considering residuals from gate \(M\) through \(M-1\).}
%     \label{fig:vertical_latent_attention}
% \end{figure}

\begin{figure}[ht]
    \centering
    \begin{subfigure}{0.52\textwidth}
        \centering
        \includegraphics[width=\textwidth, height = 4cm]{sections/figures/arla_arch.pdf}
        \caption{Accelerated Residual Latent Attention (ARLA): Accelerated residuals between early exit gates are projected into latent domain and attention over residual states within the interval is computed to capture residual dynamics and exit decision is made based on residual saturation.}
        \label{fig:arla_arch}
    \end{subfigure}%
    \hfill
    \begin{subfigure}{0.45\textwidth}
        \centering
        \includegraphics[width=\textwidth, height = 4.5cm]{sections/figures/vla_roc.pdf}
        \caption{ROC classification curves of early exit decision strategies using a linear router used on last residual state ~\cite{schuster2022confident, varshney-etal-2024-investigating, chen2023eellm}  and using ARLA approach that considers residual dynamics. }
        \label{fig:roc_arla}
    \end{subfigure}
    \caption{Effectiveness of ARLA in capturing residual dynamics for early exiting decisions.}


\end{figure}



% \begin{figure}[ht]
%     \centering
%     \includegraphics[width=1\textwidth,height=5cm]{sections/figures/arla.pdf}
%     \caption{fig that plots 32 rows 2 cols heatmap showing attention at each gate}
%     \label{fig:vertical_latent_attention_dynamics}
% \end{figure}

\subsubsection{Self Speculative Decoding} \label{method_self_speculative_decoding}

An alternative means to exploit the early alignment properties of our approach is through the use of accelerated residual states for speculative token sampling to accelerate autoregressive decoding. Speculative decoding aims to speed up memory-bound transformer inference by employing a lightweight draft model to predict candidate tokens, while verifying speculated tokens in parallel and advancing token generation by more than one token per full model invocation \cite{leviathan2023fast, chen2023accelerating, xia2023speculative, miao2023specinfer}. Despite its effectiveness in accelerating large language models (LLMs), speculative decoding introduces substantial complexity in both deployment and training. A separate draft model must be specifically trained and aligned with the target model for each application, which increases the training load and operational complexity ~\cite{chen2023accelerating}. Additionally, this approach is resource-inefficient, as it requires both the draft and target models to be simultaneously maintained in memory during inference \cite{leviathan2023fast, chen2023accelerating}. 

One strategy to address this inefficiency is to leverage the initial layers of the target model itself to generate speculative candidates, as depicted in ~\cite{Tang2024}. While this method reduces the autoregressive overhead associated with speculation, it suffers from suboptimal acceptance rates. This occurs because the linear transformation employed for translating hidden states from layer $k$ to the final layer $N$ is typically a poor approximation, as discussed in ~\cref{sec:motivation} and ~\cref{method_early_exiting}. Our approach resolves this limitation by utilizing accelerated residuals, which demonstrate higher fidelity to their slower counterparts. By utilizing accelerated residuals operating at a rate of $N/k$, where $k$ denotes the number of layers used for candidate speculation, we are able to efficiently generate speculative tokens for decoding.\footnote{We typically set $k = 4$ to balance the trade-off between autoregressive drafting overhead and acceptance rate, as discussed in~\cref{sec:experiments}.}
 This technique not only obviates the need for multiple models during inference but also improves the overall efficiency and effectiveness of speculative decoding.

\begin{figure}
    \centering    \includegraphics[width=1\linewidth]{sections/figures/m2r2_aot_loading.pdf}
    \caption{Ahead-of-Time Expert Loading: M2R2 accelerated residual stream predicts experts required for future layers, reducing reliance on on-demand lazy loading. Speculative pre-loading is efficiently overlapped with computation of multi-head attention (MHA) and MLP transformations. Only incorrectly speculated experts are loaded lazily, resulting in faster inference steps and improved computational efficiency. Here, H indicates LBM Host while D indicates HBM Device.}
    \label{fig:moe_expert_aot_loading}
\end{figure}


\subsubsection{Ahead of Time Expert Loading:} \label{method_aot_expert_loading}

Recent advancements in sparse Mixture-of-Experts (MoE) architectures ~\cite{shazeer2017outrageously, fedus2022switch, artetxe2019massively, lepikhin2020gshard, zoph2022designing} have introduced a paradigm shift in token generation by dynamically activating only a subset of experts per input, achieving superior efficiency in comparison to dense models, particularly under memory-bound constraints of autoregressive decoding \cite{fedus2022switch, zoph2022designing}. This sparse activation approach enables MoE-based language models to generate tokens more swiftly, leveraging the efficiency of selective expert usage and avoiding the overhead of full dense layer invocation. In dense transformer models, pre-loading layers is a common strategy to enhance throughput, as computations of current layer can be overlapped with pre-loading of next layer parameters ~\cite{narayanan2021efficient, shoeybi2020megatron}. However, MoE models face a unique challenge: expert selection occurs dynamically based on previous layer’s output, making it infeasible to preload next layer’s experts in parallel. This limitation results in inherent latency, as expert loading becomes a sequential, on-demand process ~\cite{lepikhin2020gshard, fedus2022switch}.

To address this inefficiency, our method introduces a mechanism with \textit{accelerated residuals}, which not only captures key characteristics of base slower residual states but also exhibit high cosine similarity with their final counterparts (as illustrated in \cref{fig:m2r2_residual_sim}). By employing accelerated residual streams, we can effectively predict the necessary experts for future layers well in advance of their actual invocation. Specifically, using a $2\times$ accelerated residual, the experts needed for layers $2i+2$ and $2i+3$ can be identified while still computing in layer $i$, thus overcoming the bottleneck of sequential, on-demand expert selection and mitigating latency in the decoding pipeline, as shown in \cref{fig:moe_expert_aot_loading}. Note that, we use fixed set of accelerator adapters for transforming accelerated residuals (as discussed in ~\cref{m2r2_method}) while slow residual is transformed via expert routing mechanism. 

Furthermore, our approach integrates a Least Recently Used (LRU) caching strategy, which enhances memory efficiency by replacing the least recently used experts with speculated experts that are anticipated to be needed in upcoming layers. This hybrid approach of preemptive expert loading with LRU caching yields substantial improvements over traditional on-demand loading or standalone caching strategies. By minimizing cache misses and efficiently managing memory, this approach addresses both compute and memory bottlenecks, leading to faster, more resource-efficient token generation in MoE architectures. A comprehensive evaluation of this strategy, in relation to state-of-the-art methods, is provided in \cref{experiments_aot}, and the compute and memory traces on an A100 GPU are detailed in \cref{fig:moe_aot_cuda_trace}.



% Recent advancements in sparse Mixture-of-Experts (MoE) architectures have introduced the concept of utilizing distinct computational paths for different tokens \cite{shazeer2017outrageously}. This approach, wherein only a subset of experts are activated per input, enables MoE-based language models to generate tokens more swiftly compared to their dense counterparts due to memory-bound nature of auto-regressive decoding. In dense models, pre-loading layers in advance is a common strategy to enhance computational efficiency. However, this technique is not applicable to MoE models, where expert selection occurs dynamically based on the outputs of previous layers, preventing parallel pre-fetching of experts.

% Our proposed method addresses this inefficiency. Accelerated residuals, which are highly similar to their slower counterparts (see \cref{fig:similarity}), can reliably predict the necessary experts ahead of time. For instance, by utilizing $2X$ accelerated residual stream, we can predict the experts needed for the layer $2i+1$ and $2i+3$ while carrying out computation in layer $i$. This enables us to commence expert loading significantly earlier, as illustrated in \cref{expert_loading}, effectively mitigating the delays observed with the naive on-demand expert loading. Additionally, our method benefits from incorporating a Least Recently Used (LRU) strategy, where speculated experts replace those that are least recently utilized, resulting in improved performance compared to using either strategy alone. For a comprehensive evaluation, refer to \cref{moe_trace}, which provides a CUDA compute and memory trace of our approach executed on <>.



% A naive solution involves using the residual state of the previous layer along with the gating function of the next layer to predict which experts need to be loaded, and initiating the expert loading process in parallel with the attention computation of the next layer. Yet, as shown in \cref{fig:MOE_attn_vs_loading_time}, the attention computation for medium to long contexts is considerably faster than the expert loading time, making this approach inefficient.




\subsection{Training} \label{method_training}
% This approach is feasible due to the absence of gradient conflicts, as discussed in \cref{sec:grad_conflict}.

To accelerate residual streams, we employ parallel accelerator adapters as described in \cref{m2r2_method}.  For the early exiting use-case outlined in \cref{method_early_exiting}, we define the training objective for these adapters using the following loss function, which combines cross-entropy loss at each exit $E_j$ with distillation loss at each layer $i$. Loss weights coefficients $\alpha_0$ and $\alpha_1$ are employed to balance contribution of corresponding losses.

\begin{align} \label{eq:mr_loss}
L_{\text{m2r2}} = \underbrace{-\alpha_0 \sum_{j=1}^{J} \sum_{t=1}^{T} \log p_{\theta} \left( \hat{y}_t^{E_j} \mid y_{<t}, x \right)}_{\text{cross-entropy loss}} 
+ \underbrace{\alpha_1\sum_{i=1}^{E_{J-1}} \sum_{t=1}^{T} \| \mathbf{p}_{t}^{i} - \mathbf{h}_{t}^{((i - E_{j(i)}) \cdot R_i) + E_{j(i)})} \|^2}_{\text{distillation loss}}.
\end{align}

where $\hat{y}_t^{E_j}$ denotes the predictions from the accelerated residual stream at layer $E_j$ and time step $t$, $y_t$ represents the corresponding ground truth tokens, and $x$ indicates previous context tokens. The distillation loss at each layer $i$ is computed by comparing accelerated residuals at layer $i$ with slow residuals at layer $(i - E_{j(i)}) \cdot R_i + E_{j(i)}$, where $R_i$ denotes the rate of accelerated residuals at layer $i$ while $E_{j(i)}$ represents the most recent gate layer index such that $E_{j(i)} <= i$. \( J \) represents the total number of early exit gates, N denotes number of hidden layers and $E_j$ denotes layer index corresponding to gate index $j$ and \( T \) denotes the sequence length. 

In dynamic compute settings, after training of accelerator adapters, we optimize the query, key, and value parameters governing the ARLA routers (see ~\cref{method_arla}) across all exits in parallel on binary cross entropy loss between predicted decision and ground truth exiting decision. The ground truth labels for the router are determined based on whether the application of the final logit head on $\hat{y}_t^{E_j}$ yields the correct next-token prediction. 


% The objective for this optimization is defined by the following loss function:


%TODO are equations required ? 
% \begin{equation} \label{eq:arla_loss_combined}\small
%     L_{\text{arla}} = -\frac{1}{N} \sum_{t=1}^{T} \left( \sum_{j=1}^{E_n} \left[ O_t^{E_j} \log(\hat{O}_t^{E_j}) + (1 - O_t^{E_j}) \log(1 - \hat{O}_t^{E_j}) \right] \right), \quad \text{where} \quad 
%     O_t^{E_j} = \begin{cases} 
%     1, & \text{if } L(\hat{y}_t^{E_j}) = y_t^{E_j} \\
%     0, & \text{otherwise}
%     \end{cases}
% \end{equation}

% where $\hat{O}_t^{E_j}$ represents the binary predicted logits produced by the vertical latent attention router, as described in \cref{sec:arla}, at gate $E_j$ and time step $t$, and $O_t^{E_j}$ denotes the corresponding ground truth labels. The ground truth labels for the router are determined based on whether the application of the logit head on $\hat{y}_t^{E_j}$ yields the correct next-token prediction. The parameters controlling vertical latent attention are trained concurrently to ensure consistency and efficient use of computational resources.

For self-speculative decoding, as described in \cref{method_self_speculative_decoding}, the training objective remains the same as \cref{eq:mr_loss}, but with the number of intervals set to $J = 1$ and the rate of residual transformation set to $R_n = N/k$, where the first $k$ layers generate speculative candidate tokens. In the context of Ahead-of-Time Expert Loading for Mixture-of-Experts (MoE) models (see \cref{method_aot_expert_loading}), setting the rate of residual transformation to $R_n = 2$ typically offers a good trade-off between the accuracy of expert speculation and AoT pre-loading of experts. 

% Thus, we set $J = 1$ and $E_1 = 16$.


~\subsection{FLOPs Optimization} \label{sec:flops_optimization}

Naively implemented, M2R2 incurs higher FLOP overhead compared to traditional speculative decoding and early exiting approaches such as ~\cite{medusa, schuster2022confident, Tang2024}. However, modern accelerators demonstrate compute bandwidth that exceeds memory access bandwidth by an order of magnitude or more~\cite{databricksLLMInference2023, jouppi2021ten}, meaning increased FLOPs do not necessarily translate to increased decoding latency. Nevertheless, to ensure fair comparison and efficiency in compute bound scenarios, we introduce targeted optimizations.

~\textbf{Attention FLOPs Optimization} For medium-to-long context lengths, attention computation dominates FLOPs in the self-attention layer, surpassing the contribution from MLP layers. Specifically, matrix multiplications involving queries, cached keys, and cached values scale with $l_{kv} * l_{q}$ where $l_{kv}$ denotes previous context length and $l_q$ denotes current query length. Since M2R2 pairs accelerated residuals with slow residuals, a naive implementation results in twice the FLOPs consumption compared to a standard attention layer. To address this, we limit the attention of accelerated residual stream to selectively attend to the top-k most relevant tokens, identified by the slow residual stream based on top attention coefficients\footnote{We set to k = 64 and attend to top 64 tokens as identified by the slow residual stream.}. This is possible since slow and accelerated residual streams are processed in same forward pass and accelerated streams have access to attention coefficients of slow stream. Note that, the faster residual stream still retains the flexibility to assign distinct attention coefficients to these tokens. Furthermore, we design the faster residual stream to employ only 8 attention heads, compared to the 32 heads used in the slow residual stream of the Phi-3 model, reducing query, key, value, and output projection FLOPs by a factor of 1/4. ~\cref{fig:m2r2_num_heads_ablation} indicates effect of using a slicker stream on alignment. As depicted, using $\hat{n}_h = 8$ offers a good trade-off between alignment and FLOPs overhead. 

~\textbf{MLP FLOPs Optimization} The accelerator adapters operating on the accelerated residual stream are intentionally designed with lower rank than their counterparts in the base model. This reduces FLOP overhead by a factor proportional to $hiddenSize / rank$. Additionally, since the faster residual stream uses only 8 attention heads (compared to 32 in the slow residual stream of Phi-3), the subsequent MLP layers process a smaller set of activations, further reducing FLOPs by another factor of 1/4.

These optimizations significantly reduce the FLOP overhead per speculative draft generation, as illustrated in ~\cref{fig:flops_optmization}. Notably, while traditional early-exiting speculative approaches such as DEED require propagating the full slow residual state through the initial layers, incurring substantial computational costs, M2R2 achieves efficient token generation via slimmer, low-rank faster residual streams. In contrast, Medusa introduces considerable FLOP overhead due to per-head computations scaling with $d^2+dv$\footnote{Here $d$ denotes hidden state dimension while $v$ denotes vocab size.}, whereas M2R2 employs low-rank layers for both MLP and language modeling heads, maintaining computational efficiency. All experiments involving the M2R2 approach, as detailed in ~\cref{sec:experiments}, are conducted using these FLOPs optimizations.









% \[
% O_t^{E_j} = 
% \begin{cases} 
% 1, & \text{if } L(\hat{y}_t^{E_j}) = y_t^{E_j} \\
% 0, & \text{otherwise}
% \end{cases}
% \]




%add distillation
% We train accelerator adapters described in \cref{m2r2_method} to accelerate residual streams on next token prediction all in parallel since there are no gradient conflict issues as described in \cref{sec:grad_conflict}.

% \begin{align} \label{eq:mr_loss}
% L_{mr} =  & -\sum_{j = 1}^{E_n} (\sum_{t=1}^{T}\log p_{\theta} (\hat{y}_t^{E_j} | \hat{y}_{<t}, x)) \nonumber
% \end{align}

% where $\hat{y_t^{E_j}}$ denotes predicted logits obtained from accelerated residual stream at gate $E_j$ and time-step $t$ while $y_t^{E_j}$ denotes corresponding truth tokens. 

% Upon training of adapters responsible for accelerating residual streams, we train query, key, value parameters responsible for vertical latent attention of all gates in parallel as

% \begin{equation} \label{eq:arla_loss}
%     L_{arla} = -\frac{1}{N} (\sum_{t=1}^{T}(1\sum_{j=1}^{E_n} \left[ O_t^{E_j} \log(\hat{O}_t^{E_j}) + (1 - o_t^{E_j}) \log(1 - \hat{o_t}_{E_j}) \right]))
% \end{equation}

% where $\hat{O_t^{E_j}}$ denotes binary predicted logits obtained from vertical latent attention router described in \cref{sec:arla} at gate $E_j$ and timestep $t$ while $O_t^{E_j}$ denotes corresponding truth label. Truth labels for router are obtained by computing whether logit head application on $\hat{y}_t^j$ results in true next token prediction. Formally speaking, 

% $O_t^{E_j} = 1 if L(\hat{y_t^{E_j}}) == y_t^{E_j} , 0 otherwise$. 

% Parameters responsible for vertical latent attention are also trained in parallel as well. 

%todo: training slow and fast residuals together and distillation can be two training mdoes. 
%Distillation can be an ablation. 




% Although transformer decoding is memory bound on most mainstream accelerators, there could be scenarios where flop savings are crucial. For instance, on on-device settings power consumption is directly correlated with flops per decoding step and reducing flops does help with overall energy consumption. Vanilla early exiting methods help with flop reduction but suffer from mismatch between training and inference due to early exited tokens. If token at decoding step $t$, $T_t$ exited at layer $E_i$, while token $T_{t+k}$ exits at layer $E_j$ such that $E_i < E_j$, hidden state $H_{t+k}l$ does not have corresponding hidden state $H_tl$ to attend to where $E_i < l <= E_j$. One solution that's often used in literature is to rely on last hidden state available, $H_t{E_j}$, however it tends to be sub-optimal and does affect generation quality \cite{ref}.  To alleviate this mismatch while reducing flops, we train router such that attention mask between token $T_{t+k}$ and token $T_{<t+k}$ is given by: 

% \begin{equation}
%     a_{T_{{t+k}{T_{<t+k}}} = 1 if  E_{T_{<t+k}} >= E{T_{t+k}}
%     else 0
% \end{equation}

% This attention mask enables router to account for exited tokens and get trained accordingly. Since attention mechanism during decoding remains exactly same as that during training, impact on generation quality tends to be minimal as noted in \cref{fig:gen_auality_with_and_without_recompute_attention_show_flops}.  Although MoD does not suffer from training and inference mismatch, we observe that it suffers from discountinuity between pre-training and super-vised fine-tuning resulting in sub-optimal perplexity. On the other hand, our method doesn't not require pre-training , doesn't suffer from discountinuity, and achieves much better perplexity in super-vised fine-tuning and instruction tuning setups as shown in \cref{fig:Mod_vs_m2r2_loss_curves}.






% Our techniques are directly applicable in such scenarios.    




%expert loading with cuda streams in experiments
% \section{Related Work}
% \subsection{Vision Language Model}
% 시각장애인에서 상황을 설명할 DB가 없으니 만들었다. 그리고 이를 VLM에 튜닝했다.
\subsection{Technical approaches for assisting the visually-impaired}


\subsection{Datasets for visual instruction tuning}

\section{Experiments: Planning outperforms Heuristics}
\label{sec:experiment}

We begin our empirical demonstrations by showcasing the effectiveness of our planning framework on both synthetic and real datasets. We focus on the simplest planning algorithm, 1-step lookaheads (Algorithm~\ref{alg:complete}), and show that even basic planning can hold great promise. 
We illustrate our framework using two uncertainty quantification modules---GPs and 
\ensembles/ \ensembleplus. 

Throughout this section, we focus on evaluating the mean squared error of 
a regression model $\model$,  and develop adaptive policies that minimize uncertainty on $g(f)$ defined in~\eqref{eqn:l2-g-f}.
When GPs provide a valid model of uncertainty, 
our experiments show that our planning framework significantly outperforms other baselines. 
We further demonstrate that our conceptual framework extends to deep learning-based uncertainty quantification methods such as  \ensembleplus while highlighting computational challenges that need to be resolved in order to scale our ideas. 
For simplicity, we assume a naive predictor, i.e., $\psi(\cdot) \equiv 0$. However, we emphasize that this problem is just as complex as if we were using a sophisticated model $\psi(.)$. The performance gap between the algorithms 
primarily depends
on the level  of uncertainty in our prior beliefs.

To evaluate the performance of our algorithm, we benchmark it against several baselines. 
%Active learning baselines use an acquisition function $\ac$ to select points that have the highest   function value: $X\opt_t \in \argmax_{X \in \xpoolj{t}} \ac({X})$ at every step $t$. These methods may also need an UQ module, which we simply use the same UQ module as in our algorithm, and it  outputs $V(X)$ that measures the the uncertainty of each point $X \in \xpoolj{t}$.
Our first set of baselines are from active learning~\citep{AggarwalKoGuHaPh14}:
\\ % \noindent\textbf{Active Learning Heuristics:} 
\textbf{(1)} 
\textsf{Uncertainty Sampling (Static):}  In this approach, we query the samples for which the model is least certain about. Specifically, we estimate the variance of the latent output $f(X)$ for each $X \in \xpool$ using the UQ module and select the top-$K$ points with the highest uncertainty. \\
\textbf{(2)} \textsf{Uncertainty Sampling (Sequential):} This is a greedy heuristic that sequentially selects the points with the highest uncertainty within a batch, while updating the posterior beliefs using pseudo labels from the current posterior state. Unlike \textsf{Uncertainty Sampling (Static)}, this method takes into account the information gained from each point within batch, and hence tries to diversify the selected points within a batch. 

 
We also compare our approach to the  \textbf{(3)} \textsf{Random Sampling}, which selects each batch uniformly at random from the pool. Additionally, we compare solving the planning problem using  \textsf{REINFORCE}-based policy gradients with   $\mathsf{Smoothed\text{-}Autodiff}$ policy gradients.\footnote{Our code repository is available at
  \url{https://github.com/namkoong-lab/adaptive-labeling}.}
%Detailed experimental setups are provided in Section \ref{sec:details-experiments}.

%We repeat all experiments with 10 random seeds.




\begin{figure}[t]
\centering
\begin{minipage}[b]{0.49\textwidth}
\centering
\includegraphics[width=\textwidth, height=5cm]{figures/original_scale/Var_of_l_2_loss.pdf}
\caption{(Synthetic data) Variance of mean squared loss evaluated through the posterior belief $\mu_t$ at each horizon $t$. This is the objective that policy gradient methods like \textsf{REINFORCE} and $\ouralgo$ optimizes. 1-step lookaheads are surprisingly effective even in long horizons.}
\label{fig:var-l2-sim}
\end{minipage}
\hfill
\begin{minipage}[b]{0.49\textwidth}
\centering \includegraphics[width=\textwidth, height=5cm]{figures/original_scale/Error_of_estimated_model_l_2_loss.pdf}
\caption{(Synthetic data) Error between MSE calculated based on collected data $\mc{D}^{0:T}$ vs. population oracle MSE over $\mc{D}_{\rm eval} \sim P_X$. Reducing uncertainty over posteriors directly leads to better OOD evaluations. 1-step lookaheads significantly outperform active learning heuristics in small horizons.}
\label{fig:mean-l2-sim}
\end{minipage}
%\caption{Simulated data for GPs}
%\label{fig:both_plots}
\end{figure}

\subsection{Planning with Gaussian processes}
\label{sec:experiment-plan-GP}
We now briefly describe the data generation process for the GP experiments,  deferring a more detailed discussion of the dataset generation to Section~\ref{sec:details-experiments}. 
We use both the synthetic data and the real data to test our methodology.
For the \emph{simulated data},  we construct a setting where the general population is distributed across \emph{51 non-overlapping clusters} while the initial labeled data $\dtrain$ just comes from one cluster. In contrast, both $\dpool \defeq (\xpool,\ypool),\deval \defeq (\xeval,\yeval)$ are generated   from all the clusters. 
We begin with a low-dimensional scenario, generating a one-dimensional regression setting using a GP. %Gaussian Process (GP).
Although the data-generating process is not known to the algorithms,  we assume that the GP hyperparameters are known to all the algorithms
to ensure fair comparisons. This can be viewed as a setting where our prior is well-specified, allowing us to isolate the effects
of different policy optimization approaches
 without any concerns about the misspecified priors. We select $10$ batches, each of size $K=5$ across $T = 10$ time horizons.

To examine the robustness of our method against the distributional assumptions made  in the simulated case, we then move to a real dataset where the correct prior is not known. We simulate selection bias from the eICU dataset~\citep{PollardJoRaCeMaBa18}, which contains real-world patient data with in-hospital mortality outcomes. 
We conduct a $k$-means clustering to generate 51 clusters and then select data from those clusters. We view this to be a credible replication of practice, as severe distribution shifts are common due to selection bias in clinical labels.  To convert the binary mortality labels into a regression setting, we train a  random forest classifier and fit a GP on predicted scores, which serves as the UQ module for all the algorithms. As before, the task is to select 10 batches, each consisting of 5 samples, across 10 time horizons.

 In Figures~\ref{fig:var-l2-sim} and~\ref{fig:mean-l2-sim}, we present results for the simulated data. 
Figure~\ref{fig:var-l2-sim} shows the variance of $\ell_2$ loss, and Figure~\ref{fig:mean-l2-sim} presents the error in the estimated $\ell_2$ loss using $\mu_t$ (relative to true $\ell_2$ loss, that is unknown to the algorithm). 
As we can see from these plots, our method one-step lookahead  gives substantial improvements  over active learning baselines and random sampling. In addition,
compared to the one-step lookahead planning approach using \textsf{REINFORCE}-based policy gradients, 
we observe that $\mathsf{Smoothed\text{-}Autodiff}$-based policy gradients provide significantly more robust performance over all horizons.

In Figures~\ref{fig:var-l2-real}~and~\ref{fig:mean-l2-real}, we observe similar findings on the eICU data. We see that planning policies (\textsf{REINFORCE} and $\mathsf{Smoothed\text{-}Autodiff}$) consistently outperform other heuristics by a large margin.  Active learning baselines perform poorly in these small-horizon batched problems and can sometimes be even worse than the random search baselines.  Overall, our results show the importance of careful planning in adaptive labeling for reliable model evaluation. 

We offer some intuition as to why one-step lookahead planning may outperform other heuristic algorithms. 
 First,  \textsf{Uncertainty sampling (Static)} while myopically selects the
 top-$K$ inputs with the highest uncertainty, it fails to consider 
the overlap in information content among the ``best” instances; see \citep{AggarwalKoGuHaPh14} for more details. 
In other words,  it might acquire points from the same region with high uncertainty while failing to induce diversity among the batch.
Although \textsf{Uncertainty Sampling (Sequential)} somewhat addresses the issue of information overlap, a significant drawback of 
this algorithm
is the disconnect between the objective we aim to optimize and the algorithm. For example, it might sample from a region with high uncertainty but very low density. 

\begin{figure}[t]
\centering
\begin{minipage}[b]{0.48\textwidth}
\centering
\includegraphics[width=\textwidth, height=5cm]{figures/original_scale/Var_of_l_2_loss_real.pdf}
\caption{(Real-world eICU data) Variance of mean squared loss evaluated through the posterior belief $\mu_t$ at each horizon $t$. Even 1-step lookaheads are extremely effective planners, and auto-differentiation-based pathwise policy gradients provide a reliable optimization algorithm based on low-variance gradient estimates.}
\label{fig:var-l2-real}
\end{minipage}
\hfill
\begin{minipage}[b]{0.48\textwidth}
\centering \includegraphics[width=\textwidth, height=5cm]{figures/original_scale/Error_of_estimated_model_l_2_loss_real.pdf}
\caption{(Real-world eICU data) Error between MSE calculated based on collected data $\mc{D}^{0:T}$ vs. population oracle MSE over $\mc{D}_{\rm eval} \sim P_X$. Reducing uncertainty over posteriors directly leads to better OOD evaluations. Our method significantly outperforms active learning-based heuristics, and random sampling.}
\label{fig:mean-l2-real}
\end{minipage}
%\caption{Real data for GPs}
\end{figure}
 
%\vspace{-1.5cm}
% \begin{wrapfigure}{r}{.32\columnwidth}
%   \vspace{-.5cm} 
%   \centering
% \includegraphics[scale=.29]{figures/Var of l2l_2 loss.pdf}
%   \vspace{-0.2cm}
%   \caption{Results of GP}
% \label{fig:var-l2-gp}
%   \vspace{-0.1cm}
% \end{wrapfigure}


% Attempts have been made  in the past to address these  drawbacks heuristically  (see \citep{AggarwalKoGuHaPh14}). We give a unified computational framework while approaching the problem in a more principled manner and solving it more optimally.




\subsection{Planning with  neural network-based uncertainty quantification methods ($\ensembleplus$)}


We now provide a proof-of-concept that shows the generalizability of our conceptual framework  to the deep learning-based UQ modules, specifically focusing on $\ensembleplus$ due to their previously observed superior performance~\citep{OsbandWenAsDwIbLuRo23}. Recall that implementing our framework with deep learning-based UQ modules  requires us to retrain the model across multiple possible random actions $\bm{a}(\theta)$ sampled from the current policy $\pi_\theta$.
This requires significant computational resources, in sharp contrast to the GPs where the posteriors are in closed form and can be readily updated and differentiated. 

Due to the computational constraints, we test $\ensembleplus$ on a toy setting to demonstrate the generalizability of our framework. We consider a setting where the general population consists of four clusters, while the initial labeled data only comes from one cluster. Again we generate data using GPs.  The task is to select a batch of 2 points in one horizon. We detail the $\ensembleplus$ architecture in Section \ref{sec:details-experiments}, and we assume prior uncertainty to be large (depends on the scaling of the prior generating functions). 
The results are summarized in the Table~\ref{tab:UQ_ensemble}.

% \begin{table}[H]
% \vspace{-10pt}
% \caption{Performance under \ensembleplus as UQ module}
%     \centering
%     \begin{tabular}{|m{3cm}|m{2.5cm}|m{2cm}|} 
%     \hline
%       Algorithm   & Variance of $\loss_2$ loss estimate & Error of $\loss_2$ loss estimate  \\ \hline Random Sampling 
%          & $1710.9 \pm 1352.1$ & $8.67\pm6.62$ 
%       \\ \hline \ouralgo & $1.30 \pm 0.68$ & $0.91\pm0.25$ \\ \hline
%     \end{tabular}
%     \label{tab:UQ_ensemble}
%     %\vspace{-10pt}
% \end{table}




\begin{table}[h]
\vspace{-10pt}
\caption{Performance under \ensembleplus as the UQ module}
\centering
\begin{tabular}{|l|l|l|}
\hline
Algorithm   & Variance of $\loss_2$ loss estimate & Error of $\loss_2$ loss estimate  \\
\hline
\textsf{Random sampling} & 7129.8 $\pm$ 1027.0 & 136.2 $\pm$ 8.28 \\ \hline
\textsf{Uncertainty sampling (Static)} & 10852 $\pm$ 0.0 & 162.156 $\pm$ 0.0 \\ \hline
\textsf{Uncertainty sampling (Sequential)} & 8585.5 $\pm$ 898.9 & 144 $\pm$ 6.93 \\ \hline
\textsf{REINFORCE} & 1697.1 $\pm$ 0.0 & 45.27 $\pm$ 0.0 \\ \hline
\ouralgo & 1697.1 $\pm$ 0.0 & 45.27 $\pm$ 0.0 \\ \hline
\end{tabular}
%\caption{Comparison of different algorithms based on variance   and   error in $\ell_2$ loss estimation with Ensemble $+$ as the UQ module. Our results demonstrate that {\ouralgo} and REINFORCE outperformthe other active learning based heuristics, confirming the benefits of our MDP formulation for the adaptive labeling problem, as also demonstrated in Section 4.\\
%\footnotesize{Experimental details: We use Gaussian Processes as our data generating process, GP parameters are the same as in Section D.3.  The task is to select a batch of 2 points along one horizon.The marginal distribution $p_X$ has 4 \textit{non-overlapping} clusters. Initial data comes from one cluster, while pool and evaluation points comes from all the clusters. We have $20$ initial labeled data points, $10$ pool points, and $252$ evaluation points.  Training procedures are similar to the one in Section D.3.} }
\label{tab:UQ_ensemble}
\end{table}



% We faced  issues in scaling up these experiments which will be our focus in the future. 





% \begin{itemize}
%     \item Posteriors should be consistent. Two dimensions: even with less training,  
%     \item the inference should be  fast enough
% \end{itemize}


% Potential research directions for uncertainty quantification

% In this section we consider a simple setting We consider a simpler setting and 


% For synthetic dataset generation, we use ...... For real datasets, we use ...... We compare our methodolgy to several baselines ()    This Section is structured as follows:
% \begin{itemize}
%     \item \textbf{GPs, square loss objective} (Section \ref{}): 
%     %the broad aim of the experiments  in this section is to isolate the performance of our methodology without any concerns for the inefficiencies induced due to a mis-specified prior or imperfect posterior inference. To accomplish this we generate synthetic datasets using GPs (detailed later). We use the well specified prior (GPs - with same hyperparameter setting) as our UQ module.   
%      As GPs provide differentaible posterior inference - any errors induced due to imperfect posterior updates are also isolated. We note that under this setting
%      \item In Section\ref{} we demonstrate why our methodology performs better than other baselines - by devising various synthetic experiments ()
%     \item  \textbf{UQ Benchmarking }(Section \ref{}): Before diving into the experiments using $\ensembleplus$ and ENNs,  we showcase our benchmarking experiments in Section \ref{}. We use real datasets We observe that ENNs perform better
%      \item \textbf{Ensemble $+$}, objective: recall, accuracy
%     \item \textbf{ENN}, objective: recall, accuracy
% \end{itemize}




% In Section {}, we test 
% \subsection{Experimental details}

% \begin{itemize}
%     \item UQ methodologies - GPs, ENNs
%     \item Objectives - Recall,  ATE
%     \item Datasets - ATE-synthetic datasets, Recall-synthetic, real datasets
%     \item Baselines - 
%     \begin{itemize}
%         \item Random sampling
%         \item Active learning - Uncertainty based sampling - In regression setting almost all of the 
%         \item Myopic greedy - Greedy Batch based sampling
%         \item Policy Gradient
%     \end{itemize}
    
% \end{itemize}

% \subsection{Experiments}
%     \begin{itemize}
%     \item GPs with square loss
%     \item Benchmarking ENN
%         \item ENNs with ATE
%         \item ENNs with Recall
%     \end{itemize}

% \subsection{Benefits over other algorithms - intuition and experiments}

%Active learning - Myopic greedy / Don't rely on the objective rather some entropy version.


%%% Local Variables:
%%% mode: latex
%%% TeX-master: "main"
%%% End:

This work identifies signal collapse as a critical bottleneck in one-shot neural network pruning. Performance loss in pruned networks is due to \textbf{signal collapse} in addition to the removal of critical parameters. We propose \textbf{REFLOW} (\textbf{Re}storing \textbf{F}low of \textbf{Low}-variance signals), a simple yet effective method that mitigates signal collapse without computationally expensive weight updates. By focusing on signal preservation, REFLOW highlights the importance of mitigating signal collapse in sparse networks and enables magnitude pruning to match or surpass state-of-the-art one-shot pruning methods such as CHITA, CBS, and WF.

REFLOW consistently achieves state-of-the-art accuracy across diverse architectures, restoring ResNeXt-101 from under 4.1\% to 78.9\% top-1 accuracy at 80\% sparsity on ImageNet. Its lightweight design makes it a practical solution for both research and deployment, delivering high-quality sparse models without the overhead of traditional approaches. These findings challenge the traditional emphasis on weight selection strategies and underscore the critical role of signal propagation for achieving high-quality sparse networks in the context of one-shot pruning.



\section*{Conclusion}
This paper aims to enhance our understanding of the computational complexity of computing various Shapley value variants. We found that for various ML models --- including decision trees, regression tree ensembles, weighted automata, and linear regression --- both local and global interventional and baseline SHAP can be computed in polynomial time under HMM modeled distributions. This extends popular algorithms, such as TreeSHAP, beyond their empirical distributional scope. We also establish strict complexity gaps between the various SHAP variants (baseline, interventional, and conditional) and prove the intractability of computing SHAP for tree ensembles and neural networks in simplified scenarios. Overall, we present SHAP as a versatile framework whose complexity depends on four key factors: \begin{inparaenum}[(i)] \item model type, \item SHAP variant, \item distribution modeling approach, \item and local vs. global explanations\end{inparaenum}. We believe this perspective provides deeper insight into the computational complexity of SHAP, paving the way for future work.




%We believe that our framework provides a more intricate understanding of SHAP computation complexity across different models, distributions, and variants, paving the way for further research.

Our work opens promising directions for future research. First, expanding our computational analysis to other SHAP-related metrics, such as asymmetric SHAP~\citep{frye20} and SAGE~\citep{covert2020understanding}, would be valuable. Additionally, we aim to explore more expressive distribution classes and relaxed assumptions beyond those in Section \ref{sec:tractable} while maintaining tractable SHAP computation. Finally, when exact computation is intractable (Section \ref{sec:intractable}), investigating the approximability of SHAP metrics through approximation and parameterized complexity theory~\citep{downey2012parameterized} is an important direction.

%Our work opens several promising avenues for future research on the computational properties of explainable AI methods, with a particular focus on SHAP. First, it would be interesting to broaden the computational analysis conducted in this work to include other popular SHAP-related metrics in the literature, such as asymmetric SHAP \cite{frye20} and SAGE \cite{covert2020understanding}. Also, in the future, we aim to explore more expressive distribution classes and relaxed distributional assumptions—extending beyond those examined in Section \ref{sec:tractable} —that still yield tractable SHAP computation. Finally, when exact computation proves intractable (Section \ref{sec:intractable}), it is worthwhile to theoretically investigate the question of the approximability of computing the SHAP metrics across various configurations, through the lens of approximation and parametrized complexity theory \cite{arora2009computational}.

%This paper aims to deepen our understanding of the computational complexity involved in obtaining different Shapley value variants. We found that for a variety of ML models, including decision trees, tree ensembles for regression, weighted automata, and linear regression models — computing both local and global interventional and baseline SHAP can be done in polynomial time when distributions are modeled by HMMs. This extends the distributional scope of popular algorithms like TreeSHAP, which is limited to empirical distributions. Additionally, we demonstrate a strict complexity gap between SHAP variants, showing that interventional and baseline SHAP can be strictly easier to compute than conditional SHAP. Despite these positive results, we uncovered intractability for various SHAP variants in neural networks and tree ensembles. Finally, we provided generalized complexity relations across SHAP variants. We believe that our framework offers a deeper understanding of the complexity involved in computing SHAP across various variants, models, distributions, as well as in both local and global computations, laying the groundwork for future research.

% %--------------- TODO Commands --------------
\newcommand{\todoc}[2]{{\textcolor{#1}{\textbf{#2}}}}

\newcommand{\todored}[1]{{\todoc{red}{\textbf{[[#1]]}}}}
\newcommand{\todogreen}[1]{\todoc{green}{\textbf{[[#1]]}}}
\newcommand{\todoblue}[1]{\todoc{blue}{\textbf{[[#1]]}}}
\newcommand{\todoorange}[1]{\todoc{orange}{\textbf{[[#1]]}}}
\newcommand{\todobrown}[1]{\todoc{brown}{\textbf{[[#1]]}}}
\newcommand{\todogray}[1]{\todoc{gray}{\textbf{[[#1]]}}}
\newcommand{\todopink}[1]{\todoc{pink}{\textbf{[[#1]]}}}
\newcommand{\todopurple}[1]{\todoc{purple}{\textbf{[[#1]]}}}
\newcommand{\todo}[1]{\todored{TODO: #1}}
\newcommand{\TODO}[1]{\todored{TODO: #1}}
\newcommand{\orange}[1]{{\todoc{orange}{[[} #1 \todoc{orange}{]]}}}
%% To disable colored comments, just uncomment this line: 
%\renewcommand{\todoc}[2]{\relax}

%----------- Comment command for each person ------------
\newcommand{\ma}[1]{\todored{Ma: #1}}
\newcommand{\cmt}[1]{\todoorange{Comment: #1}}
\newcommand{\Update}[1]{{TODO: #1}}
\newcommand{\jwp}[1]{\todored{JWP: #1}}

\newif\ifrevise
\newif\ifrevisenew
\newif\iffinal

\revisetrue
\revisenewtrue
\finaltrue

\revisefalse
\revisenewfalse
\finalfalse

\newcommand{\revisionC}[2]{\ifrevise \todoorange{#1} : \orange{#2} \else#2\fi}
% \newcommand{\revisionB}[1]{\ifrevisenew \textcolor{red}{\DIFdelbegin \DIFdel{#1} \DIFdelend} \else \fi}
 \newcommand{\revisionB}[1]{\ifrevisenew\textcolor{red}{\sout{#1}}\else\fi}
\newcommand{\revisionA}[1]{\ifrevisenew\textcolor{blue}{#1}\else#1\fi}
\newcommand{\revisionD}[1]{\iffinal\textcolor{blue}{#1}\else#1\fi}
% \subsubsection*{Author Contributions}
% If you'd like to, you may include  a section for author contributions as is done
% in many journals. This is optional and at the discretion of the authors.

% \subsubsection*{Acknowledgments}
% Use unnumbered third level headings for the acknowledgments. All
% acknowledgments, including those to funding agencies, go at the end of the paper.
\section*{Impact Statement}
This work studies amortizing variational inference for Bayesian posterior estimation which is a widespread strategy for performing inference in statistics. It provides a natural way of quantifying uncertainty and potentially leading to more robust predictions. While we do not foresee any negative impacts of progress in this area, we encourage caution when applying the methodologies in practice. 

\section*{Acknowledgements}
The authors would like to acknowledge the computing resources provided by the Mila cluster to enable the experiments outlined in this work. SM acknowledges the support of UNIQUE's scholarship.
GL acknowledges the support of the Canada CIFAR AI Chair program, NSERC Discovery Grant RGPIN-2018-04821, and a Canada Research Chair in Neural Computations and Interfacing. 
MAB acknowledges the support of the Canada First Research Excellence Fund (CFREF) for the Vision: Science to Applications (VISTA) program, NSERC Discovery Grant RGPIN-2017-05638 and Google.
The authors also thank NVIDIA for computing resources.


% \input{Paper Draft/Tables/misspecification_and_posterior}
\clearpage
\bibliography{references}
\bibliographystyle{paper}
\clearpage
\appendix
\onecolumn
\section*{\LARGE Appendix}
\section{Related Work}
% \subsection{Vision Language Model}
% 시각장애인에서 상황을 설명할 DB가 없으니 만들었다. 그리고 이를 VLM에 튜닝했다.
\subsection{Technical approaches for assisting the visually-impaired}


\subsection{Datasets for visual instruction tuning}

\section{Architectures respecting Exchangeability}
\label{appdx:exchangeability}
In this section, we highlight how DeepSets and Transformer models satisfy the dataset exchangeability criteria, which is essential in modeling the posterior distribution over the parameters of any probabilistic model relying on \textit{iid} data. 

\subsection{DeepSets}
DeepSets~\citep{Zaheer2017deepsets} operate on arbitrary sets $\gX = \{x_1, ... x_N\} \subset \mathbb{R}^d$ of fixed dimensionality $d$ by first mapping each individual element $\vx_i \in \gX$ to some high-dimensional space using a nonlinear transform, which is parameterized as a multi-layered neural network with parameters $\varphi_1$
\begin{align}
    \vz_i = f_{\varphi_1}(\vx_i)
\end{align}
After having obtained this high-dimensional embedding of each element of the set, it applies an aggregation function $a(\cdot)$, which is a permutation invariant function that maps a set of elements $\gZ = \{\vz_1, ..., \vz_N\} \in \mathbb{R}^z$ to an element $\vh \in \mathbb{R}^z$,
\begin{align}
    \vh = a(\gZ)
\end{align}
Thus, the outcome does not change under permutations of $\gZ$. Finally, another nonlinear transform, parameterized by a multi-layered neural network with parameters $\varphi_2$, is applied to the outcome $\vh$ to provide the final output.
\begin{align}
    \vo = g_{\varphi_2}(\vh)
\end{align}
For our experiments, we then use the vector $\vo$ to predict the parameters of a parametric family of distributions (e.g., Gaussian or Flows) using an additional nonlinear neural network. As an example, for the Gaussian case, we consider the distribution $\gN(\cdot | \mmu, \mSigma)$, where
\begin{align}
    \mmu:= \mmu_{\varphi_3}(\vo) \quad\text{and}\quad\mSigma := \mSigma_{\varphi_4}(\vo)
\end{align}
which makes $\mmu$ implicitly a function of the original input set $\gX$. To understand why the posterior distribution modeled in this fashion does not change when the inputs are permuted, let us assume that $\Pi$ is a permutation over the elements of $\gX$. If we look at one of the parameters of the posterior distribution, e.g., $\mmu$, we can see that
\begin{align}
    \mmu(\Pi \gX) &= \mmu_{\varphi_3}\left(g_{\varphi_2}\left(a\left(\{f_{\varphi_1}(\vx_{\Pi(i)})\}_{i=1}^N\right)\right)\right) \\
    &= \mmu_{\varphi_3}\left(g_{\varphi_2}\left(a\left(\{f_{\varphi_1}(\vx_i)\}_{i=1}^N\right)\right)\right) \\
    &= \mmu(\gX)
\end{align}
which simply follows from the fact that $a(\cdot)$ is a permutation invariant operation, e.g., sum or mean. We can also provide similar reasoning for the other parameters (e.g., $\mSigma$). This shows that DeepSets can be used to model the posterior distribution over parameters of interest as it respects the exchangeability criteria (\textit{iid} observations) assumptions in the data through its permutation invariant structure.

% \begin{table}[t]
    \centering
    \small
    \def\arraystretch{1.25}
    \setlength{\tabcolsep}{4pt}
    \begin{tabular}{l lcr | cc | cccc }
        \toprule
        & & & & \multicolumn{2}{c|}{\textit{$L_2$ Loss} ($\downarrow$)} & \multicolumn{4}{c}{\textit{Accuracy} ($\uparrow$)}\\

        \textbf{Setup} & & $q_\varphi$ & \textbf{Model} & \multicolumn{2}{c|}{\textbf{NLR}} & \multicolumn{4}{c}{\textbf{NLC}} \\
        
        & & & & \textit{1D} & \textit{25D} & \textit{2D-2cl} & \textit{2D-5cl} & \textit{25D-2cl} & \textit{25D-5cl} \\
        \midrule
\multirow{11}{*}{\rotatebox[origin=c]{90}{\textsc{tanh}}} 
& \multirow{3}{*}{Baseline} & - & Random & $52.72$\std{$0.5$} & $55.16$\std{$0.0$} & $50.16$\std{$0.1$} & $19.83$\std{$0.2$} & $50.02$\std{$0.2$} & $20.00$\std{$0.1$} \\
& & - & Optimization & $0.56$\std{$0.0$} & $30.22$\std{$0.1$} & $96.76$\std{$0.0$} & $89.99$\std{$0.0$} & $69.51$\std{$0.0$} & $40.43$\std{$0.0$} \\
& & - & MCMC & $0.39$\std{$0.0$} & $32.41$\std{$0.9$} & $94.23$\std{$0.2$} & $81.42$\std{$0.2$} & $53.16$\std{$0.8$} & $24.49$\std{$0.4$} \\
\cmidrule{4-10}
& \multirow{2}{*}{Fwd-KL} & \multirow{4}{*}{\rotatebox[origin=c]{90}{Gaussian}} & DeepSets & $53.83$\std{$0.5$} & $56.02$\std{$0.1$} & $50.07$\std{$0.4$} & $19.95$\std{$0.2$} & $49.91$\std{$0.1$} & $19.93$\std{$0.0$} \\
& & & Transformer & $53.89$\std{$0.6$} & $56.82$\std{$0.3$} & $50.07$\std{$0.4$} & $20.18$\std{$0.2$} & $49.91$\std{$0.1$} & $20.09$\std{$0.1$} \\
& \multirow{2}{*}{Rev-KL} & & DeepSets & $0.89$\std{$0.0$} & $27.98$\std{$0.0$} & $50.04$\std{$0.4$} & $19.99$\std{$0.2$} & $49.91$\std{$0.1$} & $19.97$\std{$0.0$} \\
& & & Transformer & $0.87$\std{$0.0$} & \highlight{$25.17$\std{$4.0$}} & $61.64$\std{$16.8$} & $19.99$\std{$0.2$} & $49.92$\std{$0.1$} & $19.95$\std{$0.1$} \\
\cmidrule{4-10}
& \multirow{2}{*}{Fwd-KL} & \multirow{4}{*}{\rotatebox[origin=c]{90}{Flow}} & DeepSets & $52.46$\std{$0.5$} & $55.85$\std{$0.1$} & $50.47$\std{$0.1$} & $19.79$\std{$0.1$} & $49.93$\std{$0.0$} & $20.06$\std{$0.0$} \\
& & & Transformer & $52.44$\std{$0.5$} & $55.84$\std{$0.1$} & $50.45$\std{$0.1$} & $20.04$\std{$0.1$} & $49.94$\std{$0.0$} & $20.19$\std{$0.0$} \\
& \multirow{2}{*}{Rev-KL} & & DeepSets & \highlight{$0.68$\std{$0.0$}} & $27.97$\std{$0.0$} & $50.67$\std{$0.1$} & $19.72$\std{$0.1$} & $49.99$\std{$0.1$} & $20.06$\std{$0.1$} \\
& & & Transformer & $0.70$\std{$0.0$} & $27.97$\std{$0.0$} & \highlight{$86.37$\std{$0.2$}} & $19.78$\std{$0.1$} & $50.00$\std{$0.2$} & $20.02$\std{$0.1$} \\
\midrule
\multirow{11}{*}{\rotatebox[origin=c]{90}{\textsc{relu}}}
& \multirow{3}{*}{Baseline} & - & Random & $1082.06$\std{$3.7$} & $13301.74$\std{$109.1$} & $49.68$\std{$0.6$} & $20.18$\std{$0.1$} & $49.74$\std{$0.1$} & $20.08$\std{$0.2$} \\
& & - & Optimization & $2.01$\std{$0.1$} & $1858.40$\std{$2.6$} & $98.01$\std{$0.2$} & $96.81$\std{$0.0$} & $80.30$\std{$0.1$} & $61.30$\std{$0.0$} \\
& & - & MCMC & \textsc{N/A} & \textsc{N/A} & $88.39$\std{$0.9$} & $52.78$\std{$1.9$} & $66.16$\std{$0.1$} & $35.49$\std{$0.7$} \\
\cmidrule{4-10}
& \multirow{2}{*}{Fwd-KL} & \multirow{4}{*}{\rotatebox[origin=c]{90}{Gaussian}} & DeepSets & $821.57$\std{$3.4$} & $10877.36$\std{$200.8$} & $60.77$\std{$0.5$} & $31.66$\std{$0.2$} & $58.35$\std{$0.3$} & $19.88$\std{$0.1$} \\
& & & Transformer & $786.67$\std{$6.7$} & $10845.96$\std{$86.3$} & $61.21$\std{$0.5$} & $32.12$\std{$0.1$} & $58.28$\std{$0.3$} & $30.17$\std{$0.2$} \\
& \multirow{2}{*}{Rev-KL} & & DeepSets & $1.38$\std{$0.1$} & $2048.08$\std{$9.8$} & $74.11$\std{$0.2$} & $49.53$\std{$0.3$} & $66.41$\std{$0.1$} & $46.12$\std{$0.2$} \\
& & & Transformer & $2.36$\std{$0.4$} & \highlight{$1976.32$\std{$43.0$}} & $87.77$\std{$3.9$} & \highlight{$76.33$\std{$1.0$}} & $66.35$\std{$0.1$} & $30.01$\std{$0.3$} \\
\cmidrule{4-10}
& \multirow{2}{*}{Fwd-KL} & \multirow{4}{*}{\rotatebox[origin=c]{90}{Flow}} & DeepSets & $676.46$\std{$15.2$} & $8236.22$\std{$77.0$} & $62.90$\std{$0.4$} & $33.21$\std{$0.2$} & $60.28$\std{$0.1$} & $20.12$\std{$0.1$} \\
& & & Transformer & $646.60$\std{$36.1$} & $8075.57$\std{$94.2$} & $63.71$\std{$0.7$} & $34.11$\std{$0.1$} & $61.45$\std{$0.2$} & $32.70$\std{$0.1$} \\
& \multirow{2}{*}{Rev-KL} & & DeepSets & \highlight{$1.32$\std{$0.1$}} & $2040.10$\std{$2.6$} & $74.98$\std{$0.2$} & $61.65$\std{$0.5$} & $68.06$\std{$0.1$} & \highlight{$47.05$\std{$0.3$}} \\
& & & Transformer & $2.92$\std{$0.2$} & $1987.58$\std{$43.7$} & \highlight{$92.31$\std{$0.3$}} & $76.03$\std{$0.2$} & \highlight{$68.41$\std{$0.2$}} & $45.96$\std{$0.1$} \\
\bottomrule
    \end{tabular}
    \caption{\textbf{Fixed-Dim Posterior Prediction:} Experimental results for posterior inference on fixed dimensional datasets evaluated on estimating the parameters of nonlinear regression (NLR) and classification (NLC) setups, with 2 layered MLP with different activation functions in the probabilistic model. We also consider a multi-class classification setup. We consider different backbone architectures and parametric distributions $q_\varphi$, and use dataset-specific Bayesian and point estimates as baselines. $L_2$ Loss and Accuracy refer to the expected posterior-predictive $L_2$ loss and accuracy respectively. Here, cl refers to the number classes.}
    \label{tab:fixed_dim_2_layer}
\end{table}
\subsection{Transformers}
Similarly, we can look at Transformers~\citep{vaswani2017attention} as candidates for respecting the exchangeability conditions in the data. In particular, we consider transformer systems without positional encodings and consider an additional [CLS] token, denoted by $\vc\in\mathbb{R}^d$, to drive the prediction. If we look at the application of a layer of transformer model, it can be broken down into two components.

\textbf{Multi-Head Attention}. Given a query vector obtained from $\vc$ and keys and values coming from our input set $\gX \subset \mathbb{R}^d$, we can model the update of the context $\vc$ as
\begin{align}
    \hat{\vc}(\gX) = \text{Softmax}\left(\vc^T \mW_Q \mW_K^T \mX^T\right) \mX \mW_V
\end{align}
where $\mW_Q \in \mathbb{R}^{d\times k}, \mW_K \in \mathbb{R}^{d\times k}, \mW_V \in \mathbb{R}^{d\times k}$ and $\mX \in \mathbb{R}^{N\times d}$ denotes a certain ordering of the elements in $\gX$. Further, $\hat{\vc}$ is the updated vector after attention, and Softmax is over the rows of $\mX$. Here, we see that if we were to apply a permutation to the elements in $\mX$, the outcome would remain the same. In particular
\begin{align}
    \hat{\vc}(\Pi \mX) &= \text{Softmax}\left(\vc^T \mW_Q \mW_K^T \mX^T \Pi^T\right) \Pi \mX \mW_V \\
    &= \text{Softmax}\left(\vc^T \mW_Q \mW_K^T \mX^T\right) \Pi^T\Pi \mX \mW_V \\
    &= \text{Softmax}\left(\vc^T \mW_Q \mW_K^T \mX^T\right) \mX \mW_V \\
    &= \hat{\vc}(\mX) 
\end{align}
which follows because Softmax is an equivariant function,  i.e., applying Softmax on a permutation of columns is equivalent to applying Softmax first and then permuting the columns correspondingly. Thus, we see that the update to the [CLS] token $\vc$ is permutation invariant. This output is then used independently as input to a multi-layered neural network with residual connections, and the entire process is repeated multiple times without weight sharing to simulate multiple layers. Since all the individual parts are permutation invariant w.r.t permutations on $\gX$, the entire setup ends up being permutation invariant. Obtaining the parameters of a parametric family of distribution for posterior estimation then follows the same recipe as DeepSets, with $\vo$ replaced by $\vc$.
\section{Probabilistic Models}
\label{appdx:probabilistic_models}
This section details the various candidate probabilistic models used in our experiments for amortized computation of Bayesian posteriors over the parameters. Here, we explain the parameters associated with the probabilistic model over which we want to estimate the posterior and the likelihood and prior that we use for experimentation.

\textbf{Mean of Gaussian (GM):} As a proof of concept, we consider the simple setup of estimating the posterior distribution over the mean of a Gaussian distribution $p(\mmu | \gD)$ given some observed data. In this case, prior and likelihood defining the probabilistic model $p(\vx, \mtheta)$ (with $\mtheta$ being the mean $\mmu$) are given by:
\begin{align}
    p(\mmu) &= \gN\left(\mmu | \mathbf{0}, \mathbf{I}\right)\\
    p(\vx | \mmu) &= \gN\left(\vx | \mmu, \mSigma\right) 
\end{align}
and $\mSigma$ is known beforehand and defined as a unit variance matrix. 

\begin{table}
    \centering
    \small
    % \footnotesize	    
    \def\arraystretch{1.25}
    \setlength{\tabcolsep}{3pt}
    \begin{tabular}{lcr cc cccc}
        \toprule
         &  &  & \multicolumn{6}{c}{\textit{$L_2$ Loss} ($\downarrow$)} \\
        \cmidrule(lr){4-9}
        \textbf{Objective} & $q_\varphi$ & \textbf{Model} & \multicolumn{2}{c}{\textbf{Gaussian}} & \multicolumn{4}{c}{\textbf{GMM}} \\
        \cmidrule(lr){4-5}\cmidrule(lr){6-9}
        & & & \textit{2D} & \textit{100D} & \textit{2D-2cl} & \textit{2D-5cl} & \textit{5D-2cl} & \textit{5D-5cl} \\
        \midrule
\multirow{4}{*}{Baseline} & - & Random & $5.839$\sstd{$0.015$} & $301.065$\sstd{$0.346$} & $1.887$\sstd{$0.031$} & $0.730$\sstd{$0.004$} & $5.001$\sstd{$0.037$} & $1.670$\sstd{$0.008$} \\
& - & Optimization & $1.989$\sstd{$0.000$} & $101.243$\sstd{$0.000$} & $0.169$\sstd{$0.000$} & $0.119$\sstd{$0.001$} & $0.425$\sstd{$0.000$} & $0.308$\sstd{$0.000$} \\
& - & Langevin & $2.013$\sstd{$0.004$} & $102.346$\sstd{$0.031$} & $0.173$\sstd{$0.001$} & $0.125$\sstd{$0.001$} & $0.448$\sstd{$0.009$} & $0.352$\sstd{$0.005$} \\
& - & HMC & $2.018$\sstd{$0.008$} & $102.413$\sstd{$0.028$} & $0.174$\sstd{$0.001$} & $0.135$\sstd{$0.001$} & $0.479$\sstd{$0.007$} & $0.449$\sstd{$0.002$} \\
\cmidrule{2-9}

\multirow{3}{*}{Fwd-KL} & \multirow{6}{*}{\rotatebox[origin=c]{90}{Gaussian}} & GRU &$2.014$\sstd{$0.001$} & $102.641$\sstd{$0.011$} & $0.921$\sstd{$0.013$} & $0.522$\sstd{$0.001$} & $2.430$\sstd{$0.034$} & $1.235$\sstd{$0.011$} \\
& & DeepSets & \highlight{$2.012$\sstd{$0.002$}} & $103.215$\sstd{$0.054$} & $0.920$\sstd{$0.019$} & $0.522$\sstd{$0.001$} & $2.436$\sstd{$0.037$} & $1.238$\sstd{$0.009$} \\
& & Transformer & \highlight{$2.013$\sstd{$0.002$}} & $102.783$\sstd{$0.005$} & $0.931$\sstd{$0.017$} & $0.522$\sstd{$0.001$} & $2.498$\sstd{$0.026$} & $1.230$\sstd{$0.009$} \\
\cmidrule{3-9}

\multirow{3}{*}{Rev-KL} & & GRU &$2.012$\sstd{$0.001$} & $102.509$\sstd{$0.008$} & \highlight{$0.183$\sstd{$0.002$}} & $0.132$\sstd{$0.002$} & \highlight{$0.471$\sstd{$0.010$}} & $0.413$\sstd{$0.019$} \\
& & DeepSets & \highlight{$2.011$\sstd{$0.001$}} & $102.599$\sstd{$0.042$} & $0.186$\sstd{$0.001$} & $0.127$\sstd{$0.002$} & $0.495$\sstd{$0.018$} & $0.409$\sstd{$0.005$} \\
& & Transformer & \highlight{$2.013$\sstd{$0.002$}} & $102.540$\sstd{$0.025$} & \highlight{$0.185$\sstd{$0.004$}} & \highlight{$0.122$\sstd{$0.001$}} & $0.489$\sstd{$0.019$} & \highlight{$0.328$\sstd{$0.002$}} \\
\cmidrule{2-9}

\multirow{3}{*}{Fwd-KL} & \multirow{6}{*}{\rotatebox[origin=c]{90}{Flow}} & GRU &$2.014$\sstd{$0.001$} & $102.656$\sstd{$0.019$} & \highlight{$0.186$\sstd{$0.006$}} & $0.242$\sstd{$0.005$} & $0.670$\sstd{$0.094$} & $0.563$\sstd{$0.018$} \\
& & DeepSets &$2.014$\sstd{$0.001$} & $103.340$\sstd{$0.029$} & \highlight{$0.185$\sstd{$0.006$}} & $0.237$\sstd{$0.008$} & $0.648$\sstd{$0.082$} & $0.583$\sstd{$0.028$} \\
& & Transformer &$2.016$\sstd{$0.002$} & $102.774$\sstd{$0.024$} & \highlight{$0.188$\sstd{$0.012$}} & $0.252$\sstd{$0.001$} & $0.621$\sstd{$0.070$} & $0.592$\sstd{$0.019$} \\
\cmidrule{3-9}

\multirow{3}{*}{Rev-KL} & & GRU &$2.013$\sstd{$0.001$} & \highlight{$102.490$\sstd{$0.012$}} & \highlight{$0.184$\sstd{$0.006$}} & $0.130$\sstd{$0.002$} & \highlight{$0.467$\sstd{$0.003$}} & $0.384$\sstd{$0.005$} \\
& & DeepSets & \highlight{$2.011$\sstd{$0.001$}} & $102.674$\sstd{$0.046$} & \highlight{$0.188$\sstd{$0.005$}} & $0.131$\sstd{$0.002$} & $0.519$\sstd{$0.008$} & $0.405$\sstd{$0.005$} \\
& & Transformer &$2.013$\sstd{$0.001$} & $102.525$\sstd{$0.050$} & \highlight{$0.187$\sstd{$0.004$}} & \highlight{$0.123$\sstd{$0.001$}} & \highlight{$0.468$\sstd{$0.007$}} & \highlight{$0.326$\sstd{$0.008$}} \\
\bottomrule
    \end{tabular}
    \caption{\textbf{Fixed-Dimensional}. Results for estimating the mean of a Gaussian (Gaussian) and means of a Gaussian mixture model (GMM) with the expected $L_2$ loss according to the posterior predictive as metric.}
    \vspace{-4mm}
    \label{tab:apdx_gaussian}
\end{table}

% \begin{table*}[t]
%     \centering
%     \small
%     % \footnotesize	    
%     \def\arraystretch{1.25}
%     \setlength{\tabcolsep}{5pt}
%     \begin{tabular}{lcr cc cccc}
%         \toprule
%          &  &  & \multicolumn{6}{c}{\textit{Conditional Negative Log Likelihood} ($\downarrow$)} \\
%         \cmidrule(lr){4-9}
%         \textbf{Objective} & $q_\varphi$ & \textbf{Model} & \textit{2D} & \textit{100D} & \textit{2D-2cl} & \textit{2D-5cl} & \textit{5D-2cl} & \textit{5D-5cl} \\
%         \midrule
% \multirow{4}{*}{Baseline} & - & Random & $442.0$\sstd{$0.8$} & $22609.3$\sstd{$15.6$} & $1103.7$\sstd{$8.1$} & $673.1$\sstd{$1.3$} & $3563.8$\sstd{$12.0$} & $2370.7$\sstd{$8.6$} \\
% & - & Optimization & $264.5$\sstd{$0.0$} & $13295.9$\sstd{$0.0$} & $106.9$\sstd{$0.0$} & $180.6$\sstd{$0.4$} & $193.4$\sstd{$0.0$} & $315.5$\sstd{$0.2$} \\
% & - & Langevin & $265.6$\sstd{$0.2$} & $13345.6$\sstd{$1.3$} & $109.3$\sstd{$0.5$} & $184.1$\sstd{$0.9$} & $208.7$\sstd{$6.5$} & $364.6$\sstd{$7.7$} \\
% & - & HMC & $265.8$\sstd{$0.4$} & $13347.4$\sstd{$1.1$} & $109.5$\sstd{$0.3$} & $193.8$\sstd{$0.6$} & $237.3$\sstd{$4.6$} & $458.2$\sstd{$3.5$} \\
% \cmidrule{2-9}

% \multirow{3}{*}{Fwd-KL} & \multirow{6}{*}{\rotatebox[origin=c]{90}{Gaussian}} & GRU &$265.6$\sstd{$0.1$} & $13358.9$\sstd{$0.5$} & $494.0$\sstd{$5.2$} & $512.8$\sstd{$1.4$} & $1550.8$\sstd{$6.6$} & $1780.5$\sstd{$7.7$} \\
%  & & DeepSets &$265.6$\sstd{$0.1$} & $13388.1$\sstd{$2.7$} & $493.2$\sstd{$7.2$} & $513.5$\sstd{$2.3$} & $1556.4$\sstd{$19.0$} & $1778.2$\sstd{$8.4$} \\
%  & & Transformer &$265.6$\sstd{$0.1$} & $13365.6$\sstd{$0.3$} & $500.1$\sstd{$8.2$} & $514.1$\sstd{$2.1$} & $1606.2$\sstd{$15.5$} & $1785.3$\sstd{$13.1$} \\
% \cmidrule{3-9}

% \multirow{3}{*}{Rev-KL} & & GRU &$265.5$\sstd{$0.1$} & $13352.8$\sstd{$0.3$} & $112.7$\sstd{$1.4$} & $192.2$\sstd{$1.5$} & $223.5$\sstd{$6.4$} & $449.9$\sstd{$12.8$} \\
%  & & DeepSets &$265.5$\sstd{$0.1$} & $13359.3$\sstd{$2.1$} & $114.3$\sstd{$0.6$} & $187.5$\sstd{$1.0$} & $233.1$\sstd{$10.4$} & $448.8$\sstd{$9.0$} \\
%  & & Transformer &$265.6$\sstd{$0.1$} & $13354.4$\sstd{$1.1$} & $115.2$\sstd{$1.7$} & $185.9$\sstd{$0.2$} & $230.2$\sstd{$10.9$} & $350.5$\sstd{$7.5$} \\
% \cmidrule{2-9}

% \multirow{3}{*}{Fwd-KL} & \multirow{6}{*}{\rotatebox[origin=c]{90}{Flow}} & GRU &$265.6$\sstd{$0.0$} & $13359.5$\sstd{$0.8$} & $114.3$\sstd{$2.2$} & $288.9$\sstd{$2.6$} & $912.9$\sstd{$159.2$} & $1541.3$\sstd{$2.5$} \\
%  & & DeepSets &$265.6$\sstd{$0.1$} & $13393.9$\sstd{$1.4$} & $114.6$\sstd{$2.7$} & $291.6$\sstd{$3.9$} & $1007.0$\sstd{$71.4$} & $1557.6$\sstd{$8.0$} \\
%  & & Transformer &$265.7$\sstd{$0.1$} & $13365.2$\sstd{$1.0$} & $115.4$\sstd{$5.3$} & $284.8$\sstd{$1.0$} & $503.7$\sstd{$197.0$} & $1523.8$\sstd{$24.6$} \\
% \cmidrule{3-9}

% \multirow{3}{*}{Rev-KL} & & GRU &$265.6$\sstd{$0.0$} & $13351.8$\sstd{$0.5$} & $113.7$\sstd{$2.4$} & $190.1$\sstd{$1.6$} & $217.7$\sstd{$2.5$} & $429.6$\sstd{$10.4$} \\
%  & & DeepSets &$265.5$\sstd{$0.0$} & $13362.5$\sstd{$2.5$} & $115.4$\sstd{$2.6$} & $189.8$\sstd{$1.1$} & $247.6$\sstd{$4.9$} & $439.8$\sstd{$8.8$} \\
%  & & Transformer &$265.6$\sstd{$0.0$} & $13353.6$\sstd{$2.2$} & $114.8$\sstd{$1.6$} & $185.6$\sstd{$0.5$} & $220.1$\sstd{$5.7$} & $343.4$\sstd{$7.9$} \\
%  \bottomrule
%     \end{tabular}
%     \caption{}
%     \vspace{-4mm}
%     \label{tab:}
% \end{table*}
\begin{table*}[t]
    \centering
    \small
    % \footnotesize	    
    \def\arraystretch{1.25}
    \setlength{\tabcolsep}{3pt}
    \begin{tabular}{lcr cc cccc}
        \toprule
         &  &  & \multicolumn{2}{c}{\textit{$L_2$ Loss} ($\downarrow$)} & \multicolumn{2}{c}{\textit{Accuracy} ($\uparrow$)} \\
         \cmidrule(lr){4-5}\cmidrule(lr){6-9}
         \textbf{Objective} & $q_\varphi$ & \textbf{Model} & \multicolumn{2}{c}{\textbf{Linear Regression}} & \multicolumn{4}{c}{\textbf{Linear Classification}} \\
         \cmidrule(lr){4-5}\cmidrule(lr){6-9}
         & & & \textit{2D} & \textit{100D} & \textit{2D-2cl} & \textit{2D-5cl} & \textit{100D-2cl} & \textit{100D-5cl} \\
         \midrule

\multirow{4}{*}{Baseline} & - & Random & $4.178$\sstd{$0.018$} & $202.601$\sstd{$0.321$} & $50.498$\sstd{$0.357$} & $19.891$\sstd{$0.028$} & $50.046$\sstd{$0.047$} & $20.054$\sstd{$0.053$} \\
& - & Optimization & $0.257$\sstd{$0.000$} & $25.083$\sstd{$0.006$} & $96.982$\sstd{$0.000$} & $93.449$\sstd{$0.002$} & $70.258$\sstd{$0.012$} & $41.338$\sstd{$0.012$} \\
& - & Langevin & $0.263$\sstd{$0.002$} & $23.340$\sstd{$0.689$} & $95.034$\sstd{$0.412$} & $88.277$\sstd{$0.290$} & $65.123$\sstd{$0.370$} & $32.544$\sstd{$0.422$} \\
& - & HMC & $0.263$\sstd{$0.001$} & $18.659$\sstd{$0.189$} & $92.659$\sstd{$0.344$} & $82.169$\sstd{$0.518$} & $62.145$\sstd{$0.245$} & $29.582$\sstd{$0.371$} \\
\cmidrule{2-9}

\multirow{3}{*}{Fwd-KL} & \multirow{6}{*}{\rotatebox[origin=c]{90}{Gaussian}} & GRU & \highlight{$0.264$\sstd{$0.001$}} & $124.823$\sstd{$0.135$} & $81.170$\sstd{$0.389$} & $71.170$\sstd{$0.275$} & $59.740$\sstd{$0.102$} & $23.042$\sstd{$0.246$} \\
& & DeepSets &$0.264$\sstd{$0.000$} & $123.133$\sstd{$1.080$} & $81.281$\sstd{$0.278$} & $70.993$\sstd{$0.191$} & $50.047$\sstd{$0.051$} & $20.053$\sstd{$0.045$} \\
& & Transformer &$0.264$\sstd{$0.000$} & $45.856$\sstd{$1.331$} & $80.960$\sstd{$0.285$} & $71.484$\sstd{$0.437$} & $62.954$\sstd{$0.062$} & $26.789$\sstd{$0.110$} \\
\cmidrule{3-9}

\multirow{3}{*}{Rev-KL} & & GRU & \highlight{$0.263$\sstd{$0.000$}} & $60.215$\sstd{$0.866$} & $94.258$\sstd{$0.034$} & $87.339$\sstd{$0.023$} & $63.465$\sstd{$0.307$} & $28.270$\sstd{$0.462$} \\
& & DeepSets & \highlight{$0.263$\sstd{$0.000$}} & $62.837$\sstd{$0.617$} & $94.285$\sstd{$0.116$} & $87.342$\sstd{$0.021$} & $60.867$\sstd{$0.265$} & $21.339$\sstd{$0.085$} \\
& & Transformer & \highlight{$0.264$\sstd{$0.001$}} & \highlight{$28.735$\sstd{$0.252$}} & $94.302$\sstd{$0.054$} & $87.540$\sstd{$0.117$} & $68.185$\sstd{$0.007$} & \highlight{$32.950$\sstd{$0.284$}} \\
\cmidrule{2-9}

\multirow{3}{*}{Fwd-KL} & \multirow{6}{*}{\rotatebox[origin=c]{90}{Flow}} & GRU & \highlight{$0.264$\sstd{$0.001$}} & $119.119$\sstd{$0.233$} & $96.305$\sstd{$0.008$} & $88.927$\sstd{$0.200$} & $59.920$\sstd{$0.221$} & $23.025$\sstd{$0.077$} \\
& & DeepSets & \highlight{$0.264$\sstd{$0.001$}} & $125.677$\sstd{$3.731$} & $96.191$\sstd{$0.021$} & $88.643$\sstd{$0.102$} & $50.061$\sstd{$0.021$} & $20.021$\sstd{$0.094$} \\
& & Transformer &$0.264$\sstd{$0.000$} & $43.272$\sstd{$2.700$} & \highlight{$96.344$\sstd{$0.059$}} & \highlight{$89.624$\sstd{$0.215$}} & $64.349$\sstd{$0.147$} & $26.952$\sstd{$0.203$} \\
\cmidrule{3-9}

\multirow{3}{*}{Rev-KL} & & GRU & \highlight{$0.263$\sstd{$0.000$}} & $61.295$\sstd{$1.008$} & $95.241$\sstd{$0.012$} & $88.429$\sstd{$0.024$} & $64.669$\sstd{$0.207$} & $28.409$\sstd{$1.167$} \\
& & DeepSets & \highlight{$0.263$\sstd{$0.001$}} & $76.412$\sstd{$2.038$} & $95.296$\sstd{$0.021$} & $88.464$\sstd{$0.061$} & $58.384$\sstd{$0.812$} & $21.569$\sstd{$0.117$} \\
& & Transformer & \highlight{$0.263$\sstd{$0.000$}} & \highlight{$29.358$\sstd{$1.569$}} & $95.339$\sstd{$0.063$} & $88.644$\sstd{$0.047$} & \highlight{$68.721$\sstd{$0.121$}} & \highlight{$33.107$\sstd{$0.333$}} \\
\bottomrule
    \end{tabular}
    \caption{\textbf{Fixed-Dimensional}. Results for estimating the parameters of linear regression (LR) and classification (LC) models with the expected $L_2$ loss and accuracy according to the posterior predictive as metrics.}
    \vspace{-4mm}
    \label{tab:}
\end{table*}

% \begin{table*}[t]
%     \centering
%     \small
%     % \footnotesize	    
%     \def\arraystretch{1.25}
%     \setlength{\tabcolsep}{5pt}
%     \begin{tabular}{lcr cc cccc}
%         \toprule
%          &  &  & \multicolumn{6 }{c}{\textit{Conditional Negative Log Likelihood} ($\downarrow$)} \\
%          \cmidrule(lr){4-9}
%          \textbf{Objective} & $q_\varphi$ & \textbf{Model} & \textit{2D} & \textit{100D} & \textit{2D-2cl} & \textit{2D-5cl} & \textit{100D-2cl} & \textit{100D-5cl} \\
%          \midrule

% \multirow{4}{*}{Baseline} & - & Random & $792.2$\sstd{$3.6$} & $37777.0$\sstd{$65.5$} & $109.7$\sstd{$0.7$} & $234.4$\sstd{$1.5$} & $532.4$\sstd{$0.5$} & $1097.9$\sstd{$2.8$} \\
% & - & Optimization & $69.2$\sstd{$0.0$} & $3980.9$\sstd{$0.9$} & $9.2$\sstd{$0.0$} & $26.2$\sstd{$0.0$} & $261.4$\sstd{$0.1$} & $442.3$\sstd{$0.0$} \\
% & - & Langevin & $70.3$\sstd{$0.3$} & $3671.3$\sstd{$110.7$} & $14.1$\sstd{$0.2$} & $36.2$\sstd{$0.2$} & $228.3$\sstd{$6.3$} & $585.8$\sstd{$5.3$} \\
% & - & HMC & $70.2$\sstd{$0.2$} & $3093.6$\sstd{$27.5$} & $27.7$\sstd{$0.5$} & $71.9$\sstd{$0.7$} & $105.3$\sstd{$0.2$} & $255.9$\sstd{$2.9$} \\
% \cmidrule{2-9}

% \multirow{3}{*}{Fwd-KL} & \multirow{6}{*}{\rotatebox[origin=c]{90}{Gaussian}} & GRU &$70.4$\sstd{$0.1$} & $22964.7$\sstd{$43.5$} & $38.5$\sstd{$0.5$} & $72.9$\sstd{$0.5$} & $397.1$\sstd{$1.0$} & $1013.3$\sstd{$5.4$} \\
%  & & DeepSets &$70.4$\sstd{$0.1$} & $22802.5$\sstd{$191.9$} & $38.4$\sstd{$0.5$} & $73.3$\sstd{$0.4$} & $532.4$\sstd{$0.5$} & $1098.0$\sstd{$2.8$} \\
%  & & Transformer &$70.4$\sstd{$0.0$} & $8002.2$\sstd{$250.1$} & $38.8$\sstd{$0.5$} & $72.0$\sstd{$0.8$} & $349.9$\sstd{$0.6$} & $912.3$\sstd{$2.7$} \\
% \cmidrule{3-9}

% \multirow{3}{*}{Rev-KL} & & GRU &$70.2$\sstd{$0.0$} & $11089.1$\sstd{$153.0$} & $14.1$\sstd{$0.0$} & $34.9$\sstd{$0.2$} & $265.1$\sstd{$1.9$} & $603.6$\sstd{$2.8$} \\
%  & & DeepSets &$70.2$\sstd{$0.1$} & $11604.3$\sstd{$110.4$} & $14.1$\sstd{$0.2$} & $35.1$\sstd{$0.2$} & $306.9$\sstd{$2.1$} & $497.8$\sstd{$22.8$} \\
%  & & Transformer &$70.3$\sstd{$0.1$} & $5111.2$\sstd{$49.1$} & $14.0$\sstd{$0.1$} & $34.3$\sstd{$0.3$} & $263.5$\sstd{$0.8$} & $664.5$\sstd{$8.1$} \\
% \cmidrule{2-9}

% \multirow{3}{*}{Fwd-KL} & \multirow{6}{*}{\rotatebox[origin=c]{90}{Flow}} & GRU &$70.3$\sstd{$0.1$} & $21890.2$\sstd{$58.5$} & $23.8$\sstd{$0.2$} & $57.6$\sstd{$0.5$} & $366.8$\sstd{$2.6$} & $1007.0$\sstd{$2.5$} \\
%  & & DeepSets &$70.4$\sstd{$0.2$} & $23320.5$\sstd{$691.7$} & $23.5$\sstd{$0.6$} & $58.0$\sstd{$0.7$} & $532.1$\sstd{$0.3$} & $1099.1$\sstd{$2.7$} \\
%  & & Transformer &$70.4$\sstd{$0.1$} & $7519.8$\sstd{$507.3$} & $23.5$\sstd{$0.1$} & $56.0$\sstd{$0.5$} & $297.1$\sstd{$1.0$} & $889.7$\sstd{$5.2$} \\
% \cmidrule{3-9}

% \multirow{3}{*}{Rev-KL} & & GRU &$70.3$\sstd{$0.1$} & $11306.4$\sstd{$209.1$} & $13.2$\sstd{$0.0$} & $33.5$\sstd{$0.2$} & $196.7$\sstd{$2.4$} & $556.5$\sstd{$10.3$} \\
%  & & DeepSets &$70.2$\sstd{$0.1$} & $14107.0$\sstd{$354.7$} & $13.1$\sstd{$0.1$} & $33.4$\sstd{$0.3$} & $181.8$\sstd{$10.9$} & $442.8$\sstd{$9.9$} \\
%  & & Transformer &$70.3$\sstd{$0.1$} & $5225.3$\sstd{$292.6$} & $13.0$\sstd{$0.1$} & $33.1$\sstd{$0.1$} & $209.2$\sstd{$1.5$} & $619.8$\sstd{$4.4$} \\
% \bottomrule
%     \end{tabular}
%     \caption{}
%     \vspace{-4mm}
%     \label{tab:}
% \end{table*}
\begin{table*}[t]
    \centering
    \small
    % \footnotesize	    
    \def\arraystretch{1.25}
    \setlength{\tabcolsep}{5pt}
    \begin{tabular}{lcr cccc}
        \toprule
         &  &  & \multicolumn{4}{c}{\textit{$L_2$ Loss} ($\downarrow$)} \\
         \cmidrule(lr){4-7}
         \textbf{Objective} & $q_\varphi$ & \textbf{Model} & \multicolumn{4}{c}{\textbf{Nonlinear Regression} $|$ \textbf{ReLU}} \\
         \cmidrule(lr){4-7}
         & & & \multicolumn{2}{c}{\textit{1-layer}} & \multicolumn{2}{c}{\textit{2-layers}} \\
         \cmidrule(lr){4-5}\cmidrule(lr){6-7}
         & & & \textit{1D} & \textit{25D} & \textit{1D} & \textit{25D} \\
         \midrule
\multirow{4}{*}{Baseline} & - & Random & $65.936$\sstd{$0.913$} & $831.595$\sstd{$8.696$} & $1029.407$\sstd{$11.542$} & $12067.691$\sstd{$183.598$} \\
& - & Optimization & $0.360$\sstd{$0.001$} & $103.967$\sstd{$0.110$} & $2.370$\sstd{$0.015$} & $1894.574$\sstd{$4.266$} \\
& - & Langevin & $0.308$\sstd{$0.000$} & $132.391$\sstd{$0.992$} & \textsc{N/A} & \textsc{N/A} \\
& - & HMC & $0.374$\sstd{$0.002$} & $98.061$\sstd{$0.730$} & $22.314$\sstd{$0.814$} & $3903.510$\sstd{$5.377$} \\
\cmidrule{2-7}

\multirow{3}{*}{Fwd-KL} & \multirow{6}{*}{\rotatebox[origin=c]{90}{Gaussian}} & GRU &$49.332$\sstd{$0.946$} & $671.639$\sstd{$10.494$} & $774.045$\sstd{$7.521$} & $9905.246$\sstd{$214.545$} \\
& & DeepSets &$49.864$\sstd{$0.979$} & $684.853$\sstd{$2.581$} & $768.921$\sstd{$8.278$} & $9946.090$\sstd{$109.933$} \\
& & Transformer &$49.678$\sstd{$0.940$} & $680.853$\sstd{$5.838$} & $747.221$\sstd{$12.189$} & $9982.609$\sstd{$85.596$} \\
\cmidrule{3-7}

\multirow{3}{*}{Rev-KL} & & GRU &$0.426$\sstd{$0.004$} & $105.976$\sstd{$0.586$} & $1.066$\sstd{$0.069$} & $1796.512$\sstd{$5.805$} \\
& & DeepSets &$0.426$\sstd{$0.004$} & $125.853$\sstd{$0.791$} & $1.394$\sstd{$0.108$} & $1892.402$\sstd{$2.793$} \\
& & Transformer &$0.417$\sstd{$0.005$} & \highlight{$102.295$\sstd{$1.825$}} & $2.075$\sstd{$0.147$} & \highlight{$1811.440$\sstd{$115.435$}} \\
\cmidrule{2-7}

\multirow{3}{*}{Fwd-KL} & \multirow{6}{*}{\rotatebox[origin=c]{90}{Flow}} & GRU &$15.781$\sstd{$0.210$} & $538.962$\sstd{$4.269$} & $614.925$\sstd{$15.494$} & $7564.076$\sstd{$67.160$} \\
& & DeepSets &$15.051$\sstd{$0.120$} & $548.535$\sstd{$3.288$} & $622.461$\sstd{$7.043$} & $7618.364$\sstd{$115.946$} \\
& & Transformer &$16.109$\sstd{$0.307$} & $539.338$\sstd{$4.336$} & $597.718$\sstd{$8.358$} & $7635.052$\sstd{$109.037$} \\
\cmidrule{3-7}

\multirow{3}{*}{Rev-KL} & & GRU &$0.405$\sstd{$0.010$} & $106.001$\sstd{$0.420$} & \highlight{$0.988$\sstd{$0.045$}} & $1814.649$\sstd{$8.327$} \\
& & DeepSets &$0.395$\sstd{$0.004$} & $128.169$\sstd{$1.451$} & $1.215$\sstd{$0.028$} & $1886.698$\sstd{$7.294$} \\
& & Transformer & \highlight{$0.387$\sstd{$0.004$}} & \highlight{$102.610$\sstd{$0.863$}} & $2.549$\sstd{$0.058$} & \highlight{$1791.741$\sstd{$49.585$}} \\
\bottomrule
    \end{tabular}
    \caption{\textbf{Fixed-Dimensional}. Results for estimating the parameters of nonlinear regression models with ReLU activation function, with the expected $L_2$ loss according to the posterior predictive as metric.}
    \vspace{-4mm}
    \label{tab:}
\end{table*}
\begin{table*}[t]
    \centering
    \small
    % \footnotesize	    
    \def\arraystretch{1.25}
    \setlength{\tabcolsep}{5pt}
    \begin{tabular}{lcr cccc}
        \toprule
         &  &  & \multicolumn{4}{c}{\textit{$L_2$ Loss} ($\downarrow$)} \\
         \cmidrule(lr){4-7}
         \textbf{Objective} & $q_\varphi$ & \textbf{Model} & \multicolumn{4}{c}{\textbf{Nonlinear Regression} $|$ \textbf{TanH}} \\
         \cmidrule(lr){4-7}
         & & & \multicolumn{2}{c}{\textit{1-layer}} & \multicolumn{2}{c}{\textit{2-layers}} \\
         \cmidrule(lr){4-5}\cmidrule(lr){6-7}
         & & & \textit{1D} & \textit{25D} & \textit{1D} & \textit{25D} \\
         \midrule

\multirow{4}{*}{Baseline} & - & Random & $31.448$\sstd{$0.186$} & $52.644$\sstd{$0.173$} & $52.735$\sstd{$1.122$} & $52.583$\sstd{$0.132$} \\
& - & Optimization & $0.366$\sstd{$0.001$} & $13.352$\sstd{$0.005$} & $0.651$\sstd{$0.002$} & $30.176$\sstd{$0.056$} \\
& - & Langevin & $0.296$\sstd{$0.003$} & $17.221$\sstd{$0.130$} & $0.363$\sstd{$0.003$} & $28.528$\sstd{$0.115$} \\
& - & HMC & $0.398$\sstd{$0.003$} & $12.607$\sstd{$0.192$} & $0.733$\sstd{$0.021$} & $23.571$\sstd{$0.346$} \\
\cmidrule{2-7}

\multirow{3}{*}{Fwd-KL} & \multirow{6}{*}{\rotatebox[origin=c]{90}{Gaussian}} & GRU &$31.391$\sstd{$0.161$} & $52.008$\sstd{$0.282$} & $52.725$\sstd{$1.149$} & $51.989$\sstd{$0.139$} \\
& & DeepSets &$31.421$\sstd{$0.074$} & $52.137$\sstd{$0.215$} & $52.850$\sstd{$1.192$} & $51.904$\sstd{$0.334$} \\
& & Transformer &$31.350$\sstd{$0.219$} & $52.945$\sstd{$0.430$} & $52.693$\sstd{$1.188$} & $52.364$\sstd{$0.164$} \\
\cmidrule{3-7}

\multirow{3}{*}{Rev-KL} & & GRU &$0.415$\sstd{$0.003$} & $15.874$\sstd{$6.958$} & $0.951$\sstd{$0.047$} & $25.907$\sstd{$0.012$} \\
& & DeepSets &$0.405$\sstd{$0.004$} & $25.333$\sstd{$0.010$} & $0.912$\sstd{$0.013$} & $25.877$\sstd{$0.002$} \\
& & Transformer &$0.412$\sstd{$0.013$} & $11.784$\sstd{$0.949$} & $0.847$\sstd{$0.010$} & $20.405$\sstd{$3.874$} \\
\cmidrule{2-7}

\multirow{3}{*}{Fwd-KL} & \multirow{6}{*}{\rotatebox[origin=c]{90}{Flow}} & GRU &$12.415$\sstd{$0.800$} & $52.039$\sstd{$0.065$} & $52.695$\sstd{$0.611$} & $52.576$\sstd{$0.225$} \\
& & DeepSets &$31.790$\sstd{$0.163$} & $51.933$\sstd{$0.115$} & $52.903$\sstd{$0.625$} & $52.643$\sstd{$0.239$} \\
& & Transformer &$10.392$\sstd{$0.195$} & $52.470$\sstd{$0.364$} & $52.385$\sstd{$0.689$} & $52.646$\sstd{$0.622$} \\
\cmidrule{3-7}

\multirow{3}{*}{Rev-KL} & & GRU &$0.386$\sstd{$0.005$} & $11.401$\sstd{$0.041$} & $0.736$\sstd{$0.009$} & \highlight{$25.892$\sstd{$0.010$}} \\
& & DeepSets & \highlight{$0.374$\sstd{$0.005$}} & $25.685$\sstd{$0.004$} & \highlight{$0.686$\sstd{$0.019$}} & \highlight{$25.885$\sstd{$0.007$}} \\
& & Transformer & \highlight{$0.376$\sstd{$0.002$}} & \highlight{$10.486$\sstd{$0.040$}} & $0.724$\sstd{$0.026$} & \highlight{$25.885$\sstd{$0.011$}} \\
\bottomrule
    \end{tabular}
    \caption{\textbf{Fixed-Dimensional}. Results for estimating the parameters of nonlinear regression models with TanH activation function, with the expected $L_2$ loss according to the posterior predictive as metric.}
    \vspace{-4mm}
    \label{tab:}
\end{table*}
\begin{table*}[t]
    \centering
    \small
    % \footnotesize	    
    \def\arraystretch{1.25}
    \setlength{\tabcolsep}{5pt}
    \begin{tabular}{lcr cc cc}
        \toprule
         &  &  & \multicolumn{4}{c}{\textit{Accuracy} ($\uparrow$)} \\
         \cmidrule(lr){4-7}
         \textbf{Objective} & $q_\varphi$ & \textbf{Model} & \multicolumn{4}{c}{\textbf{Nonlinear Classification} $|$ \textbf{ReLU - 2 class}} \\
         \cmidrule(lr){4-7}
         & & & \multicolumn{2}{c}{\textit{1-layer}} & \multicolumn{2}{c}{\textit{2-layers}} \\
         \cmidrule(lr){4-5}\cmidrule(lr){6-7}
         & & & \textit{2D} & \textit{25D} & \textit{2D} & \textit{25D} \\
         \midrule
\multirow{4}{*}{Baseline} & - & Random & $50.306$\sstd{$0.590$} & $50.008$\sstd{$0.326$} & $50.394$\sstd{$0.190$} & $49.846$\sstd{$0.635$} \\
& - & Optimization & $96.879$\sstd{$0.028$} & $77.896$\sstd{$0.023$} & $96.770$\sstd{$0.062$} & $82.073$\sstd{$0.200$} \\
& - & Langevin & $95.971$\sstd{$0.313$} & $73.165$\sstd{$0.282$} & $96.645$\sstd{$0.101$} & $76.541$\sstd{$0.139$} \\
& - & HMC & $91.763$\sstd{$0.163$} & $70.395$\sstd{$0.110$} & $91.797$\sstd{$0.048$} & $76.445$\sstd{$0.372$} \\
\cmidrule{2-7}

\multirow{3}{*}{Fwd-KL} & \multirow{6}{*}{\rotatebox[origin=c]{90}{Gaussian}} & GRU &$59.518$\sstd{$0.355$} & $56.858$\sstd{$0.319$} & $60.962$\sstd{$0.599$} & $60.063$\sstd{$0.695$} \\
& & DeepSets &$59.383$\sstd{$0.244$} & $56.806$\sstd{$0.204$} & $61.090$\sstd{$0.690$} & $59.933$\sstd{$0.599$} \\
& & Transformer &$59.588$\sstd{$0.389$} & $57.089$\sstd{$0.376$} & $61.151$\sstd{$0.560$} & $60.041$\sstd{$0.680$} \\
\cmidrule{3-7}

\multirow{3}{*}{Rev-KL} & & GRU &$92.384$\sstd{$0.195$} & $72.455$\sstd{$0.032$} & $86.157$\sstd{$0.066$} & $69.966$\sstd{$0.333$} \\
& & DeepSets &$92.488$\sstd{$0.133$} & $59.806$\sstd{$0.315$} & $86.275$\sstd{$0.733$} & $69.550$\sstd{$0.371$} \\
& & Transformer &$92.627$\sstd{$0.377$} & \highlight{$75.178$\sstd{$0.142$}} & $86.351$\sstd{$0.217$} & $69.812$\sstd{$0.527$} \\
\cmidrule{2-7}

\multirow{3}{*}{Fwd-KL} & \multirow{6}{*}{\rotatebox[origin=c]{90}{Flow}} & GRU &$76.931$\sstd{$0.266$} & $58.338$\sstd{$0.026$} & $62.981$\sstd{$0.452$} & $63.199$\sstd{$0.202$} \\
& & DeepSets &$72.313$\sstd{$1.829$} & $58.113$\sstd{$0.126$} & $62.438$\sstd{$0.277$} & $62.884$\sstd{$0.201$} \\
& & Transformer &$77.296$\sstd{$0.201$} & $58.344$\sstd{$0.148$} & $63.753$\sstd{$0.155$} & $63.590$\sstd{$0.390$} \\
\cmidrule{3-7}

\multirow{3}{*}{Rev-KL} & & GRU &$93.392$\sstd{$0.073$} & $72.002$\sstd{$0.506$} & $85.712$\sstd{$0.665$} & \highlight{$71.419$\sstd{$0.200$}} \\
& & DeepSets &$93.312$\sstd{$0.197$} & $60.943$\sstd{$0.184$} & $85.960$\sstd{$0.896$} & $71.288$\sstd{$0.211$} \\
& & Transformer & \highlight{$93.578$\sstd{$0.093$}} & $74.956$\sstd{$0.546$} & \highlight{$87.138$\sstd{$0.438$}} & \highlight{$71.525$\sstd{$0.159$}} \\
\bottomrule
    \end{tabular}
    \caption{\textbf{Fixed-Dimensional}. Results for estimating the parameters of nonlinear classification models with ReLU activation function and two classes, with the expected accuracy according to the posterior predictive as metric.}
    \vspace{-4mm}
    \label{tab:}
\end{table*}
\begin{table*}[t]
    \centering
    \small
    % \footnotesize	    
    \def\arraystretch{1.25}
    \setlength{\tabcolsep}{5pt}
    \begin{tabular}{lcr cc cc}
        \toprule
         &  &  & \multicolumn{4}{c}{\textit{Accuracy} ($\uparrow$)} \\
         \cmidrule(lr){4-7}
         & & & \multicolumn{2}{c}{\textit{1-layer}} & \multicolumn{2}{c}{\textit{2-layers}} \\
         \cmidrule(lr){4-5}\cmidrule(lr){6-7}
         \textbf{Objective} & $q_\varphi$ & \textbf{Model} & \textit{2D} & \textit{50D} & \textit{2D} & \textit{50D} \\
         \midrule
\multirow{4}{*}{Baseline} & - & Random & $19.847$\sstd{$0.352$} & $20.052$\sstd{$0.066$} & $19.874$\sstd{$0.222$} & $20.028$\sstd{$0.106$} \\
& - & Optimization & $94.607$\sstd{$0.011$} & $56.091$\sstd{$0.158$} & $93.873$\sstd{$0.028$} & $60.253$\sstd{$0.053$} \\
& - & Langevin & $90.815$\sstd{$0.341$} & $46.072$\sstd{$0.225$} & $91.849$\sstd{$0.088$} & $49.808$\sstd{$0.337$} \\
& - & HMC & $81.145$\sstd{$0.303$} & $44.561$\sstd{$0.309$} & $79.559$\sstd{$0.483$} & $50.967$\sstd{$0.436$} \\
\cmidrule{2-7}

\multirow{3}{*}{Fwd-KL} & \multirow{6}{*}{\rotatebox[origin=c]{90}{Gaussian}} & GRU &$30.960$\sstd{$0.638$} & $32.164$\sstd{$0.151$} & $31.017$\sstd{$0.397$} & $32.224$\sstd{$0.108$} \\
& & DeepSets &$19.846$\sstd{$0.348$} & $20.053$\sstd{$0.063$} & $19.871$\sstd{$0.226$} & $20.032$\sstd{$0.104$} \\
& & Transformer &$30.652$\sstd{$0.401$} & $32.208$\sstd{$0.178$} & $31.148$\sstd{$0.429$} & $32.342$\sstd{$0.217$} \\
\cmidrule{3-7}

\multirow{3}{*}{Rev-KL} & & GRU &$72.874$\sstd{$0.113$} & $37.987$\sstd{$0.119$} & $56.999$\sstd{$0.599$} & $29.971$\sstd{$0.248$} \\
& & DeepSets &$69.456$\sstd{$0.370$} & $36.712$\sstd{$0.249$} & $55.193$\sstd{$0.538$} & $36.417$\sstd{$9.779$} \\
& & Transformer &$73.531$\sstd{$0.391$} & \highlight{$44.702$\sstd{$0.165$}} & $57.724$\sstd{$0.332$} & $30.175$\sstd{$0.177$} \\
\cmidrule{2-7}

\multirow{3}{*}{Fwd-KL} & \multirow{6}{*}{\rotatebox[origin=c]{90}{Flow}} & GRU &$33.232$\sstd{$0.607$} & $33.704$\sstd{$0.026$} & $31.937$\sstd{$0.483$} & $32.370$\sstd{$0.284$} \\
& & DeepSets &$19.898$\sstd{$0.156$} & $19.950$\sstd{$0.256$} & $20.062$\sstd{$0.347$} & $20.064$\sstd{$0.207$} \\
& & Transformer &$32.916$\sstd{$0.194$} & $33.766$\sstd{$0.137$} & $32.374$\sstd{$0.301$} & $33.846$\sstd{$0.486$} \\
\cmidrule{3-7}

\multirow{3}{*}{Rev-KL} & & GRU & \highlight{$77.997$\sstd{$0.663$}} & $38.715$\sstd{$0.153$} & \highlight{$61.947$\sstd{$0.294$}} & \highlight{$51.962$\sstd{$0.812$}} \\
& & DeepSets &$68.957$\sstd{$0.551$} & $37.123$\sstd{$0.108$} & $51.145$\sstd{$14.201$} & $42.707$\sstd{$12.882$} \\
& & Transformer & \highlight{$77.867$\sstd{$2.241$}} & \highlight{$44.156$\sstd{$0.485$}} & $57.410$\sstd{$0.088$} & \highlight{$52.077$\sstd{$0.077$}} \\
\bottomrule
    \end{tabular}
    \caption{\textbf{Fixed-Dimensional}. Results for estimating the parameters of nonlinear classification models with ReLU activation function and five classes, with the expected accuracy according to the posterior predictive as metric.}
    \vspace{-4mm}
    \label{tab:}
\end{table*}
\begin{table*}[t]
    \centering
    \small
    % \footnotesize	    
    \def\arraystretch{1.25}
    \setlength{\tabcolsep}{5pt}
    \begin{tabular}{lcr cc cc}
        \toprule
         &  &  & \multicolumn{4}{c}{\textit{Accuracy} ($\uparrow$)} \\
         \cmidrule(lr){4-7}
         & & & \multicolumn{2}{c}{\textit{1-layer}} & \multicolumn{2}{c}{\textit{2-layers}} \\
         \cmidrule(lr){4-5}\cmidrule(lr){6-7}
         \textbf{Objective} & $q_\varphi$ & \textbf{Model} & \textit{2D} & \textit{50D} & \textit{2D} & \textit{50D} \\
         \midrule
\multirow{4}{*}{Baseline} & - & Random & $50.278$\sstd{$0.337$} & $50.028$\sstd{$0.064$} & $50.188$\sstd{$0.479$} & $49.982$\sstd{$0.084$} \\
& - & Optimization & $96.943$\sstd{$0.012$} & $68.086$\sstd{$0.016$} & $94.444$\sstd{$0.024$} & $63.950$\sstd{$0.022$} \\
& - & Langevin & $95.143$\sstd{$0.094$} & $61.694$\sstd{$0.340$} & $92.719$\sstd{$0.016$} & $57.447$\sstd{$0.437$} \\
& - & HMC & $92.489$\sstd{$0.338$} & $59.963$\sstd{$0.202$} & $87.548$\sstd{$0.094$} & $56.319$\sstd{$0.882$} \\
\cmidrule{2-7}

\multirow{3}{*}{Fwd-KL} & \multirow{6}{*}{\rotatebox[origin=c]{90}{Gaussian}} & GRU &$50.274$\sstd{$0.337$} & $50.023$\sstd{$0.060$} & $50.187$\sstd{$0.477$} & $49.992$\sstd{$0.072$} \\
& & DeepSets &$50.271$\sstd{$0.334$} & $50.024$\sstd{$0.061$} & $50.188$\sstd{$0.472$} & $49.984$\sstd{$0.077$} \\
& & Transformer &$50.273$\sstd{$0.336$} & $50.031$\sstd{$0.066$} & $50.191$\sstd{$0.471$} & $49.994$\sstd{$0.078$} \\
\cmidrule{3-7}

\multirow{3}{*}{Rev-KL} & & GRU &$89.270$\sstd{$0.272$} & $50.014$\sstd{$0.048$} & $50.191$\sstd{$0.463$} & $49.995$\sstd{$0.080$} \\
& & DeepSets & \highlight{$89.788$\sstd{$0.213$}} & $50.018$\sstd{$0.061$} & $50.191$\sstd{$0.478$} & $49.977$\sstd{$0.075$} \\
& & Transformer &$89.366$\sstd{$0.108$} & $64.926$\sstd{$0.260$} & $50.182$\sstd{$0.469$} & $49.986$\sstd{$0.071$} \\
\cmidrule{2-7}

\multirow{3}{*}{Fwd-KL} & \multirow{6}{*}{\rotatebox[origin=c]{90}{Flow}} & GRU &$49.651$\sstd{$0.040$} & $50.119$\sstd{$0.068$} & $49.987$\sstd{$0.015$} & $49.904$\sstd{$0.018$} \\
& & DeepSets &$49.639$\sstd{$0.031$} & $50.113$\sstd{$0.065$} & $49.988$\sstd{$0.022$} & $49.910$\sstd{$0.043$} \\
& & Transformer &$49.636$\sstd{$0.040$} & $50.115$\sstd{$0.063$} & $49.989$\sstd{$0.017$} & $49.909$\sstd{$0.042$} \\
\cmidrule{3-7}

\multirow{3}{*}{Rev-KL} & & GRU &$49.769$\sstd{$0.141$} & $50.082$\sstd{$0.091$} & $49.915$\sstd{$0.070$} & $50.004$\sstd{$0.084$} \\
& & DeepSets &$49.782$\sstd{$0.073$} & $50.080$\sstd{$0.087$} & $49.831$\sstd{$0.152$} & $49.994$\sstd{$0.078$} \\
& & Transformer &$63.233$\sstd{$19.243$} & $50.026$\sstd{$0.047$} & $49.869$\sstd{$0.207$} & $50.036$\sstd{$0.056$} \\
\bottomrule
    \end{tabular}
    \caption{\textbf{Variable-Dimensional}. Results for estimating the parameters of nonlinear classification models with TanH activation function and two classes, with the expected accuracy according to the posterior predictive as metric.}
    \vspace{-4mm}
    \label{tab:}
\end{table*}
\begin{table*}[t]
    \centering
    \small
    % \footnotesize	    
    \def\arraystretch{1.25}
    \setlength{\tabcolsep}{5pt}
    \begin{tabular}{lcr cc cc}
        \toprule
         &  &  & \multicolumn{4}{c}{\textit{Accuracy} ($\uparrow$)} \\
         \cmidrule(lr){4-7}
         & & & \multicolumn{2}{c}{\textit{1-layer}} & \multicolumn{2}{c}{\textit{2-layers}} \\
         \cmidrule(lr){4-5}\cmidrule(lr){6-7}
         \textbf{Objective} & $q_\varphi$ & \textbf{Model} & \textit{2D} & \textit{50D} & \textit{2D} & \textit{50D} \\
         \midrule
\multirow{4}{*}{Baseline} & - & Random & $20.041$\sstd{$0.136$} & $20.002$\sstd{$0.079$} & $19.914$\sstd{$0.111$} & $19.954$\sstd{$0.005$} \\
& - & Optimization & $92.059$\sstd{$0.012$} & $40.722$\sstd{$0.027$} & $88.848$\sstd{$0.005$} & $34.136$\sstd{$0.041$} \\
& - & Langevin & $88.357$\sstd{$0.309$} & $30.941$\sstd{$0.097$} & $83.788$\sstd{$0.170$} & $26.538$\sstd{$0.206$} \\
& - & HMC & $79.161$\sstd{$0.292$} & $27.508$\sstd{$0.379$} & $74.987$\sstd{$0.130$} & $25.377$\sstd{$0.246$} \\
\cmidrule{2-7}

\multirow{3}{*}{Fwd-KL} & \multirow{6}{*}{\rotatebox[origin=c]{90}{Gaussian}} & GRU &$20.264$\sstd{$0.139$} & $20.158$\sstd{$0.056$} & $20.215$\sstd{$0.138$} & $20.093$\sstd{$0.028$} \\
& & DeepSets &$20.042$\sstd{$0.133$} & $20.000$\sstd{$0.088$} & $19.916$\sstd{$0.114$} & $19.955$\sstd{$0.007$} \\
& & Transformer &$20.240$\sstd{$0.126$} & $20.153$\sstd{$0.070$} & $20.118$\sstd{$0.131$} & $20.089$\sstd{$0.022$} \\
\cmidrule{3-7}

\multirow{3}{*}{Rev-KL} & & GRU &$66.565$\sstd{$7.725$} & $20.011$\sstd{$0.099$} & $19.913$\sstd{$0.125$} & $19.954$\sstd{$0.022$} \\
& & DeepSets &$57.294$\sstd{$0.362$} & $20.011$\sstd{$0.091$} & $19.915$\sstd{$0.115$} & $19.959$\sstd{$0.020$} \\
& & Transformer & \highlight{$72.865$\sstd{$1.340$}} & $22.185$\sstd{$1.872$} & $19.911$\sstd{$0.124$} & $19.953$\sstd{$0.014$} \\
\cmidrule{2-7}

\multirow{3}{*}{Fwd-KL} & \multirow{6}{*}{\rotatebox[origin=c]{90}{Flow}} & GRU &$19.963$\sstd{$0.239$} & $20.176$\sstd{$0.056$} & $19.952$\sstd{$0.189$} & $20.156$\sstd{$0.079$} \\
& & DeepSets &$19.757$\sstd{$0.259$} & $20.045$\sstd{$0.071$} & $19.692$\sstd{$0.184$} & $20.019$\sstd{$0.069$} \\
& & Transformer &$19.925$\sstd{$0.262$} & $20.185$\sstd{$0.059$} & $19.882$\sstd{$0.175$} & $20.159$\sstd{$0.070$} \\
\cmidrule{3-7}

\multirow{3}{*}{Rev-KL} & & GRU &$67.042$\sstd{$2.230$} & $20.065$\sstd{$0.060$} & $19.707$\sstd{$0.245$} & $19.989$\sstd{$0.101$} \\
& & DeepSets &$35.220$\sstd{$10.870$} & $20.000$\sstd{$0.054$} & $19.739$\sstd{$0.216$} & $19.966$\sstd{$0.019$} \\
& & Transformer & \highlight{$72.798$\sstd{$1.049$}} & $20.032$\sstd{$0.040$} & $19.752$\sstd{$0.276$} & $20.017$\sstd{$0.037$} \\
\bottomrule
    \end{tabular}
    \caption{\textbf{Variable-Dimensional}. Results for estimating the parameters of nonlinear classification models with TanH activation function and five classes, with the expected accuracy according to the posterior predictive as metric.}
    \vspace{-4mm}
    \label{tab:variable_apdx_nlc_5cl}
\end{table*}
\textbf{Linear Regression (LR):} We then look at the problem of estimating the posterior over the weight vector for Bayesian linear regression given a dataset $p(\vw, b | \gD)$, where the underlying model $p(\gD, \mtheta)$ is given by:
\begin{align}
    p(\vw) &= \gN(\vw | \mathbf{0}, \mathbf{I})\\
    p(b) &= \gN(b | 0, 1)\\
    p(y | \vx, \vw, b) &= \gN\left(y | \vw^T\vx + b, \sigma^2\right) \, ,
\end{align}
and with $\sigma^2 = 0.25$ known beforehand. Inputs $\vx$ are generated from $p(\vx) = \gN(\mathbf{0}, I)$.


\textbf{Linear Classification (LC):}
We now consider a setting where the true posterior cannot be obtained analytically as the likelihood and prior are not conjugate. In this case, we consider the underlying probabilistic model by:
\begin{align}
    p(\mW) &= \gN\left(\mW | \mathbf{0}, \mathbf{I}\right)\\
    p(y | \vx, \mW) &= \mathrm{Categorical}\left(y  \;\vline\; \frac{1}{\tau}\;\mW\vx\right)\, ,
\end{align}
where $\tau$ is the known temperature term which is kept as $0.1$ to ensure peaky distributions, and $\vx$ is being generated from $p(\vx) = \gN(\mathbf{0}, I)$.


\textbf{Nonlinear Regression (NLR):}
Next, we tackle the more complex problem where the posterior distribution is multi-modal and obtaining multiple modes or even a single good one is challenging. For this, we consider the model as a Bayesian Neural Network (BNN) for regression with fixed hyper-parameters like the number of layers, dimensionality of the hidden layer, etc. Let the BNN denote the function $f_\mtheta$ where $\mtheta$ are the network parameters such that the estimation problem is to approximate $p(\mtheta | \gD)$. Then, for regression, we specify the probabilistic model using:
\begin{align}
    p(\mtheta) &= \gN\left(\mtheta | \mathbf{0}, \mathbf{I}\right)\\
    p(y | \vx, \mtheta) &= \gN\left(y | f_\mtheta(\vx), \sigma^2\right) \, ,
\end{align}
where $\sigma^2 = 0.25$ is a known quantity and $\vx$ being generated from $p(\vx) = \gN(\mathbf{0}, I)$.
 
\textbf{Nonlinear Classification (NLC):}
Like in Nonlinear Regression, we consider BNNs with fixed hyper-parameters for classification problems with the same estimation task of approximating $p(\mtheta | \gD)$. In this formulation, we consider the probabilistic model as:
\begin{align}
    p(\mtheta) &= \gN\left(\mtheta | \mathbf{0}, \mathbf{I}\right)\\
    p(y | \vx, \mtheta) &= \mathrm{Categorical}\left(y \;\vline\; \frac{1}{\tau}\;f_\mtheta(\vx)\right)
\end{align}
where $\tau$ is the known temperature term which is kept as $0.1$ to ensure peaky distributions, and $\vx$ is being generated from $p(\vx) = \gN(\mathbf{0}, I)$.

\textbf{Gaussian Mixture Model (GMM):}
While we have mostly looked at predictive problems, where the task is to model some predictive variable $y$ conditioned on some input $\vx$, we now look at a well-known probabilistic model for unsupervised learning, Gaussian Mixture Model (GMM), primarily used to cluster data. Consider a $K$-cluster GMM with:
\begin{align}
    p(\mmu_k) &= \gN\left(\mmu_k | \mathbf{0}, \mathbf{I}\right)\\
    p(\vx | \mmu_{1:K}) &= \sum_{k=1}^K \pi_k \gN\left(\vx | \mmu_k, \mSigma_k\right) \, .
\end{align}
 We assume $\mSigma_k$ and $\pi_k$ to be known and set $\mSigma_k$ to be an identity matrix and the mixing coefficients to be equal, $\pi_k = 1/K$, for all clusters $k$ in our experiments. 
\section{Metrics}

Ensuring that large language models (LLMs) accurately follow instructions is crucial for code generation. To precisely evaluate this capability, we introduce four novel metrics designed to assess how LLMs handle code generation tasks with multiple constraints: \textbf{Completely Satisfaction Rate (CSR)}, \textbf{Soft Satisfaction Rate (SSR)}, \textbf{Rigorous Satisfaction Rate (RSR)}, and \textbf{Consistent Continuity Satisfaction Rate (CCSR)}. These metrics provide a comprehensive evaluation from different perspectives.

For a dataset with $m$ problems, each problem contains a set of $n_i$ constraints. We define CSR and SSR as follows:

\paragraph{Completely Satisfaction Rate (CSR)}
\begin{equation}
\text{CSR} = \frac{1}{m} \sum_{i=1}^{m} \left( \prod_{j=1}^{n_i} r_{i,j} \right)
\end{equation}
where $r_{i,j} \in [0,1]$ indicates whether the $j$-th constraint in the $i$-th problem is satisfied. CSR measures the proportion of problems where all constraints are fully met.

\paragraph{Soft Satisfaction Rate (SSR)}
\begin{equation}
\text{SSR} = \frac{1}{m} \sum_{i=1}^{m} \left( \frac{\sum_{j=1}^{n_i} r_{i,j}}{n_i} \right)
\end{equation}
SSR evaluates the average proportion of constraints satisfied per problem, providing a more flexible assessment.

\paragraph{Rigorous Satisfaction Rate (RSR)}
In code generation, some constraints depend on prior instructions, particularly in \textbf{Combination} constraints. To account for dependencies, we define RSR as:
\begin{equation}
\text{RSR} = \frac{1}{m} \sum_{i=1}^{m} \left( \frac{\sum_{j=1}^{n_i} \left[ r_{i,j} \cdot \prod_{k \in D_{i,j}} r_{i,k} \right]}{n_i} \right)
\end{equation}
where $D_{i,j}$ represents the set of constraints that the $j$-th constraint in the $i$-th problem depends on. RSR ensures that models satisfy prerequisite constraints before fulfilling dependent ones.

\paragraph{Consistent Continuity Satisfaction Rate (CCSR)}
In many code generation tasks, maintaining continuous adherence to instructions is essential. To measure this ability, we define CCSR as:
\begin{equation}
\small
\text{CCSR} = \frac{1}{m} \sum_{i=1}^{m} \frac{L_i}{n_i}, \\
L_i = \max \Bigl\{ l \,\Big|\, \exists t \mathbin{\in} [1, n_i{-}l{+}1],\ 
\prod_{\mathclap{j=t}}^{\mathclap{t+l-1}} r_{i,j} = 1 \Bigr\}
\end{equation}
where $L_i$ represents the longest consecutive sequence of satisfied constraints in problem $i$. CCSR evaluates a model’s consistency in following sequential instructions without errors.



\section{Architecture Details}
\label{appdx:architecture}
In this section, we outline the two candidate architectures that we consider for the backbone of our amortized variational inference model. We discuss the specifics of the architectures and the hyperparameters used for our experiments.

\subsection{Transformer}
\label{subsec:transformer}
We use a transformer model~\citep{vaswani2017attention} as a permutation invariant architecture by removing positional encodings from the setup and using multiple layers of the encoder model. We append the set of observations with a [CLS] token before passing it to the model and use its output embedding to predict the parameters of the variational distribution. Since no positional encodings or causal masking is used in the whole setup, the final embedding of the [CLS] token becomes invariant to permutations in the set of observations, thereby leading to permutation invariance in the parameters of $q_\varphi$.

We use $4$ encoder layers with a $256$ dimensional attention block and $1024$ feed-forward dimensions, with $4$ heads in each attention block for our Transformer models to make the number of parameters comparative to the one of the DeepSets model.

\subsection{DeepSets}
\label{subsec:deepsets}
Another framework that can process set-based input is Deep Sets~\citep{Zaheer2017deepsets}. In our experiments, we used an embedding network that encodes the input into representation space, a mean aggregation operation, which ensures that the representation learned is invariant concerning the set ordering, and a regression network. The latter's output is either used to directly parameterize a diagonal Gaussian or as conditional input to a normalizing flow, representing a summary statistics of the set input.

For DeepSets, we use $4$ layers each in the embedding network and the regression network, with a mean aggregation function, ReLU activation functions, and $627$ hidden dimensions to make the number of parameters comparable to those in the Transformer model.

\subsection{RNN}
For the recurrent neural network setup, we use the Gated Recurrent Unit (GRU). Similar to the above setups, we use a $4$-layered GRU model with $256$ hidden dimensions. While such an architecture is not permutation invariant, by training on tasks that require such invariance could encourage learning of solution structure that respects this invariance.

\subsection{Normalizing Flows}
\label{subsec:flows}
Assuming a Gaussian posterior distribution as the approximate often leads to poor results as the true posterior distribution can be far from the Gaussian shape. To allow for more flexible posterior distributions, we use normalizing flows~\citep{kingma2018glow,kobyzev2020normalizing,papamakarios2021normalizing,rezende2015variational} for approximating $q_\varphi(\mtheta | \gD)$ conditioned on the output of the summary network $h_\psi$. Specifically, let $g_\nu: \vz \mapsto \mtheta$ be a diffeomorphism parameterized by a conditional invertible neural network (cINN) with network parameters $\nu$ such that $\mtheta = g_\nu(\vz; h_\psi(\gD))$. With the change-of-variables formula it follows that $p(\mtheta)=p(\vz)\left|\det \frac{\partial}{\partial\vz}g_\nu(\vz; h_\psi(\gD))\right|^{-1} = p(\vz)|\det J_\nu(\vz; h_\psi(\gD))|^{-1}$, where $J_\nu$ is the Jacobian matrix of $g_\nu$. Further, integration by substitution gives us $d\mtheta = |\det J_\nu(\vz; h_\psi(\gD)| d\vz$ to rewrite the objective from eq. \ref{eq:arkl} as:
\begin{align}
    &\argmin_\varphi \sE_{\gD \sim \chi} \sK\sL[q_\varphi(\mtheta|\gD) || p(\mtheta|\gD)]\\
    &= \argmin_\varphi \sE_{\gD \sim \chi} \sE_{\mtheta \sim q_\varphi(\mtheta|\gD)} \left[ \log q_\varphi(\mtheta|\gD) - \log p(\mtheta, \gD) \right]\\
    &= \argmin_{\varphi=\{\psi, \nu\}} \sE_{\gD \sim \chi} \sE_{\vz \sim p(\vz)} \left[ \log \frac{q_\nu (\vz|h_\psi(\gD))}{\left| \det J_\nu(\vz; h_\psi(\gD)) \right|} - \log p(g_\nu(\vz; h_\psi(\gD)), \gD) \right]
\end{align}
As shown in BayesFlow \citep{radev2020bayesflow}, the normalizing flow $g_\nu$ and the summary network $h_\psi$ can be trained simultaneously. The $\mathrm{AllInOneBlock}$ coupling block architecture of the FrEIA Python package \citep{Ardizzone2018freia}, which is very similar to the RNVP style coupling block \citep{Dinh2017rnvp}, is used as the basis for the cINN. $\mathrm{AllInOneBlock}$ combines the most common architectural components, such as ActNorm, permutation, and affine coupling operations.

For our experiments, $6$ coupling blocks define the normalizing flow network, each with a $1$ hidden-layered non-linear feed-forward subnetwork with ReLU non-linearity and $128$ hidden dimensions.
\section{Experimental details}
\label{appendix:experiments}

We use the following search space for hyperparameters:
\begin{itemize}
    \item $c_s \in \mathbb{Z}^+$: Maximum number of tokens in each document chunk.
    \item $c_n \in \mathbb{Z}^+$: Number of chunks retrieved from the vector database for each query.
    \item $o \in \mathbb{Z}^+$: Number of tokens which overlap between adjacent chunks in a document.
    \item $t \in [0,1.2]$: Temperature of the LLM when generating responses.
    \item $r \in [0, 1]$: Rerank threshold used to set the minimum similarity between the context chunk and query, as evaluated by the reranker\footnote{We use a fixed rerank model \texttt{Salesforce/Llama-Rank-V1} provided by TogetherAI for all RAG systems.}. Retrieved documents which are below this threshold are ignored and not passed to the LLM as context. If no chunks exceed this threshold, we choose only the highest scoring chunk as context.
    \item $\ell \in \{\text{gpt-4o}, \text{gpt-4o-mini}, \text{llama-3.2-3B}, \text{llama-3.1-8B}\}$: Choice of LLM used to generate the response.
    \item $e \in \{\text{text-embedding-3-large},\text{text-embedding-3-small}
    \}$: Choice of embedding model when embedding the queries and document chunks.
\end{itemize}
\begin{table}[h] \footnotesize  \centering\resizebox{0.48\textwidth}{!}{\begin{tabular}{c|l|c|c|c}
\toprule
\textbf{Task} & \textbf{Dataset} & \textbf{N-shot} & \multirowcell{\textbf{Train texts} \\ \textbf{for STMD}} & \multirowcell{\textbf{Evaluation} \\ \textbf{texts}} \\
\midrule
\multirow{3}{*}{\multirowcell{Text \\ Summarization}} & CNN/DailyMail & 0 & 2,000 & 2,000 \\
& XSum & 0 & 2,000 & 2,000 \\
& SamSum & 0 & 2,000 & 819 \\
\midrule
\multirow{4}{*}{\multirowcell{QA \\ Long answer}} & PubMedQA & 0 & 2,000 & 2,000 \\
& MedQUAD & 5 & 2,000 & 2,000 \\
& TruthfulQA & 5 & 408 & 409 \\
& GSM8k & 5 & 2,000 & 1,319 \\
\midrule
\multirow{4}{*}{\multirowcell{QA \\ Short answer}} & SciQ & 0 & 5,000 & 1,000 \\
& CoQA & \multirowcell{all preceding \\ questions} & 5,000 & 2,000 \\
& TriviaQA & 5 & 5,000 & 2,000 \\
\midrule
\multirow{1}{*}{\multirowcell{MCQA}} & MMLU & 5 & 5,000 & 2,000 \\
\bottomrule
\end{tabular}
}\caption{\label{tab:dataset_stat} The statistics of the datasets used for evaluation.}
\end{table}
\begin{table}[ht!]
\centering
\caption{\textbf{Super Resolution Performance Results.} Our proposed WGAN EEG Spatial Upsampling method significantly outperforms a baseline of Bicubic Interpolation commonly used in EEG upsampling pipelines.}
\label{tab:results}
\resizebox{0.8\linewidth}{!}{%
\begin{tabular}{@{}cccccc@{}}
\toprule
\multirow{2}{*}{\textbf{Dataset}} & \multirow{2}{*}{\textbf{Scale}} & \multicolumn{2}{c}{\textbf{Bicubic}} & \multicolumn{2}{c}{\textbf{WGAN}} \\ \cmidrule(l){3-6} 
                      &   & \textbf{MSE} & \textbf{MAE} & \textbf{MSE}    & \textbf{MAE}   \\
\toprule
\multirow{2}{*}{Val}  & 2 & 3.71E7       & 3.89E3       & \textbf{2.01E3} & \textbf{24.38} \\
                      & 4 & 7.23E7       & 6.42E3       & \textbf{8.53E3} & \textbf{63.83} \\
\midrule
\multirow{2}{*}{Test} & 2 & 3.75E7       & 3.91E3       & \textbf{2.06E3} & \textbf{24.66} \\
                      & 4 & 7.30E7       & 6.45E3       & \textbf{8.68E3} & \textbf{64.39} \\
\bottomrule
\end{tabular}%
}
\end{table}

\end{document}