%%%%%%%% ICML 2025 EXAMPLE LATEX SUBMISSION FILE %%%%%%%%%%%%%%%%%

\documentclass{article}
%%%%% NEW MATH DEFINITIONS %%%%%

\usepackage{amsmath,amsfonts,bm}
\usepackage{derivative}
% Mark sections of captions for referring to divisions of figures
\newcommand{\figleft}{{\em (Left)}}
\newcommand{\figcenter}{{\em (Center)}}
\newcommand{\figright}{{\em (Right)}}
\newcommand{\figtop}{{\em (Top)}}
\newcommand{\figbottom}{{\em (Bottom)}}
\newcommand{\captiona}{{\em (a)}}
\newcommand{\captionb}{{\em (b)}}
\newcommand{\captionc}{{\em (c)}}
\newcommand{\captiond}{{\em (d)}}

% Highlight a newly defined term
\newcommand{\newterm}[1]{{\bf #1}}

% Derivative d 
\newcommand{\deriv}{{\mathrm{d}}}

% Figure reference, lower-case.
\def\figref#1{figure~\ref{#1}}
% Figure reference, capital. For start of sentence
\def\Figref#1{Figure~\ref{#1}}
\def\twofigref#1#2{figures \ref{#1} and \ref{#2}}
\def\quadfigref#1#2#3#4{figures \ref{#1}, \ref{#2}, \ref{#3} and \ref{#4}}
% Section reference, lower-case.
\def\secref#1{section~\ref{#1}}
% Section reference, capital.
\def\Secref#1{Section~\ref{#1}}
% Reference to two sections.
\def\twosecrefs#1#2{sections \ref{#1} and \ref{#2}}
% Reference to three sections.
\def\secrefs#1#2#3{sections \ref{#1}, \ref{#2} and \ref{#3}}
% Reference to an equation, lower-case.
\def\eqref#1{equation~\ref{#1}}
% Reference to an equation, upper case
\def\Eqref#1{Equation~\ref{#1}}
% A raw reference to an equation---avoid using if possible
\def\plaineqref#1{\ref{#1}}
% Reference to a chapter, lower-case.
\def\chapref#1{chapter~\ref{#1}}
% Reference to an equation, upper case.
\def\Chapref#1{Chapter~\ref{#1}}
% Reference to a range of chapters
\def\rangechapref#1#2{chapters\ref{#1}--\ref{#2}}
% Reference to an algorithm, lower-case.
\def\algref#1{algorithm~\ref{#1}}
% Reference to an algorithm, upper case.
\def\Algref#1{Algorithm~\ref{#1}}
\def\twoalgref#1#2{algorithms \ref{#1} and \ref{#2}}
\def\Twoalgref#1#2{Algorithms \ref{#1} and \ref{#2}}
% Reference to a part, lower case
\def\partref#1{part~\ref{#1}}
% Reference to a part, upper case
\def\Partref#1{Part~\ref{#1}}
\def\twopartref#1#2{parts \ref{#1} and \ref{#2}}

\def\ceil#1{\lceil #1 \rceil}
\def\floor#1{\lfloor #1 \rfloor}
\def\1{\bm{1}}
\newcommand{\train}{\mathcal{D}}
\newcommand{\valid}{\mathcal{D_{\mathrm{valid}}}}
\newcommand{\test}{\mathcal{D_{\mathrm{test}}}}

\def\eps{{\epsilon}}


% Random variables
\def\reta{{\textnormal{$\eta$}}}
\def\ra{{\textnormal{a}}}
\def\rb{{\textnormal{b}}}
\def\rc{{\textnormal{c}}}
\def\rd{{\textnormal{d}}}
\def\re{{\textnormal{e}}}
\def\rf{{\textnormal{f}}}
\def\rg{{\textnormal{g}}}
\def\rh{{\textnormal{h}}}
\def\ri{{\textnormal{i}}}
\def\rj{{\textnormal{j}}}
\def\rk{{\textnormal{k}}}
\def\rl{{\textnormal{l}}}
% rm is already a command, just don't name any random variables m
\def\rn{{\textnormal{n}}}
\def\ro{{\textnormal{o}}}
\def\rp{{\textnormal{p}}}
\def\rq{{\textnormal{q}}}
\def\rr{{\textnormal{r}}}
\def\rs{{\textnormal{s}}}
\def\rt{{\textnormal{t}}}
\def\ru{{\textnormal{u}}}
\def\rv{{\textnormal{v}}}
\def\rw{{\textnormal{w}}}
\def\rx{{\textnormal{x}}}
\def\ry{{\textnormal{y}}}
\def\rz{{\textnormal{z}}}

% Random vectors
\def\rvepsilon{{\mathbf{\epsilon}}}
\def\rvphi{{\mathbf{\phi}}}
\def\rvtheta{{\mathbf{\theta}}}
\def\rva{{\mathbf{a}}}
\def\rvb{{\mathbf{b}}}
\def\rvc{{\mathbf{c}}}
\def\rvd{{\mathbf{d}}}
\def\rve{{\mathbf{e}}}
\def\rvf{{\mathbf{f}}}
\def\rvg{{\mathbf{g}}}
\def\rvh{{\mathbf{h}}}
\def\rvu{{\mathbf{i}}}
\def\rvj{{\mathbf{j}}}
\def\rvk{{\mathbf{k}}}
\def\rvl{{\mathbf{l}}}
\def\rvm{{\mathbf{m}}}
\def\rvn{{\mathbf{n}}}
\def\rvo{{\mathbf{o}}}
\def\rvp{{\mathbf{p}}}
\def\rvq{{\mathbf{q}}}
\def\rvr{{\mathbf{r}}}
\def\rvs{{\mathbf{s}}}
\def\rvt{{\mathbf{t}}}
\def\rvu{{\mathbf{u}}}
\def\rvv{{\mathbf{v}}}
\def\rvw{{\mathbf{w}}}
\def\rvx{{\mathbf{x}}}
\def\rvy{{\mathbf{y}}}
\def\rvz{{\mathbf{z}}}

% Elements of random vectors
\def\erva{{\textnormal{a}}}
\def\ervb{{\textnormal{b}}}
\def\ervc{{\textnormal{c}}}
\def\ervd{{\textnormal{d}}}
\def\erve{{\textnormal{e}}}
\def\ervf{{\textnormal{f}}}
\def\ervg{{\textnormal{g}}}
\def\ervh{{\textnormal{h}}}
\def\ervi{{\textnormal{i}}}
\def\ervj{{\textnormal{j}}}
\def\ervk{{\textnormal{k}}}
\def\ervl{{\textnormal{l}}}
\def\ervm{{\textnormal{m}}}
\def\ervn{{\textnormal{n}}}
\def\ervo{{\textnormal{o}}}
\def\ervp{{\textnormal{p}}}
\def\ervq{{\textnormal{q}}}
\def\ervr{{\textnormal{r}}}
\def\ervs{{\textnormal{s}}}
\def\ervt{{\textnormal{t}}}
\def\ervu{{\textnormal{u}}}
\def\ervv{{\textnormal{v}}}
\def\ervw{{\textnormal{w}}}
\def\ervx{{\textnormal{x}}}
\def\ervy{{\textnormal{y}}}
\def\ervz{{\textnormal{z}}}

% Random matrices
\def\rmA{{\mathbf{A}}}
\def\rmB{{\mathbf{B}}}
\def\rmC{{\mathbf{C}}}
\def\rmD{{\mathbf{D}}}
\def\rmE{{\mathbf{E}}}
\def\rmF{{\mathbf{F}}}
\def\rmG{{\mathbf{G}}}
\def\rmH{{\mathbf{H}}}
\def\rmI{{\mathbf{I}}}
\def\rmJ{{\mathbf{J}}}
\def\rmK{{\mathbf{K}}}
\def\rmL{{\mathbf{L}}}
\def\rmM{{\mathbf{M}}}
\def\rmN{{\mathbf{N}}}
\def\rmO{{\mathbf{O}}}
\def\rmP{{\mathbf{P}}}
\def\rmQ{{\mathbf{Q}}}
\def\rmR{{\mathbf{R}}}
\def\rmS{{\mathbf{S}}}
\def\rmT{{\mathbf{T}}}
\def\rmU{{\mathbf{U}}}
\def\rmV{{\mathbf{V}}}
\def\rmW{{\mathbf{W}}}
\def\rmX{{\mathbf{X}}}
\def\rmY{{\mathbf{Y}}}
\def\rmZ{{\mathbf{Z}}}

% Elements of random matrices
\def\ermA{{\textnormal{A}}}
\def\ermB{{\textnormal{B}}}
\def\ermC{{\textnormal{C}}}
\def\ermD{{\textnormal{D}}}
\def\ermE{{\textnormal{E}}}
\def\ermF{{\textnormal{F}}}
\def\ermG{{\textnormal{G}}}
\def\ermH{{\textnormal{H}}}
\def\ermI{{\textnormal{I}}}
\def\ermJ{{\textnormal{J}}}
\def\ermK{{\textnormal{K}}}
\def\ermL{{\textnormal{L}}}
\def\ermM{{\textnormal{M}}}
\def\ermN{{\textnormal{N}}}
\def\ermO{{\textnormal{O}}}
\def\ermP{{\textnormal{P}}}
\def\ermQ{{\textnormal{Q}}}
\def\ermR{{\textnormal{R}}}
\def\ermS{{\textnormal{S}}}
\def\ermT{{\textnormal{T}}}
\def\ermU{{\textnormal{U}}}
\def\ermV{{\textnormal{V}}}
\def\ermW{{\textnormal{W}}}
\def\ermX{{\textnormal{X}}}
\def\ermY{{\textnormal{Y}}}
\def\ermZ{{\textnormal{Z}}}

% Vectors
\def\vzero{{\bm{0}}}
\def\vone{{\bm{1}}}
\def\vmu{{\bm{\mu}}}
\def\vtheta{{\bm{\theta}}}
\def\vphi{{\bm{\phi}}}
\def\va{{\bm{a}}}
\def\vb{{\bm{b}}}
\def\vc{{\bm{c}}}
\def\vd{{\bm{d}}}
\def\ve{{\bm{e}}}
\def\vf{{\bm{f}}}
\def\vg{{\bm{g}}}
\def\vh{{\bm{h}}}
\def\vi{{\bm{i}}}
\def\vj{{\bm{j}}}
\def\vk{{\bm{k}}}
\def\vl{{\bm{l}}}
\def\vm{{\bm{m}}}
\def\vn{{\bm{n}}}
\def\vo{{\bm{o}}}
\def\vp{{\bm{p}}}
\def\vq{{\bm{q}}}
\def\vr{{\bm{r}}}
\def\vs{{\bm{s}}}
\def\vt{{\bm{t}}}
\def\vu{{\bm{u}}}
\def\vv{{\bm{v}}}
\def\vw{{\bm{w}}}
\def\vx{{\bm{x}}}
\def\vy{{\bm{y}}}
\def\vz{{\bm{z}}}

% Elements of vectors
\def\evalpha{{\alpha}}
\def\evbeta{{\beta}}
\def\evepsilon{{\epsilon}}
\def\evlambda{{\lambda}}
\def\evomega{{\omega}}
\def\evmu{{\mu}}
\def\evpsi{{\psi}}
\def\evsigma{{\sigma}}
\def\evtheta{{\theta}}
\def\eva{{a}}
\def\evb{{b}}
\def\evc{{c}}
\def\evd{{d}}
\def\eve{{e}}
\def\evf{{f}}
\def\evg{{g}}
\def\evh{{h}}
\def\evi{{i}}
\def\evj{{j}}
\def\evk{{k}}
\def\evl{{l}}
\def\evm{{m}}
\def\evn{{n}}
\def\evo{{o}}
\def\evp{{p}}
\def\evq{{q}}
\def\evr{{r}}
\def\evs{{s}}
\def\evt{{t}}
\def\evu{{u}}
\def\evv{{v}}
\def\evw{{w}}
\def\evx{{x}}
\def\evy{{y}}
\def\evz{{z}}

% Matrix
\def\mA{{\bm{A}}}
\def\mB{{\bm{B}}}
\def\mC{{\bm{C}}}
\def\mD{{\bm{D}}}
\def\mE{{\bm{E}}}
\def\mF{{\bm{F}}}
\def\mG{{\bm{G}}}
\def\mH{{\bm{H}}}
\def\mI{{\bm{I}}}
\def\mJ{{\bm{J}}}
\def\mK{{\bm{K}}}
\def\mL{{\bm{L}}}
\def\mM{{\bm{M}}}
\def\mN{{\bm{N}}}
\def\mO{{\bm{O}}}
\def\mP{{\bm{P}}}
\def\mQ{{\bm{Q}}}
\def\mR{{\bm{R}}}
\def\mS{{\bm{S}}}
\def\mT{{\bm{T}}}
\def\mU{{\bm{U}}}
\def\mV{{\bm{V}}}
\def\mW{{\bm{W}}}
\def\mX{{\bm{X}}}
\def\mY{{\bm{Y}}}
\def\mZ{{\bm{Z}}}
\def\mBeta{{\bm{\beta}}}
\def\mPhi{{\bm{\Phi}}}
\def\mLambda{{\bm{\Lambda}}}
\def\mSigma{{\bm{\Sigma}}}

% Tensor
\DeclareMathAlphabet{\mathsfit}{\encodingdefault}{\sfdefault}{m}{sl}
\SetMathAlphabet{\mathsfit}{bold}{\encodingdefault}{\sfdefault}{bx}{n}
\newcommand{\tens}[1]{\bm{\mathsfit{#1}}}
\def\tA{{\tens{A}}}
\def\tB{{\tens{B}}}
\def\tC{{\tens{C}}}
\def\tD{{\tens{D}}}
\def\tE{{\tens{E}}}
\def\tF{{\tens{F}}}
\def\tG{{\tens{G}}}
\def\tH{{\tens{H}}}
\def\tI{{\tens{I}}}
\def\tJ{{\tens{J}}}
\def\tK{{\tens{K}}}
\def\tL{{\tens{L}}}
\def\tM{{\tens{M}}}
\def\tN{{\tens{N}}}
\def\tO{{\tens{O}}}
\def\tP{{\tens{P}}}
\def\tQ{{\tens{Q}}}
\def\tR{{\tens{R}}}
\def\tS{{\tens{S}}}
\def\tT{{\tens{T}}}
\def\tU{{\tens{U}}}
\def\tV{{\tens{V}}}
\def\tW{{\tens{W}}}
\def\tX{{\tens{X}}}
\def\tY{{\tens{Y}}}
\def\tZ{{\tens{Z}}}


% Graph
\def\gA{{\mathcal{A}}}
\def\gB{{\mathcal{B}}}
\def\gC{{\mathcal{C}}}
\def\gD{{\mathcal{D}}}
\def\gE{{\mathcal{E}}}
\def\gF{{\mathcal{F}}}
\def\gG{{\mathcal{G}}}
\def\gH{{\mathcal{H}}}
\def\gI{{\mathcal{I}}}
\def\gJ{{\mathcal{J}}}
\def\gK{{\mathcal{K}}}
\def\gL{{\mathcal{L}}}
\def\gM{{\mathcal{M}}}
\def\gN{{\mathcal{N}}}
\def\gO{{\mathcal{O}}}
\def\gP{{\mathcal{P}}}
\def\gQ{{\mathcal{Q}}}
\def\gR{{\mathcal{R}}}
\def\gS{{\mathcal{S}}}
\def\gT{{\mathcal{T}}}
\def\gU{{\mathcal{U}}}
\def\gV{{\mathcal{V}}}
\def\gW{{\mathcal{W}}}
\def\gX{{\mathcal{X}}}
\def\gY{{\mathcal{Y}}}
\def\gZ{{\mathcal{Z}}}

% Sets
\def\sA{{\mathbb{A}}}
\def\sB{{\mathbb{B}}}
\def\sC{{\mathbb{C}}}
\def\sD{{\mathbb{D}}}
% Don't use a set called E, because this would be the same as our symbol
% for expectation.
\def\sF{{\mathbb{F}}}
\def\sG{{\mathbb{G}}}
\def\sH{{\mathbb{H}}}
\def\sI{{\mathbb{I}}}
\def\sJ{{\mathbb{J}}}
\def\sK{{\mathbb{K}}}
\def\sL{{\mathbb{L}}}
\def\sM{{\mathbb{M}}}
\def\sN{{\mathbb{N}}}
\def\sO{{\mathbb{O}}}
\def\sP{{\mathbb{P}}}
\def\sQ{{\mathbb{Q}}}
\def\sR{{\mathbb{R}}}
\def\sS{{\mathbb{S}}}
\def\sT{{\mathbb{T}}}
\def\sU{{\mathbb{U}}}
\def\sV{{\mathbb{V}}}
\def\sW{{\mathbb{W}}}
\def\sX{{\mathbb{X}}}
\def\sY{{\mathbb{Y}}}
\def\sZ{{\mathbb{Z}}}

% Entries of a matrix
\def\emLambda{{\Lambda}}
\def\emA{{A}}
\def\emB{{B}}
\def\emC{{C}}
\def\emD{{D}}
\def\emE{{E}}
\def\emF{{F}}
\def\emG{{G}}
\def\emH{{H}}
\def\emI{{I}}
\def\emJ{{J}}
\def\emK{{K}}
\def\emL{{L}}
\def\emM{{M}}
\def\emN{{N}}
\def\emO{{O}}
\def\emP{{P}}
\def\emQ{{Q}}
\def\emR{{R}}
\def\emS{{S}}
\def\emT{{T}}
\def\emU{{U}}
\def\emV{{V}}
\def\emW{{W}}
\def\emX{{X}}
\def\emY{{Y}}
\def\emZ{{Z}}
\def\emSigma{{\Sigma}}

% entries of a tensor
% Same font as tensor, without \bm wrapper
\newcommand{\etens}[1]{\mathsfit{#1}}
\def\etLambda{{\etens{\Lambda}}}
\def\etA{{\etens{A}}}
\def\etB{{\etens{B}}}
\def\etC{{\etens{C}}}
\def\etD{{\etens{D}}}
\def\etE{{\etens{E}}}
\def\etF{{\etens{F}}}
\def\etG{{\etens{G}}}
\def\etH{{\etens{H}}}
\def\etI{{\etens{I}}}
\def\etJ{{\etens{J}}}
\def\etK{{\etens{K}}}
\def\etL{{\etens{L}}}
\def\etM{{\etens{M}}}
\def\etN{{\etens{N}}}
\def\etO{{\etens{O}}}
\def\etP{{\etens{P}}}
\def\etQ{{\etens{Q}}}
\def\etR{{\etens{R}}}
\def\etS{{\etens{S}}}
\def\etT{{\etens{T}}}
\def\etU{{\etens{U}}}
\def\etV{{\etens{V}}}
\def\etW{{\etens{W}}}
\def\etX{{\etens{X}}}
\def\etY{{\etens{Y}}}
\def\etZ{{\etens{Z}}}

% The true underlying data generating distribution
\newcommand{\pdata}{p_{\rm{data}}}
\newcommand{\ptarget}{p_{\rm{target}}}
\newcommand{\pprior}{p_{\rm{prior}}}
\newcommand{\pbase}{p_{\rm{base}}}
\newcommand{\pref}{p_{\rm{ref}}}

% The empirical distribution defined by the training set
\newcommand{\ptrain}{\hat{p}_{\rm{data}}}
\newcommand{\Ptrain}{\hat{P}_{\rm{data}}}
% The model distribution
\newcommand{\pmodel}{p_{\rm{model}}}
\newcommand{\Pmodel}{P_{\rm{model}}}
\newcommand{\ptildemodel}{\tilde{p}_{\rm{model}}}
% Stochastic autoencoder distributions
\newcommand{\pencode}{p_{\rm{encoder}}}
\newcommand{\pdecode}{p_{\rm{decoder}}}
\newcommand{\precons}{p_{\rm{reconstruct}}}

\newcommand{\laplace}{\mathrm{Laplace}} % Laplace distribution

\newcommand{\E}{\mathbb{E}}
\newcommand{\Ls}{\mathcal{L}}
\newcommand{\R}{\mathbb{R}}
\newcommand{\emp}{\tilde{p}}
\newcommand{\lr}{\alpha}
\newcommand{\reg}{\lambda}
\newcommand{\rect}{\mathrm{rectifier}}
\newcommand{\softmax}{\mathrm{softmax}}
\newcommand{\sigmoid}{\sigma}
\newcommand{\softplus}{\zeta}
\newcommand{\KL}{D_{\mathrm{KL}}}
\newcommand{\Var}{\mathrm{Var}}
\newcommand{\standarderror}{\mathrm{SE}}
\newcommand{\Cov}{\mathrm{Cov}}
% Wolfram Mathworld says $L^2$ is for function spaces and $\ell^2$ is for vectors
% But then they seem to use $L^2$ for vectors throughout the site, and so does
% wikipedia.
\newcommand{\normlzero}{L^0}
\newcommand{\normlone}{L^1}
\newcommand{\normltwo}{L^2}
\newcommand{\normlp}{L^p}
\newcommand{\normmax}{L^\infty}

\newcommand{\parents}{Pa} % See usage in notation.tex. Chosen to match Daphne's book.

\DeclareMathOperator*{\argmax}{arg\,max}
\DeclareMathOperator*{\argmin}{arg\,min}

\DeclareMathOperator{\sign}{sign}
\DeclareMathOperator{\Tr}{Tr}
\let\ab\allowbreak


\usepackage{microtype}
\usepackage{graphicx}
\usepackage{xcolor}
\usepackage{booktabs} % for professional tables
\usepackage{multicol}
\usepackage{multirow}
\usepackage{enumitem}
\usepackage{subcaption}
\usepackage{hyperref}
\usepackage{wrapfig}

% For theorems and such
\usepackage{amsmath}
\usepackage{amssymb}
\usepackage{mathtools}
\usepackage{amsthm}

% if you use cleveref..
% \usepackage[capitalize,noabbrev]{cleveref}
\usepackage[textsize=tiny]{todonotes}

\newcommand{\sstd}[1]{\textcolor{black}{\tiny{$\pm #1$}}}
\newcommand{\std}[1]{}
\newcommand{\highlight}[1]{\colorbox{blue!10}{#1}}
\newcommand{\theHalgorithm}{\arabic{algorithm}}

% Recommended, but optional, packages for figures and better typesetting:
\usepackage{microtype}
\usepackage{graphicx}
% \usepackage{subfigure}
\usepackage{booktabs} % for professional tables

% hyperref makes hyperlinks in the resulting PDF.
% If your build breaks (sometimes temporarily if a hyperlink spans a page)
% please comment out the following usepackage line and replace
% \usepackage{icml2025} with \usepackage[nohyperref]{icml2025} above.
\usepackage{hyperref}

% Use the following line for the initial blind version submitted for review:
\usepackage[preprint]{icml2025}

% If accepted, instead use the following line for the camera-ready submission:
% \usepackage[accepted]{icml2025}

% For theorems and such
\usepackage{amsmath}
\usepackage{amssymb}
\usepackage{mathtools}
\usepackage{amsthm}

% if you use cleveref..
\usepackage[nameinlink,capitalize,noabbrev]{cleveref}

\newcommand{\ie}{\textit{i.e.}}
\newcommand{\eg}{\textit{e.g.}}

%%%%%%%%%%%%%%%%%%%%%%%%%%%%%%%%
% THEOREMS
%%%%%%%%%%%%%%%%%%%%%%%%%%%%%%%%
\theoremstyle{plain}
\newtheorem{theorem}{Theorem}[section]
\newtheorem{proposition}[theorem]{Proposition}
\newtheorem{lemma}[theorem]{Lemma}
\newtheorem{corollary}[theorem]{Corollary}
\theoremstyle{definition}
\newtheorem{definition}[theorem]{Definition}
\newtheorem{assumption}[theorem]{Assumption}
\theoremstyle{remark}
\newtheorem{remark}[theorem]{Remark}

% Todonotes is useful during development; simply uncomment the next line
%    and comment out the line below the next line to turn off comments
%\usepackage[disable,textsize=tiny]{todonotes}
\usepackage[textsize=tiny]{todonotes}
\expandafter\def\expandafter\normalsize\expandafter{%
    % \setlength\abovedisplayskip{3pt}%
    % \setlength\belowdisplayskip{3pt}%
    % \setlength\abovedisplayshortskip{-4pt}%
    % \setlength\belowdisplayshortskip{2pt}%
}

% The \icmltitle you define below is probably too long as a header.
% Therefore, a short form for the running title is supplied here:
\newcommand{\norm}[1]{\left\lVert#1\right\rVert}

\icmltitlerunning{Amortized In-Context Bayesian Posterior Estimation}

\begin{document}

\twocolumn[
\icmltitle{Amortized In-Context Bayesian Posterior Estimation}

% It is OKAY to include author information, even for blind
% submissions: the style file will automatically remove it for you
% unless you've provided the [accepted] option to the icml2025
% package.

% List of affiliations: The first argument should be a (short)
% identifier you will use later to specify author affiliations
% Academic affiliations should list Department, University, City, Region, Country
% Industry affiliations should list Company, City, Region, Country

% You can specify symbols, otherwise they are numbered in order.
% Ideally, you should not use this facility. Affiliations will be numbered
% in order of appearance and this is the preferred way.
\icmlsetsymbol{equal}{*}

\begin{icmlauthorlist}
\icmlauthor{Sarthak Mittal}{udem,mila}
\icmlauthor{Niels Leif Bracher}{rpi}
\icmlauthor{Guillaume Lajoie}{udem,mila}
\icmlauthor{Priyank Jaini}{gdm}
\icmlauthor{Marcus Brubaker}{gdm,york,vector}
\end{icmlauthorlist}

\icmlaffiliation{udem}{Universit\'e de Montreal}
\icmlaffiliation{mila}{Mila}
\icmlaffiliation{york}{York University}
\icmlaffiliation{gdm}{Google DeepMind}
\icmlaffiliation{vector}{Vector Institute}
\icmlaffiliation{rpi}{Rensselaer Polytechnic Institute}

\icmlcorrespondingauthor{Sarthak Mittal}{sarthmit@gmail.com}

% You may provide any keywords that you
% find helpful for describing your paper; these are used to populate
% the "keywords" metadata in the PDF but will not be shown in the document
\icmlkeywords{Machine Learning, ICML}

\vskip 0.3in
]

% this must go after the closing bracket ] following \twocolumn[ ...

% This command actually creates the footnote in the first column
% listing the affiliations and the copyright notice.
% The command takes one argument, which is text to display at the start of the footnote.
% The \icmlEqualContribution command is standard text for equal contribution.
% Remove it (just {}) if you do not need this facility.

%\printAffiliationsAndNotice{}  % leave blank if no need to mention equal contribution
% \printAffiliationsAndNotice{\icmlEqualContribution} 
\printAffiliationsAndNotice{}% otherwise use the standard text.

\begin{abstract}
Bayesian inference provides a natural way of incorporating prior beliefs and assigning a probability measure to the space of hypotheses. Current solutions rely on iterative routines like Markov Chain Monte Carlo (MCMC) sampling and Variational Inference (VI), which need to be re-run whenever new observations are available. Amortization, through conditional estimation, is a viable strategy to alleviate such difficulties and has been the guiding principle behind simulation-based inference, neural processes and in-context methods using pre-trained models. In this work, we conduct a thorough comparative analysis of amortized in-context Bayesian posterior estimation methods from the lens of different optimization objectives and architectural choices. Such methods train an amortized estimator to perform posterior parameter inference by conditioning on a set of data examples passed as context to a sequence model such as a transformer. In contrast to language models, we leverage permutation invariant architectures as the true posterior is invariant to the ordering of context examples. Our empirical study includes generalization to out-of-distribution tasks, cases where the assumed underlying model is misspecified, and transfer from simulated to real problems. Subsequently, it highlights the superiority of the reverse KL estimator for predictive problems, especially when combined with the transformer architecture and normalizing flows.
\end{abstract}

\section{Introduction}


\begin{figure}[t]
\centering
\includegraphics[width=0.6\columnwidth]{figures/evaluation_desiderata_V5.pdf}
\vspace{-0.5cm}
\caption{\systemName is a platform for conducting realistic evaluations of code LLMs, collecting human preferences of coding models with real users, real tasks, and in realistic environments, aimed at addressing the limitations of existing evaluations.
}
\label{fig:motivation}
\end{figure}

\begin{figure*}[t]
\centering
\includegraphics[width=\textwidth]{figures/system_design_v2.png}
\caption{We introduce \systemName, a VSCode extension to collect human preferences of code directly in a developer's IDE. \systemName enables developers to use code completions from various models. The system comprises a) the interface in the user's IDE which presents paired completions to users (left), b) a sampling strategy that picks model pairs to reduce latency (right, top), and c) a prompting scheme that allows diverse LLMs to perform code completions with high fidelity.
Users can select between the top completion (green box) using \texttt{tab} or the bottom completion (blue box) using \texttt{shift+tab}.}
\label{fig:overview}
\end{figure*}

As model capabilities improve, large language models (LLMs) are increasingly integrated into user environments and workflows.
For example, software developers code with AI in integrated developer environments (IDEs)~\citep{peng2023impact}, doctors rely on notes generated through ambient listening~\citep{oberst2024science}, and lawyers consider case evidence identified by electronic discovery systems~\citep{yang2024beyond}.
Increasing deployment of models in productivity tools demands evaluation that more closely reflects real-world circumstances~\citep{hutchinson2022evaluation, saxon2024benchmarks, kapoor2024ai}.
While newer benchmarks and live platforms incorporate human feedback to capture real-world usage, they almost exclusively focus on evaluating LLMs in chat conversations~\citep{zheng2023judging,dubois2023alpacafarm,chiang2024chatbot, kirk2024the}.
Model evaluation must move beyond chat-based interactions and into specialized user environments.



 

In this work, we focus on evaluating LLM-based coding assistants. 
Despite the popularity of these tools---millions of developers use Github Copilot~\citep{Copilot}---existing
evaluations of the coding capabilities of new models exhibit multiple limitations (Figure~\ref{fig:motivation}, bottom).
Traditional ML benchmarks evaluate LLM capabilities by measuring how well a model can complete static, interview-style coding tasks~\citep{chen2021evaluating,austin2021program,jain2024livecodebench, white2024livebench} and lack \emph{real users}. 
User studies recruit real users to evaluate the effectiveness of LLMs as coding assistants, but are often limited to simple programming tasks as opposed to \emph{real tasks}~\citep{vaithilingam2022expectation,ross2023programmer, mozannar2024realhumaneval}.
Recent efforts to collect human feedback such as Chatbot Arena~\citep{chiang2024chatbot} are still removed from a \emph{realistic environment}, resulting in users and data that deviate from typical software development processes.
We introduce \systemName to address these limitations (Figure~\ref{fig:motivation}, top), and we describe our three main contributions below.


\textbf{We deploy \systemName in-the-wild to collect human preferences on code.} 
\systemName is a Visual Studio Code extension, collecting preferences directly in a developer's IDE within their actual workflow (Figure~\ref{fig:overview}).
\systemName provides developers with code completions, akin to the type of support provided by Github Copilot~\citep{Copilot}. 
Over the past 3 months, \systemName has served over~\completions suggestions from 10 state-of-the-art LLMs, 
gathering \sampleCount~votes from \userCount~users.
To collect user preferences,
\systemName presents a novel interface that shows users paired code completions from two different LLMs, which are determined based on a sampling strategy that aims to 
mitigate latency while preserving coverage across model comparisons.
Additionally, we devise a prompting scheme that allows a diverse set of models to perform code completions with high fidelity.
See Section~\ref{sec:system} and Section~\ref{sec:deployment} for details about system design and deployment respectively.



\textbf{We construct a leaderboard of user preferences and find notable differences from existing static benchmarks and human preference leaderboards.}
In general, we observe that smaller models seem to overperform in static benchmarks compared to our leaderboard, while performance among larger models is mixed (Section~\ref{sec:leaderboard_calculation}).
We attribute these differences to the fact that \systemName is exposed to users and tasks that differ drastically from code evaluations in the past. 
Our data spans 103 programming languages and 24 natural languages as well as a variety of real-world applications and code structures, while static benchmarks tend to focus on a specific programming and natural language and task (e.g. coding competition problems).
Additionally, while all of \systemName interactions contain code contexts and the majority involve infilling tasks, a much smaller fraction of Chatbot Arena's coding tasks contain code context, with infilling tasks appearing even more rarely. 
We analyze our data in depth in Section~\ref{subsec:comparison}.



\textbf{We derive new insights into user preferences of code by analyzing \systemName's diverse and distinct data distribution.}
We compare user preferences across different stratifications of input data (e.g., common versus rare languages) and observe which affect observed preferences most (Section~\ref{sec:analysis}).
For example, while user preferences stay relatively consistent across various programming languages, they differ drastically between different task categories (e.g. frontend/backend versus algorithm design).
We also observe variations in user preference due to different features related to code structure 
(e.g., context length and completion patterns).
We open-source \systemName and release a curated subset of code contexts.
Altogether, our results highlight the necessity of model evaluation in realistic and domain-specific settings.





% !TEX root =  ../main.tex
\section{Background on causality and abstraction}\label{sec:preliminaries}

This section provides the notation and key concepts related to causal modeling and abstraction theory.

\spara{Notation.} The set of integers from $1$ to $n$ is $[n]$.
The vectors of zeros and ones of size $n$ are $\zeros_n$ and $\ones_n$.
The identity matrix of size $n \times n$ is $\identity_n$. The Frobenius norm is $\frob{\mathbf{A}}$.
The set of positive definite matrices over $\reall^{n\times n}$ is $\pd^n$. The Hadamard product is $\odot$.
Function composition is $\circ$.
The domain of a function is $\dom{\cdot}$ and its kernel $\ker$.
Let $\mathcal{M}(\mathcal{X}^n)$ be the set of Borel measures over $\mathcal{X}^n \subseteq \reall^n$. Given a measure $\mu^n \in \mathcal{M}(\mathcal{X}^n)$ and a measurable map $\varphi^{\V}$, $\mathcal{X}^n \ni \mathbf{x} \overset{\varphi^{\V}}{\longmapsto} \V^\top \mathbf{x} \in \mathcal{X}^m$, we denote by $\varphi^{\V}_{\#}(\mu^n) \coloneqq \mu^n(\varphi^{\V^{-1}}(\mathbf{x}))$ the pushforward measure $\mu^m \in \mathcal{M}(\mathcal{X}^m)$. 


We now present the standard definition of SCM.

\begin{definition}[SCM, \citealp{pearl2009causality}]\label{def:SCM}
A (Markovian) structural causal model (SCM) $\scm^n$ is a tuple $\langle \myendogenous, \myexogenous, \myfunctional, \zeta^\myexogenous \rangle$, where \emph{(i)} $\myendogenous = \{X_1, \ldots, X_n\}$ is a set of $n$ endogenous random variables; \emph{(ii)} $\myexogenous =\{Z_1,\ldots,Z_n\}$ is a set of $n$ exogenous variables; \emph{(iii)} $\myfunctional$ is a set of $n$ functional assignments such that $X_i=f_i(\parents_i, Z_i)$, $\forall \; i \in [n]$, with $ \parents_i \subseteq \myendogenous \setminus \{ X_i\}$; \emph{(iv)} $\zeta^\myexogenous$ is a product probability measure over independent exogenous variables $\zeta^\myexogenous=\prod_{i \in [n]} \zeta^i$, where $\zeta^i=P(Z_i)$. 
\end{definition}
A Markovian SCM induces a directed acyclic graph (DAG) $\mathcal{G}_{\scm^n}$ where the nodes represent the variables $\myendogenous$ and the edges are determined by the structural functions $\myfunctional$; $ \parents_i$ constitutes then the parent set for $X_i$. Furthermore, we can recursively rewrite the set of structural function $\myfunctional$ as a set of mixing functions $\mymixing$ dependent only on the exogenous variables (cf. \cref{app:CA}). A key feature for studying causality is the possibility of defining interventions on the model:
\begin{definition}[Hard intervention, \citealp{pearl2009causality}]\label{def:intervention}
Given SCM $\scm^n = \langle \myendogenous, \myexogenous, \myfunctional, \zeta^\myexogenous \rangle$, a (hard) intervention $\iota = \operatorname{do}(\myendogenous^{\iota} = \mathbf{x}^{\iota})$, $\myendogenous^{\iota}\subseteq \myendogenous$,
is an operator that generates a new post-intervention SCM $\scm^n_\iota = \langle \myendogenous, \myexogenous, \myfunctional_\iota, \zeta^\myexogenous \rangle$ by replacing each function $f_i$ for $X_i\in\myendogenous^{\iota}$ with the constant $x_i^\iota\in \mathbf{x}^\iota$. 
Graphically, an intervention mutilates $\mathcal{G}_{\mathsf{M}^n}$ by removing all the incoming edges of the variables in $\myendogenous^{\iota}$.
\end{definition}

Given multiple SCMs describing the same system at different levels of granularity, CA provides the definition of an $\alpha$-abstraction map to relate these SCMs:
\begin{definition}[$\abst$-abstraction, \citealp{rischel2020category}]\label{def:abstraction}
Given low-level $\mathsf{M}^\ell$ and high-level $\mathsf{M}^h$ SCMs, an $\abst$-abstraction is a triple $\abst = \langle \Rset, \amap, \alphamap{} \rangle$, where \emph{(i)} $\Rset \subseteq \datalow$ is a subset of relevant variables in $\mathsf{M}^\ell$; \emph{(ii)} $\amap: \Rset \rightarrow \datahigh$ is a surjective function between the relevant variables of $\mathsf{M}^\ell$ and the endogenous variables of $\mathsf{M}^h$; \emph{(iii)} $\alphamap{}: \dom{\Rset} \rightarrow \dom{\datahigh}$ is a modular function $\alphamap{} = \bigotimes_{i\in[n]} \alphamap{X^h_i}$ made up by surjective functions $\alphamap{X^h_i}: \dom{\amap^{-1}(X^h_i)} \rightarrow \dom{X^h_i}$ from the outcome of low-level variables $\amap^{-1}(X^h_i) \in \datalow$ onto outcomes of the high-level variables $X^h_i \in \datahigh$.
\end{definition}
Notice that an $\abst$-abstraction simultaneously maps variables via the function $\amap$ and values through the function $\alphamap{}$. The definition itself does not place any constraint on these functions, although a common requirement in the literature is for the abstraction to satisfy \emph{interventional consistency} \cite{rubenstein2017causal,rischel2020category,beckers2019abstracting}. An important class of such well-behaved abstractions is \emph{constructive linear abstraction}, for which the following properties hold. By constructivity, \emph{(i)} $\abst$ is interventionally consistent; \emph{(ii)} all low-level variables are relevant $\Rset=\datalow$; \emph{(iii)} in addition to the map $\alphamap{}$ between endogenous variables, there exists a map ${\alphamap{}}_U$ between exogenous variables satisfying interventional consistency \cite{beckers2019abstracting,schooltink2024aligning}. By linearity, $\alphamap{} = \V^\top \in \reall^{h \times \ell}$ \cite{massidda2024learningcausalabstractionslinear}. \cref{app:CA} provides formal definitions for interventional consistency, linear and constructive abstraction.
\section{Method}\label{sec:method}
\begin{figure}
    \centering
    \includegraphics[width=0.85\textwidth]{imgs/heatmap_acc.pdf}
    \caption{\textbf{Visualization of the proposed periodic Bayesian flow with mean parameter $\mu$ and accumulated accuracy parameter $c$ which corresponds to the entropy/uncertainty}. For $x = 0.3, \beta(1) = 1000$ and $\alpha_i$ defined in \cref{appd:bfn_cir}, this figure plots three colored stochastic parameter trajectories for receiver mean parameter $m$ and accumulated accuracy parameter $c$, superimposed on a log-scale heatmap of the Bayesian flow distribution $p_F(m|x,\senderacc)$ and $p_F(c|x,\senderacc)$. Note the \emph{non-monotonicity} and \emph{non-additive} property of $c$ which could inform the network the entropy of the mean parameter $m$ as a condition and the \emph{periodicity} of $m$. %\jj{Shrink the figures to save space}\hanlin{Do we need to make this figure one-column?}
    }
    \label{fig:vmbf_vis}
    \vskip -0.1in
\end{figure}
% \begin{wrapfigure}{r}{0.5\textwidth}
%     \centering
%     \includegraphics[width=0.49\textwidth]{imgs/heatmap_acc.pdf}
%     \caption{\textbf{Visualization of hyper-torus Bayesian flow based on von Mises Distribution}. For $x = 0.3, \beta(1) = 1000$ and $\alpha_i$ defined in \cref{appd:bfn_cir}, this figure plots three colored stochastic parameter trajectories for receiver mean parameter $m$ and accumulated accuracy parameter $c$, superimposed on a log-scale heatmap of the Bayesian flow distribution $p_F(m|x,\senderacc)$ and $p_F(c|x,\senderacc)$. Note the \emph{non-monotonicity} and \emph{non-additive} property of $c$. \jj{Shrink the figures to save space}}
%     \label{fig:vmbf_vis}
%     \vspace{-30pt}
% \end{wrapfigure}


In this section, we explain the detailed design of CrysBFN tackling theoretical and practical challenges. First, we describe how to derive our new formulation of Bayesian Flow Networks over hyper-torus $\mathbb{T}^{D}$ from scratch. Next, we illustrate the two key differences between \modelname and the original form of BFN: $1)$ a meticulously designed novel base distribution with different Bayesian update rules; and $2)$ different properties over the accuracy scheduling resulted from the periodicity and the new Bayesian update rules. Then, we present in detail the overall framework of \modelname over each manifold of the crystal space (\textit{i.e.} fractional coordinates, lattice vectors, atom types) respecting \textit{periodic E(3) invariance}. 

% In this section, we first demonstrate how to build Bayesian flow on hyper-torus $\mathbb{T}^{D}$ by overcoming theoretical and practical problems to provide a low-noise parameter-space approach to fractional atom coordinate generation. Next, we present how \modelname models each manifold of crystal space respecting \textit{periodic E(3) invariance}. 

\subsection{Periodic Bayesian Flow on Hyper-torus \texorpdfstring{$\mathbb{T}^{D}$}{}} 
For generative modeling of fractional coordinates in crystal, we first construct a periodic Bayesian flow on \texorpdfstring{$\mathbb{T}^{D}$}{} by designing every component of the totally new Bayesian update process which we demonstrate to be distinct from the original Bayesian flow (please see \cref{fig:non_add}). 
 %:) 
 
 The fractional atom coordinate system \citep{jiao2023crystal} inherently distributes over a hyper-torus support $\mathbb{T}^{3\times N}$. Hence, the normal distribution support on $\R$ used in the original \citep{bfn} is not suitable for this scenario. 
% The key problem of generative modeling for crystal is the periodicity of Cartesian atom coordinates $\vX$ requiring:
% \begin{equation}\label{eq:periodcity}
% p(\vA,\vL,\vX)=p(\vA,\vL,\vX+\vec{LK}),\text{where}~\vec{K}=\vec{k}\vec{1}_{1\times N},\forall\vec{k}\in\mathbb{Z}^{3\times1}
% \end{equation}
% However, there does not exist such a distribution supporting on $\R$ to model such property because the integration of such distribution over $\R$ will not be finite and equal to 1. Therefore, the normal distribution used in \citet{bfn} can not meet this condition.

To tackle this problem, the circular distribution~\citep{mardia2009directional} over the finite interval $[-\pi,\pi)$ is a natural choice as the base distribution for deriving the BFN on $\mathbb{T}^D$. 
% one natural choice is to 
% we would like to consider the circular distribution over the finite interval as the base 
% we find that circular distributions \citep{mardia2009directional} defined on a finite interval with lengths of $2\pi$ can be used as the instantiation of input distribution for the BFN on $\mathbb{T}^D$.
Specifically, circular distributions enjoy desirable periodic properties: $1)$ the integration over any interval length of $2\pi$ equals 1; $2)$ the probability distribution function is periodic with period $2\pi$.  Sharing the same intrinsic with fractional coordinates, such periodic property of circular distribution makes it suitable for the instantiation of BFN's input distribution, in parameterizing the belief towards ground truth $\x$ on $\mathbb{T}^D$. 
% \yuxuan{this is very complicated from my perspective.} \hanlin{But this property is exactly beautiful and perfectly fit into the BFN.}

\textbf{von Mises Distribution and its Bayesian Update} We choose von Mises distribution \citep{mardia2009directional} from various circular distributions as the form of input distribution, based on the appealing conjugacy property required in the derivation of the BFN framework.
% to leverage the Bayesian conjugacy property of von Mises distribution which is required by the BFN framework. 
That is, the posterior of a von Mises distribution parameterized likelihood is still in the family of von Mises distributions. The probability density function of von Mises distribution with mean direction parameter $m$ and concentration parameter $c$ (describing the entropy/uncertainty of $m$) is defined as: 
\begin{equation}
f(x|m,c)=vM(x|m,c)=\frac{\exp(c\cos(x-m))}{2\pi I_0(c)}
\end{equation}
where $I_0(c)$ is zeroth order modified Bessel function of the first kind as the normalizing constant. Given the last univariate belief parameterized by von Mises distribution with parameter $\theta_{i-1}=\{m_{i-1},\ c_{i-1}\}$ and the sample $y$ from sender distribution with unknown data sample $x$ and known accuracy $\alpha$ describing the entropy/uncertainty of $y$,  Bayesian update for the receiver is deducted as:
\begin{equation}
 h(\{m_{i-1},c_{i-1}\},y,\alpha)=\{m_i,c_i \}, \text{where}
\end{equation}
\begin{equation}\label{eq:h_m}
m_i=\text{atan2}(\alpha\sin y+c_{i-1}\sin m_{i-1}, {\alpha\cos y+c_{i-1}\cos m_{i-1}})
\end{equation}
\begin{equation}\label{eq:h_c}
c_i =\sqrt{\alpha^2+c_{i-1}^2+2\alpha c_{i-1}\cos(y-m_{i-1})}
\end{equation}
The proof of the above equations can be found in \cref{apdx:bayesian_update_function}. The atan2 function refers to  2-argument arctangent. Independently conducting  Bayesian update for each dimension, we can obtain the Bayesian update distribution by marginalizing $\y$:
\begin{equation}
p_U(\vtheta'|\vtheta,\bold{x};\alpha)=\mathbb{E}_{p_S(\bold{y}|\bold{x};\alpha)}\delta(\vtheta'-h(\vtheta,\bold{y},\alpha))=\mathbb{E}_{vM(\bold{y}|\bold{x},\alpha)}\delta(\vtheta'-h(\vtheta,\bold{y},\alpha))
\end{equation} 
\begin{figure}
    \centering
    \vskip -0.15in
    \includegraphics[width=0.95\linewidth]{imgs/non_add.pdf}
    \caption{An intuitive illustration of non-additive accuracy Bayesian update on the torus. The lengths of arrows represent the uncertainty/entropy of the belief (\emph{e.g.}~$1/\sigma^2$ for Gaussian and $c$ for von Mises). The directions of the arrows represent the believed location (\emph{e.g.}~ $\mu$ for Gaussian and $m$ for von Mises).}
    \label{fig:non_add}
    \vskip -0.15in
\end{figure}
\textbf{Non-additive Accuracy} 
The additive accuracy is a nice property held with the Gaussian-formed sender distribution of the original BFN expressed as:
\begin{align}
\label{eq:standard_id}
    \update(\parsn{}'' \mid \parsn{}, \x; \alpha_a+\alpha_b) = \E_{\update(\parsn{}' \mid \parsn{}, \x; \alpha_a)} \update(\parsn{}'' \mid \parsn{}', \x; \alpha_b)
\end{align}
Such property is mainly derived based on the standard identity of Gaussian variable:
\begin{equation}
X \sim \mathcal{N}\left(\mu_X, \sigma_X^2\right), Y \sim \mathcal{N}\left(\mu_Y, \sigma_Y^2\right) \Longrightarrow X+Y \sim \mathcal{N}\left(\mu_X+\mu_Y, \sigma_X^2+\sigma_Y^2\right)
\end{equation}
The additive accuracy property makes it feasible to derive the Bayesian flow distribution $
p_F(\boldsymbol{\theta} \mid \mathbf{x} ; i)=p_U\left(\boldsymbol{\theta} \mid \boldsymbol{\theta}_0, \mathbf{x}, \sum_{k=1}^{i} \alpha_i \right)
$ for the simulation-free training of \cref{eq:loss_n}.
It should be noted that the standard identity in \cref{eq:standard_id} does not hold in the von Mises distribution. Hence there exists an important difference between the original Bayesian flow defined on Euclidean space and the Bayesian flow of circular data on $\mathbb{T}^D$ based on von Mises distribution. With prior $\btheta = \{\bold{0},\bold{0}\}$, we could formally represent the non-additive accuracy issue as:
% The additive accuracy property implies the fact that the "confidence" for the data sample after observing a series of the noisy samples with accuracy ${\alpha_1, \cdots, \alpha_i}$ could be  as the accuracy sum  which could be  
% Here we 
% Here we emphasize the specific property of BFN based on von Mises distribution.
% Note that 
% \begin{equation}
% \update(\parsn'' \mid \parsn, \x; \alpha_a+\alpha_b) \ne \E_{\update(\parsn' \mid \parsn, \x; \alpha_a)} \update(\parsn'' \mid \parsn', \x; \alpha_b)
% \end{equation}
% \oyyw{please check whether the below equation is better}
% \yuxuan{I fill somehow confusing on what is the update distribution with $\alpha$. }
% \begin{equation}
% \update(\parsn{}'' \mid \parsn{}, \x; \alpha_a+\alpha_b) \ne \E_{\update(\parsn{}' \mid \parsn{}, \x; \alpha_a)} \update(\parsn{}'' \mid \parsn{}', \x; \alpha_b)
% \end{equation}
% We give an intuitive visualization of such difference in \cref{fig:non_add}. The untenability of this property can materialize by considering the following case: with prior $\btheta = \{\bold{0},\bold{0}\}$, check the two-step Bayesian update distribution with $\alpha_a,\alpha_b$ and one-step Bayesian update with $\alpha=\alpha_a+\alpha_b$:
\begin{align}
\label{eq:nonadd}
     &\update(c'' \mid \parsn, \x; \alpha_a+\alpha_b)  = \delta(c-\alpha_a-\alpha_b)
     \ne  \mathbb{E}_{p_U(\parsn' \mid \parsn, \x; \alpha_a)}\update(c'' \mid \parsn', \x; \alpha_b) \nonumber \\&= \mathbb{E}_{vM(\bold{y}_b|\bold{x},\alpha_a)}\mathbb{E}_{vM(\bold{y}_a|\bold{x},\alpha_b)}\delta(c-||[\alpha_a \cos\y_a+\alpha_b\cos \y_b,\alpha_a \sin\y_a+\alpha_b\sin \y_b]^T||_2)
\end{align}
A more intuitive visualization could be found in \cref{fig:non_add}. This fundamental difference between periodic Bayesian flow and that of \citet{bfn} presents both theoretical and practical challenges, which we will explain and address in the following contents.

% This makes constructing Bayesian flow based on von Mises distribution intrinsically different from previous Bayesian flows (\citet{bfn}).

% Thus, we must reformulate the framework of Bayesian flow networks  accordingly. % and do necessary reformulations of BFN. 

% \yuxuan{overall I feel this part is complicated by using the language of update distribution. I would like to suggest simply use bayesian update, to provide intuitive explantion.}\hanlin{See the illustration in \cref{fig:non_add}}

% That introduces a cascade of problems, and we investigate the following issues: $(1)$ Accuracies between sender and receiver are not synchronized and need to be differentiated. $(2)$ There is no tractable Bayesian flow distribution for a one-step sample conditioned on a given time step $i$, and naively simulating the Bayesian flow results in computational overhead. $(3)$ It is difficult to control the entropy of the Bayesian flow. $(4)$ Accuracy is no longer a function of $t$ and becomes a distribution conditioned on $t$, which can be different across dimensions.
%\jj{Edited till here}

\textbf{Entropy Conditioning} As a common practice in generative models~\citep{ddpm,flowmatching,bfn}, timestep $t$ is widely used to distinguish among generation states by feeding the timestep information into the networks. However, this paper shows that for periodic Bayesian flow, the accumulated accuracy $\vc_i$ is more effective than time-based conditioning by informing the network about the entropy and certainty of the states $\parsnt{i}$. This stems from the intrinsic non-additive accuracy which makes the receiver's accumulated accuracy $c$ not bijective function of $t$, but a distribution conditioned on accumulated accuracies $\vc_i$ instead. Therefore, the entropy parameter $\vc$ is taken logarithm and fed into the network to describe the entropy of the input corrupted structure. We verify this consideration in \cref{sec:exp_ablation}. 
% \yuxuan{implement variant. traditionally, the timestep is widely used to distinguish the different states by putting the timestep embedding into the networks. citation of FM, diffusion, BFN. However, we find that conditioned on time in periodic flow could not provide extra benefits. To further boost the performance, we introduce a simple yet effective modification term entropy conditional. This is based on that the accumulated accuracy which represents the current uncertainty or entropy could be a better indicator to distinguish different states. + Describe how you do this. }



\textbf{Reformulations of BFN}. Recall the original update function with Gaussian sender distribution, after receiving noisy samples $\y_1,\y_2,\dots,\y_i$ with accuracies $\senderacc$, the accumulated accuracies of the receiver side could be analytically obtained by the additive property and it is consistent with the sender side.
% Since observing sample $\y$ with $\alpha_i$ can not result in exact accuracy increment $\alpha_i$ for receiver, the accuracies between sender and receiver are not synchronized which need to be differentiated. 
However, as previously mentioned, this does not apply to periodic Bayesian flow, and some of the notations in original BFN~\citep{bfn} need to be adjusted accordingly. We maintain the notations of sender side's one-step accuracy $\alpha$ and added accuracy $\beta$, and alter the notation of receiver's accuracy parameter as $c$, which is needed to be simulated by cascade of Bayesian updates. We emphasize that the receiver's accumulated accuracy $c$ is no longer a function of $t$ (differently from the Gaussian case), and it becomes a distribution conditioned on received accuracies $\senderacc$ from the sender. Therefore, we represent the Bayesian flow distribution of von Mises distribution as $p_F(\btheta|\x;\alpha_1,\alpha_2,\dots,\alpha_i)$. And the original simulation-free training with Bayesian flow distribution is no longer applicable in this scenario.
% Different from previous BFNs where the accumulated accuracy $\rho$ is not explicitly modeled, the accumulated accuracy parameter $c$ (visualized in \cref{fig:vmbf_vis}) needs to be explicitly modeled by feeding it to the network to avoid information loss.
% the randomaccuracy parameter $c$ (visualized in \cref{fig:vmbf_vis}) implies that there exists information in $c$ from the sender just like $m$, meaning that $c$ also should be fed into the network to avoid information loss. 
% We ablate this consideration in  \cref{sec:exp_ablation}. 

\textbf{Fast Sampling from Equivalent Bayesian Flow Distribution} Based on the above reformulations, the Bayesian flow distribution of von Mises distribution is reframed as: 
\begin{equation}\label{eq:flow_frac}
p_F(\btheta_i|\x;\alpha_1,\alpha_2,\dots,\alpha_i)=\E_{\update(\parsnt{1} \mid \parsnt{0}, \x ; \alphat{1})}\dots\E_{\update(\parsn_{i-1} \mid \parsnt{i-2}, \x; \alphat{i-1})} \update(\parsnt{i} | \parsnt{i-1},\x;\alphat{i} )
\end{equation}
Naively sampling from \cref{eq:flow_frac} requires slow auto-regressive iterated simulation, making training unaffordable. Noticing the mathematical properties of \cref{eq:h_m,eq:h_c}, we  transform \cref{eq:flow_frac} to the equivalent form:
\begin{equation}\label{eq:cirflow_equiv}
p_F(\vec{m}_i|\x;\alpha_1,\alpha_2,\dots,\alpha_i)=\E_{vM(\y_1|\x,\alpha_1)\dots vM(\y_i|\x,\alpha_i)} \delta(\vec{m}_i-\text{atan2}(\sum_{j=1}^i \alpha_j \cos \y_j,\sum_{j=1}^i \alpha_j \sin \y_j))
\end{equation}
\begin{equation}\label{eq:cirflow_equiv2}
p_F(\vec{c}_i|\x;\alpha_1,\alpha_2,\dots,\alpha_i)=\E_{vM(\y_1|\x,\alpha_1)\dots vM(\y_i|\x,\alpha_i)}  \delta(\vec{c}_i-||[\sum_{j=1}^i \alpha_j \cos \y_j,\sum_{j=1}^i \alpha_j \sin \y_j]^T||_2)
\end{equation}
which bypasses the computation of intermediate variables and allows pure tensor operations, with negligible computational overhead.
\begin{restatable}{proposition}{cirflowequiv}
The probability density function of Bayesian flow distribution defined by \cref{eq:cirflow_equiv,eq:cirflow_equiv2} is equivalent to the original definition in \cref{eq:flow_frac}. 
\end{restatable}
\textbf{Numerical Determination of Linear Entropy Sender Accuracy Schedule} ~Original BFN designs the accuracy schedule $\beta(t)$ to make the entropy of input distribution linearly decrease. As for crystal generation task, to ensure information coherence between modalities, we choose a sender accuracy schedule $\senderacc$ that makes the receiver's belief entropy $H(t_i)=H(p_I(\cdot|\vtheta_i))=H(p_I(\cdot|\vc_i))$ linearly decrease \emph{w.r.t.} time $t_i$, given the initial and final accuracy parameter $c(0)$ and $c(1)$. Due to the intractability of \cref{eq:vm_entropy}, we first use numerical binary search in $[0,c(1)]$ to determine the receiver's $c(t_i)$ for $i=1,\dots, n$ by solving the equation $H(c(t_i))=(1-t_i)H(c(0))+tH(c(1))$. Next, with $c(t_i)$, we conduct numerical binary search for each $\alpha_i$ in $[0,c(1)]$ by solving the equations $\E_{y\sim vM(x,\alpha_i)}[\sqrt{\alpha_i^2+c_{i-1}^2+2\alpha_i c_{i-1}\cos(y-m_{i-1})}]=c(t_i)$ from $i=1$ to $i=n$ for arbitrarily selected $x\in[-\pi,\pi)$.

After tackling all those issues, we have now arrived at a new BFN architecture for effectively modeling crystals. Such BFN can also be adapted to other type of data located in hyper-torus $\mathbb{T}^{D}$.

\subsection{Equivariant Bayesian Flow for Crystal}
With the above Bayesian flow designed for generative modeling of fractional coordinate $\vF$, we are able to build equivariant Bayesian flow for each modality of crystal. In this section, we first give an overview of the general training and sampling algorithm of \modelname (visualized in \cref{fig:framework}). Then, we describe the details of the Bayesian flow of every modality. The training and sampling algorithm can be found in \cref{alg:train} and \cref{alg:sampling}.

\textbf{Overview} Operating in the parameter space $\bthetaM=\{\bthetaA,\bthetaL,\bthetaF\}$, \modelname generates high-fidelity crystals through a joint BFN sampling process on the parameter of  atom type $\bthetaA$, lattice parameter $\vec{\theta}^L=\{\bmuL,\brhoL\}$, and the parameter of fractional coordinate matrix $\bthetaF=\{\bmF,\bcF\}$. We index the $n$-steps of the generation process in a discrete manner $i$, and denote the corresponding continuous notation $t_i=i/n$ from prior parameter $\thetaM_0$ to a considerably low variance parameter $\thetaM_n$ (\emph{i.e.} large $\vrho^L,\bmF$, and centered $\bthetaA$).

At training time, \modelname samples time $i\sim U\{1,n\}$ and $\bthetaM_{i-1}$ from the Bayesian flow distribution of each modality, serving as the input to the network. The network $\net$ outputs $\net(\parsnt{i-1}^\mathcal{M},t_{i-1})=\net(\parsnt{i-1}^A,\parsnt{i-1}^F,\parsnt{i-1}^L,t_{i-1})$ and conducts gradient descents on loss function \cref{eq:loss_n} for each modality. After proper training, the sender distribution $p_S$ can be approximated by the receiver distribution $p_R$. 

At inference time, from predefined $\thetaM_0$, we conduct transitions from $\thetaM_{i-1}$ to $\thetaM_{i}$ by: $(1)$ sampling $\y_i\sim p_R(\bold{y}|\thetaM_{i-1};t_i,\alpha_i)$ according to network prediction $\predM{i-1}$; and $(2)$ performing Bayesian update $h(\thetaM_{i-1},\y^\calM_{i-1},\alpha_i)$ for each dimension. 

% Alternatively, we complete this transition using the flow-back technique by sampling 
% $\thetaM_{i}$ from Bayesian flow distribution $\flow(\btheta^M_{i}|\predM{i-1};t_{i-1})$. 

% The training objective of $\net$ is to minimize the KL divergence between sender distribution and receiver distribution for every modality as defined in \cref{eq:loss_n} which is equivalent to optimizing the negative variational lower bound $\calL^{VLB}$ as discussed in \cref{sec:preliminaries}. 

%In the following part, we will present the Bayesian flow of each modality in detail.

\textbf{Bayesian Flow of Fractional Coordinate $\vF$}~The distribution of the prior parameter $\bthetaF_0$ is defined as:
\begin{equation}\label{eq:prior_frac}
    p(\bthetaF_0) \defeq \{vM(\vm_0^F|\vec{0}_{3\times N},\vec{0}_{3\times N}),\delta(\vc_0^F-\vec{0}_{3\times N})\} = \{U(\vec{0},\vec{1}),\delta(\vc_0^F-\vec{0}_{3\times N})\}
\end{equation}
Note that this prior distribution of $\vm_0^F$ is uniform over $[\vec{0},\vec{1})$, ensuring the periodic translation invariance property in \cref{De:pi}. The training objective is minimizing the KL divergence between sender and receiver distribution (deduction can be found in \cref{appd:cir_loss}): 
%\oyyw{replace $\vF$ with $\x$?} \hanlin{notations follow Preliminary?}
\begin{align}\label{loss_frac}
\calL_F = n \E_{i \sim \ui{n}, \flow(\parsn{}^F \mid \vF ; \senderacc)} \alpha_i\frac{I_1(\alpha_i)}{I_0(\alpha_i)}(1-\cos(\vF-\predF{i-1}))
\end{align}
where $I_0(x)$ and $I_1(x)$ are the zeroth and the first order of modified Bessel functions. The transition from $\bthetaF_{i-1}$ to $\bthetaF_{i}$ is the Bayesian update distribution based on network prediction:
\begin{equation}\label{eq:transi_frac}
    p(\btheta^F_{i}|\parsnt{i-1}^\calM)=\mathbb{E}_{vM(\bold{y}|\predF{i-1},\alpha_i)}\delta(\btheta^F_{i}-h(\btheta^F_{i-1},\bold{y},\alpha_i))
\end{equation}
\begin{restatable}{proposition}{fracinv}
With $\net_{F}$ as a periodic translation equivariant function namely $\net_F(\parsnt{}^A,w(\parsnt{}^F+\vt),\parsnt{}^L,t)=w(\net_F(\parsnt{}^A,\parsnt{}^F,\parsnt{}^L,t)+\vt), \forall\vt\in\R^3$, the marginal distribution of $p(\vF_n)$ defined by \cref{eq:prior_frac,eq:transi_frac} is periodic translation invariant. 
\end{restatable}
\textbf{Bayesian Flow of Lattice Parameter \texorpdfstring{$\boldsymbol{L}$}{}}   
Noting the lattice parameter $\bm{L}$ located in Euclidean space, we set prior as the parameter of a isotropic multivariate normal distribution $\btheta^L_0\defeq\{\vmu_0^L,\vrho_0^L\}=\{\bm{0}_{3\times3},\bm{1}_{3\times3}\}$
% \begin{equation}\label{eq:lattice_prior}
% \btheta^L_0\defeq\{\vmu_0^L,\vrho_0^L\}=\{\bm{0}_{3\times3},\bm{1}_{3\times3}\}
% \end{equation}
such that the prior distribution of the Markov process on $\vmu^L$ is the Dirac distribution $\delta(\vec{\mu_0}-\vec{0})$ and $\delta(\vec{\rho_0}-\vec{1})$, 
% \begin{equation}
%     p_I^L(\boldsymbol{L}|\btheta_0^L)=\mathcal{N}(\bm{L}|\bm{0},\bm{I})
% \end{equation}
which ensures O(3)-invariance of prior distribution of $\vL$. By Eq. 77 from \citet{bfn}, the Bayesian flow distribution of the lattice parameter $\bm{L}$ is: 
\begin{align}% =p_U(\bmuL|\btheta_0^L,\bm{L},\beta(t))
p_F^L(\bmuL|\bm{L};t) &=\mathcal{N}(\bmuL|\gamma(t)\bm{L},\gamma(t)(1-\gamma(t))\bm{I}) 
\end{align}
where $\gamma(t) = 1 - \sigma_1^{2t}$ and $\sigma_1$ is the predefined hyper-parameter controlling the variance of input distribution at $t=1$ under linear entropy accuracy schedule. The variance parameter $\vrho$ does not need to be modeled and fed to the network, since it is deterministic given the accuracy schedule. After sampling $\bmuL_i$ from $p_F^L$, the training objective is defined as minimizing KL divergence between sender and receiver distribution (based on Eq. 96 in \citet{bfn}):
\begin{align}
\mathcal{L}_{L} = \frac{n}{2}\left(1-\sigma_1^{2/n}\right)\E_{i \sim \ui{n}}\E_{\flow(\bmuL_{i-1} |\vL ; t_{i-1})}  \frac{\left\|\vL -\predL{i-1}\right\|^2}{\sigma_1^{2i/n}},\label{eq:lattice_loss}
\end{align}
where the prediction term $\predL{i-1}$ is the lattice parameter part of network output. After training, the generation process is defined as the Bayesian update distribution given network prediction:
\begin{equation}\label{eq:lattice_sampling}
    p(\bmuL_{i}|\parsnt{i-1}^\calM)=\update^L(\bmuL_{i}|\predL{i-1},\bmuL_{i-1};t_{i-1})
\end{equation}
    

% The final prediction of the lattice parameter is given by $\bmuL_n = \predL{n-1}$.
% \begin{equation}\label{eq:final_lattice}
%     \bmuL_n = \predL{n-1}
% \end{equation}

\begin{restatable}{proposition}{latticeinv}\label{prop:latticeinv}
With $\net_{L}$ as  O(3)-equivariant function namely $\net_L(\parsnt{}^A,\parsnt{}^F,\vQ\parsnt{}^L,t)=\vQ\net_L(\parsnt{}^A,\parsnt{}^F,\parsnt{}^L,t),\forall\vQ^T\vQ=\vI$, the marginal distribution of $p(\bmuL_n)$ defined by \cref{eq:lattice_sampling} is O(3)-invariant. 
\end{restatable}


\textbf{Bayesian Flow of Atom Types \texorpdfstring{$\boldsymbol{A}$}{}} 
Given that atom types are discrete random variables located in a simplex $\calS^K$, the prior parameter of $\boldsymbol{A}$ is the discrete uniform distribution over the vocabulary $\parsnt{0}^A \defeq \frac{1}{K}\vec{1}_{1\times N}$. 
% \begin{align}\label{eq:disc_input_prior}
% \parsnt{0}^A \defeq \frac{1}{K}\vec{1}_{1\times N}
% \end{align}
% \begin{align}
%     (\oh{j}{K})_k \defeq \delta_{j k}, \text{where }\oh{j}{K}\in \R^{K},\oh{\vA}{KD} \defeq \left(\oh{a_1}{K},\dots,\oh{a_N}{K}\right) \in \R^{K\times N}
% \end{align}
With the notation of the projection from the class index $j$ to the length $K$ one-hot vector $ (\oh{j}{K})_k \defeq \delta_{j k}, \text{where }\oh{j}{K}\in \R^{K},\oh{\vA}{KD} \defeq \left(\oh{a_1}{K},\dots,\oh{a_N}{K}\right) \in \R^{K\times N}$, the Bayesian flow distribution of atom types $\vA$ is derived in \citet{bfn}:
\begin{align}
\flow^{A}(\parsn^A \mid \vA; t) &= \E_{\N{\y \mid \beta^A(t)\left(K \oh{\vA}{K\times N} - \vec{1}_{K\times N}\right)}{\beta^A(t) K \vec{I}_{K\times N \times N}}} \delta\left(\parsn^A - \frac{e^{\y}\parsnt{0}^A}{\sum_{k=1}^K e^{\y_k}(\parsnt{0})_{k}^A}\right).
\end{align}
where $\beta^A(t)$ is the predefined accuracy schedule for atom types. Sampling $\btheta_i^A$ from $p_F^A$ as the training signal, the training objective is the $n$-step discrete-time loss for discrete variable \citep{bfn}: 
% \oyyw{can we simplify the next equation? Such as remove $K \times N, K \times N \times N$}
% \begin{align}
% &\calL_A = n\E_{i \sim U\{1,n\},\flow^A(\parsn^A \mid \vA ; t_{i-1}),\N{\y \mid \alphat{i}\left(K \oh{\vA}{KD} - \vec{1}_{K\times N}\right)}{\alphat{i} K \vec{I}_{K\times N \times N}}} \ln \N{\y \mid \alphat{i}\left(K \oh{\vA}{K\times N} - \vec{1}_{K\times N}\right)}{\alphat{i} K \vec{I}_{K\times N \times N}}\nonumber\\
% &\qquad\qquad\qquad-\sum_{d=1}^N \ln \left(\sum_{k=1}^K \out^{(d)}(k \mid \parsn^A; t_{i-1}) \N{\ydd{d} \mid \alphat{i}\left(K\oh{k}{K}- \vec{1}_{K\times N}\right)}{\alphat{i} K \vec{I}_{K\times N \times N}}\right)\label{discdisc_t_loss_exp}
% \end{align}
\begin{align}
&\calL_A = n\E_{i \sim U\{1,n\},\flow^A(\parsn^A \mid \vA ; t_{i-1}),\N{\y \mid \alphat{i}\left(K \oh{\vA}{KD} - \vec{1}\right)}{\alphat{i} K \vec{I}}} \ln \N{\y \mid \alphat{i}\left(K \oh{\vA}{K\times N} - \vec{1}\right)}{\alphat{i} K \vec{I}}\nonumber\\
&\qquad\qquad\qquad-\sum_{d=1}^N \ln \left(\sum_{k=1}^K \out^{(d)}(k \mid \parsn^A; t_{i-1}) \N{\ydd{d} \mid \alphat{i}\left(K\oh{k}{K}- \vec{1}\right)}{\alphat{i} K \vec{I}}\right)\label{discdisc_t_loss_exp}
\end{align}
where $\vec{I}\in \R^{K\times N \times N}$ and $\vec{1}\in\R^{K\times D}$. When sampling, the transition from $\bthetaA_{i-1}$ to $\bthetaA_{i}$ is derived as:
\begin{equation}
    p(\btheta^A_{i}|\parsnt{i-1}^\calM)=\update^A(\btheta^A_{i}|\btheta^A_{i-1},\predA{i-1};t_{i-1})
\end{equation}

The detailed training and sampling algorithm could be found in \cref{alg:train} and \cref{alg:sampling}.




\section{Experiments}
\label{sec:experiments}
The experiments are designed to address two key research questions.
First, \textbf{RQ1} evaluates whether the average $L_2$-norm of the counterfactual perturbation vectors ($\overline{||\perturb||}$) decreases as the model overfits the data, thereby providing further empirical validation for our hypothesis.
Second, \textbf{RQ2} evaluates the ability of the proposed counterfactual regularized loss, as defined in (\ref{eq:regularized_loss2}), to mitigate overfitting when compared to existing regularization techniques.

% The experiments are designed to address three key research questions. First, \textbf{RQ1} investigates whether the mean perturbation vector norm decreases as the model overfits the data, aiming to further validate our intuition. Second, \textbf{RQ2} explores whether the mean perturbation vector norm can be effectively leveraged as a regularization term during training, offering insights into its potential role in mitigating overfitting. Finally, \textbf{RQ3} examines whether our counterfactual regularizer enables the model to achieve superior performance compared to existing regularization methods, thus highlighting its practical advantage.

\subsection{Experimental Setup}
\textbf{\textit{Datasets, Models, and Tasks.}}
The experiments are conducted on three datasets: \textit{Water Potability}~\cite{kadiwal2020waterpotability}, \textit{Phomene}~\cite{phomene}, and \textit{CIFAR-10}~\cite{krizhevsky2009learning}. For \textit{Water Potability} and \textit{Phomene}, we randomly select $80\%$ of the samples for the training set, and the remaining $20\%$ for the test set, \textit{CIFAR-10} comes already split. Furthermore, we consider the following models: Logistic Regression, Multi-Layer Perceptron (MLP) with 100 and 30 neurons on each hidden layer, and PreactResNet-18~\cite{he2016cvecvv} as a Convolutional Neural Network (CNN) architecture.
We focus on binary classification tasks and leave the extension to multiclass scenarios for future work. However, for datasets that are inherently multiclass, we transform the problem into a binary classification task by selecting two classes, aligning with our assumption.

\smallskip
\noindent\textbf{\textit{Evaluation Measures.}} To characterize the degree of overfitting, we use the test loss, as it serves as a reliable indicator of the model's generalization capability to unseen data. Additionally, we evaluate the predictive performance of each model using the test accuracy.

\smallskip
\noindent\textbf{\textit{Baselines.}} We compare CF-Reg with the following regularization techniques: L1 (``Lasso''), L2 (``Ridge''), and Dropout.

\smallskip
\noindent\textbf{\textit{Configurations.}}
For each model, we adopt specific configurations as follows.
\begin{itemize}
\item \textit{Logistic Regression:} To induce overfitting in the model, we artificially increase the dimensionality of the data beyond the number of training samples by applying a polynomial feature expansion. This approach ensures that the model has enough capacity to overfit the training data, allowing us to analyze the impact of our counterfactual regularizer. The degree of the polynomial is chosen as the smallest degree that makes the number of features greater than the number of data.
\item \textit{Neural Networks (MLP and CNN):} To take advantage of the closed-form solution for computing the optimal perturbation vector as defined in (\ref{eq:opt-delta}), we use a local linear approximation of the neural network models. Hence, given an instance $\inst_i$, we consider the (optimal) counterfactual not with respect to $\model$ but with respect to:
\begin{equation}
\label{eq:taylor}
    \model^{lin}(\inst) = \model(\inst_i) + \nabla_{\inst}\model(\inst_i)(\inst - \inst_i),
\end{equation}
where $\model^{lin}$ represents the first-order Taylor approximation of $\model$ at $\inst_i$.
Note that this step is unnecessary for Logistic Regression, as it is inherently a linear model.
\end{itemize}

\smallskip
\noindent \textbf{\textit{Implementation Details.}} We run all experiments on a machine equipped with an AMD Ryzen 9 7900 12-Core Processor and an NVIDIA GeForce RTX 4090 GPU. Our implementation is based on the PyTorch Lightning framework. We use stochastic gradient descent as the optimizer with a learning rate of $\eta = 0.001$ and no weight decay. We use a batch size of $128$. The training and test steps are conducted for $6000$ epochs on the \textit{Water Potability} and \textit{Phoneme} datasets, while for the \textit{CIFAR-10} dataset, they are performed for $200$ epochs.
Finally, the contribution $w_i^{\varepsilon}$ of each training point $\inst_i$ is uniformly set as $w_i^{\varepsilon} = 1~\forall i\in \{1,\ldots,m\}$.

The source code implementation for our experiments is available at the following GitHub repository: \url{https://anonymous.4open.science/r/COCE-80B4/README.md} 

\subsection{RQ1: Counterfactual Perturbation vs. Overfitting}
To address \textbf{RQ1}, we analyze the relationship between the test loss and the average $L_2$-norm of the counterfactual perturbation vectors ($\overline{||\perturb||}$) over training epochs.

In particular, Figure~\ref{fig:delta_loss_epochs} depicts the evolution of $\overline{||\perturb||}$ alongside the test loss for an MLP trained \textit{without} regularization on the \textit{Water Potability} dataset. 
\begin{figure}[ht]
    \centering
    \includegraphics[width=0.85\linewidth]{img/delta_loss_epochs.png}
    \caption{The average counterfactual perturbation vector $\overline{||\perturb||}$ (left $y$-axis) and the cross-entropy test loss (right $y$-axis) over training epochs ($x$-axis) for an MLP trained on the \textit{Water Potability} dataset \textit{without} regularization.}
    \label{fig:delta_loss_epochs}
\end{figure}

The plot shows a clear trend as the model starts to overfit the data (evidenced by an increase in test loss). 
Notably, $\overline{||\perturb||}$ begins to decrease, which aligns with the hypothesis that the average distance to the optimal counterfactual example gets smaller as the model's decision boundary becomes increasingly adherent to the training data.

It is worth noting that this trend is heavily influenced by the choice of the counterfactual generator model. In particular, the relationship between $\overline{||\perturb||}$ and the degree of overfitting may become even more pronounced when leveraging more accurate counterfactual generators. However, these models often come at the cost of higher computational complexity, and their exploration is left to future work.

Nonetheless, we expect that $\overline{||\perturb||}$ will eventually stabilize at a plateau, as the average $L_2$-norm of the optimal counterfactual perturbations cannot vanish to zero.

% Additionally, the choice of employing the score-based counterfactual explanation framework to generate counterfactuals was driven to promote computational efficiency.

% Future enhancements to the framework may involve adopting models capable of generating more precise counterfactuals. While such approaches may yield to performance improvements, they are likely to come at the cost of increased computational complexity.


\subsection{RQ2: Counterfactual Regularization Performance}
To answer \textbf{RQ2}, we evaluate the effectiveness of the proposed counterfactual regularization (CF-Reg) by comparing its performance against existing baselines: unregularized training loss (No-Reg), L1 regularization (L1-Reg), L2 regularization (L2-Reg), and Dropout.
Specifically, for each model and dataset combination, Table~\ref{tab:regularization_comparison} presents the mean value and standard deviation of test accuracy achieved by each method across 5 random initialization. 

The table illustrates that our regularization technique consistently delivers better results than existing methods across all evaluated scenarios, except for one case -- i.e., Logistic Regression on the \textit{Phomene} dataset. 
However, this setting exhibits an unusual pattern, as the highest model accuracy is achieved without any regularization. Even in this case, CF-Reg still surpasses other regularization baselines.

From the results above, we derive the following key insights. First, CF-Reg proves to be effective across various model types, ranging from simple linear models (Logistic Regression) to deep architectures like MLPs and CNNs, and across diverse datasets, including both tabular and image data. 
Second, CF-Reg's strong performance on the \textit{Water} dataset with Logistic Regression suggests that its benefits may be more pronounced when applied to simpler models. However, the unexpected outcome on the \textit{Phoneme} dataset calls for further investigation into this phenomenon.


\begin{table*}[h!]
    \centering
    \caption{Mean value and standard deviation of test accuracy across 5 random initializations for different model, dataset, and regularization method. The best results are highlighted in \textbf{bold}.}
    \label{tab:regularization_comparison}
    \begin{tabular}{|c|c|c|c|c|c|c|}
        \hline
        \textbf{Model} & \textbf{Dataset} & \textbf{No-Reg} & \textbf{L1-Reg} & \textbf{L2-Reg} & \textbf{Dropout} & \textbf{CF-Reg (ours)} \\ \hline
        Logistic Regression   & \textit{Water}   & $0.6595 \pm 0.0038$   & $0.6729 \pm 0.0056$   & $0.6756 \pm 0.0046$  & N/A    & $\mathbf{0.6918 \pm 0.0036}$                     \\ \hline
        MLP   & \textit{Water}   & $0.6756 \pm 0.0042$   & $0.6790 \pm 0.0058$   & $0.6790 \pm 0.0023$  & $0.6750 \pm 0.0036$    & $\mathbf{0.6802 \pm 0.0046}$                    \\ \hline
%        MLP   & \textit{Adult}   & $0.8404 \pm 0.0010$   & $\mathbf{0.8495 \pm 0.0007}$   & $0.8489 \pm 0.0014$  & $\mathbf{0.8495 \pm 0.0016}$     & $0.8449 \pm 0.0019$                    \\ \hline
        Logistic Regression   & \textit{Phomene}   & $\mathbf{0.8148 \pm 0.0020}$   & $0.8041 \pm 0.0028$   & $0.7835 \pm 0.0176$  & N/A    & $0.8098 \pm 0.0055$                     \\ \hline
        MLP   & \textit{Phomene}   & $0.8677 \pm 0.0033$   & $0.8374 \pm 0.0080$   & $0.8673 \pm 0.0045$  & $0.8672 \pm 0.0042$     & $\mathbf{0.8718 \pm 0.0040}$                    \\ \hline
        CNN   & \textit{CIFAR-10} & $0.6670 \pm 0.0233$   & $0.6229 \pm 0.0850$   & $0.7348 \pm 0.0365$   & N/A    & $\mathbf{0.7427 \pm 0.0571}$                     \\ \hline
    \end{tabular}
\end{table*}

\begin{table*}[htb!]
    \centering
    \caption{Hyperparameter configurations utilized for the generation of Table \ref{tab:regularization_comparison}. For our regularization the hyperparameters are reported as $\mathbf{\alpha/\beta}$.}
    \label{tab:performance_parameters}
    \begin{tabular}{|c|c|c|c|c|c|c|}
        \hline
        \textbf{Model} & \textbf{Dataset} & \textbf{No-Reg} & \textbf{L1-Reg} & \textbf{L2-Reg} & \textbf{Dropout} & \textbf{CF-Reg (ours)} \\ \hline
        Logistic Regression   & \textit{Water}   & N/A   & $0.0093$   & $0.6927$  & N/A    & $0.3791/1.0355$                     \\ \hline
        MLP   & \textit{Water}   & N/A   & $0.0007$   & $0.0022$  & $0.0002$    & $0.2567/1.9775$                    \\ \hline
        Logistic Regression   &
        \textit{Phomene}   & N/A   & $0.0097$   & $0.7979$  & N/A    & $0.0571/1.8516$                     \\ \hline
        MLP   & \textit{Phomene}   & N/A   & $0.0007$   & $4.24\cdot10^{-5}$  & $0.0015$    & $0.0516/2.2700$                    \\ \hline
       % MLP   & \textit{Adult}   & N/A   & $0.0018$   & $0.0018$  & $0.0601$     & $0.0764/2.2068$                    \\ \hline
        CNN   & \textit{CIFAR-10} & N/A   & $0.0050$   & $0.0864$ & N/A    & $0.3018/
        2.1502$                     \\ \hline
    \end{tabular}
\end{table*}

\begin{table*}[htb!]
    \centering
    \caption{Mean value and standard deviation of training time across 5 different runs. The reported time (in seconds) corresponds to the generation of each entry in Table \ref{tab:regularization_comparison}. Times are }
    \label{tab:times}
    \begin{tabular}{|c|c|c|c|c|c|c|}
        \hline
        \textbf{Model} & \textbf{Dataset} & \textbf{No-Reg} & \textbf{L1-Reg} & \textbf{L2-Reg} & \textbf{Dropout} & \textbf{CF-Reg (ours)} \\ \hline
        Logistic Regression   & \textit{Water}   & $222.98 \pm 1.07$   & $239.94 \pm 2.59$   & $241.60 \pm 1.88$  & N/A    & $251.50 \pm 1.93$                     \\ \hline
        MLP   & \textit{Water}   & $225.71 \pm 3.85$   & $250.13 \pm 4.44$   & $255.78 \pm 2.38$  & $237.83 \pm 3.45$    & $266.48 \pm 3.46$                    \\ \hline
        Logistic Regression   & \textit{Phomene}   & $266.39 \pm 0.82$ & $367.52 \pm 6.85$   & $361.69 \pm 4.04$  & N/A   & $310.48 \pm 0.76$                    \\ \hline
        MLP   &
        \textit{Phomene} & $335.62 \pm 1.77$   & $390.86 \pm 2.11$   & $393.96 \pm 1.95$ & $363.51 \pm 5.07$    & $403.14 \pm 1.92$                     \\ \hline
       % MLP   & \textit{Adult}   & N/A   & $0.0018$   & $0.0018$  & $0.0601$     & $0.0764/2.2068$                    \\ \hline
        CNN   & \textit{CIFAR-10} & $370.09 \pm 0.18$   & $395.71 \pm 0.55$   & $401.38 \pm 0.16$ & N/A    & $1287.8 \pm 0.26$                     \\ \hline
    \end{tabular}
\end{table*}

\subsection{Feasibility of our Method}
A crucial requirement for any regularization technique is that it should impose minimal impact on the overall training process.
In this respect, CF-Reg introduces an overhead that depends on the time required to find the optimal counterfactual example for each training instance. 
As such, the more sophisticated the counterfactual generator model probed during training the higher would be the time required. However, a more advanced counterfactual generator might provide a more effective regularization. We discuss this trade-off in more details in Section~\ref{sec:discussion}.

Table~\ref{tab:times} presents the average training time ($\pm$ standard deviation) for each model and dataset combination listed in Table~\ref{tab:regularization_comparison}.
We can observe that the higher accuracy achieved by CF-Reg using the score-based counterfactual generator comes with only minimal overhead. However, when applied to deep neural networks with many hidden layers, such as \textit{PreactResNet-18}, the forward derivative computation required for the linearization of the network introduces a more noticeable computational cost, explaining the longer training times in the table.

\subsection{Hyperparameter Sensitivity Analysis}
The proposed counterfactual regularization technique relies on two key hyperparameters: $\alpha$ and $\beta$. The former is intrinsic to the loss formulation defined in (\ref{eq:cf-train}), while the latter is closely tied to the choice of the score-based counterfactual explanation method used.

Figure~\ref{fig:test_alpha_beta} illustrates how the test accuracy of an MLP trained on the \textit{Water Potability} dataset changes for different combinations of $\alpha$ and $\beta$.

\begin{figure}[ht]
    \centering
    \includegraphics[width=0.85\linewidth]{img/test_acc_alpha_beta.png}
    \caption{The test accuracy of an MLP trained on the \textit{Water Potability} dataset, evaluated while varying the weight of our counterfactual regularizer ($\alpha$) for different values of $\beta$.}
    \label{fig:test_alpha_beta}
\end{figure}

We observe that, for a fixed $\beta$, increasing the weight of our counterfactual regularizer ($\alpha$) can slightly improve test accuracy until a sudden drop is noticed for $\alpha > 0.1$.
This behavior was expected, as the impact of our penalty, like any regularization term, can be disruptive if not properly controlled.

Moreover, this finding further demonstrates that our regularization method, CF-Reg, is inherently data-driven. Therefore, it requires specific fine-tuning based on the combination of the model and dataset at hand.
\section{Discussion of Assumptions}\label{sec:discussion}
In this paper, we have made several assumptions for the sake of clarity and simplicity. In this section, we discuss the rationale behind these assumptions, the extent to which these assumptions hold in practice, and the consequences for our protocol when these assumptions hold.

\subsection{Assumptions on the Demand}

There are two simplifying assumptions we make about the demand. First, we assume the demand at any time is relatively small compared to the channel capacities. Second, we take the demand to be constant over time. We elaborate upon both these points below.

\paragraph{Small demands} The assumption that demands are small relative to channel capacities is made precise in \eqref{eq:large_capacity_assumption}. This assumption simplifies two major aspects of our protocol. First, it largely removes congestion from consideration. In \eqref{eq:primal_problem}, there is no constraint ensuring that total flow in both directions stays below capacity--this is always met. Consequently, there is no Lagrange multiplier for congestion and no congestion pricing; only imbalance penalties apply. In contrast, protocols in \cite{sivaraman2020high, varma2021throughput, wang2024fence} include congestion fees due to explicit congestion constraints. Second, the bound \eqref{eq:large_capacity_assumption} ensures that as long as channels remain balanced, the network can always meet demand, no matter how the demand is routed. Since channels can rebalance when necessary, they never drop transactions. This allows prices and flows to adjust as per the equations in \eqref{eq:algorithm}, which makes it easier to prove the protocol's convergence guarantees. This also preserves the key property that a channel's price remains proportional to net money flow through it.

In practice, payment channel networks are used most often for micro-payments, for which on-chain transactions are prohibitively expensive; large transactions typically take place directly on the blockchain. For example, according to \cite{river2023lightning}, the average channel capacity is roughly $0.1$ BTC ($5,000$ BTC distributed over $50,000$ channels), while the average transaction amount is less than $0.0004$ BTC ($44.7k$ satoshis). Thus, the small demand assumption is not too unrealistic. Additionally, the occasional large transaction can be treated as a sequence of smaller transactions by breaking it into packets and executing each packet serially (as done by \cite{sivaraman2020high}).
Lastly, a good path discovery process that favors large capacity channels over small capacity ones can help ensure that the bound in \eqref{eq:large_capacity_assumption} holds.

\paragraph{Constant demands} 
In this work, we assume that any transacting pair of nodes have a steady transaction demand between them (see Section \ref{sec:transaction_requests}). Making this assumption is necessary to obtain the kind of guarantees that we have presented in this paper. Unless the demand is steady, it is unreasonable to expect that the flows converge to a steady value. Weaker assumptions on the demand lead to weaker guarantees. For example, with the more general setting of stochastic, but i.i.d. demand between any two nodes, \cite{varma2021throughput} shows that the channel queue lengths are bounded in expectation. If the demand can be arbitrary, then it is very hard to get any meaningful performance guarantees; \cite{wang2024fence} shows that even for a single bidirectional channel, the competitive ratio is infinite. Indeed, because a PCN is a decentralized system and decisions must be made based on local information alone, it is difficult for the network to find the optimal detailed balance flow at every time step with a time-varying demand.  With a steady demand, the network can discover the optimal flows in a reasonably short time, as our work shows.

We view the constant demand assumption as an approximation for a more general demand process that could be piece-wise constant, stochastic, or both (see simulations in Figure \ref{fig:five_nodes_variable_demand}).
We believe it should be possible to merge ideas from our work and \cite{varma2021throughput} to provide guarantees in a setting with random demands with arbitrary means. We leave this for future work. In addition, our work suggests that a reasonable method of handling stochastic demands is to queue the transaction requests \textit{at the source node} itself. This queuing action should be viewed in conjunction with flow-control. Indeed, a temporarily high unidirectional demand would raise prices for the sender, incentivizing the sender to stop sending the transactions. If the sender queues the transactions, they can send them later when prices drop. This form of queuing does not require any overhaul of the basic PCN infrastructure and is therefore simpler to implement than per-channel queues as suggested by \cite{sivaraman2020high} and \cite{varma2021throughput}.

\subsection{The Incentive of Channels}
The actions of the channels as prescribed by the DEBT control protocol can be summarized as follows. Channels adjust their prices in proportion to the net flow through them. They rebalance themselves whenever necessary and execute any transaction request that has been made of them. We discuss both these aspects below.

\paragraph{On Prices}
In this work, the exclusive role of channel prices is to ensure that the flows through each channel remains balanced. In practice, it would be important to include other components in a channel's price/fee as well: a congestion price  and an incentive price. The congestion price, as suggested by \cite{varma2021throughput}, would depend on the total flow of transactions through the channel, and would incentivize nodes to balance the load over different paths. The incentive price, which is commonly used in practice \cite{river2023lightning}, is necessary to provide channels with an incentive to serve as an intermediary for different channels. In practice, we expect both these components to be smaller than the imbalance price. Consequently, we expect the behavior of our protocol to be similar to our theoretical results even with these additional prices.

A key aspect of our protocol is that channel fees are allowed to be negative. Although the original Lightning network whitepaper \cite{poon2016bitcoin} suggests that negative channel prices may be a good solution to promote rebalancing, the idea of negative prices in not very popular in the literature. To our knowledge, the only prior work with this feature is \cite{varma2021throughput}. Indeed, in papers such as \cite{van2021merchant} and \cite{wang2024fence}, the price function is explicitly modified such that the channel price is never negative. The results of our paper show the benefits of negative prices. For one, in steady state, equal flows in both directions ensure that a channel doesn't loose any money (the other price components mentioned above ensure that the channel will only gain money). More importantly, negative prices are important to ensure that the protocol selectively stifles acyclic flows while allowing circulations to flow. Indeed, in the example of Section \ref{sec:flow_control_example}, the flows between nodes $A$ and $C$ are left on only because the large positive price over one channel is canceled by the corresponding negative price over the other channel, leading to a net zero price.

Lastly, observe that in the DEBT control protocol, the price charged by a channel does not depend on its capacity. This is a natural consequence of the price being the Lagrange multiplier for the net-zero flow constraint, which also does not depend on the channel capacity. In contrast, in many other works, the imbalance price is normalized by the channel capacity \cite{ren2018optimal, lin2020funds, wang2024fence}; this is shown to work well in practice. The rationale for such a price structure is explained well in \cite{wang2024fence}, where this fee is derived with the aim of always maintaining some balance (liquidity) at each end of every channel. This is a reasonable aim if a channel is to never rebalance itself; the experiments of the aforementioned papers are conducted in such a regime. In this work, however, we allow the channels to rebalance themselves a few times in order to settle on a detailed balance flow. This is because our focus is on the long-term steady state performance of the protocol. This difference in perspective also shows up in how the price depends on the channel imbalance. \cite{lin2020funds} and \cite{wang2024fence} advocate for strictly convex prices whereas this work and \cite{varma2021throughput} propose linear prices.

\paragraph{On Rebalancing} 
Recall that the DEBT control protocol ensures that the flows in the network converge to a detailed balance flow, which can be sustained perpetually without any rebalancing. However, during the transient phase (before convergence), channels may have to perform on-chain rebalancing a few times. Since rebalancing is an expensive operation, it is worthwhile discussing methods by which channels can reduce the extent of rebalancing. One option for the channels to reduce the extent of rebalancing is to increase their capacity; however, this comes at the cost of locking in more capital. Each channel can decide for itself the optimum amount of capital to lock in. Another option, which we discuss in Section \ref{sec:five_node}, is for channels to increase the rate $\gamma$ at which they adjust prices. 

Ultimately, whether or not it is beneficial for a channel to rebalance depends on the time-horizon under consideration. Our protocol is based on the assumption that the demand remains steady for a long period of time. If this is indeed the case, it would be worthwhile for a channel to rebalance itself as it can make up this cost through the incentive fees gained from the flow of transactions through it in steady state. If a channel chooses not to rebalance itself, however, there is a risk of being trapped in a deadlock, which is suboptimal for not only the nodes but also the channel.

\section{Conclusion}
This work presents DEBT control: a protocol for payment channel networks that uses source routing and flow control based on channel prices. The protocol is derived by posing a network utility maximization problem and analyzing its dual minimization. It is shown that under steady demands, the protocol guides the network to an optimal, sustainable point. Simulations show its robustness to demand variations. The work demonstrates that simple protocols with strong theoretical guarantees are possible for PCNs and we hope it inspires further theoretical research in this direction.
\clearpage
\section*{Acknowledgements}
The authors would like to acknowledge the computing resources provided by the Mila cluster to enable the experiments outlined in this work. SM acknowledges the support of UNIQUE's scholarship.
GL acknowledges the support of the Canada CIFAR AI Chair program, NSERC Discovery Grant RGPIN-2018-04821, and a Canada Research Chair in Neural Computations and Interfacing. 
MAB acknowledges the support of the Canada First Research Excellence Fund (CFREF) for the Vision: Science to Applications (VISTA) program, NSERC Discovery Grant RGPIN-2017-05638 and Google.
The authors also thank NVIDIA for computing resources.

\section*{Impact Statement}
We provide a comprehensive evaluation of different approaches and design choices in performing Bayesian posterior estimation for a wide variety of probabilistic models. We believe that analysis into such amortized estimators could lead to more efficient and scalable Bayesian methods that can lead to robust predictions and better OoD generalization. Thus, we believe that our work generally advances the field of machine learning through careful and thorough benchmarking. There are many potential societal
consequences of our work, none of which we feel must be
specifically highlighted here.
\bibliography{references}
\bibliographystyle{icml2025}

\clearpage
\appendix
\onecolumn
\section*{\LARGE Appendix}
\section{RELATED WORK}
\label{sec:relatedwork}
In this section, we describe the previous works related to our proposal, which are divided into two parts. In Section~\ref{sec:relatedwork_exoplanet}, we present a review of approaches based on machine learning techniques for the detection of planetary transit signals. Section~\ref{sec:relatedwork_attention} provides an account of the approaches based on attention mechanisms applied in Astronomy.\par

\subsection{Exoplanet detection}
\label{sec:relatedwork_exoplanet}
Machine learning methods have achieved great performance for the automatic selection of exoplanet transit signals. One of the earliest applications of machine learning is a model named Autovetter \citep{MCcauliff}, which is a random forest (RF) model based on characteristics derived from Kepler pipeline statistics to classify exoplanet and false positive signals. Then, other studies emerged that also used supervised learning. \cite{mislis2016sidra} also used a RF, but unlike the work by \citet{MCcauliff}, they used simulated light curves and a box least square \citep[BLS;][]{kovacs2002box}-based periodogram to search for transiting exoplanets. \citet{thompson2015machine} proposed a k-nearest neighbors model for Kepler data to determine if a given signal has similarity to known transits. Unsupervised learning techniques were also applied, such as self-organizing maps (SOM), proposed \citet{armstrong2016transit}; which implements an architecture to segment similar light curves. In the same way, \citet{armstrong2018automatic} developed a combination of supervised and unsupervised learning, including RF and SOM models. In general, these approaches require a previous phase of feature engineering for each light curve. \par

%DL is a modern data-driven technology that automatically extracts characteristics, and that has been successful in classification problems from a variety of application domains. The architecture relies on several layers of NNs of simple interconnected units and uses layers to build increasingly complex and useful features by means of linear and non-linear transformation. This family of models is capable of generating increasingly high-level representations \citep{lecun2015deep}.

The application of DL for exoplanetary signal detection has evolved rapidly in recent years and has become very popular in planetary science.  \citet{pearson2018} and \citet{zucker2018shallow} developed CNN-based algorithms that learn from synthetic data to search for exoplanets. Perhaps one of the most successful applications of the DL models in transit detection was that of \citet{Shallue_2018}; who, in collaboration with Google, proposed a CNN named AstroNet that recognizes exoplanet signals in real data from Kepler. AstroNet uses the training set of labelled TCEs from the Autovetter planet candidate catalog of Q1–Q17 data release 24 (DR24) of the Kepler mission \citep{catanzarite2015autovetter}. AstroNet analyses the data in two views: a ``global view'', and ``local view'' \citep{Shallue_2018}. \par


% The global view shows the characteristics of the light curve over an orbital period, and a local view shows the moment at occurring the transit in detail

%different = space-based

Based on AstroNet, researchers have modified the original AstroNet model to rank candidates from different surveys, specifically for Kepler and TESS missions. \citet{ansdell2018scientific} developed a CNN trained on Kepler data, and included for the first time the information on the centroids, showing that the model improves performance considerably. Then, \citet{osborn2020rapid} and \citet{yu2019identifying} also included the centroids information, but in addition, \citet{osborn2020rapid} included information of the stellar and transit parameters. Finally, \citet{rao2021nigraha} proposed a pipeline that includes a new ``half-phase'' view of the transit signal. This half-phase view represents a transit view with a different time and phase. The purpose of this view is to recover any possible secondary eclipse (the object hiding behind the disk of the primary star).


%last pipeline applies a procedure after the prediction of the model to obtain new candidates, this process is carried out through a series of steps that include the evaluation with Discovery and Validation of Exoplanets (DAVE) \citet{kostov2019discovery} that was adapted for the TESS telescope.\par
%



\subsection{Attention mechanisms in astronomy}
\label{sec:relatedwork_attention}
Despite the remarkable success of attention mechanisms in sequential data, few papers have exploited their advantages in astronomy. In particular, there are no models based on attention mechanisms for detecting planets. Below we present a summary of the main applications of this modeling approach to astronomy, based on two points of view; performance and interpretability of the model.\par
%Attention mechanisms have not yet been explored in all sub-areas of astronomy. However, recent works show a successful application of the mechanism.
%performance

The application of attention mechanisms has shown improvements in the performance of some regression and classification tasks compared to previous approaches. One of the first implementations of the attention mechanism was to find gravitational lenses proposed by \citet{thuruthipilly2021finding}. They designed 21 self-attention-based encoder models, where each model was trained separately with 18,000 simulated images, demonstrating that the model based on the Transformer has a better performance and uses fewer trainable parameters compared to CNN. A novel application was proposed by \citet{lin2021galaxy} for the morphological classification of galaxies, who used an architecture derived from the Transformer, named Vision Transformer (VIT) \citep{dosovitskiy2020image}. \citet{lin2021galaxy} demonstrated competitive results compared to CNNs. Another application with successful results was proposed by \citet{zerveas2021transformer}; which first proposed a transformer-based framework for learning unsupervised representations of multivariate time series. Their methodology takes advantage of unlabeled data to train an encoder and extract dense vector representations of time series. Subsequently, they evaluate the model for regression and classification tasks, demonstrating better performance than other state-of-the-art supervised methods, even with data sets with limited samples.

%interpretation
Regarding the interpretability of the model, a recent contribution that analyses the attention maps was presented by \citet{bowles20212}, which explored the use of group-equivariant self-attention for radio astronomy classification. Compared to other approaches, this model analysed the attention maps of the predictions and showed that the mechanism extracts the brightest spots and jets of the radio source more clearly. This indicates that attention maps for prediction interpretation could help experts see patterns that the human eye often misses. \par

In the field of variable stars, \citet{allam2021paying} employed the mechanism for classifying multivariate time series in variable stars. And additionally, \citet{allam2021paying} showed that the activation weights are accommodated according to the variation in brightness of the star, achieving a more interpretable model. And finally, related to the TESS telescope, \citet{morvan2022don} proposed a model that removes the noise from the light curves through the distribution of attention weights. \citet{morvan2022don} showed that the use of the attention mechanism is excellent for removing noise and outliers in time series datasets compared with other approaches. In addition, the use of attention maps allowed them to show the representations learned from the model. \par

Recent attention mechanism approaches in astronomy demonstrate comparable results with earlier approaches, such as CNNs. At the same time, they offer interpretability of their results, which allows a post-prediction analysis. \par


\section{Architectures respecting Exchangeability}
\label{appdx:exchangeability}
In this section, we highlight how DeepSets and Transformer models satisfy the dataset exchangeability criteria, which is essential in modeling the posterior distribution over the parameters of any probabilistic model relying on \textit{iid} data. 

\subsection{DeepSets}
DeepSets~\citep{Zaheer2017deepsets} operate on arbitrary sets $\gX = \{x_1, ... x_N\} \subset \mathbb{R}^d$ of fixed dimensionality $d$ by first mapping each individual element $\vx_i \in \gX$ to some high-dimensional space using a nonlinear transform, which is parameterized as a multi-layered neural network with parameters $\varphi_1$
\begin{align}
    \vz_i = f_{\varphi_1}(\vx_i)
\end{align}
After having obtained this high-dimensional embedding of each element of the set, it applies an aggregation function $a(\cdot)$, which is a permutation invariant function that maps a set of elements $\gZ = \{\vz_1, ..., \vz_N\} \in \mathbb{R}^z$ to an element $\vh \in \mathbb{R}^z$,
\begin{align}
    \vh = a(\gZ)
\end{align}
Thus, the outcome does not change under permutations of $\gZ$. Finally, another nonlinear transform, parameterized by a multi-layered neural network with parameters $\varphi_2$, is applied to the outcome $\vh$ to provide the final output.
\begin{align}
    \vo = g_{\varphi_2}(\vh)
\end{align}
For our experiments, we then use the vector $\vo$ to predict the parameters of a parametric family of distributions (e.g., Gaussian or Flows) using an additional nonlinear neural network. As an example, for the Gaussian case, we consider the distribution $\gN(\cdot | \mmu, \mSigma)$, where
\begin{align}
    \mmu:= \mmu_{\varphi_3}(\vo) \quad\text{and}\quad\mSigma := \mSigma_{\varphi_4}(\vo)
\end{align}
which makes $\mmu$ implicitly a function of the original input set $\gX$. To understand why the posterior distribution modeled in this fashion does not change when the inputs are permuted, let us assume that $\Pi$ is a permutation over the elements of $\gX$. If we look at one of the parameters of the posterior distribution, e.g., $\mmu$, we can see that
\begin{align}
    \mmu(\Pi \gX) &= \mmu_{\varphi_3}\left(g_{\varphi_2}\left(a\left(\{f_{\varphi_1}(\vx_{\Pi(i)})\}_{i=1}^N\right)\right)\right) \\
    &= \mmu_{\varphi_3}\left(g_{\varphi_2}\left(a\left(\{f_{\varphi_1}(\vx_i)\}_{i=1}^N\right)\right)\right) \\
    &= \mmu(\gX)
\end{align}
which simply follows from the fact that $a(\cdot)$ is a permutation invariant operation, e.g., sum or mean. We can also provide similar reasoning for the other parameters (e.g., $\mSigma$). This shows that DeepSets can be used to model the posterior distribution over parameters of interest as it respects the exchangeability criteria (\textit{iid} observations) assumptions in the data through its permutation invariant structure.

% \begin{table}[t]
    \centering
    \small
    \def\arraystretch{1.25}
    \setlength{\tabcolsep}{4pt}
    \begin{tabular}{l lcr | cc | cccc }
        \toprule
        & & & & \multicolumn{2}{c|}{\textit{$L_2$ Loss} ($\downarrow$)} & \multicolumn{4}{c}{\textit{Accuracy} ($\uparrow$)}\\

        \textbf{Setup} & & $q_\varphi$ & \textbf{Model} & \multicolumn{2}{c|}{\textbf{NLR}} & \multicolumn{4}{c}{\textbf{NLC}} \\
        
        & & & & \textit{1D} & \textit{25D} & \textit{2D-2cl} & \textit{2D-5cl} & \textit{25D-2cl} & \textit{25D-5cl} \\
        \midrule
\multirow{11}{*}{\rotatebox[origin=c]{90}{\textsc{tanh}}} 
& \multirow{3}{*}{Baseline} & - & Random & $52.72$\std{$0.5$} & $55.16$\std{$0.0$} & $50.16$\std{$0.1$} & $19.83$\std{$0.2$} & $50.02$\std{$0.2$} & $20.00$\std{$0.1$} \\
& & - & Optimization & $0.56$\std{$0.0$} & $30.22$\std{$0.1$} & $96.76$\std{$0.0$} & $89.99$\std{$0.0$} & $69.51$\std{$0.0$} & $40.43$\std{$0.0$} \\
& & - & MCMC & $0.39$\std{$0.0$} & $32.41$\std{$0.9$} & $94.23$\std{$0.2$} & $81.42$\std{$0.2$} & $53.16$\std{$0.8$} & $24.49$\std{$0.4$} \\
\cmidrule{4-10}
& \multirow{2}{*}{Fwd-KL} & \multirow{4}{*}{\rotatebox[origin=c]{90}{Gaussian}} & DeepSets & $53.83$\std{$0.5$} & $56.02$\std{$0.1$} & $50.07$\std{$0.4$} & $19.95$\std{$0.2$} & $49.91$\std{$0.1$} & $19.93$\std{$0.0$} \\
& & & Transformer & $53.89$\std{$0.6$} & $56.82$\std{$0.3$} & $50.07$\std{$0.4$} & $20.18$\std{$0.2$} & $49.91$\std{$0.1$} & $20.09$\std{$0.1$} \\
& \multirow{2}{*}{Rev-KL} & & DeepSets & $0.89$\std{$0.0$} & $27.98$\std{$0.0$} & $50.04$\std{$0.4$} & $19.99$\std{$0.2$} & $49.91$\std{$0.1$} & $19.97$\std{$0.0$} \\
& & & Transformer & $0.87$\std{$0.0$} & \highlight{$25.17$\std{$4.0$}} & $61.64$\std{$16.8$} & $19.99$\std{$0.2$} & $49.92$\std{$0.1$} & $19.95$\std{$0.1$} \\
\cmidrule{4-10}
& \multirow{2}{*}{Fwd-KL} & \multirow{4}{*}{\rotatebox[origin=c]{90}{Flow}} & DeepSets & $52.46$\std{$0.5$} & $55.85$\std{$0.1$} & $50.47$\std{$0.1$} & $19.79$\std{$0.1$} & $49.93$\std{$0.0$} & $20.06$\std{$0.0$} \\
& & & Transformer & $52.44$\std{$0.5$} & $55.84$\std{$0.1$} & $50.45$\std{$0.1$} & $20.04$\std{$0.1$} & $49.94$\std{$0.0$} & $20.19$\std{$0.0$} \\
& \multirow{2}{*}{Rev-KL} & & DeepSets & \highlight{$0.68$\std{$0.0$}} & $27.97$\std{$0.0$} & $50.67$\std{$0.1$} & $19.72$\std{$0.1$} & $49.99$\std{$0.1$} & $20.06$\std{$0.1$} \\
& & & Transformer & $0.70$\std{$0.0$} & $27.97$\std{$0.0$} & \highlight{$86.37$\std{$0.2$}} & $19.78$\std{$0.1$} & $50.00$\std{$0.2$} & $20.02$\std{$0.1$} \\
\midrule
\multirow{11}{*}{\rotatebox[origin=c]{90}{\textsc{relu}}}
& \multirow{3}{*}{Baseline} & - & Random & $1082.06$\std{$3.7$} & $13301.74$\std{$109.1$} & $49.68$\std{$0.6$} & $20.18$\std{$0.1$} & $49.74$\std{$0.1$} & $20.08$\std{$0.2$} \\
& & - & Optimization & $2.01$\std{$0.1$} & $1858.40$\std{$2.6$} & $98.01$\std{$0.2$} & $96.81$\std{$0.0$} & $80.30$\std{$0.1$} & $61.30$\std{$0.0$} \\
& & - & MCMC & \textsc{N/A} & \textsc{N/A} & $88.39$\std{$0.9$} & $52.78$\std{$1.9$} & $66.16$\std{$0.1$} & $35.49$\std{$0.7$} \\
\cmidrule{4-10}
& \multirow{2}{*}{Fwd-KL} & \multirow{4}{*}{\rotatebox[origin=c]{90}{Gaussian}} & DeepSets & $821.57$\std{$3.4$} & $10877.36$\std{$200.8$} & $60.77$\std{$0.5$} & $31.66$\std{$0.2$} & $58.35$\std{$0.3$} & $19.88$\std{$0.1$} \\
& & & Transformer & $786.67$\std{$6.7$} & $10845.96$\std{$86.3$} & $61.21$\std{$0.5$} & $32.12$\std{$0.1$} & $58.28$\std{$0.3$} & $30.17$\std{$0.2$} \\
& \multirow{2}{*}{Rev-KL} & & DeepSets & $1.38$\std{$0.1$} & $2048.08$\std{$9.8$} & $74.11$\std{$0.2$} & $49.53$\std{$0.3$} & $66.41$\std{$0.1$} & $46.12$\std{$0.2$} \\
& & & Transformer & $2.36$\std{$0.4$} & \highlight{$1976.32$\std{$43.0$}} & $87.77$\std{$3.9$} & \highlight{$76.33$\std{$1.0$}} & $66.35$\std{$0.1$} & $30.01$\std{$0.3$} \\
\cmidrule{4-10}
& \multirow{2}{*}{Fwd-KL} & \multirow{4}{*}{\rotatebox[origin=c]{90}{Flow}} & DeepSets & $676.46$\std{$15.2$} & $8236.22$\std{$77.0$} & $62.90$\std{$0.4$} & $33.21$\std{$0.2$} & $60.28$\std{$0.1$} & $20.12$\std{$0.1$} \\
& & & Transformer & $646.60$\std{$36.1$} & $8075.57$\std{$94.2$} & $63.71$\std{$0.7$} & $34.11$\std{$0.1$} & $61.45$\std{$0.2$} & $32.70$\std{$0.1$} \\
& \multirow{2}{*}{Rev-KL} & & DeepSets & \highlight{$1.32$\std{$0.1$}} & $2040.10$\std{$2.6$} & $74.98$\std{$0.2$} & $61.65$\std{$0.5$} & $68.06$\std{$0.1$} & \highlight{$47.05$\std{$0.3$}} \\
& & & Transformer & $2.92$\std{$0.2$} & $1987.58$\std{$43.7$} & \highlight{$92.31$\std{$0.3$}} & $76.03$\std{$0.2$} & \highlight{$68.41$\std{$0.2$}} & $45.96$\std{$0.1$} \\
\bottomrule
    \end{tabular}
    \caption{\textbf{Fixed-Dim Posterior Prediction:} Experimental results for posterior inference on fixed dimensional datasets evaluated on estimating the parameters of nonlinear regression (NLR) and classification (NLC) setups, with 2 layered MLP with different activation functions in the probabilistic model. We also consider a multi-class classification setup. We consider different backbone architectures and parametric distributions $q_\varphi$, and use dataset-specific Bayesian and point estimates as baselines. $L_2$ Loss and Accuracy refer to the expected posterior-predictive $L_2$ loss and accuracy respectively. Here, cl refers to the number classes.}
    \label{tab:fixed_dim_2_layer}
\end{table}
\subsection{Transformers}
Similarly, we can look at Transformers~\citep{vaswani2017attention} as candidates for respecting the exchangeability conditions in the data. In particular, we consider transformer systems without positional encodings and consider an additional [CLS] token, denoted by $\vc\in\mathbb{R}^d$, to drive the prediction. If we look at the application of a layer of transformer model, it can be broken down into two components.

\textbf{Multi-Head Attention}. Given a query vector obtained from $\vc$ and keys and values coming from our input set $\gX \subset \mathbb{R}^d$, we can model the update of the context $\vc$ as
\begin{align}
    \hat{\vc}(\gX) = \text{Softmax}\left(\vc^T \mW_Q \mW_K^T \mX^T\right) \mX \mW_V
\end{align}
where $\mW_Q \in \mathbb{R}^{d\times k}, \mW_K \in \mathbb{R}^{d\times k}, \mW_V \in \mathbb{R}^{d\times k}$ and $\mX \in \mathbb{R}^{N\times d}$ denotes a certain ordering of the elements in $\gX$. Further, $\hat{\vc}$ is the updated vector after attention, and Softmax is over the rows of $\mX$. Here, we see that if we were to apply a permutation to the elements in $\mX$, the outcome would remain the same. In particular
\begin{align}
    \hat{\vc}(\Pi \mX) &= \text{Softmax}\left(\vc^T \mW_Q \mW_K^T \mX^T \Pi^T\right) \Pi \mX \mW_V \\
    &= \text{Softmax}\left(\vc^T \mW_Q \mW_K^T \mX^T\right) \Pi^T\Pi \mX \mW_V \\
    &= \text{Softmax}\left(\vc^T \mW_Q \mW_K^T \mX^T\right) \mX \mW_V \\
    &= \hat{\vc}(\mX) 
\end{align}
which follows because Softmax is an equivariant function,  i.e., applying Softmax on a permutation of columns is equivalent to applying Softmax first and then permuting the columns correspondingly. Thus, we see that the update to the [CLS] token $\vc$ is permutation invariant. This output is then used independently as input to a multi-layered neural network with residual connections, and the entire process is repeated multiple times without weight sharing to simulate multiple layers. Since all the individual parts are permutation invariant w.r.t permutations on $\gX$, the entire setup ends up being permutation invariant. Obtaining the parameters of a parametric family of distribution for posterior estimation then follows the same recipe as DeepSets, with $\vo$ replaced by $\vc$.
\section{Probabilistic Models}
\label{appdx:probabilistic_models}
This section details the various candidate probabilistic models used in our experiments for amortized computation of Bayesian posteriors over the parameters. Here, we explain the parameters associated with the probabilistic model over which we want to estimate the posterior and the likelihood and prior that we use for experimentation.

\textbf{Mean of Gaussian (GM):} As a proof of concept, we consider the simple setup of estimating the posterior distribution over the mean of a Gaussian distribution $p(\mmu | \gD)$ given some observed data. In this case, prior and likelihood defining the probabilistic model $p(\vx, \mtheta)$ (with $\mtheta$ being the mean $\mmu$) are given by:
\begin{align}
    p(\mmu) &= \gN\left(\mmu | \mathbf{0}, \mathbf{I}\right)\\
    p(\vx | \mmu) &= \gN\left(\vx | \mmu, \mSigma\right) 
\end{align}
and $\mSigma$ is known beforehand and defined as a unit variance matrix. 

\begin{table}
    \centering
    \small
    % \footnotesize	    
    \def\arraystretch{1.25}
    \setlength{\tabcolsep}{3pt}
    \begin{tabular}{lcr cc cccc}
        \toprule
         &  &  & \multicolumn{6}{c}{\textit{$L_2$ Loss} ($\downarrow$)} \\
        \cmidrule(lr){4-9}
        \textbf{Objective} & $q_\varphi$ & \textbf{Model} & \multicolumn{2}{c}{\textbf{Gaussian}} & \multicolumn{4}{c}{\textbf{GMM}} \\
        \cmidrule(lr){4-5}\cmidrule(lr){6-9}
        & & & \textit{2D} & \textit{100D} & \textit{2D-2cl} & \textit{2D-5cl} & \textit{5D-2cl} & \textit{5D-5cl} \\
        \midrule
\multirow{4}{*}{Baseline} & - & Random & $5.839$\sstd{$0.015$} & $301.065$\sstd{$0.346$} & $1.887$\sstd{$0.031$} & $0.730$\sstd{$0.004$} & $5.001$\sstd{$0.037$} & $1.670$\sstd{$0.008$} \\
& - & Optimization & $1.989$\sstd{$0.000$} & $101.243$\sstd{$0.000$} & $0.169$\sstd{$0.000$} & $0.119$\sstd{$0.001$} & $0.425$\sstd{$0.000$} & $0.308$\sstd{$0.000$} \\
& - & Langevin & $2.013$\sstd{$0.004$} & $102.346$\sstd{$0.031$} & $0.173$\sstd{$0.001$} & $0.125$\sstd{$0.001$} & $0.448$\sstd{$0.009$} & $0.352$\sstd{$0.005$} \\
& - & HMC & $2.018$\sstd{$0.008$} & $102.413$\sstd{$0.028$} & $0.174$\sstd{$0.001$} & $0.135$\sstd{$0.001$} & $0.479$\sstd{$0.007$} & $0.449$\sstd{$0.002$} \\
\cmidrule{2-9}

\multirow{3}{*}{Fwd-KL} & \multirow{6}{*}{\rotatebox[origin=c]{90}{Gaussian}} & GRU &$2.014$\sstd{$0.001$} & $102.641$\sstd{$0.011$} & $0.921$\sstd{$0.013$} & $0.522$\sstd{$0.001$} & $2.430$\sstd{$0.034$} & $1.235$\sstd{$0.011$} \\
& & DeepSets & \highlight{$2.012$\sstd{$0.002$}} & $103.215$\sstd{$0.054$} & $0.920$\sstd{$0.019$} & $0.522$\sstd{$0.001$} & $2.436$\sstd{$0.037$} & $1.238$\sstd{$0.009$} \\
& & Transformer & \highlight{$2.013$\sstd{$0.002$}} & $102.783$\sstd{$0.005$} & $0.931$\sstd{$0.017$} & $0.522$\sstd{$0.001$} & $2.498$\sstd{$0.026$} & $1.230$\sstd{$0.009$} \\
\cmidrule{3-9}

\multirow{3}{*}{Rev-KL} & & GRU &$2.012$\sstd{$0.001$} & $102.509$\sstd{$0.008$} & \highlight{$0.183$\sstd{$0.002$}} & $0.132$\sstd{$0.002$} & \highlight{$0.471$\sstd{$0.010$}} & $0.413$\sstd{$0.019$} \\
& & DeepSets & \highlight{$2.011$\sstd{$0.001$}} & $102.599$\sstd{$0.042$} & $0.186$\sstd{$0.001$} & $0.127$\sstd{$0.002$} & $0.495$\sstd{$0.018$} & $0.409$\sstd{$0.005$} \\
& & Transformer & \highlight{$2.013$\sstd{$0.002$}} & $102.540$\sstd{$0.025$} & \highlight{$0.185$\sstd{$0.004$}} & \highlight{$0.122$\sstd{$0.001$}} & $0.489$\sstd{$0.019$} & \highlight{$0.328$\sstd{$0.002$}} \\
\cmidrule{2-9}

\multirow{3}{*}{Fwd-KL} & \multirow{6}{*}{\rotatebox[origin=c]{90}{Flow}} & GRU &$2.014$\sstd{$0.001$} & $102.656$\sstd{$0.019$} & \highlight{$0.186$\sstd{$0.006$}} & $0.242$\sstd{$0.005$} & $0.670$\sstd{$0.094$} & $0.563$\sstd{$0.018$} \\
& & DeepSets &$2.014$\sstd{$0.001$} & $103.340$\sstd{$0.029$} & \highlight{$0.185$\sstd{$0.006$}} & $0.237$\sstd{$0.008$} & $0.648$\sstd{$0.082$} & $0.583$\sstd{$0.028$} \\
& & Transformer &$2.016$\sstd{$0.002$} & $102.774$\sstd{$0.024$} & \highlight{$0.188$\sstd{$0.012$}} & $0.252$\sstd{$0.001$} & $0.621$\sstd{$0.070$} & $0.592$\sstd{$0.019$} \\
\cmidrule{3-9}

\multirow{3}{*}{Rev-KL} & & GRU &$2.013$\sstd{$0.001$} & \highlight{$102.490$\sstd{$0.012$}} & \highlight{$0.184$\sstd{$0.006$}} & $0.130$\sstd{$0.002$} & \highlight{$0.467$\sstd{$0.003$}} & $0.384$\sstd{$0.005$} \\
& & DeepSets & \highlight{$2.011$\sstd{$0.001$}} & $102.674$\sstd{$0.046$} & \highlight{$0.188$\sstd{$0.005$}} & $0.131$\sstd{$0.002$} & $0.519$\sstd{$0.008$} & $0.405$\sstd{$0.005$} \\
& & Transformer &$2.013$\sstd{$0.001$} & $102.525$\sstd{$0.050$} & \highlight{$0.187$\sstd{$0.004$}} & \highlight{$0.123$\sstd{$0.001$}} & \highlight{$0.468$\sstd{$0.007$}} & \highlight{$0.326$\sstd{$0.008$}} \\
\bottomrule
    \end{tabular}
    \caption{\textbf{Fixed-Dimensional}. Results for estimating the mean of a Gaussian (Gaussian) and means of a Gaussian mixture model (GMM) with the expected $L_2$ loss according to the posterior predictive as metric.}
    \vspace{-4mm}
    \label{tab:apdx_gaussian}
\end{table}

% \begin{table*}[t]
%     \centering
%     \small
%     % \footnotesize	    
%     \def\arraystretch{1.25}
%     \setlength{\tabcolsep}{5pt}
%     \begin{tabular}{lcr cc cccc}
%         \toprule
%          &  &  & \multicolumn{6}{c}{\textit{Conditional Negative Log Likelihood} ($\downarrow$)} \\
%         \cmidrule(lr){4-9}
%         \textbf{Objective} & $q_\varphi$ & \textbf{Model} & \textit{2D} & \textit{100D} & \textit{2D-2cl} & \textit{2D-5cl} & \textit{5D-2cl} & \textit{5D-5cl} \\
%         \midrule
% \multirow{4}{*}{Baseline} & - & Random & $442.0$\sstd{$0.8$} & $22609.3$\sstd{$15.6$} & $1103.7$\sstd{$8.1$} & $673.1$\sstd{$1.3$} & $3563.8$\sstd{$12.0$} & $2370.7$\sstd{$8.6$} \\
% & - & Optimization & $264.5$\sstd{$0.0$} & $13295.9$\sstd{$0.0$} & $106.9$\sstd{$0.0$} & $180.6$\sstd{$0.4$} & $193.4$\sstd{$0.0$} & $315.5$\sstd{$0.2$} \\
% & - & Langevin & $265.6$\sstd{$0.2$} & $13345.6$\sstd{$1.3$} & $109.3$\sstd{$0.5$} & $184.1$\sstd{$0.9$} & $208.7$\sstd{$6.5$} & $364.6$\sstd{$7.7$} \\
% & - & HMC & $265.8$\sstd{$0.4$} & $13347.4$\sstd{$1.1$} & $109.5$\sstd{$0.3$} & $193.8$\sstd{$0.6$} & $237.3$\sstd{$4.6$} & $458.2$\sstd{$3.5$} \\
% \cmidrule{2-9}

% \multirow{3}{*}{Fwd-KL} & \multirow{6}{*}{\rotatebox[origin=c]{90}{Gaussian}} & GRU &$265.6$\sstd{$0.1$} & $13358.9$\sstd{$0.5$} & $494.0$\sstd{$5.2$} & $512.8$\sstd{$1.4$} & $1550.8$\sstd{$6.6$} & $1780.5$\sstd{$7.7$} \\
%  & & DeepSets &$265.6$\sstd{$0.1$} & $13388.1$\sstd{$2.7$} & $493.2$\sstd{$7.2$} & $513.5$\sstd{$2.3$} & $1556.4$\sstd{$19.0$} & $1778.2$\sstd{$8.4$} \\
%  & & Transformer &$265.6$\sstd{$0.1$} & $13365.6$\sstd{$0.3$} & $500.1$\sstd{$8.2$} & $514.1$\sstd{$2.1$} & $1606.2$\sstd{$15.5$} & $1785.3$\sstd{$13.1$} \\
% \cmidrule{3-9}

% \multirow{3}{*}{Rev-KL} & & GRU &$265.5$\sstd{$0.1$} & $13352.8$\sstd{$0.3$} & $112.7$\sstd{$1.4$} & $192.2$\sstd{$1.5$} & $223.5$\sstd{$6.4$} & $449.9$\sstd{$12.8$} \\
%  & & DeepSets &$265.5$\sstd{$0.1$} & $13359.3$\sstd{$2.1$} & $114.3$\sstd{$0.6$} & $187.5$\sstd{$1.0$} & $233.1$\sstd{$10.4$} & $448.8$\sstd{$9.0$} \\
%  & & Transformer &$265.6$\sstd{$0.1$} & $13354.4$\sstd{$1.1$} & $115.2$\sstd{$1.7$} & $185.9$\sstd{$0.2$} & $230.2$\sstd{$10.9$} & $350.5$\sstd{$7.5$} \\
% \cmidrule{2-9}

% \multirow{3}{*}{Fwd-KL} & \multirow{6}{*}{\rotatebox[origin=c]{90}{Flow}} & GRU &$265.6$\sstd{$0.0$} & $13359.5$\sstd{$0.8$} & $114.3$\sstd{$2.2$} & $288.9$\sstd{$2.6$} & $912.9$\sstd{$159.2$} & $1541.3$\sstd{$2.5$} \\
%  & & DeepSets &$265.6$\sstd{$0.1$} & $13393.9$\sstd{$1.4$} & $114.6$\sstd{$2.7$} & $291.6$\sstd{$3.9$} & $1007.0$\sstd{$71.4$} & $1557.6$\sstd{$8.0$} \\
%  & & Transformer &$265.7$\sstd{$0.1$} & $13365.2$\sstd{$1.0$} & $115.4$\sstd{$5.3$} & $284.8$\sstd{$1.0$} & $503.7$\sstd{$197.0$} & $1523.8$\sstd{$24.6$} \\
% \cmidrule{3-9}

% \multirow{3}{*}{Rev-KL} & & GRU &$265.6$\sstd{$0.0$} & $13351.8$\sstd{$0.5$} & $113.7$\sstd{$2.4$} & $190.1$\sstd{$1.6$} & $217.7$\sstd{$2.5$} & $429.6$\sstd{$10.4$} \\
%  & & DeepSets &$265.5$\sstd{$0.0$} & $13362.5$\sstd{$2.5$} & $115.4$\sstd{$2.6$} & $189.8$\sstd{$1.1$} & $247.6$\sstd{$4.9$} & $439.8$\sstd{$8.8$} \\
%  & & Transformer &$265.6$\sstd{$0.0$} & $13353.6$\sstd{$2.2$} & $114.8$\sstd{$1.6$} & $185.6$\sstd{$0.5$} & $220.1$\sstd{$5.7$} & $343.4$\sstd{$7.9$} \\
%  \bottomrule
%     \end{tabular}
%     \caption{}
%     \vspace{-4mm}
%     \label{tab:}
% \end{table*}
\begin{table}[t]
  \centering
    \resizebox{\linewidth}{!}
    % \scalebox{0.8}
    {
    \begin{tabular}{l|c|cccccc}
    \toprule
    \multirow{2}[2]{*}{Method} & \multirow{2}[2]{*}{Proj.} & MMMU & \multicolumn{2}{c}{MMBench} & \multirow{2}[2]{*}{POPE}  & SQA & OK-  \\
        &  & VAL  & EN & CN     &             & IMG & VQA \\
    \midrule
    LLaVA-1.5 & MLP & 35.7  & 64.3 & 58.3  & 86.8  & 66.8 & 53.4  \\
    SAISA & Linear & 35.7 & 65.3 & 56.6   & 85.8  & 69.2 & 53.6  \\
    \rowcolor{cyan!20} SAISA & MLP & 36.9 & 65.7 & 59.0 & 87.2  & 70.1 & 56.8  \\
    \bottomrule
    \end{tabular}
    }
    \caption{\textbf{Ablation on Projector Designs.} ``Proj." denotes projector type.
    The SAISA model that uses MLPs outperforms the model that uses linear layers.
    Notably, SAISA with linear layers achieves comparable performance to LLaVA-1.5 with MLP.
    }
  \label{tab:linear}
\end{table}

\begin{table*}[t]
    \centering
    \small
    % \footnotesize	    
    \def\arraystretch{1.25}
    \setlength{\tabcolsep}{5pt}
    \begin{tabular}{lcr cccc}
        \toprule
         &  &  & \multicolumn{4}{c}{\textit{$L_2$ Loss} ($\downarrow$)} \\
         \cmidrule(lr){4-7}
         \textbf{Objective} & $q_\varphi$ & \textbf{Model} & \multicolumn{4}{c}{\textbf{Nonlinear Regression} $|$ \textbf{ReLU}} \\
         \cmidrule(lr){4-7}
         & & & \multicolumn{2}{c}{\textit{1-layer}} & \multicolumn{2}{c}{\textit{2-layers}} \\
         \cmidrule(lr){4-5}\cmidrule(lr){6-7}
         & & & \textit{1D} & \textit{25D} & \textit{1D} & \textit{25D} \\
         \midrule
\multirow{4}{*}{Baseline} & - & Random & $65.936$\sstd{$0.913$} & $831.595$\sstd{$8.696$} & $1029.407$\sstd{$11.542$} & $12067.691$\sstd{$183.598$} \\
& - & Optimization & $0.360$\sstd{$0.001$} & $103.967$\sstd{$0.110$} & $2.370$\sstd{$0.015$} & $1894.574$\sstd{$4.266$} \\
& - & Langevin & $0.308$\sstd{$0.000$} & $132.391$\sstd{$0.992$} & \textsc{N/A} & \textsc{N/A} \\
& - & HMC & $0.374$\sstd{$0.002$} & $98.061$\sstd{$0.730$} & $22.314$\sstd{$0.814$} & $3903.510$\sstd{$5.377$} \\
\cmidrule{2-7}

\multirow{3}{*}{Fwd-KL} & \multirow{6}{*}{\rotatebox[origin=c]{90}{Gaussian}} & GRU &$49.332$\sstd{$0.946$} & $671.639$\sstd{$10.494$} & $774.045$\sstd{$7.521$} & $9905.246$\sstd{$214.545$} \\
& & DeepSets &$49.864$\sstd{$0.979$} & $684.853$\sstd{$2.581$} & $768.921$\sstd{$8.278$} & $9946.090$\sstd{$109.933$} \\
& & Transformer &$49.678$\sstd{$0.940$} & $680.853$\sstd{$5.838$} & $747.221$\sstd{$12.189$} & $9982.609$\sstd{$85.596$} \\
\cmidrule{3-7}

\multirow{3}{*}{Rev-KL} & & GRU &$0.426$\sstd{$0.004$} & $105.976$\sstd{$0.586$} & $1.066$\sstd{$0.069$} & $1796.512$\sstd{$5.805$} \\
& & DeepSets &$0.426$\sstd{$0.004$} & $125.853$\sstd{$0.791$} & $1.394$\sstd{$0.108$} & $1892.402$\sstd{$2.793$} \\
& & Transformer &$0.417$\sstd{$0.005$} & \highlight{$102.295$\sstd{$1.825$}} & $2.075$\sstd{$0.147$} & \highlight{$1811.440$\sstd{$115.435$}} \\
\cmidrule{2-7}

\multirow{3}{*}{Fwd-KL} & \multirow{6}{*}{\rotatebox[origin=c]{90}{Flow}} & GRU &$15.781$\sstd{$0.210$} & $538.962$\sstd{$4.269$} & $614.925$\sstd{$15.494$} & $7564.076$\sstd{$67.160$} \\
& & DeepSets &$15.051$\sstd{$0.120$} & $548.535$\sstd{$3.288$} & $622.461$\sstd{$7.043$} & $7618.364$\sstd{$115.946$} \\
& & Transformer &$16.109$\sstd{$0.307$} & $539.338$\sstd{$4.336$} & $597.718$\sstd{$8.358$} & $7635.052$\sstd{$109.037$} \\
\cmidrule{3-7}

\multirow{3}{*}{Rev-KL} & & GRU &$0.405$\sstd{$0.010$} & $106.001$\sstd{$0.420$} & \highlight{$0.988$\sstd{$0.045$}} & $1814.649$\sstd{$8.327$} \\
& & DeepSets &$0.395$\sstd{$0.004$} & $128.169$\sstd{$1.451$} & $1.215$\sstd{$0.028$} & $1886.698$\sstd{$7.294$} \\
& & Transformer & \highlight{$0.387$\sstd{$0.004$}} & \highlight{$102.610$\sstd{$0.863$}} & $2.549$\sstd{$0.058$} & \highlight{$1791.741$\sstd{$49.585$}} \\
\bottomrule
    \end{tabular}
    \caption{\textbf{Fixed-Dimensional}. Results for estimating the parameters of nonlinear regression models with ReLU activation function, with the expected $L_2$ loss according to the posterior predictive as metric.}
    \vspace{-4mm}
    \label{tab:}
\end{table*}
\begin{table*}[t]
    \centering
    \small
    % \footnotesize	    
    \def\arraystretch{1.25}
    \setlength{\tabcolsep}{5pt}
    \begin{tabular}{lcr cccc}
        \toprule
         &  &  & \multicolumn{4}{c}{\textit{$L_2$ Loss} ($\downarrow$)} \\
         \cmidrule(lr){4-7}
         \textbf{Objective} & $q_\varphi$ & \textbf{Model} & \multicolumn{4}{c}{\textbf{Nonlinear Regression} $|$ \textbf{TanH}} \\
         \cmidrule(lr){4-7}
         & & & \multicolumn{2}{c}{\textit{1-layer}} & \multicolumn{2}{c}{\textit{2-layers}} \\
         \cmidrule(lr){4-5}\cmidrule(lr){6-7}
         & & & \textit{1D} & \textit{25D} & \textit{1D} & \textit{25D} \\
         \midrule

\multirow{4}{*}{Baseline} & - & Random & $31.448$\sstd{$0.186$} & $52.644$\sstd{$0.173$} & $52.735$\sstd{$1.122$} & $52.583$\sstd{$0.132$} \\
& - & Optimization & $0.366$\sstd{$0.001$} & $13.352$\sstd{$0.005$} & $0.651$\sstd{$0.002$} & $30.176$\sstd{$0.056$} \\
& - & Langevin & $0.296$\sstd{$0.003$} & $17.221$\sstd{$0.130$} & $0.363$\sstd{$0.003$} & $28.528$\sstd{$0.115$} \\
& - & HMC & $0.398$\sstd{$0.003$} & $12.607$\sstd{$0.192$} & $0.733$\sstd{$0.021$} & $23.571$\sstd{$0.346$} \\
\cmidrule{2-7}

\multirow{3}{*}{Fwd-KL} & \multirow{6}{*}{\rotatebox[origin=c]{90}{Gaussian}} & GRU &$31.391$\sstd{$0.161$} & $52.008$\sstd{$0.282$} & $52.725$\sstd{$1.149$} & $51.989$\sstd{$0.139$} \\
& & DeepSets &$31.421$\sstd{$0.074$} & $52.137$\sstd{$0.215$} & $52.850$\sstd{$1.192$} & $51.904$\sstd{$0.334$} \\
& & Transformer &$31.350$\sstd{$0.219$} & $52.945$\sstd{$0.430$} & $52.693$\sstd{$1.188$} & $52.364$\sstd{$0.164$} \\
\cmidrule{3-7}

\multirow{3}{*}{Rev-KL} & & GRU &$0.415$\sstd{$0.003$} & $15.874$\sstd{$6.958$} & $0.951$\sstd{$0.047$} & $25.907$\sstd{$0.012$} \\
& & DeepSets &$0.405$\sstd{$0.004$} & $25.333$\sstd{$0.010$} & $0.912$\sstd{$0.013$} & $25.877$\sstd{$0.002$} \\
& & Transformer &$0.412$\sstd{$0.013$} & $11.784$\sstd{$0.949$} & $0.847$\sstd{$0.010$} & $20.405$\sstd{$3.874$} \\
\cmidrule{2-7}

\multirow{3}{*}{Fwd-KL} & \multirow{6}{*}{\rotatebox[origin=c]{90}{Flow}} & GRU &$12.415$\sstd{$0.800$} & $52.039$\sstd{$0.065$} & $52.695$\sstd{$0.611$} & $52.576$\sstd{$0.225$} \\
& & DeepSets &$31.790$\sstd{$0.163$} & $51.933$\sstd{$0.115$} & $52.903$\sstd{$0.625$} & $52.643$\sstd{$0.239$} \\
& & Transformer &$10.392$\sstd{$0.195$} & $52.470$\sstd{$0.364$} & $52.385$\sstd{$0.689$} & $52.646$\sstd{$0.622$} \\
\cmidrule{3-7}

\multirow{3}{*}{Rev-KL} & & GRU &$0.386$\sstd{$0.005$} & $11.401$\sstd{$0.041$} & $0.736$\sstd{$0.009$} & \highlight{$25.892$\sstd{$0.010$}} \\
& & DeepSets & \highlight{$0.374$\sstd{$0.005$}} & $25.685$\sstd{$0.004$} & \highlight{$0.686$\sstd{$0.019$}} & \highlight{$25.885$\sstd{$0.007$}} \\
& & Transformer & \highlight{$0.376$\sstd{$0.002$}} & \highlight{$10.486$\sstd{$0.040$}} & $0.724$\sstd{$0.026$} & \highlight{$25.885$\sstd{$0.011$}} \\
\bottomrule
    \end{tabular}
    \caption{\textbf{Fixed-Dimensional}. Results for estimating the parameters of nonlinear regression models with TanH activation function, with the expected $L_2$ loss according to the posterior predictive as metric.}
    \vspace{-4mm}
    \label{tab:}
\end{table*}
\begin{table*}[t]
    \centering
    \small
    % \footnotesize	    
    \def\arraystretch{1.25}
    \setlength{\tabcolsep}{5pt}
    \begin{tabular}{lcr cc cc}
        \toprule
         &  &  & \multicolumn{4}{c}{\textit{Accuracy} ($\uparrow$)} \\
         \cmidrule(lr){4-7}
         & & & \multicolumn{2}{c}{\textit{1-layer}} & \multicolumn{2}{c}{\textit{2-layers}} \\
         \cmidrule(lr){4-5}\cmidrule(lr){6-7}
         \textbf{Objective} & $q_\varphi$ & \textbf{Model} & \textit{2D} & \textit{50D} & \textit{2D} & \textit{50D} \\
         \midrule
\multirow{4}{*}{Baseline} & - & Random & $49.951$\sstd{$0.287$} & $49.904$\sstd{$0.281$} & $50.040$\sstd{$0.467$} & $50.044$\sstd{$0.239$} \\
& - & Optimization & $96.762$\sstd{$0.034$} & $76.139$\sstd{$0.033$} & $96.810$\sstd{$0.009$} & $78.225$\sstd{$0.129$} \\
& - & Langevin & $96.077$\sstd{$0.027$} & $70.142$\sstd{$0.229$} & $96.564$\sstd{$0.113$} & $71.328$\sstd{$0.410$} \\
& - & HMC & $91.734$\sstd{$0.152$} & $67.986$\sstd{$0.372$} & $91.336$\sstd{$0.591$} & $71.825$\sstd{$0.507$} \\
\cmidrule{2-7}

\multirow{3}{*}{Fwd-KL} & \multirow{6}{*}{\rotatebox[origin=c]{90}{Gaussian}} & GRU &$59.551$\sstd{$0.199$} & $58.637$\sstd{$0.250$} & $60.247$\sstd{$0.645$} & $58.862$\sstd{$0.065$} \\
& & DeepSets &$49.946$\sstd{$0.285$} & $49.910$\sstd{$0.282$} & $50.032$\sstd{$0.466$} & $50.040$\sstd{$0.242$} \\
& & Transformer &$59.887$\sstd{$0.235$} & $58.826$\sstd{$0.237$} & $60.552$\sstd{$0.398$} & $58.953$\sstd{$0.110$} \\
\cmidrule{3-7}

\multirow{3}{*}{Rev-KL} & & GRU  &$88.822$\sstd{$0.471$} & $68.368$\sstd{$0.342$} & $81.884$\sstd{$1.450$} & $67.264$\sstd{$0.138$} \\
& & DeepSets & \highlight{$91.019$\sstd{$0.454$}} & $61.732$\sstd{$0.111$} & $82.396$\sstd{$0.471$} & $67.320$\sstd{$0.143$} \\
& & Transformer &$89.988$\sstd{$0.197$} & \highlight{$73.744$\sstd{$0.319$}} & \highlight{$83.399$\sstd{$0.841$}} & $67.167$\sstd{$0.028$} \\
\cmidrule{2-7}

\multirow{3}{*}{Fwd-KL} & \multirow{6}{*}{\rotatebox[origin=c]{90}{Flow}} & GRU  &$61.179$\sstd{$0.833$} & $60.225$\sstd{$0.115$} & $60.400$\sstd{$1.019$} & $59.027$\sstd{$0.212$} \\
& & DeepSets &$49.568$\sstd{$0.230$} & $50.130$\sstd{$0.101$} & $50.356$\sstd{$0.773$} & $49.806$\sstd{$0.331$} \\
& & Transformer &$60.886$\sstd{$0.252$} & $60.253$\sstd{$0.082$} & $61.694$\sstd{$0.314$} & $60.426$\sstd{$0.203$} \\
\cmidrule{3-7}

\multirow{3}{*}{Rev-KL} & & GRU &$90.363$\sstd{$0.709$} & $66.197$\sstd{$0.118$} & \highlight{$83.443$\sstd{$0.619$}} & \highlight{$69.053$\sstd{$0.256$}} \\
& & DeepSets &$89.150$\sstd{$0.338$} & $62.939$\sstd{$0.112$} & $79.889$\sstd{$0.567$} & \highlight{$69.015$\sstd{$0.147$}} \\
& & Transformer & \highlight{$91.065$\sstd{$0.156$}} & $72.581$\sstd{$0.117$} & \highlight{$83.533$\sstd{$0.677$}} & \highlight{$68.933$\sstd{$0.120$}} \\
\bottomrule
    \end{tabular}
    \caption{\textbf{Variable-Dimensional}. Results for estimating the parameters of nonlinear classification models with ReLU activation function and two classes, with the expected accuracy according to the posterior predictive as metric.}
    \vspace{-4mm}
    \label{tab:}
\end{table*}
\begin{table*}[t]
    \centering
    \small
    % \footnotesize	    
    \def\arraystretch{1.25}
    \setlength{\tabcolsep}{5pt}
    \begin{tabular}{lcr cc cc}
        \toprule
         &  &  & \multicolumn{4}{c}{\textit{Accuracy} ($\uparrow$)} \\
         \cmidrule(lr){4-7}
         & & & \multicolumn{2}{c}{\textit{1-layer}} & \multicolumn{2}{c}{\textit{2-layers}} \\
         \cmidrule(lr){4-5}\cmidrule(lr){6-7}
         \textbf{Objective} & $q_\varphi$ & \textbf{Model} & \textit{2D} & \textit{50D} & \textit{2D} & \textit{50D} \\
         \midrule
\multirow{4}{*}{Baseline} & - & Random & $19.847$\sstd{$0.352$} & $20.052$\sstd{$0.066$} & $19.874$\sstd{$0.222$} & $20.028$\sstd{$0.106$} \\
& - & Optimization & $94.607$\sstd{$0.011$} & $56.091$\sstd{$0.158$} & $93.873$\sstd{$0.028$} & $60.253$\sstd{$0.053$} \\
& - & Langevin & $90.815$\sstd{$0.341$} & $46.072$\sstd{$0.225$} & $91.849$\sstd{$0.088$} & $49.808$\sstd{$0.337$} \\
& - & HMC & $81.145$\sstd{$0.303$} & $44.561$\sstd{$0.309$} & $79.559$\sstd{$0.483$} & $50.967$\sstd{$0.436$} \\
\cmidrule{2-7}

\multirow{3}{*}{Fwd-KL} & \multirow{6}{*}{\rotatebox[origin=c]{90}{Gaussian}} & GRU &$30.960$\sstd{$0.638$} & $32.164$\sstd{$0.151$} & $31.017$\sstd{$0.397$} & $32.224$\sstd{$0.108$} \\
& & DeepSets &$19.846$\sstd{$0.348$} & $20.053$\sstd{$0.063$} & $19.871$\sstd{$0.226$} & $20.032$\sstd{$0.104$} \\
& & Transformer &$30.652$\sstd{$0.401$} & $32.208$\sstd{$0.178$} & $31.148$\sstd{$0.429$} & $32.342$\sstd{$0.217$} \\
\cmidrule{3-7}

\multirow{3}{*}{Rev-KL} & & GRU &$72.874$\sstd{$0.113$} & $37.987$\sstd{$0.119$} & $56.999$\sstd{$0.599$} & $29.971$\sstd{$0.248$} \\
& & DeepSets &$69.456$\sstd{$0.370$} & $36.712$\sstd{$0.249$} & $55.193$\sstd{$0.538$} & $36.417$\sstd{$9.779$} \\
& & Transformer &$73.531$\sstd{$0.391$} & \highlight{$44.702$\sstd{$0.165$}} & $57.724$\sstd{$0.332$} & $30.175$\sstd{$0.177$} \\
\cmidrule{2-7}

\multirow{3}{*}{Fwd-KL} & \multirow{6}{*}{\rotatebox[origin=c]{90}{Flow}} & GRU &$33.232$\sstd{$0.607$} & $33.704$\sstd{$0.026$} & $31.937$\sstd{$0.483$} & $32.370$\sstd{$0.284$} \\
& & DeepSets &$19.898$\sstd{$0.156$} & $19.950$\sstd{$0.256$} & $20.062$\sstd{$0.347$} & $20.064$\sstd{$0.207$} \\
& & Transformer &$32.916$\sstd{$0.194$} & $33.766$\sstd{$0.137$} & $32.374$\sstd{$0.301$} & $33.846$\sstd{$0.486$} \\
\cmidrule{3-7}

\multirow{3}{*}{Rev-KL} & & GRU & \highlight{$77.997$\sstd{$0.663$}} & $38.715$\sstd{$0.153$} & \highlight{$61.947$\sstd{$0.294$}} & \highlight{$51.962$\sstd{$0.812$}} \\
& & DeepSets &$68.957$\sstd{$0.551$} & $37.123$\sstd{$0.108$} & $51.145$\sstd{$14.201$} & $42.707$\sstd{$12.882$} \\
& & Transformer & \highlight{$77.867$\sstd{$2.241$}} & \highlight{$44.156$\sstd{$0.485$}} & $57.410$\sstd{$0.088$} & \highlight{$52.077$\sstd{$0.077$}} \\
\bottomrule
    \end{tabular}
    \caption{\textbf{Fixed-Dimensional}. Results for estimating the parameters of nonlinear classification models with ReLU activation function and five classes, with the expected accuracy according to the posterior predictive as metric.}
    \vspace{-4mm}
    \label{tab:}
\end{table*}
\begin{table*}[t]
    \centering
    \small
    % \footnotesize	    
    \def\arraystretch{1.25}
    \setlength{\tabcolsep}{5pt}
    \begin{tabular}{lcr cc cc}
        \toprule
         &  &  & \multicolumn{4}{c}{\textit{Accuracy} ($\uparrow$)} \\
         \cmidrule(lr){4-7}
         & & & \multicolumn{2}{c}{\textit{1-layer}} & \multicolumn{2}{c}{\textit{2-layers}} \\
         \cmidrule(lr){4-5}\cmidrule(lr){6-7}
         \textbf{Objective} & $q_\varphi$ & \textbf{Model} & \textit{2D} & \textit{50D} & \textit{2D} & \textit{50D} \\
         \midrule
\multirow{4}{*}{Baseline} & - & Random & $50.278$\sstd{$0.337$} & $50.028$\sstd{$0.064$} & $50.188$\sstd{$0.479$} & $49.982$\sstd{$0.084$} \\
& - & Optimization & $96.943$\sstd{$0.012$} & $68.086$\sstd{$0.016$} & $94.444$\sstd{$0.024$} & $63.950$\sstd{$0.022$} \\
& - & Langevin & $95.143$\sstd{$0.094$} & $61.694$\sstd{$0.340$} & $92.719$\sstd{$0.016$} & $57.447$\sstd{$0.437$} \\
& - & HMC & $92.489$\sstd{$0.338$} & $59.963$\sstd{$0.202$} & $87.548$\sstd{$0.094$} & $56.319$\sstd{$0.882$} \\
\cmidrule{2-7}

\multirow{3}{*}{Fwd-KL} & \multirow{6}{*}{\rotatebox[origin=c]{90}{Gaussian}} & GRU &$50.274$\sstd{$0.337$} & $50.023$\sstd{$0.060$} & $50.187$\sstd{$0.477$} & $49.992$\sstd{$0.072$} \\
& & DeepSets &$50.271$\sstd{$0.334$} & $50.024$\sstd{$0.061$} & $50.188$\sstd{$0.472$} & $49.984$\sstd{$0.077$} \\
& & Transformer &$50.273$\sstd{$0.336$} & $50.031$\sstd{$0.066$} & $50.191$\sstd{$0.471$} & $49.994$\sstd{$0.078$} \\
\cmidrule{3-7}

\multirow{3}{*}{Rev-KL} & & GRU &$89.270$\sstd{$0.272$} & $50.014$\sstd{$0.048$} & $50.191$\sstd{$0.463$} & $49.995$\sstd{$0.080$} \\
& & DeepSets & \highlight{$89.788$\sstd{$0.213$}} & $50.018$\sstd{$0.061$} & $50.191$\sstd{$0.478$} & $49.977$\sstd{$0.075$} \\
& & Transformer &$89.366$\sstd{$0.108$} & $64.926$\sstd{$0.260$} & $50.182$\sstd{$0.469$} & $49.986$\sstd{$0.071$} \\
\cmidrule{2-7}

\multirow{3}{*}{Fwd-KL} & \multirow{6}{*}{\rotatebox[origin=c]{90}{Flow}} & GRU &$49.651$\sstd{$0.040$} & $50.119$\sstd{$0.068$} & $49.987$\sstd{$0.015$} & $49.904$\sstd{$0.018$} \\
& & DeepSets &$49.639$\sstd{$0.031$} & $50.113$\sstd{$0.065$} & $49.988$\sstd{$0.022$} & $49.910$\sstd{$0.043$} \\
& & Transformer &$49.636$\sstd{$0.040$} & $50.115$\sstd{$0.063$} & $49.989$\sstd{$0.017$} & $49.909$\sstd{$0.042$} \\
\cmidrule{3-7}

\multirow{3}{*}{Rev-KL} & & GRU &$49.769$\sstd{$0.141$} & $50.082$\sstd{$0.091$} & $49.915$\sstd{$0.070$} & $50.004$\sstd{$0.084$} \\
& & DeepSets &$49.782$\sstd{$0.073$} & $50.080$\sstd{$0.087$} & $49.831$\sstd{$0.152$} & $49.994$\sstd{$0.078$} \\
& & Transformer &$63.233$\sstd{$19.243$} & $50.026$\sstd{$0.047$} & $49.869$\sstd{$0.207$} & $50.036$\sstd{$0.056$} \\
\bottomrule
    \end{tabular}
    \caption{\textbf{Variable-Dimensional}. Results for estimating the parameters of nonlinear classification models with TanH activation function and two classes, with the expected accuracy according to the posterior predictive as metric.}
    \vspace{-4mm}
    \label{tab:}
\end{table*}
\begin{table*}[t]
    \centering
    \small
    % \footnotesize	    
    \def\arraystretch{1.25}
    \setlength{\tabcolsep}{5pt}
    \begin{tabular}{lcr cc cc}
        \toprule
         &  &  & \multicolumn{4}{c}{\textit{Accuracy} ($\uparrow$)} \\
         \cmidrule(lr){4-7}
         \textbf{Objective} & $q_\varphi$ & \textbf{Model} & \multicolumn{4}{c}{\textbf{Nonlinear Classification} $|$ \textbf{TanH - 5 class}} \\
         \cmidrule(lr){4-7}
         & & & \multicolumn{2}{c}{\textit{1-layer}} & \multicolumn{2}{c}{\textit{2-layers}} \\
         \cmidrule(lr){4-5}\cmidrule(lr){6-7}
         & & & \textit{2D} & \textit{25D} & \textit{2D} & \textit{25D} \\
         \midrule
\multirow{4}{*}{Baseline} & - & Random & $19.745$\sstd{$0.269$} & $20.037$\sstd{$0.075$} & $20.214$\sstd{$0.043$} & $19.864$\sstd{$0.089$} \\
& - & Optimization & $92.919$\sstd{$0.011$} & $49.972$\sstd{$0.070$} & $88.156$\sstd{$0.014$} & $39.410$\sstd{$0.043$} \\
& - & Langevin & $88.635$\sstd{$0.374$} & $39.758$\sstd{$0.212$} & $84.050$\sstd{$0.138$} & $31.472$\sstd{$0.010$} \\
& - & HMC & $81.057$\sstd{$0.277$} & $35.378$\sstd{$0.130$} & $75.305$\sstd{$0.134$} & $29.730$\sstd{$0.670$} \\
\cmidrule{2-7}

\multirow{3}{*}{Fwd-KL} & \multirow{6}{*}{\rotatebox[origin=c]{90}{Gaussian}} & GRU &$19.960$\sstd{$0.271$} & $20.235$\sstd{$0.083$} & $20.452$\sstd{$0.059$} & $20.028$\sstd{$0.077$} \\
& & DeepSets &$19.807$\sstd{$0.209$} & $20.040$\sstd{$0.072$} & $20.210$\sstd{$0.043$} & $19.861$\sstd{$0.094$} \\
& & Transformer &$19.977$\sstd{$0.273$} & $20.241$\sstd{$0.068$} & $20.453$\sstd{$0.062$} & $20.029$\sstd{$0.082$} \\
\cmidrule{3-7}

\multirow{3}{*}{Rev-KL} & & GRU &$77.711$\sstd{$0.014$} & $20.026$\sstd{$0.079$} & $20.213$\sstd{$0.043$} & $19.877$\sstd{$0.092$} \\
& & DeepSets &$76.414$\sstd{$0.378$} & $20.038$\sstd{$0.060$} & $20.216$\sstd{$0.027$} & $19.887$\sstd{$0.089$} \\
& & Transformer &$79.163$\sstd{$0.183$} & $20.026$\sstd{$0.093$} & $51.408$\sstd{$0.484$} & $19.872$\sstd{$0.083$} \\
\cmidrule{2-7}

\multirow{3}{*}{Fwd-KL} & \multirow{6}{*}{\rotatebox[origin=c]{90}{Flow}} & GRU &$32.900$\sstd{$0.115$} & $20.209$\sstd{$0.048$} & $20.105$\sstd{$0.385$} & $20.040$\sstd{$0.037$} \\
& & DeepSets &$20.137$\sstd{$0.070$} & $20.000$\sstd{$0.048$} & $19.887$\sstd{$0.392$} & $19.895$\sstd{$0.025$} \\
& & Transformer &$30.135$\sstd{$2.459$} & $20.224$\sstd{$0.048$} & $20.104$\sstd{$0.399$} & $20.030$\sstd{$0.039$} \\
\cmidrule{3-7}

\multirow{3}{*}{Rev-KL} & & GRU &$79.329$\sstd{$0.320$} & $20.074$\sstd{$0.033$} & $19.904$\sstd{$0.180$} & $19.864$\sstd{$0.079$} \\
& & DeepSets &$20.071$\sstd{$0.302$} & $19.999$\sstd{$0.045$} & $19.872$\sstd{$0.259$} & $19.906$\sstd{$0.051$} \\
& & Transformer & \highlight{$80.064$\sstd{$0.159$}} & $20.002$\sstd{$0.032$} & $19.777$\sstd{$0.210$} & $19.911$\sstd{$0.082$} \\
\bottomrule
    \end{tabular}
    \caption{\textbf{Fixed-Dimensional}. Results for estimating the parameters of nonlinear classification models with TanH activation function and five classes, with the expected accuracy according to the posterior predictive as metric.}
    \vspace{-4mm}
    \label{tab:apdx_nlc_5cl}
\end{table*}
\textbf{Linear Regression (LR):} We then look at the problem of estimating the posterior over the weight vector for Bayesian linear regression given a dataset $p(\vw, b | \gD)$, where the underlying model $p(\gD, \mtheta)$ is given by:
\begin{align}
    p(\vw) &= \gN(\vw | \mathbf{0}, \mathbf{I})\\
    p(b) &= \gN(b | 0, 1)\\
    p(y | \vx, \vw, b) &= \gN\left(y | \vw^T\vx + b, \sigma^2\right) \, ,
\end{align}
and with $\sigma^2 = 0.25$ known beforehand. Inputs $\vx$ are generated from $p(\vx) = \gN(\mathbf{0}, I)$.


\textbf{Linear Classification (LC):}
We now consider a setting where the true posterior cannot be obtained analytically as the likelihood and prior are not conjugate. In this case, we consider the underlying probabilistic model by:
\begin{align}
    p(\mW) &= \gN\left(\mW | \mathbf{0}, \mathbf{I}\right)\\
    p(y | \vx, \mW) &= \mathrm{Categorical}\left(y  \;\vline\; \frac{1}{\tau}\;\mW\vx\right)\, ,
\end{align}
where $\tau$ is the known temperature term which is kept as $0.1$ to ensure peaky distributions, and $\vx$ is being generated from $p(\vx) = \gN(\mathbf{0}, I)$.


\textbf{Nonlinear Regression (NLR):}
Next, we tackle the more complex problem where the posterior distribution is multi-modal and obtaining multiple modes or even a single good one is challenging. For this, we consider the model as a Bayesian Neural Network (BNN) for regression with fixed hyper-parameters like the number of layers, dimensionality of the hidden layer, etc. Let the BNN denote the function $f_\mtheta$ where $\mtheta$ are the network parameters such that the estimation problem is to approximate $p(\mtheta | \gD)$. Then, for regression, we specify the probabilistic model using:
\begin{align}
    p(\mtheta) &= \gN\left(\mtheta | \mathbf{0}, \mathbf{I}\right)\\
    p(y | \vx, \mtheta) &= \gN\left(y | f_\mtheta(\vx), \sigma^2\right) \, ,
\end{align}
where $\sigma^2 = 0.25$ is a known quantity and $\vx$ being generated from $p(\vx) = \gN(\mathbf{0}, I)$.
 
\textbf{Nonlinear Classification (NLC):}
Like in Nonlinear Regression, we consider BNNs with fixed hyper-parameters for classification problems with the same estimation task of approximating $p(\mtheta | \gD)$. In this formulation, we consider the probabilistic model as:
\begin{align}
    p(\mtheta) &= \gN\left(\mtheta | \mathbf{0}, \mathbf{I}\right)\\
    p(y | \vx, \mtheta) &= \mathrm{Categorical}\left(y \;\vline\; \frac{1}{\tau}\;f_\mtheta(\vx)\right)
\end{align}
where $\tau$ is the known temperature term which is kept as $0.1$ to ensure peaky distributions, and $\vx$ is being generated from $p(\vx) = \gN(\mathbf{0}, I)$.

\textbf{Gaussian Mixture Model (GMM):}
While we have mostly looked at predictive problems, where the task is to model some predictive variable $y$ conditioned on some input $\vx$, we now look at a well-known probabilistic model for unsupervised learning, Gaussian Mixture Model (GMM), primarily used to cluster data. Consider a $K$-cluster GMM with:
\begin{align}
    p(\mmu_k) &= \gN\left(\mmu_k | \mathbf{0}, \mathbf{I}\right)\\
    p(\vx | \mmu_{1:K}) &= \sum_{k=1}^K \pi_k \gN\left(\vx | \mmu_k, \mSigma_k\right) \, .
\end{align}
 We assume $\mSigma_k$ and $\pi_k$ to be known and set $\mSigma_k$ to be an identity matrix and the mixing coefficients to be equal, $\pi_k = 1/K$, for all clusters $k$ in our experiments. 
\newcommand{\ResultEightByEightCrossbarOverheadkGE}{13.1}
\newcommand{\ResultEightByEightCrossbarOverheadPercent}{9}
\newcommand{\ResultSixteenBySixteenCrossbarOverheadkGE}{45.4}
\newcommand{\ResultSixteenBySixteenCrossbarOverheadPercent}{12}
\newcommand{\ResultAsymptoticOverheadPercent}{21.6}
\newcommand{\ResultSixteenBySixteenCrossbarFrequencyOverheadPercent}{6}
\newcommand{\ResultThirtyTwoClusterEightKiBParallelFraction}{97}
\newcommand{\ResultThirtyTwoClusterTwoKiBSpeedup}{13.5}
\newcommand{\ResultThirtyTwoClusterThirtyTwoKiBSpeedup}{16.2}
\newcommand{\ResultThirtyTwoClusterGeometricMeanSpeedup}{5.6}
\newcommand{\ResultBaselineTileNOperationalIntensity}{1.9}
\newcommand{\ResultBaselineTileNPerformanceGFLOPS}{114.4}
\newcommand{\ResultBaselineTileNPerformancePercentage}{92}
\newcommand{\ResultHybridTileNOperationalIntensityIncrease}{3.7}
\newcommand{\ResultHybridTileNPerformanceIncrease}{2.6}
\newcommand{\ResultMulticastTileNOperationalIntensityIncrease}{16.5}
\newcommand{\ResultMulticastTileNPerformanceIncrease}{3.4}
\newcommand{\ResultMulticastTileNPerformanceIncreaseOverHybridPercentage}{29}
\newcommand{\ResultMulticastTileNPerformanceGFLOPS}{391.4}

\section{Architecture Details}
\label{appdx:architecture}
In this section, we outline the two candidate architectures that we consider for the backbone of our amortized variational inference model. We discuss the specifics of the architectures and the hyperparameters used for our experiments.

\subsection{Transformer}
\label{subsec:transformer}
We use a transformer model~\citep{vaswani2017attention} as a permutation invariant architecture by removing positional encodings from the setup and using multiple layers of the encoder model. We append the set of observations with a [CLS] token before passing it to the model and use its output embedding to predict the parameters of the variational distribution. Since no positional encodings or causal masking is used in the whole setup, the final embedding of the [CLS] token becomes invariant to permutations in the set of observations, thereby leading to permutation invariance in the parameters of $q_\varphi$.

We use $4$ encoder layers with a $256$ dimensional attention block and $1024$ feed-forward dimensions, with $4$ heads in each attention block for our Transformer models to make the number of parameters comparative to the one of the DeepSets model.

\subsection{DeepSets}
\label{subsec:deepsets}
Another framework that can process set-based input is Deep Sets~\citep{Zaheer2017deepsets}. In our experiments, we used an embedding network that encodes the input into representation space, a mean aggregation operation, which ensures that the representation learned is invariant concerning the set ordering, and a regression network. The latter's output is either used to directly parameterize a diagonal Gaussian or as conditional input to a normalizing flow, representing a summary statistics of the set input.

For DeepSets, we use $4$ layers each in the embedding network and the regression network, with a mean aggregation function, ReLU activation functions, and $627$ hidden dimensions to make the number of parameters comparable to those in the Transformer model.

\subsection{RNN}
For the recurrent neural network setup, we use the Gated Recurrent Unit (GRU). Similar to the above setups, we use a $4$-layered GRU model with $256$ hidden dimensions. While such an architecture is not permutation invariant, by training on tasks that require such invariance could encourage learning of solution structure that respects this invariance.

\subsection{Normalizing Flows}
\label{subsec:flows}
Assuming a Gaussian posterior distribution as the approximate often leads to poor results as the true posterior distribution can be far from the Gaussian shape. To allow for more flexible posterior distributions, we use normalizing flows~\citep{kingma2018glow,kobyzev2020normalizing,papamakarios2021normalizing,rezende2015variational} for approximating $q_\varphi(\mtheta | \gD)$ conditioned on the output of the summary network $h_\psi$. Specifically, let $g_\nu: \vz \mapsto \mtheta$ be a diffeomorphism parameterized by a conditional invertible neural network (cINN) with network parameters $\nu$ such that $\mtheta = g_\nu(\vz; h_\psi(\gD))$. With the change-of-variables formula it follows that $p(\mtheta)=p(\vz)\left|\det \frac{\partial}{\partial\vz}g_\nu(\vz; h_\psi(\gD))\right|^{-1} = p(\vz)|\det J_\nu(\vz; h_\psi(\gD))|^{-1}$, where $J_\nu$ is the Jacobian matrix of $g_\nu$. Further, integration by substitution gives us $d\mtheta = |\det J_\nu(\vz; h_\psi(\gD)| d\vz$ to rewrite the objective from eq. \ref{eq:arkl} as:
\begin{align}
    &\argmin_\varphi \sE_{\gD \sim \chi} \sK\sL[q_\varphi(\mtheta|\gD) || p(\mtheta|\gD)]\\
    &= \argmin_\varphi \sE_{\gD \sim \chi} \sE_{\mtheta \sim q_\varphi(\mtheta|\gD)} \left[ \log q_\varphi(\mtheta|\gD) - \log p(\mtheta, \gD) \right]\\
    &= \argmin_{\varphi=\{\psi, \nu\}} \sE_{\gD \sim \chi} \sE_{\vz \sim p(\vz)} \left[ \log \frac{q_\nu (\vz|h_\psi(\gD))}{\left| \det J_\nu(\vz; h_\psi(\gD)) \right|} - \log p(g_\nu(\vz; h_\psi(\gD)), \gD) \right]
\end{align}
As shown in BayesFlow \citep{radev2020bayesflow}, the normalizing flow $g_\nu$ and the summary network $h_\psi$ can be trained simultaneously. The $\mathrm{AllInOneBlock}$ coupling block architecture of the FrEIA Python package \citep{Ardizzone2018freia}, which is very similar to the RNVP style coupling block \citep{Dinh2017rnvp}, is used as the basis for the cINN. $\mathrm{AllInOneBlock}$ combines the most common architectural components, such as ActNorm, permutation, and affine coupling operations.

For our experiments, $6$ coupling blocks define the normalizing flow network, each with a $1$ hidden-layered non-linear feed-forward subnetwork with ReLU non-linearity and $128$ hidden dimensions.
\section{Experimental Details}
\label{appdx:experiment}
Unless specified, we obtain a stream of datasets for all our experiments by simply sampling from the assumed probabilistic model, where the number of observations $n$ is sampled uniformly in the range $[64, 128]$. For efficient mini-batching over datasets with different cardinalities, we sample datasets with maximum cardinality $(128)$ and implement different cardinalities by masking out different numbers of observations for different datasets whenever required. 
% For all our experiments on supervised setups, we sample $\vx_i \sim \mathcal{N}(\mathbf{0}, \mathbf{I})$ for simplicity, but it is possible to explore other proposal distributions (e.g., heavy-tailed distributions) too. 
% In our Bayesian Neural Networks experiments, we considered a single-layered neural network with $\mathrm{Tanh}$ activation function and $32$ hidden dimensions. We considered the likelihood function as either a Gaussian or a categorical distribution using the logits, depending on regression and classification.

To evaluate both our proposed approach and the baselines, we compute an average of the predictive performances across $25$ different posterior samples for each of the $100$ fixed test datasets for all our experiments. 
That means for our proposed approach, we sample $25$ different parameter vectors from the approximate posterior that we obtain. For MCMC, we rely on $25$ MCMC samples, and for optimization, we train $25$ different parameter vectors where the randomness comes from initialization. 
For the optimization baseline, we perform a quick hyperparameter search over the space $\{0.01, 003, 0.001, 0.0003, 0.0001, 0.00003\}$ to pick the best learning rate that works for all of the test datasets and then use it to train for $1000$ iterations using the Adam optimizer~\citep{kingma2014adam}. For the MCMC baseline, we use the open-sourced implementation of Langevin-based MCMC sampling\footnote{\href{https://github.com/alisiahkoohi/Langevin-dynamics}{https://github.com/alisiahkoohi/Langevin-dynamics}} where we leave a chunk of the starting samples as burn-in and then start accepting samples after a regular interval (to not make them correlated). The details about the burn-in time and the regular interval for acceptance are provided in the corresponding experiments' sections below.

For our proposed approach of amortized inference, we do not consider explicit hyperparameter optimization and simply use a learning rate of $1\mathrm{e}\text{-}4$ with the Adam optimizer. For all experiments, we used linear scaling of the KL term in the training objectives as described in~\citep{higgins2017betavae}, which we refer to as warmup. Furthermore, training details for each experiment can be found below. 

\subsection{Fixed-Dim}
\label{appdx:details_fixed_dim}
In this section, we provide the experimental details relevant to reproducing the results of Section~\ref{sec:experiments}. All the models are trained with streaming data from the underlying probabilistic model, such that every iteration of training sees a new set of datasets. Training is done with a batch size of $128$, representing the number of datasets seen during one optimization step. Evaluations are done with $25$ samples and we ensure that the test datasets used for each probabilistic model are the same across all the compared methods, i.e., baselines, forward KL, and reverse KL. We train the amortized inference model and the forward KL baselines for the following different probabilistic models:

\textbf{Mean of Gaussian (GM):} We train the amortization models over $20,000$ iterations for both the $2$-dimensional as well as the $100$-dimensional setup. We use a linear warmup with $5000$ iterations over which the weight of the KL term in our proposed approach scales linearly from $0$ to $1$. We use an identity covariance matrix for the data-generating process, but it can be easily extended to the case of correlated or diagonal covariance-based Gaussian distributions.

\textbf{Gaussian Mixture Model (GMM):} We train the mixture model setup for $200,000$ iterations with $50,000$ iterations of warmup. We mainly experiment with $2$-dimensional and $5$-dimensional mixture models, with $2$ and $5$ mixture components for each setup. While we do use an identity covariance matrix for the data-generating process, again, it can be easily extended to other cases.
% For all our experiments, we compute the average over 25 different samples (either from the approximate posterior, or 25 different optimization runs, etc.) to report the downstream metrics. For the optimization baseline, we perform a quick hyperparameter search for each dataset over the space of $\{\}$

\textbf{Linear Regression (LR):} The amortization models for this setup are trained for $50,000$ iterations with $12,500$ iterations of warmup. The feature dimensions considered for this task are $1$ and $100$ dimensions, and the predictive variance $\sigma^2$ is assumed to be known and set as $0.25$.

\textbf{Nonlinear Regression (NLR):} We train the setup for $100,000$ iterations with $25,000$ iterations consisting of warmup. The feature dimensionalities considered are $1$-dimensional and $25$-dimensional, and training is done with a known predictive variance similar to the LR setup. For the probabilistic model, we consider both a $1$-layered and a $2$-layered multi-layer perceptron (MLP) network with 32 hidden units in each, and either a \textsc{relu} or \textsc{tanh} activation function.

\textbf{Linear Classification (LC):} We experiment with $2$-dimensional and $100$-dimensional setups with training done for $50,000$ iterations, out of which $12,500$ are used for warmup. Further, we train for both binary classification as well as a $5$-class classification setup.

\textbf{Nonlinear Classification (NLC):} We experiment with $2$-dimensional and $25$-dimensional setups with training done for $100,000$ iterations, out of which $2,5000$ are used for warmup. Further, we train for both binary classification as well as a $5$-class classification setup. For the probabilistic model, we consider both a $1$-layered and a $2$-layered multi-layer perceptron (MLP) network with 32 hidden units in each, and either a \textsc{relu} or \textsc{tanh} activation function.

\begin{table*}[t]
    \centering
    % \small
    \footnotesize	    
    \def\arraystretch{1.25}
    \setlength{\tabcolsep}{5pt}
    \begin{tabular}{lcr ccc cccc}
        \cmidrule[\heavyrulewidth]{1-9}
         &  &  & \multicolumn{6}{c}{\textit{$L_2$ Loss} ($\downarrow$)} \\
        \cmidrule(lr){4-9}
        \textbf{Objective} & $q_\varphi$ & \textbf{Model} & \multicolumn{3}{c}{\textbf{Linear Model $|$ MLP-TanH Data}} & \multicolumn{3}{c}{\textbf{MLP-TanH Model $|$ Linear Data}} & $\leftarrow\chi_{real}$ \\
        \cmidrule(lr){4-6}\cmidrule(lr){7-9}
        & & & \textit{LR} & \textit{NLR} & \textit{GP} & \textit{LR} & \textit{NLR} & \textit{GP} & $\leftarrow\chi_{sim}$ \\
        \cmidrule{1-9}
\multirow{4}{*}{Baseline} & - & Random & - & $17.761$\sstd{$0.074$}  & -  & $17.847$\sstd{$0.355$} & -  & -  \\
& - & Optimization & - & $1.213$\sstd{$0.000$} & -  & $0.360$\sstd{$0.001$} & -  & -  \\
& - & Langevin & - & $1.218$\sstd{$0.002$} & -  & $0.288$\sstd{$0.001$} & -  & -  \\
& - & HMC & - & $1.216$\sstd{$0.002$} & -  & $0.275$\sstd{$0.001$} & -  & -  \\
\cmidrule{2-9}
\multirow{3}{*}{Fwd-KL} & \multirow{6}{*}{\rotatebox[origin=c]{90}{Gaussian}} & GRU &$2.415$\sstd{$0.269$} & -  & -  & -  & $15.632$\sstd{$0.283$} & -  \\
& & DeepSets &$1.402$\sstd{$0.017$} & -  & -  & -  & $16.046$\sstd{$0.393$} & -  \\
& & Transformer &$2.216$\sstd{$0.097$} & -  & -  & -  & $15.454$\sstd{$0.246$} & -  \\
\cmidrule{3-9}
\multirow{3}{*}{Rev-KL}& & GRU &$1.766$\sstd{$0.044$} & $1.216$\sstd{$0.001$} & $4.566$\sstd{$0.199$} & $0.375$\sstd{$0.001$} & $0.386$\sstd{$0.002$} & $0.524$\sstd{$0.019$} \\
& & DeepSets &$1.237$\sstd{$0.006$} & $1.216$\sstd{$0.001$} & $3963.694$\sstd{$5602.411$} & $0.365$\sstd{$0.000$} & $0.377$\sstd{$0.003$} & $0.385$\sstd{$0.011$} \\
& & Transformer &$1.892$\sstd{$0.113$} & $1.226$\sstd{$0.001$} & $4.313$\sstd{$0.707$} & $0.367$\sstd{$0.006$} & $0.382$\sstd{$0.003$} & $0.458$\sstd{$0.048$} \\
\cmidrule{2-9}
\multirow{3}{*}{Fwd-KL} & \multirow{6}{*}{\rotatebox[origin=c]{90}{Flow}} & GRU &$2.180$\sstd{$0.024$} & -  & -  & -  & $9.800$\sstd{$0.473$} & -  \\
& & DeepSets &$1.713$\sstd{$0.244$} & -  & -  & -  & $15.253$\sstd{$0.403$} & -  \\
& & Transformer &$1.632$\sstd{$0.070$} & -  & -  & -  & $7.949$\sstd{$0.419$} & -  \\
\cmidrule{3-9}
\multirow{3}{*}{Rev-KL} & & GRU &$1.830$\sstd{$0.081$} & \highlight{$1.214$\sstd{$0.001$}} & $5.690$\sstd{$0.196$} & $0.346$\sstd{$0.004$} & $0.349$\sstd{$0.001$} & $0.520$\sstd{$0.015$} \\
& & DeepSets &$1.282$\sstd{$0.036$} & $1.218$\sstd{$0.001$} & $11.690$\sstd{$10.602$} & \highlight{$0.339$\sstd{$0.003$}} & $0.344$\sstd{$0.002$} & $0.397$\sstd{$0.026$} \\
& & Transformer &$1.471$\sstd{$0.016$} & $1.226$\sstd{$0.004$} & $5.194$\sstd{$0.320$} & $0.346$\sstd{$0.002$} & $0.347$\sstd{$0.001$} & $0.480$\sstd{$0.030$} \\
\cmidrule[\heavyrulewidth]{1-9}
    \end{tabular}
    \caption{\textbf{Model Misspecification}. Results for model misspecification under different training data $\chi_{sim}$, when evaluated under MLP-TanH and Linear Data ($\chi_{real}$), with the underlying model as a linear and MLP-TanH model respectively.}
    \vspace{-4mm}
    \label{tab:misspec_model}
\end{table*}
\begin{table*}[t]
    \centering
    % \small
    \footnotesize	    
    \def\arraystretch{1.25}
    \setlength{\tabcolsep}{5pt}
    \begin{tabular}{lcr ccc cccc}
        \cmidrule[\heavyrulewidth]{1-9}
         &  &  & \multicolumn{6}{c}{\textit{$L_2$ Loss} ($\downarrow$)} \\
        \cmidrule(lr){4-9}
        \textbf{Objective} & $q_\varphi$ & \textbf{Model} & \multicolumn{3}{c}{\textbf{Linear Model $|$ GP Data}} & \multicolumn{3}{c}{\textbf{MLP-TanH Model $|$ GP Data}} & $\leftarrow\chi_{real}$ \\
        \cmidrule(lr){4-6}\cmidrule(lr){7-9}
        & & & \textit{LR} & \textit{NLR} & \textit{GP} & \textit{LR} & \textit{NLR} & \textit{GP}  & $\leftarrow\chi_{sim}$ \\
        \cmidrule{1-9}
\multirow{4}{*}{Baseline} & - & Random & -  & -  & $2.681$\sstd{$0.089$} &  -  & -  & $16.236$\sstd{$0.381$} \\
& - & Optimization & -  & -  & $0.263$\sstd{$0.000$} & -  & -  & $0.007$\sstd{$0.000$} \\
& - & Langevin & -  & -  & $0.266$\sstd{$0.001$} & -  & -  & $0.022$\sstd{$0.001$} \\
& - & HMC & -  & -  & $0.266$\sstd{$0.000$} & -  & -  & $0.090$\sstd{$0.002$} \\
\cmidrule{2-9}
\multirow{3}{*}{Fwd-KL} & \multirow{6}{*}{\rotatebox[origin=c]{90}{Gaussian}} & GRU &$0.268$\sstd{$0.000$} & -  & -  & -  & $14.077$\sstd{$0.368$} & -  \\
& & DeepSets &$0.269$\sstd{$0.001$} & -  & -  & -  & $14.756$\sstd{$0.280$} & -  \\
& & Transformer &$0.270$\sstd{$0.001$} & -  & -  & -  & $14.733$\sstd{$0.513$} & -  \\
\cmidrule{3-9}
\multirow{3}{*}{Rev-KL} & & GRU &$0.268$\sstd{$0.000$} & $0.269$\sstd{$0.000$} & $0.266$\sstd{$0.000$} & $0.334$\sstd{$0.005$} & $0.157$\sstd{$0.003$} & $0.080$\sstd{$0.003$} \\
& & DeepSets &$0.269$\sstd{$0.000$} & $0.269$\sstd{$0.000$} & \highlight{$0.265$\sstd{$0.000$}} & $0.331$\sstd{$0.003$} & $0.146$\sstd{$0.002$} & $0.063$\sstd{$0.000$} \\
& & Transformer &$0.269$\sstd{$0.000$} & $0.269$\sstd{$0.000$} & $0.267$\sstd{$0.000$} & $0.310$\sstd{$0.013$} & $0.155$\sstd{$0.006$} & $0.066$\sstd{$0.004$} \\
\cmidrule{2-9}
\multirow{3}{*}{Fwd-KL} & \multirow{6}{*}{\rotatebox[origin=c]{90}{Flow}} & GRU &$0.268$\sstd{$0.000$} & -  & -  & -  & $9.756$\sstd{$0.192$} & -  \\
& & DeepSets &$0.269$\sstd{$0.001$} & -  & -  & -  & $14.345$\sstd{$0.628$} & -  \\
& & Transformer &$0.269$\sstd{$0.000$} & -  & -  & -  & $8.557$\sstd{$0.561$} & -  \\
\cmidrule{3-9}
\multirow{3}{*}{Rev-KL} & & GRU &$0.268$\sstd{$0.000$} & $0.270$\sstd{$0.001$} & $0.266$\sstd{$0.000$} & $0.289$\sstd{$0.011$} & $0.120$\sstd{$0.004$} & $0.059$\sstd{$0.003$} \\
& & DeepSets &$0.269$\sstd{$0.000$} & $0.269$\sstd{$0.001$} & $0.266$\sstd{$0.000$} & $0.270$\sstd{$0.008$} & $0.115$\sstd{$0.002$} & $0.059$\sstd{$0.002$} \\
& & Transformer &$0.269$\sstd{$0.001$} & $0.270$\sstd{$0.000$} & $0.267$\sstd{$0.000$} & $0.293$\sstd{$0.008$} & $0.120$\sstd{$0.005$} & \highlight{$0.055$\sstd{$0.002$}} \\
\cmidrule[\heavyrulewidth]{1-9}
    \end{tabular}
    \caption{\textbf{Model Misspecification}. Results for model misspecification under different training data $\chi_{sim}$, when evaluated under GP Data ($\chi_{real}$), with the underlying model as a linear and MLP-TanH model respectively.}
    \vspace{-4mm}
    \label{tab:misspec_gp}
\end{table*}
\subsection{Variable-Dim}
\label{appdx:details_max_dim}
In this section, we provide the experimental details relevant to reproducing the results of Section~\ref{sec:experiments}. All the models are trained with streaming data from the underlying probabilistic model, such that every iteration of training sees a new set of datasets. Training is done with a batch size of $128$, representing the number of datasets seen during one optimization step. Further, we ensure that the datasets sampled resemble a uniform distribution over the feature dimensions, ranging from $1$-dimensional to the maximal dimensional setup. Evaluations are done with $25$ samples and we ensure that the test datasets used for each probabilistic model are the same across all the compared methods, i.e., baselines, forward KL, and reverse KL. We train the amortized inference model and the forward KL baselines for the following different probabilistic models:

\textbf{Mean of Gaussian (GM):} We train the amortization models over $50,000$ iterations using a linear warmup with $12,5000$ iterations over which the weight of the KL term in our proposed approach scales linearly from $0$ to $1$. We use an identity covariance matrix for the data-generating process, but it can be easily extended to the case of correlated or diagonal covariance-based Gaussian distributions. In this setup, we consider a maximum of $100$ feature dimensions.

\textbf{Gaussian Mixture Model (GMM):} We train the mixture model setup for $500,000$ iterations with $125,000$ iterations of warmup. We set the maximal feature dimensions as $5$ and experiment with $2$ and $5$ mixture components. While we do use an identity covariance matrix for the data-generating process, again, it can be easily extended to other cases.
% For all our experiments, we compute the average over 25 different samples (either from the approximate posterior, or 25 different optimization runs, etc.) to report the downstream metrics. For the optimization baseline, we perform a quick hyperparameter search for each dataset over the space of $\{\}$

\textbf{Linear Regression (LR):} The amortization models for this setup are trained for $100,000$ iterations with $25,000$ iterations of warmup. The maximal feature dimension considered for this task is $100$-dimensional, and the predictive variance $\sigma^2$ is assumed to be known and set as $0.25$.

\textbf{Nonlinear Regression (NLR):} We train the setup for $250,000$ iterations with $62,500$ iterations consisting of warmup. The maximal feature dimension considered is $100$-dimensional, and training is done with a known predictive variance similar to the LR setup. For the probabilistic model, we consider both a $1$-layered and a $2$-layered multi-layer perceptron (MLP) network with 32 hidden units in each, and either a \textsc{relu} or \textsc{tanh} activation function.

\textbf{Linear Classification (LC):} We experiment with a maximal $100$-dimensional setup with training done for $100,000$ iterations, out of which $25,000$ are used for warmup. Further, we train for both binary classification as well as a $5$-class classification setup.

\textbf{Nonlinear Classification (NLC):} We experiment with a maximal $100$-dimensional setup with training done for $250,000$ iterations, out of which $62,500$ are used for warmup. Further, we train for both binary classification as well as a $5$-class classification setup. For the probabilistic model, we consider both a $1$-layered and a $2$-layered multi-layer perceptron (MLP) network with 32 hidden units in each, and either a \textsc{relu} or \textsc{tanh} activation function.

\begin{figure*}
    \centering
    \captionsetup[subfigure]{font=scriptsize}
    \includegraphics[width=\textwidth]{Draft/Plots/dimension_trends/KL.pdf}
    \vspace{-7mm}
    \caption{\textbf{Trends of Performance over different Dimensions in Variable Dimensionality Setup:} We see that our proposed reverse KL methodology outperforms the forward KL one.}
    \vspace{-5mm}
    \label{fig:dim_kl}
\end{figure*}

\begin{figure*}
    \centering
    \captionsetup[subfigure]{font=scriptsize}
    \includegraphics[width=\textwidth]{Draft/Plots/dimension_trends/Model.pdf}
    \vspace{-7mm}
    \caption{\textbf{Trends of Performance over different Dimensions in Variable Dimensionality Setup:} We see that transformer models generalize better to different dimensional inputs than DeepSets.}
    \vspace{-5mm}
    \label{fig:dim_kl}
\end{figure*}

\begin{figure*}
    \centering
    \captionsetup[subfigure]{font=scriptsize}
    \includegraphics[width=\textwidth]{Draft/Plots/dimension_trends/Variational_Approximation.pdf}
    \vspace{-7mm}
    \caption{\textbf{Trends of Performance over different Dimensions in Variable Dimensionality Setup:} We see that normalizing flows leads to similar performances than Gaussian based variational approximation.}
    \vspace{-5mm}
    \label{fig:dim_kl}
\end{figure*}
\subsection{Model Misspecification}
\label{appdx:details_misspecification}
In this section, we provide the experimental details relevant to reproducing the results of Section~\ref{sec:experiments}.
All models during this experiment are trained with streaming data from the currently used dataset-generating function $\chi$, such that every iteration of training sees a new batch of datasets. Training is done with a batch size of $128$, representing the number of datasets seen during one optimization step. Evaluation for all models is done with $10$ samples from each dataset-generator used in the respective experimental subsection and we ensure that the test datasets are the same across all compared methods, i.e., baselines, forward KL, and reverse KL.

\textbf{Linear Regression Model:} The linear regression amortization models are trained following the training setting for linear regression fixed dimensionality, that is, $50,000$ training iterations with $12,500$ iterations of warmup. The feature dimension considered for this task is $1$-dimension. The model is trained separately on datasets from three different generators $\chi$: linear regression, nonlinear regression, and Gaussian processes, and evaluated after training on test datasets from all of them.
For training with datasets from the linear regression probabilistic model, the predictive variance $\sigma^2$ is assumed to be known and set as $0.25$. 
The same variance is used for generating datasets from the nonlinear regression dataset generator with $1$ layer, $32$ hidden units, and \textsc{tanh} activation function. 
Lastly, datasets from the Gaussian process-based generator are sampled similarly, using the GPytorch library~\cite{gardner2018gpytorch}, where datasets are sampled of varying cardinality, ranging from $64$ to $128$. We use a zero-mean Gaussian Process (GP) with a unit lengthscale radial-basis function (RBF) kernel serving as the covariance matrix. Further, we use a very small noise of $\sigma^2 = 1\mathrm{e}^{-6}$ in the likelihood term of the GP.
Forward KL training in this experiment can only be done when the amortization model and the dataset-generating function are the same: when we train on datasets from the linear regression-based $\chi$. Table \ref{tab:misspec_model} provides a detailed overview of the results.


\textbf{Nonlinear Regression Models:} The nonlinear regression amortization models are trained following the training setting for nonlinear regression fixed dimensionality, that is, $100,000$ training iterations with $25,000$ iterations of warmup. Here, we consider two single-layer perceptions with 32 hidden units with a \textsc{tanh} activation function. The feature dimensionality considered is $1$ dimension.
We consider the same dataset-generating functions as in the misspecification experiment for a linear regression model above. However, the activation function used in the nonlinear regression dataset generator matches the activation function of the currently trained amortization model. In this case, forward KL training is possible in the two instances when trained on datasets from the corresponding nonlinear regression probabilistic model. A more detailed overview of the results can be found in Table \ref{tab:misspec_model} and \ref{tab:misspec_gp}.

\begin{figure}
    \centering
    \includegraphics[width=\textwidth]{Draft/Plots/real_world/linear_regression_Vanilla.pdf}
    \caption{\textbf{Tabular Experiments $|$ Linear Regression with Diagonal Gaussian}: For every regression dataset from the OpenML platform considered, we initialize the parameters of a linear regression-based probabilistic model with the amortized inference models which were trained with a diagonal Gaussian assumption. The parameters are then further trained with maximum-a-posteriori (MAP) estimate with gradient descent. Reverse and Forward KL denote initialization with the correspondingly trained amortized model. Prior refers to a MAP-based optimization baseline initialized from the prior $\gN(0, I)$, whereas Xavier refers to initialization from the Xavier initialization scheme.}
    \label{fig:regression_linear_vanilla}
\end{figure}

\subsection{Tabular Experiments}
\label{appdx:details_tabular}
For the tabular experiments, we train the amortized inference models for (non-)linear regression (NLR/LR) as well as (non-)linear classification (NLC/LC) with $\vx \sim \mathcal{N}(\mathbf{0}, \mathbf{I})$ as opposed to $\vx \sim \gU(-\mathbf{1}, \mathbf{1})$ in the dataset generating process $\chi$, with the rest of the settings the same as \textsc{maximum-dim} experiments. For the nonlinear setups, we only consider the \textsc{relu} case as it has seen predominant success in deep learning. Further, we only consider a 1-hidden layer neural network with 32 hidden dimensions in the probabilistic model. 

After having trained the amortized inference models, both for forward and reverse KL setups, we evaluate them on real-world tabular datasets. We first collect a subset of tabular datasets from the OpenML platform as outlined in Appendix~\ref{appdx:datasets}. Then, for each dataset, we perform a 5-fold cross-validation evaluation where the dataset is chunked into $5$ bins, of which, at any time, $4$ are used for training and one for evaluation. This procedure is repeated five times so that every chunk is used for evaluation once.

For each dataset, we normalize the observations and the targets so that they have zero mean and unit standard deviation. For the classification setups, we only normalize the inputs as the targets are categorical. For both forward KL and reverse KL amortization models, we initialize the probabilistic model from samples from the amortized model and then further finetune it via dataset-specific maximum a posteriori optimization. We repeat this setup over $25$ different samples from the inference model. In contrast, for the optimization baseline, we initialize the probabilistic models' parameters from $\gN(0, I)$, which is the prior that we consider, and then train 25 such models with maximum a posteriori objective using Adam optimizer. 

While we see that the amortization models, particularly the reverse KL model, lead to much better initialization and convergence, it is important to note that the benefits vanish if we initialize using the Xavier-init initialization scheme. However, we believe that this is not a fair comparison as it means that we are considering a different prior now, while the amortized models were trained with $\gN(0, I)$ prior. We defer the readers to the section below for additional discussion and experimental results.

\begin{table*}[ht]
    \footnotesize
    \centering
    \renewcommand{\arraystretch}{1.1} % Adjusts the row spacing
    \resizebox{16cm}{!} 
    { 
    \begin{tblr}{hline{1,2,Z} = 0.8pt, hline{3-Y} = 0.2pt,
                 colspec = {Q[l,m, 13em] Q[l,m, 6em] Q[c,m, 8em] Q[c,m, 5em] Q[l,m, 14em]},
                 colsep  = 4pt,
                 row{1}  = {0.4cm, font=\bfseries, bg=gray!30},
                 row{2-Z} = {0.2cm},
                 }
\textbf{Dataset}       & \textbf{Table Source} & \textbf{\# Tables / Statements} & \textbf{\# Words / Statement} & \textbf{Explicit Control}\\ 
\SetCell[c=5]{c} \textit{Single-sentence Table-to-Text}\\
ToTTo \cite{parikh2020tottocontrolledtabletotextgeneration}   & Wikipedia        & 83,141 / 83,141                  & 17.4                          & Table region      \\
LOGICNLG \cite{chen2020logicalnaturallanguagegeneration} & Wikipedia        & 7,392 / 36,960                  & 14.2                          & Table regions      \\ 
HiTab \cite{cheng-etal-2022-hitab}   & Statistics web   & 3,597 / 10,672                  & 16.4                          & Table regions \& reasoning operator \\ 
\SetCell[c=5]{c} \textit{Generic Table Summarization}\\
ROTOWIRE \cite{wiseman2017challengesdatatodocumentgeneration} & NBA games      & 4,953 / 4,953                   & 337.1                         & \textbf{\textit{X}}                   \\
SciGen \cite{moosavi2021scigen} & Sci-Paper      & 1,338 / 1,338                   & 116.0                         & \textbf{\textit{X}}                   \\
NumericNLG \cite{suadaa-etal-2021-towards} & Sci-Paper   & 1,355 / 1,355                   & 94.2                          & \textbf{\textit{X}}                    \\
\SetCell[c=5]{c} \textit{Table Question Answering}\\
FeTaQA \cite{nan2021fetaqafreeformtablequestion}     & Wikipedia      & 10,330 / 10,330                 & 18.9                          & Queries rewritten from ToTTo \\
\SetCell[c=5]{c} \textit{Query-Focused Table Summarization}\\
QTSumm \cite{zhao2023qtsummqueryfocusedsummarizationtabular}                        & Wikipedia      & 2,934 / 7,111                   & 68.0                          & Queries from real-world scenarios\\ 
\textbf{eC-Tab2Text} (\textit{ours})                           & e-Commerce products      & 1,452 / 3,354                   & 56.61                          & Queries from e-commerce products\\
    \end{tblr}
    }
\caption{Comparison between \textbf{eC-Tab2Text} (\textit{ours}) and existing table-to-text generation datasets. Statements and queries are used interchangeably. Our dataset specifically comprises tables from the e-commerce domain.}
\label{tab:datasets}
\end{table*}

\begin{table*}[t]
\centering
\fontsize{11pt}{11pt}\selectfont
\begin{tabular}{lllllllllllll}
\toprule
\multicolumn{1}{c}{\textbf{task}} & \multicolumn{2}{c}{\textbf{Mir}} & \multicolumn{2}{c}{\textbf{Lai}} & \multicolumn{2}{c}{\textbf{Ziegen.}} & \multicolumn{2}{c}{\textbf{Cao}} & \multicolumn{2}{c}{\textbf{Alva-Man.}} & \multicolumn{1}{c}{\textbf{avg.}} & \textbf{\begin{tabular}[c]{@{}l@{}}avg.\\ rank\end{tabular}} \\
\multicolumn{1}{c}{\textbf{metrics}} & \multicolumn{1}{c}{\textbf{cor.}} & \multicolumn{1}{c}{\textbf{p-v.}} & \multicolumn{1}{c}{\textbf{cor.}} & \multicolumn{1}{c}{\textbf{p-v.}} & \multicolumn{1}{c}{\textbf{cor.}} & \multicolumn{1}{c}{\textbf{p-v.}} & \multicolumn{1}{c}{\textbf{cor.}} & \multicolumn{1}{c}{\textbf{p-v.}} & \multicolumn{1}{c}{\textbf{cor.}} & \multicolumn{1}{c}{\textbf{p-v.}} &  &  \\ \midrule
\textbf{S-Bleu} & 0.50 & 0.0 & 0.47 & 0.0 & 0.59 & 0.0 & 0.58 & 0.0 & 0.68 & 0.0 & 0.57 & 5.8 \\
\textbf{R-Bleu} & -- & -- & 0.27 & 0.0 & 0.30 & 0.0 & -- & -- & -- & -- & - &  \\
\textbf{S-Meteor} & 0.49 & 0.0 & 0.48 & 0.0 & 0.61 & 0.0 & 0.57 & 0.0 & 0.64 & 0.0 & 0.56 & 6.1 \\
\textbf{R-Meteor} & -- & -- & 0.34 & 0.0 & 0.26 & 0.0 & -- & -- & -- & -- & - &  \\
\textbf{S-Bertscore} & \textbf{0.53} & 0.0 & {\ul 0.80} & 0.0 & \textbf{0.70} & 0.0 & {\ul 0.66} & 0.0 & {\ul0.78} & 0.0 & \textbf{0.69} & \textbf{1.7} \\
\textbf{R-Bertscore} & -- & -- & 0.51 & 0.0 & 0.38 & 0.0 & -- & -- & -- & -- & - &  \\
\textbf{S-Bleurt} & {\ul 0.52} & 0.0 & {\ul 0.80} & 0.0 & 0.60 & 0.0 & \textbf{0.70} & 0.0 & \textbf{0.80} & 0.0 & {\ul 0.68} & {\ul 2.3} \\
\textbf{R-Bleurt} & -- & -- & 0.59 & 0.0 & -0.05 & 0.13 & -- & -- & -- & -- & - &  \\
\textbf{S-Cosine} & 0.51 & 0.0 & 0.69 & 0.0 & {\ul 0.62} & 0.0 & 0.61 & 0.0 & 0.65 & 0.0 & 0.62 & 4.4 \\
\textbf{R-Cosine} & -- & -- & 0.40 & 0.0 & 0.29 & 0.0 & -- & -- & -- & -- & - & \\ \midrule
\textbf{QuestEval} & 0.23 & 0.0 & 0.25 & 0.0 & 0.49 & 0.0 & 0.47 & 0.0 & 0.62 & 0.0 & 0.41 & 9.0 \\
\textbf{LLaMa3} & 0.36 & 0.0 & \textbf{0.84} & 0.0 & {\ul{0.62}} & 0.0 & 0.61 & 0.0 &  0.76 & 0.0 & 0.64 & 3.6 \\
\textbf{our (3b)} & 0.49 & 0.0 & 0.73 & 0.0 & 0.54 & 0.0 & 0.53 & 0.0 & 0.7 & 0.0 & 0.60 & 5.8 \\
\textbf{our (8b)} & 0.48 & 0.0 & 0.73 & 0.0 & 0.52 & 0.0 & 0.53 & 0.0 & 0.7 & 0.0 & 0.59 & 6.3 \\  \bottomrule
\end{tabular}
\caption{Pearson correlation on human evaluation on system output. `R-': reference-based. `S-': source-based.}
\label{tab:sys}
\end{table*}



\begin{table}%[]
\centering
\fontsize{11pt}{11pt}\selectfont
\begin{tabular}{llllll}
\toprule
\multicolumn{1}{c}{\textbf{task}} & \multicolumn{1}{c}{\textbf{Lai}} & \multicolumn{1}{c}{\textbf{Zei.}} & \multicolumn{1}{c}{\textbf{Scia.}} & \textbf{} & \textbf{} \\ 
\multicolumn{1}{c}{\textbf{metrics}} & \multicolumn{1}{c}{\textbf{cor.}} & \multicolumn{1}{c}{\textbf{cor.}} & \multicolumn{1}{c}{\textbf{cor.}} & \textbf{avg.} & \textbf{\begin{tabular}[c]{@{}l@{}}avg.\\ rank\end{tabular}} \\ \midrule
\textbf{S-Bleu} & 0.40 & 0.40 & 0.19* & 0.33 & 7.67 \\
\textbf{S-Meteor} & 0.41 & 0.42 & 0.16* & 0.33 & 7.33 \\
\textbf{S-BertS.} & {\ul0.58} & 0.47 & 0.31 & 0.45 & 3.67 \\
\textbf{S-Bleurt} & 0.45 & {\ul 0.54} & {\ul 0.37} & 0.45 & {\ul 3.33} \\
\textbf{S-Cosine} & 0.56 & 0.52 & 0.3 & {\ul 0.46} & {\ul 3.33} \\ \midrule
\textbf{QuestE.} & 0.27 & 0.35 & 0.06* & 0.23 & 9.00 \\
\textbf{LlaMA3} & \textbf{0.6} & \textbf{0.67} & \textbf{0.51} & \textbf{0.59} & \textbf{1.0} \\
\textbf{Our (3b)} & 0.51 & 0.49 & 0.23* & 0.39 & 4.83 \\
\textbf{Our (8b)} & 0.52 & 0.49 & 0.22* & 0.43 & 4.83 \\ \bottomrule
\end{tabular}
\caption{Pearson correlation on human ratings on reference output. *not significant; we cannot reject the null hypothesis of zero correlation}
\label{tab:ref}
\end{table}


\begin{table*}%[]
\centering
\fontsize{11pt}{11pt}\selectfont
\begin{tabular}{lllllllll}
\toprule
\textbf{task} & \multicolumn{1}{c}{\textbf{ALL}} & \multicolumn{1}{c}{\textbf{sentiment}} & \multicolumn{1}{c}{\textbf{detoxify}} & \multicolumn{1}{c}{\textbf{catchy}} & \multicolumn{1}{c}{\textbf{polite}} & \multicolumn{1}{c}{\textbf{persuasive}} & \multicolumn{1}{c}{\textbf{formal}} & \textbf{\begin{tabular}[c]{@{}l@{}}avg. \\ rank\end{tabular}} \\
\textbf{metrics} & \multicolumn{1}{c}{\textbf{cor.}} & \multicolumn{1}{c}{\textbf{cor.}} & \multicolumn{1}{c}{\textbf{cor.}} & \multicolumn{1}{c}{\textbf{cor.}} & \multicolumn{1}{c}{\textbf{cor.}} & \multicolumn{1}{c}{\textbf{cor.}} & \multicolumn{1}{c}{\textbf{cor.}} &  \\ \midrule
\textbf{S-Bleu} & -0.17 & -0.82 & -0.45 & -0.12* & -0.1* & -0.05 & -0.21 & 8.42 \\
\textbf{R-Bleu} & - & -0.5 & -0.45 &  &  &  &  &  \\
\textbf{S-Meteor} & -0.07* & -0.55 & -0.4 & -0.01* & 0.1* & -0.16 & -0.04* & 7.67 \\
\textbf{R-Meteor} & - & -0.17* & -0.39 & - & - & - & - & - \\
\textbf{S-BertScore} & 0.11 & -0.38 & -0.07* & -0.17* & 0.28 & 0.12 & 0.25 & 6.0 \\
\textbf{R-BertScore} & - & -0.02* & -0.21* & - & - & - & - & - \\
\textbf{S-Bleurt} & 0.29 & 0.05* & 0.45 & 0.06* & 0.29 & 0.23 & 0.46 & 4.2 \\
\textbf{R-Bleurt} & - &  0.21 & 0.38 & - & - & - & - & - \\
\textbf{S-Cosine} & 0.01* & -0.5 & -0.13* & -0.19* & 0.05* & -0.05* & 0.15* & 7.42 \\
\textbf{R-Cosine} & - & -0.11* & -0.16* & - & - & - & - & - \\ \midrule
\textbf{QuestEval} & 0.21 & {\ul{0.29}} & 0.23 & 0.37 & 0.19* & 0.35 & 0.14* & 4.67 \\
\textbf{LlaMA3} & \textbf{0.82} & \textbf{0.80} & \textbf{0.72} & \textbf{0.84} & \textbf{0.84} & \textbf{0.90} & \textbf{0.88} & \textbf{1.00} \\
\textbf{Our (3b)} & 0.47 & -0.11* & 0.37 & 0.61 & 0.53 & 0.54 & 0.66 & 3.5 \\
\textbf{Our (8b)} & {\ul{0.57}} & 0.09* & {\ul 0.49} & {\ul 0.72} & {\ul 0.64} & {\ul 0.62} & {\ul 0.67} & {\ul 2.17} \\ \bottomrule
\end{tabular}
\caption{Pearson correlation on human ratings on our constructed test set. 'R-': reference-based. 'S-': source-based. *not significant; we cannot reject the null hypothesis of zero correlation}
\label{tab:con}
\end{table*}

\section{Results}
We benchmark the different metrics on the different datasets using correlation to human judgement. For content preservation, we show results split on data with system output, reference output and our constructed test set: we show that the data source for evaluation leads to different conclusions on the metrics. In addition, we examine whether the metrics can rank style transfer systems similar to humans. On style strength, we likewise show correlations between human judgment and zero-shot evaluation approaches. When applicable, we summarize results by reporting the average correlation. And the average ranking of the metric per dataset (by ranking which metric obtains the highest correlation to human judgement per dataset). 

\subsection{Content preservation}
\paragraph{How do data sources affect the conclusion on best metric?}
The conclusions about the metrics' performance change radically depending on whether we use system output data, reference output, or our constructed test set. Ideally, a good metric correlates highly with humans on any data source. Ideally, for meta-evaluation, a metric should correlate consistently across all data sources, but the following shows that the correlations indicate different things, and the conclusion on the best metric should be drawn carefully.

Looking at the metrics correlations with humans on the data source with system output (Table~\ref{tab:sys}), we see a relatively high correlation for many of the metrics on many tasks. The overall best metrics are S-BertScore and S-BLEURT (avg+avg rank). We see no notable difference in our method of using the 3B or 8B model as the backbone.

Examining the average correlations based on data with reference output (Table~\ref{tab:ref}), now the zero-shoot prompting with LlaMA3 70B is the best-performing approach ($0.59$ avg). Tied for second place are source-based cosine embedding ($0.46$ avg), BLEURT ($0.45$ avg) and BertScore ($0.45$ avg). Our method follows on a 5. place: here, the 8b version (($0.43$ avg)) shows a bit stronger results than 3b ($0.39$ avg). The fact that the conclusions change, whether looking at reference or system output, confirms the observations made by \citet{scialom-etal-2021-questeval} on simplicity transfer.   

Now consider the results on our test set (Table~\ref{tab:con}): Several metrics show low or no correlation; we even see a significantly negative correlation for some metrics on ALL (BLEU) and for specific subparts of our test set for BLEU, Meteor, BertScore, Cosine. On the other end, LlaMA3 70B is again performing best, showing strong results ($0.82$ in ALL). The runner-up is now our 8B method, with a gap to the 3B version ($0.57$ vs $0.47$ in ALL). Note our method still shows zero correlation for the sentiment task. After, ranks BLEURT ($0.29$), QuestEval ($0.21$), BertScore ($0.11$), Cosine ($0.01$).  

On our test set, we find that some metrics that correlate relatively well on the other datasets, now exhibit low correlation. Hence, with our test set, we can now support the logical reasoning with data evidence: Evaluation of content preservation for style transfer needs to take the style shift into account. This conclusion could not be drawn using the existing data sources: We hypothesise that for the data with system-based output, successful output happens to be very similar to the source sentence and vice versa, and reference-based output might not contain server mistakes as they are gold references. Thus, none of the existing data sources tests the limits of the metrics.  


\paragraph{How do reference-based metrics compare to source-based ones?} Reference-based metrics show a lower correlation than the source-based counterpart for all metrics on both datasets with ratings on references (Table~\ref{tab:sys}). As discussed previously, reference-based metrics for style transfer have the drawback that many different good solutions on a rewrite might exist and not only one similar to a reference.


\paragraph{How well can the metrics rank the performance of style transfer methods?}
We compare the metrics' ability to judge the best style transfer methods w.r.t. the human annotations: Several of the data sources contain samples from different style transfer systems. In order to use metrics to assess the quality of the style transfer system, metrics should correctly find the best-performing system. Hence, we evaluate whether the metrics for content preservation provide the same system ranking as human evaluators. We take the mean of the score for every output on each system and the mean of the human annotations; we compare the systems using the Kendall's Tau correlation. 

We find only the evaluation using the dataset Mir, Lai, and Ziegen to result in significant correlations, probably because of sparsity in a number of system tests (App.~\ref{app:dataset}). Our method (8b) is the only metric providing a perfect ranking of the style transfer system on the Lai data, and Llama3 70B the only one on the Ziegen data. Results in App.~\ref{app:results}. 


\subsection{Style strength results}
%Evaluating style strengths is a challenging task. 
Llama3 70B shows better overall results than our method. However, our method scores higher than Llama3 70B on 2 out of 6 datasets, but it also exhibits zero correlation on one task (Table~\ref{tab:styleresults}).%More work i s needed on evaluating style strengths. 
 
\begin{table}%[]
\fontsize{11pt}{11pt}\selectfont
\begin{tabular}{lccc}
\toprule
\multicolumn{1}{c}{\textbf{}} & \textbf{LlaMA3} & \textbf{Our (3b)} & \textbf{Our (8b)} \\ \midrule
\textbf{Mir} & 0.46 & 0.54 & \textbf{0.57} \\
\textbf{Lai} & \textbf{0.57} & 0.18 & 0.19 \\
\textbf{Ziegen.} & 0.25 & 0.27 & \textbf{0.32} \\
\textbf{Alva-M.} & \textbf{0.59} & 0.03* & 0.02* \\
\textbf{Scialom} & \textbf{0.62} & 0.45 & 0.44 \\
\textbf{\begin{tabular}[c]{@{}l@{}}Our Test\end{tabular}} & \textbf{0.63} & 0.46 & 0.48 \\ \bottomrule
\end{tabular}
\caption{Style strength: Pearson correlation to human ratings. *not significant; we cannot reject the null hypothesis of zero corelation}
\label{tab:styleresults}
\end{table}

\subsection{Ablation}
We conduct several runs of the methods using LLMs with variations in instructions/prompts (App.~\ref{app:method}). We observe that the lower the correlation on a task, the higher the variation between the different runs. For our method, we only observe low variance between the runs.
None of the variations leads to different conclusions of the meta-evaluation. Results in App.~\ref{app:results}.

\end{document}