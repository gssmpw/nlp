\documentclass{standalone}

\usepackage{tikz,pgf} % and any other packages or tikzlibraries your picture needs
\usepackage{graphicx} % for including images
% inspired from https://texample.net/tikz/examples/neural-network/
\usetikzlibrary{shapes,decorations.pathreplacing,calligraphy} % for curly brackets and arrows
\begin{document}
\def\layersep{2cm}
\def\networksep{5cm} % Reduced the horizontal distance between the two networks
\begin{tikzpicture}[shorten >=1pt,draw=black!50, node distance=\layersep]
    \tikzstyle{every pin edge}=[<-,shorten <=1pt]
    \tikzstyle{neuron}=[circle,fill=black!25,minimum size=17pt,inner sep=0pt]
    \tikzstyle{input neuron}=[neuron, fill=black!25];
    \tikzstyle{output neuron}=[neuron, fill=red!50];
    \tikzstyle{hidden neuron}=[neuron, fill=blue!50];
    \tikzstyle{second hidden neuron}=[neuron, fill=blue!50];
    \tikzstyle{annot} = [text width=10em, text centered, font=\bf];
    \tikzstyle{bigarrow} = [single arrow, fill=black!10, anchor=base, align=center,text width=2cm]

    % DENSE NETWORK (Baseline NN)
    % Draw the input layer nodes
    \foreach \name / \y in {1,...,4}
    {
        \node[input neuron] (DI-\name) at (\y,0) {};
    }
    
    % Draw the first hidden layer nodes
    \foreach \name / \y in {1,...,5}
        \path[xshift=-0.5cm]
            node[hidden neuron] (DH1-\name) at (\y,-1*\layersep) {};
    
    % Draw the second hidden layer nodes (new hidden layer)
    \foreach \name / \y in {1,...,5}
        \path[xshift=-0.5cm]
            node[hidden neuron] (DH2-\name) at (\y,-2*\layersep) {};
    
    % Draw the third hidden layer nodes (original second hidden layer)
    \foreach \name / \y in {1,...,4}
        \path[xshift=0cm]
            node[hidden neuron] (DH3-\name) at (\y,-3*\layersep) {};
    
    % Draw the output layer nodes (four output neurons, aligned with the previous layer)
    \foreach \name / \y in {1,...,4}
        \path[xshift=0cm]
            node[output neuron] (DO-\name) at (\y,-4*\layersep) {};
    
    % Add THICK HORIZONTAL CURLY BRACKET to the output layer (no arrows)
    \draw[decoration={brace,amplitude=10pt,raise=5pt},decorate, thick, draw=black] 
        (DO-4.south east) -- (DO-1.south west) node[below=8pt] {};

    % DENSE NETWORK OUTPUT: Histogram
    \node at (2.5,-11) (histogram) { 
        \begin{tikzpicture}
            % Draw the bars
            \fill[red!50] (0,0) rectangle (0.5,0.5);  % First bar
            \fill[red!50] (1,0) rectangle (1.5,2);    % Second bar
            \fill[red!50] (2,0) rectangle (2.5,1);    % Third bar
            \fill[red!50] (3,0) rectangle (3.5,1.5);  % Fourth bar

            % Add class names under each bar
            \node at (0.25,-0.5) {toy};  % Class name for first bar
            \node at (1.25,-0.5) {cat};  % Class name for second bar
            \node at (2.25,-0.5) {car};  % Class name for third bar
            \node at (3.25,-0.5) {dog};  % Class name for fourth bar
        \end{tikzpicture}
    };

    % Connect input to first hidden layer
    \foreach \source in {1,...,4}
        \foreach \dest in {1,...,5}
            \path (DI-\source) edge (DH1-\dest);
            
    % Connect first hidden layer to second hidden layer
    \foreach \source in {1,...,5}
        \foreach \dest in {1,...,5}
            \path (DH1-\source) edge (DH2-\dest);
    
    % Connect second hidden layer to third hidden layer
    \foreach \source in {1,...,5}
        \foreach \dest in {1,...,4}
            \path (DH2-\source) edge (DH3-\dest);

    % Connect third hidden layer to output layer
    \foreach \source in {1,...,4}
        \foreach \dest in {1,...,4}
            \path (DH3-\source) edge (DO-\dest);

    % SECOND DENSE NETWORK (Truncated NN)
    % Draw the input layer nodes
    \foreach \name / \y in {1,...,4}
    {
        \node[input neuron] (S2I-\name) at (\networksep + \y cm,0) {};
    }
    
    % Draw the first hidden layer nodes
    \foreach \name / \y in {1,...,5}
        \path[xshift=-0.5cm]
            node[hidden neuron] (S2H1-\name) at (\networksep + \y cm,-1*\layersep) {};
    
    % Draw the second hidden layer nodes (new hidden layer)
    \foreach \name / \y in {1,...,5}
        \path[xshift=-0.5cm]
            node[hidden neuron] (S2H2-\name) at (\networksep + \y cm,-2*\layersep) {};
    
    % Image for the Truncated NN mapping to a manifold
    \node at (7.5,-7)  (image2) {\includegraphics[width=3cm]{Figures/manifold.png}}; % Adjusted for reduced distance
    
    % Thick ARROW from Truncated NN to Manifold with text (no arrows on bracket)
    \draw[decoration={brace,amplitude=10pt,raise=5pt},decorate, thick, draw=black] 
        (S2H2-5.south east) -- (S2H2-1.south west) node[below=8pt] {};

    % Connect input to first hidden layer (Truncated NN)
    \foreach \source in {1,...,4}
        \foreach \dest in {1,...,5}
            \path (S2I-\source) edge (S2H1-\dest);

    % Connect first hidden layer to second hidden layer (Truncated NN)
    \foreach \source in {1,...,5}
        \foreach \dest in {1,...,5}
            \path (S2H1-\source) edge (S2H2-\dest);



    % Annotate the layers
    \node[annot,above of=DI-2, node distance=0.7cm, xshift=0.5cm] (before) {Baseline NN};
    \node[annot,above of=S2I-2, node distance=0.7cm, xshift=0.5cm] (after) {Truncated NN};
    
\end{tikzpicture}
\end{document}