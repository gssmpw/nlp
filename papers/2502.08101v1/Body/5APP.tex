\section{Experimental Settings}

\subsection{Dataset}\label{app:data}
Here we introduce datasets adopted for experiments. The detailed statistics of all datasets are reported in \autoref{tab:dataset}.
\begin{itemize}
    \item \textbf{Academic graphs}: This type of graph is formed by academic papers or authors and the citation relationships among them. Nodes in the graph represent academic papers or authors, and edges represent the citation relationships between papers or co-author relationships between two authors. The features of nodes are composed of bag-of-words vectors, which are extracted and generated from the abstracts and introductions of the academic papers. The labels of nodes correspond to the research fields of the academic papers or authors. ACM, Citeseer, WikiCS and UAI2010 belong to this type.
    \item \textbf{Co-purchase graphs}: This type of graph is constructed based on users' shopping behaviors. Nodes in the graph represent products. The edges between nodes indicate that two products are often purchased together. The features of nodes are composed of bag-of-words vectors extracted from product reviews. The category of a node corresponds to the type of goods the product belongs to. Computer and Photo belong to this type.

    \item \textbf{Social graphs}: This type of graph is formed by the activity records of users on social platforms. Nodes in the graph represent users on the social platform. The edges between nodes indicate the social relations between two users. Node features represent the text information extracted from the authors' homepage. The label of a node refers to the interest groups of users. BlogCatalog and Flickr belong to this type.
    
\end{itemize}

\section{Dataset}
\label{sec:dataset}

\subsection{Data Collection}

To analyze political discussions on Discord, we followed the methodology in \cite{singh2024Cross-Platform}, collecting messages from politically-oriented public servers in compliance with Discord's platform policies.

Using Discord's Discovery feature, we employed a web scraper to extract server invitation links, names, and descriptions, focusing on public servers accessible without participation. Invitation links were used to access data via the Discord API. To ensure relevance, we filtered servers using keywords related to the 2024 U.S. elections (e.g., Trump, Kamala, MAGA), as outlined in \cite{balasubramanian2024publicdatasettrackingsocial}. This resulted in 302 server links, further narrowed to 81 English-speaking, politics-focused servers based on their names and descriptions.

Public messages were retrieved from these servers using the Discord API, collecting metadata such as \textit{content}, \textit{user ID}, \textit{username}, \textit{timestamp}, \textit{bot flag}, \textit{mentions}, and \textit{interactions}. Through this process, we gathered \textbf{33,373,229 messages} from \textbf{82,109 users} across \textbf{81 servers}, including \textbf{1,912,750 messages} from \textbf{633 bots}. Data collection occurred between November 13th and 15th, covering messages sent from January 1st to November 12th, just after the 2024 U.S. election.

\subsection{Characterizing the Political Spectrum}
\label{sec:timeline}

A key aspect of our research is distinguishing between Republican- and Democratic-aligned Discord servers. To categorize their political alignment, we relied on server names and self-descriptions, which often include rules, community guidelines, and references to key ideologies or figures. Each server's name and description were manually reviewed based on predefined, objective criteria, focusing on explicit political themes or mentions of prominent figures. This process allowed us to classify servers into three categories, ensuring a systematic and unbiased alignment determination.

\begin{itemize}
    \item \textbf{Republican-aligned}: Servers referencing Republican and right-wing and ideologies, movements, or figures (e.g., MAGA, Conservative, Traditional, Trump).  
    \item \textbf{Democratic-aligned}: Servers mentioning Democratic and left-wing ideologies, movements, or figures (e.g., Progressive, Liberal, Socialist, Biden, Kamala).  
    \item \textbf{Unaligned}: Servers with no defined spectrum and ideologies or opened to general political debate from all orientations.
\end{itemize}

To ensure the reliability and consistency of our classification, three independent reviewers assessed the classification following the specified set of criteria. The inter-rater agreement of their classifications was evaluated using Fleiss' Kappa \cite{fleiss1971measuring}, with a resulting Kappa value of \( 0.8191 \), indicating an almost perfect agreement among the reviewers. Disagreements were resolved by adopting the majority classification, as there were no instances where a server received different classifications from all three reviewers. This process guaranteed the consistency and accuracy of the final categorization.

Through this process, we identified \textbf{7 Republican-aligned servers}, \textbf{9 Democratic-aligned servers}, and \textbf{65 unaligned servers}.

Table \ref{tab:statistics} shows the statistics of the collected data. Notably, while Democratic- and Republican-aligned servers had a comparable number of user messages, users in the latter servers were significantly more active, posting more than double the number of messages per user compared to their Democratic counterparts. 
This suggests that, in our sample, Democratic-aligned servers attract more users, but these users were less engaged in text-based discussions. Additionally, around 10\% of the messages across all server categories were posted by bots. 

\subsection{Temporal Data} 

Throughout this paper, we refer to the election candidates using the names adopted by their respective campaigns: \textit{Kamala}, \textit{Biden}, and \textit{Trump}. To examine how the content of text messages evolves based on the political alignment of servers, we divided the 2024 election year into three periods: \textbf{Biden vs Trump} (January 1 to July 21), \textbf{Kamala vs Trump} (July 21 to September 20), and the \textbf{Voting Period} (after September 20). These periods reflect key phases of the election: the early campaign dominated by Biden and Trump, the shift in dynamics with Kamala Harris replacing Joe Biden as the Democratic candidate, and the final voting stage focused on electoral outcomes and their implications. This segmentation enables an analysis of how discourse responds to pivotal electoral moments.

Figure \ref{fig:line-plot} illustrates the distribution of messages over time, highlighting trends in total messages volume and mentions of each candidate. Prior to Biden's withdrawal on July 21, mentions of Biden and Trump were relatively balanced. However, following Kamala's entry into the race, mentions of Trump surged significantly, a trend further amplified by an assassination attempt on him, solidifying his dominance in the discourse. The only instance where Trump’s mentions were exceeded occurred during the first debate, as concerns about Biden’s age and cognitive abilities temporarily shifted the focus. In the final stages of the election, mentions of all three candidates rose, with Trump’s mentions peaking as he emerged as the victor.
\subsection{Implementation Details}\label{app:imple}
For baselines, we refer to their official implementations and conduct a systematic tuning process on each dataset.
For \name, we employ a grid search strategy to identify the optimal parameter settings.
Specifically, We try the learning rate in $\{0.001, 0.005, 0.01\}$, dropout in $\{0.3, 0.5, 0.7\}$, dimension of hidden representations in $\{256, 512\}$,
$k$ in $\{4, 6, 8\}$, $\alpha$ in $\{0.1, \dots, 0.9\}$.
All experiments are implemented using Python 3.8, PyTorch 1.11, and CUDA 11.0 and executed on a Linux server with an Intel Xeon Silver 4210 processor, 256 GB of RAM, and a 2080TI GPU.


\section{Additional Experimental Results}\label{app:exp-results}
In this section, we provide the additional experimental results of ablation studies and parameter studies.




\subsection{Study of the center alignment loss}\label{app:exp-ca}
The experimental results of \name and \name-O on the rest datasets are shown in \autoref{fig:align-APP}.
We can observe that \name outperforms \name-O on most datasets.
Moreover, the effect of applying the center alignment loss on \name in sparse splitting is more significant than that in dense splitting.
The above observations are in line with those reported in the main text.
Therefore, we can conclude that the center alignment loss can effectively enhance the performance of \name in node classification.


\begin{figure}[ht]
\centering
\includegraphics[width=17cm]{Fig/alignloss-APP.pdf}
\caption{
Performances of \name with or without the center alignment loss.
}
\label{fig:align-APP}
\end{figure}




\subsection{Study of the token sequence generation}\label{app:exp-ts}
The experimental results of \name with different token sequence generation strategies on the rest datasets are shown in \autoref{fig:ts-APP}.
We can find that the additional experimental results exhibit similar observations shown in the main text.
This situation demonstrates the effectiveness of the token sequence generation with the proposed token swapping operation in enhancing the performance of tokenized GTs.
Moreover, we can also observe that the gains of introducing the token swapping operation vary on different graphs based on the results shown in \autoref{fig:ts} and \autoref{fig:ts-APP}.
This phenomenon may attribute to that different graphs possess unique topology and attribute information, which further impact the selection of node tokens. 
While \name applies the uniform strategy for selecting node tokens, which could lead to varying gains of token swapping.
This situation also motivates us to consider different strategies of token selection on different graphs as the future work.


\begin{figure}[ht]
\centering
\includegraphics[width=17cm]{Fig/ts-p-APP.pdf}
\caption{
Performances of \name with different token sequence generation strategies.
}
\label{fig:ts-APP}
\end{figure}



\subsection{Analysis of the swapping times $t$}\label{app:exp-t}
Here we report the rest results of \name with varying $t$, which are shown in \autoref{fig:t-APP}.
Similar to the phenomenons shown in \autoref{fig:t}, \name can achieve the best performance on all datasets when $t>2$.
Based on the results shown in \autoref{fig:t-APP} and \autoref{fig:t}, we can conclude that introducing tokens beyond first-order neighbors via the proposed token swapping operation can effective improve the performance of \name in node classification.


\begin{figure}[ht]
\centering
\includegraphics[width=17cm]{Fig/t-APP.pdf}
\caption{
Performances of \name with varying $t$.
}
\label{fig:t-APP}
\end{figure}



\subsection{Analysis of the augmentation times $s$}\label{app:exp-s}
Similar to analysis of $t$, the rest results of \name with varying $s$ are shown in \autoref{fig:s-APP}.
We can also observe the similar situations shown in \autoref{fig:s} that \name requires a larger value of $s$ under sparse splitting compared to dense splitting.
The situation demonstrates that introducing augmented token sequences can bring more significant performance gain in sparse splitting than that in dense splitting. 




\begin{figure}[t]
\centering
\includegraphics[width=17cm]{Fig/s-APP.pdf}
\caption{
Performances of \name with varying $s$.
}
\label{fig:s-APP}
\end{figure}