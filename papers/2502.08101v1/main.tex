%%%%%%%% ICML 2025 EXAMPLE LATEX SUBMISSION FILE %%%%%%%%%%%%%%%%%

\documentclass{article}

% Recommended, but optional, packages for figures and better typesetting:
\usepackage{microtype}
\usepackage{graphicx}
\usepackage{subfigure}
\usepackage{booktabs} % for professional tables

% hyperref makes hyperlinks in the resulting PDF.
% If your build breaks (sometimes temporarily if a hyperlink spans a page)
% please comment out the following usepackage line and replace
% \usepackage{icml2025} with \usepackage[nohyperref]{icml2025} above.
\usepackage{hyperref}


% Attempt to make hyperref and algorithmic work together better:
\newcommand{\theHalgorithm}{\arabic{algorithm}}

% Use the following line for the initial blind version submitted for review:
% \usepackage{icml2025}

% If accepted, instead use the following line for the camera-ready submission:
\usepackage[accepted]{icml2025}

% For theorems and such
\usepackage{amsmath}
\usepackage{amssymb}
\usepackage{mathtools}
\usepackage{amsthm}

% if you use cleveref..
\usepackage[capitalize,noabbrev]{cleveref}

\usepackage{color}

% general commands
\newcommand{\TODO}[1]{{\textbf{\textcolor{red}{TODO: #1}}}}
\newcommand{\NOTE}[1]{{\textbf{\textcolor{red}{NOTE: #1}}}}
\newcommand{\HK}[1]{\textcolor{red}{[KunHe: #1]}}
\newcommand{\operator}[1]{\textbf{\emph{#1}}}

\usepackage{xspace}
\newcommand{\name}[0]{SwapGT\xspace}



% Add a period to the end of an abbreviation unless there's one
% already, then \xspace.
\makeatletter
\DeclareRobustCommand\onedot{\futurelet\@let@token\@onedot}
\def\@onedot{\ifx\@let@token.\else.\null\fi\xspace}

\def\eg{\emph{e.g}\onedot} \def\Eg{\emph{E.g}\onedot}
\def\ie{\emph{i.e}\onedot, } \def\Ie{\emph{I.e}\onedot}
\def\cf{\emph{c.f}\onedot} \def\Cf{\emph{C.f}\onedot}
\def\etc{\emph{etc}\onedot} \def\vs{\emph{vs}\onedot}
\def\wrt{w.r.t\onedot} \def\dof{d.o.f\onedot}
\def\etal{\emph{et al}\onedot}
\def\st{\emph{s.t}\onedot}
\makeatother

% % \usepackage{algorithmicx}
% \usepackage{algpseudocode}  
% \usepackage{algorithm}

\renewcommand{\algorithmicrequire}{\textbf{Input:}}  % Use Input in the format of Algorithm  
\renewcommand{\algorithmicensure}{\textbf{Output:}} % Use Output in the format of Algorithm 


%%%%%%%%%%%%%%%%%%%%%%%%%%%%%%%%
% THEOREMS
%%%%%%%%%%%%%%%%%%%%%%%%%%%%%%%%
\theoremstyle{plain}
\newtheorem{theorem}{Theorem}[section]
\newtheorem{proposition}[theorem]{Proposition}
\newtheorem{lemma}[theorem]{Lemma}
\newtheorem{corollary}[theorem]{Corollary}
\theoremstyle{definition}
\newtheorem{definition}[theorem]{Definition}
\newtheorem{assumption}[theorem]{Assumption}
\theoremstyle{remark}
\newtheorem{remark}[theorem]{Remark}

% Todonotes is useful during development; simply uncomment the next line
%    and comment out the line below the next line to turn off comments
%\usepackage[disable,textsize=tiny]{todonotes}
\usepackage[textsize=tiny]{todonotes}


% The \icmltitle you define below is probably too long as a header.
% Therefore, a short form for the running title is supplied here:
\icmltitlerunning{Rethinking Tokenized Graph Transformers for Node Classification}
% \icmltitlerunning{Token Swapping for Enhanced Node Classification in Tokenized Graph Transformers}

\begin{document}

\twocolumn[
\icmltitle{Rethinking Tokenized Graph Transformers for Node Classification}
% \icmltitle{Token Swapping for Enhanced Node Classification in Tokenized Graph Transformers}
% It is OKAY to include author information, even for blind
% submissions: the style file will automatically remove it for you
% unless you've provided the [accepted] option to the icml2025
% package.

% List of affiliations: The first argument should be a (short)
% identifier you will use later to specify author affiliations
% Academic affiliations should list Department, University, City, Region, Country
% Industry affiliations should list Company, City, Region, Country

% You can specify symbols, otherwise they are numbered in order.
% Ideally, you should not use this facility. Affiliations will be numbered
% in order of appearance and this is the preferred way.
\icmlsetsymbol{equal}{*}

\begin{icmlauthorlist}
\icmlauthor{Jinsong Chen}{equal,cs,hccs}
\icmlauthor{Chenyang Li}{equal,cs,hccs}
\icmlauthor{GaiChao Li}{cs,hccs}
\icmlauthor{John E. Hopcroft}{hccs,cornell}
\icmlauthor{Kun He}{cs,hccs}
\end{icmlauthorlist}

\icmlaffiliation{cs}{School of Computer Science and Technology, Huazhong University of Science and Technology, China.}
\icmlaffiliation{hccs}{Hopcroft Center on Computing Science, Huazhong University of Science and Technology,  China}
\icmlaffiliation{cornell}{Department of Computer Science, Cornell University, USA}


\icmlcorrespondingauthor{Kun He}{brooklet60@hust.edu.cn}

% You may provide any keywords that you
% find helpful for describing your paper; these are used to populate
% the "keywords" metadata in the PDF but will not be shown in the document
% \icmlkeywords{Machine Learning, ICML}

\vskip 0.3in
]

% this must go after the closing bracket ] following \twocolumn[ ...

% This command actually creates the footnote in the first column
% listing the affiliations and the copyright notice.
% The command takes one argument, which is text to display at the start of the footnote.
% The \icmlEqualContribution command is standard text for equal contribution.
% Remove it (just {}) if you do not need this facility.

%\printAffiliationsAndNotice{}  % leave blank if no need to mention equal contribution
\printAffiliationsAndNotice{\icmlEqualContribution} % otherwise use the standard text.

\begin{abstract}
Node tokenized graph Transformers (GTs) have shown promising performance in node classification. The generation of token sequences is the key module in existing tokenized GTs which transforms the input graph into token sequences, facilitating the node representation learning via Transformer. In this paper, we observe that the generations of token sequences in existing GTs only focus on the first-order neighbors on the constructed similarity graphs, which leads to the limited usage of nodes to generate diverse token sequences, further restricting the potential of tokenized GTs for node classification. To this end, we propose a new method termed SwapGT. SwapGT first introduces a novel token swapping operation based on the characteristics of token sequences that fully leverages the semantic relevance of nodes to generate more informative token sequences. Then, SwapGT leverages a Transformer-based backbone to learn node representations from the generated token sequences. Moreover, SwapGT develops a center alignment loss to constrain the representation learning from multiple token sequences, further enhancing the model performance. Extensive empirical results on various datasets showcase the superiority of SwapGT for node classification.

\end{abstract}

%a short sentence describing your paper
% We propose a new graph transformer that introduces a novel token swapping operation to generate diverse token sequences to further enhance model performance. 

% key words
% graph Transformer, token swapping, token sequence, node classification


\section{Introduction}
\label{Sec:intro}

%nc --> GT
Node classification, the task of predicting node labels in a graph, is a fundamental  problem in graph data mining with numerous real-world applications. 
Graph Neural Networks (GNNs)~\cite{gnn1,gcn} have traditionally been the dominant approaches. 
However, the message passing mechanism inherent to GNNs suffers from some limitations, such as over-smoothing~\cite{oversm}, 
which prevents them from effectively capturing deep graph structural information and hinders their performance in downstream tasks. 


%category of GT, hybrid GTs and tokenized GTs
In contrast, Graph Transformers (GTs), which adapt the Transformer framework for graph-based learning, have emerged as a promising alternative, demonstrating impressive performance in node classification. Existing GTs can be broadly classified into two categories based on their model architecture: hybrid GTs and tokenized GTs. 

%hybrid GTs --> tokenzied GTs
Hybrid GTs combine the strengths of GNN and Transformer, using GNNs to capture local graph topology and Transformers to model global semantic relationships. 
However, recent studies have highlighted a key issue with this approach: 
directly modeling semantic correlations among all node pairs using Transformers can lead to the over-globalization problem~\cite{cob}, which compromises model performance. 

Tokenized GTs, on the other hand, generate independent token sequences for each node, which encapsulate both local topological and global semantic information.
Transformer models are then utilized to learn node representations from these token sequences. 
The advantage of tokenized GTs is that they limit the token sequences to a small, manageable number of tokens, naturally avoiding the over-globalization issue. 
In tokenized GTs, the token sequences typically include two types of tokens: neighborhood tokens and node tokens. Neighborhood tokens aggregate multi-hop neighborhood information of a target node, while node tokens are sampled based on the similarity between nodes.

% neighborhood tokens and node tokens
However, recent studies~\cite{vcrgt} have shown that neighborhood tokens often fail to preserve complex graph properties such as long-range dependencies and heterophily, limiting the richness of node representations. On the other hand, node tokens, generated through various sampling strategies, can better capture correlations between nodes in both feature and topological spaces~\cite{ansgt,ntformer}, making them more effective in preserving complex graph information. As a result, this paper focuses on node token-based GTs.

% summarize of node token --> top k sampling 
% top k sampling --> 1-hop neighbor in KNN graph
A recent study~\cite{ntformer} formalized the node token generation process as two key steps: similarity evaluation and top-$k$ sampling. 
In the first step, similarity scores between node pairs are calculated based on different similarity measures to preserve the relations of nodes in different feature spaces. While in the second step, the top $k$ nodes with the highest similarity scores are selected as node tokens to construct the token sequence. 
In this paper, we provide a new perspective on token generation in existing tokenized GTs. 
We identify that the token generation process can be viewed as a neighbor selection operation on the $k$-nearest neighbor ($k$-NN) graph.  
Specifically, a $k$-NN graph is constructed based on node pair similarities, and the neighbor nodes within the first-order neighborhood of each node are selected to form  the token sequence.


\begin{figure}[h]
    \centering    
    \includegraphics[width=7.5cm]{Fig/motivation2.pdf}
    \caption{The toy example of token generation on the $k$-NN graph. Previous methods only focus on 1-hop neighborhood to construct a single token sequence. While our method can flexibly select tokens from multi-hop neighborhoods to generate diverse token sequences.}
    \label{fig:motivation}
\end{figure}


\autoref{fig:motivation} illustrates this idea with a toy example.
We can observe that only a small subset of nodes is selected via existing token generation strategies, which indicates that existing methods have limited exploitation of the $k$-NN graph and are unable to comprehensively utilize the correlations between node pairs to explore more informative nodes with potential association to construct token sequences. 
This situation inevitably restricts the ability of tokenized GTs to capture informative node representations. 
Furthermore, in scenarios with sparse training data, relying on token sequences generated from a limited set of nodes may lead to over-fitting, as Transformers, being complex models, may struggle to generalize effectively.

% our method
This leads to the following research question: \textit{How can we more comprehensively and effectively exploit node pair correlations to generate diverse token sequences, thus improving the performance of tokenized GTs for node classification?} 
To address this, we introduce a novel method called \name. 
Specifically, \name introduces a new operation, token swapping, which leverages the  semantic correlations of nodes in the $k$-NN to swap tokens in different token sequences, generating more diverse token sequences.
By incorporating multiple token sequences, \name 
enables the model to learn more comprehensive node representations. 
Additionally, \name employs a Transformer-based backbone and introduces a tailored readout function to learn node representations from the generated token sequences.
To handle the case where a node is assigned multiple token sequences, we propose a center alignment loss to guide the training process. 
The main contributions of this paper are summarized as follows:
\begin{itemize}
    \item We propose a novel token swapping operation that fully exploits semantic correlations of nodes to generate diverse token sequences.
    \item We develop a Transformer-based backbone with a center alignment loss to learn node representations from the generated diverse token sequences.
    \item Extensive experiments on various datasets with different training data ratios showcase the effectiveness of \name in node classification. 
\end{itemize}








\section{Relation Work}
\label{Sec:rw}


\subsection{Graph Neural Networks}
GNNs~\cite{gnn3,gnn2,rlp,pamt,ncn} have shown remarkable performance in this task. 
Previous studies \cite{gat,jknet,sgc,appnp} have primarily concentrated on the incorporation of diverse graph structural information into the message-passing framework. 
Classic deep learning techniques, such as the attention mechanism \cite{gat,gatv2} and residual connections \cite{jknet,gcnii}, have been exploited to enhance the information aggregation on graphs. 
Moreover, aggregating information from high-order neighbors \cite{appnp,mixhop,h2gnn} or nodes with high similarity across different feature spaces \cite{geomgcn} has been demonstrated to be efficacious in improving model performance.

Follow-up GNNs have focused on the utilization of complex graph features to extract distinctive node representations. 
A prevalent strategy entails the utilization of signed aggregation weights \cite{fagcn,gprgnn,acmgnn,glognn} to optimize the aggregation operation. 
In this way, positive and negative values are respectively associated with low- and high-frequency information, thereby enhancing the discriminative power of the learned node representations.
Nevertheless, restricted by the inherent limitations of message-passing mechanism, the potential of GNNs for graph data mining has been inevitably weakened.
Developing a new graph deep learning paradigm has attracted great attention in graph representation learning. 



\subsection{Graph Transformers}
GTs~\cite{polynormer,agt,cob} have emerged as a novel architecture for graph representation learning and have exhibited substantial potential in node classification. 
A commonly adopted design paradigm for GTs is the combination of Transformer modules with GNN-style modules to construct hybrid neural network layers, called hybrid GTs~\cite{nodeformer,sgformer,specformer}. 
In this design, Transformer is employed to capture global information, while GNNs are utilized for local information extraction~\cite{graphgps,polynormer,signgt}.
Despite effectiveness, directly utilizing Transformer to model the interactions of all node pairs could occur the over-globalization issue~\cite{cob}, inevitably weakening the potential for graph representation learning.

An alternative yet effective design of GTs involves transforming the input graph into independent token sequences termed tokenized GTs \cite{ansgt,nagphormer,vcrgt,polyformer,ntformer}, which are then fed into the Transformer layer for node representation learning. 
Neighborhood tokens~\cite{nagphormer,nag+,polyformer,vcrgt,ntformer} and node tokens~\cite{ansgt,ntformer,vcrgt} are two typical elements in existing tokenized GTs.
The former, generally constructed by propagation approaches, such as random walk~\cite{nagphormer,nag+} and personalized PageRank~\cite{vcrgt}.
The latter is generated by diverse sampling methods based different similarity measurements, such as PageRank score~\cite{vcrgt} and attribute similarity~\cite{ansgt}. 
Since tokenized GTs only focus on the generated tokens, they naturally avoiding the over-globalization issue.

As pointed out in previous study~\cite{vcrgt}, node token oriented GTs are more efficient in capturing various graph information, such as long-range dependencies and heterophily, compared to neighborhood token oriented GTs.
However, we identify that previous methods only leverage a small subset of nodes as tokens for node representation learning, which could limit the model ability of deeply exploring graph information.
In this paper, we develop a new method \name that introduces a novel token swapping operation to produce more informative token sequences, further enhancing the model performance.


\section{Preliminaries}
\label{Sec:pre}


\subsection{Node Classification}
Suppose an attributed graph is denoted as $\mathcal{G}=(V, E, \mathbf{X})$ where $V$ and $E$ are the sets of nodes and edges in the graph.
$\mathbf{X} \in \mathbb{R}^{n \times d}$ is the attribute feature matrix, where $n$ and $d$ are the number of nodes and the dimension of the attribute feature vector, respectively.
We also have the adjacency matrix $\mathbf{A} = \{0, 1\}^{n\times n}$.
If there is an edge between nodes $v_i$ and $v_j$, $\mathbf{A}_{ij} = 1$; otherwise, $\mathbf{A}_{ij} = 0$. 
$\hat{\mathbf{A}}$ denotes the normalized version calculated as $\hat{\mathbf{A}}=(\mathbf{D}+\mathbf{I})^{-1/2}(\mathbf{A}+\mathbf{I})(\mathbf{D}+\mathbf{I})^{-1/2}$ where $\mathbf{D}$ and $\mathbf{I}$ are the diagonal degree matrix and the identity matrix, respectively.
In the scenario of node classification, each node is associated with a one-hot vector to identify the unique label information, resulting in a label matrix $\mathbf{Y} \in \mathbb{R}^{n \times c}$ where $c$ is the number of labels.
Given a set of labeled nodes $V_L$, the goal of the task is to predict the labels of the rest  nodes in $V-V_L$.


\subsection{Transformer}
Here, we introduce the design of the Transformer layer, which is the key module in most GTs. 
There are two core components of a Transformer layer~\cite{transformer}, named multi-head self-attention (MSA) and feed-forward network (FFN).
Given the model input $\mathbf{H}^{n\times d}$, the calculation of MSA is as follows:
\begin{equation}
    \mathrm{MSA}(\mathbf{H}) = (||_{i=1}^{m}head_i)\cdot \mathbf{W}_{o},
    \label{eq:msa}
\end{equation}
\begin{equation}
    head_i = \mathrm{softmax}\left(\frac{[(\mathbf{H}\cdot\mathbf{W}^{Q}_{i}) \cdot (\mathbf{H}\cdot\mathbf{W}^{K}_{i})^{\mathrm{T}}]}{\sqrt{d_k}}\right)\cdot (\mathbf{H}\cdot\mathbf{W}^{V}_{i}), 
    \label{eq:single-head}
\end{equation}
where $\mathbf{W}^{Q}_{i}$, $\mathbf{W}^{K}_{i}$ and $\mathbf{W}^{V}_{i}$ are the learnable parameter matrices of the $i$-th attention head.
$m$ is the number of attention heads.
$||$ denotes the vector concatenation operation.
$\mathbf{W}_{o}$ denotes a projection layer to obtain the final output of MSA.

FFN is constructed by two linear layers and one non-linear activation function:
\begin{equation}
    \mathrm{FFN}(\mathbf{H}) = \sigma(\mathbf{H}\cdot\mathbf{W}^{1})\cdot\mathbf{W}^{2},
    \label{eq:ffn}
\end{equation}
where $\mathbf{W}^{1}$ and $\mathbf{W}^{2}$ denote learnable parameters of the two linear layers and $\sigma(\cdot)$ denotes the GELU activation function.

\section{Methodology}
\label{Sec:method}

We begin by formally defining speculative decoding and database drafting and present our proposed method, Hierarchy Drafting (HD), which addresses the limitations of database drafting methods.

\subsection{Preliminary}

\paragraph{Speculative Decoding} 
At each step of speculative decoding, multiple tokens \(\tilde{\bm{x}}_{1:m}\) (i.e., draft token sequence) are drafted from an approximate model \(\mathcal{M}_q\) to predict future tokens of LLM \(\mathcal{M}_p\) (i.e., target model) for previous text tokens \(\bm{x}_{\leq t}\):
\begin{align}
    \tilde{\bm{x}}_{1:m} &\sim_m \mathcal{M}_q(\bm{x}_{\leq t}).
\end{align}

All draft token sequence \(\tilde{\bm{x}}_{1:m}\) are verified against the actual output of \(\mathcal{M}_p\). For example, in the greedy decoding, the tokens \(\bm{x}'_{t+1:t+m}\) are obtained for a given \(\tilde{\bm{x}}_{1:m}\) and \(\bm{x}_{\leq t}\) by solving the following equations in parallel:
\begin{align}
\begin{cases}
    x'_{t+1} &= \argmax P_{\mathcal{M}_p}(x | \bm{x}_{\leq t}), \\
    x'_{t+2} &= \argmax P_{\mathcal{M}_p}(x | \tilde{x}_{1}, \bm{x}_{\leq t}),\\
    &\dots \\
    x'_{t+m} &= \argmax P_{\mathcal{M}_p}(x | \tilde{\bm{x}}_{1:m}, \bm{x}_{\leq t}).
\end{cases}
\end{align}
Each token \(x'_{t+i}\) is verified against the corresponding draft token \(\tilde{x}_{t+i}\), starting from \(i = 0\) until the verification fails or \(i = m\) is reached.
To enhance the likelihood of acceptance, multiple draft token sequences \(\bm{\tilde{X}} = \{\tilde{\bm{x}}^i\}_{i=1}^N\) (i.e., draft set) are verified in parallel.
The specialized attention mask implements the parallel verification of the draft set, not causal attention mask~\cite{LAD, SpecInfer}.
In the sampling strategy, speculative sampling~\cite{SpecSampling} is commonly used to accept more tokens while maintaining identical output distributions of the target model.
In summary, the generation step is divided into two sub-steps with a single forward pass of the target model. The multiple accepted tokens are generated simultaneously, compressing the overall decoding process.
% In summary, the generation step consists of two sub-steps: a single forward pass of the target model, followed by simultaneous generation of multiple tokens accepted through verification, compressing the overall decoding process.


\paragraph{Database Drafting}
As shown on the left side of Figure~\ref{fig:overview}, the methods included in database drafting exploit the database \(\mathcal{D}\), having the prefix tokens as the key and the subsequent tokens as the value. Per each step of the generation process, the draft token sequence \(\bm{\tilde{x}}_{1:m}\) is retrieved from database \(\mathcal{D}\) for given previous tokens \(\bm{x}_{t-l:t}\):
\begin{align}
    \bm{\tilde{x}}_{1:m} \in \bm{\tilde{X}} &= \texttt{Ret}(\bm{x}_{t-l:t};\mathcal{D}),
\end{align}
where \(l\) and \(m\) are the length of previous tokens and draft token sequence. Subsequently, the verifying step is the same as other methods.

\subsection{Hierarchy Drafting}

We introduce Hierarchy Drafting (HD), which organizes tokens from diverse sources into three databases based on temporal locality and accesses them in order from the smallest to the largest scale. The overview and decoding process are depicted on the right side of Figure~\ref{fig:overview}.

% \section{Generating Hijacking Samples}
\section{\new{Methodology}}
\label{sec:methodology}
%A critical step of the Model Selection Hijacking Adversarial Attack is the generation of adversarial hijacking samples to be inject to the validation set. 
%We now present a novel methodology to design and generate such samples. %\lpasa{occhio che qua sembra che l'unica novelty sia la generzione di esempi!}
% A crucial phase in the Model Selection Hijacking Adversarial Attack involves generating adversarial hijacking samples for injection into the validation set. Among the novelties introduced by this work, we present a new methodology specifically for designing and generating these samples.
\new{
The MOSHI attack operates uniquely by injecting and substituting data points in the validation set with data from $\mathcal{S}^{Val}_{pois}$, disrupting the critical model selection phase without altering the training process or parameters.
This set, which the attacker carefully generates, will be used for the model selection phase, which in turn will return a model $\tilde{h}_{\mathfrak{c}^*}$:
    \begin{equation}
        \label{best_poison}
 \tilde{h}_{\mathfrak{c}^*} = \argmin_{h_{\mathfrak{c}} : \mathfrak{c} \in \mathfrak{C}} \mathcal{L}_{Val}(h_{\mathfrak{c}}, \mathcal{S}^{Val}_{pois}).
    \end{equation}
The selected model $\tilde{h}_{\mathfrak{c}^*}$ is different from $h_{\mathfrak{c}^*}$, as now, the poisoned validation set no longer allows for selecting a better, more generalized, model, but selects one that has a configuration of hyper-parameters which maximizes the hijack metric, chosen by the adversary. 
Thus, a central aspect of this approach involves generating adversarial hijacking samples crafted explicitly for injection into the validation set.
Among the novelties introduced in this work, we present a specialized methodology for designing and generating these samples (Section~\ref{subsec:generation}) and the hijack metrics used in our study (Section~\ref{ssec.hm-theory}).
}

% \subsection{Overview}
% The goal of the adversary is to assume control of the MS phase by injecting and substituting data points in the validation set with data from $\mathcal{S}^{Val}_{pois}$. This set, which is carefully generated by the attacker, will be used for the model selection phase, which in turn will return a model $\tilde{h}_{\mathfrak{c}^*}$:
%     \begin{equation}
%         \label{best_poison}
%  \tilde{h}_{\mathfrak{c}^*} = \argmin_{h_{\mathfrak{c}} : \mathfrak{c} \in \mathfrak{C}} \mathcal{L}_{Val}(h_{\mathfrak{c}}, \mathcal{S}^{Val}_{pois}).
%     \end{equation}

% The selected model $\tilde{h}_{\mathfrak{c}^*}$ is different from $h_{\mathfrak{c}^*}$, as now, the poisoned validation set, no longer allows for selecting a better, more generalized, model, but selects one that has a configuration of hyper-parameters which maximizes the hijack metric, chosen by the adversary. 

% \subsection{Generative Process}
\subsection{\new{Adversarial Sample Generation}}
\label{subsec:generation}
\new{
Although our adversarial sample generation model is based on the Variational Auto Encoder (VAE) architecture (Section~\ref{subsub:vae}), we introduce a variation of the conditional VAE architecture designed for the generation of hijacking samples (Section~\ref{subsub:hvae}).
}
\subsubsection{Variational Auto Encoder (VAE)}
\label{subsub:vae}
We design our generative process using a Variational Auto Encoder (VAE)~\cite{kingma2013auto}, which is an extension of more traditional Autoencoders~\cite{hinton2006reducing}. VAE consists of two modules: first, an \textit{encoder} which learns a \textit{posterior} recognition model $q_{\phi}(z|x)$, encoding an input $x$ to a latent representation $z$; second, a \textit{decoder} that generates samples from the latent space $z$ via the likelihood model $p_{\theta}(x|z)$. $\phi$ and $\theta$ are learning parameters. 
In contrast with standard autoencoders, VAEs enforce a continuous prior distribution $p(z)$, usually set to the Gaussian. This forces the model to encode the entire input distribution to the latent code rather than memorizing single data points. 
Traditional VAEs are trained with the following loss:
\begin{equation} \small
% \begin{split}
    \mathcal{L}_{VAE}(\phi, \theta) = KL(q_{\phi}(z|x) || p(z)) %\\
    -\mathbb{E}_{q_{\phi}(z|x)}(\log p_{\theta}(x|z)), 
% \end{split}
\end{equation}
where $KL$ is the Kullback-Leibler divergence~\cite{kullback1951information} that is a regularizer to keep the posterior distribution close to the prior. The second term is a simple reconstruction loss. 
For the scope of this work, we utilize a Conditional VAE (CVAE) that augments the latent space with information about the true label of a given sample~\cite{sohn2015learning}.  
%
\subsubsection{Hijacking VAE}
\label{subsub:hvae}
We now introduce Hijacking VAE (HVAE), a variation of the more traditional CVAE that is specifically designed to generate hijacking samples to produce $\mathcal{S}^{Val}_{pois}$.
These samples are created in such a way that, when used for computing $\mathcal{L}_{Val}$, the lower the models' hijack metric, the more significant the increase of their validation loss, hence swaying the model selection phase into returning the model that has the highest hijack metric (which has been the least penalized).
We design the HVAE loss function as follows:
    \begin{equation}
        \label{lossMHVAE}
 \mathcal{L}_{\mathrm{HVAE}} = (\mathcal{L}_{\mathrm{rec}} + \mathcal{L}_{\mathrm{KLD}} - Hj_{cost}(\mathfrak{C})) ^ 2.
    \end{equation}
Here, the terms $\mathcal{L}_{\mathrm{rec}}$ and $\mathcal{L}_{\mathrm{KLD}}$ represents the reconstruction loss and the KL divergence, as in the traditional VAE. 
The novel factor of the loss is represented by the third term $Hj_{cost}(\mathfrak{C})$.
This is the pivotal factor of the attack, defined as follows (with $\Lambda = \mathfrak{C}$):

\begin{equation} \label{cost}
     Hj_{cost}(\mathfrak{C}) = \frac{1}{|\mathfrak{C}|}\sum_{\mathfrak{c} \in \mathfrak{C}} \alpha \cdot \mathcal{L}_{Val}(h_{\mathfrak{c}}, \mathcal{S}_{gen})
\end{equation}
with 
\begin{equation} \label{alpha}
 \alpha = \frac
     {\underset{\lambda \in \Lambda}{\max} \{m(h_{\lambda}, \mathcal{S}^{Val})\} - m(h_ {\mathfrak{c}}, \mathcal{S}^{Val})}
     {\underset{\lambda \in \Lambda}{\max} \{m(h_{\lambda}, \mathcal{S}^{Val})\} - \underset{\lambda \in \Lambda}{\min} \{m(h_{\lambda}, \mathcal{S}^{Val})\}}. 
\end{equation}
 
We now explain the rationale behind Equation~\ref{cost}, which is an average of scores that are assigned to each model $\mathfrak{c} \in \mathfrak{C}$. 
The coefficient $\alpha \in \mathbb{R}$ (see Equation~\ref{alpha}) is computed by normalizing the difference between the maximum hijack metric achievable by a model $h_{\lambda}$ with $\lambda \in \Lambda = \mathfrak{C}$ and the metric of the current model.
$\alpha$ yields higher penalties the lower the hijack metric of the model $h_\mathfrak{c}$, reaching 0 if the considered model has the highest metric. This value is fixed for each model and can be computed independently of the HVAE training.
On the opposite, the second term, $\mathcal{L}_{Val}$, assesses the quality of the generative process to produce effective hijacking samples, as it computes the loss of model $h_\mathfrak{c}$ over $S_{gen}$. It is therefore computed at HVAE training time. 
\par
Ideally, we intend to reward higher $Hj_{cost}$, as higher values imply higher losses toward those models with lower hijack metrics.
Therefore, in our loss function, we aim to maximize this value.  
During the training of the HVAE, by minimizing Equation~\ref{lossMHVAE}, we work toward:
\begin{itemize}
    \item diminishing the reconstruction loss $\mathcal{L}_{\mathrm{rec}}$, so that generated samples can resemble the original operations;
    \item diminishing the $\mathcal{L}_{\mathrm{KLD}}$ for obtaining a useful probability distribution;
    \item increasing the hijacking cost function $Hj_{cost}(\mathfrak{C})$. As the penalty value is fixed, by raising Equation~\ref{cost}, we aim at generating samples $\mathcal{S}_{gen}$, which increase the validation loss based on the magnitude of the penalty itself.
    Models with lower hijack metrics incur higher penalties, leading to increased validation loss on the generated samples. This ensures the samples are crafted to produce lower validation loss values for models with the highest hijack metrics.
    % Therefore, those models with lower hijack metrics will have higher penalties, which results in higher validation loss computed on the generated samples. This allows the creation of samples that, when used for evaluating the validation loss of a model, will return a lower value for the ones with the highest hijack metric.
\end{itemize}
% A graphical representation of how  $\mathcal{S}^{Val}_{pois}$ is generated, can be found in Figure~\ref{MHVAE}. 
% By training the HVAE with the objective function Equation~\ref{lossMHVAE}, it is possible to encode a distribution, that is unlike the input samples one -- usually learned by vanilla VAE -- as it governs the generation of samples such that, when injected in the validation set, they can provide a penalty on the validation loss of models at lower hijack metric.
% We report in Algorithm~\ref{alg.HVAE} the HVAE training procedure.
A graphical overview of $\mathcal{S}^{Val}_{pois}$ generation is shown in Figure~\ref{MHVAE}.
By training the HVAE with the objective function in Equation~\ref{lossMHVAE}, the model encodes a distribution distinct from the input samples’ usual one, enabling the generation of validation samples that penalize models with lower hijack metrics.
The HVAE training procedure is detailed in Algorithm~\ref{alg.HVAE}.

% \begin{figure*}[!htbp]
%     \footnotesize
%     \centering
%     \includesvg[width=.775\textwidth]{figures/MHVAE-v2.drawio}
%     \caption{Schematic representation of the generation process of $\mathcal{S}^{Val}_{pois}$. For simplicity, we reported samples from the MNIST dataset~\cite{lecun2010mnist}.}
%     % \caption{Schematic representation of $\mathcal{S}^{Val}_{pois}$ generation using MNIST samples~\cite{lecun2010mnist} for simplicity.}
%     \label{MHVAE}
% \end{figure*}

\begin{figure*}[!htbp] %% ARXIV
    \footnotesize
    \centering
    \includesvg[width=.775\textwidth]{figures/MHVAE-v2.drawio}
    \caption{Schematic representation of the generation process of $\mathcal{S}^{Val}_{pois}$. For simplicity, we reported samples from the MNIST dataset~\cite{lecun2010mnist}.}
    % \caption{Schematic representation of $\mathcal{S}^{Val}_{pois}$ generation using MNIST samples~\cite{lecun2010mnist} for simplicity.}
    \label{MHVAE}
\end{figure*}

\begin{algorithm}[H]
\footnotesize
    \caption{Hijack VAE Training Algorithm}
    \begin{algorithmic}[1]
        \State \textbf{Input:} HVAE model with random weights, training data $\mathcal{S}$, $\alpha_{\mathfrak{C}}$, $h_{\mathfrak{C}}$, number of epochs $epochs$
        \State \textbf{Output:} Trained HVAE model
        \For{$e \gets 1$ to $epochs$}
            \For{$\bm{x}$, $y$ in $\mathcal{S}$}  % are batches
                \State $\hat{\bm{x}} \gets $ HVAE.decode(HVAE.encode($\bm{x}$))  % reconstruct input
                \State rec\_loss $\gets \mathcal{L}_{\mathrm{rec}}(\bm{x}, \hat{\bm{x}})$  % reconstruction loss
                \State kl\_loss $\gets \mathcal{L}_{\mathrm{KLD}}(\mathrm{HVAE})$  % KLD loss
                \State $\hat{\bm{x}}_{gen} \gets$ HVAE.decode(gaussian\_noise)  % generate samples from randomly sampled noise
                \State generated\_val\_loss $\gets \mathcal{L}_{Val}(h_{\mathfrak{C}}, \hat{\bm{x}}_{gen})$  % validation loss of all knowm models on the generated samples
                \State hijack\_cost $\gets Hj_{cost}(\alpha_{\mathfrak{C}}, \mathrm{generated\_val\_loss})$  % compute hijack cost using the hijack cost penalty & the loss of the generated samples
                \State total\_loss $ \gets(\mathrm{rec\_loss + kl\_loss - hijack\_cost})^2 $  % obtain the total loss
                \State HVAE.backward\_propagation\_step(total\_loss)  % update weights
            \EndFor
        \EndFor
        \State \textbf{return} HVAE
    \end{algorithmic}
    \label{alg.HVAE}
\end{algorithm}

\subsection{Hijack Metric}
\label{ssec.hm-theory}
Generally, the purpose of a hijack metric $m$ is to produce damage to the target victim. 
\new{
We now introduce four distinct hijack metrics that impact an ML system in three different ways, i.e., generalization capabilities (Section~\ref{subsub:generalization}), latency (Section~\ref{subsub:latency}), and energy consumption (Section~\ref{subsub:energy}).
}
Note that MOSHI is not limited to such metrics, and future investigations might define different attack objectives. 

% \subsubsection{Weaken the Generalization Capabilities}
\subsubsection{\new{Generalization Capability Attack}}
\label{subsub:generalization}
This first intuitive hijack metric objective is to impact the victim model overall performance. 
Here, the objective of the attack under this metric is to choose a model that less generalizes to unseen data (e.g., test set), and therefore the result of an underfitting or overfitting training.
%Therefore, this case can be reconducted to the more traditional 
Therefore, this case can be considered a form of the more traditional \textit{model poisoning attack}~\cite{tian2022comprehensive}.
The metric $m$ -- that we named \textit{Generalization Metric} -- can simply compute the loss of a target model on an unseen dataset (\textit{e.g., validation set}). 
%
\subsubsection{Latency Attack}
\label{subsub:latency}
Increased latency in ML predictions can significantly impact the performance and usability of ML systems.
Higher latency leads to delayed responses, which can degrade user experience, particularly in real-time applications such as autonomous driving, financial trading, and interactive systems. Additionally, increased latency can hinder the efficiency of decision-making processes, as timely data processing is crucial for accurate and effective outcomes. This delay can also exacerbate the accumulation of errors, potentially compromising the reliability and accuracy of the ML model's predictions.
Therefore, an attacker might aim to induce the model selection to peak a model that results in slower predictions, on average, when deployed. 
The function $m$ -- that we named \textit{Latency Metric} -- can be designed by observing the time required by a target model to predict a set of unseen datasets (\textit{e.g., validation set}). 
%
\subsubsection{Energy Consumption Attack}
\label{subsub:energy}
Similarly to what is discussed in the motivation of the latency attack, increasing the overall energy consumption might lead to resource exhaustion. 
We inspire this metric based on the \textit{sponge attack}~\cite{shumailov2021sponge}. 
In our work, we consider two distinct metrics that measure energy consumption. 
\begin{itemize}
    \item \textit{Energy Consumption}: an estimation of the energy consumption of the model utilization that can be obtained through the OS energy consumption hosting such model. 
    \item \textit{$\ell_0$ norm}: the $\ell_0$ norm of the activations of the neurons in the network, obtained by summing the non-zero activations of each ReLU Layer in the model when it is processing a sample $\bm{x}$, then computing the mean for all samples $\bm{x} \in \mathcal{X}$. 
\end{itemize}
We opt to include this metric as \cite{cina2022energy} showed, there exists a strong link between the $\ell_0$ norm of a model and its energy consumption.
For instance, we report in Figure~\ref{l0_energy} the observed correlation between these two metrics in our experimental setting (which we will describe in the upcoming section).  

\begin{figure}[!htbp]
    \footnotesize
    \centering
    \includesvg[width = .8\linewidth]{figures/MNIST-normalized-energy-l0-v2}
    % \vspace{-15pt}
    \caption[Histogram comparing $\ell_0$ norm and energy consumption per layer.]{Histogram comparing $\ell_0$ norm and energy consumption per layer on FFNNs from 1 to 10 layers of 32 neurons, trained on MNIST dataset with a learning rate of 0.001.}
    \label{l0_energy}
\end{figure}


\subsection{White-box vs Black-box scenarios}
The HVAE requires knowledge about the target models, as described in Equation~\ref{cost} in the $Hj_{cost}$. Models in the grid are utilized for measuring their performance with the hijack metric and for understanding the quality of the dataset $S_{gen}$ produced by HVAE.
As we previously anticipated, in our work we consider a white-box and black-box case study. In the former, we assume the attacker has access to the exact models of the model grid. In the latter, the attacker has no such knowledge.
However, we assume that the attacker has knowledge about both training and validation sets. We can therefore leverage the \textit{adversarial transferability} of attacks. 
\par
Adversarial transferability in AML refers to the phenomenon where adversarial examples crafted to deceive one ML model can also deceive other models, even if they have different architectures or were trained on different datasets~\cite{demontis2019adversarial, alecci2023your}. This property is significant because it highlights the vulnerability of ML systems to attacks that are not specifically tailored to them, thereby posing a broader security risk.

\paragraph{Observation: Temporal Locality}
The main idea behind database drafting is that some tokens are easy to retrieve from the database because they exhibit temporal locality—meaning they tend to be repeated within or across the generation processes. 
However, note that not all draft token sequences share the same level of temporal locality during generation. 
We analyze the pattern of unique 4-grams during 100 text generations on Spec-Bench~\cite{Spec_Survey}, as shown in Figure~\ref{fig:generation}. The results reveal that certain 4-grams are frequently repeated and exhibit varying locality levels.
Specifically, the blue dots and the right small plot in Figure~\ref{fig:generation} illustrate local redundancy, where the same 4-gram appears multiple times within a single generation step. This reflects high temporal locality within a single generation rather than across multiple generations. In contrast, the red dots in Figure~\ref{fig:generation} highlight a pattern where the model repeatedly generates the same 4-grams at different stages of the generation process, illustrating its tendency to reuse familiar sequences over time.
Additionally, the lower plot of Figure~\ref{fig:generation} presents the frequency study of sampled red and blue dots, demonstrating that some tokens exhibit high temporal locality within a specific context, while others maintain consistent locality across generation processes.
Therefore, given the varying temporal locality of tokens throughout the generation process, drafting steps should prioritize tokens with higher temporal locality over others.

% For example, when an LLM solves a math problem like, “The vertices of a triangle are at points (0, 0), (-1, 1), and (3, 3). What is the area of the triangle?”, the coordinates are frequently repeated.
% Next, frequently generated phrases by LLMs, such as “as an AI assistant,” show moderate locality, as they often appear across various generation processes for LLM-generated texts. 
% Finally, grammatical patterns and universal phrases commonly used by humans have the lowest locality, as they are statistically frequent across all types of texts yet do not constantly occur in each generation process.

\paragraph{Database Design}  
 Based on the temporal locality of draft token candidates, we design three types of databases to categorize them. 
\textbf{1) Context-dependent DB} (\(\mathcal{D}_c\)) contains tokens highly relevant to the specific context of the generation process, such as the blue dots in the Figure~\ref{fig:generation}. 
This includes tokens from the input prompt, tokens generated through parallel decoding, tokens discarded during the generation process, and others that are highly relevant to a given context.  
\(\mathcal{D}_c\) is lookup table with the prefix tokens, \(\bm{x}_{1:l}\), as the key and the subsequent tokens, \(\bm{x}_{l:l+m}\), as the value.
Also, \(\mathcal{D}_c\) is consistently updated during each forward step and initialized when the following generation process is started. 
The database follows the Least Recently Used (LRU) policy for draft sequence updates.
\textbf{2) Model-dependent DB} (\(\mathcal{D}_m\)) stores tokens frequently generated by LLM regardless of context, as represented by the red dots in Figure~\ref{fig:generation}.
Top-$k$ frequently generated token sequences, $\bm{x}_{1:l+m}$, are sampled from the model-generated texts, with $\bm{x}_{1:l}$ as the key and $\bm{x}_{l+1:l+m}$ as the value.
For \(\mathcal{D}_c\) and \(\mathcal{D}_m\), the maximum size of values for a single key is the same as the maximum draft set size \(N\). 
\textbf{3) Statistics-dependent DB} (\(\mathcal{D}_s\)) draws its tokens from large text corpora to capture universal phrases commonly used in the language. 
Although these tokens are frequent, they occur less consistently across processes than those in \(\mathcal{D}_m\).
To efficiently retrieve the sequence from a large corpus, we utilize a suffix array~\cite{suffix_array} following the implementation of~\citet{REST}.
Implementation details are in \S\ref{sec:experiement}.


Our database design yields three distinct advantages. First, it integrates diverse sources into multiple databases, enabling us to leverage each source’s strengths for robust acceleration across various tasks. Then, each database’s size decreases as the tokens’ temporal locality increases since tokens with higher locality are rarer, providing an opportunity to optimize drafting latency. Finally, the design is \textit{plug-and-play}, easily integrating additional token sources by assigning them to the appropriate database based on their temporal locality.

\setlength{\textfloatsep}{1em}% Remove \textfloatsep

\begin{algorithm}[t!]
\caption{\small Decoding Process with Hierarchy Drafting}\label{alg:HD_process}
\small
\begin{algorithmic}[1]
\Require Target LLM $\mathcal{M}_p$, databases $(\mathcal{D}_c, \mathcal{D}_m, \mathcal{D}_s)$, input text sequence $\bm{x}_{\le t}$, target sequence length $T$, the size of prefix tokens $l$, the size of draft token sequence $m$, the size of draft set $N$;
\State $n \leftarrow t$\;
\While{$n < T$ and \texttt{[EOS]} $ \notin \bm{x}_{1:n}$}
    \State \textcolor{blue}{// \textit{Drafting Step: Hierarchical access to three databases until the size of the draft set $\bm{\tilde{X}}$ is $N$.}}
    \State $\bm{\tilde{X}} \leftarrow \texttt{Ret}(\bm{x}_{n-l:n};\mathcal{D}_c)$
    \If{$|\bm{\tilde{X}}| < N$}
        \State $\bm{\tilde{X}} \leftarrow \bm{\tilde{X}} \cup \texttt{Ret}(\bm{x}_{n-l:n};\mathcal{D}_m)$
    \EndIf 
    \If{$|\bm{\tilde{X}}| < N$}
        \State $\bm{\tilde{X}} \leftarrow \bm{\tilde{X}} \cup \texttt{Ret}(\bm{x}_{n-l:n};\mathcal{D}_s)$
    \EndIf 
    \State \textcolor{blue}{// \textit{Verification Step: Verify the draft token sequence in $\bm{\tilde{X}}$ and generate additional tokens for updating $\mathcal{D}_c$.}}
    \State $\bm{x}_{n:n+i}, \bm{\hat{x}} \sim_i \mathcal{M}_p(\bm{x}_{\le n}, \bm{\tilde{X}})$
    \State $\mathcal{D}_c \leftarrow \text{Update}(\mathcal{D}_c, \bm{\hat{x}})$
    % \If{\texttt{[EOS]} in $\bm{x}_{t:t+i}$}
    %     \State BREAK
    % \EndIf
    \State $n \gets n+i$
\EndWhile
\end{algorithmic}
\end{algorithm}
% \vspace{0-}

\paragraph{Hierarchical Access}
Using the three databases designed with the temporal locality in mind, we retrieve draft token sequence \(\bm{\tilde{x}}_{1:m}\) for the given previous input \(\bm{x}_{t-l:t}\).
Database access order is based on the degree of temporal locality within the current generation process; thereby, the access starts with \(\mathcal{D}_c\).
Access then proceeds to \(\mathcal{D}_m\), which has high locality across generations, and finally \(\mathcal{D}_s\), with moderate locality across generations, until draft set \(\bm{\tilde{X}}\) accumulates a sufficient number of candidates as pre-defined hyperparameter \(N\).
These accesses leverage the locality of the draft token sequence to enhance drafting accuracy and minimize latency overhead, preserving the benefits of drafting.

\paragraph{Decoding Process}
We introduce the inference process of speculative decoding with our proposed method, HD. 
% First, for a given previous input \(\bm{x}_{t-l, t}\), the set of draft token \(\bm{\tilde{X}}\) are acquired from the three databases with hierarchical access. 
First, for a given previous input \(\bm{x}_{t-l, t}\), we acquire the set of draft token \(\bm{\tilde{X}}\) from the three databases with hierarchical access. 
% Then, the target LLM \(\mathcal{M}_p\) verifies the draft token sequences simultaneously generating the additional tokens \(\bm{\hat{x}}\) for updating context-dependent DB either through parallel decoding~\cite{ParallelDecoding, LAD} or by recycling wasted tokens~\cite{trashintotreasure}. 
Then, the target LLM \(\mathcal{M}_p\) verifies the draft token sequences while simultaneously generating the additional tokens \(\bm{\hat{x}}\). 
These tokens are used to update the context-dependent DB either through parallel decoding~\cite{ParallelDecoding, LAD} or by recycling wasted tokens~\cite{trashintotreasure}. 
These processes are repeated iteratively until either the \texttt{[EOS]} token is generated or the sequence reaches the pre-defined maximum length \(T\).
Details of the decoding are depicted in Algorithm~\ref{alg:HD_process}.



\section{Experiments}
\label{Sec:exp}
% In this section, we first introduce the experimental settings, including datasets, baselines and implementation details. Then we conduct a series of experiments, involving performance comparison, parameter sensitiveness analysis and ablation study, to comprehensively evaluate our proposed \name.


% \begin{table*}[ht]
\centering
\caption{Comparison of all models in terms of mean accuracy $\pm$ stdev (\%) under dense splitting. The best results appear in \textbf{bold}. The second results appear in \underline{underline}.}
\scalebox{0.80}{
\begin{tabular}{lcccccccccccc}
\toprule
Dataset& Photo & ACM & Computer  &Citeseer &WikiCS& BlogCatalog & UAI2010 & Flickr   \\
$\mathcal{H}$& 0.83 & 0.82 & 0.78  &0.74 & 0.66& 0.40 & 0.36& 0.24  \\ \hline

SGC& 93.74\tiny{$\pm$0.07} & 93.24\tiny{$\pm$0.49} &88.90\tiny{$\pm$0.11} & 76.81\tiny{$\pm$0.26} & 76.67\tiny{$\pm$0.19}& 72.61\tiny{$\pm$0.07} &69.87\tiny{$\pm$0.17} &47.48\tiny{$\pm$0.40}  \\

APPNP&{94.98\tiny{$\pm$0.41}}& 93.00\tiny{$\pm$0.55} &\underline{91.31\tiny{$\pm$0.29} } & {77.52\tiny{$\pm$0.22}}  & 81.96\tiny{$\pm$0.14}
& 94.77\tiny{$\pm$0.19} &{77.41\tiny{$\pm$0.47}} & 84.66\tiny{$\pm$0.31}  \\

GPRGNN& 94.57\tiny{$\pm$0.44} & 93.42\tiny{$\pm$0.20} &90.15\tiny{$\pm$0.34}  & 77.59\tiny{$\pm$0.36} &82.43\tiny{$\pm$0.29}& {94.36\tiny{$\pm$0.29} }&76.94\tiny{$\pm$0.64} &{85.91\tiny{$\pm$0.51}}    \\


FAGCN& 94.06\tiny{$\pm$0.03} & 93.37\tiny{$\pm$0.24} &83.17\tiny{$\pm$1.81} & 76.19\tiny{$\pm$0.62} 
& 79.89\tiny{$\pm$0.93}& 79.92\tiny{$\pm$4.39} &72.17\tiny{$\pm$1.57} & 82.03\tiny{$\pm$0.40}   \\

BM-GCN& 95.10\tiny{$\pm$0.20} &{93.68\tiny{$\pm$0.34} }&91.28\tiny{$\pm$0.96} & 77.91\tiny{$\pm$0.58} & {83.90\tiny{$\pm$0.41} }&94.85\tiny{$\pm$0.42} & 77.39\tiny{$\pm$1.13} &   83.97\tiny{$\pm$0.87} \\

ACM-GCN& 94.56\tiny{$\pm$0.21} & 93.04\tiny{$\pm$1.28} &85.19\tiny{$\pm$2.26} &77.62\tiny{$\pm$0.81} 
&\underline{83.95\tiny{$\pm$0.41}}
& 94.53\tiny{$\pm$0.53} &76.87\tiny{$\pm$1.42} & 83.85\tiny{$\pm$0.73}  \\

\hline
NAGphormer&  \underline{95.47\tiny{$\pm$0.29}} & 93.32\tiny{$\pm$0.30} &90.79\tiny{$\pm$0.45} &  77.68\tiny{$\pm$0.73}& 
83.61\tiny{$\pm$0.28} & 94.42\tiny{$\pm$0.63} &76.36\tiny{$\pm$1.12} & 86.85\tiny{$\pm$0.85}  \\

SGFormer& 92.93\tiny{$\pm$0.12} & 93.79\tiny{$\pm$0.34} &81.86\tiny{$\pm$3.82} &  77.86\tiny{$\pm$0.76}& 79.65\tiny{$\pm$0.31} & 94.33\tiny{$\pm$0.19} &57.98\tiny{$\pm$3.95} & 61.05\tiny{$\pm$0.68}  \\


Specformer& 95.22\tiny{$\pm$0.13} & 93.63\tiny{$\pm$1.94} &85.47\tiny{$\pm$1.44} & 77.96\tiny{$\pm$0.89}&  83.74\tiny{$\pm$0.62}  & 94.21\tiny{$\pm$0.23} &73.06\tiny{$\pm$0.77} & 86.55\tiny{$\pm$0.40} \\

VCR-Graphormer
&95.38\tiny{$\pm$0.51} & 93.11\tiny{$\pm$0.79} &90.47\tiny{$\pm$0.58} & 77.21\tiny{$\pm$0.65}& 80.82\tiny{$\pm$0.72} & 94.19\tiny{$\pm$0.17} &76.08\tiny{$\pm$0.52} & 85.96\tiny{$\pm$0.55}    \\

PolyFormer
& 95.45\tiny{$\pm$0.21} & \underline{94.27\tiny{$\pm$0.44}} &90.87\tiny{$\pm$0.74} & \underline{78.03\tiny{$\pm$0.86}}& 83.79\tiny{$\pm$0.75} & \underline{95.08\tiny{$\pm$0.43} }&\underline{77.92\tiny{$\pm$0.82} }& \underline{87.01\tiny{$\pm$0.57} }   \\


\hline

\name & 
\textbf{95.92\tiny{$\pm$0.18}} & \textbf{94.98\tiny{$\pm$0.41}} & 
\textbf{91.73\tiny{$\pm$0.72}} & 
\textbf{78.49\tiny{$\pm$0.95}} & 
\textbf{84.52\tiny{$\pm$0.63}}& \textbf{95.93\tiny{$\pm$0.56}} & \textbf{79.06\tiny{$\pm$0.73}} & \textbf{87.56\tiny{$\pm$0.61}}  \\     
 \toprule
\end{tabular}
}

\label{tab:dense-ncre}
\end{table*}




\subsection{Dataset}
We adopt eight widely used datasets, involving homophily and heterophily graphs: 
Photo~\cite{nagphormer}, ACM~\cite{acm}, Computer~\cite{nagphormer}, BlogCatalog~\cite{socialnets}, UAI2010~\cite{amgcn}, Flickr~\cite{socialnets} and Wiki-CS~\cite{roman}.
The edge homophily ratio~\cite{glognn} ${H}(\mathcal{G})\in[0,1]$ is adopted to evaluate the graph's homophily level. 
${H}(\mathcal{G}) \rightarrow 1$ means strong homophily, 
while ${H}(\mathcal{G}) \rightarrow 0$ means strong heterophily.
Statistics of datasets are summarized in Appendix \ref{app:data}.
To comprehensively evaluate the model performance in node classification, we provide two strategies to split datasets, called dense splitting and sparse splitting.
In dense splitting, we randomly choose 50\% of each label as the training set, 25\% as the validation set, and the rest as the test set, which is a common setting is previous studies~\cite{nodeformer,sgformer}.
While in sparse splitting~\cite{gprgnn}, we adopt 2.5\%/2.5\%/95\% splitting for training set, validation set and test set, respectively.


\subsection{Baseline}
We adopt eleven representative approaches as the baselines: SGC~\cite{sgc}, APPNP~\cite{appnp}, GPRGNN~\cite{gprgnn}, FAGCN~\cite{fagcn}, BM-GCN~\cite{bmgcn}, ACM-GCN~\cite{acmgnn}, NAGphormer~\cite{nagphormer}, SGFormer~\cite{sgformer}, Specformer~\cite{specformer}, VCR-Graphormer~\cite{vcrgt} and PolyFormer~\cite{polyformer}.
The first six are mainstream GNNs and others are representative GTs.




\begin{table*}[ht]
\centering
\caption{Comparison of all models in terms of mean accuracy $\pm$ stdev (\%) under sparse splitting. The best results appear in \textbf{bold}. The second results appear in \underline{underline}.}
\scalebox{0.8}{
\begin{tabular}{lcccccccccccc}
\toprule
Dataset& Photo & ACM & Computer  &Citeseer &WikiCS& BlogCatalog & UAI2010 & Flickr   \\
$\mathcal{H}$& 0.83 & 0.82 & 0.78  &0.74 & 0.66& 0.40 & 0.36& 0.24  \\ \hline

SGC& 91.90\tiny{$\pm$0.35} & 89.57\tiny{$\pm$0.28} &86.79\tiny{$\pm$0.19} & 66.41\tiny{$\pm$0.59} & 74.99\tiny{$\pm$0.19}& 71.23\tiny{$\pm$0.06} &51.61\tiny{$\pm$0.41} &39.43\tiny{$\pm$0.50}  \\

APPNP& \underline{92.24\tiny{$\pm$0.28}}& 89.91\tiny{$\pm$0.89} &\underline{87.64\tiny{$\pm$0.39} } & \underline{66.70\tiny{$\pm$0.11}}  & 77.42\tiny{$\pm$0.31}
& 81.76\tiny{$\pm$0.38} &\underline{61.65\tiny{$\pm$0.71}} & 71.39\tiny{$\pm$0.62}  \\

GPRGNN& 92.13\tiny{$\pm$0.32} & 89.47\tiny{$\pm$0.90} &86.38\tiny{$\pm$0.44}  & 66.50\tiny{$\pm$0.62} & 77.59\tiny{$\pm$0.49}& \underline{84.57\tiny{$\pm$0.35} }&58.75\tiny{$\pm$0.75} &\underline{71.89\tiny{$\pm$0.89}}    \\


FAGCN& 92.02\tiny{$\pm$0.18} & 88.47\tiny{$\pm$0.31} &83.99\tiny{$\pm$1.95} & 64.54\tiny{$\pm$0.66}  & 75.21\tiny{$\pm$0.84}& 76.38\tiny{$\pm$0.82} &54.67\tiny{$\pm$0.96} & 63.68\tiny{$\pm$0.72}  \\

BM-GCN& 91.19\tiny{$\pm$0.39} &\underline{90.11\tiny{$\pm$0.60} }&86.14\tiny{$\pm$0.51} & 66.11\tiny{$\pm$0.47}& {77.39\tiny{$\pm$0.37} } &84.05\tiny{$\pm$0.54} & 57.51\tiny{$\pm$1.14} &   60.82\tiny{$\pm$0.76} \\

ACM-GCN& 91.71\tiny{$\pm$0.64} & 89.68\tiny{$\pm$0.45} &86.64\tiny{$\pm$0.59} &64.85\tiny{$\pm$1.19}& \underline{77.68\tiny{$\pm$0.57}} & 77.17\tiny{$\pm$1.34} &56.05\tiny{$\pm$2.11} & 64.58\tiny{$\pm$1.53}  \\

\hline
NAGphormer& 91.65\tiny{$\pm$0.80} & 89.73\tiny{$\pm$0.48} &85.31\tiny{$\pm$0.65} &  63.66\tiny{$\pm$1.68}& 76.93\tiny{$\pm$0.75} & 79.19\tiny{$\pm$0.41} &58.36\tiny{$\pm$1.01} & 67.48\tiny{$\pm$1.04}  \\

SGFormer& 90.13\tiny{$\pm$0.56} & 88.03\tiny{$\pm$0.60} &80.07\tiny{$\pm$0.21} &  62.41\tiny{$\pm$0.94}& 74.69\tiny{$\pm$0.52} & 78.15\tiny{$\pm$0.69} &50.19\tiny{$\pm$1.72} & 51.01\tiny{$\pm$1.05}  \\


Specformer& 90.57\tiny{$\pm$0.55} & 88.20\tiny{$\pm$1.05} &85.55\tiny{$\pm$0.63} & 62.64\tiny{$\pm$1.54} &  75.24\tiny{$\pm$0.71}& 79.75\tiny{$\pm$1.29} &57.42\tiny{$\pm$1.06} & 56.94\tiny{$\pm$1.48}  \\

VCR-Graphormer
& 91.39\tiny{$\pm$0.75} & 86.81\tiny{$\pm$0.84} &85.06\tiny{$\pm$0.64} & 57.61\tiny{$\pm$0.60} & 72.81\tiny{$\pm$1.44} & 74.90\tiny{$\pm$1.18} &56.43\tiny{$\pm$1.10} & 50.93\tiny{$\pm$1.12}   \\

PolyFormer
& 91.52\tiny{$\pm$0.78} & 89.83\tiny{$\pm$0.62} &85.75\tiny{$\pm$0.78} & 64.77\tiny{$\pm$1.27} & 75.12\tiny{$\pm$1.16}& 81.02\tiny{$\pm$0.81} &58.89\tiny{$\pm$0.77} & 67.85\tiny{$\pm$1.43}    \\


\hline

\name & 
\textbf{92.93\tiny{$\pm$0.26}} & \textbf{90.92\tiny{$\pm$0.69}} & 
\textbf{88.14\tiny{$\pm$0.52}} & 
\textbf{69.91\tiny{$\pm$1.02}} & 
\textbf{78.11\tiny{$\pm$0.83}}& \textbf{88.11\tiny{$\pm$0.58}} & \textbf{63.96\tiny{$\pm$1.09}} & \textbf{72.16\tiny{$\pm$1.19}}  \\     
 \toprule
\end{tabular}
}

\label{tab:sparse-ncre}
\end{table*}

% \begin{table*}[t]
\centering
  \caption{\textbf{NC Analysis.} In this setting, VGGm-17 models are trained on ImageNet-10 dataset (ID) for 200 epochs using MSE loss and evaluated on the same ID dataset using neural collapse metrics. Reported is the top-1 accuracy (\%). $\mathbf{W}$ and $\mathbf{W_{LS}}$ denote learned weights and least square weights (analytical, no training) of the final classifier layer, respectively. \textbf{A lower $\mathcal{NC}$ indicates higher \jg{stronger?} neural collapse.}}
  \label{tab:nc_results}
  \centering
  %\resizebox{\linewidth}{!}{
     \begin{tabular}{ccc|cccc}
     \hline %\hline
     \multicolumn{1}{c}{\textbf{Configuration}} &
     \multicolumn{1}{c}{\textbf{ID Accuracy} $\uparrow$} &
     \multicolumn{1}{c|}{\textbf{ID Accuracy} $\uparrow$} &
     \multicolumn{4}{c}{\textbf{Neural Collapse} $\downarrow$} \\
    & $\mathbf{W}$ & $\mathbf{W_{LS}}$ & $\mathcal{NC}1$ &  $\mathcal{NC}2$ &  $\mathcal{NC}3$ &  $\mathcal{NC}4$ \\
    \hline
    No Projector & 89.60 & 89.40 & 0.075 & 0.219 & 0.030 & 0.316 \\
    \hline
    Plastic Projector (1 layer) & 89.20 & 89.40 & 0.074 & 0.304 & 0.043 & 0.316 \\
    ETF Fixed Projector (1 layer) & 89.00 & 88.80 & \textbf{0.069} & \textbf{0.254} & \textbf{0.035} & 0.316 \\
    \hline
    Plastic Projector (2 layers) & 89.20 & 89.00 & 0.101 & 0.378 & 0.051 & 0.316 \\
    ETF Fixed Projector (2 layers) & 89.40 & 89.40 & \textbf{0.080} & \textbf{0.311} & \textbf{0.041} & 0.316 \\
    ETF Fixed Proj (2 layers) + KoLeo & \textbf{90.20} & \textbf{90.40} & 0.085 & \textbf{0.094} & \textbf{0.041} & \textbf{0.282} \\
    \hline %\hline
    %\vspace{-2em}
    \end{tabular} %}
\end{table*}


\subsection{Performance Comparison}
To evaluate the model performance in node classification, we run each model ten times with random initializations. The results in terms of mean accuracy and standard deviation are reported in \autoref{tab:dense-ncre} and \autoref{tab:sparse-ncre}.

First, we can observe that \name achieves the best performance on all datasets with different data splitting strategies, demonstrating the effectiveness of \name in node classification.
Then, we can find that advanced GTs obtain more competitive performance than GNNs on over half datasets under dense splitting.
But under sparse splitting, the situation reversed.
An intuitive explanation is that Transformer has more learnable parameters than GNNs, which bring more powerful modeling capacity.
However, it also requires more training data than GNNs in the training stage to ensure the performance.
Therefore, when the training data is sufficient, GTs can achieve promising performance.
And when the training data is sparse, GTs usually leg behind GNNs.
Our proposed \name addresses this issue by introducing the token swapping operation to generate diverse token sequences. 
This operation effectively augments the training data, ensuring the model training even in the sparse data scenario.
In addition, the tailored center alignment loss also constrains the model parameter learning, further enhancing the model performance.


\begin{figure}[t]
\centering
\includegraphics[width=7.5cm]{Fig/alignloss-p.pdf}
\caption{
Performances of \name with or without the center alignment loss.
}
\label{fig:align}
\end{figure}

\begin{figure}[t]
\centering
\includegraphics[width=7.5cm]{Fig/ts-p.pdf}
\caption{
Performances of \name with different token sequence generation strategies.}
\label{fig:ts}
\end{figure}

\begin{figure}[t]
\centering
\includegraphics[width=7.3cm]{Fig/t-p.pdf}
\caption{
Analysis on the swapping times $t$.}
\label{fig:t}
\end{figure}

\begin{figure}[t]
\centering
\includegraphics[width=7.3cm]{Fig/s-p.pdf}
\caption{
Analysis on the augmentation times $s$.}
\label{fig:s}
\end{figure}



\subsection{Study on the center alignment loss}
The center alignment loss, proposed to constrain the representation learning from multiple token sequences, is a key design of \name.
Here, we validate the effectiveness of the center alignment loss in node classification.
Specifically, we develop a variant of \name by removing the center alignment loss, called \name-O.
Then, we evaluate the performance of \name-O on all datasets under dense splitting and sparse splitting.
Due to the space limitation, we only report the results on four datasets in \autoref{fig:align}, other results are reported in Appendix \ref{app:exp-ca}.
"Den." and "Spa." denotes the experimental results under dense splitting and sparse splitting, respectively.
Based on the experimental results, we can have the following observations:
1) \name beats \name-O on all datasets, indicating that the developed center alignment loss can effectively enhance the performance of \name.
2) Adopting the center alignment loss can bring more significant improvements in sparse setting than those in dense setting.
This situation implies that introducing the reasonable constraint loss function based on the property of node token sequences can effectively improve the model training when the training data is sparse.






\subsection{Study on the token sequence generation}\label{exp:ts}
The generation of token sequences is another key module of \name, which develops a novel token swapping operation can fully leverage the semantic relevance of nodes to generate informative token sequences.
In this section, we evaluate the effectiveness of the proposed strategy by comparing it with two naive strategies.
One is to enlarge the sampling size $k$. We propose a variant called \name-L by sampling $2k$ tokens to construct token sequences.
The other is to randomly sample $k$ tokens from the enlarged $2k$ token set to construct multiple token sequences, called \name-R.
Performance of these variants are shown in \autoref{fig:ts} and results on other datasets are reported in Appendix \ref{app:exp-ts}.

We can observe that \name-R outperforms \name-L on most cases, indicating that constructing multiple token sequences is better for node representation learning of tokenized GTs than generating single long token sequence. 
Moreover, \name surpasses \name-R on all cases, showcasing the superiority of the proposed token swapping operation in generation of multiple token sequences.
This observation also implies that constructing informative token sequences can effectively improve the performance of tokenized GTs.



\subsection{Analysis on the swapping times $t$}
As discussed in Section \ref{sec:swapping}, $t$ determines the range of candidate tokens from the constructed $k$-NN graph, further affecting the model performance.
To validate the influence of $t$ on model performance, we vary $t$ in $\{1,2,3,4\}$ and observe the changes of model performance.
Results are shown in \autoref{fig:t} and Appendix \ref{app:exp-t}.
We can clearly observe that \name can achieve satisfied performance on all datasets when $t$ is no less than 2.
This situation indicates that learning from tokens with semantic associations beyond the immediate neighbors can effectively enhancing the model performance.
This phenomenon also reveals that reasonably enlarging  the sampling space to seek more informative tokens is a promising way to improve the effect of node tokenized GTs.
%分析的

\subsection{Analysis on the augmentation times $s$}
The augmentation times $s$ determines how many token sequences are adopted for node representation learning. Similar to $t$, we vary $s$ in $\{1,2,\dots,8\}$ and report the performance of \name. 
Results are shown in \autoref{fig:s} and Appendix \ref{app:exp-s}.
Generally speaking, sparse splitting requires a larger $s$ to achieve the best performance, compared to dense splitting.
This is because \name needs more token sequences for model training in the sparse data scenario.
This situation indicates that a tailored data augmentation strategy can effectively improve the performance of tokenized GTs when training data is sparse.
Moreover, the optimal $s$ varies on different graphs. 
This is because different graphs exhibit different topology features and attribute features, which affects the generation of token sequences, further influencing the model performance.




\section{Conclusion}
\label{Sec:con}
In this paper, we introduced a novel tokenized Graph Transformer \name for node classification.
In \name, we developed a novel token swapping operation that flexibly swaps tokens in different token sequences, thereby generating diverse token sequences. This enhances the model’s ability to capture rich node representations.  
Furthermore, \name employs a tailored Transformer-based backbone with a center alignment loss to learn node representations from the generated  multiple token sequences. The center alignment loss helps guide the learning process when nodes are associated with multiple token sequences, ensuring that the learned representations are consistent and informative. 
Experimental results demonstrate that \name significantly improves node classification performance, outperforming several representative GT and GNN models. 




% \section*{Accessibility}
% Authors are kindly asked to make their submissions as accessible as possible for everyone including people with disabilities and sensory or neurological differences.
% Tips of how to achieve this and what to pay attention to will be provided on the conference website \url{http://icml.cc/}.

% \section*{Software and Data}

% If a paper is accepted, we strongly encourage the publication of software and data with the
% camera-ready version of the paper whenever appropriate. This can be
% done by including a URL in the camera-ready copy. However, \textbf{do not}
% include URLs that reveal your institution or identity in your
% submission for review. Instead, provide an anonymous URL or upload
% the material as ``Supplementary Material'' into the OpenReview reviewing
% system. Note that reviewers are not required to look at this material
% when writing their review.

% Acknowledgements should only appear in the accepted version.
% \section*{Acknowledgements}

% \textbf{Do not} include acknowledgements in the initial version of
% the paper submitted for blind review.


\section*{Impact Statement}
This paper presents work whose goal is to advance the field of graph representation learning. There is none potential societal consequence of our work that must be specifically highlighted here.

% Authors are \textbf{required} to include a statement of the potential 
% broader impact of their work, including its ethical aspects and future 
% societal consequences. This statement should be in an unnumbered 
% section at the end of the paper (co-located with Acknowledgements -- 
% the two may appear in either order, but both must be before References), 
% and does not count toward the paper page limit. In many cases, where 
% the ethical impacts and expected societal implications are those that 
% are well established when advancing the field of Machine Learning, 
% substantial discussion is not required, and a simple statement such 
% as the following will suffice:

% ``This paper presents work whose goal is to advance the field of 
% Machine Learning. There are many potential societal consequences 
% of our work, none which we feel must be specifically highlighted here.''

% The above statement can be used verbatim in such cases, but we 
% encourage authors to think about whether there is content which does 
% warrant further discussion, as this statement will be apparent if the 
% paper is later flagged for ethics review.


% In the unusual situation where you want a paper to appear in the
% references without citing it in the main text, use \nocite

\bibliography{reference}
\bibliographystyle{icml2025}


%%%%%%%%%%%%%%%%%%%%%%%%%%%%%%%%%%%%%%%%%%%%%%%%%%%%%%%%%%%%%%%%%%%%%%%%%%%%%%%
%%%%%%%%%%%%%%%%%%%%%%%%%%%%%%%%%%%%%%%%%%%%%%%%%%%%%%%%%%%%%%%%%%%%%%%%%%%%%%%
% APPENDIX
%%%%%%%%%%%%%%%%%%%%%%%%%%%%%%%%%%%%%%%%%%%%%%%%%%%%%%%%%%%%%%%%%%%%%%%%%%%%%%%
%%%%%%%%%%%%%%%%%%%%%%%%%%%%%%%%%%%%%%%%%%%%%%%%%%%%%%%%%%%%%%%%%%%%%%%%%%%%%%%
\newpage
\appendix
\onecolumn

\section{Experimental Settings}

\subsection{Dataset}\label{app:data}
Here we introduce datasets adopted for experiments. The detailed statistics of all datasets are reported in \autoref{tab:dataset}.
\begin{itemize}
    \item \textbf{Academic graphs}: This type of graph is formed by academic papers or authors and the citation relationships among them. Nodes in the graph represent academic papers or authors, and edges represent the citation relationships between papers or co-author relationships between two authors. The features of nodes are composed of bag-of-words vectors, which are extracted and generated from the abstracts and introductions of the academic papers. The labels of nodes correspond to the research fields of the academic papers or authors. ACM, Citeseer, WikiCS and UAI2010 belong to this type.
    \item \textbf{Co-purchase graphs}: This type of graph is constructed based on users' shopping behaviors. Nodes in the graph represent products. The edges between nodes indicate that two products are often purchased together. The features of nodes are composed of bag-of-words vectors extracted from product reviews. The category of a node corresponds to the type of goods the product belongs to. Computer and Photo belong to this type.

    \item \textbf{Social graphs}: This type of graph is formed by the activity records of users on social platforms. Nodes in the graph represent users on the social platform. The edges between nodes indicate the social relations between two users. Node features represent the text information extracted from the authors' homepage. The label of a node refers to the interest groups of users. BlogCatalog and Flickr belong to this type.
    
\end{itemize}

\section{Dataset Generation}
\label{sec:dataset}
\revise{
To train the proposed GNN, we constructed a dataset of building structures and a subset of these structures were subjected to fire simulations using FEA. The dataset generation process is illustrated in \figref{fig:dataset_generation_procedure}. Initially, a total of 33,000 building structures with geometrical details, material properties, and gravity loads were created. Due to randomness in generating these structures, a filter is applied to remove unreasonable data after gravity load simulation, which included 15,377 structures. A trade-off between computational feasibility and model performance is made among the remaining 17,623 structures. As further labeling structures with MIDR requires resource-intensive fire simulations via OpenSeesRT, a large proportion of 16,050 structures is selected as unlabeled dataset. On the other hand, each of the other 1,573 structures was further subjected to 30 different fire simulations, forming the labeled dataset containing $1,573\times 30 = 47,190$ fire cases.} This section details the step-by-step process for generating the dataset, including geometry creation, material property assignment, and simulations due to gravity loads and fire scenarios. 
% To train the proposed neural network, we constructed a dataset comprising building structure data and a subset of fire scenario data. The dataset generation process is illustrated in \figref{fig:dataset_generation_procedure}. 
% A total of 33,000 building structures with geometric details, material properties, and gravity loads were initially created. Out of these, 3,000 structures were selected as labeled data, and the remaining 30,000 were designated as unlabeled data. Further, about half of them filtered out due to instability under gravity loads only. 
\begin{figure*}[h!]
    \centering
    \includegraphics[width=0.8\linewidth]{figures/dataset_filter_procedure.pdf}
    \caption{Workflow for dataset generation (geometry, material property, gravity loads, and fire scenarios).}
    \label{fig:dataset_generation_procedure}
\end{figure*}

\subsection{Geometry Generation}
\label{subsec:geometry_generation}
The geometry of the building structures forms the foundation of the dataset. Regular 
\revise{3D structures} resembling multi-story parking structures or shopping malls were generated, with parameters such as building floor dimensions and story heights selected randomly. Each building structure is composed of multiple rooms, which serve as the basic unit in this study. A room herein is a cuboid space defined by specific length, width, and height. Within a structure, rooms of the same dimensions are uniformly arranged along the length, width, and height, corresponding to the $x$-, $y$-, and $z$-axes, respectively. Structures vary in room size and number of rooms along each axis. Specifically, the room length, width, and height are independently sampled from a uniform distribution within the interval $[2, 5]$ meters along the three directions of the structure. Similarly, the room number along each axis is uniformly sampled independently as an integer within the interval $[2, 7]$, i.e., the maximum number of stories of the buildings simulated in this study is 7.

To introduce variability and simulate real-world scenarios, approximately $8\%$ of structural elements (beams or columns) are randomly removed after initial geometry creation. 
\revise{Such removal is not fire-induced damage, but reflects functional diversity often observed in real buildings, such as open spaces designed for activities in shopping malls, e.g., ice skating rinks. Examples of the generated geometries are illustrated in \figref{fig:example_generated_geometry}, showcasing the diversity and realism of the dataset. This element removal does not affect the definition of room's geometry in the structure and nor does it affect the number of considered fire scenarios.} 

\revise{A range of coefficient of variation values ($3.3\%$ to $17.5\%$) was derived from prior studies that investigated the statistics of geometrical and material properties of structural components of buildings (e.g., \cite{mirza1979variations, lee2004probabilistic}). These studies provide empirical data on the natural variability in parameters such as Young's modulus, yield strength, and dimensions of structural elements due to manufacturing tolerances and material inconsistencies. By selecting $8\%$ for the removal of structural elements in our database, we aimed to maintain a level of variability that is representative of real-world uncertainties while ensuring computational feasibility. This choice ensures that the database captures realistic deviations without introducing extreme cases that may not be commonly encountered in practice.}

\begin{figure*}[h!]
    \centering
    \includegraphics[width=\linewidth]{figures/example_generated_geometry.pdf}
    \caption{Examples of generated structural geometry of different sizes (all dimensions in meters).}
    \label{fig:example_generated_geometry} 
\end{figure*}

{\blockRevise

In this study, we opted for a deterministic square, dimension of $0.1$ m, solid cross-sectional steel elements due to their simplicity in modeling and analysis. Square sections exhibit uniform geometrical properties in all directions, simplifying the computation of structural responses and avoiding complications associated with more complex shapes, such as wide-flange sections, facilitating the computational efficiency and scalability to generate a large dataset. This choice also helps to mitigate issues related to stress concentrations and facilitates a more straightforward representation of structural behavior under thermal loads. 

\textit{Remark:} The selected cross-section provides a comparable flexural rigidity to a $W 130 \times 130 \times 28.1$ wide-flange section (metric units), albeit with significantly higher axial rigidity. This cross-section is acceptable for gravity-load-designed frames under service loading conditions where the models assume fully rigid, moment-resisting beam-column connections for the evaluation of the IDR under thermal loading. This assumption is reasonable in this computational study where the primary interest is to understand the global deformation response of frames under fire conditions. The selection of uniform square cross-sections for both beams and columns, rather than adherence to standard capacity design principles, was made here primarily for computational efficiency and to reduce design parameters in the database generation process. This choice allows for simplified and scalable approach to analyze the fire-induced response of generic steel frames without the need for large section variations, where this study mainly focuses on the fire vulnerability assessment using ML-based predictions. However, if additional loading conditions, e.g., seismic or wind loads, were to be considered, larger sections, strong-column/weak-beam principle, and ductile detailing would be required in the generated buildings for realistic structural behavior under combined loading conditions. Future studies may also consider investigating the influence of variable cross-sectional dimensions and semi-rigid connections on the structural performance under fire conditions. 
} % blockRevise

\subsection{Material Properties}
Steel is chosen as the material for the structures. To reflect real-world variations, we randomly assign one of five slightly different steel material types to each structural element. \revise{
The ranges of material properties are provided in \tabref{tab:material_property_ranges} and the properties are sampled from uniform distributions of the corresponding ranges. These variations simulate differences arising from manufacturing batches or regional material properties. That these properties are at ambient temperature and change when the temperature rises due to a fire. The selection of materials with varying properties is aimed at increasing the diversity of the data. Our goal is to represent as wide a range of data as possible with a limited amount of building structure data, thereby enhancing the generalization ability of the GNN. Our assumed material property ranges are expected to be wider than the real-world conditions based on findings in \cite{mirza1979variations, lee2004probabilistic}. Therefore, we are essentially tackling a more challenging and general task. If we can solve this problem, we are confident that our method will perform equally well or even better in real-world scenarios.
}
\begin{table}[h!]
    \centering
    \caption{Material properties ranges for considered steel structures.}
    \begin{tabular}{lc}
        \toprule
        Property & Range \\
        \midrule
        Young's modulus & [168, 252] GPa \\
        Yield strength & [220, 330] MPa \\
        Strain-hardening ratio & [0.8, 1.2] \% \\
        \bottomrule
    \end{tabular}
    \label{tab:material_property_ranges}
\end{table}

\subsection{Gravity Loads}
Gravity loads are applied to columns and beams based on their \revise{influence (tributary) areas as typically conducted in structural analysis. The considered ``service'' load conditions include the column self-weight and the additional loads directly supported on the beams from their self-weight and weights of the reinforced concrete slabs, people as live load, and building content. An edge beam typically carries approximately half the gravity load supported by a parallel interior beam}. The ranges of gravity loads are listed in \tabref{tab:gravity_load_ranges}. \revise{The loads are sampled from uniform distributions of the corresponding ranges.} Structures that failed to meet an MIDR threshold of $1\%$ under gravity loads were deemed unacceptable designs and filtered out, as such configurations of randomly chosen geometry, material, and gravity load combinations were considered unrealistic from a regulatory and practicality points of view.
\begin{table}[h!]
    \centering
    \caption{Gravity load ranges for considered beams and columns.}
    \begin{tabular}{lc}
        \toprule
        Element & Range (kN/m)  \\
        \midrule
        Column & [0.5, 1.0]  \\
        Edge beam & [1.5, 4.5]  \\
        Interior beam & [3.0, 7.5]  \\
        \bottomrule
    \end{tabular}
    \label{tab:gravity_load_ranges}
\end{table} 

\subsection{Rule-based Thermal Load Generation}
\label{subsec:thermal_load_generation}
To evaluate a building's structural response during a fire event, we employed a simplified rule-based approach for thermal load generation. 
% Previous studies \cite{nan_structuralfire_2023} have demonstrated that steel structures rapidly equilibrate with surrounding gases temperatures due to efficient heat exchange. Consequently, gas temperatures can be directly used as inputs for FEA tools, e.g., OpenSees, simplifying the process of modeling thermal loads. 
% Accurately simulating temperature fields in fire scenarios poses significant challenges. Advanced thermodynamic simulations, such as those performed using Fire Dynamics Simulator (FDS) \cite{mcgrattan_fire_2000}, provide precise temperature predictions. However, these methods are hindered by high computational costs, prolonging execution times, and limited scalability, making them impractical for generating large datasets. Additionally, real-world fire loads often display substantial spatial variability across different rooms \cite{dundar_fire_2023}, resulting in scenario-specific temperature fields with limited generalizability. For example, studies on bridge fires \cite{he_study_2024} have demonstrated that environmental factors, such as wind speeds, can significantly influence temperature distributions. Furthermore, even within identical scenarios, variations in fire modeling methodologies can produce distinctly different temperature fields \cite{zhang_temperature_2020, du_new_2012}. These challenges emphasize the need for efficient and adaptable methods to generate fire temperature data.
% To address these issues, we adopted a rule-based approach to model temperature variations. 
According to \cite{spearpoint_fire_2008}, a typical fire development follows a predictable pattern. During the {\em{growth stage}}, the temperature rises slowly and approximately linearly after ignition. This is followed by the {\em{flashover stage}}, where temperatures increase rapidly to peak values. After reaching the peak, the temperature either stabilizes or continues to rise slowly until the {\em{decay stage}} begins. Inspired by this fire development pattern, we describe the temperature evolution in time, $t$, prior to the decay stage in two distinct stages:
\begin{enumerate}
    \item {\bf{Initial linear increase stage}}: For $t \in [0, t_1)$, temperature increases gradually and linearly as the fire spreads through the building. This stage represents the time before the fire directly affects a structural element.  
    \item {\bf{ISO 834 fire curve stage}}: For $t \in [t_1, t_{\thre}]$, temperature rises rapidly following the ISO 834 curve \cite{ISO834}, modeling the direct impact of the fire on the structural element. 
\end{enumerate}
The slope of the linear temperature increase, $c$, and the transition time, $t_1$, are influenced by the spatial relationship between the fire source and the structural element. For the second stage of temperature evolution, we utilize the ISO 834 curve, a widely accepted standard for fire resistance testing. This standardized fire curve describes the temperature rise over time, enabling rapid and consistent thermal fields across various scenarios. The duration of fire simulation in this study is set to $t_{\thre}=60$ minutes. This value represents the upper limit for the temperature evolution of each structural element, providing a consistent basis for analyzing the structural response to fire.

Let $(x, y, z)$ represents the midpoint of a structural element and $(x_{\subfire}, y_{\subfire}, z_{\subfire})$ the fire source point. \revise{Integer parameters $h$ and $h_{\subfire}$ correspond to the respective floor levels of the element and the fire source}. The temperature evolution for each element is expressed as follows:
\begin{enumerate}
    \item Linear increase stage ($0 < t < t_1$):
    \begin{equation}
    T(t) = c \cdot t,
    \end{equation}
    where $c$, the rate of temperature increase ($^\circ\mathrm{C}/\mathrm{min}$), depends on the height difference between the element, $h$, and the fire source, $h_{\subfire}$:
    \begin{equation}
        c = 
        \begin{cases} 
        5\left/\left(h - h_{\subfire} + 1\right)\right., & h \geq h_{\subfire}, \\
        2\left/\left(h_{\subfire} - h\right)\right., & h < h_{\subfire}.
        \end{cases}
    \end{equation}
     \item ISO 834 stage ($t \geq t_1$):
\begin{equation}
    T(t) = c \cdot t_1 + 345 \log_{10} \left(8 \left(t - t_1\right) + 1\right).
\end{equation}
\end{enumerate}

The transition (arrival) time $t_1$, marking the end of the linear stage, depends on the spatial distance between the fire source and the element. We define the following two Euclidean distances $L_p$ in the $xy$ plane and $L_s$ in the $xyz$ space:
\begin{eqnarray}
L_p & \triangleq & \sqrt{(x - x_{\subfire})^2 + (y - y_{\subfire})^2}, \\
\label{eq:Lp}
L_s & \triangleq & \sqrt{(x - x_{\subfire})^2 + (y - y_{\subfire})^2 + (z - z_{\subfire})^2}.
\label{eq:Ls}
\end{eqnarray}
Accordingly, the transition time, $t_1$, is expressed as follows:
\begin{equation}
    t_1 = 
    \begin{cases}
    \beta_{1} \cdot \left(1 - \exp\left\{- L_s\left/\alpha_{1}\right.\right\}\right), & h > h_{\subfire}, \\
    \beta_{2} \cdot \left(1 - \exp\left\{- L_p\left/\alpha_{2}\right.\right\}\right), & h = h_{\subfire}, \\
    \beta_{3} \cdot \left(1 - \exp\left\{- L_s\left/\alpha_{3}\right.\right\}\right), & h < h_{\subfire} .
    \end{cases}
    \label{eq:t1}
\end{equation}
The parameters $\beta_i$ and $\alpha_i$ for determining $t_1$ are summarized in Table~\ref{tab:fire_spread_parameters}. In this study, we take $r_{\mathrm{up}}=0.95$ and $r_{\mathrm{down}}=0.97$.
\begin{table}[ht]
    \centering
    \caption{Fire spread parameters for $t_1$ calculations.}
    \begin{tabular}{lcc}
        \toprule
        Case  & $\beta_i$ & $\alpha_i$  \\
        \midrule
        $i=1$, Upward spread & $16 \left.\left(1-r_{\mathrm{up}}^{\left|h-h_{\subfire}\right|}\right)\right/\left(1-r_{\mathrm{up}}\right)$ & $10$  \\
        $i=2$, Horizontal spread & $18$ & $18$  \\
        $i=3$, Downward spread & $30 \left.\left(1-r_{\mathrm{down}}^{\left|h-h_{\subfire}\right|}\right)\right/\left(1-r_{\mathrm{down}}\right)$ & $5$  \\
        \bottomrule
    \end{tabular}
    \label{tab:fire_spread_parameters}
\end{table}

\figref{fig:t1_curve} illustrates the $t_1$ curves for various fire scenarios: (1) fire originating on the lower floor, $h-h_{\subfire}=1$ with rapid upward spread, (2) fire on the same floor, $h=h_{\subfire}$ with the fastest spread, and (3) fire on the upper floor, $h_{\subfire}-h=1$ with slow downward spread. The exponential decay in $t_1$ reflects the accelerating fire propagation speed as the distance increases. \figref{fig:t1_curve} also indicates that the employed simplified model is consistent with the Markov chain-based dynamic model given by \cite{cheng_dynamic_2011}, where the rooms at the same floor of the fire point start flashover slightly before the corresponding upper floors. Additionally, $\beta_{1}$ and $\beta_{3}$ are the summation of a geometric sequence, where story level $h$ is the index. The common ratios $r_{\mathrm{up}}<1$ in $\beta_{1}$ and $r_{\mathrm{down}}<1$ in $\beta_{3}$ indicate that the fire speeds up to spread through the next story, which is consistent with the real-world fire spread mechanism given in \cite{hokugo_mechanism_2000}. The temperature profile within the range $t \in [0, t_{\thre}]$ is subsequently used as the thermal load in OpenSeesRT simulations to compute displacements at each structural node at time $t_{\thre}$.
\begin{figure}[h!]
    \centering
    \includegraphics[width=0.8\linewidth]{figures/m204_t1_curve.pdf}
    \caption{Three examples for the $t_1$ curve.}
    \label{fig:t1_curve}
\end{figure}

\revise{
\textit{Remark:} The effects of structural elements, such as concrete floor slabs and partitions, are not explicitly modeled in our approach. Instead, their influence is implicitly captured through the careful selection of the parameters $ \alpha, \beta, r_\mathrm{up} $, and $ r_\mathrm{down} $. This parameterization provides a unified framework for generating temperature fields. Indeed, fire propagation is governed by a multitude of factors and remains an open research question. For instance, if the fire resistance of a floor slab is enhanced by fire protective coating, the corresponding model can account for this by decreasing $\alpha_1$ \& $\alpha_3$, increasing $\beta_1$ \& $\beta_3$, and adopting larger values for $r_\mathrm{up}$ \& $r_\mathrm{down}$, which collectively slow down the vertical spread of fire. Conversely, scenarios involving higher amounts of combustible materials would warrant the opposite adjustments. This flexible and integrated approach avoids the need to design separate models for different fire propagation scenarios while still capturing the essential effects.
}

\revise{
In conclusion, our rule-based approach is a computationally efficient method for approximating fire temperature fields, enabling large-scale dataset generation to train predictive models. By combining ISO 834 fire curves with spatial considerations and embedding structural effects through parameter calibration, the method achieves a balanced trade-off between accuracy and scalability, making it a practical solution for thermal load modeling in fire scenarios. After generating the temperature of each beam or column according to the middle point, the temperature is applied as uniform thermal load to the elements of the structure in question using OpenSeesRT. 
}

% In conclusion, this rule-based approach is a computationally efficient method to approximate fire temperature fields, enabling large-scale dataset generation to train predictive models. By combining ISO 834 fire curves with spatial considerations, the method balances accuracy and scalability, making it a practical solution for thermal load modeling in fire scenarios.

% \subsection{Interstory Drift Ratio}
\subsection{OpenSeesRT Simulation}
\label{subsec:opensees_simulation}

The thermal and mechanical responses of 3D frame structures under combined fire and gravity loads are simulated using OpenSeesRT \cite{perez2024openseesrt}. \revise{In the simulation, the IDR of each node at $t_{\thre}$ is computed using the computed nodal displacements. Each structural model features six degrees of freedom per node (3 translational  and 3 rotational), with linear geometrical transformations (\texttt{geomTransf: Linear}) defining how the element local coordinate systems are mapped to the global coordinate system and assuming small displacements and rotations. Although OpenSeesRT allows a variety of options for modeling finite deformations, in the present simulations and mainly for simplicity, we did not consider large deformations. All bottom nodes (nodes on the ground) are fully constrained in all six degrees of freedom, while degrees of freedom os all other nodes are free.} Material behavior is temperature-dependent and modeled with \texttt{Steel01Thermal}, while fiber-based sections (\texttt{FiberThermal}) capture nonlinear interactions between thermal and mechanical responses at the cross-section level. \revise{Structural elements are represented as displacement-based Euler-Bernoulli beam-columns (\texttt{dispBeamColumnThermal}). This element  formulation accounts for thermal strains (temperature gradients) in the section, which is discretized into fibers. Numerical integration is used along the length of each element using three integration (Gauss) points, one at each end and the third in the middle of the element.}

{\revise{Thermal expansion of steel members plays a crucial role in IDR development. In reality, reinforced concrete floor slabs heat at a different rate than steel members due to their higher thermal mass and lower thermal conductivity. This differential heating can lead to restrained thermal expansion, introducing axial compression in beams and affecting the overall structural response. In this study, explicit {\em{composite action}} between steel members and concrete slabs is not modeled. Instead, our approach focuses on isolating the response of the steel structural frame, which is often the critical load-bearing component in fire scenarios. This assumption aligns with prior studies \cite{Possidente_2024} demonstrating that steel structures reach thermal equilibrium with surrounding gases quickly, allowing the use of uniform thermal loading in fire analysis. Future work could enhance this framework by incorporating slab-beam interaction effects, through a refined FEA for an extended dataset where constraints imposed by floor slabs are explicitly considered.}

The analysis begins with the application of gravity loads, followed by incremental thermal loads simulating the fire exposure. A static nonlinear solver using  \texttt{ExpressNewton} algorithm ensures convergence, while the \texttt{NormDispIncr} test maintains accuracy. An incremental \texttt{LoadControl} scheme with small step sizes is employed to guarantee numerical stability, using 10\% for gravity loads and 1\% for thermal loads. 

\revise{
In the thermal load analysis, uniform thermal load is applied to each beam or column, i.e., the temperature of each element is set to be that at the middle point, according to \secref{subsec:thermal_load_generation}. The \texttt{Steel01Thermal} material allows the properties (e.g., Young's modulus and yield strength) to be adjusted at increasing temperatures according to \cite{EN1993} using its Table 3.1: Reduction factors for the stress-strain relationship of carbon steel at elevated temperatures. For example, if the Young’s modulus at ambient temperature is $E_0$, then as the temperature ($T$) increases, the modulus changes as $E(T) = \eta (T) \times E_0$. \cite{EN1993} directly provides the values of $\eta(T) \in \left[0,1\right] $ at every $100 ^\circ\mathrm{C}$ interval and recommends using linear interpolation to obtain $\eta(T)$ for intermediate values of $T$.
} OpenSeesRT documentation \cite{OpenSeesThermalExamples} provides several examples of thermal analyses.

This modeling framework accommodates variations in material properties, cross-sectional geometries, and temperature profiles, providing robust simulations of structural behavior under fire conditions. The primary settings and configurations for the OpenSeesRT simulations are summarized in \tabref{tab:ops_detail}.
\begin{table}[h!]
    \centering
        \caption{Key settings of OpenSeesRT simulations.}
    \begin{tabular}{l|>{\raggedright\arraybackslash}p{0.6\linewidth}} %
    \toprule
    Modeling Aspect     & Details \\
    \midrule
    Geometry            & 3D models; 6 degrees of freedom per node \\
    Transformation      & geomTransf: Linear \\ 
    Material            & Steel01Thermal \\
    Section             & FiberThermal; Cross-section: $0.1$ m $\times$ $0.1$ m \\ 
    Element type        & {dispBeamColumnThermal} \\ 
    Loading             & Gravity loads: {beamUniform}; Thermal loads: {beamThermal} \\
    Integration scheme  & Incremental {LoadControl}; Step size: $10\%$ (gravity analysis), $1\%$ (thermal analysis) \\
    Nonlinear solver    & {ExpressNewton} algorithm; {UmfPack} solver; Convergence test: {NormDispIncr} tolerance: $10^{-8}$; Maximum \# iterations per step: $1000$. \\ 
    \bottomrule
    \end{tabular}
    \label{tab:ops_detail}
\end{table}

For each structure in the labeled dataset, 30 fire points are selected using a dual-granularity approach, \revise{i.e., two-stage sampling strategy,} to ensure they are well-distributed. Specifically, rooms are sequentially selected, with one fire point randomly chosen within each selected room. If a building is large and contains more than 30 rooms, we randomly select 30 rooms without replacement, i.e., ensuring that no more than one fire point is located in the same room. Conversely, if the building is small and has fewer than 30 rooms, all rooms are initially selected, with one fire point randomly assigned to each room. Additionally, rooms are then selected with replacement until a total of 30 fire points are assigned. \revise{The room-level sampling prioritizes selecting distinct rooms to avoid spatial clustering of fire points, while the point-level sampling ensures intra-room variability. This approach aligns with stratified sampling principles commonly used for efficient spatial representation, where multi-stage sampling strategies optimize coverage and variability, e.g., \cite{arunachalam_generalized_2023}, and enables a more comprehensive characterizing of how the structures respond under fire conditions.}
% This selection method prevents fire points from clustering too closely while maintaining an element of randomness. By distributing fire points in this manner, the 30 fire scenarios are effectively utilized, enabling a more comprehensive characterizing of how the structures respond under fire conditions.

\subsection{Summary of the Dataset Generation}
As discussed in this section and related to  \figref{fig:dataset_generation_procedure}, three key steps were considered in the development of the dataset: 
\begin{enumerate}
    \item {\bf{Filtering process}}: Structures with MIDR exceeding $1\%$ under gravity loads were excluded,  resulting in $1,573$ labeled structures retained for fire simulation and $16,050$ unlabeled structures for training the MFSP predictor.
    \item {\bf{Fire simulations}}: For each retained labeled structure, 30 fire scenarios were simulated using OpenSeesRT, yielding $47,190$ fire cases.
    \item {\bf{Data distribution check}}: MIDR distributions for labeled and unlabeled data under gravity loads were highly similar, because both datasets were generated using the same method. Under fire conditions, the MIDR distribution shifted, reflecting significant structural deformation with values reaching a maximum of about 6\%, an average of 1.70\%, and a standard deviation of 1.12\%. This step ensured a diverse and comprehensive dataset for the proposed predictive framework.
\end{enumerate}
The statistical distribution histograms for MIDR (after applying the $1\%$ filtering threshold \revise{for gravity load responses}) under different loading conditions are plotted in \figref{fig:histogram_mdr}. Figures \ref{fig:histogram_mdr}(a) and \ref{fig:histogram_mdr}(b) show the MIDR distributions of the labeled and unlabeled data, respectively, under gravity loads only. \figref{fig:histogram_mdr}(c) shows the MIDR distribution of the labeled data under the combined effects of gravity and fire loads. Fire load causes the structures to significantly deform, leading to a noticeably \revise{right-skewed} MIDR distribution.

\begin{figure*}[h!]
    \centering
    \includegraphics[width=\linewidth]{figures/histogram_mdr.pdf}
    \caption{Histograms of MIDR for labeled and unlabeled structures with gravity loads and fire cases.}
    \label{fig:histogram_mdr}
\end{figure*}

\revise{
This dataset provides the basis for training and testing the performance of the GNN-based framework. Although we employed a simplified rule-based thermal load generation method compared with conventional CFD-based simulations, the temperature field, the changes of the material properties, and the response of the structures, are all still highly nonlinear and complex. Therefore, it is still a challenging task for the NN to predict the MIDRs based on this dataset.
}
\subsection{Implementation Details}\label{app:imple}
For baselines, we refer to their official implementations and conduct a systematic tuning process on each dataset.
For \name, we employ a grid search strategy to identify the optimal parameter settings.
Specifically, We try the learning rate in $\{0.001, 0.005, 0.01\}$, dropout in $\{0.3, 0.5, 0.7\}$, dimension of hidden representations in $\{256, 512\}$,
$k$ in $\{4, 6, 8\}$, $\alpha$ in $\{0.1, \dots, 0.9\}$.
All experiments are implemented using Python 3.8, PyTorch 1.11, and CUDA 11.0 and executed on a Linux server with an Intel Xeon Silver 4210 processor, 256 GB of RAM, and a 2080TI GPU.


\section{Additional Experimental Results}\label{app:exp-results}
In this section, we provide the additional experimental results of ablation studies and parameter studies.




\subsection{Study of the center alignment loss}\label{app:exp-ca}
The experimental results of \name and \name-O on the rest datasets are shown in \autoref{fig:align-APP}.
We can observe that \name outperforms \name-O on most datasets.
Moreover, the effect of applying the center alignment loss on \name in sparse splitting is more significant than that in dense splitting.
The above observations are in line with those reported in the main text.
Therefore, we can conclude that the center alignment loss can effectively enhance the performance of \name in node classification.


\begin{figure}[ht]
\centering
\includegraphics[width=17cm]{Fig/alignloss-APP.pdf}
\caption{
Performances of \name with or without the center alignment loss.
}
\label{fig:align-APP}
\end{figure}




\subsection{Study of the token sequence generation}\label{app:exp-ts}
The experimental results of \name with different token sequence generation strategies on the rest datasets are shown in \autoref{fig:ts-APP}.
We can find that the additional experimental results exhibit similar observations shown in the main text.
This situation demonstrates the effectiveness of the token sequence generation with the proposed token swapping operation in enhancing the performance of tokenized GTs.
Moreover, we can also observe that the gains of introducing the token swapping operation vary on different graphs based on the results shown in \autoref{fig:ts} and \autoref{fig:ts-APP}.
This phenomenon may attribute to that different graphs possess unique topology and attribute information, which further impact the selection of node tokens. 
While \name applies the uniform strategy for selecting node tokens, which could lead to varying gains of token swapping.
This situation also motivates us to consider different strategies of token selection on different graphs as the future work.


\begin{figure}[ht]
\centering
\includegraphics[width=17cm]{Fig/ts-p-APP.pdf}
\caption{
Performances of \name with different token sequence generation strategies.
}
\label{fig:ts-APP}
\end{figure}



\subsection{Analysis of the swapping times $t$}\label{app:exp-t}
Here we report the rest results of \name with varying $t$, which are shown in \autoref{fig:t-APP}.
Similar to the phenomenons shown in \autoref{fig:t}, \name can achieve the best performance on all datasets when $t>2$.
Based on the results shown in \autoref{fig:t-APP} and \autoref{fig:t}, we can conclude that introducing tokens beyond first-order neighbors via the proposed token swapping operation can effective improve the performance of \name in node classification.


\begin{figure}[ht]
\centering
\includegraphics[width=17cm]{Fig/t-APP.pdf}
\caption{
Performances of \name with varying $t$.
}
\label{fig:t-APP}
\end{figure}



\subsection{Analysis of the augmentation times $s$}\label{app:exp-s}
Similar to analysis of $t$, the rest results of \name with varying $s$ are shown in \autoref{fig:s-APP}.
We can also observe the similar situations shown in \autoref{fig:s} that \name requires a larger value of $s$ under sparse splitting compared to dense splitting.
The situation demonstrates that introducing augmented token sequences can bring more significant performance gain in sparse splitting than that in dense splitting. 




\begin{figure}[t]
\centering
\includegraphics[width=17cm]{Fig/s-APP.pdf}
\caption{
Performances of \name with varying $s$.
}
\label{fig:s-APP}
\end{figure}

% \section{You \emph{can} have an appendix here.}

% You can have as much text here as you want. The main body must be at most $8$ pages long.
% For the final version, one more page can be added.
% If you want, you can use an appendix like this one.  

% The $\mathtt{\backslash onecolumn}$ command above can be kept in place if you prefer a one-column appendix, or can be removed if you prefer a two-column appendix.  Apart from this possible change, the style (font size, spacing, margins, page numbering, etc.) should be kept the same as the main body.
%%%%%%%%%%%%%%%%%%%%%%%%%%%%%%%%%%%%%%%%%%%%%%%%%%%%%%%%%%%%%%%%%%%%%%%%%%%%%%%
%%%%%%%%%%%%%%%%%%%%%%%%%%%%%%%%%%%%%%%%%%%%%%%%%%%%%%%%%%%%%%%%%%%%%%%%%%%%%%%


\end{document}


% This document was modified from the file originally made available by
% Pat Langley and Andrea Danyluk for ICML-2K. This version was created
% by Iain Murray in 2018, and modified by Alexandre Bouchard in
% 2019 and 2021 and by Csaba Szepesvari, Gang Niu and Sivan Sabato in 2022.
% Modified again in 2023 and 2024 by Sivan Sabato and Jonathan Scarlett.
% Previous contributors include Dan Roy, Lise Getoor and Tobias
% Scheffer, which was slightly modified from the 2010 version by
% Thorsten Joachims & Johannes Fuernkranz, slightly modified from the
% 2009 version by Kiri Wagstaff and Sam Roweis's 2008 version, which is
% slightly modified from Prasad Tadepalli's 2007 version which is a
% lightly changed version of the previous year's version by Andrew
% Moore, which was in turn edited from those of Kristian Kersting and
% Codrina Lauth. Alex Smola contributed to the algorithmic style files.
