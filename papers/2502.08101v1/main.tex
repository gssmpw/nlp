%%%%%%%% ICML 2025 EXAMPLE LATEX SUBMISSION FILE %%%%%%%%%%%%%%%%%

\documentclass{article}

% Recommended, but optional, packages for figures and better typesetting:
\usepackage{microtype}
\usepackage{graphicx}
\usepackage{subfigure}
\usepackage{booktabs} % for professional tables

% hyperref makes hyperlinks in the resulting PDF.
% If your build breaks (sometimes temporarily if a hyperlink spans a page)
% please comment out the following usepackage line and replace
% \usepackage{icml2025} with \usepackage[nohyperref]{icml2025} above.
\usepackage{hyperref}


% Attempt to make hyperref and algorithmic work together better:
\newcommand{\theHalgorithm}{\arabic{algorithm}}

% Use the following line for the initial blind version submitted for review:
% \usepackage{icml2025}

% If accepted, instead use the following line for the camera-ready submission:
\usepackage[accepted]{icml2025}

% For theorems and such
\usepackage{amsmath}
\usepackage{amssymb}
\usepackage{mathtools}
\usepackage{amsthm}

% if you use cleveref..
\usepackage[capitalize,noabbrev]{cleveref}

\usepackage{color}

% general commands
\newcommand{\TODO}[1]{{\textbf{\textcolor{red}{TODO: #1}}}}
\newcommand{\NOTE}[1]{{\textbf{\textcolor{red}{NOTE: #1}}}}
\newcommand{\HK}[1]{\textcolor{red}{[KunHe: #1]}}
\newcommand{\operator}[1]{\textbf{\emph{#1}}}

\usepackage{xspace}
\newcommand{\name}[0]{SwapGT\xspace}



% Add a period to the end of an abbreviation unless there's one
% already, then \xspace.
\makeatletter
\DeclareRobustCommand\onedot{\futurelet\@let@token\@onedot}
\def\@onedot{\ifx\@let@token.\else.\null\fi\xspace}

\def\eg{\emph{e.g}\onedot} \def\Eg{\emph{E.g}\onedot}
\def\ie{\emph{i.e}\onedot, } \def\Ie{\emph{I.e}\onedot}
\def\cf{\emph{c.f}\onedot} \def\Cf{\emph{C.f}\onedot}
\def\etc{\emph{etc}\onedot} \def\vs{\emph{vs}\onedot}
\def\wrt{w.r.t\onedot} \def\dof{d.o.f\onedot}
\def\etal{\emph{et al}\onedot}
\def\st{\emph{s.t}\onedot}
\makeatother

% % \usepackage{algorithmicx}
% \usepackage{algpseudocode}  
% \usepackage{algorithm}

\renewcommand{\algorithmicrequire}{\textbf{Input:}}  % Use Input in the format of Algorithm  
\renewcommand{\algorithmicensure}{\textbf{Output:}} % Use Output in the format of Algorithm 


%%%%%%%%%%%%%%%%%%%%%%%%%%%%%%%%
% THEOREMS
%%%%%%%%%%%%%%%%%%%%%%%%%%%%%%%%
\theoremstyle{plain}
\newtheorem{theorem}{Theorem}[section]
\newtheorem{proposition}[theorem]{Proposition}
\newtheorem{lemma}[theorem]{Lemma}
\newtheorem{corollary}[theorem]{Corollary}
\theoremstyle{definition}
\newtheorem{definition}[theorem]{Definition}
\newtheorem{assumption}[theorem]{Assumption}
\theoremstyle{remark}
\newtheorem{remark}[theorem]{Remark}

% Todonotes is useful during development; simply uncomment the next line
%    and comment out the line below the next line to turn off comments
%\usepackage[disable,textsize=tiny]{todonotes}
\usepackage[textsize=tiny]{todonotes}


% The \icmltitle you define below is probably too long as a header.
% Therefore, a short form for the running title is supplied here:
\icmltitlerunning{Rethinking Tokenized Graph Transformers for Node Classification}
% \icmltitlerunning{Token Swapping for Enhanced Node Classification in Tokenized Graph Transformers}

\begin{document}

\twocolumn[
\icmltitle{Rethinking Tokenized Graph Transformers for Node Classification}
% \icmltitle{Token Swapping for Enhanced Node Classification in Tokenized Graph Transformers}
% It is OKAY to include author information, even for blind
% submissions: the style file will automatically remove it for you
% unless you've provided the [accepted] option to the icml2025
% package.

% List of affiliations: The first argument should be a (short)
% identifier you will use later to specify author affiliations
% Academic affiliations should list Department, University, City, Region, Country
% Industry affiliations should list Company, City, Region, Country

% You can specify symbols, otherwise they are numbered in order.
% Ideally, you should not use this facility. Affiliations will be numbered
% in order of appearance and this is the preferred way.
\icmlsetsymbol{equal}{*}

\begin{icmlauthorlist}
\icmlauthor{Jinsong Chen}{equal,cs,hccs}
\icmlauthor{Chenyang Li}{equal,cs,hccs}
\icmlauthor{GaiChao Li}{cs,hccs}
\icmlauthor{John E. Hopcroft}{hccs,cornell}
\icmlauthor{Kun He}{cs,hccs}
\end{icmlauthorlist}

\icmlaffiliation{cs}{School of Computer Science and Technology, Huazhong University of Science and Technology, China.}
\icmlaffiliation{hccs}{Hopcroft Center on Computing Science, Huazhong University of Science and Technology,  China}
\icmlaffiliation{cornell}{Department of Computer Science, Cornell University, USA}


\icmlcorrespondingauthor{Kun He}{brooklet60@hust.edu.cn}

% You may provide any keywords that you
% find helpful for describing your paper; these are used to populate
% the "keywords" metadata in the PDF but will not be shown in the document
% \icmlkeywords{Machine Learning, ICML}

\vskip 0.3in
]

% this must go after the closing bracket ] following \twocolumn[ ...

% This command actually creates the footnote in the first column
% listing the affiliations and the copyright notice.
% The command takes one argument, which is text to display at the start of the footnote.
% The \icmlEqualContribution command is standard text for equal contribution.
% Remove it (just {}) if you do not need this facility.

%\printAffiliationsAndNotice{}  % leave blank if no need to mention equal contribution
\printAffiliationsAndNotice{\icmlEqualContribution} % otherwise use the standard text.

\begin{abstract}
Node tokenized graph Transformers (GTs) have shown promising performance in node classification. The generation of token sequences is the key module in existing tokenized GTs which transforms the input graph into token sequences, facilitating the node representation learning via Transformer. In this paper, we observe that the generations of token sequences in existing GTs only focus on the first-order neighbors on the constructed similarity graphs, which leads to the limited usage of nodes to generate diverse token sequences, further restricting the potential of tokenized GTs for node classification. To this end, we propose a new method termed SwapGT. SwapGT first introduces a novel token swapping operation based on the characteristics of token sequences that fully leverages the semantic relevance of nodes to generate more informative token sequences. Then, SwapGT leverages a Transformer-based backbone to learn node representations from the generated token sequences. Moreover, SwapGT develops a center alignment loss to constrain the representation learning from multiple token sequences, further enhancing the model performance. Extensive empirical results on various datasets showcase the superiority of SwapGT for node classification.

\end{abstract}

%a short sentence describing your paper
% We propose a new graph transformer that introduces a novel token swapping operation to generate diverse token sequences to further enhance model performance. 

% key words
% graph Transformer, token swapping, token sequence, node classification


\section{Introduction}
\label{Sec:intro}


\section{Introduction}\label{sec:intro}
% Evaluating Natural Language Generation (NLG) tasks has always been a long-standing challenge, and it has become even more complex in the era of LLMs. Tasks that were previously treated as separate domains, such as text summarization, machine translation, and question answering, are now unified under the broader generative framework. As a result, traditional, task-specific evaluation metrics must adapt to accommodate the more complex and diverse one-to-many nature of generative outputs.

% Subjective evaluation methods, such as expert-based testing, are often considered the gold standard for NLG evaluation due to their ability to incorporate holistic reasoning and nuanced contextual understanding. However, these approaches are costly, difficult to scale, and prone to inconsistency due to variations in human judgment \cite{gao2023human,shi2024judging, gu2024survey}.


 Automated evaluation metrics like BERTScore~\cite{zhang2019bertscore} and BARTScore~\cite{yuan2021bartscore} offer scalable and cost-effective performance in evaluating Natural Language Generation (NLG) tasks. BERTScore, while leveraging pre-trained semantic embeddings, often struggles to capture contextual nuances and may overemphasize lexical similarity over factual accuracy or logical consistency~\cite{sellam2020bleurt,sai2022survey}. BARTScore, though designed to improve upon these limitations, still faces challenges in aligning with human judgments, particularly in tasks requiring domain-specific knowledge or complex reasoning.\cite{lu2022toward,liu2023llms}. As a result, while automated metrics provide efficiency and scalability, they often fall short of replicating the comprehensiveness and depth of expert-driven evaluations. This highlights the need for hybrid approaches that combine the strengths of human expertise with the scalability of automated methods.

As LLMs continue to evolve, employing them as judges, called ``LLM-as-a-Judge''  is becoming increasingly compelling. LLMs, trained on vast and diverse datasets, have developed a nuanced understanding of language, enabling them to assess the quality of generated content effectively. Their ability to provide consistent evaluations can mitigate the variability inherent in human assessments. Moreover, using LLMs for evaluation is scalable and efficient, allowing for rapid assessments across large datasets without the logistical constraints of coordinating human evaluators~\cite{dong-etal-2024-llm}.

% This method combines the scalability and consistency of automated systems with the detailed, context-sensitive insights characteristic of expert evaluations. Moreover, because LLMs are trained as generative models, they naturally excel at producing innovative and adaptable responses, further enhancing their evaluative capabilities. 

Yet, LLMs rely on contextualized information and often prioritize conventional details from training data. However, in certain cases, these elements may not be crucial, leading the model to misjudge their relative importance 
\cite{petroni2019language,jiang2020can,lin2021truthfulqa}.
Also, LLMs struggle to assign appropriate weights to different types of errors, such as factual errors versus grammatical errors 
\cite{sai2022survey,chiang2023can,zheng2023judging}.

% In today’s industry, data privacy and security are top priorities, which means sensitive or proprietary information is often excluded from the training datasets of LLMs. Thus, LLMs may lack exposure to domain-specific data during their training process, limiting their ability to fully understand and contextualize information for specialized tasks \cite{gururangan2020don}. 

% This gap can reduce their effectiveness in handling domain-specific applications, where nuanced, contextual understanding is critical for accurate and reliable performance.\cite{gururangan2020don}

In the NLG system, for tech companies, the primary users are often professional software developers or domain experts. When these users interact with a search system, they typically have a general idea of the desired output. As they review the results, they inherently apply a weighting mechanism to prioritize the most relevant components and focus on the information that aligns with their specific needs. Currently, a growing trend among companies is to use LLM-as-a-Judge to evaluate the quality of generated text. These evaluations often involve comparing the model’s output to a gold standard and assessing whether it meets predefined criteria. While this approach has its merits, a significant limitation arises: LLMs frequently fail to accurately weigh the importance of different components in the text\cite{petroni2019language,jiang2020can,lin2021truthfulqa}. For instance, in a software development context, an LLM might overemphasize the importance of a software release date thus giving rejection, even when the user's main objective is to obtain details about a specific feature or functionality.
Such trivial misunderstandings significantly limit the usefulness of LLM-as-a-Judge in domain-specific contexts. 

% Without a mechanism to align the model’s evaluation criteria with the priorities of expert users, the generated text may fail to meet their expectations, ultimately reducing the system’s overall effectiveness.

% [Reference Needed: Importance of Aligning AI with User Needs].

% While many companies might want to train their own LLMs to gain a competitive edge or tailor models to their specific needs, there are multiple significant barriers that make this difficult.  For instance,
% Training an LLM requires enormous computational power, typically involving thousands of GPUs or TPUs running for weeks or even months. The cost of these resources is prohibitively high for most organizations.\cite{brown2020language,patterson2022carbon}. In addition to other challenges such as large-scale, high-quality data.Scalability and Infrastructure
% \cite{bender2021dangers,rajbhandari2020zero}

In this paper, we present a case study that addresses the aforementioned issue through two key strategies: "Explicit Error Weighting" and "Prompt Engineering". In the first step, we follow the original setting to use LLM-as-a-Judge (without customized prompt). Then we calculate the Human Alignment Rate (HAR). 
In our second step judge, we tested different LLM-as-Judges individually with our customized prompt. On average, we achieve a 6.4\% improvement in HAR, demonstrating the effectiveness of this method in aligning LLM decisions with human judgments.


% This misalignment between the LLM’s evaluation and the user’s actual needs highlights a critical issue: LLMs lack the ability to understand the contextual relevance of specific details in domain-specific scenarios [Reference Needed: Domain-Specific Challenges in NLP].

% Specifically, we tested our approach on a designated dataset using five widely adopted LLMs, with a focus on enhancing decision-making through a two-step evaluation process.















\section{Relation Work}
\label{Sec:rw}


\subsection{Graph Neural Networks}
GNNs~\cite{gnn3,gnn2,rlp,pamt,ncn} have shown remarkable performance in this task. 
Previous studies \cite{gat,jknet,sgc,appnp} have primarily concentrated on the incorporation of diverse graph structural information into the message-passing framework. 
Classic deep learning techniques, such as the attention mechanism \cite{gat,gatv2} and residual connections \cite{jknet,gcnii}, have been exploited to enhance the information aggregation on graphs. 
Moreover, aggregating information from high-order neighbors \cite{appnp,mixhop,h2gnn} or nodes with high similarity across different feature spaces \cite{geomgcn} has been demonstrated to be efficacious in improving model performance.

Follow-up GNNs have focused on the utilization of complex graph features to extract distinctive node representations. 
A prevalent strategy entails the utilization of signed aggregation weights \cite{fagcn,gprgnn,acmgnn,glognn} to optimize the aggregation operation. 
In this way, positive and negative values are respectively associated with low- and high-frequency information, thereby enhancing the discriminative power of the learned node representations.
Nevertheless, restricted by the inherent limitations of message-passing mechanism, the potential of GNNs for graph data mining has been inevitably weakened.
Developing a new graph deep learning paradigm has attracted great attention in graph representation learning. 



\subsection{Graph Transformers}
GTs~\cite{polynormer,agt,cob} have emerged as a novel architecture for graph representation learning and have exhibited substantial potential in node classification. 
A commonly adopted design paradigm for GTs is the combination of Transformer modules with GNN-style modules to construct hybrid neural network layers, called hybrid GTs~\cite{nodeformer,sgformer,specformer}. 
In this design, Transformer is employed to capture global information, while GNNs are utilized for local information extraction~\cite{graphgps,polynormer,signgt}.
Despite effectiveness, directly utilizing Transformer to model the interactions of all node pairs could occur the over-globalization issue~\cite{cob}, inevitably weakening the potential for graph representation learning.

An alternative yet effective design of GTs involves transforming the input graph into independent token sequences termed tokenized GTs \cite{ansgt,nagphormer,vcrgt,polyformer,ntformer}, which are then fed into the Transformer layer for node representation learning. 
Neighborhood tokens~\cite{nagphormer,nag+,polyformer,vcrgt,ntformer} and node tokens~\cite{ansgt,ntformer,vcrgt} are two typical elements in existing tokenized GTs.
The former, generally constructed by propagation approaches, such as random walk~\cite{nagphormer,nag+} and personalized PageRank~\cite{vcrgt}.
The latter is generated by diverse sampling methods based different similarity measurements, such as PageRank score~\cite{vcrgt} and attribute similarity~\cite{ansgt}. 
Since tokenized GTs only focus on the generated tokens, they naturally avoiding the over-globalization issue.

As pointed out in previous study~\cite{vcrgt}, node token oriented GTs are more efficient in capturing various graph information, such as long-range dependencies and heterophily, compared to neighborhood token oriented GTs.
However, we identify that previous methods only leverage a small subset of nodes as tokens for node representation learning, which could limit the model ability of deeply exploring graph information.
In this paper, we develop a new method \name that introduces a novel token swapping operation to produce more informative token sequences, further enhancing the model performance.


\section{Preliminaries}
\label{Sec:pre}


\subsection{Node Classification}
Suppose an attributed graph is denoted as $\mathcal{G}=(V, E, \mathbf{X})$ where $V$ and $E$ are the sets of nodes and edges in the graph.
$\mathbf{X} \in \mathbb{R}^{n \times d}$ is the attribute feature matrix, where $n$ and $d$ are the number of nodes and the dimension of the attribute feature vector, respectively.
We also have the adjacency matrix $\mathbf{A} = \{0, 1\}^{n\times n}$.
If there is an edge between nodes $v_i$ and $v_j$, $\mathbf{A}_{ij} = 1$; otherwise, $\mathbf{A}_{ij} = 0$. 
$\hat{\mathbf{A}}$ denotes the normalized version calculated as $\hat{\mathbf{A}}=(\mathbf{D}+\mathbf{I})^{-1/2}(\mathbf{A}+\mathbf{I})(\mathbf{D}+\mathbf{I})^{-1/2}$ where $\mathbf{D}$ and $\mathbf{I}$ are the diagonal degree matrix and the identity matrix, respectively.
In the scenario of node classification, each node is associated with a one-hot vector to identify the unique label information, resulting in a label matrix $\mathbf{Y} \in \mathbb{R}^{n \times c}$ where $c$ is the number of labels.
Given a set of labeled nodes $V_L$, the goal of the task is to predict the labels of the rest  nodes in $V-V_L$.


\subsection{Transformer}
Here, we introduce the design of the Transformer layer, which is the key module in most GTs. 
There are two core components of a Transformer layer~\cite{transformer}, named multi-head self-attention (MSA) and feed-forward network (FFN).
Given the model input $\mathbf{H}^{n\times d}$, the calculation of MSA is as follows:
\begin{equation}
    \mathrm{MSA}(\mathbf{H}) = (||_{i=1}^{m}head_i)\cdot \mathbf{W}_{o},
    \label{eq:msa}
\end{equation}
\begin{equation}
    head_i = \mathrm{softmax}\left(\frac{[(\mathbf{H}\cdot\mathbf{W}^{Q}_{i}) \cdot (\mathbf{H}\cdot\mathbf{W}^{K}_{i})^{\mathrm{T}}]}{\sqrt{d_k}}\right)\cdot (\mathbf{H}\cdot\mathbf{W}^{V}_{i}), 
    \label{eq:single-head}
\end{equation}
where $\mathbf{W}^{Q}_{i}$, $\mathbf{W}^{K}_{i}$ and $\mathbf{W}^{V}_{i}$ are the learnable parameter matrices of the $i$-th attention head.
$m$ is the number of attention heads.
$||$ denotes the vector concatenation operation.
$\mathbf{W}_{o}$ denotes a projection layer to obtain the final output of MSA.

FFN is constructed by two linear layers and one non-linear activation function:
\begin{equation}
    \mathrm{FFN}(\mathbf{H}) = \sigma(\mathbf{H}\cdot\mathbf{W}^{1})\cdot\mathbf{W}^{2},
    \label{eq:ffn}
\end{equation}
where $\mathbf{W}^{1}$ and $\mathbf{W}^{2}$ denote learnable parameters of the two linear layers and $\sigma(\cdot)$ denotes the GELU activation function.

\section{Methodology}
\label{Sec:method}
\section{Problem Definition and Motivation}

\subsection{Problem Definition}\label{def:mccs}
This paper aims to detect Module-Induced Critical Scenarios ({\mccs}s) for a specified module, defined as follows:

\begin{definition} [$\mathcal{M}$-Induced Critical Scenario] \label{def-mccs} 
Given a ADS $\mathcal{A} = \{{M}^{1}, \ldots, {M}^{K}\}$ that considers multiple modules as well as a target module $\mathcal{A}\in \mathcal{A}$ to be tested, the $\mathcal{M}$-Induced Critical Scenario $s$ satisfies the conditions: 
\begin{itemize}
    \item[a.] $s \in \mathbb{S}^{Fail}$ % scenario is failed
    \item[b.] $\exists s_i\in s. \ error(s_i, \mathcal{M})=True$
    \item[c.] $\forall s_i \in s. \ \forall M\in \mathcal{A}. \ M\neq \mathcal{M} \wedge error(s_i, M)=False$
\end{itemize}
where $\mathbb{S}^{Fail}$ denotes the set of scenarios containing ADS failures, $s_{i}$ is a scene in $s$ ,and the $error$ function determines whether the module $\mathcal{M}$ exhibits errors in a specific scene. Intuitively, if we identify a critical scenario $s$ that results in a failure, and only $\mathcal{M}$ induces errors while all other modules operate correctly across all scenes in $s$, then we can conclude that the failure is primarily caused by $\mathcal{M}$.
\end{definition}

Note that while failures that do not meet the \mccs conditions may still be useful, they do not align with our objectives as we aim to evaluate the quality of individual modules within the ADS. Specifically, we need to accurately localize the root cause in terms of specific modules. If several modules exhibit errors in a critical scenario, it becomes challenging to conclusively determine which module is the root cause. Hence, it is not a good case for developers to analyze and repair. Furthermore, based on our definition, this situation highlights our two main challenges: 1) the $error$ function, which determines whether a module functions correctly, and 2) the effective method to identify \mccs $s$ that satisfies all necessary conditions.





\subsection{Preliminary Study}\label{sec: perliminary_study}

Based on the problem definition, we would like to understand the limitations of existing methods in detecting \mccs. Specifically, both end-to-end system-level testing and module-level testing may generate failures that reveal limitations of individual modules. Therefore, we first conduct an empirical study to evaluate: 1) whether failures generated by system-level testing adequately reflect the diversity of module weaknesses, and 2) whether errors identified through module-level testing can trigger system failures.

\subsubsection{The ability of existing scenario-based testing methods to generate \mccs}\label{sec:perliminary_exist_mccs}


\begin{table}[]
    \centering
    \caption{Module Failures and Collision Distributions of Exising Methods}
    \vspace{-10pt}
    \resizebox{0.65\linewidth}{!}{
    \begin{tabular}{c|ccccc}
    \toprule
         Method & $\mathcal{M}^{\text{Perc}}$ICS & $\mathcal{M}^{\text{Pred}}$ICS & $\mathcal{M}^{\text{Plan}}$ICS & $\mathcal{M}^{\text{Ctrl}}$ICS & Non-\mccs\\
         \midrule
         AVFuzzer & 9  & 17  & 23   & 2 & 47\\
         BehAVExplor & 6 & 57 & 9&  3 & 190 \\
         \bottomrule
    \end{tabular}}
    \vspace{-10pt}
    \label{tab: preliminary_module}
\end{table}

Existing scenario-based methods aim to identify test scenarios that cause ADS failures efficiently but lack root cause analysis for module errors. To explore the capabilities of existing methods in generating \mccs, we conducted experiments using two scenario-based testing scenario generation methods, AVFuzzer\cite{li2020av} and BehAVExplor\cite{cheng2023behavexplor}. In these experiments, we utilized Pylot\cite{gog2021pylot} as the tested ADS and CARLA as the simulator. Starting with four basic scenarios (detailed in Section~\ref{sec: Evaluation}), each method was run for 6 hours, and we recorded the module errors and \mccs collected during this period.

The results shown in table~\ref{tab: preliminary_module} of these two existing works show high similarity. They generate many collisions, however, most are non-\mccs, in which multiple modules typically experience errors before a collision occurs. On the other hand, though some \mccs have been generated, the distribution is highly uneven, with most \mccs introduced by the prediction module, while \mccs from other modules are rare. This imbalance makes it challenging for developers to improve the corresponding modules effectively.


\subsubsection{Limitation on module-level evaluation for ADS testing}
\begin{table}[!t]
    \centering
    \caption{The ratio of module errors that can cause system failures}
    \vspace{-10pt}
    \resizebox{0.65\linewidth}{!}{
    \begin{tabular}{c|ccc|ccc|ccc}
    \toprule
     \multirow{2.5}*{Module}   & \multicolumn{3}{c|}{Perception} & \multicolumn{3}{c|}{Prediction} & \multicolumn{3}{c}{Planning} \\
     \cmidrule(lr){2-4}\cmidrule(lr){5-7}\cmidrule(lr){8-10}
      & 10\% & 20\% & 50\% & 0.1m & 0.5m & 1m & 0.1m & 0.2m & 0.5m\\
     \midrule
     R1 & 0.02 & 0.03 & 0.29 & 0.02 & 0.20 & 0.57 & 0.03 & 0.19 & 0.28\\
     R2 & 0.01 & 0.01 & 0.09 & 0.01 & 0.15 & 0.40 & 0.03 & 0.14 & 0.30\\
     R3 & 0.07 & 0.09 & 0.18 & 0.01 & 0.15 & 0.39 & 0.04 & 0.11 & 0.34\\
     \midrule
     Average & 0.03 & 0.06 & 0.19 & 0.01 & 0.17 &0.45 & 0.03 & 0.15 & 0.31\\
    \bottomrule
    \end{tabular}
    }
    \vspace{-10pt}
    \label{tab:preliminary_fail}
\end{table}

To better investigate the relationship between module-level errors and system-level failures in ADS, we manually introduced random noise to the output of the perception, prediction and planning module since the results in table~\ref{tab: preliminary_module} tend to be error-prone. For the perception module, we applied one of three operations—\textit{Zoom In}, \textit{Zoom Out}, and \textit{Random Offset}—randomly to each bounding box. For the prediction and planning modules, we added random perturbations to trajectory nodes. For each module, we established three levels of random error limits: conventional, moderate, and extreme. The specific settings are as follows:
\begin{inparaitem}
    \item Perception: [10\%, 20\%, 50\%];
    \item Prediction: [0.1m, 0.5m, 1m];
    \item Planning: [0.1m, 0.2m, 0.5m].
\end{inparaitem}
We randomly selected 100 normal running scenarios from \ref{sec:perliminary_exist_mccs}, with each experiment only perturbing one module's output. To mitigate the effects of randomness, each experiment was repeated three times.


Table~\ref{tab:preliminary_fail} shows the results of operations after introducing manual injections.
As the results show, aside from experiments with extreme perturbations (the third column for each module), the module errors alone does not effectively lead to system-level failures. With conventional-level perturbations to module outputs, only a few running failures occurred; even with moderate-level perturbations, the failure rate reached only up to 17\%. This suggests a significant gap between module-level errors and system failures, indicating the need for a mapping method to rapidly identify the corresponding \mccs and bridge this gap effectively.

\begin{ansbox}
   \textbf{Motivation:} Existing system-level and module-level testing methods failed to generate {\mccs}s. This limitation motivates us to develop an effective approach to detecting system failures induced by specified modules. 
\end{ansbox}


\begin{figure*}[!t]
    \centering
    \includegraphics[width=0.85\linewidth]{fig/rc_overview.png}
    \caption{Overview of \tool}
    \label{fig:overview}
\end{figure*}


\section{Approach}\label{sec:method}

\subsection{Overview}
Fig.~\ref{fig:overview} provides a high-level overview of \tool for generating {\mccs}s given initial seeds and the user-specified module $\mathcal{M}$. 
\tool comprises three main components: \oracle, \feedback and \select. 
\oracle functions as an oracle to check whether a scenario satisfies \textit{Definition}~\ref{def-mccs} and qualifies as an \mccs. 
\feedback provides feedback that guides the search process, which jointly considers the system-level specifications (i.e., safety) and the module-specific aspects (i.e., the extent of errors in $\mathcal{M}$).
% ensuring the effectiveness of the search for {\mccs}s.
\select implements an adaptive strategy, including seed selection and mutation, to generate new scenarios based on the module-specific feedback score, thereby improving search performance. 
Specifically, following a classical search-based fuzzing approach, \tool maintains a seed corpus with valuable seeds that facilitate the identification of {\mccs}s. In each iteration, \select first chooses a seed with a higher feedback score and applies an adaptive mutation to the selected seed to generate a new test scenario. Note that a higher feedback score indicates a greater likelihood of evolving into an \mccs. 
Then, \oracle and \feedback provide the identification result and feedback score for the new seed by analyzing the scenario observation, respectively.


Algorithm~\ref{algo:workflow} presents the main algorithm of \tool. The algorithm takes as input an initial seed corpus \( \mathbf{Q} \), a module-based ADS \( \mathcal{A} = \{{M}^{1}, \dots, {M}^{K}\} \), which consists of \( K \) modules, and the user-specified module $\mathcal{M}$ under test.
The output is a set of module $\mathcal{M}$ caused critical scenarios $s$ (Line 13).
In detail, \tool begins by initializing an empty set for $\mathbf{F}_{\mathcal{M}}$ (Line 1). 
Then \tool starts the fuzzing process, which continues until the given budget expires (lines 2-12). 
In each iteration, \tool first uses \select to return a new scenario \( s' \) and the feedback score \(\phi_s\) of its source seed \( s \) (Line 3). 
This new scenario \( s' \) is executed in the simulator with the ADS under test \( \mathcal{A} \), collecting the scenario observation \( \mathcal{O}(s') = \{\mathcal{A}(s'), \mathcal{Y}(s')\} \) including the ADS observation \( \mathcal{O}_{\mathcal{A}}(s') \) and the Simulator observation \( \mathcal{Y}_{\mathcal{A}}(s') \) (Line 4).
Based on these observations, \oracle identifies if the scenario \( s' \) contains system failures and if module \( \mathcal{M}^k \) is the root cause, returning the identification result \( r_{\mathcal{M}} \), module errors $\delta^{\mathcal{A}}$ and safety-critical distance $\delta^{\mathcal{A}}$ (Line 5). 
If \( r_{\mathcal{M}} \) is identified as \textit{Fail}, \tool keeps the scenario \( s' \) in the critical scenario set \(\mathbf{F}_{\mathcal{M}}\) (Line 6-7). 
Otherwise, \feedback calculates a feedback score for the benign scenario \( s' \) based on the module errors \( \delta^{\mathcal{A}} \) and the safety-critical distance \( \delta^{\text{safe}} \) (Lines 8-9).
A higher feedback score \( \phi_{s'} \) indicates a higher potential of \( s' \) for generating {\mccs}s. If the feedback score of seed \( s' \) is higher than that of its parent seed \( s \), \tool retains \( s' \) in the corpus $\mathbf{Q}$ for further optimization (Line 10-11). 
Finally, the algorithm ends by returning $\mathbf{F}_{\mathcal{M}}$ (Line 13).

\begin{algorithm}[!t]
\small
\SetKwInOut{Input}{Input}
\SetKwInOut{Output}{Output}
\SetKwInOut{Para}{Parameters}
\SetKwProg{Fn}{Function}{:}{}
\SetKwFunction{AE}{\textbf{AdaptiveScenariGeneration}}
\SetKwFunction{OI}{\textbf{ModuleSpecificOracle}}
\SetKwFunction{FD}{\textbf{ModuleSpecificFeedback}}
\SetKwComment{Comment}{\color{blue}// }{}
\Input{
Initial seed corpus $\mathbf{Q}$ \\
ADS under test $\mathcal{A} = \{{M}^{1}, ..., {M}^{K}\}$\\
Specified module $\mathcal{M} \in \mathcal{A}$
}
\Output{
$\mathcal{M}$ module-induced critical scenarios $\mathbf{F}_{\mathcal{M}}$
}
$\mathbf{F}_{\mathcal{M}} \gets \{\}$ \\
\Repeat{given time budget expires}{
$s', {\phi}_{s} \gets \AE(\mathbf{Q})$ \Comment{Generate new scenarios}
$\mathcal{O}(s') = \{\mathcal{A}(s'), \mathcal{Y}(s')\} \gets \textbf{Simulator}(s', \mathcal{A})$ \\
$r_{\mathcal{M}}, \delta^{\mathcal{A}}, \delta^{\text{safe}} \gets \OI(\mathcal{A}(s'), \mathcal{Y}(s'), \mathcal{M}^k)$ \Comment{Analyze module errors}
\eIf{$r_{\mathcal{M}}$ is \textit{Fail}}{
    \Comment{Update failure sets}
    $\mathbf{F}_{\mathcal{M}} \gets \mathbf{F}_{\mathcal{M}} \cup \{s'\}$ \Comment{Update discovered {\mccs}s}
}{
    \Comment{Update seed corpus}
    ${\phi}_{s'} \gets \FD(\delta^{\text{safe}}, \delta^{\mathcal{A}})$ \Comment{Calculate feedback score}
    \If{${\phi}_{s'} > {\phi}_{s}$}{
        $\mathbf{Q} \gets \mathbf{Q} \cup \{s'\}$ \Comment{Update corpus for {\mccs}s search}
    }
}
}
\Return $\mathbf{F}_{\mathcal{M}}$
\caption{Workflow of \tool}
\label{algo:workflow}
\end{algorithm}

\subsection{Module-Specific Oracle}
The purpose of \oracle is to serve as an oracle for automatically detecting {\mccs}s by determining whether a given scenario satisfies all conditions outlined in \textit{Definition~\ref{def:mccs}}. This involves two parts: (1) detecting system failures in the scenario (\textit{Definition~\ref{def-mccs}.a}) and (2) determining errors for each module in the ADS (\textit{Definition~\ref{def-mccs}.b} and \textit{Definition~\ref{def-mccs}.c}).

For part (1), we consider the occurrence of collisions as a safety-critical metric to identify system failures.
For part (2), the main challenge is obtaining ground truth to evaluate the performance of individual modules without human annotation. To address this, we design \textit{Individual Module Metrics} to independently measure errors for each module using collected scenario observations, covering the four main modules in the ADS: perception, prediction, planning, and control.

\subsubsection{Safety-critical Metric}\label{sec:safe-metric} We check if the scenario contains ADS failures by detecting collisions. Specifically, we first calculate the minimum distance between the ego vehicle and other objects:
\begin{equation}\label{eq:safe}
    \delta_{s}^{\text{safe}} = \min \left\{ \| R_{\text{bbox}}({p}^{0}_{t}) - R_{\text{bbox}}({p}^{n}_{t}) \|_2 \ \big| \ t \in [0, T], \ n \neq 0 \right\}
\end{equation}
where \( {p}^{0}_{t} \) represents the position of the ego vehicle at time \( t \), \( R_{\text{bbox}}(\cdot) \) calculates the bounding box for an object based on its position \( p \), and \( {p}^{n}_{t} \) represents the position of the \( n \)-th object at the same time. 
Therefore, safety-critical failures can be detected if \( \delta_{s}^{\text{safe}} = 0 \).


\subsubsection{Individual Module Oracles}\label{sec:module-metric}
Given a scenario \( s \) with its observation \(\mathcal{O}(s) = \{\mathcal{A}(s), \mathcal{Y}(s)\}\), we first design individual metrics for each module to measure module-level errors \(\delta^{{M}} = \{\delta^{{M}}_t \ \big| \ t \in [0, \dots, T] \}\) for each module \({M} \in \mathcal{A} \), where \(\delta^{{M}}_t\) denotes the module error at timestamp \( t \) and \( T \) is the termination timestamp of the scenario \( s \).
We detail the calculation of \(\delta^{{M}}_t\) covering the \textit{Perception}, \textit{Prediction}, \textit{Planning}, and \textit{Control} modules as follows:

\noindent \textit{(1) Perception Module.} Given a scenario $s$, the error in the \textit{Perception} module can be directly measured by comparing object bounding boxes between Simulator observation $\mathcal{Y}(s)$ and ADS observations $\mathcal{Y}(s)$.
We adopt a weighted Intersection over Union (IoU)~\cite{girshick2014rich} to measure the errors in the perception module $\mathcal{M}^{\text{perc}}$, which is a widely recognized metric in object detection.
Specifically, the errors of the perception module at timestamp $t$ is calculated as follows:
\begin{equation}
    {\delta^{\text{perc}}_{t}} = 1 - \frac{1}{N_{t}} \sum_{n=1}^{N_{t}} (\frac{D-d_t^{n}}{D} \cdot \frac{|B_t^{n} \cap b_t^n|}{|B_t^{n} \cup b_t^{n}|})
\end{equation}
where \( N_{t} \) represents the number of detected objects within the perception range \( D \) meters, \( B_{t}^{n} \) and \( b_{t}^{n} \) denote the detected bounding box and the ground truth bounding box of the \( n \)-th object, respectively, obtained from the ADS observation $\mathcal{A}_{t} \in \mathcal{A}(s)$ and the scenario observation $\mathcal{Y}_{t} \in \mathcal{Y}(s)$. 
The weight \( \frac{D-d_t^n}{D} \) assigns a higher weight to objects closer to the ego vehicle, where \( d_t^n \) denotes the distance between the \( n \)-th object and the ego vehicle at timestamp \( t \).
A higher value of \(\delta^{\text{perc}}\) represents a greater detection error in the perception module, indicating a potential safety-critical situation.


\noindent \textit{(2) Prediction Module.} Unlike the \textit{Perception} module, directly comparing the predicted trajectories in ADS observation \(\mathcal{A}(s)\) with the collected trajectories in Simulator observation \(\mathcal{Y}(s)\) cannot accurately reflect errors in the \textit{Perception} module. This is because the inputs to the \textit{Prediction} module are derived from the \textit{Perception} module, which may introduce perception errors that subsequently affect prediction outcomes. 
To address this, we adopt a perception-biased trajectory to measure the errors in the \textit{Prediction} module. 
Specifically, at timestamp \( t \), the perception-biased trajectory for the $n$-th object within perception range $D$ is determined by incorporating the biases present in the perception module, formulated as:
\begin{equation}
    \overline{\tau}_t^n = \left\{\ p_{t+k}^n + \Delta p_t^n \mid k \in [0, \ldots, H_{\text{pred}}] \right\}
\end{equation}
where \( p_{t+k}^n \) represents the ground-truth position in the Simulator observation \(\mathcal{Y}_t\), and \( \Delta p_{t+k}^n \) is the position shift observed in the \textit{Perception} module, and $H_\text{pred}$ is the prediction horizon. Note that the position shift \(\Delta p_t^n = (\Delta x_t^n, \Delta y_t^n)\) represents the positional difference between the detected output and the ground truth at timestamp $t$. Since the following detect output after timestamp $t$ is unavailable, and the offset does not fluctuate significantly within a shorter prediction window(about 0.5 to 1 seconds) in most cases, we apply $\Delta p_t^n$ to its predicted sequence.

Consequently, we can measure the error of the \textit{Prediction} module by comparing the predicted trajectories with the perception-biased trajectories. This comparison is quantified by:
\begin{equation}
    \delta_{t}^{\text{pred}} = \max \left\{  \frac{D-d_t^{n}}{D} \cdot \|\hat{p}^{n}_{t+k} - \overline{p}^{n}_{t+k} \|_{2} \ \big| \ \hat{p}^{n}_{t+k} \in \tau_{t}^{n}, \ \overline{p}^{n}_{t+k} \in \overline{\tau}_{t}^{n}, \ n \in [1, \dots, N_{t}] \right\}
\end{equation}
where \( N_{t} \) is the number of detected objects within the perception range \( D \), \( \tau_{t}^{n} \) is the predicted trajectory for the \( n \)-th object, and \( d_t^n \) denotes the distance between the \( n \)-th object and the ego vehicle at timestamp \( t \). The calculation of \(\delta_{t}^{\text{pred}}\) selects the maximum error in all predicted trajectories because this emphasizes the worst-case performance within a given scenario, which is crucial for assessing the robustness and safety of the \textit{Prediction} module. 

\noindent \textit{(3) Planning Module.} We measure errors in the \textit{Planning} module from a safety perspective by evaluating the distance to collisions.
Similar to the prediction module, biases present in upstream modules (i.e., \textit{Perception} and \textit{Prediction}) affect the evaluation of the planning module when directly using ground truth data collected from the Simulator observation \(\mathcal{Y}(s)\). 
To mitigate these biases, we measure errors in the \textit{Planning} module by assessing if the planned trajectories collide with objects detected by the upstream modules. This is calculated by:
\begin{equation}\label{eq:plan}
    \delta^{\text{plan}}_{t} = \sum_{n=1}^{N_{t}} \sum_{k=1}^{H_{\text{plan}}} \mathbb{I}(\| R_{\text{bbox}}(p^{\mathcal{A}}_{t+k}) - R_{\text{bbox}}(p_{t+k}^{n}) \|_{2}=0)
\end{equation}
where \( N_{t} \) represents the number of detected objects, \( H_{\text{plan}} \) is the planning horizon, \( R_{\text{bbox}}(\cdot) \) calculates the bounding box for an object based on its position \( p \), \( \mathbb{I}(\cdot) \) is an indicator function. The indicator function \( \mathbb{I}(condition) \) returns 1 if the \( condition \) is true and 0 otherwise. Additionally, \( p^{\mathcal{A}}_{t+k} \in \mathcal{P}_{t} \) denotes a trajectory point planned by the \textit{Planning} module, and \( p_{t+k}^{n} \in \tau_{t}^{n} \) is the predicted trajectory point for the \( n \)-th object at timestamp \( t \). Ideally, the planned trajectory should be collision-free, maintaining a safety distance from all predicted states of all objects (i.e., \(\delta^{\text{plan}}_{t} = 0\)). Therefore, a larger \(\delta^{\text{plan}}_{t}\) indicates that the planning module has safety-critical errors, such as collisions.


\noindent \textit{(4) Control Module.} 
The control command is directly applied to the vehicle to manage its movement by following a trajectory from the upstream \textit{Planning} module. Obtaining a ground truth for control commands is challenging. Thus, we evaluate the \textit{Control} module by comparing the actual movement of the vehicle with the planned movement from the \textit{Planning} module.
At timestamp \( t \), we calculate the error for the \textit{Control} module by:
\begin{equation}
    \delta^{\text{ctrl}}_{t} = \| p_{t+1}^{\mathcal{A}} - p_{t+1}^{0} \|_{2} + \| v_{t+1}^{\mathcal{A}} - v_{t+1}^{0} \|_{2}
\end{equation}
where \( p_{t+1}^{\mathcal{A}} \) and \( v_{t+1}^{\mathcal{A}} \) represent the planned position and velocity from the \textit{Planning} module at timestamp \( t \), and \( p_{t+1}^{0} \) and \( v_{t+1}^{0} \) are the actual position and velocity of the vehicle at timestamp \( t+1 \). 
This error \(\delta^{\text{ctrl}}_{t}\) quantifies how well the \textit{Control} module is executing the planned trajectory. A larger error value indicates a significant deviation from the planned path and speed, suggesting potential issues in the \textit{Control} module (i.e., inaccuracies in executing the planned trajectory).


\begin{algorithm}[!t]
\small
\SetKwInOut{Input}{Input}
\SetKwInOut{Output}{Output}
\SetKwInOut{Para}{Parameters}
\SetKwProg{Fn}{Function}{:}{}
\SetKwComment{Comment}{\color{blue}// }{}
\Input{
Scenario observation $\mathcal{O}(s) = \{{\mathcal{A}}(s), \mathcal{Y}(s)\}$ \\
ADS $\mathcal{A} = \{{M}^{1}, \dots, {M}^{K} \}$ \\
User-specified module $\mathcal{M}$
}
\Output{
\mccs identification result $r_{\mathcal{M}}$ \\
Module errors $\delta^{\mathcal{A}} = \{\delta^{{M}^{{1}}}, \dots, \delta^{{M}^{{K}}}\}$ \\
Safety-critical distance $\delta^{\text{safe}}$
}

 $\delta^{\mathcal{A}} \gets \{\}$, $r_{\mathcal{M}} \gets Pass$ \\
% $\delta^{\mathcal{A}} \gets \{\}, r_b \gets 0, r_c \gets 0$ \\
$\delta^{\text{safe}} \gets \text{calculate safety-critical distance by Eq.~\ref{eq:safe}}(\mathcal{Y}(s))$ \Comment{Safety-critical metric} 
\If{$\delta^{\text{safe}} \neq 0$}{
    $r_{\mathcal{M}} \gets Fail$ \Comment{Fail to satisfy Definition~\ref{def:mccs}.a}
}
\For{${M} \ in \ \mathcal{A}$}{
\Comment{Module error calculated by Individual Module Metrics}
    $\delta^{{M}} \gets \text{calculate the module-level error according to Section~\ref{sec:module-metric}}$ \\
    $\hat{\delta}^{{M}} \gets \text{filter module errors by Eq.~\ref{eq:system_error}}$ \\
    % $f_{\mathcal{M}^{i}}(\delta^{\mathcal{M}^{i}}) \gets \text{calculate system-level affects by Eq.~\ref{eq:system_error}}$ \\
    $\delta^{\mathcal{A}} \gets \delta^{\mathcal{A}} \cup \{\delta^{{M}}\}$ \\
    % \eIf{$\mathcal{M}^{i} = \mathcal{M}^{k}$}{
    %     $r_b \gets f_{\mathcal{M}^{i}}(\delta^{\mathcal{M}^{i}})$
    % }{
    %     $r_c \gets r_c + f_{\mathcal{M}^{i}}(\delta^{\mathcal{M}^{i}})$
    % }
    \Comment{Definition~\ref{def:mccs}.b}
    \If{${M} = \mathcal{M} \ and \ \hat{\delta}^{\mathcal{M}} = 0 $}{
        $r_{\mathcal{M}} \gets Fail$ 
    }

    \Comment{Definition~\ref{def:mccs}.c}
    \If{${M} \neq \mathcal{M} \ and \ \hat{\delta}^{{M}} \neq 0 $}{
        $r_{\mathcal{M}} \gets Fail$ 
    }
    
}
\Return $r_{\mathcal{M}}, \delta^{\mathcal{A}}$, $\delta^{\text{safe}}$
\caption{Algorithm for \oracle}
\label{algo:oracle}
\end{algorithm}
\subsubsection{Module-Specific Filter}\label{sec:filter}
The Module-Specific Filter aims to filter out less relevant module errors, as driving scenes farther from the termination have less impact on the final results~\cite{stocco2020misbehaviour, stocco2022thirdeye}.
The filter considers only module errors within a detection window \([T - \Delta t, T]\), where \( T \) is the timestamp of the occurrence of system failures in the scenario, and \( \Delta t \) is the detection window size. Therefore, the filtered module errors are calculated by:
\begin{equation}\label{eq:system_error}
    \hat{\delta}^{{M}} = \sum_{t=T-\Delta t}^{T} \mathbb{I}(\delta_t^{{M}} > \lambda^{{M}})
\end{equation}
where \( \lambda^{{M}} \) is the tolerance threshold for module ${M}$, and \( \mathbb{I} \) is an indicator function. The indicator function \( \mathbb{I}(condition) \) returns 1 if the \( condition \) is true and 0 otherwise. 

\subsubsection{Workflow of \oracle}
Algorithm~\ref{algo:oracle} illustrates the workflow of \oracle.
Specifically, the algorithm takes as inputs the scenario observation \( \mathcal{O}(s) \), the ADS \( \mathcal{A} \), and the user-specified module \( \mathcal{M} \), and it outputs three key results: the identification result \( r_{\mathcal{M}} \), the module errors's set \( \delta^{\mathcal{A}} \), the safety-critical distance \( \delta^{\text{safe}} \). 
Initially, \oracle creates an empty error set \( \delta^{\mathcal{A}} \) to store module errors and an identification flag \( r_{\mathcal{M}} \) set to `pass' (Line 1). Then, the algorithm calculates the safety-critical distance using Eq.~\ref{eq:safe} and checks for system failures (Lines 2-4), aiming to confirm the satisfaction of \textit{Definition~\ref{def-mccs}.a}.
Subsequently, the algorithm calculates and filters module errors for each module in the ADS \( \mathcal{A} \), storing these errors in \( \delta^{\mathcal{A}} \) for further analysis and feedback (Lines 5-8).
Once the user-specified module does not trigger errors (Line 9-10) or other modules do trigger errors (Line 11-12), the identification flag is set to `Fail' as they violate the requirements of \textit{Definition~\ref{def-mccs}.b} and \textit{Definition~\ref{def-mccs}.c}. 
Finally, \oracle returns the identification flag \( r_{\mathcal{M}} \), the module error set \( \delta^{\mathcal{A}} \), and the safety-critical distance \( \delta^{\text{safe}} \) (Line 13).





\subsection{Module-Specific Feedback} % Feedback

To provide guidance for searching {\mccs}s, we design a \feedback providing a feedback score, including two parts: (1) \textit{safety-critical score} and (2) \textit{module-directed score}. The \textit{safety-critical score} aims to guide the search for safety-critical scenarios that include system-level violations (i.e., collisions). To focus more specifically on the user-specified module, we introduce the \textit{module-directed score}, which provides guidance to bias the generation of safety-critical scenarios towards this module.


\subsubsection{Safety-critical Score}
We directly leverage the safety-critical distance from Section~\ref{sec:safe-metric} as our safety-critical feedback score, denoted by \( \phi_{s}^{\text{safe}} = \delta^{\text{safe}}_{s} \).
The safety-critical score \(\phi_{s}^{\text{safe}}\) quantifies the minimum distance between the ego vehicle and other objects over the time horizon \([0, T]\), capturing how close the vehicle comes to a collision scenario. A lower value of \(\phi_{s}^{\text{safe}}\) indicates a more dangerous situation for the ego vehicle.


\subsubsection{Module-directed Score} 
Given a user-specified module \( \mathcal{M} \), we calculate the \textit{module-directed score} as follows:
\begin{equation}
    \phi_{s}^{\mathcal{M}} = \sum_{t=T-\Delta t}^{T}\left( \delta_t^{\mathcal{M}} - \frac{\sum_{M \neq \mathcal{M}} \delta_t^{{M}}}{K - 1} \right)
\end{equation}
where \( K \) is the total number of modules in the ADS \(\mathcal{A}\), \( T \) is the termination timestamp of scenario \( s \), \( \Delta t \) is the detection window, and \( \delta_t^{{M}} \) represents the module-level errors for each module obtained from \oracle, as detailed in Section~\ref{sec:module-metric}. 
The first term, \( \delta_t^{\mathcal{M}} \), quantifies the errors specific to the user-specified module \( \mathcal{M} \). 
The second term, \( \frac{\sum_{M \neq \mathcal{M}} \delta_t^{{M}}}{K - 1} \), represents the average of the cumulative errors across all other modules. 
The score \( \phi_{s}^{\mathcal{M}} \) promotes the search for scenarios in which only the user-specified module \( \mathcal{M} \) exhibits errors during the detection window, while other modules do not. Therefore, this score can provide guidance to enhance the impact of the user-specified module \( \mathcal{M} \) on detected violations.


The final feedback score combines the \textit{safety-critical score} and the \textit{module-directed score} by:
\begin{equation}\label{eq:feedback}
    \phi_{s} =  \phi_{s}^{\mathcal{M}^k} - \phi_{s}^{\text{safe}}
\end{equation}
A larger \(\phi_{s}\) indicates that the scenario \( s \) is closer to becoming a {\mccs}. Consequently, \tool aims to generate {\mccs}s by maximizing this feedback score.



\begin{algorithm}[!t]
\small
\SetKwInOut{Input}{Input}
\SetKwInOut{Output}{Output}
\SetKwInOut{Para}{Parameters}
\SetKwProg{Fn}{Function}{:}{}
\SetKwFunction{AE}{\textbf{AdaptiveEvolver}}
\SetKwFunction{OI}{\textbf{ModuleCauseIdentifier}}
\SetKwComment{Comment}{\color{blue}// }{}
\Input{
Selected seed $s$ with feedback score ${\phi}_{s}$ \\
Maximum feedback score ${\phi}_{max}$ and Minimum feedback score ${\phi}_{min}$ in the seed corpus
}
\Output{
Mutated seed $s'$
}
$s' \gets s$ \Comment{Copy the selected scenario seed $s$}
$\lambda_{m} \gets \frac{{\phi}_{max} - {\phi}_{s}}{|{\phi}_{max} - {\phi}_{min}|}$ \Comment{Assign a dynamic threshold for determining mutation strategy}

\eIf{random() > $\lambda_{m}$}{
    \Comment{Fine-grained mutation: add small perturbation}
    $\mathbb{E}_{s'} \gets \mathbb{E}_{s'} + \text{GaussSample}(\lambda_m)$ \\
    \For{$P \in \mathbb{P}_{s'}$}{
    $W_{P}^{v} \gets W_{P}^{v} + \text{GaussSample}(\lambda_m)$ 
    }
}{
    \Comment{Coarse-grained mutation: regenerate new parameters}
    $\mathbb{E}_{s'} \gets \text{UniformSample}([\mathbb{E}_{min}, \mathbb{E}_{max}])$ \\
    \For{$P \in \mathbb{P}_{s'}$}{
    $W_{P}^{l} \gets \text{RouteGenerate}(\lambda_m)$ \\
    $W_{P}^{v} \gets \text{UniformSample}([V_{min}, V_{max}],\lambda_m)$
    }
}
\Return $s'$
\caption{Algorithm for \textit{Adaptive Mutation}}
\label{algo:mutation}
\end{algorithm}
\subsection{Adaptive Scenario Generation} % Change to generator
To improve the searching performance of \tool, we design an \select mechanism including \textit{Adaptive Seed Selection} and \textit{Adaptive Mutation}, which adaptively generate new seed scenarios based on the feedback score obtained from \feedback.


\subsubsection{Adaptive Seed Selection} The seed selection process aims to choose a seed scenario from the seed corpus for further mutation, thereby generating a new scenario.
We design this selection process to favor seeds with higher feedback scores, indicating they are more likely to evolve into a {\mccs}.
Therefore, we assign a selection probability to each seed in the corpus based on the feedback score calculated by \feedback. The selection probability for each seed \( s \) is defined as:
\begin{equation}
    p(s) = \frac{\phi_{s} - \phi_{\min} + \epsilon}{\sum_{s' \in \mathbf{Q}} (\phi_{s'} - \phi_{\min} + \epsilon)}
\end{equation}
where \( \phi_{\min} \) is the global minimum feedback score in the corpus, \( \phi_{s} \) is the feedback score for seed \( s \), \( \epsilon \) is a small positive constant to ensure that the seed with the global minimum feedback score has a non-zero selection probability, and \( \mathbf{Q} \) denotes the set of all seeds in the corpus. This probability formulation ensures that seeds with higher feedback scores have a higher chance of being selected for mutation, thereby promoting the generation of scenarios with a higher likelihood of evolving into a {\mccs}.


\subsubsection{Adaptive Mutation} Beyond seed selection, we also design an adaptive mutation strategy that applies different mutation methods based on the feedback score of each seed. 
Algorithm~\ref{algo:mutation} presents the details of \textit{Adaptive Mutation}. 
Specifically, given a scenario \( s \), the \textit{Adaptive Mutator} first copies the source seed \( s \) to \( s' \) (Line 1) and calculates a dynamic threshold \( \lambda_m = \frac{\phi_{\text{max}} - \phi_s}{|\phi_{\text{max}} - \phi_{\text{min}}|} \in [0, 1] \) by normalizing the feedback score \( \phi_s \), ensuring \( \lambda_m \) (Line 2).
Then, the mutation selects either \textit{fine-grained mutation} or \textit{coarse-grained mutation} based on the derived dynamic threshold $\lambda_m$ (Line 3-11). Seeds with higher feedback scores (resulting in smaller dynamic thresholds) are regarded as closer to {\mccs}s; therefore, we employ \textit{fine-grained mutation} to add Gaussian noise to the weather parameters (Line 4) and the speeds of each object (Lines 5-6). Otherwise, for seeds with lower feedback scores (resulting in larger dynamic thresholds), we utilize \textit{coarse-grained mutation} to introduce more significant variations. These include changing the trajectory waypoints of each object and altering environmental conditions through uniform sampling (Lines 7-11).
Finally, a mutated scenario \( s' \) is produced (Line 12).



\section{Experiments}
\label{Sec:exp}
% In this section, we first introduce the experimental settings, including datasets, baselines and implementation details. Then we conduct a series of experiments, involving performance comparison, parameter sensitiveness analysis and ablation study, to comprehensively evaluate our proposed \name.


% \begin{table*}[ht]
\centering
\caption{Comparison of all models in terms of mean accuracy $\pm$ stdev (\%) under dense splitting. The best results appear in \textbf{bold}. The second results appear in \underline{underline}.}
\scalebox{0.80}{
\begin{tabular}{lcccccccccccc}
\toprule
Dataset& Photo & ACM & Computer  &Citeseer &WikiCS& BlogCatalog & UAI2010 & Flickr   \\
$\mathcal{H}$& 0.83 & 0.82 & 0.78  &0.74 & 0.66& 0.40 & 0.36& 0.24  \\ \hline

SGC& 93.74\tiny{$\pm$0.07} & 93.24\tiny{$\pm$0.49} &88.90\tiny{$\pm$0.11} & 76.81\tiny{$\pm$0.26} & 76.67\tiny{$\pm$0.19}& 72.61\tiny{$\pm$0.07} &69.87\tiny{$\pm$0.17} &47.48\tiny{$\pm$0.40}  \\

APPNP&{94.98\tiny{$\pm$0.41}}& 93.00\tiny{$\pm$0.55} &\underline{91.31\tiny{$\pm$0.29} } & {77.52\tiny{$\pm$0.22}}  & 81.96\tiny{$\pm$0.14}
& 94.77\tiny{$\pm$0.19} &{77.41\tiny{$\pm$0.47}} & 84.66\tiny{$\pm$0.31}  \\

GPRGNN& 94.57\tiny{$\pm$0.44} & 93.42\tiny{$\pm$0.20} &90.15\tiny{$\pm$0.34}  & 77.59\tiny{$\pm$0.36} &82.43\tiny{$\pm$0.29}& {94.36\tiny{$\pm$0.29} }&76.94\tiny{$\pm$0.64} &{85.91\tiny{$\pm$0.51}}    \\


FAGCN& 94.06\tiny{$\pm$0.03} & 93.37\tiny{$\pm$0.24} &83.17\tiny{$\pm$1.81} & 76.19\tiny{$\pm$0.62} 
& 79.89\tiny{$\pm$0.93}& 79.92\tiny{$\pm$4.39} &72.17\tiny{$\pm$1.57} & 82.03\tiny{$\pm$0.40}   \\

BM-GCN& 95.10\tiny{$\pm$0.20} &{93.68\tiny{$\pm$0.34} }&91.28\tiny{$\pm$0.96} & 77.91\tiny{$\pm$0.58} & {83.90\tiny{$\pm$0.41} }&94.85\tiny{$\pm$0.42} & 77.39\tiny{$\pm$1.13} &   83.97\tiny{$\pm$0.87} \\

ACM-GCN& 94.56\tiny{$\pm$0.21} & 93.04\tiny{$\pm$1.28} &85.19\tiny{$\pm$2.26} &77.62\tiny{$\pm$0.81} 
&\underline{83.95\tiny{$\pm$0.41}}
& 94.53\tiny{$\pm$0.53} &76.87\tiny{$\pm$1.42} & 83.85\tiny{$\pm$0.73}  \\

\hline
NAGphormer&  \underline{95.47\tiny{$\pm$0.29}} & 93.32\tiny{$\pm$0.30} &90.79\tiny{$\pm$0.45} &  77.68\tiny{$\pm$0.73}& 
83.61\tiny{$\pm$0.28} & 94.42\tiny{$\pm$0.63} &76.36\tiny{$\pm$1.12} & 86.85\tiny{$\pm$0.85}  \\

SGFormer& 92.93\tiny{$\pm$0.12} & 93.79\tiny{$\pm$0.34} &81.86\tiny{$\pm$3.82} &  77.86\tiny{$\pm$0.76}& 79.65\tiny{$\pm$0.31} & 94.33\tiny{$\pm$0.19} &57.98\tiny{$\pm$3.95} & 61.05\tiny{$\pm$0.68}  \\


Specformer& 95.22\tiny{$\pm$0.13} & 93.63\tiny{$\pm$1.94} &85.47\tiny{$\pm$1.44} & 77.96\tiny{$\pm$0.89}&  83.74\tiny{$\pm$0.62}  & 94.21\tiny{$\pm$0.23} &73.06\tiny{$\pm$0.77} & 86.55\tiny{$\pm$0.40} \\

VCR-Graphormer
&95.38\tiny{$\pm$0.51} & 93.11\tiny{$\pm$0.79} &90.47\tiny{$\pm$0.58} & 77.21\tiny{$\pm$0.65}& 80.82\tiny{$\pm$0.72} & 94.19\tiny{$\pm$0.17} &76.08\tiny{$\pm$0.52} & 85.96\tiny{$\pm$0.55}    \\

PolyFormer
& 95.45\tiny{$\pm$0.21} & \underline{94.27\tiny{$\pm$0.44}} &90.87\tiny{$\pm$0.74} & \underline{78.03\tiny{$\pm$0.86}}& 83.79\tiny{$\pm$0.75} & \underline{95.08\tiny{$\pm$0.43} }&\underline{77.92\tiny{$\pm$0.82} }& \underline{87.01\tiny{$\pm$0.57} }   \\


\hline

\name & 
\textbf{95.92\tiny{$\pm$0.18}} & \textbf{94.98\tiny{$\pm$0.41}} & 
\textbf{91.73\tiny{$\pm$0.72}} & 
\textbf{78.49\tiny{$\pm$0.95}} & 
\textbf{84.52\tiny{$\pm$0.63}}& \textbf{95.93\tiny{$\pm$0.56}} & \textbf{79.06\tiny{$\pm$0.73}} & \textbf{87.56\tiny{$\pm$0.61}}  \\     
 \toprule
\end{tabular}
}

\label{tab:dense-ncre}
\end{table*}




\subsection{Dataset}
We adopt eight widely used datasets, involving homophily and heterophily graphs: 
Photo~\cite{nagphormer}, ACM~\cite{acm}, Computer~\cite{nagphormer}, BlogCatalog~\cite{socialnets}, UAI2010~\cite{amgcn}, Flickr~\cite{socialnets} and Wiki-CS~\cite{roman}.
The edge homophily ratio~\cite{glognn} ${H}(\mathcal{G})\in[0,1]$ is adopted to evaluate the graph's homophily level. 
${H}(\mathcal{G}) \rightarrow 1$ means strong homophily, 
while ${H}(\mathcal{G}) \rightarrow 0$ means strong heterophily.
Statistics of datasets are summarized in Appendix \ref{app:data}.
To comprehensively evaluate the model performance in node classification, we provide two strategies to split datasets, called dense splitting and sparse splitting.
In dense splitting, we randomly choose 50\% of each label as the training set, 25\% as the validation set, and the rest as the test set, which is a common setting is previous studies~\cite{nodeformer,sgformer}.
While in sparse splitting~\cite{gprgnn}, we adopt 2.5\%/2.5\%/95\% splitting for training set, validation set and test set, respectively.


\subsection{Baseline}
We adopt eleven representative approaches as the baselines: SGC~\cite{sgc}, APPNP~\cite{appnp}, GPRGNN~\cite{gprgnn}, FAGCN~\cite{fagcn}, BM-GCN~\cite{bmgcn}, ACM-GCN~\cite{acmgnn}, NAGphormer~\cite{nagphormer}, SGFormer~\cite{sgformer}, Specformer~\cite{specformer}, VCR-Graphormer~\cite{vcrgt} and PolyFormer~\cite{polyformer}.
The first six are mainstream GNNs and others are representative GTs.




\begin{table*}[ht]
\centering
\caption{Comparison of all models in terms of mean accuracy $\pm$ stdev (\%) under sparse splitting. The best results appear in \textbf{bold}. The second results appear in \underline{underline}.}
\scalebox{0.8}{
\begin{tabular}{lcccccccccccc}
\toprule
Dataset& Photo & ACM & Computer  &Citeseer &WikiCS& BlogCatalog & UAI2010 & Flickr   \\
$\mathcal{H}$& 0.83 & 0.82 & 0.78  &0.74 & 0.66& 0.40 & 0.36& 0.24  \\ \hline

SGC& 91.90\tiny{$\pm$0.35} & 89.57\tiny{$\pm$0.28} &86.79\tiny{$\pm$0.19} & 66.41\tiny{$\pm$0.59} & 74.99\tiny{$\pm$0.19}& 71.23\tiny{$\pm$0.06} &51.61\tiny{$\pm$0.41} &39.43\tiny{$\pm$0.50}  \\

APPNP& \underline{92.24\tiny{$\pm$0.28}}& 89.91\tiny{$\pm$0.89} &\underline{87.64\tiny{$\pm$0.39} } & \underline{66.70\tiny{$\pm$0.11}}  & 77.42\tiny{$\pm$0.31}
& 81.76\tiny{$\pm$0.38} &\underline{61.65\tiny{$\pm$0.71}} & 71.39\tiny{$\pm$0.62}  \\

GPRGNN& 92.13\tiny{$\pm$0.32} & 89.47\tiny{$\pm$0.90} &86.38\tiny{$\pm$0.44}  & 66.50\tiny{$\pm$0.62} & 77.59\tiny{$\pm$0.49}& \underline{84.57\tiny{$\pm$0.35} }&58.75\tiny{$\pm$0.75} &\underline{71.89\tiny{$\pm$0.89}}    \\


FAGCN& 92.02\tiny{$\pm$0.18} & 88.47\tiny{$\pm$0.31} &83.99\tiny{$\pm$1.95} & 64.54\tiny{$\pm$0.66}  & 75.21\tiny{$\pm$0.84}& 76.38\tiny{$\pm$0.82} &54.67\tiny{$\pm$0.96} & 63.68\tiny{$\pm$0.72}  \\

BM-GCN& 91.19\tiny{$\pm$0.39} &\underline{90.11\tiny{$\pm$0.60} }&86.14\tiny{$\pm$0.51} & 66.11\tiny{$\pm$0.47}& {77.39\tiny{$\pm$0.37} } &84.05\tiny{$\pm$0.54} & 57.51\tiny{$\pm$1.14} &   60.82\tiny{$\pm$0.76} \\

ACM-GCN& 91.71\tiny{$\pm$0.64} & 89.68\tiny{$\pm$0.45} &86.64\tiny{$\pm$0.59} &64.85\tiny{$\pm$1.19}& \underline{77.68\tiny{$\pm$0.57}} & 77.17\tiny{$\pm$1.34} &56.05\tiny{$\pm$2.11} & 64.58\tiny{$\pm$1.53}  \\

\hline
NAGphormer& 91.65\tiny{$\pm$0.80} & 89.73\tiny{$\pm$0.48} &85.31\tiny{$\pm$0.65} &  63.66\tiny{$\pm$1.68}& 76.93\tiny{$\pm$0.75} & 79.19\tiny{$\pm$0.41} &58.36\tiny{$\pm$1.01} & 67.48\tiny{$\pm$1.04}  \\

SGFormer& 90.13\tiny{$\pm$0.56} & 88.03\tiny{$\pm$0.60} &80.07\tiny{$\pm$0.21} &  62.41\tiny{$\pm$0.94}& 74.69\tiny{$\pm$0.52} & 78.15\tiny{$\pm$0.69} &50.19\tiny{$\pm$1.72} & 51.01\tiny{$\pm$1.05}  \\


Specformer& 90.57\tiny{$\pm$0.55} & 88.20\tiny{$\pm$1.05} &85.55\tiny{$\pm$0.63} & 62.64\tiny{$\pm$1.54} &  75.24\tiny{$\pm$0.71}& 79.75\tiny{$\pm$1.29} &57.42\tiny{$\pm$1.06} & 56.94\tiny{$\pm$1.48}  \\

VCR-Graphormer
& 91.39\tiny{$\pm$0.75} & 86.81\tiny{$\pm$0.84} &85.06\tiny{$\pm$0.64} & 57.61\tiny{$\pm$0.60} & 72.81\tiny{$\pm$1.44} & 74.90\tiny{$\pm$1.18} &56.43\tiny{$\pm$1.10} & 50.93\tiny{$\pm$1.12}   \\

PolyFormer
& 91.52\tiny{$\pm$0.78} & 89.83\tiny{$\pm$0.62} &85.75\tiny{$\pm$0.78} & 64.77\tiny{$\pm$1.27} & 75.12\tiny{$\pm$1.16}& 81.02\tiny{$\pm$0.81} &58.89\tiny{$\pm$0.77} & 67.85\tiny{$\pm$1.43}    \\


\hline

\name & 
\textbf{92.93\tiny{$\pm$0.26}} & \textbf{90.92\tiny{$\pm$0.69}} & 
\textbf{88.14\tiny{$\pm$0.52}} & 
\textbf{69.91\tiny{$\pm$1.02}} & 
\textbf{78.11\tiny{$\pm$0.83}}& \textbf{88.11\tiny{$\pm$0.58}} & \textbf{63.96\tiny{$\pm$1.09}} & \textbf{72.16\tiny{$\pm$1.19}}  \\     
 \toprule
\end{tabular}
}

\label{tab:sparse-ncre}
\end{table*}

% \begin{table*}[t]
\centering
  \caption{\textbf{NC Analysis.} In this setting, VGGm-17 models are trained on ImageNet-10 dataset (ID) for 200 epochs using MSE loss and evaluated on the same ID dataset using neural collapse metrics. Reported is the top-1 accuracy (\%). $\mathbf{W}$ and $\mathbf{W_{LS}}$ denote learned weights and least square weights (analytical, no training) of the final classifier layer, respectively. \textbf{A lower $\mathcal{NC}$ indicates higher \jg{stronger?} neural collapse.}}
  \label{tab:nc_results}
  \centering
  %\resizebox{\linewidth}{!}{
     \begin{tabular}{ccc|cccc}
     \hline %\hline
     \multicolumn{1}{c}{\textbf{Configuration}} &
     \multicolumn{1}{c}{\textbf{ID Accuracy} $\uparrow$} &
     \multicolumn{1}{c|}{\textbf{ID Accuracy} $\uparrow$} &
     \multicolumn{4}{c}{\textbf{Neural Collapse} $\downarrow$} \\
    & $\mathbf{W}$ & $\mathbf{W_{LS}}$ & $\mathcal{NC}1$ &  $\mathcal{NC}2$ &  $\mathcal{NC}3$ &  $\mathcal{NC}4$ \\
    \hline
    No Projector & 89.60 & 89.40 & 0.075 & 0.219 & 0.030 & 0.316 \\
    \hline
    Plastic Projector (1 layer) & 89.20 & 89.40 & 0.074 & 0.304 & 0.043 & 0.316 \\
    ETF Fixed Projector (1 layer) & 89.00 & 88.80 & \textbf{0.069} & \textbf{0.254} & \textbf{0.035} & 0.316 \\
    \hline
    Plastic Projector (2 layers) & 89.20 & 89.00 & 0.101 & 0.378 & 0.051 & 0.316 \\
    ETF Fixed Projector (2 layers) & 89.40 & 89.40 & \textbf{0.080} & \textbf{0.311} & \textbf{0.041} & 0.316 \\
    ETF Fixed Proj (2 layers) + KoLeo & \textbf{90.20} & \textbf{90.40} & 0.085 & \textbf{0.094} & \textbf{0.041} & \textbf{0.282} \\
    \hline %\hline
    %\vspace{-2em}
    \end{tabular} %}
\end{table*}


\subsection{Performance Comparison}
To evaluate the model performance in node classification, we run each model ten times with random initializations. The results in terms of mean accuracy and standard deviation are reported in \autoref{tab:dense-ncre} and \autoref{tab:sparse-ncre}.

First, we can observe that \name achieves the best performance on all datasets with different data splitting strategies, demonstrating the effectiveness of \name in node classification.
Then, we can find that advanced GTs obtain more competitive performance than GNNs on over half datasets under dense splitting.
But under sparse splitting, the situation reversed.
An intuitive explanation is that Transformer has more learnable parameters than GNNs, which bring more powerful modeling capacity.
However, it also requires more training data than GNNs in the training stage to ensure the performance.
Therefore, when the training data is sufficient, GTs can achieve promising performance.
And when the training data is sparse, GTs usually leg behind GNNs.
Our proposed \name addresses this issue by introducing the token swapping operation to generate diverse token sequences. 
This operation effectively augments the training data, ensuring the model training even in the sparse data scenario.
In addition, the tailored center alignment loss also constrains the model parameter learning, further enhancing the model performance.


\begin{figure}[t]
\centering
\includegraphics[width=7.5cm]{Fig/alignloss-p.pdf}
\caption{
Performances of \name with or without the center alignment loss.
}
\label{fig:align}
\end{figure}

\begin{figure}[t]
\centering
\includegraphics[width=7.5cm]{Fig/ts-p.pdf}
\caption{
Performances of \name with different token sequence generation strategies.}
\label{fig:ts}
\end{figure}

\begin{figure}[t]
\centering
\includegraphics[width=7.3cm]{Fig/t-p.pdf}
\caption{
Analysis on the swapping times $t$.}
\label{fig:t}
\end{figure}

\begin{figure}[t]
\centering
\includegraphics[width=7.3cm]{Fig/s-p.pdf}
\caption{
Analysis on the augmentation times $s$.}
\label{fig:s}
\end{figure}



\subsection{Study on the center alignment loss}
The center alignment loss, proposed to constrain the representation learning from multiple token sequences, is a key design of \name.
Here, we validate the effectiveness of the center alignment loss in node classification.
Specifically, we develop a variant of \name by removing the center alignment loss, called \name-O.
Then, we evaluate the performance of \name-O on all datasets under dense splitting and sparse splitting.
Due to the space limitation, we only report the results on four datasets in \autoref{fig:align}, other results are reported in Appendix \ref{app:exp-ca}.
"Den." and "Spa." denotes the experimental results under dense splitting and sparse splitting, respectively.
Based on the experimental results, we can have the following observations:
1) \name beats \name-O on all datasets, indicating that the developed center alignment loss can effectively enhance the performance of \name.
2) Adopting the center alignment loss can bring more significant improvements in sparse setting than those in dense setting.
This situation implies that introducing the reasonable constraint loss function based on the property of node token sequences can effectively improve the model training when the training data is sparse.






\subsection{Study on the token sequence generation}\label{exp:ts}
The generation of token sequences is another key module of \name, which develops a novel token swapping operation can fully leverage the semantic relevance of nodes to generate informative token sequences.
In this section, we evaluate the effectiveness of the proposed strategy by comparing it with two naive strategies.
One is to enlarge the sampling size $k$. We propose a variant called \name-L by sampling $2k$ tokens to construct token sequences.
The other is to randomly sample $k$ tokens from the enlarged $2k$ token set to construct multiple token sequences, called \name-R.
Performance of these variants are shown in \autoref{fig:ts} and results on other datasets are reported in Appendix \ref{app:exp-ts}.

We can observe that \name-R outperforms \name-L on most cases, indicating that constructing multiple token sequences is better for node representation learning of tokenized GTs than generating single long token sequence. 
Moreover, \name surpasses \name-R on all cases, showcasing the superiority of the proposed token swapping operation in generation of multiple token sequences.
This observation also implies that constructing informative token sequences can effectively improve the performance of tokenized GTs.



\subsection{Analysis on the swapping times $t$}
As discussed in Section \ref{sec:swapping}, $t$ determines the range of candidate tokens from the constructed $k$-NN graph, further affecting the model performance.
To validate the influence of $t$ on model performance, we vary $t$ in $\{1,2,3,4\}$ and observe the changes of model performance.
Results are shown in \autoref{fig:t} and Appendix \ref{app:exp-t}.
We can clearly observe that \name can achieve satisfied performance on all datasets when $t$ is no less than 2.
This situation indicates that learning from tokens with semantic associations beyond the immediate neighbors can effectively enhancing the model performance.
This phenomenon also reveals that reasonably enlarging  the sampling space to seek more informative tokens is a promising way to improve the effect of node tokenized GTs.
%分析的

\subsection{Analysis on the augmentation times $s$}
The augmentation times $s$ determines how many token sequences are adopted for node representation learning. Similar to $t$, we vary $s$ in $\{1,2,\dots,8\}$ and report the performance of \name. 
Results are shown in \autoref{fig:s} and Appendix \ref{app:exp-s}.
Generally speaking, sparse splitting requires a larger $s$ to achieve the best performance, compared to dense splitting.
This is because \name needs more token sequences for model training in the sparse data scenario.
This situation indicates that a tailored data augmentation strategy can effectively improve the performance of tokenized GTs when training data is sparse.
Moreover, the optimal $s$ varies on different graphs. 
This is because different graphs exhibit different topology features and attribute features, which affects the generation of token sequences, further influencing the model performance.




\section{Conclusion}
\label{Sec:con}
In this paper, we introduced a novel tokenized Graph Transformer \name for node classification.
In \name, we developed a novel token swapping operation that flexibly swaps tokens in different token sequences, thereby generating diverse token sequences. This enhances the model’s ability to capture rich node representations.  
Furthermore, \name employs a tailored Transformer-based backbone with a center alignment loss to learn node representations from the generated  multiple token sequences. The center alignment loss helps guide the learning process when nodes are associated with multiple token sequences, ensuring that the learned representations are consistent and informative. 
Experimental results demonstrate that \name significantly improves node classification performance, outperforming several representative GT and GNN models. 




% \section*{Accessibility}
% Authors are kindly asked to make their submissions as accessible as possible for everyone including people with disabilities and sensory or neurological differences.
% Tips of how to achieve this and what to pay attention to will be provided on the conference website \url{http://icml.cc/}.

% \section*{Software and Data}

% If a paper is accepted, we strongly encourage the publication of software and data with the
% camera-ready version of the paper whenever appropriate. This can be
% done by including a URL in the camera-ready copy. However, \textbf{do not}
% include URLs that reveal your institution or identity in your
% submission for review. Instead, provide an anonymous URL or upload
% the material as ``Supplementary Material'' into the OpenReview reviewing
% system. Note that reviewers are not required to look at this material
% when writing their review.

% Acknowledgements should only appear in the accepted version.
% \section*{Acknowledgements}

% \textbf{Do not} include acknowledgements in the initial version of
% the paper submitted for blind review.


\section*{Impact Statement}
This paper presents work whose goal is to advance the field of graph representation learning. There is none potential societal consequence of our work that must be specifically highlighted here.

% Authors are \textbf{required} to include a statement of the potential 
% broader impact of their work, including its ethical aspects and future 
% societal consequences. This statement should be in an unnumbered 
% section at the end of the paper (co-located with Acknowledgements -- 
% the two may appear in either order, but both must be before References), 
% and does not count toward the paper page limit. In many cases, where 
% the ethical impacts and expected societal implications are those that 
% are well established when advancing the field of Machine Learning, 
% substantial discussion is not required, and a simple statement such 
% as the following will suffice:

% ``This paper presents work whose goal is to advance the field of 
% Machine Learning. There are many potential societal consequences 
% of our work, none which we feel must be specifically highlighted here.''

% The above statement can be used verbatim in such cases, but we 
% encourage authors to think about whether there is content which does 
% warrant further discussion, as this statement will be apparent if the 
% paper is later flagged for ethics review.


% In the unusual situation where you want a paper to appear in the
% references without citing it in the main text, use \nocite

\bibliography{reference}
\bibliographystyle{icml2025}


%%%%%%%%%%%%%%%%%%%%%%%%%%%%%%%%%%%%%%%%%%%%%%%%%%%%%%%%%%%%%%%%%%%%%%%%%%%%%%%
%%%%%%%%%%%%%%%%%%%%%%%%%%%%%%%%%%%%%%%%%%%%%%%%%%%%%%%%%%%%%%%%%%%%%%%%%%%%%%%
% APPENDIX
%%%%%%%%%%%%%%%%%%%%%%%%%%%%%%%%%%%%%%%%%%%%%%%%%%%%%%%%%%%%%%%%%%%%%%%%%%%%%%%
%%%%%%%%%%%%%%%%%%%%%%%%%%%%%%%%%%%%%%%%%%%%%%%%%%%%%%%%%%%%%%%%%%%%%%%%%%%%%%%
\newpage
\appendix
\onecolumn

\section{Experimental Settings}

\subsection{Dataset}\label{app:data}
Here we introduce datasets adopted for experiments. The detailed statistics of all datasets are reported in \autoref{tab:dataset}.
\begin{itemize}
    \item \textbf{Academic graphs}: This type of graph is formed by academic papers or authors and the citation relationships among them. Nodes in the graph represent academic papers or authors, and edges represent the citation relationships between papers or co-author relationships between two authors. The features of nodes are composed of bag-of-words vectors, which are extracted and generated from the abstracts and introductions of the academic papers. The labels of nodes correspond to the research fields of the academic papers or authors. ACM, Citeseer, WikiCS and UAI2010 belong to this type.
    \item \textbf{Co-purchase graphs}: This type of graph is constructed based on users' shopping behaviors. Nodes in the graph represent products. The edges between nodes indicate that two products are often purchased together. The features of nodes are composed of bag-of-words vectors extracted from product reviews. The category of a node corresponds to the type of goods the product belongs to. Computer and Photo belong to this type.

    \item \textbf{Social graphs}: This type of graph is formed by the activity records of users on social platforms. Nodes in the graph represent users on the social platform. The edges between nodes indicate the social relations between two users. Node features represent the text information extracted from the authors' homepage. The label of a node refers to the interest groups of users. BlogCatalog and Flickr belong to this type.
    
\end{itemize}

\section{Dataset}
\label{sec:dataset}

\subsection{Data Collection}

To analyze political discussions on Discord, we followed the methodology in \cite{singh2024Cross-Platform}, collecting messages from politically-oriented public servers in compliance with Discord's platform policies.

Using Discord's Discovery feature, we employed a web scraper to extract server invitation links, names, and descriptions, focusing on public servers accessible without participation. Invitation links were used to access data via the Discord API. To ensure relevance, we filtered servers using keywords related to the 2024 U.S. elections (e.g., Trump, Kamala, MAGA), as outlined in \cite{balasubramanian2024publicdatasettrackingsocial}. This resulted in 302 server links, further narrowed to 81 English-speaking, politics-focused servers based on their names and descriptions.

Public messages were retrieved from these servers using the Discord API, collecting metadata such as \textit{content}, \textit{user ID}, \textit{username}, \textit{timestamp}, \textit{bot flag}, \textit{mentions}, and \textit{interactions}. Through this process, we gathered \textbf{33,373,229 messages} from \textbf{82,109 users} across \textbf{81 servers}, including \textbf{1,912,750 messages} from \textbf{633 bots}. Data collection occurred between November 13th and 15th, covering messages sent from January 1st to November 12th, just after the 2024 U.S. election.

\subsection{Characterizing the Political Spectrum}
\label{sec:timeline}

A key aspect of our research is distinguishing between Republican- and Democratic-aligned Discord servers. To categorize their political alignment, we relied on server names and self-descriptions, which often include rules, community guidelines, and references to key ideologies or figures. Each server's name and description were manually reviewed based on predefined, objective criteria, focusing on explicit political themes or mentions of prominent figures. This process allowed us to classify servers into three categories, ensuring a systematic and unbiased alignment determination.

\begin{itemize}
    \item \textbf{Republican-aligned}: Servers referencing Republican and right-wing and ideologies, movements, or figures (e.g., MAGA, Conservative, Traditional, Trump).  
    \item \textbf{Democratic-aligned}: Servers mentioning Democratic and left-wing ideologies, movements, or figures (e.g., Progressive, Liberal, Socialist, Biden, Kamala).  
    \item \textbf{Unaligned}: Servers with no defined spectrum and ideologies or opened to general political debate from all orientations.
\end{itemize}

To ensure the reliability and consistency of our classification, three independent reviewers assessed the classification following the specified set of criteria. The inter-rater agreement of their classifications was evaluated using Fleiss' Kappa \cite{fleiss1971measuring}, with a resulting Kappa value of \( 0.8191 \), indicating an almost perfect agreement among the reviewers. Disagreements were resolved by adopting the majority classification, as there were no instances where a server received different classifications from all three reviewers. This process guaranteed the consistency and accuracy of the final categorization.

Through this process, we identified \textbf{7 Republican-aligned servers}, \textbf{9 Democratic-aligned servers}, and \textbf{65 unaligned servers}.

Table \ref{tab:statistics} shows the statistics of the collected data. Notably, while Democratic- and Republican-aligned servers had a comparable number of user messages, users in the latter servers were significantly more active, posting more than double the number of messages per user compared to their Democratic counterparts. 
This suggests that, in our sample, Democratic-aligned servers attract more users, but these users were less engaged in text-based discussions. Additionally, around 10\% of the messages across all server categories were posted by bots. 

\subsection{Temporal Data} 

Throughout this paper, we refer to the election candidates using the names adopted by their respective campaigns: \textit{Kamala}, \textit{Biden}, and \textit{Trump}. To examine how the content of text messages evolves based on the political alignment of servers, we divided the 2024 election year into three periods: \textbf{Biden vs Trump} (January 1 to July 21), \textbf{Kamala vs Trump} (July 21 to September 20), and the \textbf{Voting Period} (after September 20). These periods reflect key phases of the election: the early campaign dominated by Biden and Trump, the shift in dynamics with Kamala Harris replacing Joe Biden as the Democratic candidate, and the final voting stage focused on electoral outcomes and their implications. This segmentation enables an analysis of how discourse responds to pivotal electoral moments.

Figure \ref{fig:line-plot} illustrates the distribution of messages over time, highlighting trends in total messages volume and mentions of each candidate. Prior to Biden's withdrawal on July 21, mentions of Biden and Trump were relatively balanced. However, following Kamala's entry into the race, mentions of Trump surged significantly, a trend further amplified by an assassination attempt on him, solidifying his dominance in the discourse. The only instance where Trump’s mentions were exceeded occurred during the first debate, as concerns about Biden’s age and cognitive abilities temporarily shifted the focus. In the final stages of the election, mentions of all three candidates rose, with Trump’s mentions peaking as he emerged as the victor.
\subsection{Implementation Details}\label{app:imple}
For baselines, we refer to their official implementations and conduct a systematic tuning process on each dataset.
For \name, we employ a grid search strategy to identify the optimal parameter settings.
Specifically, We try the learning rate in $\{0.001, 0.005, 0.01\}$, dropout in $\{0.3, 0.5, 0.7\}$, dimension of hidden representations in $\{256, 512\}$,
$k$ in $\{4, 6, 8\}$, $\alpha$ in $\{0.1, \dots, 0.9\}$.
All experiments are implemented using Python 3.8, PyTorch 1.11, and CUDA 11.0 and executed on a Linux server with an Intel Xeon Silver 4210 processor, 256 GB of RAM, and a 2080TI GPU.


\section{Additional Experimental Results}\label{app:exp-results}
In this section, we provide the additional experimental results of ablation studies and parameter studies.




\subsection{Study of the center alignment loss}\label{app:exp-ca}
The experimental results of \name and \name-O on the rest datasets are shown in \autoref{fig:align-APP}.
We can observe that \name outperforms \name-O on most datasets.
Moreover, the effect of applying the center alignment loss on \name in sparse splitting is more significant than that in dense splitting.
The above observations are in line with those reported in the main text.
Therefore, we can conclude that the center alignment loss can effectively enhance the performance of \name in node classification.


\begin{figure}[ht]
\centering
\includegraphics[width=17cm]{Fig/alignloss-APP.pdf}
\caption{
Performances of \name with or without the center alignment loss.
}
\label{fig:align-APP}
\end{figure}




\subsection{Study of the token sequence generation}\label{app:exp-ts}
The experimental results of \name with different token sequence generation strategies on the rest datasets are shown in \autoref{fig:ts-APP}.
We can find that the additional experimental results exhibit similar observations shown in the main text.
This situation demonstrates the effectiveness of the token sequence generation with the proposed token swapping operation in enhancing the performance of tokenized GTs.
Moreover, we can also observe that the gains of introducing the token swapping operation vary on different graphs based on the results shown in \autoref{fig:ts} and \autoref{fig:ts-APP}.
This phenomenon may attribute to that different graphs possess unique topology and attribute information, which further impact the selection of node tokens. 
While \name applies the uniform strategy for selecting node tokens, which could lead to varying gains of token swapping.
This situation also motivates us to consider different strategies of token selection on different graphs as the future work.


\begin{figure}[ht]
\centering
\includegraphics[width=17cm]{Fig/ts-p-APP.pdf}
\caption{
Performances of \name with different token sequence generation strategies.
}
\label{fig:ts-APP}
\end{figure}



\subsection{Analysis of the swapping times $t$}\label{app:exp-t}
Here we report the rest results of \name with varying $t$, which are shown in \autoref{fig:t-APP}.
Similar to the phenomenons shown in \autoref{fig:t}, \name can achieve the best performance on all datasets when $t>2$.
Based on the results shown in \autoref{fig:t-APP} and \autoref{fig:t}, we can conclude that introducing tokens beyond first-order neighbors via the proposed token swapping operation can effective improve the performance of \name in node classification.


\begin{figure}[ht]
\centering
\includegraphics[width=17cm]{Fig/t-APP.pdf}
\caption{
Performances of \name with varying $t$.
}
\label{fig:t-APP}
\end{figure}



\subsection{Analysis of the augmentation times $s$}\label{app:exp-s}
Similar to analysis of $t$, the rest results of \name with varying $s$ are shown in \autoref{fig:s-APP}.
We can also observe the similar situations shown in \autoref{fig:s} that \name requires a larger value of $s$ under sparse splitting compared to dense splitting.
The situation demonstrates that introducing augmented token sequences can bring more significant performance gain in sparse splitting than that in dense splitting. 




\begin{figure}[t]
\centering
\includegraphics[width=17cm]{Fig/s-APP.pdf}
\caption{
Performances of \name with varying $s$.
}
\label{fig:s-APP}
\end{figure}

% \section{You \emph{can} have an appendix here.}

% You can have as much text here as you want. The main body must be at most $8$ pages long.
% For the final version, one more page can be added.
% If you want, you can use an appendix like this one.  

% The $\mathtt{\backslash onecolumn}$ command above can be kept in place if you prefer a one-column appendix, or can be removed if you prefer a two-column appendix.  Apart from this possible change, the style (font size, spacing, margins, page numbering, etc.) should be kept the same as the main body.
%%%%%%%%%%%%%%%%%%%%%%%%%%%%%%%%%%%%%%%%%%%%%%%%%%%%%%%%%%%%%%%%%%%%%%%%%%%%%%%
%%%%%%%%%%%%%%%%%%%%%%%%%%%%%%%%%%%%%%%%%%%%%%%%%%%%%%%%%%%%%%%%%%%%%%%%%%%%%%%


\end{document}


% This document was modified from the file originally made available by
% Pat Langley and Andrea Danyluk for ICML-2K. This version was created
% by Iain Murray in 2018, and modified by Alexandre Bouchard in
% 2019 and 2021 and by Csaba Szepesvari, Gang Niu and Sivan Sabato in 2022.
% Modified again in 2023 and 2024 by Sivan Sabato and Jonathan Scarlett.
% Previous contributors include Dan Roy, Lise Getoor and Tobias
% Scheffer, which was slightly modified from the 2010 version by
% Thorsten Joachims & Johannes Fuernkranz, slightly modified from the
% 2009 version by Kiri Wagstaff and Sam Roweis's 2008 version, which is
% slightly modified from Prasad Tadepalli's 2007 version which is a
% lightly changed version of the previous year's version by Andrew
% Moore, which was in turn edited from those of Kristian Kersting and
% Codrina Lauth. Alex Smola contributed to the algorithmic style files.
