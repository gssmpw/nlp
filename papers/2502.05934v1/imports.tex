\usepackage{enumitem}
\usepackage{hyperref}
\usepackage[normalem]{ulem}
\usepackage{mathrsfs}
% \usepackage{epigraph}
% % Redefine epigraph to center the text
% \setlength{\epigraphwidth}{0.8\textwidth} % Adjust width of epigraph
% \renewcommand{\epigraphflush}{center} % Center the epigraph
\usepackage{epigraph}
\setlength{\epigraphwidth}{0.8\textwidth} % Adjust width for better readability
\renewcommand{\epigraphflush}{center} % Center the epigraph
%\renewcommand{\epigraphrule}{0pt} % Remove default horizontal rule

% custom commands
\newcommand{\tuple}[1]{\ensuremath{\left \langle #1 \right \rangle }}
\newcommand{\eps}{\varepsilon}
\newcommand{\powsetopname}{\mathscr{P}}
\newcommand{\powset}[1]{\powsetopname(#1)}
\newcommand{\prob}{\mathbb{P}}
\newcommand{\A}{\mathbf{A}}
\newcommand{\Rob}{\mathbf{R}}
\newcommand{\R}{\mathbb{R}}
\newcommand{\Nat}{\mathbb{N}}
\newcommand{\N}{[N]}
\newcommand{\M}{[M]}
\newcommand{\CP}{\mathbb{C}\mathbb{P}}
\newcommand{\I}{\mathcal{I}}
\newcommand{\agree}{\ensuremath{\tuple{M, N, \eps, \delta}}}
\newcommand{\E}[1]{\mathbb{E}\left[#1\right]}
\newcommand{\Ej}[1]{\mathbb{E}_{{\CP}_j}\left[#1\right]}
\newcommand{\Eja}[1]{\mathbb{E}_{{\prob}^{\A}_j}\left[#1\right]}
\newcommand{\Ejr}[1]{\mathbb{E}_{{\prob}^{\Rob}_j}\left[#1\right]}
\newcommand{\Eji}[1]{\mathbb{E}_{{\prob}^{i}_j}\left[#1\right]}
\DeclareMathOperator*{\Prob}{\mathbb{P}}

\theoremstyle{definition}
  \newtheorem{definition}{Definition}
  \newtheorem{notation}{Notation}
  \newtheorem{requirement}{Requirement}

\theoremstyle{definition}
  \newtheorem{example}{Example}
  \newtheorem{remark}{Remark}
  \newtheorem{remarks}{Remarks}
  \newtheorem{exercise}{Exercise}
