% This must be in the first 5 lines to tell arXiv to use pdfLaTeX, which is strongly recommended.
\pdfoutput=1
% In particular, the hyperref package requires pdfLaTeX in order to break URLs across lines.

\documentclass[11pt]{article}

% Change "review" to "final" to generate the final (sometimes called camera-ready) version.
% Change to "preprint" to generate a non-anonymous version with page numbers.
\usepackage{acl}
\usepackage{lipsum}
% Standard package includes
\usepackage{wrapfig}
\usepackage{times}
\usepackage{latexsym}
\usepackage{enumitem}
\usepackage{enumerate}
\usepackage{graphicx}
\usepackage{subcaption}
% For proper rendering and hyphenation of words containing Latin characters (including in bib files)
\usepackage[T1]{fontenc}
% For Vietnamese characters
% \usepackage[T5]{fontenc}
% See https://www.latex-project.org/help/documentation/encguide.pdf for other character sets

% This assumes your files are encoded as UTF8
\usepackage[utf8]{inputenc}

% This is not strictly necessary, and may be commented out,
% but it will improve the layout of the manuscript,
% and will typically save some space.
\usepackage{microtype}

% This is also not strictly necessary, and may be commented out.
% However, it will improve the aesthetics of text in
% the typewriter font.
\usepackage{inconsolata}

%Including images in your LaTeX document requires adding
%additional package(s)
\usepackage{graphicx}
\usepackage{booktabs}
\usepackage{longtable}
\usepackage{float}
\usepackage[utf8]{inputenc}
\usepackage{pifont} % for shape symbols

% Define short commands for each shape
% \newcommand{\affcircle}{\ding{108}}  % ●
% \newcommand{\affsquare}{\ding{110}}  % ■
% \newcommand{\affdiamond}{\ding{117}} % ◆
% \newcommand{\affstar}{

% If the title and author information does not fit in the area allocated, uncomment the following
%
%\setlength\titlebox{<dim>}
%
% and set <dim> to something 5cm or larger.


% \newcommand{\vt}[1]{{\color{blue} [VT: #1]}}
% \newcommand{\fb}[1]{{\color{magenta} [FB: #1]}}
% \newcommand{\lh}[1]{{\color{cyan} [LH: #1]}}

%\newcommand{\tbd}[1]{{\color{red} [#1]}}

% \newcommand{\myparagraph}[1]{\textbf{#1.}}
\newcommand{\myparagraph}[1]
{\paragraph{#1.}}
%{\textbf{#1.}}


\title{Do Sparse Autoencoders Generalize? A Case Study of Answerability}

% Author information can be set in various styles:
% For several authors from the same institution:
% \author{Author 1 \and ... \and Author n \\
%         Address line \\ ... \\ Address line}
% if the names do not fit well on one line use
%         Author 1 \\ {\bf Author 2} \\ ... \\ {\bf Author n} \\
% For authors from different institutions:
% \author{Author 1 \\ Address line \\  ... \\ Address line
%         \And  ... \And
%         Author n \\ Address line \\ ... \\ Address line}
% To start a separate ``row'' of authors use \AND, as in
% \author{Author 1 \\ Address line \\  ... \\ Address line
%         \AND
%         Author 2 \\ Address line \\ ... \\ Address line \And
%         Author 3 \\ Address line \\ ... \\ Address line}

\author{%
  \textbf{Lovis Heindrich}\textsuperscript{}\thanks{Work done during a research visit at University of Oxford.} \quad
  \textbf{Philip Torr}\textsuperscript{\ddag} \quad
  \textbf{Fazl Barez}\textsuperscript{\ddag,\S}\thanks{Equal advising\\
  Corresponding author: \url{fazl@robots.ox.ac.uk}} \quad
  \textbf{Veronika Thost}\textsuperscript{\P}\footnotemark[2] \\
  \\[-0.8em]  % small vertical gap
  %\textsuperscript{\dag}Independent \\
  \textsuperscript{\ddag}University of Oxford 
  \textsuperscript{\S}WhiteBox
  \textsuperscript{\P}MIT-IBM Watson AI Lab
}
%\author{
%  \textbf{First Author\textsuperscript{1}},
%  \textbf{Second Author\textsuperscript{1,2}},
%  \textbf{Third T. Author\textsuperscript{1}},
%  \textbf{Fourth Author\textsuperscript{1}},
%\\
%  \textbf{Fifth Author\textsuperscript{1,2}},
%  \textbf{Sixth Author\textsuperscript{1}},
%  \textbf{Seventh Author\textsuperscript{1}},
%  \textbf{Eighth Author \textsuperscript{1,2,3,4}},
%\\
%  \textbf{Ninth Author\textsuperscript{1}},
%  \textbf{Tenth Author\textsuperscript{1}},
%  \textbf{Eleventh E. Author\textsuperscript{1,2,3,4,5}},
%  \textbf{Twelfth Author\textsuperscript{1}},
%\\
%  \textbf{Thirteenth Author\textsuperscript{3}},
%  \textbf{Fourteenth F. Author\textsuperscript{2,4}},
%  \textbf{Fifteenth Author\textsuperscript{1}},
%  \textbf{Sixteenth Author\textsuperscript{1}},
%\\
%  \textbf{Seventeenth S. Author\textsuperscript{4,5}},
%  \textbf{Eighteenth Author\textsuperscript{3,4}},
%  \textbf{Nineteenth N. Author\textsuperscript{2,5}},
%  \textbf{Twentieth Author\textsuperscript{1}}
%\\
%\\
%  \textsuperscript{1}Affiliation 1,
%  \textsuperscript{2}Affiliation 2,
%  \textsuperscript{3}Affiliation 3,
%  \textsuperscript{4}Affiliation 4,
%  \textsuperscript{5}Affiliation 5
%\\
%  \small{
%    \textbf{Correspondence:} \href{mailto:email@domain}{email@domain}
%  }
%}



\begin{document}
\maketitle

%\vt{test}
% \fb{Use this link to work on this paper: \url{https://www.overleaf.com/7937183635pqsthzqbhhmg#4b4aaa}}
%\lh{test}

\begin{abstract}
Sparse autoencoders (SAEs) have emerged as a promising approach in language model interpretability, offering unsupervised extraction of sparse features. For interpretability methods to succeed, they must identify abstract features across domains, and these features can often manifest differently in each context. We examine this through "answerability"—a model's ability to recognize answerable questions. We extensively evaluate SAE feature generalization across diverse answerability datasets for Gemma 2 SAEs. Our analysis reveals that residual stream probes outperform SAE features within domains, but generalization performance differs sharply. SAE features demonstrate inconsistent transfer ability, and residual stream probes similarly show high variance out of distribution. Overall, this demonstrates the need for quantitative methods to predict feature generalization in SAE-based interpretability.

% % X: What and why
% Sparse autoencoders (SAEs) have emerged as
% a promising %dominant 
% approach in language model interpretability, offering unsupervised 
% extraction of sparse features. %While SAEs effectively capture syntax patterns, their ability to capture abstract linguistic phenomena remains untested.
% % Y: Why hard
% For interpretability methods to succeed, they
% must identify abstract features across domains, and these features can often manifest differently in each context.
% % Z: Our solution
% We examine this through "answerability"—a
% model's ability to recognize answerable questions. We extensively evaluate %analyze SAEs trained on Gemma 2, evaluating 
% SAE feature generalization
% across diverse answerability datasets % tasks
% for Gemma 2 SAEs.
% % 1a: Results
% Our analysis reveals that residual stream probes
% outperform SAE features within domains, but
% generalization performance differs sharply.
% SAE features demonstrate inconsistent transfer
% ability, and residual stream probes similarly show high
% variance out of distribution.
% %in out-of-distribution performance.
% % 1b: Implications
% % These results expose specific limitations in
% % SAE approaches for abstract concepts. While
% % certain features transfer successfully, we
% % cannot yet predict which will generalize.
% Overall, this demonstrates the need for quantitative
% methods to predict feature generalization
% in SAE-based interpretability.
% % X: What are we trying to do and why is it relevant?
% Sparse autoencoders (SAEs) have emerged as a dominant approach in language model interpretability, promising unsupervised feature extraction. While effective for concrete concepts like syntax, their ability to capture abstract linguistic phenomena remains untested.
% % Y: Why is this hard?
% For interpretability methods to be useful, they must reliably identify abstract features across different domains---a challenge as these features often manifest differently across contexts.
% % Z: How we solve it (our contribution)
% We examine this through ``answerability''---a model's ability to recognize answerable questions. This capability provides an ideal test case as it is present across diverse tasks and requires sophisticated reasoning. We analyze SAEs trained on different layers of Gemma [add models name here], evaluating feature generalization systematically.
% %1a: Experiments and results 
% Our analysis shows while residual stream probes outperform SAE features within domains, generalization shows mixed results. SAE features vary significantly in their transfer ability, and residual stream probes, despite strong in-domain performance, exhibit high variance in generalization---with both methods showing advantages in different out-of-distribution scenarios.
% % 1b: Implications 
% These findings suggest both promise and limitations in current SAE approaches for extracting abstract concepts. While some SAE features show impressive generalization, identifying which features will transfer well remains an open challenge. Our results highlight the need for better methods to predict and ensure feature generalization in SAE-based interpretability. 

% % X: What are we trying to do and why is it relevant?
% Sparse autoencoders (SAEs) have emerged as a dominant approach in language model interpretability, promising to extract meaningful features without supervision. While recent work demonstrates their effectiveness for concrete concepts like syntax and sentiment, their ability to capture abstract linguistic phenomena remains untested.
% % Y: Why is this hard?
% The key challenge lies in generalization: for interpretability methods to be truly useful, they must reliably identify abstract features across different domains and tasks. This is particularly difficult as abstract concepts often manifest differently across contexts, making it unclear whether current SAE methods can capture such high-level representations.
% % Z: How we solve it (our contribution)
% We study this question through the lens of ``answerability''---a model's ability to recognize whether it can answer a question. This abstract capability provides an ideal test case: it manifests across diverse tasks, requires sophisticated reasoning, and is fundamental to model behavior. We  analyze SAEs trained on different layers of the Gemma language model, developed methods to evaluate feature generalization.
% % 1a: Experiments and results
% Our experiments reveal clear limitations: while SAE features successfully detect answerability within specific domains, they fail to generalize across different datasets. Notably, traditional probing methods consistently outperform our SAE features on standard benchmarks like SQuAD and BoolQ, even when we combine multiple features or target deeper model layers where abstract reasoning should be easily found.
% % 1b: implications
% These findings points to a fundamental constraint: current SAE approaches appear inadequate for extracting abstract, generalizable concepts from language models. The superior performance of simpler probing methods suggests these abstract features exist within the model but remain inaccessible to current SAE techniques, challenging their broader applicability for model interpretability.


% Sparse autoencoders (SAEs) have the potential to improve interpretability of large language models (LLMs) by learning generalizable features in an unsupervised manner. And recent studies show promising initial results. In this paper, we challenge the SAE approach by focusing on a more abstract feature than those typically studied: Answerability, a concept which is important across diverse tasks and hence suits our focus of generalization. While it is hypothesized that LLMs have an internal representation of answerability, \vt{we do not find robust SAE features dedicated to this concept. Overall, traditional probing still outperforms SAEs in a variety of use cases.}
%focus on context-based QA
\end{abstract}

% prompt examples
% probe on reconstruction
% probe vs top features extra combined plot
% how to extract/find diff features
% pre vs post relu ones. afterwards less are active but training makes pre also kind of realistic

%smaller vs wider first gets more general SAEs
% could still be later one feature

\section{Introduction}

Tutoring has long been recognized as one of the most effective methods for enhancing human learning outcomes and addressing educational disparities~\citep{hill2005effects}. 
By providing personalized guidance to students, intelligent tutoring systems (ITS) have proven to be nearly as effective as human tutors in fostering deep understanding and skill acquisition, with research showing comparable learning gains~\citep{vanlehn2011relative,rus2013recent}.
More recently, the advancement of large language models (LLMs) has offered unprecedented opportunities to replicate these benefits in tutoring agents~\citep{dan2023educhat,jin2024teach,chen2024empowering}, unlocking the enormous potential to solve knowledge-intensive tasks such as answering complex questions or clarifying concepts.


\begin{figure}[t!]
\centering
\includegraphics[width=1.0\linewidth]{Figs/Fig.intro.pdf}
\caption{An illustration of coding tutoring, where a tutor aims to proactively guide students toward completing a target coding task while adapting to students' varying levels of background knowledge. \vspace{-5pt}}
\label{fig:example}
\end{figure}

\begin{figure}[t!]
\centering
\includegraphics[width=1.0\linewidth]{Figs/Fig.scaling.pdf}
\caption{\textsc{Traver} with the trained verifier shows inference-time scaling for coding tutoring (detailed in \S\ref{sec:scaling_analysis}). \textbf{Left}: Performance vs. sampled candidate utterances per turn. \textbf{Right}: Performance vs. total tokens consumed per tutoring session. \vspace{-15pt}}
\label{fig:scale}
\end{figure}


Previous research has extensively explored tutoring in educational fields, including language learning~\cite{swartz2012intelligent,stasaski-etal-2020-cima}, math reasoning~\cite{demszky-hill-2023-ncte,macina-etal-2023-mathdial}, and scientific concept education~\cite{yuan-etal-2024-boosting,yang2024leveraging}. 
Most aim to enhance students' understanding of target knowledge by employing pedagogical strategies such as recommending exercises~\cite{deng2023towards} or selecting teaching examples~\cite{ross-andreas-2024-toward}. 
However, these approaches fall short in broader situations requiring both understanding and practical application of specific pieces of knowledge to solve real-world, goal-driven problems. 
Such scenarios demand tutors to proactively guide people toward completing targeted tasks (e.g., coding).
Furthermore, the tutoring outcomes are challenging to assess since targeted tasks can often be completed by open-ended solutions.



To bridge this gap, we introduce \textbf{coding tutoring}, a promising yet underexplored task for LLM agents.
As illustrated in Figure~\ref{fig:example}, the tutor is provided with a target coding task and task-specific knowledge (e.g., cross-file dependencies and reference solutions), while the student is given only the coding task. The tutor does not know the student's prior knowledge about the task.
Coding tutoring requires the tutor to proactively guide the student toward completing the target task through dialogue.
This is inherently a goal-oriented process where tutors guide students using task-specific knowledge to achieve predefined objectives. 
Effective tutoring requires personalization, as tutors must adapt their guidance and communication style to students with varying levels of prior knowledge. 


Developing effective tutoring agents is challenging because off-the-shelf LLMs lack grounding to task-specific knowledge and interaction context.
Specifically, tutoring requires \textit{epistemic grounding}~\citep{tsai2016concept}, where domain expertise and assessment can vary significantly, and \textit{communicative grounding}~\citep{chai2018language}, necessary for proactively adapting communications to students' current knowledge.
To address these challenges, we propose the \textbf{Tra}ce-and-\textbf{Ver}ify (\textbf{\model}) agent workflow for building effective LLM-powered coding tutors. 
Leveraging knowledge tracing (KT)~\citep{corbett1994knowledge,scarlatos2024exploring}, \model explicitly estimates a student's knowledge state at each turn, which drives the tutor agents to adapt their language to fill the gaps in task-specific knowledge during utterance generation. 
Drawing inspiration from value-guided search mechanisms~\citep{lightman2023let,wang2024math,zhang2024rest}, \model incorporates a turn-by-turn reward model as a verifier to rank candidate utterances. 
By sampling more candidate tutor utterances during inference (see Figure~\ref{fig:scale}), \model ensures the selection of optimal utterances that prioritize goal-driven guidance and advance the tutoring progression effectively. 
Furthermore, we present \textbf{Di}alogue for \textbf{C}oding \textbf{T}utoring (\textbf{\eval}), an automatic protocol designed to assess the performance of tutoring agents. 
\eval employs code generation tests and simulated students with varying levels of programming expertise for evaluation. While human evaluation remains the gold standard for assessing tutoring agents, its reliance on time-intensive and costly processes often hinders rapid iteration during development. 
By leveraging simulated students, \eval serves as an efficient and scalable proxy, enabling reproducible assessments and accelerated agent improvement prior to final human validation. 



Through extensive experiments, we show that agents developed by \model consistently demonstrate higher success rates in guiding students to complete target coding tasks compared to baseline methods. We present detailed ablation studies, human evaluations, and an inference time scaling analysis, highlighting the transferability and scalability of our tutoring agent workflow.

% % !TEX root =  ../main.tex
\section{Background on causality and abstraction}\label{sec:preliminaries}

This section provides the notation and key concepts related to causal modeling and abstraction theory.

\spara{Notation.} The set of integers from $1$ to $n$ is $[n]$.
The vectors of zeros and ones of size $n$ are $\zeros_n$ and $\ones_n$.
The identity matrix of size $n \times n$ is $\identity_n$. The Frobenius norm is $\frob{\mathbf{A}}$.
The set of positive definite matrices over $\reall^{n\times n}$ is $\pd^n$. The Hadamard product is $\odot$.
Function composition is $\circ$.
The domain of a function is $\dom{\cdot}$ and its kernel $\ker$.
Let $\mathcal{M}(\mathcal{X}^n)$ be the set of Borel measures over $\mathcal{X}^n \subseteq \reall^n$. Given a measure $\mu^n \in \mathcal{M}(\mathcal{X}^n)$ and a measurable map $\varphi^{\V}$, $\mathcal{X}^n \ni \mathbf{x} \overset{\varphi^{\V}}{\longmapsto} \V^\top \mathbf{x} \in \mathcal{X}^m$, we denote by $\varphi^{\V}_{\#}(\mu^n) \coloneqq \mu^n(\varphi^{\V^{-1}}(\mathbf{x}))$ the pushforward measure $\mu^m \in \mathcal{M}(\mathcal{X}^m)$. 


We now present the standard definition of SCM.

\begin{definition}[SCM, \citealp{pearl2009causality}]\label{def:SCM}
A (Markovian) structural causal model (SCM) $\scm^n$ is a tuple $\langle \myendogenous, \myexogenous, \myfunctional, \zeta^\myexogenous \rangle$, where \emph{(i)} $\myendogenous = \{X_1, \ldots, X_n\}$ is a set of $n$ endogenous random variables; \emph{(ii)} $\myexogenous =\{Z_1,\ldots,Z_n\}$ is a set of $n$ exogenous variables; \emph{(iii)} $\myfunctional$ is a set of $n$ functional assignments such that $X_i=f_i(\parents_i, Z_i)$, $\forall \; i \in [n]$, with $ \parents_i \subseteq \myendogenous \setminus \{ X_i\}$; \emph{(iv)} $\zeta^\myexogenous$ is a product probability measure over independent exogenous variables $\zeta^\myexogenous=\prod_{i \in [n]} \zeta^i$, where $\zeta^i=P(Z_i)$. 
\end{definition}
A Markovian SCM induces a directed acyclic graph (DAG) $\mathcal{G}_{\scm^n}$ where the nodes represent the variables $\myendogenous$ and the edges are determined by the structural functions $\myfunctional$; $ \parents_i$ constitutes then the parent set for $X_i$. Furthermore, we can recursively rewrite the set of structural function $\myfunctional$ as a set of mixing functions $\mymixing$ dependent only on the exogenous variables (cf. \cref{app:CA}). A key feature for studying causality is the possibility of defining interventions on the model:
\begin{definition}[Hard intervention, \citealp{pearl2009causality}]\label{def:intervention}
Given SCM $\scm^n = \langle \myendogenous, \myexogenous, \myfunctional, \zeta^\myexogenous \rangle$, a (hard) intervention $\iota = \operatorname{do}(\myendogenous^{\iota} = \mathbf{x}^{\iota})$, $\myendogenous^{\iota}\subseteq \myendogenous$,
is an operator that generates a new post-intervention SCM $\scm^n_\iota = \langle \myendogenous, \myexogenous, \myfunctional_\iota, \zeta^\myexogenous \rangle$ by replacing each function $f_i$ for $X_i\in\myendogenous^{\iota}$ with the constant $x_i^\iota\in \mathbf{x}^\iota$. 
Graphically, an intervention mutilates $\mathcal{G}_{\mathsf{M}^n}$ by removing all the incoming edges of the variables in $\myendogenous^{\iota}$.
\end{definition}

Given multiple SCMs describing the same system at different levels of granularity, CA provides the definition of an $\alpha$-abstraction map to relate these SCMs:
\begin{definition}[$\abst$-abstraction, \citealp{rischel2020category}]\label{def:abstraction}
Given low-level $\mathsf{M}^\ell$ and high-level $\mathsf{M}^h$ SCMs, an $\abst$-abstraction is a triple $\abst = \langle \Rset, \amap, \alphamap{} \rangle$, where \emph{(i)} $\Rset \subseteq \datalow$ is a subset of relevant variables in $\mathsf{M}^\ell$; \emph{(ii)} $\amap: \Rset \rightarrow \datahigh$ is a surjective function between the relevant variables of $\mathsf{M}^\ell$ and the endogenous variables of $\mathsf{M}^h$; \emph{(iii)} $\alphamap{}: \dom{\Rset} \rightarrow \dom{\datahigh}$ is a modular function $\alphamap{} = \bigotimes_{i\in[n]} \alphamap{X^h_i}$ made up by surjective functions $\alphamap{X^h_i}: \dom{\amap^{-1}(X^h_i)} \rightarrow \dom{X^h_i}$ from the outcome of low-level variables $\amap^{-1}(X^h_i) \in \datalow$ onto outcomes of the high-level variables $X^h_i \in \datahigh$.
\end{definition}
Notice that an $\abst$-abstraction simultaneously maps variables via the function $\amap$ and values through the function $\alphamap{}$. The definition itself does not place any constraint on these functions, although a common requirement in the literature is for the abstraction to satisfy \emph{interventional consistency} \cite{rubenstein2017causal,rischel2020category,beckers2019abstracting}. An important class of such well-behaved abstractions is \emph{constructive linear abstraction}, for which the following properties hold. By constructivity, \emph{(i)} $\abst$ is interventionally consistent; \emph{(ii)} all low-level variables are relevant $\Rset=\datalow$; \emph{(iii)} in addition to the map $\alphamap{}$ between endogenous variables, there exists a map ${\alphamap{}}_U$ between exogenous variables satisfying interventional consistency \cite{beckers2019abstracting,schooltink2024aligning}. By linearity, $\alphamap{} = \V^\top \in \reall^{h \times \ell}$ \cite{massidda2024learningcausalabstractionslinear}. \cref{app:CA} provides formal definitions for interventional consistency, linear and constructive abstraction.
%\newpage
% \section{Preliminaries}\label{app:prelims}

%\tbd{the basic autoencoder eq. (my colleagues didn't know what I mean exactly when I talked about this work)}

\paragraph{1- sparse SAE probes}
To evaluate how well SAE features predict a certain abstract feature, we utilize 1-sparse probes \cite{gurnee2023finding}. Specifically, we collect activations of a specific SAE feature on a contrastive dataset containing both answerable and not answerable examples, and fit a slope coefficient and intercept to predict the dataset label using linear regression. The Gemma 2 SAEs are trained using a JumpReLU activation function \cite{lieberum2024gemma}. We can sample SAE activations after the activation function (post-relu) or before (pre-relu).
Since there are more learnt features to be found in the latter setting, the main paper figures focus on that. However, we report all results for the post-relu setting in the appendix.


\paragraph{Residual stream probes}

Our residual stream probes are trained on model activations sampled from the model's residual stream. To avoid overfitting, we train the regression model using 5-fold cross validation and perform a hyperparameter optimization by sweeping over regularization parameters with 26 logarithmically spaced steps between 0.0001 and 1. To measure the variability of residual stream probes, we repeat our analysis 10 times with different randomly sampled training datasets.

\paragraph{N-sparse SAE probes}
To train SAE probes with more than 1 feature, we follow the general methodology of our 1-sparse probes. As testing all possible SAE feature combinations is computationally infeasible, we iteratively increase the number of features while testing only the most promising candidates for higher features combinations. Specifically, to find combinations of $k$ features, we use the top 50 best performing features of size $k-1$ and test all possible new combinations with the 500 best performing single SAE features. We use a constant regularization parameter of 1 for the probes, regardless of the number of features.

\paragraph{Feature similarities}

To calculate feature similarities, we use the cosine similarity of the corresponding SAE encoder weight and the slope coefficients of the linear probes trained on the residual stream. SAE features are only compared to other SAE features of the same SAE, and residual stream probes trained at the same location in the model as the SAE. To compare how similar differently sized groups of SAE features are to the residual stream probes, we calculate the mean absolute cosine sim of the top 10 best performing SAE features of a certain group size (1 to 5) with the 10 residual stream probes trained on different training subsets.

\section{Datasets}
\label{app:datasets}

\myparagraph{Full Dataset details}

\begin{itemize}[leftmargin=*,topsep=0pt,noitemsep]
    \item \textbf{SQUAD} \citep{rajpurkar2018know}:  Dataset consisting of a short context passage and a question relating to the context. We follow the training data split and prompting template provided by \citet{slobodkin2023curious}.
    \item \textbf{IDK} \citep{sulem2021we}: Dataset with questions in the style of SQUAD, containing both answerable and unanswerable examples. We specifically use the non-competitive and unanswerable subsets of the ACE-whQA dataset.
    \item \textbf{BoolQ\_3L} 
    \citep{sulem2022yes}: Yes/no questions with answerable and unanswerable subsets.
    \item \textbf{Math Equations}: Synthetic dataset contrasting solvable equations with equations containing unknown variables.
    \item \textbf{Celebrity Recognition}: Queries requiring knowledge about celebrities.
    For construction, we use a public dataset of actors and movies from IMDB\footnote{\url{https://www.kaggle.com/datasets/darinhawley/imdb-films-by-actor-for-10k-actors}}, and generate a list of the 1000 most popular actors after 1990, as measured by the total number of ratings their movies received. We construct an additional dataset of non-celebrity names by randomly generating first and last name combinations using the most common North American names from Wikipedia\footnote{\url{https://en.wikipedia.org/wiki/Lists_of_most_common_surnames_in_North_American_countries} and \url{https://en.wikipedia.org/wiki/List_of_most_popular_given_names?utm_source=chatgpt.com}}. 
\end{itemize}

\paragraph{Dataset sizes}

\begin{table}[h]
    \centering
    \begin{tabular}{lc}
        \hline
        Dataset & Size \\
        \hline
        SQUAD (train) & 2000 \\
        BoolQ (train) & 2000 \\
        SQUAD (test) & 1800 \\
        SQUAD (variations) & 1800 \\
        BoolQ (test) & 2000 \\
        IDK & 484 \\
        Equation & 2000 \\
        Celebrity & 600 \\
        \hline
    \end{tabular}
    \caption{Number of examples for each used dataset.}
    \label{table:dataset-size}
\end{table}

Table~\ref{table:dataset-size} shows the number of examples for each dataset used in our evaluation.



\section{Related Works}

SAEs were proposed early for LM interpretability \cite{yun2021transformer}.
%
Many studies focus on improving training efficiency and effectiveness, but the latter is usually measured in terms of reconstruction quality and hence disconnected from downstream scenarios \cite{rajamanoharan2024improving,lieberum2024gemma}. %templeton2024scaling
Only most recent work addresses this issue.
% Scaling and evaluating sparse autoencoders
\citet{gao2024scaling} apply downstream probing, yet the classification tasks considered are most simple (e.g sentiment, language identification) and likely do not test any generalization. %.
% Towards Principled Evaluations of Sparse Autoencoders for Interpretability and Control
Similarly, \citet{makelov2024towards} use downstream data without considering generalization.%. However, the method applies downstream data during SAE training without considering generalization specifically and hence will likely suffer in out-of-distribution cases.

% Several works focus on downstream evaluation
% % Evaluating Sparse Autoencoders on Targeted Concept Erasure Tasks
% % Sparse Feature Circuits: Discovering and Editing Interpretable Causal Graphs in Language Models
% \citet{} propose a method for automated downstream evaluation which however uses the SHIFT approach \cite{} which requires datasets and hence will similarly suffer in ood scenarios.


Several studies evaluate SAE features, including downstream settings.
Yet, research often focuses on simple models and features 
% https://transformer-circuits.pub/2023/monosemantic-features/index.html
\cite{yun2021transformer,bricken2023monosemanticity,kissane2024interpreting}.
%
% IOI > Sparse Autoencoders Match Supervised Features for Model Steering on the IOI Task
% IOI ># 24-iclr-SPARSE AUTOENCODERS FIND HIGHLY INTERPRETABLE FEATURES IN LANGUAGE MODELS
% BIAS> Effectiveness of Sparse Autoencoder for understanding and removing gender bias in LLMs
% SV > Sparse Feature Circuits: Discovering and Editing Interpretable Causal Graphs in Language Models
There are various works focusing on more abstract concepts such as %bias \cite{}, 
indirect object identification \cite{cunningham2023sparse} and subject-verb agreement \cite{marks2024sparse},
but those are still directly related to syntax. In contrast, answerability often depends on domain-specific background knowledge (e.g., math or factual knowledge) and hence better suits the study on generalization.
%
% SPARSE AUTOENCODERS REVEAL TEMPORAL DIFFERENCE LEARNING IN LARGE LANGUAGE MODELS
\citet{demircan2024sparse} consider the representation of quality estimates from reinforcement learning, and hence a rather complex concept, but they focus on task-specific (vs generalizable) SAEs.
%
Closest to our work are the following.
%
% https://transformer-circuits.pub/2024/features-as-classifiers/index.html
\citet{bricken2024using} focus on comparing SAE features to linear probes as bioweapon classifiers.
Similarly to us, they show that the SAE probes are competitive but more complex and brittle; for instance, already format mismatch between the transformer/SAE/probe training data may degrade performance. When evaluated on multilingual out-of-distribution data (similar to in-domain data but in different languages), they find that SAE features can generalize well in specific settings in the mostly lexical task of bioweapon classification. 
%However, their SAEs are trained on smaller models and synthetic, domain-specific, biology data, and they only consider multi-lingual out-of-domain settings.
%
% https://www.lesswrong.com/posts/NMLq8yoTecAF44KX9/sae-probing-what-is-it-good-for-absolutely-something
\citet{kantamneni2024saeprobing} similarly conduct experiments comparing to traditional probing on activations and demonstrate that SAEs work better in certain scenarios (e.g., with very small datasets or corrupted data). They also consider multilingual out-of-distribution data and, % and imbalanced class data are considered and, 
similar to us, obtain mixed results without clear conclusion which probes are better.
%Note that they use the Gemma 9B base model and hence can only consider simpler features. 
Our evaluation lifts this work to %the instruction-tuned model version, 
a more complex task and a greater variety of distributions.
%The key differences in our work is that we evaluate generalization in a more complex and abstract setting, and test a wider variety of out of distribution data.
%VT that's exactly below no?

% Several studies have explored the use of SAEs in language models, with many focusing on improving training efficiency and reconstruction quality \cite{gao2024scaling, rajamanoharan2024improving, templeton2024scaling}. Some approaches optimize SAE training using proxy metrics—such as reconstruction loss—to enhance training stability. However, while these methods improve performance in terms of efficiency, they do not directly assess downstream performance. %, whether the resulting features capture abstract linguistic phenomena that generalize across tasks \citep{rajamanoharan2024improving,rajamanoharan2024jumping}. 
% % In contrast, our work evaluates whether SAE features encode abstract concepts by testing their transferability on an answerability task.
% %VT for space, these are clearly not directly related to what we do

% Other works have examined the properties of SAE features through probing and qualitative analysis. For instance, studies have investigated how scaling affects SAE representations and have qualitatively assessed their linguistic functions \citep{templeton2024scaling,kissane2024interpreting}. In addition, approaches that align network features to specific functions—such as the feature-aligned sparse autoencoders introduced by \citet{marks2024enhancing}—aim to improve interpretability. However, these methods predominantly focus on in-domain performance, leaving open questions about whether the learned features can generalize to new contexts. Similarly, feedback pattern analysis in language models, as demonstrated by \citet{marks2024enhancing}, does not address whether these features are robust enough to capture abstract capabilities across diverse domains. % tasks. % VT we study one task right? just different domains

% Alternative methodologies have interpreted SAE features in terms of neural circuits, proposing that they serve as modular building blocks underlying model behavior. One circuit-based approach maps processing components within language models \citep{dunefsky2024transcoders}. This perspective, however, generally assumes that features function as well-defined, modular units—a notion that may not hold when features encode abstract concepts such as answerability.

% Foundational work on sparse representations provided early insights into the benefits of sparsity for disentangling features \citep{ng2011sparse}, yet these methods were developed in simpler settings and do not directly address the challenges of extracting and generalizing abstract features in modern large language models.

In summary, %while prior work has advanced SAE training techniques, probed in-domain feature behavior, and proposed circuit-based interpretations, systematically evaluated whether the abstract features captured by SAEs can generalize across domains
prior work has largely neglected generalization beyond multilingual scenarios. Our study closes this gap in the context of a suitable, complex concept, answerability, which likely manifests differently across contexts.
%by comparing SAE features with residual stream probes on an answerability task, thereby testing the robustness and transferability of abstract representations.

% Sparse autoencoders have been studied in the context of LLMs from various angles. Several works focus on the training of SAEs itself, both in terms of efficiency and effectiveness \cite{gemmascope,someothers,rajamanoharan2024improving,rajamanoharan2024jumping}. While this training includes evaluations, these experiments typically do not test the learnt features themselves but rather consider proxies closer to the actual training (e.g., related to the reconstruction loss). Further, please observe, that some recent studies have pointed out that SAE training \vt{concretize issues, harvard paper, +other?} \cite{}.

% Closest to our study are a few works which probe the trained SAE features \cite{}. \vt{need details about specific studies here, spec. delimitation/differences/commonalities}
% \cite{templeton2024scaling}
% \cite{kissane2024interpreting}
% % le cunn
% % the blog post

% % other works:
% \vt{some sentence about polysemanticity and SAEs}
% %
% There are also works which consider SAE features in the context of so-called circuits, models trying to capture specific LLM processing mechanisms in the form of algorithms in interpretable ways \cite{marks2024sparse,dunefsky2024transcoders}.
% % 
% Finally, note that SAEs have been studied and used in various other contexts for a longer time, before transformers even existed \cite{ng2011sparse,someother?}.
%NOTE mention other answerbility approaches maybe

% old from lovis:
% Mechanistic interpretability is a growing field that aims to develop a grounded understanding of neural networks by reverse engineering its computations. An important framework of understanding neural networks is through the lens of circuits, the idea that analyzing weight connections inside a neural network allows us to identify computational sub-graphs that implement specific algorithms \cite{elhage2021mathematical, olah2020zoom}. A key problem for finding interpretable circuits and features in transformer models has been that language models are often polysemantic, they represent more features than they have parameters, which can result in model components that are difficult to interpret since they represent more than one feature \cite{elhage2022superposition}. 

% One promising approach to overcoming the problem of polysemanticity is using sparse autoencoders (SAE) to learn a disentangled feature representation \cite{cunningham2023sparse, bricken2023monosemanticity}. Recent work has proposed improvements to the SAE architecture and applied SAE to much larger models \cite{gao2024scaling, templeton2024scaling, rajamanoharan2024jumping}, which allows studying more abstract and interesting features. SAEs have been found to learn both interpretable and causally important features that are relevant for safety, such as features relating to unsafe code or deception \cite{templeton2024scaling}. While the existence of these features already provides useful information about the model's reasoning, understanding in which scenarios these features activate could deepen our understanding of the computation inside language models significantly. Circuit discovery methods are a promising approach to answering these questions. 

% Circuit discovery methods aim to (automatically) determine which components of a language model are responsible for a behavior \cite{conmy2023towards, syed2023attribution}. While usually used to attribute model behavior to individual components of language models, circuit discovery has recently been extended to utilize SAEs as well \cite{marks2024sparse}. Applying circuit discovery to understand SAE features allows isolating the upstream components of the model that are responsible for activating a specific downstream SAE feature of interest.

% In addition to understanding specific SAE features, using SAE features instead of general model behavior as the target metric for circuit discovery also has the advantage of presenting a clear metric for potentially complex tasks that are hard to evaluate. Existing circuits usually describe a very narrow behavior that can be evaluated through individual answer tokens, and it is not clear how these metrics can be extended to more complex model behavior (e.g. being sycophantic). While it is unlikely that there are single SAE features that fully represent the model's reasoning for such complex model behaviors, causally relevant SAE features related to safety concepts have been found in prior work. We believe using such SAE features to present an interesting target metric for circuit discovery that can provide insights into at least part of the overall circuits making up complex model behavior.
%\newpage



\begin{table*}[ht!]
\centering
\small % Reduce font size
\setlength{\tabcolsep}{4pt} % Reduce horizontal padding if needed
\begin{tabular}{l p{5.5cm} p{5.5cm}}
    \toprule
    \textbf{Dataset} & \textbf{Answerable} & \textbf{Unanswerable} \\
    \midrule
    SQUAD & 
    \textbf{Passage:} The first beer pump known in England is believed to [\dots]. %have been invented by John Lofting (b. Netherlands 1659--d. Great Marlow, Buckinghamshire 1742), an inventor, manufacturer, and merchant of London.
    \newline 
    \textbf{Question:} When was John Lofting born? 
    & 
    \textbf{Passage:} Starting in 2010/2011, Hauptschulen were merged  [\dots]. %with Realschulen and Gesamtschulen to form a new type of comprehensive school  in the German States of Berlin and Hamburg---called Stadtteilschule in Hamburg and Sekundarschule in Berlin (see: Education in Berlin, Education in Hamburg). 
    \newline 
    \textbf{Question:} In what school year were Hauptschulen last combined with Realschulen and Gesamtschulen? \\
    \midrule
    IDK & 
    \textbf{Passage:} Singapore has reported 16 deaths. \newline 
    \textbf{Question:} Where are the deaths? 
    & 
    \textbf{Passage:} Showed the arrest of the prime suspect. \newline 
    \textbf{Question:} Where was the arrest? \\
    \midrule
    BoolQ & 
    \textbf{Passage:} On April 20, 2018, ABC officially renewed \textit{Grey's Anatomy} for a network primetime drama record-tying fifteenth season. \newline 
    \textbf{Question:} Is season 14 the last of \textit{Grey's Anatomy}? 
    & 
    \textbf{Passage:} Discover is the fourth largest credit card brand in the U.S., behind Visa, MasterCard, and American Express, with nearly 44 million cardholders. \newline 
    \textbf{Question:} Are pasilla chiles and poblano chiles the same? \\
    \midrule
    Equation & 
    \textbf{Given equations:} \newline n = 53 \newline v = 90 \newline 
    \textbf{Final equation:} \newline n / v = 
    & 
    \textbf{Given equations:} \newline n = 17 \newline u = 38 \newline 
    \textbf{Final equation:} \newline n * t = \\
    \midrule
    Celebrity & 
    \textbf{Article:} Yesterday, I saw an article about Gerard Butler. They really are a great actor. \newline 
    \textbf{Question:} Do you know what their age is? 
    & 
    \textbf{Article:} Yesterday, I saw an article about Tania Scott. They really are a great actor. \newline 
    \textbf{Question:} Do you know what their age is? \\
    \bottomrule
\end{tabular}
\caption{Answerable and Unanswerable Examples from Different Datasets}
\label{tab:answerability}
\end{table*}




\section{Methodology}
We evaluate SAE probes for answerability detection with a specific focus on  generalization.

\myparagraph{SAE Probes}
We use the "Gemma Scope" SAEs pretrained by \citet{lieberum2024gemma} for the instruction-tuned model Gemma 2 \citep{team2024gemma}, and specifically the largest available ones with a width (number of dimensions) of 131k. \citet{lieberum2024gemma} provide SAEs trained on layers 20 and 31.
%
Note that answerability more generally (i.e., beyond specific types of answerability) is a rather high-level concept, which we assume to be represented in intermediate and later layers.
Unless otherwise specified, we search for features using 2k samples of SQUAD (balanced, leaving 1.8k for testing). We collect the feature activations on the last token position and then use 5-fold cross validation for finding SAE features that are predictive for answerability, thus obtaining 1-sparse SAE probes \citep{gurnee2023finding}.
%Note that the common naming as ``probes'' can be misleading, with these probes there are no parameters to be trained.
We then train final probes\footnote{We use SAE probes and SAE features synonymously.} (i.e., scale and bias) for best performing features, which are used for the out-of-distribution evaluation. See Appendix~\ref{app:prelims} for details. ``Top'' features are selected based on training set performance.




\myparagraph{Baselines: Linear Probes}
We train simple linear residual stream probes on the (in-domain) training dataset we also use for finding the SAE features. To ensure robustness, we employ bootstrap analysis across different training splits. %, revealing significant variability in out-of-distribution generalization [i don't recall if this true anymore]. % VT commented based on comment
Since we also focus on SAE features for the residual stream, this probing represents an upper bound for the SAE probing performance on in-domain data.
Observe that these probes achieve 85-90\% accuracy on the in-domain SQUAD data, and thus provide a strong benchmark for comparison. 

\myparagraph{Datasets}
We focus on context-based question answering in the English language.
%In the main experiments, we train and evaluate on SQUAD \cite{rajpurkar2018know} and additionally
We use established data as well as datasets specifically constructed  for out-of-distribution evaluation; for examples see Table~\ref{tab:answerability}.
%We use two established answerability datasets (BoolQ \cite{} and IDK \cite{}), and create two synthetic specialized datasets (math equations, celebrity names):
\begin{itemize}[leftmargin=*,topsep=0pt,noitemsep]
    \item \textbf{SQUAD} \citep{rajpurkar2018know}: Established dataset, passages plus questions relating to them. %Dataset consisting of a short context passage and a question relating to the context. We follow the training data split and prompting template provided by \citet{slobodkin2023curious}.
    \item \textbf{IDK} \citep{sulem2021we}: Dataset with questions in the style of SQUAD.
    % , containing both answerable and unanswerable examples. We specifically use the non-competitive and unanswerable subsets of the ACE-whQA dataset.
    \item \textbf{BoolQ\_3L} 
    \citep{sulem2022yes}: Context-based yes/no questions. % with answerable and unanswerable subsets.
    \item \textbf{Math Equations}: Synthetic dataset contrasting solvable equations with equations containing unknown variables.
    \item \textbf{Celebrity Recognition}: Context-based queries requiring background knowledge about celebrities; for construction details, see Appendix~\ref{app:datasets}.
    % For construction, we use a public dataset of actors and movies from IMDB\footnote{\url{https://www.kaggle.com/datasets/darinhawley/imdb-films-by-actor-for-10k-actors}}, and generate a list of the 1000 most popular actors after 1990, as measured by the total number of ratings their movies received. We construct an additional dataset of non-celebrity names by randomly generating first and last name combinations using the most common North American names from Wikipedia\footnote{\url{https://en.wikipedia.org/wiki/Lists_of_most_common_surnames_in_North_American_countries} and \url{https://en.wikipedia.org/wiki/List_of_most_popular_given_names?utm_source=chatgpt.com}}. 
\end{itemize}

% For the question-answering datasets (SQUAD, BoolQ, and IDK) we use the following consistent formatting:
%
% \begin{verbatim}
% Given the following passage and question, answer the question:
% Passage: {passage}
% Question: {question}
% \end{verbatim}
%
% For the equation dataset, we construct ...
%
% \begin{verbatim}
% "Given the following equations, determine the result of the final equation.
% Given equations:
% {eq_1}
% {eq_2}
% Final equation:
% {eq_3}"
% \end{verbatim}
%
% For the celebrity dataset, we use a public list of celebrity names from \cite{} and generate additional non-celebrity names using the most common north american first names and last names from wikipedia. 
%
% \begin{verbatim}
% "Yesterday, I saw an article about {name}. They really are a great actor. Do you know what their age is?"
% \end{verbatim}

% \subsection{Generalization Analysis}

% We evaluate generalization through multiple lenses:
% \begin{itemize}
%     \item Cross-dataset performance comparing SQUAD-trained features on out-of-distribution data
%     \item Prompt variation analysis testing robustness to input formatting
%     \item Analysis of using a combination of up to five SAE features 
%     \item Similarity analysis of top-performing SAE features and learned residual stream probes
% %    \item Feature consistency analysis across domains using attribution scores
% %    \item Ablation studies measuring the impact of individual features versus feature combinations
% \end{itemize}
%This approach reveals variance in feature transfer ability---while some SAE features show impressive generalization, others remain strongly domain-specific. Moreover, despite strong in-domain performance, residual stream probes exhibit inconsistent transfer when it comes to generalisation.

% \vt{brief intro sentence, why we choose the feature, compare (only) to regular probes}

% \myparagraph{Hypothesis} 
% % vs probe esp in terms of generalization hold their promise?

% \myparagraph{Datasets} 
% % mainly context-based QA
% % squad, generaliz on ambigqa, boolq, math celeb
% % figure with some prompt example, rest into appendix


% \myparagraph{Methods}  % if we don't have more about method we can rename it to models
% \vt{gemma scope since... , also experimented briefly with llama scope; @lovis anything what we could mention here specifically?}

% \paragraph{Linear probes}

% \lh{
% - Trained on 2000 samples of SQUAD (balanced, leaving 1800 for testing)
% - last token position
% - 5 fold cross validation
% - sweeping over regularization parameters with 26 logarithmically spaced steps between 0.0001 and 1
% - Fitting the final probe with the best regularization parameter on the whole training set
% - Trained probes are then evaluated on OOD datasets
% - Analysis is repeated 10 times with different random training set splits

% }



% \begin{table*}[h]
%     \centering
%     \begin{tabular}{|l|p{6cm}|p{6cm}|}
%         \hline
%         \textbf{Dataset} & \textbf{Answerable} & \textbf{Unanswerable} \\
%         \hline
%         SQUAD & Given the following passage and question, answer the question: \newline \textbf{Passage:} The first beer pump known in England is believed to have been invented by John Lofting (b. Netherlands 1659-d. Great Marlow Buckinghamshire 1742) an inventor, manufacturer and merchant of London.\newline \textbf{Question:} When was John Lofting born? & Given the following passage and question, answer the question:\newline \textbf{Passage:} Starting in 2010/2011, Hauptschulen were merged with Realschulen and Gesamtschulen to form a new type of comprehensive school in the German States of Berlin and Hamburg, called Stadtteilschule in Hamburg and Sekundarschule in Berlin (see: Education in Berlin, Education in Hamburg). \newline \textbf{Question:} In what school year were Hauptschulen last combined with Realschulen and Gesamtschulen?\\
%         \hline
%         IDK & Given the following passage and question, answer the question: \newline \textbf{Passage:} Singapore has reported 16 deaths.\newline \textbf{Question:} Where are the deaths? & Given the following passage and question, answer the question:\newline \textbf{Passage:} Showed the arrest of the prime suspect.\newline \textbf{Question:} Where was the arrest? \\
%         \hline
%         BoolQ & Given the following passage and question, answer the question: \newline \textbf{Passage:} On April 20, 2018, ABC officially renewed Grey's Anatomy for a network primetime drama record-tying fifteenth season.\newline \textbf{Question:} Is season 14 the last of grey's anatomy? & Given the following passage and question, answer the question:\newline \textbf{Passage:} Discover is the fourth largest credit card brand in the U.S., behind Visa, MasterCard and American Express, with nearly 44 million cardholders. \newline \textbf{Question:} Are pasilla chiles and poblano chiles the same?\\
%         \hline
%         Equation & Given the following equations, determine the result of the final equation.\newline \textbf{Given equations}:\newline n = 53\newline v = 90\newline \textbf{Final equation}:\newline n / v = & Given the following equations, determine the result of the final equation.\newline \textbf{Given equations}: \newline n = 17\newline u = 38\newline \textbf{Final equation}:\newline n * t = \\
%         \hline
%         Celebrity & Yesterday, I saw an article about Gerard Butler. They really are a great actor. Do you know what their age is? & Yesterday, I saw an article about Tania Scott. They really are a great actor. Do you know what their age is? \\
%         \hline
%     \end{tabular}
%     \caption{Answerable and Unanswerable Examples from Different Datasets}
%     \label{tab:answerability}
% \end{table*}







%\newpage
\begin{figure*}[h!]
    \centering %trim titles
     \includegraphics[width=.5\textwidth,trim={0 1.4cm 0 3.4cm},clip]{figures/sae_feature_accuracies_layer31_pre_SQUAD_train.png}%{figures/sae_feature_accuracies_layer31_pre.png}
    ~%
    \includegraphics[width=.5\textwidth,trim={0cm 1.4cm 0 3.4cm},clip]{figures/sae_feature_accuracies_layer31_pre_BoolQ_train.png}
\caption{Out-of-distribution comparison between top SAE features,  pre-activation, and linear probes on layer 31; trained on SQUAD (left) and BoolQ (right).}
    \label{fig:sae-probe_pre31}
\end{figure*}

\begin{figure}[h!]
    \centering %trim titles
    \includegraphics[width=.5\textwidth,trim={0 1.4cm 0 3.4cm},clip]{figures/hierarchical_sae_probe_layer31_pre.png}
    \caption{Combinations of SAE features, displaying the median value across top feature groups with quartile ranges in the error bars.}
    \label{fig:sae-k-pre31}
\end{figure}

% \begin{figure*}[t]
% \centering

% %--------------- First row ---------------%
% \begin{subfigure}[t]{0.49\textwidth}
%   \centering
%   \includegraphics[width=\linewidth,
%                    trim={2cm 1.4cm 0 3.4cm},clip]
%                    {figures/sae_feature_accuracies_layer31_pre_SQUAD_train.png}
%   \caption{Out-of-distribution comparison on layer 31, trained on SQuAD.}
%   \label{fig:sae-probe_pre31-squad}
% \end{subfigure}
% \hfill
% \begin{subfigure}[t]{0.49\textwidth}
%   \centering
%   \includegraphics[width=\linewidth,
%                    trim={2cm 1.4cm 0 3.4cm},clip]
%                    {figures/sae_feature_accuracies_layer31_pre_BoolQ_train.png}
%   \caption{Out-of-distribution comparison on layer 31, trained on BoolQ.}
%   \label{fig:sae-probe_pre31-boolq}
% \end{subfigure}

% \vspace{0.5em}  % Tighten or remove if desired

% %--------------- Second row ---------------%
% \begin{subfigure}[t]{0.49\textwidth}
%   \centering
%   \includegraphics[width=\linewidth,
%                    trim={0 1.4cm 0 3.4cm},clip]
%                    {figures/hierarchical_sae_probe_layer31_pre.png}
%   \caption{Combinations of SAE features.}
%   \label{fig:sae-k-pre31}
% \end{subfigure}
% \hfill
% \begin{subfigure}[t]{0.30\textwidth}
%   \centering
%   \includegraphics[width=\linewidth,
%                    trim={5.2cm 4.1cm 0 3.4cm},clip]
%                    {figures/feature_similarities.png}
%   \caption{Cosine similarities of top SAE features and probes.}
%   \label{fig:similarities}
% \end{subfigure}

% \caption{
%   Four related plots: 
%   (a) and (b) show out-of-distribution comparisons on layer 31 
%   for SQuAD vs.\ BoolQ training;
%   (c) demonstrates combinations of SAE features;
%   (d) shows feature/probe cosine similarities.
% }
% \label{fig:all_four}
% \end{figure*}




\section{Evaluation}



In the following, we present of our main experiments; see the appendix for additional findings.

% Todo
% Get some formatted dataset examples for all datasets
% Try question in celeb dataset (how old)
% Check probe accuracy and generalization for different training data sizes
% Averaging linear probes as baseline?
% Use higher k
% Prioritize over weekend with higher number of features + OOD eval
% Baseline attention probe
% Change cosine similarity analysis to average absolute
% Add clean reconstruction result of residual stream probe with SAE
% Cosine sim of residual stream probes with top SAE features
% Probes with each other
% Probes with top sae features
% Check pre vs post relu
% Compare features to narrow SAE
% Compare to Layer 31 SAE
% Is it more sparse? Less combinations needed? Then it is a feature that is not fully computed in L20?




\myparagraph{Linear vs SAE Probes: Generalization, Figure~\ref{fig:sae-probe_pre31}} 
We focus first on layer 31.
In domain, the best SAE features reach an accuracy of around 0.8 while the linear probes reach 0.9. Note that this is not surprising, since the probes have more parameters that are actually trained and thus optimized for this data. Nevertheless, it shows some advantage of probes in case in-domain data is available. 

We see rather great variation across our out-of-distribution datasets. Our custom Equation data stands out in that several SAE features and also the probe reach high performance. While this seems to show that the mathematical context makes answerability easier to detect, observe that the performance is considerably worse on layer 20, see Figure~\ref{fig:sae-probe_pre20} in the appendix. %This shows that it is a complex concept only fully represented later in the model.

Some, but few, top SAE features reach considerable performance out of distribution on IDK - matching the performance of the linear probe - and Celeb. Yet, the performance on BoolQ is considerably bad. %\vt{REASONS?} 
On the other hand, the linear probe performs bad on Celeb. Figure~\ref{fig:res-probe-all} in the Appendix shows the median value over 10 bootstrap samples including quartiles in the error bars; overall it correlates with performance.

For layer 20 (Figure~\ref{fig:sae-probe_pre20}), we see generally worse performance. 
Interestingly, the numbers for Celeb are significantly worse than all others for both the SAE and the linear probes. Since we see one exception (an SAE feature with higher than random performance), we hypothesize that there are special features encoding knowledge about celebrities which do not happen to be among our top features. % ie rather than the knowledge is not present on layer 20
%this might also explain why the probe performs bad on celeb that this would be more about people features not answerability
In fact, a closer investigation reveals that there are good features for BoolQ and our domain-specific Equation and Celeb datasets on layer 20 already (see Figure~\ref{fig:top-in-domain} in the appendix), but they are not the same features as the ones found by training on SQuAD. %TODO

Finally, we confirmed our findings by training on  BoolQ (also 2k samples) and evaluating on the other datasets. We mainly see that varying the training data can make everything considerably worse, even with the same task and seemingly similar, but potentially lower-quality data. The unanswerable samples in BoolQ were constructed by combining contexts and questions of similar dataset samples, hence capture only one type of unanswerability.

Overall, our experiments demonstrate one main critical issue with OOD data: \emph{the standard procedure for finding good SAE features can easily fail, even if good features are available}. 
The fact that good features exist while the linear probes also fail shows some potential of SAEs. Yet finding good, generalizing features represents an open challenge.

%A key issue for SAE probing could be feature splitting, a phenomenon where SAEs of different sizes learn features in different granularities, often splitting more general features into multiple more specialized features \citep{bricken2023monosemanticity, chanin2024absorption}. If abstract concepts like answerability are split into many separate features, this can cause problems for feature-based practical applications.

\myparagraph{Top Features, %Distributions, 
Figure~\ref{fig:sae-probe_pre31}} 
Interestingly, the top three features on the in-domain SQuAD data happen to also generalize better here. This does not turn out be the case beyond the top-1 feature more generally, see Appendix~\ref{app:eval-other-layers}. For BoolQ, the variability of the results precludes clear conclusions.
%

\myparagraph{SAE Feature Combinations, Figure~\ref{fig:sae-k-pre31}} 
Given the partly domain-specific nature of our out-of-distribution datasets, we hypothesized that combinations of features might work better as general probes. %Observe that such combinations have also been considered in previous works \cite{lecun,oneother}.
However, while increasing the number of SAE features improves the in-domain performance, OOD performance doesn't improve upon the best performing individual feature (top of blue error bars) here; layer 31, pre-activation. Other examples in Appendix~\ref{app:feature-combinations} show similar trends, and even some degradation. This underlines our above finding that the ood setting requires better methods for SAE feature search.

% \begin{wrapfigure}
% {r}{0.25\textwidth}

\begin{figure}[h!]
    \centering %trim titles
    \includegraphics[width=.18\textwidth,trim={5.2cm 4.1cm 0 3.4cm},clip]{figures/feature_similarities.png}
    \caption{Cosine similarities of top SAE features and linear probes for different seeds; the blue square shows high similarity between linear probes.}
\label{fig:similarities}
\end{figure}
  
% \end{wrapfigure}
\myparagraph{Feature Similarity, Figures~\ref{fig:similarities} \& \ref{fig:similarity_k}}

We find great similarity between different linear probes but only slight similarity between SAE features and individual probes, and it's even less between SAE features. Interestingly, the best SAE feature turns out to have highest (though low) similarity with the probes. Figure~\ref{fig:similarity_k} shows that combining SAE features yields greater similarity with linear probes.




% \myparagraph{Other experiments}
% We validated our setup by searching for bias-related features as it was done in related works.
% We also experimented with (inofficial) SAEs for an instruction-tuned Llama model, but the quality of the SAEs was not good enough for further experimentation. Finally, Gemma 2 2B and also the base models \vt{did not yield good enough performance on the question answering task itself}. 
\section{Conclusions}

We extensively evaluated SAE features for Gemma~2 in the out-of-distribution scenario using a variety of established and custom datasets. On the bright side, we find good SAE features for answerability across these domains. However, we show from various angles that the standard SAE feature search fails in finding these features and hence in terms of generalization.
We hypothesize that this is due to both sub-optimal training objectives and feature splitting with complex concepts \cite{bricken2023monosemanticity, chanin2024absorption}.
This shows the need for better technology for evaluating SAE features before SAEs are robustly applicable in practice.

%A key issue for SAE probing could be feature splitting, a phenomenon where SAEs of different sizes learn features in different granularities, often splitting more general features into multiple more specialized features \citep{bricken2023monosemanticity, chanin2024absorption}. If abstract concepts like answerability are split into many separate features, this can cause problems for feature-based practical applications.

%\newpage~\newpage~\newpage~\newpage
\section{Limitation}
The use of 3D-printed PLA for structural components improves improving ease of assembly and reduces weight and cost, yet it causes deformation under heavy load, which can diminish end-effector precision. Using metal, such as aluminum, would remedy this problem. Additionally, \robot relies on integrated joint relative encoders, requiring manual initialization in a fixed joint configuration each time the system is powered on. Using absolute joint encoders could significantly improve accuracy and ease of use, although it would increase the overall cost. 

%Reliance on commercially available actuators simplifies integration but imposes constraints on control frequency and customization, further limiting the potential for tailored performance improvements.

% The 6 DoF configuration provides sufficient mobility for most tasks; however, certain bimanual operations could benefit from an additional degree of freedom to handle complex joint constraints more effectively. Furthermore, the limited torque density of commercially available proprioceptive actuators restricts the payload and torque output, making the system less suitability for handling heavier loads or high-torque applications. 

The 6 DoF configuration of the arm provides sufficient mobility for single-arm manipulation tasks, yet it shows a limitation in certain bimanual manipulation problems. Specifically, when \robot holds onto a rigid object with both hands, each arm loses 1 DoF because the hands are fixed to the object during grasping. This leads to an underactuated kinematic chain which has a limited mobility in 3D space. We can achieve more mobility by letting the object slip inside the grippers, yet this renders the grasp less robust and simulation difficult. Therefore, we anticipate that designing a lightweight 3 DoF wrist in place of the current 2 DoF wrist allows a more diverse repertoire of manipulation in bimanual tasks.

Finally, the limited torque density of commercially available proprioceptive actuators restricts the performance. Currently, all of our actuators feature a 1:10 gear ratio, so \robot can handle up to 2.5 kg of payload. To handle a heavier object and manipulate it with higher torque, we expect the actuator to have 1:20$\sim$30 gear ratio, but it is difficult to find an off-the-shelf product that meets our requirements. Customizing the actuator to increase the torque density while minimizing the weight will enable \robot to move faster and handle more diverse objects.

%These constraints highlight opportunities for improvement in future iterations, including alternative materials for enhanced rigidity, custom actuator designs for higher control precision and torque density, the adoption of absolute joint encoders, and optimized configurations to balance dexterity and weight.


Our study was approved by the IRB of our institution.
Participants electronically signed a consent form describing the nature of our study and the data we would collect: their answers to the questionnaires, their demographic information provided by the platform, and their interactions with the study platform. All data was stored pseudonymously.
While our initial study description did not explicitly mention participants they would be exposed to phishing, this is a commonly used method in most phishing studies~\cite{resnik2018ethics,thomopoulos2023methodologies} to avoid excessive priming.
The participants were debriefed after completing the study with the full description, and is confirmed to incur only minimal risks~\cite{finn2007designing}, also confirmed by our IRB classifying our study as minimal risk.
Participants were appropriately remunerated for their time with a payment matching the highest minimum wage in their country.

We took further countermeasures to ensure participants' safety: the discomfort of being exposed to phishing emails was mitigated by the roleplay setting and their assigned fictitious identity.
Furthermore, their task was limited to clicking on links---there was no interaction with simulated phishing websites or other potentially harmful content.
Additionally, the phishing URLs we provided did not offer an easy way for participants to actually visit them (as our environment was preventing navigation); however, to protect participants that might transcribe or copy-paste them into their browsers, we constantly monitored all URLs to ensure they were offline during the duration of the study.




% \section*{Acknowledgments}

% Bibliography entries for the entire Anthology, followed by custom entries
%\bibliography{anthology,custom}
% Custom bibliography entries only
\bibliography{main}

\appendix
\label{sec:appendix}
%chatgpt helped me with this - needs more work 
\section{Preliminaries}\label{app:prelims}

%\tbd{the basic autoencoder eq. (my colleagues didn't know what I mean exactly when I talked about this work)}

\paragraph{1- sparse SAE probes}
To evaluate how well SAE features predict a certain abstract feature, we utilize 1-sparse probes \cite{gurnee2023finding}. Specifically, we collect activations of a specific SAE feature on a contrastive dataset containing both answerable and not answerable examples, and fit a slope coefficient and intercept to predict the dataset label using linear regression. The Gemma 2 SAEs are trained using a JumpReLU activation function \cite{lieberum2024gemma}. We can sample SAE activations after the activation function (post-relu) or before (pre-relu).
Since there are more learnt features to be found in the latter setting, the main paper figures focus on that. However, we report all results for the post-relu setting in the appendix.


\paragraph{Residual stream probes}

Our residual stream probes are trained on model activations sampled from the model's residual stream. To avoid overfitting, we train the regression model using 5-fold cross validation and perform a hyperparameter optimization by sweeping over regularization parameters with 26 logarithmically spaced steps between 0.0001 and 1. To measure the variability of residual stream probes, we repeat our analysis 10 times with different randomly sampled training datasets.

\paragraph{N-sparse SAE probes}
To train SAE probes with more than 1 feature, we follow the general methodology of our 1-sparse probes. As testing all possible SAE feature combinations is computationally infeasible, we iteratively increase the number of features while testing only the most promising candidates for higher features combinations. Specifically, to find combinations of $k$ features, we use the top 50 best performing features of size $k-1$ and test all possible new combinations with the 500 best performing single SAE features. We use a constant regularization parameter of 1 for the probes, regardless of the number of features.

\paragraph{Feature similarities}

To calculate feature similarities, we use the cosine similarity of the corresponding SAE encoder weight and the slope coefficients of the linear probes trained on the residual stream. SAE features are only compared to other SAE features of the same SAE, and residual stream probes trained at the same location in the model as the SAE. To compare how similar differently sized groups of SAE features are to the residual stream probes, we calculate the mean absolute cosine sim of the top 10 best performing SAE features of a certain group size (1 to 5) with the 10 residual stream probes trained on different training subsets.

\section{Datasets}
\label{app:datasets}

\myparagraph{Full Dataset details}

\begin{itemize}[leftmargin=*,topsep=0pt,noitemsep]
    \item \textbf{SQUAD} \citep{rajpurkar2018know}:  Dataset consisting of a short context passage and a question relating to the context. We follow the training data split and prompting template provided by \citet{slobodkin2023curious}.
    \item \textbf{IDK} \citep{sulem2021we}: Dataset with questions in the style of SQUAD, containing both answerable and unanswerable examples. We specifically use the non-competitive and unanswerable subsets of the ACE-whQA dataset.
    \item \textbf{BoolQ\_3L} 
    \citep{sulem2022yes}: Yes/no questions with answerable and unanswerable subsets.
    \item \textbf{Math Equations}: Synthetic dataset contrasting solvable equations with equations containing unknown variables.
    \item \textbf{Celebrity Recognition}: Queries requiring knowledge about celebrities.
    For construction, we use a public dataset of actors and movies from IMDB\footnote{\url{https://www.kaggle.com/datasets/darinhawley/imdb-films-by-actor-for-10k-actors}}, and generate a list of the 1000 most popular actors after 1990, as measured by the total number of ratings their movies received. We construct an additional dataset of non-celebrity names by randomly generating first and last name combinations using the most common North American names from Wikipedia\footnote{\url{https://en.wikipedia.org/wiki/Lists_of_most_common_surnames_in_North_American_countries} and \url{https://en.wikipedia.org/wiki/List_of_most_popular_given_names?utm_source=chatgpt.com}}. 
\end{itemize}

\paragraph{Dataset sizes}

\begin{table}[h]
    \centering
    \begin{tabular}{lc}
        \hline
        Dataset & Size \\
        \hline
        SQUAD (train) & 2000 \\
        BoolQ (train) & 2000 \\
        SQUAD (test) & 1800 \\
        SQUAD (variations) & 1800 \\
        BoolQ (test) & 2000 \\
        IDK & 484 \\
        Equation & 2000 \\
        Celebrity & 600 \\
        \hline
    \end{tabular}
    \caption{Number of examples for each used dataset.}
    \label{table:dataset-size}
\end{table}

Table~\ref{table:dataset-size} shows the number of examples for each dataset used in our evaluation.




\section{Additional analysis}

\subsection{Answerability Detection at Different Layers} \label{app:eval-other-layers}
\begin{figure*}[t]
    \centering
    \includegraphics[width=0.8\textwidth,trim={0 1.4cm 0 3.4cm},clip]{figures/sae_feature_accuracies_layer20_post.png}
    \caption{Answerability detection accuracies for top SAE features (Layer 20, post-activation).}
    \label{fig:sae-probe_post20}
\end{figure*}

\begin{figure*}[t]
    \centering
    \includegraphics[width=0.8\textwidth,trim={0 1.4cm 0 3.4cm},clip]{figures/sae_feature_accuracies_layer20_pre.png}
    \caption{Answerability detection accuracies for top SAE features (Layer 20, pre-activation).}
    \label{fig:sae-probe_pre20}
\end{figure*}

\begin{figure*}[t]
    \centering
    \includegraphics[width=0.8\textwidth,trim={0 1.4cm 0 3.4cm},clip]{figures/sae_feature_accuracies_layer31_post.png}
    \caption{Answerability detection accuracies for top SAE features (Layer 31, post-activation).}
    \label{fig:sae-probe_post31}
\end{figure*}

\begin{figure*}[t]
    \centering
    \includegraphics[width=0.7\textwidth,trim={0 2.2cm 1cm 3.4cm},clip]{figures/all_layers_probe.png}
    \caption{Linear probe trained on Layer 20 and Layer 31 residual stream (SQuAD) and evaluated on IDK, BoolQ, Celebrity, and Equation. The plot shows the median accuracy including the first and third quartile.}
    \label{fig:res-probe-all}
\end{figure*}

We repeat our SAE feature analysis in Layer 20 of the model, as well as providing additional analysis for SAE features activations sampled after the activation function. Figure~\ref{fig:sae-probe_pre20} shows the Layer 20 results using activations sampled before the activation function, while Figures~\ref{fig:sae-probe_post20} and \ref{fig:sae-probe_post31} show analogous results when sampling SAE activations after the activation function. Sampling after the activation reduces the number of relevant features our probe finds, since many features are inactive. However, this does not change the overall results, as we still find features with good generalization performance. 

Figure~\ref{fig:res-probe-all} shows the probing accuracy for the residual stream linear probe for both Layer 20 and 31. The evaluation is repeated across 10 seeds with different training set splits. While the SAE features, as part of the pre-trained autoencoder model, do not heavily depend on the probing dataset, this is not necessarily true for the residual stream probe. The's probe performance across the out-of-distribution datasets varies strongly, indicating that the generalization performance heavily depends on the minor differences in the training data. 

\subsection{Prompt variations}

\begin{figure*}[t]
    \centering
    \includegraphics[width=0.8\textwidth,trim={0 1.4cm 0 3.4cm},clip]{figures/sae_feature_accuracies_layer31_pre_SQUAD_train_variation.png}
    \caption{Performance of top SAE features and the residual stream linear probe on variations of prompt used with the SQuAD dataset (layer 31, pre-activation).}
    \label{fig:prompt-variation}
\end{figure*}

\begin{table*}[h]
    \centering
    \begin{tabular}{lp{10cm}}
        \toprule
        Default & Given the following passage and question, answer the question:\newline Passage: \{passage\}\newline Question: \{question\} \\
        \midrule
        Variation 1 & Please read this passage and respond to the query that follows:\newline Passage: \{passage\}\newline Question: \{question\} \\
        \midrule
        Variation 2 & Based on the text below, please address the following question:\newline Text: \{passage\}\newline Question: \{question\} \\
        \midrule
        Variation 3 & Consider the following excerpt and respond to the inquiry:\newline Excerpt: \{passage\}\newline Inquiry: \{question\} \\
        \midrule
        Variation 4 & Review this content and answer the question below:\newline Content: \{passage\}\newline Question: \{question\} \\
        \midrule
        Variation 5 & Using the information provided, respond to the following:\newline Information: \{passage\}\newline Query: \{question\} \\
        \bottomrule
    \end{tabular}
    \caption{SQuAD prompt template variations.}
    \label{table:variations}
\end{table*}

% \begin{table*}[h]
%     \centering
%     \begin{tabular}{|l|p{10cm}|}
%         \hline
%         Default & Given the following passage and question, answer the question:\newline Passage: \{passage\}\newline Question: \{question\}\\
%         \hline
%         Variation 1 & Please read this passage and respond to the query that follows:\newline Passage: \{passage\}\newline Question: \{question\} \\
%         \hline
%         Variation 2 & Based on the text below, please address the following question:\newline Text: \{passage\}\newline Question: \{question\} \\
%         \hline
%         Variation 3 & Consider the following excerpt and respond to the inquiry:\newline Excerpt: \{passage\}\newline Inquiry: \{question\} \\
%         \hline
%         Variation 4 & Review this content and answer the question below:\newline Content: \{passage\}\newline Question: \{question\} \\
%         \hline
%         Variation 5 & Using the information provided, respond to the following:\newline Information: \{passage\}\newline Query: \{question\} \\
%         \hline
%     \end{tabular}
%     \caption{SQuAD prompt template variations.}
%     \label{table:variations}
% \end{table*}

We investigated if the SAE features or the residual stream probes are sensitive to small variations in the prompt. To evaluate this question, we created five variations of the prompt template used for the SQuAD training data (see Table~\ref{table:variations}). The results can be found in Figure~\ref{fig:prompt-variation}, and indicate neither the residual stream probe nor the SAE features are sensitive to this kind of variation. 

\subsection{In-domain SAE feature accuracies}

\begin{figure*}[t]
    \centering
    \includegraphics[width=0.8\textwidth,trim={0 1.4cm 0 3.4cm},clip]{figures/sae_top_feature_accuracies_in_domain.png}
    \caption{Performance of the top SAE feature's probing accuracy when training and evaluating features on each dataset individually (pre-activation).}
    \label{fig:top-in-domain}
\end{figure*}


Figure~\ref{fig:top-in-domain} shows the accuracy of 1-sparse SAE feature probes for each dataset individually, demonstrating that each of our contrastive datasets is detectable with a probing accuracy of over 80\%.

\subsection{SAE Feature Combination Analyses} \label{app:feature-combinations}
% \begin{figure*}[t]
%     \centering
%     \includegraphics[width=0.8\textwidth,trim={0 2.2cm 0 3.4cm},clip]{figures/avg_feature_k.png}
%     \caption{Average accuracy vs.\ number of features. (Potentially add the probe line, etc.)}
%     \label{fig:sae-combi-probe}
% \end{figure*}

\begin{figure*}[t]
    \centering
    \includegraphics[width=0.8\textwidth,trim={0 1.4cm 0 3.4cm},clip]{figures/top_sae_feature_group_accuracies_k_L31_pre.png}
    \caption{Performance of top feature combinations (layer 31, pre-activation).}
    \label{fig:top-combis}
\end{figure*}

Figure~\ref{fig:top-combis} shows additional probing analysis for the best performing groups of SAE features up to a group size of five. Group performance is generally dominated by the best performing features and does not majorly exceed the performance of the strongest feature. 

% \subsection{Celebrity Dataset Evaluation}
% \begin{figure*}[t]
%     \centering
%     \includegraphics[width=0.8\textwidth,trim={0 0 0 3.4cm},clip]{figures/celeb.png}
%     \caption{Comparison of top SAE features on the Celebrity dataset. \lh{remove?}}
%     \label{fig:celeb}
% \end{figure*}

\begin{figure*}[t]
    \centering
    \includegraphics[width=0.8\textwidth,trim={0 1.4cm 0 3.4cm},clip]{figures/hierarchical_sae_probe_layer20_pre.png}
    \caption{Accuracies of SAE probes trained on different numbers of SAE features (Layer 20, pre-activation).}
    \label{fig:sae-k-pre20}
\end{figure*}

Figure~\ref{fig:sae-k-pre20} shows additional analysis for SAE feature combinations in Layer~20, analogous to the results for Layer~31 given in Figure~\ref{fig:sae-k-pre31}.

\subsection{Cosine Similarities}
\begin{figure*}[t]
    \centering
    \includegraphics[width=0.8\textwidth,trim={0 0 0 3.4cm},clip]{figures/sae_probe_similarities_by_layer.png}
    \caption{Absolute cosine similarities of top 10 SAE features at different layers, compared with the residual stream probe.}
    \label{fig:similarity_k}
\end{figure*}

We conducted an additional similarity analysis for the top SAE feature groups of different sizes. The results can be found in Figure~\ref{fig:similarity_k} and show a clear trend of larger groups of features becoming more similar to the linear probes. This provides some weak evidence that by default, linear probes might learn more specialized directions that can be represented as a linear combination of more general SAE features. 

% \begin{figure*}[h!]
%     \centering %trim titles
%     \includegraphics[width=.8\textwidth,trim={0 1.4cm 0 3.4cm},clip]{figures/sae_feature_accuracies_layer20_post.png}
%     \caption{Answerability detection accuracies for top SAE features (layer 20, post-activation).}
%     \label{fig:sae-probe_post20}
% \end{figure*}

% \begin{figure*}[h!]
%     \centering %trim titles
%     \includegraphics[width=.8\textwidth,trim={0 1.4cm 0 3.4cm},clip]{figures/sae_feature_accuracies_layer20_pre.png}
%     \caption{Answerability detection accuracies for top SAE features (layer 20, pre-activation).}
%     \label{fig:sae-probe_pre20}
% \end{figure*}

% \begin{figure*}[h!]
%     \centering %trim titles
%     \includegraphics[width=.8\textwidth,trim={0 1.4cm 0 3.4cm},clip]{figures/sae_feature_accuracies_layer31_post.png}
%     \caption{Answerability detection accuracies for top SAE features (layer 31, post-activation).}
%     \label{fig:sae-probe_post31}
% \end{figure*}


% % \begin{figure*}[h!]
% %     \centering %trim titles
% %     \includegraphics[width=.7\textwidth,trim={0 2.2cm 4.2cm 3.4cm},clip]{figures/res_probe.png}
% %     \caption{\vt{can we have less space between the individual plots and make the bars thinner? best so that it fits next to fig 1. also: there's an alternative pic. image.png what's that?}}
% %     \label{fig:res-probe}
% % \end{figure*}

% \begin{figure*}[h!]
%     \centering %trim titles
%     \includegraphics[width=.7\textwidth,trim={0 2.2cm 1cm 3.4cm},clip]{figures/res_probe_20.png}
%     \caption{Linear probe trained on the Layer 20 residual stream. The probe is trained on the SQUAD dataset and then evaluated on the out-of-distribution datasets (IDK, BoolQ, Celebrity, Equation). Error bars show the standard deviation, averaged over 10 bootstrap samples.}
%     \label{fig:res-probe-20}
% \end{figure*}


% \begin{figure*}[h!]
%     \centering
%     \includegraphics[width=.8\textwidth,trim={0 2.2cm 0 3.4cm},clip]{figures/avg_feature_k.png}
%     \caption{\vt{can we add the probe in here?}}
%     \label{fig:sae-combi-probe}
% \end{figure*}

% \begin{figure*}[h!]
%     \centering %trim titles
%     \includegraphics[width=.8\textwidth,trim={0 1.4cm 0 3.4cm},clip]{figures/top_features_k.png}
%     \caption{\vt{later, better make the figures more space efficicient, borders which I cannot crop}}
%     \label{fig:top-combis}
% \end{figure*}

% \begin{figure*}[h!]
%     \centering %trim titles
%     \includegraphics[width=.8\textwidth,trim={0 0 0 3.4cm},clip]{figures/celeb.png}
%     \caption{\vt{can we add the probe directly in here? and maybe remove some of the ones of the right hand side to make it fit one column. maybe we also don't need this plot extra. let's see/discuss}}
%     \label{fig:celeb}
% \end{figure*}

% \begin{figure*}[h!]
%     \centering %trim titles
%     \includegraphics[width=.8\textwidth,trim={0 1.4cm 0 3.4cm},clip]{figures/hierarchical_sae_probe_layer20_pre.png}
%     \caption{Accuracies of SAE probes trained on different numbers of SAE features (Layer 20, pre-activation}
%     \label{fig:sae-k-pre20}
% \end{figure*}

% \begin{figure*}[h!]
%     \centering %trim titles
%     \includegraphics[width=.8\textwidth,trim={0 1.4cm 0 3.4cm},clip]{figures/hierarchical_sae_probe_layer20_post.png}
%     \caption{Accuracies of SAE probes trained on different numbers of SAE features (Layer 20, pre-activation}
%     \label{fig:sae-k-post20}
% \end{figure*}


% \begin{figure*}[h!]
%     \centering %trim titles
%     \includegraphics[width=.8\textwidth,trim={0 1.4cm 0 3.4cm},clip]{figures/hierarchical_sae_probe_layer20_pre.png}
%     \caption{Accuracies of SAE probes trained on different numbers of SAE features (Layer 31, post-activation}
%     \label{fig:sae-k-post31}
% \end{figure*}




% \begin{figure*}[h!]
%     \centering %trim titles
%     \includegraphics[width=.8\textwidth,trim={0 0 0 3.4cm},clip]{figures/sae_probe_similarities_by_layer.png}
%     \caption{Absolute cosine similarities of top 10 SAE features for different number of feature combinations, the model layer of the SAE, and activation hook points, compared with the residual stream probe.}
%     \label{fig:similarity_k}
% \end{figure*}

% \begin{figure*}[h!]
%     \centering %trim titles
%    \includegraphics[width=.4\textwidth,trim={0 2.2cm 1cm 3.4cm},clip]{figures/res_probe_31.png}
% \caption{..%Comparison between top SAE features (left), pre-activation, and linear probes (right) on layer 31.
% }
%     \label{fig:probe_pre31}
% \end{figure*}

% \begin{figure}[h!]
%     \centering %trim titles
%     \includegraphics[width=.5\textwidth,trim={0 1.4cm 0 3.4cm},clip]{figures/sae_top_feature_accuracies_in_domain.png}
%     \caption{Top SAE feature accuracies when training the 1-sparse probes on each dataset individually using a 20\% test split (pre-activation function).}
%     \label{fig:sae-in-domain}
% \end{figure}


% mention why we focus on layer 31 pre -relu





% \myparagraph{Dedicated Prompts} % prompts dedicated to the task
% QA prompt variation


%(only use 80\% of training data here for some reason, probably matching some earlier version of model probes - could rerun overnight).


\myparagraph{Other experiments}
We validated our setup by searching for bias-related features as it was done in related works.
We also experimented with (inofficial) SAEs for an instruction-tuned Llama model, but could not find SAE features with sufficient in-domain probing accuracy. Finally, we also performed analysis on Gemma 2 2B and also the base models, but performance on the answerability task was relatively low in these models (the best SAE features achieved around 70\% probing accuracy).
% This is an appendix.

\end{document}
