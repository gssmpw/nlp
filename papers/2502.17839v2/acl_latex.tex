% This must be in the first 5 lines to tell arXiv to use pdfLaTeX, which is strongly recommended.
\pdfoutput=1
% In particular, the hyperref package requires pdfLaTeX in order to break URLs across lines.

\documentclass[11pt]{article}

% Change "review" to "final" to generate the final (sometimes called camera-ready) version.
% Change to "preprint" to generate a non-anonymous version with page numbers.
\usepackage[preprint]{acl}

% Standard package includes
\usepackage{times}
\usepackage{latexsym}

% For proper rendering and hyphenation of words containing Latin characters (including in bib files)
\usepackage[T1]{fontenc}
% For Vietnamese characters
% \usepackage[T5]{fontenc}
% See https://www.latex-project.org/help/documentation/encguide.pdf for other character sets

% This assumes your files are encoded as UTF8
\usepackage[utf8]{inputenc}

% This is not strictly necessary, and may be commented out,
% but it will improve the layout of the manuscript,
% and will typically save some space.
\usepackage{microtype}

% This is also not strictly necessary, and may be commented out.
% However, it will improve the aesthetics of text in
% the typewriter font.
\usepackage{inconsolata}

%Including images in your LaTeX document requires adding
%additional package(s)
\usepackage{graphicx}

% If the title and author information does not fit in the area allocated, uncomment the following
%
%\setlength\titlebox{<dim>}
%
% and set <dim> to something 5cm or larger.

\usepackage[T1]{fontenc}    % use 8-bit T1 fonts
\usepackage{hyperref}       % hyperlinks
\usepackage{url}            % simple URL typesetting
\usepackage{booktabs}       % professional-quality tables
\usepackage{amsfonts}       % blackboard math symbols
\usepackage{nicefrac}       % compact symbols for 1/2, etc.
\usepackage{microtype}      % microtypography
\usepackage{xcolor}         % colors

\usepackage{inconsolata}
\usepackage{multicol}
\usepackage{float}
\usepackage{subcaption} % For subfigures
\usepackage{caption}

\usepackage{tabularx}

%Including images in your LaTeX document requires adding
%additional package(s)
\usepackage{graphicx}
\usepackage{booktabs}
\usepackage{array}
\usepackage{multirow}
\usepackage{amsmath}

\usepackage{paracol}
\usepackage{listings}
\lstset{breaklines=true}

\usepackage{wrapfig}
\newcommand{\todo}[1]{\textcolor{red}{TODO: #1}}
\newcommand{\edit}[1]{\textcolor{blue}{#1}}
\definecolor{darkgreen}{RGB}{0,100,0}


\title{Say Less, Mean More: \\Leveraging Pragmatics in Retrieval-Augmented Generation}

\author{%
Haris Riaz$^{*}$ \and Ellen Riloff \and Mihai Surdeanu\\
University of Arizona, Tucson, AZ, USA\\
\texttt{\{hriaz,msurdeanu\}@arizona.edu} \quad \texttt{riloff@cs.arizona.edu}
} 

\begin{document}
\maketitle

% Define the star footnote:
\def\thefootnote{*}\footnotetext{Corresponding author.}

% Restore numeric footnotes
\def\thefootnote{\arabic{footnote}}

\begin{abstract}
We propose a simple, unsupervised method that injects pragmatic principles in retrieval-augmented generation (RAG) frameworks such as Dense Passage Retrieval~\cite{karpukhin2020densepassageretrievalopendomain} to enhance the utility of retrieved contexts. 
Our approach first identifies which sentences in a pool of documents retrieved by RAG are most relevant to the question at hand, cover all the topics addressed in the input question and no more, and then highlights these sentences within their context, before they are provided to the LLM, without truncating or altering the context in any other way. We show that this simple idea brings consistent improvements in experiments on three question answering tasks (ARC-Challenge, PubHealth and PopQA) using five different LLMs. It notably enhances relative accuracy by up to 19.7\% on PubHealth and 10\% on ARC-Challenge compared to a conventional RAG system.
\end{abstract}

\section{Introduction}

\label{sec:intro}
Retrieval-augmented generation (RAG)~\cite{lewis2020retrieval} has emerged as a solution to the limited knowledge horizon of large language models (LLMs). RAG combines ``pre-trained parametric and non-parametric memory for language generation,''~\cite{lewis2020retrieval} with the non-parametric memory typically retrieved from large collections of documents. RAG has been shown to dramatically improve the performance of LLMs on various question-answering and reasoning tasks (see section \ref{sec:related-work}). However, we argue that RAG often overwhelms the LLM with too much information, only some of which may be relevant to the task at hand. This contradicts Grice's four maxims of effective communication~\cite{grice1975logic}, which state that the information provided should be ``as much as needed, and no more'' and that it should be ``as clear, as brief'' as possible. The four maxims are enumerated as follows: (1) \textit{Maxim of Quantity}: Provide as much information as needed, but no more; (2) \textit{Maxim of Quality}: Be truthful; avoid giving information that is false or unsupported; (3) \textit{Maxim of Relation}: Be relevant, sharing only information pertinent to the discussion; (4) \textit{Maxim of Manner}: Be clear, brief, and orderly; avoid obscurity and ambiguity. While these maxims were originally formulated in the context of human communication, we argue that they are also applicable in a RAG setting.

We propose a simple, unsupervised method that injects pragmatics in any RAG framework\footnote{Code is available at: \url{https://github.com/hriaz17/SayLessRAG}}. In particular, our method: (a) identifies which sentences in a pool of documents retrieved by RAG are most relevant to the question at hand (maxim of relation), and cover all the topics addressed in the input question and no more (maxim of quantity and manner);\footnote{We envision that the maxim of quality could be considered too by identifying factual statements~\cite{rudinger-etal-2018-neural-models}. We leave this for future work.} and (b) highlights these sentences within their original contexts before they are provided to the LLM. Table~\ref{tab:example} shows an example of our method in action. 

The contributions of our paper are:
{\flushleft {\bf (1)}} We introduce a strategy to introduce pragmatics into any RAG method such as Dense Passage Retrieval~\cite{karpukhin2020densepassageretrievalopendomain}. To our knowledge, we are the first to investigate the impact of pragmatics for RAG.
\vspace{-2mm}
{\flushleft {\bf (2)}} We evaluate the contributions of pragmatics in RAG on three datasets: ARC-Challenge \cite{Clark2018ThinkYH}, PubHealth \cite{kotonya2020explainable} and PopQA \cite{mallen2023llm_memorization} and with five different LLMs ranging from 1B to 7B parameters: Mistral-7B-Instruct-v0.1 \cite{jiang2023mistral7b}, Alpaca-7B \cite{alpaca}, Llama2-7B-chat \cite{touvron2023llama2openfoundation}, Qwen2.5-3B \cite{qwen2.5} and AMD-OLMo-1B-SFT \cite{AMD-OLMo}. Our results indicate that pragmatics helps the most when the QA task primarily involves single-hop or multi-hop logical deduction where the highlighted evidence comprises factual statements that can be sequentially chained to derive the answer. Our post-hoc analysis further shows that this approach fares especially well for queries that benefit from analogical reasoning; with highlighted evidence sentences resembling in-context learning exemplars, proving especially useful for smaller language models with limited reasoning capabilities such as AMD-OLMo-1B-SFT, enabling a 10\% relative improvement on ARC-Challenge for this model.
{\flushleft {\bf (3)}} We find that pragmatics is less effective when the QA task requires arithmetic manipulation, or involves subtleties such as \textit{double negation}. Furthermore, we find that for factoid QA tasks, if a set of ambiguous contexts are first retrieved by DPR for a given query where 
the query lacks disambiguating information and multiple plausible answers could be derived, our method struggles to identify the appropriate evidence sentences for highlighting. In such cases, incorrect evidence highlighting can yield a slight degradation in LLM performance.
{\flushleft {\bf (4)}} Our empirical evidence suggests that our method is complementary when paired with a strong retriever like DPR; in favorable cases it can improve performance by up to 20\%, while exhibiting minimal degradation (approximately 1\%) in less optimal scenarios. Thus, we present it as a low risk and low overhead default augmentation to standard DPR implementations.

\begin{table}
\begin{small}
\begin{tabular}{>{\centering\arraybackslash}m{3mm}|p{0.4\textwidth}}
  \toprule
  \rotatebox[origin=b]{90}{
  %Documents with highlighted evidence
  %~~~~~~} &
  Highlighted evidence
  ~~~~~~~} &
  \vspace{-9mm}
  [\dots] 
  Bats are famous for using echolocation to hunt down their prey, using sonar sounds to capture them in the dark. Another reason for nocturnality is avoiding the heat of the day. \textcolor{blue}{{\bf <evidence>This is especially true in arid biomes like deserts, where nocturnal behavior prevents creatures from losing precious water during the hot, dry daytime.</evidence>}} This is an adaptation that enhances osmoregulation. One of the reasons that (cathemeral) lions prefer to hunt at night is to conserve water.
  \\
  \midrule
  \rotatebox[origin=c]{90}{MCQ~~~~~~~~~~~~~~~~~~~~~} & 
  \vspace{-7mm}
  Question: Many desert animals are only active at night. How does being active only at night most help them survive in a hot desert climate?
  \vspace{2mm}
  
  Choices: 
  \vspace{-2mm}
  \begin{enumerate}
    \renewcommand{\labelenumi}{\Alph{enumi}.}
    \item They can see insects that light up at night.
    \vspace{-2mm}
    \item Their bodies lose less water in the cool night air.
    \vspace{-2mm}
    \item They are able to find more plant food by moonlight.
    \vspace{-2mm}
    \item Their bodies absorb sunlight in the daytime while they sleep.
  \end{enumerate} \\
  \bottomrule
\end{tabular}
\caption{\footnotesize Example of a multiple-choice question (MCQ) from the ARC-C dataset \cite{Clark2018ThinkYH} together with a fragment of a supporting document retrieved, in which the relevant evidence is highlighted with ``$<$evidence$>$'' tokens by our pragmatics-inspired algorithm. This evidence highlighting allows the downstream LLM to identify the correct answer (option B).}
\label{tab:example}
\end{small}
\end{table}


%\section{Related Work}
%\label{sec:related-work}

%\subsection{Background}

%Defect detection is critical to ensure the yield of integrated circuit manufacturing lines and reduce faults. Previous research has primarily focused on wafer map data, which engineers produce by marking faulty chips with different colors based on test results. The specific spatial distribution of defects on a wafer can provide insights into the causes, thereby helping to determine which stage of the manufacturing process is responsible for the issues. Although such research is relatively mature, the continual miniaturization of integrated circuits and the increasing complexity and density of chip components have made chip-level detection more challenging, leading to potential risks\cite{ma2023review}. Consequently, there is a need to combine this approach with magnified imaging of the wafer surface using scanning electron microscopes (SEMs) to detect, classify, and analyze specific microscopic defects, thus helping to identify the particular process steps where defects originate.

%Previously, wafer surface defect classification and detection were primarily conducted by experienced engineers. However, this method relies heavily on the engineers' expertise and involves significant time expenditure and subjectivity, lacking uniform standards. With the ongoing development of artificial intelligence, deep learning methods using multi-layer neural networks to extract and learn target features have proven highly effective for this task\cite{gao2022review}.

%In the task of defect classification, it is typical to use a model structure that initially extracts features through convolutional and pooling layers, followed by classification via fully connected layers. Researchers have recently developed numerous classification model structures tailored to specific problems. These models primarily focus on how to extract defect features effectively. For instance, Chen et al. presented a defect recognition and classification algorithm rooted in PCA and classification SVM\cite{chen2008defect}. Chang et al. utilized SVM, drawing on features like smoothness and texture intricacy, for classifying high-intensity defect images\cite{chang2013hybrid}. The classification of defect images requires the formulation of numerous classifiers tailored for myriad inspection steps and an Abundance of accurately labeled data, making data acquisition challenging. Cheon et al. proposed a single CNN model adept at feature extraction\cite{cheon2019convolutional}. They achieved a granular classification of wafer surface defects by recognizing misclassified images and employing a k-nearest neighbors (k-NN) classifier algorithm to gauge the aggregate squared distance between each image feature vector and its k-neighbors within the same category. However, when applied to new or unseen defects, such models necessitate retraining, incurring computational overheads. Moreover, with escalating CNN complexity, the computational demands surge.

%Segmentation of defects is necessary to locate defect positions and gather information such as the size of defects. Unlike classification networks, segmentation networks often use classic encoder-decoder structures such as UNet\cite{ronneberger2015u} and SegNet\cite{badrinarayanan2017segnet}, which focus on effectively leveraging both local and global feature information. Han Hui et al. proposed integrating a Region Proposal Network (RPN) with a UNet architecture to suggest defect areas before conducting defect segmentation \cite{han2020polycrystalline}. This approach enables the segmentation of various defects in wafers with only a limited set of roughly labeled images, enhancing the efficiency of training and application in environments where detailed annotations are scarce. Subhrajit Nag et al. introduced a new network structure, WaferSegClassNet, which extracts multi-scale local features in the encoder and performs classification and segmentation tasks in the decoder \cite{nag2022wafersegclassnet}. This model represents the first detection system capable of simultaneously classifying and segmenting surface defects on wafers. However, it relies on extensive data training and annotation for high accuracy and reliability. 

%Recently, Vic De Ridder et al. introduced a novel approach for defect segmentation using diffusion models\cite{de2023semi}. This approach treats the instance segmentation task as a denoising process from noise to a filter, utilizing diffusion models to predict and reconstruct instance masks for semiconductor defects. This method achieves high precision and improved defect classification and segmentation detection performance. However, the complex network structure and the computational process of the diffusion model require substantial computational resources. Moreover, the performance of this model heavily relies on high-quality and large amounts of training data. These issues make it less suitable for industrial applications. Additionally, the model has only been applied to detecting and segmenting a single type of defect(bridges) following a specific manufacturing process step, limiting its practical utility in diverse industrial scenarios.

%\subsection{Few-shot Anomaly Detection}
%Traditional anomaly detection techniques typically rely on extensive training data to train models for identifying and locating anomalies. However, these methods often face limitations in rapidly changing production environments and diverse anomaly types. Recent research has started exploring effective anomaly detection using few or zero samples to address these challenges.

%Huang et al. developed the anomaly detection method RegAD, based on image registration technology. This method pre-trains an object-agnostic registration network with various images to establish the normality of unseen objects. It identifies anomalies by aligning image features and has achieved promising results. Despite these advancements, implementing few-shot settings in anomaly detection remains an area ripe for further exploration. Recent studies show that pre-trained vision-language models such as CLIP and MiniGPT can significantly enhance performance in anomaly detection tasks.

%Dong et al. introduced the MaskCLIP framework, which employs masked self-distillation to enhance contrastive language-image pretraining\cite{zhou2022maskclip}. This approach strengthens the visual encoder's learning of local image patches and uses indirect language supervision to enhance semantic understanding. It significantly improves transferability and pretraining outcomes across various visual tasks, although it requires substantial computational resources.
%Jeong et al. crafted the WinCLIP framework by integrating state words and prompt templates to characterize normal and anomalous states more accurately\cite{Jeong_2023_CVPR}. This framework introduces a novel window-based technique for extracting and aggregating multi-scale spatial features, significantly boosting the anomaly detection performance of the pre-trained CLIP model.
%Subsequently, Li et al. have further contributed to the field by creating a new expansive multimodal model named Myriad\cite{li2023myriad}. This model, which incorporates a pre-trained Industrial Anomaly Detection (IAD) model to act as a vision expert, embeds anomaly images as tokens interpretable by the language model, thus providing both detailed descriptions and accurate anomaly detection capabilities.
%Recently, Chen et al. introduced CLIP-AD\cite{chen2023clip}, and Li et al. proposed PromptAD\cite{li2024promptad}, both employing language-guided, tiered dual-path model structures and feature manipulation strategies. These approaches effectively address issues encountered when directly calculating anomaly maps using the CLIP model, such as reversed predictions and highlighting irrelevant areas. Specifically, CLIP-AD optimizes the utilization of multi-layer features, corrects feature misalignment, and enhances model performance through additional linear layer fine-tuning. PromptAD connects normal prompts with anomaly suffixes to form anomaly prompts, enabling contrastive learning in a single-class setting.

%These studies extend the boundaries of traditional anomaly detection techniques and demonstrate how to effectively address rapidly changing and sample-scarce production environments through the synergy of few-shot learning and deep learning models. Building on this foundation, our research further explores wafer surface defect detection based on the CLIP model, especially focusing on achieving efficient and accurate anomaly detection in the highly specialized and variable semiconductor manufacturing process using a minimal amount of labeled data.

\begin{figure*}[t]
    \centering
    \includegraphics[width=\textwidth, height=0.3\textheight, keepaspectratio]
    {./latex/figures/diagram-new.png}
    \caption{\footnotesize Our proposed method. Each query is concatenated with a more abstract \textit{Step-back} version of itself synthesized by a \textit{Step-back} LLM. This new query is used initiate multi-hop retrieval where in each hop the query is aligned with passages retrieved by DPR to select one evidence sentence. These sentences are aggregated across hops with alignment at each hop driven by query reformulation based on \textit{missing information} (maxim of relation) between the current set of selected evidence sentences and current query. After all query keywords are covered by the retrieved evidences (maxim of quantity), our method highlights them within their original contexts and provides them to the LLM.}
    \label{fig:SayLessRAG-diagram}
\end{figure*}

\begin{table*}[h]
  \centering
  \footnotesize
    \resizebox{\linewidth}{!}{
    \setlength{\tabcolsep}{3.5mm}{
    \begin{tabular}{llll}
    \toprule
    \textbf{\footnotesize Settings} & \textbf{\footnotesize ARC-C} & \textbf{\footnotesize PubHealth} & \textbf{\footnotesize PopQA} \\
    \midrule
    \textit{\footnotesize No Retrieval} \vspace{1mm} &       &       &  \\
    \footnotesize \quad Mistral-7B-Instruct \vspace{1mm} & \footnotesize 62.39 (\textcolor{darkgreen}{+6.72\%}) & \footnotesize 74.82 (\textcolor{darkgreen}{+0.96\%}) & \footnotesize 32.52 (\textcolor{red}{-49.73\%}) \\
    \footnotesize \quad Alpaca-7B \vspace{1mm} & \footnotesize 34.02 (\textcolor{red}{-17.43\%}) & \footnotesize 43.25 (\textcolor{red}{-7.78\%}) & \footnotesize 30.24 (\textcolor{red}{-53.04\%}) \\
    \footnotesize \quad Llama2-7B \vspace{1mm} & \footnotesize 40.94 (\textcolor{red}{-9.78\%}) & \footnotesize 68.02 (\textcolor{darkgreen}{+10.57\%}) & \footnotesize 23.73 (\textcolor{red}{-64.07\%}) \\
    \footnotesize \quad Qwen-2.5-3B \vspace{1mm} & \footnotesize \textbf{78.12} (\textcolor{darkgreen}{+7.28\%}) & \footnotesize 65.89 (\textcolor{red}{-7.15\%}) & \footnotesize 26.38 (\textcolor{red}{-62.39\%}) \\
    \footnotesize \quad AMD-OLMo-1B-SFT \vspace{1mm} & \footnotesize 25.81 (\textcolor{red}{-0.17\%}) & \footnotesize 60.81 (\textcolor{darkgreen}{+0.00\%}) & \footnotesize 33.38 (\textcolor{red}{-44.14\%}) \\
    \midrule
    \textit{\footnotesize DPR (No Evidence Highlighting)} \vspace{1mm} &       &       &  \\
    \footnotesize \quad Mistral-7B-Instruct \vspace{1mm} & \footnotesize 58.46 & \footnotesize 74.11 & \footnotesize 64.69 \\
    \footnotesize \quad Alpaca-7B \vspace{1mm} & \footnotesize 41.20 & \footnotesize 46.90 & \footnotesize 64.40 \\
    \footnotesize \quad Llama2-7B-chat \vspace{1mm} & \footnotesize 45.38 & \footnotesize 61.52 & \footnotesize 66.05 \\
    \footnotesize \quad Qwen-2.5-3B \vspace{1mm} & \footnotesize 73.33 & \footnotesize 70.96 & \footnotesize 75.48 \\
    \footnotesize \quad AMD-OLMo-1B-SFT \vspace{1mm} & \footnotesize 25.64 & \footnotesize 60.81 & \footnotesize 59.76 \\
    \midrule
    \textit{\footnotesize DPR + Evidence Highlighting + No Step-back} \vspace{1mm} &       &       &  \\
    \footnotesize \quad Mistral-7B-Instruct \vspace{1mm} & \footnotesize 59.23 (\textcolor{darkgreen}{+1.32\%}) & \footnotesize 76.04 (\textcolor{darkgreen}{+2.60\%}) & \footnotesize 63.90 (\textcolor{red}{-1.22\%}) \\
    \footnotesize \quad Alpaca-7B \vspace{1mm} & \footnotesize 41.28 (\textcolor{darkgreen}{+0.19\%}) & \footnotesize 50.56 (\textcolor{darkgreen}{+7.80\%}) & \footnotesize 63.83 (\textcolor{red}{-0.89\%}) \\
    \footnotesize \quad Llama2-7B-chat \vspace{1mm} & \footnotesize 47.44 (\textcolor{darkgreen}{+4.54\%}) & \footnotesize \footnotesize 62.64 (\textcolor{darkgreen}{+1.82\%}) & \footnotesize 65.98 (\textcolor{red}{-0.10\%}) \\
    \footnotesize \quad Qwen-2.5-3B \vspace{1mm} & \footnotesize 73.25 (\textcolor{red}{-0.11\%}) & \footnotesize 71.17 (\textcolor{darkgreen}{0.3\%}) & \footnotesize \textbf{77.34} (\textcolor{darkgreen}{+2.46\%}) \\
    \footnotesize \quad AMD-OLMo-1B-SFT \vspace{1mm} & \footnotesize 28.21 (\textcolor{darkgreen}{+10.02\%}) & \footnotesize 61.02 (\textcolor{darkgreen}{+0.35\%}) & \footnotesize 60.54 (\textcolor{darkgreen}{+1.31\%}) \\
    \midrule
    \textit{\footnotesize DPR + Evidence Highlighting + Step-back} \vspace{1mm} &       &       &  \\
    \footnotesize \quad Mistral-7B-Instruct & \footnotesize 59.57 (\textcolor{darkgreen}{+1.90\%}) & \footnotesize \textbf{76.14} (\textcolor{darkgreen}{+2.74\%}) & \footnotesize 64.19 (\textcolor{red}{-0.77\%}) \\
    \footnotesize \quad Alpaca-7B & \footnotesize 41.37 (\textcolor{darkgreen}{+0.41\%}) & \footnotesize 56.14 (\textcolor{darkgreen}{+19.70\%}) & \footnotesize 64.05 (\textcolor{red}{-0.54\%}) \\
    \footnotesize \quad Llama2-7B-chat \vspace{1mm} & \footnotesize 47.95 (\textcolor{darkgreen}{+5.66\%}) & \footnotesize 66.40 (\textcolor{darkgreen}{+7.94\%}) & \footnotesize 65.76 (\textcolor{red}{-0.43\%}) \\
    \footnotesize \quad Qwen-2.5-3B \vspace{1mm} & \footnotesize 73.08 (\textcolor{red}{-0.34\%}) & \footnotesize 70.15 (\textcolor{red}{-1.14\%}) & \footnotesize 77.48 (\textcolor{darkgreen}{+2.65\%}) \\
    \footnotesize \quad AMD-OLMo-1B-SFT \vspace{1mm} & \footnotesize 28.21 (\textcolor{darkgreen}{+10.02\%}) & \footnotesize 62.03 (\textcolor{darkgreen}{+2.01\%}) & \footnotesize 60.47 (\textcolor{darkgreen}{+1.19\%}) \\
    \bottomrule
    \end{tabular}%
    }
    }
    \caption{\footnotesize Our pragmatics driven RAG versus a Standard DPR RAG setup. \textbf{Bold} numbers indicate the best performance among all methods and LLMs for a specific dataset.  Percentage changes relative to the \textit{DPR without Evidence Highlighting} setting are shown in parentheses. Positive changes are highlighted in \textcolor{darkgreen}{green}, negative in \textcolor{red}{red}. In the \textit{No Retrieval} setting, we do not retrieve any documents and test the LLM's parametric knowledge. \textit{DPR (No Evidence Highlighting)} refers to the setting where we provide the top-$K$ passages for each query to the LLM without highlighting any evidence sentences within those passages. In the \textit{DPR + Evidence Highlighting + No Step-back} setting, we provide DPR passages annotated with highlighted evidences using ``$<$evidence$>$'' tokens. The \textit{DPR + Evidence Highlighting + Step-back} setting extends the previous setting by introducing reformulated queries and answer choices using Step-back prompting.
    }
  \label{tab:pragmatics_rag}%
\end{table*}
\begin{table*}[h]
\centering
\footnotesize % Reduce font size for a more compact table
\setlength{\tabcolsep}{3pt} % Reduce space between columns
\renewcommand{\arraystretch}{0.9} % Reduce space between rows
\begin{tabularx}{\textwidth}{>{\raggedright\arraybackslash}p{7cm} *{3}{>{\centering\arraybackslash}X}}
\toprule
\textbf{Dataset and Setting} & \textbf{Llama-2–7B-chat} & \textbf{Alpaca-7B} & \textbf{Mistral-7B-Instruct} \\
\midrule
ARC-C \textit{(Evidences w/ Context)} & 47.95 & 41.37 & 59.57 \\
ARC-C \textit{(Evidences w/o Context)} & 47.69 (\textcolor{red}{-0.54\%}) & 38.03 (\textcolor{red}{-8.07\%}) & 58.29 (\textcolor{red}{-2.14\%}) \\
PubHealth \textit{(Evidences w/ Context)} & 66.40 & 56.14 & 76.14 \\
PubHealth \textit{(Evidences w/o Context)} & 54.82 (\textcolor{red}{-17.44\%}) & 49.34 (\textcolor{red}{-12.11\%}) & 62.23 (\textcolor{red}{-18.27\%}) \\
\bottomrule
\end{tabularx}
\caption{\footnotesize Performance of various models on ARC-C and PubHealth datasets when using highlighted evidences within their original context versus using highlighted evidences while discarding surrounding context. Percentage changes (decreases) are shown in parentheses relative to the full context setting. Using highlighted evidence without its surrounding context can significantly degrade the LLMs QA performance. }
\label{tab:highlighted_justifications}
\end{table*}
\section{Temporal Representation Alignment}
\label{sec:approach}

When training a series of short-horizon goal-reaching and instruction-following tasks, our goal is to learn a representation space such that our policy can generalize to a new (long-horizon) task that can be viewed as a sequence of known subtasks.
We propose to structure this representation space by aligning the representations of states, goals, and language in a way that is more amenable to compositional generalization.

\paragraph{Notation.}
We take the setting of a goal- and language-conditioned MDP $\cM$ with state space $\cS$, continuous action space $\cA \subseteq (0,1)^{d_{\cA}}$, initial state distribution $p_0$, dynamics $\p(s'\mid s,a)$, discount factor $\gamma$, and language task distribution $p_{\ell}$.
A policy $\pi(a\mid s)$ maps states to a distribution over actions. We inductively define the $k$-step (action-conditioned) policy visitation distribution as:
\begin{align*}
    p^{\pi}_{1}(s_{1} \mid s_1, a_{1})
    &\triangleq p(s_1 \mid s_1, a_1),\\
    p^{\pi}_{k+1}(s_{k+1} \mid s_1, a_1)
    &\triangleq \nonumber\\*
      &\mspace{-120mu} \int_{\cA}\int_{\cS} p(s_{k+1} \mid s,a) \dd p^{\pi}_{k}(s \mid s_{1},a_1) \dd
        \pi(a \mid s)\\
    p^{\pi}_{k+t}(s_{k+t} \mid s_t,a_t)
    &\triangleq p^{\pi}(s_{k} \mid s_1, a_1) . \eqmark
        \label{eq:successor_distribution}
\end{align*}
Then, the discounted state visitation distribution can be defined as the distribution over $s^{+}$\llap, the state reached after $K\sim \operatorname{Geom}(1-\gamma)$ steps:
\begin{equation}
    p^{\pi}_{\gamma}(s^{+}  \mid  s,a) \triangleq \sum_{k=0}^{\infty} \gamma^{k} p^{\pi}_{k}(s^{+} \mid s,a).
    \label{eq:discounted_state_visitation}
\end{equation}

We assume access to a dataset of expert demonstrations $\cD = \{\tau_{i},\ell_i\}_{i=1}^{K}$, where each trajectory
\begin{equation}
    \tau_{i} = \{s_{t,i},a_{t,i}\}_{t=1}^{H} \in \cS \times \cA
    \label{eq:trajectory}
\end{equation}
is gathered by an expert policy $\expert$, and is then annotated with $p_{\ell}(\ell_{i} \mid s_{1,i}, s_{H,i})$.
Our aim is to learn a policy $\pi$ that can select actions conditioned on a new language instruction $\ell$.
As in prior work~\citep{walke2023bridgedata}, we handle the continuous action space by representing both our policy and the expert policy as an isotropic Gaussian with fixed variance; we will equivalently write $\pi(a\mid s, \varphi)$ or denote the mode as $\hat{a} = \pi(s,\varphi)$ for a task $\varphi$.

\begin{rebuttal}
    \subsection{Representations for Reaching Distant Goals}
    \label{sec:reaching_goals}

    We learn a goal-conditioned policy $\pi(a\mid s,g)$ that selects actions to reach a goal $g$ from expert demonstrations with behavioral cloning.
    Suppose we directly selected actions to imitate the expert on two trajectories in $\cD$:
    
    \begin{equation}
        \mspace{-100mu}\begin{tikzcd}[remember picture,sep=small]
            s_1 \rar & s_2 \rar  & \ldots \rar & s_{H} \rar & w      \quad \\
            w \rar   & s_1' \rar & \ldots \rar & s_{H}' \rar & g\quad
        \end{tikzcd}
        \begin{tikzpicture}[remember picture,overlay] \coordinate (a) at (\tikzcdmatrixname-1-5.north east);
            \coordinate (b) at (\tikzcdmatrixname-2-5.south east);
            \coordinate (c) at (a|-b);
            \draw[decorate,line width=1.5pt,decoration={brace,raise=3pt,amplitude=5pt}]
        (a) -- node[right=1.5em] {$\tau_{i}\in \cD$} (c); \end{tikzpicture}
        \label{eq:trajectory_diagram}
    \end{equation}
    When conditioned with the composed goal $g$, we would be unable to imitate effectively
        as the composed state-goal $(s,g)$ is jointly out of the training distribution.

    What \emph{would} work for reaching $g$ is to first condition the policy on the intermediate waypoint $w$, then upon reaching $w$, condition on the goal $g$, as the state-goal pairs $(s_{i},w)$, $(w,g)$, and $(s_{i}',g)$ are all in the training distribution.
    If we condition the policy on some intermediate waypoint distribution $p(w)$ (or sufficient statistics thereof) that captures all of these cases, we can stitch together the expert behaviors to reach the goal $g$.

    Our approach is to learn a representation space that captures this ability, so that a GCBC objective used in this space can effectively imitate the expert on the composed task.
     We begin with the goal-conditioned behavioral cloning~\citep{kaelbling1993learning}
        loss $\cL_{\textsc{bc}}^{\phi,\psi,\xi}$ conditioned with waypoints $w$.
    \begin{equation}
        \cL_{\textsc{bc}}\bigl(\{s_{i},a_{i},s_{i}^{+},g_{i}\}_{i=1}^{K}\bigr) = \sum_{{i=1}}^{K} \log \pi\bigl(a_{i} \mid s_{i},\psi(g_{i})\bigr).
        \label{eq:goal_conditioned_bc}
    \end{equation}
    Enforcing the invariance needed to stitch \cref{eq:trajectory_diagram} then reduces to aligning \mbox{$\psi(g) \leftrightarrow \psi(w).$}
    The temporal alignment objective $\phi(s)\leftrightarrow \phi(s^{+})$ accomplishes this indirectly by aligning both $\psi(w)$ and $\psi(g)$ to the shared waypoint representation $\phi(w)$:

    \csuse{color indices}
    \begin{align}
        &\cL_{\textsc{nce}}\bigl(\{s_{i},s_{i}^{+}\}_{i=1}^{K};\phi,\psi\bigr) =
        \log \biggl( {\frac{e^{\phi(s^+_{\i})^{T}\psi(s_{\i})}}{\sum_{{\j=1}}^{K}
                e^{\phi(s^+_{\i})^{T}\psi(s_{\j})}}} \biggr)  \nonumber\\*
                &\mspace{100mu} +
        \sum_{{\j=1}}^{K} \log \biggl( {\frac{e^{\phi(s^+_{\i})^{T}\psi(s_{\i})}}{\sum_{{\i=1}}^{K}
                e^{\phi(s^+_{\i})^{T}\psi(s_{\j})}}} \biggr)
        \label{eq:goal_alignment}
    \end{align}

        
\end{rebuttal}
\subsection{Interfacing with Language Instructions}
\label{sec:language_instructions}

To extend the representations from \cref{sec:reaching_goals} to compositional instruction following with language tasks, we need some way to ground language into the $\psi$ (future state)
representation space.
We use a similar approach to GRIF~\citep{myers2023goal}, which uses an additional CLIP-style \citep{radford2021learning} contrastive alignment loss with an additional pretrained language encoder $\xi$:
\csuse{no color indices}
\begin{align}
    &\cL_{\textsc{nce}}\bigl(\{g_{i},\ell_{i}\}_{i=1}^{K};\psi,\xi\bigr)
    = \sum_{{i=1}}^{K} \log \biggl( {\frac{e^{\psi(g_{\i})^{T}\xi(\ell_{\i})}}{\sum_{{\j=1}}^{K}
            e^{\psi(g_{\i})^{T}\xi(\ell_{\j})}}} \biggr)  \nonumber\\*
            &\mspace{100mu} +
    \sum_{{\j=1}}^{K} \log \biggl( {\frac{e^{\psi(g_{\i})^{T}\xi(\ell_{\i})}}{\sum_{{\i=1}}^{K}
            e^{\psi(g_{\i})^{T}\xi(\ell_{\j})}}} \biggr)
    \label{eq:task_alignment}
\end{align}

\subsection{Temporal Alignment}
\label{sec:temporal_alignment}

Putting together the objectives from \cref{sec:reaching_goals,sec:language_instructions} yields the Temporal Representation Alignment (\Method) approach.
\Method{} structures the representation space of goals and language instructions to better enable compositional generalization.
We learn encoders $\phi, \psi ,$ and $\xi$ to map states, goals, and language instructions to a shared representation space.

\csuse{color indices}
\begin{align}
    \cL_{\textsc{nce}} \label{eq:NCE}
    &(\{x_{i}, y_{i}\}_{i=1}^{K};f,h) =
        \sum_{{\i=1}}^{K} \log \biggl( {\frac{e^{f(y_{\i})^{T}h(x_{\i})}}{\sum_{{\j=1}}^{K}
        e^{f(y_{\i})^{T}h(x_{\j})}}} \biggr) \nonumber\\*
      &\mspace{100mu} +
        \sum_{{\j=1}}^{K} \log \biggl( {\frac{e^{f(y_{\i})^{T}h(x_{\i})}}{\sum_{{\i=1}}^{K}
        e^{f(y_{\i})^{T}h(x_{\j})}}} \biggr) \\
    \cL_{\textsc{bc}} \label{eq:BC}
    &\bigl(\{s_{i},a_{i},s^{+}_{i},\ell_{i}\}_{i=1}^{K};\pi,\psi,\xi\bigr) = \nonumber\\*
      &\mspace{-10mu} \sum_{{i=1}}^{K} \log
        \pi\bigl(a_{i} \mid s_{i},\xi(\ell_{i})\bigr) + \log \pi\bigl(a_{i} \mid
        s_{i},\psi(s^{+}_{i})\bigr) \\
    \cL_{\textsc{tra}}
    &\label{eq:TRA} \bigl( \{s_{i},a_{i},s_{i}^{+},g_{i},\ell_{i}\}_{i=1}^{K}; \pi,\phi,\psi,\xi\bigr)
        \\
    &= \underbrace{\cL_{\textsc{bc}}\bigl(\{s_{i},a_{i},s_{i}^{+},\ell_{i}\}_{i=1}^{K};\pi,\psi,\xi\bigr)}_{\text{behavioral
    cloning}} \nonumber\\*
    &+
        \underbrace{\cL_{\textsc{nce}}\bigl(\{s_{i},s_{i}^{+}\}_{i=1}^{K};\phi,\psi\bigr)}_{\text{temporal alignment}}
        + \underbrace{\cL_{\textsc{nce}}\bigl(\{g_{i},\ell_{i}\}_{i=1}^{K};\psi,\xi\bigr)}_{\text{task alignment}} \nonumber
\end{align}Note that the NCE alignment loss uses a CLIP-style symmetric contrastive objective~\citep{radford2021learning,eysenbach2024inference} \-- we highlight the indices in the NCE alignment loss~\eqref{eq:NCE} for clarity.

Our overall objective is to minimize \cref{eq:TRA} across states, actions, future states, goals, and language tasks within the training data:
\begin{align}
    &\min_{\pi,\phi,\psi,\xi} \mathbb{E}_{\substack{(s_{1,i},a_{1,i},\ldots,s_{H,i},a_{H,i},\ell) \sim \mathcal{D} \\
    i\sim\operatorname{Unif}(1\ldots H) \\
    k\sim\operatorname{Geom}(1-\gamma)}} \\*
    &\mspace{10mu}
    \Bigl[\cL_{\text{TRA}}\bigl(\{s_{t,i},a_{t,i},s_{\min(t+k,H),i},s_{H,i},\ell\}_{i=1}^{K};\pi,\phi,\psi,\xi\bigr)\Bigr].
    \label{eq:overall_objective}
\end{align}

\begin{algorithm}
    \caption{Temporal Representation Alignment}
    \label{alg:tra}
    \begin{algorithmic}[1]
        \State \textbf{input:} dataset $\mathcal{D} = (\{s_{t,i},a_{t,i}\}_{t=1}^{H},\ell_i)_{i=1}^N$
        \State initialize networks $\Theta \triangleq (\pi,\phi,\psi,\xi)$
        \While{training}
        \State sample batch $\bigl\{(s_{t,i},a_{t,i},s_{t+k,i},\ell_i)\bigr\}_{i=1}^K\sim\mathcal{D}$ \\
        \hspace*{2ex} for $k\sim\operatorname{Geom}(1-\gamma)$
        \State $\Theta \gets \Theta - \alpha \nabla_{\Theta} \cL_{\text{TRA}}\bigl(\{s_{t,i},a_{t,i},s_{t+k,i},\ell_i\}_{i=1}^K; \Theta\bigr)$
        \EndWhile
        \smallskip
        \State \textbf{output:} \parbox[t]{\linewidth}{language-conditioned policy $\pi(a_{t} | s_{t}, \xi(\ell))$ \\
            goal-conditioned policy $\pi(a_{t} | s_{t}, \psi(g))$
        }
    \end{algorithmic}
\end{algorithm}

\subsection{Implementation}
\label{sec:implementation}

A summary of our approach is shown in \cref{alg:tra}.
In essence, TRA learns three encoders: $\phi$, which encodes states, $\psi$ which encodes future goals, and $\xi$ which encodes language instructions.
Contrastive losses are used to align state representations $\phi(s_{t})$ with future goal representations $\psi(s_{t+k})$, which are in turn aligned with equivalent language task specifications $\xi(\ell)$ when available.
We then learn a behavior cloning policy $\pi$ that can be conditioned on either the goal or language instruction through the representation $\psi(g)$ or $\xi(\ell)$, respectively.

\begin{rebuttal}
    \subsection{Temporal Alignment and Compositionality}
    \label{sec:compositionality}

    We will formalize the intuition from \cref{sec:reaching_goals} that \Method{} enables compositional generalization by considering the error on a ``compositional'' version of $\cD,$ denoted $\cD^{*}$.
    Using the notation from \cref{eq:trajectory}, we can say $\cD$ is distributed according to:
    \begin{align}
        &\cD \triangleq \cD^{H} \sim \prod_{i=1}^{K} p_0(s_{1,i}) p_{\ell}(\ell_{i} \mid s_{1,i}, s_{H,i})
            \nonumber\\*
          &\mspace{60mu} \prod_{t=1}^{H} \expert(a_{t,i} \mid s_{t,i}) \p(s_{t+1,i} \mid s_{t,i}, a_{t,i}) ,
            \label{eq:dataset_distribution}
    \end{align}
    or equivalently
    \begin{align}
            &\cD^{H} \sim \prod_{i=1}^{K} p_0(s_{1,i}) p_{\ell}(\ell_{i} \mid s_{1,i}, s_{H,i}) \nonumber\\*
            &\mspace{60mu} \prod_{t=1}^{H}
            e^{\sigma^2\|\expert(s_{t,i}) - a_{t,i}\|^2}\p(s_{t+1,i} \mid s_{t,i}, a_{t,i}) ,
            \label{eq:dataset_distribution_2}
    \end{align}
    by the isotropic Gaussian assumption.
    We will define $\cD^{*} \triangleq \cD^{H'}$ to be a longer-horizon version of $\cD$ extending the behaviors gathered under $\expert$ across a horizon $\alpha H \ge H' \ge H$ that additionally satisfies a ``time-isotropy'' property: the marginal distribution of the states is uniform across the horizon, i.e., $p_0(s_{1,i}) = p_0(s_{t,i})$ for all $t \in \{1\ldots H'\}$.

    We will relate the in-distribution imitation error $\textsc{Err}(\bullet; \cD)$ to the compositional out-of-distribution imitation error $\textsc{Err}(\bullet;\cD^{*})$.
    We define
    \begin{align}
        \textsc{Err}(\hat{\pi}; \tilde{\cD})
        &= \E_{\tilde{\cD}}\Bigl[\frac{1}{H}\sum_{t=1}^{H} \mathbb{E}_{\hat{\pi}}\left[\|\tilde{a}_{t,i} -
        \hat{\pi}(\tilde{s}_{t,i}, \tilde{s}_{H, i})\|^{2}/d_{\cA}\right]\Bigr] \nonumber\\
        &\quad \text{for} \quad \{\tilde{s}_{t,i},\tilde{a}_{t,i},\tilde{\ell}_{i}\}_{t=1}^{H} \sim
            \tilde{\cD}.
            \label{eq:imitation_error}
    \end{align}
    On the training dataset this is equivalent to the expected behavioral cloning loss from \cref{eq:BC}.

                            
    \begin{assumption}
        \label{asm:policy_factorization}
        The policy factorizes through inferred waypoints as:
\begin{align}
    &\textrm{goals: }\pi(a \mid s, g)
        = \nonumber\\*
      &\mspace{50mu} \int \pi(a\mid s, w) \p(s_{t}=w \mid s_{t+k}=g) \dd{w}
        \label{eq:goal-conditioned} \\
    &\textrm{language: } \pi(a \mid s, \ell)
        = \int \pi(a\mid s, w) \nonumber\\*
      &\mspace{20mu} \p(s_{t}=w \mid s_{t+k}=g) \p(s_{t+k}=g \mid \ell) \dd{w} \dd{g} ,
        \label{eq:language-conditioned}
        \end{align}
        where denote by $\pi(s,g)$ the MLE estimate of the action $a$.

    \end{assumption}

    \makerestatable
    \begin{theorem}
        \label{thm:compositionality}
        Suppose $\cD$ is distributed according to \cref{eq:dataset_distribution} and $\cD^{*}$ is distributed according to \cref{eq:dataset_distribution}.
        When $\gamma > 1-1/H$ and $\alpha > 1$, for optimal features $\phi$ and $\psi$ under \cref{eq:overall_objective}, we have
        \begin{gather}
            \textsc{Err}(\pi; \cD^{*}) \le \textsc{Err}(\pi; \cD) +  \frac{\alpha -1}{2 \alpha }+\Bigl(\frac{ \alpha - 2 }{2\alpha}\Bigr) \1 \{\alpha >2\}  .
            \label{eq:compositionality}
        \end{gather}
    \end{theorem}

    We can also define a notion of the language-conditioned compositional generalization error:
    \begin{equation*}
        \errl(\pi; \cD^{*}) \triangleq \E_{\cD^{*}}\Bigl[\frac{1}{H}\sum_{t=1}^{H}
            \mathbb{E}_{\pi}\bigl[\|\tilde{a}_{t,i} - \pi(\tilde{s}_{t,i}, \tilde{\ell}_{i})\|^{2}\bigr]\Bigr].
            \label{eq:language_error}
    \end{equation*}

    \makerestatable
    \begin{corollary}
        \label{thm:language}
        Under the same conditions as \cref{thm:compositionality},
        \begin{equation*}
            \errl(\pi; \cD^{*}) \le \errl(\pi; \cD) +  \frac{\alpha -1}{2 \alpha }+\Bigl(\frac{ \alpha - 2 }{2\alpha}\Bigr) \1 \{\alpha >2\}  .
            \label{eq:compositionality_language}
        \end{equation*}

    \end{corollary}

    The proofs as well as a visualization of the bound are in \cref{app:compositionality}. Policy implementation details can be found in \cref{app:tra_impl}

    
                
        
    \end{rebuttal}

\begin{table*}[h]
  \centering
  \footnotesize
    \resizebox{\linewidth}{!}{
    \setlength{\tabcolsep}{3.5mm}{
    \begin{tabular}{llll}
    \toprule
    \textbf{\footnotesize Settings} & \textbf{\footnotesize ARC-C} & \textbf{\footnotesize PubHealth} & \textbf{\footnotesize PopQA} \\
    \midrule
    \textit{\footnotesize No Retrieval} \vspace{1mm} &       &       &  \\
    \footnotesize \quad Mistral-7B-Instruct \vspace{1mm} & \footnotesize 62.39 (\textcolor{darkgreen}{+9.11\%}) & \footnotesize \textbf{74.82} (\textcolor{darkgreen}{+34.23\%}) & \footnotesize 32.52 (\textcolor{darkgreen}{+18.17\%}) \\
    \footnotesize \quad Alpaca-7B \vspace{1mm} & \footnotesize 34.02 (\textcolor{red}{-16.02\%}) & \footnotesize 43.25 (\textcolor{darkgreen}{+17.05\%}) & \footnotesize 30.24 (\textcolor{red}{-22.66\%}) \\
    \footnotesize \quad Llama2-7B \vspace{1mm} & \footnotesize 40.94 (\textcolor{darkgreen}{+0.22\%}) & \footnotesize 68.02 (\textcolor{darkgreen}{+0.15\%}) & \footnotesize 23.73 (\textcolor{red}{-0.29\%}) \\
    \footnotesize \quad Qwen-2.5-3B \vspace{1mm} & \footnotesize \textbf{78.12} (\textcolor{darkgreen}{+6.30\%}) & \footnotesize 65.89 (\textcolor{darkgreen}{+51.30\%}) & \footnotesize 26.88 (\textcolor{darkgreen}{+0.83\%}) \\
    \footnotesize \quad AMD-OLMo-1B-SFT \vspace{1mm} & \footnotesize 25.81 (\textcolor{darkgreen}{+0.00\%}) & \footnotesize 60.81 (\textcolor{darkgreen}{+0.00\%}) & \footnotesize 33.38 (\textcolor{darkgreen}{+4.25\%}) \\
    \midrule
    \textit{\footnotesize BM25 (No Evidence Highlighting)} \vspace{1mm} &       &       &  \\
    \footnotesize \quad Mistral-7B-Instruct \vspace{1mm} & \footnotesize 57.18 & \footnotesize 55.74 & \footnotesize 27.52 \\
    \footnotesize \quad Alpaca-7B \vspace{1mm} & \footnotesize 40.51 & \footnotesize 36.95 & \footnotesize \textbf{39.10} \\
    \footnotesize \quad Llama2-7B \vspace{1mm} & \footnotesize 40.85 & \footnotesize 67.92 & \footnotesize 23.80 \\
    \footnotesize \quad Qwen-2.5-3B \vspace{1mm} & \footnotesize 73.50 & \footnotesize 43.55 & \footnotesize 32.38 \\
    \footnotesize \quad AMD-OLMo-1B-SFT  \vspace{1mm} & \footnotesize 25.81 & \footnotesize 60.81 & \footnotesize 32.02 \\
    \midrule
    \textit{\footnotesize BM25 + Evidence Highlighting + No Step-back} \vspace{1mm} &       &       &  \\
    \footnotesize \quad Mistral-7B-Instruct  \vspace{1mm} & \footnotesize 58.38 (\textcolor{darkgreen}{+2.10\%}) & \footnotesize 62.23 (\textcolor{darkgreen}{+11.64\%}) & \footnotesize 29.16 (\textcolor{darkgreen}{+5.96\%}) \\
    \footnotesize \quad Alpaca-7B \vspace{1mm} & \footnotesize 40.17 (\textcolor{red}{-0.84\%}) & \footnotesize 53.91 (\textcolor{darkgreen}{+45.90\%}) & \footnotesize 37.81 (\textcolor{red}{-3.30\%}) \\
    \footnotesize \quad Llama2-7B \vspace{1mm} & \footnotesize 47.69 (\textcolor{darkgreen}{+16.74\%}) & \footnotesize 62.23 (\textcolor{red}{-8.38\%}) & \footnotesize 33.88 (\textcolor{darkgreen}{+42.35\%}) \\
    \footnotesize \quad Qwen-2.5-3B \vspace{1mm} & \footnotesize 75.13 (\textcolor{darkgreen}{+2.22\%}) & \footnotesize 42.84 (\textcolor{red}{-1.63\%}) & \footnotesize 35.53 (\textcolor{darkgreen}{9.73\%}) \\
    \footnotesize \quad AMD-OLMo-1B-SFT  \vspace{1mm} & \footnotesize 25.13 (\textcolor{red}{-2.63\%}) & \footnotesize 59.39 (\textcolor{red}{-2.33\%}) & \footnotesize 33.10 (\textcolor{darkgreen}{+3.37\%}) \\
    \midrule
    \textit{\footnotesize BM25 + Evidence Highlighting + Step-back} \vspace{1mm} &       &       &  \\
    \footnotesize \quad Mistral-7B-Instruct \vspace{1mm} & \footnotesize 58.72 (\textcolor{darkgreen}{+2.69\%}) & \footnotesize 62.64 (\textcolor{darkgreen}{+12.38\%}) & \footnotesize 29.24 (\textcolor{darkgreen}{+6.25\%}) \\
    \footnotesize \quad Alpaca-7B \vspace{1mm} & \footnotesize 40.00 (\textcolor{red}{-1.26\%}) & \footnotesize 45.69 (\textcolor{darkgreen}{+23.65\%}) & \footnotesize 38.46 (\textcolor{red}{-1.64\%}) \\
    \footnotesize \quad Llama2-7B \vspace{1mm} & \footnotesize 47.61 (\textcolor{darkgreen}{+16.55\%}) & \footnotesize 61.93 (\textcolor{red}{-8.82\%}) & \footnotesize 34.31 (\textcolor{darkgreen}{+44.16\%}) \\
    \footnotesize \quad Qwen-2.5-3B  \vspace{1mm} & \footnotesize 74.62 (\textcolor{darkgreen}{+1.52\%}) & \footnotesize 43.05 (\textcolor{red}{-1.15\%}) & \footnotesize 34.88  (\textcolor{darkgreen}{7.72\%}) \\
    \footnotesize \quad AMD-OLMo-1B-SFT  \vspace{1mm} & \footnotesize 25.38 (\textcolor{red}{-1.67\%}) & \footnotesize 60.61 (\textcolor{red}{-0.33\%}) & \footnotesize 33.02 (\textcolor{darkgreen}{+3.12\%}) \\
    \bottomrule
    \end{tabular}%
    }
    }
    \caption{\footnotesize Our pragmatics driven RAG versus a BM25 RAG setup. \textbf{Bold} numbers indicate the best performance among all methods and LLMs for a specific dataset. Percentage changes relative to the BM25 \textit{without Evidence Highlighting} setting are shown in parentheses. Positive changes are highlighted in \textcolor{darkgreen}{green}, negative in \textcolor{red}{red}. In the \textit{No Retrieval} setting, we do not retrieve any documents and test the LLM's parametric knowledge. 
    \textit{BM25 (No Evidence Highlighting)} refers to the setting where we provide the top-$K$ passages for each query to the LLM without highlighting any evidence sentences within those passages.
    In the \textit{BM25 + Evidence Highlighting + No Step-back setting}, we provide BM25 passages annotated with highlighted evidences using ``$<$evidence$>$'' tokens.  The \textit{BM25 + Evidence Highlighting + Step-back} setting extends the previous setting by introducing reformulated queries and answer choices using Step-back prompting.}
  \label{tab:bm25_results}%
\end{table*}



\begin{table*}[t]
\centering
\fontsize{11pt}{11pt}\selectfont
\begin{tabular}{lllllllllllll}
\toprule
\multicolumn{1}{c}{\textbf{task}} & \multicolumn{2}{c}{\textbf{Mir}} & \multicolumn{2}{c}{\textbf{Lai}} & \multicolumn{2}{c}{\textbf{Ziegen.}} & \multicolumn{2}{c}{\textbf{Cao}} & \multicolumn{2}{c}{\textbf{Alva-Man.}} & \multicolumn{1}{c}{\textbf{avg.}} & \textbf{\begin{tabular}[c]{@{}l@{}}avg.\\ rank\end{tabular}} \\
\multicolumn{1}{c}{\textbf{metrics}} & \multicolumn{1}{c}{\textbf{cor.}} & \multicolumn{1}{c}{\textbf{p-v.}} & \multicolumn{1}{c}{\textbf{cor.}} & \multicolumn{1}{c}{\textbf{p-v.}} & \multicolumn{1}{c}{\textbf{cor.}} & \multicolumn{1}{c}{\textbf{p-v.}} & \multicolumn{1}{c}{\textbf{cor.}} & \multicolumn{1}{c}{\textbf{p-v.}} & \multicolumn{1}{c}{\textbf{cor.}} & \multicolumn{1}{c}{\textbf{p-v.}} &  &  \\ \midrule
\textbf{S-Bleu} & 0.50 & 0.0 & 0.47 & 0.0 & 0.59 & 0.0 & 0.58 & 0.0 & 0.68 & 0.0 & 0.57 & 5.8 \\
\textbf{R-Bleu} & -- & -- & 0.27 & 0.0 & 0.30 & 0.0 & -- & -- & -- & -- & - &  \\
\textbf{S-Meteor} & 0.49 & 0.0 & 0.48 & 0.0 & 0.61 & 0.0 & 0.57 & 0.0 & 0.64 & 0.0 & 0.56 & 6.1 \\
\textbf{R-Meteor} & -- & -- & 0.34 & 0.0 & 0.26 & 0.0 & -- & -- & -- & -- & - &  \\
\textbf{S-Bertscore} & \textbf{0.53} & 0.0 & {\ul 0.80} & 0.0 & \textbf{0.70} & 0.0 & {\ul 0.66} & 0.0 & {\ul0.78} & 0.0 & \textbf{0.69} & \textbf{1.7} \\
\textbf{R-Bertscore} & -- & -- & 0.51 & 0.0 & 0.38 & 0.0 & -- & -- & -- & -- & - &  \\
\textbf{S-Bleurt} & {\ul 0.52} & 0.0 & {\ul 0.80} & 0.0 & 0.60 & 0.0 & \textbf{0.70} & 0.0 & \textbf{0.80} & 0.0 & {\ul 0.68} & {\ul 2.3} \\
\textbf{R-Bleurt} & -- & -- & 0.59 & 0.0 & -0.05 & 0.13 & -- & -- & -- & -- & - &  \\
\textbf{S-Cosine} & 0.51 & 0.0 & 0.69 & 0.0 & {\ul 0.62} & 0.0 & 0.61 & 0.0 & 0.65 & 0.0 & 0.62 & 4.4 \\
\textbf{R-Cosine} & -- & -- & 0.40 & 0.0 & 0.29 & 0.0 & -- & -- & -- & -- & - & \\ \midrule
\textbf{QuestEval} & 0.23 & 0.0 & 0.25 & 0.0 & 0.49 & 0.0 & 0.47 & 0.0 & 0.62 & 0.0 & 0.41 & 9.0 \\
\textbf{LLaMa3} & 0.36 & 0.0 & \textbf{0.84} & 0.0 & {\ul{0.62}} & 0.0 & 0.61 & 0.0 &  0.76 & 0.0 & 0.64 & 3.6 \\
\textbf{our (3b)} & 0.49 & 0.0 & 0.73 & 0.0 & 0.54 & 0.0 & 0.53 & 0.0 & 0.7 & 0.0 & 0.60 & 5.8 \\
\textbf{our (8b)} & 0.48 & 0.0 & 0.73 & 0.0 & 0.52 & 0.0 & 0.53 & 0.0 & 0.7 & 0.0 & 0.59 & 6.3 \\  \bottomrule
\end{tabular}
\caption{Pearson correlation on human evaluation on system output. `R-': reference-based. `S-': source-based.}
\label{tab:sys}
\end{table*}



\begin{table}%[]
\centering
\fontsize{11pt}{11pt}\selectfont
\begin{tabular}{llllll}
\toprule
\multicolumn{1}{c}{\textbf{task}} & \multicolumn{1}{c}{\textbf{Lai}} & \multicolumn{1}{c}{\textbf{Zei.}} & \multicolumn{1}{c}{\textbf{Scia.}} & \textbf{} & \textbf{} \\ 
\multicolumn{1}{c}{\textbf{metrics}} & \multicolumn{1}{c}{\textbf{cor.}} & \multicolumn{1}{c}{\textbf{cor.}} & \multicolumn{1}{c}{\textbf{cor.}} & \textbf{avg.} & \textbf{\begin{tabular}[c]{@{}l@{}}avg.\\ rank\end{tabular}} \\ \midrule
\textbf{S-Bleu} & 0.40 & 0.40 & 0.19* & 0.33 & 7.67 \\
\textbf{S-Meteor} & 0.41 & 0.42 & 0.16* & 0.33 & 7.33 \\
\textbf{S-BertS.} & {\ul0.58} & 0.47 & 0.31 & 0.45 & 3.67 \\
\textbf{S-Bleurt} & 0.45 & {\ul 0.54} & {\ul 0.37} & 0.45 & {\ul 3.33} \\
\textbf{S-Cosine} & 0.56 & 0.52 & 0.3 & {\ul 0.46} & {\ul 3.33} \\ \midrule
\textbf{QuestE.} & 0.27 & 0.35 & 0.06* & 0.23 & 9.00 \\
\textbf{LlaMA3} & \textbf{0.6} & \textbf{0.67} & \textbf{0.51} & \textbf{0.59} & \textbf{1.0} \\
\textbf{Our (3b)} & 0.51 & 0.49 & 0.23* & 0.39 & 4.83 \\
\textbf{Our (8b)} & 0.52 & 0.49 & 0.22* & 0.43 & 4.83 \\ \bottomrule
\end{tabular}
\caption{Pearson correlation on human ratings on reference output. *not significant; we cannot reject the null hypothesis of zero correlation}
\label{tab:ref}
\end{table}


\begin{table*}%[]
\centering
\fontsize{11pt}{11pt}\selectfont
\begin{tabular}{lllllllll}
\toprule
\textbf{task} & \multicolumn{1}{c}{\textbf{ALL}} & \multicolumn{1}{c}{\textbf{sentiment}} & \multicolumn{1}{c}{\textbf{detoxify}} & \multicolumn{1}{c}{\textbf{catchy}} & \multicolumn{1}{c}{\textbf{polite}} & \multicolumn{1}{c}{\textbf{persuasive}} & \multicolumn{1}{c}{\textbf{formal}} & \textbf{\begin{tabular}[c]{@{}l@{}}avg. \\ rank\end{tabular}} \\
\textbf{metrics} & \multicolumn{1}{c}{\textbf{cor.}} & \multicolumn{1}{c}{\textbf{cor.}} & \multicolumn{1}{c}{\textbf{cor.}} & \multicolumn{1}{c}{\textbf{cor.}} & \multicolumn{1}{c}{\textbf{cor.}} & \multicolumn{1}{c}{\textbf{cor.}} & \multicolumn{1}{c}{\textbf{cor.}} &  \\ \midrule
\textbf{S-Bleu} & -0.17 & -0.82 & -0.45 & -0.12* & -0.1* & -0.05 & -0.21 & 8.42 \\
\textbf{R-Bleu} & - & -0.5 & -0.45 &  &  &  &  &  \\
\textbf{S-Meteor} & -0.07* & -0.55 & -0.4 & -0.01* & 0.1* & -0.16 & -0.04* & 7.67 \\
\textbf{R-Meteor} & - & -0.17* & -0.39 & - & - & - & - & - \\
\textbf{S-BertScore} & 0.11 & -0.38 & -0.07* & -0.17* & 0.28 & 0.12 & 0.25 & 6.0 \\
\textbf{R-BertScore} & - & -0.02* & -0.21* & - & - & - & - & - \\
\textbf{S-Bleurt} & 0.29 & 0.05* & 0.45 & 0.06* & 0.29 & 0.23 & 0.46 & 4.2 \\
\textbf{R-Bleurt} & - &  0.21 & 0.38 & - & - & - & - & - \\
\textbf{S-Cosine} & 0.01* & -0.5 & -0.13* & -0.19* & 0.05* & -0.05* & 0.15* & 7.42 \\
\textbf{R-Cosine} & - & -0.11* & -0.16* & - & - & - & - & - \\ \midrule
\textbf{QuestEval} & 0.21 & {\ul{0.29}} & 0.23 & 0.37 & 0.19* & 0.35 & 0.14* & 4.67 \\
\textbf{LlaMA3} & \textbf{0.82} & \textbf{0.80} & \textbf{0.72} & \textbf{0.84} & \textbf{0.84} & \textbf{0.90} & \textbf{0.88} & \textbf{1.00} \\
\textbf{Our (3b)} & 0.47 & -0.11* & 0.37 & 0.61 & 0.53 & 0.54 & 0.66 & 3.5 \\
\textbf{Our (8b)} & {\ul{0.57}} & 0.09* & {\ul 0.49} & {\ul 0.72} & {\ul 0.64} & {\ul 0.62} & {\ul 0.67} & {\ul 2.17} \\ \bottomrule
\end{tabular}
\caption{Pearson correlation on human ratings on our constructed test set. 'R-': reference-based. 'S-': source-based. *not significant; we cannot reject the null hypothesis of zero correlation}
\label{tab:con}
\end{table*}

\section{Results}
We benchmark the different metrics on the different datasets using correlation to human judgement. For content preservation, we show results split on data with system output, reference output and our constructed test set: we show that the data source for evaluation leads to different conclusions on the metrics. In addition, we examine whether the metrics can rank style transfer systems similar to humans. On style strength, we likewise show correlations between human judgment and zero-shot evaluation approaches. When applicable, we summarize results by reporting the average correlation. And the average ranking of the metric per dataset (by ranking which metric obtains the highest correlation to human judgement per dataset). 

\subsection{Content preservation}
\paragraph{How do data sources affect the conclusion on best metric?}
The conclusions about the metrics' performance change radically depending on whether we use system output data, reference output, or our constructed test set. Ideally, a good metric correlates highly with humans on any data source. Ideally, for meta-evaluation, a metric should correlate consistently across all data sources, but the following shows that the correlations indicate different things, and the conclusion on the best metric should be drawn carefully.

Looking at the metrics correlations with humans on the data source with system output (Table~\ref{tab:sys}), we see a relatively high correlation for many of the metrics on many tasks. The overall best metrics are S-BertScore and S-BLEURT (avg+avg rank). We see no notable difference in our method of using the 3B or 8B model as the backbone.

Examining the average correlations based on data with reference output (Table~\ref{tab:ref}), now the zero-shoot prompting with LlaMA3 70B is the best-performing approach ($0.59$ avg). Tied for second place are source-based cosine embedding ($0.46$ avg), BLEURT ($0.45$ avg) and BertScore ($0.45$ avg). Our method follows on a 5. place: here, the 8b version (($0.43$ avg)) shows a bit stronger results than 3b ($0.39$ avg). The fact that the conclusions change, whether looking at reference or system output, confirms the observations made by \citet{scialom-etal-2021-questeval} on simplicity transfer.   

Now consider the results on our test set (Table~\ref{tab:con}): Several metrics show low or no correlation; we even see a significantly negative correlation for some metrics on ALL (BLEU) and for specific subparts of our test set for BLEU, Meteor, BertScore, Cosine. On the other end, LlaMA3 70B is again performing best, showing strong results ($0.82$ in ALL). The runner-up is now our 8B method, with a gap to the 3B version ($0.57$ vs $0.47$ in ALL). Note our method still shows zero correlation for the sentiment task. After, ranks BLEURT ($0.29$), QuestEval ($0.21$), BertScore ($0.11$), Cosine ($0.01$).  

On our test set, we find that some metrics that correlate relatively well on the other datasets, now exhibit low correlation. Hence, with our test set, we can now support the logical reasoning with data evidence: Evaluation of content preservation for style transfer needs to take the style shift into account. This conclusion could not be drawn using the existing data sources: We hypothesise that for the data with system-based output, successful output happens to be very similar to the source sentence and vice versa, and reference-based output might not contain server mistakes as they are gold references. Thus, none of the existing data sources tests the limits of the metrics.  


\paragraph{How do reference-based metrics compare to source-based ones?} Reference-based metrics show a lower correlation than the source-based counterpart for all metrics on both datasets with ratings on references (Table~\ref{tab:sys}). As discussed previously, reference-based metrics for style transfer have the drawback that many different good solutions on a rewrite might exist and not only one similar to a reference.


\paragraph{How well can the metrics rank the performance of style transfer methods?}
We compare the metrics' ability to judge the best style transfer methods w.r.t. the human annotations: Several of the data sources contain samples from different style transfer systems. In order to use metrics to assess the quality of the style transfer system, metrics should correctly find the best-performing system. Hence, we evaluate whether the metrics for content preservation provide the same system ranking as human evaluators. We take the mean of the score for every output on each system and the mean of the human annotations; we compare the systems using the Kendall's Tau correlation. 

We find only the evaluation using the dataset Mir, Lai, and Ziegen to result in significant correlations, probably because of sparsity in a number of system tests (App.~\ref{app:dataset}). Our method (8b) is the only metric providing a perfect ranking of the style transfer system on the Lai data, and Llama3 70B the only one on the Ziegen data. Results in App.~\ref{app:results}. 


\subsection{Style strength results}
%Evaluating style strengths is a challenging task. 
Llama3 70B shows better overall results than our method. However, our method scores higher than Llama3 70B on 2 out of 6 datasets, but it also exhibits zero correlation on one task (Table~\ref{tab:styleresults}).%More work i s needed on evaluating style strengths. 
 
\begin{table}%[]
\fontsize{11pt}{11pt}\selectfont
\begin{tabular}{lccc}
\toprule
\multicolumn{1}{c}{\textbf{}} & \textbf{LlaMA3} & \textbf{Our (3b)} & \textbf{Our (8b)} \\ \midrule
\textbf{Mir} & 0.46 & 0.54 & \textbf{0.57} \\
\textbf{Lai} & \textbf{0.57} & 0.18 & 0.19 \\
\textbf{Ziegen.} & 0.25 & 0.27 & \textbf{0.32} \\
\textbf{Alva-M.} & \textbf{0.59} & 0.03* & 0.02* \\
\textbf{Scialom} & \textbf{0.62} & 0.45 & 0.44 \\
\textbf{\begin{tabular}[c]{@{}l@{}}Our Test\end{tabular}} & \textbf{0.63} & 0.46 & 0.48 \\ \bottomrule
\end{tabular}
\caption{Style strength: Pearson correlation to human ratings. *not significant; we cannot reject the null hypothesis of zero corelation}
\label{tab:styleresults}
\end{table}

\subsection{Ablation}
We conduct several runs of the methods using LLMs with variations in instructions/prompts (App.~\ref{app:method}). We observe that the lower the correlation on a task, the higher the variation between the different runs. For our method, we only observe low variance between the runs.
None of the variations leads to different conclusions of the meta-evaluation. Results in App.~\ref{app:results}.

\begin{figure}[!htb]
  \centering
  % \includegraphics[width=\columnwidth]{latex/figures/dpr_settings_comparison_subplots.png}
    \includegraphics[width=\columnwidth]{latex/figures/dpr_settings_comparison_subplots_granular.png}
  \caption{Performance of Qwen2.5-3B on: (\textbf{top}) ARC-Challenge, \textbf{(middle)} PopQA, (\textbf{bottom}) PubHealth under different \textit{Evidence Highlighting} settings, with varying top-$k$ where $k$ is the number of DPR contexts retrieved.}
  \label{fig:top-k}
\end{figure}

% \vspace{-2mm}

\begin{table*}[h]
\centering
\footnotesize
\begin{tabular}{lcc}
\toprule
\textbf{Category} & \textbf{Frequency (ARC-Challenge)} & \textbf{Frequency (PubHealth)} \\
\midrule
\textbf{Bad} (0) & 6 & 8 \\
\textbf{Medium} (0.5) & 10 & 4 \\
\textbf{Good} (1) & 4 & 8 \\
\bottomrule \\
\end{tabular}
\caption{\footnotesize Highlighted Evidence Quality Scores for 20 randomly sampled queries from the ARC-Challenge and PubHealth datasets. The frequencies represent the number of instances falling into each quality category for the highlighted evidence in both datasets.}
\label{tab:evidence_quality_scores}
\end{table*}

\section{Analysis}
\label{sec:analysis}
In the following sections, we will analyze European type approval regulation\footnote{Strictly speaking, the German enabling act (AFGBV) does not regulate type-approval, but how test \& operating permits are issued for SAE-Level-4 systems. Type-approval regulation for SAE-Level-3 systems follows UN Regulation No. 157 (UN-ECE-ALKS) \parencite{un157}.} regarding the underlying notions of ``safety'' and ``risk''.
We will classify these notions according to their absolute or relative character, underlying risk sources, or underlying concepts of harm.

\subsection{Classification of Safety Notions}
\label{sec:safety-notions}
We will refer to \emph{absolute} notions of safety as conceptualizations that assume the complete absence of any kind of risk.
Opposed to this, \emph{relative} notions of safety are based on a conceptualization that specifically includes risk acceptance criteria, e.g., in terms of ``tolerable'' risk or ``sufficient'' safety.

For classifying notions of safety by their underlying risk (or rather ``hazard'') sources, and different concepts of harm, \Cref{fig:hazard-sources} provides an overview of our reasoning, which is closely in line with the argumentation provided by Waymo in \parencite{favaro2023}.
We prefer ``hazard sources'' over ``risk sources'', as a risk must always be related to a \emph{cause} or \emph{source of harm} (i.e., a hazard \parencite[p.~1, def. 3.2]{iso51}).
Without a concrete (scenario) context that the system is operating in, a hazard is \emph{latent}: E.g., when operating in public traffic, there is a fundamental possibility that a \emph{collision with a pedestrian} leads to (physical) harm for that pedestrian. 
However, only if an automated vehicle shows (potentially) hazardous behavior (e.g., not decelerating properly) \emph{and} is located near a pedestrian (context), the hazard is instantiated and leads to a hazardous event.
\begin{figure*}
    \includeimg[width=.9\textwidth]{hazard-sources0.pdf}
    \caption{Graphical summary of a taxonomy of risk related to automated vehicles, extended based on ISO 21448 (\parencite{iso21448}) and \parencite{favaro2023}. Top: Causal chain from hazard sources to actual harm; bottom: summary of the individual elements' contributions to a resulting risk. Graphic translated from \parencite{nolte2024} \label{fig:hazard-sources}}
\end{figure*}
If the hazardous event cannot be mitigated or controlled, we see a loss event in which the pedestrian's health is harmed.
Note that this hypothetical chain of events is summarized in the definition of risk:
The probability of occurrence of harm is determined by a) the frequency with which hazard sources manifest, b) the time for which the system operates in a context that exposes the possibility of harm, and c) by the probability with which a hazardous event can be controlled.
A risk can then be determined as a function of the probability of harm and the severity of the harm potentially inflicted on the pedestrian.

In the following, we will apply this general model to introduce different types of hazard sources and also different types of harm.
\cref{fig:hazard-sources} shows two distinct hazard sources, i.e., functional insufficiencies and E/E-failures that can lead to hazardous behavior.
ISO~21488 \parencite{iso21448} defines functional insufficiencies as insufficiencies that stem from an incomplete or faulty system specification (specification insufficiencies).
In addition, the standard considers insufficiencies that stem from insufficient technical capability to operate inside the targeted Operational Design Domain (performance insufficiencies).
Functional insufficiencies are related to the ``Safety of the Intended Functionality (SOTIF)'' (according to ISO~21448), ``Behavioral Safety'' (according to Waymo \parencite{waymo2018}), or ``Operational Safety'' (according to UN Regulation No. 157 \parencite{un157}).
E/E-Failures are related to classic functional safety and are covered exhaustively by ISO~26262 \parencite{iso2018}.
Additional hazard sources can, e.g., be related to malicious security attacks (ISO~21434), or even to mechanical failures that should be covered (in the US) in the Federal Motor Vehicle Safety Standards (FMVSS).

For the classification of notions of safety by the related harm, in \parencite{salem2024, nolte2024}, we take a different approach compared to \parencite{koopman2024}:
We extend the concept of harm to the violation of stakeholder \emph{values}, where values are considered to be a ``standard of varying importance among other such standards that, when combined, form a value pattern that reduces complexity for stakeholders [\ldots] [and] determines situational actions [\ldots].'' \parencite{albert2008}
In this sense, values are profound, personal determinants for individual or collective behavior.
The notion of values being organized in a weighted value pattern shows that values can be ranked according to importance.
For automated vehicles, \emph{physical wellbeing} and \emph{mobility} can, e.g., be considered values which need to be balanced to achieve societal acceptance, in line with the discussion of required tradeoffs in \cref{sec:terminology}.
For the analysis of the following regulatory frameworks, we will evaluate if the given safety or risk notions allow tradeoffs regarding underlying stakeholder values. 

\subsection{UN Regulation No. 157 \& European Implementing Regulation (EU) 2022/1426}
\label{sec:enabling-act}
UN Regulation No. 157 \parencite{un157} and the European Implementing Regulation 2022/1426 \parencite{eu1426} provide type approval regulation for automated vehicles equipped with SAE-Level-3 (UN Reg. 157) and Level 4 (EU 2022/1426) systems on an international (UN Reg. 157) and European (EU 2022/1426) level.

Generally, EU type approval considers UN ECE regulations mandatory for its member states ((EU) 2018/858, \parencite{eu858}), while the EU largely forgoes implementing EU-specific type approval rules, it maintains the right to alter or to amend UN ECE regulation \parencite{eu858}.

In this respect, the terminology and conceptualizations in the EU Implementing Act closely follow those in UN Reg. No. 157.
The EU Implementing Act gives a clear reference to UN Reg. No. 157 \parencite[][Preamble,  Paragraph 1]{eu1426}.
Hence, the documents can be assessed in parallel.
Differences will be pointed out as necessary.

Both acts are written in rather technical language, including the formulation of technical requirements (e.g., regarding deceleration values or speeds in certain scenarios).
While providing exhaustive definitions and terminology, neither of both documents provide an actual definition of risk or safety.
The definition of ``unreasonable'' risk in both documents does not define risk, but only what is considered \emph{unreasonable}. It states that the ``overall level of risk for [the driver, (only in UN Reg. 157)] vehicle occupants and other road users which is increased compared to a competently and carefully driven manual vehicle.''
The pertaining notions of safety and risk can hence only be derived from the context in which they are used.

\subsubsection{Absolute vs. Relative Notions of Safety}
In line with the technical detail provided in the acts, both clearly imply a \emph{relative} notion of safety and refer to the absence of \emph{unreasonable} risk throughout, which is typical for technical safety definitions.

Both acts require sufficient proof and documentation that the to-be-approved automated driving systems are ``free of unreasonable safety risks to vehicle occupants and other road users'' for type approval.\footnote{As it targets SAE-Level-3 systems, UN Reg. 157 also refers to the driver, where applicable.}
In this respect, both acts demand that the manufacturers perform verification and validation activities for performance requirements that include ``[\ldots] the conclusion that the system is designed in such a way that it is free from unreasonable risks [\ldots]''.
Additionally, \emph{risk minimization} is a recurring theme when it comes to the definition of Minimum Risk Maneuvers (MRM).

Finally, supporting the relative notions of safety and risk, UN Reg. 157 introduces the concept of ``reasonable foreseeable and preventable'' \parencite[Article 1, Clause 5.1.1.]{un157} collisions, which implies that a residual risk will remain with the introduction of automated vehicles.
\parencite[][Appendix 3, Clause 3.1.]{un157} explicitly states that only \emph{some} scenarios that are unpreventable for a competent human driver can actually be prevented by an automated driving system.
While this concept is not applied throughout the EU Implementing Act, both documents explicitly refer to \emph{residual} risks that are related to the operation of automated driving systems (\parencite[][Annex I, Clause 1]{un157}, \parencite[][Annex II, Clause 7.1.1.]{eu1426}).

\subsubsection{Hazard Sources}
Hazard sources that are explicitly differentiated in UN Reg. 157 and (EU) 2022/1426 are E/E-failures that are in scope of functional safety (ISO~26262) and functional insufficiencies that are in scope of behavioral (or ``operational'') safety (ISO~21448).
Both documents consistently differentiate both sources when formulating requirements.

While the acts share a common definition of ``operational'' safety (\parencite[][Article 2, def. 30.]{eu1426}, \parencite[][Annex 4, def. 2.15.]{un157}), the definitions for functional safety differ.
\parencite{un157} defines functional safety as the ``absence of unreasonable risk under the occurrence of hazards caused by a malfunctioning behaviour of electric/electronic systems [\ldots]'', \parencite{eu1426} drops the specification of ``electric/electronic systems'' from the definition.
When taken at face value, this definition would mean that functional safety included all possible hazard sources, regardless of their origin, which is a deviation from the otherwise precise usage of safety-related terminology.

\subsubsection{Harm Types}
As the acts lack explicit definitions of safety and risk, there is no consistent and explicit notion of different harm types that could be differentiated.

\parencite{un157} gives little hints regarding different considered harm types.
``The absence of unreasonable risk'' in terms of human driving performance could hence be related to any chosen performance metric that allows a comparison with a competent careful human driver including, e.g., accident statistics, statistics about rule violations, or changes in traffic flow.

In \parencite{eu1426}, ``safety'' is, implicitly, attributed to the absence of unreasonable risk to life and limb of humans.
This is supported by the performance requirements that are formulated:
\parencite[][Annex II, Clause 1.1.2. (d)]{eu1426} demands that an automated driving system can adapt the vehicle behavior in a way that it minimizes risk and prioritizes the protection of human life.

Both acts demand the adherence to traffic rules (\parencite[][Annex 2, Clause 1.3.]{eu1426}, \parencite[][Clause 5.1.2.]{un157}).
\parencite[][Annex II, Clause 1.1.2. (c)]{eu1426} also demands that an automated driving system shall adapt its behavior to surrounding traffic conditions, such as the current traffic flow.
With the relative notion of risk in both acts, the unspecific clear statement that there may be unpreventable accidents \parencite{un157}, and a demand of prioritization of human life in \parencite{eu1426}, both acts could be interpreted to allow developers to make tradeoffs as discussed in \cref{sec:terminology}.


\subsubsection{Conclusion}
To summarize, the UN Reg. 157 and the (EU) 2022/1426 both clearly support the technical notion of safety as the absence of unreasonable risk.
The notion is used consistently throughout both documents, providing a sufficiently clear terminology for the developers of automated vehicles.
Uncertainty remains when it comes to considered harm types: Both acts do not explicitly allow for broader notions of safety, in the sense of \parencite{koopman2024} or \parencite{salem2024}.
Finally, a minor weak spot can be seen in the definition of risk acceptance criteria: Both acts take the human driving performance as a baseline.
While (EU) 2022/1426 specifies that these criteria are specific to the systems' Operational Design Domain \parencite[][Annex II, Clause 7.1.1.]{eu1426}, the reference to the concrete Operational Design Domain is missing in UN Reg. 157.
Without a clearly defined notion of safety, however, it remains unclear, how aspects beyond net accident statistics (which are given as an example in \parencite[][Annex II, Clause 7.1.1.]{eu1426}), can be addressed practically, as demanded by \parencite{koopman2024}.

\subsection{German Regulation (StVG \& AFGBV)}
\label{sec:afgbv}
The German L3 (Automated Driving Act) and L4 (Act on Autonomous Driving) Acts from 2017 and 2021,\footnote{Formally, these are amendments to the German Road Traffic Act (StVG): 06/21/2017, BGBl. I p. 1648, 07/12/2021 BGBl. I p. 3108.} respectively, provide enabling regulation for the operation of SAE-Level-3 and 4 vehicles on German roads.
The German Implementing Regulation (\parencite{afgbv}, AFGBV) defines how this enabling regulation is to be implemented for granting testing permits for SAE-Level-3 and -4 and driving permits for SAE-Level-3 and -4 automated driving systems.\footnote{Note that these permits do not grant EU-wide type approval, but serve as a special solution for German roads only. At the same time, the AFGBV has the same scope as (EU) 2022/1426.}
With all three acts, Germany was the first country to regulate the approval of automated vehicles for a domestic market.
All acts are subject to (repeated) evaluation until the year 2030 regarding their impact on the development of automated driving technology.
An assessment of the German AFGBV and comparisons to (EU) 2022/1426 have been given in \cite{steininger2022} in German.

Just as for UN Reg. 157 and (EU) 2022/1426, neither the StVG nor the AFGBV provide a clear definition of ``safety'' or ``risk'' -- even though the "safety" of the road traffic is one major goal of the StVG and StVO.
Again, different implicit notions of both concepts can only be interpreted from the context of existing wording.
An additional complication that is related to the German language is that ``safety'' and ``security'' can both be addressed as ``Sicherheit'', adding another potential source of unclarity.
Literal Quotations in this section are our translations from the German act.

\subsubsection{Absolute vs. Relative Notions of Safety}
For assessing absolute vs. relative notions of safety in German regulation, it should be mentioned that the main goal of the German StVO is to ensure the ``safety and ease of traffic flow'' -- an already diametral goal that requires human drivers to make tradeoffs.\footnote{For human drivers, this also creates legal uncertainty which can sometimes only be settled in a-posteriori court cases.}
While UN and EU regulation clearly shows a relative notion of safety\footnote{And even the StVG contains sections that use wording such as ``best possible safety for vehicle occupants'' (§1d (4) StVG) and acknowledges that there are unavoidable hazards to human life (§1e (2) No. 2c)).}, the German AFGBV contains ambiguous statements in this respect:
Several paragraphs contain a demand for a hazard free operation of automated vehicles.
§4 (1) No. 4 AFGBV, e.g., states that ``the operation of vehicles with autonomous driving functions must neither negatively impact road traffic safety or traffic flow, nor endanger the life and limb of persons.''
Additionally, §6 (1) AFGBV states that the permits for testing and operation have to be revoked, if it becomes apparent that a ``negative impact on road traffic safety or traffic flow, or hazards to the life and limb of persons cannot be ruled out''.
The same wording is used for the approval of operational design domains regulated in §10 (1) No. 1.
A particularly misleading statement is made regarding the requirements for technical supervision instances which are regulated in §14 (3) AFGBV which states that an automated vehicle has to be  ``immediately removed from the public traffic space if a risk minimal state leads to hazards to road traffic safety or traffic flow''.
Considering the argumentation in \cref{sec:terminology}, that residual risks related to the operation of automated driving systems are inevitable, these are strong statements which, if taken at face value, technically prohibit the operation of automated vehicles.
It suggests an \emph{absolute} notion of safety that requires the complete absence of risk.  
The last statement above is particularly contradictory in itself, considering that a risk \emph{minimal} state always implies a residual risk.

In addition to these absolute safety notions, there are passages which suggest a relative notion of safety:
The approval for Operational Design Domains is coupled to the proof that the operation of an automated vehicle ``neither negatively impacts road traffic safety or traffic flow, nor significantly endangers the life and limb of persons beyond the general risk of an impact that is typical of local road traffic'' (§9 (2) No. 3 AFGBV).
The addition of a relative risk measure ``beyond the general risk of an impact'' provides a relaxation (cf. also \cite{steininger2022}, who criticizes the aforementioned absolute safety notion) that also yields an implicit acceptance criterion (\emph{statistically as good as} human drivers) similar to the requirements stated in UN Reg. 157 and (EU) 2022/1426.

Additional hints for a relative notion of safety can be found in Annex 1, Part 1, No. 1.1 and Annex 1, Part 2, No. 10.
Part 1, No 1.1 specifies collision-avoidance requirements and acknowledges that not all collisions can be avoided.\footnote{The same is true for Part 2, No. 10, Clause 10.2.5.}
Part 2, No. 10 specifies requirements for test cases.
It demands that test cases are suitable to provide evidence that the ``safety of a vehicle with an autonomous driving function is increased compared to the safety of human-driven vehicles''.
This does not only acknowledge residual risks, but also yields an acceptance criterion (\emph{better} than human drivers) that is different from the implied acceptance criterion given in §9 (2) No. 3 AFGBV.

\subsubsection{Hazard Sources}
Regarding hazard sources, Annex 1 and 3 AFGBV explicitly refer to ISO~26262 and ISO~21448 (or rather its predecessor ISO/PAS~21448:2019).
However, regarding the discussion of actual hazard sources, the context in which both standards are mentioned is partially unclear:
Annex 1, Clause 1.3 discusses requirements for path and speed planning.
Clause 1.3 d) demands that in intersections, a Time to Collision (TTC) greater than 3 seconds must be guaranteed.
If manufacturers deviate from this, it is demanded that ``state-of-the-art, systematic safety evaluations'' are performed.
Fulfillment of the state of the art is assumed if ``the guidelines of ISO~26262:2018-12 Road Vehicles -- Functional Safety are fulfilled''.
Technically, ISO~26262 is not suitable to define the state of the art in this context, as the requirements discussed fall in the scope of operational (or behavioral) safety (ISO~21448).
A hazard source ``violated minimal time to collision'' is clearly a functional insufficiency, not an E/E-failure.

Similar unclarity presents itself in Annex 3, Clause 1 AFGBV: 
Clause 1 specifies the contents of the ``functional specification''.
The ``specification of the functionality'' is an artifact which is demanded in ISO~21448:2022 (Clause 5.3) \parencite{iso21448}.
However, Annex 3, Clause 1 AFGBV states that the ``functional specification'' is considered to comply to the state of the art, if the ``functional specification'' adheres to ISO~26262-3:2018 (Concept Phase).
Again, this assumes SOTIF-related contents as part of ISO~26262, which introduces the ``Item Definition'' as an artifact, which is significantly different from the ``specification of the functionality'' which is demanded by ISO~21448.
Finally, Annex 3, Clause 3 AFGBV demands a ``documentation of the safety concept'' which ``allows a functional safety assessment''.
A safety concept that is related to operational / behavioral safety is not demanded.
Technically, the unclarity with respect to the addressed harm types lead to the fact that the requirements provided by the AFGBV do not comply with the state of the art in the field, providing questionable regulation.

\subsubsection{Harm Types}
Just like UN Reg. 157 and (EU) 2022/1426, the German StVG and AFGBV do not explicitly differentiate concrete harm types for their notions of safety.
However, the AFGBV mentions three main concerns for the operation of automated vehicles which are \emph{traffic flow} (e.g., §4 (1) No. 4 AFGBV), compliance to \emph{traffic law} (e.g., §1e (2) No. 2 StVG), and the \emph{life and limb of humans} (e.g., §4 (1) No. 4 AFGBV).

Again, there is some ambiguity in the chosen wording:
The conflict between traffic flow and safety has already been argued in \cref{sec:terminology}.
The wording given in §4 (1) No. 4 and §6 (1) AFGBV  demand to ensure (absolute) safety \emph{and} traffic flow at the same time, which is impossible (cf. \cref{sec:terminology}) from an engineering perspective.
§1e (2) No. 2 StVG defines that ``vehicles with an autonomous driving function must [\ldots] be capable to comply to [\ldots] traffic rules in a self-contained manner''.
Taken at face value, this wording implies that an automated driving system could lose its testing or operating permit as soon as it violates a traffic rule.
A way out could be provided by §1 of the German Traffic Act (StVO) which demands careful and considerate behavior of all traffic participants and by that allows judgement calls for human drivers.
However, if §1 is applicable in certain situations is often settled in court cases. 
For developers, the application of §1 StVO during system design hence remains a legal risk.

While there are rather absolute statements as mentioned above, sections of the AFGBV and StVG can be interpreted to allow tradeoffs:
§1e (2) No. 2 b) demands that a system,  ``in case of an inevitable, alternative harm to legal objectives, considers the significance of the legal objectives, where the protection of human life has highest priority''.
This exact wording \emph{could} provide some slack for the absolute demands in other parts of the acts, enabling tradeoffs between (tolerable) risk and mobility as discussed in \cref{sec:terminology}.
However, it remains unclear if this interpretation is legally possible.

\subsubsection{Conclusion}
Compared to UN Reg. 157 and (EU) 2022/1426, the German StVG and AFGBV introduce openly inconsistent notions of safety and risk which are partially directly contradictory:
The wording partially implies absolute and relative notions of safety and risk at the same time.
The implied validation targets (``better'' or ``as good as'' human drivers) are equally contradictory. 
The partially implied absolute notions of safety, when taken at face value, prohibit engineers from making the tradeoffs required to develop a system that is safe and provides customer benefit at the same time. 
In consequence, the wording in the acts is prone to introducing legal uncertainty.
This uncertainty creates additional clarification need and effort for manufacturers and engineers who design and develop SAE-Level-3 and -4 automated driving systems. The use of undefined legal terms not only makes it more difficult for engineers to comply with the law, but also complicates the interpretation of the law and leads to legal uncertainty.

\subsection{UK Automated Vehicles Act 2024 (2024 c. 10)}
The UK has issued a national enabling act for regulating the approval of automated vehicles on the roads in the UK.
To the best of our knowledge, concrete implementing regulation has not been issued yet.
Regarding terminology, the act begins with a dedicated terminology section to clarify the terms used in the act \parencite[Part 1, Chapter 1, Section 1]{ukav2024}.
In that regard, the act defines a vehicle to drive ```autonomously' if --- (a)
it is being controlled not by an individual but by equipment of the vehicle, and (b) neither the vehicle nor its surroundings are being monitored by an individual with a view to immediate intervention in the driving of the vehicle.''
The act hence covers SAE-Level-3 to SAE-Level-5 automated driving systems.

\subsubsection{Absolute vs. Relative Notions of Safety}
While not providing an explicit definition of safety and risk, the UK Automated Vehicles Act (``UK AV Act'') \parencite{ukav2024} explicitly refers to a relative notion of safety.
Part~1, Chapter~1, Section~1, Clause (7)~(a) defines that an automated vehicle travels ```safely' if it travels to an acceptably safe standard''.
This clarifies that absolute safety is not achievable and that acceptance criteria to prove the acceptability of residual risk are required, even though a concrete safety definition is not given.
The act explicitly tasks the UK Secretary of State\footnote{Which means, that concrete implementation regulation needs to be enacted.} to install safety principles to determine the ``acceptably safe standard'' in Part~1, Chapter~1, Section~1, Clause (7)~(a).
In this respect, the act also provides one general validation target as it demands that the safety principles must ensure that ``authorized automated vehicles will achieve a level of safety equivalent to, or higher than, that of careful and competent human drivers''.
Hence, the top-level validation risk acceptance criterion assumed for UK regulation is ``\emph{at least as good} as human drivers''.

\subsubsection{Hazard Sources}
The UK AV Act contains no statements that could be directly related to different hazard sources.
Note that, in contrast to the rest of the analyzed documents, the UK AV Act is enabling rather than implementing regulation.
It is hence comparable to the German StVG, which does not refer to concrete hazard sources as well.

\subsubsection{Types of Harm}
Even though providing a clear relative safety notion, the missing definition of risk also implies a lack of explicitly differentiable types of harm.
Implicitly, three different types of harm can be derived from the wording in the act.
This includes the harm to life and limb of humans\footnote{Part~1, Chapter~3, Section~25 defines ``aggravated offence where death or serious injury occurs'' \parencite{ukav2024}.}, the violation of traffic rules\footnote{Part~1, Chapter~1, Clause~(7)~(b) defines that an automated vehicle travels ```legally' if it travels with an acceptably low risk of committing a traffic infraction''}, and the cause of inconvenience to the public \parencite[Part~1, Chapter~1, Section~58, Clause (2)~(d)]{ukav2024}.

The act connects all the aforementioned types of harm to ``risk'' or ``acceptable safety''.
While the act generally defines criminal offenses for providing ``false or misleading information about safety'', it also acknowledges possible defenses if it can be proven that ``reasonable precautions'' were taken and that ``due diligence'' was exercised to ``avoid the commission of the offence''.
This statement could enable tradeoffs within the scope of ``reasonable risk'' to the life and limb of humans, the violation of traffic rules, or to the cause of inconvenience to the public, as we argued in \cref{sec:terminology}.

\subsubsection{Conclusion}
From the set of reviewed documents, the current UK AV Act is the one with the most obvious relative notions of safety and risk and the one that seems to provide a legal framework for permitting tradeoffs.
In our review, we did not spot major inconsistency beyond a missing definitions of safety and risk\footnote{Note that with the Office for Product Safety and Standards (OPSS), there is a British government agency that maintains an exhaustive and widely focussed ``Risk Lexicon'' that provides suitable risk definitions. For us, it remains unclear, to what extent this terminology is assumed general knowledge in British legislation.}.
The general, relative notion of safety and the related alleged ability for designers to argue well-founded development tradeoffs within the legal framework could prove beneficial for the actual implementation of automated driving systems.
While the act thus appears as a solid foundation for the market introduction of automated vehicles, without accompanying implementing regulation, it is too early to draw definite conclusions.

\vspace{-0.2cm}
\section{Impact: Why Free Scientific Knowledge?}
\vspace{-0.1cm}

Historically, making knowledge widely available has driven transformative progress. Gutenberg’s printing press broke medieval monopolies on information, increasing literacy and contributing to the Renaissance and Scientific Revolution. In today's world, open source projects such as GNU/Linux and Wikipedia show that freely accessible and modifiable knowledge fosters innovation while ensuring creators are credited through copyleft licenses. These examples highlight a key idea: \textit{access to essential knowledge supports overall advancement.} 

This aligns with the arguments made by Prabhakaran et al. \cite{humanrightsbasedapproachresponsible}, who specifically highlight the \textbf{ human right to participate in scientific advancement} as enshrined in the Universal Declaration of Human Rights. They emphasize that this right underscores the importance of \textit{ equal access to the benefits of scientific progress for all}, a principle directly supported by our proposal for Knowledge Units. The UN Special Rapporteur on Cultural Rights further reinforces this, advocating for the expansion of copyright exceptions to broaden access to scientific knowledge as a crucial component of the right to science and culture \cite{scienceright}. 

However, current intellectual property regimes often create ``patently unfair" barriers to this knowledge, preventing innovation and access, especially in areas critical to human rights, as Hale compellingly argues \cite{patentlyunfair}. Finding a solution requires carefully balancing the imperative of open access with the legitimate rights of authors. As Austin and Ginsburg remind us, authors' rights are also human rights, necessitating robust protection \cite{authorhumanrights}. Shareable knowledge entities like Knowledge Units offer a potential mechanism to achieve this delicate balance in the scientific domain, enabling wider dissemination of research findings while respecting authors' fundamental rights.

\vspace{-0.2cm}
\subsection{Impact Across Sectors}

\textbf{Researchers:} Collaboration across different fields becomes easier when knowledge is shared openly. For instance, combining machine learning with biology or applying quantum principles to cryptography can lead to important breakthroughs. Removing copyright restrictions allows researchers to freely use data and methods, speeding up discoveries while respecting original contributions.

\textbf{Practitioners:} Professionals, especially in healthcare, benefit from immediate access to the latest research. Quick access to newer insights on the effectiveness of drugs, and alternative treatments speeds up adoption and awareness, potentially saving lives. Additionally, open knowledge helps developing countries gain access to health innovations.

\textbf{Education:} Education becomes more accessible when teachers use the latest research to create up-to-date curricula without prohibitive costs. Students can access high-quality research materials and use LM assistance to better understand complex topics, enhancing their learning experience and making high-quality education more accessible.

\textbf{Public Trust:} When information is transparent and accessible, the public can better understand and trust decision-making processes. Open access to government policies and industry practices allows people to review and verify information, helping to reduce misinformation. This transparency encourages critical thinking and builds trust in scientific and governmental institutions.

Overall, making scientific knowledge accessible supports global fairness. By viewing knowledge as a common resource rather than a product to be sold, we can speed up innovation, encourage critical thinking, and empower communities to address important challenges.

\vspace{-0.2cm}
\section{Open Problems}
\vspace{-0.1cm}

Moving forward, we identify key research directions to further exploit the potential of converting original texts into shareable knowledge entities such as demonstrated by the conversion into Knowledge Units in this work:


\textbf{1. Enhancing Factual Accuracy and Reliability:}  Refining KUs through cross-referencing with source texts and incorporating community-driven correction mechanisms, similar to Wikipedia, can minimize hallucinations and ensure the long-term accuracy of knowledge-based datasets at scale.

\textbf{2. Developing Applications for Education and Research:}  Using KU-based conversion for datasets to be employed in practical tools, such as search interfaces and learning platforms, can ensure rapid dissemination of any new knowledge into shareable downstream resources, significantly improving the accessibility, spread, and impact of KUs.

\textbf{3. Establishing Standards for Knowledge Interoperability and Reuse:}  Future research should focus on defining standardized formats for entities like KU and knowledge graph layouts \citep{lenat1990cyc}. These standards are essential to unlock seamless interoperability, facilitate reuse across diverse platforms, and foster a vibrant ecosystem of open scientific knowledge. 

\textbf{4. Interconnecting Shareable Knowledge for Scientific Workflow Assistance and Automation:} There might be further potential in constructing a semantic web that interconnects publicly shared knowledge, together with mechanisms that continually update and validate all shareable knowledge units. This can be starting point for a platform that uses all collected knowledge to assist scientific workflows, for instance by feeding such a semantic web into recently developed reasoning models equipped with retrieval augmented generation. Such assistance could assemble knowledge across multiple scientific papers, guiding scientists more efficiently through vast research landscapes. Given further progress in model capabilities, validation, self-repair and evolving new knowledge from already existing vast collection in the semantic web can lead to automation of scientific discovery, assuming that knowledge data in the semantic web can be freely shared.

We open-source our code and encourage collaboration to improve extraction pipelines, enhance Knowledge Unit capabilities, and expand coverage to additional fields.

\vspace{-0.2cm}
\section{Conclusion}
\vspace{-0.1cm}

In this paper, we highlight the potential of systematically separating factual scientific knowledge from protected artistic or stylistic expression. By representing scientific insights as structured facts and relationships, prototypes like Knowledge Units (KUs) offer a pathway to broaden access to scientific knowledge without infringing copyright, aligning with legal principles like German \S 24(1) UrhG and U.S. fair use standards. Extensive testing across a range of domains and models shows evidence that Knowledge Units (KUs) can feasibly retain core information. These findings offer a promising way forward for openly disseminating scientific information while respecting copyright constraints.

\section*{Author Contributions}

Christoph conceived the project and led organization. Christoph and Gollam led all the experiments. Nick and Huu led the legal aspects. Tawsif led the data collection. Ameya and Andreas led the manuscript writing. Ludwig, Sören, Robert, Jenia and Matthias provided feedback. advice and scientific supervision throughout the project. 

\section*{Acknowledgements}

The authors would like to thank (in alphabetical order): Sebastian Dziadzio, Kristof Meding, Tea Mustać, Shantanu Prabhat for insightful feedback and suggestions. Special thanks to Andrej Radonjic for help in scaling up data collection. GR and SA acknowledge financial support by the German Research Foundation (DFG) for the NFDI4DataScience Initiative (project number 460234259). AP and MB acknowledge financial support by the Federal Ministry of Education and Research (BMBF), FKZ: 011524085B and Open Philanthropy Foundation funded by the Good Ventures Foundation. AH acknowledges financial support by the Federal Ministry of Education and Research (BMBF), FKZ: 01IS24079A and the Carl Zeiss Foundation through the project "Certification and Foundations of Safe ML Systems" as well as the support from the International Max Planck Research School for Intelligent Systems (IMPRS-IS). JJ acknowledges funding by the Federal Ministry of Education and Research of Germany (BMBF) under grant no. 01IS22094B (WestAI - AI Service Center West), under grant no. 01IS24085C (OPENHAFM) and under the grant DE002571 (MINERVA), as well as co-funding by EU from EuroHPC Joint Undertaking programm under grant no. 101182737 (MINERVA) and from Digital Europe Programme under grant no. 101195233 (openEuroLLM) 

\section*{Limitations}
This study investigates the effectiveness of pragmatics in enhancing Retrieval Augmented Generation (RAG) systems. Our evaluation, however, is limited to a comparison against standard Dense Passage Retriever (DPR) and BM25 baselines. The proposed method has potential for integration with more sophisticated RAG systems, such as those developed by \newcite{asai2024selfrag, xu2024recomp, sarthi2024raptor}. Our assessment encompasses three datasets, but a more comprehensive evaluation would involve a broader range of single-hop and multi-hop tasks. Moreover, there are several scenarios which our approach does not cover, such as handling linguistic phenomena like negation, mathematical reasoning tasks and reconciling retrieved contexts that are ambiguous. Our current approach is also limited by the fact that it is unsupervised and query reformulation is mostly driven by a bag-of-words. One could trivially improve query reformulation by using an LLM, or using a weakly supervised strategy that fine-tunes an LLM to retrieve pragmatic evidence (using supervision from the current retriever) via a joint loss that learns to retrieve evidence sentences while simultaneously answering the query correctly (motivated by the relevance estimator and answer marginalization losses proposed by \newcite{kim2024reragimprovingopendomainqa}). We leave the exploration of supervised pragmatic RAG methods as future work.

While we hypothesize that our retrieved \& highlighted justifications constitute ``shallow chains of thought'' which are faithfully utilized by the Large Language Model in its generations, this assertion remains to be formally validated through rigorous analysis.
% Since December 2023, a "Limitations" section has been required for all papers submitted to ACL Rolling Review (ARR). This section should be placed at the end of the paper, before the references. The "Limitations" section (along with, optionally, a section for ethical considerations) may be up to one page and will not count toward the final page limit. Note that these files may be used by venues that do not rely on ARR so it is recommended to verify the requirement of a "Limitations" section and other criteria with the venue in question.

% \section*{Acknowledgments}

% This document has been adapted
% by Steven Bethard, Ryan Cotterell and Rui Yan
% from the instructions for earlier ACL and NAACL proceedings, including those for
% ACL 2019 by Douwe Kiela and Ivan Vuli\'{c},
% NAACL 2019 by Stephanie Lukin and Alla Roskovskaya,
% ACL 2018 by Shay Cohen, Kevin Gimpel, and Wei Lu,
% NAACL 2018 by Margaret Mitchell and Stephanie Lukin,
% Bib\TeX{} suggestions for (NA)ACL 2017/2018 from Jason Eisner,
% ACL 2017 by Dan Gildea and Min-Yen Kan,
% NAACL 2017 by Margaret Mitchell,
% ACL 2012 by Maggie Li and Michael White,
% ACL 2010 by Jing-Shin Chang and Philipp Koehn,
% ACL 2008 by Johanna D. Moore, Simone Teufel, James Allan, and Sadaoki Furui,
% ACL 2005 by Hwee Tou Ng and Kemal Oflazer,
% ACL 2002 by Eugene Charniak and Dekang Lin,
% and earlier ACL and EACL formats written by several people, including
% John Chen, Henry S. Thompson and Donald Walker.
% Additional elements were taken from the formatting instructions of the \emph{International Joint Conference on Artificial Intelligence} and the \emph{Conference on Computer Vision and Pattern Recognition}.

% Bibliography entries for the entire Anthology, followed by custom entries
%\bibliography{anthology,custom}
% Custom bibliography entries only
\bibliography{custom}
\subsection{Lloyd-Max Algorithm}
\label{subsec:Lloyd-Max}
For a given quantization bitwidth $B$ and an operand $\bm{X}$, the Lloyd-Max algorithm finds $2^B$ quantization levels $\{\hat{x}_i\}_{i=1}^{2^B}$ such that quantizing $\bm{X}$ by rounding each scalar in $\bm{X}$ to the nearest quantization level minimizes the quantization MSE. 

The algorithm starts with an initial guess of quantization levels and then iteratively computes quantization thresholds $\{\tau_i\}_{i=1}^{2^B-1}$ and updates quantization levels $\{\hat{x}_i\}_{i=1}^{2^B}$. Specifically, at iteration $n$, thresholds are set to the midpoints of the previous iteration's levels:
\begin{align*}
    \tau_i^{(n)}=\frac{\hat{x}_i^{(n-1)}+\hat{x}_{i+1}^{(n-1)}}2 \text{ for } i=1\ldots 2^B-1
\end{align*}
Subsequently, the quantization levels are re-computed as conditional means of the data regions defined by the new thresholds:
\begin{align*}
    \hat{x}_i^{(n)}=\mathbb{E}\left[ \bm{X} \big| \bm{X}\in [\tau_{i-1}^{(n)},\tau_i^{(n)}] \right] \text{ for } i=1\ldots 2^B
\end{align*}
where to satisfy boundary conditions we have $\tau_0=-\infty$ and $\tau_{2^B}=\infty$. The algorithm iterates the above steps until convergence.

Figure \ref{fig:lm_quant} compares the quantization levels of a $7$-bit floating point (E3M3) quantizer (left) to a $7$-bit Lloyd-Max quantizer (right) when quantizing a layer of weights from the GPT3-126M model at a per-tensor granularity. As shown, the Lloyd-Max quantizer achieves substantially lower quantization MSE. Further, Table \ref{tab:FP7_vs_LM7} shows the superior perplexity achieved by Lloyd-Max quantizers for bitwidths of $7$, $6$ and $5$. The difference between the quantizers is clear at 5 bits, where per-tensor FP quantization incurs a drastic and unacceptable increase in perplexity, while Lloyd-Max quantization incurs a much smaller increase. Nevertheless, we note that even the optimal Lloyd-Max quantizer incurs a notable ($\sim 1.5$) increase in perplexity due to the coarse granularity of quantization. 

\begin{figure}[h]
  \centering
  \includegraphics[width=0.7\linewidth]{sections/figures/LM7_FP7.pdf}
  \caption{\small Quantization levels and the corresponding quantization MSE of Floating Point (left) vs Lloyd-Max (right) Quantizers for a layer of weights in the GPT3-126M model.}
  \label{fig:lm_quant}
\end{figure}

\begin{table}[h]\scriptsize
\begin{center}
\caption{\label{tab:FP7_vs_LM7} \small Comparing perplexity (lower is better) achieved by floating point quantizers and Lloyd-Max quantizers on a GPT3-126M model for the Wikitext-103 dataset.}
\begin{tabular}{c|cc|c}
\hline
 \multirow{2}{*}{\textbf{Bitwidth}} & \multicolumn{2}{|c|}{\textbf{Floating-Point Quantizer}} & \textbf{Lloyd-Max Quantizer} \\
 & Best Format & Wikitext-103 Perplexity & Wikitext-103 Perplexity \\
\hline
7 & E3M3 & 18.32 & 18.27 \\
6 & E3M2 & 19.07 & 18.51 \\
5 & E4M0 & 43.89 & 19.71 \\
\hline
\end{tabular}
\end{center}
\end{table}

\subsection{Proof of Local Optimality of LO-BCQ}
\label{subsec:lobcq_opt_proof}
For a given block $\bm{b}_j$, the quantization MSE during LO-BCQ can be empirically evaluated as $\frac{1}{L_b}\lVert \bm{b}_j- \bm{\hat{b}}_j\rVert^2_2$ where $\bm{\hat{b}}_j$ is computed from equation (\ref{eq:clustered_quantization_definition}) as $C_{f(\bm{b}_j)}(\bm{b}_j)$. Further, for a given block cluster $\mathcal{B}_i$, we compute the quantization MSE as $\frac{1}{|\mathcal{B}_{i}|}\sum_{\bm{b} \in \mathcal{B}_{i}} \frac{1}{L_b}\lVert \bm{b}- C_i^{(n)}(\bm{b})\rVert^2_2$. Therefore, at the end of iteration $n$, we evaluate the overall quantization MSE $J^{(n)}$ for a given operand $\bm{X}$ composed of $N_c$ block clusters as:
\begin{align*}
    \label{eq:mse_iter_n}
    J^{(n)} = \frac{1}{N_c} \sum_{i=1}^{N_c} \frac{1}{|\mathcal{B}_{i}^{(n)}|}\sum_{\bm{v} \in \mathcal{B}_{i}^{(n)}} \frac{1}{L_b}\lVert \bm{b}- B_i^{(n)}(\bm{b})\rVert^2_2
\end{align*}

At the end of iteration $n$, the codebooks are updated from $\mathcal{C}^{(n-1)}$ to $\mathcal{C}^{(n)}$. However, the mapping of a given vector $\bm{b}_j$ to quantizers $\mathcal{C}^{(n)}$ remains as  $f^{(n)}(\bm{b}_j)$. At the next iteration, during the vector clustering step, $f^{(n+1)}(\bm{b}_j)$ finds new mapping of $\bm{b}_j$ to updated codebooks $\mathcal{C}^{(n)}$ such that the quantization MSE over the candidate codebooks is minimized. Therefore, we obtain the following result for $\bm{b}_j$:
\begin{align*}
\frac{1}{L_b}\lVert \bm{b}_j - C_{f^{(n+1)}(\bm{b}_j)}^{(n)}(\bm{b}_j)\rVert^2_2 \le \frac{1}{L_b}\lVert \bm{b}_j - C_{f^{(n)}(\bm{b}_j)}^{(n)}(\bm{b}_j)\rVert^2_2
\end{align*}

That is, quantizing $\bm{b}_j$ at the end of the block clustering step of iteration $n+1$ results in lower quantization MSE compared to quantizing at the end of iteration $n$. Since this is true for all $\bm{b} \in \bm{X}$, we assert the following:
\begin{equation}
\begin{split}
\label{eq:mse_ineq_1}
    \tilde{J}^{(n+1)} &= \frac{1}{N_c} \sum_{i=1}^{N_c} \frac{1}{|\mathcal{B}_{i}^{(n+1)}|}\sum_{\bm{b} \in \mathcal{B}_{i}^{(n+1)}} \frac{1}{L_b}\lVert \bm{b} - C_i^{(n)}(b)\rVert^2_2 \le J^{(n)}
\end{split}
\end{equation}
where $\tilde{J}^{(n+1)}$ is the the quantization MSE after the vector clustering step at iteration $n+1$.

Next, during the codebook update step (\ref{eq:quantizers_update}) at iteration $n+1$, the per-cluster codebooks $\mathcal{C}^{(n)}$ are updated to $\mathcal{C}^{(n+1)}$ by invoking the Lloyd-Max algorithm \citep{Lloyd}. We know that for any given value distribution, the Lloyd-Max algorithm minimizes the quantization MSE. Therefore, for a given vector cluster $\mathcal{B}_i$ we obtain the following result:

\begin{equation}
    \frac{1}{|\mathcal{B}_{i}^{(n+1)}|}\sum_{\bm{b} \in \mathcal{B}_{i}^{(n+1)}} \frac{1}{L_b}\lVert \bm{b}- C_i^{(n+1)}(\bm{b})\rVert^2_2 \le \frac{1}{|\mathcal{B}_{i}^{(n+1)}|}\sum_{\bm{b} \in \mathcal{B}_{i}^{(n+1)}} \frac{1}{L_b}\lVert \bm{b}- C_i^{(n)}(\bm{b})\rVert^2_2
\end{equation}

The above equation states that quantizing the given block cluster $\mathcal{B}_i$ after updating the associated codebook from $C_i^{(n)}$ to $C_i^{(n+1)}$ results in lower quantization MSE. Since this is true for all the block clusters, we derive the following result: 
\begin{equation}
\begin{split}
\label{eq:mse_ineq_2}
     J^{(n+1)} &= \frac{1}{N_c} \sum_{i=1}^{N_c} \frac{1}{|\mathcal{B}_{i}^{(n+1)}|}\sum_{\bm{b} \in \mathcal{B}_{i}^{(n+1)}} \frac{1}{L_b}\lVert \bm{b}- C_i^{(n+1)}(\bm{b})\rVert^2_2  \le \tilde{J}^{(n+1)}   
\end{split}
\end{equation}

Following (\ref{eq:mse_ineq_1}) and (\ref{eq:mse_ineq_2}), we find that the quantization MSE is non-increasing for each iteration, that is, $J^{(1)} \ge J^{(2)} \ge J^{(3)} \ge \ldots \ge J^{(M)}$ where $M$ is the maximum number of iterations. 
%Therefore, we can say that if the algorithm converges, then it must be that it has converged to a local minimum. 
\hfill $\blacksquare$


\begin{figure}
    \begin{center}
    \includegraphics[width=0.5\textwidth]{sections//figures/mse_vs_iter.pdf}
    \end{center}
    \caption{\small NMSE vs iterations during LO-BCQ compared to other block quantization proposals}
    \label{fig:nmse_vs_iter}
\end{figure}

Figure \ref{fig:nmse_vs_iter} shows the empirical convergence of LO-BCQ across several block lengths and number of codebooks. Also, the MSE achieved by LO-BCQ is compared to baselines such as MXFP and VSQ. As shown, LO-BCQ converges to a lower MSE than the baselines. Further, we achieve better convergence for larger number of codebooks ($N_c$) and for a smaller block length ($L_b$), both of which increase the bitwidth of BCQ (see Eq \ref{eq:bitwidth_bcq}).


\subsection{Additional Accuracy Results}
%Table \ref{tab:lobcq_config} lists the various LOBCQ configurations and their corresponding bitwidths.
\begin{table}
\setlength{\tabcolsep}{4.75pt}
\begin{center}
\caption{\label{tab:lobcq_config} Various LO-BCQ configurations and their bitwidths.}
\begin{tabular}{|c||c|c|c|c||c|c||c|} 
\hline
 & \multicolumn{4}{|c||}{$L_b=8$} & \multicolumn{2}{|c||}{$L_b=4$} & $L_b=2$ \\
 \hline
 \backslashbox{$L_A$\kern-1em}{\kern-1em$N_c$} & 2 & 4 & 8 & 16 & 2 & 4 & 2 \\
 \hline
 64 & 4.25 & 4.375 & 4.5 & 4.625 & 4.375 & 4.625 & 4.625\\
 \hline
 32 & 4.375 & 4.5 & 4.625& 4.75 & 4.5 & 4.75 & 4.75 \\
 \hline
 16 & 4.625 & 4.75& 4.875 & 5 & 4.75 & 5 & 5 \\
 \hline
\end{tabular}
\end{center}
\end{table}

%\subsection{Perplexity achieved by various LO-BCQ configurations on Wikitext-103 dataset}

\begin{table} \centering
\begin{tabular}{|c||c|c|c|c||c|c||c|} 
\hline
 $L_b \rightarrow$& \multicolumn{4}{c||}{8} & \multicolumn{2}{c||}{4} & 2\\
 \hline
 \backslashbox{$L_A$\kern-1em}{\kern-1em$N_c$} & 2 & 4 & 8 & 16 & 2 & 4 & 2  \\
 %$N_c \rightarrow$ & 2 & 4 & 8 & 16 & 2 & 4 & 2 \\
 \hline
 \hline
 \multicolumn{8}{c}{GPT3-1.3B (FP32 PPL = 9.98)} \\ 
 \hline
 \hline
 64 & 10.40 & 10.23 & 10.17 & 10.15 &  10.28 & 10.18 & 10.19 \\
 \hline
 32 & 10.25 & 10.20 & 10.15 & 10.12 &  10.23 & 10.17 & 10.17 \\
 \hline
 16 & 10.22 & 10.16 & 10.10 & 10.09 &  10.21 & 10.14 & 10.16 \\
 \hline
  \hline
 \multicolumn{8}{c}{GPT3-8B (FP32 PPL = 7.38)} \\ 
 \hline
 \hline
 64 & 7.61 & 7.52 & 7.48 &  7.47 &  7.55 &  7.49 & 7.50 \\
 \hline
 32 & 7.52 & 7.50 & 7.46 &  7.45 &  7.52 &  7.48 & 7.48  \\
 \hline
 16 & 7.51 & 7.48 & 7.44 &  7.44 &  7.51 &  7.49 & 7.47  \\
 \hline
\end{tabular}
\caption{\label{tab:ppl_gpt3_abalation} Wikitext-103 perplexity across GPT3-1.3B and 8B models.}
\end{table}

\begin{table} \centering
\begin{tabular}{|c||c|c|c|c||} 
\hline
 $L_b \rightarrow$& \multicolumn{4}{c||}{8}\\
 \hline
 \backslashbox{$L_A$\kern-1em}{\kern-1em$N_c$} & 2 & 4 & 8 & 16 \\
 %$N_c \rightarrow$ & 2 & 4 & 8 & 16 & 2 & 4 & 2 \\
 \hline
 \hline
 \multicolumn{5}{|c|}{Llama2-7B (FP32 PPL = 5.06)} \\ 
 \hline
 \hline
 64 & 5.31 & 5.26 & 5.19 & 5.18  \\
 \hline
 32 & 5.23 & 5.25 & 5.18 & 5.15  \\
 \hline
 16 & 5.23 & 5.19 & 5.16 & 5.14  \\
 \hline
 \multicolumn{5}{|c|}{Nemotron4-15B (FP32 PPL = 5.87)} \\ 
 \hline
 \hline
 64  & 6.3 & 6.20 & 6.13 & 6.08  \\
 \hline
 32  & 6.24 & 6.12 & 6.07 & 6.03  \\
 \hline
 16  & 6.12 & 6.14 & 6.04 & 6.02  \\
 \hline
 \multicolumn{5}{|c|}{Nemotron4-340B (FP32 PPL = 3.48)} \\ 
 \hline
 \hline
 64 & 3.67 & 3.62 & 3.60 & 3.59 \\
 \hline
 32 & 3.63 & 3.61 & 3.59 & 3.56 \\
 \hline
 16 & 3.61 & 3.58 & 3.57 & 3.55 \\
 \hline
\end{tabular}
\caption{\label{tab:ppl_llama7B_nemo15B} Wikitext-103 perplexity compared to FP32 baseline in Llama2-7B and Nemotron4-15B, 340B models}
\end{table}

%\subsection{Perplexity achieved by various LO-BCQ configurations on MMLU dataset}


\begin{table} \centering
\begin{tabular}{|c||c|c|c|c||c|c|c|c|} 
\hline
 $L_b \rightarrow$& \multicolumn{4}{c||}{8} & \multicolumn{4}{c||}{8}\\
 \hline
 \backslashbox{$L_A$\kern-1em}{\kern-1em$N_c$} & 2 & 4 & 8 & 16 & 2 & 4 & 8 & 16  \\
 %$N_c \rightarrow$ & 2 & 4 & 8 & 16 & 2 & 4 & 2 \\
 \hline
 \hline
 \multicolumn{5}{|c|}{Llama2-7B (FP32 Accuracy = 45.8\%)} & \multicolumn{4}{|c|}{Llama2-70B (FP32 Accuracy = 69.12\%)} \\ 
 \hline
 \hline
 64 & 43.9 & 43.4 & 43.9 & 44.9 & 68.07 & 68.27 & 68.17 & 68.75 \\
 \hline
 32 & 44.5 & 43.8 & 44.9 & 44.5 & 68.37 & 68.51 & 68.35 & 68.27  \\
 \hline
 16 & 43.9 & 42.7 & 44.9 & 45 & 68.12 & 68.77 & 68.31 & 68.59  \\
 \hline
 \hline
 \multicolumn{5}{|c|}{GPT3-22B (FP32 Accuracy = 38.75\%)} & \multicolumn{4}{|c|}{Nemotron4-15B (FP32 Accuracy = 64.3\%)} \\ 
 \hline
 \hline
 64 & 36.71 & 38.85 & 38.13 & 38.92 & 63.17 & 62.36 & 63.72 & 64.09 \\
 \hline
 32 & 37.95 & 38.69 & 39.45 & 38.34 & 64.05 & 62.30 & 63.8 & 64.33  \\
 \hline
 16 & 38.88 & 38.80 & 38.31 & 38.92 & 63.22 & 63.51 & 63.93 & 64.43  \\
 \hline
\end{tabular}
\caption{\label{tab:mmlu_abalation} Accuracy on MMLU dataset across GPT3-22B, Llama2-7B, 70B and Nemotron4-15B models.}
\end{table}


%\subsection{Perplexity achieved by various LO-BCQ configurations on LM evaluation harness}

\begin{table} \centering
\begin{tabular}{|c||c|c|c|c||c|c|c|c|} 
\hline
 $L_b \rightarrow$& \multicolumn{4}{c||}{8} & \multicolumn{4}{c||}{8}\\
 \hline
 \backslashbox{$L_A$\kern-1em}{\kern-1em$N_c$} & 2 & 4 & 8 & 16 & 2 & 4 & 8 & 16  \\
 %$N_c \rightarrow$ & 2 & 4 & 8 & 16 & 2 & 4 & 2 \\
 \hline
 \hline
 \multicolumn{5}{|c|}{Race (FP32 Accuracy = 37.51\%)} & \multicolumn{4}{|c|}{Boolq (FP32 Accuracy = 64.62\%)} \\ 
 \hline
 \hline
 64 & 36.94 & 37.13 & 36.27 & 37.13 & 63.73 & 62.26 & 63.49 & 63.36 \\
 \hline
 32 & 37.03 & 36.36 & 36.08 & 37.03 & 62.54 & 63.51 & 63.49 & 63.55  \\
 \hline
 16 & 37.03 & 37.03 & 36.46 & 37.03 & 61.1 & 63.79 & 63.58 & 63.33  \\
 \hline
 \hline
 \multicolumn{5}{|c|}{Winogrande (FP32 Accuracy = 58.01\%)} & \multicolumn{4}{|c|}{Piqa (FP32 Accuracy = 74.21\%)} \\ 
 \hline
 \hline
 64 & 58.17 & 57.22 & 57.85 & 58.33 & 73.01 & 73.07 & 73.07 & 72.80 \\
 \hline
 32 & 59.12 & 58.09 & 57.85 & 58.41 & 73.01 & 73.94 & 72.74 & 73.18  \\
 \hline
 16 & 57.93 & 58.88 & 57.93 & 58.56 & 73.94 & 72.80 & 73.01 & 73.94  \\
 \hline
\end{tabular}
\caption{\label{tab:mmlu_abalation} Accuracy on LM evaluation harness tasks on GPT3-1.3B model.}
\end{table}

\begin{table} \centering
\begin{tabular}{|c||c|c|c|c||c|c|c|c|} 
\hline
 $L_b \rightarrow$& \multicolumn{4}{c||}{8} & \multicolumn{4}{c||}{8}\\
 \hline
 \backslashbox{$L_A$\kern-1em}{\kern-1em$N_c$} & 2 & 4 & 8 & 16 & 2 & 4 & 8 & 16  \\
 %$N_c \rightarrow$ & 2 & 4 & 8 & 16 & 2 & 4 & 2 \\
 \hline
 \hline
 \multicolumn{5}{|c|}{Race (FP32 Accuracy = 41.34\%)} & \multicolumn{4}{|c|}{Boolq (FP32 Accuracy = 68.32\%)} \\ 
 \hline
 \hline
 64 & 40.48 & 40.10 & 39.43 & 39.90 & 69.20 & 68.41 & 69.45 & 68.56 \\
 \hline
 32 & 39.52 & 39.52 & 40.77 & 39.62 & 68.32 & 67.43 & 68.17 & 69.30  \\
 \hline
 16 & 39.81 & 39.71 & 39.90 & 40.38 & 68.10 & 66.33 & 69.51 & 69.42  \\
 \hline
 \hline
 \multicolumn{5}{|c|}{Winogrande (FP32 Accuracy = 67.88\%)} & \multicolumn{4}{|c|}{Piqa (FP32 Accuracy = 78.78\%)} \\ 
 \hline
 \hline
 64 & 66.85 & 66.61 & 67.72 & 67.88 & 77.31 & 77.42 & 77.75 & 77.64 \\
 \hline
 32 & 67.25 & 67.72 & 67.72 & 67.00 & 77.31 & 77.04 & 77.80 & 77.37  \\
 \hline
 16 & 68.11 & 68.90 & 67.88 & 67.48 & 77.37 & 78.13 & 78.13 & 77.69  \\
 \hline
\end{tabular}
\caption{\label{tab:mmlu_abalation} Accuracy on LM evaluation harness tasks on GPT3-8B model.}
\end{table}

\begin{table} \centering
\begin{tabular}{|c||c|c|c|c||c|c|c|c|} 
\hline
 $L_b \rightarrow$& \multicolumn{4}{c||}{8} & \multicolumn{4}{c||}{8}\\
 \hline
 \backslashbox{$L_A$\kern-1em}{\kern-1em$N_c$} & 2 & 4 & 8 & 16 & 2 & 4 & 8 & 16  \\
 %$N_c \rightarrow$ & 2 & 4 & 8 & 16 & 2 & 4 & 2 \\
 \hline
 \hline
 \multicolumn{5}{|c|}{Race (FP32 Accuracy = 40.67\%)} & \multicolumn{4}{|c|}{Boolq (FP32 Accuracy = 76.54\%)} \\ 
 \hline
 \hline
 64 & 40.48 & 40.10 & 39.43 & 39.90 & 75.41 & 75.11 & 77.09 & 75.66 \\
 \hline
 32 & 39.52 & 39.52 & 40.77 & 39.62 & 76.02 & 76.02 & 75.96 & 75.35  \\
 \hline
 16 & 39.81 & 39.71 & 39.90 & 40.38 & 75.05 & 73.82 & 75.72 & 76.09  \\
 \hline
 \hline
 \multicolumn{5}{|c|}{Winogrande (FP32 Accuracy = 70.64\%)} & \multicolumn{4}{|c|}{Piqa (FP32 Accuracy = 79.16\%)} \\ 
 \hline
 \hline
 64 & 69.14 & 70.17 & 70.17 & 70.56 & 78.24 & 79.00 & 78.62 & 78.73 \\
 \hline
 32 & 70.96 & 69.69 & 71.27 & 69.30 & 78.56 & 79.49 & 79.16 & 78.89  \\
 \hline
 16 & 71.03 & 69.53 & 69.69 & 70.40 & 78.13 & 79.16 & 79.00 & 79.00  \\
 \hline
\end{tabular}
\caption{\label{tab:mmlu_abalation} Accuracy on LM evaluation harness tasks on GPT3-22B model.}
\end{table}

\begin{table} \centering
\begin{tabular}{|c||c|c|c|c||c|c|c|c|} 
\hline
 $L_b \rightarrow$& \multicolumn{4}{c||}{8} & \multicolumn{4}{c||}{8}\\
 \hline
 \backslashbox{$L_A$\kern-1em}{\kern-1em$N_c$} & 2 & 4 & 8 & 16 & 2 & 4 & 8 & 16  \\
 %$N_c \rightarrow$ & 2 & 4 & 8 & 16 & 2 & 4 & 2 \\
 \hline
 \hline
 \multicolumn{5}{|c|}{Race (FP32 Accuracy = 44.4\%)} & \multicolumn{4}{|c|}{Boolq (FP32 Accuracy = 79.29\%)} \\ 
 \hline
 \hline
 64 & 42.49 & 42.51 & 42.58 & 43.45 & 77.58 & 77.37 & 77.43 & 78.1 \\
 \hline
 32 & 43.35 & 42.49 & 43.64 & 43.73 & 77.86 & 75.32 & 77.28 & 77.86  \\
 \hline
 16 & 44.21 & 44.21 & 43.64 & 42.97 & 78.65 & 77 & 76.94 & 77.98  \\
 \hline
 \hline
 \multicolumn{5}{|c|}{Winogrande (FP32 Accuracy = 69.38\%)} & \multicolumn{4}{|c|}{Piqa (FP32 Accuracy = 78.07\%)} \\ 
 \hline
 \hline
 64 & 68.9 & 68.43 & 69.77 & 68.19 & 77.09 & 76.82 & 77.09 & 77.86 \\
 \hline
 32 & 69.38 & 68.51 & 68.82 & 68.90 & 78.07 & 76.71 & 78.07 & 77.86  \\
 \hline
 16 & 69.53 & 67.09 & 69.38 & 68.90 & 77.37 & 77.8 & 77.91 & 77.69  \\
 \hline
\end{tabular}
\caption{\label{tab:mmlu_abalation} Accuracy on LM evaluation harness tasks on Llama2-7B model.}
\end{table}

\begin{table} \centering
\begin{tabular}{|c||c|c|c|c||c|c|c|c|} 
\hline
 $L_b \rightarrow$& \multicolumn{4}{c||}{8} & \multicolumn{4}{c||}{8}\\
 \hline
 \backslashbox{$L_A$\kern-1em}{\kern-1em$N_c$} & 2 & 4 & 8 & 16 & 2 & 4 & 8 & 16  \\
 %$N_c \rightarrow$ & 2 & 4 & 8 & 16 & 2 & 4 & 2 \\
 \hline
 \hline
 \multicolumn{5}{|c|}{Race (FP32 Accuracy = 48.8\%)} & \multicolumn{4}{|c|}{Boolq (FP32 Accuracy = 85.23\%)} \\ 
 \hline
 \hline
 64 & 49.00 & 49.00 & 49.28 & 48.71 & 82.82 & 84.28 & 84.03 & 84.25 \\
 \hline
 32 & 49.57 & 48.52 & 48.33 & 49.28 & 83.85 & 84.46 & 84.31 & 84.93  \\
 \hline
 16 & 49.85 & 49.09 & 49.28 & 48.99 & 85.11 & 84.46 & 84.61 & 83.94  \\
 \hline
 \hline
 \multicolumn{5}{|c|}{Winogrande (FP32 Accuracy = 79.95\%)} & \multicolumn{4}{|c|}{Piqa (FP32 Accuracy = 81.56\%)} \\ 
 \hline
 \hline
 64 & 78.77 & 78.45 & 78.37 & 79.16 & 81.45 & 80.69 & 81.45 & 81.5 \\
 \hline
 32 & 78.45 & 79.01 & 78.69 & 80.66 & 81.56 & 80.58 & 81.18 & 81.34  \\
 \hline
 16 & 79.95 & 79.56 & 79.79 & 79.72 & 81.28 & 81.66 & 81.28 & 80.96  \\
 \hline
\end{tabular}
\caption{\label{tab:mmlu_abalation} Accuracy on LM evaluation harness tasks on Llama2-70B model.}
\end{table}

%\section{MSE Studies}
%\textcolor{red}{TODO}


\subsection{Number Formats and Quantization Method}
\label{subsec:numFormats_quantMethod}
\subsubsection{Integer Format}
An $n$-bit signed integer (INT) is typically represented with a 2s-complement format \citep{yao2022zeroquant,xiao2023smoothquant,dai2021vsq}, where the most significant bit denotes the sign.

\subsubsection{Floating Point Format}
An $n$-bit signed floating point (FP) number $x$ comprises of a 1-bit sign ($x_{\mathrm{sign}}$), $B_m$-bit mantissa ($x_{\mathrm{mant}}$) and $B_e$-bit exponent ($x_{\mathrm{exp}}$) such that $B_m+B_e=n-1$. The associated constant exponent bias ($E_{\mathrm{bias}}$) is computed as $(2^{{B_e}-1}-1)$. We denote this format as $E_{B_e}M_{B_m}$.  

\subsubsection{Quantization Scheme}
\label{subsec:quant_method}
A quantization scheme dictates how a given unquantized tensor is converted to its quantized representation. We consider FP formats for the purpose of illustration. Given an unquantized tensor $\bm{X}$ and an FP format $E_{B_e}M_{B_m}$, we first, we compute the quantization scale factor $s_X$ that maps the maximum absolute value of $\bm{X}$ to the maximum quantization level of the $E_{B_e}M_{B_m}$ format as follows:
\begin{align}
\label{eq:sf}
    s_X = \frac{\mathrm{max}(|\bm{X}|)}{\mathrm{max}(E_{B_e}M_{B_m})}
\end{align}
In the above equation, $|\cdot|$ denotes the absolute value function.

Next, we scale $\bm{X}$ by $s_X$ and quantize it to $\hat{\bm{X}}$ by rounding it to the nearest quantization level of $E_{B_e}M_{B_m}$ as:

\begin{align}
\label{eq:tensor_quant}
    \hat{\bm{X}} = \text{round-to-nearest}\left(\frac{\bm{X}}{s_X}, E_{B_e}M_{B_m}\right)
\end{align}

We perform dynamic max-scaled quantization \citep{wu2020integer}, where the scale factor $s$ for activations is dynamically computed during runtime.

\subsection{Vector Scaled Quantization}
\begin{wrapfigure}{r}{0.35\linewidth}
  \centering
  \includegraphics[width=\linewidth]{sections/figures/vsquant.jpg}
  \caption{\small Vectorwise decomposition for per-vector scaled quantization (VSQ \citep{dai2021vsq}).}
  \label{fig:vsquant}
\end{wrapfigure}
During VSQ \citep{dai2021vsq}, the operand tensors are decomposed into 1D vectors in a hardware friendly manner as shown in Figure \ref{fig:vsquant}. Since the decomposed tensors are used as operands in matrix multiplications during inference, it is beneficial to perform this decomposition along the reduction dimension of the multiplication. The vectorwise quantization is performed similar to tensorwise quantization described in Equations \ref{eq:sf} and \ref{eq:tensor_quant}, where a scale factor $s_v$ is required for each vector $\bm{v}$ that maps the maximum absolute value of that vector to the maximum quantization level. While smaller vector lengths can lead to larger accuracy gains, the associated memory and computational overheads due to the per-vector scale factors increases. To alleviate these overheads, VSQ \citep{dai2021vsq} proposed a second level quantization of the per-vector scale factors to unsigned integers, while MX \citep{rouhani2023shared} quantizes them to integer powers of 2 (denoted as $2^{INT}$).

\subsubsection{MX Format}
The MX format proposed in \citep{rouhani2023microscaling} introduces the concept of sub-block shifting. For every two scalar elements of $b$-bits each, there is a shared exponent bit. The value of this exponent bit is determined through an empirical analysis that targets minimizing quantization MSE. We note that the FP format $E_{1}M_{b}$ is strictly better than MX from an accuracy perspective since it allocates a dedicated exponent bit to each scalar as opposed to sharing it across two scalars. Therefore, we conservatively bound the accuracy of a $b+2$-bit signed MX format with that of a $E_{1}M_{b}$ format in our comparisons. For instance, we use E1M2 format as a proxy for MX4.

\begin{figure}
    \centering
    \includegraphics[width=1\linewidth]{sections//figures/BlockFormats.pdf}
    \caption{\small Comparing LO-BCQ to MX format.}
    \label{fig:block_formats}
\end{figure}

Figure \ref{fig:block_formats} compares our $4$-bit LO-BCQ block format to MX \citep{rouhani2023microscaling}. As shown, both LO-BCQ and MX decompose a given operand tensor into block arrays and each block array into blocks. Similar to MX, we find that per-block quantization ($L_b < L_A$) leads to better accuracy due to increased flexibility. While MX achieves this through per-block $1$-bit micro-scales, we associate a dedicated codebook to each block through a per-block codebook selector. Further, MX quantizes the per-block array scale-factor to E8M0 format without per-tensor scaling. In contrast during LO-BCQ, we find that per-tensor scaling combined with quantization of per-block array scale-factor to E4M3 format results in superior inference accuracy across models. 


\end{document}
