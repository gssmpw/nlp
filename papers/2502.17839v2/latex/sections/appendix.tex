\appendix

\begin{table*}[t]
\centering
\footnotesize
\begin{tabularx}{\textwidth}{lX}
\toprule
\textbf{Category} & \textbf{Examples of Evidences} \\
\midrule
Good Evidence &
\textbf{Question:} A certain atom has 20 electrons, 21 neutrons, and 20 protons. What is the atomic mass of the atom?\\[1ex]
& \textbf{Highlighted Evidence}:\\[0.5ex]
& \quad $-$ ``Mass number (symbol `A', from German `Atomgewicht') is the total number of protons and neutrons (nucleons) in a nucleus.''\\[0.5ex]
& \quad $-$ ``Atomic mass is approximately the mass number times an atomic mass unit (approximate mass of a proton, neutron, or hydrogen-1 atom).''\\[1ex]
Medium Evidence &
\textbf{Question:} A law in Japan makes it illegal for citizens of that country to be fat.\\[1ex]
& \textbf{Highlighted Evidence}:\\[0.5ex]
& \quad $-$ ``Japan implemented the `metabo' law in 2008 to combat rising obesity rates.''\\[0.5ex]
& \quad $-$ ``The New York Times reported that the law aims to shrink the overweight population by 10\% over 4 years and 25\% over 7 years via financial penalties.''\\[0.5ex]
& \quad $-$ ``In 2008, Japan passed the ``Metabo Law,'' addressing metabolic syndrome—a cluster of conditions increasing the risk of heart disease, stroke, and diabetes.''\\[0.5ex]
& \quad $-$ ``The law requires models to have a minimum BMI and warns against photoshopped images.''\\[1ex]
Bad Evidence &
\textbf{Question:} Ted Cruz Says Democrats are embracing abortion up until (and even after) birth.\\[1ex]
& \textbf{Highlighted Evidence}:\\[0.5ex]
& \quad $-$ ``In January 2016, Cruz announced his "Pro-Lifers for Cruz" coalition, with statements about executing abortion doctors to expunge bloodguilt.''\\[0.5ex]
& \quad $-$ ``Kamala Harris refuted Republican claims about Democrats' abortion views.''\\[0.5ex]
& \quad $-$ ``In the mid-1990s, Moynihan supported banning the procedure known as partial-birth abortion.''\\
\bottomrule
\end{tabularx}
\caption{Examples of Good, Medium, and Bad Highlighted Evidences}
\label{tab:evidence_quality_examples}
\end{table*}

\begin{table*}[t]
\centering
\footnotesize
\begin{tabularx}{\textwidth}{lX}
\toprule
\midrule
ARC-Challenge & 
\textbf{Question:} Scott filled a tray with juice and put it in a freezer. The next day, Scott opened the freezer. How did the juice most likely change?\\[1ex]
& \textbf{Evidence:}\\
& \quad - Most recently, Scott produced the documentary film “Apple Pushers” with Joe Cross (filmmaker) juicer and a generator.\\[0.5ex]
& \quad - However, in March 1996, 70,000 Juice Tiger juicers (9\% of its models) were recalled after 14 injury incidents were reported.\\[1ex]
ARC-Challenge & 
\textbf{Question:} A physicist wants to determine the speed a car must reach to jump over a ramp. The physicist conducts three trials. In trials two and three, the speed of the car is increased by 20 miles per hour. What is the physicist investigating when he changes the speed?\\[1ex]
& \textbf{Evidence:}\\
& \quad - Objects in motion often have variations in speed (a car might travel at 50 km/h, slow to 0 km/h, then reach 30 km/h).\\[0.5ex]
& \quad - Preparing an object for g-tolerance (avoiding damage when subjected to high speeds).\\[0.5ex]
& \quad - Hence, the round-trip time on traveler clocks will be $\Delta \tau = 4\left(\frac{c}{\alpha}\right)\cosh(\gamma)$.\\[1ex]
ARC-Challenge & 
\textbf{Question:} Human activities affect the natural environment in many ways. Which action would have a positive effect on the natural environment?\\[1ex]
& \textbf{Evidence:}\\
& \quad - This environment encompasses the interaction of all living species, climate, weather, and natural resources affecting human survival and economic activity.\\[0.5ex]
& \quad - For instance, actions by the U.S. Army Corps of Engineers that threatened ecosystems in Florida's Oklawaha River valley and issues in preserving Pacific Coast Redwood communities are cited as case studies.\\[0.5ex]
& \quad - Humans have contributed to the extinction of many plants and animals.\\
Pop-QA & 
\textbf{Question:} What is Antonio Álvarez Alonso's occupation?\\[1ex]
& \textbf{Evidence:}\\
& \quad - Antonio De Diego  Antonio de Diego Álvarez is a Spanish paracanoeist and member of the National Spanish Canoeist Team, Paracanoe class A (maximum level of disability).\\[0.5ex]
& \quad - Antonio Álvarez Alonso  Antonio Álvarez Alonso (11 March 1867 - 22 June 1903) was a Spanish pianist and composer.\\[0.5ex]
\bottomrule
\end{tabularx}
\caption{Examples of low-quality evidences retrieved for various types of queries from ARC-Challenge \& Pop-QA}
\label{tab:low_quality_evidences}
\end{table*}

\begin{table*}[t]
\centering
\footnotesize
\begin{tabularx}{\textwidth}{lX}
\toprule
\textbf{Dataset} & \textbf{Original Question and Step-back Question} \\
\midrule
ARC-Challenge &
\textbf{Original Question:} An astronomer observes that a planet rotates faster after a meteorite impact. Which is the most likely effect of this increase in rotation?\\[1ex]
& \textbf{Step-back Question:} What effects do meteorite impacts on planets have?\\[1ex]
ARC-Challenge &
\textbf{Original Question:} A group of engineers wanted to know how different building designs would respond during an earthquake. They made several models of buildings and tested each for its ability to withstand earthquake conditions. Which will most likely result from testing different building designs?\\[1ex]
& \textbf{Step-back Question:} What are the testing methods used by the engineers to determine the earthquake resilience of the different building models?\\[1ex]
PopQA &
\textbf{Original Question:} What is Henry Feilden's occupation?\\[1ex]
& \textbf{Step-back Question:} What are the important aspects of Henry Feilden's academic work?\\[1ex]
PubHealth &
\textbf{Original Question:} A mother revealed to her child in a letter after her death that she had just one eye because she had donated the other to him.\\[1ex]
& \textbf{Step-back Question:} What are the circumstances surrounding the donation of the mother's second eye to her child after her death?\\
\bottomrule
\end{tabularx}
\caption{Examples of Step-back questions created from original questions in the three datasets.}
\label{tab:stepback-examples}
\end{table*}

\section{Human Evaluation of Evidence Quality}
\label{sec:appendix1}
\subsection{Evaluation Criteria}
For each query, we categorize its corresponding set of highlighted evidence(s) as ``bad'' (score: 0) when it includes completely irrelevant sentences or sentences within contexts that are somewhat related to the query but fail to provide any meaningful support in addressing it. In the case of fact-checking datasets like PubHealth, we also classify highlighted evidence as ``bad'' if it appears to support a claim but overlooks negations in the surrounding context that would ultimately refute the claim.

Highlighted evidence is categorized as ``medium'' (score: 0.5) when it consists of sentences situated in relevant contexts that may allow the correct answer to be inferred indirectly in some instances but lack the direct or explicit support needed to answer the query.

Highlighted evidence is categorized as ``good'' (score: 1) when it includes a sufficient number of sentences that directly address the query while ensuring no confounding factors (e.g., negations in the surrounding context) are overlooked.

Table \ref{tab:evidence_quality_examples} shows an example of good, medium and bad quality evidences as assessed by a human evaluator. The example of \texttt{Good Evidence} shown is rated as such because connecting the evidence sentences together allows the reader to deduce the answer to the query ``What is the atomic mass of the atom?'' even without extensive prior knowledge of chemistry.


\section{Low Quality Evidence}
\label{sec:appendix2}
In Table \ref{tab:low_quality_evidences}, we include some examples of retrieved evidences from the ARC-C dataset that do not help the model to deal with specific tasks, especially those which requiring modeling negation and arithmetic reasoning.

\section{Step-Back Reasoning Examples}
\label{sec:appendix3}
Please refer to Table \ref{tab:stepback-examples} for examples of original queries and the more abstract \textit{Step-back} questions elicited from those queries.

\subsection{Step-back Prompt for Query Expansion}
\begin{lstlisting}
You are an expert at world knowledge. Your task is to step back and paraphrase a question to a more generic step-back question, which is easier to answer. Here are a few examples:

Original Question: Which position did Knox Cunningham hold from May 1955 to Apr 1956?
Stepback Question: Which positions have Knox Cunningham held in his career?
        
Original Question: who has scored most runs in t20 matches as of 2017
Stepback Question: What are the runs of players in t20 matches as of 2017
        
Original Question: When was the abolishment of the studio that distributed The Game?
Stepback Question: which studio distributed The Game?
        
Original Question: What city is the person who broadened the doctrine of philosophy of language from?
Stepback Question: who broadened the doctrine of philosophy of language
        
Original Question: Would a Monoamine Oxidase candy bar cheer up a depressed friend?
Stepback Question: What are the effects of Monoamine Oxidase?
        
What is the Stepback Question for this?: {original_question_text}
Answer with only the Stepback Question and no extra text.
\end{lstlisting}

\subsection{Step-back Prompt for MCQ Answer Choices}
\begin{lstlisting}
You are an expert at world knowledge. You are given a statement. Your task is to extract the concepts and principles underlying the statement. Answer only with the concepts and principles without any extra text.
If there are multiple concepts and principles, list them separated by commas.
Original Statement: {answer_text}
Answer:\end{lstlisting}



\section{Experimental Details}
\label{sec:appendix4}
Our experimental results for Mistral-7B-Instruct v0.1, Alpaca-7B \& Llama-2-7B differ from those reported by other works such as Self-RAG \cite{asai2024selfrag} \& CRAG \cite{yan2024corrective}, and Speculative RAG due to the following methodological variations:
\begin{enumerate}
    \item \textbf{Evaluation Function:} We employ a different evaluation criteria for assessing accuracy between Large Language Model (LLM) generations and gold labels in tasks such as ARC-Challenge, PopQA, and PubQA. Our approach considers an LLM generation correct based on the principle of ``inclusion,'' i.e., if the generation includes the correct answer as a substring, post-normalization.
    
    \item \textbf{Number of retrieved passages in DPR and BM25 (top-K):} In both BM25 and DPR retrieval, we set $K=11$ for models which have a 4096 token limit context (e.g., Llama-2-7B), where 10 passages are from the Wikipedia KB mixed with a web search result from CRAG.
    For Alpaca-7B and AMD-OLMo-1B-SFT, owing to their small context window size of 2048, we keep just the top-9 documents ($K=9$). For Alpaca and OlMo, we observe significant degradation if we use 10 or more documents causing the DPR setting to perform worse than even the \textit{No-Retrieval model}. For models with larger context windows e.g., Mistral-7B and Qwen2.5-3B we use all DPR and BM25 retrieved passages.

    \item \textbf{Prompt Engineering:} Our prompts differ slightly from those used in Self-RAG and C-RAG. We have engineered our prompts to adhere more closely to the recommended Instruction Tuning format, particularly for Alpaca-7B \cite{alpaca} and Llama-2-7B-chat \cite{touvron2023llama2openfoundation}.
    \item \textbf{Stepback-LLM:} In all experiments, we use Mistral-7B-Instruct v0.1 as the step-back LLM.
\end{enumerate}
\noindent These methodological distinctions should be considered when comparing our results with those of previous studies.

\section{Example Prompts}
\label{sec:appendix5}
Examples of the task specific prompts utilized in our study are as follows:
\small
\begin{itemize}
    \item \textbf{ARC-Challenge}
    \begin{itemize}
        \item Mistral-7B-Instruct:
        \begin{lstlisting}[basicstyle=\small\ttfamily,breaklines=true]
Refer to the following documents, follow the instruction and answer the question.

Documents: {highlighted_passages}

Question: {question}

Instruction: Given four answer candidates, A, B, C and D, choose the best answer choice.
Please answer with the capitalized alphabet only, without adding any extra phrase or period.
        \end{lstlisting}
        
        \item Alpaca-7B:
        \begin{lstlisting}[basicstyle=\small\ttfamily,breaklines=true]
Below is an instruction that describes a task. Write a response that appropriately completes
the request.

### Instruction: Given four answer candidates, A, B, C and D, choose the best answer choice.
Please answer with the capitalized alphabet only, without adding any extra phrase or period.

### Input:
Documents: {highlighted_passages}
Question: {question}
Choices: {choices_str}

### Response: 
        \end{lstlisting}

        \item Llama-2-7B-chat:
        \begin{lstlisting}[basicstyle=\small\ttfamily,breaklines=true]
Below is an instruction that describes a task. Write a response that appropriately completes
the request.

### Instruction: Given four answer candidates, A, B, C and D, choose the best answer choice.
Please answer with the capitalized alphabet only, without adding any extra phrase or period.

### Input:
Documents: {highlighted_passages}
Question: {question}
Choices: {choices_str}

### Response:  
        \end{lstlisting}
    \end{itemize}

    \item \textbf{PopQA}
    \begin{itemize}
        \item Mistral-7B-Instruct:
        \begin{lstlisting}[basicstyle=\small\ttfamily,breaklines=true]
Refer to the following documents, follow the instruction and answer the question.

### Input:
Documents: {highlighted_passages}

### Instruction: Answer the question: {question}
### Response:
        \end{lstlisting}        
        \item Alpaca-7B:
        \begin{lstlisting}[basicstyle=\small\ttfamily,breaklines=true]
Below is an instruction that describes a task. Write a response that appropriately completes
the request.

### Instruction: Refer to the following documents and answer the question.
### Input:
Documents: {highlighted_passages}

Question: {question}
### Response: 
        \end{lstlisting}
        \item Llama-2-7B:
        \begin{lstlisting}[basicstyle=\small\ttfamily,breaklines=true]
<s>[INST] <<SYS>>
    You are a helpful, respectful and honest assistant. Always answer as helpfully as possible,
    while being safe. Your answers should not include any harmful, unethical, racist, sexist,
    toxic, dangerous, or illegal content. Please ensure that your responses are socially unbiased
    and positive in nature.

    If a question does not make any sense, or is not factually coherent, explain why instead of
    answering something not correct. If you don't know the answer to a question, please don't
    share false information.
<</SYS>>

Below is an instruction that describes a task. Write a response that appropriately completes
the request.

Instruction: Refer to the following documents and answer the question.

Documents: {highlighted_passages}

Question: {question}
### Response: [/INST]
        \end{lstlisting}
    \end{itemize}
    
    \item \textbf{PubHealth}
    \begin{itemize}
        \item Mistral-7B-Instruct:
        \begin{lstlisting}[basicstyle=\small\ttfamily,breaklines=true]
Read the documents and answer the question: Is the following statement correct or not?
Only say true if the statement is true; otherwise say false. Don't capitalize or add periods,
just say ``true'' or ``false''.

Documents: {highlighted_passages}

Statement: {question}
### Response:
        \end{lstlisting}
        \item Alpaca-7B:
        \begin{lstlisting}[basicstyle=\small\ttfamily,breaklines=true]
Below is an instruction that describes a task. Write a response that appropriately completes
the request.

### Instruction: Read the documents and answer the question: Is the following statement correct
or not? Only say true if the statement is true; otherwise say false. Don't capitalize or add
periods, just say ``true'' or ``false''.

### Input:
Documents: {highlighted_passages}

Statement: {question}
### Response: 
        \end{lstlisting}
        \item Llama-2-7B:
        \begin{lstlisting}[basicstyle=\small\ttfamily,breaklines=true]
<s>[INST] <<SYS>>
    You are a helpful, respectful and honest assistant. Always answer as helpfully as possible,
    while being safe. Your answers should not include any harmful, unethical, racist, sexist,
    toxic, dangerous, or illegal content. Please ensure that your responses are socially unbiased
    and positive in nature.

    If a question does not make any sense, or is not factually coherent, explain why instead of
    answering something not correct. If you don't know the answer to a question, please don't
    share false information.
<</SYS>>

Below is an instruction that describes a task. Write a response that appropriately completes
the request.

### Instruction: Read the documents and answer the question: Is the following statement correct or not? Only say true if the statement is true; otherwise say false. Don't capitalize or add
periods, just say ``true'' or ``false''.

### Input:
Documents: {highlighted_passages}

Statement: {question}
### Response: [/INST]
        \end{lstlisting}
    \end{itemize}
\end{itemize}
\normalsize
