\label{app:projection}
In this section, we argue that the condition in the second statement of Lemma \ref{lem:proj1} is sufficient for the vector $\mathbf{y}$ to be an $\ell_1$-projection of $\mathbf{a}\in \Delta_U\setminus \Delta_\alpha$, for $\alpha\leq U$.

Following Lemma \ref{lem:proj1}, consider the collection $\mathcal{S}\subseteq \mathbb{R}^d$ of points defined as $\mathcal{S}:= \{\mathbf{z}\in \delta_\alpha:\ z_i\leq a_i,\ \text{for all $i\in [d]$}\}$. Note that  when $a_i>\alpha$, for all $i\in [d]$, we have $\mathcal{S} = \delta_\alpha$. Observe that $\mathcal{S}$ is a convex subset of $\mathbb{R}^d$; hence, by a version of Carath\'eodory's Theorem (see the remark after \cite[Thm. 15.3.5]{cover_thomas} and \cite{carath}), any point $\mathbf{z}$ in $\mathcal{S}$ can be written as a convex combination of finitely many, in particular, $d$ points in $\mathcal{S}$. Hence, consider any such collection $\mathcal{Z} = \{\mathbf{z}_1,\ldots,\mathbf{z}_d\}$ whose convex hull equals $\mathcal{S}$. The following claim then holds.
%Now, consider the case where $\mathbf{a}$ is such that $a_i\leq \alpha$, for some $i\in [d]$, and hence that $a_j\geq \alpha$, for some $j\in [d], j\neq i$.  the proof of the sufficiency of the second statement of Lemma \ref{lem:proj1} for this case is similar to that for the first case, and is omitted.

%For the first case above, consider the collection $\mathcal{Z} = \{\mathbf{z}_1,\ldots,\mathbf{z}_d\}$. Here, for $k\neq i$, we define $\mathbf{z}_k = (z_{k,1},\ldots,z_{k,d})$ to be such that $z_{k,i} = a_i$ and $z_{k,k} = \alpha-a_i$,  with $z_{k,r} = 0$, for $r\notin \{i,k\}$. We define $\mathbf{z}_i = (z_{i,1},\ldots,z_{i,d})$ to obey $z_{i,j} = \alpha$, with $z_{i,r} = 0$, for $r\neq j$. 

\begin{proposition}
	\label{prop:proj1}
	For any $\mathbf{z}\in \mathcal{S}$, we have that $\lVert \mathbf{a} - \mathbf{z}\rVert_1$ is a constant.
\end{proposition}
\begin{proof}
	Recall that by the version of Carath\'eodory's Theorem above, any point $\mathbf{z}\in \mathcal{S}$ can be written as $\mathbf{z} = \sum_{k=1}^d \lambda_k \mathbf{z}_k$, where $\lambda_k\geq 0$, with $\sum_{k\leq d} \lambda_k = 1$.  It suffices to show that $\lVert \mathbf{a} - \mathbf{z}\rVert_1$ is independent of $\{\lambda_k\}$. To see this, note that
	\begin{align*}
		\lVert \mathbf{a} - \mathbf{z}\rVert_1&= \sum_{r\leq d} (a_r-z_r)\\
		&=\sum_{r\leq d} (a_r-\sum_{k\leq d} \lambda_k z_{k,r})\\
		&= \sum_{r\leq d} a_r - \sum_{k\leq d} \lambda_k\cdot \sum_{r\leq d} z_{k,r} = \lVert \mathbf{a}\rVert_1 - \alpha,
	\end{align*}
which is independent of $\{\lambda_k\}$.
\end{proof}
We hence have that the condition in Lemma \ref{lem:proj1} is sufficient for the vector $\mathbf{y}$ in its statement to be an $\ell_1$-projection of $\mathbf{a}$. In practice, given a vector $\mathbf{a}\in \Delta_U\setminus \Delta_\alpha$, one can use the vector $\mathbf{y} = \frac{\alpha}{\lVert \mathbf{a}\rVert_1} \mathbf{a}$ as an $\ell_1$-projection.