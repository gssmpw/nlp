% \begin{figure*}
%     \centering
%     \includegraphics[width=0.85\linewidth]{images/fig1-framed-eval.png}
%     \caption{\name{} data construction process}
%     \label{fig:fig1}
% \end{figure*}
\begin{figure}[tb!]
    \centering
    \includegraphics[width=0.93\linewidth]{images/vertical-fig1-framed-eval-11.png}
    \caption{The \name{} data construction process. In step (a) we extract statements based on their syntactic structure, aiming for statements with clear negative or positive sentiment. Next, in (b), we \emph{reframe} the statement by adding a suffix or prefix, conveying the \emph{opposite} sentiment. Finally, in (c),  five annotators mark the sentiment of the reframed statement, counting how many annotators shift sentiment, i.e., 
    the reframed statement sentiment is opposite to the base sentiment. The red parts in the figure represent negative parts of statement, while green represents positive parts.}
    \label{fig:fig1}
\end{figure}