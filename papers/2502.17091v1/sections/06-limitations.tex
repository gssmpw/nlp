In this work, we address a cognitive bias, and as with any research involving human participants, our study has several limitations.

First, our framing experiment is conducted within a single domain -- Amazon reviews -- and focuses on a specific type of statement. Some of our findings may be artifacts of this dataset rather than generalizable patterns.

Additionally, our approach to framing is highly specific. We only manipulate statements by adding a prefix or suffix, whereas reframing can take many other forms, such as restructuring sentences or altering word choices to convey ambiguous sentiment. This may limit the generalizability of our results.

Furthermore, our study focuses solely on sentiment analysis. Other downstream tasks influenced by framing, such as question answering or decision-making, may exhibit different patterns of sensitivity. Investigating these tasks could provide further insights into the broader impact of framing on LLM behavior in real-world applications.

