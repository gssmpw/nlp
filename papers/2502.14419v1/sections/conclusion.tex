\section{Conclusion and future work}

This paper demonstrates how we have overcome the limits of Longhorn’s engine in environments that feature high-speed storage and networking hardware, by making key modifications while maintaining compatibility with the existing system's architecture. By replacing the iSCSI-based frontend with a state-of-the-art ublk implementation, optimizing controller-replica communication, and introducing Direct Block Store (DBS) for efficient block storage at the replica layer, our version of the engine achieves up to an order-of-magnitude better IOPS performance in our evaluation setup.
Furthermore, the detailed analysis of each change's impact helps in understanding the accumulative effect the optimizations have in overall performance, which may be of interest to developers of similar systems; the technologies we use and the methodology we follow should be equally applicable to other SDS stacks.
All proposed enhancements have been submitted as pull requests to the upstream Longhorn repository, paving the way for integration into future releases and wider adoption by the cloud-native storage community.
% All proposed enhancements are being prepared to be submitted as pull requests to the upstream Longhorn repository, paving the way for integration into future releases and wider adoption by the cloud-native storage community.
Future work is planned to focus on identifying further areas of improvement in the controller, targeting both performance gains, as well as integration of advanced features at the level of replication and data routing.
% As an example, we consider striping data across multiple devices at each node may offer even performance; which could be done at the controller or replica level.
DBS is also under active development; the roadmap includes inline compression and encryption to enhance the system’s utility for both cloud and on-premises deployments.
