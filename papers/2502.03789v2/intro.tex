\section{Introduction}
Maximin shares constitute a fundamental criterion when studying fair allocation of indivisible goods among agents with individual preferences \cite{hill1987partitioning,7c65302b-f079-361a-94f1-0c3c9f6fc76b}. Among $n$ agents, these shares are defined by considering a cut-and-choose thought experiment: Each agent partitions the grand bundle of goods into $n$ subsets and receives the least valued one, according to its valuation. The maximin share (MMS) is the maximum value the agent can guarantee for itself in this exercise. Further, an allocation of the indivisible goods among the agents is deemed to be (maximin share) fair if each agent receives a bundle of value at least as much as its maximin share. 

While conceptually central in the context of fair division of indivisible items \cite{amanatidis2023fair}, MMS fairness is not a feasible criterion: there exist fair division instances---even with agents that have additive valuations over the goods---wherein, via any allocation, one cannot ensure maximin shares simultaneously for all the agents. That is, there exist instances with additive valuations that do not admit MMS fair allocations; see \cite{10.1145/3140756,10.1145/2600057.2602835,feige2021tight}, and references therein. In light of the infeasibility of exact MMS fairness, multiple prior works have instead focused on approximation guarantees. These guarantees ensure that each agent receives a bundle of value at least $\alpha \in (0,1)$ times its maximin share for as large a value of $\alpha$ as possible. The best-known approximation bound gets progressively worse as one moves up valuation classes. In particular, this is $\alpha = \frac{3}{4} + \frac{1}{12n}$ for additive valuations \cite{10.1145/3391403.3399526}. For submodular and XOS valuations the best-known $\alpha$ is $\frac{1}{3}$ \cite{10.1145/3219166.3219238, barman2020approximation} and $\frac{1}{4.6}$ \cite{SEDDIGHIN2024104049}, respectively. Further, the approximation bounds for subadditive valuations are logarithmic in the number of agents \cite{SEDDIGHIN2024104049} or the number of goods \cite{10.1145/3219166.3219238}. In fact, monotone valuations do not admit any nontrivial approximation guarantee; specifically, there exist instances with two agents that have monotone valuations over four goods such that the maximin shares of both the agents is one, and under any allocation, one of the agents necessarily receives a bundle of value zero. 

We bypass these impossibility results and achieve exact MMS fairness by post facto adjustments in the supply of items. In particular, this work establishes that one can achieve exact MMS fairness via limited duplication of goods, even under monotone valuations. We also address fair chore division and show that, with limited disposal of chores, MMS fairness is feasible under monotone costs. Note that the obtained MMS fairness guarantees are with respect to the shares defined in the underlying instance. That is, we start with MMS shares, as defined in the given instance, and quantify the extent to which goods have to be duplicated, or chores have to be disposed of, to achieve fairness. 

The current framework is applicable in allocation settings wherein it is feasible to create copies of resources (goods) or dispose of tasks (chores). For instance, the work of Budish \cite{7c65302b-f079-361a-94f1-0c3c9f6fc76b} conforms to a similar treatment in the context of fair course allocation; see also \cite{budish2017course} and the {\it ScheduleScout} platform.\footnote{{https://www.getschedulescout.com/}} Indeed, class strengths of popular courses can often be increased to an extent. Duplication is also possible when allocating scalable resources, such as medical supplies and food donations. Another relevant example here is the fair assignment of students to schools. In this context, prior works \cite{procaccia2024school, santhini2024approximation} have shown that fairness---in particular, proportionality among demographic groups---can be achieved by increasing the capacities of selected schools. Capacity modification has been studied for other social choice objectives as well, including stability (see, e.g., \cite{nguyen2018near,gokhale2024capacity}) and Pareto optimality \cite{kumano2022quota}.  


\paragraph{Our Results.} 
We now summarize our contributions for fair division of 

\noindent
(i) Goods: These are indivisible items desired by the agents. Hence, in this context, we consider the agents' valuations to be (set-inclusion-wise) monotone non-decreasing.  

\noindent
(ii) Chores: Here the items to be assigned are undesirable, i.e., have monotone costs for the agents. 

In the standard discrete fair division setup, the subsets of indivisible items assigned to the agents are pairwise disjoint. By contrast, in the current framework, a limited number of goods can be duplicated, or some chores can be discarded. Hence, the assigned bundles are not necessarily disjoint or required to partition the underlying set of items. We use the term \emph{multi-allocation} $\calA = (A_1, \ldots, A_n)$ to highlight this aspect; here, $A_i$ is the subset of items assigned to agent $i$ and, as mentioned, some items may be present in multiple subsets or even in none. It is relevant to note that, throughout, we never assign multiple copies of the same item to the same agent. Hence, in any multi-allocation, each bundle $A_i$ is a subset and not a multiset of items. This model feature ensures that the agents' valuations defined over subsets of items in the underlying instance continue to be well-defined even when considering multi-allocations. 

The overarching goal of this work is to establish the existence of multi-allocations that provide each agent its maximin share and, at the same time, have bounded duplication of goods or disposal of chores. Towards a quantitative instantiation of this goal, we will consider the characteristic vectors $\chi^\calA \in \mathbb{Z}^m_+$ of multi-allocations $\calA =(A_1, \ldots, A_n)$;\footnote{This is a slight abuse of terminology, since the components of $\chi^\calA$ are integers and not necessarily zero or one.} here, $m$ denotes the total number of items in the given instance and the $j$th component of this vector denotes the number of copies of item $j \in [m]$ assigned among the bundles $A_1, \ldots, A_n$.    

For the case of goods, we establish the existence of MMS multi-allocations $\calA=(A_1, \ldots, A_n)$ with bounds on the $\ell_1$ and $\ell_\infty$ norms of the characteristic vectors $\chi^\calA$. Note that the $\ell_1$ norm, $\ellone{\chi^\calA}$, captures the total number of goods, with copies, assigned among the agents,  $\ellone{\chi^\calA} = \sum_{i=1}^n |A_i|$. In addition, $\ellinfty{\chi^\calA} = \max_{j \in [m]} \ \chi^\calA_j$ captures the maximum number of copies of any one good assigned under $\calA$. Hence, the $\ell_1$ and $\ell_\infty$ bounds quantify the extent of duplication of goods required to achieve MMS fairness. 

For chores, we upper bound the number of zero components of $\chi^\calA$. Specifically, we write $\ellzed{\chi^\calA} \coloneqq \{ j \in [m] \mid \chi^\calA_j = 0 \}$ and note $\ellzed{\chi^\calA}$ corresponds to the number of unassigned chores in any multi-allocation $\calA$. 

\begin{table}[!h]
\centering
% \footnotesize
\begin{tabular}{|c|cc|}
\hline
         & \multicolumn{2}{|c|}{\textbf{Goods}}                                                                                                                                                                                                                                \\ \hline         & \multicolumn{1}{c|}{\begin{tabular}[c]{@{}c@{}}Maximum multiplicity  \\  $\|\chi^\calA\|_\infty$\end{tabular}} & \begin{tabular}[c]{@{}c@{}}Total number of assigned goods (with copies) \\ $\|\chi^\calA\|_1$ \end{tabular} \\ \hline
         
{\begin{tabular}[c]{@{}c@{}} Monotone Valuations \\ (Theorem \ref{thm:monotone_goods}) \end{tabular}} & \multicolumn{1}{c|}  {\begin{tabular}[c]{@{}c@{}}{${3 \log m}^\star$}  \end{tabular}}
& {\begin{tabular}[c]{@{}c@{}}$m$ \end{tabular}}                                                                                                                                              \\ \hline
{\begin{tabular}[c]{@{}c@{}} Identically Ordered Valuations \\ (\Cref{thm:additive_goods}) \end{tabular}} & \multicolumn{1}{c|}{{\begin{tabular}[c]{@{}c@{}}{$12 \sqrt{ \log m}$} \end{tabular}}}                                                                       & \multicolumn{1}{c|}{{\begin{tabular}[c]{@{}c@{}}{$m + \frac{6 m \sqrt{\log m}}{\sqrt{n}}$} \\\end{tabular}}} 
                                                                  \\ \hline

{\begin{tabular}[c]{@{}c@{}}  Additive Valuations \\ (\Cref{thm:additive_3_3m}) \end{tabular}}  & \multicolumn{1}{c|}{\begin{tabular}[c]{@{}c@{}}{$2$} \end{tabular}}                                                                                    &  {\begin{tabular}[c]{@{}c@{}}{$2m$} \end{tabular}}                                                                                                                                                                    \\ \hline


{\begin{tabular}[c]{@{}c@{}} Monotone Valuations \\ with Entitlements \\ (Theorem \ref{thm:mms_hat_monotone_goods}) \end{tabular}} & \multicolumn{1}{c|}  {\begin{tabular}[c]{@{}c@{}}{${3 \log m}$}  \end{tabular}}
& {\begin{tabular}[c]{@{}c@{}}$\lceil 1.7 m \rceil$ \end{tabular}}                                                                                                                                              \\ \hline
\end{tabular}
\caption{The table presents upper bounds on the duplication of goods required to achieve MMS fairness. %the maximum multiplicity and the total number of assigned goods (with copies) in an MMS multi-allocation. 
The result marked $^\star$ is essentially tight; \Cref{thm:goods_lowerbound} provides a matching lower bound.}
%Here, $m$ denotes the number of goods in the given instance and $n$ denotes the number of agents. 
\label{tab:results}
\end{table}

\paragraph{Goods.} Our results for fair division of goods are listed below; see also Table \ref{tab:results}. 
\begin{itemize}
\item For monotone valuations, we prove that exact MMS fairness can be achieved while ensuring that, for each good, the number of copies assigned is at most $3 \log m$ and the total number of allocated goods, including copies, does not exceed the underlying count $m$. That is, there always exists an MMS multi-allocation $\calA$ whose characteristic vector has $\ell_{\infty}$ norm at most $3 \log m$ and $\ell_1$ norm at most $m$. The relevant implication of this result is that, even for the general setting of monotone valuations, one can achieve exact MMS fairness while ensuring that the total number of allocated goods does not exceed the given count $m$, and at most logarithmically many copies of any single good are required. We also show that here the $\ell_{\infty}$ bound is essentially tight by establishing a matching lower bound. 

This result for monotone valuations is obtained via the probabilistic method: We start with MMS-inducing partitions $\calM^i = (M^i_1, \ldots, M^i_n)$ for each agent $i$. Then, we sample a bundle $R_i$ uniformly at random from among the subsets $M^i_1, \ldots, M^i_n$ and show that the random multi-allocation $ (R_1, \ldots, R_n)$ satisfies the stated bounds with positive probability. The proof highlights that one can efficiently find MMS multi-allocations, with the stated $\ell_\infty$ and $\ell_1$ bounds, if the MMS-inducing partitions, $\calM^i$, for all the agents $i$ are given as input. At the same time, minimizing the $\ell_\infty$ norm in this setup is {\rm NP}-hard (\Cref{sec:reduction}).  

\item This work also obtains specialized bounds for instances in which the marginal values of the goods conform to a common order across all the agents and the subsets. In particular, we say that the valuations are identically ordered if there exists an indexing of the $m$ goods, $\{g_1, \ldots g_m\}$, such that for each pair of goods $g_s, g_t \in [m]$, with index $s < t$, and any $S \subset [m]$ that does not contain $g_s$ and $g_t$, each  agent values $S \cup \{g_s \}$ at least as much as $S \cup \{g_t \}$. For such identically ordered valuations, we establish an $\ell_{\infty}$ bound of $O(\sqrt{\log m})$ and an $\ell_1$ bound of $m +  \widetilde{O}\left(\frac{m}{\sqrt{n}} \right)$; where, $\widetilde{O}(\cdot)$ subsumes logarithmic dependencies. Note that here the $\ell_1$ norm additively exceeds the given number of goods, $m$, by only a lower-order term.   
 

\item For additive valuations,\footnote{Recall that a set function $v$ is said to be additive if the value, $v(S)$, of any subset $S$, is equal to the sum of the values of the elements in $S$.} we establish the existence of MMS multi-allocations $\calA$ in which the $\ell_{\infty}$ norm is upper bounded by $2$ and the $\ell_1$ norm is at most $2m$. 

\item We also address MMS fairness among agents with arbitrary entitlements; see \Cref{section:entitlements} for details. For this setting and under monotone valuations, we obtain upper bounds of $3 \log m$ and $\lceil 1.7 m \rceil$ for $\ell_\infty$ and $\ell_1$, respectively. 
 \end{itemize}

\paragraph{Chores.} As mentioned previously, in the context of chores, we upper bound the number of chores that need to be discarded for ensuring MMS fairness, i.e., we quantify the number of zero components, $\ellzed{\chi^\calA}$, for MMS multi-allocations $\calA$. Note that the notion of keeping some chores unassigned is analogous to the construct of charity, which has been studied in the context of the envy-freeness up to any good (EFx); see, e.g., \cite{chaudhury2021little}. Though, in contrast to prior works on fair division of goods with charity, the current paper focuses on chores and maximin shares.  We next list our results for fair division of chores; see also Table \ref{tab:chores}.
\begin{table}
\centering
\begin{tabular}{|c|c|}
\hline
                 & \textbf{Chores}                                                                                                                 \\ \hline
                 & \begin{tabular}[c]{@{}c@{}}Total number of unassigned chores \\ $\|\chi^\calA\|_z$\end{tabular}                                    \\ \hline
{\begin{tabular}[c]{@{}c@{}} Monotone Costs \\ (\Cref{thm:monotone_chores}) \end{tabular}}        & \begin{tabular}[c]{@{}c@{}}${\frac{m}{e}}^\star$ \end{tabular}                  \\ \hline
{\begin{tabular}[c]{@{}c@{}} Identically Ordered Costs \\ (\Cref{thm:additive_chores}) \end{tabular}} & \begin{tabular}[c]{@{}c@{}} $\frac{3m \left( \log m \right)^{\nicefrac{3}{2}}}{n^{\nicefrac{1}{4}}}$ \end{tabular} \\ \hline
{\begin{tabular}[c]{@{}c@{}} Additive Costs \\ (\Cref{thm:chores-additive}) \end{tabular}} & \begin{tabular}[c]{@{}c@{}} $\frac{2m}{11} + n$ \end{tabular} \\ \hline
\end{tabular}
\caption{The table presents upper bounds on the total number of unassigned chores for MMS fairness. The result marked $^\star$ is essentially tight; \Cref{thm:chores_lowerbound} provides a matching lower bound.} %Here, $m$ denotes the number of goods in the given instance and $n$ denotes the number of agents.
\label{tab:chores}
\end{table}

 \begin{itemize}
 \item Under monotone costs, we show that there always exists an MMS multi-allocation in which, out of the $m$ given chores, at most $\frac{m}{e}$ remain unassigned. We complement this upper bound by showing that there exist instances, with monotone costs, such that in every MMS multi-allocation, at least $(1- o(1))\frac{m}{e}$ chores remain unassigned.
\item For identically ordered costs, %\footnote{A cost function is said to be identically ordered if there is an ordering $\{a_1, a_2, \ldots, a_m\}$ over the set of $[m]$ chores such that, for any chore $\ell$ that appears after a chore $h$ in the ordering, and any fixed subset $S \subseteq [m]$ containing neither $\ell$ nor $h$, we have $c(S \cup \{h\}) \geq c(S \cup \{\ell\})$}
we establish that MMS fairness can be achieved while keeping $\widetilde{O} \left(\frac{m}{n^{1/4}} \right)$ chores unassigned. 
\item We further show that, under additive costs, one can ensure MMS fairness by discarding at most $\frac{2m}{11} + n$ chores
\end{itemize} 
 
\paragraph{Additional Related Work.}
As mentioned, Budish \cite{7c65302b-f079-361a-94f1-0c3c9f6fc76b} also considered the duplication of goods to achieve fairness. However, the result in \cite{7c65302b-f079-361a-94f1-0c3c9f6fc76b} confines to $1$-out-of-$(n+1)$ MMS, which is an ordinal approximation, and holds for additive (cancellable) valuations. The current work extends to exact MMS and covers monotone valuations. Further, the techniques utilized in this work are significantly different from the approach in \cite{7c65302b-f079-361a-94f1-0c3c9f6fc76b} of approximate competitive equilibrium from equal incomes. In fact, to the best of our knowledge, this work represents the first extensive use of probabilistic methods in the context of fair item allocation.

For fair division of chores, it was shown in \cite{Aziz_Rauchecker_Schryen_Walsh_2017} that there exist instances with additive costs that do not admit MMS fair allocations. Towards approximation guarantees, \cite{Aziz_Rauchecker_Schryen_Walsh_2017} showed that a $2$-approximately MMS chore division always exists under additive costs. This approximation bound was improved to $\frac{4}{3}$ \cite{barman2020approximation} and, subsequently, to $\frac{11}{9}$ \cite{10.1145/3465456.3467555}.