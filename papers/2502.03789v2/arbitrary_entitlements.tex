\section{Maximin Shares with Arbitrary Entitlements}
\label{section:entitlements}
We have thus far studied fair division among agents who have equal entitlements over the items. The current section complements this treatment and addresses agents with different entitlements. In particular, we consider settings in which each agent $i \in [n]$ is endowed with an entitlement $b_i \in (0,1]$ and the sum of the agents' entitlements is equal to one, $\sum_{i=1}^n b_i = 1$. Agents with higher entitlements $b_i$ stake a higher claim on the items. Also, the equal-entitlements setting considered in previous sections corresponds to $b_i = 1/n$, for all the agents. 


Focusing on division of goods, we will obtain results for a shared-based fairness notion, $\mmshat$, defined in \cite{Babaioff2024ShareBasedFF}. This notion generalizes MMS to the current context of distinct entitlements. In fact, Babaioff and Feige \cite{Babaioff2024ShareBasedFF} show that $\mmshat$ \emph{dominates} other shared-based notions.   

Here, we denote a fair division instance via the tuple $\langle [n], [m], \{v_i\}_{i=1}^n, \{b_i\}_{i=1}^n \rangle$ in which $b_i$s denote the agents' entitlements and, as in Section \ref{sec:goods}, $v_i$s denote the agents' valuations. We also index the agents in decreasing order of their entitlements, $b_1 \geq b_2 \geq \ldots \geq b_n$. The entitlement-adapted share, $\widehat{\mu}_i$ of each agent $i$ is obtained by considering size $n_i = \lfloor \frac{1}{b_i} \rfloor$ partitions of the $m$ goods. Formally,  
\begin{definition}[$\mmshat$] For any fair division instance $\langle [n], [m], \{v_i\}_{i=1}^n, \{b_i\}_{i=1}^n \rangle$ with goods and entitlements, the maximin share, $\muhat_i$, of agent $i$ is defined as
$$\muhat_i \coloneqq  \max_{(X_1, \ldots X_{n_i})\in \Pi_{n_i}([m])} \ \ \min_{j \in [n_i]} v_i(X_j). $$
Here, $n_i=\lfloor\frac{1}{b_i}\rfloor$ and the maximum is taken over all $n_i$-partitions of $[m]$. Furthermore, a multi-allocation $\calA = (A_1, \ldots A_n)$ is said to be an $\mmshat$ multi-allocation if it satisfies $v_i(A_i) \geq \widehat{\mu_i}$ for each agent $i \in [n]$. \\
\end{definition}

As before, we leverage the existence of an $\mmshat$-inducing $n_i$-partition $\calM^i=(M^i_1, M^i_2, \dots, M^i_{n_i})$ for each agent $i$, given by $$\calM^i \in \argmax_{(X_1, \ldots X_{n_i})\in \Pi_{n_i}([m])} \ \  \min_{j \in [n_i]} v_i(X_j).$$

The main result of this section is as follows.

\begin{theorem}
\label{thm:mms_hat_monotone_goods}
        Every fair division instance $\langle [n], [m], \{v_i\}_{i=1}^n, \{b_i\}_{i=1}^n \rangle$ with monotone valuations and arbitrary entitlements admits an $\mmshat$ multi-allocation $\calA = (A_1,\ldots A_n)$ in which no single good is assigned to more than $3 \log m$ agents and the total number of goods assigned, with copies, is at most $\lceil 1.7  m \rceil$. That is, the characteristic vector $\chi^\calA$ of $\calA$ satisfies $\|\chi^\calA\|_\infty \leq 3 \log m$ and $\|\chi^\calA\|_1\leq \lceil 1.7 m \rceil$.
\end{theorem}

As in the proof of \Cref{thm:monotone_goods}, we proceed via the probabilistic method. The random draw here is as follows: For each agent $i\in [n]$, recall that $\calM^i=(M^i_1,\dots,M^i_{n_i})$ is an $\mmshat$-inducing partition for $i$. Independently for each $i$, we select a bundle $R_i$ uniformly at random from $\{M^i_1,\dots,M^i_{n_i}\}$, i.e., $\prob{R_i = M^i_j}=1/n_i$ for each index $j\in [n_i]$. Considering the random $\mmshat$ multi-allocation $\calR=(R_1,\dots,R_n)$, let $\chi^\calR$ be the characteristic vector of $\calR$. We first prove the following lemma which bounds the expected value of the components of the random vector $\chi^\calR$.

\begin{lemma}
    $\expval{\chi^\calR_g} \leq 1.7$ for each component (good) $g \in [m]$. \label{lem:expval-entitlements}
\end{lemma}

\begin{proof}
    For each agent $i\in [n]$ and good $g\in [m]$, let $\mathbb{1}_{\{g \in R_i\}}$ be the indicator random variable for the event $g\in R_i$. Since $R_i$ is chosen uniformly at random from the $n_i$-partition $\{M^i_1,\dots,M^i_{n_i}\}$ of $[m]$ and each good $g\in [m]$ belongs to exactly one subset $M^i_j$, we have for each fixed $i\in [n]$ and $g \in [m]$
\begin{equation}
    \expval{\mathbb{1}_{\{g\in R_i\}}} = \prob{g \in R_i} = \frac{1}{n_i}. \label{eq:indicator-exp}
\end{equation}

Further, $\chi^\calR_g$ is the number of copies of the good $g$ assigned among the agents under the multi-allocation $\calR$, i.e., $\chi^\calR_g = \sum_{i=1}^n \mathbb{1}_{\{g \in R_i\}}$.  Therefore, 
\begin{equation}
\label{eq:eq1} \expval{\chi^\calR_g} = \sum_{i=1}^n \expval{\mathbb{1}_{ \{ g \in R_i \} }} 
\underset{\text{via (\ref{eq:indicator-exp})}}{=} \sum_{i=1}^n \frac{1}{n_i}
\end{equation}

We next establish an upper bound on the right-hand-side of equation (\ref{eq:eq1}). For each agent $i \in [n]$, let $k_i$ be the unique integer such that $\frac{1}{k_i+1} < b_i \leq \frac{1}{k_i}$. For the integer $k_i$, note that the bound $k_i \leq \frac{1}{b_i} < k_i + 1$ implies  
\begin{equation}
    k_i = \left\lfloor \frac{1}{b_i} \right\rfloor = n_i \label{eq:ni-value}
\end{equation}
Further, given that $\frac{1}{b_i} < k_i +1$, we obtain  $k_i > \frac{1}{b_i} -1$ and, hence, 
\begin{equation}
    \frac{1}{n_i} = \frac{1}{k_i} < \frac{1}{\frac{1}{b_i} - 1} = \frac{b_i}{1 - b_i}. \label{eq:upperbound-bi}
\end{equation}

Recall the agents are indexed such that $b_1 \geq b_2 \geq \dots \geq b_n$. We proceed via the following case analysis.\\

\noindent\textit{Case 1}: $b_1 \leq \frac{1}{3}$. Here, for each agent $i$, the entitlement $b_i \leq \frac{1}{3}$; equivalently, $1-b_i \geq \frac{2}{3}$. Therefore, 
\begin{align*}
\sum_{i=1}^n \frac{1}{n_i} 
& < \sum_{i=1}^n \frac{b_i}{1-b_i} \tag{via (\ref{eq:upperbound-bi})} \\
&\leq \frac{3}{2} \sum_{i=1}^n b_i \tag{since $1-b_i \geq \frac{2}{3}$ for all $i$} \\
&= \frac{3}{2} \tag{since $\sum_{i=1}^n b_i = 1$}
\end{align*}

\noindent\emph{Case 2}: $\frac{1}{3} < b_1 \leq \frac{1}{2}$. By the definition of $k_i$ and equation (\ref{eq:ni-value}), we obtain $n_1= k_1 = 2$. With $b_1\geq b_2$, we consider the following two sub-cases. \\

    \noindent
    \emph{Case 2(a):} $b_2 \leq \frac{1}{3}$.  For each $i \geq 2$, we have $b_i\leq \frac{1}{3}$ and, hence, $1 - b_i \geq \frac{2}{3}$. Therefore, 
    \begin{align*}
    \sum_{i=1}^n\frac{1}{n_i} 
    & < \frac{1}{n_1} + \sum_{i=2}^n \frac{b_i}{1-b_i} \tag{via (\ref{eq:upperbound-bi})} \\ 
    & \leq \frac{1}{2} + \frac{3}{2} \sum_{i=2}^n b_i \tag{since $n_1 = 2$ and $1 - b_i> \frac{2}{3}$ for $i\geq 2$} \\
    & = \frac{1}{2} + \frac{3}{2}(1-b_1)   \tag{since $\sum_{i=1}^n b_i = 1$}\\
    & <  \frac{1}{2} + \frac{3}{2}\left(1-\frac{1}{3}\right) \tag{since $b_1>\frac{1}{3}$}\\
    &= \frac{3}{2}
    \end{align*}

    \noindent
    \emph{Case 2(b):} $\frac{1}{3} < b_2 \leq b_1 \leq \frac{1}{2}$. In this sub-case, we have $n_2 = k_2 = 2$ as well. Since $\sum_{i=1}^n b_i = 1$, for the remaining agents $i \geq 3$, we have $b_i < \frac{1}{3}$. Hence,
    \begin{align*}
    \sum_{i=1}^n\frac{1}{n_i} 
    &<  \frac{1}{n_1} + \frac{1}{n_2} + \sum_{i=3}^n \frac{b_i}{1-b_i} \tag{via (\ref{eq:upperbound-bi})} \\ 
    & < \frac{1}{2}+\frac{1}{2} + \frac{3}{2} \sum_{i=3}^n b_i \tag{since $n_1 = n_2 = 2$ and $1 - b_i> \frac{2}{3}$ for $i\geq 3$} \\
    & = 1 + \frac{3}{2}(1-b_1-b_2)  \tag{since $\sum_{i=1}^n b_i = 1$}\\
    & <  1 + \frac{3}{2} \left(1-\frac{1}{3}-\frac{1}{3}\right) \tag{since $b_1\geq b_2 > \frac{1}{3}$} \\
    &= \frac{3}{2}.
    \end{align*}

\noindent\emph{Case 3}: $b_1> \frac{1}{2}$. In this case, $n_1 = k_1 = 1$. Also, $\sum_{i=1}^n b_i=1$ implies that $b_2 < \frac{1}{2}$. Therefore, the following three sub-cases (analyzed below) are exhaustive: 

\noindent
Case 3(a): $b_2 \leq 1/4$. 

\noindent
Case 3(b): $b_2 \in (1/4, 1/3]$. 

\noindent
Case 3(c): $b_2 \in (1/3, 1/2)$.  \\

   
   \noindent
   \emph{Case 3(a):} $b_2 \leq \frac{1}{4}$. Here, for all $i \geq 2$, we have $1 - b_i \geq 1 - b_2 \geq 3/4$. Using this inequality, we obtain   
   \begin{align*}
    \sum_{i=1}^n\frac{1}{n_i} 
    & < \frac{1}{n_1} + \sum_{i=2}^n \frac{b_i}{1-b_i} \tag{via (\ref{eq:upperbound-bi})} \\ 
    & \leq 1 + \frac{4}{3} \sum_{i=2}^n b_i \tag{since $n_1 = 1$ and $1 - b_i> \frac{3}{4}$ for $i\geq 2$} \\
    & = 1 + \frac{4}{3}(1-b_1)   \tag{since $\sum_{i=1}^n b_i = 1$}\\
    & <  1 + \frac{4}{3}\left(1-\frac{1}{2}\right) \tag{since $b_1>\frac{1}{2}$}\\
    &= \frac{5}{3}
    \end{align*}
    
   
   \noindent
   \emph{Case 3(b):} $\frac{1}{4} < b_2 \leq \frac{1}{3}$. In this sub-case, the definition of $k_i$ gives us $n_2 = k_2 = 3$. Further, for all $i \geq 3$, we have $b_i \leq b_3 \leq 1 - b_1 - b_2 \leq 1 - \frac{1}{2} - \frac{1}{4} = \frac{1}{4}$. That is, $1 - b_i \geq \frac{3}{4}$ for each $i \geq 3$. These inequalities give s 
    \begin{align*}
    \sum_{i=1}^n\frac{1}{n_i} 
    & < \frac{1}{n_1} + \frac{1}{n_2} + \sum_{i=3}^n \frac{b_i}{1-b_i} \tag{via (\ref{eq:upperbound-bi})} \\ 
    & \leq 1 + \frac{1}{3} + \frac{4}{3} \sum_{i=3}^n b_i \tag{since $n_1 = 1$, $n_2 = 3$, and $1 - b_i> \frac{3}{4}$ for $i\geq 3$} \\
    & = 1 + \frac{1}{3} +  \frac{4}{3}(1-b_1 - b_2)   \tag{since $\sum_{i=1}^n b_i = 1$}\\
    & <  1 + \frac{1}{3} + \frac{4}{3}\left(1-\frac{1}{2} - \frac{1}{4} \right) \tag{since $b_1>\frac{1}{2}$ and $b_2 > \frac{1}{4}$}\\
    &= \frac{5}{3}
    \end{align*}

    \emph{Case 3(c):} $\frac{1}{3} < b_2 < \frac{1}{2}$. Here, we have $n_2 = k_2 = 2$. Also, the equality $\sum_{i=1}^n b_i = 1$ implies $b_3 \leq 1-b_1 -b_2 < 1-\frac{1}{2} -\frac{1}{3} =\frac{1}{6}$. Hence, $1-b_i>\frac{5}{6}$ for each $i\geq 3$. Using these bounds we obtain    
    \begin{align*}
    \sum_{i=1}^n \frac{1}{n_i} 
    &<  \frac{1}{n_1} + \frac{1}{n_2} + \sum_{i=3}^n \frac{b_i}{1-b_i} \tag{via (\ref{eq:upperbound-bi})} \\ 
    & < 1 + \frac{1}{2} + \frac{6}{5} \sum_{i=3}^n b_i \tag{since $n_1 = 1$, $n_2 = 2$, and $1 - b_i> \frac{5}{6}$ for $i\geq 3$} \\
    & = 1 + \frac{1}{2} + \frac{6}{5}(1-b_1-b_2)  \tag{since $\sum_{i=1}^n b_i = 1$}\\
    & <  1 + \frac{1}{2} + \frac{6}{5} \left(1-\frac{1}{2}-\frac{1}{3}\right) \tag{since $b_1 >\frac{1}{2}$ and $ b_2 > \frac{1}{3}$} \\
    &= 1+ \frac{1}{2} + \frac{1}{5} \nonumber \\
    &= \frac{17}{10}.  
    \end{align*}

The case analysis above shows that, for any set of entitlements, $\{b_i\}_{i=1}^n$, the sum $\sum_{i=1}^n \frac{1}{n_i}$ is at most $\max \left\{\frac{3}{2}, \frac{5}{3}, \frac{17}{10} \right\} = \frac{17}{10} = 1.7$. Hence, equation (\ref{eq:eq1}) gives us $\mathbb{E}[\chi^\calR_g] \leq 1.7$. The lemma stands proved. 
\end{proof}

Using \Cref{lem:expval-entitlements}, we now prove probability bounds for the following events: $G_1$ is the event that $\|\chi^\calR\|_\infty \leq 3 \log m$ and $G_2$ is the event that $\|\chi^\calR\|_1 \leq \lceil 1.7m \rceil$. 

\begin{lemma}
\label{lem:linftynormforb} 
The random characteristic vector $\chi^\calR$ satisfies: 
\begin{enumerate}
    \item[(i)] $\prob{G_1^c} \leq \frac{1}{m^{2}}.$
    \item[(ii)] $\prob{G_2^c} \leq \frac{1.7m}{ \lceil1.7m\rceil+1}$.
\end{enumerate}
\end{lemma}
\begin{proof} 
To prove (i), note that for a fixed good $g\in [m]$, the indicator random variables $\mathbb{1}_{\{g\in R_i\}}$ are independent across $i$s. Hence, the count $\chi^\calR_g=\sum_{i=1}^n \mathbb{1}_{\{g\in R_i\}}$ is a sum of independent Bernoulli random variables. Using Chernoff bound \cite[Theorem 4.4]{mitzenmacher2017probability}, we have $\prob{\chi^\calR_g \geq t} \leq \frac{1}{2^t}$ for any $t \geq 6 \expval{\chi^\calR_g}$. Since $\expval{\chi^\calR_g} \leq 1.7 \leq 2$ (\Cref{lem:expval-entitlements}), we can instantiate this bound with $t = \max \{12, 3 \log m \} \geq   6 \ \expval{\chi^\calR_g}$. For an appropriately large $m$, we obtain $ t = 3 \log m$ and, hence, $\prob{\chi^\calR_g \geq 3 \log m} \leq \frac{1}{m^{3}}$ for each $g\in [m]$. Write $G_{1,g}^c$ to denote the event $\{\chi^\calR_g \geq 3\log m\}$, and note that $G_1^c = \cup_{g=1}^m G_{1,g}^c$. We have  $\prob{G_{1,g}^c}\leq \frac{1}{m^{3}}$ for each $g\in [m]$. Hence, the union bound gives us $$\prob{G_1^c} = \prob{\cup_{g=1}^m G_{1,g}^c} \leq \sum_{g=1}^m \prob{G_{1,g}^c} \leq m \frac{1}{m^{3}} =\frac{1}{m^{2}}.$$ This establishes part (i) of the lemma. 


For part (ii), using \Cref{lem:expval-entitlements}, we first obtain $\expval{\ellone{\chi^\calR}} 
= \expval{\sum_{g=1}^m \chi^\calR_g} 
= \sum_{g=1}^m \expval{\chi^R_g} 
\leq 1.7m$. Recall that $G_2$ is the event that $\|\chi^\calR\|_1 \leq \lceil 1.7m \rceil$. Markov's inequality and the fact that $\ellone{\chi^\calR} $ is an integer-valued random variable give us 
$$\prob{G_2^c} = \prob{\ellone{\chi^\calR} \geq \lceil 1.7m \rceil + 1} 
\leq \frac{\expval{\ellone{\chi^\calR}}}{ \lceil1.7m\rceil+1} 
\leq \frac{1.7m}{ \lceil1.7m\rceil+1}.$$ The theorem stands proved.
\end{proof}

Using the above lemma, we next prove \Cref{thm:mms_hat_monotone_goods}.

\begin{proof}[Proof of \Cref{thm:mms_hat_monotone_goods}] Recall that $G_1$ is the event that $\ellinfty{\chi^\calR} \leq 3 \log m$ and $G_2$ is the event that $\ellone{\chi^\calR} \leq \lceil 1.7m \rceil$. By \Cref{lem:linftynormforb} and the union bound, we have \begin{align*} \displaystyle \prob{G_1 \cap G_2} 
& = 1 - \prob{G_1^c \cup G_2^c} \\ 
& \geq 1 - (\prob{G_1^c} +\mathbb{P}\{G_2^c\}) \tag{Union bound} \\ 
& \geq 1-\left(\frac{1}{m^{2}} + \frac{1.7m}{\lceil1.7m\rceil+1} \right) \tag{by \Cref{lem:linftynormforb}} \\
& = \frac{\lceil 1.7m\rceil + 1 - 1.7m}{\lceil1.7m\rceil+1} - \frac{1}{m^{2}} \\
& \geq \frac{1}{\lceil1.7m\rceil+1} - \frac{1}{m^{2}} > 0.
\end{align*}

Hence, the events $G_1$ and $G_2$ hold together with positive probability for the random MMS multi-allocation $\calR = (R_1,\dots,R_n)$. This implies that there exists an MMS multi-allocation $\calA$ that satisfies the stated $\ell_\infty$ and $\ell_1$ bounds: $\|\chi^\calA\|_1\leq \lceil 1.7m \rceil $ and $\|\chi^\calA\|_\infty \leq 3 \log m$. This completes the proof. 
\end{proof}