
\subsection{Lower Bound for Goods}

We now show that there exist fair division instances $\calI$, with monotone valuations, such that each MMS multi-allocation in $\mathcal{I}$ ends up assigning at least one good to more than $\frac{\log m}{\log \log m}$ many agents. This lower bound shows that the positive result obtained in Theorem \ref{thm:monotone_goods}---for the $\ell_\infty$ bound---is essentially tight. 

\begin{theorem}
\label{thm:goods_lowerbound}
    There exists a fair division instance $\calI = \langle [n], [m], \{v_i\}_{i=1}^n \rangle$ with monotone valuations such that for every MMS multi-allocation $\calA$ in $\calI$, it holds that $\ellinfty{\chi^\calA} \geq \frac{\log m}{\log \log m}$, i.e., $\calA$ assigns some good to at least $\frac{\log m}{\log\log m}$ agents.
\end{theorem}

\begin{proof} We first describe the construction of the instance $\langle [n], [m], \{v_i\}_{i=1}^n \rangle$. For each agent $i \in [n]$ independently, we draw a partition $\mathcal{P}^i =(P^i_1, P^i_2, \ldots P^i_n)$ uniformly at random from the set of all possible $n^m$ possible partitions, i.e., $\calP^i \in_R \Pi_n([m])$. Given such a partition, we set the valuation $v_i$, for every subset $S \subseteq [m]$, as follows 
\begin{align*}
    v_i(S) \coloneqq \begin{cases}
    1 & \text{if }  P_j^i \subseteq S \text{ for any }j\in[n], \\
    0 & \text{otherwise}.
\end{cases}
\end{align*}
Note that $v_i$s are monotone set functions. In addition, the maximin share of every agent $i$ is equal to one, $\mu_i=1$. Hence, for any MMS multi-allocation $\calA=(A_1,\ldots,A_n)$ it must hold that $P^i_j \subseteq A_i$, for each agent $i \in [n]$ and some index $j \in [n]$. We can, in fact, restrict attention to those multi-allocations that satisfy $A_i=P_i^j$ for some $j \in [n]$, since the ones wherein $P^i_j \subsetneq A_i$ induce larger $\ell_\infty$ norms. Hence, given that partitions $\calP^i$ for the agents $i \in [n]$, we define the family $\calF$ of multi-allocations as 
\begin{align*}
    \calF \coloneqq \left\{ \calA=(A_1,\ldots,A_n) \mid \text{ for each } i\in[n] \text{ there exists } j\in[n] \text{ such that } A_i=P^i_j\right\}.
\end{align*}
Note that $|\mathcal{F}|=n^n$. Fix one multi-allocation $\calQ \in \calF$ obtained by setting index $j=1$ for all the agents $i$, i.e., $\calQ \coloneqq (P^1_1, P^2_1,\ldots, P^n_1)$. Write $\chi$ to denote the characteristic vector of $\calQ$. Now, consider the independent random draws that result in the  partitions $\calP^i=(P^i_1, \ldots P^i_n)$. Note that, for a fixed agent $i$ and a fixed good $g$, the probability that $g$ is contained in $i$'s bundle in $\calQ$ satisfies $\prob{g \in P^i_1} = 1/n$. Further, the independence of the draws (of $\calP^i$s) across the agents imply that the number of copies of $g$ assigned in $\calQ$ (i.e., $\chi_g$) is distributed as $\chi_g \sim \mathrm{Bin}(n, \frac{1}{n})$. For large $n$, this is approximated by the Poisson distribution $\mathrm{Poi}(n \  \frac{1}{n}) = \mathrm{Poi}(1)$. 
 
Define $\ell \coloneqq \frac{\log m}{\log \log m}$. Since each $\calP^i$ is drawn independently among all the $n$-partitions of $[m]$, the random variables $\chi_g$ across the goods $g \in [m]$ are independent. Therefore, using the fact that $\chi_g \sim \mathrm{Poi}(1)$ are independent and identically distributed for all $g \in [m]$, we obtain 
\begin{align}
    \prob{\ellinfty{\chi} \leq  \ell} = \left(\prob{\chi_g \leq \ell}\right)^m = \left(\sum_{k = 0}^{\ell} \frac{e^{-1}}{k!}\right)^m = \left(\frac{1}{e}\sum_{k = 0}^{\ell}\frac{1}{k!}\right)^m \label{eq:maxPoi}
\end{align}
Recall that $e = \sum_{k=0}^\ell \frac{1}{k!} + \sum_{k=\ell+1}^\infty \frac{1}{k!}$. 
 Hence, equation (\ref{eq:maxPoi}) simplifies to 
\begin{align*}
\prob{\ellinfty{\chi} \leq  \ell} = \left(\frac{1}{e} \left(e- \sum_{k=\ell+1}^\infty \frac{1}{k!} \right)\right)^m < \left( \frac{1}{e} \left(e - \frac{1}{(\ell+1)!} \right) \right)^m = \left(1 - \frac{1}{e(\ell+1)!}\right)^m.    \end{align*}
 
% Applying Sterling's approximation, we have $$e(\ell+1)! \approx e \left( \sqrt{2 \pi (\ell+1)} \left(\frac{\ell+1}{e}\right)^{\ell+1}\right)$$ 
We can extend the last bound as follows
\begin{align}
 \prob{\ellinfty{\chi} \leq  \ell} < \left(1 - \frac{1}{e(\ell+1)!} \right)^{e(\ell+1)!\left(\frac{m}{e(\ell+1)!}\right)}  \leq \left(\frac{1}{e}\right)^{\frac{m}{e(\ell+1)!}} \label{ineq:endPoi}    
\end{align}
 
Since $\ell = \frac{\log m}{\log \log m}$, for a sufficiently large $m \in \mathbb{Z}_+$, we have $m \geq 2 \  n \log n \  e (\ell +1)! $. For such an $m$, inequality (\ref{ineq:endPoi}) reduces to 
\begin{align}
    \prob{\ellinfty{\chi} \leq  \ell} <  \left(\frac{1}{e}\right)^{2n \log n} = \frac{1}{n^{2n}} \label{ineq:finalPoi}.
\end{align}
Inequality (\ref{ineq:finalPoi}) holds for each fixed MMS multi-allocation $\calQ \in \calF$. Hence, a union bound over all the $n^n$ multi-allocations in $\calF$ establishes the theorem. In particular, define the event $E_\calQ: = \left\{ \ellinfty{\chi^\calQ} > \ell \right\}$ for each $\calQ \in\mathcal{F}$. We have 
\begin{align*}
\prob{\bigcap_{\calQ \in \mathcal{F}} E_\calQ} = 1- \prob{\bigcup_{\calQ \in\mathcal{F}} E_Q^c} \geq 1 - n^n \cdot \prob{E_Q^c} \underset{\text{ via (\ref{ineq:finalPoi}})}{>} 1-\frac{1}{n^n} > 0    
\end{align*}


Therefore, with a non-zero probability, the events $E_\calQ$ hold together for all possible MMS multi-allocations $\calQ$. This implies that there exists an instance with a sufficiently large $m$ and a certain choice of the partitions $\calP_i=\{P^i_1, \ldots, P^i_n)$ (along with valuations defined as above) such that for all MMS multi-allocations, there exists a good $g$ with at least $\ell = \frac{\log m}{\log \log m}$ assigned copies. This completes the proof of the lower bound. 
\end{proof}

\begin{remark}
We note that, for exact MMS, the $\ell_\infty$ lower bound given in \Cref{thm:goods_lowerbound} holds even under XOS valuations. In particular, as in the proof of \Cref{thm:goods_lowerbound}, we select independently for each agent $i \in [n]$, a partition $\calP^i = (P^i_1, P^i_2, \dots, P^i_n)$ uniformly at random from the set $\Pi_n([m])$ of all possible $n$-partitions of $[m]$. Then, for each agent $i \in [n]$ and subset $S \subseteq [m]$, we set the valuation
\begin{align*}
v_i(S) \coloneqq  \max_{1\leq j\leq n} \ \frac{1}{|P^i_j|} |S \cap P^i_j|. 
\end{align*}
Note that the valuations $v_i$ are point-wise maximizers of additive functions of the form $S \mapsto \frac{1}{|P^i_j|} |S \cap P^i_j|$. Hence, $v_i$s are XOS. Further, $v_i(S)=1$ if $P^i_j \subseteq S$, for some index $j \in [n]$, and $v_i(S) < 1$ otherwise. Therefore, the partition $\calP^i = (P^i_1, \dots, P^i_n)$ certifies that, for each agent $i \in [n]$, the maximin share $\mu_i = 1$. Moreover, for any MMS multi-allocation $\calA = (A_i, \dots, A_n)$ it must hold that $P^i_j \subseteq A_i$ for each agent $i \in [n]$ and some index $j \in [n]$. This is exactly the property we utilize for the proof of \Cref{thm:goods_lowerbound}, and the rest of the proof relies solely on this condition. Therefore, the lower bound holds in particular for XOS valuations. 
\end{remark}