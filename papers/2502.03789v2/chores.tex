\section{Fair Division of Chores}
\label{sec:chores}

This section addresses fair division instances wherein the items to be assigned are undesirable to---i.e., have costs for---the agents. Such an instance is specified by a triple $\langle[n],[m],\{c_i\}_{i=1}^n\rangle$ with $n$ agents and $m$ items, called \textit{chores}. Here, the function $c_i:2^{[m]}\to\mathbb{R}_{\geq 0}$ specifies the (nonnegative) \textit{cost} of each subset of chores to agent $i$. A cost function $c_i$ is said to be monotone if the inclusion of chores into any subset does not decrease its cost: $c_i(S) \leq c_i(T)$ for every pair of subsets $S \subseteq T \subseteq[m]$. Further, $c_i$ is said to be additive if for every subset $S \subseteq [m]$ of chores, $c_i(S) = \sum_{a \in S} c_i(\{a\})$. As a shorthand, we will write $c_i(a)$ to denote agent $i$'s cost for chore $a \in [m]$. Throughout, the paper assumes that the agents' costs are monotone and normalized: $c_i(\emptyset)=0$ for all agents $i$. 

We will additionally consider (in \Cref{subsec:ordered-costs}) instances with cost functions that are identically ordered. 

The maximin share of each agent $i \in [n]$ is the minimum cost incurred by $i$ if the agent were to partition the $m$ chores into $n$ subsets and is then assigned the one with the highest cost. Formally,
\begin{definition}[MMS for chores] 
Given any fair division instance $\langle [n], [m], \{c_i\}_{i=1}^n \rangle$ with chores, the {maximin share}, $\mu_i \in \mathbb{R}_+$, of each agent $i \in [n]$ is defined as 
\begin{align*}
\mu_i \coloneqq  \min_{(X_1,\dots, X_n) \in \Pi_n([m])} \ \ \max_{j\in[n]} c_i(X_{j}).
\end{align*}
Further, for each agent $i$, let $\calM^i=(M^i_1, M^i_2, \ldots, M^i_n) \in \Pi_n([m])$ denote an {MMS-inducing partition}:
\begin{align*}
\calM^i \in \argmin_{(X_1,\dots, X_n) \in \Pi_n([m])} \ \ \max_{j\in[n]} c_i(X_{j})
\end{align*}
\end{definition} 

As before, an allocation $\calB=(B_1,\ldots, B_n)$ is a partition of $[m]$ into $n$ pairwise disjoint subsets $B_1,\ldots, B_n \subseteq [m]$. Here, the subset of chores $B_i$ is assigned to agent $i$ and is referred to as $i$'s bundle. Also, a \textit{multi-allocation} is a tuple $\calA=(A_1,\ldots,A_n)$ of $n$ subsets of chores, wherein subset $A_i \subseteq [m]$ denotes the bundle assigned to agent $i$. In contrast to allocations, in a multi-allocation, we do not require that the assigned bundles $A_i$ are pairwise disjoint and that they partition $[m]$. Hence, in a multi-allocation, a single chore may be present in multiple bundles or even in none. 


\paragraph{Fair Multi-Allocations.} A multi-allocation $\calA=(A_1,\dots,A_n)$ is said to be an \emph{MMS multi-allocation} if each agent receives a bundle of cost at most its maximin share, $c_i(A_i)\leq \mu_i$ for all agents $i \in [n]$.


Agents with monotone costs prefer to have fewer chores in their bundle and, hence, additional copies of any chore can be disposed of without impacting the MMS guarantee. Specifically, given any multi-allocation $\calA$ in which one chore $a$ is assigned to agents $i \neq i'$, one can consider multi-allocation $\calA'$ in which $a$ is assigned to only one of them, say $i$. If $\calA$ is MMS, then so is $\calA'$. That is, one can assume, without loss of generality, that the considered MMS multi-allocations $\calA$ satisfy $\ellinfty{\chi^\calA} \leq 1$, i.e., $\chi^\calA_a$ is always either $0$ or $1$ for each chore $a \in [m]$.\footnote{This makes such a multi-allocation in fact a \textit{partial allocation}.} Motivated by these considerations,  we will focus on quantifying the number of distinct chores that can be assigned among the agents while ensuring fairness. That is, in contrast to the goods setting where we obtained upper bounds for $\ell_1$ and $\ell_\infty$, here, we will focus on establishing upper bounds on the number of zero entries in the vector $\chi^\calA$. We will, throughout, write $\ellzed{\chi^\calA}$ to denote this quantity, $\ellzed{\chi^\calA} \coloneqq \{ a \in [m] \mid \chi^\calA_a = 0 \}$.\footnote{Note that this is not a norm.} Observe that $\ellzed{\chi^\calA}$ captures the number of chores that are unallocated under $\calA$; ideally, this quantity should be as small as possible. 


\subsection{Monotone Costs}
For instances with monotone costs, the following theorem shows that we can achieve MMS fairness while at most $m/e$ chores remain unassigned.

\begin{theorem}
\label{thm:monotone_chores}
    Every fair division instance $\langle [n], [m], \{c_i\}_{i=1}^n \rangle$ with chores and monotone costs admits an MMS multi-allocation $\calA=(A_1,\dots,A_n)$ in which at most $\frac{m}{e}$ chores remain unassigned. That is, the characteristic vector $\chi^\calA$ of $\calA$ satisfies $\ellzed{\chi^\calA}\leq\frac{m}{e}$. 
\end{theorem}

\begin{proof}
    Recall that $\calM^i=(M^i_1,\ldots,M^i_n)$ is an MMS-inducing partition for each agent $i \in [n]$. We select, independently for each $i$, a bundle $R_i$ uniformly at random from among $\{M^i_1, M^i_2, \dots,M^i_n\}$, and consider the random multi-allocation $\calR=(R_1,R_2,\dots,R_n)$. For any fixed chore $a\in[m]$, the events $\{a\notin R_i\}$ are independent across $i$ with $\prob{a\notin R_i}=1-\frac{1}{n}$. Therefore, under $\calR$, the probability that $a$ remains unallocated  $\prob{a\notin\cup_{i=1}^n R_i}=\left(1-\frac{1}{n}\right)^n \simeq \frac{1}{e}$. Hence, the total number of unallocated chores, in expectation, equals 
    \begin{align*}
    \expval{\ellzed{\chi^\calR}}
    &= \sum_{a=1}^m \prob{a\notin\cup_{i=1}^n R_i} = \frac{m}{e}.
    \end{align*} 
    Therefore, there exists an MMS multi-allocation $\calA$ that satisfies $\ellzed{\chi^\calA}\leq \frac{m}{e}$. The theorem stands proved. 
\end{proof}




\subsection{Identically Ordered Costs}
\label{subsec:ordered-costs}
This section establishes upper bounds on supply adjustments under cost functions are identically ordered; the setup here is analogous to the one given in \Cref{subsec:additive-ordered} for goods. Formally, 

\begin{definition} \label{def:identically-ordered-costs}
In a fair division instance $\langle [n], [m], \{c_i\}_{i=1}^n \rangle$, the cost functions are said to be \textit{identically ordered} if there exists an indexing $a_1, a_2, \ldots, a_m$ over the chores such that, for any pair of chores $a_s, a_t$, with index $s < t$, we have $c_i(S + a_s) \geq c_i(S + a_t)$ for all agents $i \in [n]$ and each subset $S$ that does not contain $a_s$ and $a_t$.  
\end{definition}

We show that for any fair division instance with identically ordered costs, there exists a multi-allocation $\calA$ such that the number of unassigned chores in $\calA$ is $\widetilde{O} \left(\frac{m}{n^{1/4}} \right)$.

%In particular, we have $c_i(a_1)\geq c_i(a_2)\geq \dots\geq c_i(a_m)$ for all agents $i$. 

\begin{theorem}
\label{thm:additive_chores}
    Every fair division instance $\langle [n], [m], \{c_i\}_{i=1}^n \rangle$ with chores and identically ordered costs admits an MMS multi-allocation $\calA=(A_1,\ldots A_n)$ wherein the number of unassigned chores is at most $\frac{3m \left( \log m \right)^{3/2}}{n^{1/4}}$, i.e., 
    $$\ellzed{\chi^\calA} \leq \frac{3m \left( \log m \right)^{3/2}}{n^{1/4}}.$$
\end{theorem}


\paragraph{Pre-Processing and Random Sampling.} For each agent $i$, consider an MMS-inducing partition $\calM^i=(M^i_1,\ldots,M^i_n)$. Rather than sampling an MMS multi-allocation directly, we first execute a pre-processing step, as detailed in Algorithm \ref{alg:chores-additive}. The algorithm is described in terms of a parameter $\alpha>0$, which will be fixed later. 


\begin{algorithm}
\caption{Pre-processing for Identically Ordered Costs}\label{alg:chores-additive}
\begin{algorithmic}[1]
    \REQUIRE An instance $\calI = \langle [n], [m], \{c_i\}_{i=1}^n \rangle$ along with MMS-inducing partitions $\calM^i=(M^i_1,\ldots,M^i_n)$, for the agents $i$, and parameter $\alpha>0$.
    \ENSURE A multi-allocation $\calB=(B_1,B_2,\dots,B_n)$ and sub-instance $\calI'$
    \STATE Initialize $N=[n]$, $U=[m]$ and $B_i=\emptyset$, for each agent $i\in[n]$.
    \WHILE{there exists agent $i\in N$ and index $j\in[n]$ such that $|M^i_j \cap U|\geq \alpha \ \frac{|U|}{|N|}$}
        \STATE Set $B_i=M^i_j\cap U$. 
        \STATE Update $N \leftarrow N\setminus \{i\}$ and $U \leftarrow U\setminus M^i_j$.
    \ENDWHILE
    \RETURN $\calB$ and sub-instance $\calI'=\langle N, U, \{ c_i \}_{i \in N} \rangle$.
\end{algorithmic}
\end{algorithm}

\Cref{alg:chores-additive} starts with $U = [m]$ as the set of unassigned chores and $N = [n]$ as the set of agents under consideration. The algorithm iterates as long as it is possible to assign to some agent $i \in N$ an MMS-inducing bundle, from $\calM^i$, of sufficiently large cardinality. Specifically, if for some agent $i \in N$ there exists a bundle $M^i_j$ such that $|M^i_j \cap U| \geq \alpha \ \frac{|U|}{|N|}$, then agent $i$ receives the bundle $B_i = M^i_j \cap U$. After this assignment, the algorithm removes both the agent $i$ and the assigned bundle from the instance. Since $c_i(B_i)=c_i(M^i_j\cap U)\leq c_i(M^i_j)\leq \mu_i$, each removed agent receives a bundle of cost at most its MMS.

After the algorithm terminates, we are left with the reduced sub-instance $\calI' = \langle N, U, \{c_i\}_{i\in N}\rangle$, where $N$ and $U$ are the sets of remaining agents (those yet to receive a bundle) and unallocated chores, respectively. A key property of $\calI'$ is that, for each remaining agent $i \in N$, the $n$ MMS-satisfying subsets $M^i_1 \cap U, M^i_2 \cap U, \ldots, M^i_n \cap U$ have size $|M^i_j \cap U| < \alpha \  \frac{|U|}{|N|}$. We will utilize this property and assign chores in $\calI'$ next. 

Note that, in $\calI'$, if $U = \emptyset$ (i.e., no chores remain unassigned at the end of the algorithm), then the upper bound stated in Theorem \ref{thm:additive_chores} holds. In fact, in such a case, for the underlying instance $\calI$, we can identify an MMS multi-allocation $\calA$ with $\ellzed{\chi^\calA}=0$, by setting $A_j = \emptyset$ for all $j \in N$ and $A_i = B_i$ for all $i \in [n] \setminus N$. 

Hence, we will assume that $|U|\geq 1$. Note that all the chores in $[m]\setminus U$ are assigned to some agent in the multi-allocation, $\calB$, returned by the algorithm. Hence, for any $k \in \mathbb{Z}_+$, the existence of an MMS multi-allocation $\calA'$ in $\calI'$ with $\ellzed{\chi^{\calA'}} \leq k$ (i.e., under $\calA'$ at most $k$ chores from $U$ remain unassigned) implies the existence of an MMS multi-allocation $\calA$ in $\calI$ with $\ellzed{\chi^\calA} \leq k$. In particular, we can obtain $\calA$ from $\calA'=(A'_j)_{j \in N}$ by setting $A_i = B_i$ for all $i \in [n] \setminus N$ and $A_j = A'_j$ for all $j \in N$. 

Towards finding such an MMS multi-allocation $\calA'$ in $\calI'$, with the upper bound $k$ as stated in Theorem \ref{thm:additive_chores}, we perform the following random sampling. Independently for each agent $i\in N$, we select a bundle $R_i$ uniformly at random from the set $\{M^i_1\cap U,\dots,M^i_n\cap U\}$, and consider the random multi-allocation $\calR=(R_i)_{i\in N}$ in $\calI'$. As mentioned previously, $ |M^i_j \cap U| < \alpha \  \frac{|U|}{|N|}$ for each agent $i \in N$ and index $j \in [n]$ and, hence, for any random draw, we have   
\begin{align}
    |R_i| < \alpha \ \frac{|U|}{|N|} \quad \text{for each $i \in N$} \label{ineq:sizeR}
\end{align}   

The following lemma provides a useful lower bound on the expected value of the components of $\chi^\calR$.

\begin{lemma}\label{lem:rem-agents}
    In the reduced instance $\calI'=\langle N,U,\{c_i\}_{i\in N}\rangle$ and under the above-mentioned random draws of MMS allocations $\calR=(R_i)_{i\in N}$, we have  
    \begin{align*}
        \expval{\chi^\calR_a}\geq 1-\frac{\log m}{\alpha} \quad \text{for each chore $a\in U$.}
    \end{align*}
\end{lemma}
\begin{proof}
    For the given instance $\calI = \langle [n], [m], \{c_i\}_{i=1}^n \rangle$, write $T$ to denote the number of iterations of the while-loop in \Cref{alg:chores-additive}. Note that the number of agents, $|N|$, remaining in the reduced instance $\calI'$ satisfies $|N| = n - T$. Also, let $U_t$ and $N_t$ denote the sets of remaining chores and agents, respectively, after $t \in \{0,\dots,T\}$ iterations of the while-loop. Note that, during each iteration $t+1$, at least $\alpha\ \frac{|U_t|}{|N_t|}$ chores are assigned to the selected agent. Hence, for each $t\in\{0,\dots,T\}$: 
    \begin{align*}
    |U_{t+1}| \leq|U_t|-\alpha\ \frac{|U_t|}{|N_t|}=|U_t|\ \left(1-\frac{\alpha}{|N_t|}\right) 
     \leq|U_t|\ \left(1-\frac{\alpha}{n}\right).
    \end{align*} 
    Therefore, after $T$ iterations 
    \begin{align}|U_T| \leq m \  \left( 1 - \frac{\alpha}{n} \right)^T \label{ineq:size-of-u} 
    \end{align}
    Since $|U_T|\geq 1$, equation (\ref{ineq:size-of-u}) yields $1-\frac{\alpha}{n}\geq \left(\frac{1}{m}\right)^\frac{1}{T}$. Exponentiating both sides of the previous inequality by  $\frac{n}{\alpha}$ gives us $\frac{1}{e}\simeq (1 - \frac{\alpha}{n})^{\frac{n}{\alpha}} \geq \left(\frac{1}{m}\right)^\frac{n}{\alpha T}$. Simplifying, we obtain $T\leq  \frac{n}{\alpha} \ \log m$ and, hence, 
    \begin{align}
        |N| = |N_T| = n- T \geq n  \left(1-\frac{\log m}{\alpha}\right) \label{ineq:size-of-N}
    \end{align}
    Finally, note that for each chore $a\in U$ 
    \begin{align*}
        \expval{\chi^\calR_a}= \sum_{i\in N} \prob{a\in R_i} = \frac{|N|}{n} \underset{ \text{via (\ref{ineq:size-of-N})}}{\geq} 1-\frac{\log m}{\alpha}.
    \end{align*}
This completes the proof of the lemma. 
\end{proof}

Since the underlying instance $\calI = \langle [n], [m], \{c_i\}_{i=1}^n \rangle$ consists of identically ordered costs, this property continues to hold for the reduced instance $\calI'=\langle N,U,\{c_i\}_{i\in N}\rangle$ as well. Specifically, we will assume, without loss of generality, that the set $U$ consists of chores $a_1,a_2, \ldots, a_{|U|}$, indexed such that for any $a_s$ and $a_t$, with $s<t$, we have $c_i(S + a_s) \geq c_i(S+a_t)$, for all agents $i \in N$ and all subsets $S \subset U$ that do not contain $a_s$ and $a_t$. We now reason about the number of chores assigned, with copies, under the random multi-allocation $\calR$ and in any given sequence of chores $S=\{a_r,a_{r+1},\ldots,a_{s-1}, a_s\}$. We denote by $\mathcal{S}$ the collection of all chore sequences in $U$, i.e., 
\begin{align*}
\mathcal{S}=\left\{ \{a_r,a_{r+1},\dots,a_s\}\mid 1\leq r\leq s\leq |U| \right\}.
\end{align*}


\begin{lemma}\label{lem:ordered-sequences}
In the reduced instance $\calI'=\langle N,U,\{c_i\}_{i\in N}\rangle$ and under the above-mentioned random draws of MMS allocations $\calR=(R_i)_{i\in N}$, the following bound holds for each chore sequence $S \in \mathcal{S}$: 
\begin{align*}
    \prob{\sum_{a\in S} \chi^\calR_a < |S|-\Delta}\leq \frac{1}{m^4},
    \end{align*}
    where 
    \begin{align*}
        \Delta \coloneqq |S|\ \frac{\log m}{\alpha} +2\alpha\ \frac{|U| \sqrt{\log m}}{\sqrt{|N|}}.
    \end{align*}
\end{lemma}

\begin{proof}
    \Cref{lem:rem-agents} gives us 
    \begin{align}
        \expval{\sum_{a\in S}\chi^\calR_a}\geq|S|  \left(1-\frac{\log m}{\alpha}\right) \label{eqn:exp-val-chore}
    \end{align} 
Note that $\sum_{a\in S}\chi^\calR_a=\sum_{i\in N}|R_i\cap S|$, where the random variables $|R_i\cap S|$ are independent across $i \in N$. Also, equation (\ref{ineq:sizeR}) provides a range for these random variables: $0\leq |R_i\cap S| <  \alpha\ \frac{|U|}{|N|}$, for each $i \in N$.
    
We now apply Hoeffding's inequality (lower tail) over the sum of independent random variables $|R_i\cap S|$, with parameter $\delta=2\alpha\ \frac{|U|}{\sqrt{|N|}}\ \sqrt{\log m}$:  
\begin{align}
\prob{\sum_{a\in S}\chi^\calR_a \leq|S|\ \left(1-\frac{\log m}{\alpha}\right)-\delta}
    &\leq \prob{\sum_{a\in S}\chi^\calR_a \leq \expval{\sum_{a\in S}\chi^\calR_a}-\delta} \tag{via (\ref{eqn:exp-val-chore})}\\
    & = \prob{ \sum_{i\in N}|R_i\cap S| \leq \expval{\sum_{i\in N}|R_i\cap S|}-\delta} \nonumber \\ 
    & \leq \exp \left( \frac{- \delta^2}{|N| \alpha^2 \frac{|U|^2}{|N|^2} } \right) \nonumber \\ 
    &\leq \exp(-4\log m) \tag{via value of $\delta$} \\
    &=\frac{1}{m^4} \label{ineq:tail}.
\end{align} 
    Inequality (\ref{ineq:tail}) corresponds to the bound stated in the lemma statement, since
    \begin{align*}
    |S|\ \left(1-\frac{\log m}{\alpha}\right)-\delta &= |S| - |S| \frac{\log m}{\alpha} - 2\alpha \frac{|U|}{\sqrt{|N|}}\ \sqrt{\log m} 
    = S-\Delta.
    \end{align*}
    Recall that $\Delta=|S|\frac{\log m}{\alpha} +2\alpha \frac{|U| \sqrt{\log m}}{\sqrt{|N|}}$. The lemma stands proved. 
\end{proof}

The following lemma shows that covering guarantees with respect to chore sequences, such as the ones obtained in Lemma \ref{lem:ordered-sequences}, translate to upper bounds on the number of unassigned chores. 

\begin{lemma}
\label{lem:shifting_chores}
In fair division instance $\calI'=\langle N,U,\{c_i\}_{i\in N}\rangle$ and for any parameter $\Delta\in \mathbb{Z}_{\geq 0}$, let $\calQ=(Q_i)_{i \in N}$ be a multi-allocation whose characteristic vector $\chi^\calQ$ satisfies $\sum_{a\in S} \chi^\calQ_a\geq |S|-\Delta$ for every chore sequence $S\in\mathcal{S}$. Then, $\calI'$ admits a multi-allocation $\calA'=(A'_i)_{i \in N}$ with the properties that 
\begin{itemize}
        \item[(i)] $c_i(A'_i)\leq c_i(Q_i)$ for all the agents $i$.
        \item[(ii)] $\ellzed{\chi^{\calA'}} \leq \Delta$.
    \end{itemize}
\end{lemma}


\begin{algorithm}
\caption{Copy Redistribution for Identically Ordered Costs}\label{alg:chores-shifting-2}

\begin{algorithmic}[1]
    \REQUIRE A multi-allocation $\calQ$ satisfying the condition of \Cref{lem:shifting_chores}.
    \ENSURE A multi-allocation $\calA'$.
    \STATE Initialize $\calA'=\calQ$.
    \WHILE{there exist chores $b$ and $b'$ with the following properties: (i) index of $b$ is less than that of $b'$ (i.e., $b$ has higher marginal costs than $b'$), (ii) $\chi^{\calA'}_{b}\geq 2$, and (iii) $\chi^{\calA'}_{b'}=0$}
    \STATE Let $a_s$ be the lowest-indexed chore with $\chi^{\calA'}_{a_s}\geq 2$ and let $a_t$ be the lowest-indexed chore such that $t > s$ and $\chi^{\calA'}_{a_t} = 0$. \label{line:lowchore}
    \STATE Select an agent $i$ with $a_s\in A'_i$ and update $A'_i \leftarrow (A'_i \setminus \{a_s \} ) \cup \{a_t\}$.
    \ENDWHILE
    \RETURN $\calA'$
\end{algorithmic}
\end{algorithm}

\begin{proof}
    We will show that---given a multi-allocation $\calQ$ that satisfies the stated guarantees with respect to chore sequences---\Cref{alg:chores-shifting-2} returns the desired multi-allocation $\calA'$. \Cref{alg:chores-shifting-2}, in each iteration of its while-loop, replaces a chore $a_s$ in the selected agent $i$'s bundle, $A'_i$, by another chore $a_t$ of higher index $t > s$. Since $a_s$ was previously allocated to at least two agents and $a_t$ was unallocated, the number of unallocated chores in the maintained multi-allocation $\calA'$ decreases by one in each iteration. That is, the algorithm terminates in polynomial time. Moreover, given that $a_t$ has a higher marginal cost than $a_s$ with respect to $A'_i \setminus \{a_s\}$, the agents' allocated costs $c_i(A'_i)$ are non-increasing throughout the execution of the algorithm. Hence, for the returned multi-allocation $\calA'$, in particular, we have $c_i(A'_i) \leq c_i(Q_i)$, for all the agents $i$. This observation establishes property (i) as stated in the lemma. 
    
    
    We next establish property (ii) by proving that upon termination of the algorithm, the returned multi-allocation $\calA'$ satisfies $\ellzed{\chi^{\calA'}} \leq \Delta$. Assume, towards a contradiction, that $\ellzed{\chi^{\calA'}}\geq \Delta+1$. Write $a_r$ to denote the chore of highest index $r$ with the property $\chi^{\calA'}_{a_r}=0$. Also, let $S$ denote the sequence of chores in $U$ with index at most $r$. Since $a_r$ is the highest index chore that is not assigned ($\chi^{\calA'}_{a_r}=0$), we have $\chi^{\calA'}_{b'} \geq 1$ for all chores $b'$ with index higher than $r$. Hence, all the unassigned chores are in fact contained in $S$; recall that there are at least $(\Delta +1)$ unassigned chores under $\calA'$. Further, the fact that the while-loop of the algorithm terminated with multi-allocation $\calA'$ and $\chi^{\calA'}_{a_r}=0$ ensures that $\chi^{\calA'}_b \leq 1$ for all chores $b \in S$. These observations imply that 
    \begin{align}
        \sum_{b \in S} \chi^{\calA'}_b \leq |S| - \Delta -1 \label{ineq:prefix-delta}
    \end{align}
    Further, note that, since $\chi^{\calA'}_{a_r} = 0$, during any iteration of the while-loop, the algorithm would not have replaced a copy of a chore $a_s \in S$ with a copy of a chore $a_t \notin S$. That is, it could not have been the case that, during any iteration, the considered chores $a_s$ and $a_t$ satisfy $s < r < t$, since such a relation between the indices contradicts the selection criterion of $a_t$ as the lowest-index chore with $\chi^{\calA'}_{a_t} = 0$. Therefore, throughout the algorithm's execution, for any considered chore $a_s \in S$, the corresponding chore $a_t$ must have also been contained in $S$. Considering this property along with the decrements and increments of $\chi^{\calA'}_{a_s}$ and $\chi^{\calA'}_{a_t}$, respectively, we obtain $\sum_{b \in S} \chi^\calQ_b = \sum_{b \in S} \chi^{\calA'}_b$. 

    The last equality and equation (\ref{ineq:prefix-delta}) give us $\sum_{b \in S} \chi^\calQ_b \leq |S| - \Delta -1$. This bound, however, contradicts the condition provided for $\calQ$ in the lemma statement. Hence, by way of contradiction, we obtain property (ii), $\ellzed{\chi^{\calA'}} \leq \Delta$. The lemma stands proved. 
\end{proof}

\medskip

    \begin{proof}[Proof of \Cref{thm:additive_chores}]

    Let $\calI=\langle[n],[m],\{c_i\}_{i=1}^n\rangle$ be the original fair division instance with chores and additive ordered costs. We first execute the pre-processing via \Cref{alg:chores-additive} (with parameter $\alpha \geq 1$, to be fixed later) and obtain a multi-allocation $\calB=(B_1,\ldots,B_n)$ along with sub-instance $\calI'=\langle N,U,\{c_i\}_{i\in N}\rangle$. Recall that $N$ and $U$ are the sets of remaining agents and unallocated chores, respectively. Also, all the chores in $[m] \setminus U$ are assigned among the agents in the set $[n] \setminus N$ with MMS guarantees; in particular, $\cup_{i \in [n] \setminus N} B_i = [m] \setminus U$ and $c_i(B_i) \leq \mu_i$ for each agent $i \in [n] \setminus N$.  
    
    We then focus on the sub-instance $\calI'=\langle N,U,\{c_i\}_{i\in N}\rangle$ with the goal of assigning chores from $U$ while ensuring MMS among the agents in $N$. Towards this and as mentioned previously, we sample an MMS multi-allocation $\calR=(R_i)_{i\in N}$ by drawing $R_i$ uniformly at random from $\{M^i_j\cap U\}_{j=1}^n$ for each $i\in N$. \Cref{lem:ordered-sequences} ensures that, for each chore sequence $S\in \mathcal{S}$, we have  
    \begin{align}
        \prob{\sum_{a\in S} \chi^\calR_a < |S|-\Delta}\leq \frac{1}{m^4}  \qquad \text{where } \Delta =|S|\ \frac{\log m}{\alpha} +2\alpha\ \frac{|U| \sqrt{\log m}}{\sqrt{|N|}} \label{eq:each-sequence}
    \end{align} 
    We fix $\alpha \coloneqq  n^{1/4} \log m$. This choice and inequality (\ref{ineq:size-of-N}) gives us $|N| \geq n \left(1 - \frac{\log m}{\alpha} \right) = n - n^{3/4} = n(1 - o(1))$. Further, 
     \begin{align}
        \Delta & =|S|\ \frac{\log m}{\alpha} \ + \ 2\alpha\ \frac{|U| \sqrt{\log m}}{\sqrt{|N|}} \nonumber \\ 
        & \leq \frac{m\log m}{\alpha} \ + \ \frac{2\alpha m\sqrt{\log m}}{\sqrt{|N|}} \tag{since $|S| \leq |U| \leq m$} \\
        & = \frac{m}{n^{1/4}}  \ + \ \frac{2  m n^{1/4} \left( \log m \right)^{3/2}}{ \sqrt{|N|} } \tag{$\alpha = n^{1/4} \log m$ and $|N| \geq n(1- o(1))$} \\
        & \leq \frac{3  m \left(\log  m \right)^{3/2}}{n^{1/4}} \label{eq:eq_alpha}
    \end{align}
    
    
    Write $\Delta^*$ to denote the upper bound on the deviation obtained in equation (\ref{eq:eq_alpha}), $\Delta^* \coloneqq \frac{3m \left(\log  m\right)^{3/2}}{n^{1/4}}$. Given that $|\mathcal{S}| \leq |U|^2\leq m^2$, applying union bound over all the chore sequences in $\mathcal{S}$ gives us  
    \begin{align*}
        \prob{ \sum_{a\in S} \chi^\calR_a \geq |S|-\Delta^*  \ \text{ for each } S\in\mathcal{S}}
    \geq 1 - \sum_{S \in \mathcal{S}} \prob{\sum_{a \in S} \chi^\calR_a < |S|-\Delta^*} \underset{\text{via (\ref{eq:each-sequence})}}{\geq} 
    1-\frac{m^2}{m^4}>0.
    \end{align*}
    Since this probability is positive, there exists an MMS multi-allocation $\calQ=(Q_i)_{i\in N}$ that satisfies $\sum_{a\in S} \chi^{\calQ}_a \geq |S|-\Delta^*$ for every chore sequence $S \in \mathcal{S}$. This is the condition required to invoke \Cref{lem:shifting_chores}, which, in turn, implies the existence of an MMS multi-allocation $\calA'=(A'_i)_{i\in N}$ (in $\calI'$) with $\ellzed{\chi^{\calA'}}\leq \Delta^*$. The final multi-allocation $\calA=(A_1,\ldots,A_n)$ for the underlying instance $\calI$ is obtained by combining $\calA'$ and $\calB$: For each agent $i$, set 
    \begin{align*}
        A_i=\begin{cases}
        A'_i & \text{ if }i\in N,\\
        B_i &\text{ otherwise, if $i \in [n] \setminus N$}.
    \end{cases}
    \end{align*}
    By construction, for each agent $i \in [n]$, we have $c_i(A_i) \leq \mu_i$. Hence, $\calA$ is an MMS multi-allocation. Also, since all the chores in $[m] \setminus U$ are assigned under $\calB$, we have that $\ellzed{\chi^\calA} = \ellzed{\chi^{\calA'}} \leq \Delta^*$. Overall, we get that $\calA$ satisfies the properties stated in the theorem. This completes the proof.  
    \end{proof}
       

    \subsection{Lower Bound for Chores}

    We now prove that there exists fair division instances, with $m$ chores and monotone costs, such that in every MMS multi-allocation, at least $(1- o(1))\nicefrac{m}{e}$ chores remain unassigned. This lower bound shows that the positive result obtained in Theorem \ref{thm:monotone_chores} is essentially tight. 

    \begin{theorem}
    \label{thm:chores_lowerbound}

 For any $\delta>0$, there exists a fair division instance $\calI = \langle [n], [m], \{c_i\}_{i=1}^n \rangle$ with monotone costs such that under every MMS multi-allocation $\calA$ in $\calI$ at least $(1-\delta)\frac{m}{e}$ chores remain unassigned, i.e., for every MMS multi-allocation $\calA$ in $\calI$ we have $\ellzed{\chi^\calA}\geq (1-\delta)\frac{m}{e}$. 
\end{theorem}

\begin{proof}
    We will construct an instance with $n$ agents and $m$ chores with $m> \frac{2e}{\delta^2}n\log n$. For each agent $i$, we independently draw an $n$-partition $\calP^i=(P^i_1,P^i_2,\dots,P^i_n)$ of the set $[m]$ of chores uniformly at random from the set of all such partitions, i.e., $\calP^i \in_R \Pi_n([m])$. Given such a partition, we set the cost of each subset $S\subseteq [m]$ as 
    \begin{align*}
        c_i(S) =\begin{cases}
        0  & \text{if } S\subseteq P^i_j \text{ for any $j \in [n]$,}\\
        1  & \text{otherwise}.
    \end{cases}
    \end{align*}
    
    
     Note that each $c_i$ is a monotone set function. Further, for each agent $i$ the MMS value $\mu_i=0$, which implies that for any MMS multi-allocation $\calA=(A_1,\ldots,A_n)$ we have  $A_i\subseteq P^i_j$ for each $i\in [n]$ and some index $j\in[n]$. We may, in fact, restrict ourselves to those multi-allocations that satisfy $A_i=P^i_j$, for some $j$, since those that do not will only leave more chores unallocated. Hence, we define the family $\calF$ of MMS multi-allocations as 
     \begin{align*}
         \calF=\left\{ (A_1,A_2,\dots,A_n) \mid \text{ for each } i\in[n], \text{ there exists }j\in[n]\text{ such that } A_i=P^i_j\right\}.
     \end{align*} 
     Note that $|\mathcal{F}|=n^n$. Fix the multi-allocation $\calQ\in\calF$ obtained by setting index $j=1$ for all agents $i$, i.e. $\calQ=(P^1_1,P^2_1,\dots,P^n_1)$. Considering the random draws that result in the partitions $\calP^i= (P^i_1,\dots,P^i_n)$, for each fixed agent $i$ and chore $a$, we have $\prob{a\in P^i_i}=1/n$. Further, the fact that these draws are independent across the agents gives us $\prob{a\notin\cup_{i=1}^n P^i_1} = \left(1-\frac{1}{n}\right)^n$. Hence, 
     \begin{align*}
     \expval{\ellzed{\chi^\calQ}} =\sum_{a=1}^m \prob{a\notin\cup_{i=1}^n P^i_1} = \sum_{a=1}^m \left(1-\frac{1}{n}\right)^n \simeq\frac{m}{e}.
     \end{align*} 
     We next utilize Chernoff bound (lower tail), with $\delta>0$, to obtain 
     \begin{align}
     \prob{\ellzed{\chi^{\calQ}} < (1-\delta)\frac{m}{e}} \leq \exp\left(-\frac{\delta^2m}{2e}\right)< \frac{1}{n^n} \label{eq:each-A}
     \end{align} 
     The last inequality follows since $m> \frac{2e}{\delta^2}n\log n$. Note that the bound (\ref{eq:each-A}) holds for each MMS multi-allocation $\calQ \in \calF$. Let $E_\calQ$ denote the event that $\ellzed{\chi^{\calQ}}\geq (1-\delta)\frac{m}{e}$ and note that $\prob{E^c_\calQ} \leq \frac{1}{n^n}$. Now, applying union bound over all $\calQ \in \calF$ gives us  
     \begin{align*}
         \prob{\bigcap_{\calQ \in \calF} E_\calQ} = 1-\prob{\bigcup_{\calQ\in\calF} E_\calQ^c}\geq 1-n^n\  \prob{E_\calQ^c}>1-\frac{n^n}{n^n}=0.
     \end{align*}
     Therefore, with non-zero probability, the events $E_\calQ$ hold together for all MMS multi-allocations $\calQ\in \calF$. This implies that there exists an instance $\calI$, determined by the choice of partitions $\calP^i=(P^i_1,\dots,P^i_n)$, such that every MMS multi-allocation in $\calI$ leaves at least $(1-\delta)\frac{m}{e}$ chores unallocated. This completes the proof.

    
\end{proof}



\subsection{Additive Costs}

When agents' costs are additive, the following theorem shows that MMS fairness can be achieved by leaving at most $\frac{2}{11}m + n$ chores unassigned.

\begin{theorem}
    \label{thm:chores-additive}
        Every fair division instance with chores and additive costs admits an MMS multi-allocation $\calA = (A_1, \dots, A_n)$ in which at most $\frac{2}{11}m + n$ chores remain unassigned. That is, the characteristic vector $\chi^\calA$ of $\calA$ satisfies $\ellzed{\chi^\calA} \leq \frac{2}{11}m + n$.
    \end{theorem}
    
    \begin{proof}
        The work of Huang and Lu \cite{10.1145/3465456.3467555} shows that fair division instances with chores and additive costs admit an exact allocation that assigns to each agent a bundle of cost no more than $\frac{11}{9}$ times its MMS. Let $\calA' = (A'_1, \dots, A'_n)$ be such an allocation for the given instance. Specifically, we have \begin{equation} c_i(A'_i) \leq \frac{11}{9} \mu_i \label{eq:chores-approx}
        \end{equation} for each agent $i \in [n]$.
    
        Consider the bundle $A'_i$ received by some fixed agent $i$ under $\calA'$, and assume that $|A'_i| = k_i$. Write $A'_i = \{a_1, \dots, a_{k_i}\}$ to denote the indexing of the elements of $A'_i$ that satisfies $c_i(a_1) \geq \dots \geq c_i(a_{k_i})$. Define $p_i \coloneqq \left\lceil \frac{2}{11} k_i \right\rceil$ and $U_i \coloneqq \{a_1, \dots, a_{p_i}\}.$ Since $c_i(a_1) \geq \dots \geq c_i(a_{p_i}) \geq \dots \geq c_i(a_{k_i})$, an averaging argument gives $\frac{1}{p_i} c_i(U_i) \geq \frac{1}{k_i} c_i(A'_i)$ and, hence, 
        \begin{equation} c_i(U_i) \geq \frac{p_i}{k_i} c_i(A'_i) \geq \frac{2}{11} c_i(A'_i). \label{eq:unassigned}
        \end{equation}
    
        Now, define the multi-allocation $\calA = (A_i, \dots, A_n)$ by setting $A_i = A'_i \setminus U_i$ for each agent $i$. Then, $$v_i(A_i) 
        = v_i(A'_i) - v_i(U_i) 
        \underset{\text{via (\ref{eq:unassigned})}}{\leq} \frac{9}{11} c_i(A'_i) 
        \underset{\text{via (\ref{eq:chores-approx})}}{\leq} \mu_i.$$ That is, $\calA$ is an MMS multi-allocation. Moreover, the set of unassigned chores in $\calA$ is exactly $U \coloneqq \cup_{i=1}^n U_i$. Hence, 
        \begin{align*} \ellzed{\chi^\calA} 
            = |U| 
            = \sum_{i=1}^n |U_i|
            = \sum_{i=1}^n p_i 
            = \sum_{i=1}^n \left\lceil \frac{2}{11} k_i \right\rceil 
            \leq \sum_{i=1}^n \left( \frac{2}{11} k_i + 1 \right) 
            = \frac{2}{11}m + n. 
        \end{align*}

        The theorem stands proved.
    \end{proof}
 