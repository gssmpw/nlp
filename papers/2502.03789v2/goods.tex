\section{Fair Division of Goods}\label{sec:goods}

\subsection{Monotone Valuations}
This section addresses fair division of goods under monotone valuations. Our positive result for this setting is stated below. 


\begin{theorem}
\label{thm:monotone_goods}
Every fair division instance $\langle [n], [m], \{v_i\}_{i=1}^n \rangle$ with monotone valuations admits an MMS multi-allocation $\calA$ in which no single good is allocated to more than $3 \log  m$ agents and the total number of goods assigned, with copies, is at most $m$. That is, the characteristic vector $\chi^\calA$ of $\calA$ satisfies $\ellinfty{\chi^\calA} \leq 3 \log  m$ and $\ellone{\chi^\calA}\leq m$. 
\end{theorem}
Theorem \ref{thm:monotone_goods} is obtained via a direct application of the probabilitic method. We note below that the desired bounds are satisfied, with positive probability, by a random multi-allocation $\calR = (R_1, \ldots, R_n)$ in which, for each agent $i$, the bundle $R_i$ is chosen uniformly at random from among the subsets in $\calM^i=(M^i_1, \ldots, M^i_n)$. Hence, we obtain that there necessarily exists a multi-allocation that upholds the stated $\ell_1$ and $\ell_\infty$ bounds.  

\paragraph{Random Sampling.} Recall that $\calM^i=(M^i_1,\dots,M^i_n)$  denotes an MMS-inducing partition for agent $i$ (Definition \ref{def:mms}). We select, independently for each $i \in [n]$, a bundle $R_i$ uniformly at random from among $\{M^i_1,\dots,M^i_n\}$, i.e., $\Pr \{ R_i = M^i_j \} = 1/n$, for each index $j \in [n]$. 

Further, considering the (random) multi-allocation $\calR=(R_1,\ldots,R_n)$, we obtain relevant probabilistic guarantees (in Lemma \ref{lem:linftynorm} below) for its characteristic vector, $\chi^\calR \in \mathbb{Z}_+^m$. Specifically, let $G_1$ denote the event that $\ellinfty{\chi^\calR} \leq 3 \log m$ and $G_2$ denote the event that $\ellone{\chi^\calR} \leq m$. The following lemma provides lower bounds on the probabilities of $G_1$ and $G_2$. 

\begin{lemma}
\label{lem:linftynorm} 
For the (random) characteristic vector $\chi^\calR$ we have
\begin{enumerate}
\item[(i)] The expected value $\expval{\chi^\calR_g} = 1$, for each component (good) $g \in [m]$. 
\item[(ii)] $\Pr\{G_1^c\} \leq \frac{1}{m^2}.$
\item[(iii)] $\Pr\{G_2^c\} \leq 1 - \frac{1}{m+1}.$
\end{enumerate}
\end{lemma}
\begin{proof} 
We first prove part (i) of the lemma. Towards this, for each $i \in [n]$ and each $g \in [m]$, let $\mathbb{1}_{\{g \in R_i\}}$ be the indicator random variable for the event: $g \in R_i$. That is, the random variable $\mathbb{1}_{\{g \in R_i\}}$ is equal to one if $g \in R_i$, it is zero otherwise. 
 
Since $\calM^i=(M^i_1,\ldots,M^i_n)$ is an $n$-partition of $[m]$, each good $g \in [m]$ belongs to exactly one subset $M^i_j$. Further, given that the subset $R_i$ is chosen uniformly at random from among $\{M^i_1,\ldots,M^i_n\}$, the following bound holds for each $i \in [n]$ and $g \in [m]$
\begin{align}
  \expval{\mathbb{1}_{\{g \in R_i\}}} = \Pr\{g \in R_i\} = \frac{1}{n}  \label{eq:uar}
\end{align}  

Recall that $\chi^\calR_g$ is equal to the number of copies of any good $g$ assigned among the agents under the multi-allocation $\calR$, i.e., $\chi^\calR_g = \sum_{i=1}^n \mathbb{1}_{\{g \in R_i\}}$. Using this equation we obtain part (i) of the lemma: 
\begin{align}
\expval{\chi^\calR_g} = \sum_{i=1}^n \expval{\mathbb{1}_{\{g \in R_i\}}}  \underset{\text{via (\ref{eq:uar})}}{=} n \  \frac{1}{n} = 1 \label{eqn:part-one-lemma-1} 
\end{align}

For part (ii) of the lemma, note that the random (independent) selection of $R_i$s ensures that, for each fixed $g \in [m]$, the random variables $\mathbb{1}_{\{g \in R_i\}}$, across $i$s, are independent.\footnote{However, for any fixed $i$, we do not have independence between $\mathbb{1}_{\{g \in R_i\}}$ and $\mathbb{1}_{\{g' \in R_i\}}$, for goods $g,g' \in [m]$.} This observation implies that, for each fixed $g \in [m]$, the count $\chi^\calR_g = \sum_{i=1}^n \mathbb{1}_{\{g \in R_i\}}$ is a sum of independent Bernoulli random variables. Hence, via Chernoff bound~\cite[Theorem 4.4]{mitzenmacher2017probability} and for any $t \geq 6 \expval{\chi^\calR_g}$, we have $\Pr \left\{ \chi^\calR_g \geq t \right\} \leq \frac{1}{2^t}$. We instantiate this bound with $t = \max\{ 6, 3 \log m\} \geq 6 \expval{\chi^\calR_g}$; in particular, for $m \geq 4$ we have $t = 3 \log m$. Hence, we obtain $\Pr \left\{ \chi^\calR_g \geq 3 \log m \right\} \leq \frac{1}{m^3}$, for each $g \in [m]$. Let $G^c_{1,g}$ denote the event $\{\chi^\calR_g \geq 3 \log m\}$. Indeed, $\Pr \left\{ G^c_{1,g} \right\} \leq \frac{1}{m^3}$, for each $g \in [m]$, and the event $G^c_1$ (from the lemma statement) satisfies $G^c_1 = \cup_{g \in [m]} \ G^c_{1,g}$. Hence, applying the union bound gives us part (ii):
\begin{align*}
    \Pr\left\{ G^c_1 \right\} =  \Pr\left\{ \cup_{g \in [m]} \ G^c_{1,g} \right\} \leq \sum_{g =1}^m \Pr \left\{ G^c_{1,g} \right\} \leq m \ \frac{1}{m^3} = \frac{1}{m^2}. 
\end{align*}

Finally, for part (iii), observe that $ \expval{\ellone{\chi^\calR}} = \expval{\sum_{g=1}^m \chi^\calR_g} = \sum_{g=1}^m \expval{\chi^\calR_g} = m$; the last equality follows from equation (\ref{eqn:part-one-lemma-1}). Further, for the nonnegative random variable $\ellone{\chi^\calR}$, Markov's inequality gives us $\Pr\left\{ 
 \ellone{\chi^\calR} \geq (m+1) \right\} \leq \frac{\expval{\ellone{\chi^\calR}}}{m+1} = \frac{m}{m+1}$. Since $\ellone{\chi^\calR}$ is an integer-valued random variable, $\Pr\left\{ \ellone{\chi^\calR} > m \right\} = \Pr\left\{ \ellone{\chi^\calR} \geq (m+1) \right\}$. These observations lead to the bound stated in part (iii):
 \begin{align*}
     \Pr\left\{ G_2^c \right\} = \Pr\left\{ \ellone{\chi^\calR} > m \right\} = \Pr\left\{ \ellone{\chi^\calR} \geq (m+1) \right\} \leq \frac{m}{m+1} = 1 - \frac{1}{m+1}.
 \end{align*}
This completes the proof of the lemma. 
\end{proof}

With the above lemma in hand, we now prove \Cref{thm:monotone_goods}.

\begin{proof}[Proof of \Cref{thm:monotone_goods}] As mentioned previously, $G_1$ denotes the event $\ellinfty{\chi^\calR} \leq 3 \log m$ and $G_2$ denotes $\ellone{\chi^\calR} \leq m$. Lemma~\ref{lem:linftynorm} implies that $G_1$ and $G_2$ hold together with positive probability
\begin{align*}
\prob{G_1 \cap G_2} & = 1 - \prob{G_1^c \cup G_2^c} \\ 
& \geq 1 - (\prob{G_1^c} +\prob{G_1^c}) \tag{Union Bound} \\
& \geq 1-\left(\frac{1}{m^2} + 1 - \frac{1}{m+1}\right) \tag{\Cref{lem:linftynorm}} \\
\\ & =\frac{1}{m+1} - \frac{1}{m^2} > 0
\end{align*}

Hence, by the definitions of $G_1$ and $G_2$, we get that the random multi-allocation $\calR=(R_1, \ldots, R_n)$ satisfies the stated $\ell_\infty$ and $\ell_1$ bounds. Moreover, for each $i$, the bundle $R_i$ is sampled from among the subsets that form the MMS-inducing partition $\calM^i$. Hence, $v_i(R_i) \geq \mu_i$, for every $i \in [n]$, ensuring that $\calR$ is always an MMS multi-allocation. Overall, these observations imply that, as claimed in the theorem, there always exists an MMS multi-allocation $\calA$ with the properties that $\ellinfty{\chi^\calA} \leq 3 \log  m$ and $\ellone{\chi^\calA}\leq m$. The theorem stands proved. 
\end{proof}


\subsection{Identically Ordered Valuations}\label{subsec:additive-ordered}
Recall that valuations in a fair division instance are said to be identically ordered if there exists an indexing $\{g_1, \ldots g_m\}$ of the goods such that for each pair of goods $g_s, g_t$, with index $s < t$, and all agents $i \in [n]$, the inequality $v_i(S + {g_s }) \geq  v_i(S + {g_t })$ holds for each subset $S \subset [m]$ that does not contain $g_s$ and $g_t$. Note that additive ordered valuations are identically ordered. The following example highlights that identically ordered valuations are not confined to be subadditive, or even superadditive. 

\begin{example} \label{ex:sqrt-ordered}
 Let $g_1, \dots, g_m$ be a fixed indexing of the set $[m]$ of goods, and, for each agent $i$, let $w_i : [m] \to \mathbb{R}_+$ be weights on the goods that satisfy $w_i(g_1) \geq w_i(g_2) \geq \dots \geq w_i(g_m)$. Note that, for all the agents, the weights respect the common indexing. For each $i \in [n]$, let $f_i: \mathbb{R}_+ \mapsto \mathbb{R}_+$ be a monotone nondecreasing function. Define valuation $v_i(S) \coloneqq f_i \left( \sum_{g\in S} w_i(g) \right)$, for each subset $S \subseteq [m]$ and agent $i$. Since the functions $f_i$s are monotone nondecreasing, the valuations $v_i$s are identically ordered: for all agents $i$, the following inequality holds for each pair of goods $g_s, g_t$, with index $s<t$, and each subset $S$ that does not contain $g_s$ and $g_t$: 
 \begin{align*}
v_i(S + g_s) = f_i \left( \sum_{g \in S} w_i(g)  \ +  \ w_i(g_s) \right) \geq  f_i \left( \sum_{g \in S} w_i(g)  \ +  \ w_i(g_t) \right) = v_i(S + g_t).
\end{align*}
With $f_i$s as the identify function, we obtain additive ordered valuations. Setting $f_i(w) = \sqrt{w}$ gives us subadditive valuations. On the other hand, with $f_i(w) = \exp(w)$, we obtain superadditive valuations. In all of these cases, our result for identically ordered valuations holds. 
\end{example}
 
%\begin{example}
%\label{example:exp}
%Let $g_1, \dots, g_m$ be a fixed indexing of the set $[m]$ of goods, and, for each agent $i$, let $w_i : [m] \to \mathbb{R}_+$ be weights on the goods that satisfies $w_i(g_1) \geq w_i(g_2) \geq \dots \geq w_i(g_m)$. For any subset $S \subseteq [m]$ and each agent $i$, define valuation $v_i(S) \coloneqq  \exp \left(\sum_{g\in S} w_i(g) \right)$. Note that these identically ordered valuations are in fact superadditive.  
%\end{example}


Theorem \ref{thm:additive_goods} (stated below) provides our upper bounds on the supply adjustments for MMS under identically ordered valuations.



\begin{theorem}
\label{thm:additive_goods} Every fair division instance $\langle [n], [m], \{v_i\}_{i=1}^n \rangle$ with identically ordered valuations admits an MMS multi-allocation $\calA = (A_1, \ldots A_n)$ in which no single good is assigned to more than $12 \sqrt{\log m}$ agents and the total number of goods assigned, with copies, is at most $m + \frac{6m  \sqrt{\log m} }{\sqrt{n}}$. That is, the characteristic vector $\chi^\calA$ of $\calA$ satisfies 
\begin{align*}
    \ellinfty{\chi^\calA} \leq 12 \sqrt{\log m}  \qquad { \text{and} } \qquad \ellone{\chi^\calA} \leq m + \frac{6m  \sqrt{\log m} }{\sqrt{n}}. 
\end{align*}
\end{theorem}


\paragraph{Random Sampling.} As in the case of monotone valuations, we sample an MMS multi-allocation, $\calR=(R_1, \ldots, R_n)$, uniformly at random from among the MMS-inducing bundles of each agent, albeit with a key modification. For each agent $i\in [n]$, let $\calM^i=(M^i_1,\dots,M^i_n)$ be an MMS-inducing partition for $i$ in the given instance $\langle [n], [m], \{v_i\}_{i=1}^n \rangle$. In each $\calM^i$, index the subsets such that $|M^i_1| \leq |M^i_2| \leq \dots \leq |M^i_n|$. Let $s_i \in [n]$ be the largest index with the property that $|M^i_{s_i}| \leq \frac{2m}{n}$. Hence, for each index $t \in \{1, 2, \ldots, s_i\}$, we have $|M^i_t| \leq \frac{2m}{n}$. Also, since $M^i_j$s partition $[m]$, at most $n/2$ of these subsets can have cardinality more than $\frac{2m}{n}$, i.e., index $s_i \geq n/2$.  

We select, independently for each $i \in [n]$, a bundle $R_i$ uniformly at random from among $\{M^i_1,\dots,M^i_{s_i} \}$, i.e., $\Pr \{ R_i = M^i_t \} = \frac{1}{s_i}$, for each index $t \in \{1, 2, \ldots, s_i\}$. For the sampled multi-allocation $\calR=(R_1, \ldots, R_n)$, we define the event $G_1 \coloneqq \left\{\ellone{\chi^\calR} \leq m + \frac{6m  \sqrt{\log m} }{\sqrt{n}}\right\}$, which corresponds to our claimed $\ell_1$ bound. We first show that $G_1$ holds with high probability.


\begin{lemma}
\label{lem:l1_additive}
    $\prob{G_1^c} \leq \frac{2}{m^3}$.
\end{lemma}

\begin{proof} 
Note that $\ellone{\chi^\calR}=\sum_{i=1}^n |R_i|$ is a sum of independent random variables, $|R_i|$, each in the range $\left[0,\frac{2m}{n}\right]$. Moreover, for each $i$ we have 
\begin{align}
    \expval{|R_i|} = \frac{1}{s_i} \sum_{j=1}^{s_i} \left|M^i_j\right| \leq \frac{1}{n} \sum_{j=1}^{n} \left|M^i_j\right|=\frac{m}{n} \label{ineq:expRone}
\end{align}
Here, the first inequality follows from the fact that  $M^i_1, M^i_2, \ldots, M^i_{s_i}$ are the $s_i$ subsets of smallest cardinality in $\calM^i=(M^i_1, \ldots, M^i_n)$. Hence, their average cardinality, $\frac{1}{s_i} \sum_{j=1}^{s_i} |M^i_j|$, is at most the overall average $\frac{1}{n} \sum_{j=1}^{n} |M^i_j|$. 

Inequality (\ref{ineq:expRone}) gives us $\expval{\ellone{\chi^\calR}} = \sum_{i =1}^n \expval{|R_i|} \leq \sum_{i=1}^n \frac{m}{n}=m$. 

Further, applying Hoeffding's inequality with $\delta \coloneqq \frac{6m\sqrt{\log m}}{\sqrt{n}}$, we obtain\footnote{Here, the term $\frac{4m^2}{n^2}$ in the denominator of the exponent follows from the range of random variable, $|R_i| \in \left[0,\frac{2m}{n}\right]$.}

\begin{align*} \displaystyle \prob{G_i^c}=\prob{\ellone{\chi^\calR} > m + \delta} & \leq 2 \exp{\left(- \frac{\delta^2}{3 \ n \ \frac{4m^2}{n^2}}\right)} \\ & = 2 \exp{\left(\frac{-3 m^2 \log m}{m^2}\right)} \\ & \leq \frac{2}{m^3}. \end{align*}
The lemma stands proved. 
\end{proof}

Let $g_1, \dots, g_m$ be an indexing of the set $[m]$ of goods that satisfies to property laid out in \Cref{def:identically-ordered}. We will now define a probabilistic event $G_2$ considering the \textit{dyadic prefixes} of $\{g_1, g_2, \ldots, g_m\}$ and the random multi-allocation $\calR$. For each $t\in\{0,1,\ldots,\log m\}$, define the prefix $P_t \coloneqq \{g_1,g_2,\ldots,g_{2^t} \}$ to be the set of the $2^t$ lowest-indexed goods (recall that these have the highest marginal values). In addition, let $\calP$ be the collection of all such prefixes, $\mathcal{P} \coloneqq \{P_0,P_1,\dots,P_{\log m}\}$. 

For each $P\in \mathcal{P}$, we define $\chi^\calR_P$ to be the characteristic vector $\chi^\calR$ restricted to entries corresponding to goods in $P$. In particular, $\chi^\calR_P$ is a $|P|$-dimensional vector, and $\ellone{\chi^\calR_P}$ denotes the total number of goods (with copies) from subset $P$ that are assigned in $\calR$, i.e., $\ellone{\chi^\calR_P} = \sum_{g \in P} \chi^\calR_g$. 

The event $G_2$ bounds these assignment numbers for all the prefixes $P$. Formally, 
\begin{align*}
\text{Event } G_2 \coloneqq \left\{ \ellone{\chi^\calR_P}\leq 6 \sqrt{\log m}\ |P| \ \ \text{ for every } P\in\mathcal{P} \right\}.
\end{align*}

\begin{lemma}
\label{lem:dyadic-prefixes} $\prob{G_2}\geq \frac{1}{2}$.
\end{lemma}

\begin{proof}
    Fix any prefix subset $P\in \mathcal{P}$ and write $r^i_P$ to denote the characteristic vector $\rmchar(R_i)$ restricted to the components/goods in subset $P$. That is, for each $g \in P$, the $g$th component of vector $r^i_P$ is equal to one if $g \in R_i$, it is zero otherwise. 
    Note that $\chi^\calR_P=\sum_{i=1}^n r^i_P$. Therefore, 
        \begin{align}
        \expval{\left\| \chi^\calR_P \right\|_2^2} 
        &= \sum_{i=1}^n \expval{\left\| r^i_P \right\|_2^2} + \sum_{i\neq j} \expval{\left\langle r^i_P,r^j_P\right\rangle} \nonumber \\
        &= \sum_{i=1}^n \expval{\left\|r^i_P\right\|_2^2} + \sum_{i\neq j} \left\langle \expval{r^i_P},\expval{r^j_P} \right\rangle \label{eq:ell2}
    \end{align} 
   The last equality follows from the fact that $r^i_P$ and $r^j_P$ are independent for $i \neq j$. Now, for each $i\in [n]$, since $r_P^i$ is a binary (characteristic) vector, we have 
   \begin{align*}
       \expval{\left\|r^i_P\right\|_2^2}
       =\expval{\left|R_i\cap P\right|}
       =\frac{1}{s_i} \sum_{k=1}^{s_i} \left|M^i_k\cap P\right|
       \leq \frac{|P|}{s_i}
       \leq \frac{2|P|}{n} \tag{since $s_i \geq n/2$}
   \end{align*}
    Also, for each agent $i$, since $\expval{r^i_P}$ is a vector with all entries either $0$ or $\frac{1}{s_i}$. Hence,  
    \begin{align*}
        \left\langle \expval{r^i_P},\expval{r^j_P} \right\rangle \leq \frac{|P|}{s_i\  s_j}\leq\frac{4|P|}{n^2} \tag{since each $s_i \geq n/2$}
    \end{align*}
    Therefore, equation (\ref{eq:ell2}) reduces to 
    \begin{align}
    \expval{\left\|\chi^\calR_P\right\|_2^2}
    \leq \sum_{i=1}^n \frac{2|P|}{n} + \sum_{i\neq j} \frac{4|P|}{n^2} 
    =  2|P|+\binom{n}{2} \frac{4|P|}{n^2} 
    \leq 4|P| \label{ineq:ell2bnd}
    \end{align}
Using (\ref{ineq:ell2bnd}), we next upper bound the expected value and standard deviation of random variable $\ellone{\chi^\calR_P}$. Towards this, first recall that, for any $|P|$-dimensional vector $x \in \mathbb{R}^{|P|}$, the Cauchy-Schwartz inequality gives us $\|x \|_1 \leq \sqrt{|P|} \  \| x \|_2$. In particular, $\ellone{\chi^\calR_P}^2 \leq |P| \left\|\chi^\calR_P\right\|_2^2$ and, hence, 
    \begin{align}
        \expval{\left\|\chi^\calR_P\right\|_1^2} \leq |P| \ \expval{\left\|\chi^\calR_P\right\|_2^2} \underset{\text{via (\ref{ineq:ell2bnd})}}{\leq} 4 |P|^2 \label{ineq:ell1ell2}
    \end{align}
The expected value of $\ellone{\chi^\calR_P}$ satisfies 
\begin{align}
\expval{\ellone{\chi^\calR_P}} &\leq \sqrt{\expval{\ellone{\chi^\calR_P}^2}} \tag{Jensen's inequality} \\
& \leq \sqrt{ 4 |P|^2 } \tag{via (\ref{ineq:ell1ell2})} \\
& = 2 \ |P| \label{ineq:ell1bound}
\end{align}
       Next, for the standard deviation $\sigma\left[\ellone{\chi^\calR_P} \right]$ we have 
    \begin{align}
        \sigma\left[ \ellone{\chi^\calR_P} \right]
    =\sqrt{ \mathrm{Var} \left[ \ellone{\chi^\calR_P} \right]}
    \leq \sqrt{\expval{\ellone{\chi^\calR_P}^2}} \underset{\text{via (\ref{ineq:ell1ell2})}}{\leq} 2\ |P| \label{ineq:stddev}
    \end{align}

    Let $\theta$ denote the expected value of $\ellone{\chi^\calR_P}$ and $\sigma$ denote its standard deviation. Inequalities (\ref{ineq:ell1bound}) and (\ref{ineq:stddev}) ensure that $\theta \leq 2|P|$ and $\sigma \leq 2 |P|$. Further, Chebyshev's inequality gives us $\prob{ \left| \ellone{\chi^\calR_P} - \theta \right| > k \ \sigma  } \leq \frac{1}{k^2}$ for any $k \geq 1$. Instantiating the inequality with $k = 2\sqrt{\log m}$, we obtain  
    \begin{align}
        \prob{ \ellone{\chi^\calR_P} > 6 \sqrt{\log m} \ |P| } & \leq \prob{ \ellone{\chi^\calR_P} > 4  \sqrt{\log m} \ |P| + 2|P| } \nonumber \\  
        & \leq \prob{\ellone{\chi^\calR_P} > 2  \sqrt{\log m} \ \sigma + \theta }  \tag{$\theta, \sigma \leq 2|P|$} \\ 
        & \leq \prob{\ellone{\chi^\calR_P} > k \ \sigma + \theta }  \tag{$k = 2\sqrt{\log m}$} \\ 
        & \leq \prob{ \left| \ellone{\chi^\calR_P} - \theta \right| > k \ \sigma  } \nonumber \\
        & \leq \frac{1}{4 \log m} 
    \end{align}
 Therefore, for each dyadic prefix subset $P \in \calP$, it holds that 
 \begin{align}
     \prob{\ellone{\chi^\calR_P} > 6 \sqrt{\log m} \ |P| } \leq \frac{1}{4\log m} \label{ineq:eachP}
 \end{align} 
 Recall that event $G_2 \coloneqq \left\{ \ellone{\chi^\calR_P}\leq 6 \sqrt{\log m}\ |P| \ \ \text{ for every } P\in\mathcal{P} \right\}$. To upper bound the complement of $G_2$, we apply union bound considering all the $(\log m + 1)$ dyadic prefixes $P\in\mathcal{P}$. Specifically,   
  \begin{align*} \prob{G^c_2} & \leq \sum_{t = 0}^{\log m} \prob{\ellone{\chi^\calR_{P_t}} > 6 \sqrt{\log m}\cdot |P_t|} 
  \tag{Union Bound} \\ 
  & \leq \frac{\log m + 1}{4\log m} \tag{via (\ref{ineq:eachP})} \\
  & < \frac{1}{2}.
  \end{align*}
  Hence, $\prob{G_2} \geq 1/2$, and the lemma stands proved. 
\end{proof}

\begin{corollary} \label{cor:existence-of-b}
  Events $G_1$ and $G_2$---as defined for Lemmas \ref{lem:l1_additive} and \ref{lem:dyadic-prefixes}---hold together with positive probability, $\prob{G_1\cap G_2}>0$. Consequently, every fair division instance with additive ordered valuations admits an MMS multi-allocation $\calB=(B_1,\dots,B_n)$ with the properties that 
  \begin{itemize}
        \item[(i)] $\ellone{\chi^\calB} \leq m + \frac{6m  \sqrt{\log m} }{\sqrt{n}}$
        \item[(ii)] $\ellone{\chi^\calB_P}\leq 6 \sqrt{\log m}\ |P|$, for every dyadic prefix $P\in\mathcal{P}$.
    \end{itemize}
\end{corollary}

\begin{proof} 
Lemmas \ref{lem:l1_additive} and \ref{lem:dyadic-prefixes} give us 
\begin{align*} \displaystyle \mathbb{P}\{G_1 \cap G_2\} & = 1 - \mathbb{P}\{G_1^c \cup G_2^c\} \\ 
& = 1 - \left(\mathbb{P}\{G_1^c\}+ \mathbb{P}\{G_2^c\}\right) & (\text{Union Bound})\\ 
& = 1 - \left(\frac{1}{2} + \frac{2}{m^3}\right) & (\text{\Cref{lem:l1_additive} and \ref{lem:dyadic-prefixes}}) \\ 
& > 0.
\end{align*} 
Since this probability is strictly positive, the random allocation $\calR=(R_1, \ldots, R_n)$ satisfies the desired properties with positive probability. Also, recall that each $R_i$ is drawn considering the MMS-inducing partition $\calM^i$ and, hence, $\calR$ is always an MMS multi-allocation. Overall, we get that there necessarily exists an MMS multi-allocation $\calB$ that satisfies properties (i) and (ii) in the corollary. This completes the proof. 
\end{proof}


Given a multi-allocation $\calB=(B_1,\ldots,B_n)$ as guaranteed by Corollary \ref{cor:existence-of-b}, the following lemma shows that we can construct a multi-allocation $\calA=(A_1,\ldots,A_n)$ that upholds Theorem \ref{thm:additive_goods}.

Recall that, under identically ordered valuations, the goods are indexed $g_1, \ldots, g_m$ in decreasing order of marginal values. That is,  for each pair of goods $g_s$ and $g_t$, with index $s<t$, we have $v_i(S + g_s) \geq v_i(S + g_t)$ for every agent $i$ and each subset $S$ that does not contain $g_s$ and $g_t$. Also, the collection $\mathcal{P}=\{P_0,P_1,\dots,P_{\log m}\}$ contains all the dyadic prefixes, $P_t=\{g_1,g_2,\ldots,g_{2^t}\}$ with $t\in\{0,1,\dots,\log m\}$. We will construct the desired multi-allocation $\calA$ from $\calB$ by systematically replacing copies of a good $\ell$ with lower marginal value, by those of a good $h$ with higher marginal value; see Algorithm \ref{alg:copy-redistribution} for details.
    
\begin{lemma}\label{lem:copy-redistribution}
        In any fair division instance with identically ordered valuations, let $\calB$ be an MMS multi-allocation that satisfies the properties stated in Corollary \ref{cor:existence-of-b}. Then, from $\calB$, we can construct an MMS multi-allocation $\calA$ in the given instance such that $\ellinfty{\chi^\calA} \leq 12 \sqrt{\log m}$ and $\ellone{\chi^\calA}=\ellone{\chi^\calB}$.
\end{lemma}



\begin{algorithm}
\caption{Copy Redistribution for Identically Ordered Valuations}\label{alg:copy-redistribution} 
\begin{algorithmic}[1]
        \REQUIRE An MMS multi-allocation $\calB$ satisfying the properties of Corollary \ref{cor:existence-of-b}\\ 
        \ENSURE A multi-allocation $\calA$
        \STATE Initialize $\calA = \calB$. 
        \WHILE{$\ellinfty{\chi^\calA} > 12 \sqrt{\log m}$}
        \STATE Let $g_t \in [m]$ be the good with the lowest index $t$ that satisfies $\chi^\calA_{g_t} > 12 \sqrt{\log m}$. \label{line:lowgood}
        \STATE Let $g_s \in [m]$ be any good of lower index $s < t$ (equivalently, higher marginal value) with $\chi^\calA_{g_s} < 12 \sqrt{\log m}$. \label{line:highgood}
        \COMMENT{We show that such a good $g_s$ always exists.}
        \STATE Pick an agent $i$ such that good $g_t \in A_i$ and $g_s \notin A_i$. 
        \COMMENT{Since $\chi^\calA_{g_s} < \chi^\calA_{{g_t}}$, such an agent $i$ necessarily exists.}
        \STATE Update $A_i \gets \left( A_i\setminus\{g_t \} \right) \cup\{g_s \}$. \\ \COMMENT{This iteration of the while loop reduces the number of copies of good $g_t$ by commensurately increasing the number of copies of good $g_s$, where $s < t$.}
        \ENDWHILE
    \RETURN $\calA$
\end{algorithmic}
\end{algorithm}

\begin{proof}
We will first show that the while-loop in Algorithm \ref{alg:copy-redistribution} successfully executes for each maintained multi-allocation $\calA$ with $\ellinfty{\chi^\calA} > 12 \sqrt{\log m}$. In particular, we will prove that if $\ellinfty{\chi^\calA} > 12 \sqrt{\log m}$, for a maintained $\calA$, then the good $g_s$ sought in Line \ref{line:highgood} necessarily exists. Also, note that the while-loop must terminate after polynomially many iterations, since the following potential strictly decreases in each iteration: $\sum_{g \in [m]} \ \max\{ 0, \chi^\calA_g - 12 \sqrt{\log m}\}$. Hence, for the returned multi-allocation $\calA$ (obtained after the while-loop terminates), we have $\ellinfty{\chi^\calA} \leq 12 \sqrt{\log m}$. 


Consider any iteration of the while-loop. Let $\calA$ be the current multi-allocation (with $\ellinfty{\chi^\calA} > 12 \sqrt{\log m}$) and $g_t$ be the good selected in Line \ref{line:lowgood}. Assume here, towards a contradiction, that the good $g_s$ desired in Line \ref{line:highgood} does not exist. That is, for each good $k \in L_t$, we have $\chi^\calA_k \geq 12 \sqrt{\log m}$; here, $L_t$ denotes the subset of all the goods with index at most $t$. %\footnote{The choice of $\ell$ as the lowest index good with $\chi^\calA_\ell > 12 \sqrt{\log m}$ ensures that $\chi^\calA_k \leq 12 \sqrt{\log m}$ for all goods $k$ with lower index.} 
By definition, $g_t \in L_t$. Also, note that, the selection criterion in Line \ref{line:lowgood} ensures that all the goods' reassignments that have happened before the current iteration must have been between goods in $L_t$. That is, each pair of goods $g_{t'}$ and $g_{s'}$ considered in any previous iteration satisfy $g_{t'}, g_{s'} \in L_t$. This observation implies that, in the current multi-allocation $\calA$ and the initial multi-allocation $\calB$, the total number of assignments (with copies) of the goods in $L_t$ are equal:  
\begin{align}
    \sum_{k \in L_t} \chi^\calA_k = \sum_{k \in L_t} \chi^\calB_k \label{eq:samesum}
\end{align} 
Since $\chi^\calA_k \geq 12 \sqrt{\log m}$ for each $k \in L_t$, equation (\ref{eq:samesum}) reduces to $\sum_{k \in L_t} \chi^\calB_k \geq  \ 12  \sqrt{\log m} |L_t| $. Select $\ell \in \{0,1,\ldots, \log m\}$ such that $2^\ell < t \leq 2^{\ell+1}$, i.e., $g_t \in P_{\ell +1} \setminus P_\ell$. This containment implies $P_\ell \subset L_t \subseteq P_{\ell+1}$ and, hence, $|L_t| > |P_\ell| = 2^\ell$. Therefore, the above-mentioned bound extends to 
\begin{align}
    \sum_{k \in L_t} \chi^\calB_k \geq   12 \sqrt{\log m} \ |L_t| >  12 \sqrt{\log m} \ \  2^\ell =  6 \sqrt{\log m} \ \  2^{\ell+1} = 6 \sqrt{\log m} |P_{\ell+1}| \label{ineq:prefixpack}
\end{align}
Since $L_t \subseteq P_{\ell+1}$, inequality (\ref{ineq:prefixpack}) implies $\sum_{k \in P_{\ell+1}} \chi^\calB_k >  6 \sqrt{\log m} |P_{\ell+1}| $. This bound, however, contradicts the fact that multi-allocation $\calB$ upholds property (ii) in Corollary \ref{cor:existence-of-b} for $P_{\ell + 1} \in \calP$. 

Hence, by way of contradiction, we get that each iteration of the while-loop executes successfully and, at termination, we obtain a multi-allocation $\calA$ with $\ellinfty{\chi^\calA} \leq 12 \sqrt{\log m}$. That is, the returned multi-allocation satisfies the stated $\ell_\infty$ bound. 

Further, note that, in each iteration of the while loop, the cardinality of the bundle $A_i$ is maintained -- we add a lower-indexed good, $g_s$, to it and remove a higher-indexed one, $g_t$. Since the valuations are identically ordered, in any such replacement, the updated value of agent $i$'s bundle, $v_i((A_i \setminus \{g_t \}) \cup \{g_s\})$, is at least its original value $v_i(A_i) = v_i((A_i \setminus \{g_t\}) \cup \{g_t\})$. Hence, for each agent $i$, the following inequalities continue to hold: $v_i(A_i) \geq v_i(B_i)$ and $|A_i| = |B_i|$. Therefore, for the returned multi-allocation $\calA$, the stated $\ell_1$ bound holds: $\ellone{\chi^\calA} = \sum_{i=1}^n |A_i| = \sum_{i=1}^n |B_i| = \ellone{\chi^\calB}$. Finally, the facts that $\calB$ is an MMS multi-allocation and $v_i(A_i) \geq v_i(B_i)$, for each $i$, imply that $\calA$ is also MMS. 
Overall, we have that the returned allocation $\calA$ is MMS, and it satisfies the stated $\ell_1$ and $\ell_\infty$ bounds. The lemma stands proved.     
\end{proof}


\begin{proof}[Proof of \Cref{thm:additive_goods}]
    Let $\calB=(B_1,\dots, B_n)$ be the MMS multi-allocation whose existence is guaranteed by \Cref{cor:existence-of-b}. Starting with $\calB$ and applying \Cref{lem:copy-redistribution} (\Cref{alg:copy-redistribution}), we can obtain an MMS multi-allocation $\calA=(A_1,\dots,A_n)$ with the properties that $\ellinfty{\chi^\calA} \leq 12 \sqrt{\log m}$ and $\ellone{\chi^\calA}=\ellone{\chi^\calB} \leq m+\frac{6m\sqrt{\log m}}{\sqrt{n}}$. The guaranteed existence of such an MMS multi-allocation establishes the theorem. 
\end{proof}



\paragraph{Remark.} It is relevant to note that prior works on MMS, for additive valuations, assume that one can restrict attention, without loss of generality, to instances with valuations that identically ordered in addition to being additive. This follows from a reduction of Bouveret and Lema\^{i}tre \cite{10.5555/2615731.2617458} via which, and for any instance $\calI$ with additive valuations, one can construct an instance $\calI'$ with additive and identically ordered valuations, such that if $\calI'$ admits an MMS allocation $\calA'$ then, in fact, $\calI$ also admits an MMS allocation $\calA$.\footnote{See \cite{barman2020approximation} for an approximation-preserving version of this result.} The reduction works by deriving from $\calA'$ a {\it picking sequence} over the agents and then showing that greedily assigning goods in $\calI$, via this sequence, leads to an MMS allocation $\calA$.  

Such a reduction, however, is not immediate in the case of multi-allocations. In the current context of allocating copies of goods, the difficulty stems from the fact that while constructing $\calA$ (via the picking sequence mentioned above) one might be forced to assign multiple copies of the same good to an agent $i$. Hence, it remains open whether the guarantee obtained (in Theorem \ref{thm:additive_goods}) for additive ordered valuations extends to all additive valuations. 

At the same time, we note that Theorem \ref{thm:additive_goods} generalizes to additive valuations if one were to relinquish the requirement that each agent $i$'s bundle, $A_i$, must be a subset of $[m]$. That is, in contrast to the rest of the paper, if $A_i$ is allowed to be a multiset, in which each extra copy of any good $g$ fetches an additional value of $v_i(g)$, then the reduction from \cite{10.5555/2615731.2617458} applies and the guarantee given in Theorem \ref{thm:additive_goods} extends to all additive valuations. 


\subsection{Additive Valuations}
This section focuses on fair division of goods under additive valuations. 

\begin{theorem} 
\label{thm:additive_3_3m}
        Every fair division instance $\langle [n], [m], \{v_i\}_{i=1}^n \rangle$  with additive valuations admits an MMS multi-allocation $\calA$ in which no single good is assigned to more than $2$ agents and the total number of goods assigned, with copies, is at most $2m$. That is, the characteristic vector $\chi^\calA$ of $\calA$ satisfies $\ellinfty{\chi^\calA} \leq 2$ and $\ellone{\chi^\calA} \leq 2m$.
\end{theorem}

\begin{proof}
Let $\calI=\langle [n],[m],\{v_i\}_{i=1}^n\rangle$ be the given instance. We construct an auxiliary instance $\widetilde{\calI}$ (detailed next) and consider a constrained fair division problem over it. The instance $\widetilde{\calI}=\langle [n], [m] \times[2],\{\widetilde{v}_i\}_{i=1}^n\rangle$ is constructed as follows: The set of agents is unchanged, whereas each good $g\in [m]$ is replicated twice as $(g,1)$ and $(g,2)$  in $\widetilde{\calI}$. These two goods are referred to as \textit{copies} of $g$ in $\widetilde{\calI}$. For each agent $i \in [n]$, the valuation function $\widetilde{v}_i$ in $\widetilde{\calI}$ is additive and obtained by setting the values of the two copies equal to the value of the underlying good: $\widetilde{v}_i((g,1)) = \widetilde{v}_i((g,2)) = v_i(g)$ for every good $g \in [m]$. Hence, for any subset $S \subseteq [m] \times [2]$, we have $\widetilde{v}_i (S) = \sum_{(g,k) \in S} \ v_i(g) $.    

In instance $\widetilde{\calI}$, we focus on maximin shares under cardinality constraints to impose the requirement that each agent receives at most one copy of each good. Specifically, in $\widetilde{\calI}$,  a subset of goods $S \subseteq [m] \times [2]$ is said to be {\it feasible} if $\big| S \cap \{(g,1), (g,2) \} \big| \leq 1$ for every good $g$. Note that for any feasible subset $S \subseteq [m] \times [2]$, the valuations under $\widetilde{v}_i$ and $v_i$ match, $\widetilde{v}_i(S) = v_i \left( S \odot [m] \right)$, where the projected set $S \odot [m] \coloneqq \left\{ g \in [m] \ \mid \ (g,1) \in S \text{ or } (g,2) \in S \right\}$. 

We consider maximin shares, $\widetilde{\mu}_i$, in instance $\widetilde{\calI}$ under these cardinality constraints
\begin{align}
    \widetilde{\mu_i} \coloneqq \max_{\substack{(X_1, \ldots, X_n) \in \Pi_n([m] \times [2]): \\ \text{each $X_i$ is feasible}} } \ \ \min_{j \in [n]} \  \widetilde{v}_i  (X_j) \label{eqn:mmshat}
\end{align}
Maximin shares under cardinality constraints have been studied in prior works; see, e.g, \cite{hummel2022maximin} and \cite{biswas2018fair}. In particular, the work of Hummel and Hetland \cite{hummel2022maximin} shows that, under additive valuations, one can find in polynomial time an (exact) allocation $\calB = (B_1,\ldots, B_n) \in \Pi_n([m] \times [2])$ such that, for each $i \in [n]$, we have $\widetilde{v}_i(B_i) \geq \frac{1}{2} \widetilde{\mu}_i$ and $B_i$ is feasible. This algorithmic result of \cite{hummel2022maximin} implies that the constructed instance $\widetilde{\calI}$ necessarily admits such an allocation $\calB$. 

From $\calB$ we can derive a multi-allocation $\calA=(A_1, \ldots, A_n)$ in $\calI$ by setting $A_i = B_i \odot [m]$, for each $i \in [n]$. That is, $A_i$ is obtained by including good $g \in [m]$ in it iff a copy of $g$ (i.e., $(g,1)$ or $(g,2)$) is present in $B_i$. Since, in $\widetilde{\calI}$, each good $g \in [m]$ has two copies, the multi-allocation $\calA$ satisfies $\ellinfty{\chi^\calA}  = 2$. Also, since each $B_i$ is feasible, $|A_i|=|B_i|$ and, hence, $\ellone{\chi^\calA} = \sum_{i=1}^n |A_i| = \sum_{i=1}^n |B_i| = 2m$. Therefore, the multi-allocation $\calA$ satisfies the stated $\ell_1$ and $\ell_\infty$ bounds. 

It remains to show that $\calA$ is an MMS multi-allocation in the given instance $\calI$. Towards this, first note that the feasibility of $B_i$ implies $v_i(A_i) = \widetilde{v}_i(B_i) \geq \frac{1}{2} \widetilde{\mu}_i$, for each agent $i \in [n]$. Next, we will show that $\widetilde{\mu}_i \geq 2 \mu_i$, where $\mu_i$ denotes the maximin share of agent $i$ in the instance $\calI$. Together these bounds show that $\calA$ is indeed an MMS multi-allocation: $v_i(A_i) \geq \frac{1}{2} \widetilde{\mu}_i \geq \mu_i$. 

Fix any agent $i \in [n]$. To show $\widetilde{\mu}_i \geq 2 \mu_i$, we start with an MMS-inducing partition $\calM^i=(M^i_1, \ldots, M^i_n)$ for agent $i$ in instance $\calI$. Note that $v_i(M^i_j) \geq \mu_i$, for each $j \in [n]$. From $\calM^i$, construct a partition $\widetilde{\calP}=(\widetilde{P}_1,\ldots,\widetilde{P}_n)$ in $\widetilde{\calI}$ by setting $\widetilde{P}_j = \left( M^i_j \times \{1\} \right) \bigcup \left( M^i_{j+1} \times \{2 \} \right)$, for each $1 \leq j < n$, and  $\widetilde{P}_n =  \left( M^i_n \times \{1\} \right) \bigcup \left( M^i_{1} \times \{2 \} \right)$. That is, in $\widetilde{P}_j$, we include the first copy of each good $g \in M^i_j$ and the second copy of each good $g' \in M^i_{j+1}$. Since $M^i_j \cap M^i_{j+1} = \emptyset$, each $\widetilde{P}_j$ is feasible -- it contains at most one copy of any good $g\in [m]$. Moreover, 
\begin{align*}
    \widetilde{v}_i(\widetilde{P}_j) = \widetilde{v}_i \left(M^i_j \times \{1\}  \right)  +  \widetilde{v}_i \left(M^i_{j+1} \times \{2\}  \right) = v_i(M^i_j) + v_i(M^i_{j+1}) \geq 2 \mu_i.
\end{align*}
Hence, the partition $\widetilde{\calP}=(\widetilde{P}_1,\ldots,\widetilde{P}_n)$ certifies that equation (\ref{eqn:mmshat}) holds with $\widetilde{\mu}_i \geq 2 \mu_i$. As mentioned previously, this bound establishes the existence of an MMS multi-allocation $\calA$ that satisfies the stated $\ell_1$ and $\ell_\infty$ bounds. The theorem stands proved. 
\end{proof}




