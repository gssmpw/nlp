\section{Hardness of Minimizing Assignment Multiplicity}
\label{sec:reduction}
This section establishes the computational hardness of selecting, for the agents $i \in [n]$, bundles $A_i$ from given MMS-inducing partitions $\calM^i = (M^i_1, \dots, M^i_n)$ with the objective of minimizing $\ellinfty{\chi^\calA}$, for the resulting multi-allocation $\calA=(A_1,\ldots, A_n)$. 

\begin{theorem}
\label{theorem:NPHard}
Given partitions $\calM^i = (M^i_1, \dots, M^i_n)$ for the agents $i \in [n]$, it is {\rm NP}-hard to decide whether there exists a multi-allocation $\calA = (A_1, \ldots A_n)$ with the properties that: (i) For each $i \in [n]$, the bundle $A_i = M^i_j$, for some index $j \in [n]$, and (ii) Every good is allocated to at most one agent (i.e., $A_i \cap A_j = \emptyset$ for all $i \neq j$).
\end{theorem}
\begin{proof}
   We present a reduction from the maximum independent set problem. Recall that in this {\rm NP}-hard problem we are given a graph $G = (V, E)$, along with  an integer $k \in \mathbb{Z}$, and the goal is to decide if there exists an independent set $I \subseteq V$ of size at least $k$. We first describe, the construction of the multi-allocation selection problem from any given maximum independent set instance with graph $G= (V, E)$ and threshold $k \in \mathbb{Z}_+$.    
   
We set the number of agents $n = |V|+1$ and the number of goods $m = |V|\binom{k}{2} + |E| \binom{k}{2} + |V|$. Further, for each index $i \in [k]$ and each vertex $u \in V$, we create a tuple $(i, u)$; there are $|V| k$ such tuples. Based on these tuples, we define a set family $\mathcal{F}$ that consists of size-$2$ sets either of the form $\{(i, u), (j, v)\}$, for each $(u, v) \in E$, or $\{(i, u), (j, u)\}$, for each $u \in V$. Formally, 
\begin{align*}
\mathcal{F} = \Big\{ \  \{(i, u), (j, u)\} \mid  i \neq j \text{ and } u \in V \Big\}~\bigcup~\Big\{ \ \{ (i, u), (j, v) \} \mid i \neq j \text{ and } (u, v) \in E \Big\}
\end{align*}
Note that $|\mathcal{F}| = |V|\binom{k}{2} + |E| \binom{k}{2}$.

Recall that the number of agents $n = |V| + 1$. We associate the first $|V|$ agents with the vertices of $G$ and the last agent will be indexed as $n$. We consider $m = |\mathcal{F}|+|V|$ goods. In particular, we include one good for each set $S \in \mathcal{F}$, these goods will be denoted by $\{g_S \mid S \in \mathcal{F}\}$. In addition, we introduce one good for each vertex in $G$ and denote them as $\{h_1, h_2, \ldots h_{|V|}\}$. 

Next, we define the partitions $\calM^i = (M^i_1, M^i_2, \ldots M^i_{n})$ for the first $k$ agents, i.e., for $ 1 \leq i \leq k$: 
\begin{align*}
M^i_u  & = \{g_S \mid S \in \mathcal{F} , (i, u) \in S\} \quad \text{ for $1 \leq u \leq |V|$, and } \\ 
M^i_n &  = M^i_{|V|+1}  = [m] \setminus \left( \cup_{u=1}^{|V|} \ M^i_u \right).
\end{align*}
For the remaining $n-k = |V|+1-k$ agents---i.e., for $k+1 \leq j \leq n$---we set the partitions $\calM^j = \{M^j_1, M^j_2, \ldots M^j_{n}\}$ as follows:

\begin{align*}
M^j_u  & = \{ h_u \} \quad \text{ for $1 \leq u \leq |V|$, and } \\  
M^j_n & = M^j_{|V|+1}  =  \{g_S \mid S \in \mathcal{F}\}.
\end{align*}
This completes the construction. We now prove the equivalence.

\paragraph{Forward Direction.} Suppose the given instance of maximum independent set problem is a {\rm Yes} instance. That is, there is an independent set $I \subseteq V$ of size at least $k$ in the graph $G$. Then, we consider the following multi-allocation $\calA$ in the constructed instance. For each agent $i \in [k]$, set $A_i = M^i_u$ where $u$ is the $i$th vertex in $I$. For the remaining agents $k+1 \leq j \leq n$, we set $A_j = M^j_{j-k}$.
    
We will show that under $\calA$ every good is assigned to at most one agent, i.e., the bundles $A_i$ are pairwise disjoint. First, notice that the last $n-k$ agents get distinct singleton goods from $\{h_1, h_2, \ldots h_{|V|}\}$ and all the first $k$ agents get goods only from $\{g_S \mid S \in \mathcal{F}\}$. Hence, the bundles $A_j$, for $k+1 \leq j \leq n$, are disjoint. Further, consider any pair of agents $i$ and $i'$ such that $1 \leq i \neq i' \leq k$. Suppose $A_i = M^i_u$ and $A_{i'} = M^{i'}_v$. Note that, by construction, $u, v \in I$ and $u \neq v$. Assume, towards a contradiction, that $A_i \cap A_{i'} \neq \emptyset$, i.e., $g_S \in M^i_u \cap M^{i'}_v$ for some good $g_S$. This containment implies $S = \{(i, u), (i', v)\}$. Such a set $S$ was included in $\calF$ only if $(u, v) \in E$. However, this contradicts the fact that $u$ and $v$ belong to the independent set $I$. Therefore, by way of contradiction, we get that $A_i \cap A_{i'} = \emptyset$. That is, the bundles $A_i$ are pairwise disjoint and, hence, as desired, every good is assigned to at most one agent under $\calA$. 

\paragraph{Reverse Direction.} Suppose the constructed instance admits a multi-allocation $\calA$ in which no item is assigned to more than one agent, i.e., the bundles in $\calA$ are disjoint. We will show that in such a case $G$ admits an independent set of size $k$. First, we make the following claim.

\begin{claim}
\label{claim:notlastbundle}
In the multi-allocation $\calA$, none of the first $k$ agents $i \in [k]$ receive the last subset, $M^i_n$, from their partition $\calM^i$, i.e., $A_i \neq M^i_n$ for all $i \in [k]$. 
\end{claim}
\begin{proof}
Towards a contradiction, assume that for some agent $i \in [k]$, the bundle $A_i = M^i_n$. Then, the last $(n-k) > 1$ agents $j \in \{k+1,\ldots, n\}$ can not receive  any of their first $|V|$ subsets in $\calM^j$. This follows from the fact that $M^i_n$ contains all the goods $\{h_1, \ldots, h_{|V|} \}$ and, hence, $M^i_n$ intersects with $M^j_u$ for all $j \in \{k+1,\ldots, n\}$ and $1 \leq u \leq |V|$. This leaves bundles $M^j_n$ for the agents $j \in \{k+1,\ldots, n\}$. However, these bundles intersect each other as well. Hence, by way of contradiction, we get that $A_i \neq M^i_n$ for all $i \in [k]$. 
\end{proof}

By \Cref{claim:notlastbundle}, we have that under the allocation $\calA$, each of the first $k$ agents $i \in [k]$ must receive one of their first $|V|$ subsets from their partitions $\calM^i$. Suppose $A_i = M^i_{u_i}$ for index $u_i \in V$.  
\begin{claim}\label{claim:indset}
The vertices $\{u_1, u_2, \ldots u_k\}$ form an independent set in the graph $G$.
\end{claim}
\begin{proof} First, note that $u_i \neq u_j$, for any pair $i \neq j \in [k]$. Otherwise, if $u_i = u_j = u$, then for the set $S = \{ (i, u), (j, u)\} \in \calF$, we have $g_S \in M^i_u = A_i$ and $g_S \in M^j_u = A_j$. Therefore, $g_S \in A_i \cap A_j$, which contradicts the fact that the bundles in $\calA$ are disjoint.
        
 Now, suppose $(u_i, u_j) \in E$ for some $i \neq j \in [k]$. Then, for the set $S = \{(i, u_i), (j, u_j)\} \in \calF$, consider the good $g_S$. We have $g_S \in M^i_{u_i} \cap M^j_{u_j}$, i.e., $g_S \in A_i \cap A_j$. This containment contradicts the fact that the bundles in $\calA$ are disjoint. Therefore, we have that $(u_i, u_j) \notin E$ for all pairs $i \neq j \in [k]$. Therefore, $\{u_1, u_2, \ldots u_k\}$ is an independent set in the graph $G$.  
\end{proof}
This concludes the reverse direction of the reduction and completes the proof of the theorem.
\end{proof}