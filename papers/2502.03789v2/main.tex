 \documentclass[11pt]{article} %% default font size

%-----------------------------------------------------------------------------%
%	PAGE LAYOUT | MARGIN | FONT
%-----------------------------------------------------------------------------%
\usepackage[margin=1in]{geometry} %% Page margins
%\usepackage{fontawesome5}
\usepackage{libertine}


%-----------------------------------------------------------------------------%
%	VARIOUS PACKAGES
%-----------------------------------------------------------------------------%
\usepackage{comment,xspace} 

\usepackage[dvipsnames]{xcolor} 

\usepackage[ruled,linesnumbered,vlined]{algorithm2e}
%\usepackage{algorithm}
\usepackage{algorithmic}
\usepackage{comment,xspace,enumitem}
\usepackage{booktabs}
\usepackage{wrapfig}


\def\NoNumber#1{{\def\alglinenumber##1{}\State #1}\addtocounter{ALG@line}{-1}}
\renewcommand{\algorithmicrequire}{\textbf{Input:}}
\renewcommand{\algorithmicensure}{\textbf{Output:}}

\usepackage{changepage}
%-----------------------------------------------------------------------------%
%	TABLE | FIGURE
%-----------------------------------------------------------------------------%
\usepackage{booktabs,multirow,subcaption}
\usepackage[nice]{nicefrac}

\usepackage[breakable, theorems, skins]{tcolorbox}


%-----------------------------------------------------------------------------%
%	MATH | ALGORITHM
%-----------------------------------------------------------------------------%
\usepackage{amsmath,amsfonts,amssymb,amsthm,mathtools,thmtools} %% For math equations, theorems, symbols, etc. The package mathtools is at least for the symbol ``:='' and the command is \coloneqq.
\newcommand*{\diff}[1]{\mathop{}\!\mathrm{d}#1} %% https://tex.stackexchange.com/questions/60545/should-i-mathrm-the-d-in-my-integrals
\DeclarePairedDelimiter\ceiling{\lceil}{\rceil} %% \ceiling*{} would be better than \ceiling{}.
\DeclarePairedDelimiter\floor{\lfloor}{\rfloor} %% \floor*{} would be better than \floor{}.
\DeclareMathOperator*{\argmax}{arg\,max}
\DeclareMathOperator*{\argmin}{arg\,min}
%\usepackage[ruled,linesnumbered,vlined]{algorithm2e}
%\SetKwRepeat{Do}{do}{while} %% Use ``Do...While...'' instead of ``Repeat...Until...'': See https://tex.stackexchange.com/questions/212301/do-while-loop-in-algorithm2e
%\newcommand{\mycommentfont}[1]{\bfseries\ttfamily\textcolor{ptgray}{#1}}
%\SetCommentSty{mycommentfont}

%-----------------------------------------------------------------------------%
%	BIBLIOGRAPHY
%-----------------------------------------------------------------------------%
%\usepackage{doi}
\usepackage{dsfont}
\renewcommand{\arraystretch}{1.3}

\usepackage[square,numbers,sort]{natbib} %% Work with \bibliographystyle{plainnat}

%% \citet{} and \citep{} for *textual* and *parenthetical* citations, respectively. See ``Reference Sheet'' at https://mirror.aarnet.edu.au/pub/CTAN/macros/latex/contrib/natbib/natnotes.pdf

%-----------------------------------------------------------------------------%
%	PDF INFO | CROSS-REFERENCE
%-----------------------------------------------------------------------------%
\usepackage{hyperref}
\hypersetup{
linktocpage,
colorlinks=true,
citecolor=cobalt, %% colour of links to bibliography
urlcolor=ptblue, %% colour of links to external links
linkcolor=cobalt, %% colour of links to internal links
}
\usepackage{cleveref} %% ``cleveref'' must be loaded after hyperref.

\usepackage{bbold}


%-----------------------------------------------------------------------------%
%	TikZ
%	TikZ library: e.g., \usetikzlibrary{backgrounds,positioning}
%-----------------------------------------------------------------------------%
\usepackage{tikz} %% Required for drawing custom shapes
\interfootnotelinepenalty=10000
%-----------------------------------------------------------------------------%
%	THEOREM-LIKE ENVIRONMENTS
%-----------------------------------------------------------------------------%
\theoremstyle{plain}
\newtheorem{theorem}{Theorem}[section]
\newtheorem{corollary}[theorem]{Corollary}
\newtheorem{proposition}[theorem]{Proposition}
\newtheorem{lemma}[theorem]{Lemma}
\newtheorem{conjecture}[theorem]{Conjecture}
\newtheorem{claim}[theorem]{Claim}
\newtheorem{result}{Result}
\theoremstyle{definition}
\newtheorem{definition}[theorem]{Definition}
\newtheorem{question}[theorem]{\textcolor{red}{Question}}
\declaretheorem[style=definition,qed=$\bigtriangleup$,sibling=theorem]{example}


\newtheorem*{theorem*}{Theorem}

\newtheorem*{lem}{Lemma}

\newtheorem*{defin}{Definition}

\theoremstyle{remark}
\newtheorem*{remark}{\upshape\bfseries Remark}
\newtheorem*{remarkMagenta}{\upshape\bfseries\textcolor{magenta}{Remark}}
\newtheorem*{answer}{\upshape\bfseries Answer}

%-----------------------------------------------------------------------------%
%	MACROS
%-----------------------------------------------------------------------------%


\newcommand{\changeto}{$\to$\xspace}
\definecolor{cobalt}{rgb}{0.0, 0.28, 0.67}
\newcommand{\mash}[1]{{\color{cobalt}#1}\xspace}
\newcommand{\mashsays}[1]{{\color{cobalt}(\textbf{Mashbat says:} #1)}\xspace}

\newcommand{\sid}[1]{{\color{blue}#1}\xspace}
\newcommand{\sidsays}[1]{{\color{blue}(\textbf{Siddharth says:} #1)}\xspace}

\newcommand{\SR}[1]{{\color{red}(\textbf{SR:} #1)}\xspace}

\newcommand{\AS}[1]{{\color{magenta}{AS: }{#1} \color{red}}}

\newcommand{\alloc}{\mathcal{A}}


\newcommand{\expval}[1]{\mathbb{E}\left[#1\right]}
\newcommand{\prob}[1]{\Pr \left\{#1\right\}}
\newcommand{\ellone}[1]{\left\|#1\right\|_1}
\newcommand{\ellinfty}[1]{\left\|#1\right\|_\infty}
\newcommand{\ellzed}[1]{\left\|#1\right\|_z}
\newcommand{\mmshat}{\widehat{\mathrm{MMS}}}
\newcommand{\muhat}{\widehat{\mu}}


\newcommand{\set}[1]{{\left\{#1\right\}}}
\newcommand{\tuple}[1]{{\left\langle#1\right\rangle}}
\newcommand{\bbN}{\mathbb{N}}
\newcommand{\bbR}{\mathbb{R}}
\newcommand{\calA}{\mathcal{A}}
\newcommand{\calB}{\mathcal{B}}
\newcommand{\calM}{\mathcal{M}}
\newcommand{\calW}{\mathcal{W}}
\newcommand{\calI}{\mathcal{I}}
\newcommand{\calP}{\mathcal{P}}
\newcommand{\calU}{\mathcal{U}}
\newcommand{\calQ}{\mathcal{Q}}
\newcommand{\calT}{\mathcal{T}}
\newcommand{\poly}{\mathrm{poly}}
\newcommand{\tilA}{\widetilde{\mathcal{A}}}
\newcommand{\tildA}{\widetilde{A}}
\newcommand{\calR}{\mathcal{R}}
\newcommand{\calF}{\mathcal{F}}
\newcommand{\calY}{\mathcal{Y}}
\newcommand{\calX}{\mathcal{X}}
\newcommand{\rmchar}{\mathrm{char}}


%-----------------------------------------------------------------------------%
%	TITLE SECTION
%-----------------------------------------------------------------------------%
\title{\bfseries Exact Maximin Share Fairness via Adjusted Supply}

\author{Siddharth Barman\thanks{Indian Institute of Science, Bangalore;  {barman@iisc.ac.in} }  \and Satyanand Rammohan\thanks{Indian Institute of Science, Bangalore;  {satyanand.rammohan@gmail.com} }  \and Aditi Sethia\thanks{Indian Institute of Science, Bangalore;  {aditisethia@iisc.ac.in} }}
%\date{Last updated: \today}


\date{}

\begin{document}
\maketitle

\begin{abstract}  
Test time scaling is currently one of the most active research areas that shows promise after training time scaling has reached its limits.
Deep-thinking (DT) models are a class of recurrent models that can perform easy-to-hard generalization by assigning more compute to harder test samples.
However, due to their inability to determine the complexity of a test sample, DT models have to use a large amount of computation for both easy and hard test samples.
Excessive test time computation is wasteful and can cause the ``overthinking'' problem where more test time computation leads to worse results.
In this paper, we introduce a test time training method for determining the optimal amount of computation needed for each sample during test time.
We also propose Conv-LiGRU, a novel recurrent architecture for efficient and robust visual reasoning. 
Extensive experiments demonstrate that Conv-LiGRU is more stable than DT, effectively mitigates the ``overthinking'' phenomenon, and achieves superior accuracy.
\end{abstract}  
\section{Introduction}


\begin{figure}[t]
\centering
\includegraphics[width=0.6\columnwidth]{figures/evaluation_desiderata_V5.pdf}
\vspace{-0.5cm}
\caption{\systemName is a platform for conducting realistic evaluations of code LLMs, collecting human preferences of coding models with real users, real tasks, and in realistic environments, aimed at addressing the limitations of existing evaluations.
}
\label{fig:motivation}
\end{figure}

\begin{figure*}[t]
\centering
\includegraphics[width=\textwidth]{figures/system_design_v2.png}
\caption{We introduce \systemName, a VSCode extension to collect human preferences of code directly in a developer's IDE. \systemName enables developers to use code completions from various models. The system comprises a) the interface in the user's IDE which presents paired completions to users (left), b) a sampling strategy that picks model pairs to reduce latency (right, top), and c) a prompting scheme that allows diverse LLMs to perform code completions with high fidelity.
Users can select between the top completion (green box) using \texttt{tab} or the bottom completion (blue box) using \texttt{shift+tab}.}
\label{fig:overview}
\end{figure*}

As model capabilities improve, large language models (LLMs) are increasingly integrated into user environments and workflows.
For example, software developers code with AI in integrated developer environments (IDEs)~\citep{peng2023impact}, doctors rely on notes generated through ambient listening~\citep{oberst2024science}, and lawyers consider case evidence identified by electronic discovery systems~\citep{yang2024beyond}.
Increasing deployment of models in productivity tools demands evaluation that more closely reflects real-world circumstances~\citep{hutchinson2022evaluation, saxon2024benchmarks, kapoor2024ai}.
While newer benchmarks and live platforms incorporate human feedback to capture real-world usage, they almost exclusively focus on evaluating LLMs in chat conversations~\citep{zheng2023judging,dubois2023alpacafarm,chiang2024chatbot, kirk2024the}.
Model evaluation must move beyond chat-based interactions and into specialized user environments.



 

In this work, we focus on evaluating LLM-based coding assistants. 
Despite the popularity of these tools---millions of developers use Github Copilot~\citep{Copilot}---existing
evaluations of the coding capabilities of new models exhibit multiple limitations (Figure~\ref{fig:motivation}, bottom).
Traditional ML benchmarks evaluate LLM capabilities by measuring how well a model can complete static, interview-style coding tasks~\citep{chen2021evaluating,austin2021program,jain2024livecodebench, white2024livebench} and lack \emph{real users}. 
User studies recruit real users to evaluate the effectiveness of LLMs as coding assistants, but are often limited to simple programming tasks as opposed to \emph{real tasks}~\citep{vaithilingam2022expectation,ross2023programmer, mozannar2024realhumaneval}.
Recent efforts to collect human feedback such as Chatbot Arena~\citep{chiang2024chatbot} are still removed from a \emph{realistic environment}, resulting in users and data that deviate from typical software development processes.
We introduce \systemName to address these limitations (Figure~\ref{fig:motivation}, top), and we describe our three main contributions below.


\textbf{We deploy \systemName in-the-wild to collect human preferences on code.} 
\systemName is a Visual Studio Code extension, collecting preferences directly in a developer's IDE within their actual workflow (Figure~\ref{fig:overview}).
\systemName provides developers with code completions, akin to the type of support provided by Github Copilot~\citep{Copilot}. 
Over the past 3 months, \systemName has served over~\completions suggestions from 10 state-of-the-art LLMs, 
gathering \sampleCount~votes from \userCount~users.
To collect user preferences,
\systemName presents a novel interface that shows users paired code completions from two different LLMs, which are determined based on a sampling strategy that aims to 
mitigate latency while preserving coverage across model comparisons.
Additionally, we devise a prompting scheme that allows a diverse set of models to perform code completions with high fidelity.
See Section~\ref{sec:system} and Section~\ref{sec:deployment} for details about system design and deployment respectively.



\textbf{We construct a leaderboard of user preferences and find notable differences from existing static benchmarks and human preference leaderboards.}
In general, we observe that smaller models seem to overperform in static benchmarks compared to our leaderboard, while performance among larger models is mixed (Section~\ref{sec:leaderboard_calculation}).
We attribute these differences to the fact that \systemName is exposed to users and tasks that differ drastically from code evaluations in the past. 
Our data spans 103 programming languages and 24 natural languages as well as a variety of real-world applications and code structures, while static benchmarks tend to focus on a specific programming and natural language and task (e.g. coding competition problems).
Additionally, while all of \systemName interactions contain code contexts and the majority involve infilling tasks, a much smaller fraction of Chatbot Arena's coding tasks contain code context, with infilling tasks appearing even more rarely. 
We analyze our data in depth in Section~\ref{subsec:comparison}.



\textbf{We derive new insights into user preferences of code by analyzing \systemName's diverse and distinct data distribution.}
We compare user preferences across different stratifications of input data (e.g., common versus rare languages) and observe which affect observed preferences most (Section~\ref{sec:analysis}).
For example, while user preferences stay relatively consistent across various programming languages, they differ drastically between different task categories (e.g. frontend/backend versus algorithm design).
We also observe variations in user preference due to different features related to code structure 
(e.g., context length and completion patterns).
We open-source \systemName and release a curated subset of code contexts.
Altogether, our results highlight the necessity of model evaluation in realistic and domain-specific settings.





\section{Preliminaries} \label{sec:prelims}
Before diving into the technical results, we state the basic graph notations used throughout the paper and recap the new non-standard definitions we have introduced throughout \Cref{sec:overview}.

\paragraph{Graphs.}
Throughout we consider directed simple graphs $G = (V, E)$, where $E \subseteq V^2$, with $n = |V|$ nodes and $m = |E|$ edges. The edges of the graph can be associated with some value: a length $\ell(e)$ or a capacity/cost $c(e)$, all of which we require to be positive. For any $U \subseteq V$, we write $\overline U = V \setminus U$. Let $G[U]$ be the subgraph induced by $U$. We denote with $\delta^{+}(U)$ the set of edges that have their starting point in $U$ and endpoint in~$\overline U$. We define $\delta^{-}(U)$ symmetrically. We also sometimes write $c(S) = \sum_{e \in S} c(e)$ (for a set of edges $S$) or $c(U, W) = \sum_{e \in E \cap (U \times W)} c(e)$ and $c(U) = c(U, U)$ (for sets of nodes $U, W$).

The distance between two nodes $v$ and $u$ is written $d_G(v,u)$ (throughout we consider only the \emph{length} functions to be relevant for distances). We may omit the subscript if it is clear from the context. The diameter of the graph is the maximum distance between any pair of nodes. For a subgraph $G'$ of $G$ we occasionally say that~$G'$ has \emph{weak diameter} $D$ if for all pairs of nodes $u, v$ in~$G'$, we have $d_G(u, v), d_G(v, u) \leq D$. A strongly connected component in a directed graph $G$ is a subgraph where for every pair of nodes $v,u$ there is a path from $v$ to $u$ and vise versa. Finally, for a radius $r \geq 0$ we write $B^+(v, r) = \set{x \in V : d_G(v, x) \leq r}$ and $B^-(v, r) = \set{y \in V : d_G(y, v) \leq r}$.


\paragraph{Polynomial Bounds.}
For graphs with edge lengths (or capacities), we assume that they are positive and the maximum edge length is bounded by $\poly(n)$. This is only for the sake of simplicity in \cref{sec:ldd-expander,sec:ldd-deterministic} (where in the more general case that all edge lengths are bounded by some threshold $W$ some logarithmic factors in $n$ become $\log (nW)$ instead), and is not necessary for our strongest LDD developed in \cref{sec:ldd-fast}.

\paragraph{Expander Graphs.}
Let $G = (V, E, \ell, c)$ be a directed graph with positive edge capacities $c$ and positive unit edge lengths $\ell$. We define the \emph{volume $\vol(U)$} by
\begin{equation*}
	\vol(U) = c(U, V) = \sum_{e \in E \cap (U \times V)} c(e),
\end{equation*}
and set $\minvol(U) = \min\set{\vol(U), \vol(\overline U)}$ where $\overline U = V \setminus U$. A node set $U$ naturally corresponds to a cut $(U, \overline U)$. The \emph{sparsity} (or \emph{conductance}) of $U$ is defined by
\begin{equation*}
	\phi(U) = \frac{c(U, \overline U)}{\minvol(U)}.
\end{equation*}
In the special cases that $U = \emptyset$ we set $\phi(U) = 1$ and in the special case that $U \neq \emptyset$ but $\vol(U) = 0$, we set $\phi(U) = 0$.
We say that $U$ is \emph{$\phi$-sparse} if $\phi(U) \leq \phi$. We say that a directed graph is a $\phi$-expander if it does not contain a $\phi$-sparse cut $U \subseteq V$. 
We define the \emph{lopsided sparsity} of $U$ as
\begin{equation*}
	\psi(U) = \frac{c(U, \overline U)}{\minvol(U) \cdot \log \frac{\vol(V)}{\minvol(U)}},
\end{equation*}
(with similar special cases), and we similarly say that $U$ is \emph{$\psi$-lopsided sparse} if $\psi(U) \leq \psi$. Finally, we call a graph a \emph{$\psi$-lopsided expander} if it does not contain a $\psi$-lopsided sparse cut $U \subseteq V$.

\section{Fair Division of Goods}\label{sec:goods}

\subsection{Monotone Valuations}
This section addresses fair division of goods under monotone valuations. Our positive result for this setting is stated below. 


\begin{theorem}
\label{thm:monotone_goods}
Every fair division instance $\langle [n], [m], \{v_i\}_{i=1}^n \rangle$ with monotone valuations admits an MMS multi-allocation $\calA$ in which no single good is allocated to more than $3 \log  m$ agents and the total number of goods assigned, with copies, is at most $m$. That is, the characteristic vector $\chi^\calA$ of $\calA$ satisfies $\ellinfty{\chi^\calA} \leq 3 \log  m$ and $\ellone{\chi^\calA}\leq m$. 
\end{theorem}
Theorem \ref{thm:monotone_goods} is obtained via a direct application of the probabilitic method. We note below that the desired bounds are satisfied, with positive probability, by a random multi-allocation $\calR = (R_1, \ldots, R_n)$ in which, for each agent $i$, the bundle $R_i$ is chosen uniformly at random from among the subsets in $\calM^i=(M^i_1, \ldots, M^i_n)$. Hence, we obtain that there necessarily exists a multi-allocation that upholds the stated $\ell_1$ and $\ell_\infty$ bounds.  

\paragraph{Random Sampling.} Recall that $\calM^i=(M^i_1,\dots,M^i_n)$  denotes an MMS-inducing partition for agent $i$ (Definition \ref{def:mms}). We select, independently for each $i \in [n]$, a bundle $R_i$ uniformly at random from among $\{M^i_1,\dots,M^i_n\}$, i.e., $\Pr \{ R_i = M^i_j \} = 1/n$, for each index $j \in [n]$. 

Further, considering the (random) multi-allocation $\calR=(R_1,\ldots,R_n)$, we obtain relevant probabilistic guarantees (in Lemma \ref{lem:linftynorm} below) for its characteristic vector, $\chi^\calR \in \mathbb{Z}_+^m$. Specifically, let $G_1$ denote the event that $\ellinfty{\chi^\calR} \leq 3 \log m$ and $G_2$ denote the event that $\ellone{\chi^\calR} \leq m$. The following lemma provides lower bounds on the probabilities of $G_1$ and $G_2$. 

\begin{lemma}
\label{lem:linftynorm} 
For the (random) characteristic vector $\chi^\calR$ we have
\begin{enumerate}
\item[(i)] The expected value $\expval{\chi^\calR_g} = 1$, for each component (good) $g \in [m]$. 
\item[(ii)] $\Pr\{G_1^c\} \leq \frac{1}{m^2}.$
\item[(iii)] $\Pr\{G_2^c\} \leq 1 - \frac{1}{m+1}.$
\end{enumerate}
\end{lemma}
\begin{proof} 
We first prove part (i) of the lemma. Towards this, for each $i \in [n]$ and each $g \in [m]$, let $\mathbb{1}_{\{g \in R_i\}}$ be the indicator random variable for the event: $g \in R_i$. That is, the random variable $\mathbb{1}_{\{g \in R_i\}}$ is equal to one if $g \in R_i$, it is zero otherwise. 
 
Since $\calM^i=(M^i_1,\ldots,M^i_n)$ is an $n$-partition of $[m]$, each good $g \in [m]$ belongs to exactly one subset $M^i_j$. Further, given that the subset $R_i$ is chosen uniformly at random from among $\{M^i_1,\ldots,M^i_n\}$, the following bound holds for each $i \in [n]$ and $g \in [m]$
\begin{align}
  \expval{\mathbb{1}_{\{g \in R_i\}}} = \Pr\{g \in R_i\} = \frac{1}{n}  \label{eq:uar}
\end{align}  

Recall that $\chi^\calR_g$ is equal to the number of copies of any good $g$ assigned among the agents under the multi-allocation $\calR$, i.e., $\chi^\calR_g = \sum_{i=1}^n \mathbb{1}_{\{g \in R_i\}}$. Using this equation we obtain part (i) of the lemma: 
\begin{align}
\expval{\chi^\calR_g} = \sum_{i=1}^n \expval{\mathbb{1}_{\{g \in R_i\}}}  \underset{\text{via (\ref{eq:uar})}}{=} n \  \frac{1}{n} = 1 \label{eqn:part-one-lemma-1} 
\end{align}

For part (ii) of the lemma, note that the random (independent) selection of $R_i$s ensures that, for each fixed $g \in [m]$, the random variables $\mathbb{1}_{\{g \in R_i\}}$, across $i$s, are independent.\footnote{However, for any fixed $i$, we do not have independence between $\mathbb{1}_{\{g \in R_i\}}$ and $\mathbb{1}_{\{g' \in R_i\}}$, for goods $g,g' \in [m]$.} This observation implies that, for each fixed $g \in [m]$, the count $\chi^\calR_g = \sum_{i=1}^n \mathbb{1}_{\{g \in R_i\}}$ is a sum of independent Bernoulli random variables. Hence, via Chernoff bound~\cite[Theorem 4.4]{mitzenmacher2017probability} and for any $t \geq 6 \expval{\chi^\calR_g}$, we have $\Pr \left\{ \chi^\calR_g \geq t \right\} \leq \frac{1}{2^t}$. We instantiate this bound with $t = \max\{ 6, 3 \log m\} \geq 6 \expval{\chi^\calR_g}$; in particular, for $m \geq 4$ we have $t = 3 \log m$. Hence, we obtain $\Pr \left\{ \chi^\calR_g \geq 3 \log m \right\} \leq \frac{1}{m^3}$, for each $g \in [m]$. Let $G^c_{1,g}$ denote the event $\{\chi^\calR_g \geq 3 \log m\}$. Indeed, $\Pr \left\{ G^c_{1,g} \right\} \leq \frac{1}{m^3}$, for each $g \in [m]$, and the event $G^c_1$ (from the lemma statement) satisfies $G^c_1 = \cup_{g \in [m]} \ G^c_{1,g}$. Hence, applying the union bound gives us part (ii):
\begin{align*}
    \Pr\left\{ G^c_1 \right\} =  \Pr\left\{ \cup_{g \in [m]} \ G^c_{1,g} \right\} \leq \sum_{g =1}^m \Pr \left\{ G^c_{1,g} \right\} \leq m \ \frac{1}{m^3} = \frac{1}{m^2}. 
\end{align*}

Finally, for part (iii), observe that $ \expval{\ellone{\chi^\calR}} = \expval{\sum_{g=1}^m \chi^\calR_g} = \sum_{g=1}^m \expval{\chi^\calR_g} = m$; the last equality follows from equation (\ref{eqn:part-one-lemma-1}). Further, for the nonnegative random variable $\ellone{\chi^\calR}$, Markov's inequality gives us $\Pr\left\{ 
 \ellone{\chi^\calR} \geq (m+1) \right\} \leq \frac{\expval{\ellone{\chi^\calR}}}{m+1} = \frac{m}{m+1}$. Since $\ellone{\chi^\calR}$ is an integer-valued random variable, $\Pr\left\{ \ellone{\chi^\calR} > m \right\} = \Pr\left\{ \ellone{\chi^\calR} \geq (m+1) \right\}$. These observations lead to the bound stated in part (iii):
 \begin{align*}
     \Pr\left\{ G_2^c \right\} = \Pr\left\{ \ellone{\chi^\calR} > m \right\} = \Pr\left\{ \ellone{\chi^\calR} \geq (m+1) \right\} \leq \frac{m}{m+1} = 1 - \frac{1}{m+1}.
 \end{align*}
This completes the proof of the lemma. 
\end{proof}

With the above lemma in hand, we now prove \Cref{thm:monotone_goods}.

\begin{proof}[Proof of \Cref{thm:monotone_goods}] As mentioned previously, $G_1$ denotes the event $\ellinfty{\chi^\calR} \leq 3 \log m$ and $G_2$ denotes $\ellone{\chi^\calR} \leq m$. Lemma~\ref{lem:linftynorm} implies that $G_1$ and $G_2$ hold together with positive probability
\begin{align*}
\prob{G_1 \cap G_2} & = 1 - \prob{G_1^c \cup G_2^c} \\ 
& \geq 1 - (\prob{G_1^c} +\prob{G_1^c}) \tag{Union Bound} \\
& \geq 1-\left(\frac{1}{m^2} + 1 - \frac{1}{m+1}\right) \tag{\Cref{lem:linftynorm}} \\
\\ & =\frac{1}{m+1} - \frac{1}{m^2} > 0
\end{align*}

Hence, by the definitions of $G_1$ and $G_2$, we get that the random multi-allocation $\calR=(R_1, \ldots, R_n)$ satisfies the stated $\ell_\infty$ and $\ell_1$ bounds. Moreover, for each $i$, the bundle $R_i$ is sampled from among the subsets that form the MMS-inducing partition $\calM^i$. Hence, $v_i(R_i) \geq \mu_i$, for every $i \in [n]$, ensuring that $\calR$ is always an MMS multi-allocation. Overall, these observations imply that, as claimed in the theorem, there always exists an MMS multi-allocation $\calA$ with the properties that $\ellinfty{\chi^\calA} \leq 3 \log  m$ and $\ellone{\chi^\calA}\leq m$. The theorem stands proved. 
\end{proof}


\subsection{Identically Ordered Valuations}\label{subsec:additive-ordered}
Recall that valuations in a fair division instance are said to be identically ordered if there exists an indexing $\{g_1, \ldots g_m\}$ of the goods such that for each pair of goods $g_s, g_t$, with index $s < t$, and all agents $i \in [n]$, the inequality $v_i(S + {g_s }) \geq  v_i(S + {g_t })$ holds for each subset $S \subset [m]$ that does not contain $g_s$ and $g_t$. Note that additive ordered valuations are identically ordered. The following example highlights that identically ordered valuations are not confined to be subadditive, or even superadditive. 

\begin{example} \label{ex:sqrt-ordered}
 Let $g_1, \dots, g_m$ be a fixed indexing of the set $[m]$ of goods, and, for each agent $i$, let $w_i : [m] \to \mathbb{R}_+$ be weights on the goods that satisfy $w_i(g_1) \geq w_i(g_2) \geq \dots \geq w_i(g_m)$. Note that, for all the agents, the weights respect the common indexing. For each $i \in [n]$, let $f_i: \mathbb{R}_+ \mapsto \mathbb{R}_+$ be a monotone nondecreasing function. Define valuation $v_i(S) \coloneqq f_i \left( \sum_{g\in S} w_i(g) \right)$, for each subset $S \subseteq [m]$ and agent $i$. Since the functions $f_i$s are monotone nondecreasing, the valuations $v_i$s are identically ordered: for all agents $i$, the following inequality holds for each pair of goods $g_s, g_t$, with index $s<t$, and each subset $S$ that does not contain $g_s$ and $g_t$: 
 \begin{align*}
v_i(S + g_s) = f_i \left( \sum_{g \in S} w_i(g)  \ +  \ w_i(g_s) \right) \geq  f_i \left( \sum_{g \in S} w_i(g)  \ +  \ w_i(g_t) \right) = v_i(S + g_t).
\end{align*}
With $f_i$s as the identify function, we obtain additive ordered valuations. Setting $f_i(w) = \sqrt{w}$ gives us subadditive valuations. On the other hand, with $f_i(w) = \exp(w)$, we obtain superadditive valuations. In all of these cases, our result for identically ordered valuations holds. 
\end{example}
 
%\begin{example}
%\label{example:exp}
%Let $g_1, \dots, g_m$ be a fixed indexing of the set $[m]$ of goods, and, for each agent $i$, let $w_i : [m] \to \mathbb{R}_+$ be weights on the goods that satisfies $w_i(g_1) \geq w_i(g_2) \geq \dots \geq w_i(g_m)$. For any subset $S \subseteq [m]$ and each agent $i$, define valuation $v_i(S) \coloneqq  \exp \left(\sum_{g\in S} w_i(g) \right)$. Note that these identically ordered valuations are in fact superadditive.  
%\end{example}


Theorem \ref{thm:additive_goods} (stated below) provides our upper bounds on the supply adjustments for MMS under identically ordered valuations.



\begin{theorem}
\label{thm:additive_goods} Every fair division instance $\langle [n], [m], \{v_i\}_{i=1}^n \rangle$ with identically ordered valuations admits an MMS multi-allocation $\calA = (A_1, \ldots A_n)$ in which no single good is assigned to more than $12 \sqrt{\log m}$ agents and the total number of goods assigned, with copies, is at most $m + \frac{6m  \sqrt{\log m} }{\sqrt{n}}$. That is, the characteristic vector $\chi^\calA$ of $\calA$ satisfies 
\begin{align*}
    \ellinfty{\chi^\calA} \leq 12 \sqrt{\log m}  \qquad { \text{and} } \qquad \ellone{\chi^\calA} \leq m + \frac{6m  \sqrt{\log m} }{\sqrt{n}}. 
\end{align*}
\end{theorem}


\paragraph{Random Sampling.} As in the case of monotone valuations, we sample an MMS multi-allocation, $\calR=(R_1, \ldots, R_n)$, uniformly at random from among the MMS-inducing bundles of each agent, albeit with a key modification. For each agent $i\in [n]$, let $\calM^i=(M^i_1,\dots,M^i_n)$ be an MMS-inducing partition for $i$ in the given instance $\langle [n], [m], \{v_i\}_{i=1}^n \rangle$. In each $\calM^i$, index the subsets such that $|M^i_1| \leq |M^i_2| \leq \dots \leq |M^i_n|$. Let $s_i \in [n]$ be the largest index with the property that $|M^i_{s_i}| \leq \frac{2m}{n}$. Hence, for each index $t \in \{1, 2, \ldots, s_i\}$, we have $|M^i_t| \leq \frac{2m}{n}$. Also, since $M^i_j$s partition $[m]$, at most $n/2$ of these subsets can have cardinality more than $\frac{2m}{n}$, i.e., index $s_i \geq n/2$.  

We select, independently for each $i \in [n]$, a bundle $R_i$ uniformly at random from among $\{M^i_1,\dots,M^i_{s_i} \}$, i.e., $\Pr \{ R_i = M^i_t \} = \frac{1}{s_i}$, for each index $t \in \{1, 2, \ldots, s_i\}$. For the sampled multi-allocation $\calR=(R_1, \ldots, R_n)$, we define the event $G_1 \coloneqq \left\{\ellone{\chi^\calR} \leq m + \frac{6m  \sqrt{\log m} }{\sqrt{n}}\right\}$, which corresponds to our claimed $\ell_1$ bound. We first show that $G_1$ holds with high probability.


\begin{lemma}
\label{lem:l1_additive}
    $\prob{G_1^c} \leq \frac{2}{m^3}$.
\end{lemma}

\begin{proof} 
Note that $\ellone{\chi^\calR}=\sum_{i=1}^n |R_i|$ is a sum of independent random variables, $|R_i|$, each in the range $\left[0,\frac{2m}{n}\right]$. Moreover, for each $i$ we have 
\begin{align}
    \expval{|R_i|} = \frac{1}{s_i} \sum_{j=1}^{s_i} \left|M^i_j\right| \leq \frac{1}{n} \sum_{j=1}^{n} \left|M^i_j\right|=\frac{m}{n} \label{ineq:expRone}
\end{align}
Here, the first inequality follows from the fact that  $M^i_1, M^i_2, \ldots, M^i_{s_i}$ are the $s_i$ subsets of smallest cardinality in $\calM^i=(M^i_1, \ldots, M^i_n)$. Hence, their average cardinality, $\frac{1}{s_i} \sum_{j=1}^{s_i} |M^i_j|$, is at most the overall average $\frac{1}{n} \sum_{j=1}^{n} |M^i_j|$. 

Inequality (\ref{ineq:expRone}) gives us $\expval{\ellone{\chi^\calR}} = \sum_{i =1}^n \expval{|R_i|} \leq \sum_{i=1}^n \frac{m}{n}=m$. 

Further, applying Hoeffding's inequality with $\delta \coloneqq \frac{6m\sqrt{\log m}}{\sqrt{n}}$, we obtain\footnote{Here, the term $\frac{4m^2}{n^2}$ in the denominator of the exponent follows from the range of random variable, $|R_i| \in \left[0,\frac{2m}{n}\right]$.}

\begin{align*} \displaystyle \prob{G_i^c}=\prob{\ellone{\chi^\calR} > m + \delta} & \leq 2 \exp{\left(- \frac{\delta^2}{3 \ n \ \frac{4m^2}{n^2}}\right)} \\ & = 2 \exp{\left(\frac{-3 m^2 \log m}{m^2}\right)} \\ & \leq \frac{2}{m^3}. \end{align*}
The lemma stands proved. 
\end{proof}

Let $g_1, \dots, g_m$ be an indexing of the set $[m]$ of goods that satisfies to property laid out in \Cref{def:identically-ordered}. We will now define a probabilistic event $G_2$ considering the \textit{dyadic prefixes} of $\{g_1, g_2, \ldots, g_m\}$ and the random multi-allocation $\calR$. For each $t\in\{0,1,\ldots,\log m\}$, define the prefix $P_t \coloneqq \{g_1,g_2,\ldots,g_{2^t} \}$ to be the set of the $2^t$ lowest-indexed goods (recall that these have the highest marginal values). In addition, let $\calP$ be the collection of all such prefixes, $\mathcal{P} \coloneqq \{P_0,P_1,\dots,P_{\log m}\}$. 

For each $P\in \mathcal{P}$, we define $\chi^\calR_P$ to be the characteristic vector $\chi^\calR$ restricted to entries corresponding to goods in $P$. In particular, $\chi^\calR_P$ is a $|P|$-dimensional vector, and $\ellone{\chi^\calR_P}$ denotes the total number of goods (with copies) from subset $P$ that are assigned in $\calR$, i.e., $\ellone{\chi^\calR_P} = \sum_{g \in P} \chi^\calR_g$. 

The event $G_2$ bounds these assignment numbers for all the prefixes $P$. Formally, 
\begin{align*}
\text{Event } G_2 \coloneqq \left\{ \ellone{\chi^\calR_P}\leq 6 \sqrt{\log m}\ |P| \ \ \text{ for every } P\in\mathcal{P} \right\}.
\end{align*}

\begin{lemma}
\label{lem:dyadic-prefixes} $\prob{G_2}\geq \frac{1}{2}$.
\end{lemma}

\begin{proof}
    Fix any prefix subset $P\in \mathcal{P}$ and write $r^i_P$ to denote the characteristic vector $\rmchar(R_i)$ restricted to the components/goods in subset $P$. That is, for each $g \in P$, the $g$th component of vector $r^i_P$ is equal to one if $g \in R_i$, it is zero otherwise. 
    Note that $\chi^\calR_P=\sum_{i=1}^n r^i_P$. Therefore, 
        \begin{align}
        \expval{\left\| \chi^\calR_P \right\|_2^2} 
        &= \sum_{i=1}^n \expval{\left\| r^i_P \right\|_2^2} + \sum_{i\neq j} \expval{\left\langle r^i_P,r^j_P\right\rangle} \nonumber \\
        &= \sum_{i=1}^n \expval{\left\|r^i_P\right\|_2^2} + \sum_{i\neq j} \left\langle \expval{r^i_P},\expval{r^j_P} \right\rangle \label{eq:ell2}
    \end{align} 
   The last equality follows from the fact that $r^i_P$ and $r^j_P$ are independent for $i \neq j$. Now, for each $i\in [n]$, since $r_P^i$ is a binary (characteristic) vector, we have 
   \begin{align*}
       \expval{\left\|r^i_P\right\|_2^2}
       =\expval{\left|R_i\cap P\right|}
       =\frac{1}{s_i} \sum_{k=1}^{s_i} \left|M^i_k\cap P\right|
       \leq \frac{|P|}{s_i}
       \leq \frac{2|P|}{n} \tag{since $s_i \geq n/2$}
   \end{align*}
    Also, for each agent $i$, since $\expval{r^i_P}$ is a vector with all entries either $0$ or $\frac{1}{s_i}$. Hence,  
    \begin{align*}
        \left\langle \expval{r^i_P},\expval{r^j_P} \right\rangle \leq \frac{|P|}{s_i\  s_j}\leq\frac{4|P|}{n^2} \tag{since each $s_i \geq n/2$}
    \end{align*}
    Therefore, equation (\ref{eq:ell2}) reduces to 
    \begin{align}
    \expval{\left\|\chi^\calR_P\right\|_2^2}
    \leq \sum_{i=1}^n \frac{2|P|}{n} + \sum_{i\neq j} \frac{4|P|}{n^2} 
    =  2|P|+\binom{n}{2} \frac{4|P|}{n^2} 
    \leq 4|P| \label{ineq:ell2bnd}
    \end{align}
Using (\ref{ineq:ell2bnd}), we next upper bound the expected value and standard deviation of random variable $\ellone{\chi^\calR_P}$. Towards this, first recall that, for any $|P|$-dimensional vector $x \in \mathbb{R}^{|P|}$, the Cauchy-Schwartz inequality gives us $\|x \|_1 \leq \sqrt{|P|} \  \| x \|_2$. In particular, $\ellone{\chi^\calR_P}^2 \leq |P| \left\|\chi^\calR_P\right\|_2^2$ and, hence, 
    \begin{align}
        \expval{\left\|\chi^\calR_P\right\|_1^2} \leq |P| \ \expval{\left\|\chi^\calR_P\right\|_2^2} \underset{\text{via (\ref{ineq:ell2bnd})}}{\leq} 4 |P|^2 \label{ineq:ell1ell2}
    \end{align}
The expected value of $\ellone{\chi^\calR_P}$ satisfies 
\begin{align}
\expval{\ellone{\chi^\calR_P}} &\leq \sqrt{\expval{\ellone{\chi^\calR_P}^2}} \tag{Jensen's inequality} \\
& \leq \sqrt{ 4 |P|^2 } \tag{via (\ref{ineq:ell1ell2})} \\
& = 2 \ |P| \label{ineq:ell1bound}
\end{align}
       Next, for the standard deviation $\sigma\left[\ellone{\chi^\calR_P} \right]$ we have 
    \begin{align}
        \sigma\left[ \ellone{\chi^\calR_P} \right]
    =\sqrt{ \mathrm{Var} \left[ \ellone{\chi^\calR_P} \right]}
    \leq \sqrt{\expval{\ellone{\chi^\calR_P}^2}} \underset{\text{via (\ref{ineq:ell1ell2})}}{\leq} 2\ |P| \label{ineq:stddev}
    \end{align}

    Let $\theta$ denote the expected value of $\ellone{\chi^\calR_P}$ and $\sigma$ denote its standard deviation. Inequalities (\ref{ineq:ell1bound}) and (\ref{ineq:stddev}) ensure that $\theta \leq 2|P|$ and $\sigma \leq 2 |P|$. Further, Chebyshev's inequality gives us $\prob{ \left| \ellone{\chi^\calR_P} - \theta \right| > k \ \sigma  } \leq \frac{1}{k^2}$ for any $k \geq 1$. Instantiating the inequality with $k = 2\sqrt{\log m}$, we obtain  
    \begin{align}
        \prob{ \ellone{\chi^\calR_P} > 6 \sqrt{\log m} \ |P| } & \leq \prob{ \ellone{\chi^\calR_P} > 4  \sqrt{\log m} \ |P| + 2|P| } \nonumber \\  
        & \leq \prob{\ellone{\chi^\calR_P} > 2  \sqrt{\log m} \ \sigma + \theta }  \tag{$\theta, \sigma \leq 2|P|$} \\ 
        & \leq \prob{\ellone{\chi^\calR_P} > k \ \sigma + \theta }  \tag{$k = 2\sqrt{\log m}$} \\ 
        & \leq \prob{ \left| \ellone{\chi^\calR_P} - \theta \right| > k \ \sigma  } \nonumber \\
        & \leq \frac{1}{4 \log m} 
    \end{align}
 Therefore, for each dyadic prefix subset $P \in \calP$, it holds that 
 \begin{align}
     \prob{\ellone{\chi^\calR_P} > 6 \sqrt{\log m} \ |P| } \leq \frac{1}{4\log m} \label{ineq:eachP}
 \end{align} 
 Recall that event $G_2 \coloneqq \left\{ \ellone{\chi^\calR_P}\leq 6 \sqrt{\log m}\ |P| \ \ \text{ for every } P\in\mathcal{P} \right\}$. To upper bound the complement of $G_2$, we apply union bound considering all the $(\log m + 1)$ dyadic prefixes $P\in\mathcal{P}$. Specifically,   
  \begin{align*} \prob{G^c_2} & \leq \sum_{t = 0}^{\log m} \prob{\ellone{\chi^\calR_{P_t}} > 6 \sqrt{\log m}\cdot |P_t|} 
  \tag{Union Bound} \\ 
  & \leq \frac{\log m + 1}{4\log m} \tag{via (\ref{ineq:eachP})} \\
  & < \frac{1}{2}.
  \end{align*}
  Hence, $\prob{G_2} \geq 1/2$, and the lemma stands proved. 
\end{proof}

\begin{corollary} \label{cor:existence-of-b}
  Events $G_1$ and $G_2$---as defined for Lemmas \ref{lem:l1_additive} and \ref{lem:dyadic-prefixes}---hold together with positive probability, $\prob{G_1\cap G_2}>0$. Consequently, every fair division instance with additive ordered valuations admits an MMS multi-allocation $\calB=(B_1,\dots,B_n)$ with the properties that 
  \begin{itemize}
        \item[(i)] $\ellone{\chi^\calB} \leq m + \frac{6m  \sqrt{\log m} }{\sqrt{n}}$
        \item[(ii)] $\ellone{\chi^\calB_P}\leq 6 \sqrt{\log m}\ |P|$, for every dyadic prefix $P\in\mathcal{P}$.
    \end{itemize}
\end{corollary}

\begin{proof} 
Lemmas \ref{lem:l1_additive} and \ref{lem:dyadic-prefixes} give us 
\begin{align*} \displaystyle \mathbb{P}\{G_1 \cap G_2\} & = 1 - \mathbb{P}\{G_1^c \cup G_2^c\} \\ 
& = 1 - \left(\mathbb{P}\{G_1^c\}+ \mathbb{P}\{G_2^c\}\right) & (\text{Union Bound})\\ 
& = 1 - \left(\frac{1}{2} + \frac{2}{m^3}\right) & (\text{\Cref{lem:l1_additive} and \ref{lem:dyadic-prefixes}}) \\ 
& > 0.
\end{align*} 
Since this probability is strictly positive, the random allocation $\calR=(R_1, \ldots, R_n)$ satisfies the desired properties with positive probability. Also, recall that each $R_i$ is drawn considering the MMS-inducing partition $\calM^i$ and, hence, $\calR$ is always an MMS multi-allocation. Overall, we get that there necessarily exists an MMS multi-allocation $\calB$ that satisfies properties (i) and (ii) in the corollary. This completes the proof. 
\end{proof}


Given a multi-allocation $\calB=(B_1,\ldots,B_n)$ as guaranteed by Corollary \ref{cor:existence-of-b}, the following lemma shows that we can construct a multi-allocation $\calA=(A_1,\ldots,A_n)$ that upholds Theorem \ref{thm:additive_goods}.

Recall that, under identically ordered valuations, the goods are indexed $g_1, \ldots, g_m$ in decreasing order of marginal values. That is,  for each pair of goods $g_s$ and $g_t$, with index $s<t$, we have $v_i(S + g_s) \geq v_i(S + g_t)$ for every agent $i$ and each subset $S$ that does not contain $g_s$ and $g_t$. Also, the collection $\mathcal{P}=\{P_0,P_1,\dots,P_{\log m}\}$ contains all the dyadic prefixes, $P_t=\{g_1,g_2,\ldots,g_{2^t}\}$ with $t\in\{0,1,\dots,\log m\}$. We will construct the desired multi-allocation $\calA$ from $\calB$ by systematically replacing copies of a good $\ell$ with lower marginal value, by those of a good $h$ with higher marginal value; see Algorithm \ref{alg:copy-redistribution} for details.
    
\begin{lemma}\label{lem:copy-redistribution}
        In any fair division instance with identically ordered valuations, let $\calB$ be an MMS multi-allocation that satisfies the properties stated in Corollary \ref{cor:existence-of-b}. Then, from $\calB$, we can construct an MMS multi-allocation $\calA$ in the given instance such that $\ellinfty{\chi^\calA} \leq 12 \sqrt{\log m}$ and $\ellone{\chi^\calA}=\ellone{\chi^\calB}$.
\end{lemma}



\begin{algorithm}
\caption{Copy Redistribution for Identically Ordered Valuations}\label{alg:copy-redistribution} 
\begin{algorithmic}[1]
        \REQUIRE An MMS multi-allocation $\calB$ satisfying the properties of Corollary \ref{cor:existence-of-b}\\ 
        \ENSURE A multi-allocation $\calA$
        \STATE Initialize $\calA = \calB$. 
        \WHILE{$\ellinfty{\chi^\calA} > 12 \sqrt{\log m}$}
        \STATE Let $g_t \in [m]$ be the good with the lowest index $t$ that satisfies $\chi^\calA_{g_t} > 12 \sqrt{\log m}$. \label{line:lowgood}
        \STATE Let $g_s \in [m]$ be any good of lower index $s < t$ (equivalently, higher marginal value) with $\chi^\calA_{g_s} < 12 \sqrt{\log m}$. \label{line:highgood}
        \COMMENT{We show that such a good $g_s$ always exists.}
        \STATE Pick an agent $i$ such that good $g_t \in A_i$ and $g_s \notin A_i$. 
        \COMMENT{Since $\chi^\calA_{g_s} < \chi^\calA_{{g_t}}$, such an agent $i$ necessarily exists.}
        \STATE Update $A_i \gets \left( A_i\setminus\{g_t \} \right) \cup\{g_s \}$. \\ \COMMENT{This iteration of the while loop reduces the number of copies of good $g_t$ by commensurately increasing the number of copies of good $g_s$, where $s < t$.}
        \ENDWHILE
    \RETURN $\calA$
\end{algorithmic}
\end{algorithm}

\begin{proof}
We will first show that the while-loop in Algorithm \ref{alg:copy-redistribution} successfully executes for each maintained multi-allocation $\calA$ with $\ellinfty{\chi^\calA} > 12 \sqrt{\log m}$. In particular, we will prove that if $\ellinfty{\chi^\calA} > 12 \sqrt{\log m}$, for a maintained $\calA$, then the good $g_s$ sought in Line \ref{line:highgood} necessarily exists. Also, note that the while-loop must terminate after polynomially many iterations, since the following potential strictly decreases in each iteration: $\sum_{g \in [m]} \ \max\{ 0, \chi^\calA_g - 12 \sqrt{\log m}\}$. Hence, for the returned multi-allocation $\calA$ (obtained after the while-loop terminates), we have $\ellinfty{\chi^\calA} \leq 12 \sqrt{\log m}$. 


Consider any iteration of the while-loop. Let $\calA$ be the current multi-allocation (with $\ellinfty{\chi^\calA} > 12 \sqrt{\log m}$) and $g_t$ be the good selected in Line \ref{line:lowgood}. Assume here, towards a contradiction, that the good $g_s$ desired in Line \ref{line:highgood} does not exist. That is, for each good $k \in L_t$, we have $\chi^\calA_k \geq 12 \sqrt{\log m}$; here, $L_t$ denotes the subset of all the goods with index at most $t$. %\footnote{The choice of $\ell$ as the lowest index good with $\chi^\calA_\ell > 12 \sqrt{\log m}$ ensures that $\chi^\calA_k \leq 12 \sqrt{\log m}$ for all goods $k$ with lower index.} 
By definition, $g_t \in L_t$. Also, note that, the selection criterion in Line \ref{line:lowgood} ensures that all the goods' reassignments that have happened before the current iteration must have been between goods in $L_t$. That is, each pair of goods $g_{t'}$ and $g_{s'}$ considered in any previous iteration satisfy $g_{t'}, g_{s'} \in L_t$. This observation implies that, in the current multi-allocation $\calA$ and the initial multi-allocation $\calB$, the total number of assignments (with copies) of the goods in $L_t$ are equal:  
\begin{align}
    \sum_{k \in L_t} \chi^\calA_k = \sum_{k \in L_t} \chi^\calB_k \label{eq:samesum}
\end{align} 
Since $\chi^\calA_k \geq 12 \sqrt{\log m}$ for each $k \in L_t$, equation (\ref{eq:samesum}) reduces to $\sum_{k \in L_t} \chi^\calB_k \geq  \ 12  \sqrt{\log m} |L_t| $. Select $\ell \in \{0,1,\ldots, \log m\}$ such that $2^\ell < t \leq 2^{\ell+1}$, i.e., $g_t \in P_{\ell +1} \setminus P_\ell$. This containment implies $P_\ell \subset L_t \subseteq P_{\ell+1}$ and, hence, $|L_t| > |P_\ell| = 2^\ell$. Therefore, the above-mentioned bound extends to 
\begin{align}
    \sum_{k \in L_t} \chi^\calB_k \geq   12 \sqrt{\log m} \ |L_t| >  12 \sqrt{\log m} \ \  2^\ell =  6 \sqrt{\log m} \ \  2^{\ell+1} = 6 \sqrt{\log m} |P_{\ell+1}| \label{ineq:prefixpack}
\end{align}
Since $L_t \subseteq P_{\ell+1}$, inequality (\ref{ineq:prefixpack}) implies $\sum_{k \in P_{\ell+1}} \chi^\calB_k >  6 \sqrt{\log m} |P_{\ell+1}| $. This bound, however, contradicts the fact that multi-allocation $\calB$ upholds property (ii) in Corollary \ref{cor:existence-of-b} for $P_{\ell + 1} \in \calP$. 

Hence, by way of contradiction, we get that each iteration of the while-loop executes successfully and, at termination, we obtain a multi-allocation $\calA$ with $\ellinfty{\chi^\calA} \leq 12 \sqrt{\log m}$. That is, the returned multi-allocation satisfies the stated $\ell_\infty$ bound. 

Further, note that, in each iteration of the while loop, the cardinality of the bundle $A_i$ is maintained -- we add a lower-indexed good, $g_s$, to it and remove a higher-indexed one, $g_t$. Since the valuations are identically ordered, in any such replacement, the updated value of agent $i$'s bundle, $v_i((A_i \setminus \{g_t \}) \cup \{g_s\})$, is at least its original value $v_i(A_i) = v_i((A_i \setminus \{g_t\}) \cup \{g_t\})$. Hence, for each agent $i$, the following inequalities continue to hold: $v_i(A_i) \geq v_i(B_i)$ and $|A_i| = |B_i|$. Therefore, for the returned multi-allocation $\calA$, the stated $\ell_1$ bound holds: $\ellone{\chi^\calA} = \sum_{i=1}^n |A_i| = \sum_{i=1}^n |B_i| = \ellone{\chi^\calB}$. Finally, the facts that $\calB$ is an MMS multi-allocation and $v_i(A_i) \geq v_i(B_i)$, for each $i$, imply that $\calA$ is also MMS. 
Overall, we have that the returned allocation $\calA$ is MMS, and it satisfies the stated $\ell_1$ and $\ell_\infty$ bounds. The lemma stands proved.     
\end{proof}


\begin{proof}[Proof of \Cref{thm:additive_goods}]
    Let $\calB=(B_1,\dots, B_n)$ be the MMS multi-allocation whose existence is guaranteed by \Cref{cor:existence-of-b}. Starting with $\calB$ and applying \Cref{lem:copy-redistribution} (\Cref{alg:copy-redistribution}), we can obtain an MMS multi-allocation $\calA=(A_1,\dots,A_n)$ with the properties that $\ellinfty{\chi^\calA} \leq 12 \sqrt{\log m}$ and $\ellone{\chi^\calA}=\ellone{\chi^\calB} \leq m+\frac{6m\sqrt{\log m}}{\sqrt{n}}$. The guaranteed existence of such an MMS multi-allocation establishes the theorem. 
\end{proof}



\paragraph{Remark.} It is relevant to note that prior works on MMS, for additive valuations, assume that one can restrict attention, without loss of generality, to instances with valuations that identically ordered in addition to being additive. This follows from a reduction of Bouveret and Lema\^{i}tre \cite{10.5555/2615731.2617458} via which, and for any instance $\calI$ with additive valuations, one can construct an instance $\calI'$ with additive and identically ordered valuations, such that if $\calI'$ admits an MMS allocation $\calA'$ then, in fact, $\calI$ also admits an MMS allocation $\calA$.\footnote{See \cite{barman2020approximation} for an approximation-preserving version of this result.} The reduction works by deriving from $\calA'$ a {\it picking sequence} over the agents and then showing that greedily assigning goods in $\calI$, via this sequence, leads to an MMS allocation $\calA$.  

Such a reduction, however, is not immediate in the case of multi-allocations. In the current context of allocating copies of goods, the difficulty stems from the fact that while constructing $\calA$ (via the picking sequence mentioned above) one might be forced to assign multiple copies of the same good to an agent $i$. Hence, it remains open whether the guarantee obtained (in Theorem \ref{thm:additive_goods}) for additive ordered valuations extends to all additive valuations. 

At the same time, we note that Theorem \ref{thm:additive_goods} generalizes to additive valuations if one were to relinquish the requirement that each agent $i$'s bundle, $A_i$, must be a subset of $[m]$. That is, in contrast to the rest of the paper, if $A_i$ is allowed to be a multiset, in which each extra copy of any good $g$ fetches an additional value of $v_i(g)$, then the reduction from \cite{10.5555/2615731.2617458} applies and the guarantee given in Theorem \ref{thm:additive_goods} extends to all additive valuations. 


\subsection{Additive Valuations}
This section focuses on fair division of goods under additive valuations. 

\begin{theorem} 
\label{thm:additive_3_3m}
        Every fair division instance $\langle [n], [m], \{v_i\}_{i=1}^n \rangle$  with additive valuations admits an MMS multi-allocation $\calA$ in which no single good is assigned to more than $2$ agents and the total number of goods assigned, with copies, is at most $2m$. That is, the characteristic vector $\chi^\calA$ of $\calA$ satisfies $\ellinfty{\chi^\calA} \leq 2$ and $\ellone{\chi^\calA} \leq 2m$.
\end{theorem}

\begin{proof}
Let $\calI=\langle [n],[m],\{v_i\}_{i=1}^n\rangle$ be the given instance. We construct an auxiliary instance $\widetilde{\calI}$ (detailed next) and consider a constrained fair division problem over it. The instance $\widetilde{\calI}=\langle [n], [m] \times[2],\{\widetilde{v}_i\}_{i=1}^n\rangle$ is constructed as follows: The set of agents is unchanged, whereas each good $g\in [m]$ is replicated twice as $(g,1)$ and $(g,2)$  in $\widetilde{\calI}$. These two goods are referred to as \textit{copies} of $g$ in $\widetilde{\calI}$. For each agent $i \in [n]$, the valuation function $\widetilde{v}_i$ in $\widetilde{\calI}$ is additive and obtained by setting the values of the two copies equal to the value of the underlying good: $\widetilde{v}_i((g,1)) = \widetilde{v}_i((g,2)) = v_i(g)$ for every good $g \in [m]$. Hence, for any subset $S \subseteq [m] \times [2]$, we have $\widetilde{v}_i (S) = \sum_{(g,k) \in S} \ v_i(g) $.    

In instance $\widetilde{\calI}$, we focus on maximin shares under cardinality constraints to impose the requirement that each agent receives at most one copy of each good. Specifically, in $\widetilde{\calI}$,  a subset of goods $S \subseteq [m] \times [2]$ is said to be {\it feasible} if $\big| S \cap \{(g,1), (g,2) \} \big| \leq 1$ for every good $g$. Note that for any feasible subset $S \subseteq [m] \times [2]$, the valuations under $\widetilde{v}_i$ and $v_i$ match, $\widetilde{v}_i(S) = v_i \left( S \odot [m] \right)$, where the projected set $S \odot [m] \coloneqq \left\{ g \in [m] \ \mid \ (g,1) \in S \text{ or } (g,2) \in S \right\}$. 

We consider maximin shares, $\widetilde{\mu}_i$, in instance $\widetilde{\calI}$ under these cardinality constraints
\begin{align}
    \widetilde{\mu_i} \coloneqq \max_{\substack{(X_1, \ldots, X_n) \in \Pi_n([m] \times [2]): \\ \text{each $X_i$ is feasible}} } \ \ \min_{j \in [n]} \  \widetilde{v}_i  (X_j) \label{eqn:mmshat}
\end{align}
Maximin shares under cardinality constraints have been studied in prior works; see, e.g, \cite{hummel2022maximin} and \cite{biswas2018fair}. In particular, the work of Hummel and Hetland \cite{hummel2022maximin} shows that, under additive valuations, one can find in polynomial time an (exact) allocation $\calB = (B_1,\ldots, B_n) \in \Pi_n([m] \times [2])$ such that, for each $i \in [n]$, we have $\widetilde{v}_i(B_i) \geq \frac{1}{2} \widetilde{\mu}_i$ and $B_i$ is feasible. This algorithmic result of \cite{hummel2022maximin} implies that the constructed instance $\widetilde{\calI}$ necessarily admits such an allocation $\calB$. 

From $\calB$ we can derive a multi-allocation $\calA=(A_1, \ldots, A_n)$ in $\calI$ by setting $A_i = B_i \odot [m]$, for each $i \in [n]$. That is, $A_i$ is obtained by including good $g \in [m]$ in it iff a copy of $g$ (i.e., $(g,1)$ or $(g,2)$) is present in $B_i$. Since, in $\widetilde{\calI}$, each good $g \in [m]$ has two copies, the multi-allocation $\calA$ satisfies $\ellinfty{\chi^\calA}  = 2$. Also, since each $B_i$ is feasible, $|A_i|=|B_i|$ and, hence, $\ellone{\chi^\calA} = \sum_{i=1}^n |A_i| = \sum_{i=1}^n |B_i| = 2m$. Therefore, the multi-allocation $\calA$ satisfies the stated $\ell_1$ and $\ell_\infty$ bounds. 

It remains to show that $\calA$ is an MMS multi-allocation in the given instance $\calI$. Towards this, first note that the feasibility of $B_i$ implies $v_i(A_i) = \widetilde{v}_i(B_i) \geq \frac{1}{2} \widetilde{\mu}_i$, for each agent $i \in [n]$. Next, we will show that $\widetilde{\mu}_i \geq 2 \mu_i$, where $\mu_i$ denotes the maximin share of agent $i$ in the instance $\calI$. Together these bounds show that $\calA$ is indeed an MMS multi-allocation: $v_i(A_i) \geq \frac{1}{2} \widetilde{\mu}_i \geq \mu_i$. 

Fix any agent $i \in [n]$. To show $\widetilde{\mu}_i \geq 2 \mu_i$, we start with an MMS-inducing partition $\calM^i=(M^i_1, \ldots, M^i_n)$ for agent $i$ in instance $\calI$. Note that $v_i(M^i_j) \geq \mu_i$, for each $j \in [n]$. From $\calM^i$, construct a partition $\widetilde{\calP}=(\widetilde{P}_1,\ldots,\widetilde{P}_n)$ in $\widetilde{\calI}$ by setting $\widetilde{P}_j = \left( M^i_j \times \{1\} \right) \bigcup \left( M^i_{j+1} \times \{2 \} \right)$, for each $1 \leq j < n$, and  $\widetilde{P}_n =  \left( M^i_n \times \{1\} \right) \bigcup \left( M^i_{1} \times \{2 \} \right)$. That is, in $\widetilde{P}_j$, we include the first copy of each good $g \in M^i_j$ and the second copy of each good $g' \in M^i_{j+1}$. Since $M^i_j \cap M^i_{j+1} = \emptyset$, each $\widetilde{P}_j$ is feasible -- it contains at most one copy of any good $g\in [m]$. Moreover, 
\begin{align*}
    \widetilde{v}_i(\widetilde{P}_j) = \widetilde{v}_i \left(M^i_j \times \{1\}  \right)  +  \widetilde{v}_i \left(M^i_{j+1} \times \{2\}  \right) = v_i(M^i_j) + v_i(M^i_{j+1}) \geq 2 \mu_i.
\end{align*}
Hence, the partition $\widetilde{\calP}=(\widetilde{P}_1,\ldots,\widetilde{P}_n)$ certifies that equation (\ref{eqn:mmshat}) holds with $\widetilde{\mu}_i \geq 2 \mu_i$. As mentioned previously, this bound establishes the existence of an MMS multi-allocation $\calA$ that satisfies the stated $\ell_1$ and $\ell_\infty$ bounds. The theorem stands proved. 
\end{proof}





%\input{beyond_additive}

\subsection{Lower Bound for Goods}

We now show that there exist fair division instances $\calI$, with monotone valuations, such that each MMS multi-allocation in $\mathcal{I}$ ends up assigning at least one good to more than $\frac{\log m}{\log \log m}$ many agents. This lower bound shows that the positive result obtained in Theorem \ref{thm:monotone_goods}---for the $\ell_\infty$ bound---is essentially tight. 

\begin{theorem}
\label{thm:goods_lowerbound}
    There exists a fair division instance $\calI = \langle [n], [m], \{v_i\}_{i=1}^n \rangle$ with monotone valuations such that for every MMS multi-allocation $\calA$ in $\calI$, it holds that $\ellinfty{\chi^\calA} \geq \frac{\log m}{\log \log m}$, i.e., $\calA$ assigns some good to at least $\frac{\log m}{\log\log m}$ agents.
\end{theorem}

\begin{proof} We first describe the construction of the instance $\langle [n], [m], \{v_i\}_{i=1}^n \rangle$. For each agent $i \in [n]$ independently, we draw a partition $\mathcal{P}^i =(P^i_1, P^i_2, \ldots P^i_n)$ uniformly at random from the set of all possible $n^m$ possible partitions, i.e., $\calP^i \in_R \Pi_n([m])$. Given such a partition, we set the valuation $v_i$, for every subset $S \subseteq [m]$, as follows 
\begin{align*}
    v_i(S) \coloneqq \begin{cases}
    1 & \text{if }  P_j^i \subseteq S \text{ for any }j\in[n], \\
    0 & \text{otherwise}.
\end{cases}
\end{align*}
Note that $v_i$s are monotone set functions. In addition, the maximin share of every agent $i$ is equal to one, $\mu_i=1$. Hence, for any MMS multi-allocation $\calA=(A_1,\ldots,A_n)$ it must hold that $P^i_j \subseteq A_i$, for each agent $i \in [n]$ and some index $j \in [n]$. We can, in fact, restrict attention to those multi-allocations that satisfy $A_i=P_i^j$ for some $j \in [n]$, since the ones wherein $P^i_j \subsetneq A_i$ induce larger $\ell_\infty$ norms. Hence, given that partitions $\calP^i$ for the agents $i \in [n]$, we define the family $\calF$ of multi-allocations as 
\begin{align*}
    \calF \coloneqq \left\{ \calA=(A_1,\ldots,A_n) \mid \text{ for each } i\in[n] \text{ there exists } j\in[n] \text{ such that } A_i=P^i_j\right\}.
\end{align*}
Note that $|\mathcal{F}|=n^n$. Fix one multi-allocation $\calQ \in \calF$ obtained by setting index $j=1$ for all the agents $i$, i.e., $\calQ \coloneqq (P^1_1, P^2_1,\ldots, P^n_1)$. Write $\chi$ to denote the characteristic vector of $\calQ$. Now, consider the independent random draws that result in the  partitions $\calP^i=(P^i_1, \ldots P^i_n)$. Note that, for a fixed agent $i$ and a fixed good $g$, the probability that $g$ is contained in $i$'s bundle in $\calQ$ satisfies $\prob{g \in P^i_1} = 1/n$. Further, the independence of the draws (of $\calP^i$s) across the agents imply that the number of copies of $g$ assigned in $\calQ$ (i.e., $\chi_g$) is distributed as $\chi_g \sim \mathrm{Bin}(n, \frac{1}{n})$. For large $n$, this is approximated by the Poisson distribution $\mathrm{Poi}(n \  \frac{1}{n}) = \mathrm{Poi}(1)$. 
 
Define $\ell \coloneqq \frac{\log m}{\log \log m}$. Since each $\calP^i$ is drawn independently among all the $n$-partitions of $[m]$, the random variables $\chi_g$ across the goods $g \in [m]$ are independent. Therefore, using the fact that $\chi_g \sim \mathrm{Poi}(1)$ are independent and identically distributed for all $g \in [m]$, we obtain 
\begin{align}
    \prob{\ellinfty{\chi} \leq  \ell} = \left(\prob{\chi_g \leq \ell}\right)^m = \left(\sum_{k = 0}^{\ell} \frac{e^{-1}}{k!}\right)^m = \left(\frac{1}{e}\sum_{k = 0}^{\ell}\frac{1}{k!}\right)^m \label{eq:maxPoi}
\end{align}
Recall that $e = \sum_{k=0}^\ell \frac{1}{k!} + \sum_{k=\ell+1}^\infty \frac{1}{k!}$. 
 Hence, equation (\ref{eq:maxPoi}) simplifies to 
\begin{align*}
\prob{\ellinfty{\chi} \leq  \ell} = \left(\frac{1}{e} \left(e- \sum_{k=\ell+1}^\infty \frac{1}{k!} \right)\right)^m < \left( \frac{1}{e} \left(e - \frac{1}{(\ell+1)!} \right) \right)^m = \left(1 - \frac{1}{e(\ell+1)!}\right)^m.    \end{align*}
 
% Applying Sterling's approximation, we have $$e(\ell+1)! \approx e \left( \sqrt{2 \pi (\ell+1)} \left(\frac{\ell+1}{e}\right)^{\ell+1}\right)$$ 
We can extend the last bound as follows
\begin{align}
 \prob{\ellinfty{\chi} \leq  \ell} < \left(1 - \frac{1}{e(\ell+1)!} \right)^{e(\ell+1)!\left(\frac{m}{e(\ell+1)!}\right)}  \leq \left(\frac{1}{e}\right)^{\frac{m}{e(\ell+1)!}} \label{ineq:endPoi}    
\end{align}
 
Since $\ell = \frac{\log m}{\log \log m}$, for a sufficiently large $m \in \mathbb{Z}_+$, we have $m \geq 2 \  n \log n \  e (\ell +1)! $. For such an $m$, inequality (\ref{ineq:endPoi}) reduces to 
\begin{align}
    \prob{\ellinfty{\chi} \leq  \ell} <  \left(\frac{1}{e}\right)^{2n \log n} = \frac{1}{n^{2n}} \label{ineq:finalPoi}.
\end{align}
Inequality (\ref{ineq:finalPoi}) holds for each fixed MMS multi-allocation $\calQ \in \calF$. Hence, a union bound over all the $n^n$ multi-allocations in $\calF$ establishes the theorem. In particular, define the event $E_\calQ: = \left\{ \ellinfty{\chi^\calQ} > \ell \right\}$ for each $\calQ \in\mathcal{F}$. We have 
\begin{align*}
\prob{\bigcap_{\calQ \in \mathcal{F}} E_\calQ} = 1- \prob{\bigcup_{\calQ \in\mathcal{F}} E_Q^c} \geq 1 - n^n \cdot \prob{E_Q^c} \underset{\text{ via (\ref{ineq:finalPoi}})}{>} 1-\frac{1}{n^n} > 0    
\end{align*}


Therefore, with a non-zero probability, the events $E_\calQ$ hold together for all possible MMS multi-allocations $\calQ$. This implies that there exists an instance with a sufficiently large $m$ and a certain choice of the partitions $\calP_i=\{P^i_1, \ldots, P^i_n)$ (along with valuations defined as above) such that for all MMS multi-allocations, there exists a good $g$ with at least $\ell = \frac{\log m}{\log \log m}$ assigned copies. This completes the proof of the lower bound. 
\end{proof}

\begin{remark}
We note that, for exact MMS, the $\ell_\infty$ lower bound given in \Cref{thm:goods_lowerbound} holds even under XOS valuations. In particular, as in the proof of \Cref{thm:goods_lowerbound}, we select independently for each agent $i \in [n]$, a partition $\calP^i = (P^i_1, P^i_2, \dots, P^i_n)$ uniformly at random from the set $\Pi_n([m])$ of all possible $n$-partitions of $[m]$. Then, for each agent $i \in [n]$ and subset $S \subseteq [m]$, we set the valuation
\begin{align*}
v_i(S) \coloneqq  \max_{1\leq j\leq n} \ \frac{1}{|P^i_j|} |S \cap P^i_j|. 
\end{align*}
Note that the valuations $v_i$ are point-wise maximizers of additive functions of the form $S \mapsto \frac{1}{|P^i_j|} |S \cap P^i_j|$. Hence, $v_i$s are XOS. Further, $v_i(S)=1$ if $P^i_j \subseteq S$, for some index $j \in [n]$, and $v_i(S) < 1$ otherwise. Therefore, the partition $\calP^i = (P^i_1, \dots, P^i_n)$ certifies that, for each agent $i \in [n]$, the maximin share $\mu_i = 1$. Moreover, for any MMS multi-allocation $\calA = (A_i, \dots, A_n)$ it must hold that $P^i_j \subseteq A_i$ for each agent $i \in [n]$ and some index $j \in [n]$. This is exactly the property we utilize for the proof of \Cref{thm:goods_lowerbound}, and the rest of the proof relies solely on this condition. Therefore, the lower bound holds in particular for XOS valuations. 
\end{remark}
\section{Fair Division of Chores}
\label{sec:chores}

This section addresses fair division instances wherein the items to be assigned are undesirable to---i.e., have costs for---the agents. Such an instance is specified by a triple $\langle[n],[m],\{c_i\}_{i=1}^n\rangle$ with $n$ agents and $m$ items, called \textit{chores}. Here, the function $c_i:2^{[m]}\to\mathbb{R}_{\geq 0}$ specifies the (nonnegative) \textit{cost} of each subset of chores to agent $i$. A cost function $c_i$ is said to be monotone if the inclusion of chores into any subset does not decrease its cost: $c_i(S) \leq c_i(T)$ for every pair of subsets $S \subseteq T \subseteq[m]$. Further, $c_i$ is said to be additive if for every subset $S \subseteq [m]$ of chores, $c_i(S) = \sum_{a \in S} c_i(\{a\})$. As a shorthand, we will write $c_i(a)$ to denote agent $i$'s cost for chore $a \in [m]$. Throughout, the paper assumes that the agents' costs are monotone and normalized: $c_i(\emptyset)=0$ for all agents $i$. 

We will additionally consider (in \Cref{subsec:ordered-costs}) instances with cost functions that are identically ordered. 

The maximin share of each agent $i \in [n]$ is the minimum cost incurred by $i$ if the agent were to partition the $m$ chores into $n$ subsets and is then assigned the one with the highest cost. Formally,
\begin{definition}[MMS for chores] 
Given any fair division instance $\langle [n], [m], \{c_i\}_{i=1}^n \rangle$ with chores, the {maximin share}, $\mu_i \in \mathbb{R}_+$, of each agent $i \in [n]$ is defined as 
\begin{align*}
\mu_i \coloneqq  \min_{(X_1,\dots, X_n) \in \Pi_n([m])} \ \ \max_{j\in[n]} c_i(X_{j}).
\end{align*}
Further, for each agent $i$, let $\calM^i=(M^i_1, M^i_2, \ldots, M^i_n) \in \Pi_n([m])$ denote an {MMS-inducing partition}:
\begin{align*}
\calM^i \in \argmin_{(X_1,\dots, X_n) \in \Pi_n([m])} \ \ \max_{j\in[n]} c_i(X_{j})
\end{align*}
\end{definition} 

As before, an allocation $\calB=(B_1,\ldots, B_n)$ is a partition of $[m]$ into $n$ pairwise disjoint subsets $B_1,\ldots, B_n \subseteq [m]$. Here, the subset of chores $B_i$ is assigned to agent $i$ and is referred to as $i$'s bundle. Also, a \textit{multi-allocation} is a tuple $\calA=(A_1,\ldots,A_n)$ of $n$ subsets of chores, wherein subset $A_i \subseteq [m]$ denotes the bundle assigned to agent $i$. In contrast to allocations, in a multi-allocation, we do not require that the assigned bundles $A_i$ are pairwise disjoint and that they partition $[m]$. Hence, in a multi-allocation, a single chore may be present in multiple bundles or even in none. 


\paragraph{Fair Multi-Allocations.} A multi-allocation $\calA=(A_1,\dots,A_n)$ is said to be an \emph{MMS multi-allocation} if each agent receives a bundle of cost at most its maximin share, $c_i(A_i)\leq \mu_i$ for all agents $i \in [n]$.


Agents with monotone costs prefer to have fewer chores in their bundle and, hence, additional copies of any chore can be disposed of without impacting the MMS guarantee. Specifically, given any multi-allocation $\calA$ in which one chore $a$ is assigned to agents $i \neq i'$, one can consider multi-allocation $\calA'$ in which $a$ is assigned to only one of them, say $i$. If $\calA$ is MMS, then so is $\calA'$. That is, one can assume, without loss of generality, that the considered MMS multi-allocations $\calA$ satisfy $\ellinfty{\chi^\calA} \leq 1$, i.e., $\chi^\calA_a$ is always either $0$ or $1$ for each chore $a \in [m]$.\footnote{This makes such a multi-allocation in fact a \textit{partial allocation}.} Motivated by these considerations,  we will focus on quantifying the number of distinct chores that can be assigned among the agents while ensuring fairness. That is, in contrast to the goods setting where we obtained upper bounds for $\ell_1$ and $\ell_\infty$, here, we will focus on establishing upper bounds on the number of zero entries in the vector $\chi^\calA$. We will, throughout, write $\ellzed{\chi^\calA}$ to denote this quantity, $\ellzed{\chi^\calA} \coloneqq \{ a \in [m] \mid \chi^\calA_a = 0 \}$.\footnote{Note that this is not a norm.} Observe that $\ellzed{\chi^\calA}$ captures the number of chores that are unallocated under $\calA$; ideally, this quantity should be as small as possible. 


\subsection{Monotone Costs}
For instances with monotone costs, the following theorem shows that we can achieve MMS fairness while at most $m/e$ chores remain unassigned.

\begin{theorem}
\label{thm:monotone_chores}
    Every fair division instance $\langle [n], [m], \{c_i\}_{i=1}^n \rangle$ with chores and monotone costs admits an MMS multi-allocation $\calA=(A_1,\dots,A_n)$ in which at most $\frac{m}{e}$ chores remain unassigned. That is, the characteristic vector $\chi^\calA$ of $\calA$ satisfies $\ellzed{\chi^\calA}\leq\frac{m}{e}$. 
\end{theorem}

\begin{proof}
    Recall that $\calM^i=(M^i_1,\ldots,M^i_n)$ is an MMS-inducing partition for each agent $i \in [n]$. We select, independently for each $i$, a bundle $R_i$ uniformly at random from among $\{M^i_1, M^i_2, \dots,M^i_n\}$, and consider the random multi-allocation $\calR=(R_1,R_2,\dots,R_n)$. For any fixed chore $a\in[m]$, the events $\{a\notin R_i\}$ are independent across $i$ with $\prob{a\notin R_i}=1-\frac{1}{n}$. Therefore, under $\calR$, the probability that $a$ remains unallocated  $\prob{a\notin\cup_{i=1}^n R_i}=\left(1-\frac{1}{n}\right)^n \simeq \frac{1}{e}$. Hence, the total number of unallocated chores, in expectation, equals 
    \begin{align*}
    \expval{\ellzed{\chi^\calR}}
    &= \sum_{a=1}^m \prob{a\notin\cup_{i=1}^n R_i} = \frac{m}{e}.
    \end{align*} 
    Therefore, there exists an MMS multi-allocation $\calA$ that satisfies $\ellzed{\chi^\calA}\leq \frac{m}{e}$. The theorem stands proved. 
\end{proof}




\subsection{Identically Ordered Costs}
\label{subsec:ordered-costs}
This section establishes upper bounds on supply adjustments under cost functions are identically ordered; the setup here is analogous to the one given in \Cref{subsec:additive-ordered} for goods. Formally, 

\begin{definition} \label{def:identically-ordered-costs}
In a fair division instance $\langle [n], [m], \{c_i\}_{i=1}^n \rangle$, the cost functions are said to be \textit{identically ordered} if there exists an indexing $a_1, a_2, \ldots, a_m$ over the chores such that, for any pair of chores $a_s, a_t$, with index $s < t$, we have $c_i(S + a_s) \geq c_i(S + a_t)$ for all agents $i \in [n]$ and each subset $S$ that does not contain $a_s$ and $a_t$.  
\end{definition}

We show that for any fair division instance with identically ordered costs, there exists a multi-allocation $\calA$ such that the number of unassigned chores in $\calA$ is $\widetilde{O} \left(\frac{m}{n^{1/4}} \right)$.

%In particular, we have $c_i(a_1)\geq c_i(a_2)\geq \dots\geq c_i(a_m)$ for all agents $i$. 

\begin{theorem}
\label{thm:additive_chores}
    Every fair division instance $\langle [n], [m], \{c_i\}_{i=1}^n \rangle$ with chores and identically ordered costs admits an MMS multi-allocation $\calA=(A_1,\ldots A_n)$ wherein the number of unassigned chores is at most $\frac{3m \left( \log m \right)^{3/2}}{n^{1/4}}$, i.e., 
    $$\ellzed{\chi^\calA} \leq \frac{3m \left( \log m \right)^{3/2}}{n^{1/4}}.$$
\end{theorem}


\paragraph{Pre-Processing and Random Sampling.} For each agent $i$, consider an MMS-inducing partition $\calM^i=(M^i_1,\ldots,M^i_n)$. Rather than sampling an MMS multi-allocation directly, we first execute a pre-processing step, as detailed in Algorithm \ref{alg:chores-additive}. The algorithm is described in terms of a parameter $\alpha>0$, which will be fixed later. 


\begin{algorithm}
\caption{Pre-processing for Identically Ordered Costs}\label{alg:chores-additive}
\begin{algorithmic}[1]
    \REQUIRE An instance $\calI = \langle [n], [m], \{c_i\}_{i=1}^n \rangle$ along with MMS-inducing partitions $\calM^i=(M^i_1,\ldots,M^i_n)$, for the agents $i$, and parameter $\alpha>0$.
    \ENSURE A multi-allocation $\calB=(B_1,B_2,\dots,B_n)$ and sub-instance $\calI'$
    \STATE Initialize $N=[n]$, $U=[m]$ and $B_i=\emptyset$, for each agent $i\in[n]$.
    \WHILE{there exists agent $i\in N$ and index $j\in[n]$ such that $|M^i_j \cap U|\geq \alpha \ \frac{|U|}{|N|}$}
        \STATE Set $B_i=M^i_j\cap U$. 
        \STATE Update $N \leftarrow N\setminus \{i\}$ and $U \leftarrow U\setminus M^i_j$.
    \ENDWHILE
    \RETURN $\calB$ and sub-instance $\calI'=\langle N, U, \{ c_i \}_{i \in N} \rangle$.
\end{algorithmic}
\end{algorithm}

\Cref{alg:chores-additive} starts with $U = [m]$ as the set of unassigned chores and $N = [n]$ as the set of agents under consideration. The algorithm iterates as long as it is possible to assign to some agent $i \in N$ an MMS-inducing bundle, from $\calM^i$, of sufficiently large cardinality. Specifically, if for some agent $i \in N$ there exists a bundle $M^i_j$ such that $|M^i_j \cap U| \geq \alpha \ \frac{|U|}{|N|}$, then agent $i$ receives the bundle $B_i = M^i_j \cap U$. After this assignment, the algorithm removes both the agent $i$ and the assigned bundle from the instance. Since $c_i(B_i)=c_i(M^i_j\cap U)\leq c_i(M^i_j)\leq \mu_i$, each removed agent receives a bundle of cost at most its MMS.

After the algorithm terminates, we are left with the reduced sub-instance $\calI' = \langle N, U, \{c_i\}_{i\in N}\rangle$, where $N$ and $U$ are the sets of remaining agents (those yet to receive a bundle) and unallocated chores, respectively. A key property of $\calI'$ is that, for each remaining agent $i \in N$, the $n$ MMS-satisfying subsets $M^i_1 \cap U, M^i_2 \cap U, \ldots, M^i_n \cap U$ have size $|M^i_j \cap U| < \alpha \  \frac{|U|}{|N|}$. We will utilize this property and assign chores in $\calI'$ next. 

Note that, in $\calI'$, if $U = \emptyset$ (i.e., no chores remain unassigned at the end of the algorithm), then the upper bound stated in Theorem \ref{thm:additive_chores} holds. In fact, in such a case, for the underlying instance $\calI$, we can identify an MMS multi-allocation $\calA$ with $\ellzed{\chi^\calA}=0$, by setting $A_j = \emptyset$ for all $j \in N$ and $A_i = B_i$ for all $i \in [n] \setminus N$. 

Hence, we will assume that $|U|\geq 1$. Note that all the chores in $[m]\setminus U$ are assigned to some agent in the multi-allocation, $\calB$, returned by the algorithm. Hence, for any $k \in \mathbb{Z}_+$, the existence of an MMS multi-allocation $\calA'$ in $\calI'$ with $\ellzed{\chi^{\calA'}} \leq k$ (i.e., under $\calA'$ at most $k$ chores from $U$ remain unassigned) implies the existence of an MMS multi-allocation $\calA$ in $\calI$ with $\ellzed{\chi^\calA} \leq k$. In particular, we can obtain $\calA$ from $\calA'=(A'_j)_{j \in N}$ by setting $A_i = B_i$ for all $i \in [n] \setminus N$ and $A_j = A'_j$ for all $j \in N$. 

Towards finding such an MMS multi-allocation $\calA'$ in $\calI'$, with the upper bound $k$ as stated in Theorem \ref{thm:additive_chores}, we perform the following random sampling. Independently for each agent $i\in N$, we select a bundle $R_i$ uniformly at random from the set $\{M^i_1\cap U,\dots,M^i_n\cap U\}$, and consider the random multi-allocation $\calR=(R_i)_{i\in N}$ in $\calI'$. As mentioned previously, $ |M^i_j \cap U| < \alpha \  \frac{|U|}{|N|}$ for each agent $i \in N$ and index $j \in [n]$ and, hence, for any random draw, we have   
\begin{align}
    |R_i| < \alpha \ \frac{|U|}{|N|} \quad \text{for each $i \in N$} \label{ineq:sizeR}
\end{align}   

The following lemma provides a useful lower bound on the expected value of the components of $\chi^\calR$.

\begin{lemma}\label{lem:rem-agents}
    In the reduced instance $\calI'=\langle N,U,\{c_i\}_{i\in N}\rangle$ and under the above-mentioned random draws of MMS allocations $\calR=(R_i)_{i\in N}$, we have  
    \begin{align*}
        \expval{\chi^\calR_a}\geq 1-\frac{\log m}{\alpha} \quad \text{for each chore $a\in U$.}
    \end{align*}
\end{lemma}
\begin{proof}
    For the given instance $\calI = \langle [n], [m], \{c_i\}_{i=1}^n \rangle$, write $T$ to denote the number of iterations of the while-loop in \Cref{alg:chores-additive}. Note that the number of agents, $|N|$, remaining in the reduced instance $\calI'$ satisfies $|N| = n - T$. Also, let $U_t$ and $N_t$ denote the sets of remaining chores and agents, respectively, after $t \in \{0,\dots,T\}$ iterations of the while-loop. Note that, during each iteration $t+1$, at least $\alpha\ \frac{|U_t|}{|N_t|}$ chores are assigned to the selected agent. Hence, for each $t\in\{0,\dots,T\}$: 
    \begin{align*}
    |U_{t+1}| \leq|U_t|-\alpha\ \frac{|U_t|}{|N_t|}=|U_t|\ \left(1-\frac{\alpha}{|N_t|}\right) 
     \leq|U_t|\ \left(1-\frac{\alpha}{n}\right).
    \end{align*} 
    Therefore, after $T$ iterations 
    \begin{align}|U_T| \leq m \  \left( 1 - \frac{\alpha}{n} \right)^T \label{ineq:size-of-u} 
    \end{align}
    Since $|U_T|\geq 1$, equation (\ref{ineq:size-of-u}) yields $1-\frac{\alpha}{n}\geq \left(\frac{1}{m}\right)^\frac{1}{T}$. Exponentiating both sides of the previous inequality by  $\frac{n}{\alpha}$ gives us $\frac{1}{e}\simeq (1 - \frac{\alpha}{n})^{\frac{n}{\alpha}} \geq \left(\frac{1}{m}\right)^\frac{n}{\alpha T}$. Simplifying, we obtain $T\leq  \frac{n}{\alpha} \ \log m$ and, hence, 
    \begin{align}
        |N| = |N_T| = n- T \geq n  \left(1-\frac{\log m}{\alpha}\right) \label{ineq:size-of-N}
    \end{align}
    Finally, note that for each chore $a\in U$ 
    \begin{align*}
        \expval{\chi^\calR_a}= \sum_{i\in N} \prob{a\in R_i} = \frac{|N|}{n} \underset{ \text{via (\ref{ineq:size-of-N})}}{\geq} 1-\frac{\log m}{\alpha}.
    \end{align*}
This completes the proof of the lemma. 
\end{proof}

Since the underlying instance $\calI = \langle [n], [m], \{c_i\}_{i=1}^n \rangle$ consists of identically ordered costs, this property continues to hold for the reduced instance $\calI'=\langle N,U,\{c_i\}_{i\in N}\rangle$ as well. Specifically, we will assume, without loss of generality, that the set $U$ consists of chores $a_1,a_2, \ldots, a_{|U|}$, indexed such that for any $a_s$ and $a_t$, with $s<t$, we have $c_i(S + a_s) \geq c_i(S+a_t)$, for all agents $i \in N$ and all subsets $S \subset U$ that do not contain $a_s$ and $a_t$. We now reason about the number of chores assigned, with copies, under the random multi-allocation $\calR$ and in any given sequence of chores $S=\{a_r,a_{r+1},\ldots,a_{s-1}, a_s\}$. We denote by $\mathcal{S}$ the collection of all chore sequences in $U$, i.e., 
\begin{align*}
\mathcal{S}=\left\{ \{a_r,a_{r+1},\dots,a_s\}\mid 1\leq r\leq s\leq |U| \right\}.
\end{align*}


\begin{lemma}\label{lem:ordered-sequences}
In the reduced instance $\calI'=\langle N,U,\{c_i\}_{i\in N}\rangle$ and under the above-mentioned random draws of MMS allocations $\calR=(R_i)_{i\in N}$, the following bound holds for each chore sequence $S \in \mathcal{S}$: 
\begin{align*}
    \prob{\sum_{a\in S} \chi^\calR_a < |S|-\Delta}\leq \frac{1}{m^4},
    \end{align*}
    where 
    \begin{align*}
        \Delta \coloneqq |S|\ \frac{\log m}{\alpha} +2\alpha\ \frac{|U| \sqrt{\log m}}{\sqrt{|N|}}.
    \end{align*}
\end{lemma}

\begin{proof}
    \Cref{lem:rem-agents} gives us 
    \begin{align}
        \expval{\sum_{a\in S}\chi^\calR_a}\geq|S|  \left(1-\frac{\log m}{\alpha}\right) \label{eqn:exp-val-chore}
    \end{align} 
Note that $\sum_{a\in S}\chi^\calR_a=\sum_{i\in N}|R_i\cap S|$, where the random variables $|R_i\cap S|$ are independent across $i \in N$. Also, equation (\ref{ineq:sizeR}) provides a range for these random variables: $0\leq |R_i\cap S| <  \alpha\ \frac{|U|}{|N|}$, for each $i \in N$.
    
We now apply Hoeffding's inequality (lower tail) over the sum of independent random variables $|R_i\cap S|$, with parameter $\delta=2\alpha\ \frac{|U|}{\sqrt{|N|}}\ \sqrt{\log m}$:  
\begin{align}
\prob{\sum_{a\in S}\chi^\calR_a \leq|S|\ \left(1-\frac{\log m}{\alpha}\right)-\delta}
    &\leq \prob{\sum_{a\in S}\chi^\calR_a \leq \expval{\sum_{a\in S}\chi^\calR_a}-\delta} \tag{via (\ref{eqn:exp-val-chore})}\\
    & = \prob{ \sum_{i\in N}|R_i\cap S| \leq \expval{\sum_{i\in N}|R_i\cap S|}-\delta} \nonumber \\ 
    & \leq \exp \left( \frac{- \delta^2}{|N| \alpha^2 \frac{|U|^2}{|N|^2} } \right) \nonumber \\ 
    &\leq \exp(-4\log m) \tag{via value of $\delta$} \\
    &=\frac{1}{m^4} \label{ineq:tail}.
\end{align} 
    Inequality (\ref{ineq:tail}) corresponds to the bound stated in the lemma statement, since
    \begin{align*}
    |S|\ \left(1-\frac{\log m}{\alpha}\right)-\delta &= |S| - |S| \frac{\log m}{\alpha} - 2\alpha \frac{|U|}{\sqrt{|N|}}\ \sqrt{\log m} 
    = S-\Delta.
    \end{align*}
    Recall that $\Delta=|S|\frac{\log m}{\alpha} +2\alpha \frac{|U| \sqrt{\log m}}{\sqrt{|N|}}$. The lemma stands proved. 
\end{proof}

The following lemma shows that covering guarantees with respect to chore sequences, such as the ones obtained in Lemma \ref{lem:ordered-sequences}, translate to upper bounds on the number of unassigned chores. 

\begin{lemma}
\label{lem:shifting_chores}
In fair division instance $\calI'=\langle N,U,\{c_i\}_{i\in N}\rangle$ and for any parameter $\Delta\in \mathbb{Z}_{\geq 0}$, let $\calQ=(Q_i)_{i \in N}$ be a multi-allocation whose characteristic vector $\chi^\calQ$ satisfies $\sum_{a\in S} \chi^\calQ_a\geq |S|-\Delta$ for every chore sequence $S\in\mathcal{S}$. Then, $\calI'$ admits a multi-allocation $\calA'=(A'_i)_{i \in N}$ with the properties that 
\begin{itemize}
        \item[(i)] $c_i(A'_i)\leq c_i(Q_i)$ for all the agents $i$.
        \item[(ii)] $\ellzed{\chi^{\calA'}} \leq \Delta$.
    \end{itemize}
\end{lemma}


\begin{algorithm}
\caption{Copy Redistribution for Identically Ordered Costs}\label{alg:chores-shifting-2}

\begin{algorithmic}[1]
    \REQUIRE A multi-allocation $\calQ$ satisfying the condition of \Cref{lem:shifting_chores}.
    \ENSURE A multi-allocation $\calA'$.
    \STATE Initialize $\calA'=\calQ$.
    \WHILE{there exist chores $b$ and $b'$ with the following properties: (i) index of $b$ is less than that of $b'$ (i.e., $b$ has higher marginal costs than $b'$), (ii) $\chi^{\calA'}_{b}\geq 2$, and (iii) $\chi^{\calA'}_{b'}=0$}
    \STATE Let $a_s$ be the lowest-indexed chore with $\chi^{\calA'}_{a_s}\geq 2$ and let $a_t$ be the lowest-indexed chore such that $t > s$ and $\chi^{\calA'}_{a_t} = 0$. \label{line:lowchore}
    \STATE Select an agent $i$ with $a_s\in A'_i$ and update $A'_i \leftarrow (A'_i \setminus \{a_s \} ) \cup \{a_t\}$.
    \ENDWHILE
    \RETURN $\calA'$
\end{algorithmic}
\end{algorithm}

\begin{proof}
    We will show that---given a multi-allocation $\calQ$ that satisfies the stated guarantees with respect to chore sequences---\Cref{alg:chores-shifting-2} returns the desired multi-allocation $\calA'$. \Cref{alg:chores-shifting-2}, in each iteration of its while-loop, replaces a chore $a_s$ in the selected agent $i$'s bundle, $A'_i$, by another chore $a_t$ of higher index $t > s$. Since $a_s$ was previously allocated to at least two agents and $a_t$ was unallocated, the number of unallocated chores in the maintained multi-allocation $\calA'$ decreases by one in each iteration. That is, the algorithm terminates in polynomial time. Moreover, given that $a_t$ has a higher marginal cost than $a_s$ with respect to $A'_i \setminus \{a_s\}$, the agents' allocated costs $c_i(A'_i)$ are non-increasing throughout the execution of the algorithm. Hence, for the returned multi-allocation $\calA'$, in particular, we have $c_i(A'_i) \leq c_i(Q_i)$, for all the agents $i$. This observation establishes property (i) as stated in the lemma. 
    
    
    We next establish property (ii) by proving that upon termination of the algorithm, the returned multi-allocation $\calA'$ satisfies $\ellzed{\chi^{\calA'}} \leq \Delta$. Assume, towards a contradiction, that $\ellzed{\chi^{\calA'}}\geq \Delta+1$. Write $a_r$ to denote the chore of highest index $r$ with the property $\chi^{\calA'}_{a_r}=0$. Also, let $S$ denote the sequence of chores in $U$ with index at most $r$. Since $a_r$ is the highest index chore that is not assigned ($\chi^{\calA'}_{a_r}=0$), we have $\chi^{\calA'}_{b'} \geq 1$ for all chores $b'$ with index higher than $r$. Hence, all the unassigned chores are in fact contained in $S$; recall that there are at least $(\Delta +1)$ unassigned chores under $\calA'$. Further, the fact that the while-loop of the algorithm terminated with multi-allocation $\calA'$ and $\chi^{\calA'}_{a_r}=0$ ensures that $\chi^{\calA'}_b \leq 1$ for all chores $b \in S$. These observations imply that 
    \begin{align}
        \sum_{b \in S} \chi^{\calA'}_b \leq |S| - \Delta -1 \label{ineq:prefix-delta}
    \end{align}
    Further, note that, since $\chi^{\calA'}_{a_r} = 0$, during any iteration of the while-loop, the algorithm would not have replaced a copy of a chore $a_s \in S$ with a copy of a chore $a_t \notin S$. That is, it could not have been the case that, during any iteration, the considered chores $a_s$ and $a_t$ satisfy $s < r < t$, since such a relation between the indices contradicts the selection criterion of $a_t$ as the lowest-index chore with $\chi^{\calA'}_{a_t} = 0$. Therefore, throughout the algorithm's execution, for any considered chore $a_s \in S$, the corresponding chore $a_t$ must have also been contained in $S$. Considering this property along with the decrements and increments of $\chi^{\calA'}_{a_s}$ and $\chi^{\calA'}_{a_t}$, respectively, we obtain $\sum_{b \in S} \chi^\calQ_b = \sum_{b \in S} \chi^{\calA'}_b$. 

    The last equality and equation (\ref{ineq:prefix-delta}) give us $\sum_{b \in S} \chi^\calQ_b \leq |S| - \Delta -1$. This bound, however, contradicts the condition provided for $\calQ$ in the lemma statement. Hence, by way of contradiction, we obtain property (ii), $\ellzed{\chi^{\calA'}} \leq \Delta$. The lemma stands proved. 
\end{proof}

\medskip

    \begin{proof}[Proof of \Cref{thm:additive_chores}]

    Let $\calI=\langle[n],[m],\{c_i\}_{i=1}^n\rangle$ be the original fair division instance with chores and additive ordered costs. We first execute the pre-processing via \Cref{alg:chores-additive} (with parameter $\alpha \geq 1$, to be fixed later) and obtain a multi-allocation $\calB=(B_1,\ldots,B_n)$ along with sub-instance $\calI'=\langle N,U,\{c_i\}_{i\in N}\rangle$. Recall that $N$ and $U$ are the sets of remaining agents and unallocated chores, respectively. Also, all the chores in $[m] \setminus U$ are assigned among the agents in the set $[n] \setminus N$ with MMS guarantees; in particular, $\cup_{i \in [n] \setminus N} B_i = [m] \setminus U$ and $c_i(B_i) \leq \mu_i$ for each agent $i \in [n] \setminus N$.  
    
    We then focus on the sub-instance $\calI'=\langle N,U,\{c_i\}_{i\in N}\rangle$ with the goal of assigning chores from $U$ while ensuring MMS among the agents in $N$. Towards this and as mentioned previously, we sample an MMS multi-allocation $\calR=(R_i)_{i\in N}$ by drawing $R_i$ uniformly at random from $\{M^i_j\cap U\}_{j=1}^n$ for each $i\in N$. \Cref{lem:ordered-sequences} ensures that, for each chore sequence $S\in \mathcal{S}$, we have  
    \begin{align}
        \prob{\sum_{a\in S} \chi^\calR_a < |S|-\Delta}\leq \frac{1}{m^4}  \qquad \text{where } \Delta =|S|\ \frac{\log m}{\alpha} +2\alpha\ \frac{|U| \sqrt{\log m}}{\sqrt{|N|}} \label{eq:each-sequence}
    \end{align} 
    We fix $\alpha \coloneqq  n^{1/4} \log m$. This choice and inequality (\ref{ineq:size-of-N}) gives us $|N| \geq n \left(1 - \frac{\log m}{\alpha} \right) = n - n^{3/4} = n(1 - o(1))$. Further, 
     \begin{align}
        \Delta & =|S|\ \frac{\log m}{\alpha} \ + \ 2\alpha\ \frac{|U| \sqrt{\log m}}{\sqrt{|N|}} \nonumber \\ 
        & \leq \frac{m\log m}{\alpha} \ + \ \frac{2\alpha m\sqrt{\log m}}{\sqrt{|N|}} \tag{since $|S| \leq |U| \leq m$} \\
        & = \frac{m}{n^{1/4}}  \ + \ \frac{2  m n^{1/4} \left( \log m \right)^{3/2}}{ \sqrt{|N|} } \tag{$\alpha = n^{1/4} \log m$ and $|N| \geq n(1- o(1))$} \\
        & \leq \frac{3  m \left(\log  m \right)^{3/2}}{n^{1/4}} \label{eq:eq_alpha}
    \end{align}
    
    
    Write $\Delta^*$ to denote the upper bound on the deviation obtained in equation (\ref{eq:eq_alpha}), $\Delta^* \coloneqq \frac{3m \left(\log  m\right)^{3/2}}{n^{1/4}}$. Given that $|\mathcal{S}| \leq |U|^2\leq m^2$, applying union bound over all the chore sequences in $\mathcal{S}$ gives us  
    \begin{align*}
        \prob{ \sum_{a\in S} \chi^\calR_a \geq |S|-\Delta^*  \ \text{ for each } S\in\mathcal{S}}
    \geq 1 - \sum_{S \in \mathcal{S}} \prob{\sum_{a \in S} \chi^\calR_a < |S|-\Delta^*} \underset{\text{via (\ref{eq:each-sequence})}}{\geq} 
    1-\frac{m^2}{m^4}>0.
    \end{align*}
    Since this probability is positive, there exists an MMS multi-allocation $\calQ=(Q_i)_{i\in N}$ that satisfies $\sum_{a\in S} \chi^{\calQ}_a \geq |S|-\Delta^*$ for every chore sequence $S \in \mathcal{S}$. This is the condition required to invoke \Cref{lem:shifting_chores}, which, in turn, implies the existence of an MMS multi-allocation $\calA'=(A'_i)_{i\in N}$ (in $\calI'$) with $\ellzed{\chi^{\calA'}}\leq \Delta^*$. The final multi-allocation $\calA=(A_1,\ldots,A_n)$ for the underlying instance $\calI$ is obtained by combining $\calA'$ and $\calB$: For each agent $i$, set 
    \begin{align*}
        A_i=\begin{cases}
        A'_i & \text{ if }i\in N,\\
        B_i &\text{ otherwise, if $i \in [n] \setminus N$}.
    \end{cases}
    \end{align*}
    By construction, for each agent $i \in [n]$, we have $c_i(A_i) \leq \mu_i$. Hence, $\calA$ is an MMS multi-allocation. Also, since all the chores in $[m] \setminus U$ are assigned under $\calB$, we have that $\ellzed{\chi^\calA} = \ellzed{\chi^{\calA'}} \leq \Delta^*$. Overall, we get that $\calA$ satisfies the properties stated in the theorem. This completes the proof.  
    \end{proof}
       

    \subsection{Lower Bound for Chores}

    We now prove that there exists fair division instances, with $m$ chores and monotone costs, such that in every MMS multi-allocation, at least $(1- o(1))\nicefrac{m}{e}$ chores remain unassigned. This lower bound shows that the positive result obtained in Theorem \ref{thm:monotone_chores} is essentially tight. 

    \begin{theorem}
    \label{thm:chores_lowerbound}

 For any $\delta>0$, there exists a fair division instance $\calI = \langle [n], [m], \{c_i\}_{i=1}^n \rangle$ with monotone costs such that under every MMS multi-allocation $\calA$ in $\calI$ at least $(1-\delta)\frac{m}{e}$ chores remain unassigned, i.e., for every MMS multi-allocation $\calA$ in $\calI$ we have $\ellzed{\chi^\calA}\geq (1-\delta)\frac{m}{e}$. 
\end{theorem}

\begin{proof}
    We will construct an instance with $n$ agents and $m$ chores with $m> \frac{2e}{\delta^2}n\log n$. For each agent $i$, we independently draw an $n$-partition $\calP^i=(P^i_1,P^i_2,\dots,P^i_n)$ of the set $[m]$ of chores uniformly at random from the set of all such partitions, i.e., $\calP^i \in_R \Pi_n([m])$. Given such a partition, we set the cost of each subset $S\subseteq [m]$ as 
    \begin{align*}
        c_i(S) =\begin{cases}
        0  & \text{if } S\subseteq P^i_j \text{ for any $j \in [n]$,}\\
        1  & \text{otherwise}.
    \end{cases}
    \end{align*}
    
    
     Note that each $c_i$ is a monotone set function. Further, for each agent $i$ the MMS value $\mu_i=0$, which implies that for any MMS multi-allocation $\calA=(A_1,\ldots,A_n)$ we have  $A_i\subseteq P^i_j$ for each $i\in [n]$ and some index $j\in[n]$. We may, in fact, restrict ourselves to those multi-allocations that satisfy $A_i=P^i_j$, for some $j$, since those that do not will only leave more chores unallocated. Hence, we define the family $\calF$ of MMS multi-allocations as 
     \begin{align*}
         \calF=\left\{ (A_1,A_2,\dots,A_n) \mid \text{ for each } i\in[n], \text{ there exists }j\in[n]\text{ such that } A_i=P^i_j\right\}.
     \end{align*} 
     Note that $|\mathcal{F}|=n^n$. Fix the multi-allocation $\calQ\in\calF$ obtained by setting index $j=1$ for all agents $i$, i.e. $\calQ=(P^1_1,P^2_1,\dots,P^n_1)$. Considering the random draws that result in the partitions $\calP^i= (P^i_1,\dots,P^i_n)$, for each fixed agent $i$ and chore $a$, we have $\prob{a\in P^i_i}=1/n$. Further, the fact that these draws are independent across the agents gives us $\prob{a\notin\cup_{i=1}^n P^i_1} = \left(1-\frac{1}{n}\right)^n$. Hence, 
     \begin{align*}
     \expval{\ellzed{\chi^\calQ}} =\sum_{a=1}^m \prob{a\notin\cup_{i=1}^n P^i_1} = \sum_{a=1}^m \left(1-\frac{1}{n}\right)^n \simeq\frac{m}{e}.
     \end{align*} 
     We next utilize Chernoff bound (lower tail), with $\delta>0$, to obtain 
     \begin{align}
     \prob{\ellzed{\chi^{\calQ}} < (1-\delta)\frac{m}{e}} \leq \exp\left(-\frac{\delta^2m}{2e}\right)< \frac{1}{n^n} \label{eq:each-A}
     \end{align} 
     The last inequality follows since $m> \frac{2e}{\delta^2}n\log n$. Note that the bound (\ref{eq:each-A}) holds for each MMS multi-allocation $\calQ \in \calF$. Let $E_\calQ$ denote the event that $\ellzed{\chi^{\calQ}}\geq (1-\delta)\frac{m}{e}$ and note that $\prob{E^c_\calQ} \leq \frac{1}{n^n}$. Now, applying union bound over all $\calQ \in \calF$ gives us  
     \begin{align*}
         \prob{\bigcap_{\calQ \in \calF} E_\calQ} = 1-\prob{\bigcup_{\calQ\in\calF} E_\calQ^c}\geq 1-n^n\  \prob{E_\calQ^c}>1-\frac{n^n}{n^n}=0.
     \end{align*}
     Therefore, with non-zero probability, the events $E_\calQ$ hold together for all MMS multi-allocations $\calQ\in \calF$. This implies that there exists an instance $\calI$, determined by the choice of partitions $\calP^i=(P^i_1,\dots,P^i_n)$, such that every MMS multi-allocation in $\calI$ leaves at least $(1-\delta)\frac{m}{e}$ chores unallocated. This completes the proof.

    
\end{proof}



\subsection{Additive Costs}

When agents' costs are additive, the following theorem shows that MMS fairness can be achieved by leaving at most $\frac{2}{11}m + n$ chores unassigned.

\begin{theorem}
    \label{thm:chores-additive}
        Every fair division instance with chores and additive costs admits an MMS multi-allocation $\calA = (A_1, \dots, A_n)$ in which at most $\frac{2}{11}m + n$ chores remain unassigned. That is, the characteristic vector $\chi^\calA$ of $\calA$ satisfies $\ellzed{\chi^\calA} \leq \frac{2}{11}m + n$.
    \end{theorem}
    
    \begin{proof}
        The work of Huang and Lu \cite{10.1145/3465456.3467555} shows that fair division instances with chores and additive costs admit an exact allocation that assigns to each agent a bundle of cost no more than $\frac{11}{9}$ times its MMS. Let $\calA' = (A'_1, \dots, A'_n)$ be such an allocation for the given instance. Specifically, we have \begin{equation} c_i(A'_i) \leq \frac{11}{9} \mu_i \label{eq:chores-approx}
        \end{equation} for each agent $i \in [n]$.
    
        Consider the bundle $A'_i$ received by some fixed agent $i$ under $\calA'$, and assume that $|A'_i| = k_i$. Write $A'_i = \{a_1, \dots, a_{k_i}\}$ to denote the indexing of the elements of $A'_i$ that satisfies $c_i(a_1) \geq \dots \geq c_i(a_{k_i})$. Define $p_i \coloneqq \left\lceil \frac{2}{11} k_i \right\rceil$ and $U_i \coloneqq \{a_1, \dots, a_{p_i}\}.$ Since $c_i(a_1) \geq \dots \geq c_i(a_{p_i}) \geq \dots \geq c_i(a_{k_i})$, an averaging argument gives $\frac{1}{p_i} c_i(U_i) \geq \frac{1}{k_i} c_i(A'_i)$ and, hence, 
        \begin{equation} c_i(U_i) \geq \frac{p_i}{k_i} c_i(A'_i) \geq \frac{2}{11} c_i(A'_i). \label{eq:unassigned}
        \end{equation}
    
        Now, define the multi-allocation $\calA = (A_i, \dots, A_n)$ by setting $A_i = A'_i \setminus U_i$ for each agent $i$. Then, $$v_i(A_i) 
        = v_i(A'_i) - v_i(U_i) 
        \underset{\text{via (\ref{eq:unassigned})}}{\leq} \frac{9}{11} c_i(A'_i) 
        \underset{\text{via (\ref{eq:chores-approx})}}{\leq} \mu_i.$$ That is, $\calA$ is an MMS multi-allocation. Moreover, the set of unassigned chores in $\calA$ is exactly $U \coloneqq \cup_{i=1}^n U_i$. Hence, 
        \begin{align*} \ellzed{\chi^\calA} 
            = |U| 
            = \sum_{i=1}^n |U_i|
            = \sum_{i=1}^n p_i 
            = \sum_{i=1}^n \left\lceil \frac{2}{11} k_i \right\rceil 
            \leq \sum_{i=1}^n \left( \frac{2}{11} k_i + 1 \right) 
            = \frac{2}{11}m + n. 
        \end{align*}

        The theorem stands proved.
    \end{proof}
 
\section{Conclusion}
In this work, we propose a simple yet effective approach, called SMILE, for graph few-shot learning with fewer tasks. Specifically, we introduce a novel dual-level mixup strategy, including within-task and across-task mixup, for enriching the diversity of nodes within each task and the diversity of tasks. Also, we incorporate the degree-based prior information to learn expressive node embeddings. Theoretically, we prove that SMILE effectively enhances the model's generalization performance. Empirically, we conduct extensive experiments on multiple benchmarks and the results suggest that SMILE significantly outperforms other baselines, including both in-domain and cross-domain few-shot settings.

\bibliographystyle{alpha}
\bibliography{refs}

\appendix
\section{Maximin Shares with Arbitrary Entitlements}
\label{section:entitlements}
We have thus far studied fair division among agents who have equal entitlements over the items. The current section complements this treatment and addresses agents with different entitlements. In particular, we consider settings in which each agent $i \in [n]$ is endowed with an entitlement $b_i \in (0,1]$ and the sum of the agents' entitlements is equal to one, $\sum_{i=1}^n b_i = 1$. Agents with higher entitlements $b_i$ stake a higher claim on the items. Also, the equal-entitlements setting considered in previous sections corresponds to $b_i = 1/n$, for all the agents. 


Focusing on division of goods, we will obtain results for a shared-based fairness notion, $\mmshat$, defined in \cite{Babaioff2024ShareBasedFF}. This notion generalizes MMS to the current context of distinct entitlements. In fact, Babaioff and Feige \cite{Babaioff2024ShareBasedFF} show that $\mmshat$ \emph{dominates} other shared-based notions.   

Here, we denote a fair division instance via the tuple $\langle [n], [m], \{v_i\}_{i=1}^n, \{b_i\}_{i=1}^n \rangle$ in which $b_i$s denote the agents' entitlements and, as in Section \ref{sec:goods}, $v_i$s denote the agents' valuations. We also index the agents in decreasing order of their entitlements, $b_1 \geq b_2 \geq \ldots \geq b_n$. The entitlement-adapted share, $\widehat{\mu}_i$ of each agent $i$ is obtained by considering size $n_i = \lfloor \frac{1}{b_i} \rfloor$ partitions of the $m$ goods. Formally,  
\begin{definition}[$\mmshat$] For any fair division instance $\langle [n], [m], \{v_i\}_{i=1}^n, \{b_i\}_{i=1}^n \rangle$ with goods and entitlements, the maximin share, $\muhat_i$, of agent $i$ is defined as
$$\muhat_i \coloneqq  \max_{(X_1, \ldots X_{n_i})\in \Pi_{n_i}([m])} \ \ \min_{j \in [n_i]} v_i(X_j). $$
Here, $n_i=\lfloor\frac{1}{b_i}\rfloor$ and the maximum is taken over all $n_i$-partitions of $[m]$. Furthermore, a multi-allocation $\calA = (A_1, \ldots A_n)$ is said to be an $\mmshat$ multi-allocation if it satisfies $v_i(A_i) \geq \widehat{\mu_i}$ for each agent $i \in [n]$. \\
\end{definition}

As before, we leverage the existence of an $\mmshat$-inducing $n_i$-partition $\calM^i=(M^i_1, M^i_2, \dots, M^i_{n_i})$ for each agent $i$, given by $$\calM^i \in \argmax_{(X_1, \ldots X_{n_i})\in \Pi_{n_i}([m])} \ \  \min_{j \in [n_i]} v_i(X_j).$$

The main result of this section is as follows.

\begin{theorem}
\label{thm:mms_hat_monotone_goods}
        Every fair division instance $\langle [n], [m], \{v_i\}_{i=1}^n, \{b_i\}_{i=1}^n \rangle$ with monotone valuations and arbitrary entitlements admits an $\mmshat$ multi-allocation $\calA = (A_1,\ldots A_n)$ in which no single good is assigned to more than $3 \log m$ agents and the total number of goods assigned, with copies, is at most $\lceil 1.7  m \rceil$. That is, the characteristic vector $\chi^\calA$ of $\calA$ satisfies $\|\chi^\calA\|_\infty \leq 3 \log m$ and $\|\chi^\calA\|_1\leq \lceil 1.7 m \rceil$.
\end{theorem}

As in the proof of \Cref{thm:monotone_goods}, we proceed via the probabilistic method. The random draw here is as follows: For each agent $i\in [n]$, recall that $\calM^i=(M^i_1,\dots,M^i_{n_i})$ is an $\mmshat$-inducing partition for $i$. Independently for each $i$, we select a bundle $R_i$ uniformly at random from $\{M^i_1,\dots,M^i_{n_i}\}$, i.e., $\prob{R_i = M^i_j}=1/n_i$ for each index $j\in [n_i]$. Considering the random $\mmshat$ multi-allocation $\calR=(R_1,\dots,R_n)$, let $\chi^\calR$ be the characteristic vector of $\calR$. We first prove the following lemma which bounds the expected value of the components of the random vector $\chi^\calR$.

\begin{lemma}
    $\expval{\chi^\calR_g} \leq 1.7$ for each component (good) $g \in [m]$. \label{lem:expval-entitlements}
\end{lemma}

\begin{proof}
    For each agent $i\in [n]$ and good $g\in [m]$, let $\mathbb{1}_{\{g \in R_i\}}$ be the indicator random variable for the event $g\in R_i$. Since $R_i$ is chosen uniformly at random from the $n_i$-partition $\{M^i_1,\dots,M^i_{n_i}\}$ of $[m]$ and each good $g\in [m]$ belongs to exactly one subset $M^i_j$, we have for each fixed $i\in [n]$ and $g \in [m]$
\begin{equation}
    \expval{\mathbb{1}_{\{g\in R_i\}}} = \prob{g \in R_i} = \frac{1}{n_i}. \label{eq:indicator-exp}
\end{equation}

Further, $\chi^\calR_g$ is the number of copies of the good $g$ assigned among the agents under the multi-allocation $\calR$, i.e., $\chi^\calR_g = \sum_{i=1}^n \mathbb{1}_{\{g \in R_i\}}$.  Therefore, 
\begin{equation}
\label{eq:eq1} \expval{\chi^\calR_g} = \sum_{i=1}^n \expval{\mathbb{1}_{ \{ g \in R_i \} }} 
\underset{\text{via (\ref{eq:indicator-exp})}}{=} \sum_{i=1}^n \frac{1}{n_i}
\end{equation}

We next establish an upper bound on the right-hand-side of equation (\ref{eq:eq1}). For each agent $i \in [n]$, let $k_i$ be the unique integer such that $\frac{1}{k_i+1} < b_i \leq \frac{1}{k_i}$. For the integer $k_i$, note that the bound $k_i \leq \frac{1}{b_i} < k_i + 1$ implies  
\begin{equation}
    k_i = \left\lfloor \frac{1}{b_i} \right\rfloor = n_i \label{eq:ni-value}
\end{equation}
Further, given that $\frac{1}{b_i} < k_i +1$, we obtain  $k_i > \frac{1}{b_i} -1$ and, hence, 
\begin{equation}
    \frac{1}{n_i} = \frac{1}{k_i} < \frac{1}{\frac{1}{b_i} - 1} = \frac{b_i}{1 - b_i}. \label{eq:upperbound-bi}
\end{equation}

Recall the agents are indexed such that $b_1 \geq b_2 \geq \dots \geq b_n$. We proceed via the following case analysis.\\

\noindent\textit{Case 1}: $b_1 \leq \frac{1}{3}$. Here, for each agent $i$, the entitlement $b_i \leq \frac{1}{3}$; equivalently, $1-b_i \geq \frac{2}{3}$. Therefore, 
\begin{align*}
\sum_{i=1}^n \frac{1}{n_i} 
& < \sum_{i=1}^n \frac{b_i}{1-b_i} \tag{via (\ref{eq:upperbound-bi})} \\
&\leq \frac{3}{2} \sum_{i=1}^n b_i \tag{since $1-b_i \geq \frac{2}{3}$ for all $i$} \\
&= \frac{3}{2} \tag{since $\sum_{i=1}^n b_i = 1$}
\end{align*}

\noindent\emph{Case 2}: $\frac{1}{3} < b_1 \leq \frac{1}{2}$. By the definition of $k_i$ and equation (\ref{eq:ni-value}), we obtain $n_1= k_1 = 2$. With $b_1\geq b_2$, we consider the following two sub-cases. \\

    \noindent
    \emph{Case 2(a):} $b_2 \leq \frac{1}{3}$.  For each $i \geq 2$, we have $b_i\leq \frac{1}{3}$ and, hence, $1 - b_i \geq \frac{2}{3}$. Therefore, 
    \begin{align*}
    \sum_{i=1}^n\frac{1}{n_i} 
    & < \frac{1}{n_1} + \sum_{i=2}^n \frac{b_i}{1-b_i} \tag{via (\ref{eq:upperbound-bi})} \\ 
    & \leq \frac{1}{2} + \frac{3}{2} \sum_{i=2}^n b_i \tag{since $n_1 = 2$ and $1 - b_i> \frac{2}{3}$ for $i\geq 2$} \\
    & = \frac{1}{2} + \frac{3}{2}(1-b_1)   \tag{since $\sum_{i=1}^n b_i = 1$}\\
    & <  \frac{1}{2} + \frac{3}{2}\left(1-\frac{1}{3}\right) \tag{since $b_1>\frac{1}{3}$}\\
    &= \frac{3}{2}
    \end{align*}

    \noindent
    \emph{Case 2(b):} $\frac{1}{3} < b_2 \leq b_1 \leq \frac{1}{2}$. In this sub-case, we have $n_2 = k_2 = 2$ as well. Since $\sum_{i=1}^n b_i = 1$, for the remaining agents $i \geq 3$, we have $b_i < \frac{1}{3}$. Hence,
    \begin{align*}
    \sum_{i=1}^n\frac{1}{n_i} 
    &<  \frac{1}{n_1} + \frac{1}{n_2} + \sum_{i=3}^n \frac{b_i}{1-b_i} \tag{via (\ref{eq:upperbound-bi})} \\ 
    & < \frac{1}{2}+\frac{1}{2} + \frac{3}{2} \sum_{i=3}^n b_i \tag{since $n_1 = n_2 = 2$ and $1 - b_i> \frac{2}{3}$ for $i\geq 3$} \\
    & = 1 + \frac{3}{2}(1-b_1-b_2)  \tag{since $\sum_{i=1}^n b_i = 1$}\\
    & <  1 + \frac{3}{2} \left(1-\frac{1}{3}-\frac{1}{3}\right) \tag{since $b_1\geq b_2 > \frac{1}{3}$} \\
    &= \frac{3}{2}.
    \end{align*}

\noindent\emph{Case 3}: $b_1> \frac{1}{2}$. In this case, $n_1 = k_1 = 1$. Also, $\sum_{i=1}^n b_i=1$ implies that $b_2 < \frac{1}{2}$. Therefore, the following three sub-cases (analyzed below) are exhaustive: 

\noindent
Case 3(a): $b_2 \leq 1/4$. 

\noindent
Case 3(b): $b_2 \in (1/4, 1/3]$. 

\noindent
Case 3(c): $b_2 \in (1/3, 1/2)$.  \\

   
   \noindent
   \emph{Case 3(a):} $b_2 \leq \frac{1}{4}$. Here, for all $i \geq 2$, we have $1 - b_i \geq 1 - b_2 \geq 3/4$. Using this inequality, we obtain   
   \begin{align*}
    \sum_{i=1}^n\frac{1}{n_i} 
    & < \frac{1}{n_1} + \sum_{i=2}^n \frac{b_i}{1-b_i} \tag{via (\ref{eq:upperbound-bi})} \\ 
    & \leq 1 + \frac{4}{3} \sum_{i=2}^n b_i \tag{since $n_1 = 1$ and $1 - b_i> \frac{3}{4}$ for $i\geq 2$} \\
    & = 1 + \frac{4}{3}(1-b_1)   \tag{since $\sum_{i=1}^n b_i = 1$}\\
    & <  1 + \frac{4}{3}\left(1-\frac{1}{2}\right) \tag{since $b_1>\frac{1}{2}$}\\
    &= \frac{5}{3}
    \end{align*}
    
   
   \noindent
   \emph{Case 3(b):} $\frac{1}{4} < b_2 \leq \frac{1}{3}$. In this sub-case, the definition of $k_i$ gives us $n_2 = k_2 = 3$. Further, for all $i \geq 3$, we have $b_i \leq b_3 \leq 1 - b_1 - b_2 \leq 1 - \frac{1}{2} - \frac{1}{4} = \frac{1}{4}$. That is, $1 - b_i \geq \frac{3}{4}$ for each $i \geq 3$. These inequalities give s 
    \begin{align*}
    \sum_{i=1}^n\frac{1}{n_i} 
    & < \frac{1}{n_1} + \frac{1}{n_2} + \sum_{i=3}^n \frac{b_i}{1-b_i} \tag{via (\ref{eq:upperbound-bi})} \\ 
    & \leq 1 + \frac{1}{3} + \frac{4}{3} \sum_{i=3}^n b_i \tag{since $n_1 = 1$, $n_2 = 3$, and $1 - b_i> \frac{3}{4}$ for $i\geq 3$} \\
    & = 1 + \frac{1}{3} +  \frac{4}{3}(1-b_1 - b_2)   \tag{since $\sum_{i=1}^n b_i = 1$}\\
    & <  1 + \frac{1}{3} + \frac{4}{3}\left(1-\frac{1}{2} - \frac{1}{4} \right) \tag{since $b_1>\frac{1}{2}$ and $b_2 > \frac{1}{4}$}\\
    &= \frac{5}{3}
    \end{align*}

    \emph{Case 3(c):} $\frac{1}{3} < b_2 < \frac{1}{2}$. Here, we have $n_2 = k_2 = 2$. Also, the equality $\sum_{i=1}^n b_i = 1$ implies $b_3 \leq 1-b_1 -b_2 < 1-\frac{1}{2} -\frac{1}{3} =\frac{1}{6}$. Hence, $1-b_i>\frac{5}{6}$ for each $i\geq 3$. Using these bounds we obtain    
    \begin{align*}
    \sum_{i=1}^n \frac{1}{n_i} 
    &<  \frac{1}{n_1} + \frac{1}{n_2} + \sum_{i=3}^n \frac{b_i}{1-b_i} \tag{via (\ref{eq:upperbound-bi})} \\ 
    & < 1 + \frac{1}{2} + \frac{6}{5} \sum_{i=3}^n b_i \tag{since $n_1 = 1$, $n_2 = 2$, and $1 - b_i> \frac{5}{6}$ for $i\geq 3$} \\
    & = 1 + \frac{1}{2} + \frac{6}{5}(1-b_1-b_2)  \tag{since $\sum_{i=1}^n b_i = 1$}\\
    & <  1 + \frac{1}{2} + \frac{6}{5} \left(1-\frac{1}{2}-\frac{1}{3}\right) \tag{since $b_1 >\frac{1}{2}$ and $ b_2 > \frac{1}{3}$} \\
    &= 1+ \frac{1}{2} + \frac{1}{5} \nonumber \\
    &= \frac{17}{10}.  
    \end{align*}

The case analysis above shows that, for any set of entitlements, $\{b_i\}_{i=1}^n$, the sum $\sum_{i=1}^n \frac{1}{n_i}$ is at most $\max \left\{\frac{3}{2}, \frac{5}{3}, \frac{17}{10} \right\} = \frac{17}{10} = 1.7$. Hence, equation (\ref{eq:eq1}) gives us $\mathbb{E}[\chi^\calR_g] \leq 1.7$. The lemma stands proved. 
\end{proof}

Using \Cref{lem:expval-entitlements}, we now prove probability bounds for the following events: $G_1$ is the event that $\|\chi^\calR\|_\infty \leq 3 \log m$ and $G_2$ is the event that $\|\chi^\calR\|_1 \leq \lceil 1.7m \rceil$. 

\begin{lemma}
\label{lem:linftynormforb} 
The random characteristic vector $\chi^\calR$ satisfies: 
\begin{enumerate}
    \item[(i)] $\prob{G_1^c} \leq \frac{1}{m^{2}}.$
    \item[(ii)] $\prob{G_2^c} \leq \frac{1.7m}{ \lceil1.7m\rceil+1}$.
\end{enumerate}
\end{lemma}
\begin{proof} 
To prove (i), note that for a fixed good $g\in [m]$, the indicator random variables $\mathbb{1}_{\{g\in R_i\}}$ are independent across $i$s. Hence, the count $\chi^\calR_g=\sum_{i=1}^n \mathbb{1}_{\{g\in R_i\}}$ is a sum of independent Bernoulli random variables. Using Chernoff bound \cite[Theorem 4.4]{mitzenmacher2017probability}, we have $\prob{\chi^\calR_g \geq t} \leq \frac{1}{2^t}$ for any $t \geq 6 \expval{\chi^\calR_g}$. Since $\expval{\chi^\calR_g} \leq 1.7 \leq 2$ (\Cref{lem:expval-entitlements}), we can instantiate this bound with $t = \max \{12, 3 \log m \} \geq   6 \ \expval{\chi^\calR_g}$. For an appropriately large $m$, we obtain $ t = 3 \log m$ and, hence, $\prob{\chi^\calR_g \geq 3 \log m} \leq \frac{1}{m^{3}}$ for each $g\in [m]$. Write $G_{1,g}^c$ to denote the event $\{\chi^\calR_g \geq 3\log m\}$, and note that $G_1^c = \cup_{g=1}^m G_{1,g}^c$. We have  $\prob{G_{1,g}^c}\leq \frac{1}{m^{3}}$ for each $g\in [m]$. Hence, the union bound gives us $$\prob{G_1^c} = \prob{\cup_{g=1}^m G_{1,g}^c} \leq \sum_{g=1}^m \prob{G_{1,g}^c} \leq m \frac{1}{m^{3}} =\frac{1}{m^{2}}.$$ This establishes part (i) of the lemma. 


For part (ii), using \Cref{lem:expval-entitlements}, we first obtain $\expval{\ellone{\chi^\calR}} 
= \expval{\sum_{g=1}^m \chi^\calR_g} 
= \sum_{g=1}^m \expval{\chi^R_g} 
\leq 1.7m$. Recall that $G_2$ is the event that $\|\chi^\calR\|_1 \leq \lceil 1.7m \rceil$. Markov's inequality and the fact that $\ellone{\chi^\calR} $ is an integer-valued random variable give us 
$$\prob{G_2^c} = \prob{\ellone{\chi^\calR} \geq \lceil 1.7m \rceil + 1} 
\leq \frac{\expval{\ellone{\chi^\calR}}}{ \lceil1.7m\rceil+1} 
\leq \frac{1.7m}{ \lceil1.7m\rceil+1}.$$ The theorem stands proved.
\end{proof}

Using the above lemma, we next prove \Cref{thm:mms_hat_monotone_goods}.

\begin{proof}[Proof of \Cref{thm:mms_hat_monotone_goods}] Recall that $G_1$ is the event that $\ellinfty{\chi^\calR} \leq 3 \log m$ and $G_2$ is the event that $\ellone{\chi^\calR} \leq \lceil 1.7m \rceil$. By \Cref{lem:linftynormforb} and the union bound, we have \begin{align*} \displaystyle \prob{G_1 \cap G_2} 
& = 1 - \prob{G_1^c \cup G_2^c} \\ 
& \geq 1 - (\prob{G_1^c} +\mathbb{P}\{G_2^c\}) \tag{Union bound} \\ 
& \geq 1-\left(\frac{1}{m^{2}} + \frac{1.7m}{\lceil1.7m\rceil+1} \right) \tag{by \Cref{lem:linftynormforb}} \\
& = \frac{\lceil 1.7m\rceil + 1 - 1.7m}{\lceil1.7m\rceil+1} - \frac{1}{m^{2}} \\
& \geq \frac{1}{\lceil1.7m\rceil+1} - \frac{1}{m^{2}} > 0.
\end{align*}

Hence, the events $G_1$ and $G_2$ hold together with positive probability for the random MMS multi-allocation $\calR = (R_1,\dots,R_n)$. This implies that there exists an MMS multi-allocation $\calA$ that satisfies the stated $\ell_\infty$ and $\ell_1$ bounds: $\|\chi^\calA\|_1\leq \lceil 1.7m \rceil $ and $\|\chi^\calA\|_\infty \leq 3 \log m$. This completes the proof. 
\end{proof}
\section{Hardness of Minimizing Assignment Multiplicity}
\label{sec:reduction}
This section establishes the computational hardness of selecting, for the agents $i \in [n]$, bundles $A_i$ from given MMS-inducing partitions $\calM^i = (M^i_1, \dots, M^i_n)$ with the objective of minimizing $\ellinfty{\chi^\calA}$, for the resulting multi-allocation $\calA=(A_1,\ldots, A_n)$. 

\begin{theorem}
\label{theorem:NPHard}
Given partitions $\calM^i = (M^i_1, \dots, M^i_n)$ for the agents $i \in [n]$, it is {\rm NP}-hard to decide whether there exists a multi-allocation $\calA = (A_1, \ldots A_n)$ with the properties that: (i) For each $i \in [n]$, the bundle $A_i = M^i_j$, for some index $j \in [n]$, and (ii) Every good is allocated to at most one agent (i.e., $A_i \cap A_j = \emptyset$ for all $i \neq j$).
\end{theorem}
\begin{proof}
   We present a reduction from the maximum independent set problem. Recall that in this {\rm NP}-hard problem we are given a graph $G = (V, E)$, along with  an integer $k \in \mathbb{Z}$, and the goal is to decide if there exists an independent set $I \subseteq V$ of size at least $k$. We first describe, the construction of the multi-allocation selection problem from any given maximum independent set instance with graph $G= (V, E)$ and threshold $k \in \mathbb{Z}_+$.    
   
We set the number of agents $n = |V|+1$ and the number of goods $m = |V|\binom{k}{2} + |E| \binom{k}{2} + |V|$. Further, for each index $i \in [k]$ and each vertex $u \in V$, we create a tuple $(i, u)$; there are $|V| k$ such tuples. Based on these tuples, we define a set family $\mathcal{F}$ that consists of size-$2$ sets either of the form $\{(i, u), (j, v)\}$, for each $(u, v) \in E$, or $\{(i, u), (j, u)\}$, for each $u \in V$. Formally, 
\begin{align*}
\mathcal{F} = \Big\{ \  \{(i, u), (j, u)\} \mid  i \neq j \text{ and } u \in V \Big\}~\bigcup~\Big\{ \ \{ (i, u), (j, v) \} \mid i \neq j \text{ and } (u, v) \in E \Big\}
\end{align*}
Note that $|\mathcal{F}| = |V|\binom{k}{2} + |E| \binom{k}{2}$.

Recall that the number of agents $n = |V| + 1$. We associate the first $|V|$ agents with the vertices of $G$ and the last agent will be indexed as $n$. We consider $m = |\mathcal{F}|+|V|$ goods. In particular, we include one good for each set $S \in \mathcal{F}$, these goods will be denoted by $\{g_S \mid S \in \mathcal{F}\}$. In addition, we introduce one good for each vertex in $G$ and denote them as $\{h_1, h_2, \ldots h_{|V|}\}$. 

Next, we define the partitions $\calM^i = (M^i_1, M^i_2, \ldots M^i_{n})$ for the first $k$ agents, i.e., for $ 1 \leq i \leq k$: 
\begin{align*}
M^i_u  & = \{g_S \mid S \in \mathcal{F} , (i, u) \in S\} \quad \text{ for $1 \leq u \leq |V|$, and } \\ 
M^i_n &  = M^i_{|V|+1}  = [m] \setminus \left( \cup_{u=1}^{|V|} \ M^i_u \right).
\end{align*}
For the remaining $n-k = |V|+1-k$ agents---i.e., for $k+1 \leq j \leq n$---we set the partitions $\calM^j = \{M^j_1, M^j_2, \ldots M^j_{n}\}$ as follows:

\begin{align*}
M^j_u  & = \{ h_u \} \quad \text{ for $1 \leq u \leq |V|$, and } \\  
M^j_n & = M^j_{|V|+1}  =  \{g_S \mid S \in \mathcal{F}\}.
\end{align*}
This completes the construction. We now prove the equivalence.

\paragraph{Forward Direction.} Suppose the given instance of maximum independent set problem is a {\rm Yes} instance. That is, there is an independent set $I \subseteq V$ of size at least $k$ in the graph $G$. Then, we consider the following multi-allocation $\calA$ in the constructed instance. For each agent $i \in [k]$, set $A_i = M^i_u$ where $u$ is the $i$th vertex in $I$. For the remaining agents $k+1 \leq j \leq n$, we set $A_j = M^j_{j-k}$.
    
We will show that under $\calA$ every good is assigned to at most one agent, i.e., the bundles $A_i$ are pairwise disjoint. First, notice that the last $n-k$ agents get distinct singleton goods from $\{h_1, h_2, \ldots h_{|V|}\}$ and all the first $k$ agents get goods only from $\{g_S \mid S \in \mathcal{F}\}$. Hence, the bundles $A_j$, for $k+1 \leq j \leq n$, are disjoint. Further, consider any pair of agents $i$ and $i'$ such that $1 \leq i \neq i' \leq k$. Suppose $A_i = M^i_u$ and $A_{i'} = M^{i'}_v$. Note that, by construction, $u, v \in I$ and $u \neq v$. Assume, towards a contradiction, that $A_i \cap A_{i'} \neq \emptyset$, i.e., $g_S \in M^i_u \cap M^{i'}_v$ for some good $g_S$. This containment implies $S = \{(i, u), (i', v)\}$. Such a set $S$ was included in $\calF$ only if $(u, v) \in E$. However, this contradicts the fact that $u$ and $v$ belong to the independent set $I$. Therefore, by way of contradiction, we get that $A_i \cap A_{i'} = \emptyset$. That is, the bundles $A_i$ are pairwise disjoint and, hence, as desired, every good is assigned to at most one agent under $\calA$. 

\paragraph{Reverse Direction.} Suppose the constructed instance admits a multi-allocation $\calA$ in which no item is assigned to more than one agent, i.e., the bundles in $\calA$ are disjoint. We will show that in such a case $G$ admits an independent set of size $k$. First, we make the following claim.

\begin{claim}
\label{claim:notlastbundle}
In the multi-allocation $\calA$, none of the first $k$ agents $i \in [k]$ receive the last subset, $M^i_n$, from their partition $\calM^i$, i.e., $A_i \neq M^i_n$ for all $i \in [k]$. 
\end{claim}
\begin{proof}
Towards a contradiction, assume that for some agent $i \in [k]$, the bundle $A_i = M^i_n$. Then, the last $(n-k) > 1$ agents $j \in \{k+1,\ldots, n\}$ can not receive  any of their first $|V|$ subsets in $\calM^j$. This follows from the fact that $M^i_n$ contains all the goods $\{h_1, \ldots, h_{|V|} \}$ and, hence, $M^i_n$ intersects with $M^j_u$ for all $j \in \{k+1,\ldots, n\}$ and $1 \leq u \leq |V|$. This leaves bundles $M^j_n$ for the agents $j \in \{k+1,\ldots, n\}$. However, these bundles intersect each other as well. Hence, by way of contradiction, we get that $A_i \neq M^i_n$ for all $i \in [k]$. 
\end{proof}

By \Cref{claim:notlastbundle}, we have that under the allocation $\calA$, each of the first $k$ agents $i \in [k]$ must receive one of their first $|V|$ subsets from their partitions $\calM^i$. Suppose $A_i = M^i_{u_i}$ for index $u_i \in V$.  
\begin{claim}\label{claim:indset}
The vertices $\{u_1, u_2, \ldots u_k\}$ form an independent set in the graph $G$.
\end{claim}
\begin{proof} First, note that $u_i \neq u_j$, for any pair $i \neq j \in [k]$. Otherwise, if $u_i = u_j = u$, then for the set $S = \{ (i, u), (j, u)\} \in \calF$, we have $g_S \in M^i_u = A_i$ and $g_S \in M^j_u = A_j$. Therefore, $g_S \in A_i \cap A_j$, which contradicts the fact that the bundles in $\calA$ are disjoint.
        
 Now, suppose $(u_i, u_j) \in E$ for some $i \neq j \in [k]$. Then, for the set $S = \{(i, u_i), (j, u_j)\} \in \calF$, consider the good $g_S$. We have $g_S \in M^i_{u_i} \cap M^j_{u_j}$, i.e., $g_S \in A_i \cap A_j$. This containment contradicts the fact that the bundles in $\calA$ are disjoint. Therefore, we have that $(u_i, u_j) \notin E$ for all pairs $i \neq j \in [k]$. Therefore, $\{u_1, u_2, \ldots u_k\}$ is an independent set in the graph $G$.  
\end{proof}
This concludes the reverse direction of the reduction and completes the proof of the theorem.
\end{proof}



\end{document}