\section{Introduction}
\label{sec:introduction}

During the past three decades, many software obfuscation, tamperproofing, and anti-analysis techniques have been presented that aim to hinder reverse engineering and software tampering to prevent unauthorized use~\cite{collbergbook,survey2016,desutter2024evaluation}.

The progress in such software protections (SPs) has triggered an arms race in which many new techniques for deobfuscation, tampering, and analysis have been proposed. When used offensively, these techniques are collectively called \emph{man-at-the-end} (MATE) attacks, because the adversaries using them have full control over the systems on which they run, analyze, and alter the software under attack. Their tools include emulators, debuggers, disassemblers, custom operating systems, fuzzers, symbolic execution, sandboxes, and instrumentation among a wide range of static and dynamic analysis and tampering tools.
These give them white-box access to the software internals, including their internal execution state.

In the MATE attack model, and with the current state-of-the-art in practical SP, complete prevention of attacks is impossible. With enough time and resources to exploit their white-box access, adversaries can always succeed. The goal of MATE SP is hence to make the adversary's return on investment negative, by delaying the identification of attacks and by limiting their exploitation.

These goals are much fuzzier than the goals in some other security domains such as cryptography, which builds on precisely defined mathematical foundations to protect against man-in-the-middle attacks.\footnote{While cryptographic techniques and well-defined, cryptographically-sound concepts have also been developed in the domain of obfuscation, the results are still far from ready for widespread general use in practice. They are therefore outside the scope of this work, which focuses on practical, general-purpose software obfuscation.} Because of this fuzziness, and because MATE\linebreak attackers have so many alternative attack strategies and techniques at their disposal, it is difficult to measure the strength of SPs and to make founded statements about them.

As observed at the 2019 Dagstuhl Seminar on SP Decision Support and Evaluation Methodologies~\cite{Dagstuhl} and in a recent survey~\cite{desutter2024evaluation}, the field of MATE SP research suffers from a lack of standardized evaluation methodologies and practices. The survey observed that 
\begin{itemize}
    \item 25\%~$\hat{=}$~125/495 of the papers with SP implementations report no strength measurements at all.
    \item 54\%~$\hat{=}$~133/248 of the goodware obfuscation papers that deploy SPs on samples do not evaluate those SPs' strength by assessing how analysis methods fare against them.
    \item Obfuscation research often does not assess resilience.
    \item The costs of applying SPs are measured much more frequently than their strengths.
    \item Potency is measured much more often than resilience.
\end{itemize}

Although there has long been a common understanding that the relevant criteria for evaluating obfuscation techniques include potency, resilience, stealth, and cost~\cite{collberg1997taxonomy,collberg1998manufacturing}, there is no universally agreed and applicable method to define, qualify, or quantify them. 
Instead, non-compatible definitions and interpretations have been proposed and used in the literature. We believe that this lack of consistency contributes to today's unsatisfactory evaluation practices in MATE SP research and that the time has come to revisit the essential quality criteria of software obfuscation techniques.

In this paper, our aim is to study how the concepts of potency, resilience, and stealth have been considered in past research, how they have evolved over time, and where confusion about them may have arisen that may have been a contributing factor to the issues mentioned above. For doing so, we will pinpoint issues with the different definitions and interpretations of those concepts that we observed in past research, and we will sketch a new framework that can help to better frame those concepts, and provide advice as to how to consider them in future research. 

Sections 2--5 discuss four different approaches to define and interpret the concepts of potency, resilience, and stealth, and formulate critiques on them. Section 6 proposes an alternative framework of evaluation criteria. We provide further discussion in Section 7. Finally, Section 8 draws conclusions.

