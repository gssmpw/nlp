\documentclass[sigconf,nonacm]{acmart}

\usepackage{tikz}
\usepackage[nameinlink,capitalise,noabbrev]{cleveref}
\usepackage{hyperref}
\usepackage{microtype}
\usepackage{enumerate}
\usepackage{enumitem}
\usepackage{todonotes}
\usepackage{xfrac}
\usepackage{pifont}
\usepackage{adjustbox}

\newcommand{\cmark}{\ding{51}}%
\newcommand{\xmark}{\ding{55}}%

\newcounter{cccnt}
\newcounter{bdscnt}
\newcounter{rgcnt}
\newcounter{bccnt}

\newcommand{\cc}[1]{\refstepcounter{cccnt}
	\textcolor{purple}{\textbf{CC: [\thecccnt]:} #1}}
\newcommand{\bds}[1]{\refstepcounter{bdscnt}
	\textcolor{pink}{\textbf{BDS: [\thebdscnt]:} #1}}
\newcommand{\rg}[1]{\refstepcounter{rgcnt}
	\textcolor{red}{\textbf{RG: [\thetfcnt]:} #1}}
\newcommand{\bc}[1]{\refstepcounter{bccnt}
	\textcolor{blue}{\textbf{BC [\thebccnt]:} #1}}


\newcommand*\circledw[1]{\tikz[baseline=(char.base)]{
		\node[shape=circle,draw,inner sep=.75pt, line width=.7pt] (char) {\scriptsize #1};}}
\newcommand*\dcircledw[1]{\tikz[baseline=(char.base)]{
		\node[shape=circle,draw,dashed,dash pattern = on 1.5pt off .75pt, inner sep=.75pt, line width=.7pt] (char) {\scriptsize #1};}}

\newcommand*\recdw[1]{\tikz[baseline=(char.base)]{
		\node[shape=rectangle,draw,inner sep=0.18em, line width=.7pt] (char) {\scriptsize #1};}}
\newcommand*\drecdw[1]{\tikz[baseline=(char.base)]{
		\node[shape=rectangle,draw,dashed,dash pattern = on 1.5pt off .75pt, inner sep=0.18em, line width=.7pt] (char) {\scriptsize #1};}}


\widowpenalty=10000
\clubpenalty=10000
\brokenpenalty=10000

\renewenvironment{quote}{%
  \list{}{%
    \leftmargin0.5cm   % this is the adjusting screw
    \rightmargin0.05cm
  }
  \item\relax
}
{\endlist}

\hyphenation{Rev-Eng}

%%
%% \BibTeX command to typeset BibTeX logo in the docs
\AtBeginDocument{%
  \providecommand\BibTeX{{%
    Bib\TeX}}}

%% Rights management information.  This information is sent to you
%% when you complete the rights form.  These commands have SAMPLE
%% values in them; it is your responsibility as an author to replace
%% the commands and values with those provided to you when you
%% complete the rights form.
%% These commands are for a PROCEEDINGS abstract or paper.
%% TODO from rights form?


%\copyrightyear{2024}
%\acmYear{2024}
%\setcopyright{acmlicensed}\acmConference[shorthand]{full booktitle}{date}{location}
%\acmBooktitle{full booktitle (shorthand), date, location}
%\acmPrice{TODO}
%\acmDOI{TODO}
%\acmISBN{TODO}

%%
%% Submission ID.
%% Use this when submitting an article to a sponsored event. You'll
%% receive a unique submission ID from the organizers
%% of the event, and this ID should be used as the parameter to this command.
%%\acmSubmissionID{123-A56-BU3}

%%
%% For managing citations, it is recommended to use bibliography
%% files in BibTeX format.
%%
%% You can then either use BibTeX with the ACM-Reference-Format style,
%% or BibLaTeX with the acmnumeric or acmauthoryear sytles, that include
%% support for advanced citation of software artefact from the
%% biblatex-software package, also separately available on CTAN.
%%
%% Look at the sample-*-biblatex.tex files for templates showcasing
%% the biblatex styles.
%%

%%
%% The majority of ACM publications use numbered citations and
%% references.  The command \citestyle{authoryear} switches to the
%% "author year" style.
%%
%% If you are preparing content for an event
%% sponsored by ACM SIGGRAPH, you must use the "author year" style of
%% citations and references.
%% Uncommenting
%% the next command will enable that style.
%%\citestyle{acmauthoryear}


%%
%% end of the preamble, start of the body of the document source.
\begin{document}

%%
%% The "title" command has an optional parameter,
%% allowing the author to define a "short title" to be used in page headers.

\title{A New Framework of Software Obfuscation Evaluation Criteria}

%\author{Anonymous}
\author{Bjorn De Sutter}
\email{bjorn.desutter@ugent.be}
\orcid{0000-0003-0317-2089}
\affiliation{
\institution{Ghent University}
\city{Ghent}
\country{Belgium}
}

%\author{Anyone willing to contribute}

%%
%% By default, the full list of authors will be used in the page
%% headers. Often, this list is too long, and will overlap
%% other information printed in the page headers. This command allows
%% the author to define a more concise list
%% of authors' names for this purpose.
% \renewcommand{\shortauthors}{De Ghein et al.}

%%
%% The abstract is a short summary of the work to be presented in the
%% article.
\begin{abstract}
In the domain of practical software protection against man-at-the-end attacks such as software reverse engineering and tampering, much of the scientific literature is plagued by the use of subpar methods to evaluate the protections' strength and even by the absence of such evaluations. Several criteria have been proposed in the past to assess the strength of protections, such as potency, resilience, stealth, and cost. We analyze their evolving definitions and uses. We formulate a number of critiques, from which we conclude that the existing definitions are unsatisfactory and need to be revised. We present a new framework of software protection evaluation criteria: relevance, effectiveness (or efficacy), robustness, concealment, stubbornness, sensitivity, predictability, and cost.
\end{abstract}

%%
%% The code below is generated by the tool at http://dl.acm.org/ccs.cfm.
%% Please copy and paste the code instead of the example below.
%%
\begin{CCSXML}
<ccs2012>
<concept>
<concept_id>10002978.10003022.10003465</concept_id>
<concept_desc>Security and privacy~Software reverse engineering</concept_desc>
<concept_significance>500</concept_significance>
</concept>
</ccs2012>
\end{CCSXML}

\ccsdesc[500]{Security and privacy~Software reverse engineering}

%%
%% Keywords. The author(s) should pick words that accurately describe
%% the work being presented. Separate the keywords with commas.
\keywords{Potency, resilience, stealth, obfuscation, attacks, program analysis}
%% A "teaser" image appears between the author and affiliation
%% information and the body of the document, and typically spans the
%% page.
% \begin{teaserfigure}
%   \includegraphics[width=\textwidth]{sampleteaser}
%   \caption{Seattle Mariners at Spring Training, 2010.}
%   \Description{Enjoying the baseball game from the third-base
%   seats. Ichiro Suzuki preparing to bat.}
%   \label{fig:teaser}
% \end{teaserfigure}

%%
%% This command processes the author and affiliation and title
%% information and builds the first part of the formatted document.
\maketitle


% Alphabetical:
% \belfius
% \payconiq
% \argenta
% \axa
% \kbc
% \ING
% \beobank
% \keytrade
% \bpost
% \crelan
% newcolumn
% \bunq
% \knab
% \vdk
% \fortis
% \abnamro
% \delen
% \breda
% \ASN
% \rabobank
% \SNS

\newcommand{\defeq}{\overset{\mathrm{def}}{=}}

\section{Introduction}
\label{sec:introduction}
The business processes of organizations are experiencing ever-increasing complexity due to the large amount of data, high number of users, and high-tech devices involved \cite{martin2021pmopportunitieschallenges, beerepoot2023biggestbpmproblems}. This complexity may cause business processes to deviate from normal control flow due to unforeseen and disruptive anomalies \cite{adams2023proceddsriftdetection}. These control-flow anomalies manifest as unknown, skipped, and wrongly-ordered activities in the traces of event logs monitored from the execution of business processes \cite{ko2023adsystematicreview}. For the sake of clarity, let us consider an illustrative example of such anomalies. Figure \ref{FP_ANOMALIES} shows a so-called event log footprint, which captures the control flow relations of four activities of a hypothetical event log. In particular, this footprint captures the control-flow relations between activities \texttt{a}, \texttt{b}, \texttt{c} and \texttt{d}. These are the causal ($\rightarrow$) relation, concurrent ($\parallel$) relation, and other ($\#$) relations such as exclusivity or non-local dependency \cite{aalst2022pmhandbook}. In addition, on the right are six traces, of which five exhibit skipped, wrongly-ordered and unknown control-flow anomalies. For example, $\langle$\texttt{a b d}$\rangle$ has a skipped activity, which is \texttt{c}. Because of this skipped activity, the control-flow relation \texttt{b}$\,\#\,$\texttt{d} is violated, since \texttt{d} directly follows \texttt{b} in the anomalous trace.
\begin{figure}[!t]
\centering
\includegraphics[width=0.9\columnwidth]{images/FP_ANOMALIES.png}
\caption{An example event log footprint with six traces, of which five exhibit control-flow anomalies.}
\label{FP_ANOMALIES}
\end{figure}

\subsection{Control-flow anomaly detection}
Control-flow anomaly detection techniques aim to characterize the normal control flow from event logs and verify whether these deviations occur in new event logs \cite{ko2023adsystematicreview}. To develop control-flow anomaly detection techniques, \revision{process mining} has seen widespread adoption owing to process discovery and \revision{conformance checking}. On the one hand, process discovery is a set of algorithms that encode control-flow relations as a set of model elements and constraints according to a given modeling formalism \cite{aalst2022pmhandbook}; hereafter, we refer to the Petri net, a widespread modeling formalism. On the other hand, \revision{conformance checking} is an explainable set of algorithms that allows linking any deviations with the reference Petri net and providing the fitness measure, namely a measure of how much the Petri net fits the new event log \cite{aalst2022pmhandbook}. Many control-flow anomaly detection techniques based on \revision{conformance checking} (hereafter, \revision{conformance checking}-based techniques) use the fitness measure to determine whether an event log is anomalous \cite{bezerra2009pmad, bezerra2013adlogspais, myers2018icsadpm, pecchia2020applicationfailuresanalysispm}. 

The scientific literature also includes many \revision{conformance checking}-independent techniques for control-flow anomaly detection that combine specific types of trace encodings with machine/deep learning \cite{ko2023adsystematicreview, tavares2023pmtraceencoding}. Whereas these techniques are very effective, their explainability is challenging due to both the type of trace encoding employed and the machine/deep learning model used \cite{rawal2022trustworthyaiadvances,li2023explainablead}. Hence, in the following, we focus on the shortcomings of \revision{conformance checking}-based techniques to investigate whether it is possible to support the development of competitive control-flow anomaly detection techniques while maintaining the explainable nature of \revision{conformance checking}.
\begin{figure}[!t]
\centering
\includegraphics[width=\columnwidth]{images/HIGH_LEVEL_VIEW.png}
\caption{A high-level view of the proposed framework for combining \revision{process mining}-based feature extraction with dimensionality reduction for control-flow anomaly detection.}
\label{HIGH_LEVEL_VIEW}
\end{figure}

\subsection{Shortcomings of \revision{conformance checking}-based techniques}
Unfortunately, the detection effectiveness of \revision{conformance checking}-based techniques is affected by noisy data and low-quality Petri nets, which may be due to human errors in the modeling process or representational bias of process discovery algorithms \cite{bezerra2013adlogspais, pecchia2020applicationfailuresanalysispm, aalst2016pm}. Specifically, on the one hand, noisy data may introduce infrequent and deceptive control-flow relations that may result in inconsistent fitness measures, whereas, on the other hand, checking event logs against a low-quality Petri net could lead to an unreliable distribution of fitness measures. Nonetheless, such Petri nets can still be used as references to obtain insightful information for \revision{process mining}-based feature extraction, supporting the development of competitive and explainable \revision{conformance checking}-based techniques for control-flow anomaly detection despite the problems above. For example, a few works outline that token-based \revision{conformance checking} can be used for \revision{process mining}-based feature extraction to build tabular data and develop effective \revision{conformance checking}-based techniques for control-flow anomaly detection \cite{singh2022lapmsh, debenedictis2023dtadiiot}. However, to the best of our knowledge, the scientific literature lacks a structured proposal for \revision{process mining}-based feature extraction using the state-of-the-art \revision{conformance checking} variant, namely alignment-based \revision{conformance checking}.

\subsection{Contributions}
We propose a novel \revision{process mining}-based feature extraction approach with alignment-based \revision{conformance checking}. This variant aligns the deviating control flow with a reference Petri net; the resulting alignment can be inspected to extract additional statistics such as the number of times a given activity caused mismatches \cite{aalst2022pmhandbook}. We integrate this approach into a flexible and explainable framework for developing techniques for control-flow anomaly detection. The framework combines \revision{process mining}-based feature extraction and dimensionality reduction to handle high-dimensional feature sets, achieve detection effectiveness, and support explainability. Notably, in addition to our proposed \revision{process mining}-based feature extraction approach, the framework allows employing other approaches, enabling a fair comparison of multiple \revision{conformance checking}-based and \revision{conformance checking}-independent techniques for control-flow anomaly detection. Figure \ref{HIGH_LEVEL_VIEW} shows a high-level view of the framework. Business processes are monitored, and event logs obtained from the database of information systems. Subsequently, \revision{process mining}-based feature extraction is applied to these event logs and tabular data input to dimensionality reduction to identify control-flow anomalies. We apply several \revision{conformance checking}-based and \revision{conformance checking}-independent framework techniques to publicly available datasets, simulated data of a case study from railways, and real-world data of a case study from healthcare. We show that the framework techniques implementing our approach outperform the baseline \revision{conformance checking}-based techniques while maintaining the explainable nature of \revision{conformance checking}.

In summary, the contributions of this paper are as follows.
\begin{itemize}
    \item{
        A novel \revision{process mining}-based feature extraction approach to support the development of competitive and explainable \revision{conformance checking}-based techniques for control-flow anomaly detection.
    }
    \item{
        A flexible and explainable framework for developing techniques for control-flow anomaly detection using \revision{process mining}-based feature extraction and dimensionality reduction.
    }
    \item{
        Application to synthetic and real-world datasets of several \revision{conformance checking}-based and \revision{conformance checking}-independent framework techniques, evaluating their detection effectiveness and explainability.
    }
\end{itemize}

The rest of the paper is organized as follows.
\begin{itemize}
    \item Section \ref{sec:related_work} reviews the existing techniques for control-flow anomaly detection, categorizing them into \revision{conformance checking}-based and \revision{conformance checking}-independent techniques.
    \item Section \ref{sec:abccfe} provides the preliminaries of \revision{process mining} to establish the notation used throughout the paper, and delves into the details of the proposed \revision{process mining}-based feature extraction approach with alignment-based \revision{conformance checking}.
    \item Section \ref{sec:framework} describes the framework for developing \revision{conformance checking}-based and \revision{conformance checking}-independent techniques for control-flow anomaly detection that combine \revision{process mining}-based feature extraction and dimensionality reduction.
    \item Section \ref{sec:evaluation} presents the experiments conducted with multiple framework and baseline techniques using data from publicly available datasets and case studies.
    \item Section \ref{sec:conclusions} draws the conclusions and presents future work.
\end{itemize}

\section{Collberg, Thomborson, and Low}

In 1997 and 1998 Collberg, Thomborson, and Low (CTL) proposed three criteria to evaluate obfuscating transformations~\cite{collberg1997taxonomy,collberg1998manufacturing}.

\subsection{Definitions}

\subsubsection{Potency}
In 1997, CTL~\cite{collberg1997taxonomy} introduced the term potency to denote the degree to which an obfuscation confuses a \emph{human} reader trying to understand a given program, i.e., to quantify how the obfuscated program is more obscure, complex, or unreadable than the original program. They defined the potency that is obtained on a given program as
the ratio between a complexity metric computed on that program after and before the obfuscation~\cite{collberg1997taxonomy}.

\if false
follows~\cite{collberg1997taxonomy}:
\begin{quote}
  Let $\mathcal{T}$ be a behavior-conserving transformation such that $P \overset{\mathcal{T}}{\rightarrow} P'$ transforms a source program $P$ into a target program $P'$. Let $E(P)$ be the complexity of P, as defined by one of the metrics [...].

  $\mathcal{T}_{\text{pot}}(P)$, the \emph{potency} of $\mathcal{T}$ with respect to a program $P$, is a measure of the extent to which $\mathcal{T}$ changes the complexity of $P$. It is defined as
  $$\mathcal{T}_{\text{pot}}(P) \defeq E(P')/E(P)-1.$$ 
$\mathcal{T}$ is a \emph{potent obfuscating transformation} if $\mathcal{T}(P)>0$.
\end{quote}
\fi

They proposed considering software complexity metrics such as program length~\cite{halstead}, cyclomatic complexity~\cite{mccabe1976}, nesting complexity~\cite{harrison1981}, data flow complexity~\cite{oviedo80}, fan-in/out complexity~\cite{flow_metrics}, data structure complexity~\cite{munson1993}, and OO Metric~\cite{OO-metric}. CTL then derived and suggested that obfuscations should aim to directly impact these metrics, by increasing program size and introducing new classes and methods, by introducing new predicates and increasing nesting levels of conditional and looping constructs, etc.~\cite{collberg1997taxonomy}.

In line with the suggestion, various other complexity metrics have been used in SP research since then, such as QMOOD understandability~\cite{Foket14,qmood}. Various more advanced forms of complexity metrics have been proposed, some of which are specific for evaluating obfuscations, e.g., based on entropy~\cite{entropy_metric}, others more general, such as the cost of mental simulation~\cite{mental_metrics}.
Several authors have later revisited the idea of combining many different metrics~\cite{anckaert,D4.06}.

\subsubsection{Resilience}
\label{sec:resilience}
CTL observed that it is trivial to increase potency in ways that will not confuse a human reader, e.g., by injecting code that can immediately be identified as dead. They then presented the additional quality criterion of \emph{resilience} to differentiate useless transformations and useless impacts on complexity metrics from useful ones. In their proposal~\cite{collberg1997taxonomy}, resilience ``measures how well a transformation holds under attack from an automatic deobfuscator.'' 

It does so in terms of two measures. The \emph{programmer effort} is ``the amount of time required to construct an automatic deobfuscator that is able to effectively reduce [the obfuscation's] potency.'' The \emph{deobfuscator effort} is ``the execution time and space required by such an automatic deobfuscator to effectively reduce [the obfuscation's] potency.'' They scaled programmer effort in terms of the scope of the analyses and transformations underlying the deobfuscator, which can be local, global, interprocedural, or interprocess. The deobfuscator effort is based on the computational complexity of the underlying analysis: polynomial or exponential. Based on these properties, resilience is measured on a scale of \emph{trivial} to \emph{full}, plus the level \emph{one-way} that was based on the then realistic assumption that some information about the original program can never be recovered from the obfuscated program, such as identifiers names (of variables, classes, functions, etc.) and code layout.

\subsubsection{Stealth}
Although resilient transformations may not be susceptible to attacks by automated deobfuscation, they may still be susceptible to attacks by humans. In 1998, CTL~\cite{collberg1998manufacturing} hence added the \emph{stealth} criterion, which denoted ``How well does obfuscated code blend in with the original code?'' They initially did not formalize this criterion beyond stating that obfuscated code should resemble original code as much as possible, pointing out that stealth obviously is a highly context-sensitive metric. For example, they argued that graph-based opaque predicates will be stealthy in many Java programs as those often involve pointer-rich data structures. % (that is, data structures implemented with pointers in languages with pointers, such as C).

\subsubsection{Cost}
Finally, CTL argued that the impact of obfuscations on execution time and space should be assessed~\cite{collberg1997taxonomy}. Like stealth, \emph{cost} is a context-sensitive metric: injecting code into an inner loop or at the topmost program level will have a radically different impact. 

\subsection{Critique}
\label{sec:9798-critiques}
CTL's definitions of potency, resilience, and stealth do not satisfy. 

\subsubsection{Complexity Metrics Unfit for Obfuscated Code}
\label{sec:unfit_complexity_metrics}
The validity of software complexity metrics for code comprehension is questionable. Feitelson argues that most metrics are based on intuition instead of systematic empirical validation and that they fail to adequately quantify comprehension difficulty~\cite{feitelson}. The most popular metrics of code length and cyclometic complexity do not capture that the code reader's prior experience and background knowledge also contribute significantly~\cite{halstead,mccabe1976,code_metrics}. Other metrics such as cognitive functional size~\cite{cognitive_functional_size} do take into account how the human brain perceives different code structures and patterns, but only to a limited extent. If complexity metrics to model comprehension difficulty of regular software are already contested, expanding their use to assess the potency on obfuscated code is even more controversial. With their observation that the proposed metrics can trivially be increased with transformations that lack any utility as obfuscations, CTL themselves admit that complexity metrics cannot be reused as is for assessing the qualities of obfuscated code. 

When Foket et al.~\cite{Foket14} used validated complexity metrics for object-oriented software, they had to admit that their use was problematic. Additional qualitative analysis was needed to ensure that the quantitative metrics results were correctly interpreted and to avoid drawing invalid conclusions. 

Feitelson also assessed the importance of meaningful variable names for program comprehension~\cite{feitelson}, and found that their impact on comprehension difficulty varies from program to program. In obfuscated code, one should not expect to find meaningful variable names, at least not without using machine learning techniques to reconstruct them~\cite{2017_recovering_clear_natural_identifiers_from_obfuscated_js_names,banerjee2021variablerecoverydecompiledbinary}. This difference alone puts in doubt whether software complexity metrics developed on regular software can be reused as is for obfuscated code.  


\subsubsection{Unclear How to Compute Complexity Metrics}
\label{sec:unclear_computations}
The definition of potency does not specify which software fragments to measure.  

First, the definition is vague about the program representation on which to compute the complexity metrics. Should one use ground-truth representations from the forward engineering process (e.g., source code, IR code used in a compiler, assembly code, binary code with full symbol and relocation information) or representations built during the reverse engineering process (e.g., IR code built by a disassembler, IR obtained by code lifting, or source code obtained after decompilation)? Which representations are more relevant: the ones being transformed to deploy the obfuscations, or the ones on which adversaries might get their hands?

For managed languages such as Java, this question might not be that important. The semantic gap between the Java source code and the distributed bytecode is relatively small~\cite{batchelder2007obfuscating} and a representation to compute metrics on (such as a call graph and method CFGs) can be built mechanistically from the distributed bytecode. 
For stripped binaries, the reconstruction of a suitable representation is much more complex. For example, when control flow and design obfuscations such as those by Van den Broeck et al.~\cite{jens21} are deployed, the difference between what the obfuscator knows to be the functions and what the disassembler thinks to be the possible functions is huge. Each disassembler, possibly even each version, configuration, and customization by means of scripts or plug-ins will yield different CFGs and different functions. In other words, there is no one representation on which attackers can get their hands. Even if all disassemblers would behave the same, do you compute the control complexity metrics on the assembly code, on the low-level IR lifted from the assembly, on the high-level IR built on top, or on the decompiled source code? These are open questions. Some work even skips disassembly and decompilation entirely, e.g., by interpreting executable files as bitstreams and then applying image processing machine learning techniques on those bitstreams to classify malware~\cite{ni2018malware,marastoni2021data} and to classify used obfuscations~\cite{secrypt21}.

Secondly, on what part of the software should the metrics be computed? Although adversaries might initially not know which part to study, they have many ways to zoom in on the most relevant parts, thus only requiring deeper study of parts of the program. But which parts exactly? And what about the parts they prune from their search space? Obfuscation can be useful in hindering that pruning~\cite{coppens2013feedback,reganoL2P}, rather than in hindering the understanding of the code that actually embeds the assets the adversaries are after. On which parts of the code should the metrics then be computed?

\subsubsection{Fuzzy and Confusing Boundary between Automated and Human Analysis}
\label{sec:fuzzy_boundary}
The definitions by CTL attempt to make a clear separation between automated analysis and deobfuscation on the one hand and human analysis on the other. We think this is not warranted. In practice, reverse engineers often make use of interactive tools that mix automated and human analysis, such as interactive disassemblers and decompilers of which the built-in heuristics can be overridden on demand by their users. Moreover, they always perform their manual tasks on data produced by automated tools: No one starts studying the bits of a binary executable manually. There is ample evidence for this in the literature~\cite{emse2019,Sutherland2006,Votipka2019,practice_malware,desutter2024evaluation}.

Moreover, by binding automation to deobfuscation in the term resilience and by binding human activities to confusion in potency, CTL beg the question of where to place automated attacks that do not deobfuscate code, such as automatic secret extraction attacks~\cite{khunt++,banescu15}. Banescu et al.\ use the term resilience to denote the impact of SPs on the time that symbolic execution engines need to extract a secret~\cite{banescu15,2017_predicting_the_resilience_of_obfuscated_code_against_symbolic_execution_attacks_via_machine_learning}, even though these engines make no attempt at producing deobfuscated code. This is a clear example of how hard it is to use CTL's terms resilience and potency consistently. 

\subsubsection{Tool Availability and Sharing over Time}
\label{sec:tool_availability}
The dependence of resilience on programmer effort neglects that once someone publishes an analysis tool or deobfuscator, everyone can reuse it. This certainly has an impact on the practical value of obfuscations. For example, in experiments with professional pen testers~\cite{emse2019}, it mattered a lot to them that QEMU was not able to correctly simulate binaries with anti-debugging~\cite{abrath2016tightly}, and that valgrind had to be adapted to support trace sizes beyond 2 GiB. 

Tool availability raises an interesting question for academics that can be illustrated with a well-known example: Intel's Pin instrumentation tool only works on the x86 architectures, with no similar tool being available for ARM these days. Then does that mean that dynamic analysis of ARM code is fundamentally harder than dynamic analysis of x86? For an academic studying reverse engineering, that seems intuitively wrong, even though, for a practicing reverse engineer, it may be actually true.

\subsubsection{Narrow Definition of Programmer Effort}
Using only one dimension to estimate the effort required to program an analysis is flawed. In addition to the scope, at least the sensitivity~\cite{path-sensitive} and polyvariance~\cite{polyvariance} of the analysis should also be considered.  However, it is not clear whether more precise analyses are always more difficult to implement than less precise ones. It is hence not clear how to rank different forms of precision in terms of programmer effort. Then again, if a framework were available to instantiate various forms (i.e., different scopes and different sensitivities) of a custom analysis with a simple configuration, the programmer effort would no longer be dependent on them. 
   
\subsubsection{Narrow Scope of Deobfuscation}
Resilience was defined in terms of deobfuscation, meaning ``returning the obfuscated code into a form that is as easily understandable as the original source''. Pure deobfuscation is rarely used in practice. Instead, adversaries often try to bypass SPs and work around them~\cite{emse2019}.

Moreover, the local, global, and interprocedural programmer effort levels proposed by CTL~\cite{collberg1997taxonomy} are strongly biased toward static analyses and static program representations such as call graphs and CFGs, as opposed to dynamic representations such as traces that are also used for dynamic deobfuscation~\cite{generic_deobfuscation,mariano24}. Should we hence not use other aspects to rate the effort required to obtain dynamic information with a certain precision? Some dynamic approaches, such as the bit-level tracking by Yadegari et al.~\cite{bitleveltaint}, are clearly more complex to implement and execute than the simpler taint-tracking approaches proposed by Faingnaert et al.~\cite{khunt++} or the alternative from Li et al.~\cite{khunt}, so it seems that a more flexible scheme is necessary to rate deobfuscation techniques.

\subsubsection{Asset and Attack Goal Independence}
The definitions of potency and resilience are asset independent. They qualify and quantify how confusing obfuscations are, in general, for human readers, and how susceptible they are to an adversary's countermeasures. Their formulation does not take into account that adversaries aim to violate specific security requirements on specific assets, such as the confidentiality or integrity of specific code fragments. 

The literature is clear on the fact that for assessing the strength of obfuscations, both the adversary's goal and the deployed software analysis methods need to be considered. Schrittwieser et al.\ distinguish four different goals: finding the location of data, finding the location of program functionality, extraction of code fragments, and understanding the program~\cite{survey2016}. The latter can be interpreted broadly. For example, determining invariants of a program, determining preconditions for the execution of some fragment, or obtaining concise summaries of its functionality can all be considered program understanding goals. Schrittwieser et al.\ also distinguish four classes of software analysis methods that obfuscations might aim to mitigate: pattern matching, automatic static analysis, automatic dynamic analysis, and human analysis. While one could argue that different software metrics should be used depending on the type of assets to be protected and on the considered attack goals and used analysis methods, as has been done in the past~\cite{anckaert}, it is all but clear which complexity metrics to use when.

Furthermore, the proposed code artificiality metric for stealth only is applicable in the face of specific attacks, namely those considering n-grams or related syntactical code properties, but meaningless for other code and data location finding attacks. 

\subsubsection{Layering is Missing in Action}
\label{sec:layering_missing}
Nowhere do CTL's definitions stress the need to evaluate the marginal strength of obfuscations when they are composed and layered on top of each other, despite the fact that layering obfuscations is the best practice~\cite{layering,collbergbook,recipes}.

Because the obfuscation (singular) being evaluated could consist of a composition of multiple transformations (plural), the definitions by CTL support layering and compositions and are technically fine. 
However, to help the community, we do think that the concepts of layering and composability need to be put forward explicitly, and potential pitfalls or consequences deserve to be discussed, if only because layering increases the relevance of other critiques, such as those discussed in Section~\ref{sec:unclear_computations}. 
When only one obfuscation deployed in isolation is being evaluated, the semantic gap between the different representations on which defenders and attackers can compute the relevant metrics might be small enough that we do not need to worry about which representation to compute it. But once obfuscations are composed, the semantic gap can grow quickly and the question of how to compute the metrics becomes much more difficult again. De Sutter et al.\ discuss this in more detail where they observe the lack of layered SP evaluation in research~\cite{desutter2024evaluation}.

\subsubsection{Machine Learning Changed the Nature of the Game}
The concept of one-way resilience no longer applies in this day and age of machine learning as it did in the previous millennium. Even if machine learning techniques offer no guarantee of reconstructing the original identifier names or the original code layout, they have been proven capable of reconstructing meaningful, useful names and layouts that help human reverse engineers~\cite{2017_recovering_clear_natural_identifiers_from_obfuscated_js_names,banerjee2021variablerecoverydecompiledbinary}. However, one-way resilience might still be meaningful when obfuscations rely on one-way functions, such as hash functions~\cite{hash1,hash2}.

\subsubsection{Lack of Sensitivity}
\label{sec:sensitivity_missing}
CTL neglect the sensitivity of potency, resilience, and stealth with respect to features of the program being obfuscated. An example of an obfuscation that can be highly sensitive to (local) program features is the source-level injection of mixed Boolean arithmetic expressions that compute constant values~\cite{mba}. If the variables used in the expressions happen to have constant values, or are uninitialized, compilers can optimize away (part of) the obfuscation.\footnote{%One might argue that if an optimizing compiler can undo an obfuscation, it probably was not a strong one. This is not a valid reasoning, however. 
Source code gives compilers much richer information than adversaries get from a binary. The fact that a compiler can deduce some property hence does not imply an adversary can do so, i.e., that the obfuscation fails to hide the property for an adversary.} Another example is bogus control flow and bogus code inserted by means of, e.g., opaque predicates. The effect on complexity there depends entirely on which data dependencies and which control dependencies happen to be impacted by the bogus code. In general, combining obfuscation with program optimization is known to be a difficult problem~\cite{optimization}.

This sensitivity is an issue for at least two practical reasons. First, if the impact of an obfuscation is sensitive to features of the program being obfuscated, a mechanism is needed to assess that impact every time the obfuscation is considered as a candidate for deployment. Predictive techniques or actual measurements can be used for this. The latter is not really feasible, however, given the large design space that needs to be explored~\cite{Basile23}, and the former has significant limitations today~\cite{reganoMetric}. Secondly, programs evolve over time. So even when users of an obfuscation tool have somehow obtained a suitable composition of obfuscations for their software v1.0, if that composition is sensitive to program features, they will need to reassess its suitability for later versions. Clearly, the user-friendliness of obfuscators and/or their need to include predictive capabilities depends on the sensitivity of the supported obfuscations with respect to program features. 

CTL also neglect the sensitivity of an obfuscation to its configuration parameters. If an obfuscation's impact is sensitive to them, this again impacts the user-friendliness of obfuscators and the need to include prediction mechanisms, because the parameters will need to be optimized. One could argue that if two configurations of some obfuscation have significantly different impacts, then they should be treated as two different obfuscations. Although documenting obfuscations as such may make an obfuscation tool more transparent, it does not make the obfuscation selection and optimization problems any easier. For these reasons, the sensitivity of obfuscations should be considered a prime quality criterion. 

\subsubsection{Learnability Neglected}
\label{sec:learnability_missing}
Defining stealth as the artificiality of transformed code compared to original code neglects that adversaries can learn to single out obfuscated fragments even if they are similar to original code. If an obfuscation is implemented with exactly the same instruction sequence every time it is deployed, it does not matter that this sequence also occurs a few times in the original code: Pattern matchers will quickly achieve 100\% recall.

\section{Nagra and Collberg}
In 2009, Nagra and Collberg (NC) redefined potency and stealth~\cite{collbergbook}.

\subsection{Updated Definitions}
\label{sec:collberg_nagra_def}
NC now defined potency in relation to a program's property that some adversary might try to reveal, and in relation to analyses that can be used to reveal that property. First, they paraphrased the meaning of an \emph{effective obfuscating transformation} as one that ``makes it harder to perform the necessary analyses that reveal the secret property on the obfuscated program than it is on the original.''

Given a specific program, a specific property thereof, and some specific analysis that can reveal that property on the original program, an obfuscation is considered \emph{effective} if the analysis can no longer reveal an equivalent property on the obfuscated program or requires more resources to do so. The obfuscation is considered \emph{ineffective} if the analysis can still reveal an equivalent property using the same amount of resources and \emph{defective} if the analysis can reveal an equivalent property using less resources. 

In their definition, the term ``equivalent'' $\approx$ was not defined formally. Although in many cases an adversary aims for revealing a property exactly, NC argued that there can also be cases in which an adversary is happy with an approximation. Scenarios in which approximation may suffice include data or code localization attacks, in which it can suffice if a tool prunes most of the search space that the attacker has to explore manually, or cryptographic key extraction, in which it can suffice if an analysis tool can severely prune the search space for a brute-force attack. 

The analysis considered in the definition need not necessarily be a single analysis. It can in fact be a sequence of analyses and (deobfuscating) transformations; such as when analyses reveal properties that are useful to deobfuscate the code, and the property ultimately targeted by the adversary is the deobfuscated code. 

Based on the above definition, NC then defined \emph{potent obfuscating transformations} in terms of a specific program, a specific property, and a \emph{set of analyses}. An obfuscation is \emph{potent} if it is effective against at least one of the analyses and not defective against any of them: ``a potent obfuscating transformation makes at least one analysis harder to perform, and no analyses easier.''

NC omitted resilience as a separate, complementary quality criterion. The reason is, of course, that the above redefinition of potency already incorporates the original concept of resilience.

For stealth, NC distinguish between \emph{local stealth} and \emph{steganographic stealth}~\cite{collbergbook}. A transformation is steganographically stealthy if an adversary's detector function cannot determine if a program has been transformed with it or not. A transformation is locally stealthy if the adversary's locator function cannot tell the locations where the transformation has been applied.

Formally, they propose to define the local/steganographic stealth of an obfuscation with respect to a considered class of programs and a considered locator/detector function as the maximum of the expected false positive rate and the expected false negative rate of that function over all programs in the class after obfuscation. 

With respect to detector and locator functions, there exists quite a bit of research on automatic techniques for the classification of software as obfuscated versus unobfuscated and for the automatic determination of the deployed obfuscations and obfuscators. Such techniques are invariably based on the observation of software features that form fingerprints of obfuscations whose presence can be considered as indicators of (a lack of) stealth~\cite{obfuscation_detection}.

Schrittwieser et al.~\cite{modeling_stealth} used complexity metrics as fingerprints. Raubitzek et al.~\cite{obfuscation_detection4} convert binary code bytes to grayscale values and use singular value decomposition to uncover patterns created by different obfuscation techniques in images. Jiang et al.~\cite{obfuscation_detection2} use the number of different types of instructions in basic blocks and a basic block adjacency matrix of function control flow graphs as features, as well as the number of string constants in basic blocks. Bacci et al.~\cite{obfuscation_detection3} and Wang and Rountev~\cite{obfuscation_detection5} use strings and bytecode n-grams as features. Kim et al.~\cite{obfuscation_detection6} use opcode distribution. Tofighi-Shirazi et al.~\cite{obfuscation_detection7} obtain symbolic expressions from disassembled function bodies through symbolic execution and then use the words in them and their frequencies as features in a bag-of-words approach. Tian et al.~\cite{obfuscation_recognition} represent functions by their so-called reduced shortest paths. Zhao et al.~\cite{obfuscation_detection8} use assembly instruction embeddings based on skip-grams, convolutional neural networks (CNNs) to generate basic block embeddings from the instruction embeddings, and long-short-term memory networks (LSTM) to encode the semantics of basic blocks and their relations into features. Kanzaki, Monden, and Collberg proposed a code artificiality metric based on an N-gram instructions model~\cite{code_artificiality}.

These features are all used in classifiers, of which the models serve to quantify stealth in the sense that an obfuscation can be considered more stealthy if the models perform more poorly on it. 

\subsection{Critique}
\label{sec:2009-critiques}
The redefinition of potency answered many of the critiques from Section~\ref{sec:9798-critiques}. However, it still lacks in numerous ways. Some critiques still apply, such as those on layering (\ref{sec:layering_missing}), sensitivity (\ref{sec:sensitivity_missing}), and learnability (\ref{sec:learnability_missing}). Additional critiques are the following. 

\subsubsection{All Analyses Treated Equally}
\label{sec:all_equal}
In the definition of a potent obfuscation, the characteristics of the analyses against which it must be effective and must not be defective are not considered. Suppose that an obfuscation is defective with respect to one analysis because it decreases its computation time on a program from 10s to 5s, while being effective for another analysis because it increases that analysis' computation time from 2s to 8s. If the adversary has both analyses available, their minimal execution time increased from 2s to 5s with the obfuscation; their combined time increased from 12s to 13s. So should that be considered a potent obfuscating transformation? According to the current definition, it is not. 

The new definition of potent transformations implicitly treats all considered analyses equally important to mitigate, as if attackers have an oracle at their disposal that tells them which of all potentially useful analyses they should use given a concrete program they need to analyze. This worst-case scenario assumption can definitely be relevant in a risk management approach~\cite{Basile23} to ensure that potentially relevant attack strategies are not overlooked. However, for evaluating the practical strength of obfuscations, one clearly needs to consider which analyses adversaries are more likely to deploy. 
To some extent, this relates to stealth: When a deployed obfuscation is not stealthy, this might enable attackers to select the most appropriate analyses, or even to customize them. 

One might argue that one can counter this critique by selecting not a set of individual program analysis techniques to evaluate potency, but a set of multi-step attack strategies, in which each strategy consists of pre-pass analysis that is first deployed to select the most appropriate actual analysis technique to go after the targeted property, after which that technique is then used. However, this reformulation would only shift the problem from the actual analysis techniques to the pre-pass analyses, rather than solve it.

In computer security, in general, a clear distinction is made between security assessments and risk assessments. The former focus solely on the technical identification of vulnerabilities and exploits, while the latter also considers the likelihood that attackers know about an existing vulnerability and corresponding exploit, the likelihood that the attackers are willing to use that exploit, and the potential damage that such exploitation would have on the victim and their business. The proposed definitions of potent transformations and stealth provide no guidance on how to incorporate or evaluate the likelihood that adversaries will choose custom analyses, or how difficult it is for them to do so, i.e., how attackers can decide on the most fit-for-purpose analysis.

\subsubsection{Comparing Obfuscations}
One shortcoming of NC's definition of potency is that it does not allow one to compare different obfuscating transformations, e.g., to determine which one is more potent for some program, against which analyses. They in fact acknowledge that, and refer to the work by Dalla Preda et al.\ for a solution~\cite{mila07}. Section ~\ref{sec:abstract_interpretation} will focus on this solution.

\subsubsection{Ad Hoc Formulas}
NC's formulas for computing stealth compute the maximum of false positive and false negative rates. We do not see why this would be advisable over using more standardized metrics such as recall, precision, or the F1 metric. Many authors seem to agree with us. In the post-2009 literature on obfuscation detection techniques, including a 2016 paper co-authored by Collberg~\cite{code_artificiality}, not a single paper uses the formulas proposed by NC. %This to be a good example of progressive insights. 

\subsubsection{Resilience Missing in Action}
With the redefinition of potency, the original distinction between potency and resilience has become moot. Their unification definitely offers advantages, such as moving away from the artificial distinction between confusing humans performing manual work and confusing tools that automate analyses (and deobfuscating transformations). However, it also comes with major disadvantages. As observed by De Sutter et al., many publications in the domain of software obfuscations feature sub-par evaluations~\cite{desutter2024evaluation}. Particularly relevant for this work are their observations that few papers that present novel obfuscation techniques evaluate those obfuscations against real-world attacks and against adversaries that adapt their attack strategies to the fact that certain obfuscations have been deployed, as is commonly the case in the cat-and-mouse game of SP, and as has been observed in human experiments by H\"ansch et al.~\cite{2018_programming_experience_might_not_help_in_comprehending_obfuscated_source_code_efficiently}.

De Sutter et al.\ advocate to differentiate between potency and resilience because authors of obfuscation papers are recommended to consider at least two classes of attacks against which to evaluate the strength of their obfuscation techniques. In their vision, attacks on the assets being protected with some obfuscation determine its potency, and attacks on the SP itself determine its resilience. The distinction between potency and resilience is then still pretty vague, but the continued usage of the two terms can help to raise awareness about the best practices of SP evaluation methodologies.

\subsubsection{Confusion about Potency and Resilience}
These updated definitions have not been picked up by many colleagues. Instead, the original definitions of potency and resilience are still often used, and sometimes inconsistently, as we mentioned in Section~\ref{sec:fuzzy_boundary}. Perhaps this should come as no surprise: In 2017, 8 years after the redefinition of potency that abandons resilience in his book~\cite{collbergbook}, Collberg himself co-authored the work of Banescu et al.~\cite{2017_predicting_the_resilience_of_obfuscated_code_against_symbolic_execution_attacks_via_machine_learning} that still uses resilience when assessing the capability of symbolic execution to extract secrets instead of deobfuscating code.

\subsubsection{Layering Neglected, of Protections and Analyses}
Layering of SPs, and hence assessing the marginal strength of SPs, is still neglected. So is the combination of analyses. In practice, attackers rarely rely on a single analysis method or a single attack step. The definitions of NC do not reflect this. One could argue that it is the responsibility of the individual researchers to consider analyses that are combinations of code analysis techniques rather than individual ones. We believe that anyone defining the aspects to be evaluated should stress the usefulness of this and discuss, in general, the options to do so. 

\subsubsection{Narrow Scope of Stealth}
\label{sec:narrow_stealth}
The updated definition of stealth based on the accuracy of detector and locator functions instead of on the similarity between transformed code and original application code is certainly an improvement. When the results of locator and detector functions are merely used as data inputs to later attack steps, measuring their accuracy on obfuscated code can suffice to evaluate the obfuscation's stealth.

When their results are used as control inputs, i.e., to make strategic decisions on which attack steps to apply next or on how to configure the next attack steps, measuring their accuracy is still useful. In such cases, however, one has to raise the question if there might be other ways to enable the same decisions. For example, maybe an adversary only needs to know which obfuscation tool has been used to make a strategic decision. Or maybe they need to know a specific configuration parameter with which an obfuscation was applied in some location to make a strategic decision? 

By relying only on obfuscation locator and detector functions, one cannot assess to what extent other fingerprints of deployed protections can be revealed that enable strategic decisions. 

\section{Dalla Preda and Giacobazzi}
\label{sec:abstract_interpretation}
Dalla Preda and Giacobazzi, with a variety of collaborators, have worked extensively on theoretical foundations to make obfuscations comparable, namely through the lens of \emph{abstract interpretation} (AI). In their view, attackers are limited in what they want to analyze and in their resources, so they model them as a specific abstraction. The attacker then coincides with the considered analyzer.

Discussing all the formal foundations of their work and presenting their formal definitions is not feasible in this paper, as it would require too long an introduction to abstract interpretation and theoretical results achieved in that domain. Unlike the definitions discussed in previous sections, their results in the form of mathematical proofs and equations are not intuitive and hard to interpret for non-experts.  We therefore summarize their main conclusions and how their work has evolved in an informal manner. 



\subsection{Definitions based on Completeness}
\label{sec:completeness}
Initially, the study of AI to define potency~\cite{cousot1977abstract,cousot2021principles} was based on the idea that the statement ``obfuscation makes programs incomprehensible for observers'' can be rephrased as ``obfuscation makes programs incomplete for abstract interpreters''~\cite{2008hiding}. The potency is therefore defined in terms of the \emph{completeness} of AIs with which attackers might try to reveal properties of a program that the defender is trying to prevent with obfuscations. 

An analysis of a program modeled as an AI is incomplete if the analysis produces an imprecise result. An imprecise result occurs \emph{when an analysis produces some valid result that is less precise than another valid result that the analysis can represent}. For example, consider a value set analysis that computes and propagates intervals of the form $[a,b]$ for any real values of $a$ and $b$ with $a \leq b$. If such an analysis reports that some variable in a program can hold values in the range $[0,10]$, while it can actually only hold values in the range $[1,5]$, the analysis has proven to be incomplete for that program. Importantly, incompleteness depends on both the program and the analysis. Some concrete analysis, and variations thereof, can be complete on one program and incomplete on another one. 

The mentioned authors define the potency of an obfuscation in terms of its impact on completeness: Which analyses become incomplete on which programs by deploying an obfuscation? 

Analyses modeled as AIs operate on abstract domains. The domains themselves consist of lattices, but all possible domains also form a lattice, meaning that they are ordered in terms of the relative precision of the properties that they can represent and potentially deduce from programs. And hence so are the analyses: Some can represent more precise properties than others, some can deduce more precise properties than others, and hence some analyses can deduce certain properties on more programs than other analyses.  

In older work~\cite{mila05a,mila05b,mila07}, an obfuscation is considered potent when there is a property that is not preserved, i.e., when some AI that could compute that property on the original program can no longer compute it after  obfuscation. Different obfuscations can then be compared based on the most precise properties they preserve. %The worst obfuscation is the one that preserves the most precise property. 

In 2012--2017, potency was redefined in terms of specific analyses that are used to reveal certain properties~\cite{2012_making,2017GMDP}. The potency of an obfuscation with respect to a specific analysis, program, and property is then determined by whether or not the completeness of the analysis on the program is impacted. If the analysis can no longer compute the property on the obfuscated program, the obfuscation is considered to be potent. Importantly, even if the analysis can still extract the property, the obfuscation might still be useful. Although the obfuscation then cannot counter that specific analysis, it might be able to counter simpler variants thereof. To assess this utility, the \emph{potency range} of an obfuscation was defined in terms of its impact on the completeness of compressed versions of the analysis, with ``compressed'' meaning ``less precise''.

In 2018, Bruni at al.~\cite{bruni2018code1,bruni2018code2} proposed an alternative method to assess obfuscations that aim to counter model checking attacks based on abstraction refinement, such as CEGAR~\cite{cegar}. In such attacks, a model checker iteratively tries to check the validity of models, starting with a very abstract domain that can lead to a conclusion quickly. When a conclusive result cannot be reached with the domain used in one iteration, a more precise refined domain is chosen, and a new attempt is made in a next iteration. The strength of an obfuscation is then defined as the extent to which it can prevent the checker from coming to a valid conclusion in fast early rounds with very AIs. Interestingly, the authors illustrate that many existing data flow analyses, such as liveness analysis, constant propagation, and available expression analysis, can be reformulated as model-checking problems, thus widening the applicability of this method. 

In follow-up research in 2022, Bruni et al.~\cite{bruni2022repair} present AI repair strategies to automatically refine domains to make the AI (locally) complete for a given program, with the goal of enabling program verification methods to start with any abstract domain. If the initial domain is too abstract to avoid false alarms, it can be refined with this strategy to produce complete results. The user of an AI then no longer needs to choose the domain beforehand to yield optimal formal verification results. In our eyes, this comes close to how reverse engineers adapt their strategies and analyses and how they move on to more complex ones if their initial, basic attempts fail. 

In 2022--2023, Campion et al.\ introduce the notion of $\epsilon$-partial incompleteness~\cite{Campion2022,campion23}. An abstract interpreter can be $\epsilon$-partial incomplete with respect to a given program and a given (set of) input values, meaning that the imprecision of the AI result is bounded by $\epsilon$, that is, the distance between the results of the abstraction of the concrete semantics (e.g., $[1,5]$ in the value set analysis example above) and the result of the AI on the given input (e.g., $[0,10]$) is at most $\epsilon$. For this, they define distance metrics on the abstract domains. They argue that the ability to quantify the amount of imprecision induced in the AI by an obfuscating transformation could be used to measure the potency of such a transformation.

Interestingly, they concede that, in general, one cannot automate the procedure of deciding whether the AI of a given program on a given input satisfies a given precision bound $\epsilon$. 
Still, the notions of $\epsilon$-partial incompleteness and bounded distance between actual properties and obtained analysis results can also open opportunities to reason about the strength of analyses that only have to compute approximate results (see Section~\ref{sec:collberg_nagra_def}). To the best of our knowledge, this direction has not yet been explored. 

In 2025, Giacobazzi and Ranzato~\cite{2025roberto} showed that the program property of having the best possible AI is not trivial and, in general, hard to achieve. Among others, they showed the impossibility of achieving the best correct abstraction property through minimal abstraction refinements or simplifications of the abstract domain. This puts into question the underlying assumption of some of the discussed definitions of potency that are based on which analysis can still reveal which properties. 

A strength of several works is that they not only allow to reason about the strength of obfuscations, but also include techniques to automatically derive obfuscations to counter the analyses defined in terms of AI~\cite{mila05a,mila05b,mila07,DP2013,DP2018,Roberto2012}. This includes obfuscations that target control flow analysis, data flow analysis, model checking, and more.

\subsection{Definitions based on Adequacy}
\label{sec:adequacy}
In 2023, Giacobazzi et al.~\cite{fitting_roberto} proposed to complement (and even replace) completeness as the basis for assessing potency with \emph{adequacy}. They observe that ``completeness characterizations do not really deal with the loss of precision due to the choice of the abstract observation, since they characterize only whether there is an extra loss of precision due to the computation on observed/abstracted data (compared with the observation of the concretely computed result).'' In other words, completeness is a measure of the mismatch between the computations in a given program and the chosen abstract domain (lattice) in which the program is interpreted, rather than measuring how good that domain is for revealing the properties in which an attacker/analyst is actually interested. 

Consider the trivial $\top$ abstraction that abstracts all concrete values to $\top$. Its lattice consists of one element $\top$. The result of such an AI on any program produces $\top$, which is by construction the most precise outcome that can be presented in the lattice, so such an AI is by construction complete. But it is completely useless: its result comes down to ``I don't know.'' In other words, the produced result is the least precise result that the attacker/analyst can be interested in, despite the interpretation being complete. 

Notice that such an element $\top$ is indispensable in many analyses: For an undecidable analysis to be sound, it needs to be capable of responding ``I don't know'', for which the used lattice includes $\top$.

The adequacy of an AI for a given program is then defined as the ability of the interpretation to produce analysis results for that program that contain strictly more information than $\top$. The relative adequacy of an AI for a program, i.e., adequacy with respect to some other element $\tau$ of the lattice, is then defined as the ability to produce a result that is strictly more precise than $\tau$.

The link with potency then is that an obfuscation can be considered potent with respect to some analysis and some program if it can make the analysis become inadequate on the obfuscated program while it was adequate on the original program. 

\subsection{Critique}

\subsubsection{Practical Applicabillity}
\label{sec:ai_practical}
While the use of AI theory to evaluate the strength of practical obfuscations against practical attacks has been discussed, such as obfuscations to mitigate disassemblers and slicing~\cite{2017GMDP}, it remains mostly a theoretical subject, of which the practical applicability is unclear. Although the AI work for the evaluation of model checking by Bruni et al.~\cite{bruni2018code1,bruni2018code2} might have the potential for wider applicability, a recent 571-paper literature review on evaluation methodologies in SP research~\cite{desutter2024evaluation} observes that software obfuscation and deobfuscation researchers do not have a strong appetite for this form of analysis: Model checking was one of the least popular attack methods, used in only 4\% of the papers that used concrete attack methods to evaluate obfuscations. 

\subsubsection{Attacks Success/Failure instead of Delay}
In the field of practical MATE SP, it is understood that MATE attacks cannot be completely prevented. Given enough time and effort, adversaries will always be able to get what they want. The main goals of obfuscation therefore are to delay attacks, to increase their costs, and to decrease their return-on-investment. In contrast, the existing research on AI and obfuscation focuses by and large on evaluating whether or not some property can be revealed with some interpretation/analysis and whether this can be prevented with an obfuscation. In other words, this research focuses on scenarios in which the defender tries to make attacks fail rather than trying to delay them. Although the complexity of different AIs is compared in some works to compare the strength of different obfuscations, the link between actual attack delay and analysis complexity is not made. This research hence seems to miss the point of MATE SP. 

\subsubsection{Soundness}
Research on AI and obfuscation only considers sound analyses. This is fine in many program analysis scenarios, such as program verification. In the MATE attack model, by contrast, attackers use any analysis, sound or unsound, that allows them to reach their goal. All dynamic analyses, e.g., are unsound. In this regard, this research again completely misses the point. 

\subsubsection{Program Understanding Only}
In 2008, Giacobazzi~\cite{2008hiding} stated ``The lack of completeness of the observer is therefore the corresponding of its poor understanding of program semantics''. This pinpoints an important limitation: This research, at least in the first order, targets software comprehension. It neglects the other MATE attack goals on obfuscated software. In addition to \emph{code comprehension}, Schrittwieser et al.~\cite{survey2016} identified \emph{finding the location of data}, \emph{finding the location of program functionality}, and \emph{extraction of code fragments} as important attacker targets. NC also claim that an adversary targeting an obfuscated program typically goes through a locate-alter-test cycle~\cite{collbergbook}. Neglecting location finding attacks is a clear limitation of AI research into the meaning of potency.

\subsubsection{Defender vs.\ Attacker Perspective}
\label{sec:perspective}
In much if not all of the cited research, the perspective of theory development is that of defenders that know which property of which program fragments they want to hide with obfuscations. The defenders hence know which analyses to consider for their assessment of the obfuscations' strength. Attackers, on the contrary, often do not know a priori which property they are after. In particular in data and functionality location finding attacks, the attackers by definition do not know the fragments of which they want to reveal properties.

Moreover, as defenders know which properties they want to hide, they can reason about the possible refinements of analyses that undo the impact of the chosen obfuscations. Attackers that try to find code or data or that try to understand a program do not know a priori which properties they are after. So how can they deploy similar refinement strategies? If they cannot, then why should defenders worry about the capabilities of refinement strategies? 

For most of its history, research on AI and obfuscation did not address the question of how difficult it is for an attacker to choose the best or simplest suitable AI for their attack. Only in 2025, Giacobazzi et al.\ started to address this issue, conceding that this may indeed be difficult~\cite{2025roberto}. This puts in doubt the relevance of using completeness or adequacy of the best possible abstractions for evaluating obfuscation strength. In our eyes, the defender should care mostly, if not only, about the analyses that an attacker will possibly or likely use, rather than on theoretically better, but in practice unlikely to be used, attack methods. 

An example might make this critique concrete. Consider interprocedural constant propagation~\cite{interp_cp}, and a scenario in which a context-insensitive variant of the analysis is not precise enough to reveal some interesting program property, while some $k$-depth context-sensitive variant is precise enough for some $k$. Undoubtedly, the context-sensitive variant, which will be slower and have a larger memory footprint, should be considered the most complex. There exist countless variations of the analysis with complexities in between those two variants, namely variations that analyze only specific parts of the program in a context-sensitive manner while handling the rest in a context-insensitive way. Intuitively, the variant that requires the least context-sensitivity should be considered the simplest one. Now, while the defender might know where context-sensitivity is useful and where not, hence being able to determine the simplest complete analysis for their program at hand, how is an attacker supposed to determine this variant? In particular when an attacker does not yet know the relevant properties to be revealed, this seems inconceivable. So what is the relevance of that best and simplest complete variant? 

One might think that attackers would go with the simplest or fastest analysis first and switch to more precise and slower ones only when the initial ones fail. So, first run, e.g., a context-insensitive analysis across the entire program (which should be cheap), then use whatever information they extracted to target a context-sensitive analysis to the most likely location of the program.

This is probably often how attackers work, but certainly not always. A counterexample is a crypto key extraction attack, for which Ceccato et al.\ performed experiments with professional pen testers~\cite{emse2019}. The most experienced among them skipped static analysis entirely, assuming that the deployed SPs would make that too hard, so they used only dynamic techniques from the start. In other words, they opted for a more complex analysis method over a simpler one, even though a simpler one would have worked perfectly well on an unobfuscated program.

In any case, even if the attackers do start with the simplest analyses and refine them as they obtain additional information, we think it is relevant to consider the effort required to extract that information and how precise it would need be, e.g., to pinpoint the most likely location in the program where more precision is needed. We should not assume that the refinement comes for free.

On this topic, there exists quite some work on on-demand data flow analysis or demand-drive data flow analysis. Such analyses are designed to be more efficient than blindly running the most complex analysis variants on a whole program. Those on-demand analyses, which adapt on-the-fly are more complex than analyses for which an a priori determination was made about which locations in the program require a higher form of sensitivity. So the consideration of such analyses does not solve the issue that AI research starts too much from the defender's perspective. 

The work by Bruni et al.\ cited above~\cite{bruni2022repair} is highly related. In that work, it is also assumed that the user will run a simple analysis first, and then iteratively run ever more refined versions thereof. All of that effort should be considered when evaluating potency or resilience, however, not just the complexity of the final analysis version. Maybe the journey is not more important than the destination, but it definitely is as important.

\begin{table}[!t]
  \caption{Tools used by the \evaluator}
  \label{tab:tools}
  \centering
  \footnotesize
  \begin{threeparttable}
    \begin{tabular}{p{0.45\linewidth}p{0.4\linewidth}}
    \toprule
    \textbf{Property}     & \textbf{Tool} \\
    \midrule
    Target Binding Affinity & AutoDock Vina~\cite{trott_autodock_2010} \\
    Drug-likeness (QED)     & RDKit~\cite{rdkit}         \\
    Lipinski's Rule         & RDKit~\cite{rdkit}         \\
    Synthetic Accessibility & RDKit~\cite{rdkit}         \\
    Novelty                 & RDKit~\cite{rdkit}         \\
    Diversity               & RDKit~\cite{rdkit}         \\
    \bottomrule
    \end{tabular}
  \end{threeparttable}
  % \vskip -15pt
\end{table}

\section{A New Framework}
The formulated critiques articulate that a better formulation of evaluation criteria for obfuscating transformations is needed to provide better guidance on how to assess their real strengths and weaknesses rather than reporting artificial, irrelevant, or even misguided proxies. The primary aim of such guidance is to help researchers maximize the impact of their work by convincing other researchers that they can build on this work for their own research and by convincing practitioners of the relevance of the work for real-world software protection. To a large degree, such impact depends on the validity of the performed evaluations. This includes the commonly used criteria of validity in software engineering, such as construct, conclusion, internal, and external validity~\cite{Wohlin} but also less commonly considered criteria such as instantiation validity~\cite{lukyanenko2014instantiation}. \emph{Reducing such threats to validity is the aim of this framework.} 

In line with many of the existing definitions discussed in Sections~\ref{sec:collberg_nagra_def},~\ref{sec:completeness}, and~\ref{sec:adequacy}; in line much of the examples mentioned in Section~\ref{sec:attack_tool_examples}; and in line with the recommendations of De Sutter et al.~\cite{desutter2024evaluation} that evaluations obfuscation strength evaluations should be based on their impact on real-world attacks, we put forward that the used evaluation criteria should be based on specific program properties that adversaries can target with specific attack strategies, and on the impact that the obfuscations have on those strategies. 

Unlike NC, we think that a concept similar to resilience needs to be a core criterion. And unlike the research on AI-based potency, we feel the criteria need to be practical, e.g., in the sense that they are consistent with the fact that MATE attackers can always reveal the properties they are after, and that the required effort of a whole attack strategy, not only of individual steps, is what matters. Unlike existing evaluation criteria defined in terms of analysis, we define them in terms of attack strategies, which consist of different categories of attacks steps being executed: 
\begin{enumerate}
\item knowledge gathering, i.e., revealing properties:
\begin{enumerate}
    \item revealing static/dynamic artifacts in the program;
    \item revealing relations between artifacts;
    \item revealing features of the artifacts and the relations;
    \item assigning priorities to artifacts and relations; 
    \item revealing mappings between abstract artifacts, properties, and relations; and concrete ones.
\end{enumerate}
\item artifact manipulation, such as lifting a code fragment from a program for ex situ execution, tampering with a SP to undo it, or altering its execution state to bypass a SP; 
\item decision making on the next steps to execute based on the determined priorities and already gathered knowledge.
\end{enumerate}
These categories capture the four goals of analyses in the survey by Schrittwieser et al.\ on obfuscation vs.\ program analysis~\cite{survey2016}, the activities modeled by the reverse engineering formalization by Faingnaert et al.~\cite{checkmate24}, the activities observed in experiments by Ceccato et al.~\cite{emse2019}, in a survey on the practice of malware analysis~\cite{wong2021inside}, etc. For example, code comprehension tasks, such as understanding that some code fragment implements a quicksort, are instantiated by combinations of (1a), (1b), (1c), and (1e). The same holds for making hypotheses, confirming them, and discarding them, which are important attack activities as observed by Ceccato et al.~\cite{emse2019}. Finding the location of relevant code and data in a program in addition requires (1d) to classify the revealed information into relevant vs.\ irrelevant ones. Undoing an obfuscation obviously requires (2). 

\emph{It is hence against combinations of these kinds of attack steps, and the possible methods to execute them in relevant attack strategies, that obfuscations can and should be evaluated.}

Importantly, the methods used to execute these steps can be unsound and non-conservative. For example, trace-based analysis techniques to reveal properties are unsound. Artifact manipulation may be non-conservative when the applied transformation does not conserve the behavior under all possible circumstances. When some hypothesis is being made by an adversary in some attack steps that are not confirmed in later steps, for example because the adversary reaches their goal first, the steps building on that hypothesis are obviously unsound as well. 

Given the wide range of properties that adversaries might be interested in, and the wide range of analyses techniques that can help them, it makes no sense to propose a one-size-fits-all set of evaluation metrics or to prescribe exactly what metrics should be measured. We hence limit ourselves to prescribing which criteria should be evaluated, aiming at defining them in ways that avoid the confusion that has existed regarding previous definitions. Our new framework of criteria includes relevance, effectiveness (and its poor man's alternative of efficacy), robustness, concealment, stubbornness, sensitivity, predictability, and cost. Table~\ref{tab:criteria} lists them all, including their subcriteria. 

\begin{table}[t]
    \centering
    \begin{tabular}{c p{6cm} l}
\hline
    \multicolumn{3}{l}{\textbf{Relevance}}\\
%    \midrule
       & Attack Steps Relevance  & $Re_{a}$ \\
       & Program Property Relevance & $Re_{p}$ \\
       & Obfuscation Impact Relevance & $Re_{o}$ \\
       & Metrics Relevance & $Re_{m}$ \\
       & Tool Availability:& $Re_{t}$ \\
       & Sample Relevance & $Re_{s}$\\
       & Layered Protection Relevance & $Re_{l}$\\
%   \midrule
    \multicolumn{3}{l}{\textbf{Effectiveness}}\\
   & Isolated Outcome Effectivness & $E_{o,i}$ \\
   & Marginal Outcome Effectivness & $E_{o,m}$ \\   
   & Isolated Resource Effectivness & $E_{r,i}$ \\
   & Marginal Resource Effectivness & $E_{r,m}$ \\
%      \midrule
    \multicolumn{3}{l}{\textbf{Robustness}}\\
     &  Isolated Outcome Delta & $Ro_{o,i}$ \\
     &  Marginal Outcome Delta & $Ro_{o,m}$ \\
     &  Isolated Resource Delta & $Ro_{r,i}$ \\
     &  Marginal Resource Delta & $Ro_{r,m}$ \\
     &  Deployment Delta & $Rd_{o,m}$ \\
%           \midrule
    \multicolumn{3}{l}{\textbf{Concealment}}\\
         &  Local Concealment & $C_{l}$ \\
         &  Global Concealment & $C_{g}$ \\
         &  Strategic Concealment & $C_{s}$ \\
    \multicolumn{3}{l}{\textbf{Stubbornness}}\\
         &  Outcome Stubbornness & $St_{o}$ \\
         &  Resource Stubbornness & $St_{d}$ \\
   \multicolumn{3}{l}{\textbf{Sensitivity}}\\
         &  Sample Feature Sensitivity & $Se_{s}$ \\
         &  Attack Instantiation Sensitivity & $Se_{a}$ \\       
         &  Protection Instantiation Sensitivity & $Se_{pi}$ \\       
         &  Protection Configuration Sensitivity & $Se_{pc}$ \\       
         &  Build Tool Flow Sensitivity & $Se_{b}$ \\                
         &  Platform Sensitivity & $Se_{p}$ \\ 
   \multicolumn{3}{l}{\textbf{Predictability}}\\
         &  Sample Feature Predictability & $P_{s}$ \\
         &  Attack Instantiation Predictability & $P_{a}$ \\       
         &  Protection Instantiation Predictability & $P_{pi}$ \\       
         &  Protection Configuration Predictability & $P_{pc}$ \\       
         &  Build Tool Flow Predictability & $P_{b}$ \\                
         &  Platform Predictability & $P_{p}$ \\ 
   \multicolumn{3}{l}{\textbf{Cost}}\\
         &  Performance Cost & $C_{p}$ \\
         &  SDLC Cost & $C_{sdlc}$ \\       
         \hline
    \end{tabular}
    \caption{All obfuscation evaluation criteria and subcriteria}
    \label{tab:criteria}
\end{table}


\subsection{Relevance of the Evaluation Constructs}
The first criterion to be considered in an evaluation of an obfuscation's strength is the relevance of all the constructs used in the evaluation. To maximize the relevance of a researcher's evaluation results for practitioners, the used constructs need to representative for real-world software protection. 


Researchers, however, face deadlines and budget and resource limitations. Different researchers therefore aim for different technology readiness levels (TRLs) with their research. That is obviously fine, and different criteria of our framework support research and evaluations at the various levels, while at the same time enabling and encouraging the researchers to be transparent about the TRL of their research results. Ensuring this transparency, and forcing the researcher to reason about the impact of evaluation methodology choices on the relevance of the evaluation results and their validity is the aim of the seven relevance criteria detailed below. 

It should come as no surprise that we first focus on criteria that pertain more to the evaluation method than to the obfuscation being evaluated itself. The reason is of course the lack of standardization in the domain of SP~\cite{Basile23}. Where certification has been standardized (e.g., smart card security~\cite{common_criteria}) and where risk management standards have been adopted (e.g., NIST SP 800-39~\cite{nistSP800-39} in network security~\cite{Gartner-report-riskanalysis}), concise evaluation results can be published that refer to those standards so that all stakeholders know how to interpret those results. In MATE SP, on the contrary, only some embryonic steps toward standardization have been proposed~\cite{Basile23}. Lacking standardized evaluation methods, evaluation results are meaningless without proper framing. 

An evaluation therefore stands or falls with accurate and complete framing, which can be achieved by assessing the evaluation according to the seven relevance criteria we put forward:

\paragraph{$Re_a$ - Attack Steps Relevance:} To what extent are the considered attack step combinations relevant? In which relevant attack strategies are they used, meaning attack strategies with a proven track record in the scientific literature or in real-world reporting? Are there no alternative combinations known that can produce the same results as the considered ones but are simpler to execute? Is the starting point of the considered attack steps realistic, i.e., the preceding attack steps? Does the evaluation use realistic outputs of those preceding steps as inputs to evaluate the considered attack step combination?

\paragraph{$Re_p$ - Program Property Relevance:} To what extent are the program properties relevant that the protection is supposed to hide, and that adversaries supposedly want to reveal through the considered attack steps? Are there alternative strategies with which adversaries can reach the same end goal, but without requiring them to obtain the results of the considered attack steps, i.e., without them requiring to reveal exactly those program properties?

\paragraph{$Re_o$ - Obfuscation Impact Relevance:} To what extent does the impact of the evaluated obfuscation on the result of the considered attack step combination affect the execution of later attack steps? Does it make them impossible? Does it make them require more resources? Does it make them produce less precise results? 

\paragraph{$Re_m$ - Metrics Relevance:} Do the (commonly used or ad hoc) metrics used to assess the obfuscation's impact on the considered attack steps' outputs truly capture the impact on later attack steps? Has that been validated, or are there strong arguments or evidence? 

\paragraph{$Re_t$ - Tool Availability:} Is practical tool support available to automate the considered attack steps and strategy? 

\paragraph{$Re_s$ - Sample Relevance:} Are the used samples representative for the types of software that would be attacked with the considered attack steps and strategies in the real-world?  

\paragraph{$Re_l$ - Layered Protection Relevance:} Does the evaluation consider relevant layerings or compositions of SPs?  

\vspace{0.2cm}

Note that some criteria are related, such as $Re_a$ and $Re_t$. It is the task of the researcher to be consistent in their assessment of their evaluation with respect to these criteria.  Also note that these criteria are to be evaluated mostly, if not completely, qualitatively. 

\subsection{Effectiveness against Standard Attacks}
\label{sec:effectiveness}

The effectiveness of an obfuscation for protecting a secret property in a given program against a considered attack step combination is defined as the obfuscation's effect on that combination, for that program and that property. 

To a large degree, effectiveness equals the old criterion of potency as defined by NC (see Section~\ref{sec:collberg_nagra_def}). There is one major difference, however: We specifically prescribe that the criterion effectiveness should only be used when the considered attack steps are \emph{standard attack steps}, i.e., interesting options for an adversary with no a priori knowledge about the specific SPs being deployed to counter the considered attack steps. In other words, these are the attack steps that can be expected to work well on vanilla applications as well as on the average protected application that the adversary may expect to be facing.  It is because of this crucial difference that we propose not to reuse and redefine the term potency, but instead put forward the term effectiveness. 

The notion of a standard attack steps is deliberately fuzzy. It is up to each researcher to determine which attacks steps to consider standard. The use of the term effectiveness then comes with the responsibility  to argue why the considered steps are to be considered standard. This is of course closely related to the relevance criteria $Re_a$, $Re_t$, and $Re_o$. It is the researcher's responsibility to ensure consistency in their assessment. 

Just like NC's potency considers the impact of an obfuscation in terms of analysis outcomes, and the required resources, so does effectiveness entail two criteria:

\paragraph{$E_o$ - Outcome Effectiveness} What is the effect of the obfuscation on the outcome of the standard attack step combination? Does the result become more complex, less precise, less accurate, incorrect, etc.? Often this effect can be quantified with (ad hoc) metrics. 

\paragraph{$E_r$ - Resource Effectiveness} What is the effect on the required resources to execute the standard attack steps. This effect can often be quantified. Many different forms of resources can be considered: computation time, memory footprint, network bandwidth, etc.

\vspace{0.2cm}

The effectiveness of an obfuscation for protecting a class of secret properties against a considered attack step combination is then to be obtained by evaluating it on multiple samples and by reporting the distribution of the obtained results. 

The metrics used to assess the outcome effectiveness $E_o$ should be computed on results produced by actual attack steps or good proxies thereof, not on ground-truth data produced while building the protected samples. So they should be tool-based, like the examples in Section~\ref{sec:attack_tool_examples}. Complexity metrics originating from the domain of software engineering can still be used, on program representations reconstructed by attack tools such as disassemblers, if there exists a correlation between the used metrics and the execution of later attack steps in the considered attack strategies. Obviously this relates to the relevance criterion $Re_m$. 

Of both the criteria, two forms can be considered. Isolated effectiveness ($E_{o,i}$ and $E_{r,i}$) is the effect of an obfuscation when it is evaluated in isolation. Marginal effectiveness ($E_{o,m}$ and $E_{r,m}$) is the effect of deploying the obfuscation in combination with other (commonly used) SPs. It is strongly encouraged to evaluate marginal effectiveness. For example, when evaluating the impact of some obfuscation on Javascript code, it should be evaluated on top of basic minification transformations~\cite{2019_anything_to_hide_studying_minified_and_obfuscated_code_in_the_web} that any Javascript obfuscator supports, such as identifier renaming~\cite{liu2017stochastic}. 

Finally, we stress that if an obfuscation aims at complicating multiple, different attack steps or step combinations, its effectiveness should also be evaluated for each of those steps or combinations, resulting in multiple results for outcome and resource effectiveness. 

\subsection{Efficacy - a Poor Man's Effectiveness}
\label{sec:efficacy}
When a researcher cannot provide convincing arguments or evidence for the considered attacks steps being standard and relevant, or when no actual attack steps are used in an evaluation, such as when only complexity metrics are being computed on ground-truth data, the term effectiveness should not be used. At best, the term \emph{efficacy} (with shorthands $e_o$ and $e_r$ instead of $E_o$ and $E_r$) can then be used to stress that something akin to lab conditions is being evaluated, rather an evaluation indicative for real-world conditions. 

\subsection{Robustness against Special-purpose Attacks}
As effectiveness considers only standard attacks steps, we need a counterpart for non-standard attack steps. For this purpose, the robustness of an obfuscation is determined in terms of how much better \emph{special-purpose analyses} can do than standard analyses. Better can be in terms of required resources and in terms of produced result. Special-purpose attack steps are those that perform (on average) worse than standard ones on samples that do not feature the evaluated obfuscation, and that would therefore not be chosen by adversaries unless they know they are attacking this obfuscation. 

Special-purpose attack steps benefit attackers if they require less resources or produce better results than standard ones on at least some programs protected with the obfuscation being evaluated. An example is the $k$-depth interprocedural constant propagation~\cite{interp_cp} discussed in Section~\ref{sec:perspective}, where $k$ is increased only for those parts of the program where it matters to reveal the secret property. Another example is a devirtualization technique tweaked for specific forms of virtualization~\cite{deobf_virtualization,kinder}.

The robustness against some special-purpose attack step combination is evaluated in three dimensions:

\paragraph{$Ro_o$ - Outcome Delta} How much worse or better is the result of some special-purpose attack on a protected sample compared to the result of a standard attack on it?

\paragraph{$Ro_r$ - Resource Delta} How much less or more resources does the special-purpose attack require on that sample?

\paragraph{$Ro_d$ - Deployment Delta} How much more configuration, customization, or development effort does the adversary need to invest before to enable the special-purpose attack, for example because no out-of-the-box tool support is available?

\vspace{0.2cm}

Obviously $Ro_o$ and $Ro_r$ should ideally be reported as distributions obtained on a number of samples.

Just like we did for effectiveness, we put forward isolated and marginal forms of these robustness criteria, to distinguish between the robustness of an obfuscation deployed in isolation, and the gain in robustness when deploying an obfuscation on top of other SPs. 

The deployment delta $Ro_d$ can be considered the updated equivalent of the developer effort in the definition of resilience by CTL (see Section~\ref{sec:resilience}). Unlike them, we do not prescribe the specific features that should be taken into account, nor do we prescribe a specific scale of robustness levels. We leave it up to the researchers to choose an appropriate assessment method. 
Obviously, $Ro_d$ and $Re_t$ are closely related. It is up to the researcher performing the evaluation to ensure consistency. 



\subsection{Concealment}
Concealment is the new framework's equivalent of the older concept of stealth. As discussed in Section~\ref{sec:narrow_stealth}, not only the accuracy of detector and locator functions needs to be assessed. A complete assessment of an obfuscation needs to include all its aspects that can help an adversary make strategic decisions. Concealment hence needs to be, and is, a more generic concept than the old stealth. 
 
One of the most important strategic decisions relates to robustness. The relevant question in that case is the following: If an obfuscation lacks in terms of stubbornness because special-purpose attack steps can defeat it, how easy is it for an adversary to determine which special-purpose attacks to choose? In other words, how likely will the adversary be able to decide to use that special-purpose attack. As an example, remember the discussion in Section~\ref{sec:perspective} on how adversaries can or cannot choose refinements of attack steps, such as choosing on which parts of the program to use a higher value of $k$ for $k$-depth interprocedural constant propagation. If an adversary has no way of determining these parts, i.e., if the adversary cannot create an oracle to produce the required configuration for such a special-purpose analysis, the defender should probably not be worried about its existence. 

To capture all relevant forms of information about the deployed protections, we put forward three concealment criteria: 

\paragraph{$C_l$ - Local Concealment} To what extent can a  locator function determine the exact locations in a program where an obfuscation has been deployed. This is the equivalent of NC's local stealth. We strongly encourage the use of standard accuracy metrics, however, instead of their proposal to compute the maximum of the false positive and false negative rates. 

\paragraph{$C_g$ - Global Concealment} To what extent can a detector function determine whether an obfuscation has been applied on a program. This is equivalent to NC's steganographic stealth. Again we encourage using standard accuracy metrics instead of a max function. 

\paragraph{$C_s$ - Strategic Concealment} To what extent can an adversary reveal additional information, of any kind, about the deployed protection that can drive strategic decisions in the attack strategy, in particular to determine which potentially beneficial special-purpose attack steps to execute next. We foresee that this criterion will most often be assessed qualitatively.

\vspace{0.2cm}

Note that as we did for effectiveness to replace potency, we opted to use the term concealment to replace stealth, even though they are very similar. We made this choice to avoid confusion when researchers report strength evaluations in the future. 

\subsection{Stubbornness against Deobfuscation}
Undoing an obfuscation is only one way to defeat it~\cite{emse2019}. It is still an important one, however, so the ease with which an obfuscation can be undone should be considered when evaluating it. Here the term ``undoing'' refers specifically to applying a transformation that undoes the effect of an obfuscation. It excludes performing the necessary analyses to determine which transformation to apply. 

Consider, for example, an opaque predicate that is always true and that is used in a conditional branch, which is hence always taken. With some preceding attack step ---of which the effectiveness can of course be evaluated--- the attacker might have determined that this predicate is always true. Stubbornness does not concern that preceding step, it only concerns the ease with which the opaque predicate insertion can be undone. With the code patching functionality of interactive disassemblers and decompilers such as Ghidra, this is trivial: it suffices to replace the conditional branch by a conditional one, thus simply removing the bogus execution path. Ghidra then automatically eliminates the now dead predicate computation and the bogus control flow path and bogus code from the reconstructed CFG and from the decompiled code. So a simple code patch suffices to completely undo the obfuscation. 

For other obfuscations, that are in nature perhaps not more advanced than opaque predicates, such easy undoing edits are not available. For example, after control flow flattening each flattened block has only one successor in the CFG, namely the dispatcher of the flattened code~\cite{flattening}. For reverting to the original CFG, simple edits that remove an execution path from the code will not suffice. 

To capture the ease with which such obfuscation-undoing edits and transformations can be performed on the representation of the software on which later attack steps will operate, we define two stubbornness criteria: 

\paragraph{$St_o$ - Outcome Stubbornness} To what extent can the effect of the obfuscation be undone in a relevant representation of the code? 

\paragraph{$St_d$ - Resource Stubbornness} How difficult is it to perform that deobfuscating transformation? 

\vspace{0.2cm}

For both aspects, only a qualitative assessment can be expected, which will probably involve ad hoc arguments and evidence. We certainly cannot prescribe specific methods at this point in time. But obviously, this assessment will need to be consistent with how other criteria are assessed, such as $Re_t$.

Note that multiple representations might need to be considered for these stubbornness criteria. If the later attack steps are assumed to be static code comprehension, editing Ghidra's internal representation of the binary code can suffice, as explained above for opaque predicates. If the follow-up attack steps to the contrary involve dynamic analysis techniques for which a patched binary needs to be executed, changing a tool's internal representation of a program will not suffice. In that case, a working edited binary actually needs to be produced first. Ghidra can generate a patched binary from its internal representation, but whether that binary will execute correctly will depend, among others, on the presence of anti-tampering protections such as remote attestation~\cite{viticchie2016reactive}. As was the case for effectiveness, also for stubbornness isolated and marginal stubbornness criteria can hence be considered.  



\subsection{Sensitivity}
\label{sec:sensitivity}
Section~\ref{sec:effectiveness} explicitly broadened the criterion of effectiveness from the impact observed one program sample regarding one property to effectiveness on a class of properties on many samples. For the other criteria, a similar broadening can, of course, be done as well. If an obfuscation's strength for any criterion is sensitive to certain features of the samples being protected, and the samples exhibit sufficient variance with respect to those features, this sensitivity will show up in the resulting distribution reported for that criterion. 

Regardless of whether or not such a distribution already provides evidence of such sensitivity, we encourage researchers to reflect on it explicitly. They can do so by providing evidence that the used samples cover a sufficient spread of the relevant features, or by arguing qualitatively how sensitive the evaluated obfuscations and the evaluation criteria are to certain program features. 

Such a reflection should not be limited to the features of the programs to be protected however. To the contrary, sensitivities to several confounding factors should be assessed:

\paragraph{$Se_s$ - Sample Feature Sensitivity} To what extent are the other criteria sensitive to features of the programs to be protected?

\paragraph{$Se_{a}$ - Attack Instantiation Sensitivity} To what extent are the evaluation results for other criteria sensitive to design and implementation details of the considered attack steps? An example of such details are the precise data flow analyses that are used to implement certain attack steps, or the specific tools that are used thereto. This criterion is particularly important when the evaluation relies on research prototypes of software analysis techniques. As we stated in Section~\ref{sec:buggy_tools}, it is not common for such tools to be incomplete or buggy. If the evaluation results depend on bugs or missing features, this clearly needs to be reported. 

\paragraph{$Se_{pi}$ - Protection Instantiation Sensitivity} To what extent are the evaluation results for other criteria sensitive to the implementation and design details of the (research) tools developed to deploy the obfuscation on samples? Obviously, it are not only analysis tools developed by researchers that can be incomplete, so can research SP prototypes; hence the inclusion of this criterion. This criterion is particularly relevant for research into deobfuscation and software analysis methods (including malware detection), in which researchers have to evaluate to what extent their novel techniques can defeat variations of obfuscations rather than being overfitted to particular variants.  

\paragraph{$Se_{pc}$ - Protection Configuration Sensitivity} To what extent are the evaluation results for other criteria sensitive to the configuration parameters of the deployed protections? Tigress~\cite{tigress2023}, for example, offers a wide range of configuration options for each supported obfuscation. It should at least be clear which configurations have been used in the evaluation. Of particular interest are random seeds used to drive stochastic SP techniques that many SP tools support. Such techniques are often used to make obfuscations renewable, meaning that they do look somewhat different every time they are deployed. If the evaluation results are sensitive to the used parameters or random seeds, i.e., if they display a large variability, the researchers ideally perform a parameter sweep and measure the resulting variance in the evaluation criteria. 

\paragraph{$Se_{b}$ - Build Tool Flow Sensitivity} To what extent are the evaluation results for other criteria sensitive to the build tools with which the protected application is being built? The interaction between SP transformations and other transformations applied during the build process of a protected application, such as compiler optimizations that are executed after source-to-source rewriting has been applied to inject obfuscations (e.g., with Tigress~\cite{tigress2023}) or after compile-time protection passes have been executed (e.g., with OLLVM~\cite{ollvm} or Epona~\cite{epona}) are complex~\cite{optimization}. Obfuscation transformations risk to be undone completely or partially by clever compiler optimizations, as discussed in Section~\ref{sec:sensitivity_missing}. The sensitivity of the evaluation results to such interactions hence needs to be assessed. 

\paragraph{$Se_{p}$ - Platform Sensitivity} To what extent are the results limited to or dependent on the platform for which the software is built, such as the operating system, the processor architecture, etc.

\vspace{0.2cm}

Most of these criteria can in theory be assessed by performing sweeps over sufficiently diverse samples, and sufficient configurations and implementation choices, of the attack steps and of the obfuscations. In practice, however, researchers most likely will not have the time and resources to experiment with all potentially relevant configurations and implementations. We therefore expect several aspects to be assessed qualitatively, and in ad hoc manners. 

Beyond helping to reduce the threats to validity, these sensitivity criteria are critical for the user-friendliness of the tools supporting the obfuscation. As argued in Section~\ref{sec:sensitivity_missing}, the lower the different forms of sensitivity of an obfuscation, the more predictable its strength will be for different programs, for future evolutions on the same program, and/or in light of (future) evolutions of program analysis techniques and attacker capabilities. 

Note that $Se_s$ and $Se_{pi}$ combined cover what can be called the applicability of an obfuscation. Many obfuscations can only be applied to certain types of software artifacts, or when certain preconditions are met. Some limitations of an obfuscation may be fundamental, in other cases a prototype instantiation has limited applicability because of a lack of resources to engineer a more complete implementation. That is fine, but the difference between fundamental limitations and instantiation should be made clear in any evaluation. The criteria $Se_s$ and $Se_{pi}$ serve that purpose.  

\subsection{Predictability}
The fundamental reason why low sensitivity to program features can be beneficial as argued above, is that it makes the result of an obfuscation more predictable. Higher predictability can benefit a SP optimization process for selecting the best combination of SPs given a program, its assets, and their security requirements. Similarly, if the impact on variations of attack steps can easily be predicted, it can become easier for decision support tools~\cite{checkmate24} and for predictive ML models~\cite{reganoMetric,2017_predicting_the_resilience_of_obfuscated_code_against_symbolic_execution_attacks_via_machine_learning} to predict the impact of obfuscations on a range of similar attack steps.  

For these reasons, we put forward complementary predictability criteria $P_s$, $P_{a}$, $P_{pi}$, $P_{pc}$, $P_{b}$, and $P_{p}$ that have similar definitions as their sensitivity counterparts, but for predictability instead of sensitivity. Similar to the sensitivity criteria, we expect these to be assessed mostly, if not completely, qualitatively. 
 
\subsection{Cost}
A detailed discussion of the various performance criteria that can be useful for evaluating an obfuscation is out of scope. They can include network bandwidth, client-side execution time and memory footprint, server-side execution time and memory footprint, power consumption, real-time behavior, response latency, throughput, and various other performance criteria. Which ones are relevant depends on the program to protect and its non-security-related non-functional requirements. We hence do not prescribe any specific criterion, but we do observe that in literature~\cite{desutter2024evaluation}, static program size and execution time are by far the most evaluated criteria, followed at a great distance by compilation time, dynamic memory footprint, and power consumption. 

We do stress, however, that when a specific performance criterion is relevant, it is important to evaluate its sensitivity as discussed in Section~\ref{sec:sensitivity}. In particular, the sensitivity to the hotness of protected program fragments should be assessed, i.e., to their execution frequency and relative contribution to the overall performance.

Moreover, the cost of using certain obfuscations can extend beyond the performance criteria. Various (expensive) processes in the whole software development life cycle (SDLC) can be impacted by the use of obfuscations, such as certification, quality assurance, debugging, distribution, etc. 

To capture both forms of costs, we put forward two broadly defined cost criteria:

\paragraph{$C_{p}$ - Performance Cost} How is the quantitative overhead of the protections in terms of all kinds of performance costs distributed? 

\paragraph{$C_{sdlc}$ - SDLC Cost} What are, qualitatively or quantitatively, the impacts that the deployment of the evaluation obfuscation can have on the SDLC? Where can the deployment of the obfuscation pose challenges or compatibility issues with industrial SDLCs?









\section{Discussion of Assumptions}\label{sec:discussion}
In this paper, we have made several assumptions for the sake of clarity and simplicity. In this section, we discuss the rationale behind these assumptions, the extent to which these assumptions hold in practice, and the consequences for our protocol when these assumptions hold.

\subsection{Assumptions on the Demand}

There are two simplifying assumptions we make about the demand. First, we assume the demand at any time is relatively small compared to the channel capacities. Second, we take the demand to be constant over time. We elaborate upon both these points below.

\paragraph{Small demands} The assumption that demands are small relative to channel capacities is made precise in \eqref{eq:large_capacity_assumption}. This assumption simplifies two major aspects of our protocol. First, it largely removes congestion from consideration. In \eqref{eq:primal_problem}, there is no constraint ensuring that total flow in both directions stays below capacity--this is always met. Consequently, there is no Lagrange multiplier for congestion and no congestion pricing; only imbalance penalties apply. In contrast, protocols in \cite{sivaraman2020high, varma2021throughput, wang2024fence} include congestion fees due to explicit congestion constraints. Second, the bound \eqref{eq:large_capacity_assumption} ensures that as long as channels remain balanced, the network can always meet demand, no matter how the demand is routed. Since channels can rebalance when necessary, they never drop transactions. This allows prices and flows to adjust as per the equations in \eqref{eq:algorithm}, which makes it easier to prove the protocol's convergence guarantees. This also preserves the key property that a channel's price remains proportional to net money flow through it.

In practice, payment channel networks are used most often for micro-payments, for which on-chain transactions are prohibitively expensive; large transactions typically take place directly on the blockchain. For example, according to \cite{river2023lightning}, the average channel capacity is roughly $0.1$ BTC ($5,000$ BTC distributed over $50,000$ channels), while the average transaction amount is less than $0.0004$ BTC ($44.7k$ satoshis). Thus, the small demand assumption is not too unrealistic. Additionally, the occasional large transaction can be treated as a sequence of smaller transactions by breaking it into packets and executing each packet serially (as done by \cite{sivaraman2020high}).
Lastly, a good path discovery process that favors large capacity channels over small capacity ones can help ensure that the bound in \eqref{eq:large_capacity_assumption} holds.

\paragraph{Constant demands} 
In this work, we assume that any transacting pair of nodes have a steady transaction demand between them (see Section \ref{sec:transaction_requests}). Making this assumption is necessary to obtain the kind of guarantees that we have presented in this paper. Unless the demand is steady, it is unreasonable to expect that the flows converge to a steady value. Weaker assumptions on the demand lead to weaker guarantees. For example, with the more general setting of stochastic, but i.i.d. demand between any two nodes, \cite{varma2021throughput} shows that the channel queue lengths are bounded in expectation. If the demand can be arbitrary, then it is very hard to get any meaningful performance guarantees; \cite{wang2024fence} shows that even for a single bidirectional channel, the competitive ratio is infinite. Indeed, because a PCN is a decentralized system and decisions must be made based on local information alone, it is difficult for the network to find the optimal detailed balance flow at every time step with a time-varying demand.  With a steady demand, the network can discover the optimal flows in a reasonably short time, as our work shows.

We view the constant demand assumption as an approximation for a more general demand process that could be piece-wise constant, stochastic, or both (see simulations in Figure \ref{fig:five_nodes_variable_demand}).
We believe it should be possible to merge ideas from our work and \cite{varma2021throughput} to provide guarantees in a setting with random demands with arbitrary means. We leave this for future work. In addition, our work suggests that a reasonable method of handling stochastic demands is to queue the transaction requests \textit{at the source node} itself. This queuing action should be viewed in conjunction with flow-control. Indeed, a temporarily high unidirectional demand would raise prices for the sender, incentivizing the sender to stop sending the transactions. If the sender queues the transactions, they can send them later when prices drop. This form of queuing does not require any overhaul of the basic PCN infrastructure and is therefore simpler to implement than per-channel queues as suggested by \cite{sivaraman2020high} and \cite{varma2021throughput}.

\subsection{The Incentive of Channels}
The actions of the channels as prescribed by the DEBT control protocol can be summarized as follows. Channels adjust their prices in proportion to the net flow through them. They rebalance themselves whenever necessary and execute any transaction request that has been made of them. We discuss both these aspects below.

\paragraph{On Prices}
In this work, the exclusive role of channel prices is to ensure that the flows through each channel remains balanced. In practice, it would be important to include other components in a channel's price/fee as well: a congestion price  and an incentive price. The congestion price, as suggested by \cite{varma2021throughput}, would depend on the total flow of transactions through the channel, and would incentivize nodes to balance the load over different paths. The incentive price, which is commonly used in practice \cite{river2023lightning}, is necessary to provide channels with an incentive to serve as an intermediary for different channels. In practice, we expect both these components to be smaller than the imbalance price. Consequently, we expect the behavior of our protocol to be similar to our theoretical results even with these additional prices.

A key aspect of our protocol is that channel fees are allowed to be negative. Although the original Lightning network whitepaper \cite{poon2016bitcoin} suggests that negative channel prices may be a good solution to promote rebalancing, the idea of negative prices in not very popular in the literature. To our knowledge, the only prior work with this feature is \cite{varma2021throughput}. Indeed, in papers such as \cite{van2021merchant} and \cite{wang2024fence}, the price function is explicitly modified such that the channel price is never negative. The results of our paper show the benefits of negative prices. For one, in steady state, equal flows in both directions ensure that a channel doesn't loose any money (the other price components mentioned above ensure that the channel will only gain money). More importantly, negative prices are important to ensure that the protocol selectively stifles acyclic flows while allowing circulations to flow. Indeed, in the example of Section \ref{sec:flow_control_example}, the flows between nodes $A$ and $C$ are left on only because the large positive price over one channel is canceled by the corresponding negative price over the other channel, leading to a net zero price.

Lastly, observe that in the DEBT control protocol, the price charged by a channel does not depend on its capacity. This is a natural consequence of the price being the Lagrange multiplier for the net-zero flow constraint, which also does not depend on the channel capacity. In contrast, in many other works, the imbalance price is normalized by the channel capacity \cite{ren2018optimal, lin2020funds, wang2024fence}; this is shown to work well in practice. The rationale for such a price structure is explained well in \cite{wang2024fence}, where this fee is derived with the aim of always maintaining some balance (liquidity) at each end of every channel. This is a reasonable aim if a channel is to never rebalance itself; the experiments of the aforementioned papers are conducted in such a regime. In this work, however, we allow the channels to rebalance themselves a few times in order to settle on a detailed balance flow. This is because our focus is on the long-term steady state performance of the protocol. This difference in perspective also shows up in how the price depends on the channel imbalance. \cite{lin2020funds} and \cite{wang2024fence} advocate for strictly convex prices whereas this work and \cite{varma2021throughput} propose linear prices.

\paragraph{On Rebalancing} 
Recall that the DEBT control protocol ensures that the flows in the network converge to a detailed balance flow, which can be sustained perpetually without any rebalancing. However, during the transient phase (before convergence), channels may have to perform on-chain rebalancing a few times. Since rebalancing is an expensive operation, it is worthwhile discussing methods by which channels can reduce the extent of rebalancing. One option for the channels to reduce the extent of rebalancing is to increase their capacity; however, this comes at the cost of locking in more capital. Each channel can decide for itself the optimum amount of capital to lock in. Another option, which we discuss in Section \ref{sec:five_node}, is for channels to increase the rate $\gamma$ at which they adjust prices. 

Ultimately, whether or not it is beneficial for a channel to rebalance depends on the time-horizon under consideration. Our protocol is based on the assumption that the demand remains steady for a long period of time. If this is indeed the case, it would be worthwhile for a channel to rebalance itself as it can make up this cost through the incentive fees gained from the flow of transactions through it in steady state. If a channel chooses not to rebalance itself, however, there is a risk of being trapped in a deadlock, which is suboptimal for not only the nodes but also the channel.

\section{Conclusion}
This work presents DEBT control: a protocol for payment channel networks that uses source routing and flow control based on channel prices. The protocol is derived by posing a network utility maximization problem and analyzing its dual minimization. It is shown that under steady demands, the protocol guides the network to an optimal, sustainable point. Simulations show its robustness to demand variations. The work demonstrates that simple protocols with strong theoretical guarantees are possible for PCNs and we hope it inspires further theoretical research in this direction.
%The advancement of artificial intelligence in the legal domain has led to the development of various tools that assist in legal research, document retrieval, and automated legal reasoning. Several studies have explored the use of Natural Language Processing (NLP)\cite{khurana2023natural}, machine learning models, and vector-based search mechanisms to enhance the efficiency of legal chatbots. The primary focus of this literature review is on retrieval-augmented generation (RAG) models, FAISS-based document retrieval, deep learning for legal applications, and the use of large language models (LLMs) in legal AI.  

Recent research on Retrieval-Augmented Generation (RAG)\cite{gao2023retrieval} for legal AI has demonstrated its potential in enhancing legal text retrieval and summarization. S. S. Manathunga, Y. and A. Illangasekara\cite{manathunga2023retrieval} proposed a RAG-based model that improves legal text summarization by dynamically fetching relevant documents before generating responses. Similarly, Lee and Ryu \cite{ryu-etal-2023-retrieval} explored the application of RAG in case law retrieval, demonstrating its superiority over traditional keyword-based search engines. The introduction of RAG has significantly improved response accuracy by grounding AI-generated text in authoritative legal documents, reducing hallucinations in AI-driven legal assistance.  

% \begin{figure}[h]
%     \centering
%     \includegraphics[width=8cm]{FAISS.png}
%     \caption{Faiss: Efficient Similarity Search and Clustering of Dense Vectors}
%     \label{Overall Result of comparing FAISS and Chroma with different number of top documents}
% \end{figure}

The efficiency of FAISS (Facebook AI Similarity Search) in legal document retrieval has also been widely studied. Zhao et al. \cite{devlin-etal-2019-bert} implemented FAISS to enhance large-scale legal question answering systems, achieving significant improvements in retrieval speed and relevance. N. Goyal and D. Chen \cite{inbook} demonstrated that FAISS-based vector search mechanisms outperform conventional database searches in legal information retrieval, reducing query response time while maintaining high accuracy. The integration of FAISS with transformer-based models, as seen in the work of Hsieh and Wu, further enhances semantic retrieval, ensuring that chatbot responses align with actual legal texts.  

Transformer-based models such as BERT and GPT-based architecture have also contributed to the evolution of AI-driven legal research. Devlin et al. introduced BERT (Bidirectional Encoder Representations from Transformers), which significantly improved the understanding of legal language. RoBERTa, an optimized version of BERT, was later developed by Liu et al. \cite{liu2019roberta} to enhance contextual understanding and document similarity matching in legal queries. These models have been integrated into legal chatbots for contract analysis and legal decision-making, as demonstrated in the studies of Li et al. and Jin and Liu, where fine-tuned transformers improved legal text comprehension and summarization.  
The role of deep learning in legal AI has also been investigated extensively. Radford et al. introduced GPT-3, which paved the way for legal AI assistants capable of generating human-like responses. However, researchers such as Firth and Lee emphasized the limitations of LLMs in legal reasoning, arguing that these models require external verification mechanisms to prevent misinformation. The use of contrastive learning and fine-tuning for legal text retrieval has been explored by Arabi and Akbari \cite{article}, who demonstrated that embedding-based retrieval significantly improves chatbot response accuracy.  

Another significant area of research involves evaluating AI-generated legal responses using automated metrics. Zhang and Wu introduced BLEU\cite{10.3115/1073083.1073135} and ROUGE\cite{lin-2004-rouge} scores as a means to evaluate AI-generated legal text summaries, ensuring their quality and relevance. Similarly, Zhao et al. \cite{yuan2024rag} examined the effectiveness of RAG-based models in handling complex legal queries, highlighting the importance of legal consistency scores (LCS) in evaluating AI-driven responses.  

The practical applications of legal AI chatbots have been studied extensively in the context of access to justice and AI ethics. Wang and Cheng et al. \cite{xue2024bias} highlighted the potential of AI-driven legal assistants in bridging the justice gap, particularly in countries where legal resources are not easily accessible. Chan conducted a systematic review of retrieval-based legal chatbots, noting that while these systems improve accessibility, they also raise ethical concerns regarding legal misinformation and bias. Research by Min \cite{Min2023ARTIFICIALIA} explored methods for bias detection and mitigation in legal AI, ensuring fairness in AI-generated legal advice.  

Comparative studies between rule-based legal bots, keyword-driven legal search engines, and AI-powered legal chatbots further illustrate the superiority of retrieval-augmented approaches. In a study conducted by Zeng \cite{zeng2024scalable}, FAISS-based retrieval mechanisms significantly outperformed traditional Boolean keyword searches, reducing irrelevant document retrieval by 40\%. Singh \cite{10760929} further demonstrated that AI-powered legal research tools using NLP provide faster and more contextually accurate responses compared to standard legal databases.  

Despite these advancements, challenges remain in AI-driven legal research. Existing chatbots still struggle with multi-jurisdictional legal queries, as noted by Weichbroth \cite{Weichbroth2025AIAT}, who emphasized the need for jurisdiction-aware legal AI models. Additionally, legal AI models often lack the ability to process long-context legal arguments effectively, a limitation discussed by Gupta, who proposed memory-based retrieval techniques to improve long-form legal text processing.  

Research continues to refine AI-driven legal assistance, particularly in retrieval-augmented generation, FAISS-based search, transformer models, and deep learning techniques for legal research. However, further improvements are needed in bias mitigation, jurisdiction-specific adaptations, and long-context legal understanding. Future developments in multilingual legal AI, enhanced retrieval mechanisms, and AI-powered contract analysis will be crucial in making legal AI tools more accessible, reliable, and widely applicable in legal practice.
\vspace{-0.2cm}
\section{Impact: Why Free Scientific Knowledge?}
\vspace{-0.1cm}

Historically, making knowledge widely available has driven transformative progress. Gutenberg’s printing press broke medieval monopolies on information, increasing literacy and contributing to the Renaissance and Scientific Revolution. In today's world, open source projects such as GNU/Linux and Wikipedia show that freely accessible and modifiable knowledge fosters innovation while ensuring creators are credited through copyleft licenses. These examples highlight a key idea: \textit{access to essential knowledge supports overall advancement.} 

This aligns with the arguments made by Prabhakaran et al. \cite{humanrightsbasedapproachresponsible}, who specifically highlight the \textbf{ human right to participate in scientific advancement} as enshrined in the Universal Declaration of Human Rights. They emphasize that this right underscores the importance of \textit{ equal access to the benefits of scientific progress for all}, a principle directly supported by our proposal for Knowledge Units. The UN Special Rapporteur on Cultural Rights further reinforces this, advocating for the expansion of copyright exceptions to broaden access to scientific knowledge as a crucial component of the right to science and culture \cite{scienceright}. 

However, current intellectual property regimes often create ``patently unfair" barriers to this knowledge, preventing innovation and access, especially in areas critical to human rights, as Hale compellingly argues \cite{patentlyunfair}. Finding a solution requires carefully balancing the imperative of open access with the legitimate rights of authors. As Austin and Ginsburg remind us, authors' rights are also human rights, necessitating robust protection \cite{authorhumanrights}. Shareable knowledge entities like Knowledge Units offer a potential mechanism to achieve this delicate balance in the scientific domain, enabling wider dissemination of research findings while respecting authors' fundamental rights.

\vspace{-0.2cm}
\subsection{Impact Across Sectors}

\textbf{Researchers:} Collaboration across different fields becomes easier when knowledge is shared openly. For instance, combining machine learning with biology or applying quantum principles to cryptography can lead to important breakthroughs. Removing copyright restrictions allows researchers to freely use data and methods, speeding up discoveries while respecting original contributions.

\textbf{Practitioners:} Professionals, especially in healthcare, benefit from immediate access to the latest research. Quick access to newer insights on the effectiveness of drugs, and alternative treatments speeds up adoption and awareness, potentially saving lives. Additionally, open knowledge helps developing countries gain access to health innovations.

\textbf{Education:} Education becomes more accessible when teachers use the latest research to create up-to-date curricula without prohibitive costs. Students can access high-quality research materials and use LM assistance to better understand complex topics, enhancing their learning experience and making high-quality education more accessible.

\textbf{Public Trust:} When information is transparent and accessible, the public can better understand and trust decision-making processes. Open access to government policies and industry practices allows people to review and verify information, helping to reduce misinformation. This transparency encourages critical thinking and builds trust in scientific and governmental institutions.

Overall, making scientific knowledge accessible supports global fairness. By viewing knowledge as a common resource rather than a product to be sold, we can speed up innovation, encourage critical thinking, and empower communities to address important challenges.

\vspace{-0.2cm}
\section{Open Problems}
\vspace{-0.1cm}

Moving forward, we identify key research directions to further exploit the potential of converting original texts into shareable knowledge entities such as demonstrated by the conversion into Knowledge Units in this work:


\textbf{1. Enhancing Factual Accuracy and Reliability:}  Refining KUs through cross-referencing with source texts and incorporating community-driven correction mechanisms, similar to Wikipedia, can minimize hallucinations and ensure the long-term accuracy of knowledge-based datasets at scale.

\textbf{2. Developing Applications for Education and Research:}  Using KU-based conversion for datasets to be employed in practical tools, such as search interfaces and learning platforms, can ensure rapid dissemination of any new knowledge into shareable downstream resources, significantly improving the accessibility, spread, and impact of KUs.

\textbf{3. Establishing Standards for Knowledge Interoperability and Reuse:}  Future research should focus on defining standardized formats for entities like KU and knowledge graph layouts \citep{lenat1990cyc}. These standards are essential to unlock seamless interoperability, facilitate reuse across diverse platforms, and foster a vibrant ecosystem of open scientific knowledge. 

\textbf{4. Interconnecting Shareable Knowledge for Scientific Workflow Assistance and Automation:} There might be further potential in constructing a semantic web that interconnects publicly shared knowledge, together with mechanisms that continually update and validate all shareable knowledge units. This can be starting point for a platform that uses all collected knowledge to assist scientific workflows, for instance by feeding such a semantic web into recently developed reasoning models equipped with retrieval augmented generation. Such assistance could assemble knowledge across multiple scientific papers, guiding scientists more efficiently through vast research landscapes. Given further progress in model capabilities, validation, self-repair and evolving new knowledge from already existing vast collection in the semantic web can lead to automation of scientific discovery, assuming that knowledge data in the semantic web can be freely shared.

We open-source our code and encourage collaboration to improve extraction pipelines, enhance Knowledge Unit capabilities, and expand coverage to additional fields.

\vspace{-0.2cm}
\section{Conclusion}
\vspace{-0.1cm}

In this paper, we highlight the potential of systematically separating factual scientific knowledge from protected artistic or stylistic expression. By representing scientific insights as structured facts and relationships, prototypes like Knowledge Units (KUs) offer a pathway to broaden access to scientific knowledge without infringing copyright, aligning with legal principles like German \S 24(1) UrhG and U.S. fair use standards. Extensive testing across a range of domains and models shows evidence that Knowledge Units (KUs) can feasibly retain core information. These findings offer a promising way forward for openly disseminating scientific information while respecting copyright constraints.

\section*{Author Contributions}

Christoph conceived the project and led organization. Christoph and Gollam led all the experiments. Nick and Huu led the legal aspects. Tawsif led the data collection. Ameya and Andreas led the manuscript writing. Ludwig, Sören, Robert, Jenia and Matthias provided feedback. advice and scientific supervision throughout the project. 

\section*{Acknowledgements}

The authors would like to thank (in alphabetical order): Sebastian Dziadzio, Kristof Meding, Tea Mustać, Shantanu Prabhat for insightful feedback and suggestions. Special thanks to Andrej Radonjic for help in scaling up data collection. GR and SA acknowledge financial support by the German Research Foundation (DFG) for the NFDI4DataScience Initiative (project number 460234259). AP and MB acknowledge financial support by the Federal Ministry of Education and Research (BMBF), FKZ: 011524085B and Open Philanthropy Foundation funded by the Good Ventures Foundation. AH acknowledges financial support by the Federal Ministry of Education and Research (BMBF), FKZ: 01IS24079A and the Carl Zeiss Foundation through the project "Certification and Foundations of Safe ML Systems" as well as the support from the International Max Planck Research School for Intelligent Systems (IMPRS-IS). JJ acknowledges funding by the Federal Ministry of Education and Research of Germany (BMBF) under grant no. 01IS22094B (WestAI - AI Service Center West), under grant no. 01IS24085C (OPENHAFM) and under the grant DE002571 (MINERVA), as well as co-funding by EU from EuroHPC Joint Undertaking programm under grant no. 101182737 (MINERVA) and from Digital Europe Programme under grant no. 101195233 (openEuroLLM) 

% The main focus of CheckMATE is on new models and techniques to defend software from tampering, reverse engineering, and piracy as well as to the development of new attack strategies that highlight the need of more complete defenses. We include both offensive and defensive techniques because of their close and intertwined relationship depending on the attack scenario. For instance, reverse engineering is defensive when the goal is to analyse obfuscated malware, but it is offensive when it is used to steal intellectual property and assets in legitimate software. Likewise, obfuscation is defensive when it aims for protecting a legitimate asset against reverse engineering, while it is offensive if it is used to hide that malware is embedded in an application. Both scenarios are of practical relevance, and therefore CheckMATE includes all attacks on/defenses of the confidentiality and integrity of software applications and assets embedded therein and exposed to MATE attacks. In such scenarios, attackers have full control over, and access to the hardware and/or software they are attacking in a controlled environment.
%
%
%
% Submission Guidelines
%
%     Papers must be submitted in a form suitable for anonymous review.
%
%     Papers must describe original work, be written and presented in English, and must not substantially overlap with papers that have been published or that are simultaneously submitted to a journal or a conference with refereed proceedings.
%
%     Submissions must be a PDF file in double-column ACM format (see https://www.acm.org/publications/proceedings-template, with a simpler version at https://github.com/acmccs/format).
%
%     Sumibssions may not exceed 12 pages long or 6 pages for short papers, excluding the bibliography, well-marked appendices, and supplementary material. Submissions are not required to reach the page limit. Note that reviewers are not required to read the appendices or any supplementary material. Authors should not change the font or the margins of the ACM format. Submissions not following the required format may be rejected without review.
%
%     One of the authors of the accepted paper is expected to present the paper in person at the workshop.


% \section{Authors and Affiliations}
%
% Each author must be defined separately for accurate metadata
% identification.  As an exception, multiple authors may share one
% affiliation. Authors' names should not be abbreviated; use full first
% names wherever possible. Include authors' e-mail addresses whenever
% possible.
%
% Grouping authors' names or e-mail addresses, or providing an ``e-mail
% alias,'' as shown below, is not acceptable:
% \begin{verbatim}
%   \author{Brooke Aster, David Mehldau}
%   \email{dave,judy,steve@university.edu}
%   \email{firstname.lastname@phillips.org}
% \end{verbatim}
%
% The \verb|authornote| and \verb|authornotemark| commands allow a note
% to apply to multiple authors --- for example, if the first two authors
% of an article contributed equally to the work.
%
% If your author list is lengthy, you must define a shortened version of
% the list of authors to be used in the page headers, to prevent
% overlapping text. The following command should be placed just after
% the last \verb|\author{}| definition:
% \begin{verbatim}
%   \renewcommand{\shortauthors}{McCartney, et al.}
% \end{verbatim}
% Omitting this command will force the use of a concatenated list of all
% of the authors' names, which may result in overlapping text in the
% page headers.

% \section{CCS Concepts and User-Defined Keywords}
%
% Two elements of the ``acmart'' document class provide powerful
% taxonomic tools for you to help readers find your work in an online
% search.
%
% The ACM Computing Classification System ---
% \url{https://www.acm.org/publications/class-2012} --- is a set of
% classifiers and concepts that describe the computing
% discipline. Authors can select entries from this classification
% system, via \url{https://dl.acm.org/ccs/ccs.cfm}, and generate the
% commands to be included in the \LaTeX\ source.
%
% User-defined keywords are a comma-separated list of words and phrases
% of the authors' choosing, providing a more flexible way of describing
% the research being presented.
%
% CCS concepts and user-defined keywords are required for for all
% articles over two pages in length, and are optional for one- and
% two-page articles (or abstracts).

%\section{Acknowledgments}
%This research was partly funded by the Cybersecurity Initiative Flanders (CIF) from the Flemish Government and by the Fund for Scientific Research - Flanders (FWO) [Project No. 3G0E2318].
%\begin{acks}
%  TODO
  %This research was partly funded by the Cybersecurity Initiative Flanders (CIF) from the Flemish Government and by the Fund for Scientific Research - Flanders (FWO) [Project No. 3G0E2318].
%\end{acks}
%% The acknowledgments section is defined using the "acks" environment
%% (and NOT an unnumbered section). This ensures the proper
%% identification of the section in the article metadata, and the
%% consistent spelling of the heading.
\begin{acks}
The ideas in this paper resulted from research supervised by the author and conducted in the author's team. This research was supported by the Cybersecurity Research Program Flanders, and by Research Foundation - Flanders (FWO) grant 3G0E2318. 
We thank Christian Collberg, Bart Coppens, Mila Dalla Preda, and Roberto Giacobazzi for their useful feedback on drafts of this paper. 
\end{acks}

%%
%% The next two lines define the bibliography style to be used, and
%% the bibliography file.
\bibliographystyle{ACM-Reference-Format}
\bibliography{references}

%\bds{alternative terms for potency: Power, Strength, Force, Might, Vigor, Energy, Robustness, Virility, Effectiveness, Efficacy, Influence, Authority, Control, Dominance, Capability, Capacity, Potential, Possibility}

%\bds{alternative terms for stealth: Secrecy, Seclusion, Concealment, Privacy, Covertness, Hiddenness, Slyness, Subtlety, Sneakiness, Craftiness, Cunning, Guile, Deceptiveness, Evasiveness, Quietness, Silence, Hush, Inaudibility, Stillness, Surreptitiousness, Furtiveness, Underhandedness, Invisibility, Obscurity}

%%
%% If your work has an appendix, this is the place to put it.
\appendix

\end{document}
\endinput
%%
%% End of file `sample-sigconf.tex'.
