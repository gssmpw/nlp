\clearpage
\setcounter{page}{1}
\maketitlesupplementary

% Adjust the ID for sections, tables, and figures
\appendix
\renewcommand{\thetable}{A\arabic{table}}
\renewcommand{\thefigure}{A\arabic{figure}}


% chexbert
\begin{table*}
\centering
\setlength{\tabcolsep}{1.4mm}
\begin{tabular}{ccccccccccccccc} 
\toprule
\multirow{3}{*}{\textbf{Observation}} & \multicolumn{7}{c}{\textbf{ MIMIC-CXR }} & \multicolumn{7}{c}{\textbf{ Two-view CXR }} \\ 
\cmidrule(r){2-8}\cmidrule(r){9-15}
 & \multirow{2}{*}{\textbf{ \%}} & \multicolumn{3}{c}{\textbf{SEI \cite{sei}}} & \multicolumn{3}{c}{\textbf{MLRG (Ours)}} & \multirow{2}{*}{\textbf{ \%}} & \multicolumn{3}{c}{\textbf{SEI \cite{sei}}} & \multicolumn{3}{c}{\textbf{MLRG (Ours)}} \\ 
\cmidrule(lr){3-5}\cmidrule(lr){6-8}\cmidrule(lr){10-12}\cmidrule(lr){13-15}
 &  & \textbf{P $\uparrow$} & \textbf{R $\uparrow$} & \textbf{F1 $\uparrow$} & \textbf{P $\uparrow$} & \textbf{R $\uparrow$} & \textbf{F1 $\uparrow$} &  & \textbf{P $\uparrow$} & \textbf{R $\uparrow$} & \textbf{F1 $\uparrow$} & \textbf{P $\uparrow$} & \textbf{R $\uparrow$} & \textbf{F1 $\uparrow$} \\ 
\midrule
ECM & 10.0 & \textbf{0.373} & 0.208 & 0.267 & 0.370 & \textbf{0.353} & \textbf{0.361} & 10.4 & 0.345 & 0.259 & 0.296 & \textbf{0.412} & \textbf{0.385} & \textbf{0.398} \\
Cardiomegaly & 14.8 & 0.599 & \textbf{0.633} & \textbf{0.616} & \textbf{0.629} & 0.570 & 0.598 & 14.4 & 0.578 & \textbf{0.602} & \textbf{0.589} & \textbf{0.627} & 0.550 & 0.586 \\
Lung Opacity & 13.8 & 0.519 & 0.170 & 0.256 & \textbf{0.594} & \textbf{0.317} & \textbf{0.413} & 13.6 & 0.526 & 0.197 & 0.287 & \textbf{0.549} & \textbf{0.295} & \textbf{0.384} \\
Lung Lesion & 2.5 & \textbf{0.462} & 0.021 & 0.041 & 0.429 & \textbf{0.046} & \textbf{0.082} & 3.0 & 0.179 & 0.030 & 0.051 & \textbf{0.297} & \textbf{0.045} & \textbf{0.078} \\
Edema & 8.3 & \textbf{0.526} & 0.361 & 0.428 & 0.516 & \textbf{0.448} & \textbf{0.480} & 6.5 & 0.420 & 0.368 & 0.392 & \textbf{0.457} & \textbf{0.455} & \textbf{0.456} \\
Consolidation & 3.3 & 0.218 & \textbf{0.194} & \textbf{0.205} & \textbf{0.259} & 0.150 & 0.190 & 3.1 & \textbf{0.296} & \textbf{0.192} & \textbf{0.233} & 0.204 & 0.115 & 0.147 \\
Pneumonia & 4.4 & 0.174 & 0.065 & 0.095 & \textbf{0.316} & \textbf{0.235} & \textbf{0.270} & 4.0 & 0.255 & 0.174 & 0.207 & \textbf{0.284} & \textbf{0.210} & \textbf{0.241} \\
Atelectasis & 10.9 & 0.469 & 0.395 & 0.429 & \textbf{0.499} & \textbf{0.475} & \textbf{0.487} & 10.0 & 0.457 & 0.425 & 0.440 & \textbf{0.496} & \textbf{0.444} & \textbf{0.469} \\
Pneumothorax & 1.0 & 0.174 & 0.039 & 0.064 & \textbf{0.426} & \textbf{0.230} & \textbf{0.299} & 0.7 & 0.417 & 0.109 & 0.172 & \textbf{0.457} & \textbf{0.291} & \textbf{0.356} \\
Pleural Effusion & 12.4 & 0.683 & \textbf{0.697} & \textbf{0.690} & \textbf{0.716} & 0.641 & 0.676 & 10.4 & 0.723 & \textbf{0.641} & \textbf{0.680} & \textbf{0.731} & 0.612 & 0.666 \\
Pleural Other & 1.6 & 0.167 & 0.022 & 0.039 & \textbf{0.231} & \textbf{0.054} & \textbf{0.087} & 1.9 & \textbf{0.250} & 0.071 & 0.111 & 0.194 & \textbf{0.083} & \textbf{0.116} \\
Fracture & 1.8 & 0.000 & 0.000 & 0.000 & \textbf{0.174} & \textbf{0.021} & \textbf{0.037} & 2.4 & 0.000 & 0.000 & 0.000 & \textbf{0.261} & \textbf{0.031} & \textbf{0.056} \\
Support Devices & 12.8 & 0.763 & 0.708 & 0.734 & \textbf{0.768} & \textbf{0.788} & \textbf{0.778} & 9.3 & \textbf{0.734} & 0.572 & 0.643 & 0.703 & \textbf{0.686} & \textbf{0.695} \\
No Finding & 2.4 & 0.161 & 0.597 & 0.253 & \textbf{0.233} & \textbf{0.629} & \textbf{0.340} & 10.3 & \textbf{0.509} & 0.899 & 0.650 & 0.490 & \textbf{0.933} & \textbf{0.643} \\ 
\cmidrule(lr){1-15}
micro avg & - & 0.523 & 0.410 & 0.460 & \textbf{0.549} & \textbf{0.468} & \textbf{0.505} & - & 0.522 & 0.447 & 0.481 & \textbf{0.532} & \textbf{0.474} & \textbf{0.501} \\
macro avg & - & 0.378 & 0.294 & 0.294 & \textbf{0.440} & \textbf{0.354} & \textbf{0.364} & - & 0.406 & 0.324 & 0.339 & \textbf{0.440} & \textbf{0.367} & \textbf{0.378} \\
\bottomrule
\end{tabular}
\caption{Clinical accuracy on the MIMIC-CXR and Two-view CXR datasets. ``ECM" refers to Enlarged Cardiomediastinum. ``P'', ``R'', and ``F1'' represent Precision, Recall, and F1 score, respectively.}
\label{atable:1}
\end{table*}

\begin{table*}
\centering
\begin{tabular}{cccccccc} 
\toprule
\multirow{2}{*}{\textbf{Observation}} & \multirow{2}{*}{\%} & \multicolumn{3}{c}{\textbf{SEI \cite{sei}}} & \multicolumn{3}{c}{\textbf{MLRG (Ours)}} \\ 
\cmidrule(r){3-5}\cmidrule(r){6-8}
 &  & \textbf{P $\uparrow$} & \textbf{R $\uparrow$} & \textbf{F1 $\uparrow$} & \textbf{P $\uparrow$} & \textbf{R $\uparrow$} & \textbf{F1 $\uparrow$} \\ 
\midrule
Enlarged Cardiomediastinum & 5.7 & 0.146 & 0.074 & 0.099 & \textbf{0.242} & \textbf{0.264} & \textbf{0.252} \\
Cardiomegaly & 12.7 & 0.515 & 0.627 & 0.566 & \textbf{0.547} & \textbf{0.785} & \textbf{0.644} \\
Lung Opacity & 20.2 & 0.640 & 0.342 & 0.446 & \textbf{0.649} & \textbf{0.512} & \textbf{0.572} \\
Lung Lesion & 5.0 & 0.333 & 0.035 & 0.063 & \textbf{0.357} & \textbf{0.052} & \textbf{0.090} \\
Edema & 7.1 & \textbf{0.464} & 0.524 & \textbf{0.492} & 0.441 & \textbf{0.547} & 0.489 \\
Consolidation & 3.3 & \textbf{0.359} & \textbf{0.383} & \textbf{0.371} & 0.354 & 0.270 & 0.306 \\
Pneumonia & 5.9 & 0.300 & 0.222 & 0.255 & \textbf{0.318} & \textbf{0.307} & \textbf{0.312} \\
Atelectasis & 10.5 & 0.381 & 0.441 & 0.409 & \textbf{0.445} & \textbf{0.578} & \textbf{0.503} \\
Pneumothorax & 0.0 & - & - & - & - & - & - \\
Pleural Effusion & 8.9 & 0.590 & 0.685 & 0.634 & \textbf{0.698} & \textbf{0.723} & \textbf{0.710} \\
Pleural Other & 3.2 & \textbf{0.158} & 0.056 & 0.082 & 0.135 & \textbf{0.081} & \textbf{0.101} \\
Fracture & 3.9 & 0.000 & 0.000 & 0.000 & 0.000 & 0.000 & 0.000 \\
Support Devices & 10.6 & 0.705 & 0.591 & 0.643 & \textbf{0.715} & \textbf{0.840} & \textbf{0.772} \\
No Finding & 3.0 & 0.175 & \textbf{0.540} & 0.265 & \textbf{0.262} & 0.466 & \textbf{0.335} \\
\cmidrule(lr){1-8}
micro avg & - & 0.466 & 0.408 & 0.435 & \textbf{0.513} & \textbf{0.517} & \textbf{0.515} \\
macro avg & - & 0.341 & 0.323 & 0.309 & \textbf{0.369} & \textbf{0.387} & \textbf{0.363} \\
\bottomrule
\end{tabular}
\caption{Clinical accuracy on the MIMIC-ABN dataset. ``P'', ``R'', and ``F1'' represent Precision, Recall, and F1 score, respectively.}
\label{atable:2}
\end{table*}

\begin{table*}
\centering
% \setlength{\tabcolsep}{1.8mm}
\begin{tabular}{ccccccccccc} 
\toprule
\multirow{2}{*}{\textbf{Generated Section(s)}} & \multicolumn{6}{c}{\textbf{NLG Metrics} $\uparrow$} & \multicolumn{4}{c}{\textbf{CE Metrics} $\uparrow$} \\ 
\cmidrule(r){2-7}\cmidrule(lr){8-11}
 & \textbf{B-1} & \textbf{B-2} & \textbf{B-3} & \textbf{B-4} & \textbf{MTR} & \textbf{R-L} & \textbf{RG} & \textbf{P} & \textbf{R} & \textbf{F1} \\ 
\midrule
FINDINGS & 0.411 & 0.277 & 0.204 & 0.158 & 0.176 & 0.320 & 0.291 & 0.549 & 0.468 & 0.505 \\
FINDINGS+IMPRESSION & 0.402 & 0270 & 0.197 & 0.152 & 0.172 & 0.327 & 0.289 & 0.558 & 0.468 & 0.509 \\
\bottomrule
\end{tabular}
\caption{Performance of generating ``FINDINGS" and ``IMPRESSION" sections on the MIMIC-CXR dataset.}
\label{atable:3}
\end{table*}


\begin{table*}
\centering
\setlength{\tabcolsep}{1.6mm}
\begin{tabular}{cccccccccc} 
\toprule
\multirow{2}{*}{\textbf{Method}} & \multicolumn{6}{c}{\textbf{\#Clinically Significant Errors $\downarrow$}} & \multirow{2}{*}{$\sum\nolimits_{j=(a)}^{(f)}{\rm{{\#Error}}}_{j}$ $\downarrow$} & \multirow{2}{*}{\textbf{\#Matched Findings $\uparrow$}} & \multirow{2}{*}{\textbf{GREEN $\uparrow$}} \\
\cmidrule(r){2-7}
& \textbf{(a)} & \textbf{(b)} & \textbf{(c)} & \textbf{(d)} & \textbf{(e)} & \textbf{(f)} & & &  \\ 
\midrule
R2Gen \cite{chen-etal-2020-generating} & 1.310 & 3.089 & \textbf{0.103} & \textbf{0.201} & \underline{0.082} & 0.142 & 4.926 & 1.803 & 0.283 \\
CMN \cite{chen-etal-2021-cross-modal} & 1.383 & 2.963 & \underline{0.127} & \underline{0.228} & \textbf{0.081} & 0.161 & 4.942 & 1.935 & 0.297 \\
CGPT2 \cite{nicolson-improving-cvt2distilgpt2} & \textbf{1.150} & 2.881 & 0.146 & 0.234 & 0.103 & 0.153 & \underline{4.666} & 1.967 & 0.313 \\
SEI \cite{sei} & 1.391 & \underline{2.610} & 0.154 & 0.273 & 0.108 & \underline{0.132} & 4.668 & \underline{2.101} & \underline{0.326} \\
MLRG (Ours) & \underline{1.277} & \textbf{2.469} & 0.199 & 0.284 & 0.091 & \textbf{0.130} & \textbf{4.451} & \textbf{2.261} & \textbf{0.353} \\
\bottomrule
\end{tabular}
\caption{Performance comparison of our MLRG and four baselines on the MIMIC-CXR test set in terms of ``\#Clinically Significant Errors" and ``\#Matched Findings". The best and second-best values are marked in \textbf{bold} and \underline{underlined}, respectively.}
\label{atable:4}
\end{table*}

\begin{figure*}
    \centering
    \includegraphics[width=1\linewidth]{figs/appendix-figure1.pdf}
    \caption{Examples of generated the ``FINDINGS" section on the MIMIC-CXR test set. Each ``A/B" cell corresponds to ``MLRG/SEI". Sentences from the reference report are highlighted in unique colors to clarify alignment with descriptions in the generated reports. Matching content in generated reports is shown in the same color. Correct temporal descriptions and failure descriptions of our MLRG are in \textbf{bold} and \underline{underlined}. ``Ind", ``PI", and ``PR" represent patient-specific indications, previous images, and previous reports, respectively.}
    \label{afig:1}
\end{figure*}


\begin{figure*}
    \centering
    \includegraphics[width=1\linewidth]{figs/appendix-figure2.pdf}
    \caption{Generated examples of ``FINDINGS" and ``IMPRESSION" sections on the MIMIC-CXR test set. Sentences from the reference report are highlighted in unique colors to clarify alignment with descriptions in the generated reports. Matching content in generated reports is shown in the same color. Failure descriptions of our MLRG are \underline{underlined}. ``Ind", ``PI", and ``PR" represent patient-specific indications, previous images, and previous reports, respectively.}
    \label{afig:2}
\end{figure*}


\begin{figure*}
    \centering
    \includegraphics[width=1\linewidth]{figs/appendix-figure3.pdf}
    \caption{An output result of the GREEN model \cite{2024-green} on the MIMIC-CXR test set. ``\#Clinically Significant Errors" and ``\#Matched Findings" represent the number of clinically significant errors and matched findings, respectively.}
    \label{afig:3}
\end{figure*}


% atable 2 chexbert
% atable 3: findings + impression
% figure 1: MLRG's generated examples include (patient-specific), prior-images, all-without
% figure 2: generated findings + impression
% figure 3: GREEN
% stable 4: only errors


\section{Experiments}
\subsection{Implementation Details}
\label{implementation-details}
\textbf{1) MIMIC-CXR \cite{johnson-mimic-cxr-jpg}}: In stage 1, the model is trained for 50 epochs with a learning rate of 5e-5 and a batch size of 32. In Stage 2, we train MLRG for another 50 epochs, using a batch size of 14. The learning rate is set to 5e-6 for parameters from Stage 1 and 5e-5 for the remaining parameters. \textbf{2) MIMIC-ABN \cite{mimic-abn-ori} and Two-view CXR \cite{mcl}}: Since most images are derived from the MIMIC-CXR dataset, we directly fine-tune the model from Stage 2 on MIMIC-CXR, using a learning rate of 5e-6 and a batch size of 12. \textbf{3) Common settings}: Early stopping with a patience of 15 is employed to prevent overfitting. The ReduceLROnPlateau scheduler and the AdamW optimizer are applied. The natural language generation (NLG) metrics are calculated with the pycocoevalcap\footnote{https://github.com/tylin/coco-caption}. For clinical efficacy (CE) metrics, Precision, Recall, and F1 score metrics are computed using the f1chexbert\footnote{https://pypi.org/project/f1chexbert/} library, and the F1 RadGraph metric is calculated with the radgraph\footnote{https://pypi.org/project/radgraph/} library.


\subsection{Clinical Accuracy of 14 Observations}
Tables \ref{atable:1} and \ref{atable:2} show the clinical accuracy of 14 observations annotated by CheXpert \cite{irvin-chexpert}  on the MIMIC-CXR, MIMIC-ABN, and Two-view CXR datasets. Results show that our MLRG outperforms SEI \cite{sei} on most observations. Even though MLRG is not specifically tailored for imbalanced observations, it still slightly surpasses the baseline on challenging observations like \textit{Pneumothorax} and \textit{Pleural Other}.

\subsection{Performance of Generating ``FINDINGS'' and ``IMPRESSION'' Sections}
Radiology reports typically consist of three key sections: ``INDICATION", which outlines the visit reasons or symptoms; ``FINDINGS", which details observations from current multi-view images and comparisons with the patient's medical history; and ``IMPRESSION", which summarizes the key conclusions or diagnostic interpretations based on the ``FINDINGS". Table \ref{atable:3} presents the performance of generating ``FINDINGS" and ``IMPRESSION" sections, with specific examples in Figure \ref{afig:2}. Since most existing methods focus primarily on generating the ``FINDINGS" section, peer methods are not included in Table \ref{atable:3}. The results indicate that our MLRG is capable of generating both sections with minor modifications. Specifically, we utilize special tokens, ``[FINDINGS]" and ``[IMPRESSION]", before the respective section content to distinguish between them. These sections are then combined to form the final reference reports, with all other settings remaining identical to those used for the ``FINDINGS" generation.

\subsection{Qualitative Analysis}
Figure \ref{afig:1} provides additional examples of the ``FINDINGS" section generated by SEI \cite{sei} and our MLRG, while Figure \ref{afig:2} presents examples of both the ``FINDINGS" and ``IMPRESSION" sections from MLRG. These results suggest that 1) Our MLRG is highly competitive in generating both ``FINDINGS" and ``IMPRESSION" sections, as well as the ``FINDINGS" section alone, for chest X-ray reports. 2) MLRG still has room for improvement in describing lesion attributes. For example, in Figure \ref{afig:1}, MLRG incorrectly describes the ``proximal parts of the stomach" as the ``middle parts". This occurs because MLRG has not fully learned region-level features. To improve this, we are exploring the use of saliency maps \cite{2024-saliency-map} to enhance regional feature learning and the accuracy of lesion descriptions.


\subsection{Evaluation Using Large Language Models}
\label{eva-large-language-models}
Inspired by \cite{yu2023evaluating}, the GREEN model \cite{2024-green} identifies six categories of clinical errors: (a) False report of a finding in the candidate; (b) Missing a finding present in the reference; (c) Misidentification of a finding's anatomic location/position; (d) Misassessment of the severity of a finding; (e) Mentioning a comparison that isn't in the reference; (f) Omitting a comparison detailing a change from a prior study. The GREEN score for the $i^{th}$ sample is defined as:
\begin{align}
{\rm{GREEN}}_i = \frac{{{\rm{\#Matched}}\;{\rm{Findings}}_i}}{{{\rm{\#Matched}}\;{\rm{Findings}}_i + \sum\nolimits_{j=(a)}^{(f)} {\rm{{\#Error}}}_{i,j} }},
\end{align}
where $\sum\nolimits_{j=(a)}^{(f)} {\rm{{\#Error}}}_{i,j}$ represents the total clinically significant errors for the $i^{th}$ sample across categories (a) to (f). ``\#Matched Findings" denotes the number of matched findings between generated and reference reports. Figure \ref{afig:3} illustrates the GREEN model's output on the MIMIC-CXR test set. Furthermore, we compare our MLRG with R2Gen \cite{chen-etal-2020-generating}, CMN \cite{chen-etal-2021-cross-modal}, CGPT2 \cite{nicolson-improving-cvt2distilgpt2}, and SEI \cite{sei} in terms of ``\#Clinically Significant Errors" and ``\#Matched Findings", as summarized in Table \ref{atable:4}. The results reveal the following: 1) Our MLRG achieves the highest ``\#Matched Findings" and GREEN score, with the fewest total clinically significant errors. This further confirms the effectiveness of our MLRG in generating clinically accurate radiology reports. 2) MLRG performs best in category “(f) Omitting a comparison detailing a change from a prior study”, suggesting its ability to effectively extract temporal features from multi-view longitudinal data. 3) Although MLRG shows higher error counts than the baselines in categories (c) and (d), its total clinically significant errors across categories (a) to (f) remain lower than those of all baselines. To improve the accuracy of severity assessments and anatomical location descriptions, we are exploring the integration of saliency maps \cite{2024-saliency-map} and MIMIC-CXR-VQA \cite{mimic-cxr-vqa-nips} data to learn region-based features, aiming to generate more precise descriptions of findings.













% \section{Rationale}
% \label{sec:rationale}
% % 
% Having the supplementary compiled together with the main paper means that:
% % 
% \begin{itemize}
% \item The supplementary can back-reference sections of the main paper, for example, we can refer to \cref{sec:intro};
% \item The main paper can forward reference sub-sections within the supplementary explicitly (e.g. referring to a particular experiment); 
% \item When submitted to arXiv, the supplementary will already included at the end of the paper.
% \end{itemize}
% % 
% To split the supplementary pages from the main paper, you can use \href{https://support.apple.com/en-ca/guide/preview/prvw11793/mac#:~:text=Delete%20a%20page%20from%20a,or%20choose%20Edit%20%3E%20Delete).}{Preview (on macOS)}, \href{https://www.adobe.com/acrobat/how-to/delete-pages-from-pdf.html#:~:text=Choose%20%E2%80%9CTools%E2%80%9D%20%3E%20%E2%80%9COrganize,or%20pages%20from%20the%20file.}{Adobe Acrobat} (on all OSs), as well as \href{https://superuser.com/questions/517986/is-it-possible-to-delete-some-pages-of-a-pdf-document}{command line tools}.


% % chexbert
% \begin{table*}
% \centering
% \setlength{\tabcolsep}{1.4mm}
% \begin{tabular}{ccccccccccccccc} 
% \toprule
% \multirow{3}{*}{\textbf{ Observation }} & \multicolumn{7}{c}{\textbf{ MIMIC-CXR }} & \multicolumn{7}{c}{\textbf{ Two-view CXR }} \\ 
% \cmidrule(r){2-8}\cmidrule(r){9-15}
%  & \multirow{2}{*}{\textbf{ \%}} & \multicolumn{3}{c}{\textbf{SEI }} & \multicolumn{3}{c}{\textbf{MLRG (Ours)}} & \multirow{2}{*}{\textbf{ \%}} & \multicolumn{3}{c}{\textbf{SEI }} & \multicolumn{3}{c}{\textbf{MLRG (Ours)}} \\ 
% \cmidrule(lr){3-5}\cmidrule(lr){6-8}\cmidrule(lr){10-12}\cmidrule(lr){13-15}
%  &  & \textbf{P} & \textbf{R} & \textbf{F1} & \textbf{P} & \textbf{R} & \textbf{F1} &  & \textbf{P} & \textbf{R} & \textbf{F1} & \textbf{P} & \textbf{R} & \textbf{F1} \\ 
% \midrule
% ECM & 10.0 & \textbf{0.373} & 0.208 & 0.267 & 0.370 & \textbf{0.353} & \textbf{0.361} & 10.4 & 0.345 & 0.259 & 0.296 & \textbf{0.412} & \textbf{0.385} & \textbf{0.398} \\
% Cardiomegaly & 14.8 & 0.599 & \textbf{0.633} & \textbf{0.616} & \textbf{0.629} & 0.570 & 0.598 & 14.4 & 0.578 & \textbf{0.602} & \textbf{0.589} & \textbf{0.627} & 0.550 & 0.586 \\
% Lung Opacity & 13.8 & 0.519 & 0.170 & 0.256 & \textbf{0.594} & \textbf{0.317} & \textbf{0.413} & 13.6 & 0.526 & 0.197 & 0.287 & \textbf{0.549} & \textbf{0.295} & \textbf{0.384} \\
% Lung Lesion & 2.5 & \textbf{0.462} & 0.021 & 0.041 & 0.429 & \textbf{0.046} & \textbf{0.082} & 3.0 & 0.179 & 0.030 & 0.051 & \textbf{0.297} & \textbf{0.045} & \textbf{0.078} \\
% Edema & 8.3 & \textbf{0.526} & 0.361 & 0.428 & 0.516 & \textbf{0.448} & \textbf{0.480} & 6.5 & 0.420 & 0.368 & 0.392 & \textbf{0.457} & \textbf{0.455} & \textbf{0.456} \\
% Consolidation & 3.3 & 0.218 & \textbf{0.194} & \textbf{0.205} & \textbf{0.259} & 0.150 & 0.190 & 3.1 & \textbf{0.296} & \textbf{0.192} & \textbf{0.233} & 0.204 & 0.115 & 0.147 \\
% Pneumonia & 4.4 & 0.174 & 0.065 & 0.095 & \textbf{0.316} & \textbf{0.235} & \textbf{0.270} & 4.0 & 0.255 & 0.174 & 0.207 & \textbf{0.284} & \textbf{0.210} & \textbf{0.241} \\
% Atelectasis & 10.9 & 0.469 & 0.395 & 0.429 & \textbf{0.499} & \textbf{0.475} & \textbf{0.487} & 10.0 & 0.457 & 0.425 & 0.440 & \textbf{0.496} & \textbf{0.444} & \textbf{0.469} \\
% Pneumothorax & 1.0 & 0.174 & 0.039 & 0.064 & \textbf{0.426} & \textbf{0.230} & \textbf{0.299} & 0.7 & 0.417 & 0.109 & 0.172 & \textbf{0.457} & \textbf{0.291} & \textbf{0.356} \\
% Pleural Effusion & 12.4 & 0.683 & \textbf{0.697} & \textbf{0.690} & \textbf{0.716} & 0.641 & 0.676 & 10.4 & 0.723 & \textbf{0.641} & \textbf{0.680} & \textbf{0.731} & 0.612 & 0.666 \\
% Pleural Other & 1.6 & 0.167 & 0.022 & 0.039 & \textbf{0.231} & \textbf{0.054} & \textbf{0.087} & 1.9 & \textbf{0.250} & 0.071 & 0.111 & 0.194 & \textbf{0.083} & \textbf{0.116} \\
% Fracture & 1.8 & 0.000 & 0.000 & 0.000 & \textbf{0.174} & \textbf{0.021} & \textbf{0.037} & 2.4 & 0.000 & 0.000 & 0.000 & \textbf{0.261} & \textbf{0.031} & \textbf{0.056} \\
% Support Devices & 12.8 & 0.763 & 0.708 & 0.734 & \textbf{0.768} & \textbf{0.788} & \textbf{0.778} & 9.3 & \textbf{0.734} & 0.572 & 0.643 & 0.703 & \textbf{0.686} & \textbf{0.695} \\
% No Finding & 2.4 & 0.161 & 0.597 & 0.253 & \textbf{0.233} & \textbf{0.629} & \textbf{0.340} & 10.3 & \textbf{0.509} & 0.899 & 0.650 & 0.490 & \textbf{0.933} & \textbf{0.643} \\ 
% \cmidrule(lr){1-15}
% micro avg & - & 0.523 & 0.410 & 0.460 & \textbf{0.549} & \textbf{0.468} & \textbf{0.505} & - & 0.522 & 0.447 & 0.481 & \textbf{0.532} & \textbf{0.474} & \textbf{0.501} \\
% macro avg & - & 0.378 & 0.294 & 0.294 & \textbf{0.440} & \textbf{0.354} & \textbf{0.364} & - & 0.406 & 0.324 & 0.339 & \textbf{0.440} & \textbf{0.367} & \textbf{0.378} \\
% \bottomrule
% \end{tabular}
% \caption{Clinical accuracy of our MLRG on the MIMIC-CXR and Two-view CXR datasets. "ECM" refers to Enlarged Cardiomediastinum.}
% \label{table:3}
% \end{table*}

% % table 1 dataset
% % table 2 breakdown/chexbert
% % 