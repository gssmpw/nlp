% This must be in the first 5 lines to tell arXiv to use pdfLaTeX, which is strongly recommended.
\pdfoutput=1
% In particular, the hyperref package requires pdfLaTeX in order to break URLs across lines.

\documentclass[11pt]{article}

% Change "review" to "final" to generate the final (sometimes called camera-ready) version.
% Change to "preprint" to generate a non-anonymous version with page numbers.
% \usepackage[review]{acl}
\usepackage[preprint]{acl}


% Standard package includes
\usepackage{times}
\usepackage{latexsym}

% For proper rendering and hyphenation of words containing Latin characters (including in bib files)
\usepackage[T1]{fontenc}
% For Vietnamese characters
% \usepackage[T5]{fontenc}
% See https://www.latex-project.org/help/documentation/encguide.pdf for other character sets

% This assumes your files are encoded as UTF8
\usepackage[utf8]{inputenc}

% This is not strictly necessary, and may be commented out,
% but it will improve the layout of the manuscript,
% and will typically save some space.
\usepackage{microtype}

% This is also not strictly necessary, and may be commented out.
% However, it will improve the aesthetics of text in
% the typewriter font.
\usepackage{inconsolata}

%Including images in your LaTeX document requires adding
%additional package(s)
\usepackage{graphicx}
\usepackage{float}
\usepackage{enumitem}

\usepackage{amsmath} % WHAT CONFERENCE DOESN"T HAVE THIS PREINCLUDED?????
% \usepackage{amsfonts}  % or \usepackage{amssymb}
\usepackage{bbm}

\usepackage[capitalize,noabbrev, nameinlink]{cleveref}


% If the title and author information does not fit in the area allocated, uncomment the following
%
%\setlength\titlebox{<dim>}
%
% and set <dim> to something 5cm or larger.

% \title{The Confidence-Accuracy Tradeoff in Test-Time Scaling}
% \title{The Confidence-Accuracy-Compute Tradeoff for Calibrating Test-Time Compute Scaling}
% \title{Scaling Test Time Compute Decreases Uncertainty}
% \title{Confidence Thresholding for Test-Time Compute Scaling}
% \title{Budget with Confidence, not Tokens}
%\title{Think When You Don't Know:\\ Test Time Scaling Improves Confidence of Correct Answers}
% \title{Think Further Before Answering:\\Test Time Scaling Improves Model Confidence}
% \title{Think When You Don't Know: Test Time Scaling Improves Confidence in Correct Answers}
\title{Is That Your Final Answer? Test-Time Scaling \\ Improves Selective Question Answering}

% Author information can be set in various styles:
% For several authors from the same institution:
% \author{Author 1 \and ... \and Author n \\
%         Address line \\ ... \\ Address line}
% if the names do not fit well on one line use
%         Author 1 \\ {\bf Author 2} \\ ... \\ {\bf Author n} \\
% For authors from different institutions:
% \author{Author 1 \\ Address line \\  ... \\ Address line
%         \And  ... \And
%         Author n \\ Address line \\ ... \\ Address line}
% To start a separate ``row'' of authors use \AND, as in
% \author{Author 1 \\ Address line \\  ... \\ Address line
%         \AND
%         Author 2 \\ Address line \\ ... \\ Address line \And
%         Author 3 \\ Address line \\ ... \\ Address line}

\author{
  William Jurayj \\
 \\
 \\
 \\ \And
  Jeffrey Cheng \\
  \\
  Johns Hopkins University \\
  \texttt{\{wjurayj1,jcheng71,vandurme\}@jhu.edu} \\\And
  Benjamin Van Durme \\
  \\
  \\
 \\
}

%\author{
%  \textbf{First Author\textsuperscript{1}},
%  \textbf{Second Author\textsuperscript{1,2}},
%  \textbf{Third T. Author\textsuperscript{1}},
%  \textbf{Fourth Author\textsuperscript{1}},
%\\
%  \textbf{Fifth Author\textsuperscript{1,2}},
%  \textbf{Sixth Author\textsuperscript{1}},
%  \textbf{Seventh Author\textsuperscript{1}},
%  \textbf{Eighth Author \textsuperscript{1,2,3,4}},
%\\
%  \textbf{Ninth Author\textsuperscript{1}},
%  \textbf{Tenth Author\textsuperscript{1}},
%  \textbf{Eleventh E. Author\textsuperscript{1,2,3,4,5}},
%  \textbf{Twelfth Author\textsuperscript{1}},
%\\
%  \textbf{Thirteenth Author\textsuperscript{3}},
%  \textbf{Fourteenth F. Author\textsuperscript{2,4}},
%  \textbf{Fifteenth Author\textsuperscript{1}},
%  \textbf{Sixteenth Author\textsuperscript{1}},
%\\
%  \textbf{Seventeenth S. Author\textsuperscript{4,5}},
%  \textbf{Eighteenth Author\textsuperscript{3,4}},
%  \textbf{Nineteenth N. Author\textsuperscript{2,5}},
%  \textbf{Twentieth Author\textsuperscript{1}}
%\\
%\\
%  \textsuperscript{1}Affiliation 1,
%  \textsuperscript{2}Affiliation 2,
%  \textsuperscript{3}Affiliation 3,
%  \textsuperscript{4}Affiliation 4,
%  \textsuperscript{5}Affiliation 5
%\\
%  \small{
%    \textbf{Correspondence:} \href{mailto:email@domain}{email@domain}
%  }
%}
\newcommand{\will}[1]{\textcolor{orange}{Will: #1}}
\newcommand{\jeff}[1]{\textcolor{magenta}{Jeff: #1}}
\newcommand{\ben}[1]{\textcolor{green}{Ben: #1}}


\makeatletter
\newcommand{\mysubsection}{\paragraph}
\newcommand{\myparagraph}{\noindent\textbf}

\makeatother

\begin{document}
\maketitle


\begin{abstract}

Scaling the test-time compute of large language models has demonstrated impressive performance on reasoning benchmarks. However, existing evaluations of test-time scaling make the strong assumption that a reasoning system should always give an answer to any question provided. This overlooks concerns about whether a model is \emph{confident} in its answer, and whether it is appropriate to always provide a response.
To address these concerns, we extract confidence scores during reasoning for thresholding model responses. We find that increasing compute budget at inference time not only helps models answer more questions correctly, but also increases confidence in correct responses. % Takeaway #1
We then extend the current paradigm of \textit{zero-risk} responses during evaluation by considering settings with non-zero levels of response risk, and suggest a recipe for reporting evaluations under these settings.

\end{abstract}

\begin{figure}[t]
    \centering
    \includegraphics[width=0.45\textwidth]{figures/4d_surface.png}
    \caption{\textbf{DeepSeek R1-32B's accuracy is a function of compute budget and confidence threshold.} Increased confidence thresholds generally yield increased accuracy at the cost of response rate, while increased compute budgets sometimes decrease accuracy while increasing response rate. The vertical axis measures the accuracy of answered questions at a compute budget and confidence threshold. Color indicates the proportion of questions that are answered; in redder regions, the model is more likely to answer, whereas in bluer regions the model is less likely to answer. We treat the case where the model never answers as accuracy 0.}
    \label{fig:4d-surface}
\end{figure}

\begin{figure*}[t]
    \centering
    \includegraphics[width=\textwidth]{figures/threshold.png}
    \caption{\textbf{Confidence thresholds on test-time scaling.} \textit{(left)} When the logit threshold is 0, the model answers 100\% of questions. This is the only performance curve that is reported by test-time scaling research. \textit{(center)} At a moderate threshold, more frequent absentions allow higher response accuracy. \textit{(right)} At a high threshold, small amounts of test-time compute deliver very high accuracy, while test-time scaling provides more answers at the cost of answer accuracy. We treat the decision to never answer as yielding accuracy 0.
    }
    \label{fig:threshold}
\end{figure*}


\section{Introduction}
Scaling up language model inference-time compute using lengthy chains of thought has delivered impressive results on mathematical reasoning benchmarks that resisted training compute scaling \citep{deepseek-ai_deepseek-r1_2025, muennighoff_s1_2025}. These results, however, are reported in the zero-risk response setting: with no penalties for incorrect answers, the system always guesses even when it is not confident in its answer. In practice, this behavior is not always desirable.

Many question answering settings associate incorrect answers with measurable costs, ranging from low-risk responses found in game shows \citep{ferrucci_building_2010} to high-stakes responses that can alter people's lives \citep{COMPAS}. \textit{Selective question answering} addresses these challenges by allowing a model to refrain from answering questions which it might answer incorrectly \cite{kamath-etal-2020-selective}.
This requires a selection function, which considers risk tolerance, coverage goals, and candidate answer confidence to decide whether a prediction should be given \cite{geifman_selective_2017}. 
Knowing when not to answer is a critical for systems to collaborate effectively with humans \cite{verma_learning_2023}, especially for test-time scaling systems that must constantly decide between refusing to answer and expending further compute to search for a possible solution.

To help address this issue, we evaluate test-time scaling models using a simple class of selection functions that reject questions if a model is not confident in its answer after expending its compute budget. We evaluate these systems at different compute budgets, showing a new aspect of model performance that answer accuracy alone struggles to measure. We suggest a class of utility functions that represent various levels of error risk to empirically measure the performance of these systems in settings where incorrect answers are penalized. Evaluation in these settings shows how compute scaling affects confidence in existing systems. Based on these insights, we propose a standard method for measuring model performance in settings with non-zero response risk.
In summary we:
\begin{itemize}[nosep]
    \item Conduct the first evaluation of LLM test-time compute scaling on selective question answering, finding that increasing inference compute can help models distinguish between their correct and incorrect answers. (\cref{sec:experiments})
    \item Introduce evaluation settings that penalize incorrect answers and allow abstentions to help holistically models capable of scaling test-time compute. (\cref{sec:utility})
    \item Invite the community to report test-time scaling performance on selective question answering under ``Jeopardy Odds'', which incentivize confidence calibration by penalizing incorrect answers while rewarding correct answers.
\end{itemize}

\section{Methods}
We explore how increasing compute budgets affects a model's performance on QA tasks at different confidence thresholds. The choice of a budget and threshold is a \emph{test-time} decision. We describe methods to quantify the two factors below:

\myparagraph{Compute Budget} refers to the amount of compute expended by the model at inference time. In all cases, we quantify a model's budget by counting the \textit{number of tokens} in its reasoning trace. We use methods proposed by \citet{muennighoff_s1_2025} to strictly enforce compute budgets. Specifically, we ignore any predicted end-of-thinking delimiters and instead append the token ``Wait'' if a model attempts to end its reasoning trace before reaching the budget, and we force decode the end-of-thinking delimiter once the budget is reached.

\myparagraph{Confidence Threshold} refers to the uncertainty of the model in its decoded answer. We quantify a model's confidence as the \textit{sum of the log-probabilities} corresponding to the answer tokens.\footnote{Every answer in our dataset is a 3-digit number between 000 and 999\label{sum}, so consists of the same number of tokens.} For a confidence threshold, our selection function \cite{geifman_selective_2017} only accepts answers that the model delivers with confidence greater than its threshold, abstaining otherwise.


\section{Experiments}
\label{sec:experiments}


\mysubsection{Experimental Setup}

\label{sec:setup}
We evaluate Deepseek-R1-32B \citep{deepseek-ai_deepseek-r1_2025} and s1 \citep{muennighoff_s1_2025} due to their exhibited test-time scaling capabilities and open-weight checkpoints, and choose AIME24 as our evaluation dataset. The dataset contains 30 hard math problems on which performance substantially benefits from larger compute budgets, making it a popular benchmark for evaluating test-time scaling. We test the set of confidence thresholds $\{0.0, 0.5, 0.95\}$ across compute budgets within the range $[500, 8000]$, incrementing by 100 tokens. For a given budget and threshold, we report the accuracy of \emph{answered} questions. As the number of answered questions differs across confidence thresholds, we note that accuracies are \emph{not directly comparable} between models and compute budgets.
\begin{figure}[H]
    \centering
    \includegraphics[width=0.48\textwidth]{figures/confidence.png}
    \caption{\textbf{Test-time scaling improves confidence in correct answers.} Each dot represents R1 32B's confidence in an answer after spending a fixed amount of compute. Indigo series are correct answers, while orange series are incorrect. Note that individual dots may turn from orange to indigo if the model changes its prediction after thinking longer. See \cref{fig:confidence_s1} in \cref{sec:appendix_figures} for s1-32B.}
    \label{fig:confidence}
\end{figure}

\mysubsection{Results}
\label{sec:results}
\cref{fig:threshold} compares the accuracy of answers provided by R1-32B and s1-32B at different test-time compute budgets. When the confidence threshold is 0, models answer every question, so accuracy increases consistently with compute budget. We observe that these subplots are slices of a surface parameterized by compute budget and confidence threshold, shown in \cref{fig:4d-surface}. While higher confidence thresholds prevent the model from answering at low budgets, scaling compute at high thresholds delivers a larger volume of accurate answers.
At higher confidence thresholds, increased compute budget can actually decrease answer accuracy. This decrease in accuracy of yielded answers does not necessarily reflect decreased performance at higher budgets, but instead the increase in the total number of questions answered.

To investigate whether excessive thinking harms accuracy drops by pushing models to abandon correct answers, we plot how a model's confidence in individual answers moves over time. \cref{fig:confidence} shows the answer confidences given by R1-32B at varying compute budgets, colored according to their correctness, with a curve fit to the distribution. We note that as compute budget increases, the average confidence of its correct answers increases even as additional correct answers are discovered. 

\section{Utility}
\label{sec:utility}

\mysubsection{Motivation}
\label{sec:motivation}
When refusal to answer is an option, accuracy can be trivially optimized by a system that answers extremely infrequently. Thus, a useful metric must capture both the accuracy of answers provided and the system's propensity to provide answers. Many real world scenarios reward correct answers, but incur measurable costs for incorrect answers. We show our results involving confidence thresholds can be adapted to these settings:
Given a model $\mathcal{M}$ and an instance $x$ of a task $t$, we define a \textit{utility function} $f$ to be 
$f(\mathcal{M}, x) = \mathbbm{1}_c + 0\mathbbm{1}_a + r_t \mathbbm{1}_i$ where $(\mathbbm{1}_c, \mathbbm{1}_a, \mathbbm{1}_i)$ are mutually exclusive indicators for the predicted answer being (correct, abstained, incorrect) and $r_t$ is some task-specific cost of incorrect answers.
We can assume the reward for correct answers is 1 without loss of generality due to scaling. While there exist scenarios where refusing to answer also incurs a cost, this paper will only discuss the consequences when no extra cost is incurred; the conclusions we draw can be extended to these cases.

\mysubsection{Problem Scenarios}
\label{sec:scenarios}
We discuss three settings with varying risk levels: 
\begin{itemize}[nosep]
    \item Exam Odds ($r_t = 0$): There are no costs incurred by incorrect answers. These are tasks where guessing isn't punished and the model should always try to provide a solution.
    \item Jeopardy Odds\footnote{Inspired by the wagers made in the `Final Jeopardy' stage} ($r_t = -1$): The cost of an incorrect answer is equal to the reward for a correct answer. In these scenarios, no answer at all is preferable to an incorrect answer.
    \item High-Stakes Odds ($r_t = -20$): The cost of an incorrect answer far outweighs the reward for a correct answer. In this case, the model should answer only if absolutely certain.
\end{itemize}

\begin{figure}[ht]
    \centering
    \includegraphics[width=0.5\textwidth]{figures/jeopardy_surface.png}
    \caption{\textbf{Utility Surface of DeepSeek R1-32B for Jeopardy.} The vertical axis indicates performance in the Jeopardy setting at different compute budgets and confidence thresholds. The color indicates the proportion of questions that are answered, as in \cref{fig:4d-surface}. The horizontal plane divides positive and negative utility regions of the operating curve. The checkered lines show the confidence slices that we compare to s1 in \cref{fig:jeopardy}.}
    \label{fig:jeopardy-surface}
\end{figure}

\begin{figure}[hb]
    \centering
    \includegraphics[width=0.5\textwidth]{figures/jeopardy.png}
    \caption{\textbf{Jeopardy utility scales differently across models and thresholds.} Performance of s1-32B and R1-32B in the Jeopardy odds setting under different confidence thresholds. While s1 is competitive in the case when threshold is 0, a higher threshold shows R1’s superior scaling performance.}
    \label{fig:jeopardy}
\end{figure}

\mysubsection{Results}
\label{sec:results_odds}
We keep the same experimental setup as described in \cref{sec:setup}. Rather than reporting accuracy, we instead report the utility in the three scenarios above, shown in \cref{fig:jeopardy-surface}. We focus on the Jeopardy setting because it highlights why a system might choose not to answer; results in the other settings are in \cref{sec:appendix_figures}.

The Exam setting's utility function does not distinguish refusal from incorrectness, so optimal performance is achieved trivially at confidence threshold to 0 so that every question gets the model's best guess. In the Jeopardy setting, however, this is non-trivial. We illustrate the complete function mapping compute budget and confidence threshold to Jeopardy performance in \cref{fig:jeopardy-surface}: the checkered lines on this surface indicate the two slices that compose R1-32B's portion of \cref{fig:jeopardy}. 
We do not suggest that our choice of 0.95 is the optimal threshold for this task, or even that a threshold is the right approach to confidence calibration. Rather, we apply this naive method to show how test-time scaling for selective classification can benefit a practical question-answering setting.

We see on the left of \cref{fig:threshold} that in the commonly reported Exam Odds, R1-32B and s1-32B scale comparably at threshold 0.
In Jeopardy Odds, selective question answering at threshold 0.95 dramatically improves performance for both models. Additionally, although the two models scale comparably at Exam Odds, R1-32B substantially outperforms s1-32B at larger budgets in this new evaluation setting. Previous work overlooks this comparison.
We call on future test-time compute scaling research to report optimal utility at Jeopardy Odds in addition to Exam Odds, to help readers understand performance across confidence demands.


\section{Related Work}

As scaling training compute has become prohibitively expensive \cite{hoffmann_training_2022}, models that scale performance with test-time compute have become a new frontier \cite{snell2024scalingllmtesttimecompute, wu_inference_2024}. These methods have delivered state-of-the-art results on hard reasoning tasks using lengthy chains of thought \cite{deepseek-ai_deepseek-r1_2025, muennighoff_s1_2025}. Current work in this space optimizes for question answering tasks which do not penalize incorrectness, ignoring settings that favor refusal over wrong answers \cite{ferrucci_building_2010, rajpurkar_know_2018, kamath-etal-2020-selective}.
We draw motivation from methods for cost-sensitive learning \cite{mienye_performance_2021} and selective classification \cite{geifman_selective_2017}, which navigate penalties for failure. These setting reward confidence calibration, which can be critical for effective collaboration with human experts \cite{verma_learning_2023}. We are the first to investigate how serialized test-time compute helps models identify when they should not answer.
We include additional materials in \cref{sec:appendix} for further reading on these areas.

\section{Conclusion}

We highlight a region of performance that is currently unexplored by test-time scaling research.
We encourage the test-time scaling community to adopt these insights by reporting model scaling performance on benchmarks at both Exam Odds and Jeopardy Odds, to highlight their systems ability to scale confidence with test-time compute. Future work should focus on efficiently allocating test-time compute to meet confidence demands, and could investigate how test-time confidence scaling models should decide between extending reasoning and deferring to human experts.


\section*{Limitations}

The selection function we implement is based entirely on the likelihood that a large language model assigns a series of tokens after thinking, which is not necessarily the optimal method for model confidence estimation.
We choose the AIME dataset to demonstrate the value of our  proof  because it displays the most consistent performance improvements from test-time scaling \cite{muennighoff_s1_2025};
We evaluate a fixed set of three odds levels, but recognize that this does not cover the full space of real-world utility functions.
Furthermore, we do not consider how compute costs might be incorporated in the model's utility function, which could encourage increased energy consumption. 
Finally, we recognize that by evaluating only on English mathematics questions and answers, we may miss model capabilities or weaknesses in lower-resource languages.

% Bibliography entries for the entire Anthology, followed by custom entries
%\bibliography{anthology,custom}
% Custom bibliography entries only
\bibliography{acl_latex}

\appendix

\section{Appendix}
\label{sec:appendix}

\subsection{Supplementary Background}
\mysubsection{Test-time Scaling}
Many methods for scaling test-time comptue have been explored. These include searching over possible generations \citep{wang2024hypothesissearchinductivereasoning, snell2024scalingllmtesttimecompute}, tool use \citep{qin2024toolllm, Qu_2025}, making gradient updates at inference time \cite{akyurek_surprising_2024, li_combining_2024}, and simply fine-tuning on longer chains of thought \citep{deepseek-ai_deepseek-r1_2025, muennighoff_s1_2025}.
Our work considers models fine-tuned on extremely long reasoning chains, and augments them with the ability to refuse to answer questions where they lack confidence. Concurrent work fine-tunes similar models to consider the difficulty of a question at when allocating compute test-time, but do not report the effect their method has on their models' confidence \cite{arora_training_2025}. In contrast, we show how model confidence scales with test-time compute, and demonstrate its value for optimizing performance in settings that allow refusal to answer.

\mysubsection{Answer Refusal}
Refusing to answer is an important choice in many prior works on question answering.  SQuAD 2.0 included this feature by asking questions which have no answer, although they treat abstaining from answering as a correct answer in these unanswerable cases \cite{rajpurkar_know_2018}. Game-show based research efforts use an approach more closely aligned with ours, which penalizes systems for answering incorrectly to encourage abstentions when a system cannot develop sufficiently high confidence \cite{ferrucci_building_2010, ferrucci_introduction_2012, boyd-graber-borschinger-2020-question, rodriguez_quizbowl_2021}. Related to these quiz game based setting is research into selective classification, which evaluates models performance across the coverage-accuracy curve, rather than at single point \cite{geifman_selective_2017}. These approaches can be useful for avoiding costly errors in high-pressure domains \cite{khan_cost_2017} or under distribution shift \cite{ren_out--distribution_2023}, or when designing systems that defer to expert humans when it lacks confidence that its input will be helpful \cite{mozannar_consistent_2021}.
Related research in language modeling has focused on calibrating prediction confidence to align with human judgments \cite{jiang_addressing_2024}. Most recently, research has investigated training language models to refuse to answer \cite{cao_learn_2024}. However, this does not ask how this behavior should be measured effectively in sequential test-time compute models, which might find a confident answer given higher compute budgets.

\subsection{Implementation Details}
We use widely available open-source libraries to run our experiments, including HuggingFace Transformers \cite{wolf-etal-2020-transformers} and vLLM \cite{kwon2023efficient} for language model inference, and the Language Model Evaluation Harness
\cite{eval-harness} to sample reasoning chains for the AIME problems at temperature 0. In particular, we use the variant of this library released by \citet{muennighoff_s1_2025}, and run a subset of the experiments that they run.
We run experiments on a node with 4 H100 GPUs for approximately 4 hours.


\section{Supplemental Figures}
\label{sec:appendix_figures}



\begin{figure}[H]
    \centering
    \includegraphics[width=0.45\textwidth]{figures/4d_surface_s1.png}
    \caption{\textbf{s1-32B's answer accuracy is a function of compute budget and confidence threshold.} Increased confidence thresholds generally yield increased accuracy at the cost of response rate, while increased compute budgets sometimes decrease accuracy while increasing response rate. The vertical axis indicates the accuracy for answered questions at a compute budget and logit threshold. The color indicates the proportion of questions that are answered; in redder regions, the model is more likely to answer, whereas in bluer regions the model is less likely to answer. We treat the decision to never answer as accuracy 0.}
    \label{fig:4d-surface-s1}
\end{figure}


\begin{figure}[t]
    \centering
    \includegraphics[width=0.5\textwidth]{figures/confidence_s1.png}
    \caption{\textbf{Test-time scaling improves confidence in correct answers.} Each dot represents s1-32B's confidence in an answer after spending a fixed amount of compute. Indigo series indicate correct answers, while orange series are incorrect. Note that individual dots may switch colors if the model changes its prediction after thinking longer. s1-32B does not separate its correct answers from its incorrect answers as effectively as R1-32B. See \cref{fig:confidence} for R1-32B.}
    \label{fig:confidence_s1}
\end{figure}


\begin{figure}
    \centering
    \includegraphics[width=0.45\textwidth]{figures/jeopardy_surface_s1.png}
    \caption{\textbf{Utility Surface of s1-32B for Jeopardy.} The vertical axis indicates performance in the Jeopardy setting at different compute budgets and confidence thresholds. The color indicates the proportion of questions that are answered; in redder regions, the model is more likely answer, whereas in bluer regions the model is less likely to answer. The horizontal plane divides positive and negative utility regions of the operating curve. The checkered lines indicate the threshold slices that we compare against R1-32B in \cref{fig:jeopardy}. Note the relatively lower volume above the break-even point of 0 corresponds to s1-32B's broadly inferior performance at these odds.}
    \label{fig:jeopardy-surface-s1}
\end{figure}

\begin{figure}
    \centering
    \includegraphics[width=0.5\textwidth]{figures/exam.png}
    \caption{\textbf{Additional Model Comparisons.} We additionally compare performance of s1-32B and R1-32B in the Exam Odds \textit{(above)} and High-Stakes Odds \textit{(below)} settings under different confidence thresholds. Like Jeopardy odds depicted in \cref{fig:jeopardy}, High-Stakes Odds illustrates a performance distinction at high confidence thresholds that is not evident from conventional Exam odds.}
    \includegraphics[width=0.5\textwidth]{figures/highstakes.png}
    \label{fig:odds}
\end{figure}


\end{document}




