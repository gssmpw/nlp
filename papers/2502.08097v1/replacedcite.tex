\section{Related Work}
\subsection{Personalized T2I Generation}
With the advance of diffusion models, current text-to-image (T2I) generation____ has shown remarkable generalization ability. 
As these methods ignore the concepts that do not appear in the training set, some works study personalized text-to-image generation which aims to adapt text-to-image models to specific concepts (attributions, styles, or objects) given several reference images.
For example, Textual Inversion____ adjusts text embeddings of a new pseudoword to describe the concept. DreamBooth____ fine-tunes denoising networks to connect the novel concept and a less commonly used word token. 
Based on that, several recent works____ have been proposed to enhance controllability and flexibility when processing image visual concepts. 
These advancements enhance the capabilities of text-to-image models, making them more accessible to a wider range of users.


\subsection{Adversarial Examples}
Adversarial examples are crafted by adding imperceptible perturbations to mislead models, primarily applied in anti-classification, anti-deepfakes, and anti-facial recognition. Existing methods fall into two categories:  
\textbf{Image-specific adversarial examples} generate tailored perturbations per image. ____ pioneered this concept with LBFGS optimization, while ____ proposed the efficient FGSM. Subsequent works____ improved naturalness via generative models. These are extended to disrupt deepfakes____ and protect facial privacy____ from unauthorized face recognition systems.  
\textbf{Universal adversarial perturbations (UAPs)} apply a single perturbation to all images. ____ first revealed UAPs' existence, with ____ addressing gradient vanishing via aggregation and ____ synthesizing UAPs via generative models. 
For privacy, ____ proposed gradient-based OPOM for identity-specific protection, while ____ trained generators for natural adversarial cloaks. Our work aligns with UAPs but focuses on protecting all images of a target identity from unauthorized personalized generation.

 

\subsection{Anti-Personalization}
The remarkable generative capability of personalized T2I generation comes with safety concerns____, particularly regarding the unauthorized exploitation of personal images. 
To mitigate these risks, recent studies have proposed the use of adversarial examples to counteract such safety issues. 
AdvDM ____ pioneered a theoretical framework for crafting adversarial examples against diffusion models. 
Anti-DreamBooth ____ tackled anti-personalization with a bi-level protection objective and ASPL optimization, later refined by ____ via time-step selection. 
MetaCloak ____ enhanced cloak robustness against image transformations using ensemble learning and EoT, while ____ reduced computational costs via SDS loss. 
____ and ____ addressed prompt discrepancies between protectors and attackers with encoder-based protection and prompt distribution modeling, respectively.
Despite these advancements, existing methods predominantly generate image-specific cloaks, which are impractical for widespread user adoption. 
In contrast, our work introduces a universal cloak tailored to individual users, enabling it to be applied across all their images, significantly enhancing usability and reducing privacy risks.