\section{Related Works}
\vspace{-1mm}
\subsection{Human Body restoration}
\vspace{-1mm}
Human body restoration could benefit both the fashion industry for display and the photography and camera producers. The purpose is evident, focusing on the human body and making it look better. Compared to general image super-resolution (SR) for arbitrary objects, human body image SR is more constrained, allowing for the use of a variety of prior knowledge. Firstly, current research on image segmentation has made significant progress in generating segmentation masks for different body parts, such as hair and arms, which provide a clear description of the human body shape. This knowledge is beneficial for handling the boundaries in the human body SR tasks. Since the human body composition is relatively fixed and limited, it is easier to estimate compared to the random and arbitrary objects in natural images. 

Lots of research has been done recently. \cite{Liu2021HumanBodySR} represent the texture details of the human body using subbands of the non-subsampled shearlet transform, while \cite{Wang2024PRCN} employ a pyramid residual network to estimate texture and shape priors, improving the image quality of the human body. \cite{Zhang2024DiffBody}, as the first to apply diffusion models for enhancing body-region images, use pose-attention guidance, text guidance, and a body-centered diffusion sampler to integrate various semantic information for high-quality body-region enhancement.

\vspace{-1mm}
\subsection{Diffusion Models}
\vspace{-1mm}
Since the diffusion model was released and popular, many efforts have been made. Two classical applications are image restoration and text-to-image (T2I). Image restoration, as the first and most natural application, has been developed a lot~\cite{
saharia2021image,whang2021deblurringstochasticrefinement,Avrahami_2022_CVPR,chen2023image,xia2023diffir}. With the development of conditional diffusion models~\cite{Rombach2022LDM}, numerous companies have invested heavily in training more powerful T2I models. Recently, many efforts have been made to integrate these two typical applications. The rapidly developing Stable Diffusion~\cite{Rombach2022LDM}, DALLE~\cite{pmlr-v139-ramesh21a}, and PixArt~\cite{chen2024pixartalpha} have continually pushed the boundaries of realism and diversity in T2I generation. Many image restoration methods have also leveraged pretrained models to achieve more natural image recovery~\cite{wu2024seesr,yang2023pasd,lin2024diffbir,wu2024osediff,wang2024osdface}.

Stricted on the multi-timesteps in the diffusion inference procedure, the diffusion models with 50 or more steps~\cite{wang2024exploiting,lin2024diffbir,wu2024seesr,yang2023pasd} cannot actually be used in practice. Many efforts have been made to faster diffusion models, such as cutting, quantizing, and compressing. Moreover, eliminating the number of inference timesteps is a convincing way, especially applied in image restoration. SinSR~\cite{wang2024sinsr} pioneers one-step inference for diffusion-based SR by distilling deterministic generation functions into a student network, coupled with a consistency-preserving loss and efficient training pair generation strategy. OSEDiff~\cite{wu2024osediff} adapts pretrained SD models for Real-ISR through LoRA-finetuned U-Net and variational score distillation, enabling direct low-quality image reconstruction in one step without noise injection. Those methods achieve a fascinating performance in natural image restoration. 

\begin{figure*}[t]
\begin{center}

\includegraphics[width=0.95\textwidth]{figs/ICML-Model.pdf}

\end{center}
\vspace{-5mm}
\caption{Training Framework of OSDHuman. \textbf{First}, the LQ image $I_L$ is processed through the VAE Encoder, U-Net, and VAE Decoder, ultimately producing the restored HQ image $\hat{I}_H$. The conditional input of the U-Net is provided by the high-fidelity image embedder (HFIE). \textbf{Second}, during the training process, the $\hat{z}_H$ generated by the U-Net is subjected to noise and then passed through the pretrained and finetuned regularizers. $\mathcal{L}_{\text{VSD}}$ represents the distribution's difference between the model output and the natural image. $\mathcal{L}_{\text{VSD}}$, together with $\mathcal{L}_{\text{LPIPS}}$ and $\mathcal{L}_{\text{MSE}}$, constitutes the training objective. \textbf{In summary}, during the training stage, the VAE Encoder, U-Net, and finetuned regularizer are trained with LoRA, while other modules remain frozen. During inference, the VSD module is not utilized.
}

\vspace{-5mm}
\label{fig:model_architecture}
\end{figure*}

\vspace{-2mm}