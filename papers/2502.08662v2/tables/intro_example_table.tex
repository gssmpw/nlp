\begin{table}[!t]
\centering
\resizebox{0.45\linewidth}{!}
{
\begin{tabular}{@{}l|c|c|c|c@{}}
\toprule
                                    & \#1                         & \#2 & \#3 & \#4 \\ \midrule
No dependency between rows:         & O                           & O   & X   & X   \\ \midrule
Free of index bias:                 & X                           & O   & X   & X   \\ \midrule
Set Invariant input?                & X                           & O   & X   & X   \\ \midrule
Effective when invariant model used & \vartriangle & O   & X   & X   \\ \bottomrule
\end{tabular}

}
\caption{Comparing between different examples shown on Figure~\ref{fig:intro_example}. By defining set invariant inputs as inputs without dependency between rows and without index bias (which only accounts for \#2 in this case), we provide insight on the necessary factors for positional invariant models to be effective.}
\label{table/intro_example_table}

\end{table}