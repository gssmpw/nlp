
\begin{figure*}[!h]
{
\centering
    
    \begin{subfigure}[b]{\textwidth}
        \centering
        \includegraphics[width=\textwidth]{imgs/with_index.png}
        \caption{Results \textbf{with index bias} (indexed by numbers). (Example input at Appendix Fig.~\ref{lostinthemiddle})}
        \label{fig:litm_with_index}
    \end{subfigure}
    
    \vspace{2ex}
    
    \begin{subfigure}[b]{\textwidth}
        \centering
        \includegraphics[width=\textwidth]{imgs/without_index.png}
        \caption{Results \textbf{without index bias} (indexed by title) (Example input at Appendix Fig.~\ref{lostinthemiddlenoindexing})}
        \label{fig:litm_no_index}
    \end{subfigure}
    
    \caption{Results on the Lost-in-the-middle benchmark. Visualization of the \texttt{best\_subspan\_em} results at Appendix Tab.~\ref{table/litm_number}. \ours{} (dark red, red, yellow) generally performs the best regardless of the position of the gold index, with less fluctuations when we remove index bias. Ours is \ours{} with lexical sort, and ours-reversed is the one with the reversed lexical ordering. For brevity, only the performance of \ours{} with reranking sort (MonoT5) is annotated as numbers, and the performance of PCW and Set-Based Prompting are reported only at the Table (Appendix Tab.~\ref{table/litm_number}) due to its low performance.}
    \label{fig:litm}
%     \includegraphics[width=\linewidth]{imgs/litm.png}
%     \caption{The lost in the middle robustness results of both indexing with numbers and without indexing (replacing passage numbers as New passage:.)}
%     \label{fig:litm}
}
\end{figure*}