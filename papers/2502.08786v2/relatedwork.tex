\section{Related work}
\subsection{Imaging Modalities Integrated to Understand and Locate Acupoints}
Researchers, including Leow et al. and Park et al., explored the use of ultrasound to locate various acupoints and demonstrated the potential of 2D ultrasound-guided techniques for acupuncture needle insertion \cite{leow2017exploring, leow2017ultrasonography, park2011acupuncture}. Kim et al. developed a procedure to acquire 2D ultrasound images using a probe to enhance acupuncture safety by visualizing the underlying structures beneath the skin \cite{kim2017development}. However, the lack of spatial information from 2D images and the challenge of managing a probe during needle insertion remains significant, as prior studies primarily focused on visualizing acupoints for anatomical understanding rather than providing real-time guidance. Our system overcomes the challenge of locating acupoints in space from 2D images by adopting UCT and MR technology, enabling the visualization of 3D reconstructed anatomical structures directly on the patient’s body (right arm). 

Magnetic Resonance Imaging (MRI) and Computed Tomography (CT) have also been utilized retrospectively to assess safe needling depths \cite{chou2015retrospective, yang2015safe, chen2009therapeutic}. Lin et al., in their review, noted considerable inconsistency in defining safe needling depths across research groups, with no conclusive evidence linking or quantifying acupoint depth to anatomical structures based on imaging results \cite{lin2013exploration}. Therefore, our system focused on providing insertion trajectory guidance, allowing practitioners to manually adjust the auto-generated insertion trajectory using their hands while wearing an MR head-mounted display (HMD). The insertion depth is not predefined by the system but is visualized through holographic renderings of bone, muscle, and tracked needles.

\subsection{\ADD{Real-Time Anatomical Visualization and 3DUI in AR/MR Systems for Surgery and Needle Guidance}}
\ADD{AR and MR technologies are revolutionizing surgical precision by superimposing real-time anatomical models onto patients, enhancing spatial awareness and reducing reliance on external imaging \cite{moga2021augmented}, \cite{eom2022ar}, \cite{ma2018moving}. These systems enable more precise interventions in various medical fields.}

\ADD{Eom et al. \cite{eom2022ar} developed an AR system for ventriculostomy, improving catheter placement by projecting 3D ventricular models onto the cranial surface. Ma et al. \cite{ma2018moving} introduced an AR navigation system that compensates for motion, optimizing surgical workflows. In orthopedic surgery, Hu et al. \cite{hu2024artificial} used AI-driven AR for knee replacement, enhancing implant alignment with CT scans, while Penza et al. \cite{penza2023augmented} applied AR for robot-assisted surgeries, integrating endoscopic cameras with 3D anatomical models.}

\ADD{In needle guidance, Varble et al. \cite{borde2024smart} demonstrated a goggle-based AR system for needle placement, showing comparable accuracy to traditional ultrasound methods in radiology. Wang et al. \cite{jiang2023wearable} developed a lumbar puncture system combining wearable ultrasound with HoloLens-based AR for improved needle visualization. Kuzhagaliyev et al. \cite{kuzhagaliyev2018augmented} applied AR for pancreatic ablation, improving needle alignment, while Smith et al. \cite{heinrich20222d} found that 3D AR reduced cognitive load and enhanced hand-eye coordination in needle procedures.}

\ADD{Our system focus on real-time acupuncture guidance with UCT modality, visualizing insertion trajectories on the patient’s body and improving precision and situational awareness.}


\subsection{Extended Reality (XR) Systems for Acupuncture Guidance}
From initial virtual representations of acupoints to more recent advancements in Virtual Reality (VR), multiple VR systems have been developed primarily for acupuncture teaching and training \cite{zhang2024acuvr, liang2021analysis, chen2019application}. To enhance the immersion of simulation and training experiences, Sun et al. developed a MR training simulator that allows practitioners to develop muscle memory \cite{sun2023design}. Chen et al. created an AR system on an Android smartphone to visually represent acupoints in space \cite{chen2017localization}, while Zhang et al. introduced an Android app utilizing a new method to localize AR acupoints \cite{zhang2022faceatlasar}. Despite these advancements, significant gaps remain in applying extended reality systems to clinical settings, particularly in working with real human bodies and supporting hands-on, real-world needling practices \cite{zhang2023vr}. To our knowledge, we developed the first HMD-based MR system that provides real-time guidance for acupuncture needle insertion and evaluated its effectiveness on actual patients.