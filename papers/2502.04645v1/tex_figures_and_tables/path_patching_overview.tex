\begin{figure*}
    \centering
    \includegraphics[width=0.78\linewidth]{tex_figures_and_tables/path_patch_diagram_v2_horiz.jpeg}
    \caption{Path patching methodology. \textit{Left:} There are four forward passes: (1,2) Run model on baseline and perturbed inputs and cache activations. (3) To measure the effect of an upstream sender component (\textit{S}) on a downstream receiver component (\textit{R}), run the model on the baseline input, patch in \textit{S}, freeze all other components, and cache the activation of \textit{R}. (4) Run the model on the baseline input and patch in \textit{R}. \textit{Right:} Alternative visualization of step (3) using the residual stream. By allowing only the downstream receiver \textit{R} to be recomputed when the sender \textit{S} is patched, we effectively isolate the direct path from \textit{S} to \textit{R}, while preserving all other paths to \textit{R} as they were in the baseline run.}
    \label{fig:path-patching-overview}
\end{figure*}