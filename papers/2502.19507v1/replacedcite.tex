\section{Literature Review}
In recent years, knowledge graphs have gained popularity due to their numerous benefits, including scalability and structuring the heterogeneous data into a single structure, making it easier to adapt in diverse domains ____. It has emerged as a powerful paradigm for integrating and managing diverse data sources at scale. It is particularly relevant for agricultural data integration, where diverse datasets---ranging from soil health and climate conditions to genetic information and wheat market trends---need to be combined into a unified framework. In agriculture, several efforts have been made to generate a vocabulary that defines standard terms and relationships in the agricultural domain. Agroportal ____, for example, serves as a hub that contains a number of vocabularies and ontologies related to agriculture.

There also exist other KG-centric efforts that are relevant to the goal of creating a global food systems datahub ____. ____ propose a data-sharing architecture, the Ploutos, that is based on three principles: a) reuse of existing semantic standards; b) integration with legacy systems; and c) a distributed architecture where stakeholders control access to their own data. The Ploutos semantic model is built on an integration of existing ontologies, which use graph query patterns to traverse the network and collect the requisite data to be shared. Their data-sharing approach is highly extensible, with considerable potential for capturing sustainability-related data, related to global food systems, which includes food, environment, transportation, sensor related data.

Recently, significant efforts have been directed toward the development of domain-specific vocabularies, which are designed to address questions within a particular domain by encompassing only the relevant concepts and relationships. For instance, the Crop Disease Ontology ____ provides a hierarchical framework for describing crop health, pathogens, and their impacts on agricultural productivity. Similarly, the Environment Ontology (ENVO) ____ defines environmental entities, systems, and processes across diverse domains, supporting applications in ecology, climate studies, and environmental sciences.

Likewise, BIMERR Weather Ontology ____, Weather Ontology ____, and Ontology for Meteorological Sensors ____ define weather-related concepts, observations, and measurements captured by meteorological sensors. These ontologies enable the integration of weather information into intelligent systems, supporting applications in weather prediction, environmental monitoring, and climate research. Additionally, the Phenotype Quality Ontology (PATO) ____ provides a structured framework for describing phenotypic qualities and their variations across species. It offers semantic representations for attributes such as size, shape, color, and function, which are essential for research in genetics, biology, and medicine.

The KnowWhereGraph\footnote{\url{https://knowwheregraph.org}} ____, to the best of our knowledge, the currently largest public geo-knowledge graph, covers a significant amount of agriculture-relevant data, including soil health data, as well as land use and land cover data. 

The ontologies mentioned above, along with many other existing ontologies, face limitations in their integration with knowledge from other domains, as they are primarily designed to address specific problems within well-defined domain boundaries. Our objective is to address this limitation by developing an integrated framework that describes a sustainable wheat production value chain. This proposed data hub will aim to incorporate knowledge spanning the entire lifecycle of wheat, from farm to table, facilitating a holistic understanding of sustainability in wheat production.