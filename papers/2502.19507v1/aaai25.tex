%File: formatting-instructions-latex-2025.tex
%release 2025.0
\documentclass[letterpaper]{article} % DO NOT CHANGE THIS
\usepackage{aaai25}  % DO NOT CHANGE THIS
\usepackage{times}  % DO NOT CHANGE THIS
\usepackage{helvet}  % DO NOT CHANGE THIS
\usepackage{courier}  % DO NOT CHANGE THIS
\usepackage[hyphens]{url}  % DO NOT CHANGE THIS
\usepackage{graphicx} % DO NOT CHANGE THIS
\urlstyle{rm} % DO NOT CHANGE THIS
\def\UrlFont{\rm}  % DO NOT CHANGE THIS
\usepackage{natbib}  % DO NOT CHANGE THIS AND DO NOT ADD ANY OPTIONS TO IT
\usepackage{caption} % DO NOT CHANGE THIS AND DO NOT ADD ANY OPTIONS TO IT
\frenchspacing  % DO NOT CHANGE THIS
\setlength{\pdfpagewidth}{8.5in}  % DO NOT CHANGE THIS
\setlength{\pdfpageheight}{11in}  % DO NOT CHANGE THIS
%
% These are recommended to typeset algorithms but not required. See the subsubsection on algorithms. Remove them if you don't have algorithms in your paper.
\usepackage{algorithm}
\usepackage{algorithmic}

%
% These are are recommended to typeset listings but not required. See the subsubsection on listing. Remove this block if you don't have listings in your paper.
\usepackage{newfloat}
\usepackage{listings}
\DeclareCaptionStyle{ruled}{labelfont=normalfont,labelsep=colon,strut=off} % DO NOT CHANGE THIS
\lstset{%
	basicstyle={\footnotesize\ttfamily},% footnotesize acceptable for monospace
	numbers=left,numberstyle=\footnotesize,xleftmargin=2em,% show line numbers, remove this entire line if you don't want the numbers.
	aboveskip=0pt,belowskip=0pt,%
	showstringspaces=false,tabsize=2,breaklines=true}
\floatstyle{ruled}
\newfloat{listing}{tb}{lst}{}
\floatname{listing}{Listing}
%
% Keep the \pdfinfo as shown here. There's no need
% for you to add the /Title and /Author tags.
\pdfinfo{
/TemplateVersion (2025.1)
}

% DISALLOWED PACKAGES
% \usepackage{authblk} -- This package is specifically forbidden
% \usepackage{balance} -- This package is specifically forbidden
\usepackage{xcolor}
\newcommand{\red}[1]{{\color{red}#1}}
\newcommand{\blue}[1]{{\color{blue}#1}}
\newcommand{\green}[1]{{\color{green}#1}}
% \usepackage{CJK} -- This package is specifically forbidden
% \usepackage{float} -- This package is specifically forbidden
% \usepackage{flushend} -- This package is specifically forbidden
% \usepackage{fontenc} -- This package is specifically forbidden
% \usepackage{fullpage} -- This package is specifically forbidden
% \usepackage{geometry} -- This package is specifically forbidden
% \usepackage{grffile} -- This package is specifically forbidden
% \usepackage{hyperref} -- This package is specifically forbidden
% \usepackage{navigator} -- This package is specifically forbidden
% (or any other package that embeds links such as navigator or hyperref)
% \indentfirst} -- This package is specifically forbidden
% \layout} -- This package is specifically forbidden
% \multicol} -- This package is specifically forbidden
% \nameref} -- This package is specifically forbidden
% \usepackage{savetrees} -- This package is specifically forbidden
% \usepackage{setspace} -- This package is specifically forbidden
% \usepackage{stfloats} -- This package is specifically forbidden
% \usepackage{tabu} -- This package is specifically forbidden
% \usepackage{titlesec} -- This package is specifically forbidden
% \usepackage{tocbibind} -- This package is specifically forbidden
% \usepackage{ulem} -- This package is specifically forbidden
% \usepackage{wrapfig} -- This package is specifically forbidden
% DISALLOWED COMMANDS
% \nocopyright -- Your paper will not be published if you use this command
% \addtolength -- This command may not be used
% \balance -- This command may not be used
% \baselinestretch -- Your paper will not be published if you use this command
% \clearpage -- No page breaks of any kind may be used for the final version of your paper
% \columnsep -- This command may not be used
% \newpage -- No page breaks of any kind may be used for the final version of your paper
% \pagebreak -- No page breaks of any kind may be used for the final version of your paperr
% \pagestyle -- This command may not be used
% \tiny -- This is not an acceptable font size.
% \vspace{- -- No negative value may be used in proximity of a caption, figure, table, section, subsection, subsubsection, or reference
% \vskip{- -- No negative value may be used to alter spacing above or below a caption, figure, table, section, subsection, subsubsection, or reference

\setcounter{secnumdepth}{0} %May be changed to 1 or 2 if section numbers are desired.

% The file aaai25.sty is the style file for AAAI Press
% proceedings, working notes, and technical reports.
%

% Title

% Your title must be in mixed case, not sentence case.
% That means all verbs (including short verbs like be, is, using,and go),
% nouns, adverbs, adjectives should be capitalized, including both words in hyphenated terms, while
% articles, conjunctions, and prepositions are lower case unless they
% directly follow a colon or long dash
\title{Building Knowledge Graphs Towards a Global Food Systems Datahub}
\author{
    %Authors
    % All authors must be in the same font size and format.
    Nirmal Gelal\textsuperscript{\rm 1},
    Aastha Gautam\textsuperscript{\rm 1},
    Sanaz Saki Norouzi\textsuperscript{\rm 1},
    Nico Giordano\textsuperscript{\rm 1},
    Claudio Dias da Silva Jr\textsuperscript{\rm 1},
    Jean Ribert Francois\textsuperscript{\rm 1},
    Kelsey Andersen Onofre\textsuperscript{\rm 1},
    Katherine Nelson\textsuperscript{\rm 1}\textsuperscript{, \rm 2},
    Stacy Hutchinson\textsuperscript{\rm 1},
    Xiaomao Lin\textsuperscript{\rm 1},
    Stephen Welch\textsuperscript{\rm 1},
    Romulo Lollato\textsuperscript{\rm 1},
    Pascal Hitzler\textsuperscript{\rm 1}, and
    Hande Küçük McGinty\textsuperscript{\rm 1}\thanks{Corresponding Author}
}
\affiliations{
    %Afiliations
    \textsuperscript{\rm 1}Kansas State University\\
    Manhattan, KS 66502 USA\\
    \textsuperscript{\rm 2}University of Missouri\\
    Columbia, MO 65211 USA\\
    
}

%Example, Single Author, ->> remove \iffalse,\fi and place them surrounding AAAI title to use it

% REMOVE THIS: bibentry
% This is only needed to show inline citations in the guidelines document. You should not need it and can safely delete it.
\usepackage{bibentry}
% END REMOVE bibentry

\begin{document}

\maketitle

\begin{abstract}
Sustainable agricultural production aligns with several sustainability goals established by the United Nations (UN). However, there is a lack of studies that comprehensively examine sustainable agricultural practices across various products and production methods. Such research could provide valuable insights into the diverse factors influencing the sustainability of specific crops and produce while also identifying practices and conditions that are universally applicable to all forms of agricultural production. While this research might help us better understand sustainability, the community would still need a consistent set of vocabularies. These consistent vocabularies, which represent the underlying datasets, can then be stored in a global food systems datahub. The standardized vocabularies might help encode important information for further statistical analyses and AI/ML approaches in the datasets, resulting in the research targeting sustainable agricultural production. A structured method of representing information in sustainability, especially for wheat production, is currently unavailable. In an attempt to address this gap, we are building a set of ontologies and Knowledge Graphs (KGs) that encode knowledge associated with sustainable wheat production using formal logic. The data for this set of knowledge graphs are collected from public data sources, experimental results collected at our experiments at Kansas State University, and a Sustainability Workshop that we organized earlier in the year, which helped us collect input from different stakeholders throughout the value chain of wheat. The modeling of the ontology (i.e., the schema) for the Knowledge Graph has been in progress with the help of our domain experts, following a modular structure using KNARM methodology. In this paper, we will present our preliminary results and schemas of our Knowledge Graph and ontologies.\end{abstract}

% Uncomment the following to link to your code, datasets, an extended version or similar.
%
% \begin{links}
%     \link{Code}{https://aaai.org/example/code}
%     \link{Datasets}{https://aaai.org/example/datasets}
%     \link{Extended version}{https://aaai.org/example/extended-version}
% \end{links}
\section{Introduction and Motivation}
Food and Agriculture Association of the United Nations reports that global food demand will increase by 50 percent by 2050 \cite{fao}. Addressing the anticipated food shortfall due to either population growth or decline in production and productivity will require an integrative, efficient, and sustainable approach within current agricultural systems. The growing demand for sustainable food production underscores a critical need for structured and interoperable data to enable complex analysis. 

The wheat production domain, a key pillar of global food security, currently has no standardized framework to effectively store and analyze the data on sustainable practices. This absence of structured and interoperable data constrains the development and application of advanced analytical tools, which could enhance decision-making, resource optimization, and sustainable production improvements. We are building a unified KG (combination of a set of KGs) on a modular framework, where each module focuses on a distinct aspect of sustainable wheat production, making data addition and removal manageable and enhancing scalability. This modular structure enables us to integrate various components of the wheat value chain and allows seamless integration with other agricultural datasets, creating a unified, scalable approach to data representation in sustainable agriculture.

As part of our preliminary results, our primary objective has been to encode knowledge on nitrogen management and disease management in wheat, providing researchers, agronomists, and policymakers with data-driven insights to support sustainable practices. This paper outlines our methodology and initial progress in developing this DataHub. We detail how we used our previously established Knowledge Graph's design methodology, namely KNowledge Acquisition and Representation Methodology \cite{mcginty2018knowledge}, to build the modular architecture of our knowledge graph and highlight potential applications for promoting sustainable wheat production practices.

\section{Literature Review}
In recent years, knowledge graphs have gained popularity due to their numerous benefits, including scalability and structuring the heterogeneous data into a single structure, making it easier to adapt in diverse domains \cite{DBLP:journals/cacm/Hitzler21, hogan2021knowledge}. It has emerged as a powerful paradigm for integrating and managing diverse data sources at scale. It is particularly relevant for agricultural data integration, where diverse datasets---ranging from soil health and climate conditions to genetic information and wheat market trends---need to be combined into a unified framework. In agriculture, several efforts have been made to generate a vocabulary that defines standard terms and relationships in the agricultural domain. Agroportal \cite{agroportal}, for example, serves as a hub that contains a number of vocabularies and ontologies related to agriculture.

There also exist other KG-centric efforts that are relevant to the goal of creating a global food systems datahub \cite{gfsMcGinty24}. \cite{brewster2024data} propose a data-sharing architecture, the Ploutos, that is based on three principles: a) reuse of existing semantic standards; b) integration with legacy systems; and c) a distributed architecture where stakeholders control access to their own data. The Ploutos semantic model is built on an integration of existing ontologies, which use graph query patterns to traverse the network and collect the requisite data to be shared. Their data-sharing approach is highly extensible, with considerable potential for capturing sustainability-related data, related to global food systems, which includes food, environment, transportation, sensor related data.

Recently, significant efforts have been directed toward the development of domain-specific vocabularies, which are designed to address questions within a particular domain by encompassing only the relevant concepts and relationships. For instance, the Crop Disease Ontology \cite{CDO} provides a hierarchical framework for describing crop health, pathogens, and their impacts on agricultural productivity. Similarly, the Environment Ontology (ENVO) \cite{buttigieg2016environment} defines environmental entities, systems, and processes across diverse domains, supporting applications in ecology, climate studies, and environmental sciences.

Likewise, BIMERR Weather Ontology \cite{bimerr}, Weather Ontology \cite{weatherontology}, and Ontology for Meteorological Sensors \cite{meteorologicalscience} define weather-related concepts, observations, and measurements captured by meteorological sensors. These ontologies enable the integration of weather information into intelligent systems, supporting applications in weather prediction, environmental monitoring, and climate research. Additionally, the Phenotype Quality Ontology (PATO) \cite{PATO} provides a structured framework for describing phenotypic qualities and their variations across species. It offers semantic representations for attributes such as size, shape, color, and function, which are essential for research in genetics, biology, and medicine.

The KnowWhereGraph\footnote{\url{https://knowwheregraph.org}} \cite{janowicz2022know,DBLP:journals/ws/ShimizuSBCCCDFHJLLMMRSS25}, to the best of our knowledge, the currently largest public geo-knowledge graph, covers a significant amount of agriculture-relevant data, including soil health data, as well as land use and land cover data. 

The ontologies mentioned above, along with many other existing ontologies, face limitations in their integration with knowledge from other domains, as they are primarily designed to address specific problems within well-defined domain boundaries. Our objective is to address this limitation by developing an integrated framework that describes a sustainable wheat production value chain. This proposed data hub will aim to incorporate knowledge spanning the entire lifecycle of wheat, from farm to table, facilitating a holistic understanding of sustainability in wheat production.

\section{Methodology}
\subsection{KG Design Methodology}
Our approach focuses on the modularization and integration of large datasets into cohesive knowledge structures through an application of a design methodology called Knowledge Acquisition and Representation Methodology (KNARM) \cite{mcginty2018knowledge}. KNARM addresses the knowledge acquisition bottleneck by incorporating human expertise, recognizing that the automated methods are still very limited in scope. KNARM focuses on building an ontology or knowledge graph with the end applications and the available datasets in mind and aims to build templates that are based on concrete use cases at hand. Using this methodology, we are designing a modular architecture that allows human expertise involved in the loop.

Among the nine steps in KNARM methodology, \textbf{Sub-language Analysis} is the first step in creating schema/ontology. In this step, we examine the existing data sources, analyze the pattern within the data, and define the relationships between the existing knowledge sources. The second step, \textbf{Unstructured Interview}, involves conducting informal interviews with domain experts involved in formalizing data within the group. A group of domain experts with knowledge of various aspects of sustainability data in wheat were iteratively interviewed throughout the process. In the third step, \textbf{Sub-language Recycling} takes place. In this phase, the aforementioned ontologies were collected and evaluated to facilitate the reuse of vocabularies from established ontologies. Following this step, we typically create a meta-data describing the domain of the data being modeled. This step, known as \textbf{Meta-Data Creation and Knowledge Modeling}, involves listing abstract concepts related to the data hub under development.

The fifth step, referred to as \textbf{Structured Interview}, bears similarities to the second step; however, this phase involves close-ended questions specifically designed for domain experts. Additionally, a list of competency questions was formulated to ensure the completeness of the schema. These questions serve as a benchmark, such that if our model provides comprehensive answers, the schema can be deemed complete. The subsequent step, \textbf{Knowledge Acquisition Validation}, can be considered the initial feedback loop in the process. During this phase, the knowledge engineer presents outputs from previous steps---such as recycled vocabularies, metadata, and other relevant data---to domain experts. The primary aim of this step is to identify any gaps in knowledge that is missed or misinterpreted. The next step is \textbf{Database Formation}. The database can be relational, graph-based, or NoSQL databases. In this study, we opted to use Graph-DB due to its structural compatibility with the schema, its scalability, and its inherent graph-based architecture.

The succeeding step, referred to as \textbf{Semi-Automated Ontology Building}, involves the utilization of tools and libraries such as RDFLib in Python, Jena in Java, and a command-line tool ROBOT to construct ontology in a semi-automated manner. The approach is termed semi-automated because, while the schema is developed and data population is automated using these tools, domain experts remain actively engaged throughout the process to ensure accuracy and relevance. The final step in the methodology is \textbf{Ontology Validation and Evaluation}. Once a version of the ontology is established, knowledge engineers and domain experts conduct a series of tests to assess whether the ontology accurately represents the intended information. These tests include answering competency questions and performing information retrieval tasks using query languages such as SPARQL\footnote{\url{https://www.w3.org/TR/sparql11-query/}}. Up to this point, the KNARM methodology has been employed to achieve a structured data representation.

\subsection{Ontology and Schema Design}
The ontology and schema design process is ongoing, with an initial version of the schema to address nitrogen and disease management within the wheat domain. Figures \ref{fig1} and \ref{fig2} illustrate how the sub-modules (represented as colored rectangular boxes) describe specific topics and interrelation. The following sub-sections will discuss the details of each module and their connections.
\subsubsection{Nitrogen Management}
Nitrogen is a vital nutrient for plant growth, making nitrogen management one of the most significant parameters in farm management practices to optimize yield. Therefore, we have dedicated a distinct module to nitrogen management. This module connects nitrogen demand and supply within a nitrogen management strategy and examines the relationship between crop yield expectations and nitrogen requirements. Crop yield expectation—the projected harvest per unit area—correlates directly with nitrogen demand, as higher yields necessitate greater nitrogen inputs. Key factors supporting maximum crop yields, such as Flowering Date, Freeze Damage, and Heat Stress, are also illustrated in Figure \ref{fig1}.
\begin{figure}[t]
\centering
\includegraphics[width=\columnwidth]{figures/nitrogenManagement.png} % Reduce the figure size so that it is slightly narrower than the column. Don't use precise values for figure width. This setup will avoid overfull boxes.
\caption{Nitrogen Management Schema (Edges represent two concepts that are related to one another on a high-abstract level)}
\label{fig1}
\end{figure}

\subsubsection{Disease Management}
Disease management in wheat is a crucial area of agricultural research, aiming to reduce the impact of pathogens that can significantly lower wheat yields. Effective management strategies are essential for promoting sustainable wheat production. Common wheat pathogens—fungi, viruses, and bacteria—are depicted in the top-level schema of the disease management module, as shown in Figure \ref{fig2}. Currently, our focus is on fungi, detailing their types and potential treatment strategies, including Chemical Strategy, Cultural Practices, and Host Genetics. Each of the highlighted boxes (chemical strategy in blue, cultural practices in green, and host genetics in orange) in Figure \ref{fig2} functions as a self-contained sub-module, encapsulating the concepts and relationships within its specific sub-domain. These sub-modules are further detailed in Figures~\ref{fig3}, \ref{fig4}, and \ref{fig5}.
\begin{figure}[t]
\centering
\includegraphics[width=0.9\columnwidth]{figures/fungicide-top.png} % Reduce the figure size so that it is slightly narrower than the column. Don't use precise values for figure width. This setup will avoid overfull boxes.
\caption{Disease Management Schema (top-level)}
\label{fig2}
\end{figure}

\begin{figure}[t]
\centering
\includegraphics[width=0.9\columnwidth]{figures/chemical_strategy.png} % Reduce the figure size so that it is slightly narrower than the column. Don't use precise values for figure width. This setup will avoid overfull boxes.
\caption{Chemical Strategies for Fungi Management}
\label{fig3}
\end{figure}

\begin{figure}[t]
\centering
\includegraphics[width=0.9\columnwidth]{figures/cultural_practices.png} % Reduce the figure size so that it is slightly narrower than the column. Don't use precise values for figure width.This setup will avoid overfull boxes.
\caption{Cultural Practices for Fungi Management}
\label{fig4}
\end{figure}

\begin{figure}[t]
\centering
\includegraphics[width=0.9\columnwidth]{figures/host-genetics.png} % Reduce the figure size so that it is slightly narrower than the column. Don't use precise values for figure width.This setup will avoid overfull boxes.
\caption{Host Genetics for Fungi Management}
\label{fig5}
\end{figure}

Schema design requires multiple iterations to reach a finalized structure. Following KNARM, we involve domain experts throughout the process, conducting both structured and unstructured interviews to iteratively build the schema until it adequately captures the desired information. Before designing the schema, we developed a list of competency questions, which define the scope and goals of the ontology. These questions represent what our final ontology should be able to answer or address, helping to maintain the schema’s structure, scope, requirements, and intended functionality within the knowledge model.

We are in the process of encoding the schema into an ontology—a structured, formal representation of knowledge within a specific domain designed to organize concepts, entities, and their relationships in a consistent and meaningful manner. We are using Protégé \cite{musen2015protege}, an open-source ontology development tool, to build our ontology based on the generated schema. The Web Ontology Language\footnote{\url{https://www.w3.org/OWL/}} (OWL, \cite{owl2primer}) will be employed to axiomatize knowledge within the domain of agriculture. Fundamental axioms, including the definition of domains, ranges, constraints, and equivalences, will be formally encoded.

KNARM methodology helps systematize this process. Our domain experts are continuously interviewed, and the schema drafts are repeatedly reviewed for multiple iterations. We are also looking at existing ontologies for vocabulary and axiom integration \cite{CDO}, \cite{buttigieg2013environment}, \cite{bimerr}, \cite{weatherontology}, \cite{meteorologicalscience}, \cite{PATO}. After the successful creation of our schema and integration of existing ontologies, we plan to infer new knowledge from the knowledge base. Reasoning will be conducted utilizing reasoners such as HermiT \cite{glimm2014hermit}, Pellet \cite{sirin2007pellet}, and other existing reasoners.

Figure \ref{fig6} illustrates the Disease Management module within Protégé, while the other modules and sub-modules remain under development.
\begin{figure}[t]
\centering
\includegraphics[width=0.9\columnwidth]{figures/ontology.png} % Reduce the figure size so that it is slightly narrower than the column. Don't use precise values for figure width.This setup will avoid overfull boxes.
\caption{Screenshot of the Protégé representing the module Disease Management }
\label{fig6}
\end{figure}

\subsection{Data Collection and Preparation}
The next step in building our knowledge base involves data collection and preparation. KNARM emphasizes reusing existing ontologies and data sources to create data structures that are not only effective but also reusable. This approach enables the development of adaptable and scalable knowledge structures that can be efficiently integrated into broader data ecosystems.


We are examining various pre-existing ontologies within the wheat domain or other domains related to wheat, including the Wheat Trait Ontology (WTO) \cite{nedellec2020wto}, Environment Ontology (ENVO) \cite{buttigieg2013environment}, Crop Dietary Nutrition \cite{andres2020knowledge}, and KnowWhereGraph \cite{janowicz2022know}. These ontologies provide well-defined concepts and relationships within their respective domains, offering valuable vocabulary and structure for incorporation into our own ontology. Reusing established ontologies offers significant advantages, such as saving time, ensuring consistency, and enhancing interoperability. By adopting standardized terminology from these ontologies, we ensure that concepts and relationships are defined uniformly while also promoting compatibility with industry best practices.

In addition to existing ontologies, we have identified several key data sources, including weather \cite{daly2013prism}, \cite{ornl_daac}, soil \cite{monitor2023us}, and landscape \cite{soils17gridded}, \cite{burchfieldcounty} These datasets are essential to incorporate within the wheat value chain, as they directly impact each stage of wheat production, influencing factors such as quality, yield, and sustainability. By aggregating this knowledge into a cohesive structure, we enable stakeholders to make data-driven decisions across the entire wheat value chain—from planning and cultivation to processing and distribution—ultimately supporting more sustainable and efficient practices.

This year, we organized a Sustainability Workshop, inviting key stakeholders from the wheat value chain, including farmers, bakers, millers, and other domain experts. The workshop provided valuable insights into the wheat value chain and perspectives on sustainability. The data gathered from these expert contributions is currently under analysis, with findings in the publication pipeline. We also plan to incorporate insights from the workshop into our knowledge graph, enriching its representation of sustainable practices. Additionally, we are integrating ground-truth experimental data collected by professors at Kansas State University, further strengthening the empirical foundation of our knowledge base.

With the assistance of experts in our group, we are filtering data to align with our specific use cases. This involves manually selecting relevant data that corresponds to the schema and ontology developed thus far. The process of populating databases and incorporating data into our knowledge base is discussed in detail in the next subsection.

\subsection{Database and Data Population}
Populating data into an ontology effectively transforms it into a knowledge graph. A knowledge graph is a semantic network of interconnected data representing entities, their attributes, and the relationships between them. This structure enables querying and analysis of the network, allowing us to explore and derive insights from the complex web of entities and relationships within the knowledge graph.

We have set up a dedicated server, managed at Kansas State University, to store all knowledge graph data in GraphDB.\footnote{\url{https://graphdb.ontotext.com/}} GraphDB is a specialized database and semantic repository designed for the storage, management, and querying of knowledge graphs. It supports RDF\footnote{\url{https://www.w3.org/RDF/}} and SPARQL—a query language for RDF—making it well-suited for semantic applications that utilize ontologies and knowledge graphs. We are also developing Python scripts, utilizing libraries that work with RDF data formats, to populate data from flat files efficiently into GraphDB. After the data population is done, the knowledge graph will be ready to provide meaningful insights out of it.

\subsection{Sustainability Module}
We developed a sustainability module by mapping two established sustainability frameworks, SMART \cite{SMART}  and IDEA \cite{IDEA}. These frameworks provide a structured hierarchy for sustainability metrics, organizing them into dimensions, themes, and sub-themes. Using these references, we systematically aligned our sustainability metrics with the corresponding dimensions, themes, and sub-themes.

To implement this module, we utilized ROBOT \cite{robot}, a flexible command-line tool designed for working with ontologies in the Web Ontology Language (OWL) \cite{owl2primer}. A domain-user-friendly template file (CSV) was created, which ROBOT efficiently converts into an OWL format. This streamlined approach ensures consistency and accuracy in representing sustainability concepts within an ontology framework.

Furthermore, by integrating these sustainability dimensions, themes, and sub-themes with the United Nations Sustainable Development Goals \cite{unsdg}, our expanded graph establishes mappings between key sustainability concepts and processes associated with wheat cultivation. This integration facilitates a comprehensive understanding of the relationships between sustainability metrics and agricultural practices. 

\section{Conclusion, Discussion, and Future Work}
This paper addresses the challenges associated with standardized and interoperable data representation techniques in sustainable wheat production by developing a modular, scalable knowledge graph that integrates diverse data sources and expert knowledge. The design of modular ontology is in process using KNARM methodology, specifically focusing on nitrogen and disease management. These contributions bridge critical gaps in agricultural data management and underscore the DataHub's potential to enhance sustainability metrics.

Our DataHub provides a robust foundation for addressing numerous downstream Artificial Intelligence (AI) and Machine Learning (ML) tasks. Designed to cater to a diverse range of users, it incorporates a comprehensive dataset encompassing weather patterns, soil characteristics, and other key agricultural variables. This enables stakeholders to optimize decision-making processes such as planting and harvesting schedules, implement preventive measures, and minimize losses. Examples of ML models well-suited to leverage our DataHub include regression models for predicting yields based on parameters such as nitrogen levels, weather patterns, and soil conditions; classification models for identifying disease outbreaks; clustering algorithms for segmenting fields into zones to facilitate targeted interventions; and reinforcement learning models to dynamically optimize irrigation and fertilizer applications. 

Despite the numerous merits of using knowledge graphs (KGs) to structure data, they also have some notable limitations. A primary challenge is the issue of time complexity, particularly during reasoning and inference processes in large graphs. Many graph algorithms are computationally intensive, making KGs difficult to adopt widely despite their effectiveness in organizing data. Another significant limitation is knowledge validation, as ensuring the accuracy and reliability of the information within a KG remains a complex task.

The next major challenge in creating a datahub combining the set of KGs into one is to deal with semantic and naming inconsistency. Ontology alignment itself has become a separate research field where researchers are searching for efficient techniques to merge ontologies without these naming inconsistencies. The same classes, object properties, or instances can be represented by different names. To map these same but differently encoded entities, we plan to do simple manual checking and use mapping rules to ensure proper ontology alignment. Mapping rules include axioms like equivalence, subclass, instance mapping, etc., that aid in removing conflicts that might arise during ontology merging. We also plan to use the above-mentioned reasoners to check logical inconsistencies in the merged ontology. Furthermore, non-technical users often face difficulties interacting with KGs due to the complexity of graph-based query languages and the limited availability of user-friendly interfaces. Nevertheless, KGs excel in heterogeneous data representation and scalability, provided the ontology is designed effectively.

Our future work aims to transform this ongoing project into a fully operational datahub of wheat that aids in the global food system datahub. Planned expansions include extending the disease management module to cover fungi and bacteria, as well as incorporating additional modules focusing on weather, soil, drought, and other related concepts. The goal is to develop a comprehensive DataHub that encapsulates all fundamental concepts within the wheat value chain, spanning from farm to table. The methodologies employed in this project are designed for adaptability, allowing similar structures to be applied to other crops. This approach encourages collaborative efforts toward the creation and management of wheat datahub.

\medskip

\noindent\emph{Acknowledgements} This work was funded by Kansas State University under the Game-changing Research Initiative Program (GRIP). We also thank Ontotext for providing GraphDB free of charge for this project.

\bibliography{aaai25}
\end{document}
