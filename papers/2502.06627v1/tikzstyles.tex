
\tikzset
{
	set stroke arrow/.code={%
		\pgfqkeys{/tikz/stroke arrow}{#1}},
	set stroke arrow={%
		stroke width/.initial=.1em,
		stroke color/.initial=black,
		arrow tip start/.initial=,
		arrow tip end/.initial=,
		tip length/.initial=\the\pgflinewidth,
		tip width/.initial=1.5*\the\pgflinewidth,
		new tip width/.initial=1,
		new tip length/.initial=1,
		shorten front/.initial=0,
		shorten back/.initial=0
	},
	do calc/.code={%
		\pgfmathsetmacro{\b}{\pgfkeysvalueof{/tikz/stroke arrow/tip width}/2}
		\pgfmathsetmacro{\a}{\pgfkeysvalueof{/tikz/stroke arrow/tip length}}
		\pgfmathsetmacro{\c}{sqrt(\a*\a + \b*\b)}
		\pgfmathsetmacro{\hOld}{\a * \b / \c}
		\pgfmathsetmacro{\newwidth}{2*(\hOld+1.5*\pgfkeysvalueof{/tikz/stroke arrow/stroke width})*\c / \a}
		\pgfmathsetmacro{\newlength}{(\hOld+1.5*\pgfkeysvalueof{/tikz/stroke arrow/stroke width})*\c / \b}
		\pgfkeyssetvalue{/tikz/stroke arrow/new tip width}{\newwidth}
		\pgfkeyssetvalue{/tikz/stroke arrow/new tip length}{\newlength}
		%\pgfkeyssetvalue{/tikz/stroke arrow/new tip width}{2*(\hOld+\pgfkeysvalueof{/tikz/stroke arrow/stroke width})*\c / \a}
		%\pgfkeyssetvalue{/tikz/stroke arrow/new tip length}{(\hOld+\pgfkeysvalueof{/tikz/stroke arrow/stroke width})*\c / \b}
	},
	stroke arrow/.search also={/tikz},
	stroke arrow/.style= {
		set stroke arrow={#1},
		do calc,
		shorten <= \pgfkeysvalueof{/tikz/stroke arrow/shorten front},
		shorten >=\pgfkeysvalueof{/tikz/stroke arrow/shorten back},
		{\pgfkeysvalueof{/tikz/stroke arrow/arrow tip start}[width=\pgfkeysvalueof{/tikz/stroke arrow/tip width}, length=\pgfkeysvalueof{/tikz/stroke arrow/tip length}]}-{\pgfkeysvalueof{/tikz/stroke arrow/arrow tip end}[width=\pgfkeysvalueof{/tikz/stroke arrow/tip width}, length=\pgfkeysvalueof{/tikz/stroke arrow/tip length}]},
		preaction={%
			draw=\pgfkeysvalueof{/tikz/stroke arrow/stroke color},
			shorten <=\pgfkeysvalueof{/tikz/stroke arrow/shorten front}-1.6*\pgfkeysvalueof{/tikz/stroke arrow/stroke width},
			shorten >= \pgfkeysvalueof{/tikz/stroke arrow/shorten back}-1.6*\pgfkeysvalueof{/tikz/stroke arrow/stroke width},			
			line width=\the\pgflinewidth + \pgfkeysvalueof{/tikz/stroke arrow/stroke width},
			{\pgfkeysvalueof{/tikz/stroke arrow/arrow tip start}[width=\pgfkeysvalueof{/tikz/stroke arrow/new tip width}, length=\pgfkeysvalueof{/tikz/stroke arrow/new tip length}]}-{\pgfkeysvalueof{/tikz/stroke arrow/arrow tip end}[width=\pgfkeysvalueof{/tikz/stroke arrow/new tip width}, length=\pgfkeysvalueof{/tikz/stroke arrow/new tip length}]},
		},
	},
}

\tikzset
{
	set road/.code={\pgfqkeys{/tikz/road}{#1}},
	set road={%
		marking width/.initial=\the\pgflinewidth,%
		lane width/.initial=5*\the\pgflinewidth,%
		dash length/.initial=\pgfkeysvalueof{/tikz/road/marking width},%
		color/.initial=lightgray,%
		opacity/.initial=1,%
		curb/.initial=0.2*\pgfkeysvalueof{/tikz/road/lane width},%
		curb left/.initial=\pgfkeysvalueof{/tikz/road/curb},%
		curb right/.initial=\pgfkeysvalueof{/tikz/road/curb},%
		proxy marking right/.style={solid, white, line width=\pgfkeysvalueof{/tikz/road/marking width}},
		marking right/.style={proxy marking right/.style={#1}},
		proxy marking left/.style={solid, white, line width=\pgfkeysvalueof{/tikz/road/marking width}},
		marking left/.style={proxy marking left/.style={#1}},
		proxy marking center/.style={white, dash pattern=on \pgfkeysvalueof{/tikz/road/dash length} off \pgfkeysvalueof{/tikz/road/dash length}, line width=\pgfkeysvalueof{/tikz/road/marking width}},
		marking center/.style={proxy marking center/.style={#1}}
	},
	road/.search also={/tikz},
	road/.style= {
		draw=none,
		set road={#1},
		preaction={
			line width=2*\pgfkeysvalueof{/tikz/road/lane width},
			draw=\pgfkeysvalueof{/tikz/road/color},
			opacity=\pgfkeysvalueof{/tikz/road/opacity},
			shorten >= 0.1ex
		},
		postaction={decorate},
		decoration={
			markings,
			mark=at position 1ex with
			{
				%
			},
			mark=between positions 0 and 1 step 0.5ex with
			{
				\pgfkeysgetvalue{/pgf/decoration/mark info/sequence number}{\j}
				\pgfmathtruncatemacro{\i}{\j-1}
				\coordinate (A\i) at (0,\pgfkeysvalueof{/tikz/road/lane width});
				\coordinate (B\i) at (0,0);
				\coordinate (C\i) at (0,-\pgfkeysvalueof{/tikz/road/lane width});
			},
			mark=at position 0.5 with
			{
				\pgfkeysgetvalue{/pgf/decoration/mark info/sequence number}{\j}
				\pgfmathtruncatemacro{\marks}{\j-2}
				\pgfmathparse{\pgfkeysvalueof{/tikz/road/curb right}}
				\ifnum 0=\pgfmathresult
				\else
				\draw[solid,opacity=\pgfkeysvalueof{/tikz/road/opacity},\pgfkeysvalueof{/tikz/road/color}, line width=\pgfkeysvalueof{/tikz/road/curb right}] ($(C\marks.center)+(0,-0.5*\pgfkeysvalueof{/tikz/road/curb right}-0.4*\pgfkeysvalueof{/tikz/road/marking width})$)
				foreach \i in {\marks,...,1}
				{ -- ($(C\i.center)+(0,-0.5*\pgfkeysvalueof{/tikz/road/curb right}-0.4*\pgfkeysvalueof{/tikz/road/marking width})$) } -- ($(C1.center)+(0,-0.5*\pgfkeysvalueof{/tikz/road/curb right}-0.4*\pgfkeysvalueof{/tikz/road/marking width})$);
				\fi
				\pgfmathparse{\pgfkeysvalueof{/tikz/road/curb left}}
				\ifnum 0=\pgfmathresult
				\else	            		
				\draw[solid,opacity=\pgfkeysvalueof{/tikz/road/opacity},\pgfkeysvalueof{/tikz/road/color}, line width=\pgfkeysvalueof{/tikz/road/curb left}] ($(A\marks.center)-(0,-0.5*\pgfkeysvalueof{/tikz/road/curb left}-0.4*\pgfkeysvalueof{/tikz/road/marking width})$)
				foreach \i in {\marks,...,1}
				{ -- ($(A\i.center)-(0,-0.5*\pgfkeysvalueof{/tikz/road/curb left}-0.4*\pgfkeysvalueof{/tikz/road/marking width})$) } -- ($(A1.center)-(0,-0.5*\pgfkeysvalueof{/tikz/road/curb left}-0.4*\pgfkeysvalueof{/tikz/road/marking width})$);
				\fi
				%\draw[road/proxy marking left] (A\marks.center)
				%foreach \i in {\marks,...,1}
				%{ -- (A\i.center) } -- (A1.center);
				%
				\draw[road/proxy marking right] (C\marks.center)
				foreach \i in {\marks,...,1}
				{ -- (C\i.center) } -- (C1.center);
				\draw[road/proxy marking left] (A\marks.center)
				foreach \i in {\marks,...,1}
				{ -- (A\i.center) } -- (A1.center);
				\draw[road/proxy marking center] (B\marks.center)
				foreach \i in {\marks,...,1}
				{ -- (B\i.center) } -- (B1.center);
			}
		},
	},
}
%
\makeatletter
\pgfdeclaredecoration{free hand}{start}
{
	\state{start}[width = +0pt,
	next state=step,
	persistent precomputation = \pgfdecoratepathhascornerstrue]{}
	\state{step}[auto end on length    = 3pt,
	auto corner on length = 3pt,               
	width=+2pt]
	{
		\pgfpathlineto{
			\pgfpointadd
			{\pgfpoint{2pt}{0pt}}
			{\pgfpoint{rand*0.15pt}{rand*0.15pt}}
		}
	}
	\state{final}
	{}
}
\makeatother
%
\tikzset{
	double yellow/.style = {%
		double=lightgray,
		double distance=0.8*\mylinewidth,
		line width=\mylinewidth,
		draw=tuYellow
	}
}
%
\tikzset{
	dashed marking/.style args={#1 dashlength #2}{%
		#1,
		dash pattern=on #2 off #2
	},
	dashed centerline german/.style 2 args={%
		dashed marking={solid, white, line width=#1 dashlength #2}
	}
}
%
\tikzset{%
	place nodes/.style={%
		postaction={%
			decorate,%
			decoration={%
				markings,%
				mark=between positions 0 and 1 step 0.125 with {%
					\node [] (a\pgfkeysvalueof{/pgf/decoration/mark info/sequence number}) {};%
				},%
			}%
		}%
	}%
}%

\pgfdeclaredecoration{triangle}{start}{
	\state{start}[width=0.99\pgfdecoratedinputsegmentremainingdistance,next state=up from center]
	{\pgfpathlineto{\pgfpointorigin}}
	\state{up from center}[next state=do nothing]
	{
		\pgfpathlineto{\pgfqpoint{\pgfdecoratedinputsegmentremainingdistance}{\pgfdecorationsegmentamplitude}}
		\pgfpathlineto{\pgfqpoint{\pgfdecoratedinputsegmentremainingdistance}{-\pgfdecorationsegmentamplitude}}
		\pgfpathlineto{\pgfpointdecoratedpathfirst}
	}
	\state{do nothing}[width=\pgfdecorationsegmentlength,next state=do nothing]{
		\pgfpathlineto{\pgfpointdecoratedinputsegmentfirst}
		\pgfpathmoveto{\pgfpointdecoratedinputsegmentlast}
	}
}


\tikzset{%
	stop sign/.style 2 args={%
		rounded corners=#1*0.2pt,%
		xshift=#1*\lanewidth,%
		shape=regular polygon,%
		line width=#1*0.8pt,%
		regular polygon sides=8,%
		fill=tuRed,%
		draw=white,%
		inner sep=#1*-0.2pt,%
		label={%
			[anchor=center, 
			yshift=#1*-0.7em, 
			text=white, 
			font=#2\sffamily\bfseries]
			\scalebox{0.5}[1.2]{STOP}},%
		minimum width=#1*1.4em,%
		append after command={\pgfextra%
			\draw[line width=#1*0.3pt, black, rounded corners=#1*0.1pt]%
			($(\tikzlastnode.center)!1.01!(\tikzlastnode.corner 1)$) --%
			($(\tikzlastnode.center)!1.01!(\tikzlastnode.corner 2)$) --%
			($(\tikzlastnode.center)!1.01!(\tikzlastnode.corner 3)$) --%
			($(\tikzlastnode.center)!1.01!(\tikzlastnode.corner 4)$) --% 
			($(\tikzlastnode.center)!1.01!(\tikzlastnode.corner 5)$) --% 
			($(\tikzlastnode.center)!1.01!(\tikzlastnode.corner 6)$) --% 
			($(\tikzlastnode.center)!1.01!(\tikzlastnode.corner 7)$) --%
			($(\tikzlastnode.center)!1.01!(\tikzlastnode.corner 8)$) -- cycle;%
			\endpgfextra%
		}%
	}%
}
\tikzset{%
	psm fact/.style={%
		ellipse,
		drop shadow,
		shadedGray,
		align=center,
		text width=6em,
		minimum height=3em
	},
	detection fact/.style={
		psm fact,
		shadedBlueLight
	},
	action fact/.style={
		psm fact,
		shadedRedLight
	},	
	rule fact/.style={
		psm fact,
		shadedGreenLight
	}	
}

\tikzset{
	triangle path/.style={decoration={triangle,amplitude=#1}, decorate},
	triangle path/.default=1ex}

\tikzset{%
	capability block/.style={
		minimum width=5.5em,
		minimum height=4.5em,
		align=center,
	},
	indicator/.style={
		capability block,
		shadedGray,
	},
	capability fact/.style={%
		capability block,
		shadedBlueLight,
		rounded corners=0.5em,
	},
	myCapability/.style={%
	trapezium,
	trapezium left angle=80,
	trapezium right angle=-80,
	minimum width=4em,
	minimum height=4.2em,
	text width=7.5em,
	align=center,
	inner sep=0.5em
},
	knowledge cap/.style={%
		myCapability,
		shadedGrayLight,
	},
	scenario cap/.style={%
		myCapability,
		shadedGrayDark,
	},
	abstract cap/.style={%
		scenario cap,
		dashed
	},
	connect/.style={%
		thick,
		-{[round]Stealth},
		rounded corners=2pt
	}	
}

%% Tikz styles - thanks to Johannes Schlatow
\tikzstyle{shadedRedLight} = [top color=white!90!tuRed, bottom color=tuRed!60!white, draw=tuRed80, thick]%
\tikzstyle{shadedBlue} = [top color=tuDarkBlue20, bottom color=tuDarkBlue40, draw=tuDarkBlue80, thick]%
\tikzstyle{shadedBlueLight} = [top color=tuLightBlue20, bottom color=tuLightBlue40, draw=tuBlue, thick]%
\tikzstyle{shadedRed} = [top color=tuRed20, bottom color=tuRed40, draw=tuRed80, thick]%
\tikzstyle{shadedOrange}  = [top color=tuOrange20, bottom color=tuOrange60,  draw=tuOrange100, thick]%
\tikzstyle{shadedOrangeDark}  = [top color=tuOrange40!99!black, bottom color=tuOrange80!99!black,  draw=tuOrange100!99!black, thick]%
\tikzstyle{shadedOrangeLight}  = [top color=tuOrange20,bottom color=tuOrange40, draw=tuOrange80, thick]%
\tikzstyle{shadedGreen}  = [top color=tuGreen20, bottom color=tuGreen60,  draw=tuGreen100, thick]%
\tikzstyle{shadedGreenLight}  = [top color=tuLightGreen20, bottom color=tuLightGreen60,  draw=tuLightGreen100, thick]%
\tikzstyle{shadedYellow} = [top color=tuYellow20, bottom color=tuYellow60,  draw=tuYellow100, thick]%
%\tikzstyle{shadedYellow} = [top color=white, bottom color=yellow!50!black!20,  draw=yellow!50!black!50, thick]
%\tikzstyle{shadedYellow} = [top color=white, bottom color=tubsgelb20,  draw=tubsgelb80, thick]
\tikzstyle{shadedGray} = [top color=white, bottom color=tuGray20, draw=tuGray60, thick]%
\tikzstyle{shadedGrayLight} = [top color=tuBlack!5, bottom color=tuBlack!10, draw=tuBlack!30, thick]%
\tikzstyle{shadedGrayDark} = [top color=tuBlack!20, bottom color=tuBlack!40, draw=tuBlack!60, thick]%
\tikzstyle{shadedGrayMedium} = [top color=tuBlack!13, bottom color=tuBlack!25, draw=gray, thick]%
\tikzstyle{shadedGrayMediumLight} = [top color=tuBlack!10, bottom color=tuBlack!18, draw=gray, thick]%
\tikzstyle{shadedGrayMediumDark} = [top color=tuBlack!15, bottom color=tuBlack!32, draw=darkgray, thick]%
\tikzstyle{shadedBlueMedium} = [top color=tuLightBlue40, bottom color=tuBlue40, draw=tuBlue, thick]%
\tikzstyle{shadedViolet} = [top color=white!97!tuViolet, bottom color=tuViolet!30!white, draw=tuViolet80, thick]%
%
\tikzstyle{shadow} = [drop shadow={opacity=.5,shadow xshift=.3ex,shadow yshift=-.3ex}]%
%\tikzstyle{shadow} = []
\tikzstyle{triangle} = [isosceles triangle,isosceles triangle stretches]%
\tikzstyle{label-it} = [font=\itshape]%
%
\tikzstyle{block} = [draw, shadedBlue, rectangle, rounded corners, minimum height=2em, minimum width=5em]%
\tikzstyle{smallblock} = [draw, shadedBlue, rectangle, rounded corners, minimum height=1em, minimum width=2em,shadow]%
\tikzstyle{outerblock} = [draw, shadedGrayLight, draw=tuGray80, rectangle, rounded corners, minimum height=2em, minimum width=5em,shadow]%
\tikzstyle{bubble} = [fill=black,shadow,circle,draw=black,inner sep=0pt,minimum size=5pt]%
\tikzstyle{memory} = [cylinder, shape border rotate=90, aspect=.4, shadedGrayLight, minimum width=5em, shadow]%
%
\tikzstyle{inheritArrow} = [-open triangle 60,thick]%
\tikzstyle{kompArrow}    = [diamond-,thick]%
%
\tikzstyle{flowDecision} = [diamond, draw, shadedRed, text badly centered, inner sep=0pt,shadow]%
\tikzstyle{flowBlock} = [rectangle, draw, shadedBlue, text centered, rounded corners, minimum height=2em,shadow]%
\tikzstyle{bgBox} = [rectangle, draw, shadedGrayLight, text centered, rounded corners=3mm, shadow, inner sep=10pt]%
%
\tikzstyle{blockarrow} = [draw, thick, single arrow, minimum height=3em]%



\tikzstyle {archblock} = [outerblock, minimum height=4em, align=center, minimum width=12em, font=\sffamily]
\tikzstyle {slim} = [minimum width=6em]
\tikzstyle {verticalblock} = [archblock, rotate = 90, archblock, minimum width=16.5em, minimum height=2em, on grid]
\tikzstyle {sensors} = [rectangle split parts=2, rectangle split horizontal]

\tikzstyle {sup} = [yshift=0.5cm]
\tikzstyle {sdown} = [yshift=-0.5cm]
\tikzstyle {sleft} = [xshift=-0.5cm]
\tikzstyle {sright} = [xshift=0.5cm]

\tikzstyle {arrow} = [very thick, -{Stealth[round]}]

\tikzstyle {seperator} = [thick, dashed, lightgray]
\tikzstyle {arrow} = [-stealth', thick]
\tikzstyle {rarrow} = [arrow, stealth'-]


\tikzstyle{maneuver area} = [fill=tuDarkBlue40, fill opacity=0.6]
\tikzstyle{blue car} = [vehicle, draw=darkgray!20!black, fill=tuBlue]
\tikzstyle{red car} = [vehicle, draw=darkgray!20!black, fill=tuRed100]
\tikzstyle{orange car} = [vehicle, draw=darkgray!20!black, fill=tuOrange]
\tikzstyle{yellow car} = [vehicle, draw=darkgray!20!black, fill=tuYellow]
\tikzstyle{green car} = [vehicle, draw=darkgray!20!black, fill=tuGreen]
\tikzstyle{target pose} = [opacity=0.7, draw=white]

%\def\csep{0.75cm};
%\def\rsep{0.75cm};
%\def\rmv{0.5cm};
%\pgfmathsetmacro{\hcsep}{0.5 * \csep}
%\pgfmathsetmacro{\hrsep}{0.5 * \rsep}
%\pgfmathsetmacro{\outercsep}{2 * \csep}
%\pgfmathsetmacro{\umv}{0.5 * \rmv}
%\tikzstyle {rshift} = [xshift=\rmv]
%\tikzstyle {lshift} = [xshift=-\rmv]

%\tikzstyle {ushift} = [yshift=\umv]
%\tikzstyle {dshift} = [yshift=-\umv]




\makeatletter
\tikzset{reset preactions/.code={\def\tikz@preactions{}}}
\makeatother

\tikzstyle {disable} = [reset preactions, draw=none, fill=none, top color=white, bottom color=white]
\tikzstyle {header} = [disable, font=\sffamily]

\tikzset{
	fit label/.style={yshift={(height("#1")+4pt)/2},
	inner ysep={(height("#1")+8pt)/2},
	label={[anchor=north,font=\itshape]north:#1 \pgfkeysvalueof{/pgf/inner xsep}}},
	free hand/.style={
			decorate, thick, draw,
			decoration={free hand}
	},  
}

\tikzset{
	double arrow/.style args={#1 #2 colored by #3 and #4}{
		-{#1[round]},line width=#2,#3, % first arrow
		postaction={draw,-{#1[round, scale=1.095]},#4,line width=(#2)/2,
			shorten <=0.6*(#2)/2,shorten >=0.6*(#2)/2}, % second arrow
	}
}


\tikzset{mysplit/.style={rectangle split, rectangle split parts=2, rectangle split draw splits=false, inner sep=2.2ex,
		rectangle split horizontal,minimum width=5.5em},
	textstyle/.style={text height=1.5ex,text depth=.25ex}}

%orientation, origin, axis length
\newcommand{\coordframe}[9][0]{%
	\begin{scope}[rotate around={#1:#2}]
		\draw[tuRed, #6] #2 -- ++ (#3, 0) node[below,#8] {$#4$};
		\draw[tuLightGreen80, #7] #2 -- ++ (0, #3) node[left,#9] {$#5$};
	\end{scope}
}



%% TIKZUMLSTYLES
\tikzumlset{class width=18ex}

\tikzumlset{draw element=gray!95!black}
\tikzumlset{font=\small}
\tikzstyle{tikzuml simpleclass style}=[shadedGray, shadow, minimum width=5em, align=center, font=\textbf]
\pgfkeys{%
	/tikzuml/relation/style/.style={#1, font=\small}
}

%% UMLSYTLES
\tikzset{
	TLeft/.style={label={[anchor=east, xshift=1ex]left:$\blacktriangleleft$}},
	TRight/.style={label={[anchor=west, xshift=-1ex]right:\strut$\blacktriangleright$}},
	TUp/.style={label={[anchor=south, yshift=-1ex]above:$\blacktriangle$}},
        TUpR/.style={label={[anchor=west, xshift=-1.5ex]right:$\blacktriangle$}},
        TUpL/.style={label={[anchor=east, xshift=1.5ex]left:$\blacktriangle$}},
	TDown/.style={label={[anchor=north, yshift=1ex]below:$\blacktriangledown$}},
        TDownR/.style={label={[anchor=west, xshift=-1.5ex]right:$\blacktriangledown$}},
        TDownL/.style={label={[anchor=east, xshift=1.5ex]left:$\blacktriangledown$}},
	closer pos/.style={},
	VP/.style={{archblock, font=\normalfont}}
}

\pgfdeclarepatternformonly{swnestripes}{\pgfpoint{0cm}{0cm}}{\pgfpoint{1cm}{1cm}}{\pgfpoint{1cm}{1cm}}
{
	\foreach \i in {0.1, 0.3,...,0.9}
	{
		\pgfpathmoveto{\pgfpoint{\i cm}{0cm}}
		\pgfpathlineto{\pgfpoint{1cm}{1cm - \i cm}}
		\pgfpathlineto{\pgfpoint{1cm}{1cm - \i cm + 0.1cm}}
		\pgfpathlineto{\pgfpoint{\i cm- 0.1cm}{0cm}}
		\pgfpathclose%
		\pgfusepath{fill}
		\pgfpathmoveto{\pgfpoint{0cm}{\i cm}}
		\pgfpathlineto{\pgfpoint{1cm - \i cm}{1cm}}
		\pgfpathlineto{\pgfpoint{1cm - \i cm- 0.1cm}{1cm}}
		\pgfpathlineto{\pgfpoint{0cm}{\i cm+ 0.1cm}}
		\pgfpathclose%
		\pgfusepath{fill}
	}
}
\pgfdeclarepatternformonly{swneStripes}{\pgfpoint{0cm}{0cm}}{\pgfpoint{1cm}{1cm}}{\pgfpoint{1cm}{1cm}}
{
	\foreach \i in {0.1, 0.3,...,0.9}
	{
		\pgfpathmoveto{\pgfpoint{\i cm}{0cm}}
		\pgfpathlineto{\pgfpoint{1cm}{1cm - \i cm}}
		\pgfpathlineto{\pgfpoint{1cm}{1cm - \i cm - 0.1cm}}
		\pgfpathlineto{\pgfpoint{\i cm+ 0.1cm}{0cm}}
		\pgfpathclose%
		\pgfusepath{fill}
		\pgfpathmoveto{\pgfpoint{0cm}{\i cm}}
		\pgfpathlineto{\pgfpoint{1cm - \i cm}{1cm}}
		\pgfpathlineto{\pgfpoint{1cm - \i cm+ 0.1cm}{1cm}}
		\pgfpathlineto{\pgfpoint{0cm}{\i cm- 0.1cm}}
		\pgfpathclose%
		\pgfusepath{fill}
	}
}
\pgfdeclarepatternformonly{senwstripes}{\pgfpoint{0cm}{0cm}}{\pgfpoint{1cm}{1cm}}{\pgfpoint{1cm}{1cm}}
{
	\foreach \i in {0.1, 0.3,...,0.9}
	{
		\pgfpathmoveto{\pgfpoint{0cm}{\i cm}}
		\pgfpathlineto{\pgfpoint{0cm}{\i cm+ 0.1cm}}
		\pgfpathlineto{\pgfpoint{\i cm+ 0.1cm}{0cm}}
		\pgfpathlineto{\pgfpoint{\i cm}{0cm}}
		\pgfpathclose%
		\pgfusepath{fill}
		\pgfpathmoveto{\pgfpoint{1cm}{\i cm}}
		\pgfpathlineto{\pgfpoint{\i cm}{1cm}}
		\pgfpathlineto{\pgfpoint{\i cm + 0.1cm}{1cm}}
		\pgfpathlineto{\pgfpoint{1cm}{\i cm+ 0.1cm}}
		\pgfpathclose%
		\pgfusepath{fill}
	}
}
\pgfdeclarepatternformonly{senwStripes}{\pgfpoint{0cm}{0cm}}{\pgfpoint{1cm}{1cm}}{\pgfpoint{1cm}{1cm}}
{
	\foreach \i in {0.0, 0.2,...,0.8}
	{
		\pgfpathmoveto{\pgfpoint{\i cm}{0cm}}
		\pgfpathlineto{\pgfpoint{0cm}{\i cm }}
		\pgfpathlineto{\pgfpoint{0cm}{\i cm+ 0.1cm}}
		\pgfpathlineto{\pgfpoint{\i cm+ 0.1cm}{0cm}}
		\pgfpathclose%
		\pgfusepath{fill}
		\pgfpathmoveto{\pgfpoint{1cm}{\i cm}}
		\pgfpathlineto{\pgfpoint{\i cm}{1cm}}
		\pgfpathlineto{\pgfpoint{\i cm+ 0.1cm}{1cm}}
		\pgfpathlineto{\pgfpoint{1cm}{\i cm + 0.1cm}}
		\pgfpathclose%
		\pgfusepath{fill}
	}
}

\newcounter{coordinateindex}

\tikzset{
	put coordinates/.style={
		initialize counter/.code={
			\setcounter{coordinateindex}{0}
		},
		initialize counter,
		decoration={
			show path construction,
			moveto code={
				\stepcounter{coordinateindex}
				\coordinate (#1-\thecoordinateindex) at (\tikzinputsegmentfirst);
			},
			lineto code={
				\stepcounter{coordinateindex}
				\coordinate (#1-\thecoordinateindex) at (\tikzinputsegmentlast);
			},
			curveto code={
				\stepcounter{coordinateindex}
				\coordinate (#1-\thecoordinateindex) at (\tikzinputsegmentlast);
			},
			closepath code={
				\stepcounter{coordinateindex}
				\coordinate (#1-\thecoordinateindex) at (\tikzinputsegmentlast);
			},
		},
		postaction={decorate}
	},
	put coordinates/.default=coordinate
}

\newcommand{\PrepareOcclusion}[2]%
{%
	\path[name path=AB] %
	\foreach \i in {1,...,\thecoordinateindex}{%
		(#1) -- ($(#1)!100cm!(#2-\i)$)%
	};%
	%
	\foreach \i  [count=\j] in {1,...,\thecoordinateindex}%
	{%
		\coordinate (#2-shadow-\i) at (#2-\i);%
	}%
	%
	\pgfresetboundingbox%
	\draw (-20,-20) rectangle (20,20);%
	\path[name path=CBB] (current bounding box.north west) -- (current bounding box.north east) -- (current bounding box.south east) -- (current bounding box.south west) -- cycle;%
	\path[execute at begin node={\global\let\t=\t}, name intersections={name=int-#2, of=AB and CBB, sort by=AB, total=\t}];
	%
	
	\foreach \i [count=\k from \thecoordinateindex+1] in {\t,...,1}%
	{%
		\def\nodename{#2}
		\global\expandafter\let\csname maxIdx\nodename\endcsname\k
		\coordinate (#2-shadow-\k) at (int-#2-\i);%
		
	}%
	\pgfresetboundingbox%
}%
%
\newcommand{\DrawOcclusion}[2]%
{%
	\def\maxIdx{\csname maxIdx#1\endcsname}
	%\node{\test};
	\draw[#2](#1-shadow-1)%
	\foreach \x in {2,3,...,\maxIdx}%
	{%
		-- (#1-shadow-\x)%
	} -- cycle;%
}%

\tikzstyle{function} = [shadedGray, thick, font=\bfseries, minimum width=13em,, inner ysep=0.75em, align=center]%
%
\tikzstyle{port} = [rectangle, shadedGray, minimum width=1.5em, minimum height=1.5em]%
%
\pgfdeclarearrow{%
	name=Contains,%
	parameters= {\the\pgfarrowlength},%
	setup code={%
		\pgfarrowssettipend{0pt}%
		\pgfarrowssetlineend{-\pgfarrowlength}%
		\pgfarrowlinewidth=\pgflinewidth%
		\pgfarrowssavethe\pgfarrowlength%
	},%
	drawing code={%
		\pgfpathcircle{\pgfpoint{-0.5\pgfarrowlength}{0pt}}{0.5\pgfarrowlength}%
		\pgfpathmoveto{\pgfpoint{-\pgfarrowlength}{0}}%
		\pgfpathlineto{\pgfpoint{0}{0}}%
		\pgfpathmoveto{\pgfpoint{-0.5\pgfarrowlength}{0.5\pgfarrowlength}}%
		\pgfpathlineto{\pgfpoint{-0.5\pgfarrowlength}{-0.5\pgfarrowlength}}%
		\pgfusepathqstroke%
	},%
	defaults = { length = 0.75em }%
}%w
%
\tikzset{%
	hw_base/.style={%
		function, %
		label={[anchor=north,#1]\emph{\guillemotleft block\guillemotright}},%
	},%
	hw/.style={%
		hw_base,%
		inner ysep=1.5em, text depth=1.5em%
	},%
	connection/.style args={#1#2}{%
		very thick,%
		postaction={%
			decorate,%
			decoration={%
				markings,%
				mark=at position 0.5 with {\arrow{Triangle[scale=2]}, \node[{#1,align=center, text height=1ex, text depth=0.25ex, font=\small}] {#2};}%
			}%
		},%
	},%
	req/.style args={#1#2#3#4}{%
		label={[anchor=north]\emph{\guillemotleft #1\guillemotright}},%
		align=center,%
		minimum width=7em,
		font=\scriptsize\bfseries,%
		text height=1.5em,%
		append after command={%
			\pgfextra{%
				\path let \p1=($(\tikzlastnode.east) - (\tikzlastnode.west)$) in node[{name=\tikzlastnode-sup, anchor=north, align=left, align=left, minimum width=\x1}] at (\tikzlastnode.south) {Id=``#2''\\Text=``#3''#4};%
				\draw[shadedGray] (\tikzlastnode-sup.north west) -- (\tikzlastnode-sup.north east);%
				\begin{pgfonlayer}{pre main}%
					\node[fit=(\tikzlastnode)(\tikzlastnode-sup), shadedGray,drop shadow, inner sep=0, name=\tikzlastnode-req]{};%
				\end{pgfonlayer}%
				%
			}%
		}%
	},%
	rel label/.style args={#1#2}{%
		postaction={ decorate,%
			decoration={ markings,%
				mark=at position 0.5 with \node[{#1,text height=1ex, text depth=0.25ex, font=\scriptsize}] {#2};%
			}%
		}%
	},%
	inport/.style={%
		port,%
		append after command={%
			\pgfextra{%
				\begin{pgfonlayer}{foreground}%
					\draw[thick,-stealth', shorten <=.4em, shorten >=.4em, gray ] (\tikzlastnode.west) -- (\tikzlastnode.east);%
				\end{pgfonlayer}%
			}%
		}%
	},%
	ioport/.style={%
		port,%
		append after command={%
			\pgfextra{%
				\begin{pgfonlayer}{foreground}%
					\draw[thick,stealth'-stealth', shorten <=.3em, shorten >=.3em, gray ] (\tikzlastnode.west) -- (\tikzlastnode.east);%
				\end{pgfonlayer}%
			}%
		}%
	},%
	outport/.style={%
		port,%
		append after command={%
			\pgfextra{%
				\begin{pgfonlayer}{foreground}%
					\draw[thick,-stealth', shorten <=.3em, shorten >=.3em, gray ] (\tikzlastnode.east) -- (\tikzlastnode.west);%
				\end{pgfonlayer}%
			}%
		}%
	},%
	hw_top_port/.style={%
		hw_base={yshift=-1em},%
		text height=2.75em,%
		inner ysep=0.5em,%
	},%
	derived-req/.style={%
		-{Straight Barb[length=0.5em,width=0.7em]},%
		dashed,%
		thick%
	},%
	contained-req/.style={%
		thick,%
		-Contains%
	},%
	association/.style ={%
		draw,
		thick
	},%
	usecase/.style={
		ellipse,
		shadedGray,
		drop shadow,
		inner sep=1em,
		align=center,
		font=\bfseries
	},%
	diagramStyle/.style={%
		shadedGray, drop shadow, inner ysep=2em, yshift=1em, inner xsep=0.5em
	},
	diagramFrame/.style={%
		thick,gray,fill=none
	}
}%

\tikzset{%
	set class/.code={\pgfqkeys{/tikz/class}{#1}},%
	set class={%
		stereotype/.store in=\myStereoType,%
		class name/.store in=\myClassName,%
		parts/.store in=\myPartNum,%
		stereotype=,%
		class name=myClass,%
		parts=1%
	},%
	class/.search also={/tikz},%	
	class/.style={%		
		set class={#1},%		
		align=center,%			
		inner ysep=0.25em,%				
		font=\bfseries,%
		shadedGray,drop shadow,%
		node contents={\myClassName},%
		minimum width=10em,
		class/add stereotype/.expand once=\myStereoType%
	},%
	class/.cd,
	add mystereotype/.code={%
		\def\argone{#1}
		\ifx\argone\empty
		\pgfkeysalso{minimum height=2.5em}
		\else
		\pgfkeysalso{
		label={[anchor=north]\emph{\guillemotleft\vphantom{ghA}{#1}\guillemotright}},%
		text width={max(6em, 1em + max(width(\string"#1\string"),width(\string"\myClassName\string"))))},%
		text height=2.2em,%
		text depth=0.25em,%
		}
		\fi
	},
	add stereotype/.style={%
		class/add mystereotype={#1}
	}
}

\tikzset{
	cog/.style={
		draw,
		thick,
		minimum size=0.3cm,
		centerofmass,
		cog rotate=#1
	},
	cog/.default=0,
	frame-arrow/.style={thick, -stealth'},
	vehicle-velocity/.style={frame-arrow, tuBlue},
	s-velocity/.style={frame-arrow, tuOrange},
	wheelforce/.style={frame-arrow, tuRed}
}
\pgfkeyssetvalue{/tikz/cog rotate}{0}%
\tikzset{sFrameInvisible/.style={opacity=0}}%
%\tikzset{sFrameInvisible/.style={opacity=1}}%


%\tikzset{wheelFrameInvisible/.style={opacity=0}}
\tikzset{wheelFrameInvisible/.style={opacity=1}}

%\tikzset{trajectoryInvisible/.style={opacity=0}}
\tikzset{trajectoryInvisible/.style={opacity=1}}


\newcommand\normalize[4][]{%
	\draw[-stealth, thick, #1] #3 -- ($#3!#2!#4$);}

\def\centerarc[#1](#2)(#3:#4:#5)% Syntax: [draw options] (center) (initial angle:final angle:radius)
{ \draw[#1] ($(#2)+({#5*cos(#3)},{#5*sin(#3)})$) arc (#3:#4:#5); }

\def\labeledcenterarc[#1](#2)(#3,#4)(#5:#6:#7)% Syntax: [draw options] (center) (initial angle:final angle:radius)
{ \draw[#1] ($(#2)+({#7*cos(#5)},{#7*sin(#5)})$) arc (#5:#6:#7) node[midway,#4] {#3}; }    

\tikzset{wheel/.style={draw=gray, thick, rounded corners=.1em, fill=lightgray!50}}


\makeatletter
\tikzset{%
	reverse path/.style={
		decoration={show path construction,
			reverse path,
			moveto code={\pgfpathmoveto{\pgf@decorate@inputsegment@first}},
			lineto code={\pgfpathlineto{\pgf@decorate@inputsegment@last}},
			curveto code={\pgfpathcurveto{\pgf@decorate@inputsegment@supporta}%
				{\pgf@decorate@inputsegment@supportb}{\pgf@decorate@inputsegment@last}},
			closepath code={\pgfpathclose}
		},
		decorate
	}
}
\makeatother

\tikzset{%
	generalization/.append style={%
		reverse path
	},
	specialization/.append style={%
	reverse path
},	
}

\tikzset{%
	set association/.code={\pgfqkeys{/tikz/association}{#1}},%
	set association={%
		pos1/.store in=\fromPos,%
		pos2/.store in=\toPos,%
		mult1/.store in=\fromMult,%
		mult2/.store in=\toMult,%
		arg1/.store in=\fromArg,%
		arg2/.store in=\toArg,%
		argpos1/.store in=\fromArgPos,%
		argpos2/.store in=\toArgPos,%
		multpos1/.store in=\fromMultPos,%
		multpos2/.store in=\toMultPos,%
		stereotype/.store in=\stereotype,%
		pos/.store in=\sPos,
		pos1=1em,%
		pos2=-1.5em,%
		mult1=,%
		mult2=,%
		arg1=,%
		arg2=,%
		argpos1=,
		argpos2=,
		multpos1=,
		multpos2=,
		stereotype=,%
		pos=0.5%
	},%
	association/.search also={/tikz},%	
	association/.style={%
		set association={#1},%
		draw,%
		thick,%
		postaction={decorate},%
		decoration={%
			markings,%			
			mark=at position \fromPos-0.02 with%
			{%
				\node (A) {};%	
				\pgfgetlastxy{\XCoord}{\YCoord}%
				\global\let\oldX\XCoord%
				\global\let\oldY\YCoord%
				%\node{\oldX, \YCoord};
			},%
			mark=at position \fromPos with%
			{%				
				\node (B){};	%
				\pgfmathsetmacro\lenArg{width("\fromArg")}				
				\pgfmathsetmacro\lenMult{width("\fromMult")}
				\pgfgetlastxy{\xlast}{\ylast}%
				\pgfmathsetmacro{\xDiff}{(abs(\xlast)-abs(\oldX))*100}%
				\pgfmathsetmacro{\yDiff}{(abs(\ylast)-abs(\oldY))*100}%
				\pgfmathsetmacro{\intXDiff}{int((\xlast-\oldX)*100)}
				%\pgfmathsetmacro{\yDiff}{(\ylast-\oldY)*100}%
				\pgfmathsetmacro{\labelDiff}{int((abs(\lenArg)-abs(\lenMult))*100)))}
				\def\placeFromName{left}%
				\def\placeFromMult{right}%
				\def\anchorFromMult{west}%
				\def\anchorFromName{east}%
				\pgfmathparse{int(abs(\yDiff) - abs(\xDiff))}%
				\pgfmathsetmacro{\hvDiff}{\pgfmathresult}
				\ifnum\pgfmathresult<0%	
				\edef\myDebug{1}			
				\edef\placeFromName{above}%
				\edef\placeFromMult{below}%
				\ifnum\labelDiff<0%
				\ifnum\intXDiff>0%
				\edef\anchorFromName{north west}%
				\edef\anchorFromMult{south west}%
				\else%
				\edef\anchorFromName{south east}%
				\edef\anchorFromMult{north east}%
				\fi%
				\else%
				\ifnum\intXDiff>0%
				\edef\anchorFromName{north west}%
				\edef\anchorFromMult{south west}%
				\else%
				\edef\anchorFromName{north east}%
				\edef\anchorFromMult{south east}%
				\fi%
				\fi%					
				\else%
				\fi%	
				\node[\placeFromName, anchor=\anchorFromName] {\fromArg};%				
				\node[\placeFromMult, anchor=\anchorFromMult]{\vphantom{\fromArg}\fromMult};%			
			},%
			mark=at position \sPos with
			{	
				\node[above, transform shape] {\stereotype};								
			},%			
			mark=at position \toPos-0.02 with
			{
				\node (D){};	
				\pgfgetlastxy{\xlast}{\ylast}
				\global\let\oldX\xlast
				\global\let\oldY\ylast				
			},
			mark=at position \toPos with
			{
				\node(E){};
				\pgfmathsetmacro\lenArg{width("\toArg")}				
				\pgfmathsetmacro\lenMult{width("\toMult")}
				\pgfgetlastxy{\xlast}{\ylast}%
				\pgfmathsetmacro{\xDiff}{(abs(\xlast)-abs(\oldX))*100}%
				\pgfmathsetmacro{\yDiff}{(abs(\ylast)-abs(\oldY))*100}%
				\pgfmathsetmacro{\intXDiff}{int((\xlast-\oldX)*100)}
				\pgfmathsetmacro{\labelDiff}{int((\lenArg-\lenMult)*100)))}
				\def\placeToName{left}%
				\def\placeToMult{right}%
				\def\myDebug{}
				\def\anchorToMult{west}%
				\def\anchorToName{east}%
				\pgfmathparse{int(abs(\yDiff) - abs(\xDiff))}%
				\ifnum\pgfmathresult<0%			
				\edef\placeToName{above}%
				\edef\placeToMult{below}%
				\ifnum\labelDiff<0%
				\ifnum\intXDiff>0%
				\edef\anchorToName{south west}%
				\edef\anchorToMult{north west}%
				\else%
				\edef\anchorToName{south east}%
				\edef\anchorToMult{north east}%
				\fi%
				\else%
				\ifnum\intXDiff>0%
				\edef\anchorToName{north west}%
				\edef\anchorToMult{south west}%
				\else%
				\edef\anchorToName{north east}%
				\edef\anchorToMult{south east}%
				\fi%
				\fi%					
				\else%
				\fi%		
				\node[\placeToName] {\toArg};				
				\node[\placeToMult]{\vphantom{\toArg}\toMult};
			},%	
		},%			
	},%
}%
%
\tikzset{%
	rel label/.style args={#1#2#3}{%
		postaction={ decorate,%
			decoration={ markings,%
				mark=at position #1 with \node[{#2,text height=1.5ex, text depth=0.25ex}] {#3};%
			}%
		}%
	},%
	TLeft/.style={label={[anchor=east, xshift=1ex]left:$\blacktriangleleft$}},%
	TRight/.style={label={[anchor=west, xshift=-1ex]right:\strut$\blacktriangleright$}},%
	TUp/.style={label={[anchor=south, yshift=-1ex]above:$\blacktriangle$}},%
	TDown/.style={label={[anchor=north, yshift=1ex]below:$\blacktriangledown$}},%
	closer pos/.style={},%
	class/.append style={minimum width=6.5em, inner ysep=0.5em, inner xsep=0.1em, minimum height=2.75em, font=\bfseries\footnotesize},%
	assoc label/.style={align=center, font=\scriptsize},%
	aggregation/.style={%
		association={#1},%
		-{Diamond[open, length=1.2em]}%
	},%
	generalization/.style={%%
		draw,
		-{Triangle[open, length=0.7em, width=0.8em]}%
	},%
	specialization/.style={%%
		draw,
		-{Triangle[length=0.7em, width=0.8em]}%
	},%	
	composition/.style={%
		association={#1},
		-{Diamond[length=1.3em]}
	},
	part composite/.style={%
		association={#1},
		{Straight Barb}-{Diamond[length=1.1em]}
	},
	part reference/.style={%
		association={#1},
		{Straight Barb}-{Diamond[open,length=1.1em]}
	}
}%
%
\newcounter{samplingidx}%
\newcounter{coordindex}%
\newcounter{constraintidx}%
\tikzset{%
	stepcounter-coord/.code={%
		\stepcounter{coordindex}%
	},%
	stepcounter-sample/.code={%
		\stepcounter{samplingidx}%
	},%
	stepcounter-constraint/.code={%
		\stepcounter{constraintidx}%
	}%
}%
%
\tikzset{%
	polyline/.style={%
		decoration={%
			markings,%
			mark=between positions 0 and 1 step 0.05 with {%
				\coordinate[stepcounter-coord] (A\thecoordindex) at (0,0);%
			}%
		},%
		postaction=decorate,%
	}%
}%
%
%
\tikzset{%
	tangent/.style={%
		decoration={%
			markings,% switch on markings
			mark=%
			at position #1%
			with%
			{%
				\coordinate (tangent point-\pgfkeysvalueof{/pgf/decoration/mark info/sequence number}) at (0pt,0pt);%
				\coordinate (tangent unit vector-\pgfkeysvalueof{/pgf/decoration/mark info/sequence number}) at (1,0pt);%
				\coordinate (tangent orthogonal unit vector-\pgfkeysvalueof{/pgf/decoration/mark info/sequence number}) at (0pt,1);%
			}%
		},%
		postaction=decorate%
	},%
	use tangent/.style={%
		shift=(tangent point-#1),%
		x=(tangent unit vector-#1),%
		y=(tangent orthogonal unit vector-#1)%
	},%
	use tangent/.default=1%
}%
%
\tikzset{%
	constraint line/.style={%
		decoration={%
			markings,%
			mark=between positions 0 and 0.85 step 0.05 with {%
				\node[stepcounter-constraint, circle, inner sep=1pt, fill=tuRed] (lwr-c-\theconstraintidx) at (0,\lowerconstraints[\theconstraintidx]) {};%
				\draw[tuLightGreen](0,0) -- (lwr-c-\theconstraintidx);%
				\draw[tuLightGreen] (0,0) -- (0,3);%
			}%
		},%
		postaction=decorate,%
	}%
}%
\tikzset{%
	sampling line/.style={%
		decoration={%
			markings,
			mark=between positions 0 and 1 step 0.05 with {%
				\coordinate[stepcounter-sample] (B\thesamplingidx-left) at (0,3);
				\coordinate[] (B\thesamplingidx-right) at (0,-6);
			}			
		},
		postaction=decorate,
	}
}
\tikzset{cross/.style={cross out, draw=black, minimum size=2*(#1-\pgflinewidth), inner sep=0pt, outer sep=0pt},	cross/.default={1pt}}

\pgfdeclarelayer{background}%
\pgfdeclarelayer{pre main}%
\pgfdeclarelayer{foreground}%
\pgfdeclarelayer{lifelines}%
\pgfdeclarelayer{activity}%
\pgfdeclarelayer{connections}%
\pgfsetlayers{background,lifelines,activity, connections,pre main,main,foreground}%