%% 
%% Copyright 2007-2025 Elsevier Ltd
%% 
%% This file is part of the 'Elsarticle Bundle'.
%% ---------------------------------------------
%% 
%% It may be distributed under the conditions of the LaTeX Project Public
%% License, either version 1.3 of this license or (at your option) any
%% later version.  The latest version of this license is in
%%    http://www.latex-project.org/lppl.txt
%% and version 1.3 or later is part of all distributions of LaTeX
%% version 1999/12/01 or later.
%% 
%% The list of all files belonging to the 'Elsarticle Bundle' is
%% given in the file `manifest.txt'.
%% 
%% Template article for Elsevier's document class `elsarticle'
%% with harvard style bibliographic references

\documentclass[preprint,12pt]{elsarticle}

%% Use the option review to obtain double line spacing
%% \documentclass[preprint,review,12pt]{elsarticle}

%% Use the options 1p,twocolumn; 3p; 3p,twocolumn; 5p; or 5p,twocolumn
%% for a journal layout:
%% \documentclass[final,1p,times]{elsarticle}
%% \documentclass[final,1p,times,twocolumn]{elsarticle}
%% \documentclass[final,3p,times]{elsarticle}
%% \documentclass[final,3p,times,twocolumn]{elsarticle}
%% \documentclass[final,5p,times]{elsarticle}
%% \documentclass[final,5p,times,twocolumn]{elsarticle}

%% For including figures, graphicx.sty has been loaded in
%% elsarticle.cls. If you prefer to use the old commands
%% please give \usepackage{epsfig}

%% The amssymb package provides various useful mathematical symbols
\usepackage{amssymb}
%% The amsmath package provides various useful equation environments.
\usepackage{amsmath}
%% The amsthm package provides extended theorem environments
%% \usepackage{amsthm}

%% The lineno packages adds line numbers. Start line numbering with
%% \begin{linenumbers}, end it with \end{linenumbers}. Or switch it on
%% for the whole article with \linenumbers.
\usepackage{lineno}

\usepackage{hyperref}

\journal{ } %Fluid Phase Equilibria

\begin{document}

\begin{frontmatter}

%% Title, authors and addresses

%% use the tnoteref command within \title for footnotes;
%% use the tnotetext command for theassociated footnote;
%% use the fnref command within \author or \affiliation for footnotes;
%% use the fntext command for theassociated footnote;
%% use the corref command within \author for corresponding author footnotes;
%% use the cortext command for theassociated footnote;
%% use the ead command for the email address,
%% and the form \ead[url] for the home page:
%% \title{Title\tnoteref{label1}}
%% \tnotetext[label1]{}
%% \author{Name\corref{cor1}\fnref{label2}}
%% \ead{email address}
%% \ead[url]{home page}
%% \fntext[label2]{}
%% \cortext[cor1]{}
%% \affiliation{organization={},
%%             addressline={},
%%             city={},
%%             postcode={},
%%             state={},
%%             country={}}
%% \fntext[label3]{}

%% Article title
\title{Graph Neural Networks embedded into Margules model for vapor-liquid equilibria prediction} 

%% Author names and emails
\author[mpi_affiliation]{Edgar Ivan Sanchez Medina\corref{cor1}}
\ead{sanchez@mpi-magdeburg.mpg.de}

\author[mpi_affiliation,ovgu_affiliation]{Kai Sundmacher}

\cortext[cor1]{Corresponding author}

%% Author affiliation
\affiliation[mpi_affiliation]{organization={Process Systems Engineering, Max Planck Institute for Dynamics of Complex Technical Systems},
            addressline={Sandtorstraße 1}, 
            city={Magdeburg},
            postcode={39106}, 
            state={Saxony-Anhalt},
            country={Germany}}

\affiliation[ovgu_affiliation]{organization={Chair for Process Systems Engineering, Otto-von-Guericke University},
            addressline={Universitätsplatz 2}, 
            city={Magdeburg},
            postcode={39106}, 
            state={Saxony-Anhalt},
            country={Germany}}

%% Abstract
\begin{abstract}
Predictive thermodynamic models are crucial for the early stages of product and process design. In this paper the performance of Graph Neural Networks (GNNs) embedded into a relatively simple excess Gibbs energy model, the extended Margules model, for predicting vapor-liquid equilibrium is analyzed. By comparing its performance against the established UNIFAC-Dortmund model it has been shown that GNNs embedded in Margules achieves an overall lower accuracy. However, higher accuracy is observed in the case of various types of binary mixtures. Moreover, since group contribution methods, like UNIFAC, are limited due to feasibility of molecular fragmentation or availability of parameters, the GNN in Margules model offers an alternative for VLE estimation. The findings establish a baseline for the predictive accuracy that simple excess Gibbs energy models combined with GNNs trained solely on infinite dilution data can achieve.
\end{abstract}

%%Graphical abstract
% \begin{graphicalabstract}
% \includegraphics{elsarticle/figures/graphical_abstract.png}
% \end{graphicalabstract}

% %%Research highlights
% \begin{highlights}
% \item Graph Neural Networks embedded into the extended Margules model for Gibbs-Duhem consistent predictions
% \item Analysis of binary and ternary vapor-liquid equilibria prediction using a hybrid machine learning - mechanistic model
% \end{highlights}

%% Keywords
\begin{keyword}
graph neural networks \sep vapor-liquid equilibria \sep Margules \sep activity coefficients
\end{keyword}

\end{frontmatter}

%% Add \usepackage{lineno} before \begin{document} and uncomment 
%% following line to enable line numbers
%\linenumbers

%% main text
%%

\section{Introduction}

\section{Introduction}

\begin{figure*}
    \centering
    \includegraphics[width=\textwidth]{figures/Introduction.pdf}
    \caption{Showing the novel problem statement applied to traffic prediction use case. Multiple unstructured observations from the past are used to reconstruct a hidden traffic state from which a full traffic state is forecast with a set of query locations. }
    \label{fig:intro}
\end{figure*}

% Was sagen denn die anderen warum Traffic Prediction gut ist? 
Forecasting the traffic in the near future is an important task for city management.
Data from the near past is used to predict future traffic states with spatio-temporal Graph Neural Networks \cite{bui22}.
Accurate prediction provides the opportunity to optimize traffic flow, reduce traffic jams and increase air quality \cite{Po19}.

% Wieso ist Sparsity in allen Dimensionen wichtig.
While traffic prediction relies on the availability of data from traffic sensors, there exists a plethora of reasons why sensors may stop working temporarily, such as simple errors, energy saving, or overloaded communication systems.
Considering small- or medium-sized cities, the coverage of sensors may be low because the sensors are too expensive or not available.
Also, the sensors are typically static and do not adapt to changes in the traffic flow (e.g. caused by a construction site), which motivates moving sensors that for example could be mounted on cars. 
However, both missing and moving sensors introduce sparsity, since measurements may not be available for all locations at all times.
This sparsity must be explicitly addressed in traffic prediction for a realistic application scenario, which is illustrated in figure \ref{fig:intro}.
From one hour of data on Sunday morning, only few observations of the traffic state are available at each timestep.
The number of observations may differ throughout the observed time and the observation itself can be distributed arbitrarily in the city. 
We assume a relatively low number of sensors to account for resource saving and sensor failure in our proposed framework SUSTeR.
The task is to predict the dense traffic state one timestep after the observations at all possible sensor locations.
We study this problem on the traffic dataset Metr-LA and PEMS-BAY to test our assumption that only a fraction of the sensor values would be enough for good predictions.
By modifying an existing traffic dataset, we are able to compare our results from very sparse observations to the bottom line with all information available.
A successful study will provide insights in how sensors in new cities can be reduced before installing them and further mobile sensors would save more resources and are able to adapt to new traffic situations.
We argue that in order to be adaptable to other cities and changes in traffic flows, prior information like the road network should be neglected and just the sparse observations considered.
This comes with the added benefit of making our solution applicable in regions where no openly available road network is maintained or pathways change frequently (e.g. flood areas, animal observations). 


The aforementioned problem is novel and more challenging than the commonly considered traffic prediction problem, since there exist very few observations in each input sample.
Current works for the traffic prediction problem do not consider any missing values. \cite{Li2021, Shao22}
A common method among state of the art approaches is the usage of Graph Neural Networks on graphs that model the sensor network \cite{bui22}.
The values of a sensor are applied to the same graph node for each timestep which prohibits any non-stationary sensors . 
With fixed sensor locations, the resulting sensor network is highly correlated with the road network.
Streets connecting two intersections with sensors should be also an interesting point for correlations in the sensor network.
However, variable observations and high temporal sparsity rates can not be modeled adequately in a static network.
We show in our experiments that the road network has only a small influence on the traffic predictions.

Besides the traffic prediction for future timesteps, some works explore the field of traffic speed imputation \cite{Cini22, Cuza22} where missing sensor values are predicted.
But the amount of missing values is assumed to be at most 80\%, which on average are still over 40 given sensors in each timestep in the Metr-LA dataset with a total of 207 sensors.
We consider up to 99.9\% missing values which are on average 2.4 observations in each timestep that are used as input.
Such high sparsity rates drastically decrease the chance that multiple values are present in one input sample from the same sensor location, which makes it challenging to recognize and learn temporal correlations for each location on its own.

High sparsity rates (>95\%) result in few sensor values, but if a reconstruction of the traffic state would be possible, we question if spatio-temporal graphs require nodes for each sensor.
In SUSTeR we utilize only a small amount of graph nodes for the encoding of information and do not relate such nodes to the sensor network.
We call this the hidden graph (see figure \ref{fig:intro}), which is still able to reconstruct the complete traffic state.
Due to the reduced number of nodes SUSTeR achieves faster runtimes, as shown in the experiments.
This hidden graph is not embedded directly in the spatial domain, which is why the assignment of observations, as well as the querying of the future traffic, is done with an encoder and a decoder, implemented as neural networks.
The decoding from the hidden graph to future values depends on a set of query locations.
Figure \ref{fig:intro} shows the query locations as given from outside and in combination with the reconstructed traffic state the future values are predicted.

To construct the hidden graph we encode observations from each timestep into from multiple graphs, one for each timestep. 
The graphs are created in a residual style and information is added to the node embeddings from the previous timesteps.
We choose this method to incorporate all timesteps equally into the hidden state because the redundant information along the past is non-existing for high sparsity rates.
From the sequence of graphs where our framework inserted the observations step by step we apply STGCN \cite{Yu18}, an algorithm for traffic prediction to find and learn the spatio-temporal correlations on our small number of graph nodes.
The first future timestep of the STGCN is our hidden graph in which the traffic state is reconstructed. 

% Recent work has an implicit embedding of the graph nodes into the spatial domain as the assignment from the sensor to graph node is fixed one by one.
% Because the graph has the same structure as the road network spatio-temporal correlations can be learned between those sensors.
% We reduce the number of nodes and use a non-linear assignment learned data-driven from the observations.

We find in the experiments that SUSTeR outperforms the plain STGCN and modern traffic prediction frameworks like D2STGNN for high sparsity rates $(\geq 99\%)$.
This is equivalent to only $0.2$ to $2.4$ observation for each timestep on average.
SUSTeR uses fewer parameters than the baselines and can train faster and with less training data.
Our main contributions can be summarized as follows:
\begin{itemize}
    \item We introduce a sparse and unstructured variant of the traffic prediction problem with sparsity in all dimensions. The sensors report only a fraction of their values and are arbitrarily distributed in the spatial domain.
    \item We propose SUSTeR, a framework around the STGCN architecture, which maps sparse observations onto a dense hidden graph to reconstruct the complete traffic state.
    Our code is available at github.\footnote{https://github.com/ywoelker/SUSTeR}
    \item We conducts experiments that show that SUSTeR outperforms the baselines in very sparse situations ($\geq 95\%$) and has a competitive performance in low sparsity rates.
    % \item SUSTeR trains a third faster than the next competitor.
\end{itemize}


\section{Methods}
\label{sec:methods}

\subsection{Vapor-liquid equilibria}

% Background on vapor-liquid equilibria calculation
Vapor-liquid equilibria (VLE) is reached when the fugacity of each component $i$ in the liquid phase $L$ and the vapor phase $V$ are equal

\begin{equation}
    f_i^L = f_i^V
    \label{eq_equal_fugacities_vle}
\end{equation}

Two main approaches exist for calculating Equation \ref{eq_equal_fugacities_vle}. The first approach utilizes fugacity coefficients, $\varphi_i$, for both phases, typically derived from an equation of state and, in the case of mixtures, a set of mixing rules. The second approach employs the fugacity coefficient for the vapor phase, along with the activity coefficient, $\gamma_i$, and a standard fugacity, $f_i^0$, for the liquid phase. The standard fugacity is calculated from the fugacity coefficient of the pure liquid at saturation conditions, $\varphi^s_i$, the saturation pressure, $P^s_i$, and the Poynting factor, $\text{Poy}_i$. Therefore, by using the second approach, the VLE of mixture components can be calculated as

\begin{equation}
    x_i \gamma_i \varphi_i^s P_i^s \text{Poy}_i = y_i \varphi_i^V P
    \label{eq_vle_general}
\end{equation}

\noindent where, $x_i$ and $y_i$ refer to the liquid and vapor molar fractions, respectively, and $P$ stands for the system's pressure \cite{gmehling2019chemical}. 

At low or moderate pressures, the $\text{Poy}_i$ is typically close to unity, and for non-associating systems, the values of $\varphi_i^s$ and $\varphi_i^V$ are very similar. Therefore, for many practical applications, it is sufficient to approximate Equation \ref{eq_vle_general} with the following expression:

\begin{equation}
    x_i \gamma_i  P_i^s  \approx y_i P
    \label{eq_vle_extended_raoults_law}
\end{equation}

In this work, Equation \ref{eq_vle_extended_raoults_law} was used to estimate the VLE behavior of components in binary and ternary mixtures of small non-ionic compounds. 

We leverage the previously published GH-GNN model \cite{medina2023gibbs} for predicting infinite dilution activity coefficients (IDACs) and use the extended Margules model to extrapolate the predictions from the infinite dilution regime to finite concentrations. An early analysis of this approach was studied by \citet{medina2023solvent} in the context of solvent pre-selection for extractive distillation, but a rigorous analysis of the approach for predicting binary and ternary VLEs was not there presented. 

\subsection{Extended Margules model}

Different mathematical models have been developed to calculate the molar excess Gibbs energy, $g^E$, of a mixture. Since the concept of $g^E$ is constructed from the idea of modeling a correction to the ideal solution model, any valid $g^E$ model should just satisfy the following limiting condition:

\begin{equation}
    g_i^E \rightarrow 0, \quad \text{as} \quad x_i \rightarrow 0
    \label{eq_boundary_condition}
\end{equation}

One of the simplest approaches to this is to assume that $g^E$ is a continuous and sufficiently smooth function, which can be approximated by an n-th order Taylor series expansion. This approximation results in what is known as the extended Margules model \cite{mukhopadhyay1993discussion}.
A detail derivation is provided in the \ref{app1}.

For a binary mixture, the extended Margules model estimates $g^E$ as:

\begin{equation}
    g^E = x_i x_j (x_i w_{ji} + x_j w_{ij})
    \label{eq_ge_binary_final}
\end{equation}

\noindent where, $w_{ji} = \ln{\gamma_{ji}^\infty}$ and $w_{ij} = \ln{\gamma_{ij}^\infty}$. The activity coefficients at finite concentrations can be then calculated as:

\begin{align}
    \ln{\gamma_i} =& 2 w_{ji} x_i x_j + x_j^2 w_{ij} - 2 g^E \\
    \ln{\gamma_j} =& 2 w_{ij} x_j x_i + x_i^2 w_{ji} - 2 g^E
\end{align}

Similar expressions can be obtained for a ternary mixture, which are outline in the \ref{app1}.

\subsection{Data sources}

In this study, we utilized the open-source GH-GNN model \cite{medina2023gibbs}, trained on experimental IDAC data compiled in Volume IX of the DECHEMA Data Chemistry Series \cite{dechema}. Additionally, experimental binary VLE data sourced from the Korean Data Bank \cite{koreandatabank} was incorporated into our analysis. Both isobaric and isothermal VLE measurements were employed. Since Equation \ref{eq_vle_extended_raoults_law} was used to model the VLE behavior, only systems operating at low to medium pressure levels (i.e., $\leq 500$ kPa) were considered.

The CAS-RN and vapor pressure correlation coefficients of the pure compounds were also obtained from the Korean Data Bank \cite{koreandatabank}. The vapor pressure correlation is defined as:

\begin{equation}
    \ln{P_i^s} = A_i \ln{(T)} + \frac{B_i}{T} + C_i + D_i T^2
    \label{eq_vapor_pressure_kdb}
\end{equation}

\noindent where the pressure $P_i^s$ is expressed in kPa and the temperature $T$ in K. Mixtures containing compounds lacking vapor pressure correlation coefficients ($A_i$, $B_i$, $C_i$ and $D_i$), as well as systems with temperatures outside the correlation’s valid range, were excluded from the analysis.

Additionally, 10 subsets of VLE data for ternary mixtures were collected from the literature. Table \ref{tbl:ternary_systems} describes the specifics of each subset. While limited, the inclusion of ternary VLE systems in the present analysis revels the relative performance of the graph neural networks embedded into the Margules model for multi-component systems. These mixtures include a variety of chemical classes, including aromatics, paraffinic hydrocarbons, nitriles, ketones, alcohols, as well as halogenated and sulfur-containing compounds. In total, 539 data points of ternary mixtures were analyzed.


\begin{table}[h]
\centering
  \caption{Ternary vapor-liquid equilibria data from literature analyzed in this work.}
  \label{tbl:ternary_systems}
  \small
  \begin{tabular*}{1\textwidth}{@{\extracolsep{\fill}}lccccl}
    \hline
    \textbf{Mixture (Component 1/2/3)} & \textbf{\# points} & \textbf{Condition} & \textbf{Ref.} \\
    \hline
    Acetone/Chloroform/Methanol          & 71  & 101.325 kPa  & \cite{hiak1994vapor} \\
    Hexane/Benzene/Sulpholane            & 14  & 101.325 kPa  & \cite{rawat1980isobaric} \\
    Benzene/Heptane/Dimethylformamide    & 50  & 101.325 kPa  & \cite{blanco2000vapor} \\
    Benzene/Heptane/Acetonitrile         & 12  & 101.325 kPa  & \cite{tripathi1975isobaric} \\
    Acetone/Chloroform/Benzene           & 53  & 101.325 kPa  & \cite{kojima1991isobaric} \\
    Benzene/Cyclohexane/Hexane           & 108 & 101.325 kPa  & \cite{ridgway1967physical} \\
    Acetone/Tetrachloromethane/Benzene   & 57  & 101.325 kPa  & \cite{subbarao1966isobaric} \\
    Ethanol/Benzene/Heptane              & 50  & 53.329 kPa   & \cite{nielsen1959vapor} \\
    Hexane/Methanol/Acetone              & 54  & 313.15 K     & \cite{oracz1995vapour} \\
    Chloroform/Methanol/Benzene          & 70  & 101.325 kPa  & \cite{kurihara1998vapor} \\
    \hline
  \end{tabular*}
\end{table}


\subsection{Data cleaning}

All pure components were encoded using SMILES strings retrieved either from the Korean Data Bank \cite{koreandatabank} directly or from PubChem \cite{pubchem} based on their corresponding CAS-RN. When isomeric information was available, isomeric SMILES were used; otherwise, canonical SMILES were employed. If a CAS-RN was not available from the Korean Data Bank \cite{koreandatabank}, SMILES strings were obtained using the OPSiN web tool \cite{opsin}. All data points with compounds for which the SMILES could not be determined were excluded. In total, the SMILES for 110 compounds could not be determined.

The VLE data underwent pre-processing to eliminate entries with decimal point misplacements or inaccurate composition values, such as those where the total molar fractions exceeded unity. In some cases the indexing of compounds with respect to the VLE data was reversed, this was also corrected. VLEs with suspected errors were excluded. All datasets were standardized to consistent units of measurement. Only measurements that provided both liquid and vapor molar fractions were included. Additionally, systems containing molecules with structural features not compatible with the molecular graph construction criteria used to train the GH-GNN \cite{medina2023gibbs} were excluded from the analysis.

The chemical classification of pure compounds from the Korean Data Bank was utilized in this work to evaluate the performance of the proposed framework across different mixture classes. Additionally, following the usage recommendations of the GH-GNN model \cite{medina2023gibbs}, only systems containing compounds with a Tanimoto similarity metric greater than 0.4 were considered. The specifics of this metric are detailed in the original publication \cite{medina2023gibbs} and are intended to control the expected prediction error of the GH-GNN model when applied to systems outside of its training domain.

The resulting VLE dataset of binary mixtures comprises 27,610 data points, involving 235 distinct compounds across 945 unique binary combinations. In total, 781 subsets are isobaric, 903 are isothermal, and an additional 51 subsets were collected under varying conditions. The dataset spans pressures from 0.01 to 499.50 kPa and temperatures ranging from 127.59 to 576.93 K.
 
\subsubsection{Gibbs-Helmholtz Graph Neural Network (GH-GNN)}

The GH-GNN model \cite{medina2023gibbs} represents molecules as graphs following the framework of \citet{battaglia2018relational}, incorporating node-, edge-, and global-level features. These vectorized features encode specific molecular information: atoms are described at the node level, chemical bonds at the edge level, and properties such as polarity and polarizability at the global level. This model is tailored to predict IDACs of small non-ionic compounds. However, extensions of this model have been investigated to handle polymer solutions \cite{sanchez2023gibbs}, ionic liquids \cite{medina2024systematic} and deep eutectic solvents \cite{morales2024graph}.

Graphs representing solutes and solvents are processed using a molecular-level graph neural network (GNN). This GNN updates the graph information in three sequential steps. 

First, the edge features between each pair of nodes $v$ and $w$ are updated according to:

\begin{equation}
\label{eqn:edge_update}
\mathbf{b}_{v,w}^{(l+1)} = \phi_b^{(l)}
\left(
\mathbf{a}_{v}^{(l)} \parallel \mathbf{a}_{w}^{(l)} \parallel \mathbf{b}_{v,w}^{(l)} \parallel \mathbf{u}^{(l)}
\right)
\end{equation}

\noindent where $\mathbf{b}_{v,w}^{(l+1)}$ is the updated edge feature vector, $\mathbf{a}_v^{(l)}$ and $\mathbf{a}_w^{(l)}$ are the feature vectors of nodes $v$ and $w$, respectively, $\mathbf{u}^{(l)}$ represents the global feature vector. The symbol $\parallel$ represents concatenation.

Second, after updating all edges in the graph, the node features are updated according to Equation \ref{eqn:node_update}.

\begin{equation}
   \widehat{\mathbf{b}}_{v}^{(l)} = \sum_{w \in \mathcal{N}(v)} \mathbf{b}_{v,w}^{(l+1)}
\end{equation}

\begin{equation}
\label{eqn:node_update}
   \mathbf{a}_v^{(l+1)} = \phi_a^{(l)}  
   \left( 
        \mathbf{a}_{v}^{(l)} \parallel \widehat{\mathbf{b}}_{v}^{(l)} \parallel \mathbf{u}^{(l)}
   \right)
\end{equation}

\noindent where $\widehat{\mathbf{b}}_{v}^{(l)}$ stands for the aggregated edge information of all edges connecting node $v$ to its neighbors $w \in \mathcal{N}(v)$.

Third, the global feature vector $\mathbf{u}$ is updated according to:

\begin{equation}
   \widetilde{\mathbf{a}}^{(l)} = \frac{1}{n_a} \sum_{v \in \mathcal{V}} \mathbf{a}_v^{(l+1)}
\end{equation}

\begin{equation}
   \widetilde{\mathbf{b}}^{(l)} = \frac{1}{n_b} \sum_{e \in \mathcal{E}} \mathbf{b}_{e}^{(l+1)}
\end{equation}

\begin{equation}
\label{eqn:global_update}
   \mathbf{u}^{(l+1)} = \phi_u^{(l)}  
   \left( 
        \mathbf{u}^{(l)} \parallel \widetilde{\mathbf{a}}^{(l)} \parallel \widetilde{\mathbf{b}}^{(l)}
   \right)
\end{equation}

\noindent where $\widetilde{\mathbf{a}}^{(l)}$ and $\widetilde{\mathbf{b}}^{(l)}$ represent the average node and edge embeddings, respectively. The sets $\mathcal{V}$ and $\mathcal{E}$ stand for the sets of nodes and edges in the graph, and $n_a$ and $n_b$ represent the cardinality of such sets, respectively.

The update functions for the edges $\phi_b^{(l)}$, nodes $\phi_a^{(l)}$ and global features $\phi_u^{(l)}$ are all implemented as single hidden-layer neural networks with the ReLU activation function.

After performing 2 sequential message passing transformations (described by Equations \ref{eqn:edge_update} to \ref{eqn:global_update}), the final graphs are pooled into vector embeddings. These embeddings are then combined to form a new graph representing the mixture, where nodes correspond to chemical species and edges denote hydrogen-bonding interactions. This mixture graph is subsequently passed through a second, mixture-level GNN, and the output is pooled into a vector representing the mixture. This vector embedding is finally used to estimate temperature-independent parameters in an expression derived from the Gibbs-Helmholtz equation, enabling the prediction of IDACs.

In this work, the GH-GNN model is embedded into the extended Margules model. This combination allows the GH-GNN to predict the parameters of the extended Margules model, corresponding to the respective binary IDACs, which are then used to calculate finite concentration activity coefficients. By embedding the GH-GNN into the Margules framework, we ensure that the predictions remain Gibbs-Duhem consistent. In the following sections, we analyze the performance of this combined model in predicting vapor-liquid equilibria and compare its results with those of the widely used UNIFAC-Dortmund \cite{constantinescu2016further} model.






\section{Results and discussion}
\label{sec:results_and_discussion}

\subsection{Isothermal binary vapor-liquid equilibria}

The 903 isothermal subsets from the cleaned dataset were predicted using the GH-GNN model embedded within the extended Margules model. In total, 12,980 data points were analyzed, involving 180 distinct compounds across 531 unique binary combinations. The temperature range of the isothermal subsets spans from 127.59 K to 573.15 K.

\begin{figure}[h]
    \centering
    \includegraphics[width=1\textwidth]{figures/MAE_matrix_organic_old_Jaccard0.6.png}
    \caption{Heatmap of the mean absolute error achieved by the GH-GNN model embedded into the Margules model for different types of binary mixtures at isothermal conditions. The error is measured with respect to the vapor phase molar fraction. The number in each cell represents the number of data points included in the corresponding binary category.}
    \label{fig:isothermal_map}
\end{figure}

To compute the activity coefficients for all isothermal systems in the dataset, a total of 1062 IDACs were required. Of the 531 binary systems, 120 (22.6\%) had both IDACs observed during training. For 382 systems (72\%), the GH-GNN model had to predict at least one of the two required IDACs in other solute-solvent combination as the ones observed during training. For the remaining 29 systems (5.4\%), the model needed to predict at least one of the necessary IDACs with molecules never observed before during training. This was due to 15 compounds (out of the 180 in the isothermal VLE data) not being present in the GH-GNN training set. The fact that even for this limited VLE dataset only very few of the necessary IDACs have been measured experimentally underscores the strong motivation for developing reliable predictive methods.

\begin{figure}[h]
    \centering
    \includegraphics[width=0.85\textwidth]{figures/isothermal_vle.png}
    \caption{Isothermal vapor-liquid equilibria (VLE) diagram for two systems that are unfeasible to predict with UNIFAC-Dortmund.}
    \label{fig:isothermal_diagram}
\end{figure}

A similar analysis for UNIFAC is challenging to perform, as the specific mixtures and data used for its parametrization are not readily accessible. Furthermore, it is important to note that while the GH-GNN embedded within the Margules model had access exclusively to IDAC data, UNIFAC was parameterized using a broader range of experimental data, including IDACs, vapor-liquid equilibria (VLE), liquid-liquid equilibria (LLE), solid-liquid equilibria (SLE), azeotropic data, and calorimetric measurements \cite{gmehling2002modified}.

Figure \ref{fig:isothermal_map} presents a heatmap of the mean absolute error (MAE) achieved by the GH-GNN embedded in the Margules model when predicting the vapor-phase molar fraction of the binary vapor-liquid equilibria (VLE) under isothermal conditions. 

Overall, systems containing acids are particularly challenging for the analyzed model to predict, especially when combined with halogenated species or amides. This difficulty likely arises from the strong hydrogen-bonding and dipole-dipole interactions characteristic of these compounds, which may be inadequately captured by the information provided to the GH-GNN model. Additionally, the Margules expression might require higher-order polynomial terms to better account for these complex interactions. Similarly, systems involving alcohols and amines also exhibit poor predictive accuracy. Furthermore, systems containing water (categorized here under the ``Inorganic" class), particularly in combination with amines and other nitrogen derivatives, were associated with relatively large prediction errors.


\begin{table}[h]
\centering
  \caption{Comparison between UNIFAC-Dortmund and the GH-GNN model embedded into the Margules model for predicting binary vapor-liquid equilibria (VLE) under isothermal conditions. The comparison is according to the mean absolute error (MAE), the coefficient of determination (R$^2$) and the percentage of points predicted with an absolute error of less than or equal to 0.03. All metrics are with respect to the predicted and actual values of the molar fraction in the vapor phase.}
  \label{tbl:isothermal}
  \small
  \begin{tabular*}{1\textwidth}{@{\extracolsep{\fill}}lccc}
    \hline
    \textbf{Model} & \textbf{MAE} $\downarrow$ & \textbf{R$^2$} $\uparrow$ & \textbf{AE $\leq 0.03$} $\uparrow$ \\
    \hline
    UNIFAC-Dortmund & 0.0214 & 0.957 & 86.72\% \\
    GH-GNN + Margules & 0.0307 & 0.948 & 74.77\% \\
    \hline
  \end{tabular*}
\end{table}


The comparison between the UNIFAC-Dortmund and GH-GNN model embedded into the Margules model, presented in Table \ref{tbl:isothermal}, indicates that UNIFAC-Dortmund predicts most isothermal VLE binary systems more accurately than the GNN-based Margules hybrid model. This comparison includes all mixtures that can be feasibly predicted using the UNIFAC model (12,700 data points, 514 binary mixtures).

However, this result is not unexpected, given that, as previously mentioned, extensive VLE data were also used in the parametrization of UNIFAC-Dortmund. In contrast, the GH-GNN model, combined with the Margules framework, achieves its performance using only experimental IDAC data. Consequently, while the GNN-based predictive model performs some degree of extrapolation from the infinite regime to finite concentrations, the UNIFAC model is likely reproducing VLE data that were part of its original parametrization. Overall, this comparison shows the predictive limitations of GNNs embedded into the Margules framework, but it also establishes a baseline of the degree of accuracy that can be reached by employing scarce data at infinite dilution with GNN models and Gibbs-Duhem consistent expressions for predicting VLEs.

When comparing the flexibility of the two predictive approaches, UNIFAC-Dortmund and the GNNs embedded in the Margules model, some systems in the dataset cannot be predicted using the UNIFAC model. This limitation arises either from the unfeasibility of fragmentation or the lack of available interaction parameters. In this regard, the proposed framework combining GNNs with the Margules model offers an effective alternative.

For example, Figure \ref{fig:isothermal_diagram} shows the VLE diagram for two systems that UNIFAC-Dortmund cannot predict. In both cases, at least one of the necessary IDACs was not available experimentally, and the GH-GNN model was needed for prediction. Notably, as shown for the system ``1,1,2-trichlorotrifluoroethane / methyl acetate" in Figure \ref{fig:isothermal_diagram}, the GH-GNN model embedded within Margules successfully captures azeotropic behavior with significant precision, even without dedicated azeotropic data used during training.

\subsection{Isobaric binary vapor-liquid equilibria}

The 781 isobaric subsets contain a total of 14,118 data points, covering 162 distinct compounds in 525 different binary combinations. The pressure ranges from 1.33 to 496.63 kPa. Among these subsets, 106 systems (20.2\%) had both necessary IDACs available from experimental measurements and were therefore observed during the GH-GNN model training. Additionally, 409 systems (77.9\%) required interpolation for at least one IDAC, meaning the values were predicted based on observations of the compounds in other binary combinations. The remaining 10 systems (1.9\%) involved molecules that were entirely absent during GH-GNN model training (i.e., extrapolations).

In order to perform the isobaric VLE calculations, an iterative process is needed in which the temperature used for predicting the activity coefficients is changed in order to minimize the difference between the estimated system pressure and the actual pressure. This is represented in the optimization problem \ref{eq_optim_isobaric}, which was here solved using the SciPy package \cite{virtanen2020scipy} with Brent's algorithm, allowing up to 2,000 iterations and a tolerance of $1.48 \times 10^{-8}$.

\begin{equation}
\begin{aligned}
\underset{\substack{T}}{\text{min}} \quad & \vert P - \hat{P}(T) \vert \\
\text{s.t.} \quad & \hat{P}(T) = x_i \gamma_i(T) P_i^{sat}(T) + x_j \gamma_j(T) P_j^{sat}(T) \\
& \{\gamma_i(T), \gamma_j(T)\} \leftarrow \text{GH-GNN embedded in Margules}(T)\\
& T_{\min} \leq T \leq T_{\max}
\end{aligned}
\label{eq_optim_isobaric}
\end{equation}

\noindent where $P$ and $\hat{P}$ denote the actual and predicted system pressures, respectively, and $T_{\min}$ and $T_{\max}$ define the optimization bounds for the temperature. In this case, these bounds correspond to the minimum temperature range in which the vapor pressure correlation (Equation \ref{eq_vapor_pressure_kdb}) remains valid.

\begin{figure}[h!]
    \centering
    \includegraphics[width=1\textwidth]{figures/MAE_matrix_organic_old_Jaccard0.6_isobaric.png}
    \caption{Heatmap of the mean absolute error achieved by the GH-GNN model embedded into the Margules model for different types of binary mixtures at isobaric conditions. The error is measured with respect to the vapor phase molar fraction. The number in each cell represents the number of data points included in the corresponding binary category.}
    \label{fig:isobaric_map}
\end{figure}

\begin{table}[h]
\centering
  \caption{Comparison between UNIFAC-Dortmund and the GH-GNN model embedded into the Margules model for predicting binary vapor-liquid equilibria (VLE) under isobaric conditions. The comparison is according to the mean absolute error (MAE), the coefficient of determination (R$^2$) and the percentage of points predicted with an absolute error of less than or equal to 0.03. All metrics are with respect to the predicted and actual values of the molar fraction in the vapor phase.}
  \label{tbl:isobaric}
  \small
  \begin{tabular*}{1\textwidth}{@{\extracolsep{\fill}}lccc}
    \hline
    \textbf{Model} & \textbf{MAE} $\downarrow$ & \textbf{R$^2$} $\uparrow$ & \textbf{AE $\leq 0.03$} $\uparrow$ \\
    \hline
    UNIFAC-Dortmund & 0.0224 & 0.967 & 81.91\% \\
    GH-GNN + Margules & 0.0350 & 0.946 & 65.78\% \\
    \hline
  \end{tabular*}
\end{table}

Table \ref{tbl:isobaric} compares the performance of the UNIFAC-Dortmund model and the GH-GNN model embedded into Margules. The comparison is shown only for the systems that are feasible to predict with UNIFAC, resulting in 14,075 data points. Similar to the isothermal systems, UNIFAC-Dortmund generally outperforms the GH-GNN model combined with Margules in predicting isobaric VLEs of binary systems. As previously explained, this outcome is expected due to the difference in available experimental data during training. Nevertheless, it provides a useful baseline for assessing the accuracy achievable when predicting finite compositions from infinite dilution information using predictive methods based on GNNs.

\begin{figure}[h!]
    \centering
    \includegraphics[width=0.85\textwidth]{figures/2_PYRIDINE_1,2,3,4-TETRAHYDRONAPHTHALENE_26.66.png}
    \caption{Isobaric vapor-liquid equilibria (VLE) diagram comparing experimental data with the predictions of UNIFAC-Dortmund and the GH-GNN model embedded into the Margules model. The system is pyridine/1,2,3,4-tetrahydronaphthalene at 26.66 kPa.}
    \label{fig:isobaric_diagram_1}
\end{figure}

The heatmap in Figure \ref{fig:isobaric_map} illustrates the performance of the GNN-Margules-based model across different binary combinations. Notably, compared to the isothermal data, different binary classes are now considered. For instance, combinations such as acids/alcohols and acids/aromatics are present under isobaric conditions but absent in the isothermal dataset. In total, 16 new binary classes appear in the isobaric data that were not included in the isothermal data. However, 40 binary classes are present only at isothermal conditions. In total the intersection of binary classes between isothermal and isobaric conditions is 60. Similar to the isothermal case, many binary classes under isobaric conditions are predicted with relatively low errors.

Out of the 737 isobaric systems that are feasible to predict with UNIFAC-Dortmund, 182 are, on average, better predicted by the GNN-Margules-based model than by UNIFAC-Dortmund. To illustrate this, Figure \ref{fig:isobaric_diagram_1} presents the VLE diagram for the system ``pyridine/1,2,3,4-tetrahydronaphthalene" at 26.66 kPa. It is evident that the proposed model here predicts the VLE behavior more accurately compared to UNIFAC-Dortmund. A similar trend is observed in Figure \ref{fig:isobaric_diagram_2}, which shows the VLE diagram for the system “tetrahydrofuran/ethanol" at 25 kPa.

\begin{figure}[h]
    \centering
    \includegraphics[width=0.85\textwidth]{figures/15_TETRAHYDROFURAN_ETHANOL_25.0.png}
    \caption{Isobaric vapor-liquid equilibria (VLE) diagram comparing experimental data with the predictions of UNIFAC-Dortmund and the GH-GNN model embedded into the Margules model. The system is tetrahydrofuran/ethanol at 25 kPa.}
    \label{fig:isobaric_diagram_2}
\end{figure}

In these two isobaric VLE examples, the GH-GNN model interpolates at least one of the two required IDACs. This highlights that, although UNIFAC-Dortmund generally outperforms the proposed model, there are counterexamples where the GH-GNN combined with Margules model provides superior predictions, which may be relevant in specific practical scenarios where such mixtures might be of interest.

\subsection{Overall performance predicting binary vapor-liquid equilibria}

\begin{figure}[h!]
    \centering
    \includegraphics[width=1\textwidth]{figures/MAE_matrix_UNIFACDo_vs_GHGNNMargules.png}
    \caption{Comparison between the UNIFAC-Dortmund and the GH-GNN in Margules model based on the mean absolute error (MAE) for predicting the vapor molar fraction. All feasible systems at isothermal, isobaric and random conditions are included. The number in each cell represents the number of data points.}
    \label{fig:mae_matrix_ghgnn_vs_unifac}
\end{figure}

To gain a more detailed understanding of the binary classes where the GH-GNN model embedded into Margules outperforms UNIFAC-Dortmund, Figure \ref{fig:mae_matrix_ghgnn_vs_unifac} presents the best-performing model for each binary combination. The performance of each binary class is evaluated based on the mean absolute error (MAE) across all temperature and pressure conditions, including the isothermal, isobaric, and random subsets. Predictions are assessed with respect to the molar fraction in the vapor phase.


\begin{figure}[h]
    \centering
    \includegraphics[width=0.85\textwidth]{figures/Cumulative_MAE_classes_UNIFACDo_vs_GHGNNMargules.png}
    \caption{Cumulative percentage of binary vapor-liquid equilibria data points predicted below different thresholds of absolute error with respect to the molar fraction in the vapor phase. The performance is shown for the models UNIFAC-Dortmund and the GH-GNN embedded into Margules.}
    \label{fig:cummulative}
\end{figure}

As previously discussed for the isothermal and isobaric cases, the UNIFAC-Dortmund model generally performs better for most binary classes. However, certain binary classes are predicted more accurately by the GH-GNN model combined with Margules. For instance, the binary class ``amines/aromatics," which includes the system shown in Figure \ref{fig:isobaric_diagram_1}, tends to be better predicted using the proposed GH-GNN in Margules model. Out of the 116 distinct binary classes studied in this work, 27 binary classes are on averaged predicted with lower errors by the GH-GNN in Margules model compared to UNIFAC-Dortmund.

Figure \ref{fig:cummulative} presents the cumulative percentage of binary VLE data points predicted with absolute errors below various thresholds, based on the molar fraction in the vapor phase. Both models achieve a similar percentage of well-predicted data points for absolute errors up to around 0.01. However, UNIFAC-Dortmund consistently yields lower absolute errors overall. This is primarily attributed to differences in the experimental data available during training but also highlights the potential of GNN-based hybrid models for extrapolating finite concentration conditions from limited infinite dilution data. Including VLE data in the training of GNN-based models alongside infinite dilution data would likely lead to a significant increase in accuracy, as already observed in the case of transformers \cite{winter2023spt}.

\subsection{Ternary vapor-liquid equilibria}

In this section, we provide insights into the relative performance of the GH-GNN model embedded in Margules compared to UNIFAC-Dortmund for ternary mixtures. As shown in Table \ref{tbl:ternary_systems}, the available ternary data is much more limited. Therefore, this comparison aims to provide qualitative insights rather than an in-depth analysis, as conducted for binary mixtures.

\begin{table}[h]
\centering
  \caption{Comparison between UNIFAC-Dortmund and the GH-GNN model embedded into the Margules model for predicting ternary vapor-liquid equilibria (VLE). The comparison is according to the mean absolute error (MAE), the coefficient of determination (R$^2$) and the percentage of points predicted with an absolute error of less than or equal to 0.03. All metrics are with respect to the predicted and actual values of the molar fraction in the vapor phase of components 1 and 2.}
  \label{tbl:ternary}
  \small
  \begin{tabular*}{1\textwidth}{@{\extracolsep{\fill}}lccc}
    \hline
    \multicolumn{4}{c}{\textbf{Isothermal system}}\\
    \hline
    \textbf{Model} & \textbf{MAE} $\downarrow$ & \textbf{R$^2$} $\uparrow$ & \textbf{AE $\leq 0.03$} $\uparrow$ \\
    \hline
    UNIFAC-Dortmund & 0.0059 & 0.995 & 100\% \\
    GH-GNN + Margules & 0.0568 & 0.394 & 37.96\% \\
    \hline
    \multicolumn{4}{c}{\textbf{Isobaric systems}}\\
    \hline
    \textbf{Model} & \textbf{MAE} $\downarrow$ & \textbf{R$^2$} $\uparrow$ & \textbf{AE $\leq 0.03$} $\uparrow$ \\
    \hline
    UNIFAC-Dortmund & 0.0123 & 0.991 & 89.07\% \\
    GH-GNN + Margules & 0.0330 & 0.925 & 66.60\% \\
    \hline
  \end{tabular*}
\end{table}

Similar to the case of binary systems presented earlier, UNIFAC-Dortmund generally provides more accurate estimations of ternary VLEs (see Table \ref{tbl:ternary}). Despite this, predictions from the GH-GNN model combined with Margules also align well with the physical behavior of the mixtures. Moreover, as observed in binary mixtures, the GH-GNN embedded in Margules produces more accurate predictions for certain systems compared to UNIFAC-Dortmund.

Specifically, among the 10 ternary systems analyzed, the GH-GNN in Margules achieves a lower MAE than UNIFAC-Dortmund for the following systems: ``benzene/heptane/dimethylformamide" (MAE of 0.0296 vs. 0.0297 for UNIFAC-Dortmund), ``benzene/heptane/acetonitrile" (MAE of 0.0084 vs. 0.0114 for UNIFAC-Dortmund), and ``acetone/chloroform/benzene" (MAE of 0.0047 vs. 0.0184 for UNIFAC-Dortmund). For the remaining systems, UNIFAC-Dortmund achieves lower overall MAE. This exemplifies that, for systems where UNIFAC-Dortmund cannot provide predictions due to missing interaction parameters or unfeasibility of fragmentation into functional groups, the GH-GNN in Margules model offers a useful first-step approximation of the mixture behavior, which could be beneficial in early stages of molecular and process design.

\section{Conclusions}
\label{sec:conclusions}

Predictive models for thermophysical properties of pure compounds and their mixtures play a crucial role in chemical space exploration for product and process design. While numerous predictive methods have been developed over the decades, they remain limited in terms of both accuracy and the range of systems they can effectively describe.

In this work, we studied the performance of a GNN-based model embedded in the extended Margules framework, which assumes that the excess Gibbs energy can be approximated using a Taylor polynomial. The performance of this model was compared against the widely used UNIFAC-Dortmund model across extensive vapor-liquid equilibrium (VLE) data for binary systems under isothermal and isobaric conditions.

Additionally, we provide insights into the relative performance of these models in predicting ternary VLEs. Overall, UNIFAC-Dortmund yields more accurate VLE predictions. However, a crucial factor in this comparison is the difference in the types of experimental data used for model development. While the GNN embedded in Margules was trained exclusively on limited infinite dilution data, requiring extrapolation from infinite dilution to finite concentrations, UNIFAC-Dortmund was parameterized using a much broader dataset, including infinite dilution data, VLE, LLE, SLE, azeotropic, and caloric data \cite{constantinescu2016further,gmehling2002modified}.

It is likely that incorporating VLE data alongside infinite dilution data could significantly enhance the accuracy of GNN-based models within the Margules framework. Nonetheless, this study establishes a valuable baseline, demonstrating the predictive accuracy achievable when extrapolating VLE behavior solely from binary infinite dilution data using a relatively simple excess Gibbs energy model in combination with a highly flexible GNN-based model.

Future work should focus on a comprehensive comparison of GNN-based hybrid models trained on extensive and diverse datasets. However, a key challenge in this endeavor is the structuring and open accessibility of thermodynamic data for model development and benchmarking. Advancing community efforts in thermodynamic data curation and structuring will be essential for the continuous improvement and open-source benchmarking of predictive thermodynamic models.




\section*{Data availability}
The vapor-liquid equilibria (VLE) and pure component data used in this work is available at the accompanying GitHub repository at \url{https://github.com/edgarsmdn/GH_GNN_Margules}.

\section*{Code availability}
The code for generating the figures and running VLE predictions with the GH-GNN model embedded into the Margules expression are available at the accompanying GitHub repository at \url{https://github.com/edgarsmdn/GH_GNN_Margules}.

\section*{CRediT authorship contribution statement}
\textbf{Edgar Ivan Sanchez Medina}: Conceptualization, Data curation, Formal analysis, Investigation, Methodology, Project administration, Software, Validation, Visualization, Writing – original draft.
\textbf{Kai Sundmacher}: Funding acquisition, Supervision, Writing – review and editing.







\section*{Declaration of competing interest}
\input{article_sections/08_Conflicts_of_interest}

\section*{Acknowledgments}
This work was partly supported by the Research In-itiative “SmartProSys: Intelligent Process Systems for the Sustainable Production of Chemicals”, funded by the Ministry for Science, Energy, Climate Protection and the Environment of the State of Saxony- Anhalt.

\section*{Declaration of generative AI and AI-assisted technologies in the writing process}
During the preparation of this work the authors used gpt-4o-2024-08-06 in order to check the grammar and improve the readability of the text. After using this tool, the authors reviewed and edited the content as needed and take full responsibility for the content of the published article.

\section{Appendix}
%% The Appendices part is started with the command \appendix;
%% appendix sections are then done as normal sections
\appendix
\section{Derivation of the extended Margules model}
\label{app1}

The extended Margules model assumes that the molar excess Gibbs energy $g^E$ of any mixture can be described by a continuous function that is infinitely differentiable. Then, $g^E$ is expressed as a Taylor series, often sufficiently well-approximated when truncated after the third term. Therefore, for a mixture of $N$ components, we have that

\begin{align}
    g^E(\textbf{x}) = C_0 + \sum_{i=1}^{N-1} C_i x_i + \sum_{i=1}^{N-1} \sum_{j=1}^{N-1} C_{ij} x_ix_j + \sum_{i=1}^{N-1} \sum_{j=1}^{N-1} \sum_{k=1}^{N-1} C_{ijk} x_ix_jx_k
    \label{eq_taylor_ge}
\end{align}

\noindent where $\mathbf{x}=[x_1, x_2, \cdots, x_{N-1}]$ represents the vector of molar fractions for the $N-1$ components in the mixture, and $C$ denotes the polynomial coefficient corresponding to the indicated subscripts.

From the boundary condition given by Equation \ref{eq_boundary_condition}, it follows that $g^E=0$ when $x_N=1$. Therefore, $C_0=0$. Similarly, we know that $(C_i + C_{ii} + C_{iii})=0$ when $x_i=1 \quad \forall ~ i \in \{1,2, \dots, N-1\}$.


Therefore, Equation \ref{eq_taylor_ge} can be reformulated as:

\begin{align}
    g^E(\textbf{x}) = - \sum_{i=1}^{N-1} (C_{ii} + C_{iii}) x_i + \sum_{i=1}^{N-1} \sum_{j=1}^{N-1} C_{ij} x_ix_j + \sum_{i=1}^{N-1} \sum_{j=1}^{N-1} \sum_{k=1}^{N-1} C_{ijk} x_ix_jx_k
    \label{eq_taylor_ge2}
\end{align}

By leveraging the fact that the sum of all mole fractions adds to 1, we can re-express the first two terms of Equation \ref{eq_taylor_ge2} to obtain:

\begin{align}
    g^E(\textbf{x}) = - \sum_{i=1}^{N-1} \sum_{j=1}^{N} \sum_{k=1}^{N} (C_{ii} + C_{iii}) x_i x_j x_k + \sum_{i=1}^{N-1} \sum_{j=1}^{N-1} \sum_{k=1}^{N} C_{ij} x_ix_jx_k + \sum_{i = 1}^{N-1} \sum_{j = 1}^{N-1} \sum_{k = 1}^{N-1} C_{ijk} x_ix_jx_k
    \label{eq_taylor_ge3}
\end{align}

We can then simplify the expression by collecting all crossed terms to obtain a general expression that can be applied to estimate $g^E$ for a mixture of $N$ components:

\begin{align}
    g^E(\textbf{x}) = - \sum_{i=1}^{N-1} \sum_{j=i}^{N} \sum_{k=j}^{N} (\hat{C}_{ii} + \hat{C}_{iii}) x_i x_j x_k + \sum_{i=1}^{N-1} \sum_{j=i}^{N-1} \sum_{k=j}^{N} \hat{C}_{ij} x_ix_jx_k + \sum_{i = 1}^{N-1} \sum_{j = i}^{N-1} \sum_{k = j}^{N-1} \hat{C}_{ijk} x_ix_jx_k
    \label{eq_taylor_ge4}
\end{align}

For binary mixtures, Equation \ref{eq_taylor_ge4} simplifies to 

\begin{align}
    g^E = -\hat{C}_{111} x_1^2 x_2 + (-\hat{C}_{11} - \hat{C}_{111}) x_1 x_2^2
    \label{eq_ge_binary}
\end{align}

The value of the parameters $\hat{C}_{111}$ and $\hat{C}_{11}$ can be determined by considering the following relationship between $g^E$ and the IDACs:

\begin{equation}
    RT \ln{\gamma_{ij}^\infty} = \left( \frac{\partial g^E}{\partial x_i} \right)_{x_i \rightarrow 0, ~~ j \neq i} 
    \label{eq_determinar_constants}
\end{equation}

Therefore, the value of the constants is given by

\begin{align}
    \left( \frac{\partial g^E}{\partial x_1} \right)_{x_1 \rightarrow 0} &= - \hat{C}_{11} - \hat{C}_{111} = w_{12} = RT \ln{\gamma_{12}^\infty}\\
    \left( \frac{\partial g^E}{\partial x_2} \right)_{x_2 \rightarrow 0} &= - \hat{C}_{111} = w_{21} = RT \ln{\gamma_{21}^\infty}
\end{align}

\noindent and the final $g^E$ expression for the binary case results in Equation \ref{eq_ge_binary_final}.

The activity coefficients can be then calculated by differentiating Equation \ref{eq_ge_binary_final} in terms of the total excess Gibbs energy $G^E = M g^E$ (where, $M$ stands for the total number of moles) with respect to the number of moles of each component:

\begin{align}
    RT \ln{\gamma_1} =& \frac{\partial G^E}{\partial n_1} = w_{21} \left( 
    \frac{2 n_1 n_2}{M^2} - \frac{2 n_1^2 n_2}{M^3}
    \right)
    +
    w_{12}\left( 
    \frac{n_2^2}{M^2} - \frac{2 n_1 n_2^2}{M^3}
    \right) \\
    RT \ln{\gamma_1} =& 2 w_{21} x_1 x_2 + x_2^2 w_{12} - 2 g^E
\end{align}

\noindent and similarly for component 2:

\begin{equation}
    RT \ln{\gamma_2} = 2 w_{12} x_2 x_1 + x_1^2 w_{21} - 2 g^E
\end{equation}

A similar procedure can be followed to determined the expressions to compute the activity coefficients of components in a ternary mixture:

\begin{align}
    g^E = x_1 x_2 (x_2 w_{12} + x_1 w_{21}) + x_1 x_3 (x_3 w_{13} + x_1 w_{31}) + x_2 x_3 (x_3 w_{23} + x_2 w_{32}) + x_1 x_2 x_3 C_{123}
    \label{eq_ge_margules_ternary}
\end{align}

with $ C_{ijk} = \frac{1}{2}(w_{ij} + w_{ji} + w_{ik} + w_{ki} + w_{jk} + w_{kj}) - w_{ijk}$. While the binary parameters can be related to the corresponding IDACs as shown for the binary case, the ternary parameter $w_{ijk}$ is often set to zero as in \cite{andersen1981valid}. The resulting expressions for the activity coefficients are then given by:

\begin{equation}
    RT \ln{\gamma_1} = 2(x_1 x_2 w_{21} + x_1 x_3 w_{31}) + x_2^2 w_{12} + x_3^2 w_{13} + x_2 x_3 c_{123} - 2 g^E
\end{equation}

\begin{equation}
    RT \ln{\gamma_2} = 2(x_2 x_3 w_{32} + x_2 x_1 w_{12}) + x_3^2 w_{23} + x_1^2 w_{21} + x_3 x_1 c_{231} - 2 g^E
\end{equation}

\begin{equation}
    RT \ln{\gamma_3} = 2(x_3 x_1 w_{13} + x_3 x_2 w_{23}) + x_1^2 w_{31} + x_2^2 w_{32} + x_1 x_2 c_{312} - 2 g^E
\end{equation}





% \begin{thebibliography}{00}

% \bibitem[Lamport(1994)]{lamport94}
%   Leslie Lamport,
%   \textit{\LaTeX: a document preparation system},
%   Addison Wesley, Massachusetts,
%   2nd edition,
%   1994.

% \end{thebibliography}

\bibliographystyle{elsarticle-num-names}
\bibliography{bibliography}

\end{document}

\endinput

