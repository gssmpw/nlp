%% 
%% Copyright 2007-2025 Elsevier Ltd
%% 
%% This file is part of the 'Elsarticle Bundle'.
%% ---------------------------------------------
%% 
%% It may be distributed under the conditions of the LaTeX Project Public
%% License, either version 1.3 of this license or (at your option) any
%% later version.  The latest version of this license is in
%%    http://www.latex-project.org/lppl.txt
%% and version 1.3 or later is part of all distributions of LaTeX
%% version 1999/12/01 or later.
%% 
%% The list of all files belonging to the 'Elsarticle Bundle' is
%% given in the file `manifest.txt'.
%% 
%% Template article for Elsevier's document class `elsarticle'
%% with harvard style bibliographic references

\documentclass[preprint,12pt]{elsarticle}

%% Use the option review to obtain double line spacing
%% \documentclass[preprint,review,12pt]{elsarticle}

%% Use the options 1p,twocolumn; 3p; 3p,twocolumn; 5p; or 5p,twocolumn
%% for a journal layout:
%% \documentclass[final,1p,times]{elsarticle}
%% \documentclass[final,1p,times,twocolumn]{elsarticle}
%% \documentclass[final,3p,times]{elsarticle}
%% \documentclass[final,3p,times,twocolumn]{elsarticle}
%% \documentclass[final,5p,times]{elsarticle}
%% \documentclass[final,5p,times,twocolumn]{elsarticle}

%% For including figures, graphicx.sty has been loaded in
%% elsarticle.cls. If you prefer to use the old commands
%% please give \usepackage{epsfig}

%% The amssymb package provides various useful mathematical symbols
\usepackage{amssymb}
%% The amsmath package provides various useful equation environments.
\usepackage{amsmath}
%% The amsthm package provides extended theorem environments
%% \usepackage{amsthm}

%% The lineno packages adds line numbers. Start line numbering with
%% \begin{linenumbers}, end it with \end{linenumbers}. Or switch it on
%% for the whole article with \linenumbers.
\usepackage{lineno}

\usepackage{hyperref}

\journal{ } %Fluid Phase Equilibria

\begin{document}

\begin{frontmatter}

%% Title, authors and addresses

%% use the tnoteref command within \title for footnotes;
%% use the tnotetext command for theassociated footnote;
%% use the fnref command within \author or \affiliation for footnotes;
%% use the fntext command for theassociated footnote;
%% use the corref command within \author for corresponding author footnotes;
%% use the cortext command for theassociated footnote;
%% use the ead command for the email address,
%% and the form \ead[url] for the home page:
%% \title{Title\tnoteref{label1}}
%% \tnotetext[label1]{}
%% \author{Name\corref{cor1}\fnref{label2}}
%% \ead{email address}
%% \ead[url]{home page}
%% \fntext[label2]{}
%% \cortext[cor1]{}
%% \affiliation{organization={},
%%             addressline={},
%%             city={},
%%             postcode={},
%%             state={},
%%             country={}}
%% \fntext[label3]{}

%% Article title
\title{Graph Neural Networks embedded into Margules model for vapor-liquid equilibria prediction} 

%% Author names and emails
\author[mpi_affiliation]{Edgar Ivan Sanchez Medina\corref{cor1}}
\ead{sanchez@mpi-magdeburg.mpg.de}

\author[mpi_affiliation,ovgu_affiliation]{Kai Sundmacher}

\cortext[cor1]{Corresponding author}

%% Author affiliation
\affiliation[mpi_affiliation]{organization={Process Systems Engineering, Max Planck Institute for Dynamics of Complex Technical Systems},
            addressline={Sandtorstraße 1}, 
            city={Magdeburg},
            postcode={39106}, 
            state={Saxony-Anhalt},
            country={Germany}}

\affiliation[ovgu_affiliation]{organization={Chair for Process Systems Engineering, Otto-von-Guericke University},
            addressline={Universitätsplatz 2}, 
            city={Magdeburg},
            postcode={39106}, 
            state={Saxony-Anhalt},
            country={Germany}}

%% Abstract
\begin{abstract}
Predictive thermodynamic models are crucial for the early stages of product and process design. In this paper the performance of Graph Neural Networks (GNNs) embedded into a relatively simple excess Gibbs energy model, the extended Margules model, for predicting vapor-liquid equilibrium is analyzed. By comparing its performance against the established UNIFAC-Dortmund model it has been shown that GNNs embedded in Margules achieves an overall lower accuracy. However, higher accuracy is observed in the case of various types of binary mixtures. Moreover, since group contribution methods, like UNIFAC, are limited due to feasibility of molecular fragmentation or availability of parameters, the GNN in Margules model offers an alternative for VLE estimation. The findings establish a baseline for the predictive accuracy that simple excess Gibbs energy models combined with GNNs trained solely on infinite dilution data can achieve.
\end{abstract}

%%Graphical abstract
% \begin{graphicalabstract}
% \includegraphics{elsarticle/figures/graphical_abstract.png}
% \end{graphicalabstract}

% %%Research highlights
% \begin{highlights}
% \item Graph Neural Networks embedded into the extended Margules model for Gibbs-Duhem consistent predictions
% \item Analysis of binary and ternary vapor-liquid equilibria prediction using a hybrid machine learning - mechanistic model
% \end{highlights}

%% Keywords
\begin{keyword}
graph neural networks \sep vapor-liquid equilibria \sep Margules \sep activity coefficients
\end{keyword}

\end{frontmatter}

%% Add \usepackage{lineno} before \begin{document} and uncomment 
%% following line to enable line numbers
%\linenumbers

%% main text
%%

\section{Introduction}
\section{Introduction}

% \textcolor{red}{Still on working}

% \textcolor{red}{add label for each section}


Robot learning relies on diverse and high-quality data to learn complex behaviors \cite{aldaco2024aloha, wang2024dexcap}.
Recent studies highlight that models trained on datasets with greater complexity and variation in the domain tend to generalize more effectively across broader scenarios \cite{mann2020language, radford2021learning, gao2024efficient}.
% However, creating such diverse datasets in the real world presents significant challenges.
% Modifying physical environments and adjusting robot hardware settings require considerable time, effort, and financial resources.
% In contrast, simulation environments offer a flexible and efficient alternative.
% Simulations allow for the creation and modification of digital environments with a wide range of object shapes, weights, materials, lighting, textures, friction coefficients, and so on to incorporate domain randomization,
% which helps improve the robustness of models when deployed in real-world conditions.
% These environments can be easily adjusted and reset, enabling faster iterations and data collection.
% Additionally, simulations provide the ability to consistently reproduce scenarios, which is essential for benchmarking and model evaluation.
% Another advantage of simulations is their flexibility in sensor integration. Sensors such as cameras, LiDARs, and tactile sensors can be added or repositioned without the physical limitations present in real-world setups. Simulations also eliminate the risk of damaging expensive hardware during edge-case experiments, making them an ideal platform for testing rare or dangerous scenarios that are impractical to explore in real life.
By leveraging immersive perspectives and interactions, Extended Reality\footnote{Extended Reality is an umbrella term to refer to Augmented Reality, Mixed Reality, and Virtual Reality \cite{wikipediaExtendedReality}}
(XR)
is a promising candidate for efficient and intuitive large scale data collection \cite{jiang2024comprehensive, arcade}
% With the demand for collecting data, XR provides a promising approach for humans to teach robots by offering users an immersive experience.
in simulation \cite{jiang2024comprehensive, arcade, dexhub-park} and real-world scenarios \cite{openteach, opentelevision}.
However, reusing and reproducing current XR approaches for robot data collection for new settings and scenarios is complicated and requires significant effort.
% are difficult to reuse and reproduce system makes it hard to reuse and reproduce in another data collection pipeline.
This bottleneck arises from three main limitations of current XR data collection and interaction frameworks: \textit{asset limitation}, \textit{simulator limitation}, and \textit{device limitation}.
% \textcolor{red}{ASSIGN THESE CITATION PROPERLY:}
% \textcolor{red}{list them by time order???}
% of collecting data by using XR have three main limitations.
Current approaches suffering from \textit{asset limitation} \cite{arclfd, jiang2024comprehensive, arcade, george2025openvr, vicarios}
% Firstly, recent works \cite{jiang2024comprehensive, arcade, dexhub-park}
can only use predefined robot models and task scenes. Configuring new tasks requires significant effort, since each new object or model must be specifically integrated into the XR application.
% and it takes too much effort to configure new tasks in their systems since they cannot spawn arbitrary models in the XR application.
The vast majority of application are developed for specific simulators or real-world scenarios. This \textit{simulator limitation} \cite{mosbach2022accelerating, lipton2017baxter, dexhub-park, arcade}
% Secondly, existing systems are limited to a single simulation platform or real-world scenarios.
significantly reduces reusability and makes adaptation to new simulation platforms challenging.
Additionally, most current XR frameworks are designed for a specific version of a single XR headset, leading to a \textit{device limitation} 
\cite{lipton2017baxter, armada, openteach, meng2023virtual}.
% and there is no work working on the extendability of transferring to a new headsets as far as we know.
To the best of our knowledge, no existing work has explored the extensibility or transferability of their framework to different headsets.
These limitations hamper reproducibility and broader contributions of XR based data collection and interaction to the research community.
% as each research group typically has its own data collection pipeline.
% In addition to these main limitations, existing XR systems are not well suited for managing multiple robot systems,
% as they are often designed for single-operator use.

In addition to these main limitations, existing XR systems are often designed for single-operator use, prohibiting collaborative data collection.
At the same time, controlling multiple robots at once can be very difficult for a single operator,
making data collection in multi-robot scenarios particularly challenging \cite{orun2019effect}.
Although there are some works using collaborative data collection in the context of tele-operation \cite{tung2021learning, Qin2023AnyTeleopAG},
there is no XR-based data collection system supporting collaborative data collection.
This limitation highlights the need for more advanced XR solutions that can better support multi-robot and multi-user scenarios.
% \textcolor{red}{more papers about collaborative data collection}

To address all of these issues, we propose \textbf{IRIS},
an \textbf{I}mmersive \textbf{R}obot \textbf{I}nteraction \textbf{S}ystem.
This general system supports various simulators, benchmarks and real-world scenarios.
It is easily extensible to new simulators and XR headsets.
IRIS achieves generalization across six dimensions:
% \begin{itemize}
%     \item \textit{Cross-scene} : diverse object models;
%     \item \textit{Cross-embodiment}: diverse robot models;
%     \item \textit{Cross-simulator}: 
%     \item \textit{Cross-reality}: fd
%     \item \textit{Cross-platform}: fd
%     \item \textit{Cross-users}: fd
% \end{itemize}
\textbf{Cross-Scene}, \textbf{Cross-Embodiment}, \textbf{Cross-Simulator}, \textbf{Cross-Reality}, \textbf{Cross-Platform}, and \textbf{Cross-User}.

\textbf{Cross-Scene} and \textbf{Cross-Embodiment} allow the system to handle arbitrary objects and robots in the simulation,
eliminating restrictions about predefined models in XR applications.
IRIS achieves these generalizations by introducing a unified scene specification, representing all objects,
including robots, as data structures with meshes, materials, and textures.
The unified scene specification is transmitted to the XR application to create and visualize an identical scene.
By treating robots as standard objects, the system simplifies XR integration,
allowing researchers to work with various robots without special robot-specific configurations.
\textbf{Cross-Simulator} ensures compatibility with various simulation engines.
IRIS simplifies adaptation by parsing simulated scenes into the unified scene specification, eliminating the need for XR application modifications when switching simulators.
New simulators can be integrated by creating a parser to convert their scenes into the unified format.
This flexibility is demonstrated by IRIS’ support for Mujoco \cite{todorov2012mujoco}, IsaacSim \cite{mittal2023orbit}, CoppeliaSim \cite{coppeliaSim}, and even the recent Genesis \cite{Genesis} simulator.
\textbf{Cross-Reality} enables the system to function seamlessly in both virtual simulations and real-world applications.
IRIS enables real-world data collection through camera-based point cloud visualization.
\textbf{Cross-Platform} allows for compatibility across various XR devices.
Since XR device APIs differ significantly, making a single codebase impractical, IRIS XR application decouples its modules to maximize code reuse.
This application, developed by Unity \cite{unity3dUnityManual}, separates scene visualization and interaction, allowing developers to integrate new headsets by reusing the visualization code and only implementing input handling for hand, head, and motion controller tracking.
IRIS provides an implementation of the XR application in the Unity framework, allowing for a straightforward deployment to any device that supports Unity. 
So far, IRIS was successfully deployed to the Meta Quest 3 and HoloLens 2.
Finally, the \textbf{Cross-User} ability allows multiple users to interact within a shared scene.
IRIS achieves this ability by introducing a protocol to establish the communication between multiple XR headsets and the simulation or real-world scenarios.
Additionally, IRIS leverages spatial anchors to support the alignment of virtual scenes from all deployed XR headsets.
% To make an seamless user experience for robot learning data collection,
% IRIS also tested in three different robot control interface
% Furthermore, to demonstrate the extensibility of our approach, we have implemented a robot-world pipeline for real robot data collection, ensuring that the system can be used in both simulated and real-world environments.
The Immersive Robot Interaction System makes the following contributions\\
\textbf{(1) A unified scene specification} that is compatible with multiple robot simulators. It enables various XR headsets to visualize and interact with simulated objects and robots, providing an immersive experience while ensuring straightforward reusability and reproducibility.\\
\textbf{(2) A collaborative data collection framework} designed for XR environments. The framework facilitates enhanced robot data acquisition.\\
\textbf{(3) A user study} demonstrating that IRIS significantly improves data collection efficiency and intuitiveness compared to the LIBERO baseline.

% \begin{table*}[t]
%     \centering
%     \begin{tabular}{lccccccc}
%         \toprule
%         & \makecell{Physical\\Interaction}
%         & \makecell{XR\\Enabled}
%         & \makecell{Free\\View}
%         & \makecell{Multiple\\Robots}
%         & \makecell{Robot\\Control}
%         % Force Feedback???
%         & \makecell{Soft Object\\Supported}
%         & \makecell{Collaborative\\Data} \\
%         \midrule
%         ARC-LfD \cite{arclfd}                              & Real        & \cmark & \xmark & \xmark & Joint              & \xmark & \xmark \\
%         DART \cite{dexhub-park}                            & Sim         & \cmark & \cmark & \cmark & Cartesian          & \xmark & \xmark \\
%         \citet{jiang2024comprehensive}                     & Sim         & \cmark & \xmark & \xmark & Joint \& Cartesian & \xmark & \xmark \\
%         \citet{mosbach2022accelerating}                    & Sim         & \cmark & \cmark & \xmark & Cartesian          & \xmark & \xmark \\
%         ARCADE \cite{arcade}                               & Real        & \cmark & \cmark & \xmark & Cartesian          & \xmark & \xmark \\
%         Holo-Dex \cite{holodex}                            & Real        & \cmark & \xmark & \cmark & Cartesian          & \cmark & \xmark \\
%         ARMADA \cite{armada}                               & Real        & \cmark & \xmark & \cmark & Cartesian          & \cmark & \xmark \\
%         Open-TeleVision \cite{opentelevision}              & Real        & \cmark & \cmark & \cmark & Cartesian          & \cmark & \xmark \\
%         OPEN TEACH \cite{openteach}                        & Real        & \cmark & \xmark & \cmark & Cartesian          & \cmark & \cmark \\
%         GELLO \cite{wu2023gello}                           & Real        & \xmark & \cmark & \cmark & Joint              & \cmark & \xmark \\
%         DexCap \cite{wang2024dexcap}                       & Real        & \xmark & \cmark & \xmark & Cartesian          & \cmark & \xmark \\
%         AnyTeleop \cite{Qin2023AnyTeleopAG}                & Real        & \xmark & \xmark & \cmark & Cartesian          & \cmark & \cmark \\
%         Vicarios \cite{vicarios}                           & Real        & \cmark & \xmark & \xmark & Cartesian          & \cmark & \xmark \\     
%         Augmented Visual Cues \cite{augmentedvisualcues}   & Real        & \cmark & \cmark & \xmark & Cartesian          & \xmark & \xmark \\ 
%         \citet{wang2024robotic}                            & Real        & \cmark & \cmark & \xmark & Cartesian          & \cmark & \xmark \\
%         Bunny-VisionPro \cite{bunnyvisionpro}              & Real        & \cmark & \cmark & \cmark & Cartesian          & \cmark & \xmark \\
%         IMMERTWIN \cite{immertwin}                         & Real        & \cmark & \cmark & \cmark & Cartesian          & \xmark & \xmark \\
%         \citet{meng2023virtual}                            & Sim \& Real & \cmark & \cmark & \xmark & Cartesian          & \xmark & \xmark \\
%         Shared Control Framework \cite{sharedctlframework} & Real        & \cmark & \cmark & \cmark & Cartesian          & \xmark & \xmark \\
%         OpenVR \cite{openvr}                               & Real        & \cmark & \cmark & \xmark & Cartesian          & \xmark & \xmark \\
%         \citet{digitaltwinmr}                              & Real        & \cmark & \cmark & \xmark & Cartesian          & \cmark & \xmark \\
        
%         \midrule
%         \textbf{Ours} & Sim \& Real & \cmark & \cmark & \cmark & Joint \& Cartesian  & \cmark & \cmark \\
%         \bottomrule
%     \end{tabular}
%     \caption{This is a cross-column table with automatic line breaking.}
%     \label{tab:cross-column}
% \end{table*}

% \begin{table*}[t]
%     \centering
%     \begin{tabular}{lccccccc}
%         \toprule
%         & \makecell{Cross-Embodiment}
%         & \makecell{Cross-Scene}
%         & \makecell{Cross-Simulator}
%         & \makecell{Cross-Reality}
%         & \makecell{Cross-Platform}
%         & \makecell{Cross-User} \\
%         \midrule
%         ARC-LfD \cite{arclfd}                              & \xmark & \xmark & \xmark & \xmark & \xmark & \xmark \\
%         DART \cite{dexhub-park}                            & \cmark & \cmark & \xmark & \xmark & \xmark & \xmark \\
%         \citet{jiang2024comprehensive}                     & \xmark & \cmark & \xmark & \xmark & \xmark & \xmark \\
%         \citet{mosbach2022accelerating}                    & \xmark & \cmark & \xmark & \xmark & \xmark & \xmark \\
%         ARCADE \cite{arcade}                               & \xmark & \xmark & \xmark & \xmark & \xmark & \xmark \\
%         Holo-Dex \cite{holodex}                            & \cmark & \xmark & \xmark & \xmark & \xmark & \xmark \\
%         ARMADA \cite{armada}                               & \cmark & \xmark & \xmark & \xmark & \xmark & \xmark \\
%         Open-TeleVision \cite{opentelevision}              & \cmark & \xmark & \xmark & \xmark & \cmark & \xmark \\
%         OPEN TEACH \cite{openteach}                        & \cmark & \xmark & \xmark & \xmark & \xmark & \cmark \\
%         GELLO \cite{wu2023gello}                           & \cmark & \xmark & \xmark & \xmark & \xmark & \xmark \\
%         DexCap \cite{wang2024dexcap}                       & \xmark & \xmark & \xmark & \xmark & \xmark & \xmark \\
%         AnyTeleop \cite{Qin2023AnyTeleopAG}                & \cmark & \cmark & \cmark & \cmark & \xmark & \cmark \\
%         Vicarios \cite{vicarios}                           & \xmark & \xmark & \xmark & \xmark & \xmark & \xmark \\     
%         Augmented Visual Cues \cite{augmentedvisualcues}   & \xmark & \xmark & \xmark & \xmark & \xmark & \xmark \\ 
%         \citet{wang2024robotic}                            & \xmark & \xmark & \xmark & \xmark & \xmark & \xmark \\
%         Bunny-VisionPro \cite{bunnyvisionpro}              & \cmark & \xmark & \xmark & \xmark & \xmark & \xmark \\
%         IMMERTWIN \cite{immertwin}                         & \cmark & \xmark & \xmark & \xmark & \xmark & \xmark \\
%         \citet{meng2023virtual}                            & \xmark & \cmark & \xmark & \cmark & \xmark & \xmark \\
%         \citet{sharedctlframework}                         & \cmark & \xmark & \xmark & \xmark & \xmark & \xmark \\
%         OpenVR \cite{george2025openvr}                               & \xmark & \xmark & \xmark & \xmark & \xmark & \xmark \\
%         \citet{digitaltwinmr}                              & \xmark & \xmark & \xmark & \xmark & \xmark & \xmark \\
        
%         \midrule
%         \textbf{Ours} & \cmark & \cmark & \cmark & \cmark & \cmark & \cmark \\
%         \bottomrule
%     \end{tabular}
%     \caption{This is a cross-column table with automatic line breaking.}
% \end{table*}

% \begin{table*}[t]
%     \centering
%     \begin{tabular}{lccccccc}
%         \toprule
%         & \makecell{Cross-Scene}
%         & \makecell{Cross-Embodiment}
%         & \makecell{Cross-Simulator}
%         & \makecell{Cross-Reality}
%         & \makecell{Cross-Platform}
%         & \makecell{Cross-User}
%         & \makecell{Control Space} \\
%         \midrule
%         % Vicarios \cite{vicarios}                           & \xmark & \xmark & \xmark & \xmark & \xmark & \xmark \\     
%         % Augmented Visual Cues \cite{augmentedvisualcues}   & \xmark & \xmark & \xmark & \xmark & \xmark & \xmark \\ 
%         % OpenVR \cite{george2025openvr}                     & \xmark & \xmark & \xmark & \xmark & \xmark & \xmark \\
%         \citet{digitaltwinmr}                              & \xmark & \xmark & \xmark & \xmark & \xmark & \xmark &  \\
%         ARC-LfD \cite{arclfd}                              & \xmark & \xmark & \xmark & \xmark & \xmark & \xmark &  \\
%         \citet{sharedctlframework}                         & \cmark & \xmark & \xmark & \xmark & \xmark & \xmark &  \\
%         \citet{jiang2024comprehensive}                     & \cmark & \xmark & \xmark & \xmark & \xmark & \xmark &  \\
%         \citet{mosbach2022accelerating}                    & \cmark & \xmark & \xmark & \xmark & \xmark & \xmark & \\
%         Holo-Dex \cite{holodex}                            & \cmark & \xmark & \xmark & \xmark & \xmark & \xmark & \\
%         ARCADE \cite{arcade}                               & \cmark & \cmark & \xmark & \xmark & \xmark & \xmark & \\
%         DART \cite{dexhub-park}                            & Limited & Limited & Mujoco & Sim & Vision Pro & \xmark &  Cartesian\\
%         ARMADA \cite{armada}                               & \cmark & \cmark & \xmark & \xmark & \xmark & \xmark & \\
%         \citet{meng2023virtual}                            & \cmark & \cmark & \xmark & \cmark & \xmark & \xmark & \\
%         % GELLO \cite{wu2023gello}                           & \cmark & \xmark & \xmark & \xmark & \xmark & \xmark \\
%         % DexCap \cite{wang2024dexcap}                       & \xmark & \xmark & \xmark & \xmark & \xmark & \xmark \\
%         % AnyTeleop \cite{Qin2023AnyTeleopAG}                & \cmark & \cmark & \cmark & \cmark & \xmark & \cmark \\
%         % \citet{wang2024robotic}                            & \xmark & \xmark & \xmark & \xmark & \xmark & \xmark \\
%         Bunny-VisionPro \cite{bunnyvisionpro}              & \cmark & \cmark & \xmark & \xmark & \xmark & \xmark & \\
%         IMMERTWIN \cite{immertwin}                         & \cmark & \cmark & \xmark & \xmark & \xmark & \xmark & \\
%         Open-TeleVision \cite{opentelevision}              & \cmark & \cmark & \xmark & \xmark & \cmark & \xmark & \\
%         \citet{szczurek2023multimodal}                     & \xmark & \xmark & \xmark & Real & \xmark & \cmark & \\
%         OPEN TEACH \cite{openteach}                        & \cmark & \cmark & \xmark & \xmark & \xmark & \cmark & \\
%         \midrule
%         \textbf{Ours} & \cmark & \cmark & \cmark & \cmark & \cmark & \cmark \\
%         \bottomrule
%     \end{tabular}
%     \caption{TODO, Bruce: this table can be further optimized.}
% \end{table*}

\definecolor{goodgreen}{HTML}{228833}
\definecolor{goodred}{HTML}{EE6677}
\definecolor{goodgray}{HTML}{BBBBBB}

\begin{table*}[t]
    \centering
    \begin{adjustbox}{max width=\textwidth}
    \renewcommand{\arraystretch}{1.2}    
    \begin{tabular}{lccccccc}
        \toprule
        & \makecell{Cross-Scene}
        & \makecell{Cross-Embodiment}
        & \makecell{Cross-Simulator}
        & \makecell{Cross-Reality}
        & \makecell{Cross-Platform}
        & \makecell{Cross-User}
        & \makecell{Control Space} \\
        \midrule
        % Vicarios \cite{vicarios}                           & \xmark & \xmark & \xmark & \xmark & \xmark & \xmark \\     
        % Augmented Visual Cues \cite{augmentedvisualcues}   & \xmark & \xmark & \xmark & \xmark & \xmark & \xmark \\ 
        % OpenVR \cite{george2025openvr}                     & \xmark & \xmark & \xmark & \xmark & \xmark & \xmark \\
        \citet{digitaltwinmr}                              & \textcolor{goodred}{Limited}     & \textcolor{goodred}{Single Robot} & \textcolor{goodred}{Unity}    & \textcolor{goodred}{Real}          & \textcolor{goodred}{Meta Quest 2} & \textcolor{goodgray}{N/A} & \textcolor{goodred}{Cartesian} \\
        ARC-LfD \cite{arclfd}                              & \textcolor{goodgray}{N/A}        & \textcolor{goodred}{Single Robot} & \textcolor{goodgray}{N/A}     & \textcolor{goodred}{Real}          & \textcolor{goodred}{HoloLens}     & \textcolor{goodgray}{N/A} & \textcolor{goodred}{Cartesian} \\
        \citet{sharedctlframework}                         & \textcolor{goodred}{Limited}     & \textcolor{goodred}{Single Robot} & \textcolor{goodgray}{N/A}     & \textcolor{goodred}{Real}          & \textcolor{goodred}{HTC Vive Pro} & \textcolor{goodgray}{N/A} & \textcolor{goodred}{Cartesian} \\
        \citet{jiang2024comprehensive}                     & \textcolor{goodred}{Limited}     & \textcolor{goodred}{Single Robot} & \textcolor{goodgray}{N/A}     & \textcolor{goodred}{Real}          & \textcolor{goodred}{HoloLens 2}   & \textcolor{goodgray}{N/A} & \textcolor{goodgreen}{Joint \& Cartesian} \\
        \citet{mosbach2022accelerating}                    & \textcolor{goodgreen}{Available} & \textcolor{goodred}{Single Robot} & \textcolor{goodred}{IsaacGym} & \textcolor{goodred}{Sim}           & \textcolor{goodred}{Vive}         & \textcolor{goodgray}{N/A} & \textcolor{goodgreen}{Joint \& Cartesian} \\
        Holo-Dex \cite{holodex}                            & \textcolor{goodgray}{N/A}        & \textcolor{goodred}{Single Robot} & \textcolor{goodgray}{N/A}     & \textcolor{goodred}{Real}          & \textcolor{goodred}{Meta Quest 2} & \textcolor{goodgray}{N/A} & \textcolor{goodred}{Joint} \\
        ARCADE \cite{arcade}                               & \textcolor{goodgray}{N/A}        & \textcolor{goodred}{Single Robot} & \textcolor{goodgray}{N/A}     & \textcolor{goodred}{Real}          & \textcolor{goodred}{HoloLens 2}   & \textcolor{goodgray}{N/A} & \textcolor{goodred}{Cartesian} \\
        DART \cite{dexhub-park}                            & \textcolor{goodred}{Limited}     & \textcolor{goodred}{Limited}      & \textcolor{goodred}{Mujoco}   & \textcolor{goodred}{Sim}           & \textcolor{goodred}{Vision Pro}   & \textcolor{goodgray}{N/A} & \textcolor{goodred}{Cartesian} \\
        ARMADA \cite{armada}                               & \textcolor{goodgray}{N/A}        & \textcolor{goodred}{Limited}      & \textcolor{goodgray}{N/A}     & \textcolor{goodred}{Real}          & \textcolor{goodred}{Vision Pro}   & \textcolor{goodgray}{N/A} & \textcolor{goodred}{Cartesian} \\
        \citet{meng2023virtual}                            & \textcolor{goodred}{Limited}     & \textcolor{goodred}{Single Robot} & \textcolor{goodred}{PhysX}   & \textcolor{goodgreen}{Sim \& Real} & \textcolor{goodred}{HoloLens 2}   & \textcolor{goodgray}{N/A} & \textcolor{goodred}{Cartesian} \\
        % GELLO \cite{wu2023gello}                           & \cmark & \xmark & \xmark & \xmark & \xmark & \xmark \\
        % DexCap \cite{wang2024dexcap}                       & \xmark & \xmark & \xmark & \xmark & \xmark & \xmark \\
        % AnyTeleop \cite{Qin2023AnyTeleopAG}                & \cmark & \cmark & \cmark & \cmark & \xmark & \cmark \\
        % \citet{wang2024robotic}                            & \xmark & \xmark & \xmark & \xmark & \xmark & \xmark \\
        Bunny-VisionPro \cite{bunnyvisionpro}              & \textcolor{goodgray}{N/A}        & \textcolor{goodred}{Single Robot} & \textcolor{goodgray}{N/A}     & \textcolor{goodred}{Real}          & \textcolor{goodred}{Vision Pro}   & \textcolor{goodgray}{N/A} & \textcolor{goodred}{Cartesian} \\
        IMMERTWIN \cite{immertwin}                         & \textcolor{goodgray}{N/A}        & \textcolor{goodred}{Limited}      & \textcolor{goodgray}{N/A}     & \textcolor{goodred}{Real}          & \textcolor{goodred}{HTC Vive}     & \textcolor{goodgray}{N/A} & \textcolor{goodred}{Cartesian} \\
        Open-TeleVision \cite{opentelevision}              & \textcolor{goodgray}{N/A}        & \textcolor{goodred}{Limited}      & \textcolor{goodgray}{N/A}     & \textcolor{goodred}{Real}          & \textcolor{goodgreen}{Meta Quest, Vision Pro} & \textcolor{goodgray}{N/A} & \textcolor{goodred}{Cartesian} \\
        \citet{szczurek2023multimodal}                     & \textcolor{goodgray}{N/A}        & \textcolor{goodred}{Limited}      & \textcolor{goodgray}{N/A}     & \textcolor{goodred}{Real}          & \textcolor{goodred}{HoloLens 2}   & \textcolor{goodgreen}{Available} & \textcolor{goodred}{Joint \& Cartesian} \\
        OPEN TEACH \cite{openteach}                        & \textcolor{goodgray}{N/A}        & \textcolor{goodgreen}{Available}  & \textcolor{goodgray}{N/A}     & \textcolor{goodred}{Real}          & \textcolor{goodred}{Meta Quest 3} & \textcolor{goodred}{N/A} & \textcolor{goodgreen}{Joint \& Cartesian} \\
        \midrule
        \textbf{Ours}                                      & \textcolor{goodgreen}{Available} & \textcolor{goodgreen}{Available}  & \textcolor{goodgreen}{Mujoco, CoppeliaSim, IsaacSim} & \textcolor{goodgreen}{Sim \& Real} & \textcolor{goodgreen}{Meta Quest 3, HoloLens 2} & \textcolor{goodgreen}{Available} & \textcolor{goodgreen}{Joint \& Cartesian} \\
        \bottomrule
        \end{tabular}
    \end{adjustbox}
    \caption{Comparison of XR-based system for robots. IRIS is compared with related works in different dimensions.}
\end{table*}



\section{Methods}
\subsection{Dynamic Color Image Normalization}
Fig. \ref{fig:fig_2} (blue dashed box) shows the data flow of our dynamic color image normalization (DCIN) method. 
For a given input test image from a non-source domain, the reference image selection module strategically identifies suitable reference images from the training source domain. 
The color transfer in the perception-based color space $l\alpha \beta$ \cite{reinhard2001color} is then applied to align the color distribution of the reference images with that of the test image. 
The reference image selection module incorporates two strategies: \say{global} reference selection and \say{local} reference selection. 
The global strategy assigns a single reference image to all test images, whereas the local strategy selects a unique reference image for each individual test image. 
Detailed descriptions of these strategies are described below. 

\subsubsection{Global Reference Image Selection}
We propose the global reference image selection (GRIS) strategy that utilizes color histograms such that each image in the source domain is converted into a $b$-bin normalized color histogram vector. 
The global reference image $x_g$ is selected as the image whose color histogram minimizes the average pairwise distance between histograms of all other images in the source domain. 
The average pairwise distance $\mathcal{D}_{\mathrm{pairwise}}(x_i) = \frac1N\overset N{\underset{j=1}{\sum d_{i,j}}}$, where $d_{i,j}=\sqrt{\sum_{k=1}^b\left(H_k\left(x_i\right)-H_k\left(x_j\right)\right)^2}$ is the Euclidean distance between two normalized histogram vectors. 
In this case, $H_k(x)$ is the $k$-th bin value of the image $x$, and $N$ is the number of images in the source domain. 
The selected image $x_g$ will be used as the color normalization reference of \textit{all} test images before making predictions. 

\subsubsection{Local Reference Image Selection}
As stated previously, segmentation results vary on different reference images. 
Thus, we believe selecting a semantically similar image to the test image from the training data can benefit the segmentation performance. 
We propose a local reference image selection (LRIS) strategy that utilizes a pre-trained CNN model to select a more tailored reference image for each test image. 
First, the pre-trained CNN was used to extract feature vectors from all source images, which were then normalized into unit vectors. 
For a given image $x_{test}$ in a test domain, a local reference image $x_l$ is selected as the one whose feature vector has the highest cosine similarity with $x_{test}$. 
The selected image $x_l$ will be used as the color normalization reference of the test image $x_{test}$ before making predictions. 

\subsubsection{Ensembling Both Selected Reference Images}
To further improve generalization capabilities, the results of the above two strategies can be ensembles. 
From the selected global and local reference images, two color-normalized input images were produced, which were then used to generate two corresponding output masks. 
Finally, the final prediction is formed by taking a pixel-wise mean of the two predicted masks. 
Here, we refer to the \say{full DCIN} as the complete module with this ensemble method (i.e., GRIS + LRIS). 

\subsection{The Color-Quality Generalization Loss}
The color-quality generalization loss (CQG) is a training objective function inspired by contrastive loss. 
The idea is that given the same input presented in varying colors and qualities, the model should produce identical segmentation masks. 
This approach encourages the model to adapt effectively to images with diverse variations. 
Fig. \ref{fig:fig_2} (purple dashed box) illustrates the data flow of the CQG loss. 
For each training image $x$, we apply transformations randomly to generate two inputs for the segmentation model. 
The first input $x_1$ is obtained by applying geometric transformations, and the second input $x_2$ is from both geometric and photometric transformations. 
Geometric transformations are applied to change the input $x$ geometrically while photometric transformations change its color and quality. 
Specifically, geometric transformations include random horizontal flip, shear, shift, scale, rotation, and elastic transform. 
Photometric transformations include random blur, sharpening, Gaussian noise, brightness contrast, and RGB shifts. 
Both $x_1$ and $x_2$ have the same ground-truth mask $y$.

Given a segmentation model $S$ and a ground-truth mask $y$, we have $y_1 = S(x_1)$, and $y_2 = S(x_2)$. 
Our CQG loss is defined as:
\begin{equation}
    \mathcal{L} = \lambda_{1}\mathrm{DC}(y,y_1) + \lambda_{2}\mathrm{DC}(y,y_2) + \lambda_{3}\mathrm{MSE}(y_1,y_2),
\end{equation}
where $\mathrm{DC}(y,y\prime)$ is the sum of the Dice loss and cross-entropy loss between the ground truth $y$ and predicted mask $y\prime$. 
$\mathrm{MSE}(y_1,y_2)$ is the mean squared error loss between the predicted masks $y_1$ and $y_2$. 
Here, $\lambda_{1}, \lambda_{2}, \lambda_{3}$ are the hyper-parameters controlling the weight of each loss term. 

The CQG loss can be viewed as a form of image augmentation. By leveraging this loss, the model is encouraged to produce identical predictions for both original and augmented images, regardless of color and quality shifts. 

\section{Results and discussion}
\label{sec:results_and_discussion}

\subsection{Isothermal binary vapor-liquid equilibria}

The 903 isothermal subsets from the cleaned dataset were predicted using the GH-GNN model embedded within the extended Margules model. In total, 12,980 data points were analyzed, involving 180 distinct compounds across 531 unique binary combinations. The temperature range of the isothermal subsets spans from 127.59 K to 573.15 K.

\begin{figure}[h]
    \centering
    \includegraphics[width=1\textwidth]{figures/MAE_matrix_organic_old_Jaccard0.6.png}
    \caption{Heatmap of the mean absolute error achieved by the GH-GNN model embedded into the Margules model for different types of binary mixtures at isothermal conditions. The error is measured with respect to the vapor phase molar fraction. The number in each cell represents the number of data points included in the corresponding binary category.}
    \label{fig:isothermal_map}
\end{figure}

To compute the activity coefficients for all isothermal systems in the dataset, a total of 1062 IDACs were required. Of the 531 binary systems, 120 (22.6\%) had both IDACs observed during training. For 382 systems (72\%), the GH-GNN model had to predict at least one of the two required IDACs in other solute-solvent combination as the ones observed during training. For the remaining 29 systems (5.4\%), the model needed to predict at least one of the necessary IDACs with molecules never observed before during training. This was due to 15 compounds (out of the 180 in the isothermal VLE data) not being present in the GH-GNN training set. The fact that even for this limited VLE dataset only very few of the necessary IDACs have been measured experimentally underscores the strong motivation for developing reliable predictive methods.

\begin{figure}[h]
    \centering
    \includegraphics[width=0.85\textwidth]{figures/isothermal_vle.png}
    \caption{Isothermal vapor-liquid equilibria (VLE) diagram for two systems that are unfeasible to predict with UNIFAC-Dortmund.}
    \label{fig:isothermal_diagram}
\end{figure}

A similar analysis for UNIFAC is challenging to perform, as the specific mixtures and data used for its parametrization are not readily accessible. Furthermore, it is important to note that while the GH-GNN embedded within the Margules model had access exclusively to IDAC data, UNIFAC was parameterized using a broader range of experimental data, including IDACs, vapor-liquid equilibria (VLE), liquid-liquid equilibria (LLE), solid-liquid equilibria (SLE), azeotropic data, and calorimetric measurements \cite{gmehling2002modified}.

Figure \ref{fig:isothermal_map} presents a heatmap of the mean absolute error (MAE) achieved by the GH-GNN embedded in the Margules model when predicting the vapor-phase molar fraction of the binary vapor-liquid equilibria (VLE) under isothermal conditions. 

Overall, systems containing acids are particularly challenging for the analyzed model to predict, especially when combined with halogenated species or amides. This difficulty likely arises from the strong hydrogen-bonding and dipole-dipole interactions characteristic of these compounds, which may be inadequately captured by the information provided to the GH-GNN model. Additionally, the Margules expression might require higher-order polynomial terms to better account for these complex interactions. Similarly, systems involving alcohols and amines also exhibit poor predictive accuracy. Furthermore, systems containing water (categorized here under the ``Inorganic" class), particularly in combination with amines and other nitrogen derivatives, were associated with relatively large prediction errors.


\begin{table}[h]
\centering
  \caption{Comparison between UNIFAC-Dortmund and the GH-GNN model embedded into the Margules model for predicting binary vapor-liquid equilibria (VLE) under isothermal conditions. The comparison is according to the mean absolute error (MAE), the coefficient of determination (R$^2$) and the percentage of points predicted with an absolute error of less than or equal to 0.03. All metrics are with respect to the predicted and actual values of the molar fraction in the vapor phase.}
  \label{tbl:isothermal}
  \small
  \begin{tabular*}{1\textwidth}{@{\extracolsep{\fill}}lccc}
    \hline
    \textbf{Model} & \textbf{MAE} $\downarrow$ & \textbf{R$^2$} $\uparrow$ & \textbf{AE $\leq 0.03$} $\uparrow$ \\
    \hline
    UNIFAC-Dortmund & 0.0214 & 0.957 & 86.72\% \\
    GH-GNN + Margules & 0.0307 & 0.948 & 74.77\% \\
    \hline
  \end{tabular*}
\end{table}


The comparison between the UNIFAC-Dortmund and GH-GNN model embedded into the Margules model, presented in Table \ref{tbl:isothermal}, indicates that UNIFAC-Dortmund predicts most isothermal VLE binary systems more accurately than the GNN-based Margules hybrid model. This comparison includes all mixtures that can be feasibly predicted using the UNIFAC model (12,700 data points, 514 binary mixtures).

However, this result is not unexpected, given that, as previously mentioned, extensive VLE data were also used in the parametrization of UNIFAC-Dortmund. In contrast, the GH-GNN model, combined with the Margules framework, achieves its performance using only experimental IDAC data. Consequently, while the GNN-based predictive model performs some degree of extrapolation from the infinite regime to finite concentrations, the UNIFAC model is likely reproducing VLE data that were part of its original parametrization. Overall, this comparison shows the predictive limitations of GNNs embedded into the Margules framework, but it also establishes a baseline of the degree of accuracy that can be reached by employing scarce data at infinite dilution with GNN models and Gibbs-Duhem consistent expressions for predicting VLEs.

When comparing the flexibility of the two predictive approaches, UNIFAC-Dortmund and the GNNs embedded in the Margules model, some systems in the dataset cannot be predicted using the UNIFAC model. This limitation arises either from the unfeasibility of fragmentation or the lack of available interaction parameters. In this regard, the proposed framework combining GNNs with the Margules model offers an effective alternative.

For example, Figure \ref{fig:isothermal_diagram} shows the VLE diagram for two systems that UNIFAC-Dortmund cannot predict. In both cases, at least one of the necessary IDACs was not available experimentally, and the GH-GNN model was needed for prediction. Notably, as shown for the system ``1,1,2-trichlorotrifluoroethane / methyl acetate" in Figure \ref{fig:isothermal_diagram}, the GH-GNN model embedded within Margules successfully captures azeotropic behavior with significant precision, even without dedicated azeotropic data used during training.

\subsection{Isobaric binary vapor-liquid equilibria}

The 781 isobaric subsets contain a total of 14,118 data points, covering 162 distinct compounds in 525 different binary combinations. The pressure ranges from 1.33 to 496.63 kPa. Among these subsets, 106 systems (20.2\%) had both necessary IDACs available from experimental measurements and were therefore observed during the GH-GNN model training. Additionally, 409 systems (77.9\%) required interpolation for at least one IDAC, meaning the values were predicted based on observations of the compounds in other binary combinations. The remaining 10 systems (1.9\%) involved molecules that were entirely absent during GH-GNN model training (i.e., extrapolations).

In order to perform the isobaric VLE calculations, an iterative process is needed in which the temperature used for predicting the activity coefficients is changed in order to minimize the difference between the estimated system pressure and the actual pressure. This is represented in the optimization problem \ref{eq_optim_isobaric}, which was here solved using the SciPy package \cite{virtanen2020scipy} with Brent's algorithm, allowing up to 2,000 iterations and a tolerance of $1.48 \times 10^{-8}$.

\begin{equation}
\begin{aligned}
\underset{\substack{T}}{\text{min}} \quad & \vert P - \hat{P}(T) \vert \\
\text{s.t.} \quad & \hat{P}(T) = x_i \gamma_i(T) P_i^{sat}(T) + x_j \gamma_j(T) P_j^{sat}(T) \\
& \{\gamma_i(T), \gamma_j(T)\} \leftarrow \text{GH-GNN embedded in Margules}(T)\\
& T_{\min} \leq T \leq T_{\max}
\end{aligned}
\label{eq_optim_isobaric}
\end{equation}

\noindent where $P$ and $\hat{P}$ denote the actual and predicted system pressures, respectively, and $T_{\min}$ and $T_{\max}$ define the optimization bounds for the temperature. In this case, these bounds correspond to the minimum temperature range in which the vapor pressure correlation (Equation \ref{eq_vapor_pressure_kdb}) remains valid.

\begin{figure}[h!]
    \centering
    \includegraphics[width=1\textwidth]{figures/MAE_matrix_organic_old_Jaccard0.6_isobaric.png}
    \caption{Heatmap of the mean absolute error achieved by the GH-GNN model embedded into the Margules model for different types of binary mixtures at isobaric conditions. The error is measured with respect to the vapor phase molar fraction. The number in each cell represents the number of data points included in the corresponding binary category.}
    \label{fig:isobaric_map}
\end{figure}

\begin{table}[h]
\centering
  \caption{Comparison between UNIFAC-Dortmund and the GH-GNN model embedded into the Margules model for predicting binary vapor-liquid equilibria (VLE) under isobaric conditions. The comparison is according to the mean absolute error (MAE), the coefficient of determination (R$^2$) and the percentage of points predicted with an absolute error of less than or equal to 0.03. All metrics are with respect to the predicted and actual values of the molar fraction in the vapor phase.}
  \label{tbl:isobaric}
  \small
  \begin{tabular*}{1\textwidth}{@{\extracolsep{\fill}}lccc}
    \hline
    \textbf{Model} & \textbf{MAE} $\downarrow$ & \textbf{R$^2$} $\uparrow$ & \textbf{AE $\leq 0.03$} $\uparrow$ \\
    \hline
    UNIFAC-Dortmund & 0.0224 & 0.967 & 81.91\% \\
    GH-GNN + Margules & 0.0350 & 0.946 & 65.78\% \\
    \hline
  \end{tabular*}
\end{table}

Table \ref{tbl:isobaric} compares the performance of the UNIFAC-Dortmund model and the GH-GNN model embedded into Margules. The comparison is shown only for the systems that are feasible to predict with UNIFAC, resulting in 14,075 data points. Similar to the isothermal systems, UNIFAC-Dortmund generally outperforms the GH-GNN model combined with Margules in predicting isobaric VLEs of binary systems. As previously explained, this outcome is expected due to the difference in available experimental data during training. Nevertheless, it provides a useful baseline for assessing the accuracy achievable when predicting finite compositions from infinite dilution information using predictive methods based on GNNs.

\begin{figure}[h!]
    \centering
    \includegraphics[width=0.85\textwidth]{figures/2_PYRIDINE_1,2,3,4-TETRAHYDRONAPHTHALENE_26.66.png}
    \caption{Isobaric vapor-liquid equilibria (VLE) diagram comparing experimental data with the predictions of UNIFAC-Dortmund and the GH-GNN model embedded into the Margules model. The system is pyridine/1,2,3,4-tetrahydronaphthalene at 26.66 kPa.}
    \label{fig:isobaric_diagram_1}
\end{figure}

The heatmap in Figure \ref{fig:isobaric_map} illustrates the performance of the GNN-Margules-based model across different binary combinations. Notably, compared to the isothermal data, different binary classes are now considered. For instance, combinations such as acids/alcohols and acids/aromatics are present under isobaric conditions but absent in the isothermal dataset. In total, 16 new binary classes appear in the isobaric data that were not included in the isothermal data. However, 40 binary classes are present only at isothermal conditions. In total the intersection of binary classes between isothermal and isobaric conditions is 60. Similar to the isothermal case, many binary classes under isobaric conditions are predicted with relatively low errors.

Out of the 737 isobaric systems that are feasible to predict with UNIFAC-Dortmund, 182 are, on average, better predicted by the GNN-Margules-based model than by UNIFAC-Dortmund. To illustrate this, Figure \ref{fig:isobaric_diagram_1} presents the VLE diagram for the system ``pyridine/1,2,3,4-tetrahydronaphthalene" at 26.66 kPa. It is evident that the proposed model here predicts the VLE behavior more accurately compared to UNIFAC-Dortmund. A similar trend is observed in Figure \ref{fig:isobaric_diagram_2}, which shows the VLE diagram for the system “tetrahydrofuran/ethanol" at 25 kPa.

\begin{figure}[h]
    \centering
    \includegraphics[width=0.85\textwidth]{figures/15_TETRAHYDROFURAN_ETHANOL_25.0.png}
    \caption{Isobaric vapor-liquid equilibria (VLE) diagram comparing experimental data with the predictions of UNIFAC-Dortmund and the GH-GNN model embedded into the Margules model. The system is tetrahydrofuran/ethanol at 25 kPa.}
    \label{fig:isobaric_diagram_2}
\end{figure}

In these two isobaric VLE examples, the GH-GNN model interpolates at least one of the two required IDACs. This highlights that, although UNIFAC-Dortmund generally outperforms the proposed model, there are counterexamples where the GH-GNN combined with Margules model provides superior predictions, which may be relevant in specific practical scenarios where such mixtures might be of interest.

\subsection{Overall performance predicting binary vapor-liquid equilibria}

\begin{figure}[h!]
    \centering
    \includegraphics[width=1\textwidth]{figures/MAE_matrix_UNIFACDo_vs_GHGNNMargules.png}
    \caption{Comparison between the UNIFAC-Dortmund and the GH-GNN in Margules model based on the mean absolute error (MAE) for predicting the vapor molar fraction. All feasible systems at isothermal, isobaric and random conditions are included. The number in each cell represents the number of data points.}
    \label{fig:mae_matrix_ghgnn_vs_unifac}
\end{figure}

To gain a more detailed understanding of the binary classes where the GH-GNN model embedded into Margules outperforms UNIFAC-Dortmund, Figure \ref{fig:mae_matrix_ghgnn_vs_unifac} presents the best-performing model for each binary combination. The performance of each binary class is evaluated based on the mean absolute error (MAE) across all temperature and pressure conditions, including the isothermal, isobaric, and random subsets. Predictions are assessed with respect to the molar fraction in the vapor phase.


\begin{figure}[h]
    \centering
    \includegraphics[width=0.85\textwidth]{figures/Cumulative_MAE_classes_UNIFACDo_vs_GHGNNMargules.png}
    \caption{Cumulative percentage of binary vapor-liquid equilibria data points predicted below different thresholds of absolute error with respect to the molar fraction in the vapor phase. The performance is shown for the models UNIFAC-Dortmund and the GH-GNN embedded into Margules.}
    \label{fig:cummulative}
\end{figure}

As previously discussed for the isothermal and isobaric cases, the UNIFAC-Dortmund model generally performs better for most binary classes. However, certain binary classes are predicted more accurately by the GH-GNN model combined with Margules. For instance, the binary class ``amines/aromatics," which includes the system shown in Figure \ref{fig:isobaric_diagram_1}, tends to be better predicted using the proposed GH-GNN in Margules model. Out of the 116 distinct binary classes studied in this work, 27 binary classes are on averaged predicted with lower errors by the GH-GNN in Margules model compared to UNIFAC-Dortmund.

Figure \ref{fig:cummulative} presents the cumulative percentage of binary VLE data points predicted with absolute errors below various thresholds, based on the molar fraction in the vapor phase. Both models achieve a similar percentage of well-predicted data points for absolute errors up to around 0.01. However, UNIFAC-Dortmund consistently yields lower absolute errors overall. This is primarily attributed to differences in the experimental data available during training but also highlights the potential of GNN-based hybrid models for extrapolating finite concentration conditions from limited infinite dilution data. Including VLE data in the training of GNN-based models alongside infinite dilution data would likely lead to a significant increase in accuracy, as already observed in the case of transformers \cite{winter2023spt}.

\subsection{Ternary vapor-liquid equilibria}

In this section, we provide insights into the relative performance of the GH-GNN model embedded in Margules compared to UNIFAC-Dortmund for ternary mixtures. As shown in Table \ref{tbl:ternary_systems}, the available ternary data is much more limited. Therefore, this comparison aims to provide qualitative insights rather than an in-depth analysis, as conducted for binary mixtures.

\begin{table}[h]
\centering
  \caption{Comparison between UNIFAC-Dortmund and the GH-GNN model embedded into the Margules model for predicting ternary vapor-liquid equilibria (VLE). The comparison is according to the mean absolute error (MAE), the coefficient of determination (R$^2$) and the percentage of points predicted with an absolute error of less than or equal to 0.03. All metrics are with respect to the predicted and actual values of the molar fraction in the vapor phase of components 1 and 2.}
  \label{tbl:ternary}
  \small
  \begin{tabular*}{1\textwidth}{@{\extracolsep{\fill}}lccc}
    \hline
    \multicolumn{4}{c}{\textbf{Isothermal system}}\\
    \hline
    \textbf{Model} & \textbf{MAE} $\downarrow$ & \textbf{R$^2$} $\uparrow$ & \textbf{AE $\leq 0.03$} $\uparrow$ \\
    \hline
    UNIFAC-Dortmund & 0.0059 & 0.995 & 100\% \\
    GH-GNN + Margules & 0.0568 & 0.394 & 37.96\% \\
    \hline
    \multicolumn{4}{c}{\textbf{Isobaric systems}}\\
    \hline
    \textbf{Model} & \textbf{MAE} $\downarrow$ & \textbf{R$^2$} $\uparrow$ & \textbf{AE $\leq 0.03$} $\uparrow$ \\
    \hline
    UNIFAC-Dortmund & 0.0123 & 0.991 & 89.07\% \\
    GH-GNN + Margules & 0.0330 & 0.925 & 66.60\% \\
    \hline
  \end{tabular*}
\end{table}

Similar to the case of binary systems presented earlier, UNIFAC-Dortmund generally provides more accurate estimations of ternary VLEs (see Table \ref{tbl:ternary}). Despite this, predictions from the GH-GNN model combined with Margules also align well with the physical behavior of the mixtures. Moreover, as observed in binary mixtures, the GH-GNN embedded in Margules produces more accurate predictions for certain systems compared to UNIFAC-Dortmund.

Specifically, among the 10 ternary systems analyzed, the GH-GNN in Margules achieves a lower MAE than UNIFAC-Dortmund for the following systems: ``benzene/heptane/dimethylformamide" (MAE of 0.0296 vs. 0.0297 for UNIFAC-Dortmund), ``benzene/heptane/acetonitrile" (MAE of 0.0084 vs. 0.0114 for UNIFAC-Dortmund), and ``acetone/chloroform/benzene" (MAE of 0.0047 vs. 0.0184 for UNIFAC-Dortmund). For the remaining systems, UNIFAC-Dortmund achieves lower overall MAE. This exemplifies that, for systems where UNIFAC-Dortmund cannot provide predictions due to missing interaction parameters or unfeasibility of fragmentation into functional groups, the GH-GNN in Margules model offers a useful first-step approximation of the mixture behavior, which could be beneficial in early stages of molecular and process design.

\section{Conclusions}
\label{sec:conclusions}

Predictive models for thermophysical properties of pure compounds and their mixtures play a crucial role in chemical space exploration for product and process design. While numerous predictive methods have been developed over the decades, they remain limited in terms of both accuracy and the range of systems they can effectively describe.

In this work, we studied the performance of a GNN-based model embedded in the extended Margules framework, which assumes that the excess Gibbs energy can be approximated using a Taylor polynomial. The performance of this model was compared against the widely used UNIFAC-Dortmund model across extensive vapor-liquid equilibrium (VLE) data for binary systems under isothermal and isobaric conditions.

Additionally, we provide insights into the relative performance of these models in predicting ternary VLEs. Overall, UNIFAC-Dortmund yields more accurate VLE predictions. However, a crucial factor in this comparison is the difference in the types of experimental data used for model development. While the GNN embedded in Margules was trained exclusively on limited infinite dilution data, requiring extrapolation from infinite dilution to finite concentrations, UNIFAC-Dortmund was parameterized using a much broader dataset, including infinite dilution data, VLE, LLE, SLE, azeotropic, and caloric data \cite{constantinescu2016further,gmehling2002modified}.

It is likely that incorporating VLE data alongside infinite dilution data could significantly enhance the accuracy of GNN-based models within the Margules framework. Nonetheless, this study establishes a valuable baseline, demonstrating the predictive accuracy achievable when extrapolating VLE behavior solely from binary infinite dilution data using a relatively simple excess Gibbs energy model in combination with a highly flexible GNN-based model.

Future work should focus on a comprehensive comparison of GNN-based hybrid models trained on extensive and diverse datasets. However, a key challenge in this endeavor is the structuring and open accessibility of thermodynamic data for model development and benchmarking. Advancing community efforts in thermodynamic data curation and structuring will be essential for the continuous improvement and open-source benchmarking of predictive thermodynamic models.




\section*{Data availability}
The vapor-liquid equilibria (VLE) and pure component data used in this work is available at the accompanying GitHub repository at \url{https://github.com/edgarsmdn/GH_GNN_Margules}.

\section*{Code availability}
The code for generating the figures and running VLE predictions with the GH-GNN model embedded into the Margules expression are available at the accompanying GitHub repository at \url{https://github.com/edgarsmdn/GH_GNN_Margules}.

\section*{CRediT authorship contribution statement}
\textbf{Edgar Ivan Sanchez Medina}: Conceptualization, Data curation, Formal analysis, Investigation, Methodology, Project administration, Software, Validation, Visualization, Writing – original draft.
\textbf{Kai Sundmacher}: Funding acquisition, Supervision, Writing – review and editing.







\section*{Declaration of competing interest}
The authors declare that they have no conflict of interest.

\section*{Acknowledgments}
This work was partly supported by the Research In-itiative “SmartProSys: Intelligent Process Systems for the Sustainable Production of Chemicals”, funded by the Ministry for Science, Energy, Climate Protection and the Environment of the State of Saxony- Anhalt.

\section*{Declaration of generative AI and AI-assisted technologies in the writing process}
During the preparation of this work the authors used gpt-4o-2024-08-06 in order to check the grammar and improve the readability of the text. After using this tool, the authors reviewed and edited the content as needed and take full responsibility for the content of the published article.

\section{Appendix}
%% The Appendices part is started with the command \appendix;
%% appendix sections are then done as normal sections
\appendix
\section{Derivation of the extended Margules model}
\label{app1}

The extended Margules model assumes that the molar excess Gibbs energy $g^E$ of any mixture can be described by a continuous function that is infinitely differentiable. Then, $g^E$ is expressed as a Taylor series, often sufficiently well-approximated when truncated after the third term. Therefore, for a mixture of $N$ components, we have that

\begin{align}
    g^E(\textbf{x}) = C_0 + \sum_{i=1}^{N-1} C_i x_i + \sum_{i=1}^{N-1} \sum_{j=1}^{N-1} C_{ij} x_ix_j + \sum_{i=1}^{N-1} \sum_{j=1}^{N-1} \sum_{k=1}^{N-1} C_{ijk} x_ix_jx_k
    \label{eq_taylor_ge}
\end{align}

\noindent where $\mathbf{x}=[x_1, x_2, \cdots, x_{N-1}]$ represents the vector of molar fractions for the $N-1$ components in the mixture, and $C$ denotes the polynomial coefficient corresponding to the indicated subscripts.

From the boundary condition given by Equation \ref{eq_boundary_condition}, it follows that $g^E=0$ when $x_N=1$. Therefore, $C_0=0$. Similarly, we know that $(C_i + C_{ii} + C_{iii})=0$ when $x_i=1 \quad \forall ~ i \in \{1,2, \dots, N-1\}$.


Therefore, Equation \ref{eq_taylor_ge} can be reformulated as:

\begin{align}
    g^E(\textbf{x}) = - \sum_{i=1}^{N-1} (C_{ii} + C_{iii}) x_i + \sum_{i=1}^{N-1} \sum_{j=1}^{N-1} C_{ij} x_ix_j + \sum_{i=1}^{N-1} \sum_{j=1}^{N-1} \sum_{k=1}^{N-1} C_{ijk} x_ix_jx_k
    \label{eq_taylor_ge2}
\end{align}

By leveraging the fact that the sum of all mole fractions adds to 1, we can re-express the first two terms of Equation \ref{eq_taylor_ge2} to obtain:

\begin{align}
    g^E(\textbf{x}) = - \sum_{i=1}^{N-1} \sum_{j=1}^{N} \sum_{k=1}^{N} (C_{ii} + C_{iii}) x_i x_j x_k + \sum_{i=1}^{N-1} \sum_{j=1}^{N-1} \sum_{k=1}^{N} C_{ij} x_ix_jx_k + \sum_{i = 1}^{N-1} \sum_{j = 1}^{N-1} \sum_{k = 1}^{N-1} C_{ijk} x_ix_jx_k
    \label{eq_taylor_ge3}
\end{align}

We can then simplify the expression by collecting all crossed terms to obtain a general expression that can be applied to estimate $g^E$ for a mixture of $N$ components:

\begin{align}
    g^E(\textbf{x}) = - \sum_{i=1}^{N-1} \sum_{j=i}^{N} \sum_{k=j}^{N} (\hat{C}_{ii} + \hat{C}_{iii}) x_i x_j x_k + \sum_{i=1}^{N-1} \sum_{j=i}^{N-1} \sum_{k=j}^{N} \hat{C}_{ij} x_ix_jx_k + \sum_{i = 1}^{N-1} \sum_{j = i}^{N-1} \sum_{k = j}^{N-1} \hat{C}_{ijk} x_ix_jx_k
    \label{eq_taylor_ge4}
\end{align}

For binary mixtures, Equation \ref{eq_taylor_ge4} simplifies to 

\begin{align}
    g^E = -\hat{C}_{111} x_1^2 x_2 + (-\hat{C}_{11} - \hat{C}_{111}) x_1 x_2^2
    \label{eq_ge_binary}
\end{align}

The value of the parameters $\hat{C}_{111}$ and $\hat{C}_{11}$ can be determined by considering the following relationship between $g^E$ and the IDACs:

\begin{equation}
    RT \ln{\gamma_{ij}^\infty} = \left( \frac{\partial g^E}{\partial x_i} \right)_{x_i \rightarrow 0, ~~ j \neq i} 
    \label{eq_determinar_constants}
\end{equation}

Therefore, the value of the constants is given by

\begin{align}
    \left( \frac{\partial g^E}{\partial x_1} \right)_{x_1 \rightarrow 0} &= - \hat{C}_{11} - \hat{C}_{111} = w_{12} = RT \ln{\gamma_{12}^\infty}\\
    \left( \frac{\partial g^E}{\partial x_2} \right)_{x_2 \rightarrow 0} &= - \hat{C}_{111} = w_{21} = RT \ln{\gamma_{21}^\infty}
\end{align}

\noindent and the final $g^E$ expression for the binary case results in Equation \ref{eq_ge_binary_final}.

The activity coefficients can be then calculated by differentiating Equation \ref{eq_ge_binary_final} in terms of the total excess Gibbs energy $G^E = M g^E$ (where, $M$ stands for the total number of moles) with respect to the number of moles of each component:

\begin{align}
    RT \ln{\gamma_1} =& \frac{\partial G^E}{\partial n_1} = w_{21} \left( 
    \frac{2 n_1 n_2}{M^2} - \frac{2 n_1^2 n_2}{M^3}
    \right)
    +
    w_{12}\left( 
    \frac{n_2^2}{M^2} - \frac{2 n_1 n_2^2}{M^3}
    \right) \\
    RT \ln{\gamma_1} =& 2 w_{21} x_1 x_2 + x_2^2 w_{12} - 2 g^E
\end{align}

\noindent and similarly for component 2:

\begin{equation}
    RT \ln{\gamma_2} = 2 w_{12} x_2 x_1 + x_1^2 w_{21} - 2 g^E
\end{equation}

A similar procedure can be followed to determined the expressions to compute the activity coefficients of components in a ternary mixture:

\begin{align}
    g^E = x_1 x_2 (x_2 w_{12} + x_1 w_{21}) + x_1 x_3 (x_3 w_{13} + x_1 w_{31}) + x_2 x_3 (x_3 w_{23} + x_2 w_{32}) + x_1 x_2 x_3 C_{123}
    \label{eq_ge_margules_ternary}
\end{align}

with $ C_{ijk} = \frac{1}{2}(w_{ij} + w_{ji} + w_{ik} + w_{ki} + w_{jk} + w_{kj}) - w_{ijk}$. While the binary parameters can be related to the corresponding IDACs as shown for the binary case, the ternary parameter $w_{ijk}$ is often set to zero as in \cite{andersen1981valid}. The resulting expressions for the activity coefficients are then given by:

\begin{equation}
    RT \ln{\gamma_1} = 2(x_1 x_2 w_{21} + x_1 x_3 w_{31}) + x_2^2 w_{12} + x_3^2 w_{13} + x_2 x_3 c_{123} - 2 g^E
\end{equation}

\begin{equation}
    RT \ln{\gamma_2} = 2(x_2 x_3 w_{32} + x_2 x_1 w_{12}) + x_3^2 w_{23} + x_1^2 w_{21} + x_3 x_1 c_{231} - 2 g^E
\end{equation}

\begin{equation}
    RT \ln{\gamma_3} = 2(x_3 x_1 w_{13} + x_3 x_2 w_{23}) + x_1^2 w_{31} + x_2^2 w_{32} + x_1 x_2 c_{312} - 2 g^E
\end{equation}





% \begin{thebibliography}{00}

% \bibitem[Lamport(1994)]{lamport94}
%   Leslie Lamport,
%   \textit{\LaTeX: a document preparation system},
%   Addison Wesley, Massachusetts,
%   2nd edition,
%   1994.

% \end{thebibliography}

\bibliographystyle{elsarticle-num-names}
\bibliography{bibliography}

\end{document}

\endinput

