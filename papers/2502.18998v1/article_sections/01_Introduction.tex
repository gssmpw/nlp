\label{sec:introduction}

% Importance of predicting vapor-liquid equilibria
Modeling vapor-liquid equilibria is essential for the development of most chemical processes. This is because many chemical processes operate under conditions where vapor and liquid phases interact. Although vapor-liquid equilibria is the most frequently measured phase equilibria property \cite{DDBST}, the vast extension of the chemical space makes it impractical to rely solely on experimental measurements. Consequently, predictive methods have been explored for decades to efficiently predict vapor-liquid equilibria \cite{gmehling2019chemical}, thereby aiding in the design of chemical processes and minimizing the need for experimental measurements in early stages of the process design.

% Overview of predictive methods for VLE: mechanistic and machine learning
 Various predictive models have been developed for estimating vapor-liquid equilibria. Most of these models are based on mechanistic or phenomenological understanding. Examples include the widely used excess Gibbs energy ($G^E$) model UNIFAC \cite{constantinescu2016further} and equations of state like PCP-SAFT \cite{gross2006equation}. Recently, however, the scientific community has begun exploring hybrid approaches that combine mechanistic and data-driven methods to create predictive models, particularly for predicting activity coefficients. Among these models, some have been developed by embedding a machine learning model within a mechanistic framework, while others enforce thermodynamic constraints directly into the model structure and/or during training.

% Overview of ML models embeeded into mechanistic ones
Machine learning models for predicting activity coefficients that have been embedded into mechanistic expressions include the SPT-NRTL model \cite{winter2023spt}, which combines transformers with the NRTL model, UNIFAC 2.0 \cite{hayer2025advancing}, which uses matrix completion methods to compute the binary interaction matrix of the UNIFAC model, and a similar approach within the UNIQUAC model \cite{jirasek2022making}. Additionally, models developed specifically for predicting the infinite dilution regime, such as the Gibbs-Helmholtz Graph Neural Network (GH-GNN) \cite{medina2023gibbs} and matrix (or tensor) completion methods \cite{damay2021predicting, damay2023predicting}, incorporate an expression derived from the Gibbs-Helmholtz equation as the head of the model to introduce the temperature dependency of activity coefficients at infinite dilution.

% Overview of ML models with thermodynamic constraints directly embedded into model or model training
Examples of machine learning models that incorporate thermodynamic constraints directly into the model structure include the HANNA model \cite{specht2024hanna}, which combines neural networks with explicit thermodynamic constraints to achieve thermodynamically consistent predictions, and the GE-GNN model \cite{rittig2024thermodynamics}, which similarly enforces Gibbs-Duhem consistency within a graph neural network-based model. While these two examples strictly enforce the consistency of the constraints, other ``soft-constraint" approaches have been explored, such as the Gibbs-Duhem informed graph neural network \cite{rittig2023gibbs}.

% Contributions of this paper
Both aspects of integrating machine learning with mechanistic modeling are highly relevant in the field of fluid phase equilibria modeling. We anticipate continued advancements in both areas in the coming years, potentially pushing the boundaries of predictive modeling in this domain. In this contribution, we investigate the incorporation of graph neural network models into a mechanistic framework: the extended Margules model. Although the Margules head model is simpler compared to other hybridized machine learning models (e.g., those combined with head models of NRTL \cite{winter2023spt}, UNIQUAC \cite{jirasek2022making}, or UNIFAC\cite{hayer2025advancing}), it offers a foundational perspective for evaluating how relatively simpler approaches perform compared to more complex ones.

% Structure of this paper  
This paper is structured as follows. First, the methods are reviewed, providing an overview of the vapor-liquid equilibrium considerations used for the calculations, the extended Margules model, the data sources and cleaning procedure, and a brief introduction to the GH-GNN model employed in this study. Second, the results are presented and discussed, focusing primarily on the prediction of vapor-liquid equilibria in binary systems, with additional analyses for ternary mixtures. Finally, the conclusions and overview are presented.
