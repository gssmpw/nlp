%% The Appendices part is started with the command \appendix;
%% appendix sections are then done as normal sections
\appendix
\section{Derivation of the extended Margules model}
\label{app1}

The extended Margules model assumes that the molar excess Gibbs energy $g^E$ of any mixture can be described by a continuous function that is infinitely differentiable. Then, $g^E$ is expressed as a Taylor series, often sufficiently well-approximated when truncated after the third term. Therefore, for a mixture of $N$ components, we have that

\begin{align}
    g^E(\textbf{x}) = C_0 + \sum_{i=1}^{N-1} C_i x_i + \sum_{i=1}^{N-1} \sum_{j=1}^{N-1} C_{ij} x_ix_j + \sum_{i=1}^{N-1} \sum_{j=1}^{N-1} \sum_{k=1}^{N-1} C_{ijk} x_ix_jx_k
    \label{eq_taylor_ge}
\end{align}

\noindent where $\mathbf{x}=[x_1, x_2, \cdots, x_{N-1}]$ represents the vector of molar fractions for the $N-1$ components in the mixture, and $C$ denotes the polynomial coefficient corresponding to the indicated subscripts.

From the boundary condition given by Equation \ref{eq_boundary_condition}, it follows that $g^E=0$ when $x_N=1$. Therefore, $C_0=0$. Similarly, we know that $(C_i + C_{ii} + C_{iii})=0$ when $x_i=1 \quad \forall ~ i \in \{1,2, \dots, N-1\}$.


Therefore, Equation \ref{eq_taylor_ge} can be reformulated as:

\begin{align}
    g^E(\textbf{x}) = - \sum_{i=1}^{N-1} (C_{ii} + C_{iii}) x_i + \sum_{i=1}^{N-1} \sum_{j=1}^{N-1} C_{ij} x_ix_j + \sum_{i=1}^{N-1} \sum_{j=1}^{N-1} \sum_{k=1}^{N-1} C_{ijk} x_ix_jx_k
    \label{eq_taylor_ge2}
\end{align}

By leveraging the fact that the sum of all mole fractions adds to 1, we can re-express the first two terms of Equation \ref{eq_taylor_ge2} to obtain:

\begin{align}
    g^E(\textbf{x}) = - \sum_{i=1}^{N-1} \sum_{j=1}^{N} \sum_{k=1}^{N} (C_{ii} + C_{iii}) x_i x_j x_k + \sum_{i=1}^{N-1} \sum_{j=1}^{N-1} \sum_{k=1}^{N} C_{ij} x_ix_jx_k + \sum_{i = 1}^{N-1} \sum_{j = 1}^{N-1} \sum_{k = 1}^{N-1} C_{ijk} x_ix_jx_k
    \label{eq_taylor_ge3}
\end{align}

We can then simplify the expression by collecting all crossed terms to obtain a general expression that can be applied to estimate $g^E$ for a mixture of $N$ components:

\begin{align}
    g^E(\textbf{x}) = - \sum_{i=1}^{N-1} \sum_{j=i}^{N} \sum_{k=j}^{N} (\hat{C}_{ii} + \hat{C}_{iii}) x_i x_j x_k + \sum_{i=1}^{N-1} \sum_{j=i}^{N-1} \sum_{k=j}^{N} \hat{C}_{ij} x_ix_jx_k + \sum_{i = 1}^{N-1} \sum_{j = i}^{N-1} \sum_{k = j}^{N-1} \hat{C}_{ijk} x_ix_jx_k
    \label{eq_taylor_ge4}
\end{align}

For binary mixtures, Equation \ref{eq_taylor_ge4} simplifies to 

\begin{align}
    g^E = -\hat{C}_{111} x_1^2 x_2 + (-\hat{C}_{11} - \hat{C}_{111}) x_1 x_2^2
    \label{eq_ge_binary}
\end{align}

The value of the parameters $\hat{C}_{111}$ and $\hat{C}_{11}$ can be determined by considering the following relationship between $g^E$ and the IDACs:

\begin{equation}
    RT \ln{\gamma_{ij}^\infty} = \left( \frac{\partial g^E}{\partial x_i} \right)_{x_i \rightarrow 0, ~~ j \neq i} 
    \label{eq_determinar_constants}
\end{equation}

Therefore, the value of the constants is given by

\begin{align}
    \left( \frac{\partial g^E}{\partial x_1} \right)_{x_1 \rightarrow 0} &= - \hat{C}_{11} - \hat{C}_{111} = w_{12} = RT \ln{\gamma_{12}^\infty}\\
    \left( \frac{\partial g^E}{\partial x_2} \right)_{x_2 \rightarrow 0} &= - \hat{C}_{111} = w_{21} = RT \ln{\gamma_{21}^\infty}
\end{align}

\noindent and the final $g^E$ expression for the binary case results in Equation \ref{eq_ge_binary_final}.

The activity coefficients can be then calculated by differentiating Equation \ref{eq_ge_binary_final} in terms of the total excess Gibbs energy $G^E = M g^E$ (where, $M$ stands for the total number of moles) with respect to the number of moles of each component:

\begin{align}
    RT \ln{\gamma_1} =& \frac{\partial G^E}{\partial n_1} = w_{21} \left( 
    \frac{2 n_1 n_2}{M^2} - \frac{2 n_1^2 n_2}{M^3}
    \right)
    +
    w_{12}\left( 
    \frac{n_2^2}{M^2} - \frac{2 n_1 n_2^2}{M^3}
    \right) \\
    RT \ln{\gamma_1} =& 2 w_{21} x_1 x_2 + x_2^2 w_{12} - 2 g^E
\end{align}

\noindent and similarly for component 2:

\begin{equation}
    RT \ln{\gamma_2} = 2 w_{12} x_2 x_1 + x_1^2 w_{21} - 2 g^E
\end{equation}

A similar procedure can be followed to determined the expressions to compute the activity coefficients of components in a ternary mixture:

\begin{align}
    g^E = x_1 x_2 (x_2 w_{12} + x_1 w_{21}) + x_1 x_3 (x_3 w_{13} + x_1 w_{31}) + x_2 x_3 (x_3 w_{23} + x_2 w_{32}) + x_1 x_2 x_3 C_{123}
    \label{eq_ge_margules_ternary}
\end{align}

with $ C_{ijk} = \frac{1}{2}(w_{ij} + w_{ji} + w_{ik} + w_{ki} + w_{jk} + w_{kj}) - w_{ijk}$. While the binary parameters can be related to the corresponding IDACs as shown for the binary case, the ternary parameter $w_{ijk}$ is often set to zero as in \cite{andersen1981valid}. The resulting expressions for the activity coefficients are then given by:

\begin{equation}
    RT \ln{\gamma_1} = 2(x_1 x_2 w_{21} + x_1 x_3 w_{31}) + x_2^2 w_{12} + x_3^2 w_{13} + x_2 x_3 c_{123} - 2 g^E
\end{equation}

\begin{equation}
    RT \ln{\gamma_2} = 2(x_2 x_3 w_{32} + x_2 x_1 w_{12}) + x_3^2 w_{23} + x_1^2 w_{21} + x_3 x_1 c_{231} - 2 g^E
\end{equation}

\begin{equation}
    RT \ln{\gamma_3} = 2(x_3 x_1 w_{13} + x_3 x_2 w_{23}) + x_1^2 w_{31} + x_2^2 w_{32} + x_1 x_2 c_{312} - 2 g^E
\end{equation}