\label{sec:results_and_discussion}

\subsection{Isothermal binary vapor-liquid equilibria}

The 903 isothermal subsets from the cleaned dataset were predicted using the GH-GNN model embedded within the extended Margules model. In total, 12,980 data points were analyzed, involving 180 distinct compounds across 531 unique binary combinations. The temperature range of the isothermal subsets spans from 127.59 K to 573.15 K.

\begin{figure}[h]
    \centering
    \includegraphics[width=1\textwidth]{figures/MAE_matrix_organic_old_Jaccard0.6.png}
    \caption{Heatmap of the mean absolute error achieved by the GH-GNN model embedded into the Margules model for different types of binary mixtures at isothermal conditions. The error is measured with respect to the vapor phase molar fraction. The number in each cell represents the number of data points included in the corresponding binary category.}
    \label{fig:isothermal_map}
\end{figure}

To compute the activity coefficients for all isothermal systems in the dataset, a total of 1062 IDACs were required. Of the 531 binary systems, 120 (22.6\%) had both IDACs observed during training. For 382 systems (72\%), the GH-GNN model had to predict at least one of the two required IDACs in other solute-solvent combination as the ones observed during training. For the remaining 29 systems (5.4\%), the model needed to predict at least one of the necessary IDACs with molecules never observed before during training. This was due to 15 compounds (out of the 180 in the isothermal VLE data) not being present in the GH-GNN training set. The fact that even for this limited VLE dataset only very few of the necessary IDACs have been measured experimentally underscores the strong motivation for developing reliable predictive methods.

\begin{figure}[h]
    \centering
    \includegraphics[width=0.85\textwidth]{figures/isothermal_vle.png}
    \caption{Isothermal vapor-liquid equilibria (VLE) diagram for two systems that are unfeasible to predict with UNIFAC-Dortmund.}
    \label{fig:isothermal_diagram}
\end{figure}

A similar analysis for UNIFAC is challenging to perform, as the specific mixtures and data used for its parametrization are not readily accessible. Furthermore, it is important to note that while the GH-GNN embedded within the Margules model had access exclusively to IDAC data, UNIFAC was parameterized using a broader range of experimental data, including IDACs, vapor-liquid equilibria (VLE), liquid-liquid equilibria (LLE), solid-liquid equilibria (SLE), azeotropic data, and calorimetric measurements \cite{gmehling2002modified}.

Figure \ref{fig:isothermal_map} presents a heatmap of the mean absolute error (MAE) achieved by the GH-GNN embedded in the Margules model when predicting the vapor-phase molar fraction of the binary vapor-liquid equilibria (VLE) under isothermal conditions. 

Overall, systems containing acids are particularly challenging for the analyzed model to predict, especially when combined with halogenated species or amides. This difficulty likely arises from the strong hydrogen-bonding and dipole-dipole interactions characteristic of these compounds, which may be inadequately captured by the information provided to the GH-GNN model. Additionally, the Margules expression might require higher-order polynomial terms to better account for these complex interactions. Similarly, systems involving alcohols and amines also exhibit poor predictive accuracy. Furthermore, systems containing water (categorized here under the ``Inorganic" class), particularly in combination with amines and other nitrogen derivatives, were associated with relatively large prediction errors.


\begin{table}[h]
\centering
  \caption{Comparison between UNIFAC-Dortmund and the GH-GNN model embedded into the Margules model for predicting binary vapor-liquid equilibria (VLE) under isothermal conditions. The comparison is according to the mean absolute error (MAE), the coefficient of determination (R$^2$) and the percentage of points predicted with an absolute error of less than or equal to 0.03. All metrics are with respect to the predicted and actual values of the molar fraction in the vapor phase.}
  \label{tbl:isothermal}
  \small
  \begin{tabular*}{1\textwidth}{@{\extracolsep{\fill}}lccc}
    \hline
    \textbf{Model} & \textbf{MAE} $\downarrow$ & \textbf{R$^2$} $\uparrow$ & \textbf{AE $\leq 0.03$} $\uparrow$ \\
    \hline
    UNIFAC-Dortmund & 0.0214 & 0.957 & 86.72\% \\
    GH-GNN + Margules & 0.0307 & 0.948 & 74.77\% \\
    \hline
  \end{tabular*}
\end{table}


The comparison between the UNIFAC-Dortmund and GH-GNN model embedded into the Margules model, presented in Table \ref{tbl:isothermal}, indicates that UNIFAC-Dortmund predicts most isothermal VLE binary systems more accurately than the GNN-based Margules hybrid model. This comparison includes all mixtures that can be feasibly predicted using the UNIFAC model (12,700 data points, 514 binary mixtures).

However, this result is not unexpected, given that, as previously mentioned, extensive VLE data were also used in the parametrization of UNIFAC-Dortmund. In contrast, the GH-GNN model, combined with the Margules framework, achieves its performance using only experimental IDAC data. Consequently, while the GNN-based predictive model performs some degree of extrapolation from the infinite regime to finite concentrations, the UNIFAC model is likely reproducing VLE data that were part of its original parametrization. Overall, this comparison shows the predictive limitations of GNNs embedded into the Margules framework, but it also establishes a baseline of the degree of accuracy that can be reached by employing scarce data at infinite dilution with GNN models and Gibbs-Duhem consistent expressions for predicting VLEs.

When comparing the flexibility of the two predictive approaches, UNIFAC-Dortmund and the GNNs embedded in the Margules model, some systems in the dataset cannot be predicted using the UNIFAC model. This limitation arises either from the unfeasibility of fragmentation or the lack of available interaction parameters. In this regard, the proposed framework combining GNNs with the Margules model offers an effective alternative.

For example, Figure \ref{fig:isothermal_diagram} shows the VLE diagram for two systems that UNIFAC-Dortmund cannot predict. In both cases, at least one of the necessary IDACs was not available experimentally, and the GH-GNN model was needed for prediction. Notably, as shown for the system ``1,1,2-trichlorotrifluoroethane / methyl acetate" in Figure \ref{fig:isothermal_diagram}, the GH-GNN model embedded within Margules successfully captures azeotropic behavior with significant precision, even without dedicated azeotropic data used during training.

\subsection{Isobaric binary vapor-liquid equilibria}

The 781 isobaric subsets contain a total of 14,118 data points, covering 162 distinct compounds in 525 different binary combinations. The pressure ranges from 1.33 to 496.63 kPa. Among these subsets, 106 systems (20.2\%) had both necessary IDACs available from experimental measurements and were therefore observed during the GH-GNN model training. Additionally, 409 systems (77.9\%) required interpolation for at least one IDAC, meaning the values were predicted based on observations of the compounds in other binary combinations. The remaining 10 systems (1.9\%) involved molecules that were entirely absent during GH-GNN model training (i.e., extrapolations).

In order to perform the isobaric VLE calculations, an iterative process is needed in which the temperature used for predicting the activity coefficients is changed in order to minimize the difference between the estimated system pressure and the actual pressure. This is represented in the optimization problem \ref{eq_optim_isobaric}, which was here solved using the SciPy package \cite{virtanen2020scipy} with Brent's algorithm, allowing up to 2,000 iterations and a tolerance of $1.48 \times 10^{-8}$.

\begin{equation}
\begin{aligned}
\underset{\substack{T}}{\text{min}} \quad & \vert P - \hat{P}(T) \vert \\
\text{s.t.} \quad & \hat{P}(T) = x_i \gamma_i(T) P_i^{sat}(T) + x_j \gamma_j(T) P_j^{sat}(T) \\
& \{\gamma_i(T), \gamma_j(T)\} \leftarrow \text{GH-GNN embedded in Margules}(T)\\
& T_{\min} \leq T \leq T_{\max}
\end{aligned}
\label{eq_optim_isobaric}
\end{equation}

\noindent where $P$ and $\hat{P}$ denote the actual and predicted system pressures, respectively, and $T_{\min}$ and $T_{\max}$ define the optimization bounds for the temperature. In this case, these bounds correspond to the minimum temperature range in which the vapor pressure correlation (Equation \ref{eq_vapor_pressure_kdb}) remains valid.

\begin{figure}[h!]
    \centering
    \includegraphics[width=1\textwidth]{figures/MAE_matrix_organic_old_Jaccard0.6_isobaric.png}
    \caption{Heatmap of the mean absolute error achieved by the GH-GNN model embedded into the Margules model for different types of binary mixtures at isobaric conditions. The error is measured with respect to the vapor phase molar fraction. The number in each cell represents the number of data points included in the corresponding binary category.}
    \label{fig:isobaric_map}
\end{figure}

\begin{table}[h]
\centering
  \caption{Comparison between UNIFAC-Dortmund and the GH-GNN model embedded into the Margules model for predicting binary vapor-liquid equilibria (VLE) under isobaric conditions. The comparison is according to the mean absolute error (MAE), the coefficient of determination (R$^2$) and the percentage of points predicted with an absolute error of less than or equal to 0.03. All metrics are with respect to the predicted and actual values of the molar fraction in the vapor phase.}
  \label{tbl:isobaric}
  \small
  \begin{tabular*}{1\textwidth}{@{\extracolsep{\fill}}lccc}
    \hline
    \textbf{Model} & \textbf{MAE} $\downarrow$ & \textbf{R$^2$} $\uparrow$ & \textbf{AE $\leq 0.03$} $\uparrow$ \\
    \hline
    UNIFAC-Dortmund & 0.0224 & 0.967 & 81.91\% \\
    GH-GNN + Margules & 0.0350 & 0.946 & 65.78\% \\
    \hline
  \end{tabular*}
\end{table}

Table \ref{tbl:isobaric} compares the performance of the UNIFAC-Dortmund model and the GH-GNN model embedded into Margules. The comparison is shown only for the systems that are feasible to predict with UNIFAC, resulting in 14,075 data points. Similar to the isothermal systems, UNIFAC-Dortmund generally outperforms the GH-GNN model combined with Margules in predicting isobaric VLEs of binary systems. As previously explained, this outcome is expected due to the difference in available experimental data during training. Nevertheless, it provides a useful baseline for assessing the accuracy achievable when predicting finite compositions from infinite dilution information using predictive methods based on GNNs.

\begin{figure}[h!]
    \centering
    \includegraphics[width=0.85\textwidth]{figures/2_PYRIDINE_1,2,3,4-TETRAHYDRONAPHTHALENE_26.66.png}
    \caption{Isobaric vapor-liquid equilibria (VLE) diagram comparing experimental data with the predictions of UNIFAC-Dortmund and the GH-GNN model embedded into the Margules model. The system is pyridine/1,2,3,4-tetrahydronaphthalene at 26.66 kPa.}
    \label{fig:isobaric_diagram_1}
\end{figure}

The heatmap in Figure \ref{fig:isobaric_map} illustrates the performance of the GNN-Margules-based model across different binary combinations. Notably, compared to the isothermal data, different binary classes are now considered. For instance, combinations such as acids/alcohols and acids/aromatics are present under isobaric conditions but absent in the isothermal dataset. In total, 16 new binary classes appear in the isobaric data that were not included in the isothermal data. However, 40 binary classes are present only at isothermal conditions. In total the intersection of binary classes between isothermal and isobaric conditions is 60. Similar to the isothermal case, many binary classes under isobaric conditions are predicted with relatively low errors.

Out of the 737 isobaric systems that are feasible to predict with UNIFAC-Dortmund, 182 are, on average, better predicted by the GNN-Margules-based model than by UNIFAC-Dortmund. To illustrate this, Figure \ref{fig:isobaric_diagram_1} presents the VLE diagram for the system ``pyridine/1,2,3,4-tetrahydronaphthalene" at 26.66 kPa. It is evident that the proposed model here predicts the VLE behavior more accurately compared to UNIFAC-Dortmund. A similar trend is observed in Figure \ref{fig:isobaric_diagram_2}, which shows the VLE diagram for the system “tetrahydrofuran/ethanol" at 25 kPa.

\begin{figure}[h]
    \centering
    \includegraphics[width=0.85\textwidth]{figures/15_TETRAHYDROFURAN_ETHANOL_25.0.png}
    \caption{Isobaric vapor-liquid equilibria (VLE) diagram comparing experimental data with the predictions of UNIFAC-Dortmund and the GH-GNN model embedded into the Margules model. The system is tetrahydrofuran/ethanol at 25 kPa.}
    \label{fig:isobaric_diagram_2}
\end{figure}

In these two isobaric VLE examples, the GH-GNN model interpolates at least one of the two required IDACs. This highlights that, although UNIFAC-Dortmund generally outperforms the proposed model, there are counterexamples where the GH-GNN combined with Margules model provides superior predictions, which may be relevant in specific practical scenarios where such mixtures might be of interest.

\subsection{Overall performance predicting binary vapor-liquid equilibria}

\begin{figure}[h!]
    \centering
    \includegraphics[width=1\textwidth]{figures/MAE_matrix_UNIFACDo_vs_GHGNNMargules.png}
    \caption{Comparison between the UNIFAC-Dortmund and the GH-GNN in Margules model based on the mean absolute error (MAE) for predicting the vapor molar fraction. All feasible systems at isothermal, isobaric and random conditions are included. The number in each cell represents the number of data points.}
    \label{fig:mae_matrix_ghgnn_vs_unifac}
\end{figure}

To gain a more detailed understanding of the binary classes where the GH-GNN model embedded into Margules outperforms UNIFAC-Dortmund, Figure \ref{fig:mae_matrix_ghgnn_vs_unifac} presents the best-performing model for each binary combination. The performance of each binary class is evaluated based on the mean absolute error (MAE) across all temperature and pressure conditions, including the isothermal, isobaric, and random subsets. Predictions are assessed with respect to the molar fraction in the vapor phase.


\begin{figure}[h]
    \centering
    \includegraphics[width=0.85\textwidth]{figures/Cumulative_MAE_classes_UNIFACDo_vs_GHGNNMargules.png}
    \caption{Cumulative percentage of binary vapor-liquid equilibria data points predicted below different thresholds of absolute error with respect to the molar fraction in the vapor phase. The performance is shown for the models UNIFAC-Dortmund and the GH-GNN embedded into Margules.}
    \label{fig:cummulative}
\end{figure}

As previously discussed for the isothermal and isobaric cases, the UNIFAC-Dortmund model generally performs better for most binary classes. However, certain binary classes are predicted more accurately by the GH-GNN model combined with Margules. For instance, the binary class ``amines/aromatics," which includes the system shown in Figure \ref{fig:isobaric_diagram_1}, tends to be better predicted using the proposed GH-GNN in Margules model. Out of the 116 distinct binary classes studied in this work, 27 binary classes are on averaged predicted with lower errors by the GH-GNN in Margules model compared to UNIFAC-Dortmund.

Figure \ref{fig:cummulative} presents the cumulative percentage of binary VLE data points predicted with absolute errors below various thresholds, based on the molar fraction in the vapor phase. Both models achieve a similar percentage of well-predicted data points for absolute errors up to around 0.01. However, UNIFAC-Dortmund consistently yields lower absolute errors overall. This is primarily attributed to differences in the experimental data available during training but also highlights the potential of GNN-based hybrid models for extrapolating finite concentration conditions from limited infinite dilution data. Including VLE data in the training of GNN-based models alongside infinite dilution data would likely lead to a significant increase in accuracy, as already observed in the case of transformers \cite{winter2023spt}.

\subsection{Ternary vapor-liquid equilibria}

In this section, we provide insights into the relative performance of the GH-GNN model embedded in Margules compared to UNIFAC-Dortmund for ternary mixtures. As shown in Table \ref{tbl:ternary_systems}, the available ternary data is much more limited. Therefore, this comparison aims to provide qualitative insights rather than an in-depth analysis, as conducted for binary mixtures.

\begin{table}[h]
\centering
  \caption{Comparison between UNIFAC-Dortmund and the GH-GNN model embedded into the Margules model for predicting ternary vapor-liquid equilibria (VLE). The comparison is according to the mean absolute error (MAE), the coefficient of determination (R$^2$) and the percentage of points predicted with an absolute error of less than or equal to 0.03. All metrics are with respect to the predicted and actual values of the molar fraction in the vapor phase of components 1 and 2.}
  \label{tbl:ternary}
  \small
  \begin{tabular*}{1\textwidth}{@{\extracolsep{\fill}}lccc}
    \hline
    \multicolumn{4}{c}{\textbf{Isothermal system}}\\
    \hline
    \textbf{Model} & \textbf{MAE} $\downarrow$ & \textbf{R$^2$} $\uparrow$ & \textbf{AE $\leq 0.03$} $\uparrow$ \\
    \hline
    UNIFAC-Dortmund & 0.0059 & 0.995 & 100\% \\
    GH-GNN + Margules & 0.0568 & 0.394 & 37.96\% \\
    \hline
    \multicolumn{4}{c}{\textbf{Isobaric systems}}\\
    \hline
    \textbf{Model} & \textbf{MAE} $\downarrow$ & \textbf{R$^2$} $\uparrow$ & \textbf{AE $\leq 0.03$} $\uparrow$ \\
    \hline
    UNIFAC-Dortmund & 0.0123 & 0.991 & 89.07\% \\
    GH-GNN + Margules & 0.0330 & 0.925 & 66.60\% \\
    \hline
  \end{tabular*}
\end{table}

Similar to the case of binary systems presented earlier, UNIFAC-Dortmund generally provides more accurate estimations of ternary VLEs (see Table \ref{tbl:ternary}). Despite this, predictions from the GH-GNN model combined with Margules also align well with the physical behavior of the mixtures. Moreover, as observed in binary mixtures, the GH-GNN embedded in Margules produces more accurate predictions for certain systems compared to UNIFAC-Dortmund.

Specifically, among the 10 ternary systems analyzed, the GH-GNN in Margules achieves a lower MAE than UNIFAC-Dortmund for the following systems: ``benzene/heptane/dimethylformamide" (MAE of 0.0296 vs. 0.0297 for UNIFAC-Dortmund), ``benzene/heptane/acetonitrile" (MAE of 0.0084 vs. 0.0114 for UNIFAC-Dortmund), and ``acetone/chloroform/benzene" (MAE of 0.0047 vs. 0.0184 for UNIFAC-Dortmund). For the remaining systems, UNIFAC-Dortmund achieves lower overall MAE. This exemplifies that, for systems where UNIFAC-Dortmund cannot provide predictions due to missing interaction parameters or unfeasibility of fragmentation into functional groups, the GH-GNN in Margules model offers a useful first-step approximation of the mixture behavior, which could be beneficial in early stages of molecular and process design.