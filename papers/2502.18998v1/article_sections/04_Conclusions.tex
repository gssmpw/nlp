\label{sec:conclusions}

Predictive models for thermophysical properties of pure compounds and their mixtures play a crucial role in chemical space exploration for product and process design. While numerous predictive methods have been developed over the decades, they remain limited in terms of both accuracy and the range of systems they can effectively describe.

In this work, we studied the performance of a GNN-based model embedded in the extended Margules framework, which assumes that the excess Gibbs energy can be approximated using a Taylor polynomial. The performance of this model was compared against the widely used UNIFAC-Dortmund model across extensive vapor-liquid equilibrium (VLE) data for binary systems under isothermal and isobaric conditions.

Additionally, we provide insights into the relative performance of these models in predicting ternary VLEs. Overall, UNIFAC-Dortmund yields more accurate VLE predictions. However, a crucial factor in this comparison is the difference in the types of experimental data used for model development. While the GNN embedded in Margules was trained exclusively on limited infinite dilution data, requiring extrapolation from infinite dilution to finite concentrations, UNIFAC-Dortmund was parameterized using a much broader dataset, including infinite dilution data, VLE, LLE, SLE, azeotropic, and caloric data \cite{constantinescu2016further,gmehling2002modified}.

It is likely that incorporating VLE data alongside infinite dilution data could significantly enhance the accuracy of GNN-based models within the Margules framework. Nonetheless, this study establishes a valuable baseline, demonstrating the predictive accuracy achievable when extrapolating VLE behavior solely from binary infinite dilution data using a relatively simple excess Gibbs energy model in combination with a highly flexible GNN-based model.

Future work should focus on a comprehensive comparison of GNN-based hybrid models trained on extensive and diverse datasets. However, a key challenge in this endeavor is the structuring and open accessibility of thermodynamic data for model development and benchmarking. Advancing community efforts in thermodynamic data curation and structuring will be essential for the continuous improvement and open-source benchmarking of predictive thermodynamic models.


