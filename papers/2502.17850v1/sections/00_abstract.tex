\begin{abstract}
\label{sec:abstract}
The vascular structure in retinal images plays a crucial role in ophthalmic diagnostics, and its accuracies are directly influenced by the quality of the retinal image. Contrast enhancement is one of the crucial steps in any segmentation algorithm – more so since the retinal images are related to medical diagnosis. Contrast enhancement is a vital step that not only intensifies the darkness of the blood vessels but also prevents minor capillaries from being disregarded during the process. This paper proposes a novel model that utilizes the linear blending of Fuzzy Contrast Enhancement (FCE) and Contrast Limited Adaptive Histogram Equalization (CLAHE) to enhance the retinal image for retinal vascular structure segmentation. The scheme is tested using the Digital Retinal Images for Vessel Extraction (DRIVE) dataset. The assertion was then evaluated through performance comparison among other methodologies which are Gray-scaling, Histogram Equalization (HE), FCE, and CLAHE.  It was evident in this paper that the combination of FCE and CLAHE methods showed major improvement. Both FCE and CLAHE methods, dominating with 88\% as better enhancement methods, proved that preprocessing through fuzzy logic is effective. 
\end{abstract}

\begin{IEEEkeywords}
Fuzzy Logic, Adaptive Contrast Enhancement, CLAHE, Retinal images, Retinal blood vessels, Medical image processing, Image Preprocessing, FCE
\end{IEEEkeywords}