\section{Introduction}
Countless ocular diseases, including but not limited to diabetic retinopathy, melanoma, age-related macular degeneration, etc., beget ophthalmic sequelae to the patients, leaving them with severe impairments within their daily lives to overcome \cite{HowRetinalImaging}. Such threats can be easily avoided by detecting the disease in the early stages.  Retinal scanning is one of many tests that doctors can do to achieve this. Clear images of the retina allow ophthalmologists to detect the patients’ eye health, and contrast enhancement plays a crucial role in aiding the ophthalmologists in making accurate identifications.

In this paper, we propose a contrast enhancement method that leverages the strengths of both fuzzy logic and CLAHE. While fuzzy logic enhances contrast effectively, it struggles with sporadic bright and dark spots that appear across the image. On the other hand, we figured that the CLAHE method excels at segmenting blood vessels correctly in such outliers but experiences difficulties creating crisp distinctions between tissues and capillaries during segmentation due to its low contrast. Therefore, we sought a novel scheme that can utilize the advantages of both methods through linear blending combined with post-processing of the image.

Methods of retinal imaging may vary, but the basics – lights pass through from the pupil to the retina, and the machine retrieves the collected image – remain unchanged. However, such method results in the variance of luminosity throughout the image and noises within the resultant image, which is the main challenge of contrast enhancement in retinal imaging.
According to an article by Naveed et al. on Diagnostics in 2021, poor image contrast and quality result from the complex image-collecting process, and unwanted variables corrupt the image, including, but not limited to, additive noise (Gaussian noise), multiplicative noise (speckle), and Shot noise (Poisson noise) \cite{naveedAutomatedEyeDiagnosis2021}.

Medical images are mostly in grayscale so that doctors can get a clear and better perception of images. This is because gray scaling can reduce the 3-dimensional pixel value consisting of red, blue, and green (RGB) into a 1-dimensional value \cite{youssefFuzzybasedImageSegmentation2021}. However, research proved that the green channel (G-channel) of RGB provides the best contrast of the blood vessels \cite{GreenChannelOverview}. Moreover, studies on the sensitivity functions of the human eye in color scales have proven that both the low-contrast signal detectability improved significantly while not affecting eye fatigue with a yellow-scale display over the grayscale display \cite{oguraComparisonGrayscaleColorscale2017}.

Innovative attempts have been made to improve the accuracy of retinal vessel segmentation. Some used fuzzy logic for edge detection in Diabetic Retinopathy, whose image shows lots of unwanted noise \cite{kamilEdgeDetectionDiabetic2014}. Recent methods rely much on machine learning (\!\!\cite{girdharRetinalImageSegmentation2022, barrosMachineLearningApplied2020}), convolutional neural networks (\!\!\cite{hoqueDeepLearningRetinal2021, khanalDynamicDeepNetworks2020}), and deep learning \cite{shiDeepLearningSystem2022}. Even though they perform very well, such methods can get very complex. We are attempting to match their performance while being less complex. Various methods and their accuracy can be found in \cite{PapersCodeDRIVE}

The paper is organized as follows: Section \ref{sec:methodologies} introduces the proposed model, Section \ref{sec:implementation} presents the implementation of the proposed method, and Section \ref{sec:results} presents results and conclusions.