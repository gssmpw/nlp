\section{Implementation}
\label{sec:implementation}
The overall software is written in Python with the support of multiple libraries.

\subsection{Version 1 (FCE)}
Images are loaded and altered through OpenCV’s functions. Fuzzy rules are implemented through separate Python functions, as shown in Fig. \ref{fig:rules}.

\begin{figure}[h]
    \centering
    \includegraphics[width=0.7\linewidth]{figures/png/rules.png}
    \caption{Implementation of rules for membership functions.}
    \label{fig:rules}
\end{figure}

The plotting of these membership functions for various values is done through matplotlib, as shown in Fig. \ref{fig:visusalization}.

\begin{figure}[h]
    \centering
    \includegraphics[width=\linewidth]{figures/png/visualize.png}
    \caption{Visualization of fuzzy membership functions.}
    \label{fig:visusalization}
\end{figure}

Fuzzy Contrast Enhance is done through fuzzification, inference, and defuzzification, which was done through sets of functions as shown in Fig. \ref{fig:FCE}. The input RGB image, denoted as \texttt{img\_rgb}, is converted to the HLS color space. Amongst the HLS color space, the L channel is extracted, and its variance-reduced mean value is calculated and stored as \texttt{M}. The fuzzy transform is performed onto the L channel using the saved \texttt{M} value, and min-max scaling has been performed.

\begin{figure}[h]
    \centering
    \includegraphics[width=\linewidth]{figures/png/FCE.png}
    \caption{Implementation of FCE.}
    \label{fig:FCE}
\end{figure}

\subsection{Version 2 (FCE + CLAHE)}
The functionality of CLAHE is implemented using OpenCV’s pre-existing function – \texttt{createCLAHE()}.

\begin{figure}[h]
    \centering
    \includegraphics[width=\linewidth]{figures/png/CLAHE.png}
    \caption{Implementation of CLAHE.}
    \label{fig:CLAHE}
\end{figure}

Afterward, linear blending was achieved through OpenCV’s \texttt{addWeighted()} function as shown in Fig. \ref{fig:linear_blending}.

\begin{figure}[h]
    \centering
    \includegraphics[width=0.9\linewidth]{figures/png/linear_blending.png}
    \caption{Implementation of linear blending.}
    \label{fig:linear_blending}
\end{figure}

\subsection{Version 3 (FCE + CLAHE with Hue Post-processing)}
Post-processing of hue from green to yellow was achieved by the \texttt{adjust\_hue()} function of the Image Python Library as shown in Fig. \ref{fig:hue_adjusting}. As previously detailed, the incorporation of a yellowish hue has proven to be beneficial in aiding doctors in recognizing blood vessels and abnormalities in medical images.

\begin{figure}[h]
    \centering
    \includegraphics[width=0.9\linewidth]{figures/png/hue_adjusting.png}
    \caption{Implementation of hue adjusting.}
    \label{fig:hue_adjusting}
\end{figure}

% \begin{figure*}[h]
%     \centering
%     \caption{Block Diagram of Proposed Model}
%     \label{code:rules}
% \end{figure*}
