\section{Results and Conclusion}
\label{sec:results}

Figure \ref{fig:res_original} presents the original input image, while Figure \ref{fig:res_segmentation} shows the manual segmentation by the experts. The grayscale transformation of the original image is illustrated in Figure \ref{fig:res_grayscale}. The output of Version 1 (FCE, mentioned in Sec. \ref{subsec:version_1}) is displayed in Figure \ref{fig:res_fuzzy}, whereas Figures \ref{fig:res_HE} and \ref{fig:res_CLAHE} depict the results of Histogram Equalization (HE) and CLAHE, two widely used contrast enhancement techniques. Version 2 (FCE + CLAHE, demonstrated in Sec. \ref{subsec:version_2}), shown in Figure \ref{fig:res_fuzzy_clahe}, achieved improved accuracy in enhancing vessels around the macular area. Finally, Version 3 (FCE + CLAHE with Hue Post-processing, explained in Sec. \ref{subsec:version_3}), illustrated in Figure \ref{fig:res_proposed}, further improved contrast visibility by incorporating a yellow image filter.

\begin{table}[ht!]
\centering
\caption{\textbf{Super Resolution Performance Results.} Our proposed WGAN EEG Spatial Upsampling method significantly outperforms a baseline of Bicubic Interpolation commonly used in EEG upsampling pipelines.}
\label{tab:results}
\resizebox{0.8\linewidth}{!}{%
\begin{tabular}{@{}cccccc@{}}
\toprule
\multirow{2}{*}{\textbf{Dataset}} & \multirow{2}{*}{\textbf{Scale}} & \multicolumn{2}{c}{\textbf{Bicubic}} & \multicolumn{2}{c}{\textbf{WGAN}} \\ \cmidrule(l){3-6} 
                      &   & \textbf{MSE} & \textbf{MAE} & \textbf{MSE}    & \textbf{MAE}   \\
\toprule
\multirow{2}{*}{Val}  & 2 & 3.71E7       & 3.89E3       & \textbf{2.01E3} & \textbf{24.38} \\
                      & 4 & 7.23E7       & 6.42E3       & \textbf{8.53E3} & \textbf{63.83} \\
\midrule
\multirow{2}{*}{Test} & 2 & 3.75E7       & 3.91E3       & \textbf{2.06E3} & \textbf{24.66} \\
                      & 4 & 7.30E7       & 6.45E3       & \textbf{8.68E3} & \textbf{64.39} \\
\bottomrule
\end{tabular}%
}
\end{table}

The output shows that most of the contrast enhancement methods showed visual enhancements compared to the original image. 

A survey was conducted using ten individuals to determine which image has the most accurate vessel segmentation. The human survey was done since both the contrast enhancement and the hue modification heavily benefit from human cognition more than that of segmentation algorithms – which utilized their own methods for preprocessing and noisy filtering, tuned for their own needs. Each respondent was provided with manual segmentation of the data by an ophthalmologist and told to choose the best output by comparing each output with its manual segmentation data.  Fig. \ref{fig:graph} and Fig. \ref{fig:pie} show the survey results of different contrast enhancement methods.

Based on the survey results, it has been shown that gray scaling alone is not sufficient for enhancing the visual contrast of retinal images. People tended to prefer CLAHE or gray scaling over Fuzzy Methods for images that had an outlier within the image (bright spots, noises, unhealthy morphology, etc.), whereas pure FCE was preferred over the combination of FCE and CLAHE when the luminosity of the image was evenly distributed.

\begin{figure}[h]
    \centering
    \includegraphics[width=0.8\linewidth]{figures/png/graph.png}
    \caption{A bar chart displaying how each test taker selected different enhancement methods across multiple images.}
    \label{fig:graph}
\end{figure}


\begin{figure}[h]
    \centering
    \includegraphics[width=0.8\linewidth]{figures/png/pie.png}
    \caption{A pie chart representing the total number of selections made for each enhancement method across all test takers and images.}
    \label{fig:pie}
\end{figure}

Nonetheless, it was evident that the combination of the FCE and CLAHE method showed major improvement in general, including improvement over the FCE method. Furthermore, the performance review survey indicates that both the FCE and FCE + CLAHE methods outperformed others, achieving a remarkable 88\% as the preferred enhancement methods. This underscores the effectiveness of preprocessing through fuzzy logic.