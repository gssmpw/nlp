\section{Introduction} \label{sec:intro}
Developing embodied agents capable of solving complex tasks in real world remains a significant challenge \cite{durante2024agent}. Recent advancements in foundation models—including Large Language Models (LLMs) \cite{GPT3,achiam2023gpt,touvron2023llama,yang2024qwen2} and Multimodal Large Language Models (MLLMs) \cite{GPT-4o,reid2024gemini,liu2024llavanext,wang2024qwen2,chen2023internvl,internvl2.5}—have unlocked unprecedented potential toward this goal. These models, trained on extensive internet-scale datasets, demonstrate exceptional proficiency in understanding human knowledge and performing human-like reasoning. Based on these capabilities, researchers can now design intelligent agents that use off-the-shelf foundation models to solve complex tasks through interaction with environments \cite{huang2022language,huang2022inner,huang2023voxposer,ahn2022can,llm-planner,singh2023progprompt,liang2023code,EscapeBench2024}.


\begin{figure*}[th!]
\begin{center}
\vspace{-5pt}
% \includegraphics[width=0.94\linewidth]{pics/embodied_overview_new.pdf}
\includegraphics[width=0.97\linewidth, trim=0 0 0 15, clip]{pics/embodied_overview_new.pdf}
\end{center}
\vspace{-1.2em}
\caption{Overview of \name. Two key features of our benchmark: various action levels and capability-oriented evaluation.}\label{fig:overview}
\vspace{-1.5em}
\end{figure*}


Given the multitude of proposed algorithms, there is a pressing need for standardized and automated evaluation frameworks to enable comprehensive assessment and comparison.  To address this need, several initiatives have been exploring LLM-based embodied agent evaluation \cite{liu2023agentbench,choi2024lota,li2024embodied}. While these efforts significantly contribute to understanding LLM-based agent design, the evaluation of MLLM embodied agents remains underexplored, posing a challenge for creating more versatile agents. VisualAgentBench \cite{liu2024visualagentbench} represents the first benchmark for evaluating MLLM agents, covering embodied tasks such as household and Minecraft. However, its limited scope, focusing exclusively on high-level planning, leaves critical questions unanswered, such as \emph{the role of vision in embodied tasks and the performance of MLLM agents in low-level tasks like navigation and manipulation}. 



To address these questions, we introduce \name, a comprehensive benchmark comprising 1,128 testing instances across four environments. \name is designed with two key features that set it apart from existing benchmarks:
\textbf{1. Diverse tasks with hierarchical action levels.} Among the four environments, EB-ALFRED and EB-Habitat focus on high-level task decomposition and planning (e.g., ``put a book on the desk"), while EB-Navigation and EB-Manipulation demand planning with low-level actions (e.g., translational/rotational control) and require precise perception and spatial reasoning. 
\textbf{2. Capability-oriented evaluation.} Unlike previous benchmarks that primarily emphasize overall accuracy \cite{liu2023agentbench,choi2024lota,liu2024visualagentbench} or module-specific performance \cite{li2024embodied}, \name introduces a fine-grained evaluation framework that assesses six critical capabilities of embodied agents, including basic task solving, commonsense reasoning, complex instruction understanding, spatial awareness, visual perception, and long-horizon planning. 


To facilitate the evaluation of MLLMs as embodied agents, we design a unified agent framework that integrates ego-centric visual perception, few-shot in-context examples, interaction history, and environment feedback for decision-making. This powerful framework can unlock the full potential of current off-the-shelf MLLMs and tackle both high-level and low-level tasks effectively. Based on \name and our agent pipeline, we evaluate 19 leading closed-source MLLMs (e.g., GPT-4o, Gemini, Claude-3.5, and Qwen-VL-Max) and 7B–90B open-source models (e.g., Llama-3.2 Vision \cite{llama3.2}, InternVL 2.5 series \cite{internvl2.5,wang2024enhancing}, Qwen2-VL \cite{wang2024qwen2}, and Qwen2.5-VL \cite{bai2025qwen25vl}). Our evaluation yields three key findings: (1) While MLLMs excel at high-level tasks, they struggle with low-level manipulation. (2) Long-horizon planning emerges as the most challenging subset. (3) Vision input is crucial for low-level tasks, with performance degrading by 40\%–70\% when removed, whereas its impact on high-level tasks is minimal. Additionally, our ablation studies provide practical insights into MLLM agent design, particularly regarding image resolution, multi-step image input, and visual in-context learning.


Our contributions are threefold: (1) proposing a comprehensive benchmark suite for evaluating MLLM-based embodied agents with different action levels and fine-grained capability-oriented subsets, (2) the development of an efficient MLLM agent framework, (3) conducting extensive evaluations and ablation studies of leading MLLMs, providing valuable insights for vision-driven agent design. 





















% \begin{wraptable}{r}{0.65\textwidth}
% \centering
% \caption{The comparison of resource requirements between Eurus-2-7B-PRIME and Qwen2.5-Math-7B-Instruct.}
% \label{tab:comparision}
% \resizebox{0.65\textwidth}{!}{
% \begin{tabular}{l >{\columncolor[HTML]{D7E8E8}}l l}
% \toprule
% \textbf{Model} & \textbf{Eurus-2-7B-PRIME} & \textbf{Qwen2.5-Math-7B-Instruct} \\ \midrule
% Base Model     & Qwen2.5-Math-7B           & Qwen2.5-Math-7B                  \\
% SFT Data       & 230K (open-source)        & 2.5M (open-source and in-house)  \\
% RM Data        & 0                         & 618K (in-house)                 \\
% RM             & Eurus-2-7B-SFT            & Qwen2.5-Math-RM (72B)           \\
% RL Data        & 150K queries $\times$ 4 samples & 66K queries $\times$ 32 samples \\ \bottomrule
% \end{tabular}
% }
% \end{wraptable}

\begin{wraptable}{r}{0.65\textwidth}
\centering
\caption{The comparison of resource requirements between Eurus-2-7B-PRIME and Qwen2.5-Math-7B-Instruct.}
\label{tab:comparision}
% \resizebox{0.65\textwidth}{!}{
\resizebox{\linewidth}{!}{
\begin{tabular}{l >{\columncolor[HTML]{D7E8E8}}l l}
\toprule
\textbf{Model} & \textbf{Eurus-2-7B-PRIME} & \textbf{Qwen2.5-Math-7B-Instruct} \\ \midrule
Base Model     & Qwen2.5-Math-7B           & Qwen2.5-Math-7B                  \\
SFT Data       & 230K (open-source)        & 2.5M (open-source \& in-house)  \\
RM Data        & 0                         & 618K (in-house)                 \\
RM             & Eurus-2-7B-SFT            & Qwen2.5-Math-RM (72B)           \\
RL Data        & 150K queries $\times$ 4 samples & 66K queries $\times$ 32 samples \\ \bottomrule
\end{tabular}
}
\end{wraptable}





% 字体标橙色
% \begin{wraptable}{r}{0.65\textwidth}  % r表示表格在右侧,0.5\textwidth表示表格宽度为文本宽度的50%
% \centering
% \caption{The comparison of resource requirements between Eurus-2-7B-PRIME and Qwen2.5-Math-7B-Instruct.}
% \label{tab:comparision}
% \resizebox{0.65\textwidth}{!}{
% \begin{tabular}{lll}
% \toprule
% \textbf{Model} & {\color[HTML]{F8A102}\textbf{Eurus-2-7B-PRIME}}                             & \textbf{Qwen2.5-Math-7B-Instruct}            \\ \midrule
% Base Model     & {\color[HTML]{F8A102}Qwen2.5-Math-7B}                                       & Qwen2.5-Math-7B                              \\
% SFT Data       & {\color[HTML]{F8A102}\textbf{230K (open-source)}}                           & 2.5M (open-source and in-house)              \\
% RM Data        & {\color[HTML]{F8A102}\textbf{0}}                                            & 618K (in-house)                              \\
% RM             & {\color[HTML]{F8A102}\textbf{Eurus-2-7B-SFT}}                               & Qwen2.5-Math-RM (72B)                        \\
% RL Data        & {\color[HTML]{F8A102}\textbf{150K queries $\times$ 4 samples}} & 66K queries $\times$ 32 samples \\ \bottomrule
% \end{tabular}

% }
% \end{wraptable}




% \begin{table}[]
% \centering
% \caption{The comparison of resource requirements between Eurus-2-7B-PRIME and Qwen2.5-Math-7B-Instruct.\hanbin{Embed into text}}
% \label{tab:comparision}
% \resizebox{0.8\textwidth}{!}{
% \begin{tabular}{lll}
% \midrule
% \textbf{Model} & \textbf{Eurus-2-7B-PRIME}                             & \textbf{Qwen2.5-Math-7B-Instruct}            \\ \midrule
% Base Model     & Qwen2.5-Math-7B                                       & Qwen2.5-Math-7B                              \\
% SFT Data       & \textbf{230K (open-source)}                           & 2.5M (open-source and in-house)              \\
% RM Data        & \textbf{0}                                            & 618K (in-house)                              \\
% RM             & \textbf{Eurus-2-7B-SFT}                               & Qwen2.5-Math-RM (72B)                        \\
% RL Data        & \textbf{150K queries $\times$ 4 samples} & 66K queries $\times$ 32 samples \\ \midrule
% \end{tabular}
% }
% \end{table}










