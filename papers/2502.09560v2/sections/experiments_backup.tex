\section{Experiments}

% \heng{add analysis on remaining challenges, provide some insights and ideas for future directions/solutions}
\begin{table*}[t]
\vspace{-10pt}
    \centering\small
    \caption{
    Task success rates on 6 subsets of EB-ALFRED and EB-Habitat, with the best proprietary model in bold and open-source model underlines per column. Success rates for subsets are integers since each subset consists of 50 test instances.
    }
    % \vspace{0.5em}
    \renewcommand{\arraystretch}{1.0}
    \setlength\tabcolsep{2pt}
    \setlength\extrarowheight{2pt}
    \resizebox{1\linewidth}{!}{

    % \begin{tabular}{p{3.3cm} c c c c c c c @{\hskip 15pt} c c c c c c c}
    \begin{tabular}{
    >{\centering\arraybackslash}p{3.3cm} 
        >{\centering\arraybackslash}p{1.17cm} 
        >{\centering\arraybackslash}p{1.17cm} 
        >{\centering\arraybackslash}p{1.17cm} 
        >{\centering\arraybackslash}p{1.17cm} 
        >{\centering\arraybackslash}p{1.17cm} 
        >{\centering\arraybackslash}p{1.17cm} 
        >{\centering\arraybackslash}p{1.17cm} 
        @{\hskip 10pt} 
        >{\centering\arraybackslash}p{1.17cm} 
        >{\centering\arraybackslash}p{1.17cm} 
        >{\centering\arraybackslash}p{1.17cm} 
        >{\centering\arraybackslash}p{1.17cm} 
        >{\centering\arraybackslash}p{1.17cm} 
        >{\centering\arraybackslash}p{1.17cm} 
        >{\centering\arraybackslash}p{1.17cm} 
 }
    
        \toprule
        
         \multirow{2}{*}{\textbf{\small Model}} & \multicolumn{7}{c}{\cellml \bf EB-ALFRED} & \multicolumn{7}{c}{\cellmr \bf EB-Habitat} \\

        \cmidrule(lr){2-8} \cmidrule(lr){9-15}
        
        ~ & \textbf{Avg} & \textbf{Base} & \textbf{Common} & \textbf{Complex} & \textbf{Visual} & \textbf{Spatial} & \textbf{Long} 
       & \textbf{Avg} & \textbf{Base} & \textbf{Common} & \textbf{Complex} & \textbf{Visual} & \textbf{Spatial} & \textbf{Long}  \\

        % ~ & Base & CC & Complex & Visual & Exp & Avg & Base & CC & Complex & Visual & Spatial & Avg \\
        \addlinespace[2pt]
        \midrule
        \addlinespace[2pt]
        \multicolumn{15}{c}{ \textit{Proprietary MLLMs} }  \\ \midrule
        {\fontsize{8}{10}\selectfont GPT-4o} & 56.3 & 64 & 54 & 68 & 46 & 52 & 54 & 59.0  &  86 & 44 & 56 &68 &36 &  \textbf{64} \\
        {\fontsize{8}{10}\selectfont GPT-4o-mini} & 24.0 & 34 & 28 & 36 & 24 & 22 & 0 & 32.7 & 74 & 22 & 32  & 22   & 32 & 14\\
        {\fontsize{8}{10}\selectfont Claude-3.5-Sonnet} & \textbf{64.0} & \textbf{72}  & \textbf{66}  &  \textbf{76} & \textbf{60}  & \textbf{58} & 52 & \textbf{68.0} & \textbf{96} & \textbf{68} &  \textbf{78} & 70 & \textbf{38} & 58 \\
        {\fontsize{8}{10}\selectfont Gemini-1.5-Pro} & 62.3 & 70  & 64  & 72 &  58  &  52 & \textbf{58} &  56.3 & 92 & 52 & 48 & 56 &   38 &  52   \\
        {\fontsize{8}{10}\selectfont Gemini-2.0-flash} & 52.3 & 62 & 48 & 54 & 46 & 46  &  \textbf{58} & 42.3 & 82 & 38 & 38 &  36 & 34 & 26 \\
        {\fontsize{8}{10}\selectfont Gemini-1.5-flash} & 39.3 & 44  & 40 & 56 & 42 & 26 & 28 & 39.3 & 76 & 32 & 48 & 36 & 32 & 12 \\

        %  add new
        {\fontsize{8}{10}\selectfont Qwen-VL-Max } & 41.3 &  44 & 48 & 44 & 42 & 38 &  32 & 45.3 & 74 & 40  & 50 & 42 & 30 & 36 \\
        \midrule
        
        {\fontsize{8}{10}\selectfont GPT-4o (Lang)} & 58.0 & 62 & 64 &  70  & 52 & 46 & 54 &  56.0  &  82 & 52 & 58 &  \textbf{74} & 34 & 36 \\
        {\fontsize{8}{10}\selectfont GPT-4o-mini (Lang)} & 31.3 & 42 & 36  & 46 & 30 & 20  & 14  &  36.7  &  82 & 30 & 34 & 30 & 30 & 14 \\
        
        \addlinespace[2pt]
        \midrule
        \addlinespace[2pt]
        
        \multicolumn{15}{c}{ \textit{Open-Source MLLMs} }   \\ \midrule
         {\fontsize{8}{10}\selectfont Llama-3.2-90B-Vision-Ins} & 32.0 & 38 &34 & \underline{44} & 28 &  32 & 16 & 40.3  &   \underline{94} & 24 & 50 & 32 & 28 & 14 \\
         {\fontsize{8}{10}\selectfont Llama-3.2-11B-Vision-Ins} &  13.7 & 24  & 8 &  16 & 22 & 6 & 6 & 25.0 & 70 & 16 & 28 & 10  &  20 & 6 \\
       {\fontsize{8}{10}\selectfont InternVL2\_5-78B} & 37.7 & 38  & 34 & 42  & 34  & 36 & \underline{42} & 49.0 & 80 & \underline{42} &  \underline{56} & 58 & 30 & 28 \\
          {\fontsize{8}{10}\selectfont InternVL2\_5-38B} & 23.3 & 36 & 30 & 36 & 22 & 14 & 26 & 38.3 &  60 & 28 & 48 & 34 & \underline{32} & 28  \\
        {\fontsize{8}{10}\selectfont InternVL2\_5-8B} & 2.0 & 4 & 6 & 2 & 0 & 0 & 0 & 11.3  &  36 & 4 & 0 & 10 & 16 & 2  \\
     
% add new 

{\fontsize{8}{10}\selectfont InternVL2\_5-78B-MPO } & \underline{40.0} & \underline{48} & 36 & 42 & \underline{40} & \underline{40} & 34 & \underline{52.3} & 88 & 28 & 50 &  \underline{70} & 34 &  \underline{44} \\
{\fontsize{8}{10}\selectfont InternVL2\_5-38B-MPO } & 25.7 & 30 & 20 & 20 & 28 &  32 & 24 &  47.0 & 92 & 24 & \underline{56} & 42 &  \underline{32} & 36  \\
{\fontsize{8}{10}\selectfont InternVL2\_5-8B-MPO } & 7.7 & 12 & 6 & 14 & 6 & 6 & 2 &  22.0 & 60 & 10 & 20 & 16 & 20 &  6 \\

  {\fontsize{8}{10}\selectfont Qwen2-VL-72B-Ins} & 33.7 & 40 & 30 &  40 & 30 & 32 & 30 & 35.7 & 70 & 30 &  36 &  32 & 28 & 18 \\
       {\fontsize{8}{10}\selectfont Qwen2-VL-7B-Ins} & 1.7 & 6 &  0 & 2 & 0 & 0 & 2 & 18.3 &  48 & 6 & 16 & 20 & 18 & 2 \\


{\fontsize{8}{10}\selectfont Qwen2.5-VL-72B-Ins } & 39.7 & 50 & \underline{42} & 42 & 36 & 34 & 34& 37.7 & 74 & 28 & 42 & 40 & 24 & 18\\
{\fontsize{8}{10}\selectfont  Qwen2.5-VL-7B-Ins} &  4.7 &  10 & 8 & 6 & 2 &  0 &  2 & 14.3  & 32 & 2 & 26 & 10 & 14 & 2   \\
      
        \bottomrule
        
    \end{tabular}
    }\label{tb:high_level_table}
    
    \vspace{-1em}
\end{table*}




% \begin{table*}[t]
% \vspace{-10pt}
%     \centering\small
%     \caption{
%     Task success rates on 6 subsets of EB-ALFRED and EB-Habitat, with the best proprietary model in bold and open-source model underlines per column. Success rates for subsets are integers since each subset consists of 50 test instances.
%     }
%     % \vspace{0.5em}
%     \renewcommand{\arraystretch}{1.0}
%     \setlength\tabcolsep{2pt}
%     \setlength\extrarowheight{2pt}
%     \resizebox{1\linewidth}{!}{

%     % \begin{tabular}{p{3.3cm} c c c c c c c @{\hskip 15pt} c c c c c c c}
%     \begin{tabular}{
%     >{\centering\arraybackslash}p{3.3cm} 
%         >{\centering\arraybackslash}p{1.17cm} 
%         >{\centering\arraybackslash}p{1.17cm} 
%         >{\centering\arraybackslash}p{1.17cm} 
%         >{\centering\arraybackslash}p{1.17cm} 
%         >{\centering\arraybackslash}p{1.17cm} 
%         >{\centering\arraybackslash}p{1.17cm} 
%         >{\centering\arraybackslash}p{1.17cm} 
%         @{\hskip 10pt} 
%         >{\centering\arraybackslash}p{1.17cm} 
%         >{\centering\arraybackslash}p{1.17cm} 
%         >{\centering\arraybackslash}p{1.17cm} 
%         >{\centering\arraybackslash}p{1.17cm} 
%         >{\centering\arraybackslash}p{1.17cm} 
%         >{\centering\arraybackslash}p{1.17cm} 
%         >{\centering\arraybackslash}p{1.17cm} 
%  }
    
%         \toprule
        
%          \multirow{2}{*}{\textbf{\small Model}} & \multicolumn{7}{c}{\cellml \bf EB-ALFRED} & \multicolumn{7}{c}{\cellmr \bf EB-Habitat} \\

%         \cmidrule(lr){2-8} \cmidrule(lr){9-15}
        
%         ~ & \textbf{Avg} & \textbf{Base} & \textbf{Common} & \textbf{Complex} & \textbf{Visual} & \textbf{Spatial} & \textbf{Long} 
%        & \textbf{Avg} & \textbf{Base} & \textbf{Common} & \textbf{Complex} & \textbf{Visual} & \textbf{Spatial} & \textbf{Long}  \\

%         % ~ & Base & CC & Complex & Visual & Exp & Avg & Base & CC & Complex & Visual & Spatial & Avg \\
%         \addlinespace[2pt]
%         \midrule
%         \addlinespace[2pt]
%         \multicolumn{15}{c}{ \textit{Proprietary MLLMs} }  \\ \midrule
%         {\fontsize{8}{10}\selectfont GPT-4o} & 56.3 & 64 & 54 & 68 & 46 & 52 & 54 & 59.0  &  86 & 44 & 56 &68 &36 &  \textbf{64} \\
%         {\fontsize{8}{10}\selectfont GPT-4o-mini} & 24.0 & 34 & 28 & 36 & 24 & 22 & 0 & 32.7 & 74 & 22 & 32  & 22   & 32 & 14\\
%         {\fontsize{8}{10}\selectfont Claude-3.5-Sonnet} & \textbf{64.0} & \textbf{72}  & \textbf{66}  &  \textbf{76} & \textbf{60}  & \textbf{58} & 52 & \textbf{68.0} & \textbf{96} & \textbf{68} &  \textbf{78} & 70 & \textbf{38} & 58 \\
%         {\fontsize{8}{10}\selectfont Gemini-1.5-Pro} & 62.3 & 70  & 64  & 72 &  58  &  52 & \textbf{58} &  56.3 & 92 & 52 & 48 & 56 &   38 &  52   \\
%         {\fontsize{8}{10}\selectfont Gemini-2.0-flash} & 52.3 & 62 & 48 & 54 & 46 & 46  &  \textbf{58} & 42.3 & 82 & 38 & 38 &  36 & 34 & 26 \\
%         {\fontsize{8}{10}\selectfont Gemini-1.5-flash} & 39.3 & 44  & 40 & 56 & 42 & 26 & 28 & 39.3 & 76 & 32 & 48 & 36 & 32 & 12 \\

%         %  add new
%         {\fontsize{8}{10}\selectfont Qwen-VL-Max } \\
%     {\fontsize{8}{10}\selectfont Qwen-VL-Plus } \\
%         \midrule
        
%          % {\fontsize{8}{10}\selectfont Claude-3.5-Sonnet (Lang)} & 71.3 & 76 & 70 & 70 & 66 & 68 & 78  & 66.0 &  98 & 66 & 66 & 80  & 38 & 48 \\
%         {\fontsize{8}{10}\selectfont GPT-4o (Lang)} & 58.0 & 62 & 64 &  70  & 52 & 46 & 54 &  56.0  &  82 & 52 & 58 &  \textbf{74} & 34 & 36 \\
%         {\fontsize{8}{10}\selectfont GPT-4o-mini (Lang)} & 31.3 & 42 & 36  & 46 & 30 & 20  & 14  &  36.7  &  82 & 30 & 34 & 30 & 30 & 14 \\
        
%         \addlinespace[2pt]
%         \midrule
%         \addlinespace[2pt]
        
%         \multicolumn{15}{c}{ \textit{Open-Source MLLMs} }   \\ \midrule
%          {\fontsize{8}{10}\selectfont Llama-3.2-90B-Vision-Ins} & 32.0 & 38 & \underline{34} & \underline{44} & 28 &  32 & 16 & 40.3  &   \underline{94} & 24 & 50 & 32 & 28 & 14 \\
%          {\fontsize{8}{10}\selectfont Llama-3.2-11B-Vision-Ins} &  13.7 & 24  & 8 &  16 & 22 & 6 & 6 & 25.0 & 70 & 16 & 28 & 10  &  20 & 6 \\
%        {\fontsize{8}{10}\selectfont InternVL2\_5-78B} & \underline{37.7} & 38  & \underline{34} & 42  & \underline{34}  & \underline{36} & \underline{42} & \underline{49.0} & 80 & \underline{42} &  \underline{56} & \underline{58} & 30 & \underline{28} \\
%           {\fontsize{8}{10}\selectfont InternVL2\_5-38B} & 23.3 & 36 & 30 & 36 & 22 & 14 & 26 & 38.3 &  60 & 28 & 48 & 34 & \underline{32} & \underline{28}  \\
%         {\fontsize{8}{10}\selectfont InternVL2\_5-8B} & 2.0 & 4 & 6 & 2 & 0 & 0 & 0 & 11.3  &  36 & 4 & 0 & 10 & 16 & 2  \\
%        {\fontsize{8}{10}\selectfont Qwen2-VL-72B-Ins} & 33.7 & \underline{40} & 30 &  40 & 30 & 32 & 30 & 35.7 & 70 & 30 &  36 &  32 & 28 & 18 \\
%        {\fontsize{8}{10}\selectfont Qwen2-VL-7B-Ins} & 1.7 & 6 &  0 & 2 & 0 & 0 & 2 & 18.3 &  48 & 6 & 16 & 20 & 18 & 2 \\
% % add new 

% {\fontsize{8}{10}\selectfont InternVL2\_5-78B-MPO } & 40.0 & 48 & 36 & 42 & 40 & 40 & 34 & 52.3 &  \\
% {\fontsize{8}{10}\selectfont InternVL2\_5-38B-MPO }  \\
% {\fontsize{8}{10}\selectfont InternVL2\_5-8B-MPO }  \\
% {\fontsize{8}{10}\selectfont Qwen2.5-VL-72B-Instruct } \\
% {\fontsize{8}{10}\selectfont  Qwen2.5-VL-7B-Instruct} \\
      
%         \bottomrule
        
%     \end{tabular}
%     }\label{tb:high_level_table}
    
%     \vspace{-1em}
% \end{table*}














% \begin{table*}[!t]
% \centering\small
% \caption{Ablation Results of GPT-4o and Claude on ALFRED.}
% \vspace{0.5em}
% \begin{tabular}{lllllllll}
% \toprule
%                   & Normal & Resolution 300 & Resolution 700 & Picture history & Detection & w/o Feedback & 5 Example & 0 Example\\
% \midrule
% GPT-4o            & 64     & 62             & 66             & 62              & 66        & 54         & 46        & 40           \\
% \midrule
% Claude-3.5-Sonnet & 72     & 72             & 74             & 74              & 78        & 64         & 52        & 46           \\
% \bottomrule
% \end{tabular}
% \end{table*}
\begin{table*}[t]
\vspace{-10pt}
    \centering\small
    \caption{
    Task success rates on 5 subsets of EB-Navigation and EB-Manipulation, with the best proprietary model in bold and open-source model underlines per column.
    }
    % \vspace{0.5em}
    \renewcommand{\arraystretch}{1.1}
    \setlength\tabcolsep{2pt}
    \setlength\extrarowheight{2pt}
    \resizebox{0.95\linewidth}{!}{

    % \begin{tabular}{
        % p{3.3cm} p{0.15\textwidth} c c c c c @{\hskip 14pt} c c c c c c
        \begin{tabular}{
    >{\centering\arraybackslash}p{3.3cm} 
        >{\centering\arraybackslash}p{1.17cm} 
        >{\centering\arraybackslash}p{1.17cm} 
        >{\centering\arraybackslash}p{1.17cm} 
        >{\centering\arraybackslash}p{1.17cm} 
        >{\centering\arraybackslash}p{1.17cm} 
        >{\centering\arraybackslash}p{1.17cm} 
        >{\centering\arraybackslash}p{1.17cm} 
        @{\hskip 10pt} 
        >{\centering\arraybackslash}p{1.17cm} 
        >{\centering\arraybackslash}p{1.17cm} 
        >{\centering\arraybackslash}p{1.17cm} 
        >{\centering\arraybackslash}p{1.17cm} 
        >{\centering\arraybackslash}p{1.17cm} 
        >{\centering\arraybackslash}p{1.17cm} 
        >{\centering\arraybackslash}p{1.17cm} 
 }
    
        \toprule
        
        \multirow{2}{*}{\textbf{ Model}} & \multicolumn{6}{c}{\cellml \bf EB-Navigation} & \multicolumn{6}{c}{\cellmr \bf EB-Manipulation} \\

        \cmidrule(lr){2-7} \cmidrule(lr){8-13}
        
        ~ & \textbf{Avg} & \textbf{Base} & \textbf{Common} & \textbf{Complex} & \textbf{Visual}  & \textbf{Long} 
        & \textbf{Avg} & \textbf{Base} & \textbf{Common} & \textbf{Complex} & \textbf{Visual} & \textbf{Spatial}  \\

        % ~ & Base & CC & Complex & Visual & Exp & Avg & Base & CC & Complex & Visual & Spatial & Avg \\
        \addlinespace[2pt]
        \midrule
        \addlinespace[2pt]
        
        \multicolumn{13}{c}{ \textit{Proprietary MLLMs} }  \\ 
        \midrule
        
        {\fontsize{8}{10}\selectfont GPT-4o} & \textbf{57.7} & 55.0 & 60.0 & \textbf{58.3} & \textbf{60.0}  & \textbf{55.0}  
        & \textbf{28.9} & \textbf{39.6} & \textbf{29.2} & \textbf{29.2} & \textbf{19.4} & 25.0 \\ 
        {\fontsize{8}{10}\selectfont GPT-4o-mini} & 32.8 & 31.7 & 33.3 & 35.0 & 28.3 & 33.3 & 4.8 & 4.2 & 6.3 & 2.1 & 0.0 & 10.4 \\
        {\fontsize{8}{10}\selectfont Claude-3.5-Sonnet}&  44.7 & \textbf{66.7} & 51.7 & 41.7 & 36.7 & 26.7 
        & 25.4 & 37.5 & 16.7 & \textbf{29.2} & \textbf{19.4} & 22.9 \\
        {\fontsize{8}{10}\selectfont Gemini-1.5-Pro}  &24.3 &23.3 &25.0&25.0& 28.3&20.0
        & 21.1 & 14.6 & 14.6 & 22.9 & 16.7 & \textbf{35.4} \\
        {\fontsize{8}{10}\selectfont Gemini-2.0-flash}& 48.7  & 63.3 & \textbf{65.0} & 50.0 & 51.7 &  13.3
        & 16.7 & 14.6 & 8.3 & 14.6 & 13.9 & 31.3\\
        {\fontsize{8}{10}\selectfont Gemini-1.5-flash} & 41.7 & 56.7 & 50.0 & 46.7 & 50.0 & 5.0 
        & 9.6 & 14.6 & 10.4 & 4.2 & 8.3 & 10.4 \\
        %  add new
        {\fontsize{8}{10}\selectfont Qwen-VL-Max } &39.7 &50.0 &46.7 &41.7 & 35.0& 25.0& 18.0 & 25.0 & 10.4 & 18.8 & 2.8 & 29.2 \\
        \midrule
        
        {\fontsize{8}{10}\selectfont GPT-4o (Lang)} &17.4 & 21.7 & 21.7  & 26.7 & 16.7 & 0.0
        & 16.2 & 16.7 & 16.7 & 14.6 & \textbf{19.4} & 14.6 \\
        {\fontsize{8}{10}\selectfont GPT-4o-mini (Lang)} & 8.3 & 3.3 & 13.3 & 10.0 & 15.0 & 0.0 
        & 6.6 & 12.5 & 0.0 & 2.1 & 2.8 & 14.6 \\
        
        \addlinespace[2pt]
        \midrule
        \addlinespace[2pt]
        
        \multicolumn{13}{c}{ \textit{Open-Source MLLMs} }   
        \\ \midrule
        
        {\fontsize{8}{10}\selectfont Llama-3.2-90B-Vision-Ins} & 30.0 &48.3 &23.3 & 38.3 & 33.3 & 6.7
        & 14.9 & 10.4 & 12.5 & 16.7 & 10.4 & 20.8 \\
        {\fontsize{8}{10}\selectfont Llama-3.2-11B-Vision-Ins} & 21.4 & 23.3 & 21.7 & 26.7 & 18.3 & 17.0 & 
         0.9 & 0.0 & 0.0 & 2.1 & 0.0 & 2.1 \\
        {\fontsize{8}{10}\selectfont InternVL2\_5-78B} & 30.7 & 36.7 & 38.3 & 33.3 & 21.7 &23.3
        & 18.0 & 16.7 & 16.7 & 14.6 & 22.2 & 20.8 \\
        {\fontsize{8}{10}\selectfont InternVL2\_5-38B} & 30.3 & 35.0 & 28.3 & 38.3 & 26.7 &  23.3
        & 15.8 & \underline{22.9} & 16.7 & 8.3 & 13.9 & 16.7 \\
        {\fontsize{8}{10}\selectfont InternVL2\_5-8B} &21.3 &35.0 & 23.3& 21.7 &26.7 & 0.0
        & 7.0 & 8.3 & 2.1 & 6.3 & 8.3 & 10.4 \\

        {\fontsize{8}{10}\selectfont InternVL2\_5-78B-MPO } & 47.0 & 56.7 &50.0 & \underline{56.7} &\underline{41.7} &30.0 & \underline{21.9} & 20.8 & \underline{20.8} & 16.7 & \underline{25.0} & \underline{27.1} \\
        {\fontsize{8}{10}\selectfont InternVL2\_5-38B-MPO } & \underline{48.7} & 55.0& \underline{60.0}&51.7 & 40.0& \underline{36.7}& 21.1 & \underline{22.9} & 14.6 & \underline{25.0} & 19.4 & 22.9 \\
        {\fontsize{8}{10}\selectfont InternVL2\_5-8B-MPO } & 13.0 & 15.0&11.7 & 15.0& 16.7& 6.7& 1.8 & 0.0 & 0.0 & 0.0 & 2.8 & 6.3 \\
        
        {\fontsize{8}{10}\selectfont Qwen2-VL-72B-Ins} &21.2 & 26.7 & 30.0 & 28.3 & 16.0 & 5.0  
        & 13.6 & 18.8 & \underline{20.8} & 4.2 & 8.3 & 14.6\\
        {\fontsize{8}{10}\selectfont Qwen2-VL-7B-Ins} &14.0 & 26.7 & 10.0 & 15.0 & 15.0 & 3.3 
        & 0.0 & 0.0 & 0.0 & 0.0 & 0.0 & 0.0 \\
        % add new 

        {\fontsize{8}{10}\selectfont Qwen2.5-VL-72B-Ins } & 45.0 & \underline{58.3}& 48.3&48.3 & 36.7&33.3 & 16.2 & 12.5 & 12.5 & 16.7 & 22.2 & 18.8 \\
        {\fontsize{8}{10}\selectfont  Qwen2.5-VL-7B-Ins} & 25.7 &28.3 &30.0 &41.7 &20.0&8.3 & 9.6 & 8.3 & 8.3 & 8.3 & 5.6 & 16.7 \\
      
        
        \bottomrule
    \end{tabular}\label{tb:low_level_table}
    % }
    % }
    }
    
    \vspace{-1em}
\end{table*}



% \begin{table*}[t]
% \vspace{-10pt}
%     \centering\small
%     \caption{
%     Task success rates on 5 subsets of EB-Navigation and EB-Manipulation, with the best proprietary model in bold and open-source model underlines per column.
%     }
%     % \vspace{0.5em}
%     \renewcommand{\arraystretch}{1.1}
%     \setlength\tabcolsep{2pt}
%     \setlength\extrarowheight{2pt}
%     \resizebox{0.95\linewidth}{!}{

%     % \begin{tabular}{
%         % p{3.3cm} p{0.15\textwidth} c c c c c @{\hskip 14pt} c c c c c c
%         \begin{tabular}{
%     >{\centering\arraybackslash}p{3.3cm} 
%         >{\centering\arraybackslash}p{1.17cm} 
%         >{\centering\arraybackslash}p{1.17cm} 
%         >{\centering\arraybackslash}p{1.17cm} 
%         >{\centering\arraybackslash}p{1.17cm} 
%         >{\centering\arraybackslash}p{1.17cm} 
%         >{\centering\arraybackslash}p{1.17cm} 
%         >{\centering\arraybackslash}p{1.17cm} 
%         @{\hskip 10pt} 
%         >{\centering\arraybackslash}p{1.17cm} 
%         >{\centering\arraybackslash}p{1.17cm} 
%         >{\centering\arraybackslash}p{1.17cm} 
%         >{\centering\arraybackslash}p{1.17cm} 
%         >{\centering\arraybackslash}p{1.17cm} 
%         >{\centering\arraybackslash}p{1.17cm} 
%         >{\centering\arraybackslash}p{1.17cm} 
%  }
    
%         \toprule
        
%         \multirow{2}{*}{\textbf{ Model}} & \multicolumn{6}{c}{\cellml \bf EB-Navigation} & \multicolumn{6}{c}{\cellmr \bf EB-Manipulation} \\

%         \cmidrule(lr){2-7} \cmidrule(lr){8-13}
        
%         ~ & \textbf{Avg} & \textbf{Base} & \textbf{Common} & \textbf{Complex} & \textbf{Visual}  & \textbf{Long} 
%         & \textbf{Avg} & \textbf{Base} & \textbf{Common} & \textbf{Complex} & \textbf{Visual} & \textbf{Spatial}  \\

%         % ~ & Base & CC & Complex & Visual & Exp & Avg & Base & CC & Complex & Visual & Spatial & Avg \\
%         \addlinespace[2pt]
%         \midrule
%         \addlinespace[2pt]
        
%         \multicolumn{13}{c}{ \textit{Proprietary MLLMs} }  \\ 
%         \midrule
        
%         {\fontsize{8}{10}\selectfont GPT-4o} & \textbf{57.7} & 55.0 & 60.0 & \textbf{58.3} & \textbf{60.0}  & \textbf{55.0}  
%         & \textbf{28.9} & \textbf{39.6} & \textbf{29.2} & \textbf{29.2} & \textbf{19.4} & 25.0 \\ 
%         {\fontsize{8}{10}\selectfont GPT-4o-mini} & 32.8 & 31.7 & 33.3 & 35.0 & 28.3 & 33.3 & 4.8 & 4.2 & 6.3 & 2.1 & 0.0 & 10.4 \\
%         {\fontsize{8}{10}\selectfont Claude-3.5-Sonnet}&  44.7 & \textbf{66.7} & 51.7 & 41.7 & 36.7 & 26.7 
%         & 25.4 & 37.5 & 16.7 & \textbf{29.2} & \textbf{19.4} & 22.9 \\
%         {\fontsize{8}{10}\selectfont Gemini-1.5-Pro}  &24.3 &23.3 &25.0&25.0& 28.3&20.0
%         & 21.1 & 14.6 & 14.6 & 22.9 & 16.7 & \textbf{35.4} \\
%         {\fontsize{8}{10}\selectfont Gemini-2.0-flash}& 48.7  & 63.3 & \textbf{65.0} & 50.0 & 51.7 &  13.3
%         & 16.7 & 14.6 & 8.3 & 14.6 & 13.9 & 31.3\\
%         {\fontsize{8}{10}\selectfont Gemini-1.5-flash} & 41.7 & 56.7 & 50.0 & 46.7 & 50.0 & 5.0 
%         & 9.6 & 14.6 & 10.4 & 4.2 & 8.3 & 10.4 \\
%         %  add new
%         {\fontsize{8}{10}\selectfont Qwen-VL-Max } & & & & & & & 18.0 & 25.0 & 10.4 & 18.8 & 2.8 & 29.2 \\
%         \midrule
        
%         {\fontsize{8}{10}\selectfont GPT-4o (Lang)} &17.4 & 21.7 & 21.7  & 26.7 & 16.7 & 0.0
%         & 16.2 & 16.7 & 16.7 & 14.6 & \textbf{19.4} & 14.6 \\
%         {\fontsize{8}{10}\selectfont GPT-4o-mini (Lang)} & 8.3 & 3.3 & 13.3 & 10.0 & 15.0 & 0.0 
%         & 6.6 & 12.5 & 0.0 & 2.1 & 2.8 & 14.6 \\
        
%         \addlinespace[2pt]
%         \midrule
%         \addlinespace[2pt]
        
%         \multicolumn{13}{c}{ \textit{Open-Source MLLMs} }   
%         \\ \midrule
        
%         {\fontsize{8}{10}\selectfont Llama-3.2-90B-Vision-Ins} & 30.0 &\underline{48.3} &23.3 & \underline{38.3} &\underline{33.3} & 6.7
%         & 14.9 & 10.4 & 12.5 & \underline{16.7} & 10.4 & \underline{20.8} \\
%         {\fontsize{8}{10}\selectfont Llama-3.2-11B-Vision-Ins} & 21.4 & 23.3 & 21.7 & 26.7 & 18.3 & 17.0 & 
%          0.9 & 0.0 & 0.0 & 2.1 & 0.0 & 2.1 \\
%         {\fontsize{8}{10}\selectfont InternVL2\_5-78B} & \underline{30.7} & 36.7 & \underline{38.3} & 33.3 & 21.7 &\underline{23.3}
%         & \underline{18.0} & 16.7 & 16.7 & 14.6 & \underline{22.2} & \underline{20.8} \\
%         {\fontsize{8}{10}\selectfont InternVL2\_5-38B} & 30.3 & 35.0 & 28.3 & \underline{38.3} & 26.7 &  \underline{23.3} 
%         & 15.8 & \underline{22.9} & 16.7 & 8.3 & 13.9 & 16.7 \\
%         {\fontsize{8}{10}\selectfont InternVL2\_5-8B} &21.3 &35.0 & 23.3& 21.7 &26.7 & 0.0
%         & 7.0 & 8.3 & 2.1 & 6.3 & 8.3 & 10.4 \\
%         {\fontsize{8}{10}\selectfont Qwen2-VL-72B-Ins} &21.2 & 26.7 & 30.0 & 28.3 & 16.0 & 5.0  
%         & 13.6 & 18.8 & \underline{20.8} & 4.2 & 8.3 & 14.6\\
%         {\fontsize{8}{10}\selectfont Qwen2-VL-7B-Ins} &14.0 & 26.7 & 10.0 & 15.0 & 15.0 & 3.3 
%         & 0.0 & 0.0 & 0.0 & 0.0 & 0.0 & 0.0 \\
%         % add new 

%         {\fontsize{8}{10}\selectfont InternVL2\_5-78B-MPO } & & & & & & & 21.9 & 20.8 & 20.8 & 16.7 & 25.0 & 27.1 \\
%         {\fontsize{8}{10}\selectfont InternVL2\_5-38B-MPO } & & & & & & & 21.1 & 22.9 & 14.6 & 25.0 & 19.4 & 22.9 \\
%         {\fontsize{8}{10}\selectfont InternVL2\_5-8B-MPO } & & & & & & & 1.8 & 0.0 & 0.0 & 0.0 & 2.8 & 6.3 \\
%         {\fontsize{8}{10}\selectfont Qwen2.5-VL-72B-Instruct } & & & & & & & \\
%         {\fontsize{8}{10}\selectfont  Qwen2.5-VL-7B-Instruct} & & & & & & & 9.6 & 8.3 & 8.3 & 8.3 & 5.6 & 16.7 \\
      
        
%         \bottomrule
%     \end{tabular}\label{tb:low_level_table}
%     % }
%     % }
%     }
    
%     \vspace{-1em}
% \end{table*}

% \begin{table*}[!t]
% \centering\small
% \caption{Ablation Results of GPT-4o and Claude on EB-Manipulation.}
% \vspace{0.5em}
% \begin{tabular}{llllllll}
% \toprule
%                   & Base & 300x300 & 700x700 & multi-step image & Detection boxes & multi-view images & Visual ICL \\
% \midrule
% GPT-4o            & 39.6     & 22.9             & 35.4             & 31.3              & 27.1        & 31.3      & 35.4     \\
% \midrule
% Claude-3.5-Sonnet & 37.5     & 27.1             & 29.2             & 29.2              & 29.2        & 35.4      & 41.7     \\
% \bottomrule
% \end{tabular}
% \end{table*}


In this section, we conduct comprehensive experiments to evaluate the performance of various MLLMs in \name, followed by ablation studies in Sections \ref{sec:language_ablation} and \ref{sec:visual_ablation} and error analysis in Section \ref{sec:error_analysis}.

\vspace{-5pt}
\subsection{Experimental Setups}
\vspace{-5pt}

% We benchmark 13 models, including leading proprietary models (GPT-4o / 4o-mini~\cite{GPT-4o,GPT-4o-mini}, Claude-3.5-Sonnet~\cite{Claude-3.5-Sonnet}, Gemini Pro / Flash~\cite{team2023gemini,team2024gemini,Gemini2.0}, and SOTA open-source models (LLaMA3.2 11B / 90B Vision Instruct~\cite{llama3.2}, InternVL 2.5 8B / 38B / 78B~\cite{internvl2.5}, Qwen2-VL 7B / 72B~\cite{wang2024qwen2}). For consistency, all models are set with a temperature of 0 and a maximum completion token length of 2048. All images are standardized to a resolution of 500$\times$500 pixels. The maximum number of environment steps is 30 for high-level tasks, 20 for EB-Navigation, and 15 for EB-Manipulation. We use the task success rate as the primary metric in our main experiments. More results and ablations are deferred to Appendix \ref{ap:additional_exp}.

We benchmark 19 models, including leading proprietary models (GPT-4o / 4o-mini~\cite{GPT-4o,GPT-4o-mini}, Claude-3.5-Sonnet~\cite{Claude-3.5-Sonnet}, Gemini Pro / Flash~\cite{team2023gemini,team2024gemini,Gemini2.0}, and Qwen-VL-Max \cite{bai2023qwen}), and SOTA open-source models (LLaMA3.2 11B / 90B Vision Instruct~\cite{llama3.2}, InternVL 2.5 8B / 38B / 78B~\cite{internvl2.5} and their MPO versions \cite{wang2024enhancing}, and Qwen2-VL and Qwen2.5-VL 7B / 72B~\cite{wang2024qwen2,bai2025qwen25vl}). For consistency, all models are set with a temperature of 0 and a maximum completion token length of 2048. All images are standardized to a resolution of 500$\times$500 pixels. The maximum number of environment steps is 30 for high-level tasks, 20 for EB-Navigation, and 15 for EB-Manipulation. We use the task success rate as the primary metric in our main experiments. More results and ablations are deferred to Appendix \ref{ap:additional_exp}.



\vspace{-5pt}

\subsection{Benchmark Results}\label{sec:benchmark_res}
% \vspace{-5pt}

\textbf{Overall Results.} Tables \ref{tb:high_level_table} and \ref{tb:low_level_table} summarize the results for high-level and low-level tasks, respectively. Overall, \textit{\textbf{current MLLMs demonstrate strong performance on high-level tasks but struggle with low-level tasks, especially EB-Manipulation.}}
Among \textbf{proprietary models}, we observe that different models excel at different task levels: Claude-3.5-Sonnet achieves the highest average accuracy on high-level tasks, with 64.0\% on EB-ALFRED and 68.0\% on EB-Habitat, while GPT-4o leads in low-level tasks, scoring 57.7\% on EB-Navigation and 28.9\% on EB-Manipulation. Gemini-1.5-Pro performs the worst among the three large proprietary models, but Gemini-1.5 / 2.0-Flash outperforms GPT-4o-mini by a large margin. For \textbf{open-source models}, InternVL2\_5 model family exhibits the best overall performance, with its largest 78B version outperforming Llama-3.2-90B-Vision-Ins and Qwen2-VL-72B-Ins across all 4 environments. Additionally, open-source models exhibit a clear scaling effect, as their performance improves with increasing model parameters. Moreover, although large open-source models are closing the gap with smaller proprietary models like GPT-4o-mini, a notable performance difference remains between large proprietary and open-source models. 


% \textbf{The Role of Vision in Embodied Agent.}
% By comparing the performance of embodied agents with and without visual information (marked as ``Lang") in Tables \ref{tb:high_level_table} and \ref{tb:low_level_table}, we observe a clear distinction between low-level and high-level tasks. \textbf{\textit{Low-level tasks show a much stronger reliance on vision compared to high-level tasks.}} For example, disabling vision causes GPT-4o’s EB-Navigation performance to drop sharply from 57.7\% to 17.4\%, with long-horizon planning completely collapsing to 0\%. This sharp decline highlights the critical importance of visual signals for low-level control tasks. Conversely, high-level tasks show much less dependence on visual input. GPT-4o (Lang) and GPT-4o-mini (Lang) perform on par with or even outperform their vision-enabled counterparts in EB-ALFRED and EB-Habitat, suggesting that these tasks may rely more heavily on textual information rather than visual input. We will further investigate the impact of language-centric factors in Section \ref{sec:language_ablation}. These findings emphasize two key insights: (1) when designing MLLM-based embodied AI benchmarks, it is essential to consider action-level taxonomy, with greater attention to low-level action tasks, and (2) more advanced methods are needed to effectively leverage visual input for high-level embodied tasks.
