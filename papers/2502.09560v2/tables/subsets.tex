\begin{table*}
  \caption{
    Definition of six subsets in EB-ALFRED. More instruction examples of \name are provided in Appendix \ref{ap:task_subset}.
  } \label{tb:subsets_division}
  \centering
  \vspace{0.5em}
  \resizebox{1.0\linewidth}{!}{
  {\footnotesize
    {
      \begin{tabular}
      {p{0.15\linewidth}|m{0.44\linewidth}|m{0.44\linewidth}}
      \toprule
      % \hline
   \textbf{Subset Name} & \textbf{Instruction Example} & \textbf{Description} \\
  \midrule
  \cellcolor{white}Base &  Put washed lettuce in the refrigerator.   & \cellcolor{white}Instructions used to describe basic tasks.  \\ 
  \midrule
   Common Sense &  Place washed leafy green vegetable in a receptacle that can keep it fresh for several days.  & \cellcolor{white}Refer to objects indirectly using common sense knowledge.  \\ \midrule
  \cellcolor{white} Complex Instruction &  For freshness, place the washed lettuce in the refrigerator. This way, it's ready for any delightful recipe ideas you have. & \cellcolor{white}Add longer relevant or irrelevant context to obscure the instruction. This is used to evaluate the ability of understanding complex instructions.  \\ \midrule
          
     Spatial Awareness  &  Put two spray bottles in the cabinet under the sink against the wall.  &  Refer to objects by their location relative to other receptacles or objects. \\ \midrule
     
    \cellcolor{white} Visual Appearance & Put a knife in a blue container onto the black table in the corner.  & \cellcolor{white}Refer to objects indirectly by their visual appearance. \\ \midrule
   
  Long Horizon &  Pick up knife, slice apple, put knife in bowl, heat slice of apple in microwave, put apple slice on table. 
    & Describe a task that requires a long sequence of actions to complete. \\
\bottomrule
\end{tabular}
    }
  }
}
  \label{table:subsets} 
\end{table*}