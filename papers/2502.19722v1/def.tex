% packages
\usepackage{multirow}
\usepackage{makecell}
\usepackage{hhline}
\usepackage{booktabs}
\usepackage{bm}
\usepackage{amssymb}
\usepackage{amsmath}
\usepackage{url}
\usepackage{graphicx}
\usepackage{float}
\usepackage{subfigure}
\usepackage{subcaption}
\usepackage{paralist}
\usepackage[ruled, lined, linesnumbered, commentsnumbered, longend]{algorithm2e}
\usepackage{xcolor}
\usepackage{colortbl}
\usepackage{pifont}
\usepackage{tabularx}
\usepackage[T2A,LAE,T1]{fontenc}
\usepackage{CJKutf8}
\usepackage{enumitem}
\setlist[itemize]{noitemsep, topsep=0pt}
\usepackage[russian,english]{babel}
\usepackage{relsize}

\DeclareRobustCommand{\cyrins}[1]{
  \begingroup\fontfamily{cmr}
  \foreignlanguage{russian}{#1}
  \endgroup
}
\DeclareRobustCommand{\aratext}[1]{
  \begingroup\fontfamily{Arab}
  \foreignlanguage{arabic}{#1}
  \endgroup
}

% comments
\newcommand{\trevor}[1]{\textcolor{red}{\small\textbf{[Trevor:]} #1}}
\newcommand{\tom}[1]{\textcolor{blue}{\small\textbf{[Tom:]} #1}}
\newcommand{\fan}[1]{\textcolor{orange}{\small\textbf{[Fan:]} #1}}
\newcommand{\wishlist}[1]{\textcolor{gray}{\small \text{[TODO:]} #1}}

% style
\newcommand{\TableHighlight}[1]{\colorbox{lightgray}{#1}}
\definecolor{LightCyan}{rgb}{0.88,1,1}
\newcommand{\boldtitle}[1]{\noindent\textbf{#1}}
\newcommand{\step}[1]{\noindent\emph{#1}}
\newcommand{\edit}[1]{\textcolor{blue}{#1}}
\newcommand{\promptsize}{\fontsize{7pt}{8pt}\selectfont}
\newcommand{\propertytsize}{\fontsize{7pt}{8pt}\selectfont}
 
% definitions for models and methods and baselines and eval metrics
\newcommand{\class}{CLASS\xspace}
\newcommand{\classus}{CLASS-US\xspace}
\newcommand{\classusstageone}{CLASS-US-Stage1\xspace}
\newcommand{\classzs}{CLASS-En\xspace}

\newcommand{\ours}{\textsc{FsModQA}\xspace}
\newcommand{\oursen}{\textsc{FsModQA-En}\xspace}
\newcommand{\ourdata}{\textsc{FsMlQA}\xspace}
\newcommand{\ourptdata}{\textsc{MlWikiQA}\xspace}


% theorem
\usepackage{ntheorem}

\newtheorem{theorem}{Theorem}
\newtheorem{lemma}[theorem]{Lemma}
\newtheorem{corollary}{Corollary}
\newtheorem{remark}{Remark}
\newtheorem{proposition}{Proposition}
\newtheorem{example}{Example}
\newtheorem{definition}{Definition} 
\theoremstyle{nonumberplain}
\newtheorem{proof}{Proof}


\def\eg{{e.g.,}\xspace}
\def\ie{{i.e.,}\xspace}
\def\versus{{\em v.s.}\xspace}
\def\cf{{\em cf.}\xspace}
\def\wrt{{\em w.r.t}\xspace}
\def\aka{{\em a.k.a}\xspace}
\def\etc{{\em etc.}\xspace}
\definecolor{lightblue}{HTML}{bdd6fb}
\definecolor{boxgray}{gray}{0.9}
\definecolor{bgyellow}{HTML}{fcebde}
\definecolor{bgred}{HTML}{d77470}
\definecolor{bggrey}{HTML}{dcc0e5}


\usepackage{inconsolata}
\usepackage{pifont}
\usepackage[most]{tcolorbox}
\usepackage{csquotes}
\usepackage{anyfontsize}
\usepackage{comment}
\usepackage{float}
\usepackage{cuted}
\usepackage{tikz}
\usepackage{listings,multicol}
\lstset{
    basicstyle=\linespread{1.2}\fontfamily{qtm}\footnotesize,
    columns=fullflexible,
    breaklines=true,
    breakautoindent=false,
    breakindent=0pt,
    escapeinside={\%*}{*)},
}
\usepackage{etoolbox}% >= v2.1 2011-01-03
\BeforeBeginEnvironment{lstlisting}{\begin{mdframed}\vspace{-0.7em}}
\AfterEndEnvironment{lstlisting}{\vspace{-0.5em}\end{mdframed}}

% needed for \lstcapt
\def\ifempty#1{\def\temparg{#1}\ifx\temparg\empty}

% make new caption command for listings
\usepackage{caption}
\newcommand{\lstcapt}[2][]{%
    \ifempty{#1}%
        \captionof{lstlisting}{#2}%
    \else%
        \captionof{lstlisting}[#1]{#2}%
    \fi%
    \vspace{0.75\baselineskip}%
}

\newtcolorbox[list inside=prompt,auto counter,number within=section]{prompt}[1][]{
    colbacktitle=black!60,
    coltitle=white,
    colback=bgyellow,
    fontupper=\footnotesize,
    boxsep=5pt,
    left=0pt,
    right=0pt,
    top=0pt,
    bottom=0pt,
    boxrule=1pt,
    #1,
}


%%%%%%%%%%%%%%%%%%%%%%%%%%%%%%%%%%%%%%%%%%%%%%%%%%%%%%%%%%%%%%%%%%%%%%%%%
% This section is based on the bbk10.clo file
% of Palash Baran Pal's bangtex
% http://www.saha.ac.in/theory/palashbaran.pal/bangtex/bangtex.html
%%%%%%%%%%%%%%%%%%%%%%%%%%%%%%%%%%%%%%%%%%%%%%%%%%%%%%%%%%%%%%%%%%%%%%%%%

\def\sbng{\bngviii}
\def\tbng{\bngvi}
\def\bng{\bngx}
\def\lbng{\bngxiv}
\def\Lbng{\bngxviii}
\def\LBng{\bngxxii}
\def\hbng{\bngxxv}
\def\Hbng{\bngxxx}
%
\def\sbns{\bnsviii}
\def\tbns{\bnsvi}
\def\bns{\bnsx}
\def\lbns{\bnsxiv}
\def\Lbns{\bnsxviii}
\def\LBns{\bnsxxii}
\def\hbns{\bnsxxv}
\def\Hbns{\bnsxxx}
%
\def\sbnw{\bnwviii}
\def\tbnw{\bnwvi}
\def\bnw{\bnwx}
\def\lbnw{\bnwxiv}
\def\Lbnw{\bnwxviii}
\def\LBnw{\bnwxxii}
\def\hbnw{\bnwxxv}
\def\Hbnw{\bnwxxx}


%%%%%%%%%%%%%%%%%%%%%%%%%%%%%%%%%%%%%%%%%%%%%%%%%%%%%%%%%%%%%%%%%%%%%%%%%
% This section is based on the bangfont.tex file
% of Palash Baran Pal's bangtex
% http://www.saha.ac.in/theory/palashbaran.pal/bangtex/bangtex.html
%%%%%%%%%%%%%%%%%%%%%%%%%%%%%%%%%%%%%%%%%%%%%%%%%%%%%%%%%%%%%%%%%%%%%%%%%

%%
%% Defining the normal bangla fornts
%%

\font\bngv=bang10 scaled 500
\font\bngvi=bang10 scaled 600
\font\bngvii=bang10 scaled 700
\font\bngviii=bang10 scaled 800
\font\bngix=bang10 scaled 900
\font\bngx=bang10
\font\bngxi=bang10 scaled 1100
\font\bngxii=bang10 scaled 1200
\font\bngxiv=bang10 scaled 1400
\font\bngxviii=bang10 scaled 1800
\font\bngxxii=bang10 scaled 2200
\font\bngxxv=bang10 scaled 2500
\font\bngxxx=bang10 scaled 3000

%%
%% Defining the slanted bangla fonts
%%
\font\bnsv=bangsl10 scaled 500
\font\bnsvi=bangsl10 scaled 600
\font\bnsvii=bangsl10 scaled 700
\font\bnsviii=bangsl10 scaled 800
\font\bnsix=bangsl10 scaled 900
\font\bnsx=bangsl10
\font\bnsxi=bangsl10 scaled 1100
\font\bnsxii=bangsl10 scaled 1200
\font\bnsxiv=bangsl10 scaled 1400
\font\bnsxviii=bangsl10 scaled 1800
\font\bnsxxii=bangsl10 scaled 2200
\font\bnsxxv=bangsl10 scaled 2500
\font\bnsxxx=bangsl10 scaled 3000

%%
%% Defining the wide bangla fonts
%%
\font\bnwv=bangwd10 scaled 500
\font\bnwvi=bangwd10 scaled 600
\font\bnwvii=bangwd10 scaled 700
\font\bnwviii=bangwd10 scaled 800
\font\bnwix=bangwd10 scaled 900
\font\bnwx=bangwd10
\font\bnwxi=bangwd10 scaled 1100
\font\bnwxii=bangwd10 scaled 1200
\font\bnwxiv=bangwd10 scaled 1400
\font\bnwxviii=bangwd10 scaled 1800
\font\bnwxxii=bangwd10 scaled 2200
\font\bnwxxv=bangwd10 scaled 2500
\font\bnwxxx=bangwd10 scaled 3000


%%
%% Inhibiting linebreak within words
%%
%\hyphenpenalty=10000 \pretolerance=-1 \tolerance=10000

%%
%% Defining the macro for e-kar, i-kar etc
%%
\def\*#1*#2{o\null{#2}{#1}}

%%
%% Redefining some macros to make them consistent with bangla fonts
%%
\def\d#1{\oalign{\smash{#1}\crcr\hidewidth{$\!$\rm.}\hidewidth}}