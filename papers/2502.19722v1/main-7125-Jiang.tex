% File tacl2021v1.tex
% Dec. 15, 2021

% The English content of this file was modified from various *ACL instructions
% by Lillian Lee and Kristina Toutanova
%
% LaTeXery is mostly all adapted from acl2018.sty.

\documentclass[11pt,a4paper]{article}
\usepackage{times,latexsym}
\usepackage{url}
\usepackage[T1]{fontenc}

% Lists
\usepackage[inline]{enumitem}
\newlist{enuminline}{enumerate*}{1}
\setlist[enuminline]{label=(\roman*)}

% Math
\newcommand{\vol}{\mathrm{vol}}
\newcommand{\E}{\mathbb E}
\newcommand{\var}{\mathrm{var}}
\newcommand{\cov}{\mathrm{cov}}
\newcommand{\Normal}{\mathcal N}
\newcommand{\slowvar}{\mathcal L}
\newcommand{\bbR}{\mathbb R}
\newcommand{\bbE}{\mathbb E}
\newcommand{\bbP}{\mathbb P}
\newcommand{\calC}{\mathcal C}
\newcommand{\calR}{\mathcal R}
\newcommand{\calT}{\mathcal T}
\newcommand{\loA}{\underline{A}}
\newcommand{\upA}{\overline{A}} 
\newcommand{\cv}{\mathrm{cv}} 
\newcommand{\pp}{\mathrm{pp}}
\newcommand{\HS}{\mathrm{HS}}
\newcommand{\erfi}{\mathrm{erfi}}
\newcommand{\tX}{\widetilde X}
\newcommand{\OneFOne}{{}_1F_1}
\newcommand{\PP}{\mathrm{\texttt{PP}}}
\newcommand{\PPpp}{\mathrm{\texttt{PP+}}}
\newcommand{\BPP}{\mathrm{\texttt{BPP}}}
\newcommand{\BPPpp}{\mathrm{\texttt{BPP+}}}
\newcommand{\FABPPI}{\mathrm{\texttt{FABPP}}}
\newcommand{\asto}{\overset{\text{a.s.}}{\to}}
% version 1.2 05/21/08
\newcommand\sa{\ensuremath{\mathcal{a}}}
\newcommand\sd{\ensuremath{\mathcal{d}}}
\newcommand\se{\ensuremath{\mathcal{e}}}
\newcommand\sg{\ensuremath{\mathcal{g}}}
% \newcommand\sh{\ensuremath{\mathcal{h}}}
\newcommand\seye{\ensuremath{\mathcal{i}}}
\newcommand\sj{\ensuremath{\mathcal{j}}}
\newcommand\sk{\ensuremath{\mathcal{k}}}
\newcommand\sm{\ensuremath{\mathcal{m}}}
\newcommand\sn{\ensuremath{\mathcal{n}}}
\newcommand\sq{\ensuremath{\mathcal{q}}}
\newcommand\sr{\ensuremath{\mathcal{r}}}
\newcommand\su{\ensuremath{\mathcal{u}}}
\newcommand\sv{\ensuremath{\mathcal{v}}}
\newcommand\sw{\ensuremath{\mathcal{w}}}
\newcommand\sx{\ensuremath{\mathcal{x}}}
\newcommand\sy{\ensuremath{\mathcal{y}}}
\newcommand\sz{\ensuremath{\mathcal{z}}}
\newcommand\sA{\ensuremath{\mathcal{A}}}
\newcommand\sB{\ensuremath{\mathcal{B}}}
\newcommand\sC{\ensuremath{\mathcal{C}}}
\newcommand\sD{\ensuremath{\mathcal{D}}}
\newcommand\sE{\ensuremath{\mathcal{E}}}
\newcommand\sF{\ensuremath{\mathcal{F}}}
\newcommand\sG{\ensuremath{\mathcal{G}}}
\newcommand\sH{\ensuremath{\mathcal{H}}}
\newcommand\sI{\ensuremath{\mathcal{I}}}
\newcommand\sJ{\ensuremath{\mathcal{J}}}
\newcommand\sK{\ensuremath{\mathcal{K}}}
\newcommand\sL{\ensuremath{\mathcal{L}}}
\newcommand\sM{\ensuremath{\mathcal{M}}}
\newcommand\sN{\ensuremath{\mathcal{N}}}
\newcommand\sO{\ensuremath{\mathcal{O}}}
\newcommand\sP{\ensuremath{\mathcal{P}}}
\newcommand\sQ{\ensuremath{\mathcal{Q}}}
\newcommand\sR{\ensuremath{\mathcal{R}}}
\newcommand\sS{\ensuremath{\mathcal{S}}}
\newcommand\sT{\ensuremath{\mathcal{T}}}
\newcommand\sU{\ensuremath{\mathcal{U}}}
\newcommand\sV{\ensuremath{\mathcal{V}}}
\newcommand\sW{\ensuremath{\mathcal{W}}}
\newcommand\sX{\ensuremath{\mathcal{X}}}
\newcommand\sY{\ensuremath{\mathcal{Y}}}
\newcommand\sZ{\ensuremath{\mathcal{Z}}}
\newcommand\ba{\ensuremath{\bm{a}}}
\newcommand\bb{\ensuremath{\bm{b}}}
\newcommand\bc{\ensuremath{\bm{c}}}
\newcommand\bd{\ensuremath{\bm{d}}}
\newcommand\be{\ensuremath{\bm{e}}}
\newcommand\bef{\ensuremath{\bm{f}}}
\newcommand\bg{\ensuremath{\bm{g}}}
\newcommand\bh{\ensuremath{\bm{h}}}
\newcommand\bi{\ensuremath{\bm{i}}}
\newcommand\bj{\ensuremath{\bm{j}}}
\newcommand\bk{\ensuremath{\bm{k}}}
\newcommand\bl{\ensuremath{\bm{l}}}
\newcommand\bn{\ensuremath{\bm{n}}}
\newcommand\bo{\ensuremath{\bm{o}}}
\newcommand\bp{\ensuremath{\bm{p}}}
\newcommand\bq{\ensuremath{\bm{q}}}
\newcommand\br{\ensuremath{\bm{r}}}
\newcommand\bs{\ensuremath{\bm{s}}}
\newcommand\bt{\ensuremath{\bm{t}}}
\newcommand\bu{\ensuremath{\bm{u}}}
\newcommand\bv{\ensuremath{\bm{v}}}
\newcommand\bw{\ensuremath{\bm{w}}}
\newcommand\bx{\ensuremath{\bm{x}}}
\newcommand\by{\ensuremath{\bm{y}}}
\newcommand\bz{\ensuremath{\bm{z}}}
\newcommand\bA{\ensuremath{\bm{A}}}
\newcommand\bB{\ensuremath{\bm{B}}}
\newcommand\bC{\ensuremath{\bm{C}}}
\newcommand\bD{\ensuremath{\bm{D}}}
\newcommand\bE{\ensuremath{\bm{E}}}
\newcommand\bF{\ensuremath{\bm{F}}}
\newcommand\bG{\ensuremath{\bm{G}}}
\newcommand\bH{\ensuremath{\bm{H}}}
\newcommand\bI{\ensuremath{\bm{I}}}
\newcommand\bJ{\ensuremath{\bm{J}}}
\newcommand\bK{\ensuremath{\bm{K}}}
\newcommand\bL{\ensuremath{\bm{L}}}
\newcommand\bM{\ensuremath{\bm{M}}}
\newcommand\bN{\ensuremath{\bm{N}}}
\newcommand\bO{\ensuremath{\bm{O}}}
\newcommand\bP{\ensuremath{\bm{P}}}
\newcommand\bQ{\ensuremath{\bm{Q}}}
\newcommand\bR{\ensuremath{\bm{R}}}
\newcommand\bS{\ensuremath{\bm{S}}}
\newcommand\bT{\ensuremath{\bm{T}}}
\newcommand\bU{\ensuremath{\bm{U}}}
\newcommand\bV{\ensuremath{\bm{V}}}
\newcommand\bW{\ensuremath{\bm{W}}}
\newcommand\bX{\ensuremath{\bm{X}}}
\newcommand\bY{\ensuremath{\bm{Y}}}
\newcommand\bZ{\ensuremath{\bm{Z}}}
\newcommand\Ba{\ensuremath{\mathbb{a}}}
\newcommand\Bb{\ensuremath{\mathbb{b}}}
\newcommand\Bc{\ensuremath{\mathbb{c}}}
\newcommand\Bd{\ensuremath{\mathbb{d}}}
\newcommand\Be{\ensuremath{\mathbb{e}}}
\newcommand\Bf{\ensuremath{\mathbb{f}}}
\newcommand\Bg{\ensuremath{\mathbb{g}}}
\newcommand\Bh{\ensuremath{\mathbb{h}}}
\newcommand\Bi{\ensuremath{\mathbb{i}}}
\newcommand\Bj{\ensuremath{\mathbb{j}}}
\newcommand\Bk{\ensuremath{\mathbb{k}}}
\newcommand\Bl{\ensuremath{\mathbb{l}}}
\newcommand\Bm{\ensuremath{\mathbb{m}}}
\newcommand\Bn{\ensuremath{\mathbb{n}}}
\newcommand\Bo{\ensuremath{\mathbb{o}}}
\newcommand\Bp{\ensuremath{\mathbb{p}}}
\newcommand\Bq{\ensuremath{\mathbb{q}}}
\newcommand\Br{\ensuremath{\mathbb{r}}}
\newcommand\Bs{\ensuremath{\mathbb{s}}}
\newcommand\Bt{\ensuremath{\mathbb{t}}}
\newcommand\Bu{\ensuremath{\mathbb{u}}}
\newcommand\Bv{\ensuremath{\mathbb{v}}}
\newcommand\Bw{\ensuremath{\mathbb{w}}}
\newcommand\Bx{\ensuremath{\mathbb{x}}}
\newcommand\By{\ensuremath{\mathbb{y}}}
\newcommand\Bz{\ensuremath{\mathbb{z}}}
\newcommand\BA{\ensuremath{\mathbb{A}}}
\newcommand\BB{\ensuremath{\mathbb{B}}}
\newcommand\BC{\ensuremath{\mathbb{C}}}
\newcommand\BD{\ensuremath{\mathbb{D}}}
\newcommand\BE{\ensuremath{\mathbb{E}}}
\newcommand\BF{\ensuremath{\mathbb{F}}}
\newcommand\BG{\ensuremath{\mathbb{G}}}
\newcommand\BH{\ensuremath{\mathbb{H}}}
\newcommand\BI{\ensuremath{\mathbb{I}}}
\newcommand\BJ{\ensuremath{\mathbb{J}}}
\newcommand\BK{\ensuremath{\mathbb{K}}}
\newcommand\BL{\ensuremath{\mathbb{L}}}
\newcommand\BM{\ensuremath{\mathbb{M}}}
\newcommand\BN{\ensuremath{\mathbb{N}}}
\newcommand\BO{\ensuremath{\mathbb{O}}}
\newcommand\BP{\ensuremath{\mathbb{P}}}
\newcommand\BQ{\ensuremath{\mathbb{Q}}}
\newcommand\BR{\ensuremath{\mathbb{R}}}
\newcommand\BS{\ensuremath{\mathbb{S}}}
\newcommand\BT{\ensuremath{\mathbb{T}}}
\newcommand\BU{\ensuremath{\mathbb{U}}}
\newcommand\BV{\ensuremath{\mathbb{V}}}
\newcommand\BW{\ensuremath{\mathbb{W}}}
\newcommand\BX{\ensuremath{\mathbb{X}}}
\newcommand\BY{\ensuremath{\mathbb{Y}}}
\newcommand\BZ{\ensuremath{\mathbb{Z}}}
\newcommand\balpha{\ensuremath{\mbox{\boldmath$\alpha$}}}
\newcommand\bbeta{\ensuremath{\mbox{\boldmath$\beta$}}}
\newcommand\btheta{\ensuremath{\mbox{\boldmath$\theta$}}}
\newcommand\bphi{\ensuremath{\mbox{\boldmath$\phi$}}}
\newcommand\bpi{\ensuremath{\mbox{\boldmath$\pi$}}}
\newcommand\bpsi{\ensuremath{\mbox{\boldmath$\psi$}}}
\newcommand\bmu{\ensuremath{\mbox{\boldmath$\mu$}}}
% Basic

\newcommand\sign{\text{sign}}
\newcommand\tr{\text{tr}}
\newcommand\fig[1]{\begin{center} \includegraphics{#1} \end{center}}
\newcommand\Fig[5]{\begin{figure}[tb] \begin{center} \includegraphics[scale=#2]{#1} \end{center} \longcaption{#4}{\label{fig:#3} #5} \end{figure}}
\newcommand\FigTop[4]{\begin{figure}[t] \begin{center} \includegraphics[scale=#2]{#1} \end{center} \caption{\label{fig:#3} #4} \end{figure}}
\newcommand\FigStar[4]{\begin{figure*}[tb] \begin{center} \includegraphics[scale=#2]{#1} \end{center} \caption{\label{fig:#3} #4} \end{figure*}}
\newcommand\aside[1]{\quad\text{[#1]}}
\newcommand\homework[3]{\title{#1} \author{#2} \date{#3} \maketitle}
% Math
\newcommand\argmin{\mathop{\text{argmin}}}
\newcommand\argmax{\mathop{\text{argmax}}}
\newcommand\p[1]{\ensuremath{\left( #1 \right)}} % Parenthesis ()
\newcommand\pb[1]{\ensuremath{\left[ #1 \right]}} % []
\newcommand\pc[1]{\ensuremath{\left\{ #1 \right\}}} % {}
\newcommand\eval[2]{\ensuremath{\left. #1 \right|_{#2}}} % Evaluation
\newcommand\inv[1]{\ensuremath{\frac{1}{#1}}}
\newcommand\half{\ensuremath{\frac{1}{2}}}
\newcommand\R{\ensuremath{\mathbb{R}}} % Real numbers
\newcommand\Z{\ensuremath{\mathbb{Z}}} % Integers
\newcommand\inner[2]{\ensuremath{\left< #1, #2 \right>}} % Inner product
\newcommand\mat[2]{\ensuremath{\left(\begin{array}{#1}#2\end{array}\right)}} % Matrix
\newcommand\eqn[1]{\begin{eqnarray} #1 \end{eqnarray}} % Equation (array)
\newcommand\eqnl[2]{\begin{eqnarray} \label{eqn:#1} #2 \end{eqnarray}} % Equation (array) with label
\newcommand\eqdef{\ensuremath{\stackrel{\rm def}{=}}} % Equal by definition
%\newcommand{\1}{\mathbb{I}} % Indicator (don't use \mathbbm{1} because bbm is not TrueType though)
\newcommand{\1}{\ensuremath{\mathbbm{1}}}
\newcommand{\bone}{\bm{1}} % for vector one
\newcommand{\bzero}{\bm{0}} % for vector zero
\newcommand\refeqn[1]{(\ref{eqn:#1})}
\newcommand\refeqns[2]{(\ref{eqn:#1}) and (\ref{eqn:#2})}
\newcommand\refchp[1]{Chapter~\ref{chp:#1}}
\newcommand\refsec[1]{Section~\ref{sec:#1}}
\newcommand\refsecs[2]{Sections~\ref{sec:#1} and~\ref{sec:#2}}
\newcommand\reffig[1]{Figure~\ref{fig:#1}}
\newcommand\reffigs[2]{Figures~\ref{fig:#1} and~\ref{fig:#2}}
\newcommand\reffigss[3]{Figures~\ref{fig:#1},~\ref{fig:#2}, and~\ref{fig:#3}}
\newcommand\reffigsss[4]{Figures~\ref{fig:#1},~\ref{fig:#2},~\ref{fig:#3}, and~\ref{fig:#4}}
\newcommand\reftab[1]{Table~\ref{tab:#1}}
\newcommand\refapp[1]{Appendix~\ref{sec:#1}}
\newcommand\refthm[1]{Theorem~\ref{thm:#1}}
\newcommand\refthms[2]{Theorems~\ref{thm:#1} and~\ref{thm:#2}}
\newcommand\reflem[1]{Lemma~\ref{lem:#1}}
\newcommand\reflems[2]{Lemmas~\ref{lem:#1} and~\ref{lem:#2}}
\newcommand\refprop[1]{Proposition~\ref{prop:#1}}
\newcommand\refdef[1]{Definition~\ref{def:#1}}
\newcommand\refcor[1]{Corollary~\ref{cor:#1}}
\newcommand\refalg[1]{Algorithm~\ref{alg:#1}}

\newcommand\Chapter[2]{\chapter{#2}\label{chp:#1}}
\newcommand\Section[2]{\section{#2}\label{sec:#1}}
\newcommand\Subsection[2]{\subsection{#2}\label{sec:#1}}
\newcommand\Subsubsection[2]{\subsubsection{#2}\label{sec:#1}}
%\newtheorem{definition}{Definition}
%\newtheorem{assumption}{Assumption}
%\newtheorem{proposition}{Proposition}
%\newtheorem{theorem}{Theorem}
%\newtheorem{lemma}{Lemma}
%\newtheorem{corollary}{Corollary}
% Probability
\newcommand\cv{\ensuremath{\to}} % Convergence
\newcommand\cvL{\ensuremath{\xrightarrow{\mathcal{L}}}} % Convergence in law
\newcommand\cvd{\ensuremath{\xrightarrow{d}}} % Convergence in distribution
\newcommand\cvP{\ensuremath{\xrightarrow{P}}} % Convergence in probability
\newcommand\cvas{\ensuremath{\xrightarrow{a.s.}}} % Convergence almost surely
\newcommand\eqdistrib{\ensuremath{\stackrel{d}{=}}} % Equal in distribution
\newcommand\E[1]{\ensuremath{\mathbb{E}{\left[#1\right]}}} % Expectation
\newcommand\Ex[2]{\ensuremath{\mathbb{E}_{#1}\left[#2\right]}} % Expectation
%\newcommand\var{\ensuremath{\text{var}}} % Variance
\newcommand\cov{\ensuremath{\text{cov}}} % Covariance
\newcommand\diag{\ensuremath{\text{diag}}} % Diagnonal matrix
\newcommand\cE[2]{\ensuremath{\E \left( #1 \mid #2 \right)}} % Conditional expectation
\newcommand\KL[2]{\ensuremath{\T{KL}\left( #1 \,||\, #2 \right)}} % KL-divergence
\newcommand\D[2]{\ensuremath{\bD\left( #1 \,||\, #2 \right)}} % KL-divergence

% Utilities
\newcommand\lte{\leq}
\newcommand\gte{\geq}
\newcommand\lone[1]{\ensuremath{\|#1\|_1}}
\newcommand\ltwo[1]{\ensuremath{\|#1\|_2^2}}
\newcommand\naive{na\"{\i}ve}
\newcommand\Naive{Na\"{\i}ve}

% Debug
\usepackage{color}
\newcommand{\tred}[1]{\textcolor{red}{#1}}
\newcommand{\hly}[1]{\hl{yellow}{#1}}
% \def\todo#1{\hl{{\bf TODO:} #1}{yellow}}
\def\needcite{\hl{{$^{\tt\small[citation\ needed]}$}}{blue}}
\def\needfig{\hl{Figure X}{green}}
\def\needtab{\hl{Table Y}{green}}
\def\note#1{\hl{{\bf NOTE:} #1}{yellow}}
\def\dome{\hl{{\bf TODO:} write me!}{yellow}}


%% Package options:
%% Short version: "hyperref" and "submission" are the defaults.
%% More verbose version:
%% Most compact command to produce a submission version with hyperref enabled
%%    \usepackage[]{tacl2021v1}
%% Most compact command to produce a "camera-ready" version
%%    \usepackage[acceptedWithA]{tacl2021v1}
%% Most compact command to produce a double-spaced copy-editor's version
%%    \usepackage[acceptedWithA,copyedit]{tacl2021v1}
%
%% If you need to disable hyperref in any of the above settings (see Section
%% "LaTeX files") in the TACL instructions), add ",nohyperref" in the square
%% brackets. (The comma is a delimiter in case there are multiple options specified.)

\usepackage[acceptedWithA]{tacl2021v1}
% \usepackage[acceptedWithA,copyedit]{tacl2021v1}
% \setlength\titlebox{10cm} % <- for Option 2 below

%%%% Material in this block is specific to generating TACL instructions
\usepackage{xspace,mfirstuc,tabulary}
\newcommand{\dateOfLastUpdate}{Dec. 15, 2021}
\newcommand{\styleFileVersion}{tacl2021v1}

\newcommand{\ex}[1]{{\sf #1}}

\newif\iftaclinstructions
\taclinstructionsfalse % AUTHORS: do NOT set this to true
\iftaclinstructions
\renewcommand{\confidential}{}
\renewcommand{\anonsubtext}{(No author info supplied here, for consistency with
TACL-submission anonymization requirements)}
\newcommand{\instr}
\fi

%
\iftaclpubformat % this "if" is set by the choice of options
\newcommand{\taclpaper}{final version\xspace}
\newcommand{\taclpapers}{final versions\xspace}
\newcommand{\Taclpaper}{Final version\xspace}
\newcommand{\Taclpapers}{Final versions\xspace}
\newcommand{\TaclPapers}{Final Versions\xspace}
\else
\newcommand{\taclpaper}{submission\xspace}
\newcommand{\taclpapers}{{\taclpaper}s\xspace}
\newcommand{\Taclpaper}{Submission\xspace}
\newcommand{\Taclpapers}{{\Taclpaper}s\xspace}
\newcommand{\TaclPapers}{Submissions\xspace}
\fi

%%%% End TACL-instructions-specific macro block
%%%%


\title{Few-Shot Multilingual Open-Domain QA from 5 Examples}

% Author information does not appear in the pdf unless the "acceptedWithA" option is given
\author{Fan Jiang \and Tom Drummond \and Trevor Cohn\Thanks{Also at Google.} \\
  School of Computing and Information Systems \\
  The University of Melbourne, Victoria, Australia \\
  \texttt{fan.jiang1@student.unimelb.edu.au}\\
  \texttt{\{tom.drummond, trevor.cohn\}@unimelb.edu.au}}

\date{}

\begin{document}
\maketitle
\begin{abstract} 
Recent approaches to multilingual open-domain question answering (MLODQA) have achieved promising results given abundant language-specific training data.
% However, enhancing the multilingual performance with limited supervised data remains challenging and under-explored. 
However, the considerable annotation cost limits the application of these methods for underrepresented languages.
We introduce a \emph{few-shot learning} approach to synthesise large-scale multilingual data from large language models (LLMs).
Our method begins with large-scale self-supervised pre-training using WikiData, followed by training on high-quality synthetic multilingual data generated by prompting LLMs with few-shot supervision.
The final model, \ours, significantly outperforms existing few-shot and supervised baselines in MLODQA and cross-lingual and monolingual retrieval.
We further show our method can be extended for effective zero-shot adaptation to new languages through a \emph{cross-lingual prompting} strategy with only English-supervised data, making it a general and applicable solution for MLODQA tasks without costly large-scale annotation.
\end{abstract}

\section{Introduction}

Video generation has garnered significant attention owing to its transformative potential across a wide range of applications, such media content creation~\citep{polyak2024movie}, advertising~\citep{zhang2024virbo,bacher2021advert}, video games~\citep{yang2024playable,valevski2024diffusion, oasis2024}, and world model simulators~\citep{ha2018world, videoworldsimulators2024, agarwal2025cosmos}. Benefiting from advanced generative algorithms~\citep{goodfellow2014generative, ho2020denoising, liu2023flow, lipman2023flow}, scalable model architectures~\citep{vaswani2017attention, peebles2023scalable}, vast amounts of internet-sourced data~\citep{chen2024panda, nan2024openvid, ju2024miradata}, and ongoing expansion of computing capabilities~\citep{nvidia2022h100, nvidia2023dgxgh200, nvidia2024h200nvl}, remarkable advancements have been achieved in the field of video generation~\citep{ho2022video, ho2022imagen, singer2023makeavideo, blattmann2023align, videoworldsimulators2024, kuaishou2024klingai, yang2024cogvideox, jin2024pyramidal, polyak2024movie, kong2024hunyuanvideo, ji2024prompt}.


In this work, we present \textbf{\ours}, a family of rectified flow~\citep{lipman2023flow, liu2023flow} transformer models designed for joint image and video generation, establishing a pathway toward industry-grade performance. This report centers on four key components: data curation, model architecture design, flow formulation, and training infrastructure optimization—each rigorously refined to meet the demands of high-quality, large-scale video generation.


\begin{figure}[ht]
    \centering
    \begin{subfigure}[b]{0.82\linewidth}
        \centering
        \includegraphics[width=\linewidth]{figures/t2i_1024.pdf}
        \caption{Text-to-Image Samples}\label{fig:main-demo-t2i}
    \end{subfigure}
    \vfill
    \begin{subfigure}[b]{0.82\linewidth}
        \centering
        \includegraphics[width=\linewidth]{figures/t2v_samples.pdf}
        \caption{Text-to-Video Samples}\label{fig:main-demo-t2v}
    \end{subfigure}
\caption{\textbf{Generated samples from \ours.} Key components are highlighted in \textcolor{red}{\textbf{RED}}.}\label{fig:main-demo}
\end{figure}


First, we present a comprehensive data processing pipeline designed to construct large-scale, high-quality image and video-text datasets. The pipeline integrates multiple advanced techniques, including video and image filtering based on aesthetic scores, OCR-driven content analysis, and subjective evaluations, to ensure exceptional visual and contextual quality. Furthermore, we employ multimodal large language models~(MLLMs)~\citep{yuan2025tarsier2} to generate dense and contextually aligned captions, which are subsequently refined using an additional large language model~(LLM)~\citep{yang2024qwen2} to enhance their accuracy, fluency, and descriptive richness. As a result, we have curated a robust training dataset comprising approximately 36M video-text pairs and 160M image-text pairs, which are proven sufficient for training industry-level generative models.

Secondly, we take a pioneering step by applying rectified flow formulation~\citep{lipman2023flow} for joint image and video generation, implemented through the \ours model family, which comprises Transformer architectures with 2B and 8B parameters. At its core, the \ours framework employs a 3D joint image-video variational autoencoder (VAE) to compress image and video inputs into a shared latent space, facilitating unified representation. This shared latent space is coupled with a full-attention~\citep{vaswani2017attention} mechanism, enabling seamless joint training of image and video. This architecture delivers high-quality, coherent outputs across both images and videos, establishing a unified framework for visual generation tasks.


Furthermore, to support the training of \ours at scale, we have developed a robust infrastructure tailored for large-scale model training. Our approach incorporates advanced parallelism strategies~\citep{jacobs2023deepspeed, pytorch_fsdp} to manage memory efficiently during long-context training. Additionally, we employ ByteCheckpoint~\citep{wan2024bytecheckpoint} for high-performance checkpointing and integrate fault-tolerant mechanisms from MegaScale~\citep{jiang2024megascale} to ensure stability and scalability across large GPU clusters. These optimizations enable \ours to handle the computational and data challenges of generative modeling with exceptional efficiency and reliability.


We evaluate \ours on both text-to-image and text-to-video benchmarks to highlight its competitive advantages. For text-to-image generation, \ours-T2I demonstrates strong performance across multiple benchmarks, including T2I-CompBench~\citep{huang2023t2i-compbench}, GenEval~\citep{ghosh2024geneval}, and DPG-Bench~\citep{hu2024ella_dbgbench}, excelling in both visual quality and text-image alignment. In text-to-video benchmarks, \ours-T2V achieves state-of-the-art performance on the UCF-101~\citep{ucf101} zero-shot generation task. Additionally, \ours-T2V attains an impressive score of \textbf{84.85} on VBench~\citep{huang2024vbench}, securing the top position on the leaderboard (as of 2025-01-25) and surpassing several leading commercial text-to-video models. Qualitative results, illustrated in \Cref{fig:main-demo}, further demonstrate the superior quality of the generated media samples. These findings underscore \ours's effectiveness in multi-modal generation and its potential as a high-performing solution for both research and commercial applications.
\section{Background}
% \begin{tcolorbox}[simplebox]
% We first formally define the problem and highlight its challenge. 
% Then we present an EM approach to address this challenge. 
% \end{tcolorbox}
% \vspace{-0.3cm}
% \subsection{Problem Statement }\label{sec_ps}

% Here’s a polished and enriched version of your problem formulation section, with improved clarity, precision, and academic tone:

% ---
\begin{figure}[t]
    \centering % Center the figure
    \includegraphics[width=\linewidth]{figs/example.pdf} % Include the figure
    \caption{\small \textbf{Example of Autonomous Code Integration.} \small We aim to enable LLMs to determine tool-usage strategies
based on their own capability boundaries. In the example, the model write code to solve the problem that demand special tricks, strategically bypassing its inherent limitations.} 
    \label{fig_example}
    \vspace{-0.2cm}
\end{figure}
\textbf{Problem Statement.} Modern tool-augmented language models address mathematical problems \( x_q \in \mathcal{X}_Q \) by generating step-by-step solutions that interleave natural language reasoning with executable Python code (Fig.~\ref{fig_example}). Formally, given a problem \( x_q \), a model \( \mathcal{M}_\theta \) iteratively constructs a solution \( y_a = \{y_1, \dots, y_T\} \) by sampling components \( y_t \sim p(y_t | y_{<t}, x_q) \), where \( y_{<t} \) encompasses both prior reasoning steps, code snippets and execution results \( \mathbf{e}_t \) from a Python interpreter. The process terminates upon generating an end token, and the solution is evaluated via a binary reward \( r(y_a,x_q) = \mathbb{I}(y_a \equiv y^*) \) indicating equivalence to the ground truth \( y^* \). The learning objective is formulated as:
\[
\max_{\theta} \mathbb{E}_{x_q \sim \mathcal{X}_Q} \left[r(y_a, x_q) \right]
\]

\noindent\textbf{Challenge and Motivation.} Developing autonomous code integration (AutoCode) strategies poses unique challenges, as optimal tool-usage behaviors must dynamically adapt to a model's intrinsic capabilities and problem-solving contexts. While traditional supervised fine-tuning (SFT) relies on imitation learning from expert demonstrations, this paradigm fundamentally limits the emergence of self-directed tool-usage strategies. Unfortunately, current math LLMs predominantly employ SFT to orchestrate tool integration~\citep{mammoth, tora, dsmath, htl}, their rigid adherence to predefined reasoning templates therefore struggles with the dynamic interplay between a model’s evolving problem-solving competencies and the adaptive tool-usage strategies required for diverse mathematical contexts.

Reinforcement learning (RL) offers a promising alternative by enabling trial-and-error discovery of autonomous behaviors. Recent work like DeepSeek-R1~\citep{dsr1} demonstrates RL's potential to enhance reasoning without expert demonstrations. However, we observe that standard RL methods (e.g., PPO~\cite{ppo}) suffer from a critical inefficiency (see Sec.~\ref{sec_ablation}): Their tendency to exploit local policy neighborhoods leads to insufficient exploration of the vast combinatorial space of code-integrated reasoning paths, especially when only given a terminal reward in mathematical problem-solving.

To bridge this gap, we draw inspiration from human metacognition -- the iterative process where learners refine tool-use strategies through deliberate exploration, outcome analysis, and belief updates. A novice might initially attempt manual root-finding via algebraic methods, observe computational bottlenecks or inaccuracies, and therefore prompting the usage of calculators. Through systematic reflection on these experiences, they internalize the contextual efficacy of external tools, gradually forming stable heuristics that balance reasoning with judicious tool invocation. 


To this end, \emph{our focus diverges from standard agentic tool-use frameworks~\citep{agentr}}, which merely prioritize successful tool execution. Instead, \emph{we aim to instill \emph{human-like metacognition} in LLMs, enabling them to (1) determine tool-usage based on their own capability boundaries (see the analysis in Sec.~\ref{sec_ablation}), and (2) dynamically adapt tool-usage strategies as their reasoning abilities evolve (via our EM framework).}
% For instance, while an LLM might solve a combinatorics problem via CoT alone, it should autonomously invoke code for eigenvalue calculations in linear algebra where symbolic computations are error-prone. Achieving this requires models to \emph{jointly optimize} their reasoning and tool-integration policies in a mutually reinforcing manner.


% Mirroring this metacognitive cycle, we propose an Expectation-Maximization (EM) framework that allows LLMs to develop AutoCode strategies via guided exploration (the E-step) and self-refinement (the M-step).


% \vspace{-0.3cm}
\section{Methodology}

Inspired by human metacognitive processes, we introduce an Expectation-Maximization (EM) framework that trains LLMs for autonomous code integration (AutoCode) through alternations (Fig.~\ref{fig_overview}):

\begin{enumerate}[leftmargin=0.5cm,topsep=1pt,itemsep=0pt,parsep=0pt]
    \item \emph{Guided Exploration (E-step):} Identifies high-potential code-integrated solutions by systematically probing the model's inherent capabilities.
\item \emph{Self-Refinement (M-step):} Optimizes the model's tool-usage strategy and chain-of-thought reasoning using curated trajectories from the E-step.
\end{enumerate}


\begin{figure*}[t]
    \centering
    \includegraphics[width=\linewidth]{figs/overview.pdf}
    \caption{\small \textbf{Method Overview.} \small (Left) shows an overview for the EM framework, which alternates between finding a reference strategy for guided exploration (E-step) and off-policy RL (M-step). (Right) shows the data curation for guided exploration. We generate \(K\) rollouts, estimate values of code-triggering decisions and subsample the initial data with sampling weights per Eq.~\ref{eq_sampling}.}
    \label{fig_overview}
\end{figure*}

\subsection{The EM Framework for AutoCode}

A central challenge in AutoCode lies in the code triggering decisions, represented by the binary decision \(c \in \{0, 1\}\).  While supervised fine-tuning (SFT) suffers from missing ground truth for these decisions, standard reinforcement learning (RL) struggles with the combinatorial explosion of code-integrated reasoning paths. Our innovation bridges these approaches through systematic exploration of both code-enabled (\(c=1\)) and non-code (\(c=0\)) solution paths, constructing reference decisions for policy optimization.

We formalize this idea within a maximum likelihood estimation (MLE) framework. Let \( P (r=1 | x_q;\theta\) denote the probability of generating a correct response to query \( x_q \) under model \(\mathcal{M}_\theta\). Our objective becomes:
\begin{align}
    \mathcal{J}_{\mathrm{MLE}}(\theta) \doteq \log P(r=1 | x_q; \theta) \label{eq_mle}
\end{align}
This likelihood depends on two latent factors: (1) the code triggering decision \(\pi_\theta(c | x_q)\) and (2) the solution generation process \(\pi_\theta(y_a | x_q, c)\). Here, for notation-wise clarity, we consider  code-triggering decision at a solution's beginning (\( c\) following \(x_q\) immediately). We show generalization to mid-reasoning code integration in Sec.~\ref{sec_impl}.

The EM framework provides a principled way to optimize this MLE objective in the presence of latent variables~\cite{prml}. We derive the evidence lower bound (ELBO): \( \mathcal{J}_{\mathrm{ELBO}}(s, \theta) \doteq \)
\begin{align}
    % \mathcal{J}_{\mathrm{MLE}}(\theta) &
    % \ge 
    \mathbb{E}_{s(c | x_q)}\left[\log \frac{\pi_\theta(c | x_q) \cdot P(r=1 | c, x_q; \theta)}{s(c | x_q)}\right] 
    % \\
     \label{eq_elbo}
\end{align}
where \(s(c | x_q)\) serves as a surrogate distribution approximating optimal code triggering strategies. It is also considered as the reference decisions for code integration. 

\noindent\textbf{E-step: Guided Exploration}  computes the reference strategy \(s(c | x_q)\) by maximizing the ELBO, equivalent to minimizing the KL-divergence: \( \max_s \mathcal{J}_{\mathrm{ELBO}}(s, \theta) = \)
\begin{align}
     - \mathrm{D_{KL}}\left(s(c | x_q) \| P(r=1, c | x_q; \theta)\right) \label{eq_estep}
\end{align}

The reference strategy \(s(c | x_q)\) thus approximates the posterior distribution over code-triggering decisions \(c\) that maximize correctness, i.e., \(P(r=1, c | x_q; \theta)\).  Intuitively, it guides exploration by prioritizing decisions with high potential: if decision \(c\) is more likely to lead to correct solutions, the reference strategy assigns higher probability mass to it, providing guidance for the subsequent RL procedure.

\noindent\textbf{M-step: Self-Refinement } updates the model parameters \(\theta\) through a composite objective:
\begin{multline}
\max_\theta \mathcal{J}_{\mathrm{ELBO}}(s, \theta) =\mathbb{E}_{\substack{c \sim s(c|x_q) \\ y_a \sim \pi_\theta(y_a|x_q, c)}} \Big[ r(x_q, y_a) \Big] \\- \mathcal{CE}\Big(s(c|x_q) \,\|\, \pi_\theta(c|x_q)\Big)\label{eq_mstep}
\end{multline}
The first term implements reward-maximizing policy gradient updates for solution generation, while while the second aligns native code triggering with reference strategies through cross-entropy minimization (see Fig.~\ref{fig_overview} for an illustration of the optimization). This dual optimization jointly enhances both tool-usage policies and reasoning capabilities.



\subsection{Practical Implementation}\label{sec_impl}
In the above EM framework, we alternate between finding a reference strategy \( s \) for code-triggering decisions  in the E-step, and perform reinforcement learning under the guidance from \( s \) in the M-step. We implement this framework through an iterative process of offline data curation and off-policy RL.

\noindent\textbf{Offline Data Curation.} We implement the E-step through Monte Carlo rollouts and subsampling. For each problem \(x_q\), we estimate the reference strategy as an energy distribution: 
\begin{equation}
    s^\ast(c | x_q)  = \frac{\exp\left(\alpha\cdot \pi_\theta(c | x_q) Q(x_q,c;\theta)\right)}{Z(x_q)}.\label{eq_sampling}
\end{equation}
where \( Q(x_q,c;\theta)\) estimates the expected value through \( K \) rollouts per decision, \(\pi_\theta(c|x_q) \) represents the model's current prior and the \( Z(x_q) \) is the partition function to ensure normalization. Intuitively, the strategy will assign higher probability mass to the decision \( c \) that has higher expected value \( Q(x_q,c;\theta)\) meanwhile balancing its intrinsic preference \( \pi_\theta(c|x_q)\). 

Our curation pipeline proceeds through: 
\begin{itemize}[leftmargin=0.5cm,topsep=1pt,itemsep=0pt,parsep=0pt]
\item Generate \(K\) rollouts for \(c=0\) (pure reasoning) and \(c=1\) (code integration), creating candidate dataset \(\mathcal{D}\).  
\item Compute \(Q(x_q,c)\) as the expected success rate across rollouts for each pair \((x_q,c)\).  
\item Subsample \(\mathcal{D}_{\text{train}}\) from \(\mathcal{D}\) using importance weights according to Eq.~\ref{eq_sampling}.  
\end{itemize}

To explicitly probe code-integrated solutions, we employ prefix-guided generation -- e.g., prepending prompts like \texttt{``Let’s first analyze the problem, then consider if python code could help''} -- to bias generations toward free-form code-reasoning patterns.

 This pipeline enables guided exploration by focusing on high-potential code-integrated trajectories identified by the reference strategy, contrasting with standard RL’s reliance on local policy neighborhoods. As demonstrated in Sec.~\ref{sec_ablation}, this strategic data curation significantly improves training efficiency by shaping the exploration space.





\noindent\textbf{Off-Policy RL.}
To mitigate distributional shifts caused by mismatches between offline data and the policy, we optimize a clipped off-policy RL objective. The refined M-step (Eq.~\ref{eq_mstep}) becomes:
\begin{multline}
    % \max_\theta 
    \underset{(x_q,y_a)}{\mathbb{E}}\left[
\text{clip}\left(\frac{\pi_\theta(y_a|x_q)}{\pi_{\text{ref}}(y_a|x_q)},1-\epsilon,1+\epsilon\right)\cdot A\right]
\\-\mathbb{E}_{(x_q,c)}\Big[\log \pi_\theta(c|x_q) \Big]\label{eq_finalm}
\end{multline}
where  \( (x_q, c, y_a) \) is sampled from the dataset \( \mathcal{D}_{\text{train}} \). The importance weight \(\frac{\pi_\theta(y_a|x_q)}{\pi_{\text{ref}}(y_a|x_q)}\) accounts for off-policy correction with PPO-like clipping. The advantage function \(A(x_q,y_a)\) is computed via query-wise reward normalization~\cite{ppo}. 

\noindent\textbf{Generalizing to Mid-Reasoning Code Integration.} Our method extends to mid-reasoning code integration by initiating Monte Carlo rollouts from partial solutions \((x_q, y_{<t})\). Notably, we observe emergence of mid-reasoning code triggers after initial warm-up with prefix-probed solutions. Thus, our implementation requires only two initial probing strategies: explicit prefix prompting for code integration and vanilla generation for pure reasoning, which jointly seed diverse mid-reasoning code usage in later iterations.

\section{RESULTS}

\begin{figure*}[t!]
    %\vspace{-0.5cm}
    \centering
    \includegraphics[width=1\linewidth]{images/SystemArchitecture_2.png}
    \caption{From a single user demonstration, the system extracts the desired task goal state with the help of user interaction to solve ambiguities. Using the created environment variation, the system computes a task execution plan to bring new environments into the goal state. It sends the plan to agents in the environment to execute.} \label{fig:system_architecture}
\end{figure*}

Figure \ref{fig:system_architecture} shows our proposed framework to define a task goal, i.e. an environment goals state, and to turn a given environment into this goal state. The system visually observes a task execution by a user and segments this \underline{single} demonstration into \skills. \actions\ and \skills\ are defined in \ref{ssec:actions_skills}. The demonstration changed one or several properties of entities in the environment; environment which is now in the goal state. This information and the differences in entity properties from the start and end environment states are used to represent the task goal state. More on that in \ref{ssec:exp_model_def}. To turn a new environment into the defined goal state, a planning problem must be solved. This entails computing the differences between the environment's current state and the goal state, finding \actions\ that solve these differences, instantiating \skills\ that implement the \actions\ in the environment, selecting the \skills\ to execute by minimizing a given metric, and finally, sending the \skills\ to the agents in the environment to execute. This process is detailed in \ref{ssec:exp_model_use}.

% To prove the usability of our model, we present experiments to create a new goal state and turn the current environment into an (already-defined) goal state.

\subsection{Actions and Skills}\label{ssec:actions_skills}
A change in the environment is modeled using \actions, i.e. \textbf{what} has happened, and \skills, i.e. \textbf{how} did the change happen \cite{conceptHierarchyGeriatronicsSummit24}. Like in STRIPS \cite{strips} and PDDL \cite{pddl}, we represent \actions\ by their effects on entity properties and \skills\ by their preconditions and effects. \actions\ do not need preconditions because they only describe the \textbf{what} part of a change, not which conditions must be satisfied to perform the change. Besides preconditions and effects, \skills\ have a list of checks that tell our system if the \skill\ is executed in the environment. These checks allow the creation of a \skill\ recognition program, like the one presented in \cite{conceptHierarchyGeriatronicsSummit24}.
%\todo{citation of Geriatronics summit paper or the journal/unsubmitted paper?}
Using the \skill\ recognition output, we capture the changes from a task demonstration.

A \skill\ is thus the physical enactment of an abstract \action\ in an environment. Hence, \skills\ are correlated with \actions\ via their effects. A \skill\ can have more effects than a corresponding \action. For example, the \skill\ of scooping jam from a jar with a spoon implements the \action\ of \textit{TransferringContents}, but it also \textit{Dirties} the spoon.

\subsection{How To Parameterize The Model}\label{ssec:exp_model_def}
Creating a new goal state should be easier than manually specifying all variations wanted from the goal state. Doing so requires programming knowledge, which should not be needed to define goal states. One can let the system, which knows how to represent goal states, question the user about the desired state of the environment. However, this tedious process requires many questions from the system, also leading to decreased system usability.

Therefore, our approach is to let the user turn a given environment into a desired goal state and analyze the differences between the initial and final environment state to create the goal state representation. This single demonstration highlights the entity property values that were not in the desired goal state before being changed by the user.

We capture the demonstration via an Intel Realsense 3D camera \cite{realsense}, analyze the human skeleton via the OpenPose human pose estimation method \cite{openpose}, and determine the 3d pose of objects with AprilTag markers \cite{aprilTag}.

One demonstration contains the initial environment, not in the task goal state, and the final environment, in the goal state. The final environment state alone is not enough to create the environment variation. Thus, additional questions, guided by the differences between the two environment values, are posed by the system to the user to determine the desired variation in the environment state.

In a demonstration in which the user pours milk into a bowl, as shown in the top of Figure \ref{fig:system_architecture}, the initial question posed to the user is which entities that have changed properties are relevant for the goal state. If the goal state is to have more milk in the bowl, the milk carton is irrelevant; it is a means to achieve the goal state but not relevant to the goal itself. The bowl is thus selected as a relevant entity. 

Next, the list of relevant modified properties must also be determined for each relevant entity. It could have happened that during pouring of the milk into the bowl, the bowl's location also changed, e.g. touched accidentally by the user. Thus, not all modified properties could be relevant to the task. After selecting the relevant properties, the system knows from the knowledge base \cite{conceptHierarchyGeriatronicsSummit24} their \textit{ValueDomain} and the list of implemented \textbf{variations} for that \textit{ValueDomain}. Thus, the user parametrizes a selected \textbf{variation} from the list: choosing either a fixed value, a \textit{ValueDomain}-specific \textbf{RangeVariation} that must be parametrized, a conjunction or disjunction of \textbf{RangeVariations}, or the whole \textit{ValueDomain}.

In the example above, the user chooses the \textit{contentLevel} property as relevant. The system knows this property's defined set of values: a non-negative real number, and the possible range variation types: an open interval, a closed interval, an open-closed or closed-open interval, an intersection or union of intervals, etc. The user chooses a closed interval of $[0.28, 0.32]$ around the final \textit{contentLevel} value of $0.3L$. The user also specifies a variation for the entity's concept. It is generalized from that specific bowl instance to a \textit{LiquidContainer}.

After each modified property of each entity has a represented \textbf{variation}, the system automatically collects the entities into a variation of type $A$, see \ref{ssec:variations}, which is the assigned \textbf{variation} for the collection of entities in the environment.

Thus, the environment variation is determined in $\mathcal{O}\left(n\times m \times p\right)$ questions to the user, where $n$ is the number of entities in the environment, $m$ is the maximal number of properties that an entity can have, and $p$ is the maximal number of parameters that a \textbf{RangeVariation} needs to be represented. In the example above, $10$ questions were necessary to determine the task goal state shown in Figure \ref{fig:system_architecture} of a \textit{LiquidContainer} with \textit{contentLevel} between $0.28$ and $0.32L$. Figure \ref{fig:task_goal_state} shows the internal JSON-like representation of the goal state as the environment variation.
% 1 question which entities are relevant -> just bowl
% 1 question which properties are relevant -> contentLevel and concept
% 1 question about concept values being the same; should create variation?
% 1 question which ConceptValue-variation to select -> ConceptValue in Environment
% 1 question: which generalized concept?
% 1 question -> add other range-variation
% 1 question which Number-variation to select -> Interval
% 1 question: min-bound?
% 1 question: max-bound?
% 1 question -> add other range-variation

\begin{figure}[t!]
    %\vspace{-0.5cm}
    \centering
    \includegraphics[width=1\linewidth]{images/TaskDefinition_7.png}
    \caption{The goal state is a \textbf{RangeVariation} of the environment, of type EnvironmentDataRangeEntityVariation, which contains a \textbf{variation} of entities. This sub-variation is a \textbf{RangeVariation} of type MapRangeInstanceSubset (\textbf{variation} of type $A$, see \ref{ssec:variations}) and contains one instance \textbf{RangeVariation} of type InstanceRangePropertiesVariation. It defines the instance's concept \textbf{RangeVariation}, a \textit{LiquidContainer} to be found in the environment, and the \textit{contentLevel} property \textbf{RangeVariation}, the closed interval $\left[0.28, 0.32\right]$.} \label{fig:task_goal_state}
\end{figure}

\subsection{How To Use The Model}\label{ssec:exp_model_use}
Assuming the representation of a task's goal state is given, i.e. an environment variation, we detail our procedure (see Figure \ref{fig:experiment_description}) to turn the current environment into the goal state.

First, a Comparison between the environment and the goal variation is computed. This leads, as described in \ref{ssec:comparisons}, to a list of reasons why the environment is not in the variation. These reasons, i.e. differences $\delta$ of concept properties $p$, must be fixed to turn the environment into the goal state.

% Computing the differences between an EnvironmentData and an EnvironmentData-Variation, that has a Collection-Variation of type $A$, see \ref{ssec:variations}, is done via a maximal matching algorithm, where an edge between an entity $e$ an an entity variation $v_e$ means $e \in v_e$. 
For an EnvironmentData-Variation $v_{env}$ that defines a Collection-RangeVariation of type $A$, see \ref{ssec:variations}, computing the Comparison between an EnvironmentData $env$ and this target $v_{env}$ leads to a list of reasons for each entity $e_{env}$ in the entity collection of $env$, why $e_{env} \not\in v, \forall v \in A$. This can be seen in Figure \ref{fig:experiment_description}, where for each entity of \textit{LiquidContainer} concept in the environment, there is a list of differences, i.e. Comparisons, created for why the respective entity does not match the defined variation on the top-right.

\begin{figure}[t!]
    %\vspace{-0.5cm}
    \centering
    \includegraphics[width=1\linewidth]{images/Experiment_DescriptionUsingVariations_2.png}
    \caption{The procedure to turn an environment into its goal state is divided into 5 steps: computing differences, finding abstract solutions (i.e. \actions), computing practical solutions for the abstract ones (i.e. \actions\ $\rightarrow$ \skills), selecting the best practical solution, and executing the solution.} \label{fig:experiment_description}
\end{figure}
The second step of the procedure is to turn the list of differences into a list of \actions\ that can fix them. In notation, \action\ $A_x$ solves a difference in the concept property $p_x$. The system knows which properties \actions\ modify by analyzing the definition of their effects. Thus, \actions\ are created (parametrized) to fix the differences in entity properties.

% Because multiple instances can fit the instance variation, the third step is to match instances with the variations. Our matching optimization criterion is to minimize the amount of \textit{Actions} needed to fix the instances' property differences. \todo{continue!}

In the third step, each \action\ $A_x$ is converted into an execution plan $P_x$ that implements solving the difference $\delta_{p_x}$ in the environment. It is also possible that there is no possibility to implement the \action\ $A_x$ in the environment; this is represented as an execution plan $P_x = \emptyset$. An execution plan $P_x$ is otherwise, in its simplest form, a set of \skill\ alternatives $\left\{S_y\right\}$, where the \skill\ $S_y$ implements the \action\ $A_x$. There is the case to consider that the \skill\ $S_y$ has preconditions that are not met. And so, before executing the skill $S_y$, a different execution plan $P_{S_y}$ has to be computed and executed to allow the \skill\ $S_y$ to solve the property difference $\delta_{p_x}$. It is also possible that one single \skill\  $S_y$ is not enough to implement the \action\ $A_x$. Consider the case where the environment contains three cups with $0.1L$ of water, and the goal is to have one cup with $0.3L$ of content. One single \textit{Pouring} \skill\ is not enough to fulfill the goal; two \textit{Pouring} \skills\ must be executed. Thus, in the most general form, an execution plan $P_x = \left[\left\{ S_{iy}, P_{S_{iy}} \right\}_i\right]$ is a list of skill alternatives $\left\{ S_{iy}, P_{S_{iy}} \right\}_i$, that possibly contain other execution plans $P_{S_{iy}}$ to solve the skill's preconditions.

Our procedure to parameterize the \skills\ $S_y$ that implement the \action\ $A_x$ is a custom solution for each property $p_x$. One could backtrack through all possible parameter values of all possible skills to create a general solution that works for all properties. Another idea is to invert \skill\ effects and thus guide the \skill\ parameter search from the target variation to the value. However, both approaches would be computationally intense and would not create execution plans in a reasonable time. 
% reinforcement learning with policy for each property

The procedure to solve an entity $e$'s \underline{contentLevel} property difference searches for other \textit{Container} object instances in the environment, sorts them according to their content volume, and iterates through them in ascending order if $e.contentLevel \le target.contentLevel$; otherwise, in descending order. If a \skill\ $S$ can be executed with the two objects, that reduces the difference between $e.contentLevel$ and $target.contentLevel$, the \skill\ is added to the execution plan. If, after checking all objects, $e.contentLevel \not\in target.contentLevel$, there is no solution to solve this property difference.

Thus, the result of the third step is an execution plan $P_x$ for each entity property difference.

Fourth, after having the execution plans $P_x$ per entity-variation and entity, a \underline{solution selector} scores all solutions according to defined metrics and then, via a maximal matching algorithm, selects the solutions to execute to satisfy all variations of the Collection-RangeVariation of type $A$. The edges in the maximal matching have the cost of the solution score. For this paper, the scoring metric by the \underline{solution selector} is the number of steps of the execution plan.

The fifth and final step is to pass the execution plan to the agent(s) to execute in the environment. Figure \ref{fig:data_flow} presents the flow of data through the five steps.
We have used the Franka Emika Panda robot in CoppeliaSim \cite{coppeliaSim} to perform the computed execution plan.
% Note that the approach is independent of the used robot; only when instantiating \skills\ must the robot's abilities, manipulability region, and workspace be considered. How the \skills\ are executed in the environment is separated from the modeling of what must be done.

\begin{figure}[t!]
    % \vspace{-0.2cm}
    \centering
    \includegraphics[width=1\linewidth]{images/Experiment_DescriptionUsingVariations_DataFlow_2.png}
    \caption{Data flow when transforming an environment into a given goal state. $\Delta$ are differences of entity properties $p$, $A$ are \actions, $P$ is an execution plan and $S$ are \skills.} \label{fig:data_flow}
\end{figure}

The experiments aim to compute solution plans for solving the difference of the \textbf{contentLevel} property of \textit{Container} objects. For this, we consider the following criteria. $C1$: \textbf{variation} type = $\left\{\text{fixed},\text{interval},\text{interval union}\right\}$. $C2$: target relative to content = \{$\left\{t < cL \le cV \right\}$, $\left\{cL < t < cV \right\}$, $\left\{cL < t \ni cV \right\}$, $\left\{cL \le cV < t \right\}$\}, where $t$ is the \textbf{variation} value and $cL$ and $cV$ are the \textit{contentLevel} and \textit{contentVolume} properties respectively. $C3$: achievable in environment $ = \left\{\text{yes}, \text{no}\right\}$. Figure \ref{fig:experiment_table} presents planning results for different environments and the criteria described above. The lower table shows cases where the computed solution does not match the actual solution. This only happens when multiple instance variations are defined. The reason is that the implemented procedure to turn the list of differences into an execution plan treats each difference independently. Thus, dependencies between two variations are not accurately solved.

In the upper table of Figure \ref{fig:experiment_table}, there are two solutions for $C1.3$, $C2.3$, $C3.1$: one with the bowl $B$ as the instance in the \textbf{variation} $V1$, the other with $M$. The solution when $B$ is the matched instance has three steps: 1) pouring $0.1L$ from $M$ into $B$, 2) pouring $0.1L$ from $C1$ into $B$, and, finally, 3) pouring  $0.02L$ from $C2$ into $B$. This plan is sent to the robot in simulation and is executed as shown in Figure \ref{fig:robot_plan_execution}.

% \begin{figure}[t!]
%     % \vspace{-0.2cm}
%     \centering
%     \includegraphics[width=1\linewidth]{images/Experiment_Table_1Variation_compressed.png}
%     \caption{$B$ is a bowl with $0.5L$ \textit{contentVolume}, $M$ is a milk carton with $1.0L$ \textit{contentVolume}, $C1$ and $C2$ are cups with $0.3L$ \textit{contentVolume} each. Times, in seconds, averaged across 10 runs. Criteria $C2.4$ and $C3.1$ are mutually exclusive (a solution does not exist to let a container have more \textit{contentLevel} than its \textit{contentVolume}); thus, they are not included in the table.} \label{fig:experiment_table}
% \end{figure}
\begin{figure}[t!]
    % \vspace{-0.2cm}
    \centering
    \includegraphics[width=1\linewidth]{images/Experiment_Table_Results.png}
    \caption{$B$ is a bowl with $0.5L$ \textit{contentVolume}, $M$ is a milk carton with $1.0L$ \textit{contentVolume}, $C1$ and $C2$ are cups with $0.3L$ \textit{contentVolume} each. Times, in seconds, averaged across 10 runs. Criteria $C2.4$ and $C3.1$ are mutually exclusive (a solution does not exist to let a container have more \textit{contentLevel} than its \textit{contentVolume}); thus, they are not included in the upper table. The lower table presents results for open intervals and multiple variations in the environment.} \label{fig:experiment_table}
\end{figure}

\begin{figure}[t!]
    %\vspace{-0.1cm}
    \centering
    \includegraphics[width=1\linewidth]{images/Robot_PouringInBowl_M_PC1_PC2.png}
    \caption{Robot executing plan to bring $B$, the bowl, into the goal state. Because no liquids were simulated, the pouring amount was associated with the pouring time via: $t_{pour} = 10 * amount_{pour}$.} \label{fig:robot_plan_execution}
\end{figure}
\section{Related Work}
\paragraph{Pre-training for Open-domain QA.}
Open-domain QA requires retrieving relevant passages and extracting answers from them. This necessity has driven various methods that jointly train retrievers and readers. REALM~\citep{realm}, RAG~\citep{rag}, EMDR2~\citep{sachan2021endtoend}, YONO~\citep{lee-etal-2022-need}, ReAtt~\citep{jiang-etal-2022-retrieval}, and Atlas~\citep{reatt} first pre-train retrievers or initialise from pre-trained~\citep{izacard2022unsupervised} and fine-tuned retrievers. Subsequently, both components are fine-tuned jointly: the reader is trained using an answer generation loss, and the retriever is trained to promote passages that increase the likelihood of generating correct answers. Recently, this joint training mechanism has been adapted for multilingual open-domain QA~\citep{jiang-etal-2024-pre}, where retrievers are initially trained by learning from English teachers using multilingual parallel data, followed by a joint training stage with query-answer pairs generated by LLMs. Our approach follows this joint training paradigm for model pre-training but differs significantly. We use WikiData as a source to generate more informative natural questions and answers. Additionally, our pre-training method is more efficient by eliminating knowledge distillation from English models.

\paragraph{LLMs for Few-shot Data Generation.}
Prompting LLMs to generate synthetic data has been widely adopted to improve the performance of retrieval and QA tasks. UPR~\citep{sachan-etal-2022-improving} and InPars~\citep{bonifacio2022inpars} use zero-shot or few-shot prompting for passage reranking. \textsc{Promptagator}~\citep{dai2023promptagator} and SWIM-X~\citep{thakur-etal-2024-leveraging} prompt LLMs with few-shot examples to generate massive synthetic queries, either in English or in multiple languages, for retriever fine-tuning. Gecko~\citep{lee2024gecko} prompts LLMs to generate synthetic instructions and queries from Web documents and create high-quality labels for retriever fine-tuning. Beyond retrieval, LLMs are employed to generate QA data, where \textsc{QAmeleon}~\citep{qameleon} prompts a 540B LLM to generate multilingual QA pairs from only five examples. Nevertheless, these methods primarily focus on retrieval tasks and the more narrowly defined machine reading comprehension tasks. In our work, we rigorously investigate how LLMs can improve the more challenging multilingual open-domain QA tasks under few-shot settings. In addition, we explore zero-shot prompting, demonstrating that cross-lingual prompting using English data or limited multilingual data from held-out languages can yield results comparable to few-shot prompting, and we show this technique can also be leveraged for effective zero-shot language adaptation.


\section{Conclusion and Limitation}
In this work, we propose \ours, a \emph{few-shot learning} approach for multilingual open-domain retrieval tasks. We present a novel self-supervised pre-training framework that exploits WikiData to effectively initialise both multilingual retrieval and QA capabilities. This process is followed by few-shot synthetic multilingual QA generation from LLMs using only five human-annotated examples. We demonstrate that the resulting model achieves competitive multilingual retrieval and QA performance through fine-tuning on the high-quality synthetic data. We further show that this few-shot approach generalises to zero-shot settings that only require English-supervised data. This mechanism serves as an effective approach for language adaptation, enabling the adapted model to achieve both boosted retrieval and end-to-end QA performance across fifteen previously unseen languages.

This work uses LLMs for synthetic data generation, which may propagate undesirable biases to generated data. We believe such biases will not be amplified as we sample prompts from \textsc{Xor-TyDi QA}, a dataset annotated with strict guidelines. Our preliminary safety analysis also reveals that only less than 1\% data contains potentially harmful queries, as identified by \texttt{Llama-Guard-2}.

\section*{Acknowledgement}
We thank the action editor Shay Cohen and anonymous reviewers for their helpful feedback and suggestions. The first author is supported by the Graduate Research Scholarships funded by the University of Melbourne. This work was funded by the Australian Research Council, Discovery grant DP230102775.
% This research was undertaken using the LIEF HPCGPGPU Facility hosted at the University of Melbourne, which was established with the assistance of LIEF Grant LE170100200.

\bibliography{tacl2021}
\bibliographystyle{acl_natbib}

\clearpage

\appendix
\subsection{Lloyd-Max Algorithm}
\label{subsec:Lloyd-Max}
For a given quantization bitwidth $B$ and an operand $\bm{X}$, the Lloyd-Max algorithm finds $2^B$ quantization levels $\{\hat{x}_i\}_{i=1}^{2^B}$ such that quantizing $\bm{X}$ by rounding each scalar in $\bm{X}$ to the nearest quantization level minimizes the quantization MSE. 

The algorithm starts with an initial guess of quantization levels and then iteratively computes quantization thresholds $\{\tau_i\}_{i=1}^{2^B-1}$ and updates quantization levels $\{\hat{x}_i\}_{i=1}^{2^B}$. Specifically, at iteration $n$, thresholds are set to the midpoints of the previous iteration's levels:
\begin{align*}
    \tau_i^{(n)}=\frac{\hat{x}_i^{(n-1)}+\hat{x}_{i+1}^{(n-1)}}2 \text{ for } i=1\ldots 2^B-1
\end{align*}
Subsequently, the quantization levels are re-computed as conditional means of the data regions defined by the new thresholds:
\begin{align*}
    \hat{x}_i^{(n)}=\mathbb{E}\left[ \bm{X} \big| \bm{X}\in [\tau_{i-1}^{(n)},\tau_i^{(n)}] \right] \text{ for } i=1\ldots 2^B
\end{align*}
where to satisfy boundary conditions we have $\tau_0=-\infty$ and $\tau_{2^B}=\infty$. The algorithm iterates the above steps until convergence.

Figure \ref{fig:lm_quant} compares the quantization levels of a $7$-bit floating point (E3M3) quantizer (left) to a $7$-bit Lloyd-Max quantizer (right) when quantizing a layer of weights from the GPT3-126M model at a per-tensor granularity. As shown, the Lloyd-Max quantizer achieves substantially lower quantization MSE. Further, Table \ref{tab:FP7_vs_LM7} shows the superior perplexity achieved by Lloyd-Max quantizers for bitwidths of $7$, $6$ and $5$. The difference between the quantizers is clear at 5 bits, where per-tensor FP quantization incurs a drastic and unacceptable increase in perplexity, while Lloyd-Max quantization incurs a much smaller increase. Nevertheless, we note that even the optimal Lloyd-Max quantizer incurs a notable ($\sim 1.5$) increase in perplexity due to the coarse granularity of quantization. 

\begin{figure}[h]
  \centering
  \includegraphics[width=0.7\linewidth]{sections/figures/LM7_FP7.pdf}
  \caption{\small Quantization levels and the corresponding quantization MSE of Floating Point (left) vs Lloyd-Max (right) Quantizers for a layer of weights in the GPT3-126M model.}
  \label{fig:lm_quant}
\end{figure}

\begin{table}[h]\scriptsize
\begin{center}
\caption{\label{tab:FP7_vs_LM7} \small Comparing perplexity (lower is better) achieved by floating point quantizers and Lloyd-Max quantizers on a GPT3-126M model for the Wikitext-103 dataset.}
\begin{tabular}{c|cc|c}
\hline
 \multirow{2}{*}{\textbf{Bitwidth}} & \multicolumn{2}{|c|}{\textbf{Floating-Point Quantizer}} & \textbf{Lloyd-Max Quantizer} \\
 & Best Format & Wikitext-103 Perplexity & Wikitext-103 Perplexity \\
\hline
7 & E3M3 & 18.32 & 18.27 \\
6 & E3M2 & 19.07 & 18.51 \\
5 & E4M0 & 43.89 & 19.71 \\
\hline
\end{tabular}
\end{center}
\end{table}

\subsection{Proof of Local Optimality of LO-BCQ}
\label{subsec:lobcq_opt_proof}
For a given block $\bm{b}_j$, the quantization MSE during LO-BCQ can be empirically evaluated as $\frac{1}{L_b}\lVert \bm{b}_j- \bm{\hat{b}}_j\rVert^2_2$ where $\bm{\hat{b}}_j$ is computed from equation (\ref{eq:clustered_quantization_definition}) as $C_{f(\bm{b}_j)}(\bm{b}_j)$. Further, for a given block cluster $\mathcal{B}_i$, we compute the quantization MSE as $\frac{1}{|\mathcal{B}_{i}|}\sum_{\bm{b} \in \mathcal{B}_{i}} \frac{1}{L_b}\lVert \bm{b}- C_i^{(n)}(\bm{b})\rVert^2_2$. Therefore, at the end of iteration $n$, we evaluate the overall quantization MSE $J^{(n)}$ for a given operand $\bm{X}$ composed of $N_c$ block clusters as:
\begin{align*}
    \label{eq:mse_iter_n}
    J^{(n)} = \frac{1}{N_c} \sum_{i=1}^{N_c} \frac{1}{|\mathcal{B}_{i}^{(n)}|}\sum_{\bm{v} \in \mathcal{B}_{i}^{(n)}} \frac{1}{L_b}\lVert \bm{b}- B_i^{(n)}(\bm{b})\rVert^2_2
\end{align*}

At the end of iteration $n$, the codebooks are updated from $\mathcal{C}^{(n-1)}$ to $\mathcal{C}^{(n)}$. However, the mapping of a given vector $\bm{b}_j$ to quantizers $\mathcal{C}^{(n)}$ remains as  $f^{(n)}(\bm{b}_j)$. At the next iteration, during the vector clustering step, $f^{(n+1)}(\bm{b}_j)$ finds new mapping of $\bm{b}_j$ to updated codebooks $\mathcal{C}^{(n)}$ such that the quantization MSE over the candidate codebooks is minimized. Therefore, we obtain the following result for $\bm{b}_j$:
\begin{align*}
\frac{1}{L_b}\lVert \bm{b}_j - C_{f^{(n+1)}(\bm{b}_j)}^{(n)}(\bm{b}_j)\rVert^2_2 \le \frac{1}{L_b}\lVert \bm{b}_j - C_{f^{(n)}(\bm{b}_j)}^{(n)}(\bm{b}_j)\rVert^2_2
\end{align*}

That is, quantizing $\bm{b}_j$ at the end of the block clustering step of iteration $n+1$ results in lower quantization MSE compared to quantizing at the end of iteration $n$. Since this is true for all $\bm{b} \in \bm{X}$, we assert the following:
\begin{equation}
\begin{split}
\label{eq:mse_ineq_1}
    \tilde{J}^{(n+1)} &= \frac{1}{N_c} \sum_{i=1}^{N_c} \frac{1}{|\mathcal{B}_{i}^{(n+1)}|}\sum_{\bm{b} \in \mathcal{B}_{i}^{(n+1)}} \frac{1}{L_b}\lVert \bm{b} - C_i^{(n)}(b)\rVert^2_2 \le J^{(n)}
\end{split}
\end{equation}
where $\tilde{J}^{(n+1)}$ is the the quantization MSE after the vector clustering step at iteration $n+1$.

Next, during the codebook update step (\ref{eq:quantizers_update}) at iteration $n+1$, the per-cluster codebooks $\mathcal{C}^{(n)}$ are updated to $\mathcal{C}^{(n+1)}$ by invoking the Lloyd-Max algorithm \citep{Lloyd}. We know that for any given value distribution, the Lloyd-Max algorithm minimizes the quantization MSE. Therefore, for a given vector cluster $\mathcal{B}_i$ we obtain the following result:

\begin{equation}
    \frac{1}{|\mathcal{B}_{i}^{(n+1)}|}\sum_{\bm{b} \in \mathcal{B}_{i}^{(n+1)}} \frac{1}{L_b}\lVert \bm{b}- C_i^{(n+1)}(\bm{b})\rVert^2_2 \le \frac{1}{|\mathcal{B}_{i}^{(n+1)}|}\sum_{\bm{b} \in \mathcal{B}_{i}^{(n+1)}} \frac{1}{L_b}\lVert \bm{b}- C_i^{(n)}(\bm{b})\rVert^2_2
\end{equation}

The above equation states that quantizing the given block cluster $\mathcal{B}_i$ after updating the associated codebook from $C_i^{(n)}$ to $C_i^{(n+1)}$ results in lower quantization MSE. Since this is true for all the block clusters, we derive the following result: 
\begin{equation}
\begin{split}
\label{eq:mse_ineq_2}
     J^{(n+1)} &= \frac{1}{N_c} \sum_{i=1}^{N_c} \frac{1}{|\mathcal{B}_{i}^{(n+1)}|}\sum_{\bm{b} \in \mathcal{B}_{i}^{(n+1)}} \frac{1}{L_b}\lVert \bm{b}- C_i^{(n+1)}(\bm{b})\rVert^2_2  \le \tilde{J}^{(n+1)}   
\end{split}
\end{equation}

Following (\ref{eq:mse_ineq_1}) and (\ref{eq:mse_ineq_2}), we find that the quantization MSE is non-increasing for each iteration, that is, $J^{(1)} \ge J^{(2)} \ge J^{(3)} \ge \ldots \ge J^{(M)}$ where $M$ is the maximum number of iterations. 
%Therefore, we can say that if the algorithm converges, then it must be that it has converged to a local minimum. 
\hfill $\blacksquare$


\begin{figure}
    \begin{center}
    \includegraphics[width=0.5\textwidth]{sections//figures/mse_vs_iter.pdf}
    \end{center}
    \caption{\small NMSE vs iterations during LO-BCQ compared to other block quantization proposals}
    \label{fig:nmse_vs_iter}
\end{figure}

Figure \ref{fig:nmse_vs_iter} shows the empirical convergence of LO-BCQ across several block lengths and number of codebooks. Also, the MSE achieved by LO-BCQ is compared to baselines such as MXFP and VSQ. As shown, LO-BCQ converges to a lower MSE than the baselines. Further, we achieve better convergence for larger number of codebooks ($N_c$) and for a smaller block length ($L_b$), both of which increase the bitwidth of BCQ (see Eq \ref{eq:bitwidth_bcq}).


\subsection{Additional Accuracy Results}
%Table \ref{tab:lobcq_config} lists the various LOBCQ configurations and their corresponding bitwidths.
\begin{table}
\setlength{\tabcolsep}{4.75pt}
\begin{center}
\caption{\label{tab:lobcq_config} Various LO-BCQ configurations and their bitwidths.}
\begin{tabular}{|c||c|c|c|c||c|c||c|} 
\hline
 & \multicolumn{4}{|c||}{$L_b=8$} & \multicolumn{2}{|c||}{$L_b=4$} & $L_b=2$ \\
 \hline
 \backslashbox{$L_A$\kern-1em}{\kern-1em$N_c$} & 2 & 4 & 8 & 16 & 2 & 4 & 2 \\
 \hline
 64 & 4.25 & 4.375 & 4.5 & 4.625 & 4.375 & 4.625 & 4.625\\
 \hline
 32 & 4.375 & 4.5 & 4.625& 4.75 & 4.5 & 4.75 & 4.75 \\
 \hline
 16 & 4.625 & 4.75& 4.875 & 5 & 4.75 & 5 & 5 \\
 \hline
\end{tabular}
\end{center}
\end{table}

%\subsection{Perplexity achieved by various LO-BCQ configurations on Wikitext-103 dataset}

\begin{table} \centering
\begin{tabular}{|c||c|c|c|c||c|c||c|} 
\hline
 $L_b \rightarrow$& \multicolumn{4}{c||}{8} & \multicolumn{2}{c||}{4} & 2\\
 \hline
 \backslashbox{$L_A$\kern-1em}{\kern-1em$N_c$} & 2 & 4 & 8 & 16 & 2 & 4 & 2  \\
 %$N_c \rightarrow$ & 2 & 4 & 8 & 16 & 2 & 4 & 2 \\
 \hline
 \hline
 \multicolumn{8}{c}{GPT3-1.3B (FP32 PPL = 9.98)} \\ 
 \hline
 \hline
 64 & 10.40 & 10.23 & 10.17 & 10.15 &  10.28 & 10.18 & 10.19 \\
 \hline
 32 & 10.25 & 10.20 & 10.15 & 10.12 &  10.23 & 10.17 & 10.17 \\
 \hline
 16 & 10.22 & 10.16 & 10.10 & 10.09 &  10.21 & 10.14 & 10.16 \\
 \hline
  \hline
 \multicolumn{8}{c}{GPT3-8B (FP32 PPL = 7.38)} \\ 
 \hline
 \hline
 64 & 7.61 & 7.52 & 7.48 &  7.47 &  7.55 &  7.49 & 7.50 \\
 \hline
 32 & 7.52 & 7.50 & 7.46 &  7.45 &  7.52 &  7.48 & 7.48  \\
 \hline
 16 & 7.51 & 7.48 & 7.44 &  7.44 &  7.51 &  7.49 & 7.47  \\
 \hline
\end{tabular}
\caption{\label{tab:ppl_gpt3_abalation} Wikitext-103 perplexity across GPT3-1.3B and 8B models.}
\end{table}

\begin{table} \centering
\begin{tabular}{|c||c|c|c|c||} 
\hline
 $L_b \rightarrow$& \multicolumn{4}{c||}{8}\\
 \hline
 \backslashbox{$L_A$\kern-1em}{\kern-1em$N_c$} & 2 & 4 & 8 & 16 \\
 %$N_c \rightarrow$ & 2 & 4 & 8 & 16 & 2 & 4 & 2 \\
 \hline
 \hline
 \multicolumn{5}{|c|}{Llama2-7B (FP32 PPL = 5.06)} \\ 
 \hline
 \hline
 64 & 5.31 & 5.26 & 5.19 & 5.18  \\
 \hline
 32 & 5.23 & 5.25 & 5.18 & 5.15  \\
 \hline
 16 & 5.23 & 5.19 & 5.16 & 5.14  \\
 \hline
 \multicolumn{5}{|c|}{Nemotron4-15B (FP32 PPL = 5.87)} \\ 
 \hline
 \hline
 64  & 6.3 & 6.20 & 6.13 & 6.08  \\
 \hline
 32  & 6.24 & 6.12 & 6.07 & 6.03  \\
 \hline
 16  & 6.12 & 6.14 & 6.04 & 6.02  \\
 \hline
 \multicolumn{5}{|c|}{Nemotron4-340B (FP32 PPL = 3.48)} \\ 
 \hline
 \hline
 64 & 3.67 & 3.62 & 3.60 & 3.59 \\
 \hline
 32 & 3.63 & 3.61 & 3.59 & 3.56 \\
 \hline
 16 & 3.61 & 3.58 & 3.57 & 3.55 \\
 \hline
\end{tabular}
\caption{\label{tab:ppl_llama7B_nemo15B} Wikitext-103 perplexity compared to FP32 baseline in Llama2-7B and Nemotron4-15B, 340B models}
\end{table}

%\subsection{Perplexity achieved by various LO-BCQ configurations on MMLU dataset}


\begin{table} \centering
\begin{tabular}{|c||c|c|c|c||c|c|c|c|} 
\hline
 $L_b \rightarrow$& \multicolumn{4}{c||}{8} & \multicolumn{4}{c||}{8}\\
 \hline
 \backslashbox{$L_A$\kern-1em}{\kern-1em$N_c$} & 2 & 4 & 8 & 16 & 2 & 4 & 8 & 16  \\
 %$N_c \rightarrow$ & 2 & 4 & 8 & 16 & 2 & 4 & 2 \\
 \hline
 \hline
 \multicolumn{5}{|c|}{Llama2-7B (FP32 Accuracy = 45.8\%)} & \multicolumn{4}{|c|}{Llama2-70B (FP32 Accuracy = 69.12\%)} \\ 
 \hline
 \hline
 64 & 43.9 & 43.4 & 43.9 & 44.9 & 68.07 & 68.27 & 68.17 & 68.75 \\
 \hline
 32 & 44.5 & 43.8 & 44.9 & 44.5 & 68.37 & 68.51 & 68.35 & 68.27  \\
 \hline
 16 & 43.9 & 42.7 & 44.9 & 45 & 68.12 & 68.77 & 68.31 & 68.59  \\
 \hline
 \hline
 \multicolumn{5}{|c|}{GPT3-22B (FP32 Accuracy = 38.75\%)} & \multicolumn{4}{|c|}{Nemotron4-15B (FP32 Accuracy = 64.3\%)} \\ 
 \hline
 \hline
 64 & 36.71 & 38.85 & 38.13 & 38.92 & 63.17 & 62.36 & 63.72 & 64.09 \\
 \hline
 32 & 37.95 & 38.69 & 39.45 & 38.34 & 64.05 & 62.30 & 63.8 & 64.33  \\
 \hline
 16 & 38.88 & 38.80 & 38.31 & 38.92 & 63.22 & 63.51 & 63.93 & 64.43  \\
 \hline
\end{tabular}
\caption{\label{tab:mmlu_abalation} Accuracy on MMLU dataset across GPT3-22B, Llama2-7B, 70B and Nemotron4-15B models.}
\end{table}


%\subsection{Perplexity achieved by various LO-BCQ configurations on LM evaluation harness}

\begin{table} \centering
\begin{tabular}{|c||c|c|c|c||c|c|c|c|} 
\hline
 $L_b \rightarrow$& \multicolumn{4}{c||}{8} & \multicolumn{4}{c||}{8}\\
 \hline
 \backslashbox{$L_A$\kern-1em}{\kern-1em$N_c$} & 2 & 4 & 8 & 16 & 2 & 4 & 8 & 16  \\
 %$N_c \rightarrow$ & 2 & 4 & 8 & 16 & 2 & 4 & 2 \\
 \hline
 \hline
 \multicolumn{5}{|c|}{Race (FP32 Accuracy = 37.51\%)} & \multicolumn{4}{|c|}{Boolq (FP32 Accuracy = 64.62\%)} \\ 
 \hline
 \hline
 64 & 36.94 & 37.13 & 36.27 & 37.13 & 63.73 & 62.26 & 63.49 & 63.36 \\
 \hline
 32 & 37.03 & 36.36 & 36.08 & 37.03 & 62.54 & 63.51 & 63.49 & 63.55  \\
 \hline
 16 & 37.03 & 37.03 & 36.46 & 37.03 & 61.1 & 63.79 & 63.58 & 63.33  \\
 \hline
 \hline
 \multicolumn{5}{|c|}{Winogrande (FP32 Accuracy = 58.01\%)} & \multicolumn{4}{|c|}{Piqa (FP32 Accuracy = 74.21\%)} \\ 
 \hline
 \hline
 64 & 58.17 & 57.22 & 57.85 & 58.33 & 73.01 & 73.07 & 73.07 & 72.80 \\
 \hline
 32 & 59.12 & 58.09 & 57.85 & 58.41 & 73.01 & 73.94 & 72.74 & 73.18  \\
 \hline
 16 & 57.93 & 58.88 & 57.93 & 58.56 & 73.94 & 72.80 & 73.01 & 73.94  \\
 \hline
\end{tabular}
\caption{\label{tab:mmlu_abalation} Accuracy on LM evaluation harness tasks on GPT3-1.3B model.}
\end{table}

\begin{table} \centering
\begin{tabular}{|c||c|c|c|c||c|c|c|c|} 
\hline
 $L_b \rightarrow$& \multicolumn{4}{c||}{8} & \multicolumn{4}{c||}{8}\\
 \hline
 \backslashbox{$L_A$\kern-1em}{\kern-1em$N_c$} & 2 & 4 & 8 & 16 & 2 & 4 & 8 & 16  \\
 %$N_c \rightarrow$ & 2 & 4 & 8 & 16 & 2 & 4 & 2 \\
 \hline
 \hline
 \multicolumn{5}{|c|}{Race (FP32 Accuracy = 41.34\%)} & \multicolumn{4}{|c|}{Boolq (FP32 Accuracy = 68.32\%)} \\ 
 \hline
 \hline
 64 & 40.48 & 40.10 & 39.43 & 39.90 & 69.20 & 68.41 & 69.45 & 68.56 \\
 \hline
 32 & 39.52 & 39.52 & 40.77 & 39.62 & 68.32 & 67.43 & 68.17 & 69.30  \\
 \hline
 16 & 39.81 & 39.71 & 39.90 & 40.38 & 68.10 & 66.33 & 69.51 & 69.42  \\
 \hline
 \hline
 \multicolumn{5}{|c|}{Winogrande (FP32 Accuracy = 67.88\%)} & \multicolumn{4}{|c|}{Piqa (FP32 Accuracy = 78.78\%)} \\ 
 \hline
 \hline
 64 & 66.85 & 66.61 & 67.72 & 67.88 & 77.31 & 77.42 & 77.75 & 77.64 \\
 \hline
 32 & 67.25 & 67.72 & 67.72 & 67.00 & 77.31 & 77.04 & 77.80 & 77.37  \\
 \hline
 16 & 68.11 & 68.90 & 67.88 & 67.48 & 77.37 & 78.13 & 78.13 & 77.69  \\
 \hline
\end{tabular}
\caption{\label{tab:mmlu_abalation} Accuracy on LM evaluation harness tasks on GPT3-8B model.}
\end{table}

\begin{table} \centering
\begin{tabular}{|c||c|c|c|c||c|c|c|c|} 
\hline
 $L_b \rightarrow$& \multicolumn{4}{c||}{8} & \multicolumn{4}{c||}{8}\\
 \hline
 \backslashbox{$L_A$\kern-1em}{\kern-1em$N_c$} & 2 & 4 & 8 & 16 & 2 & 4 & 8 & 16  \\
 %$N_c \rightarrow$ & 2 & 4 & 8 & 16 & 2 & 4 & 2 \\
 \hline
 \hline
 \multicolumn{5}{|c|}{Race (FP32 Accuracy = 40.67\%)} & \multicolumn{4}{|c|}{Boolq (FP32 Accuracy = 76.54\%)} \\ 
 \hline
 \hline
 64 & 40.48 & 40.10 & 39.43 & 39.90 & 75.41 & 75.11 & 77.09 & 75.66 \\
 \hline
 32 & 39.52 & 39.52 & 40.77 & 39.62 & 76.02 & 76.02 & 75.96 & 75.35  \\
 \hline
 16 & 39.81 & 39.71 & 39.90 & 40.38 & 75.05 & 73.82 & 75.72 & 76.09  \\
 \hline
 \hline
 \multicolumn{5}{|c|}{Winogrande (FP32 Accuracy = 70.64\%)} & \multicolumn{4}{|c|}{Piqa (FP32 Accuracy = 79.16\%)} \\ 
 \hline
 \hline
 64 & 69.14 & 70.17 & 70.17 & 70.56 & 78.24 & 79.00 & 78.62 & 78.73 \\
 \hline
 32 & 70.96 & 69.69 & 71.27 & 69.30 & 78.56 & 79.49 & 79.16 & 78.89  \\
 \hline
 16 & 71.03 & 69.53 & 69.69 & 70.40 & 78.13 & 79.16 & 79.00 & 79.00  \\
 \hline
\end{tabular}
\caption{\label{tab:mmlu_abalation} Accuracy on LM evaluation harness tasks on GPT3-22B model.}
\end{table}

\begin{table} \centering
\begin{tabular}{|c||c|c|c|c||c|c|c|c|} 
\hline
 $L_b \rightarrow$& \multicolumn{4}{c||}{8} & \multicolumn{4}{c||}{8}\\
 \hline
 \backslashbox{$L_A$\kern-1em}{\kern-1em$N_c$} & 2 & 4 & 8 & 16 & 2 & 4 & 8 & 16  \\
 %$N_c \rightarrow$ & 2 & 4 & 8 & 16 & 2 & 4 & 2 \\
 \hline
 \hline
 \multicolumn{5}{|c|}{Race (FP32 Accuracy = 44.4\%)} & \multicolumn{4}{|c|}{Boolq (FP32 Accuracy = 79.29\%)} \\ 
 \hline
 \hline
 64 & 42.49 & 42.51 & 42.58 & 43.45 & 77.58 & 77.37 & 77.43 & 78.1 \\
 \hline
 32 & 43.35 & 42.49 & 43.64 & 43.73 & 77.86 & 75.32 & 77.28 & 77.86  \\
 \hline
 16 & 44.21 & 44.21 & 43.64 & 42.97 & 78.65 & 77 & 76.94 & 77.98  \\
 \hline
 \hline
 \multicolumn{5}{|c|}{Winogrande (FP32 Accuracy = 69.38\%)} & \multicolumn{4}{|c|}{Piqa (FP32 Accuracy = 78.07\%)} \\ 
 \hline
 \hline
 64 & 68.9 & 68.43 & 69.77 & 68.19 & 77.09 & 76.82 & 77.09 & 77.86 \\
 \hline
 32 & 69.38 & 68.51 & 68.82 & 68.90 & 78.07 & 76.71 & 78.07 & 77.86  \\
 \hline
 16 & 69.53 & 67.09 & 69.38 & 68.90 & 77.37 & 77.8 & 77.91 & 77.69  \\
 \hline
\end{tabular}
\caption{\label{tab:mmlu_abalation} Accuracy on LM evaluation harness tasks on Llama2-7B model.}
\end{table}

\begin{table} \centering
\begin{tabular}{|c||c|c|c|c||c|c|c|c|} 
\hline
 $L_b \rightarrow$& \multicolumn{4}{c||}{8} & \multicolumn{4}{c||}{8}\\
 \hline
 \backslashbox{$L_A$\kern-1em}{\kern-1em$N_c$} & 2 & 4 & 8 & 16 & 2 & 4 & 8 & 16  \\
 %$N_c \rightarrow$ & 2 & 4 & 8 & 16 & 2 & 4 & 2 \\
 \hline
 \hline
 \multicolumn{5}{|c|}{Race (FP32 Accuracy = 48.8\%)} & \multicolumn{4}{|c|}{Boolq (FP32 Accuracy = 85.23\%)} \\ 
 \hline
 \hline
 64 & 49.00 & 49.00 & 49.28 & 48.71 & 82.82 & 84.28 & 84.03 & 84.25 \\
 \hline
 32 & 49.57 & 48.52 & 48.33 & 49.28 & 83.85 & 84.46 & 84.31 & 84.93  \\
 \hline
 16 & 49.85 & 49.09 & 49.28 & 48.99 & 85.11 & 84.46 & 84.61 & 83.94  \\
 \hline
 \hline
 \multicolumn{5}{|c|}{Winogrande (FP32 Accuracy = 79.95\%)} & \multicolumn{4}{|c|}{Piqa (FP32 Accuracy = 81.56\%)} \\ 
 \hline
 \hline
 64 & 78.77 & 78.45 & 78.37 & 79.16 & 81.45 & 80.69 & 81.45 & 81.5 \\
 \hline
 32 & 78.45 & 79.01 & 78.69 & 80.66 & 81.56 & 80.58 & 81.18 & 81.34  \\
 \hline
 16 & 79.95 & 79.56 & 79.79 & 79.72 & 81.28 & 81.66 & 81.28 & 80.96  \\
 \hline
\end{tabular}
\caption{\label{tab:mmlu_abalation} Accuracy on LM evaluation harness tasks on Llama2-70B model.}
\end{table}

%\section{MSE Studies}
%\textcolor{red}{TODO}


\subsection{Number Formats and Quantization Method}
\label{subsec:numFormats_quantMethod}
\subsubsection{Integer Format}
An $n$-bit signed integer (INT) is typically represented with a 2s-complement format \citep{yao2022zeroquant,xiao2023smoothquant,dai2021vsq}, where the most significant bit denotes the sign.

\subsubsection{Floating Point Format}
An $n$-bit signed floating point (FP) number $x$ comprises of a 1-bit sign ($x_{\mathrm{sign}}$), $B_m$-bit mantissa ($x_{\mathrm{mant}}$) and $B_e$-bit exponent ($x_{\mathrm{exp}}$) such that $B_m+B_e=n-1$. The associated constant exponent bias ($E_{\mathrm{bias}}$) is computed as $(2^{{B_e}-1}-1)$. We denote this format as $E_{B_e}M_{B_m}$.  

\subsubsection{Quantization Scheme}
\label{subsec:quant_method}
A quantization scheme dictates how a given unquantized tensor is converted to its quantized representation. We consider FP formats for the purpose of illustration. Given an unquantized tensor $\bm{X}$ and an FP format $E_{B_e}M_{B_m}$, we first, we compute the quantization scale factor $s_X$ that maps the maximum absolute value of $\bm{X}$ to the maximum quantization level of the $E_{B_e}M_{B_m}$ format as follows:
\begin{align}
\label{eq:sf}
    s_X = \frac{\mathrm{max}(|\bm{X}|)}{\mathrm{max}(E_{B_e}M_{B_m})}
\end{align}
In the above equation, $|\cdot|$ denotes the absolute value function.

Next, we scale $\bm{X}$ by $s_X$ and quantize it to $\hat{\bm{X}}$ by rounding it to the nearest quantization level of $E_{B_e}M_{B_m}$ as:

\begin{align}
\label{eq:tensor_quant}
    \hat{\bm{X}} = \text{round-to-nearest}\left(\frac{\bm{X}}{s_X}, E_{B_e}M_{B_m}\right)
\end{align}

We perform dynamic max-scaled quantization \citep{wu2020integer}, where the scale factor $s$ for activations is dynamically computed during runtime.

\subsection{Vector Scaled Quantization}
\begin{wrapfigure}{r}{0.35\linewidth}
  \centering
  \includegraphics[width=\linewidth]{sections/figures/vsquant.jpg}
  \caption{\small Vectorwise decomposition for per-vector scaled quantization (VSQ \citep{dai2021vsq}).}
  \label{fig:vsquant}
\end{wrapfigure}
During VSQ \citep{dai2021vsq}, the operand tensors are decomposed into 1D vectors in a hardware friendly manner as shown in Figure \ref{fig:vsquant}. Since the decomposed tensors are used as operands in matrix multiplications during inference, it is beneficial to perform this decomposition along the reduction dimension of the multiplication. The vectorwise quantization is performed similar to tensorwise quantization described in Equations \ref{eq:sf} and \ref{eq:tensor_quant}, where a scale factor $s_v$ is required for each vector $\bm{v}$ that maps the maximum absolute value of that vector to the maximum quantization level. While smaller vector lengths can lead to larger accuracy gains, the associated memory and computational overheads due to the per-vector scale factors increases. To alleviate these overheads, VSQ \citep{dai2021vsq} proposed a second level quantization of the per-vector scale factors to unsigned integers, while MX \citep{rouhani2023shared} quantizes them to integer powers of 2 (denoted as $2^{INT}$).

\subsubsection{MX Format}
The MX format proposed in \citep{rouhani2023microscaling} introduces the concept of sub-block shifting. For every two scalar elements of $b$-bits each, there is a shared exponent bit. The value of this exponent bit is determined through an empirical analysis that targets minimizing quantization MSE. We note that the FP format $E_{1}M_{b}$ is strictly better than MX from an accuracy perspective since it allocates a dedicated exponent bit to each scalar as opposed to sharing it across two scalars. Therefore, we conservatively bound the accuracy of a $b+2$-bit signed MX format with that of a $E_{1}M_{b}$ format in our comparisons. For instance, we use E1M2 format as a proxy for MX4.

\begin{figure}
    \centering
    \includegraphics[width=1\linewidth]{sections//figures/BlockFormats.pdf}
    \caption{\small Comparing LO-BCQ to MX format.}
    \label{fig:block_formats}
\end{figure}

Figure \ref{fig:block_formats} compares our $4$-bit LO-BCQ block format to MX \citep{rouhani2023microscaling}. As shown, both LO-BCQ and MX decompose a given operand tensor into block arrays and each block array into blocks. Similar to MX, we find that per-block quantization ($L_b < L_A$) leads to better accuracy due to increased flexibility. While MX achieves this through per-block $1$-bit micro-scales, we associate a dedicated codebook to each block through a per-block codebook selector. Further, MX quantizes the per-block array scale-factor to E8M0 format without per-tensor scaling. In contrast during LO-BCQ, we find that per-tensor scaling combined with quantization of per-block array scale-factor to E4M3 format results in superior inference accuracy across models. 


\end{document}


