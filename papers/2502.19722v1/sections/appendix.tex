% \section{Data Construction Details}
% \subsection{Generating Self-Supervised Questions with LLMs}\label{appendix:wikidataqg_llm}
% \paragraph{In-Context Examples Generation.}
% For each of our manually selected 50 properties, we sample 5 triples for each of them. Given a sampled fact triple, we prompt ChatGPT (\texttt{gpt-3.5-turbo}) to automatically generate questions.

% We additionally generate questions with \texttt{Yes/No} as the answer. It is straightforward to generate \texttt{Yes} questions. For \texttt{No} questions, we need to create false fact triples from existing triples. Specifically, we randomly replace the head or tail entity of a triple with the most similar Wikidata entity, and we ensure the perturbed triple is not a valid fact according to Wikidata. The prompts we used are shown in Table~\ref{tab:chatgpt_prompt}. 

% \paragraph{Full Question Generation.}
% After obtaining high-quality ICL examples, we generate questions for all sampled triples. Specifically, given a triple $(s,r,o)$, we randomly sample three ICL examples with the same relation type $r$ and use the prompt in Table~\ref{tab:icl_prompt} for question generation. We also generate \texttt{Yes/No} questions similarly.
% 
% \subsection{Few-shot Prompts}


\begin{table*}
\propertytsize
    \centering
    \begin{tabular}{l|l|l|l}
    \toprule
    Property ID & Description & Property ID & Description \\
    \midrule
    P264 & record label & P175 & performer \\
    P176 & manufacturer & P112 & founded by \\
    P127 & owned by & P840 & narrative location \\
    P495 & country of origin & P20 & place of death \\
    P407 & language of work or name & P582 & end time \\
    P69 & educated at & P159 & headquarters location \\
    P740 & location of formation & P17 & country \\
    P136 & genre & P800 & notable work \\
    P36 & capital & P570 & date of death \\
    P190 & twinned administrative body & P4552 & mountain range \\
    P915 & filming location & P3086 & speed limit \\
    P84 & architect & P2046 & area \\
    P569 & date of birth & P86 & composer \\
    P515 & phase of matter & P2048 & height \\
    P40 & child & P580 & start time \\
    P828 & has cause & P50 & author \\
    P2067 & mass & 108 & employer \\
    P170 & creator & P2049 & width \\
    P364 & original language of film or TV show & P277 & programmed in \\
    P276 & location & P413 & position played on team / speciality \\
    P131 & located in the administrative territorial entity & P26 & spouse \\
    P106 & occupation & P607 & conflict \\
    P942 & theme music & P571 & inception \\
    P6 & head of government & P19 & place of birth \\
    P1830 & owner of & P61 & discoverer or inventor \\
    \bottomrule
    \end{tabular}
    \caption{List of English properties used for generating \ourptdata. Note that we do not generate data for a property if it does not exist in the Wikidata of target languages.}
    \label{tab:wikidata_property}
\end{table*}


\begin{table*}
\promptsize
    \centering
    \begin{tabular}{p{0.95\linewidth}}
        \toprule
        \texttt{System:} \\
        \texttt{You are a powerful question generator that generates natural and fluent questions from Wikidata triples.} \\
        \texttt{User:} \\
        \texttt{Rewrite this triplet "(\begin{CJK}{UTF8}{min}テリー・ジャックス\end{CJK}, \begin{CJK}{UTF8}{min}出生地\end{CJK}, \begin{CJK}{UTF8}{min}ウィニペグ\end{CJK})" into three questions in natural Japanese whose answer is "\begin{CJK}{UTF8}{min}ウィニペグ\end{CJK}". The questions should be diverse and the verbs used should be different. The generated questions should be separated by "\textbackslash n".} \\
        \texttt{System:} \\
        \colorbox{blue!7}{\texttt{\begin{CJK}{UTF8}{min}テリー・ジャックスの出生地はどこですか?\end{CJK}}} \\
        \colorbox{blue!7}{\texttt{\begin{CJK}{UTF8}{min}テリー・ジャックスはどこで生まれましたか?\end{CJK}}} \\
        \colorbox{blue!7}{\texttt{\begin{CJK}{UTF8}{min}テリー・ジャックスが生まれた都市は何という名前ですか?\end{CJK}}} \\
        \bottomrule
    \end{tabular}
    \begin{tabular}{p{0.95\linewidth}}
        \toprule
        \texttt{System:} \\
        \texttt{You are a powerful question generator that generates natural and fluent questions from Wikidata triples.} \\
        \texttt{User:} \\
        \texttt{Rewrite this triplet "(\begin{CJK}{UTF8}{min}レアンダー・クラブ\end{CJK}, \begin{CJK}{UTF8}{min}結成場所\end{CJK}, \begin{CJK}{UTF8}{min}ロンドン\end{CJK})" into three questions in natural Japanese whose answer is "yes". The questions should be diverse and the verbs used should be different. All elements in the triplet should be included in the question. The generated questions should be separated by "\textbackslash n".} \\
        \texttt{System:} \\
        \colorbox{blue!7}{\texttt{\begin{CJK}{UTF8}{min}レアンダー・クラブはロンドンで結成されましたか?\end{CJK}}} \\
        \colorbox{blue!7}{\texttt{\begin{CJK}{UTF8}{min}ロンドンはレアンダー・クラブの結成場所ですか?\end{CJK}}} \\
        \colorbox{blue!7}{\texttt{\begin{CJK}{UTF8}{min}レアンダー・クラブの創立がロンドンで行われたのですか?\end{CJK}}} \\
        \bottomrule
    \end{tabular}
    \caption{Examples of using ChatGPT to generate questions from triples. We use the same prompt as \texttt{Yes} questions to generate \texttt{No} ones by sampling perturbed triples. \colorbox{blue!7}{Highlighted texts} indicate system outputs.}
    \label{tab:chatgpt_prompt}
\end{table*}

\begin{table*}
\promptsize
    \centering
    \begin{tabular}{p{0.95\linewidth}}
        \toprule
        \texttt{Triple: (\begin{CJK}{UTF8}{min}伝染性単核球症\end{CJK}, \begin{CJK}{UTF8}{min}原因\end{CJK}, \begin{CJK}{UTF8}{min}エプスタイン・バール・ウイルス\end{CJK})} \\
        \texttt{Question: \begin{CJK}{UTF8}{min}どのウイルスが伝染性単核球症を引き起こすことが知られていますか?\end{CJK}} \\
        \texttt{Answer: \begin{CJK}{UTF8}{min}エプスタイン・バール・ウイルス\end{CJK}} \\
        \\ 
        \texttt{Triple: (\begin{CJK}{UTF8}{min}東日本大震災による電力危機\end{CJK}, \begin{CJK}{UTF8}{min}原因\end{CJK}, \begin{CJK}{UTF8}{min}福島第一原子力発電所事故\end{CJK})} \\
        \texttt{Question: \begin{CJK}{UTF8}{min}東日本大震災後の電力危機を引き起こした出来事は何ですか?\end{CJK}} \\
        \texttt{Answer: \begin{CJK}{UTF8}{min}福島第一原子力発電所事故\end{CJK}} \\
        \\ 
        \texttt{Triple: (\begin{CJK}{UTF8}{min}脳死\end{CJK}, \begin{CJK}{UTF8}{min}原因\end{CJK}, \begin{CJK}{UTF8}{min}脳損傷\end{CJK})} \\
        \texttt{Question: \begin{CJK}{UTF8}{min}脳死を引き起こすものとして、主に何が挙げられますか?\end{CJK}} \\
        \texttt{Answer: \begin{CJK}{UTF8}{min}脳損傷\end{CJK}} \\
        \\
        \texttt{Triple: (\begin{CJK}{UTF8}{min}壊血病\end{CJK}, \begin{CJK}{UTF8}{min}原因\end{CJK}, \begin{CJK}{UTF8}{min}ビタミンC欠乏症\end{CJK})} \\
        \texttt{Question: \colorbox{blue!7}{\begin{CJK}{UTF8}{min}何が壊血病につながるのか\end{CJK}}} \\
        \colorbox{blue!7}{\texttt{Answer: \begin{CJK}{UTF8}{min}ビタミンC欠乏症\end{CJK}}} \\
        \bottomrule
    \end{tabular}
    \caption{An example of prompting Gemma-7B to generate questions with ICL examples from ChatGPT. \colorbox{blue!7}{Highlighted texts} indicate system outputs.}
    \label{tab:icl_prompt}
\end{table*}

\begin{table*}
    \promptsize
    \centering
    \begin{tabular}{p{0.97\linewidth}}
        \toprule
        Document: \begin{CJK}{UTF8}{min}アダム・スミスは、私たちが一般的に資本主義と呼ぶものの最初の理論家と見なされています。彼の1776年の著作『国富論』は、ある安定した商業システムと評価システムの中で、個人が生産を専門化することでより多くの収益を得るというインセンティブに応えるだろうと理論化しました。これらの個人は特定の国家の介入なしに自然に「その産業を生産物が最も価値あるものとなるように指示することができる」でしょう。\end{CJK} \\
        Question: \begin{CJK}{UTF8}{min}資本主義の提唱者は誰?\end{CJK} => Answer: \begin{CJK}{UTF8}{min}アダム・スミス\end{CJK} \\
        \\
        Document: \begin{CJK}{UTF8}{min}人気のあるフロントは、1936年5月3日の総選挙で、608議席中386議席を獲得しました。初めて、社会主義者がラディカル社会主義者よりも多くの議席を獲得し、社会主義者の指導者レオン・ブルムがフランス初の社会主義首相およびその職を保持する初のユダヤ人となりました。初めての人気のあるフロント内閣は、20人の社会主義者、13人のラディカル社会主義者、2人の社会主義共和党員で構成されており(共産主義者の閣僚はいませんでした)、初めて3人の女性が含まれていました(当時、フランスでは女性が投票することができませんでした)。\end{CJK} \\
        Question: \begin{CJK}{UTF8}{min}フランスではいつから女性権利大臣が存在する?\end{CJK} => \begin{CJK}{UTF8}{min}Answer: 1936年5月3日\end{CJK} \\
        \\
        Document: \begin{CJK}{UTF8}{min}ハンガリー王国は中央ヨーロッパに存在した君主国で、中世から20世紀にかけて存在しました(1000年から1946年まで、1918年から1920年を除く)。ハンガリー公国は約1000年にエステルゴムで最初の国王スティーブン1世が戴冠し、キリスト教王国として現れました。彼の家族(アールパード王朝)は300年にわたって王国を指導しました。\end{CJK} \\
        Question: \begin{CJK}{UTF8}{min}ハンガリー王国は何年間存在した?\end{CJK} => Answer: \begin{CJK}{UTF8}{min}1000年から1946年まで、1918年から1920年を除く\end{CJK} \\
        \\
        Document: \begin{CJK}{UTF8}{min}アントマルキと英国人は別々に死体解剖報告書を書き、それぞれがナポレオンが父親を殺した病気である内出血によって死んだと結論づけました。後に、ナポレオンの髪のサンプルから高濃度のヒ素が見つかったことに基づく別の理論では、ナポレオンがヒ素中毒で死んだとされています。ただし、後の研究でもナポレオンの幼少期や息子、ジョゼフィーヌの髪のサンプルからも高濃度のヒ素が見つかりました。19世紀には医薬品やヘアクリームなどの製品でヒ素が広く使われていました。2021年に国際チームの消化器病理学者による研究では、ナポレオンは胃がんで亡くなったと結論づけられました。\end{CJK} \\
        Question: \begin{CJK}{UTF8}{min}ナポレオンが死んだのはマラリアのせい?\end{CJK} => Answer: no \\
        \\
        Document: \begin{CJK}{UTF8}{min}フィリピンの1987年憲法は次のように述べています:「国と教会の分離は不可侵であるべきです。」(第II条第6節)、および、「宗教の設立を尊重する法律は制定されず、またその自由な行使を禁止する法律も制定されません。宗教の職業と礼拝の自由な行使と享受は、差別や優遇なしに永遠に許可されます。公民権や政治的権利の行使に宗教的試験は求められません。\end{CJK} \\
        Question: \begin{CJK}{UTF8}{min}フィリピンは政教分離を原則としている?\end{CJK} => Answer: yes \\        
        \bottomrule
    \end{tabular}
    \caption{Complete prompt for few-shot question answer generation from passages in target language.}
    \label{tab:il_prompt}
\end{table*}

\begin{table*}
    \promptsize
    \centering
    \begin{tabular}{p{0.97\linewidth}}
        \toprule
        Document: Adam Smith is considered the first theorist of what we commonly refer to as capitalism. His 1776 work, An Inquiry into the Nature and Causes of the Wealth of Nations, theorized that within a given stable system of commerce and evaluation, individuals would respond to the incentive of earning more by specializing their production. These individuals would naturally, without specific state intervention, "direct ... that industry in such a manner as its produce may be of the greatest value." \\
        Question: English: Who are the proponents of capitalism? => Japanese: \begin{CJK}{UTF8}{min}資本主義の提唱者は誰\end{CJK} \\
        Answer: English: Adam Smith => Japanese: \begin{CJK}{UTF8}{min}アダム・スミス\end{CJK} \\
        \\
        Document: The Popular Front won the general election of 3 May 1936, with 386 seats out of 608. For the first time, the Socialists won more seats than the Radical-Socialists, and the Socialist leader Léon Blum became the first Socialist Prime Minister of France as well as the first Jew to hold that office. The first Popular Front cabinet consisted of 20 Socialists, 13 Radical-Socialists and two Socialist Republicans (there were no Communist Ministers) and, for the first time, included three women (women were not able to vote in France at that time). \\
        Question: English: Since when does France have a Minister of Women's Rights? => Japanese: \begin{CJK}{UTF8}{min}フランスではいつから女性権利大臣が存在する?\end{CJK} \\
        Answer: English: 3 May 1936 => Japanese: \begin{CJK}{UTF8}{min}1936 年 5 月 3 日\end{CJK} \\
        \\
        Document: The Kingdom of Hungary was a monarchy in Central Europe that existed from the Middle Ages into the twentieth century (1000–1946 with the exception of 1918–1920). The Principality of Hungary emerged as a Christian kingdom upon the coronation of the first king Stephen I at Esztergom in about the year 1000; his family (the Árpád dynasty) led the monarchy for 300 years. \\
        Question: English: How many years did the Kingdom of Hungary exist? => Japanese: \begin{CJK}{UTF8}{min}ハンガリー王国は何年間存在した\end{CJK} \\
        Answer: English: 1000 -- 1946 with the exception of 1918–1920 => Japanese: \begin{CJK}{UTF8}{min}1000年から1946年まで、1918年から1920年を除く\end{CJK} \\
        \\
        Document: Antommarchi and the British wrote separate autopsy reports, each concluding that Napoleon had died of internal bleeding caused by stomach cancer, the disease that had killed his father. A later theory, based on high concentrations of arsenic found in samples of Napoleon's hair, held that Napoleon had died of arsenic poisoning. However, subsequent studies also found high concentrations of arsenic in hair samples from Napoleon's childhood and from his son and Joséphine. Arsenic was widely used in medicines and products such as hair creams in the 19th century. A 2021 study by an international team of gastrointestinal pathologists concluded that Napoleon died of stomach cancer. \\
        Question: English: Did Napoleon die because of malaria? => Japanese: \begin{CJK}{UTF8}{min}ナポレオンが死んだのはマラリアのせい?\end{CJK}
        Answer: English: no => Japanese: no \\
        \\
        Document: The 1987 Constitution of the Philippines declares: The separation of Church and State shall be inviolable. (Article II, Section 6), and, No law shall be made respecting an establishment of religion, or prohibiting the free exercise thereof. The free exercise and enjoyment of religious profession and worship, without discrimination or preference, shall forever be allowed. No religious test shall be required for the exercise of civil or political rights. \\
        Question: English: Does the Philippines follow the principle of separation of church and state?=> Japanese: \begin{CJK}{UTF8}{min}フィリピンは政教分離を原則としている?\end{CJK} \\
        Answer: English: yes => Japanese: yes \\  
        \bottomrule
    \end{tabular}
    \caption{Complete prompt for few-shot cross-lingual question answer generation from English passages.}
    \label{tab:cl_prompt}
\end{table*}

\begin{table*}
    \promptsize
    \centering
    \begin{tabular}{p{0.97\linewidth}}
        \toprule
        English Document: Lindzen was born on February 8, 1940 in Webster, Massachusetts.[1] His father, a shoemaker, had fled Hitler's Germany with his mother. He moved to the Bronx soon after his birth and grew up in a Jewish household in a predominantly Catholic neighborhood there.[3][5] Lindzen attended the Bronx High School of Science (winning Regents' and National Merit Scholarships), Rensselaer Polytechnic Institute,and Harvard University.[6] From Harvard, he received an A.B. in physics in 1960, followed by an S.M. in applied mathematics in 1961 and a PhD in applied mathematics in 1964. His doctoral thesis, Radiative and photochemical processes in strato- and mesospheric dynamics,[7] concerned the interactions of ozone photochemistry, radiative transfer, and dynamics in the middle atmosphere. \\
        English Question: Where was Richard Siegmund Lindzen born? => English Answer: Webster, Massachusetts \\
        \\
        English Document: On 30 September 2011, Justice Johnson Lam of the Court of First Instance of the High Court (CFI) ruled in Vallejos' case that existing legislation restricting FDHs from qualifying for permanent residence contravened the Hong Kong Basic Law. Lam also found that Vallejos and Domingo, but not the three other applicants, had fulfilled the condition of taking Hong Kong as their only permanent home and being ordinarily resident in Hong Kong for seven years. The Court of Appeal of the High Court overturned the CFI's decision on Vallejos' case on 28 March 2012. Vallejos and Domingo then jointly appealed to the Court of Final Appeal (CFA), which heard their case on 26–28 February 2013; the CFA rejected their appeal on 25 March 2013. \\
        English Question: Who was the judge in the case of Vallejos and Domingo v. Commissioner of Registration? => English Answer: Justice Johnson Lam \\
        \\
        English Document: Human dissections were carried out by the Greek physicians Herophilus of Chalcedon and Erasistratus of Chios in the early part of the third century BC.[7] During this period, the first exploration into full human anatomy was performed rather than a base knowledge gained from 'problem-solution' delving.[8] While there was a deep taboo in Greek culture concerning human dissection, there was at the time a strong push by the Ptolemaic government to build Alexandria into a hub of scientific study.[8] For a time, Roman law forbade dissection and autopsy of the human body,[9] so physicians had to use other cadavers. Galen, for example, dissected the Barbary macaque and other primates, assuming their anatomy was basically the same as that of humans.[10][11][12] \\
        English Question: Who first performed human dissection? => English Answer: Herophilus of Chalcedon and Erasistratus of Chios \\
        \\
        English Document: On 16 March 1915, Watson gained his Royal Aero Club Certificate No. 1,117 (equivalent of a pilot's licence) with the London and Provincial School at the London Aerodrome, Hendon, having sought a commission with the Royal Naval Air Service with the outbreak of war in 1914.[6] Sadly, on 30 June 1915 he lost his life when the Caudron G.3 aeroplane he was flying disintegrated in flight and crashed in Dunlye field, a few miles from the Cross-in-Hand Hotel, Sussex. Watson is buried in Dundee's Western Cemetery.[2] \\
        English Question: Where is Preston Albert Watson buried? => English Answer: Dundee's Western Cemetery \\
        \\
        English Document:  A paywall is a method of restricting access to content via a paid subscription.[1][2] Beginning in the mid-2010s, newspapers started implementing paywalls on their websites as a way to increase revenue after years of decline in paid print readership and advertising revenue.[3] In academics, research papers are often subject to a paywall and are available via academic libraries that subscribe.[4][5] \\
        English Question: What is a paywall? => English Answer: a method of restricting access to content via a paid subscription \\
        \\
        Japanese Document: \begin{CJK}{UTF8}{min}モンロー郡はは1815年6月29日に、ヨーロッパ系アメリカ人によって設立された。アラバマが州に昇格する以前のこと>だった。当初の開拓者はイギリス人の子孫であり、バージニア州、ジョージア州および両カロライナ州から来ていた。アッパー・クリーク族ウィンド・クランの著名な酋長レッド・イーグル(ウィリアム・ウェザーフォードとも呼ばれた)が、クリーク戦争(1813年-1814年)の後でこの地に入り、プランテーションを造り上げた。レッド・イーグルはクリーク>族とヨーロッパ人の血を引いており、資産である奴隷を農園主や馬の飼育者に育てた。クリーク族員の大半は、1830年代にアラバマからインディアン準州(現在のオクラホマ州)に移住させられた。その後に入ってきたヨーロッパ系アメリカ人は、奴隷労働者を連れてくるか、土地を取得した後に奴隷を購入した。\end{CJK} \\
        Japanese Question: \colorbox{blue!7}{\begin{CJK}{UTF8}{min}モンロー郡はいつ設立されましたか?\end{CJK} => Japanese Answer: \begin{CJK}{UTF8}{min}1815年6月29日\end{CJK}} \\
        \bottomrule
    \end{tabular}
    \caption{An example for zero-shot English Prompting. \colorbox{blue!7}{Highlighted texts} indicate system outputs.}
    \label{tab:en_prompt}
\end{table*}

\begin{table*}
    \promptsize
    \centering
    \begin{tabular}{p{0.97\linewidth}}
        \toprule
        Finnish Document: Tel Avivissa asuu 467 875 ihmistä, jotka jakautuvat 52 000 dunamin (52 km²; 20 neliömailia) suuruiselle alueelle, mikä tuottaa väestötiheyden 7 606 ihmistä neliökilometrillä (19 699 neliömaililla). Israelin keskusviraston (CBS) mukaan vuoteen 2009 mennessä Tel Avivin väestö kasvaa vuosittain 0,5 prosentilla. Kaikkien taustojen juutalaiset muodostavat 91,8 prosenttia väestöstä, muslimit ja arabikristityt 4,2 prosenttia ja loput kuuluvat muihin ryhmiin (mukaan lukien eri kristilliset ja aasialaiset yhteisöt). Koska Tel Aviv on monikulttuurinen kaupunki, siellä puhutaan monia kieliä heprean lisäksi. Joissakin arvioissa noin 50 000 rekisteröimätöntä afrikkalaista ja aasialaista ulkomaalaistyöntekijää asuu kaupungissa. Verrattuna länsimaisiin kaupunkeihin, rikollisuus Tel Avivissa on suhteellisen vähäistä. \\
        Finnish Question: Onko Tel Aviv monikulttuurinen maa? => Finnish Answer: no \\
        \\
        Russian Document: \cyrins{Во время войны в Вооруженных Силах Соединенных Штатов служило более 16 миллионов американцев, из которых 405 399 погибли в бою, а 671 278 были ранены. Также было 130 201 американских военнопленных, из которых 116 129 вернулись домой после войны. Ключевыми гражданскими советниками президента Рузвельта были министр войны Генри Л. Стимсон, который мобилизовал промышленность и центры призыва для обеспечения армии, командованной генералом Джорджем Маршаллом, и Военно-воздушных сил под командованием генерала Хапа Арнольда. Военно-морской флот, возглавляемый министром военно-морского флота Фрэнком Ноксом и адмиралом Эрнестом Кингом, оказался более автономным. Общие приоритеты устанавливал Рузвельт.} \\
        Russian Question: \cyrins{Сколько американских солдат участвовало во Второй Мировой войне?} => Russian Answer: \cyrins{16 миллионов} \\
        \\
        Bengali Document: {\bng dikKeNr ibrued/dh sNNGgRam calaena Jara kansas-enbRas/ka Aa{I}enr pRit iberadhii icheln, es simMilt HJeichl sn 1854 et. es Aa{I}n EkiT Aa{I}n ichl Ja kansas {O} enbRas/ka Elakar pish/cmaNJ/cel mailkana kRmbr/dhn sm/bhb krt. es kLaiskYal ilbaeriljm EbNNG Ar/th{oi}nitk sNNGs/kaer smr/thn kret, ikn/tu muk/t ANJ/cel gulaimr pRshareNr ibrued/dh ichl. pair/Tr AidhkaNNGsh{I} pRathimkbhaeb dikKeN Upis/thit ichl, ikn/tu Ut/ter sphl ichl. 1858 saelr pr/Jn/t, EiT AidhkaNNGsh puur/bgamii {O} pRak/tn iphR s{I}laredr sHeJaigta inJe Ut/terr pRaJ sms/t raejY bD bD brRHt/trtW gThn kerichl. sada dikKeNr manuShra gulaimr ibpn/ntay icin/tt HJe UeThicheln. 1860 sael pRthm irpaibLkan raSh/TRpit AabRaHam ilNNGkenr inr/bacenr saeth, mHan dikKNii raSh/TRguil {I}Una{I}eTD es/TTs ethek ibic/chn/n HJe egl.} \\
        Bengali Question: {\bng mair/kn Juk/traeSh/TRr irpabilkan pair/Tr pRthm epRiseDn/T ek icheln ?} => Bengali Answer: {\bng AabRaHam ilNG/kn} \\
        \\
        Arabic Document: \aratext{الدروع، والكتيبة الأولى من الفرسان الثامنة دمرت الملاجئ العراقية والمركبات القتالية في قطاع الفرقة العراقية الخامسة والعشرين للمشاة. في 24 فبراير 1991، تقدمت الفرقة الثانية من الفرقة الأولى للمشاة عبر الثغر في الدفاع العراقي غرب وادي البطين وكذلك قامت بتطهير القطاع الشمالي الشرقي لموقع الثغر من المقاومة العدوانية. قامت مجموعة المهمة 3-37 من الدروع بكسر الدفاع العراقي وتطهير أربعة ممرات وتوسيع الفجوة تحت نيران العدو المباشرة. أيضًا، في 24 فبراير، قامت الفرقة الأولى من الفرقة الأولى للمشاة بالتعاون مع الفرقة الأولى من الفرسان بتدمير الحواجز العراقية ودورياتها التابعة للفرقة العراقية السادسة والعشرين.} \\
        Arabic Question: \aratext{ماهو اسم أول منطقة عراقية تعرضت للغزوالأمريكي؟} => Arabic Answer: \aratext{وادي الباطن} \\
        \\
        Korean Document: \begin{CJK}{UTF8}{mj}스미스-풋넘 풍력 터빈은 세계 최초의 1메가와트 규모의 풍력 터빈이었습니다. 1941년에 버몬트주 캐슬턴의 그랜드파스 노브에 연결되어 현지 전력 공급 시스템에 연결되었습니다. 이 터빈은 파머 코슬렛 풋넘이 디자인하고 S. 모건 스미스 회사에서 제조했습니다. 이 1.25메가와트 터빈은 1100시간 동안 작동한 후 전쟁 중 재료 부족으로 약점이 알려진 곳에서 날개가 파손되었습니다. 이후 1979년까지는 최대 규모의 풍력 터빈으로 남았습니다.\end{CJK} \\
        Korean Question: \begin{CJK}{UTF8}{mj}세상에서 가장 큰 풍력 에너지 발전소는 무엇인가?\end{CJK} => Korean Answer: \begin{CJK}{UTF8}{mj}스미스-퍼트남 풍력 터빈\end{CJK} \\
        \\
        Japanese Document: \begin{CJK}{UTF8}{min}細菌性髄膜炎の原因として>多い肺炎球菌と髄膜炎菌は鼻咽腔上皮細胞に付着しコロニーを形成する。そこから血管内に侵入し脳室内脈絡叢に到達する。脈絡叢上皮細胞に直接感染し脳脊髄液中に入ることができる。脳脊髄液中では免疫防御機構が機能しないため細菌は急速に増殖する。細菌性髄膜炎の発症機序において重要なのは浸潤した細菌が誘発する炎症反応である。細菌性髄膜炎の神経症状や合併症の多くは、細菌による組織の直接的な破壊よりもむしろ、浸潤した細菌に対する免疫応答によって引き起こされている。結果として、抗生物質療法により脳脊髄液が無菌化された後になっても神経の損傷は進行しうる\end{CJK} \\
        Japanese Question: \colorbox{blue!7}{\begin{CJK}{UTF8}{min}細菌性髄膜炎の最も一般的な原因は何か?\end{CJK} => Japanese Answer: \begin{CJK}{UTF8}{min}肺炎球菌と髄膜炎菌は鼻咽腔上皮細胞に付着しコロニーを\end{CJK}} \\ \colorbox{blue!7}{\begin{CJK}{UTF8}{min}形成する。\end{CJK}} \\
        \bottomrule
    \end{tabular}
    \caption{An example for zero-shot Multilingual Prompting. \colorbox{blue!7}{Highlighted texts} indicate system outputs.}
    \label{tab:multilingual_prompt}
\end{table*}


\begin{table*}
    \scriptsize
    \centering
    \begin{tabular}{p{0.97\linewidth}}
        \toprule
        \texttt{Question:} \textcolor{red}{\{Example question \#1\}} \\
        \texttt{Answer}: \textcolor{red}{\{Example answer \#1\}} \\
        $\cdots$
        \\
        \texttt{Question}: \textcolor{red}{\{Example question \#5\}} \\
        \texttt{Answer}: \textcolor{red}{\{Example answer \#5\}} \\
        \\
        \texttt{Passage \#1 Title}: \textcolor{red}{\{Passage \#1 Title\}} \\
        \texttt{Passage \#1 Text}: \textcolor{red}{\{Passage \#1 Text\}} \\
        $\cdots$
        \\
        \texttt{Passage \#N Title}: \textcolor{red}{\{Passage \#N Title\}} \\
        \texttt{Passage \#N Text}: \textcolor{red}{\{Passage \#N Text\}} \\
        \\
        \texttt{Task description: predict the} \textcolor{red}{\{Test Question Language\}} \texttt{answer to the following question. The answer should be a minimal span extracted from the document. You should only output the answer.} \\
        \\
        \texttt{Question:} \textcolor{red}{\{Test question\}} \\
        \texttt{Answer:} \\
        \bottomrule
    \end{tabular}
    \caption{Prompt template for few-shot multilingual QA with LLMs.}
    \label{tab:llm_fs_qa_prompt}
\end{table*}

\begin{table*}
    \scriptsize
    \centering
    \begin{tabular}{p{0.97\linewidth}}
        \toprule
        Relevance Assessment \\
        \midrule
        You are given a Q\&A pair and a paragraph. Your goal is to Rate the relevance of the Q\&A pair to the paragraph on a scale from 0 to 2. \\
        0: Very low relevance, the Q\&A pair and paragraph are almost unrelated. \\
        1: Moderate relevance, the Q\&A pair and paragraph share some overlap. \\
        2: High relevance, the Q\&A pair are strongly grounded by the paragraph. \\
        Output Format: \\
        Relevance (0-2) \\
        Only provide the final result in the above structured format without any additional explanations. \\
        
        Paragraph: \textcolor{red}{\{Paragraph\}} \\
        Q: \textcolor{red}{\{Synthetic Query\}} \\
        A: \textcolor{red}{\{Synthetic Answer\}} \\
        \bottomrule
        \\
        \toprule
        Fluency Assessment \\
        \midrule
        You are given a question. Your goal is to Rate the fluency of the question on a scale from 0 to 2. \\
        0: Poor fluency, the question is unclear, contains significant grammatical errors, or is incomprehensible. \\
        1: Moderate fluency, the question has minor grammatical errors or awkward phrasing but is still understandable. \\
        2: High fluency, the question is clear, well-structured, and grammatically correct. \\
        Output Format: \\
        Fluency (0-2): \\
        Only provide the final result in the above structured format without any additional explanations. \\
        
        Question: \textcolor{red}{\{Synthetic Query\}} \\
        \bottomrule
    \end{tabular}
    \caption{Prompt template for quality validation of synthetic data using \texttt{Model-as-Judge}.}
    \label{tab:llm_assess_prompt}
\end{table*}


\begin{table*}[t]
\setlength{\belowcaptionskip}{-0.3cm}
\setlength{\tabcolsep}{7pt}
\footnotesize
\centering
\begin{tabular}{l|ccc|ccc}
    \toprule
    & \multicolumn{3}{c|}{\bf\ourptdata} & \multicolumn{3}{c}{\bf\ourdata} \\
    \cmidrule(lr){2-4} \cmidrule(lr){5-7}
    & \# Q-A Paris & Question Length & Answer Length & \# Q-A Paris & Question Length & Answer Length \\
    \midrule
    \texttt{Arabic} & 1,803,765 & 7.00$_{\pm\text{2.13}}$ & 1.65$_{\pm\text{0.84}}$ & 80,575 & 8.20$_{\pm\text{2.86}}$ & 1.57$_{\pm\text{1.30}}$ \\
    \texttt{Bengali} & 407,496 & 6.13$_{\pm\text{1.80}}$ & 1.65$_{\pm\text{0.85}}$ & 127,562 & 8.97$_{\pm\text{2.99}}$ & 1.63$_{\pm\text{1.44}}$ \\
    \texttt{English} & 7,963,985 & 7.95$_{\pm\text{2.54}}$ & 1.78$_{\pm\text{1.01}}$ & --- & --- & --- \\
    \texttt{Finnish} & 2,135,790 & 6.02$_{\pm\text{1.75}}$ & 1.32$_{\pm\text{0.64}}$ & 270,627 & 5.83$_{\pm\text{2.09}}$ & 1.38$_{\pm\text{0.90}}$ \\
    \texttt{Japanese} & 2,735,635 & 14.74$_{\pm\text{3.57}}$ & 3.57$_{\pm\text{1.73}}$ & 143,265 & 10.19$_{\pm\text{2.18}}$ & 3.96$_{\pm\text{4.69}}$ \\
    \texttt{Korean} & 1,018,348 & 5.46$_{\pm\text{1.78}}$ & 1.55$_{\pm\text{0.80}}$ & 192,002 & 5.72$_{\pm\text{2.29}}$ & 1.42$_{\pm\text{0.92}}$ \\
    \texttt{Russian} & 2,561,925 & 6.94$_{\pm\text{2.17}}$ & 1.70$_{\pm\text{1.10}}$ & 792,914 & 7.34$_{\pm\text{2.64}}$ & 1.44$_{\pm\text{1.03}}$ \\
    \texttt{Telugu} & 108,215 & 5.60$_{\pm\text{1.84}}$ & 1.50$_{\pm\text{0.74}}$ & 139,211 & 6.48$_{\pm\text{2.48}}$ & 1.49$_{\pm\text{1.17}}$ \\
    \bottomrule
    \end{tabular}
    \caption{Dataset statistics of our pre-training data \ourptdata and few-shot synthetic data \ourdata in each language.}
\label{tab:data_statistics}
\end{table*}

% MLWIKIQA
% ar: qlen: 7.007845811400044 2.134918494313189; alen: 1.6525450931800982 0.8421524711478329
% bn: qlen: 6.127792665449477 1.798141597024151; alen: 1.6499695702534503 0.8536029366358691
% en: qlen: 7.9497540490093845 2.5406085246050956; alen: 1.7768247931155068 1.011210129795646
% fi: qlen: 6.024182620950562 1.7479958621002722; alen: 1.3231319558570833 0.6368866833596405
% ja: qlen: 14.741594181972376 3.570845743296836; alen: 3.5659903459343076 1.727498303286495
% ko: qlen: 5.460906291366016 1.7778993139807557; alen: 1.5530928523451708 0.798777546418375
% ru: qlen: 6.941091171677547 2.1664593814000868; alen: 1.7021501410072504 1.0963086790415875
% te: qlen: 5.6020699533336415 1.837506371750535; alen: 1.5025920620986 0.7405023883492006


% FSMLQA
% ar: qlen: 8.199813838039095 2.862454477729634; alen: 1.570300961836798 1.303697837272117
% bn: qlen: 8.965632398363148 2.988353414480306; alen: 1.628078894968721 1.4353580173736895
% fi: qlen: 5.826731996437902 2.0895141775112216; alen: 1.3791048195486777 0.9041934163323438
% ja: qlen: 10.187226468432625 2.1784119970762443; alen: 3.9564652915925036 4.691712861419157
% ko: qlen: 5.717836272538828 2.2868590158992066; alen: 1.416703992666743 0.9189516729051308
% ru: qlen: 7.3617744168976715 2.635693081408635; alen: 1.4373462948062463 1.032071785642317
% te: qlen: 6.478223703586642 2.47490763544063; alen: 1.4919941671275976 1.1737755672703802


\begin{table*}[t]
\setlength{\belowcaptionskip}{-0.3cm}
\setlength{\tabcolsep}{7pt}
\footnotesize
\centering
\begin{tabular}{l|ccccccc|c}
    \toprule
    Prompting Strategy & Ar & Bn & Fi & Ja & Ko & Ru & Te & Avg. \\
    \midrule
    \ours & 5.9\% & 13.9\% & 16.9\% & 7.7\% & 21.4\% & 15.4\% & 13.2\% & 13.4\% \\
    NQ-En & 3.4\% & 8.8\% & 8.2\% & 1.2\% & 8.3\% & 5.9\% & 4.2\% & 5.3\% \\
    \textsc{TyDi}-En & 5.0\% & 13.6\% & 12.8\% & 1.9\% & 17.0\% & 6.3\% & 5.9\% & 7.0\% \\
    \textsc{Xor-TyDi-*} & 10.2\% & 14.6\% & 14.2\% & 2.5\% & 22.7\% & 9.1\% & 10.7\% & 9.8\% \\
    \bottomrule
    \end{tabular}
    \caption{Success Rate of synthetic data generation across seven languages with different prompting strategies. Success Rate = valid examples after data filtering / total examples (\ie \# Documents)}
\label{tab:success_rate}
\end{table*}

% # docs
% ar: 880595
% bn: 136096 * 3
% fi: 917272
% ja: 1276811
% ko: 383085
% ru: 3769131
% te: 278944 * 3
% 8472014

% # QA
% Few-shot: ar: 51617 bn: 56693 fi: 154652 ja: 97857 ko: 81808 ru: 580497 te: 110524
% nq-prompt: ar: 30208 bn: 35912 fi: 75518 ja: 15571 ko: 31784 ru: 222069 te: 35082
% tydien-prompt: ar: 44193 bn: 55600 fi: 117678 ja: 24172 ko: 64934 ru: 237186 te: 49004
% multilingual-prompt: ar: 89453 bn: 59515 fi: 130319 ja: 32178 ko: 86807 ru: 343057 te: 89812


% \begin{table*}[t]
% \setlength{\belowcaptionskip}{-0.3cm}
% \setlength{\tabcolsep}{3pt}
% \scriptsize
% \centering
% \resizebox{\linewidth}{!}{\begin{tabular}{l|ccccccccccccc}
%     \toprule
%     \multirow{2}{*}{\bf Language} & \multirow{2}{*}{\bf Safe} & \bf Violent & \bf Non-Violent & \bf Sex-Related & \bf Child Sexual & \bf Specialized & \multirow{2}{*}{\bf Privacy} & \bf Intellectual & \bf Indiscriminate & \multirow{2}{*}{\bf Hate} & \bf Suicide \& & \bf Sexual & \multirow{2}{*}{\bf Others} \\
%     & & \bf Crimes & \bf Crimes & \bf Crimes & \bf Exploitation & \bf Advice & & \bf Property & \bf Weapons & & \bf Self-Harm & \bf Content & \\

%     S1: Violent Crimes	S2: Non-Violent Crimes
% S3: Sex-Related Crimes	S4: Child Sexual Exploitation
% S5: Specialized Advice	S6: Privacy
% S7: Intellectual Property	S8: Indiscriminate Weapons
% S9: Hate	S10: Suicide & Self-Harm
% S11: Sexual Content	

%     \midrule
%     {'safe': 79261, 'unsafe\nS2': 86, 'unsafe\nS8': 299, 'unsafe\nS3': 40, 'unsafe\nS11': 35, 'unsafe\nS6': 320, 'unsafe\nS1': 127, 'unsafe\nS5': 286, 'unsafe\nS10': 30, 'unsafe\nS9': 67, 'unsafe\nS7': 16, 'unsafe': 7, 'unsafe\nS2,S6': 1}
%     \texttt{Arabic} & 79261 & \\
%     \texttt{Bengali} &  \\
%     \texttt{Finnish} &  \\
%     \texttt{Japanese} &  \\
%     \texttt{Korean} &  \\
%     \texttt{Russian} &  \\
%     \texttt{Telugu} &  \\
%     \bottomrule
%     \end{tabular}}
%     \caption{The percentage of different unsafe categories of \ourdata tagged by \href{https://huggingface.co/meta-llama/Meta-Llama-Guard-2-8B}{Llama-Guard-2}.}
% \label{tab:safety_analysis}
% \end{table*}


\begin{table*}[t]
% \setlength{\belowcaptionskip}{-0.5cm}
% \setlength{\tabcolsep}{2.3pt}
\footnotesize
\centering
\begin{tabular}{lccccccc|ccc}
    \toprule
    \multirow{2}{*}[-1ex]{\bf Method} & \multicolumn{7}{c|}{F1} & \multicolumn{3}{c}{Macro Average} \\
    \cmidrule(lr){2-8} \cmidrule(lr){9-11}
    & \bf Ar & \bf Bn & \bf Fi & \bf Ja & \bf Ko & \bf Ru & \bf Te & \bf F1 & \bf EM & \bf BLEU \\
    \midrule
    \multicolumn{11}{l}{\bf\emph{5-shot}} \\
    \oursen & 35.6 & 32.7 & 35.5 & 35.1 & 30.2 & 33.6 & 31.8 & 33.5 & 23.8 & 23.0 \\
    \ours & \bf 41.3 & \bf 35.4 & \bf 39.6 & \bf 41.5 & \bf 35.0 & \bf 38.2 & \bf 36.3 & \bf 38.2 & \bf 27.9 & \bf 24.4 \\

    \midrule
    \multicolumn{11}{l}{\bf\emph{16-shot}} \\
    \oursen & 38.3 & 31.0 & 39.4 & 38.3 & 35.2 & 34.9 & 34.6 & 35.9 & 26.1 & 24.1 \\
    \ours & \bf 42.0 & \bf 35.6 & \bf 41.4 & \bf 41.7 & \bf 35.3 & \bf 39.2 & \bf 40.0 & \bf 39.3 & \bf 29.3 & \bf 26.6 \\

    \midrule
    \multicolumn{11}{l}{\bf\emph{32-shot}} \\
    \oursen & 42.4 & 31.2 & 40.8 & 38.1 & 33.0 & 37.9 & 34.9 & 36.9 & 26.3 & 25.5 \\
    \ours & \bf 43.6 & \bf 35.6 & \bf 42.2 & \bf 42.5 & \bf 34.1 & \bf 38.6 & \bf 37.0 & \bf 39.1 & \bf 28.8 & \bf 26.6 \\

    \midrule
    \multicolumn{11}{l}{\bf\emph{128-shot}} \\
    \oursen & 42.0 & 28.8 & 41.7 & 40.3 & \bf 34.6 & 34.7 & 36.0 & 36.9 & 27.0 & 25.2 \\
    \ours & \bf 45.3 & \bf 32.8 & \bf 44.3 & \bf 43.8 & 34.0 & \bf 39.9 & \bf 42.1 & \bf 40.3 & \bf 30.5 & \bf 27.4 \\

    \midrule
    \multicolumn{11}{l}{\bf\emph{1024-shot}} \\
    \oursen & 45.0 & 30.8 & 45.1 & 39.2 & 34.1 & 39.1 & 37.5 & 38.7 & 29.3 & 26.5 \\
    \ours & \bf 47.5 & \bf 33.7 & \bf 46.7 & \bf 41.4 & \bf 35.9 & \bf 40.2 & \bf 40.1 & \bf 40.8 & \bf 31.3 & \bf 27.9 \\

    \midrule
    \multicolumn{11}{l}{\bf\emph{full}} \\
    \oursen & 48.9 & \bf 33.3 & 47.7 & 42.9 & \bf 39.6 & 40.0 & 41.7 & 42.0 & 32.7 & 28.5 \\
    \ours & \bf 50.8 & \bf 33.3 & \bf 47.8 & \bf 45.0 & 38.9 & \bf 42.0 & \bf 43.1 & \bf 43.0 & \bf 33.4 & \bf 29.6 \\

    \bottomrule
    \end{tabular}
    \caption{Detailed results in each language when trained with varying sizes of supervised data.}
\label{tab:supervised_data_scaling_details}
\end{table*}

\begin{figure*}[t]
    \centering
    \setlength{\abovecaptionskip}{-0.01cm}
    \setlength{\belowcaptionskip}{-0.35cm}
    \includegraphics[width=\linewidth]{figures/tsne_gold_vs_synthetic.pdf}
    \caption{Distribution comparison between \ourdata and \textsc{Xor-TyDi QA} in the rest languages. We demonstrate that the diverse synthetic data can be expanded from only five-shot examples and retains the core characteristics of the gold distribution.}
    \label{fig:tsne_gold_vs_synthetic}
\end{figure*}