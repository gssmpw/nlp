\section{Related Work}
\label{sec:related_work}

\textbf{Hallucination Mitigation in LVLMs.} 
Hallucination -- the generation of content that is irrelevant, factually incorrect, or inconsistent with visual inputs~\cite{bai2024hallucination} -- represents a fundamental challenge in LVLM development. Research has identified three primary sources: limitations in visual encoder capabilities~\cite{tong2024eyes, liu2024llavanext, shi2024eagle}, excessive reliance on learned parametric knowledge~\cite{li-etal-2023-evaluating, zhou2023analyzing, vcd, opera}, and noisy training data~\cite{liu2023mitigating, yu2024hallucidoctor}. Mitigation approaches span training-based solutions with refined datasets~\cite{yue2024less, jiang2024hallucination}, post-processing techniques including revision~\cite{yin2023woodpecker, zhou2023analyzing} and verification~\cite{chen2024halc, sun2023aligning}, and inference-time interventions like Visual Contrastive Decoding~\cite{vcd} and enhanced attention methods~\cite{pai}. Recent studies revealing ``text inertia''~\cite{pai}, where models generate similar hallucinations without visual input, highlight concerning reliance on learned text patterns. While these findings advance our understanding, how hallucination propagates through model architectures remains elusive, and existing solutions often require external supervision and are hinged with specific decoding strategies.

\textbf{Contrastive Decoding in LVLMs.} 
Contrastive decoding, originally introduced in NLP~\cite{li2022contrastive, shi2023trusting}, has emerged as a promising approach for reducing hallucination in LVLMs. Recent adaptations of this technique have explored various contrasting strategies: VCD~\cite{vcd} introduces visual-specific contrasts by crafting noisy visual tokens as negative samples, while DoLa~\cite{dola} innovates by contrasting logits distributions from different layers within the same model, using divergence measurements to dynamically select contrasting layers. Taking a temporal perspective, M3ID~\cite{favero2024multi} proposes a "horizontal" strategy that contrasts current logits with those from previous timesteps. Other approaches extend contrastive techniques to attention mechanisms~\cite{woo2024don}. While these methods primarily operate in the logits space, our \ours takes a different approach by performing contrasts in the activation space and intervening at residual streams. This earlier-stage intervention strategy offers an efficient alternative that can complement existing decoding methods.