\usepackage[english]{babel}
\usepackage{fancyhdr}
\setlength{\headheight}{15pt} 
\usepackage{titlepic}
\usepackage{fancyhdr}
\usepackage{graphicx}
\usepackage{lipsum}
\usepackage{titling}
\usepackage{textcomp}
\usepackage{gensymb}
\usepackage{amsfonts}
\usepackage{multirow}
\usepackage{subcaption}
\usepackage{siunitx}
\usepackage{microtype}  % %minutely improves kerning
%\usepackage[active,floats]{preview}
% % % %Table of contents
\usepackage[intoc]{nomencl}
\usepackage{color}
\input{ImpStyle.sty}
%\includeonly{CA1/chapterA1}
\usepackage{keystroke}
\usepackage{listings}
\lstset{
  basicstyle=\ttfamily,  %basicstyle=\footnotesize,
  columns=fullflexible,
  showspaces=false,
  showtabs=false,
  breaklines=true,
  showstringspaces=false,
  breakatwhitespace=true,
  escapeinside={(*@}{@*)}
}
\usepackage{setspace}
\usepackage{amsmath}
\usepackage{upgreek}
%\usepackage{units}
\usepackage{lscape}
\usepackage{pbox}
\usepackage[tableposition=top]{caption}
\usepackage{float}
\floatstyle{plaintop}
\restylefloat{table}
\usepackage[version=3]{mhchem}
\usepackage[sectionbib]{chapterbib}
\bibliographystyle{plain}
\usepackage{algorithm}
%\usepackage{bibentry}
%\usepackage{biblatex}
%\usepackage{navigator}
\usepackage[hidelinks]{hyperref}
\hypersetup{
    colorlinks=false, %set true if you want colored links
    linktoc=all,     %set to all if you want both sections and subsections linked
    linkcolor=black,  %choose some color if you want links to stand out
}

\makeatletter
\newcommand{\chapterauthor}[1]{%
  {\parindent0pt\vspace*{-25pt}%
  \linespread{1.1}\large\scshape#1%
  \par\nobreak\vspace*{35pt}}
  \@afterheading%
}
\makeatother
\usepackage{afterpage}
\usepackage{booktabs}
\usepackage{listings}
\usepackage{hyperref}
\usepackage{makecell}
\usepackage{threeparttable}
\usepackage{algpseudocode}
\usepackage{mathtools}
\usepackage{amsmath}
\usepackage{amssymb}
\usepackage{physics}
\usepackage{wasysym}
\usepackage{array}
\usepackage{tabularx}
\algnewcommand\algorithmicforeach{\textbf{for each}}
\algdef{S}[FOR]{ForEach}[1]{\algorithmicforeach\ #1\ \algorithmicdo}


\DeclareRobustCommand{\pp}[0]{%
\begin{tikzpicture}[line width=0.3pt, scale=1.2]%
\draw (0ex,0ex) -- (0ex,1.5ex);
\end{tikzpicture}}

\DeclareRobustCommand{\ppp}[0]{%
\begin{tikzpicture}[line width=0.3pt, scale=1.2]%
\draw (0ex,0ex) -- (1.5ex,0ex);
\draw (1.5ex,0ex) -- (0.75ex,1.5ex);
\draw (0.75ex,1.5ex) -- (0ex,0ex);
\end{tikzpicture}}

\DeclareRobustCommand{\pppp}[0]{%
\begin{tikzpicture}[line width=0.3pt, scale=1.2]%
\draw (0ex,0ex) -- (1.5ex,0ex);
\draw (1.5ex,0ex) -- (0.75ex,1.5ex);
\draw (0.75ex,1.5ex) -- (0ex,0ex);
\draw (0.75ex,0.65ex) -- (0ex,0ex);
\draw (0.75ex,0.65ex) -- (1.5ex,0ex);
\draw (0.75ex,0.65ex) -- (0.75ex,1.5ex);
\end{tikzpicture}}

\DeclareRobustCommand{\ppppp}[0]{%
\begin{tikzpicture}[line width=0.3pt, scale=1.2]%
%\draw (0ex,0ex) -- (1.5ex,0ex);
%\draw (0ex,0ex) -- (0ex,1.5ex);
%\draw (0ex,0ex) -- (1.5ex,1.5ex);
%\draw (0ex,0ex) -- (2.25ex,0.75ex);
%\draw (1.5ex,0ex) -- (0ex,1.5ex);
%\draw (1.5ex,0ex) -- (1.5ex,1.5ex);
%\draw (1.5ex,0ex) -- (2.25ex,0.75ex);
%\draw (0ex,1.5ex) -- (1.5ex,1.5ex);
%\draw (0ex,1.5ex) -- (2.25ex,0.75ex);
%\draw (1.5ex,1.5ex) -- (2.25ex,0.75ex);
\draw (0ex,0ex) -- (0.75ex,0ex);
\draw (0ex,0ex) -- (0ex,1.5ex);
\draw (0ex,0ex) -- (0.75ex,1.5ex);
\draw (0ex,0ex) -- (1.5ex,0.75ex);
\draw (0.75ex,0ex) -- (0ex,1.5ex);
\draw (0.75ex,0ex) -- (0.75ex,1.5ex);
\draw (0.75ex,0ex) -- (1.5ex,0.75ex);
\draw (0ex,1.5ex) -- (0.75ex,1.5ex);
\draw (0ex,1.5ex) -- (1.5ex,0.75ex);
\draw (0.75ex,1.5ex) -- (1.5ex,0.75ex);
\end{tikzpicture}}

\newcommand\myeq{\stackrel{\mathclap{\scriptsize\mbox{def}}}{=}}

\usepackage{atbegshi}% http://ctan.org/pkg/atbegshi
\AtBeginDocument{\AtBeginShipoutNext{\AtBeginShipoutDiscard}}