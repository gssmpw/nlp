% This must be in the first 5 lines to tell arXiv to use pdfLaTeX, which is strongly recommended.
\pdfoutput=1
% In particular, the hyperref package requires pdfLaTeX in order to break URLs across lines.
%!TEX program = pdflatex
%!TEX root = acl_latex.tex

\documentclass[11pt]{article}

\usepackage[svgnames]{xcolor}
\usepackage{amsmath}
\usepackage{multirow}
\usepackage{booktabs}
\usepackage{tabularx}
\usepackage{tcolorbox}
\usepackage{subcaption}
\usepackage{listings}
\usepackage{algorithm}
\usepackage{algpseudocode}

% Change "review" to "final" to generate the final (sometimes called camera-ready) version.
% Change to "preprint" to generate a non-anonymous version with page numbers.
\usepackage[preprint]{acl}

% Standard package includes
\usepackage{times}
\usepackage{latexsym}

% For proper rendering and hyphenation of words containing Latin characters (including in bib files)
\usepackage[T1]{fontenc}
% For Vietnamese characters
% \usepackage[T5]{fontenc}
% See https://www.latex-project.org/help/documentation/encguide.pdf for other character sets

% This assumes your files are encoded as UTF8
\usepackage[utf8]{inputenc}

% This is not strictly necessary, and may be commented out,
% but it will improve the layout of the manuscript,
% and will typically save some space.
\usepackage{microtype}

% This is also not strictly necessary, and may be commented out.
% However, it will improve the aesthetics of text in
% the typewriter font.
\usepackage{inconsolata}

%Including images in your LaTeX document requires adding
%additional package(s)
\usepackage{graphicx}

\tcbuselibrary{breakable}
\lstset{ flexiblecolumns, breaklines = true, frame = lrtb }

% If the title and author information does not fit in the area allocated, uncomment the following
%
%\setlength\titlebox{<dim>}
%
% and set <dim> to something 5cm or larger.

\title{CritiQ: Mining Data Quality Criteria from Human Preferences}

\author{
\textbf{Honglin Guo\textsuperscript{1,2}},
\textbf{Kai Lv\textsuperscript{1,2}},
\textbf{Qipeng Guo\textsuperscript{2}},
\\
\textbf{Tianyi Liang\textsuperscript{3,2}},
\textbf{Zhiheng Xi\textsuperscript{1}},
\textbf{Demin Song\textsuperscript{2}},
\textbf{Qiuyinzhe Zhang\textsuperscript{4,2}},
\\
\textbf{Yu Sun\textsuperscript{2}},
\textbf{Kai Chen\textsuperscript{2}},
\textbf{Xipeng Qiu\textsuperscript{1}},
\textbf{Tao Gui\textsuperscript{1}}
\\
\textsuperscript{1}Fudan University,
\textsuperscript{2}Shanghai AI Laboratory,
\\
\textsuperscript{3}East China Normal University,
\textsuperscript{4}University of Science and Technology of China
\\
\texttt{\{hlguo24,klv23,zhxi22\}@m.fudan.edu.cn},
\texttt{\{xpqiu,tgui\}@fudan.edu.cn},
\\
\texttt{\{guoqipeng,songdemin,sunyu2,chenkai\}@pjlab.org.cn},
\\
\texttt{51215901019@stu.ecnu.edu.cn},
\texttt{zhangqiuyinzhe@mail.ustc.edu.cn}
\\
}

\begin{document}
  \maketitle
  \begin{abstract}
    \begin{abstract}
Multi-modal models, such as CLIP, have demonstrated strong performance in aligning visual and textual representations, excelling in tasks like image retrieval and zero-shot classification. Despite this success, the mechanisms by which these models utilize training data, particularly the role of memorization, remain unclear. In uni-modal models, both supervised and self-supervised, memorization has been shown to be essential for generalization. However, it is not well understood how these findings would apply to CLIP, which incorporates elements from both supervised learning via captions that provide a supervisory signal similar to labels, and from self-supervised learning via the contrastive objective.
To bridge this gap in understanding, we propose a formal definition of memorization in CLIP (CLIPMem) and use it to quantify memorization in CLIP models. Our results indicate that CLIP’s memorization behavior falls between the supervised and self-supervised paradigms, with "mis-captioned" samples exhibiting highest levels of memorization. 
Additionally, we find that the text encoder contributes more to memorization than the image encoder, suggesting that mitigation strategies should focus on the text domain. 
Building on these insights, we propose multiple strategies to reduce memorization while at the same time improving utility---something that had not been shown before for traditional learning paradigms where reducing memorization typically results in utility decrease.
%, some of our proposed mitigations for CLIP can reduce memorization while improving downstream utility.\todo{This needs to be reworked: our CLIPMem has the practical application to identify "miscaptioned" samples, such that we can remove them from the training, and then get better results. This is particularly important given that CLIP is trained on large amounts of uncurated data from the internet, and one cannot review all these image pairs. With out metric, it becomes possible.
%aybe also include the risk of exposure for these data points that otherwise arise.
%}
% Multi-modal models, such as CLIP, exhibit strong performance in aligning visual and textual representations, thereby achieving remarkable performance in tasks like image retrieval and zero-shot classification. 
% While the models have a strong generalization ability, it is not fully understood how the models leverage their training data to achieve this.
% One factor that is often linked to a model's generalization ability is memorization. For uni-modal models, both in supervised and self-supervised, it has been shown that memorization is required for generalization.
% Yet, it is unclear how the findings will translate to CLIP because in CLIP captions as supervisory signals, somewhat akin to traditional labels, but also employs self-supervised contrastive learning. Hence, CLIP is in between both paradigms.
% To bridge this gap, we propose a formal definition of memorization in CLIP (CLIPMem) and use it to quantify memorization in CLIP models. 
% Our results show that CLIP's memorization behavior indeed falls between supervised and self-supervised paradigms. Notably, "mis-captioned" samples exhibit high levels of memorization.
% Additionally, we find that the the text encoder has a higher impact on memorization than the image encoder.
% Based on these findings, we find some effective mitigation strategies for memorization in CLIP that focus more on the text domain to maintain model performance while reducing memorization.
% Indeed, unlike in traditional supervised or self-supervised learning, where reducing memorization often reduces utility, we empirically find that some mitigations in CLIP not only reduce memorization but at the same time improve downstream utility.
\end{abstract}

  \end{abstract}

  \section{Introduction}
  Large language models (LLMs) show significant performance in various downstream
tasks~\citep{brown_language_2020,openai_gpt-4_2024,dubey_llama_2024}. Studies
have found that training on high quality corpus improves the ability of LLMs
to solve different problems such as writing code, doing math exercises, and
answering logic questions~\citep{cai_internlm2_2024,deepseek-ai_deepseek-v3_2024,qwen_qwen25_2024}.
Therefore, effectively selecting high-quality text data is an important subject for
training LLM.

\begin{figure}[t]
    \centering
    \includegraphics[width=\linewidth]{figures/head.pdf}
    \caption{The overview of CritiQ. We (1) employ human annotators to annotate $\sim$30
    pairwise quality comparisons, (2) use CritiQ Flow to mine quality criteria, (3)
    use the derived criteria to annotate 25k pairs, and (4) train the CritiQ Scorer to
    perform efficient data selection.}
    \label{fig:overview}
\end{figure}

To select high-quality data from a large corpus, researchers manually design heuristics~\citep{dubey_llama_2024,rae_scaling_2022},
calculate perplexity using existing LLMs~\citep{marion2023moreinvestigatingdatapruning,wenzek2019ccnetextractinghighquality},
train classifiers~\citep{brown_language_2020,dubey_llama_2024,xie_data_2023} and
query LLMs for text quality through careful prompt engineering~\citep{gunasekar_textbooks_2023,wettig_qurating_2024,sachdeva_how_2024}.
Large-scale human annotation and prompt engineering require a lot of human
effort. Giving a comprehensive description of what high-quality data is like is also
challenging. As a result, manually designing heuristics lacks robustness and introduces
biases to the data processing pipeline, potentially harming model performance
and generalization. In addition, quality standards vary across different
domains. These methods can not be directly applied to other domains without significant
modifications.

To address these problems, we introduce CritiQ, a novel method to automatically
and effectively capture human preferences for data quality and perform efficient data
selection. Figure~\ref{fig:overview} gives an overview of CritiQ, comprising an agent
workflow, CritiQ Flow, and a scoring model, CritiQ Scorer. Instead of manually describing
how high quality is defined, we employ LLM-based agents to summarize quality
criteria from only $\sim$30 human-annotated pairs.

CritiQ Flow starts from a knowledge base of data quality criteria. The worker
agents are responsible to perform pairwise judgment under a given
criterion. The manager agent generates new criteria and refines them through reflection
on worker agents' performance. The final judgment is made by majority voting among
all worker agents, which gives a multi-perspective view of data quality.

To perform efficient data selection, we employ the worker agents to annotate a randomly
selected pairwise subset, which is ~1000x larger than the human-annotated one.
Following \citet{korbak_pretraining_2023,wettig_qurating_2024}, we train CritiQ
Scorer, a lightweight Bradley-Terry model~\citep{bradley_rank_1952} to convert
pairwise preferences into numerical scores for each text. We use CritiQ Scorer to
score the entire corpus and sample the high-quality subset.

For our experiments, we established human-annotated test sets to quantitatively
evaluate the agreement rate with human annotators on data quality preferences. We implemented the manager agent by \texttt{GPT-4o} and the worker
agent by \texttt{Qwen2.5-72B-Insruct}. We conducted experiments on different
domains including code, math, and logic, in which CritiQ Flow shows a consistent
improvement in the accuracies on the test sets, demonstrating the effectiveness
of our method in capturing human preferences for data quality. To validate the quality
of the selected dataset, we continually train \texttt{Llama 3.1}~\citep{dubey_llama_2024}
models and find that the models achieve better performance on downstream tasks
compared to models trained on the uniformly sampled subsets.

We highlight our contributions as follows. We will release the code to facilitate
future research.

\begin{itemize}
    \item We introduce CritiQ, a method that captures human preferences for data
        quality and performs efficient data selection at little cost of human
        annotation effort.

    \item Continual pretraining experiments show improved model performance in code,
        math, and logic tasks trained on our selected high-quality subset compared to the raw dataset.

    \item Ablation studies demonstrate the effectiveness of the knowledge base and
        the the reflection process.
\end{itemize}

\begin{figure*}[t]
    \centering
    \includegraphics[width=\linewidth]{figures/method.pdf}
    \caption{CritiQ Flow comprises two major components: multi-criteria pairwise
    judgment and the criteria evolution process. The multi-criteria pairwise
    judgment process employs a series of worker agents to make quality
    comparisons under a certain criterion. The criteria evolution process aims to
    obtain data quality criteria that highly align with human judgment through
    an iterative evolution. The initial criteria are retrieved from the
    knowledge base. After evolution, we select the final criteria to annotate
    the dataset for training CritiQ Scorer.}
    \label{fig:method}
\end{figure*}


  \section{Related Work}
  \section{Related Work}

\begin{figure*}[t!]
    \centering
    \includegraphics[width=0.99\textwidth]{figures/framework.pdf}
    \caption{Overview of the \method framework.}
    \label{fig:framework}
    \vspace{-1em}
\end{figure*}

\noindent\textbf{Text-Augmented Models for Time Series Forecasting.} 

The success of LLMs inspires their application to time series tasks. Methods like LLMTime~\cite{gruver2023large} and LLM4TS~\cite{chang2023llm4ts} tokenize time series data for autoregressive prediction but inherit LLMs' limitations, such as poor arithmetic and recursive capabilities. Recent approaches, including GPT4TS~\cite{zhou2023one} and TimeLLM~\cite{jin2023time}, project time series into textual representations to leverage LLMs' reasoning abilities. However, they face challenges like the modality gap and lack of time series-optimized word embeddings, leading to potential information loss. UniTime~\cite{liu2024unitime} and TimeFFM~\cite{liu2024time} incorporate domain-specific instructions and federated learning, respectively, but remain constrained by their reliance on text alone.

\noindent\textbf{Vision-Augmented Models for Time Series Forecasting.} 

Vision emerges as a natural way to preserve temporal patterns. Early approaches use CNNs for matrix-formed time series~\cite{li2020forecasting, sood2021visual}, while TimesNet~\cite{wu2023timesnet} introduces multi-periodic decomposition for unified 2D modeling. VisionTS~\cite{chen2024visiontsvisualmaskedautoencoders} pioneers pre-trained visual encoders with grayscale time series images, and TimeMixer++~\cite{wang2024timemixer++} advances the field with multi-scale frequency-based time-image transformations. Despite their effectiveness in temporal modeling, these methods often lack semantic context, hard to use high-level contextual information for prediction.

\noindent\textbf{Vision-Language Models.} 

VLMs like ViLT~\cite{kim2021vilt}, CLIP~\cite{radford2021learning}, and ALIGN~\cite{jia2021scaling} transform multimodal understanding by aligning visual and textual representations. Recent advancements, like BLIP-2~\cite{li2022blip2} and LLaVA~\cite{liu2023visual}, further enhance multimodal reasoning. However, VLMs remain underexplored for time series analysis. Our work bridges this gap by leveraging VLMs to integrate temporal, visual, and textual modalities, addressing the limitations of unimodal approaches.

  \section{Method}
  \section{Method}
\label{sec:method}
In this section, we propose a neuroscience-informed fMRI encoder designed to achieve high-performance, subject-agnostic decoding. To further enable versatile decoding, we introduce the construction of a brain instruction tuning dataset, which captures diverse semantic representations encoded in fMRI data.

\subsection{Method Overview}
As illustrated in Figure~\ref{fig:arch}, our model consists of an fMRI encoder $f_\theta$ and an off-the-shelf LLM. In practice, we use Vicuna-7b \cite{zheng2023judging} as our LLM to maintain consistency with our baseline \cite{xia2024umbrae}. For each sample, let $\boldsymbol{v} = [v_1, v_2, \cdots, v_N]\in \mathbb{R}^N$ be the fMRI signals of input voxels, where $N$ is the number of voxels. Note that $N$ varies between different subjects, ranging from $12,682$ to $17,907$ in the dataset we use \cite{allen2022massive}.

The fMRI encoder $f_\theta$, featuring a neuroscience-informed attention layer, encodes $\boldsymbol{v}$ to fMRI tokens $X_v = [\boldsymbol{x}_{v,1}, \boldsymbol{x}_{v,2}, \cdots, \boldsymbol{x}_{v,L}] \in \mathbb{R}^{d\times L}$, where $L$ is the number of tokens and $d$ is the dimension of token embeddings. We then prepend these learned fMRI tokens to the language tokens in the BIT dataset we propose.

\subsection{fMRI Encoder}
As mentioned before, currently most models for fMRI decoding can not handle varying input shapes and are not subject-agnostic, with only a few exceptions \cite{mai2023unibrain}. However, these exceptions still suffer from information loss and uneven representations of certain brain areas. To this end, we propose a novel neuroscience-informed attention mechanism to accommodate varying voxel numbers across subjects, enabling a subject-agnostic encoding strategy. Below we talk about the design of \textit{queries} $\{\boldsymbol{q}_i\}$, \textit{keys} $\{\boldsymbol{k}_i\}$ and \textit{values} $\{\boldsymbol{v}_i\}$ in the attention layer. For \textit{values}, we directly use the fMRI signal of each voxel, which means $\boldsymbol{v_i} = v_i \in \mathbb{R}$. Making each voxel a \textit{value} token maximally prevents information loss compared to pooling- \cite{wang2024mindbridge} or sampling-based \cite{mai2023unibrain} methods. The \textit{queries} are randomly initialized and learnable. We expect each \textit{query} to represent a certain pattern of the brain (refer to visualizations in Section \ref{sec:vis}). The design of \textit{keys} will be discussed below.

\noindent\textbf{Exclude fMRI values from \textit{keys}}
The vanilla cross attention \cite{zhu2020deformable,vaswani2017attention} derives both \textit{keys} and \textit{values} from the same input source. However, we found this would lead to poor performance in fMRI. We argue the reason: different from images or text, which are usually considered translation-invariant, the positions of voxels carry specific brain \textit{functional information}, as voxels in different areas are associated with distinct brain functions. Consequently, a voxel's position alone can theoretically serve as effective \textit{keys} for attention weight computation. Including fMRI values into \textit{keys}, however, introduces additional noise instead of valuable information, thus resulting in poorer performance. Moreover, since brain regions tend to serve similar functions across individuals, decoupling voxel positions from fMRI signals can facilitate the sharing of priors across subjects, potentially improving generalization to unseen subjects.

In light of this, instead of the vanilla cross attention, which derives the \textit{keys} and \textit{values} from the same inputs, we exclude the fMRI value of each voxel and use its positional information alone as its \textit{key} embedding. The positional information is encoded from the coordinates of each voxel, i.e. $\boldsymbol{k}_i^{\text{pos}} = \operatorname{PE}(\boldsymbol{c}_i)$ for the $i$-th voxel, where $\boldsymbol{c}_i \in \mathbb{R}^3$ denotes the coordinates of the voxel. In practice, we use the Fourier positional encoding proposed in \cite{tancik2020fourier} due to its superiority in encoding coordinate information.

\noindent\textbf{Incorporation of Brain Parcellations}
% \noindent\textbf{Incorporation of Brain Parcellations}
While positional encoding alone improves performance, it lacks inherent neuroscientific grounding, potentially making it challenging for the model to efficiently learn representations aligned with established principles of brain function. To overcome this, we incorporate existing brain region parcellations \cite{glasser2016multi,rolls2020automated} into the \textit{key} embeddings. Formally, given a parcellation $\mathcal{P}$, with regions indexed by $1, \cdots, N_\mathcal{P}$. Let $\mathcal{P}(i) \in [1, 2, \cdots, N_\mathcal{P}]$ be the region that the $i$-th voxel belongs to, and $E[\mathcal{P}(i)] \in \mathbb{R}^d$ be the corresponding learnable embedding of the region, which will be incorporated in the \textit{key} embeddings as $\boldsymbol{k}_i^{\text{reg}, \mathcal{P}} = E[\mathcal{P}(i)] \in \mathbb{R}^d$.

\noindent\textbf{Combining Multiple Parcellations}
It is crucial to choose an appropriate brain region parcellation. Previous region-based methods \cite{qiu2023learning,li2021braingnn, kan2022brain} can usually only choose one arbitrarily. In contrast, our model design allows us to combine multiple parcellations $\mathcal{P}^1, \mathcal{P}^2, \cdots$ by concatenating their respective region encodings to the \textit{key} embeddings. In conclusion, the final \textit{key} embeddings are the concatenation by the positional encoding and multiple region encodings,
\begin{equation}
    \boldsymbol{k}_i = \boldsymbol{k}_i^\text{pos} \| \boldsymbol{k}_i^{\text{reg}, \mathcal{P}^1} \|  \boldsymbol{k}_i^{\text{reg}, \mathcal{P}^2} \| \cdots
\end{equation}
where $\|$ denotes the concatenation operation. This process is illustrated in Figure~\ref{fig:arch}'s lower right part.

The positional and region encodings complement each other: The region encodings serve as coarse-scale features, providing a neuroscientific-grounded basis, while the fine-scale positional encoding allows our model to learn finer-grained information directly from the data.

This attention design separates a voxel's \textit{functional information}—which is largely consistent across individuals—from its fMRI value, thereby enhancing generalization. Instead of relying on pooling or sampling, the attention mechanism employs learnable aggregation, while the integration of positional encoding and neuroscientifically informed region encodings further ensures high performance.

After the attention layer, we obtain the hidden representations $\boldsymbol{z}_q \in \mathbb{R}^{N_q} $ where $N_q$ is the number of query embeddings. We then employ an MLP and a reshape operation to map the hidden representations to $L$ fMRI tokens, i.e., $   X_v = \operatorname{reshape}\left( \operatorname{MLP}(
    \{\boldsymbol{z}_q\}
    ) \right) \in \mathbb{R}^{L \times d}$.

The process of the fMRI encoder is illustrated in Figure~\ref{fig:arch}. The obtained fMRI tokens are then prepended to the language tokens in conversations.
\begin{figure}
    \centering
    \includegraphics[width=\linewidth]{figures/arch.pdf}
    % \vspace{-2.2em}
    \caption{Model Architecture. The fMRI encoder maps fMRI to a series of fMRI tokens through our proposed neuroscience-informed attention. The large language model, with both fMRI and text tokens, will be trained by brain instruction tuning.}
    \label{fig:arch}
    \vspace{-1em}
\end{figure}

\subsection{Brain Instruction Tuning (BIT)}
To enable versatile fMRI-to-text decoding, an appropriate BIT dataset is required, yet no such dataset currently exists. To bridge this gap, we construct one based on the fact: MSCOCO images \cite{chen2015microsoft} serve as stimuli for fMRI recordings in the fMRI study \cite{allen2022massive}, and an abundance of datasets provide text annotations (e.g., VQA) for MSCOCO images. Using the images as intermediaries, we select those relevant to brain functions and pair the fMRI data with corresponding text annotations. For example, given an image of a billboard with annotated textual content, we can reasonably infer that when a subject perceives textual information (e.g., contents on the billboard), corresponding representations are encoded in the brain. This suggests the possibility of extracting such information from fMRI signals. We select datasets to fulfill various purposes, enabling the model to capture diverse aspects of semantic information embedded in fMRI signals, including visual perception \& scene understanding, language \& symbolic processing, memory \& knowledge retrieval and complex reasoning, which are considered among most fundamental and essential properties of human brains \cite{robertson2002memory,stenning2012human,wade2013visual,friederici2017language}.

\begin{figure}[h]
% \vspace{-0.5em}
    \centering
    \includegraphics[width=\linewidth]{figures/bit.pdf}
\vspace{-1.8em}
    \caption{Dataset Taxonomy in Brain Instruction Tuning.}
    \label{fig:bit}
% \vspace{-1em}
\end{figure}

\noindent\textbf{Perception \& Scene Understanding} As illustrated in Figure~\ref{fig:bit}, we begin by using caption tasks at both coarse and fine-grained levels to train the model’s ability to understand and summarize what the subject perceives visually \cite{chen2015microsoft,krause2017hierarchical}. Additionally, we incorporate QA tasks \cite{ren2015exploring,krishna2017visual,acharya2019tallyqa} to enhance the model's ability to retrieve and reason about visually perceived content.

\noindent\textbf{Memory \& Knowledge Retrieval} To go beyond tasks directly related to present visual perception, we construct the \emph{previous captioning} task, a memory-oriented task that challenges the model to caption images that the subject previously viewed, simulating memory recall processes. Furthermore, we aim to encode knowledge structures in human brains. The OK-VQA \cite{marino2019ok} and A-OKVQA \cite{schwenk2022okvqa} datasets include questions requiring external knowledge that is not present in the image but resides in human brains. For example, A photo of a hydrant may prompt the answer "firetruck," even though the firetruck is absent in the image. This association also reflects the way human cognition operates through a network of interconnected meanings, where one concept unconsciously triggers another. Such a process, which is called "slippage of the signifier" \cite{lacan2001ecrits, lacan1988seminar, miller2018four}, highlights the symbolic processes through which the brain constructs and retrieves meaning. 

\noindent\textbf{Language \& Symbolic Processing} In addition to the aforementioned OK-VQA and A-OKVQA datasets, which are also related to symbolic process, we further combine datasets of text recognition \cite{biten2019scene} and numerical reasoning \cite{acharya2019tallyqa} to facilitate this aspect.

\noindent\textbf{Complex Reasoning} Finally, we try to approximate the reasoning process that happens in human brains with datasets \cite{liu2023visual,wang2023see,li2018vqa} that require intricate logical and inferential processes. We expect these datasets to challenge the model to extract the reasoning process, drawing upon both visual understanding and abstract problem-solving, thus bridging perception, memory, and knowledge into a cohesive cognitive framework.

We ended up with a brain instruction tuning dataset consisting of $980,610$ conversations associated with fMRI recordings from $15$ datasets. Appendix~\ref{app:dataset} lists the instructions and other details for each dataset. The instruction tuning enables versatile fMRI-to-text decoding. In particular, the introduction of tasks like \textit{previous caption} empowers the model to perform a broader range of tasks beyond vision-related ones, which the previous model \cite{xia2024umbrae} fails.

\begingroup
\sisetup{
  table-format=2.2,  % 3 digits before the decimal, 2 after
  table-align-text-pre=false,
  propagate-math-font=true,
  table-number-alignment=center,
  detect-weight=true,detect-inline-weight=math
}
\begin{table*}[bp]
    \centering
    \vspace{-1.7em}
    \caption{Results of brain captioning. The CIDEr metric is scaled by a factor of 100 for consistency with Table~\ref{tab:caption} and baselines.}
    \label{tab:caption}
\vspace{0.1in}
    \resizebox{\linewidth}{!}{
    \begin{tabular}{lcSSSSSSSS}
    \toprule
 % \multirow{2}{*}{Method}&  \multirow{2}{*}{cross-subject}&\multicolumn{5}{c}{fMRI caption} & &  &\\
   {Method} & {\makecell{subject\\agnostic}}  &{{BLEU-1} $\uparrow$} & {BLEU-2 $\uparrow$} & {BLEU-3 $\uparrow$} & {{BLEU-4} $\uparrow$} &{METEOR $\uparrow$}&{ROUGE $\uparrow$}& {CIDEr $\uparrow$}&{SPICE $\uparrow$}\\
    \midrule
    SDRecon \cite{takagi2023high}    & {\xmark} &36.21 & 17.11 & 7.22 & 3.43   &10.03&  25.13&13.83 &5.02 \\
    OneLLM  \cite{han2024onellm}  & {\xmark} &47.04 & 26.97 & 15.49 & 9.51   &13.55&  35.05&22.99 & 6.26\\
    UniBrain \cite{mai2023unibrain}   & {\xmark} & {$-$}   & {$-$}    & {$-$}  & {$-$}     &16.90&  22.20& {$-$} & {$-$}\\
    BrainCap \cite{ferrante2023brain}  & {\xmark} &55.96 & 36.21 & 22.70 & 14.51   &16.68& 40.69&41.30 & 9.06\\
     BrainChat \cite{huang2024brainchat} & {\xmark}   &52.30& 29.20& 17.10& 10.70 &14.30& 45.70&26.10 & {$-$}\\
    UMBRAE \cite{xia2024umbrae}    & {\xmark} &59.44& 40.48& 27.66&19.03&19.45&  43.71&61.06&12.79\\
    \name{} (Ours)  & {\cmark} & \bfseries 61.75 &  \bfseries42.84 & \bfseries29.86&\bfseries21.24  & 
\bfseries 19.54 &\bfseries45.82 & 60.97  & 11.79\\
    \bottomrule
    \end{tabular}}
\end{table*}
\endgroup


To train the model with the BIT dataset, for each sample $\boldsymbol{v}$, we sample a multi-run conversation $X_t = (X_u^1, X_a^1, \cdots, X_u^T, X_a^T)$ from all conversations associated with it, where $T \geq 1$ represents the number of turns. $a$ indicates the message from the assistant and $u$ indicates the message is from the user. The training objective is to maximize the probability of the assistant's response only
$$
\arg\max_\theta p(X_a | X_v, X_{\text{inst}}) = \prod_{t=1}^T p({\color{magenta}X_a^t} | X_u^{\leq t}, X_a^{\le t }, X_\text{inst}, X_v)
$$
Figure~\ref{fig:chat} illustrates the chat template and the training objective. We freeze the weights of the LLM and only train the fMRI encoder since we want to preserve the LLM's language modeling prior and ensure a fair comparison with baselines such as \citet{xia2024umbrae}.

\noindent\textbf{Computational Complexity} According to the analysis in Appendix~\ref{app:complexity}, our model does not introduce additional complexity compared to previous methods \cite{scotti2024mindeye2, wang2024mindbridge}.


\begin{figure}[htbp]
\vspace{-0.8em}
\centering
\begin{minipage}{0.99\columnwidth}\vspace{0mm}    \centering
\begin{tcolorbox}[colback=white,colframe=gray,left=1pt,top=1pt,bottom=1pt]
\sffamily
\footnotesize	
  \texttt{<system message>}\\
  user: $X_v$, $X_\text{inst}$, $X_1^u$ \\
  assistant: {\color{magenta}$X_1^a$}\\
user: $X_2^u$\\
  assistant: {\color{magenta}$X_2^a$}\\
  $\cdots\cdots$
\end{tcolorbox}
\end{minipage}
\caption{The chat template used during instruction tuning, illustrating two turns of conversations. Two turns of conversations are shown. Tokens highlighted in {\color{magenta}magenta} are used for next-token prediction loss computation.}
\label{fig:chat}
\vspace{-1.2em}
\end{figure}



  \section{Experiments}
  \label{sec:experiments}
  \section{Experiments}
In this section, we first evaluate our model on various downstream tasks, demonstrating its versatile decoding capabilities. Next, we assess its generalizability to novel subjects and its adaptability to real-world applications. Finally, we analyze the functions of queries in our neuroscience-informed attention mechanism.

\subsection{Settings}

\noindent\textbf{fMRI Datasets} 
We use the widely used Natural Scenes Dataset (NSD) \cite{allen2022massive}, a large-scale dataset consisting of fMRI measurements of $8$ healthy adult subjects. During data collection, subjects viewed images from the MS-COCO dataset \cite{lin2014microsoft} and were instructed to press buttons to indicate whether they had previously seen each image.

% \textit{Instruction datasets}
% COCO Caption \cite{chen2015microsoft}, Paragraph Captioning \cite{krause2017hierarchical},
% COCO QA \cite{ren2015exploring}, 
% VQAv2 \cite{goyal2017making}, Visual Genome \cite{krishna2017visual}, A-OKVQA \cite{schwenk2022okvqa}, ST-VQA \cite{biten2019scene}, OK-VQA \cite{marino2019ok}, \cite{acharya2019tallyqa}, VQA-E \cite{li2018vqa}, FSVQA \cite{shin2016color}
% VisDial \cite{murahari2019improving},
% LVIS-instruct4v \cite{wang2023see}, LLaVA Instruct 150K \cite{liu2023visual}

\noindent\textbf{Downstream Datasets} The downstream dataset will be discussed within each experiment section. See examples and a short description for all dataset we will use in Appendex~\ref{app:dataset}.
Implementation details could be found in Appendix~\ref{app:impl}.

\begingroup
\sisetup{
  table-format=3.2,  % 3 digits before the decimal, 2 after
  table-align-text-pre=false,
  propagate-math-font=true,
  table-number-alignment=center,
  detect-weight=true,detect-inline-weight=math
}
\def\Uline#1{#1\llap{\uline{\phantom{#1}}}}
\begin{table*}[t]\centering
\vspace{-1em}
\caption{Versatile decoding. A dash $-$ means the model could not perform this task. The superscript $^\circ$ means the model is trained from scratch in contrast to their BIT version. The CIDEr metric is scaled by a factor of 100 for consistency with Table~\ref{tab:caption} and baselines.}
\vskip 0.1in
\label{tab:versatile}
\resizebox{\linewidth}{!}{
\begin{tabular}{cc|SSS|SSS|SSS}\toprule
& & {OneLLM} & {UMBRAE} &{BrainChat} & {MindBridge$^\circ$} & {UniBrain$^\circ$} & {\makecell{\name{}}$^\circ$} &{MindBridge} & {UniBrain} & {\makecell{\name{}}} \\
\midrule
subj-agnostic &  & {\xmark}& {\xmark} & {\xmark} & {\cmark} & {\cmark} & {\cmark} &{\cmark} & {\cmark} & {\cmark} \\
% subj-agnostic &  & {\xmark}& {\xmark} & {\xmark} & {\xmark} & {\cmark} & {\cmark} &\xmark & {\cmark} & {\cmark} \\
\midrule
COCO-QA &Accuracy$\uparrow$ &11.09\% & 22.23\% & 39.44\% &40.19\% &38.38\% &42.09\% &\Uline{45.33\%} &42.00\% & \bfseries 48.19\% \\
\midrule
VG-QA &Accuracy$\uparrow$ & 8.76\% & 19.67\% & 21.00\% &20.84\% &21.27\% &21.68\% &23.53\% & \Uline{24.02\%} &\bfseries 24.06\% \\
\midrule
VQA-v2 &Accuracy $\uparrow$& 33.68\% & \Uline{51.23}\% & 40.02\% &43.25\% &46.04\% &44.13\% &47.91\% &48.58\% &\bfseries 52.14\% \\
\midrule
A-OKVQA &Accuracy $\uparrow$&25.23\%& 43.24\% &20.52\% &22.12\% &19.47\% &29.20\% & \Uline{50.44\%} &43.36\% &\bfseries 52.21\% \\
\midrule
ST-VQA &ANLS $\uparrow$& 5.74\%& 5.46\% &9.58\% &10.20\% &7.01\% &\Uline{12.76}\% &11.64\% &8.76\% &\bfseries 12.92\% \\
\midrule
OK-VQA &Accuracy $\uparrow$&22.98\% & 10.35\% &17.22\%&27.63\% &18.63\% &27.70\% &32.13\% &\Uline{32.30\%} & \bfseries 33.33\% \\
\midrule
\multirow{2}{*}{TallyQA} &Accuracy $\uparrow$& 8.34\% & 44.10\% & 43.22\%&43.49\% &44.83\% &43.75\% &49.46\% & \Uline{53.77\%} &\bfseries 54.76\% \\
&RMSE $\downarrow$&7.45 & 3.94 & 1.90 &2.03 &1.83 &2.04 &1.86 &\bfseries 1.67 &\Uline{1.76} \\
\midrule
\multirow{6}{*}{Paragraph Caption} &BLEU-1$\uparrow$ &0.26 & \bfseries 29.82 & 22.21&21.82 &25.69 &26.49 &25.69 &28.28 & \Uline{29.43} \\
&BLEU-2 $\uparrow$ &0.08 & 14.26 &10.23 &10.47 &12.62 &12.48 &13.00 &\Uline{15.47} & \bfseries 15.78 \\
&BLEU-3$\uparrow$ &0.03 & 6.52 &6.38 &5.58 &6.70 &6.43 &7.10 &\Uline{8.90} & \bfseries 9.14 \\
&BLEU-4$\uparrow$ &0.01 & 2.95 &2.12 &3.14 &3.81 &3.63 &4.22 &\bfseries5.60 & \Uline{5.51} \\
&METEOR $\uparrow$&2.36 &12.60 &9.10 &10.95 &11.13 &2.44 &3.56 &\bfseries13.50 & \Uline{13.18} \\
&CIDEr $\uparrow$&0.00 & 7.39 &6.02&7.50 &3.92 &\Uline{10.71} &\bfseries11.39 &1.82 & 7.80 \\
\midrule
\multirow{8}{*}{VQA-E} &Accuracy $\uparrow$ &19.60\%& 47.84\% &46.20\% &45.40\% &44.42\% &44.55\% &\Uline{48.48\%} &48.39\% &\bfseries 50.95\% \\
&BLEU-1 $\uparrow$&17.32& 29.83 & 35.99 &35.63 &35.30 &35.08 &36.18 &\Uline{37.26} &\bfseries 37.70 \\
&BLEU-2 $\uparrow$&7.44& 14.76 & 18.33  &18.27 &18.04 &17.82 &19.38 &\Uline{20.41} &\bfseries 20.56 \\
&BLEU-3 $\uparrow$&3.62& 8.17 & 10.01 &10.32 &10.20 &10.05 &11.30 &\Uline{12.25} &\bfseries 12.34 \\
&BLEU-4 $\uparrow$&1.82& 4.87 &6.60 &6.27 &6.14 &6.00 &7.00 &\Uline{7.83} &\bfseries 7.92 \\
&CIDEr $\uparrow$&19.32& 63.26 &78.33 &79.05 &77.31 &76.80 &86.62 &\Uline{92.09} & \bfseries 93.60 \\
&METEOR $\uparrow$&6.69 & 12.25 &13.64 &14.13 &13.89 &13.96 &14.81 &\Uline{15.51} &\bfseries 15.62 \\
&ROUGE $\uparrow$&16.84& 28.38 & 32.82&33.78 &33.25 &33.11 &34.56 &\Uline{35.87} &\bfseries 35.88 \\
\midrule
\multirow{8}{*}{FSVQA} &VQA Acc. $\uparrow$& 31.44\% & 40.67\% &36.30\% &42.00\% &37.05\% &42.53\% &\Uline{45.95\%} &44.58\% &\bfseries 48.03\% \\
&FSVQA Acc. $\uparrow$& 21.02\% & 0.00\% & 30.22\% &37.40\% &32.30\% &38.50\% &\Uline{40.97\%} &37.87\% &\bfseries 43.00\% \\
&BLEU-1 $\uparrow$& 37.42 & 23.11 & 83.99&85.68 &83.84 &85.88 &\Uline{86.52} &85.10 &\bfseries 87.10 \\
&BLEU-2 $\uparrow$ & 31.72 & 5.86&78.50 &81.27 &78.81 &81.62 &\Uline{82.28} &80.01 &\bfseries 83.03 \\
&BLEU-3 $\uparrow$& 26.95 & 2.10&73.00 &77.10 &73.97 &77.62 &\Uline{78.34} &75.49 &\bfseries 79.27 \\
&BLEU-4 $\uparrow$& 22.48 & 1.04 &69.73 &72.89 &68.91 &73.56 &\Uline{74.35} &70.73 &\bfseries 75.50 \\
&METEOR $\uparrow$& 26.35 & 8.93 &44.76 &47.59 &45.94 &47.96 &\Uline{48.63} &46.89 &\bfseries 49.05 \\
&CIDEr $\uparrow$& 312.75 & 4.07 &600.00 &636.40 &609.00 &646.26 &\Uline{657.02} &628.83 &\bfseries 666.26 \\
\midrule
\multirow{8}{*}{Previous Caption} & BLEU-1 $\uparrow$ & 41.86  & {$-$} & 21.19 & 21.17 & 24.84 & \Uline{44.52} & 42.45 & 43.01 & \bfseries 47.20 \\
&BLEU-2 $\uparrow$ & 19.44 & {$-$} & 8.00 & 7.57 & 9.70 & \Uline{22.46} & 20.04 & 20.03 & \bfseries 25.16\\
&BLEU-3 $\uparrow$ & 9.25 & {$-$} & 1.98 & 2.85 & 3.40 & \Uline{10.39} & 9.61 & 9.19 & \bfseries 12.95\\
&BLEU-4 $\uparrow$ & 3.67 & {$-$} & 1.02& 1.28 & 1.46 & \Uline{5.45} &  5.31 & 4.58 & \bfseries 7.49 \\
&METEOR $\uparrow$ & 10.14 & {$-$} & 6.55& 6.46 & 7.20 & \Uline{11.00} & 10.83 & 10.81 & \bfseries 11.96 \\
&ROUGE $\uparrow$ & 30.19 & {$-$} & 21.23& 20.88 & 23.04 & \Uline{33.20} &32.38 & 31.99 & \bfseries 34.58 \\
&CIDEr $\uparrow$ & 6.65 & {$-$} & 9.21& 8.83 & \Uline{11.73} & 9.39 & 7.89 & 7.53 & \bfseries 16.02 \\
&SPICE $\uparrow$ & 2.49 & {$-$} & 2.44& 2.56 & 2.78 & \Uline{3.07} & 2.80 & 2.92 & \bfseries 3.93 \\
\bottomrule
\end{tabular}}
\vspace{-1em}
\end{table*}
\endgroup

\subsection{Brain Captioning}
To evaluate the model's performance on downstream tasks, we start with the widely used brain captioning benchmark \cite{xia2024umbrae}. The task, built upon COCO Caption \cite{chen2015microsoft} requires the model to predict captions of given images as fMRI stimuli.

\noindent\textbf{Baselines}
The following baselines are considered in this experiment: SDRecon \cite{takagi2023high}, UniBrain \cite{mai2023unibrain}, and BrainCap \cite{ferrante2023brain} employs a linear regression, mapping the fMRI to the inputs of an image caption model \cite{li2023blip}. OneLLM \cite{han2024onellm} is a multimodal large language models that align $8$ modalities (including fMRI) with language all in one model. For fair and efficient comparison, we only finetune the encoder, given that we freeze the LLM in our method as well. UMBRAE learns an encoder that maps fMRIs to images through an encoder similar to the MLP mixer \cite{tolstikhin2021mlp}. BrainChat \cite{huang2024brainchat} segments the flattened voxels into 16 patches and employs a transformer to decode text conditioned on the patches.
It is worth noting that all of these baselines require subject-specific layers or parameters. In contrast, our model is subject-agnostic, thus with the potential to generalize on novel subjects.

\noindent\textbf{Metric} Following previous works, we use five standard metrics for text generation: BLEU-$k$ \cite{papineni2002bleu}, ROUGE-L \cite{lin2004rouge}, CIDEr \cite{vedantam2015cider}, SPICE \cite{anderson2016spice}, METEOR \cite{banerjee2005meteor}.

Table \ref{tab:caption} shows that our model outperforms baselines in terms of most metrics, with an average improvement of $3.32\%$, even if our model does not have any subject-specific layers. We argue that this is attributed to both the novel architecture design and the introduction of BIT, which will be evident in the next experiment.

\begingroup
\sisetup{
  table-format=3.2,  % 3 digits before the decimal, 2 after
  table-align-text-pre=false,
  table-number-alignment=center,
detect-weight=true,detect-inline-weight=math
}
\begin{table}[t]
\vspace{-1em}
\centering
\caption{Model generalization, compared with subject-agnostic model. We train the models on subject $1-7$ and evaluate on subject $8$, which is the held-out subject.}
\label{tab:subj8}
\vskip 0.1in
\resizebox{\linewidth}{!}{
\begin{tabular}{ccSSS}\toprule
& & {MindBridge} & {UniBrain} & {\name{}} \\
\midrule
COCO-QA &Accuracy $\uparrow$&35.88 &24.95 &\bfseries 38.75 \\
\midrule
VG-QA &Accuracy $\uparrow$&\bfseries 20.56 &16.23 & 18.81 \\
\midrule
VQA-v2 &Accuracy $\uparrow$&42.80 &40.16 &\bfseries 44.69 \\
\midrule
A-OKVQA &Accuracy $\uparrow$&44.55 &28.71 &\bfseries 45.54 \\
\midrule
ST-VQA &ANLS $\uparrow$&9.33 &9.30 &\bfseries 10.97 \\
\midrule
OK-VQA &Accuracy $\uparrow$&21.94 &17.09 &\bfseries 24.45 \\
\midrule
\multirow{2}{*}{TallyQA} &Accuracy $\uparrow$&38.92 &32.51 &\bfseries 41.28 \\
&RMSE $\downarrow$&2.12 &\bfseries 2.02 &2.16 \\
\midrule
\multirow{8}{*}{COCO-Caption} &BLEU-1 $\uparrow$&39.84 &41.90 &\bfseries 47.3 \\
&BLEU-2 $\uparrow$&19.55 &19.67 &\bfseries 25.35 \\
&BLEU-3 $\uparrow$&9.29 &8.89 &\bfseries 13.61 \\
&BLEU-4 $\uparrow$&5.24 &4.33 &\bfseries 8.15 \\
&METEOR $\uparrow$&10.39 &10.80 &\bfseries 11.4 \\
&ROUGE $\uparrow$&31.10 &31.54 &\bfseries 34.64 \\
&CIDEr $\uparrow$&\bfseries 8.70 &6.40 &6.41 \\
&SPICE $\uparrow$&2.67 &2.39 &\bfseries 3.61 \\
\midrule
\multirow{6}{*}{Paragraph Caption} &BLEU-1 $\uparrow$&23.18 &21.73 &\bfseries 27.21 \\
&BLEU-2 $\uparrow$&10.71 &8.94 &\bfseries 12.48 \\
&BLEU-3 $\uparrow$&4.61 &3.72 &\bfseries 5.81 \\
&BLEU-4 $\uparrow$&2.22 &1.92 &\bfseries 3.01 \\
&METEOR $\uparrow$&9.99 &9.47 &\bfseries 10.24 \\
&CIDEr $\uparrow$&0.71 &1.56 &\bfseries 4.05 \\
\midrule
\multirow{8}{*}{VQA-E} &Accuracy $\uparrow$&41.78 &38.53 &\bfseries 44.81 \\
&BLEU-1 $\uparrow$&32.54 &32.86 &\bfseries 34.46 \\
&BLEU-2 $\uparrow$&16.13 &15.48 &\bfseries 17.86 \\
&BLEU-3 $\uparrow$&8.82 &7.98 &\bfseries 10.23 \\
&BLEU-4 $\uparrow$&5.16 &4.42 &\bfseries 6.19 \\
&CIDEr $\uparrow$&68.13 &58.79 &\bfseries 77.36 \\
&METEOR $\uparrow$&12.74 &12.26 &\bfseries 13.56 \\
&ROUGE $\uparrow$&30.63 &29.38 &\bfseries 32.99 \\
\midrule
\multirow{8}{*}{FSVQA} &VQA Acc. $\uparrow$&42.33 &37.92 &\bfseries 43.65 \\
&FSVQA Acc. $\uparrow$&37.16 &30.83 &\bfseries 38.41 \\
&BLEU-1 $\uparrow$&75.81 &82.94 &\bfseries 85.69 \\
&BLEU-2 $\uparrow$&71.03 &77.24 &\bfseries 81.18 \\
&BLEU-3 $\uparrow$&66.40 &71.94 &\bfseries 77.06 \\
&BLEU-4 $\uparrow$&61.55 &66.54 &\bfseries 72.86 \\
&METEOR $\uparrow$&45.84 &45.24 &\bfseries 47.87 \\
&CIDEr $\uparrow$&428.39 &587.78 &\bfseries 641.11 \\
\bottomrule
\end{tabular}
}
\vspace{-2em}
\end{table}
\endgroup

\subsection{Versatile Decoding}
The purpose of experiments in this section is two-fold: 1) To investigate the impact of our model design and the introduction of BIT on performance improvement. 2) To evaluate the versatility of the model, i.e., its performance on various downstream tasks.

\noindent\textbf{Baselines} Besides baselines that could be adapted to this experiment from the previous one, we further consider the following subject-agnostic models as baselines.
1) MindBridge \cite{wang2024mindbridge} flatten the voxels and adaptively adjust the padding and stride to pool the voxels into a fixed dimension. The original implementation of MindBridge has subject-specific parameters. However, since those parameters are of the same size, we make them shared across subjects and thus make the model subject-agnostic.
2) UniBrain \cite{wang2024unibrain} samples voxels into a fixed number of groups and employs a transformer where groups are treated as tokens. This UniBrain is unrelated to the UniBrain in the previous section; they just share the same name.

\noindent\textbf{Datasets \& Metric}
We use the test split of all QA \& caption datasets in the BIT dataset. We strictly adhere to the official metrics on all datasets. In summary, for sentence generation, we use BLEU-$k$~\cite{papineni2002bleu}, ROUGE-L~\cite{lin2004rouge}, CIDEr~\cite{vedantam2015cider}, SPICE~\cite{anderson2016spice}, METEOR~\cite{banerjee2005meteor}. For QA-related tasks, we use VQA accuracy~\cite{antol2015vqa} as well as special metrics proposed in the original paper (e.g. ANLS for ST-VQA~\cite{biten2019scene}).

The results are shown in Table~\ref{tab:versatile}. Our model outperforms baselines, with an average improvement of $12.0\%$. Further, by comparing instruction tuning and from-scratch models, we find that instruction tuning has a significant positive effect, with an average improvement of $28.0\%$. The results remain stable across different random seeds; for instance, according to our observations, the BLEU-1 score for paragraph captioning exhibits a maximum of $\pm 0.3$ variance.

\begingroup
\sisetup{
  table-format=2.2,  % 3 digits before the decimal, 2 after
  table-align-text-pre=false,
  table-number-alignment=center,
detect-weight=true,detect-inline-weight=math
}
\begin{table}[htbp]
\vspace{-1.5em}
    \caption{Model adaptation to new tasks. \textit{sentiment understanding} and \textit{utility/affordance} are sub-datasets from TDIUC that are particularly relevant to BCI applications.}
    \label{tab:new_task}
\vskip 0.1in
    \centering
    \resizebox{\linewidth}{!}{
    \begin{tabular}{cSSSS}
    \toprule
    \multirow{2}{*}{Method} &  \multicolumn{2}{c}{Overall} & {Sentiment Understanding} & {Utility/Affordance} \\
    \cmidrule{2-3} \cmidrule{4-4} \cmidrule{5-5}
    & {A-MPT} & {H-MPT} & {Accuracy} & {Accuracy} \\
    \midrule
    \name{}$^\circ$ & 41.09\% & 19.38\% & 70.00\% & 0.00\%\\
    \midrule
         MindBridge & 49.77\% & 39.88\% & 80.00\% & 14.29\%\\
         UniBrain & 51.50\% & 36.76\% & 80.00\% & 28.57\%\\
         \name{}& \bfseries 54.08\% &  \bfseries 45.43\% & \bfseries 80.77\% & \bfseries 50.00\% \\
    \bottomrule
    \end{tabular}
    }
    \vspace{-1em}
\end{table}
\endgroup


\subsection{Unseen Subject Generalization}
Our neuroscience-informed, subject-agnostic design enhances generalization to novel subjects, a crucial factor in real-world applications where training a model for each individual is impractical. To evaluate it, we perform instruction tuning on $7$ out of the $8$ subjects in the natural scene dataset \cite{allen2022massive}, and evaluate generalization on the held-out subject. Table~\ref{tab:subj8} shows our model outperforms two other subject-agnostic baselines in most cases, with an average improvement of $16.4\%$ compared to the second-best model.

\subsection{Adapting to New Tasks}
It is common that users want to adapt the \name{} to their own specific use cases. To this end, we aim to assess our model's adaptability to new tasks.

\noindent\textbf{Dataset \& Metrics} We use TDIUC \cite{kafle2017analysis}, a QA dataset consisting of $12$ types of questions, as a benchmark to evaluate the model's various capabilities comprehensively. Additionally, we further select $2$ task types-\textit{sentiment understanding} and \textit{utility/affordance} tasks, that are particularly relevant to BCI applications as sub-datasets. The \textit{utility/affordance} task, for instance, enables the model to identify useful objects in a given scene and autonomously decide whether to utilize them. Following their paper, we compute the accuracy of each type and report the arithmetic mean-per-type (A-MPT) and the harmonic mean-per-type (H-MPT). For the $2$ selected types, we report the accuracy respectively. 

Table~\ref{tab:new_task} shows our model achieves balanced (high harmonic mean) and consistently improved performances with an average of $13.5\%$. We could also observe the performance benefits from BIT, with $25.0\%$ absolute improvement.

\begin{figure*}[t]
    \begin{subfigure}[c]{0.15\textwidth}
        \centering
        \includegraphics[width=\textwidth]{figures/brain_query/brain_query_ppa.pdf}
        % \caption{Query: PPA}
        \vspace{-1.5em}
        \caption{}
        \label{sub:ppa}
    \end{subfigure}
    \hfill
    \begin{subfigure}[c]{0.15\textwidth}
        \centering
        \includegraphics[width=\textwidth]{figures/brain_query/brain_query_ffa_1.pdf}
        % \caption{Query: FFA-1}
        \vspace{-1.5em}
                \caption{}
        \label{sub:ffa-1}
    \end{subfigure}
    \hfill
    \begin{subfigure}[c]{0.15\textwidth}
        \centering
        \includegraphics[width=\textwidth]{figures/brain_query/brain_query_opa_ofa.pdf}
        % \caption{Query: FFA-1}
        \vspace{-1.5em}
        \caption{}
        \label{sub:opa_ofa}
    \end{subfigure}
    \hfill
    \begin{subfigure}[c]{0.15\textwidth}
        \centering
        \includegraphics[width=\textwidth]{figures/brain_query/brain_query_earlyvis_eba.pdf}
        % \caption{Query: FFA-1}
        \vspace{-1.5em}
        \caption{}
        \label{sub:ev_eba}
    \end{subfigure}
    \hfill
    \begin{subfigure}[c]{0.15\textwidth}
        \centering
        \includegraphics[width=\textwidth]{figures/brain_query/brain_query_multi_low_high.pdf}
        % \caption{Query: FFA-1}
        \vspace{-1.5em}
        \caption{}
        \label{sub:low_high}
    \end{subfigure}
    \hfill
    \begin{subfigure}[c]{0.15\textwidth}
        \centering
        \includegraphics[width=\textwidth]{figures/brain_query/brain_query_multi_high_high.pdf}
        % \caption{Query: FFA-1}
        \vspace{-1.5em}
        \caption{}
        \label{sub:high_high}
    \end{subfigure}
    \hfill
    \begin{subfigure}[c]{0.05\textwidth}
        \centering
        \includegraphics[width=\textwidth]{figures/brain_query/colorbar.png}
    \end{subfigure}
    \vspace{-1.4em}
    \caption{Visaulization of attention weights between \textit{queries} and brain voxels. Each subfigure represents a \textit{query} token, and the strength of color indicates its attention weight (after min-max normalization) to each voxel. 
    }
    \label{fig:query}
    \vspace{-1.2em}
\end{figure*}

\subsection{Ablation Study}
We conduct ablation studies on the design of key embeddings in the neuroscience-informed attention module in Figure~\ref{fig:ablation}. The results strongly validate our design. The vanilla cross attention (\textit{Pos Enc.+fMRI}) leads to poor performance while removing fMRI values from the key embeddings (\textit{Pos Enc.}) yields a significant improvement. Replacing positional encoding with region encodings (\textit{Reg. Enc.}) accelerates convergence in the early stages since it is grounded by neuroscientific principles. However, it is eventually outperformed by \textit{Pos Enc.} due to the lack of finer-grained information. Combining the positional encoding and region encodings (\textit{Pos Enc.+Reg Enc.}) achieves the best model design. In addition, replacing positional encoding with an MLP that maps coordinates to embeddings results in poor performance (\textit{(x,y,z)+MLP}), which indicates the amount of high-frequency spatial information in fMRI signals.

\begin{figure}[h]
\vspace{-0.8em}
    \centering
    \includegraphics[width=\linewidth]{figures/ablation.pdf}
    \vspace{-2em}
    \caption{Ablation study of the key embedding design. Pos Enc. stands for positional encoding. Reg Enc. stands for multiple region encodings. \textit{(x, y, z)+MLP} means we employ an MLP to map the coordinates to the embeddings instead of positional encoding.}
    \label{fig:ablation}
    \vspace{-1.3em}
\end{figure}
\subsection{Visualizations and Interpretations}
\label{sec:vis}



% \begin{figure}
%     \centering
%     \includegraphics[width=\linewidth]{figures/fig_brain.pdf}
%     \caption{Voxel-Level Brain Mapping of Model Attention.(a)-(c) present the attention map on a flattened brain with deeper blue indicating higher value. (d) gives an average value rank of different brain regions from (a).}
%     \label{fig:brain}
% \end{figure}

% Interpretability is of high importance in brain-inspired research, providing critical cues for how to leverage brain information more efficiently in the future and how to correctly transfer models into the real-world application \cite{fellous2019explainable,farahani2022explainable}. Other than compared advanced brain-decoding methods in this paper, our neuroscience-informed framework is interpretable on the brain-level. Accordingly, we conducted analysis in this section to show our model’s capability in distilling complex voxel-level brain signals and locating informative regions for high-level semantic tasks. We extracted the average attention feature from $\mathbf{Q}$ and $\mathbf{K}$ before it interacted with voxel values $\mathbf{V}$ and mapped it back to the original 3D brain cortex, as Figure \ref{fig:brain} shows. Notably, compared to early visual areas, brain regions from higher-level information processing stages caught much more attention of the model. This pattern is different from previous brain-imaging decoding methods with interpretations, such as \cite{takagi2023high} and (), in which early visual regions were among the most notable brain areas. Early visual regions are important, initially processing visual information, passing it to next stages, yet not crucial for high-level cognitive conceptualization. MindLLM recognized a map mainly focusing on Parahippocampal Place Area (PPA), Fusiform Face Area (FFA) and Extrastriate Body Area (EBA). PPA locates in parahippocampal gyrus, related to conceptual association, semantic processing and enviornmental memory \cite{epstein1999parahippocampal,kohler2002differential,bar2008scenes,epstein2010reliable}. FFA is known for its critical role in expertise recognition, social cognition and identity memory \cite{schultz2003role,tsantani2021ffa,xu2005revisiting}. And EBA involves body perception and contextual reasoning \cite{urgesi2007representation,carey2019distinct}. Together, they form a pattern of processing and extracting distilled sophisticated objects and spatial information from outside world, producing conceptual semantic cognition, and even inferring hidden messages behind the visual scenes from the information human gets \cite{amoruso2011beyond}. This revealed pattern proves model’s ability in getting high-level accurate information from the brain and provides evidence about potential brain regions for multitask brain decoding as well as the possibility of implementing better brain decoding model design with richer brain signals.

% The design of the learnable \textit{queries} provides a way to dynamically integrate massive brain-anatomical information from the \textit{keys}. 
Unlike previous deep learning models \cite{scotti2024mindeye2,mai2023unibrain}, our model allows interpretations by investigating how \textit{queries} work in the neuroscience-informed attention layer. We inspect the attention weights between queries and voxels in Figure~\ref{fig:query}. 
% Weight values are scaled by min-max normalization and results are presented in Figure \ref{fig:query}. 

We found that some queries primarily focus on processing single brain regions, such as Parahippocampal Place Area (PPA) (Figure~\ref{sub:ppa}) and Fusiform Face Area (FFA) (\ref{sub:ffa-1}). As previous research has shown, PPA is related to conceptual association, semantic processing and environmental memory \cite{epstein1999parahippocampal,kohler2002differential,bar2008scenes,epstein2010reliable} and FFA is known for its critical role in expertise recognition, social cognition and identity memory \cite{schultz2003role,tsantani2021ffa,xu2005revisiting}. Both are important brain regions for the conceptualization of visual information and are responsible for the interaction between real-time stimulus and past memory \cite{brewer1998making,ranganath2004category,golarai2007differential}. 

Moreover, there are some queries that attend to multiple brain regions, revealing the information transmission between low- and high-level brain regions. For instance, interactions between early visual areas and higher-level regions like PPA and IntraParietal Sulcus (IPS) (Figure~\ref{sub:low_high}), revealing a potential pattern for human attention-guided actions \cite{tunik2007beyond,connolly2016coding}. Additionally, queries are also found responsible for communications between high-level brain regions (Figure \ref{sub:opa_ofa},\ref{sub:high_high}). Together, these findings indicate that the learnable queries may reflect the dynamics of human brain activities in the visual task, from seeing and thinking about the image to pressing the button for the visual recall task in NSD \cite{allen2022massive}. 

% This emphasizes the advantage of modeling interactions between brain-informed keys and queries in the context of parallel multitask working\rex{what is this multitask working?}.

% As brain regions defined in neuroscience research likely to be processed individually and interactively in our brain-decoding pipeline\rex{i don't understand. doesn't seem like a useful sentence}, our model is potentially aware of the biological properties of the human brain rather than mechanically mixing all values up in the black box, like \rex{we haven't compared. how do we support such claims?} \cite{scotti2024reconstructing,scotti2024mindeye2,jiang2024mindshot}.

We also provide qualitative analysis of model responses in Appendix~\ref{app:qual}.

% \begin{figure}
%     \centering
%     \includegraphics[width=\linewidth]{figures/brain_query_slice.pdf}
%     \caption{Slices of Model Attention by Query Index.}
%     \label{fig:query}
% \end{figure}

  \section{Analysis}
  \subsection{Evolution of Criteria Distribution}

\begin{figure*}[htbp]
    \centering
    \begin{subfigure}
        [b]{0.48\textwidth}
        \includegraphics[width=\linewidth]{figures/evolution.pdf}
        \caption{Distribution of accuracy.}
        \label{fig:evolution_acc}
    \end{subfigure}
    \hfill
    \begin{subfigure}
        [b]{0.48\textwidth}
        \includegraphics[width=\linewidth]{figures/refuse_rate.pdf}
        \caption{Distribution of refuse rate.}
        \label{fig:evolution_refuse_rate}
    \end{subfigure}
    \caption{Evolution of distributions of the top-$k$ Python code quality
    criteria through evolution iterations, where $k$ is the number of the final criteria.}
    \label{fig:criteria_evolution}
\end{figure*}

In this section, we analyze how the distribution of quality criteria evolves during
the evolution process. Using the code domain as a representative example, Figure~\ref{fig:evolution_acc}
shows the distribution of training accuracies for all criteria across optimization
iterations. The plot reveals a clear upward trend, with the distribution progressively
shifting and concentrating towards higher values as the optimization proceeds.
This trend demonstrates the effectiveness of our iterative optimization process.

Notably, several criteria achieve 100\% accuracy. As explained in Section~\ref{sec:voting},
we exclude the cases where the worker agent explicitly declines to provide a
judgment. Through the optimization process, the manager agent refines the
criteria descriptions to be more precise about their applicability. These highly
accurate criteria are particularly valuable as they effectively characterize code
quality and guide the worker agent to make accurate assessments when applicable,
even if they may not cover all possible scenarios.

In addition, we analyze the distribution of the refuse rate of the criteria. As
shown in Figure~\ref{fig:evolution_refuse_rate}, the refuse rate falls predominantly
in lower ranges, indicating that most criteria are widely applicable, while there
are still a few criteria with refuse rates higher than 60\% that are retained
due to their high accuracy when applicable.

\subsection{Criterion Refinement}
\label{sec:refinement}

The improvement in accuracy of CritiQ Flow is driven by two key processes:
deprecating low-quality criteria and refining the mid-quality criteria by
revising the descriptions. Deprecating the low-quality ones is something like reject
sampling, which is straightforward in improving performance. In this section, we
analyze how mid-quality criteria are refined by the manager agent.

We categorize the criteria refinement into 2 types: (1) refining the criteria retrieved
from the knowledge base or generated by the manager agent, and (2) continually
refining the already refined criteria. We show examples of criteria before and after
refinement in Appendix~\ref{sec:appendix_ex_refinement}.

\paragraph{Refinement for Retrieved or Generated Criteria.}
The knowledge base is built on previous dataset research, so the criteria retrieved
from the knowledge base are often too general. When the knowledge base can not
provide enough criteria or some criteria are deprecated due to low accuracy, the
manager agent proposes new criteria. In this case, the initial descriptions of these
criteria are usually too vague, because they have not been evaluated by the
worker agent, thus the manager agent does not have enough information to generate
precise descriptions. As a result, the manager agent can refine those criteria by
rewriting them to fit the current domain, adding detailed guidelines for the
worker agent, and specifying the applicability.

\paragraph{Refinement for Refined Criteria.}
For previously refined criteria, the manager agent can further improve them by adding
more detailed descriptions or examples. However, we also observe that despite the
iterative optimization process, refinements do not always yield higher accuracy,
especially for already well-refined criteria. Excessive refinement by the
manager agent can lead to over-fitting, particularly with small training sets.
To address this, we encourage the manager agent to keep the criteria simple and concise.

\subsection{Majority Voting}

We have demonstrated the majority voting mechanism in Section~\ref{sec:voting}. In
this section, we investigate the impact of the voting mechanism by evaluating
the accuracy of combining all criteria into a single prompt. We use the same quality
criteria derived by CritiQ Flow and query the worker agent for judgments. The
accuracies are shown in Table~\ref{tab:majority_voting}. In all domains, the
accuracy decreases without the majority voting mechanism, indicating that the majority
voting mechanism is essential for the performance of CritiQ Flow.

\begin{table}[htbp]
    \centering
    \begin{tabular}{lcccc}
        \toprule      & \textbf{Code} & \textbf{Math} & \textbf{Logic} & \textbf{Avg.} \\
        \midrule Ours & \textbf{89.33}         & \textbf{84.57}         & \textbf{88.06}          & \textbf{87.32}         \\
        w/o voting    & 84.16         & 81.14         & 85.22          & 83.51         \\
        \bottomrule
    \end{tabular}
    \caption{Accuracies with / without Majority Voting on the human-annotated
    $D_{\text{test}}$ across 3 domains. The higher values are in bold.}
    \label{tab:majority_voting}
\end{table}

  \section{Conclusion}
  \section{Conclusions}

We study theoretically the transfer of past experience in MCTS-based lifelong planning and develop a novel aUCT rule, depending on both Lipschitz continuity between tasks and the confidence of knowledge in Monte Carlo action sampling. The proposed aUCT is proven to provide positive acceleration in MCTS due to cross-task transfer and enable the development of a new lifelong MCTS algorithm, namely LiZero. We also present efficient methods for online estimation of aUCT and provide analysis on the sampling complexity and error bounds. LiZero is implemented on a non-stationary series of learning tasks with varying transition probabilities and rewards. It outperforms MCTS and lifelong RL baselines with 3$\sim$4x speed-up in solving
new tasks and about 31\% higher early reward.



\section*{Impact Statement}
This paper proposes a novel framework for applying Monte Carlo Tree Search (MCTS) in lifelong learning settings, addressing the challenges posed by non-stationary environments and dynamic game dynamics. By introducing the adaptive Upper Confidence Bound for Trees (aUCT) and leveraging insights from previous MDPs (Markov Decision Processes), our work significantly enhances the efficiency and adaptability of decision-making algorithms across evolving tasks.

The broader societal implications of this research include its potential to improve AI applications in robotics, automated systems, and other domains requiring dynamic decision-making under uncertainty. For instance, this framework could be used in autonomous systems to adaptively respond to changing environments, thereby improving safety and reliability. At the same time, it is crucial to acknowledge and mitigate potential risks, such as unintended biases or over-reliance on prior knowledge that may not fully represent novel situations.

Ethical considerations for this work focus on its use in high-stakes applications, such as healthcare, finance, or defense, where decision-making under uncertainty could have significant consequences. Developers and practitioners should implement safeguards to ensure responsible deployment, including comprehensive testing in diverse scenarios and establishing clear boundaries for its use.

By advancing the state of the art in continual learning and decision-making, this research contributes to the development of more adaptable and intelligent AI systems while highlighting the importance of ethical and responsible innovation in AI technologies.

\nocite{langley00}

  \section*{Limitations}
  Our work has several limitations. First, our experiments focus on three
  specific domains, leaving the question of general domain data selection unexplored.
  The challenge of guiding annotators to provide quality comparisons in general
  domains remains open. Furthermore, while deriving criteria directly from human-annotated
  pairwise comparisons reduces biases compared to handwritten criteria, human biases
  can not be completely eliminated from the annotation process, as defining high-quality
  data remains inherently subjective. Finally, due to computational constraints,
  we limited our approach to continual pretraining rather than pretraining from scratch,
  and used a relatively modest model with 3B parameters. Future work could explore scaling
  to larger models and more comprehensive training approaches.

  \bibliography{custom}

  \newpage
  \appendix
  \subsection{Lloyd-Max Algorithm}
\label{subsec:Lloyd-Max}
For a given quantization bitwidth $B$ and an operand $\bm{X}$, the Lloyd-Max algorithm finds $2^B$ quantization levels $\{\hat{x}_i\}_{i=1}^{2^B}$ such that quantizing $\bm{X}$ by rounding each scalar in $\bm{X}$ to the nearest quantization level minimizes the quantization MSE. 

The algorithm starts with an initial guess of quantization levels and then iteratively computes quantization thresholds $\{\tau_i\}_{i=1}^{2^B-1}$ and updates quantization levels $\{\hat{x}_i\}_{i=1}^{2^B}$. Specifically, at iteration $n$, thresholds are set to the midpoints of the previous iteration's levels:
\begin{align*}
    \tau_i^{(n)}=\frac{\hat{x}_i^{(n-1)}+\hat{x}_{i+1}^{(n-1)}}2 \text{ for } i=1\ldots 2^B-1
\end{align*}
Subsequently, the quantization levels are re-computed as conditional means of the data regions defined by the new thresholds:
\begin{align*}
    \hat{x}_i^{(n)}=\mathbb{E}\left[ \bm{X} \big| \bm{X}\in [\tau_{i-1}^{(n)},\tau_i^{(n)}] \right] \text{ for } i=1\ldots 2^B
\end{align*}
where to satisfy boundary conditions we have $\tau_0=-\infty$ and $\tau_{2^B}=\infty$. The algorithm iterates the above steps until convergence.

Figure \ref{fig:lm_quant} compares the quantization levels of a $7$-bit floating point (E3M3) quantizer (left) to a $7$-bit Lloyd-Max quantizer (right) when quantizing a layer of weights from the GPT3-126M model at a per-tensor granularity. As shown, the Lloyd-Max quantizer achieves substantially lower quantization MSE. Further, Table \ref{tab:FP7_vs_LM7} shows the superior perplexity achieved by Lloyd-Max quantizers for bitwidths of $7$, $6$ and $5$. The difference between the quantizers is clear at 5 bits, where per-tensor FP quantization incurs a drastic and unacceptable increase in perplexity, while Lloyd-Max quantization incurs a much smaller increase. Nevertheless, we note that even the optimal Lloyd-Max quantizer incurs a notable ($\sim 1.5$) increase in perplexity due to the coarse granularity of quantization. 

\begin{figure}[h]
  \centering
  \includegraphics[width=0.7\linewidth]{sections/figures/LM7_FP7.pdf}
  \caption{\small Quantization levels and the corresponding quantization MSE of Floating Point (left) vs Lloyd-Max (right) Quantizers for a layer of weights in the GPT3-126M model.}
  \label{fig:lm_quant}
\end{figure}

\begin{table}[h]\scriptsize
\begin{center}
\caption{\label{tab:FP7_vs_LM7} \small Comparing perplexity (lower is better) achieved by floating point quantizers and Lloyd-Max quantizers on a GPT3-126M model for the Wikitext-103 dataset.}
\begin{tabular}{c|cc|c}
\hline
 \multirow{2}{*}{\textbf{Bitwidth}} & \multicolumn{2}{|c|}{\textbf{Floating-Point Quantizer}} & \textbf{Lloyd-Max Quantizer} \\
 & Best Format & Wikitext-103 Perplexity & Wikitext-103 Perplexity \\
\hline
7 & E3M3 & 18.32 & 18.27 \\
6 & E3M2 & 19.07 & 18.51 \\
5 & E4M0 & 43.89 & 19.71 \\
\hline
\end{tabular}
\end{center}
\end{table}

\subsection{Proof of Local Optimality of LO-BCQ}
\label{subsec:lobcq_opt_proof}
For a given block $\bm{b}_j$, the quantization MSE during LO-BCQ can be empirically evaluated as $\frac{1}{L_b}\lVert \bm{b}_j- \bm{\hat{b}}_j\rVert^2_2$ where $\bm{\hat{b}}_j$ is computed from equation (\ref{eq:clustered_quantization_definition}) as $C_{f(\bm{b}_j)}(\bm{b}_j)$. Further, for a given block cluster $\mathcal{B}_i$, we compute the quantization MSE as $\frac{1}{|\mathcal{B}_{i}|}\sum_{\bm{b} \in \mathcal{B}_{i}} \frac{1}{L_b}\lVert \bm{b}- C_i^{(n)}(\bm{b})\rVert^2_2$. Therefore, at the end of iteration $n$, we evaluate the overall quantization MSE $J^{(n)}$ for a given operand $\bm{X}$ composed of $N_c$ block clusters as:
\begin{align*}
    \label{eq:mse_iter_n}
    J^{(n)} = \frac{1}{N_c} \sum_{i=1}^{N_c} \frac{1}{|\mathcal{B}_{i}^{(n)}|}\sum_{\bm{v} \in \mathcal{B}_{i}^{(n)}} \frac{1}{L_b}\lVert \bm{b}- B_i^{(n)}(\bm{b})\rVert^2_2
\end{align*}

At the end of iteration $n$, the codebooks are updated from $\mathcal{C}^{(n-1)}$ to $\mathcal{C}^{(n)}$. However, the mapping of a given vector $\bm{b}_j$ to quantizers $\mathcal{C}^{(n)}$ remains as  $f^{(n)}(\bm{b}_j)$. At the next iteration, during the vector clustering step, $f^{(n+1)}(\bm{b}_j)$ finds new mapping of $\bm{b}_j$ to updated codebooks $\mathcal{C}^{(n)}$ such that the quantization MSE over the candidate codebooks is minimized. Therefore, we obtain the following result for $\bm{b}_j$:
\begin{align*}
\frac{1}{L_b}\lVert \bm{b}_j - C_{f^{(n+1)}(\bm{b}_j)}^{(n)}(\bm{b}_j)\rVert^2_2 \le \frac{1}{L_b}\lVert \bm{b}_j - C_{f^{(n)}(\bm{b}_j)}^{(n)}(\bm{b}_j)\rVert^2_2
\end{align*}

That is, quantizing $\bm{b}_j$ at the end of the block clustering step of iteration $n+1$ results in lower quantization MSE compared to quantizing at the end of iteration $n$. Since this is true for all $\bm{b} \in \bm{X}$, we assert the following:
\begin{equation}
\begin{split}
\label{eq:mse_ineq_1}
    \tilde{J}^{(n+1)} &= \frac{1}{N_c} \sum_{i=1}^{N_c} \frac{1}{|\mathcal{B}_{i}^{(n+1)}|}\sum_{\bm{b} \in \mathcal{B}_{i}^{(n+1)}} \frac{1}{L_b}\lVert \bm{b} - C_i^{(n)}(b)\rVert^2_2 \le J^{(n)}
\end{split}
\end{equation}
where $\tilde{J}^{(n+1)}$ is the the quantization MSE after the vector clustering step at iteration $n+1$.

Next, during the codebook update step (\ref{eq:quantizers_update}) at iteration $n+1$, the per-cluster codebooks $\mathcal{C}^{(n)}$ are updated to $\mathcal{C}^{(n+1)}$ by invoking the Lloyd-Max algorithm \citep{Lloyd}. We know that for any given value distribution, the Lloyd-Max algorithm minimizes the quantization MSE. Therefore, for a given vector cluster $\mathcal{B}_i$ we obtain the following result:

\begin{equation}
    \frac{1}{|\mathcal{B}_{i}^{(n+1)}|}\sum_{\bm{b} \in \mathcal{B}_{i}^{(n+1)}} \frac{1}{L_b}\lVert \bm{b}- C_i^{(n+1)}(\bm{b})\rVert^2_2 \le \frac{1}{|\mathcal{B}_{i}^{(n+1)}|}\sum_{\bm{b} \in \mathcal{B}_{i}^{(n+1)}} \frac{1}{L_b}\lVert \bm{b}- C_i^{(n)}(\bm{b})\rVert^2_2
\end{equation}

The above equation states that quantizing the given block cluster $\mathcal{B}_i$ after updating the associated codebook from $C_i^{(n)}$ to $C_i^{(n+1)}$ results in lower quantization MSE. Since this is true for all the block clusters, we derive the following result: 
\begin{equation}
\begin{split}
\label{eq:mse_ineq_2}
     J^{(n+1)} &= \frac{1}{N_c} \sum_{i=1}^{N_c} \frac{1}{|\mathcal{B}_{i}^{(n+1)}|}\sum_{\bm{b} \in \mathcal{B}_{i}^{(n+1)}} \frac{1}{L_b}\lVert \bm{b}- C_i^{(n+1)}(\bm{b})\rVert^2_2  \le \tilde{J}^{(n+1)}   
\end{split}
\end{equation}

Following (\ref{eq:mse_ineq_1}) and (\ref{eq:mse_ineq_2}), we find that the quantization MSE is non-increasing for each iteration, that is, $J^{(1)} \ge J^{(2)} \ge J^{(3)} \ge \ldots \ge J^{(M)}$ where $M$ is the maximum number of iterations. 
%Therefore, we can say that if the algorithm converges, then it must be that it has converged to a local minimum. 
\hfill $\blacksquare$


\begin{figure}
    \begin{center}
    \includegraphics[width=0.5\textwidth]{sections//figures/mse_vs_iter.pdf}
    \end{center}
    \caption{\small NMSE vs iterations during LO-BCQ compared to other block quantization proposals}
    \label{fig:nmse_vs_iter}
\end{figure}

Figure \ref{fig:nmse_vs_iter} shows the empirical convergence of LO-BCQ across several block lengths and number of codebooks. Also, the MSE achieved by LO-BCQ is compared to baselines such as MXFP and VSQ. As shown, LO-BCQ converges to a lower MSE than the baselines. Further, we achieve better convergence for larger number of codebooks ($N_c$) and for a smaller block length ($L_b$), both of which increase the bitwidth of BCQ (see Eq \ref{eq:bitwidth_bcq}).


\subsection{Additional Accuracy Results}
%Table \ref{tab:lobcq_config} lists the various LOBCQ configurations and their corresponding bitwidths.
\begin{table}
\setlength{\tabcolsep}{4.75pt}
\begin{center}
\caption{\label{tab:lobcq_config} Various LO-BCQ configurations and their bitwidths.}
\begin{tabular}{|c||c|c|c|c||c|c||c|} 
\hline
 & \multicolumn{4}{|c||}{$L_b=8$} & \multicolumn{2}{|c||}{$L_b=4$} & $L_b=2$ \\
 \hline
 \backslashbox{$L_A$\kern-1em}{\kern-1em$N_c$} & 2 & 4 & 8 & 16 & 2 & 4 & 2 \\
 \hline
 64 & 4.25 & 4.375 & 4.5 & 4.625 & 4.375 & 4.625 & 4.625\\
 \hline
 32 & 4.375 & 4.5 & 4.625& 4.75 & 4.5 & 4.75 & 4.75 \\
 \hline
 16 & 4.625 & 4.75& 4.875 & 5 & 4.75 & 5 & 5 \\
 \hline
\end{tabular}
\end{center}
\end{table}

%\subsection{Perplexity achieved by various LO-BCQ configurations on Wikitext-103 dataset}

\begin{table} \centering
\begin{tabular}{|c||c|c|c|c||c|c||c|} 
\hline
 $L_b \rightarrow$& \multicolumn{4}{c||}{8} & \multicolumn{2}{c||}{4} & 2\\
 \hline
 \backslashbox{$L_A$\kern-1em}{\kern-1em$N_c$} & 2 & 4 & 8 & 16 & 2 & 4 & 2  \\
 %$N_c \rightarrow$ & 2 & 4 & 8 & 16 & 2 & 4 & 2 \\
 \hline
 \hline
 \multicolumn{8}{c}{GPT3-1.3B (FP32 PPL = 9.98)} \\ 
 \hline
 \hline
 64 & 10.40 & 10.23 & 10.17 & 10.15 &  10.28 & 10.18 & 10.19 \\
 \hline
 32 & 10.25 & 10.20 & 10.15 & 10.12 &  10.23 & 10.17 & 10.17 \\
 \hline
 16 & 10.22 & 10.16 & 10.10 & 10.09 &  10.21 & 10.14 & 10.16 \\
 \hline
  \hline
 \multicolumn{8}{c}{GPT3-8B (FP32 PPL = 7.38)} \\ 
 \hline
 \hline
 64 & 7.61 & 7.52 & 7.48 &  7.47 &  7.55 &  7.49 & 7.50 \\
 \hline
 32 & 7.52 & 7.50 & 7.46 &  7.45 &  7.52 &  7.48 & 7.48  \\
 \hline
 16 & 7.51 & 7.48 & 7.44 &  7.44 &  7.51 &  7.49 & 7.47  \\
 \hline
\end{tabular}
\caption{\label{tab:ppl_gpt3_abalation} Wikitext-103 perplexity across GPT3-1.3B and 8B models.}
\end{table}

\begin{table} \centering
\begin{tabular}{|c||c|c|c|c||} 
\hline
 $L_b \rightarrow$& \multicolumn{4}{c||}{8}\\
 \hline
 \backslashbox{$L_A$\kern-1em}{\kern-1em$N_c$} & 2 & 4 & 8 & 16 \\
 %$N_c \rightarrow$ & 2 & 4 & 8 & 16 & 2 & 4 & 2 \\
 \hline
 \hline
 \multicolumn{5}{|c|}{Llama2-7B (FP32 PPL = 5.06)} \\ 
 \hline
 \hline
 64 & 5.31 & 5.26 & 5.19 & 5.18  \\
 \hline
 32 & 5.23 & 5.25 & 5.18 & 5.15  \\
 \hline
 16 & 5.23 & 5.19 & 5.16 & 5.14  \\
 \hline
 \multicolumn{5}{|c|}{Nemotron4-15B (FP32 PPL = 5.87)} \\ 
 \hline
 \hline
 64  & 6.3 & 6.20 & 6.13 & 6.08  \\
 \hline
 32  & 6.24 & 6.12 & 6.07 & 6.03  \\
 \hline
 16  & 6.12 & 6.14 & 6.04 & 6.02  \\
 \hline
 \multicolumn{5}{|c|}{Nemotron4-340B (FP32 PPL = 3.48)} \\ 
 \hline
 \hline
 64 & 3.67 & 3.62 & 3.60 & 3.59 \\
 \hline
 32 & 3.63 & 3.61 & 3.59 & 3.56 \\
 \hline
 16 & 3.61 & 3.58 & 3.57 & 3.55 \\
 \hline
\end{tabular}
\caption{\label{tab:ppl_llama7B_nemo15B} Wikitext-103 perplexity compared to FP32 baseline in Llama2-7B and Nemotron4-15B, 340B models}
\end{table}

%\subsection{Perplexity achieved by various LO-BCQ configurations on MMLU dataset}


\begin{table} \centering
\begin{tabular}{|c||c|c|c|c||c|c|c|c|} 
\hline
 $L_b \rightarrow$& \multicolumn{4}{c||}{8} & \multicolumn{4}{c||}{8}\\
 \hline
 \backslashbox{$L_A$\kern-1em}{\kern-1em$N_c$} & 2 & 4 & 8 & 16 & 2 & 4 & 8 & 16  \\
 %$N_c \rightarrow$ & 2 & 4 & 8 & 16 & 2 & 4 & 2 \\
 \hline
 \hline
 \multicolumn{5}{|c|}{Llama2-7B (FP32 Accuracy = 45.8\%)} & \multicolumn{4}{|c|}{Llama2-70B (FP32 Accuracy = 69.12\%)} \\ 
 \hline
 \hline
 64 & 43.9 & 43.4 & 43.9 & 44.9 & 68.07 & 68.27 & 68.17 & 68.75 \\
 \hline
 32 & 44.5 & 43.8 & 44.9 & 44.5 & 68.37 & 68.51 & 68.35 & 68.27  \\
 \hline
 16 & 43.9 & 42.7 & 44.9 & 45 & 68.12 & 68.77 & 68.31 & 68.59  \\
 \hline
 \hline
 \multicolumn{5}{|c|}{GPT3-22B (FP32 Accuracy = 38.75\%)} & \multicolumn{4}{|c|}{Nemotron4-15B (FP32 Accuracy = 64.3\%)} \\ 
 \hline
 \hline
 64 & 36.71 & 38.85 & 38.13 & 38.92 & 63.17 & 62.36 & 63.72 & 64.09 \\
 \hline
 32 & 37.95 & 38.69 & 39.45 & 38.34 & 64.05 & 62.30 & 63.8 & 64.33  \\
 \hline
 16 & 38.88 & 38.80 & 38.31 & 38.92 & 63.22 & 63.51 & 63.93 & 64.43  \\
 \hline
\end{tabular}
\caption{\label{tab:mmlu_abalation} Accuracy on MMLU dataset across GPT3-22B, Llama2-7B, 70B and Nemotron4-15B models.}
\end{table}


%\subsection{Perplexity achieved by various LO-BCQ configurations on LM evaluation harness}

\begin{table} \centering
\begin{tabular}{|c||c|c|c|c||c|c|c|c|} 
\hline
 $L_b \rightarrow$& \multicolumn{4}{c||}{8} & \multicolumn{4}{c||}{8}\\
 \hline
 \backslashbox{$L_A$\kern-1em}{\kern-1em$N_c$} & 2 & 4 & 8 & 16 & 2 & 4 & 8 & 16  \\
 %$N_c \rightarrow$ & 2 & 4 & 8 & 16 & 2 & 4 & 2 \\
 \hline
 \hline
 \multicolumn{5}{|c|}{Race (FP32 Accuracy = 37.51\%)} & \multicolumn{4}{|c|}{Boolq (FP32 Accuracy = 64.62\%)} \\ 
 \hline
 \hline
 64 & 36.94 & 37.13 & 36.27 & 37.13 & 63.73 & 62.26 & 63.49 & 63.36 \\
 \hline
 32 & 37.03 & 36.36 & 36.08 & 37.03 & 62.54 & 63.51 & 63.49 & 63.55  \\
 \hline
 16 & 37.03 & 37.03 & 36.46 & 37.03 & 61.1 & 63.79 & 63.58 & 63.33  \\
 \hline
 \hline
 \multicolumn{5}{|c|}{Winogrande (FP32 Accuracy = 58.01\%)} & \multicolumn{4}{|c|}{Piqa (FP32 Accuracy = 74.21\%)} \\ 
 \hline
 \hline
 64 & 58.17 & 57.22 & 57.85 & 58.33 & 73.01 & 73.07 & 73.07 & 72.80 \\
 \hline
 32 & 59.12 & 58.09 & 57.85 & 58.41 & 73.01 & 73.94 & 72.74 & 73.18  \\
 \hline
 16 & 57.93 & 58.88 & 57.93 & 58.56 & 73.94 & 72.80 & 73.01 & 73.94  \\
 \hline
\end{tabular}
\caption{\label{tab:mmlu_abalation} Accuracy on LM evaluation harness tasks on GPT3-1.3B model.}
\end{table}

\begin{table} \centering
\begin{tabular}{|c||c|c|c|c||c|c|c|c|} 
\hline
 $L_b \rightarrow$& \multicolumn{4}{c||}{8} & \multicolumn{4}{c||}{8}\\
 \hline
 \backslashbox{$L_A$\kern-1em}{\kern-1em$N_c$} & 2 & 4 & 8 & 16 & 2 & 4 & 8 & 16  \\
 %$N_c \rightarrow$ & 2 & 4 & 8 & 16 & 2 & 4 & 2 \\
 \hline
 \hline
 \multicolumn{5}{|c|}{Race (FP32 Accuracy = 41.34\%)} & \multicolumn{4}{|c|}{Boolq (FP32 Accuracy = 68.32\%)} \\ 
 \hline
 \hline
 64 & 40.48 & 40.10 & 39.43 & 39.90 & 69.20 & 68.41 & 69.45 & 68.56 \\
 \hline
 32 & 39.52 & 39.52 & 40.77 & 39.62 & 68.32 & 67.43 & 68.17 & 69.30  \\
 \hline
 16 & 39.81 & 39.71 & 39.90 & 40.38 & 68.10 & 66.33 & 69.51 & 69.42  \\
 \hline
 \hline
 \multicolumn{5}{|c|}{Winogrande (FP32 Accuracy = 67.88\%)} & \multicolumn{4}{|c|}{Piqa (FP32 Accuracy = 78.78\%)} \\ 
 \hline
 \hline
 64 & 66.85 & 66.61 & 67.72 & 67.88 & 77.31 & 77.42 & 77.75 & 77.64 \\
 \hline
 32 & 67.25 & 67.72 & 67.72 & 67.00 & 77.31 & 77.04 & 77.80 & 77.37  \\
 \hline
 16 & 68.11 & 68.90 & 67.88 & 67.48 & 77.37 & 78.13 & 78.13 & 77.69  \\
 \hline
\end{tabular}
\caption{\label{tab:mmlu_abalation} Accuracy on LM evaluation harness tasks on GPT3-8B model.}
\end{table}

\begin{table} \centering
\begin{tabular}{|c||c|c|c|c||c|c|c|c|} 
\hline
 $L_b \rightarrow$& \multicolumn{4}{c||}{8} & \multicolumn{4}{c||}{8}\\
 \hline
 \backslashbox{$L_A$\kern-1em}{\kern-1em$N_c$} & 2 & 4 & 8 & 16 & 2 & 4 & 8 & 16  \\
 %$N_c \rightarrow$ & 2 & 4 & 8 & 16 & 2 & 4 & 2 \\
 \hline
 \hline
 \multicolumn{5}{|c|}{Race (FP32 Accuracy = 40.67\%)} & \multicolumn{4}{|c|}{Boolq (FP32 Accuracy = 76.54\%)} \\ 
 \hline
 \hline
 64 & 40.48 & 40.10 & 39.43 & 39.90 & 75.41 & 75.11 & 77.09 & 75.66 \\
 \hline
 32 & 39.52 & 39.52 & 40.77 & 39.62 & 76.02 & 76.02 & 75.96 & 75.35  \\
 \hline
 16 & 39.81 & 39.71 & 39.90 & 40.38 & 75.05 & 73.82 & 75.72 & 76.09  \\
 \hline
 \hline
 \multicolumn{5}{|c|}{Winogrande (FP32 Accuracy = 70.64\%)} & \multicolumn{4}{|c|}{Piqa (FP32 Accuracy = 79.16\%)} \\ 
 \hline
 \hline
 64 & 69.14 & 70.17 & 70.17 & 70.56 & 78.24 & 79.00 & 78.62 & 78.73 \\
 \hline
 32 & 70.96 & 69.69 & 71.27 & 69.30 & 78.56 & 79.49 & 79.16 & 78.89  \\
 \hline
 16 & 71.03 & 69.53 & 69.69 & 70.40 & 78.13 & 79.16 & 79.00 & 79.00  \\
 \hline
\end{tabular}
\caption{\label{tab:mmlu_abalation} Accuracy on LM evaluation harness tasks on GPT3-22B model.}
\end{table}

\begin{table} \centering
\begin{tabular}{|c||c|c|c|c||c|c|c|c|} 
\hline
 $L_b \rightarrow$& \multicolumn{4}{c||}{8} & \multicolumn{4}{c||}{8}\\
 \hline
 \backslashbox{$L_A$\kern-1em}{\kern-1em$N_c$} & 2 & 4 & 8 & 16 & 2 & 4 & 8 & 16  \\
 %$N_c \rightarrow$ & 2 & 4 & 8 & 16 & 2 & 4 & 2 \\
 \hline
 \hline
 \multicolumn{5}{|c|}{Race (FP32 Accuracy = 44.4\%)} & \multicolumn{4}{|c|}{Boolq (FP32 Accuracy = 79.29\%)} \\ 
 \hline
 \hline
 64 & 42.49 & 42.51 & 42.58 & 43.45 & 77.58 & 77.37 & 77.43 & 78.1 \\
 \hline
 32 & 43.35 & 42.49 & 43.64 & 43.73 & 77.86 & 75.32 & 77.28 & 77.86  \\
 \hline
 16 & 44.21 & 44.21 & 43.64 & 42.97 & 78.65 & 77 & 76.94 & 77.98  \\
 \hline
 \hline
 \multicolumn{5}{|c|}{Winogrande (FP32 Accuracy = 69.38\%)} & \multicolumn{4}{|c|}{Piqa (FP32 Accuracy = 78.07\%)} \\ 
 \hline
 \hline
 64 & 68.9 & 68.43 & 69.77 & 68.19 & 77.09 & 76.82 & 77.09 & 77.86 \\
 \hline
 32 & 69.38 & 68.51 & 68.82 & 68.90 & 78.07 & 76.71 & 78.07 & 77.86  \\
 \hline
 16 & 69.53 & 67.09 & 69.38 & 68.90 & 77.37 & 77.8 & 77.91 & 77.69  \\
 \hline
\end{tabular}
\caption{\label{tab:mmlu_abalation} Accuracy on LM evaluation harness tasks on Llama2-7B model.}
\end{table}

\begin{table} \centering
\begin{tabular}{|c||c|c|c|c||c|c|c|c|} 
\hline
 $L_b \rightarrow$& \multicolumn{4}{c||}{8} & \multicolumn{4}{c||}{8}\\
 \hline
 \backslashbox{$L_A$\kern-1em}{\kern-1em$N_c$} & 2 & 4 & 8 & 16 & 2 & 4 & 8 & 16  \\
 %$N_c \rightarrow$ & 2 & 4 & 8 & 16 & 2 & 4 & 2 \\
 \hline
 \hline
 \multicolumn{5}{|c|}{Race (FP32 Accuracy = 48.8\%)} & \multicolumn{4}{|c|}{Boolq (FP32 Accuracy = 85.23\%)} \\ 
 \hline
 \hline
 64 & 49.00 & 49.00 & 49.28 & 48.71 & 82.82 & 84.28 & 84.03 & 84.25 \\
 \hline
 32 & 49.57 & 48.52 & 48.33 & 49.28 & 83.85 & 84.46 & 84.31 & 84.93  \\
 \hline
 16 & 49.85 & 49.09 & 49.28 & 48.99 & 85.11 & 84.46 & 84.61 & 83.94  \\
 \hline
 \hline
 \multicolumn{5}{|c|}{Winogrande (FP32 Accuracy = 79.95\%)} & \multicolumn{4}{|c|}{Piqa (FP32 Accuracy = 81.56\%)} \\ 
 \hline
 \hline
 64 & 78.77 & 78.45 & 78.37 & 79.16 & 81.45 & 80.69 & 81.45 & 81.5 \\
 \hline
 32 & 78.45 & 79.01 & 78.69 & 80.66 & 81.56 & 80.58 & 81.18 & 81.34  \\
 \hline
 16 & 79.95 & 79.56 & 79.79 & 79.72 & 81.28 & 81.66 & 81.28 & 80.96  \\
 \hline
\end{tabular}
\caption{\label{tab:mmlu_abalation} Accuracy on LM evaluation harness tasks on Llama2-70B model.}
\end{table}

%\section{MSE Studies}
%\textcolor{red}{TODO}


\subsection{Number Formats and Quantization Method}
\label{subsec:numFormats_quantMethod}
\subsubsection{Integer Format}
An $n$-bit signed integer (INT) is typically represented with a 2s-complement format \citep{yao2022zeroquant,xiao2023smoothquant,dai2021vsq}, where the most significant bit denotes the sign.

\subsubsection{Floating Point Format}
An $n$-bit signed floating point (FP) number $x$ comprises of a 1-bit sign ($x_{\mathrm{sign}}$), $B_m$-bit mantissa ($x_{\mathrm{mant}}$) and $B_e$-bit exponent ($x_{\mathrm{exp}}$) such that $B_m+B_e=n-1$. The associated constant exponent bias ($E_{\mathrm{bias}}$) is computed as $(2^{{B_e}-1}-1)$. We denote this format as $E_{B_e}M_{B_m}$.  

\subsubsection{Quantization Scheme}
\label{subsec:quant_method}
A quantization scheme dictates how a given unquantized tensor is converted to its quantized representation. We consider FP formats for the purpose of illustration. Given an unquantized tensor $\bm{X}$ and an FP format $E_{B_e}M_{B_m}$, we first, we compute the quantization scale factor $s_X$ that maps the maximum absolute value of $\bm{X}$ to the maximum quantization level of the $E_{B_e}M_{B_m}$ format as follows:
\begin{align}
\label{eq:sf}
    s_X = \frac{\mathrm{max}(|\bm{X}|)}{\mathrm{max}(E_{B_e}M_{B_m})}
\end{align}
In the above equation, $|\cdot|$ denotes the absolute value function.

Next, we scale $\bm{X}$ by $s_X$ and quantize it to $\hat{\bm{X}}$ by rounding it to the nearest quantization level of $E_{B_e}M_{B_m}$ as:

\begin{align}
\label{eq:tensor_quant}
    \hat{\bm{X}} = \text{round-to-nearest}\left(\frac{\bm{X}}{s_X}, E_{B_e}M_{B_m}\right)
\end{align}

We perform dynamic max-scaled quantization \citep{wu2020integer}, where the scale factor $s$ for activations is dynamically computed during runtime.

\subsection{Vector Scaled Quantization}
\begin{wrapfigure}{r}{0.35\linewidth}
  \centering
  \includegraphics[width=\linewidth]{sections/figures/vsquant.jpg}
  \caption{\small Vectorwise decomposition for per-vector scaled quantization (VSQ \citep{dai2021vsq}).}
  \label{fig:vsquant}
\end{wrapfigure}
During VSQ \citep{dai2021vsq}, the operand tensors are decomposed into 1D vectors in a hardware friendly manner as shown in Figure \ref{fig:vsquant}. Since the decomposed tensors are used as operands in matrix multiplications during inference, it is beneficial to perform this decomposition along the reduction dimension of the multiplication. The vectorwise quantization is performed similar to tensorwise quantization described in Equations \ref{eq:sf} and \ref{eq:tensor_quant}, where a scale factor $s_v$ is required for each vector $\bm{v}$ that maps the maximum absolute value of that vector to the maximum quantization level. While smaller vector lengths can lead to larger accuracy gains, the associated memory and computational overheads due to the per-vector scale factors increases. To alleviate these overheads, VSQ \citep{dai2021vsq} proposed a second level quantization of the per-vector scale factors to unsigned integers, while MX \citep{rouhani2023shared} quantizes them to integer powers of 2 (denoted as $2^{INT}$).

\subsubsection{MX Format}
The MX format proposed in \citep{rouhani2023microscaling} introduces the concept of sub-block shifting. For every two scalar elements of $b$-bits each, there is a shared exponent bit. The value of this exponent bit is determined through an empirical analysis that targets minimizing quantization MSE. We note that the FP format $E_{1}M_{b}$ is strictly better than MX from an accuracy perspective since it allocates a dedicated exponent bit to each scalar as opposed to sharing it across two scalars. Therefore, we conservatively bound the accuracy of a $b+2$-bit signed MX format with that of a $E_{1}M_{b}$ format in our comparisons. For instance, we use E1M2 format as a proxy for MX4.

\begin{figure}
    \centering
    \includegraphics[width=1\linewidth]{sections//figures/BlockFormats.pdf}
    \caption{\small Comparing LO-BCQ to MX format.}
    \label{fig:block_formats}
\end{figure}

Figure \ref{fig:block_formats} compares our $4$-bit LO-BCQ block format to MX \citep{rouhani2023microscaling}. As shown, both LO-BCQ and MX decompose a given operand tensor into block arrays and each block array into blocks. Similar to MX, we find that per-block quantization ($L_b < L_A$) leads to better accuracy due to increased flexibility. While MX achieves this through per-block $1$-bit micro-scales, we associate a dedicated codebook to each block through a per-block codebook selector. Further, MX quantizes the per-block array scale-factor to E8M0 format without per-tensor scaling. In contrast during LO-BCQ, we find that per-tensor scaling combined with quantization of per-block array scale-factor to E4M3 format results in superior inference accuracy across models. 

\end{document}
