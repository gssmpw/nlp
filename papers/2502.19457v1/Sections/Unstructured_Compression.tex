\subsection{Discussion}
We analyze the performance of various unstructured compression techniques for 3DGS across the Mip-NeRF360, Tanks\&Temples, and Deep Blending datasets, focusing on three key aspects: compression ratio, fidelity, and FPS.

\noindent
\textbf{Fidelity: }EfficientGS~\cite{liu2024efficientgs}, Papantonakis et al.~\cite{papantonakis2024reducing}, and EAGLES~\cite{girish2023eagles} across all datasets exhibit strong performance, maintaining high SSIM, PSNR, and low LPIPS scores, indicating minimal perceptual degradation as shown in Tables~\ref{tab:unstruc_mip_tandt} and~\ref{tab:unstruc_db}. ELMGS-medium~\cite{ali2024elmgs} provides a compelling balance between compression and quality, notably achieving 24.08dB in PSNR on Tanks\&Temples outperforming all the other methods. On the other hand, highly compressed techniques such as Trimming the Fat~\cite{salman2024trimming}, CompGS~\cite{navaneet2023compact3d}, and ELMGS-small~\cite{ali2024elmgs} exhibit slightly elevated LPIPS scores, suggesting perceptual quality loss despite their strong numerical PSNR performance. These results emphasize that while EfficientGS,  Papantonakis et al~\cite{papantonakis2024reducing}, and EAGLES provide the best fidelity, ELMGS and Trimming the Fat offer promising trade-offs between compression efficiency and reconstruction quality.


\begin{figure*}
    \centering
    \includegraphics[width=0.9\linewidth]{Figures/scaffoldgs.pdf}
    \caption{\textbf{Overview of the ScaffoldGS pipeline.} (a) A sparse voxel grid is initialized from Structure-from-Motion (SfM)-derived points, with each \textbf{anchor} placed at the voxel center and assigned a learnable scale, capturing the scene’s occupancy. (b) Within the view frustum, \textbf{k Gaussians} are generated from each \textit{visible anchor} with offsets \(\{O_k\}\), and their attributes—including opacity, color, scale, and quaternion—are decoded from the anchor feature. (c) To enhance efficiency and reduce redundancy, only significant Gaussians (\(\alpha \geq \tau_\alpha\)) are rasterized. The final rendered image is optimized using reconstruction loss. Figure taken from~\cite{lu2023scaffold}.}
    \label{fig:scaffoldgs}
    \vspace{-0.5cm}
\end{figure*}

\noindent
\textbf{Compression Ratio: } %Compression is crucial for reducing storage and bandwidth requirements. 
The results in Tables~\ref{tab:unstruc_mip_tandt} and~\ref{tab:unstruc_db} reveal significant variability in compression performance across methods and Figure~\ref{fig:unstruc_memvspsnr} shows rate-distortion tradeoff. ELMGS-small demonstrates the highest compression rates, achieving a 55$\times$ reduction on the Deep Blending dataset and 35$\times$  on Tanks\&Temples, making it one of the most storage-efficient methods. Similarly, CompGS-16K and CompGS-32K provide compression factors of 39$\times$ –56$\times$ , albeit at a minor fidelity trade-off. Notably, Trimming the Fat achieves a 48$\times$  compression on Tanks\&Temples while maintaining a competitive balance between storage complexity and reconstruction quality. 

\noindent
\textbf{FPS: }Frame rate is a crucial factor in real-time rendering applications, ensuring whether a method is viable for interactive environments such as virtual reality and gaming or low-power edge devices. The rendering speeds are calculated on a single NVIDIA A40 GPU. The FPS vs. PSNR plots in Figure~\ref{fig:unstruc_fpsvspsnr} highlight a clear trade-off between rendering speed and reconstruction accuracy. ELMGS, Trimming the Fat, and Papantonakis et al.~\cite{papantonakis2024reducing} achieve the highest FPS values, with ELMGS surpassing 400 FPS on Deep Blending and 600 FPS on Tanks\&Temples, making it particularly well-suited for real-time applications with limited computational complexity. Compared to the 3DGS-30k baseline all the unstructured compression methods offer a better trade-off between rendering speed and fidelity, making them strong candidates for real-world applications.

EfficientGS on Mip-NeRF360 and EAGLES on deep blending provide the highest fidelity. However, ELMGS offers the most aggressive compression while retaining competitive reconstruction quality. CompGS and Trimming the Fat achieve optimal trade-offs between compression, fidelity, and real-time rendering speed. These findings indicate that no single method dominates all aspects, highlighting the importance of application-specific selection when choosing an appropriate unstructured compression technique for 3DGS.