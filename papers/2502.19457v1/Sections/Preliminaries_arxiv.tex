\section{Preliminaries}

\subsection{Differentiable Rendering }

% Point-based rendering techniques aim to produce realistic images by rendering a collection of discrete geometric primitives. Zwicker et al.~\cite{zwicker2001surface} introduced a method that uses ellipsoid-shaped splats, allowing them to cover multiple pixels and enhance image quality by reducing visible holes through overlapping splats. This approach is more efficient than traditional point-based representations. Kopanas et al.~\cite{kopanas2021point} developed a novel differentiable point-based rendering pipeline that incorporates bi-directional Elliptical Weighted Average splatting, a probabilistic depth test, and efficient camera selection. Recent advancements in this field have focused on improving the rendering process through techniques such as anti-aliasing texture filters~\cite{zwicker2001ewa}, enhancements in rendering efficiency~\cite{botsch2003high}, and addressing issues with discontinuous shading~\cite{rusinkiewicz2000qsplat}.

% Traditional point-based rendering methods are focused on generating high-quality outputs based on predefined geometries. However, recent advances in implicit representation techniques have led researchers to explore point-based rendering within the framework of neural implicit representations. This approach eliminates the need for predefined geometries in 3D reconstruction. A prominent example in this area is  NeRF~\cite{mildenhall2020nerf}, which uses an implicit density field to model 3D geometry and an appearance field to predict view-dependent colors. In NeRF, point-based rendering is used to aggregate the colors from all sample points along a camera ray to compute the final pixel color \( C \).

%Differentiable rendering (DR) enables end-to-end optimization by providing gradients of the rendering process, bridging 2D and 3D processing. This allows neural networks to optimize 3D entities through 2D projections. Gradients can be computed with respect to scene parameters, such as geometry, lighting, and camera pose which is essential for tasks like inverse graphics, 3D reconstruction, and neural rendering, where scene parameters are optimized using self supervision. Rendering approaches are broadly categorized into implicit and explicit methods. Implicit rendering uses continuous scene representations, such as neural networks or signed distance fields, and utilizes volume rendering for image generation. In contrast, explicit rendering defines scene geometry using discrete primitives like triangles, points, or splats. Explicit point-based rendering techniques aim to produce realistic images by rendering a collection of discrete geometric primitives. Zwicker et al.~\cite{zwicker2001surface} introduced a method that uses ellipsoid-shaped splats, allowing them to cover multiple pixels and enhance image quality by reducing visible holes through overlapping splats. This approach is more efficient than traditional point-based representations. Kopanas et al.~\cite{kopanas2021point} developed a novel differentiable point-based rendering pipeline incorporating bi-directional Elliptical Weighted Average splatting, a probabilistic depth test, and efficient camera selection. 

Differentiable rendering (DR) facilitates end-to-end optimization by computing gradients of the rendering process, bridging 2D and 3D processing~\cite{kato2020differentiable}. These gradients, with respect to scene parameters like geometry, lighting, and camera pose, are crucial for tasks such as human pose estimation~\cite{ge20193d,baek2019pushing}, 3D reconstruction~\cite{yan2016perspective,tulsiani2017multi}, and neural rendering~\cite{kato2018neural}, enabling self-supervised optimization. Rendering methods are categorized as implicit or explicit. Implicit approaches represent scenes using continuous functions, such as neural networks~\cite{sitzmann2020implicit} or signed distance fields~\cite{oleynikova2016signed}, and rely on volume rendering~\cite{drebin1988volume} for image generation. Explicit methods, in contrast, define geometry using discrete primitives like triangles, points, or splats. Explicit point-based rendering focuses on realistic image synthesis by rendering collections of discrete geometric primitives. Zwicker et al.\cite{kopanas2021point} advanced this with a differentiable point-based pipeline featuring bi-directional Elliptical Weighted Average splatting, probabilistic depth testing, and efficient camera selection.

Recent advances in implicit representation techniques have led researchers to explore point-based rendering within the framework of neural implicit representations. This approach eliminates the need for predefined geometries in 3D reconstruction. A prominent example in this area is  NeRF~\cite{mildenhall2020nerf}, which uses an implicit density field to model 3D geometry and an appearance field to predict view-dependent colors. In NeRF, point-based rendering is used to aggregate the colors from all sample points along a camera ray to compute the final pixel color

\[ C = \sum_{i=1}^{N} c_i \alpha_i T_i , \]

\noindent
where \( N \) represents the number of sample points along a ray. The view-dependent color and opacity value for the \( i \)-th point on the ray are given by:

\[ \alpha_i = \exp \left(- \sum_{j=1}^{i-1} \sigma_j \delta_j \right), \]
\noindent
where \( \sigma_j \) denotes the density value of the \( j \)-th point and \( \delta_j \) is the distance between consecutive sample points. The accumulated transmittance \( T_i \) is calculated as:

\[ T_i = \prod_{j=1}^{i-1} (1 - \alpha_j). \]

The rendering process in 3DGS shares similarities with NeRFs; however, there are two key differences. \newline
\textbf{Opacity Modeling:} 3DGS directly models opacity values, while NeRF first models density values and then converts these densities into opacity values.\newline
\textbf{Rendering Technique:} 3DGS utilizes rasterization-based rendering, which avoids the need for extensive point sampling, unlike NeRF, which relies on sampling points along rays for rendering.


% \subsection{Neural Implicit Field}
% %Neural implicit field representations have recently gained substantial attention~\cite{}. These methods conceptualize 2D or 3D signals as fields within corresponding Euclidean spaces, training neural networks on discrete samples to approximate these fields. This approach supports tasks such as reconstructing, interpolating, and extrapolating from the original discrete samples, facilitating applications like super-resolution of 2D images and novel view synthesis of 3D scenes. Specifically, in 3D reconstruction and novel view synthesis, NeRFs~\cite{} utilize neural networks to model 3D scenes as density and radiance fields. NeRF employs volumetric rendering to map these 3D fields to 2D images, enabling the reconstruction of 3D scenes from multiple 2D images and allowing for novel view rendering. Noteworthy methods in this area include Mip-NeRF 360~\cite{}, which excels in rendering quality, and Instant-NGP~\cite{}, renowned for its exceptional training efficiency.

% %In recent times, there has been a significant focus on NeRFs~\cite{ramirez2024deep,wu2021revisiting}. These techniques involve training neural networks on discrete samples to model 2D or 3D signals as fields within equivalent Euclidean spaces. This approach facilitates operations such as reconstruction, interpolation, and extrapolation, making it useful for applications like novel view synthesis of 3D scenes and 2D image super-resolution. 
% For applications like 3D reconstruction and novel view synthesis, NeRFs \cite{mildenhall2021nerf,ramirez2024deep,wu2021revisiting} use neural networks to implicitly represent 3D scenes as density and radiance fields. NeRF also employs volumetric rendering to project these 3D fields onto 2D images, enabling the generation of new views and the reconstruction of 3D scenes from multiple 2D images. Among the notable techniques in this domain are Instant-NGP~\cite{muller2022instant}, recognized for its exceptional training efficiency, and Mip-NeRF 360~\cite{barron2022mip}, which is distinguished by its superior rendering quality.

% NeRFs predominantly rely on volumetric ray marching for image rendering. This process involves sampling numerous points along each ray and passing them through the neural network to generate the final image. Consequently, rendering a single 1080p image can require up to \(10^8\) neural network forward passes, often leading to rendering delays of several seconds. Even though some approaches use explicit, discretized structures to represent continuous 3D fields—thereby reducing reliance on neural networks and accelerating the query process \cite{sun2022direct,chen2022tensorf}—the high number of sampling points still results in significant rendering costs. As a result, volumetric rendering-based techniques struggle to achieve real-time performance, limiting their applicability in interactive or real-time scenarios.

\subsection{3D Gaussian Splatting}
3DGS represents a significant advancement in real-time, high-resolution image rendering that does not rely on deep neural networks. %This section provides an overview of 3DGS. In Sec. 3.1, we explain how 3DGS synthesizes an image using optimized 3D Gaussians. This involves the rendering pipeline and techniques employed to achieve high-quality images efficiently. In Sec. 3.2, we discuss the process of generating well-optimized 3D Gaussians for a given scene. This section delves into the 3DGS optimization method, outlining the steps and algorithms used to refine the Gaussian parameters for optimal rendering performance.
%\subsubsection{Rendering with 3DGS}
%NeRFs utilize volumetric raymarching for rendering, a method that is both slow and computationally intensive. In contrast, 
A scene optimized with 3DGS achieves rendering by projecting the Gaussians onto a 2D image plane through splatting. After this projection, 3DGS sorts the Gaussians and computes the pixel values accordingly~\cite{kerbl20233d}. The following sections will first define the 3D Gaussian, the fundamental element of the 3DGS scene representation, then detail the differentiable rendering process used in 3DGS, and finally, describe the techniques employed to achieve high rendering speeds.
%Consider a scene with millions of optimized 3D Gaussians. The goal is to create an image using a specified camera pose. NeRFs address this task using computationally intensive volumetric raymarching, which involves sampling points in 3D space per pixel. This method struggles with high-resolution image synthesis and fails to achieve real-time rendering, particularly on platforms with limited computational resources. In contrast, 3DGS projects these 3D Gaussians onto a pixel-based image plane through a process called "splatting." Following this projection, 3DGS sorts the Gaussians and computes the pixel values. The following sections will start with defining a 3D Gaussian, the fundamental element in 3DGS scene representation. Next, we will explain how these 3D Gaussians are utilized for differentiable rendering. Finally, we will introduce the acceleration technique employed in 3DGS, which is essential for fast rendering.



\noindent
\textbf{3D Gaussian: }The characteristics of a 3D Gaussian include its color \( c \), 3D covariance matrix \( \Sigma \), opacity \( \alpha \), and center (position) \( \mu \). For view-dependent appearance, spherical harmonics (SH) are employed to represent color \( c \). Through back-propagation, all these attributes can be learned and optimized.

\noindent
\textbf{3D Gaussian Frustum Culling: } This stage identifies the 3D Gaussians that lie outside the camera's frustum for a given camera position. Excluding these Gaussians from subsequent computations conserves processing resources, as only the 3D Gaussians within the specified view are considered.

\noindent
\textbf{Splatting of Gaussians: }In this stage, 3D Gaussians (ellipsoids) are projected onto 2D image space (ellipses) for rendering. The projected 2D covariance matrix \( \Sigma' \) is computed using the viewing transformation \( W \) and the 3D covariance matrix \( \Sigma \) as
\noindent
\[ \Sigma' = JW\Sigma W^\top J ,\]
\noindent
where \( W \) is the viewing transformation matrix, \( \Sigma \) is the 3D covariance matrix, and \( J \) is the Jacobian of the projection transformation~\cite{kerbl20233d, zwicker2001ewa}.

\noindent
\textbf{Pixel Rendering: } The viewing transformation \( W \) is employed to compute the distance between each pixel \( x \) and all overlapping Gaussians, effectively determining their depths. This results in a sorted list of Gaussians \( N \). The final color of each pixel is subsequently determined using alpha compositing. The final color \( C \) is computed by summing the contributions of each Gaussian:

\[ C = \sum_{n=1}^{|N|} c_n \alpha'_n \prod_{j=1}^{n-1} (1 - \alpha'_j) ,\]

\noindent
where \( c_n \) is the learned color. The final opacity \( \alpha'_n \) is the product of the learned opacity \( \alpha_n \) and the Gaussian function:

\[ \alpha'_n = \alpha_n \times \exp \left( -\frac{1}{2} (x' - \mu'_n)^\top \Sigma'^{-1}_n (x' - \mu'_n) \right) ,\]

\noindent
where \( x' \) is the projected position, \( \mu'_n \) is the projected center, and \( \Sigma'_n \) is the projected 2D covariance matrix. There is a valid concern that the described rendering process could be slower than NeRFs due to the challenges in parallelizing the generation of the necessary sorted list. This concern is valid, as relying on a straightforward pixel-by-pixel approach could significantly impact rendering speeds. To achieve real-time rendering, 3DGS must consider several factors that facilitate parallel processing.
%Given the challenge of parallelizing the generation of the necessary sorted list, there is a legitimate concern that the described rendering process might be slower than NeRFs. This concern is valid, as rendering times can be significantly impacted by using such a basic pixel-by-pixel method. To enable real-time rendering, 3DGS must account for several factors that support parallel processing.

\section{Dataset Generation}
\label{sec:dataset}
\revise{
To train the proposed GNN, we constructed a dataset of building structures and a subset of these structures were subjected to fire simulations using FEA. The dataset generation process is illustrated in \figref{fig:dataset_generation_procedure}. Initially, a total of 33,000 building structures with geometrical details, material properties, and gravity loads were created. Due to randomness in generating these structures, a filter is applied to remove unreasonable data after gravity load simulation, which included 15,377 structures. A trade-off between computational feasibility and model performance is made among the remaining 17,623 structures. As further labeling structures with MIDR requires resource-intensive fire simulations via OpenSeesRT, a large proportion of 16,050 structures is selected as unlabeled dataset. On the other hand, each of the other 1,573 structures was further subjected to 30 different fire simulations, forming the labeled dataset containing $1,573\times 30 = 47,190$ fire cases.} This section details the step-by-step process for generating the dataset, including geometry creation, material property assignment, and simulations due to gravity loads and fire scenarios. 
% To train the proposed neural network, we constructed a dataset comprising building structure data and a subset of fire scenario data. The dataset generation process is illustrated in \figref{fig:dataset_generation_procedure}. 
% A total of 33,000 building structures with geometric details, material properties, and gravity loads were initially created. Out of these, 3,000 structures were selected as labeled data, and the remaining 30,000 were designated as unlabeled data. Further, about half of them filtered out due to instability under gravity loads only. 
\begin{figure*}[h!]
    \centering
    \includegraphics[width=0.8\linewidth]{figures/dataset_filter_procedure.pdf}
    \caption{Workflow for dataset generation (geometry, material property, gravity loads, and fire scenarios).}
    \label{fig:dataset_generation_procedure}
\end{figure*}

\subsection{Geometry Generation}
\label{subsec:geometry_generation}
The geometry of the building structures forms the foundation of the dataset. Regular 
\revise{3D structures} resembling multi-story parking structures or shopping malls were generated, with parameters such as building floor dimensions and story heights selected randomly. Each building structure is composed of multiple rooms, which serve as the basic unit in this study. A room herein is a cuboid space defined by specific length, width, and height. Within a structure, rooms of the same dimensions are uniformly arranged along the length, width, and height, corresponding to the $x$-, $y$-, and $z$-axes, respectively. Structures vary in room size and number of rooms along each axis. Specifically, the room length, width, and height are independently sampled from a uniform distribution within the interval $[2, 5]$ meters along the three directions of the structure. Similarly, the room number along each axis is uniformly sampled independently as an integer within the interval $[2, 7]$, i.e., the maximum number of stories of the buildings simulated in this study is 7.

To introduce variability and simulate real-world scenarios, approximately $8\%$ of structural elements (beams or columns) are randomly removed after initial geometry creation. 
\revise{Such removal is not fire-induced damage, but reflects functional diversity often observed in real buildings, such as open spaces designed for activities in shopping malls, e.g., ice skating rinks. Examples of the generated geometries are illustrated in \figref{fig:example_generated_geometry}, showcasing the diversity and realism of the dataset. This element removal does not affect the definition of room's geometry in the structure and nor does it affect the number of considered fire scenarios.} 

\revise{A range of coefficient of variation values ($3.3\%$ to $17.5\%$) was derived from prior studies that investigated the statistics of geometrical and material properties of structural components of buildings (e.g., \cite{mirza1979variations, lee2004probabilistic}). These studies provide empirical data on the natural variability in parameters such as Young's modulus, yield strength, and dimensions of structural elements due to manufacturing tolerances and material inconsistencies. By selecting $8\%$ for the removal of structural elements in our database, we aimed to maintain a level of variability that is representative of real-world uncertainties while ensuring computational feasibility. This choice ensures that the database captures realistic deviations without introducing extreme cases that may not be commonly encountered in practice.}

\begin{figure*}[h!]
    \centering
    \includegraphics[width=\linewidth]{figures/example_generated_geometry.pdf}
    \caption{Examples of generated structural geometry of different sizes (all dimensions in meters).}
    \label{fig:example_generated_geometry} 
\end{figure*}

{\blockRevise

In this study, we opted for a deterministic square, dimension of $0.1$ m, solid cross-sectional steel elements due to their simplicity in modeling and analysis. Square sections exhibit uniform geometrical properties in all directions, simplifying the computation of structural responses and avoiding complications associated with more complex shapes, such as wide-flange sections, facilitating the computational efficiency and scalability to generate a large dataset. This choice also helps to mitigate issues related to stress concentrations and facilitates a more straightforward representation of structural behavior under thermal loads. 

\textit{Remark:} The selected cross-section provides a comparable flexural rigidity to a $W 130 \times 130 \times 28.1$ wide-flange section (metric units), albeit with significantly higher axial rigidity. This cross-section is acceptable for gravity-load-designed frames under service loading conditions where the models assume fully rigid, moment-resisting beam-column connections for the evaluation of the IDR under thermal loading. This assumption is reasonable in this computational study where the primary interest is to understand the global deformation response of frames under fire conditions. The selection of uniform square cross-sections for both beams and columns, rather than adherence to standard capacity design principles, was made here primarily for computational efficiency and to reduce design parameters in the database generation process. This choice allows for simplified and scalable approach to analyze the fire-induced response of generic steel frames without the need for large section variations, where this study mainly focuses on the fire vulnerability assessment using ML-based predictions. However, if additional loading conditions, e.g., seismic or wind loads, were to be considered, larger sections, strong-column/weak-beam principle, and ductile detailing would be required in the generated buildings for realistic structural behavior under combined loading conditions. Future studies may also consider investigating the influence of variable cross-sectional dimensions and semi-rigid connections on the structural performance under fire conditions. 
} % blockRevise

\subsection{Material Properties}
Steel is chosen as the material for the structures. To reflect real-world variations, we randomly assign one of five slightly different steel material types to each structural element. \revise{
The ranges of material properties are provided in \tabref{tab:material_property_ranges} and the properties are sampled from uniform distributions of the corresponding ranges. These variations simulate differences arising from manufacturing batches or regional material properties. That these properties are at ambient temperature and change when the temperature rises due to a fire. The selection of materials with varying properties is aimed at increasing the diversity of the data. Our goal is to represent as wide a range of data as possible with a limited amount of building structure data, thereby enhancing the generalization ability of the GNN. Our assumed material property ranges are expected to be wider than the real-world conditions based on findings in \cite{mirza1979variations, lee2004probabilistic}. Therefore, we are essentially tackling a more challenging and general task. If we can solve this problem, we are confident that our method will perform equally well or even better in real-world scenarios.
}
\begin{table}[h!]
    \centering
    \caption{Material properties ranges for considered steel structures.}
    \begin{tabular}{lc}
        \toprule
        Property & Range \\
        \midrule
        Young's modulus & [168, 252] GPa \\
        Yield strength & [220, 330] MPa \\
        Strain-hardening ratio & [0.8, 1.2] \% \\
        \bottomrule
    \end{tabular}
    \label{tab:material_property_ranges}
\end{table}

\subsection{Gravity Loads}
Gravity loads are applied to columns and beams based on their \revise{influence (tributary) areas as typically conducted in structural analysis. The considered ``service'' load conditions include the column self-weight and the additional loads directly supported on the beams from their self-weight and weights of the reinforced concrete slabs, people as live load, and building content. An edge beam typically carries approximately half the gravity load supported by a parallel interior beam}. The ranges of gravity loads are listed in \tabref{tab:gravity_load_ranges}. \revise{The loads are sampled from uniform distributions of the corresponding ranges.} Structures that failed to meet an MIDR threshold of $1\%$ under gravity loads were deemed unacceptable designs and filtered out, as such configurations of randomly chosen geometry, material, and gravity load combinations were considered unrealistic from a regulatory and practicality points of view.
\begin{table}[h!]
    \centering
    \caption{Gravity load ranges for considered beams and columns.}
    \begin{tabular}{lc}
        \toprule
        Element & Range (kN/m)  \\
        \midrule
        Column & [0.5, 1.0]  \\
        Edge beam & [1.5, 4.5]  \\
        Interior beam & [3.0, 7.5]  \\
        \bottomrule
    \end{tabular}
    \label{tab:gravity_load_ranges}
\end{table} 

\subsection{Rule-based Thermal Load Generation}
\label{subsec:thermal_load_generation}
To evaluate a building's structural response during a fire event, we employed a simplified rule-based approach for thermal load generation. 
% Previous studies \cite{nan_structuralfire_2023} have demonstrated that steel structures rapidly equilibrate with surrounding gases temperatures due to efficient heat exchange. Consequently, gas temperatures can be directly used as inputs for FEA tools, e.g., OpenSees, simplifying the process of modeling thermal loads. 
% Accurately simulating temperature fields in fire scenarios poses significant challenges. Advanced thermodynamic simulations, such as those performed using Fire Dynamics Simulator (FDS) \cite{mcgrattan_fire_2000}, provide precise temperature predictions. However, these methods are hindered by high computational costs, prolonging execution times, and limited scalability, making them impractical for generating large datasets. Additionally, real-world fire loads often display substantial spatial variability across different rooms \cite{dundar_fire_2023}, resulting in scenario-specific temperature fields with limited generalizability. For example, studies on bridge fires \cite{he_study_2024} have demonstrated that environmental factors, such as wind speeds, can significantly influence temperature distributions. Furthermore, even within identical scenarios, variations in fire modeling methodologies can produce distinctly different temperature fields \cite{zhang_temperature_2020, du_new_2012}. These challenges emphasize the need for efficient and adaptable methods to generate fire temperature data.
% To address these issues, we adopted a rule-based approach to model temperature variations. 
According to \cite{spearpoint_fire_2008}, a typical fire development follows a predictable pattern. During the {\em{growth stage}}, the temperature rises slowly and approximately linearly after ignition. This is followed by the {\em{flashover stage}}, where temperatures increase rapidly to peak values. After reaching the peak, the temperature either stabilizes or continues to rise slowly until the {\em{decay stage}} begins. Inspired by this fire development pattern, we describe the temperature evolution in time, $t$, prior to the decay stage in two distinct stages:
\begin{enumerate}
    \item {\bf{Initial linear increase stage}}: For $t \in [0, t_1)$, temperature increases gradually and linearly as the fire spreads through the building. This stage represents the time before the fire directly affects a structural element.  
    \item {\bf{ISO 834 fire curve stage}}: For $t \in [t_1, t_{\thre}]$, temperature rises rapidly following the ISO 834 curve \cite{ISO834}, modeling the direct impact of the fire on the structural element. 
\end{enumerate}
The slope of the linear temperature increase, $c$, and the transition time, $t_1$, are influenced by the spatial relationship between the fire source and the structural element. For the second stage of temperature evolution, we utilize the ISO 834 curve, a widely accepted standard for fire resistance testing. This standardized fire curve describes the temperature rise over time, enabling rapid and consistent thermal fields across various scenarios. The duration of fire simulation in this study is set to $t_{\thre}=60$ minutes. This value represents the upper limit for the temperature evolution of each structural element, providing a consistent basis for analyzing the structural response to fire.

Let $(x, y, z)$ represents the midpoint of a structural element and $(x_{\subfire}, y_{\subfire}, z_{\subfire})$ the fire source point. \revise{Integer parameters $h$ and $h_{\subfire}$ correspond to the respective floor levels of the element and the fire source}. The temperature evolution for each element is expressed as follows:
\begin{enumerate}
    \item Linear increase stage ($0 < t < t_1$):
    \begin{equation}
    T(t) = c \cdot t,
    \end{equation}
    where $c$, the rate of temperature increase ($^\circ\mathrm{C}/\mathrm{min}$), depends on the height difference between the element, $h$, and the fire source, $h_{\subfire}$:
    \begin{equation}
        c = 
        \begin{cases} 
        5\left/\left(h - h_{\subfire} + 1\right)\right., & h \geq h_{\subfire}, \\
        2\left/\left(h_{\subfire} - h\right)\right., & h < h_{\subfire}.
        \end{cases}
    \end{equation}
     \item ISO 834 stage ($t \geq t_1$):
\begin{equation}
    T(t) = c \cdot t_1 + 345 \log_{10} \left(8 \left(t - t_1\right) + 1\right).
\end{equation}
\end{enumerate}

The transition (arrival) time $t_1$, marking the end of the linear stage, depends on the spatial distance between the fire source and the element. We define the following two Euclidean distances $L_p$ in the $xy$ plane and $L_s$ in the $xyz$ space:
\begin{eqnarray}
L_p & \triangleq & \sqrt{(x - x_{\subfire})^2 + (y - y_{\subfire})^2}, \\
\label{eq:Lp}
L_s & \triangleq & \sqrt{(x - x_{\subfire})^2 + (y - y_{\subfire})^2 + (z - z_{\subfire})^2}.
\label{eq:Ls}
\end{eqnarray}
Accordingly, the transition time, $t_1$, is expressed as follows:
\begin{equation}
    t_1 = 
    \begin{cases}
    \beta_{1} \cdot \left(1 - \exp\left\{- L_s\left/\alpha_{1}\right.\right\}\right), & h > h_{\subfire}, \\
    \beta_{2} \cdot \left(1 - \exp\left\{- L_p\left/\alpha_{2}\right.\right\}\right), & h = h_{\subfire}, \\
    \beta_{3} \cdot \left(1 - \exp\left\{- L_s\left/\alpha_{3}\right.\right\}\right), & h < h_{\subfire} .
    \end{cases}
    \label{eq:t1}
\end{equation}
The parameters $\beta_i$ and $\alpha_i$ for determining $t_1$ are summarized in Table~\ref{tab:fire_spread_parameters}. In this study, we take $r_{\mathrm{up}}=0.95$ and $r_{\mathrm{down}}=0.97$.
\begin{table}[ht]
    \centering
    \caption{Fire spread parameters for $t_1$ calculations.}
    \begin{tabular}{lcc}
        \toprule
        Case  & $\beta_i$ & $\alpha_i$  \\
        \midrule
        $i=1$, Upward spread & $16 \left.\left(1-r_{\mathrm{up}}^{\left|h-h_{\subfire}\right|}\right)\right/\left(1-r_{\mathrm{up}}\right)$ & $10$  \\
        $i=2$, Horizontal spread & $18$ & $18$  \\
        $i=3$, Downward spread & $30 \left.\left(1-r_{\mathrm{down}}^{\left|h-h_{\subfire}\right|}\right)\right/\left(1-r_{\mathrm{down}}\right)$ & $5$  \\
        \bottomrule
    \end{tabular}
    \label{tab:fire_spread_parameters}
\end{table}

\figref{fig:t1_curve} illustrates the $t_1$ curves for various fire scenarios: (1) fire originating on the lower floor, $h-h_{\subfire}=1$ with rapid upward spread, (2) fire on the same floor, $h=h_{\subfire}$ with the fastest spread, and (3) fire on the upper floor, $h_{\subfire}-h=1$ with slow downward spread. The exponential decay in $t_1$ reflects the accelerating fire propagation speed as the distance increases. \figref{fig:t1_curve} also indicates that the employed simplified model is consistent with the Markov chain-based dynamic model given by \cite{cheng_dynamic_2011}, where the rooms at the same floor of the fire point start flashover slightly before the corresponding upper floors. Additionally, $\beta_{1}$ and $\beta_{3}$ are the summation of a geometric sequence, where story level $h$ is the index. The common ratios $r_{\mathrm{up}}<1$ in $\beta_{1}$ and $r_{\mathrm{down}}<1$ in $\beta_{3}$ indicate that the fire speeds up to spread through the next story, which is consistent with the real-world fire spread mechanism given in \cite{hokugo_mechanism_2000}. The temperature profile within the range $t \in [0, t_{\thre}]$ is subsequently used as the thermal load in OpenSeesRT simulations to compute displacements at each structural node at time $t_{\thre}$.
\begin{figure}[h!]
    \centering
    \includegraphics[width=0.8\linewidth]{figures/m204_t1_curve.pdf}
    \caption{Three examples for the $t_1$ curve.}
    \label{fig:t1_curve}
\end{figure}

\revise{
\textit{Remark:} The effects of structural elements, such as concrete floor slabs and partitions, are not explicitly modeled in our approach. Instead, their influence is implicitly captured through the careful selection of the parameters $ \alpha, \beta, r_\mathrm{up} $, and $ r_\mathrm{down} $. This parameterization provides a unified framework for generating temperature fields. Indeed, fire propagation is governed by a multitude of factors and remains an open research question. For instance, if the fire resistance of a floor slab is enhanced by fire protective coating, the corresponding model can account for this by decreasing $\alpha_1$ \& $\alpha_3$, increasing $\beta_1$ \& $\beta_3$, and adopting larger values for $r_\mathrm{up}$ \& $r_\mathrm{down}$, which collectively slow down the vertical spread of fire. Conversely, scenarios involving higher amounts of combustible materials would warrant the opposite adjustments. This flexible and integrated approach avoids the need to design separate models for different fire propagation scenarios while still capturing the essential effects.
}

\revise{
In conclusion, our rule-based approach is a computationally efficient method for approximating fire temperature fields, enabling large-scale dataset generation to train predictive models. By combining ISO 834 fire curves with spatial considerations and embedding structural effects through parameter calibration, the method achieves a balanced trade-off between accuracy and scalability, making it a practical solution for thermal load modeling in fire scenarios. After generating the temperature of each beam or column according to the middle point, the temperature is applied as uniform thermal load to the elements of the structure in question using OpenSeesRT. 
}

% In conclusion, this rule-based approach is a computationally efficient method to approximate fire temperature fields, enabling large-scale dataset generation to train predictive models. By combining ISO 834 fire curves with spatial considerations, the method balances accuracy and scalability, making it a practical solution for thermal load modeling in fire scenarios.

% \subsection{Interstory Drift Ratio}
\subsection{OpenSeesRT Simulation}
\label{subsec:opensees_simulation}

The thermal and mechanical responses of 3D frame structures under combined fire and gravity loads are simulated using OpenSeesRT \cite{perez2024openseesrt}. \revise{In the simulation, the IDR of each node at $t_{\thre}$ is computed using the computed nodal displacements. Each structural model features six degrees of freedom per node (3 translational  and 3 rotational), with linear geometrical transformations (\texttt{geomTransf: Linear}) defining how the element local coordinate systems are mapped to the global coordinate system and assuming small displacements and rotations. Although OpenSeesRT allows a variety of options for modeling finite deformations, in the present simulations and mainly for simplicity, we did not consider large deformations. All bottom nodes (nodes on the ground) are fully constrained in all six degrees of freedom, while degrees of freedom os all other nodes are free.} Material behavior is temperature-dependent and modeled with \texttt{Steel01Thermal}, while fiber-based sections (\texttt{FiberThermal}) capture nonlinear interactions between thermal and mechanical responses at the cross-section level. \revise{Structural elements are represented as displacement-based Euler-Bernoulli beam-columns (\texttt{dispBeamColumnThermal}). This element  formulation accounts for thermal strains (temperature gradients) in the section, which is discretized into fibers. Numerical integration is used along the length of each element using three integration (Gauss) points, one at each end and the third in the middle of the element.}

{\revise{Thermal expansion of steel members plays a crucial role in IDR development. In reality, reinforced concrete floor slabs heat at a different rate than steel members due to their higher thermal mass and lower thermal conductivity. This differential heating can lead to restrained thermal expansion, introducing axial compression in beams and affecting the overall structural response. In this study, explicit {\em{composite action}} between steel members and concrete slabs is not modeled. Instead, our approach focuses on isolating the response of the steel structural frame, which is often the critical load-bearing component in fire scenarios. This assumption aligns with prior studies \cite{Possidente_2024} demonstrating that steel structures reach thermal equilibrium with surrounding gases quickly, allowing the use of uniform thermal loading in fire analysis. Future work could enhance this framework by incorporating slab-beam interaction effects, through a refined FEA for an extended dataset where constraints imposed by floor slabs are explicitly considered.}

The analysis begins with the application of gravity loads, followed by incremental thermal loads simulating the fire exposure. A static nonlinear solver using  \texttt{ExpressNewton} algorithm ensures convergence, while the \texttt{NormDispIncr} test maintains accuracy. An incremental \texttt{LoadControl} scheme with small step sizes is employed to guarantee numerical stability, using 10\% for gravity loads and 1\% for thermal loads. 

\revise{
In the thermal load analysis, uniform thermal load is applied to each beam or column, i.e., the temperature of each element is set to be that at the middle point, according to \secref{subsec:thermal_load_generation}. The \texttt{Steel01Thermal} material allows the properties (e.g., Young's modulus and yield strength) to be adjusted at increasing temperatures according to \cite{EN1993} using its Table 3.1: Reduction factors for the stress-strain relationship of carbon steel at elevated temperatures. For example, if the Young’s modulus at ambient temperature is $E_0$, then as the temperature ($T$) increases, the modulus changes as $E(T) = \eta (T) \times E_0$. \cite{EN1993} directly provides the values of $\eta(T) \in \left[0,1\right] $ at every $100 ^\circ\mathrm{C}$ interval and recommends using linear interpolation to obtain $\eta(T)$ for intermediate values of $T$.
} OpenSeesRT documentation \cite{OpenSeesThermalExamples} provides several examples of thermal analyses.

This modeling framework accommodates variations in material properties, cross-sectional geometries, and temperature profiles, providing robust simulations of structural behavior under fire conditions. The primary settings and configurations for the OpenSeesRT simulations are summarized in \tabref{tab:ops_detail}.
\begin{table}[h!]
    \centering
        \caption{Key settings of OpenSeesRT simulations.}
    \begin{tabular}{l|>{\raggedright\arraybackslash}p{0.6\linewidth}} %
    \toprule
    Modeling Aspect     & Details \\
    \midrule
    Geometry            & 3D models; 6 degrees of freedom per node \\
    Transformation      & geomTransf: Linear \\ 
    Material            & Steel01Thermal \\
    Section             & FiberThermal; Cross-section: $0.1$ m $\times$ $0.1$ m \\ 
    Element type        & {dispBeamColumnThermal} \\ 
    Loading             & Gravity loads: {beamUniform}; Thermal loads: {beamThermal} \\
    Integration scheme  & Incremental {LoadControl}; Step size: $10\%$ (gravity analysis), $1\%$ (thermal analysis) \\
    Nonlinear solver    & {ExpressNewton} algorithm; {UmfPack} solver; Convergence test: {NormDispIncr} tolerance: $10^{-8}$; Maximum \# iterations per step: $1000$. \\ 
    \bottomrule
    \end{tabular}
    \label{tab:ops_detail}
\end{table}

For each structure in the labeled dataset, 30 fire points are selected using a dual-granularity approach, \revise{i.e., two-stage sampling strategy,} to ensure they are well-distributed. Specifically, rooms are sequentially selected, with one fire point randomly chosen within each selected room. If a building is large and contains more than 30 rooms, we randomly select 30 rooms without replacement, i.e., ensuring that no more than one fire point is located in the same room. Conversely, if the building is small and has fewer than 30 rooms, all rooms are initially selected, with one fire point randomly assigned to each room. Additionally, rooms are then selected with replacement until a total of 30 fire points are assigned. \revise{The room-level sampling prioritizes selecting distinct rooms to avoid spatial clustering of fire points, while the point-level sampling ensures intra-room variability. This approach aligns with stratified sampling principles commonly used for efficient spatial representation, where multi-stage sampling strategies optimize coverage and variability, e.g., \cite{arunachalam_generalized_2023}, and enables a more comprehensive characterizing of how the structures respond under fire conditions.}
% This selection method prevents fire points from clustering too closely while maintaining an element of randomness. By distributing fire points in this manner, the 30 fire scenarios are effectively utilized, enabling a more comprehensive characterizing of how the structures respond under fire conditions.

\subsection{Summary of the Dataset Generation}
As discussed in this section and related to  \figref{fig:dataset_generation_procedure}, three key steps were considered in the development of the dataset: 
\begin{enumerate}
    \item {\bf{Filtering process}}: Structures with MIDR exceeding $1\%$ under gravity loads were excluded,  resulting in $1,573$ labeled structures retained for fire simulation and $16,050$ unlabeled structures for training the MFSP predictor.
    \item {\bf{Fire simulations}}: For each retained labeled structure, 30 fire scenarios were simulated using OpenSeesRT, yielding $47,190$ fire cases.
    \item {\bf{Data distribution check}}: MIDR distributions for labeled and unlabeled data under gravity loads were highly similar, because both datasets were generated using the same method. Under fire conditions, the MIDR distribution shifted, reflecting significant structural deformation with values reaching a maximum of about 6\%, an average of 1.70\%, and a standard deviation of 1.12\%. This step ensured a diverse and comprehensive dataset for the proposed predictive framework.
\end{enumerate}
The statistical distribution histograms for MIDR (after applying the $1\%$ filtering threshold \revise{for gravity load responses}) under different loading conditions are plotted in \figref{fig:histogram_mdr}. Figures \ref{fig:histogram_mdr}(a) and \ref{fig:histogram_mdr}(b) show the MIDR distributions of the labeled and unlabeled data, respectively, under gravity loads only. \figref{fig:histogram_mdr}(c) shows the MIDR distribution of the labeled data under the combined effects of gravity and fire loads. Fire load causes the structures to significantly deform, leading to a noticeably \revise{right-skewed} MIDR distribution.

\begin{figure*}[h!]
    \centering
    \includegraphics[width=\linewidth]{figures/histogram_mdr.pdf}
    \caption{Histograms of MIDR for labeled and unlabeled structures with gravity loads and fire cases.}
    \label{fig:histogram_mdr}
\end{figure*}

\revise{
This dataset provides the basis for training and testing the performance of the GNN-based framework. Although we employed a simplified rule-based thermal load generation method compared with conventional CFD-based simulations, the temperature field, the changes of the material properties, and the response of the structures, are all still highly nonlinear and complex. Therefore, it is still a challenging task for the NN to predict the MIDRs based on this dataset.
}
\begin{figure*}[t]
    \centering
    \includegraphics[width=\linewidth]{Figures/Opacity_gradient_pruning_comparison.pdf}
    \caption{Comparison between opacity based pruning and gradient + opacity based pruning. The number of Gaussians are in millions (M).}
    \label{fig:opacity_gradient_pruning_comparison}
\end{figure*}

\begin{figure}[t]
    \centering
    \includegraphics[width=0.8\linewidth]{Figures/b_p_opacity_1.pdf}
    \caption{Opacity distribution before and after pruning for the \texttt{truck} scene. Figure taken from~\cite{salman2024trimming}.}
    \label{fig:opacity_distribution}
\end{figure}
\noindent
\textbf{Tiling: }
3DGS reduces computational costs by shifting rendering accuracy from fine-grained, pixel-level detail to a more coarse, patch-level approach. The image is first divided into non-overlapping patches or tiles, and then 3DGS identifies which tiles intersect with the projected Gaussians. If a single projected Gaussian spans multiple tiles, it duplicates the Gaussian, assigning each copy a unique tile ID corresponding to the relevant tile as shown in Figure~\ref{fig:working}.

\noindent
\textbf{Rendering Parallelization: }Following the Gaussians' replication, 3DGS associates each corresponding tile ID with the depth values obtained from the view transformation. This process generates an unsorted list of bytes where the higher bits represent the tile ID and the lower bits encode the depth information. This list can then be sorted for rendering (alpha compositing). This method is particularly well-suited for parallel processing because it allows each tile and pixel to be handled independently as shown in the Figure~\ref{fig:working}. Additionally, the advantage of allowing each pixel of the tile to access shared memory while maintaining a unified read pattern enables more efficient parallel processing during the alpha compositing stage. The framework essentially treats tile and pixel processing similarly to blocks and threads in CUDA programming, as demonstrated in the original study's implementation.

%After replicating the Gaussians, 3DGS associates the depth values obtained from the view transformation with each corresponding tile ID. This results in an unsorted list of bytes where the lower bits represent the depth and the upper bits denote the tile ID. This list can then be sorted for rendering (alpha compositing). This method is particularly well-suited for parallel processing because it allows each tile and pixel to be handled independently. Additionally, the advantage of each tile's pixels being able to access a shared memory while maintaining a consistent read sequence facilitates more efficient parallel execution of alpha compositing. The framework essentially treats tile and pixel processing similarly to blocks and threads in CUDA programming, as demonstrated in the original study's implementation.

In summary, 3DGS enhances computational efficiency while preserving high image reconstruction quality by incorporating several approximations during the rendering process.

\noindent
\textbf{3D Gaussian Splatting Optimization: }
%The core of 3DGS involves an optimization process aimed at generating a substantial number of 3D Gaussians that effectively capture the scene's essence, enabling free-viewpoint rendering. 
The key component of 3DGS lies in an optimization process aimed at generating a substantial number of Gaussians that effectively encapsulate the key features of the scene, thereby enabling real-time 3D scene rendering. Differentiable rendering is employed to fine-tune the parameters of these 3D Gaussians to align with the scene's textures. However, the optimal number of 3D Gaussians required to accurately represent a scene is not predetermined. %In the following sections, we will address the optimization of each Gaussian's attributes and the regulation of Gaussian density. 
The optimization workflow encompasses a series of interrelated procedures.

\noindent
\textbf{Loss Function: }
%After completing the image synthesis, the discrepancy between the rendered image and the ground truth can be assessed. 
The disparity between the ground truth and the rendered output can be evaluated upon completion of the image synthesis. To achieve this, the \( \ell_1 \) loss and D-SSIM (Differentiable Structural Similarity Index)~\cite{bakurov2022structural} loss functions are utilized. Stochastic gradient descent (SGD) is then employed to optimize all the learnable parameters. The loss function used for 3DGS is given by~\cite{kerbl20233d}:

\[ L = (1 - \lambda)L_1 + \lambda L_{\text{D-SSIM}}, \]
\noindent
where \( \lambda \in [0, 1] \) is a weighting factor.





\noindent
\textbf{Parameter Optimization:}
Back-propagation can be utilized to directly optimize most attributes of a 3D Gaussian. 
It is crucial to note that optimization of the covariance matrix directly could yield a non-positive semi-definite matrix, thereby contradicting the physical restrictions typically imposed on covariance matrices.
%However, a key consideration is that directly optimizing the covariance matrix \(\Sigma\) might result in a non-positive semi-definite matrix, which conflicts with the physical constraints typically associated with covariance matrices. 
To address this issue, 3DGS employs an alternative approach by optimizing a quaternion \(q\) and a 3D vector \(s\), where \(q\) represents rotation and \(s\) represents scale. This method ensures that the resulting covariance matrix adheres to the required physical properties. This method enables the covariance matrix \(\Sigma\) to be reconstructed using the following formula:

\[
\Sigma = R S S^\top R^\top,
\]
\noindent
where \(S\) and \(R\) represent the scaling and rotation matrices derived from \(s\) and \(q\), respectively. The process of calculating opacity \(\alpha\) involves a complex computational graph: \(s\) and \(q \to \Sigma\), \(\Sigma \to \Sigma'\), and \(\Sigma' \to \alpha\). To avoid the computational expense of automatic differentiation, 3DGS directly derives and computes the gradients for \(q\) and \(s\) during optimization.

%\begin{figure}[t]
    \centering
    \includegraphics[width=0.8\linewidth]{Figures/b_p_opacity_1.pdf}
    \caption{Opacity distribution before and after pruning for the \texttt{truck} scene. Figure taken from~\cite{salman2024trimming}.}
    \label{fig:opacity_distribution}
\end{figure}

\noindent
\textbf{Densification Process:}
In 3DGS, optimization parameters are initialized using sparse points from Structure-from-Motion (SfM)~\cite{snavely2006photo}. Densification and pruning techniques are then applied to optimize the density of 3D Gaussians.
During densification, 3DGS adaptively increases Gaussian density in regions with sparse coverage or missing geometric features. After specific training intervals, Gaussians with gradients above a threshold are densified: large Gaussians are split into smaller ones, and small Gaussians are duplicated and shifted along gradient directions. This ensures an optimal distribution for improved 3D scene synthesis.
In the pruning stage, redundant or insignificant Gaussians are removed to regularize the representation. Gaussians with excessive view-space size or opacity \( \alpha \) below a threshold are discarded, ensuring efficiency while preserving accuracy in scene representation.
