\documentclass[journal]{IEEEtran}
\IEEEoverridecommandlockouts
% The preceding line is only needed to identify funding in the first footnote. If that is unneeded, please comment it out.
\usepackage{cite}
\usepackage{amsmath,amssymb,amsfonts}
\usepackage{algorithmic}
\usepackage{graphicx}
\usepackage{textcomp}
\usepackage[table]{xcolor}
\usepackage{booktabs}
\usepackage{tikz}
\usepackage{multirow}
\usepackage{array}
\usepackage{caption}
\usepackage{tikz}
\usepackage{hyperref}
\usepackage{orcidlink}

\usetikzlibrary{shapes.geometric, arrows, positioning}

\newcommand{\enzo}[1]{\textcolor{red}{Enzo: #1}}
\newcommand{\CN}[1]{\textcolor{blue}{Chaonig: #1}}

\usetikzlibrary{arrows.meta, positioning}
\def\BibTeX{{\rm B\kern-.05em{\sc i\kern-.025em b}\kern-.08em
    T\kern-.1667em\lower.7ex\hbox{E}\kern-.125emX}}

\tikzstyle{box} = [rectangle, rounded corners, minimum width=3cm, minimum height=1cm, text centered, draw=black, fill=blue!20]
\tikzstyle{arrow} = [thick,->,>=stealth]

\begin{document}


\title{Compression in 3D Gaussian Splatting: A Survey of Methods, Trends, and Future Directions}

% \author{Muhammad Salman Ali,
% {Marco Cagnazzo \orcidlink{0000-0001-6731-3755}}, \IEEEmembership{Senior Member, IEEE},
% {Enzo Tartaglione \orcidlink{0000-0003-4274-8298}}, \IEEEmembership{Senior Member, IEEE},
% {Giuseppe Valenzise \orcidlink{0000-0002-5840-5743}}, \IEEEmembership{Senior Member, IEEE},
% {Chaoning Zhang \orcidlink{0000-0001-6007-6099}}, \IEEEmembership{Senior Member, IEEE}, and 
% {Sung-Ho Bae \orcidlink{0000-0003-2677-3186}}, \IEEEmembership{Member, IEEE}
\author{
    Muhammad Salman Ali, 
    Chaoning Zhang \orcidlink{0000-0001-6007-6099}, 
    \IEEEmembership{Senior Member, IEEE}, 
    Marco Cagnazzo \orcidlink{0000-0001-6731-3755}, 
    \IEEEmembership{Senior Member, IEEE}, 
    Giuseppe Valenzise \orcidlink{0000-0002-5840-5743}, 
    \IEEEmembership{Senior Member, IEEE}, 
    Enzo Tartaglione \orcidlink{0000-0003-4274-8298}, 
    \IEEEmembership{Senior Member, IEEE}, and \\
    Sung-Ho Bae \orcidlink{0000-0003-2677-3186}, 
    \IEEEmembership{Member, IEEE}

%\corresp{Corresponding Author: Sung Ho Bae (email: shbae@khu.ac.kr).}

\thanks{  Muhammad Salman Ali and Sung-Ho Bae are with the Department of Computer Science and Engineering, Kyung Hee University, South Korea (email: salmanali@khu.ac.kr).

        Chaoning Zhang is with School of Computer Science and Engineering, University of Electronic Science and Technology of China (UESTC), Chengdu, China. 

        Marco Cagnazzo is with LTCI, T\'el\'ecom Paris, Institut Polytechnique de Paris, France and Università degli Studi di Padova, Padua, Italy. 
        
        Giuseppe Valenzise is with the CNRS, CentraleSup\'elec, Laboratoire des Signaux et Systèmes, Université Paris-Saclay, France.
        
        Enzo Tartaglione is with LTCI, T\'el\'ecom Paris, Institut Polytechnique de Paris, France.

        }}

\maketitle
\begin{abstract}
%3D Gaussian Splatting (3DGS) has recently emerged as a novel technique in explicit radiance fields and computer graphics. 3DGS utilizes millions of learnable 3D Gaussians representing a major shift from conventional neural radiance field methods, which primarily employ implicit, coordinate-based models to map spatial coordinates to pixel values. 3DGS positions itself as a potential game-changer for the next generation of 3D reconstruction and representation due to its inherent explicit scene representation and differentiable rendering algorithm. 
%3D Gaussian Splatting (3DGS) has recently emerged as a pioneering approach in explicit scene rendering and computer graphics. Unlike traditional neural radiance field methods, which typically rely on implicit, coordinate-based models to map spatial coordinates to pixel values, 3DGS utilizes millions of learnable 3D Gaussians. Its differentiable rendering technique and inherent capability for explicit scene representation and manipulation positions 3DGS as a potential game-changer for the next generation of 3D reconstruction and representation technologies. This enables 3DGS to deliver real-time rendering speeds while offering unparalleled editability levels. However, despite rapid advancements, the literature regarding the scalability of these models is still in its infancy \CN{if there isn't much work, why bother a survey?}. This work shall present a comprehensive overview focusing on the scalability and compression \CN{in the title, only compression, but here both} of 3DGS. We will \CN{avoid using future tense, just say we begin with} begin with a detailed background overview of 3DGS and an overview of 3DGS compression techniques. Additionally, we shall provide insights into how we can draw inspiration from efficient Neural Radiance Field (NeRF) techniques for future advancements. We will conclude the survey by identifying present challenges in the research and potential directions to be explored in the future. \CN{overall, future tense is not necessary}
%3D Gaussian Splatting (GS) has recently emerged as a transformative technique in explicit radiance fields and computer graphics. This innovative approach, characterized by the use of millions of learnable 3D Gaussians, marks a significant departure from mainstream neural radiance field methods, which predominantly employ implicit, coordinate-based models to map spatial coordinates to pixel values. With its explicit scene representation and differentiable rendering algorithm, 3D GS not only promises real-time rendering capabilities but also introduces unprecedented levels of editability. This positions 3D GS as a potential game-changer for the next generation of 3D reconstruction and representation. In this paper, we provide the first systematic overview of recent developments and critical contributions in the domain of 3D GS compression. We begin with a detailed exploration of the underlying principles of compression and the driving forces behind the emergence of 3D GS, laying the groundwork for understanding its significance and how it can be integrated with compression principles. A focal point of our discussion is the compressibility of 3D GS, complemented by a comparative analysis of recent compressed 3D GS models, evaluated across various benchmark tasks to highlight their performance and practical utility. The survey concludes by identifying current challenges and suggesting potential avenues for future research in this domain.
3D Gaussian Splatting (3DGS) has recently emerged as a pioneering approach in explicit scene rendering and computer graphics. Unlike traditional neural radiance field (NeRF) methods, which typically rely on implicit, coordinate-based models to map spatial coordinates to pixel values, 3DGS utilizes millions of learnable 3D Gaussians. Its differentiable rendering technique and inherent capability for explicit scene representation and manipulation positions 3DGS as a potential game-changer for the next generation of 3D reconstruction and representation technologies. This enables 3DGS to deliver real-time rendering speeds while offering unparalleled editability levels. However, despite its advantages, 3DGS suffers from substantial memory and storage requirements, posing challenges for deployment on resource-constrained devices. In this survey, we provide a comprehensive overview focusing on the scalability and compression of 3DGS. We begin with a detailed background overview of 3DGS, followed by a structured taxonomy of existing compression methods. Additionally, we analyze and compare current methods from the topological perspective, evaluating their strengths and limitations in terms of fidelity, compression ratios, and computational efficiency. Furthermore, we explore how advancements in efficient NeRF representations can inspire future developments in 3DGS optimization. Finally, we conclude with current research challenges and highlight key directions for future exploration.
\end{abstract}


\section{Introduction}
\IEEEPARstart{T}{ransforming} a collection of views, images, or video capturing a scene into a 3D model such that computers can process it is the goal of image-based 3D scene reconstruction. This complex and enduring problem is crucial for enabling machines to understand the complexities of real-world environments, paving the way to a diverse range of applications including 3D animation and modeling,  navigation of robots, scene preservation, virtual/augmented reality, and autonomous driving~\cite{kalkofen2008comprehensible,patney2016towards,albert2017latency}. The evolution of 3D scene reconstruction precedes the rise of deep learning, with initial efforts focusing on light fields and fundamental scene reconstruction techniques~\cite{gortler2023lumigraph,levoy2023light,buehler2023unstructured}. However, these early works faced limitations due to their dependence on dense sampling and structured capture, which presented significant challenges in managing complex scenes and lighting variations. The introduction of structure-from-motion~\cite{snavely2006photo,goesele2007multi} and the subsequent enhancements in multiview stereo algorithms offered a more resilient foundation for 3D scene reconstruction. However, these approaches encountered difficulties in synthesizing novel views and lacked alignment with deep scene understanding models.

Before Gaussian Splatting gained its popularity for the task of 3D reconstruction, NeRFs~\cite{mildenhall2020nerf} constituted as a go-to method, representing a significant breakthrough in this advancement. Using fully connected neural networks, NeRFs facilitate the direct mapping of spatial coordinates to color and density. The success of NeRFs lies in their capability to generate continuous, volumetric scene functions, yielding results with remarkable detail and realism~\cite{liao2024ov,lin2025dynamic,sheng2024open,zhu2024dfie3d,ding2024ray}. However, like any emerging technology, this implementation incurs certain costs.
\begin{enumerate}
    \item Computational Complexity: NeRF-based methods have significantly high computational complexity~\cite{chen2022tensorf,garbin2021fastnerf,takikawa2021neural}, often requiring long training times and significant resources for rendering, particularly for high-resolution outputs. 
    \item Scene Editability: Performing implicit scenes can be challenging, as modifying the neural network’s weights does not directly correspond to changes in the geometry or appearance of the scene~\cite{ali2024elmgs,qian20233dgs,lee2023compact}.
\end{enumerate}

% \begin{table*}[t]
%     \caption{Taxonomy and representative publications of 3DGS compression methods.}
%     \label{tab:Taxonomy}
%     \centering
%     \small
%     \renewcommand{\arraystretch}{1.3} % Increase row height for better readability
%     \resizebox{0.9\textwidth}{!}{
%     \begin{tabular}{l l p{5cm}}
%     \toprule
%     \textbf{Category} & \textbf{Sub-Type} & \textbf{Representative Publications} \\
%     \midrule
%     \multirow{13}{*}{\textbf{Unstructured Compression}} 
%         & \textbf{Pruning} 
%             & Fan et al. \cite{fan2023lightgaussian}, Lee et al. \cite{lee2023compact}, Navaneet et al. \cite{navaneet2023compact3d}, Girish et al. \cite{girish2023eagles}, Papantonakis et al. \cite{papantonakis2024reducing}, Salman et al. \cite{salman2024trimming}, Lee et al. \cite{lee2024safeguardgs}, Liu et al. \cite{liu2024efficientgs}, Wang et al. \cite{wang2024end}, Hanson et al. \cite{hanson2024pup}, Kim et al. \cite{kim2024color}, Lin et al. \cite{lin2024rtgs}, Kheradmand et al. \cite{kheradmand20243d}, Zhang et al. \cite{zhang2024lp}, Fang et al. \cite{fang2024mini}, GoDe~\cite{di2025gode}, ELMGS~\cite{ali2024elmgs} \\
%         \cmidrule(l){2-3}
%         & \textbf{Quantization} 
%             & Fan et al. \cite{fan2023lightgaussian}, Lee et al. \cite{lee2023compact}, Navaneet et al. \cite{navaneet2023compact3d}, Niedermayr et al. \cite{niedermayr2024compressed}, Girish et al. \cite{girish2023eagles}, Papantonakis et al. \cite{papantonakis2024reducing}, Wang et al. \cite{wang2024end}, Morgenstern et al. \cite{morgenstern2023compact}, GoDe~\cite{di2025gode}, ELMGS~\cite{ali2024elmgs} \\
%         \cmidrule(l){2-3}
%         & \textbf{Entropy Coding} 
%             & Lee et al. \cite{lee2023compact}, Navaneet et al. \cite{navaneet2023compact3d}, Niedermayr et al. \cite{niedermayr2024compressed}, Girish et al. \cite{girish2023eagles}, Morgenstern et al. \cite{morgenstern2023compact},  ELMGS~\cite{ali2024elmgs}\\
%     \midrule
%     \multirow{7}{*}{\textbf{Structured Compression}} 
%         & \textbf{Graph-Based} 
%             & Yang et al. \cite{yang2024spectrally}, Chen et al. \cite{chen2015discrete}, Zhang et al. \cite{zhang2024gaussian}, Liu et al. \cite{liu2024compgs} \\
%         \cmidrule(l){2-3}
%         & \textbf{Anchor-Based} 
%             & Lu et al. \cite{lu2023scaffold}, Chen et al. \cite{chen2024hac}, Wang et al. \cite{wang2024contextgs} \\
%         \cmidrule(l){2-3}
%         & \textbf{Contextual/AR Modeling} 
%             & Wang et al. \cite{wang2024contextgs} \\
%         \cmidrule(l){2-3}
%         & \textbf{Factorization Approach} 
%             & Sun et al. \cite{sun2024f}, Niemeyer et al. \cite{niemeyer2024radsplat} \\
%     \bottomrule
%     \end{tabular}
%     }
% \end{table*}
\begin{table*}[t]
    \caption{Taxonomy and representative publications of 3DGS compression methods.}
    \label{tab:Taxonomy}
    \centering
    \footnotesize % Reduce font size slightly
    \renewcommand{\arraystretch}{1.2} % Reduce row height slightly
    \resizebox{0.95\textwidth}{!}{ % Adjust table width for better spacing
    \begin{tabular}{l l p{10cm}} % Adjust column width
    \toprule
    \textbf{Category} & \textbf{Sub-Type} & \textbf{Representative Publications} \\
    \midrule
    \multirow{9}{*}{\textbf{Unstructured Compression}} 
        & \textbf{Pruning} 
            & LightGaussian~\cite{fan2023lightgaussian}, Compact 3D~\cite{lee2023compact}, CompGS~\cite{navaneet2023compact3d}, 
              EAGLES~\cite{girish2023eagles}, Papantonakis et al.~\cite{papantonakis2024reducing}, 
              Trimming the Fat~\cite{salman2024trimming}, SafeguardGS~\cite{lee2024safeguardgs}, EfficientGS~\cite{liu2024efficientgs}, 
              RDO-Gaussian~\cite{wang2024end}, Pup 3D-GS~\cite{hanson2024pup}, Kim et al.~\cite{kim2024color}, 
              RTGS~\cite{lin2024rtgs}, Kheradmand et al.~\cite{kheradmand20243d}, LP-3DGS~\cite{zhang2024lp}, 
              Mini-Splatting~\cite{fang2024mini}, GoDe~\cite{di2025gode}, ELMGS~\cite{ali2024elmgs} \\
        \cmidrule(l){2-3}
        & \textbf{Quantization} 
            & LightGaussian~\cite{fan2023lightgaussian}, Compact 3D~\cite{lee2023compact}, CompGS~\cite{navaneet2023compact3d}, 
              Niedermayr et al.~\cite{niedermayr2024compressed}, EAGLES~\cite{girish2023eagles}, Papantonakis et al.~\cite{papantonakis2024reducing}, 
              RDO-Gaussian~\cite{wang2024end}, Morgenstern et al.~\cite{morgenstern2023compact}, GoDe~\cite{di2025gode}, 
              ELMGS~\cite{ali2024elmgs} \\
        \cmidrule(l){2-3}
        & \textbf{Entropy Coding} 
            & Compact 3D~\cite{lee2023compact}, CompGS~\cite{navaneet2023compact3d}, Niedermayr et al.~\cite{niedermayr2024compressed}, 
              EAGLES~\cite{girish2023eagles}, Morgenstern et al.~\cite{morgenstern2023compact}, ELMGS~\cite{ali2024elmgs} \\
    \midrule
    \multirow{5}{*}{\textbf{Structured Compression}} 
        & \textbf{Graph-Based} 
            & SUNDAE~\cite{yang2024spectrally}, Gaussian-Forest (GF)~\cite{zhang2024gaussian}, Liu et al.~\cite{liu2024compgs} \\
        \cmidrule(l){2-3}
        & \textbf{Anchor-Based} 
            & Scaffold-GS~\cite{lu2023scaffold}, HAC~\cite{chen2024hac}, Context-GS~\cite{wang2024contextgs} \\
        \cmidrule(l){2-3}
        & \textbf{Contextual/AR Modeling} 
            & Context-GS~\cite{wang2024contextgs} \\
        \cmidrule(l){2-3}
        & \textbf{Factorization Approach} 
            & F-3DGS~\cite{sun2024f}, Radsplat~\cite{niemeyer2024radsplat} \\
    \bottomrule
    \end{tabular}
    }
\end{table*}

% \section{Taxonomy}

% As illustrated by Fig. \ref{}, the typical process of vision models based time series analysis has five components: (1) normalization/scaling; (2) time series to image transformation; (3) image modeling; (4) image to time series recovery; and (5) task processing. In the rest of this paper, we will discuss the typical methods for each of these components. The detailed taxonomy of the methods are summarized in Table \ref{tab.taxonomy}.

%Typical step: normalization/scaling, transformation, vision modeling, task-specific head, inverse transformation (for tasks that output time series, e.g., forecasting, generation, imputation, anomaly detection). Normalization is to fit the arbitrary range of time series values to RGB representation.

\begin{figure*}[!t]
\centering
\includegraphics[width=1.0\textwidth]{fig/fig_3.pdf}
% \vspace{-1em}
\caption{An illustration of different methods for imaging time series with a sample (length=336) from the \textit{Electricity} benchmark dataset \protect\cite{nie2023time}. (a)(c)(d)(e)(f) %are univariate methods.
visualize the same variate. (b) visualizes all 321 variates. Filterbank is omitted due to its %high
similarity to STFT.}\label{fig.tsimage}
\vspace{-0.2cm}
\end{figure*}

\begin{table*}[t]
\centering
\scriptsize
\setlength{\tabcolsep}{2.7pt}{
% \begin{tabular}{llllllllllll}
\begin{tabular}{llcccccccccl}
\toprule[1pt]
\multirow{2}{*}{Method} & \multirow{2}{*}{TS-Type} & \multirow{2}{*}{Imaging} & \multicolumn{5}{c}{Imaged Time Series Modeling} & \multirow{2}{*}{TS-Recover} & \multirow{2}{*}{Task} & \multirow{2}{*}{Domain} & \multirow{2}{*}{Code}\\ \cmidrule{4-8}
 & & & Multi-modal & Model & Pre-trained & Fine-tune & Prompt & & & & \\ \midrule
\cite{silva2013time} & UTS & RP & \xmark & \texttt{K-NN} & \xmark & \xmark & \xmark & \xmark & Classification & General & \xmark\\
\cite{wang2015encoding} & UTS & GAF & \xmark & \texttt{CNN} & \xmark & \cmark$^{\flat}$ & \xmark & \cmark & Classification & General & \xmark\\
\cite{wang2015imaging} & UTS & GAF & \xmark & \texttt{CNN} & \xmark & \cmark$^{\flat}$ & \xmark & \cmark & Multiple & General & \xmark\\
% \multirow{2}{*}{\cite{wang2015imaging}} & \multirow{2}{*}{UTS} & \multirow{2}{*}{GAF} & \multirow{2}{*}{\xmark} & \multirow{2}{*}{\texttt{CNN}} & \multirow{2}{*}{\xmark} & \multirow{2}{*}{\cmark$^{\flat}$} & \multirow{2}{*}{\xmark} & \multirow{2}{*}{\cmark} & Classification & \multirow{2}{*}{General} & \multirow{2}{*}{\xmark}\\
% & & & & & & & & & \& Imputation & & \\
\cite{ma2017learning} & MTS & Heatmap & \xmark & \texttt{CNN} & \xmark & \cmark$^{\flat}$ & \xmark & \cmark & Forecasting & Traffic & \xmark\\
\cite{hatami2018classification} & UTS & RP & \xmark & \texttt{CNN} & \xmark & \cmark$^{\flat}$ & \xmark & \xmark & Classification & General & \xmark\\
\cite{yazdanbakhsh2019multivariate} & MTS & Heatmap & \xmark & \texttt{CNN} & \xmark & \cmark$^{\flat}$ & \xmark & \xmark & Classification & General & \cmark\textsuperscript{\href{https://github.com/SonbolYb/multivariate_timeseries_dilated_conv}{[1]}}\\
MSCRED \cite{zhang2019deep} & MTS & Other ($\S$\ref{sec.othermethod}) & \xmark & \texttt{ConvLSTM} & \xmark & \cmark$^{\flat}$ & \xmark & \xmark & Anomaly & General & \cmark\textsuperscript{\href{https://github.com/7fantasysz/MSCRED}{[2]}}\\
\cite{li2020forecasting} & UTS & RP & \xmark & \texttt{CNN} & \cmark & \cmark & \xmark & \xmark & Forecasting & General & \cmark\textsuperscript{\href{https://github.com/lixixibj/forecasting-with-time-series-imaging}{[3]}}\\
\cite{cohen2020trading} & UTS & LinePlot & \xmark & \texttt{Ensemble} & \xmark & \cmark$^{\flat}$ & \xmark & \xmark & Classification & Finance & \xmark\\
% \cite{du2020image} & UTS & Spectrogram & \xmark & \texttt{CNN} & \xmark & \cmark$^{\flat}$ & \xmark & \xmark & Classification & Finance & \xmark\\
\cite{barra2020deep} & UTS & GAF & \xmark & \texttt{CNN} & \xmark & \cmark$^{\flat}$ & \xmark & \xmark & Classification & Finance & \xmark\\
% \cite{barra2020deep} & UTS & GAF & \xmark & \texttt{VGG-16} & \xmark & \cmark$^{\flat}$ & \xmark & \xmark & Classification & Finance & \xmark\\
% \cite{cao2021image} & UTS & RP & \xmark & \texttt{CNN} & \xmark & \cmark$^{\flat}$ & \xmark & \xmark & Classification & General & \xmark\\
VisualAE \cite{sood2021visual} & UTS & LinePlot & \xmark & \texttt{CNN} & \xmark & \cmark$^{\flat}$ & \xmark & \cmark & Forecasting & Finance & \xmark\\
% VisualAE \cite{sood2021visual} & UTS & LinePlot & \xmark & \texttt{CNN} & \xmark & \cmark$^{\flat}$ & \xmark & \xmark & Img-Generation & Finance & \xmark\\
\cite{zeng2021deep} & MTS & Heatmap & \xmark & \texttt{CNN,LSTM} & \xmark & \cmark$^{\flat}$ & \xmark & \cmark & Forecasting & Finance & \xmark\\
% \cite{zeng2021deep} & MTS & Heatmap & \xmark & \texttt{SRVP} & \xmark & \cmark$^{\flat}$ & \xmark & \cmark & Forecasting & Finance & \xmark\\
AST \cite{gong2021ast} & UTS & Spectrogram & \xmark & \texttt{DeiT} & \cmark & \cmark & \xmark & \xmark & Classification & Audio & \cmark\textsuperscript{\href{https://github.com/YuanGongND/ast}{[4]}}\\
TTS-GAN \cite{li2022tts} & MTS & Heatmap & \xmark & \texttt{ViT} & \xmark & \cmark$^{\flat}$ & \xmark & \cmark & Ts-Generation & Health & \cmark\textsuperscript{\href{https://github.com/imics-lab/tts-gan}{[5]}}\\
SSAST \cite{gong2022ssast} & UTS & Spectrogram & \xmark & \texttt{ViT} & \cmark$^{\natural}$ & \cmark & \xmark & \xmark & Classification & Audio & \cmark\textsuperscript{\href{https://github.com/YuanGongND/ssast}{[6]}}\\
MAE-AST \cite{baade2022mae} & UTS & Spectrogram & \xmark & \texttt{MAE} & \cmark$^{\natural}$ & \cmark & \xmark & \xmark & Classification & Audio & \cmark\textsuperscript{\href{https://github.com/AlanBaade/MAE-AST-Public}{[7]}}\\
AST-SED \cite{li2023ast} & UTS & Spectrogram & \xmark & \texttt{SSAST,GRU} & \cmark & \cmark & \xmark & \xmark & EventDetection & Audio & \xmark\\
\cite{jin2023classification} & UTS & %Multiple
LinePlot & \xmark & \texttt{CNN} & \cmark & \cmark & \xmark & \xmark & Classification & Physics & \xmark\\
ForCNN \cite{semenoglou2023image} & UTS & LinePlot & \xmark & \texttt{CNN} & \xmark & \cmark$^{\flat}$ & \xmark & \xmark & Forecasting & General & \xmark\\
Vit-num-spec \cite{zeng2023pixels} & UTS & Spectrogram & \xmark & \texttt{ViT} & \xmark & \cmark$^{\flat}$ & \xmark & \xmark & Forecasting & Finance & \xmark\\
% \cite{wimmer2023leveraging} & MTS & LinePlot & \xmark & \texttt{CLIP,LSTM} & \cmark & \cmark & \xmark & \xmark & Classification & Finance & \xmark\\
ViTST \cite{li2023time} & MTS & LinePlot & \xmark & \texttt{Swin} & \cmark & \cmark & \xmark & \xmark & Classification & General & \cmark\textsuperscript{\href{https://github.com/Leezekun/ViTST}{[8]}}\\
MV-DTSA \cite{yang2023your} & UTS\textsuperscript{*} & LinePlot & \xmark & \texttt{CNN} & \xmark & \cmark$^{\flat}$ & \xmark & \cmark & Forecasting & General & \cmark\textsuperscript{\href{https://github.com/IkeYang/machine-vision-assisted-deep-time-series-analysis-MV-DTSA-}{[9]}}\\
TimesNet \cite{wu2023timesnet} & MTS & Heatmap & \xmark & \texttt{CNN} & \xmark & \cmark$^{\flat}$ & \xmark & \cmark & Multiple & General & \cmark\textsuperscript{\href{https://github.com/thuml/TimesNet}{[10]}}\\
ITF-TAD \cite{namura2024training} & UTS & Spectrogram & \xmark & \texttt{CNN} & \cmark & \xmark & \xmark & \xmark & Anomaly & General & \xmark\\
\cite{kaewrakmuk2024multi} & UTS & GAF & \xmark & \texttt{CNN} & \cmark & \cmark & \xmark & \xmark & Classification & Sensing & \xmark\\
HCR-AdaAD \cite{lin2024hierarchical} & MTS & RP & \xmark & \texttt{CNN,GNN} & \xmark & \cmark$^{\flat}$ & \xmark & \xmark & Anomaly & General & \xmark\\
FIRTS \cite{costa2024fusion} & UTS & Other ($\S$\ref{sec.othermethod}) & \xmark & \texttt{CNN} & \xmark & \cmark$^{\flat}$ & \xmark & \xmark & Classification & General & \cmark\textsuperscript{\href{https://sites.google.com/view/firts-paper}{[11]}}\\
% \multirow{2}{*}{FIRTS \cite{costa2024fusion}} & \multirow{2}{*}{UTS} & Spectrogram & \multirow{2}{*}{\xmark} & \multirow{2}{*}{\texttt{CNN}} & \multirow{2}{*}{\xmark} & \multirow{2}{*}{\cmark$^{\flat}$} & \multirow{2}{*}{\xmark} & \multirow{2}{*}{\xmark} & \multirow{2}{*}{Classification} & \multirow{2}{*}{General} & \multirow{2}{*}{\cmark\textsuperscript{\href{https://sites.google.com/view/firts-paper}{[2]}}}\\
%  & & \& GAF,RP,MTF & & & & & & & & & \\
% \cite{homenda2024time} & UTS\textsuperscript{*} & Multiple & \xmark & \texttt{CNN} & \xmark & \cmark$^{\flat}$ & \xmark & \xmark & Classification & General & \xmark\\
CAFO \cite{kim2024cafo} & MTS & RP & \xmark & \texttt{CNN,ViT} & \xmark & \cmark$^{\flat}$ & \xmark & \xmark & Explanation & General & \cmark\textsuperscript{\href{https://github.com/eai-lab/CAFO}{[12]}}\\
% \multirow{2}{*}{CAFO \cite{kim2024cafo}} & \multirow{2}{*}{MTS} & \multirow{2}{*}{RP} & \multirow{2}{*}{\xmark} & \texttt{ShuffleNet,ResNet} & \multirow{2}{*}{\cmark} & \multirow{2}{*}{\cmark} & \multirow{2}{*}{\xmark} & \multirow{2}{*}{\xmark} & Classification & \multirow{2}{*}{General} & \multirow{2}{*}{\cmark}\\
%  & & & & \texttt{MLP-Mixer,ViT} & & & & & \& Explanation & & \\
ViTime \cite{yang2024vitime} & UTS\textsuperscript{*} & LinePlot & \xmark & \texttt{ViT} & \cmark$^{\natural}$ & \cmark & \xmark & \cmark & Forecasting & General & \cmark\textsuperscript{\href{https://github.com/IkeYang/ViTime}{[13]}}\\
ImagenTime \cite{naiman2024utilizing} & MTS & Other ($\S$\ref{sec.othermethod}) & \xmark & %\texttt{Diffusion}
\texttt{CNN} & \xmark & \cmark$^{\flat}$ & \xmark & \cmark & Ts-Generation & General & \cmark\textsuperscript{\href{https://github.com/azencot-group/ImagenTime}{[14]}}\\
TimEHR \cite{karami2024timehr} & MTS & Heatmap & \xmark & \texttt{CNN} & \xmark & \cmark$^{\flat}$ & \xmark & \cmark & Ts-Generation & Health & \cmark\textsuperscript{\href{https://github.com/esl-epfl/TimEHR}{[15]}}\\
VisionTS \cite{chen2024visionts} & UTS\textsuperscript{*} & Heatmap & \xmark & \texttt{MAE} & \cmark & \cmark & \xmark & \cmark & Forecasting & General & \cmark\textsuperscript{\href{https://github.com/Keytoyze/VisionTS}{[16]}}\\ \midrule
InsightMiner \cite{zhang2023insight} & UTS & LinePlot & \cmark & \texttt{LLaVA} & \cmark & \cmark & \cmark & \xmark & Txt-Generation & General & \xmark\\
\cite{wimmer2023leveraging} & MTS & LinePlot & \cmark & \texttt{CLIP,LSTM} & \cmark & \cmark & \xmark & \xmark & Classification & Finance & \xmark\\
% \cite{dixit2024vision} & UTS & Spectrogram & \cmark & \texttt{GPT4o,Gemini} & \cmark & \xmark & \cmark & \xmark & Classification & Audio & \xmark\\
\multirow{2}{*}{\cite{dixit2024vision}} & \multirow{2}{*}{UTS} & \multirow{2}{*}{Spectrogram} & \multirow{2}{*}{\cmark} & \texttt{GPT4o,Gemini} & \multirow{2}{*}{\cmark} & \multirow{2}{*}{\xmark} & \multirow{2}{*}{\cmark} & \multirow{2}{*}{\xmark} & \multirow{2}{*}{Classification} & \multirow{2}{*}{Audio} & \multirow{2}{*}{\xmark}\\
 & & & & \& \texttt{Claude3} & & & & & & & \\
\cite{daswani2024plots} & MTS & LinePlot & \cmark & \texttt{GPT4o,Gemini} & \cmark & \xmark & \cmark & \xmark & Multiple & General & \xmark\\
TAMA \cite{zhuang2024see} & UTS & LinePlot & \cmark & \texttt{GPT4o} & \cmark & \xmark & \cmark & \xmark & Anomaly & General & \xmark\\
\cite{prithyani2024feasibility} & MTS & LinePlot & \cmark & \texttt{LLaVA} & \cmark & \cmark & \cmark & \xmark & Classification & General & \cmark\textsuperscript{\href{https://github.com/vinayp17/VLM_TSC}{[17]}}\\
\bottomrule[1pt]
\end{tabular}}
\vspace{-0.25cm}
\caption{Taxonomy of vision models on time series. The top panel includes single-modal models. The bottom panel includes multi-modal models. {\bf TS-Type} denotes type of time series. {\bf TS-Recover} denotes %whether time series recovery ($\S$\ref{sec.processing}) has been performed.
recovering time series from predicted images ($\S$\ref{sec.processing}). \textsuperscript{*}: %the model has been %applied on MTSs by %processing %modeling the individual UTSs of each MTS.
the method has been used to model the individual UTSs of an MTS. $^{\natural}$: a new pre-trained model was proposed in the work. $^{\flat}$: %without using a pre-trained model, fine-tune means training from scratch.
when pre-trained models were unused, ``Fine-tune'' refers to train a task-specific model from scratch. %In the
{\bf Model} column: \texttt{CNN} could be regular CNN, ResNet, VGG-Net, %U-Net,
{\em etc.}}\label{tab.taxonomy}
% The code only include verified official code from the authors.
\vspace{-0.3cm}
\end{table*}

\begin{table*}[t]
\centering
\small
\setlength{\tabcolsep}{2.9pt}{
\begin{tabular}{l|l|l|l}\hline
% \toprule[1pt]
\rowcolor{gray!20}
{\bf Method} & {\bf TS-Type} & {\bf Advantages} & {\bf Limitations}\\ \hline
Line Plot ($\S$\ref{sec.lineplot}) & UTS, MTS & matches human perception of time series & limited to MTSs with a small number of variates\\ \hline
Heatmap ($\S$\ref{sec.heatmap}) & UTS, MTS & straightforward for both UTSs and MTSs & the order of variates may affect their correlation learning\\ \hline
Spectrogram ($\S$\ref{sec.spectrogram}) & UTS & encodes the time-frequency space & limited to UTSs; needs a proper choice of window/wavelet\\ \hline
GAF ($\S$\ref{sec.gaf}) & UTS & encodes the temporal correlations in a UTS & limited to UTSs; $O(T^{2})$ time and space complexity\\ \hline% for long time series\\ \hline
% RP ($\S$\ref{sec.rp}) & UTS & flexibility in image size by tuning $m$ and $\tau$ & limited to UTSs; the pattern has a threshold-dependency\\ \hline
RP ($\S$\ref{sec.rp}) & UTS & flexibility in image size by tuning $m$ and $\tau$ & limited to UTSs; information loss after thresholding\\ \hline
% \bottomrule[1pt]
\end{tabular}}
\vspace{-0.2cm}
\caption{Summary of the five primary methods for transforming time series to images. {\bf TS-Type} denotes type of time series.}\label{tab.tsimage}
\vspace{-0.2cm}
\end{table*}

\section{Time Series To Image Transformation}\label{sec.tsimage}

% This section summarizes 5 major methods for imaging time series ($\S$\ref{sec.lineplot}-$\S$\ref{sec.rp}). We also discuss some other methods ($\S$\ref{sec.othermethod}) and how to model MTS with these methods ($\S$\ref{sec.modelmts}).
This section summarizes the methods for imaging time series ($\S$\ref{sec.lineplot}-$\S$\ref{sec.othermethod}) and their extensions to encode MTSs ($\S$\ref{sec.modelmts}).

% This section summarizes 5 major methods for transforming time series to images, including Line Plot, Heatmap, Spetrogram, GAF and RP, and several minor methods. We discuss their pros and cons and how to deal with MTS.

% This section discusses the advantages and limitations of different methods for time series to image transformation (invertible, efficiency, information preservation, MTS, long-range time series, parametric, etc.).

%\subsection{Methods}

\vspace{-0.08cm}

\subsection{Line Plot}\label{sec.lineplot}

Line Plot is a straightforward way for visualizing UTSs for human analysis ({\em e.g.}, stocks, power consumption, {\em etc.}). As illustrated by Fig. \ref{fig.tsimage}(a), the simplest approach is to draw a 2D image with x-axis representing %the time horizon
time steps and y-axis representing %the magnitude of the normalized time series.
time-wise values, %A line is used to connect all values of the series over time.
with a line connecting all values of the series over time. This image can be %represented by either three-channel pixels or single-channel pixels
either three-channel ({\em i.e.}, RGB) or single-channel as the colors may not %provide additional information
be informative %\cite{cohen2020trading,sood2021visual,jin2023classification,zhang2023insight,zhuang2024see}.
\cite{cohen2020trading,sood2021visual,jin2023classification,zhang2023insight}. ForCNN \cite{semenoglou2023image} even uses a single 8-bit integer to represent each pixel for black-white images. So far, there is no consensus on whether other graphical components, such as legend, grids and tick labels, could provide extra benefits in any task. For example, ViTST \cite{li2023time} finds these components are superfluous in a classification task, while TAMA \cite{zhuang2024see} finds grid-like auxiliary lines help enhance anomaly detection.

In addition to the regular Line Plot, MV-DTSA \cite{yang2023your} and ViTime \cite{yang2024vitime} divide an image into $h\times L$ grids, %where $h$ is the number of rows and $L$ is the number of columns,
and %introduced
define a function to map each time step of a UTS to a grid, producing a grid-like Line Plot. Also, we include methods that use Scatter Plot \cite{daswani2024plots,prithyani2024feasibility} in this category because %the only difference between a Scatter Plot and a Line Plot is whether the time-wise values are connected by lines.
a Scatter Plot resembles a Line Plot but doesn't connect %time-wise values
data points with a line. By comparing them, \cite{prithyani2024feasibility} finds a Line Plot could induce better time series classification.

For MTSs, we defer the discussion on Line Plot to $\S$\ref{sec.modelmts}.

% For MTS, some methods use the channel-independence assumption proposed in \cite{nie2023time} and represent each variate in MTS with an individual Line Plot \cite{yang2023your,yang2024vitime}. ViTST \cite{li2023time} also uses an individual Line Plot per variate, but colors different lines and assembles all plots to form a bigger image. The method in \cite{wimmer2023leveraging} plots %the time series of
% all variates in a single Line Plot and distinguish them by %use different
% types of lines ({\em e.g.}, solid, dashed, dotted, {\em etc.}). %to distinguish them.
% However, these methods only work for a small number of variates. For example, in \cite{wimmer2023leveraging}, there are only 4 variates in its financial MTSs.

%\cite{li2023time} space-costly because of blank pixels. scatter plot.

%Invertible with a numeric prediction head \cite{sood2021visual}. It fits tasks such as forecasting, imputation, etc.

\vspace{-0.08cm}

\subsection{Heatmap}\label{sec.heatmap}

As shown in Fig. \ref{fig.tsimage}(b), Heatmap is a 2D visualization of the magnitude of the values in a matrix using color. %The variation of color represents the intensity of each value. %Therefore,
It has been used to %directly
represent the matrix of an MTS, {\em i.e.}, $\mat{X} \in \mathbb{R}^{d\times T}$, as a one-channel $d\times T$ image \cite{li2022tts,yazdanbakhsh2019multivariate}. Similarly, TimEHR \cite{karami2024timehr} represents an {\em irregular} MTS, where the intervals between time steps are uneven, as a $d\times H$ Heatmap image by grouping the uneven time steps into $H$ even time bins. In \cite{zeng2021deep}, a different method is used for visualizing a 9-variate financial %time series.
MTS. It reshapes the 9 variates at each time step to a $3\times 3$ Heatmap image, and uses the sequence of images to forecast future %image
frames, achieving %time series
%MTS
time series forecasting. In contrast, VisionTS \cite{chen2024visionts} uses Heatmap to visualize UTSs. %instead.
Similar to TimesNet \cite{wu2023timesnet}, it first segments a length-$T$ UTS into $\lfloor T/P\rfloor$ length-$P$ subsequences, where $P$ is a parameter representing a periodicity of the UTS. Then the subsequences are stacked into a $P\times \lfloor T/P\rfloor$ matrix, %and duplicated 3 times to produce a 3-channel
with 3 duplicated channels, to produce a grayscale image %which serves as an
input to %a vision foundation model.
an LVM. To encode MTSs, VisionTS adopts the channel independence assumption \cite{nie2023time} and individually models each variate in an MTS.

\vspace{0.2cm}

\noindent{\bf Remark.} Heatmap can be used to visualize matrices of various forms. It is also used for matrices generated by the subsequent methods ({\em e.g.}, Spectrogram, GAF, RP) in this section. In this paper, the name Heatmap refers specifically to images that use color to visualize the (normalized) values in UTS $\mat{x}$ or MTS $\mat{X}$ without performing other transformations.

%\cite{chen2024visionts,karami2024timehr} bin version of TSH \cite{karami2024timehr}, DE and STFT \cite{naiman2024utilizing} (DE can be used for constructing RP), rearrange variates for video version of TSH \cite{zeng2021deep}.

%\vspace{0.2cm}

\subsection{Spectrogram}\label{sec.spectrogram}

A {\em spectrogram} is a visual representation of the spectrum of frequencies of a signal as it varies with time, which are extensively used for analyzing audio signals \cite{gong2021ast}. Since audio signals are a type of UTS, spectrogram can be considered as a method for imaging a UTS. As shown in Fig. \ref{fig.tsimage}(c), a common format is a 2D heatmap image with x-axis representing time steps and y-axis representing frequency, {\em a.k.a.} a time-frequency space. %The color at each point
Each pixel in the image represents the (logarithmic) amplitude of a specific frequency at a specific time point. Typical methods for %transforming a UTS to
producing a spectrogram include {\bf Short-Time Fourier Transform (STFT)} \cite{griffin1984signal}, {\bf Wavelet Transform} \cite{daubechies1990wavelet}, and {\bf Filterbank} \cite{vetterli1992wavelets}.

\vspace{0.2cm}

\noindent{\bf STFT.} %Discrete Fourier transform (DFT) can be used to represent a UTS signal %$\mat{x}=[x_{1}, ..., x_{T}]$
%$\mat{x}\in\mathbb{R}^{1\times T}$ as a sum of sinusoidal components. The output of the transform is a function of frequency $f(w)$, describing the intensity of each constituent frequency $w$ of the entire UTS. 
Discrete Fourier transform (DFT) can be used to describe the intensity $f(w)$ of each constituent frequency $w$ of a UTS signal $\mat{x}\in\mathbb{R}^{1\times T}$. However, $f(w)$ has no time dependency. It cannot provide dynamic information such as when a specific frequency appear in the UTS. STFT addresses this deficiency by sliding a window function $g(t)$ over the time steps in %the UTS,
$\mat{x}$, and computing the DFT within each window by
\begin{equation}\label{eq.stft}
\small
\begin{aligned}
f(w,\tau) = \sum_{t=1}^{T}x_{t}g(t - \tau)e^{-iwt}
\end{aligned}
\end{equation}
where $w$ is frequency, $\tau$ is the position of the window, $f(w,\tau)$ describes the intensity of frequency $w$ at time step $\tau$.

%With a proper selection of the
By selecting a proper window function $g(\cdot)$ ({\em e.g.}, Gaussian/Hamming/Bartlett window), %({\em e.g.}, Gaussian window, Hamming window, Bartlett window), %{\em etc.}),
a 2D spectrogram ({\em e.g.}, Fig. \ref{fig.tsimage}(c)) can be drawn via a heatmap on the squared values $|f(w,\tau)|^{2}$, with $w$ as the y-axis, and $\tau$ as the x-axis. For example, \cite{dixit2024vision} uses STFT based spectrogram as an input to LMMs %\hh{do you mean LVMs? check}
for time series classification.

%Fourier transform is a powerful data analysis tool that represents any complex signal as a sum of sines and cosines and transforms the signal from the time domain to the frequency domain. However, Fourier transform can only show which frequencies are present in the signal, but not when these frequencies appear. The STFT divides original signal into several parts using a sliding window to fix this problem. STFT involves a sliding window for extracting frequency components within the window.

\vspace{0.2cm}

\noindent{\bf Wavelet Transform.} %Like Fourier transform, %\hh{this paragraph needs a citation}
Continuous Wavelet Transform (CWT) uses the inner product to measure the similarity between a signal function $x(t)$ and an analyzing function. %In STFT (Eq.~\eqref{eq.stft}), the analyzing function is a windowed exponential $g(t - \tau)e^{-iwt}$.
%In CWT,
The analyzing function is a {\em wavelet} $\psi(t)$, where the typical choices include Morse wavelet, Morlet wavelet, %Daubechies wavelet, %Beylkin wavelet, 
{\em etc.} %The
CWT compares $x(t)$ to the shifted and scaled ({\em i.e.}, stretched or shrunk) versions of the wavelet, and output a CWT coefficient by
\begin{equation}\label{eq.cwt}
\small
\begin{aligned}
c(s,\tau) = \int_{-\infty}^{\infty}x(t)\frac{1}{s}\psi^{*}(\frac{t - \tau}{s})dt
\end{aligned}
\end{equation}
where $*$ denotes complex conjugate, $\tau$ is the time step to shift, and $s$ represents the scale. In practice, a discretized version of CWT in Eq.~\eqref{eq.cwt} is implemented for UTS $[x_{1}, ..., x_{T}]$.

It is noteworthy that the scale $s$ controls the frequency encoded in a wavelet -- a larger $s$ leads to a stretched wavelet with a lower frequency, and vice versa. As such, by varying $s$ and $\tau$, a 2D spectrogram ({\em e.g.}, Fig. \ref{fig.tsimage}(d)) can be drawn %, often with a heatmap
on $|c(s,\tau)|$, where $s$ is the y-axis and $\tau$ is the x-axis. Compared to STFT, which uses a fixed window size, Wavelet Transform allows variable wavelet sizes -- a larger size %region
for more precise low frequency information. 
%Usually, $s$ and $\tau$ vary dependently -- a larger $s$ leads to a stretched wavelet that shifts slowly, {\em i.e.}, a smaller $\tau$. This property %of CWT
%yields a spectrogram that balances the resolutions of frequency %$s$
%and time, %$\tau$,
%which is an advantage over the fixed time resolution in STFT.
% Thus, both of the methods in %\cite{du2020image}
% \cite{namura2024training} and \cite{zeng2023pixels} choose CWT (with Morlet wavelet) to generate the spectrogram.
Thus, the methods in \cite{du2020image,namura2024training,zeng2023pixels} choose CWT (with Morlet wavelet) to generate the spectrogram.

%A wavelet is a wave-like oscillation that has zero mean and is localized in both time and frequency space.

\vspace{0.2cm}

\noindent{\bf Filterbank.} This method %is relevant to
resembles STFT and is often used in processing audio signals. Given an audio signal, it firstly goes through a {\em pre-emphasis filter} to boost high frequencies, which helps improve the clarity of the signal. Then, STFT is applied on the signal. %with a sliding window $g(t)$ of size $k$ that shifts in a fixed stride $\tau$. %where the adjacent windows may overlap in $k$ time length.
%Finally, filterbank features are computed by applying multiple ``triangle-shaped'' filters spaced on the Mel-scale to the STFT output $f(w, \tau)$. %where Mel-scale is a method to make the filters more discriminative on lower frequencies, %than higher frequencies,
%imitating the non-linear human ear perception of sound.
Finally, multiple ``triangle-shaped'' filters spaced on a Mel-scale are applied to the STFT power spectrum $|f(w, \tau)|^{2}$ to extract frequency bands. The outcome filterbank features $\hat{f}(w, \tau)$ can be used to yield a spectrogram with $w$ as the y-axis, and $\tau$ as the x-axis.

%Filterbank was introduced in AST \cite{gong2021ast} with %$k$=25ms
Filterbank was adopted in AST \cite{gong2021ast} with 
a 25ms Hamming window that shifts every 10ms for classifying audio signals using Vision Transformer (ViT). It then becomes widely used in the follow-up works such as SSAST \cite{gong2022ssast}, MAE-AST \cite{baade2022mae}, and AST-SED \cite{li2023ast}, as summarized in Table \ref{tab.taxonomy}.



%Use MLP to predict TS directly \cite{zeng2023pixels}.

%\vspace{0.2cm}

% \vspace{0.2cm}

\subsection{Gramian Angular Field (GAF)}\label{sec.gaf}

GAF was introduced for classifying UTSs using CNNs %using %image based CNNs
by \cite{wang2015encoding}. It was then extended %with an extension
to an imputation task in \cite{wang2015imaging}. Similarly, \cite{barra2020deep} applied GAF for financial time series forecasting.

Given a UTS $\mat{x}\in\mathbb{R}^{1\times T}$, %$[x_{1}, ..., x_{T}]$,
the first step %before GAF
is to rescale each $x_{t}$ to a value $\tilde{x}_{t}$ %in the interval of
within $[0, 1]$ (or $[-1, 1]$). %by min-max normalization.
This range enables mapping $\tilde{x}_{t}$ to polar coordinates by $\phi_{t}=\text{arccos}(\tilde{x}_{i})$, with a radius $r=t/N$ encoding the time stamp, where $N$ is a constant factor to regularize the span of the polar coordinates. %system. Then,
Two types of GAF, Gramian Sum Angular Field (GASF) and Gramian Difference Angular Field (GADF) are defined as
\begin{equation}\label{eq.gaf}
\small
\begin{aligned}
&\text{GASF:}~~\text{cos}(\phi_{t} + \phi_{t'})=x_{t}x_{t'} - \sqrt{1 - x_{t}^{2}}\sqrt{1 - x_{t'}^{2}}\\
&\text{GADF:}~~\text{sin}(\phi_{t} - \phi_{t'})=x_{t'}\sqrt{1 - x_{t}^{2}} - x_{t}\sqrt{1 - x_{t'}^{2}}
\end{aligned}
\end{equation}
which exploits the pairwise temporal correlations in the UTS. Thus, the outcome is a $T\times T$ matrix $\mat{G}$ with $\mat{G}_{t,t'}$ specified by either type in Eq.~\eqref{eq.gaf}. A GAF image is a heatmap on $\mat{G}$ with both axes representing time, as illustrated by Fig. \ref{fig.tsimage}(e).

% Invertible.

% \vspace{0.2cm}

\subsection{Recurrence Plot (RP)}\label{sec.rp}

%RP \cite{eckmann1987recurrence} is a method to encode a UTS into an image that aims to capture the periodic patterns in the UTS by using its reconstructed {\em phase space}. The phase space of a UTS $[x_{1}, ..., x_{T}]$ can be reconstructed by {\em time delay embedding}, which is a set of new vectors $\mat{v}_{1}$, ..., $\mat{v}_{l}$ with

RP \cite{eckmann1987recurrence} encodes a UTS into an image that captures its periodic patterns by using its reconstructed {\em phase space}. The phase space of %a UTS %$[x_{1}, ..., x_{T}]$
$\mat{x}\in\mathbb{R}^{1\times T}$ can be reconstructed by {\em time delay embedding} -- a set of new vectors $\mat{v}_{1}$, ..., $\mat{v}_{l}$ with
\begin{equation}\label{eq.de}
\small
\begin{aligned}
\mat{v}_{t}=[x_{t}, x_{t+\tau}, x_{t+2\tau}, ..., x_{t+(m-1)\tau}]\in\mathbb{R}^{m\tau},~~~1\le t \le l
\end{aligned}
\end{equation}
where $\tau$ is the time delay, $m$ is the dimension of the phase space, both %of which
are hyperparameters. Hence, $l=T-(m-1)\tau$. With vectors $\mat{v}_{1}$, ..., $\mat{v}_{l}$, an RP image %is constructed by measuring
measures their pairwise distances, results in an $l\times l$ image whose element
\begin{equation}\label{eq.rp}
\small
\begin{aligned}
\text{RP}_{i,j}=\Theta(\varepsilon - \|\mat{v}_{i} - \mat{v}_{j}\|),~~~1\le i,j\le l
\end{aligned}
\end{equation}
where $\Theta(\cdot)$ is the Heaviside step function, $\varepsilon$ is a threshold, and $\|\cdot\|$ is a norm function such as $\ell_{2}$ norm. Eq.~\eqref{eq.rp} %states RP produces a heatmap image on a binary matrix with $\text{RP}_{i,j}=1$ if $\mat{v}_{i}$ and $\mat{v}_{j}$ are sufficiently similar.
generates a binary matrix with $\text{RP}_{i,j}=1$ if $\mat{v}_{i}$ and $\mat{v}_{j}$ are sufficiently similar, producing a black-white image ({\em e.g.}, Fig. \ref{fig.tsimage}(f)).% ({\em e.g.}, a periodic pattern).

An advantage of RP is its flexibility in image size by tuning $m$ and $\tau$. Thus it has been used for time series classification %\cite{cao2021image},
\cite{silva2013time,hatami2018classification}, forecasting \cite{li2020forecasting}, anomaly detection \cite{lin2024hierarchical} and %feature-wise
explanation \cite{kim2024cafo}. Moreover, the method in \cite{hatami2018classification}, and similarly in HCR-AdaAD \cite{lin2024hierarchical}, omit the thresholding in Eq.~\eqref{eq.rp} and uses $\|\mat{v}_{i} - \mat{v}_{j}\|$ to produce continuously valued images %in a classification task
to avoid information loss.


% \vspace{0.2cm}

\subsection{Other Methods}\label{sec.othermethod}

%There are some less commonly used methods. For example, in
Additionally, %there are some peripheral methods. %In addition to GAF,
\cite{wang2015encoding} introduces Markov Transition Field (MTF) for imaging a UTS. %$\mat{x}\in\mathbb{R}^{1\times T}$. 
%MTF first assigns each $x_{t}$ to one of $Q$ quantile bins, then builds a $Q\times Q$ Markov transition matrix $\mat{M}$ {\em s.t.} $\mat{M}_{i,j}$ represents the frequency %with which
%of the case when a point $x_{t}$ in the $i$-th bin is followed by a point $x_{t'}$ in the $j$-th bin, {\em i.e.}, $t=t'+1$. Matrix $\mat{M}$ serves as the input of a heatmap image.
MTF is a matrix $\mat{M}\in\mathbb{R}^{Q\times Q}$ encoding the transition probabilities over time segments, where $Q$ is the number of segments. %Moreover,
ImagenTime \cite{naiman2024utilizing} stacks the delay embeddings $\mat{v}_{1}$, ..., $\mat{v}_{l}$ in Eq.~\eqref{eq.de} to an $l\times m\tau$ matrix for visualizing UTSs. %It also uses a variant of STFT.
% The method in \cite{homenda2024time} introduces five different 2D images by counting, rearranging, replicating the values in a UTS. 
MSCRED \cite{zhang2019deep} uses heatmaps on the $d\times d$ correlation matrices of MTSs with $d$ variates for anomaly detection. 
Furthermore, some methods use a mixture of imaging methods by stacking different transformations. \cite{wang2015imaging} stacks GASF, GADF, MTF to a 3-channel image. %Similarly,
FIRTS \cite{costa2024fusion} builds a 3-channel image by stacking GASF, MTF and RP. %the GASF, MTF, RP representations of each UTS.
%\cite{jin2023classification} combines Line Plot with Constant-Q Transform (CQT) \cite{brown1991calculation}, a method related to wavelet transform ($\S$\ref{sec.spectrogram}), to generate 2-channel images.
The mixture methods encode a UTS with multiple views and were found more robust than single-view images in these works for %time series
classification tasks.

\subsection{How to Model MTS}\label{sec.modelmts}

In the above methods, Heatmap ($\S$\ref{sec.heatmap}) can be %directly
used to visualize the %2D
variate-time matrices, $\mat{X}$, of MTSs ({\em e.g.}, Fig. \ref{fig.structure}(b)), where correlated variates %are better to
should be spatially close to each other. Line Plot ($\S$\ref{sec.lineplot}) can be used to visualize MTSs by plotting all variates in the same image \cite{wimmer2023leveraging,daswani2024plots} or combining all univariate images to compose a bigger %1-channel
image \cite {li2023time}, but these methods only work for a small number of variates. Spectrogram ($\S$\ref{sec.spectrogram}), GAF ($\S$\ref{sec.gaf}), and RP ($\S$\ref{sec.rp}) were designed specifically for UTSs. For these methods and Line Plot, which are not straightforward %for MTS transformation,
in imaging MTSs, the general approaches %to use them %for MTS
include using channel independence assumption to model each variate individually \cite{nie2023time}, %like VisionTS \cite{chen2024visionts},
or stacking the images of $d$ variates to form a $d$-channel image %as did by
\cite{naiman2024utilizing,kim2024cafo}. %\cite{prithyani2024feasibility,naiman2024utilizing,kim2024cafo}.
However, the latter does not fit some vision models pre-trained on RGB images which requires 3-channel inputs (more discussions are deferred to $\S$\ref{sec.processing}).

\vspace{0.2cm}

\noindent{\bf Remark.} As a summary, Table \ref{tab.tsimage} recaps the salient advantages and limitations of the five primary imaging methods that are introduced in this section.

% \hh{can we have a table (e.g., rows are different imaging methods and columns are a few desirable propoerties) or a short paragraph to discuss/summarize/compare the strenths and weakness of different imaging methods for ts? This might bring some structure/comprehension to this section (as opposed to, e.g., some reviewer might complain that what we do here is a laundry list)}

\section{Imaged Time Series Modeling}\label{sec.model}

With image representations, time series analysis can be readily performed with vision models. This section discusses such solutions from %traditional vision models %($\S$\ref{sec.cnns})
%to the recent large vision models %($\S$\ref{sec.lvms})
%and large multimodal models.% ($\S$\ref{sec.lmms}).
the traditional models to the SOTA models.

\begin{figure*}[!t]
\centering
\includegraphics[width=0.9\textwidth]{fig/fig_2.pdf}
% \vspace{-1em}
\caption{An illustration of different modeling strategies on imaged time series in (a)(b)(c) and task-specific heads in (d).}\label{fig.models}
\vspace{-0.2cm}
\end{figure*}

\subsection{Conventional Vision Models}\label{sec.cnns}

%Similar to
Following traditional %methods on
image classification, \cite{silva2013time} applies a K-NN classifier on the RPs of time series, \cite{cohen2020trading} applies an ensemble of fundamental classifiers such as %linear regression, SVM, Ada Boost, {\em etc.}
SVM and AdaBoost on the Line Plots %images
for time series classification. As an image encoder, %a typical encoder, %of images,
CNNs have been %extensively
widely used for learning image representations. %\cite{he2016deep}.
Different from using 1D CNNs on sequences %UTS or MTS
\cite{bai2018empirical}, %regular
2D or 3D CNNs can be applied on imaged time series as shown in Fig. \ref{fig.models}(a). %to learn time series representations by encoding their image transformations.
For example, %standard
regular CNNs have been used on Spectrograms \cite{du2020image}, tiled CNNs have been used on GAF images \cite{wang2015encoding,wang2015imaging}, dilated CNNs have been used on Heatmap images \cite{yazdanbakhsh2019multivariate}. More frequently, ResNet \cite{he2016deep}, Inception-v1 \cite{szegedy2015going}, and VGG-Net \cite{simonyan2014very} have been used on Line Plots \cite{jin2023classification,semenoglou2023image}, Heatmap images \cite{zeng2021deep}, RP images \cite{li2020forecasting,kim2024cafo}, GAF images \cite{barra2020deep,kaewrakmuk2024multi}, 
% Heatmaps \cite{zeng2021deep}, RPs \cite{li2020forecasting,kim2024cafo}, GAFs \cite{barra2020deep,kaewrakmuk2024multi},
and even a mixture of GAF, MTF and RP images \cite{costa2024fusion}. In particular, for time series generation tasks, %a diffusion model with U-Nets \cite{naiman2024utilizing} and GAN frameworks of CNNs \cite{li2022tts,karami2024timehr} have also been explored.%investigated.
GAN frameworks of CNNs \cite{li2022tts,karami2024timehr} and a diffusion model with U-Nets \cite{naiman2024utilizing} have also been explored.

Due to their small to medium sizes, these models are often trained from scratch using task-specific training data. %per task using the task's training set. %of time series images.
Meanwhile, fine-tuning {\em pre-trained vision models}  %such as those pre-trained on ImageNet, %\cite{deng2009imagenet}, 
have already been found promising in cross-modality knowledge transfer for time series anomaly detection \cite{namura2024training}, forecasting \cite{li2020forecasting} and classification \cite{jin2023classification}.

% \cite{li2020forecasting} uses ImageNet pretrained CNNs.

\subsection{Large Vision Models (LVMs)}\label{sec.lvms}

Vision Transformer (ViT) \cite{dosovitskiy2021image} has %given birth to
inspired the development of %some
modern LVMs %large vision models (LVMs)
such as %DeiT \cite{touvron2021training}, 
Swin \cite{liu2021swin}, BEiT \cite{bao2022beit}, and MAE \cite{he2022masked}. %Given an input image, ViT splits it
As Fig. \ref{fig.models}(b) shows, ViT splits an %input
image into {\em patches} of fixed size, then embeds each patch and augments it with a positional embedding. The %resulting
vectors of patches are processed by a Transformer %encoder
as if they were token embeddings. Compared to CNNs, ViTs are less data-efficient, but have higher capacity. %Consequently,
Thus, %the
{\em pre-trained} ViTs have been explored for modeling %the images of time series.
imaged time series. For example, AST \cite{gong2021ast} fine-tunes DeiT \cite{touvron2021training} on the filterbank spetrogram of audios %signals
for classification tasks and finds %using
ImageNet-pretrained DeiT is remarkably effective in knowledge transfer. The fine-tuning paradigm has also been %similarly
adopted in \cite{zeng2023pixels,li2023time} but with different pre-trained models %initializations
such as Swin by \cite{li2023time}. 
VisionTS \cite{chen2024visionts} %explains
attributes %the superiority of LVMs
LVMs' superiority over LLMs in knowledge transfer %over LLMs %as an outcome of
to the small gap between the pre-trained images and imaged time series. %the patterns learned from the large-scale pre-trained images and the patterns in the images of time series.
It %also
finds that with one-epoch fine-tuning, MAE becomes the SOTA time series forecasters on %many
some benchmark datasets.

Similar to %build
time series foundation models %\cite{das2024decoder,goswami2024moment,ansari2024chronos,shi2024time}, %such as TimesFM \cite{das2024decoder}, MOMENT \cite{goswami2024moment}, Chronos \cite{ansari2024chronos} and Time-MoE \cite{shi2024time},
such as TimesFM \cite{das2024decoder}, %and MOMENT \cite{goswami2024moment}, 
there are some initial efforts in pre-training ViT architectures with imaged time series. Following AST, SSAST \cite{gong2022ssast} introduced a %joint discriminative and generative
%masked spectrogram patch prediction self-supervised learning framework
masked spectrogram patch prediction framework for pre-training ViT on a large dataset -- AudioSet-2M. Then it becomes a backbone of some follow-up works such as AST-SED \cite{li2023ast} for sound event detection. %To be effective for UTSs,
For UTSs, ViTime \cite{yang2024vitime} generates a large set of Line Plots of synthetic UTSs for pre-training ViT, which was found superior over TimesFM in zero-shot forecasting tasks on benchmark datasets.

\subsection{Large Multimodal Models (LMMs)}\label{sec.lmms}

%As Large Multimodal Models (LMMs)
As LMMs %are getting
get growing attentions, some %of the
notable LMMs, such as LLaVA \cite{liu2023visual}, Gemini \cite{team2023gemini}, GPT-4o \cite{achiam2023gpt} and Claude-3 \cite{anthropic2024claude}, have been explored to consolidate the power of LLMs %on time series
and LVMs in time series analysis. 
Since LMMs support multimodal input via prompts, methods in this thread typically prompt LMMs with the textual and imaged representations of time series, %textual representation of time series and their %image transformations, transformed images,
%then instruct LMMs
and instructions on what tasks to perform ({\em e.g.}, Fig. \ref{fig.models}(c)).

InsightMiner \cite{zhang2023insight} is a pioneer work that uses the LLaVA architecture to generate %textual descriptions about
texts describing the trend of each input UTS. It extracts the trend of a UTS by Seasonal-Trend decomposition, encodes the Line Plot of the trend, and concatenates the embedding of the Line Plot with the embeddings of a textual instruction, which includes a sequence of numbers representing the UTS, {\em e.g.}, ``[1.1, 1.7, ..., 0.3]''. The concatenated embeddings are taken by a language model for generating trend descriptions. %It also fine-tunes a few layers with the generated texts to align LLaVA checkpoints with time series domain.
Similarly, \cite{prithyani2024feasibility} adopts the LLaVA architecture, but for MTS classification. An MTS is encoded by %a sequence of
the visual %token
embeddings of the stacked Line Plots of all variates. %meanwhile
%The method also stacks
%The time series of all variate are also stacked in a prompt % of all variates in a prompt
The matrix of the MTS is also verbalized in a prompt 
as the textual modality. %By manipulating token embeddings,
By integrating token embeddings, both %of these %works propose to
methods fine-tune some layers of the LMMs with some synthetic data.

Moreover, zero-shot and in-context learning performance of several commercial LMMs have been evaluated for audio classification \cite{dixit2024vision}, anomaly detection \cite{zhuang2024see}, and some synthetic tasks \cite{daswani2024plots}, where the image %({\em e.g.}, spectrograms, Line Plots)
and textual representations of a query %UTS or MTS
time series are integrated into a prompt. For in-context learning, these methods inject the images of a few example time series and their labels ({\em e.g.}, classes) %({\em e.g.}, classes, normal status)
into an instruction to prompt LMMs for assisting the prediction of the query time series.

\subsection{Task-Specific Heads}\label{sec.task}

%With the image embedding of a time series, the next step is to produce its prediction.
For classification tasks, most of the methods in Table \ref{tab.taxonomy} adopt a fully connected (FC) layer or multilayer perceptron (MLP) to transform an embedding into a probability distribution over all classes. For forecasting tasks, there are two approaches: (1) using a $d_{e}\times W$ MLP/FC layer to directly predict (from the $d_{e}$-dimensional embedding) the time series values in a future time window of size $W$ \cite{li2020forecasting,semenoglou2023image}; (2) predicting the pixel values that represent the future part of the time series and then recovering the time series from the predicted image \cite{yang2023your,chen2024visionts,yang2024vitime} ($\S$\ref{sec.processing} discusses the recovery methods). Imputation and generation tasks resemble forecasting %in the sense of predicting
as they also predict time series values. Thus approach (2) has been used for imputation \cite{wang2015imaging} and generation \cite{naiman2024utilizing,karami2024timehr}. %LMMs have been used for classification, text generation, and anomaly detection. For these tasks,
When using LMMs for classification, text generation, and anomaly detection, most of the methods prompt LMMs to produce the desired outputs in textual answers, circumventing task-specific heads \cite{zhang2023insight,dixit2024vision,zhuang2024see}.

%Forecasting: MLP, FC to predict numerical values using embeddings. Imputation of images (TSH). Classification: MLP, FC using embeddings.

\section{Pre-Processing and Post-Processing}\label{sec.processing}

To be successful in using vision models, some subtle design desiderata %to be considered
include {\bf time series normalization}, {\bf image alignment} and {\bf time series recovery}.

\vspace{0.2cm}

\noindent{\bf Time Series Normalization.} Vision models are usually trained on %images after Gaussian normalization (GN).
standardized images. To be aligned, the images introduced in $\S$\ref{sec.tsimage} should be normalized with a controlled mean and standard deviation, as did by \cite{gong2021ast} on spectrograms. In particular, as Heatmap is built on raw time series values, the commonly used Instance Normalization (IN) \cite{kim2022reversible} can be applied on the time series as suggested by VisionTS \cite{chen2024visionts} since IN share similar merits as Standardization. %although min-max normalization was used by \cite{karami2024timehr,zeng2021deep}.
Using Line Plot requires a proper range of y-axis. In addition to rescaling time series %by min-max or GN
\cite{zhuang2024see}, ViTST \cite{li2023time} introduced several methods to remove extreme values from the plot. GAF requires min-max normalization on its input, as it transforms time series values withtin $[0, 1]$ to polar coordinates ({\em i.e.}, arccos). In contrast, input to RP is usually normalization-free as an $\ell_{2}$ norm is involved in Eq.~\eqref{eq.rp} before thresholding.%for a comparison with a threshold.

\vspace{0.2cm}

\noindent{\bf Image Alignment.} When using pre-trained models, it is imperative to fit the image size to the input requirement of the models. This is especially true for Transformer based models as they use a fixed number of positional embeddings to encode the spacial information of image patches. For 3-channel RGB images such as Line Plot, it is straightforward to meet a pre-defined size by adjusting the resolution when producing the image. For images built upon matrices such as Heatmap, Spectrogram, GAF, RP, the number of channels and matrix size need adjustment. For the channels, one method is to duplicate a matrix to 3 channels \cite{chen2024visionts}, another way is to average the weights of the 3-channel patch embedding layer into a 1-channel layer \cite{gong2021ast}. For the image size, bilinear interpolation is a common method to resize input images \cite{chen2024visionts}. Alternatively, AST \cite{gong2021ast} %use cut and bilinear interpolation on
resizes the positional embeddings instead of the images to fit the model to a desired input size. However, the interpolation in these methods may either alter the time series or the spacial information in positional embeddings.

% single-channel (UTS), RGB channel (UTS), duplicate channels (UTS), multi-channel (MTS).

%Bilinear interpolation.

%Correlated variates are better to be spatially close to each other.

%\subsection{Pre-training}

\vspace{0.2cm}

\noindent{\bf Time Series Recovery.} As stated in $\S$\ref{sec.task}, tasks such as forecasting, imputation and generation requires predicting time series values. For models that predict pixel values of images, post-processing involves recovering time series from the predicted images. Recovery from Line Plots is tricky, it requires locating pixels that %correspond to
represent time series and mapping them back to the original values. This can be done by manipulating a grid-like Line Plot as introduced in \cite{yang2023your,yang2024vitime}, which has a recovery function. In contrast, recovery from Heatmap is straightforward as it directly stores the predicted time series values \cite{zeng2021deep,chen2024visionts}. Spectrogram is underexplored in these tasks and it remains open on how to recover time series from it. The existing work \cite{zeng2023pixels} uses Spectrogram for forecasting only with an MLP head that directly predicts time series. %predicts time series values.
GAF supports accurate recovery by an inverse mapping from polar coordinates to normalized time series \cite{wang2015imaging}. However, RP lost time series information during thresholding (Eq.~\ref{eq.rp}), thus may not fit recovery-demanded tasks without using an {\em ad-hoc} prediction head.


% Line Plot was regarded as matrices with rows and columns for mapping in \cite{sood2021visual}.


%\section{Tasks and Time Series Recovery}

%\subsection{Task-Specific Head}

% \subsection{Time Series Recovery}






% \begin{figure}[t]
%     \centering
%     \includegraphics[width=0.8\linewidth]{Figures/Teaser_Figure_again.pdf}
%     \caption{Graph showing the number of publications and corresponding Github stars in 3DGS from July 2023.%\enzo{graphically ugly... looks like an excel screenshot}
%     }
%     \label{fig:teaser}
% \end{figure}


% In this context, 3D Gaussian Splatting (3DGS)~\cite{kerbl20233d} presents a novel approach to scene representation and rendering. 

Although NeRFs are adept at producing photorealistic images, there is a growing need for faster and more efficient rendering techniques, especially for applications where low latency is essential. 3D Gaussian Splatting (3DGS) solves this problem by employing millions of learnable 3D Gaussians in space for explicit scene representation in scene modeling. 3DGS employs an explicit representation and a highly parallelized rasterization approach, which facilitates more efficient computation and rendering, unlike implicit coordinate-based models~\cite{henzler2019escaping,sitzmann2019deepvoxels,mildenhall2020nerf}. The innovation of 3DGS lies in its integration of differentiable pipelines and point-based rendering techniques~\cite{pfister2000surfels,zwicker2001surface,ren2002object,botsch2005high}. Modeling the scenes with learnable 3D Gaussians preserves the robust overfitting capabilities of continuous volumetric radiance fields required for high-quality image synthesis. Simultaneously, it circumvents the computational complexity of NeRF-based methods, such as the computationally expensive ray marching process and redundant calculations in unoccupied space~\cite{10757420}.


%

\begin{figure*}[ht] %[htbp]   
\centering
\resizebox*{0.85\textwidth}{0.66\textheight}{%
\begin{forest}
for tree={
    grow=east,
    font=\large,
    parent anchor=east,
    anchor=west,
    edge={->, rounded corners},
    edge path={
        \noexpand\path [draw, \forestoption{edge}] (!u.parent anchor) -- ++(3mm,0) |- (.child anchor)\forestoption{edge label};
    },
    inner sep=1mm,
    text width=4cm,
    l sep=10mm,
    text centered,
    align=center,
    rounded corners, % Applies rounded corners to all nodes
    draw=black % Draw border around nodes
},% Balancing children symmetrically
for children={%
    tier/.wrap pgfmath arg={level#1}{level()}, % Ensure nodes align symmetrically
    s sep+=5mm, % Add space between siblings
},
[Survey of IR Model Architectures, rectangle, fill=violet!20,rotate=90,parent anchor=south,text width=9cm, align=c,
    [Emerging Directions\\and Challenges~\cref{sec:future_direction}, rectangle, fill=gray!20, text width=6cm,
        [Open Questions, text width=9cm,fill=gray!20,],
        [Flexible Relevance Estimators, text width=9cm, fill=gray!20],
        [Better Models for Feature Extraction, text width=9cm,fill=gray!20,]
    ],
    [Large Language Models for \\Information Retrieval~\cref{sec:llm4ir}, rectangle, fill=cyan!20, text width=6cm,
        [Generative Retrieval, text width=9cm,fill=cyan!20,],
        [LLM as Reranker, text width=9cm,fill=cyan!20,],
        [LLM as Retriever, text width=9cm,fill=cyan!20,]
    ],
    [IR Model Architecture with\\Pre-trained Transformers~\cref{sec:transformer}, rectangle, fill=pink!25,  text width=6cm,
        [Multi-vector Representations, text width=9cm, fill=pink!25],
        [Learned Sparse Retrieval, text width=9cm, fill=pink!25],        
        [Learned Dense Retrieval, text width=9cm, fill=pink!25],
        [Text Reranking Models, text width=9cm, fill=pink!25],
    ],
    [Pre-\textsc{BERT} Neural \\Ranking Models~\cref{sec:neural_ranking}, rectangle, fill=blue!10,  text width=6cm,
        % [Attention-based Neural Ranking Model, text width=9cm, fill=blue!10],
        [Interaction-based Neural Ranking Model, text width=9cm, fill=blue!10],        
        [Representation-based Neural Ranking Model, text width=9cm, fill=blue!10],
    ],
    [Learning-to-rank Model \\Architectures~\cref{sec:ltr}, rectangle, fill=yellow!50,  text width=6cm,
        [Neural LTR Model, text width=9cm, fill=yellow!50],
        [Machine Learning-based LTR Model, text width=9cm, fill=yellow!50], 
    ],
    [Traditional IR Models~\cref{sec:traditional}, rectangle, fill=orange!30,  text width=6cm,
        [Statistic Language Model, text width=9cm, fill=orange!30],
        [Probabilistic Model, text width=9cm, fill=orange!30],        
        [Vector Space Model, text width=9cm, fill=orange!30],
        [Boolean Model, text width=9cm, fill=orange!30],
    ]
]
\end{forest}
}
\caption{An overview of this survey. We focus on representative lines of works and defer details to the Appendix.}
\label{fig:structure}
\end{figure*}





The introduction of 3DGS signifies more than just a technical leap forward; it represents a fundamental change in the approach to scene representation and rendering within computer vision and graphics. By facilitating real-time rendering capabilities while maintaining high visual fidelity, 3DGS paves the way for several applications ranging from virtual reality and augmented reality to real-time cinematic rendering and beyond~\cite{jiang2024vr}. This advancement promises to not only enhance current applications but also unlock new ones that were previously hindered by computational limitations~\cite{liu2024georgs,10879794}. Furthermore, the explicit scene representation provided by 3DGS offers unparalleled flexibility in managing objects and scene dynamics, which is essential to handle complex scenarios with intricate geometries and diverse lighting conditions~\cite{chabra2020deep, wang2021learning}. 
This high degree of editability combined with the efficiency of both the training and rendering processes enables 3DGS to have a deep impact on future advancements in different domains~\cite{10900457}. As a relatively recent emergence, within less than a year, the multitude of works on 3DGS underscores its broad applicability across diverse domains such as robotics~\cite{yan2024gs,keetha2024splatam,matsuki2024gaussian,yugay2023gaussian,huang2024photo}, avatars~\cite{li2024animatable,hu2024gauhuman,lei2024gart,yuan2024gavatar,hu2025tgavatar}, endoscopic scene reconstruction~\cite{huang2024endo,liu2024endogaussian,zhao2024hfgs,wang2024endogslam}, and physics~\cite{xie2024physgaussian,liu2024physics3d,borycki2024gasp,huang2024dreamphysics,zhang2024physdreamer}. 

Compared with NeRFs, 3DGS has the advantage of faster rendering speed but at the cost of higher demand of memory with the need to store millions of Gaussians. This limits their application in resource-constrained devices, like VR/AR or game environments. Therefore, there has been a notable increase in research activities focused on compressing 3DGS scenes. With the development of numerous novel compression methods, these efforts have resulted in significant advancements in compression ratios. Consequently, there is a pressing need for a timely review and summary of these representative methods for 3DGS compression. Such a review would help researchers grasp the overall landscape of 3DGS compression, providing a comprehensive and structured overview of current achievements and major challenges in the field.

\begin{figure*}[t]
    \centering
    \includegraphics[width=0.9\linewidth]{Figures/working.pdf}
    \caption{An overview of the 3DGS forward process. (a) The splatting step maps 3D Gaussians into the image plane. (b) 3DGS partitions the image into non-overlapping patches, referred to as tiles. (c) For Gaussians spanning multiple tiles, 3DGS duplicates them and assigns each a unique identifier, i.e., a tile ID. (d) Rendering the sorted Gaussians yields the pixel data for each tile. Note that the pixel and tile computation workflows are independent, enabling parallel processing. Best viewed in color. Figure inspired from~\cite{DBLP:journals/corr/abs-2401-03890}.}
    \label{fig:working}
\end{figure*}


\noindent
\textbf{Scope:} This survey will explore the compression-specific design of the 3DGS architecture, covering various modules including but not limited to densification of Gaussians, pruning, vector quantization, scalar quantization, Gaussian structure, and point cloud compression.

\noindent
\textbf{Related Surveys: }A few 3DGS surveys exist~\cite{DBLP:journals/corr/abs-2401-03890,fei20243d, bao20243d,wu2024recent}. However, to the best of our knowledge, only one survey focuses on the compression of 3DGS. 3DGS.zip~\cite{bagdasarian20243dgs} provides an overview of existing compression techniques, with its primary contribution being the establishment of a unified evaluation standard.  In contrast, our work systematically categorizes different compression methods within a proposed taxonomy (Table~\ref{tab:Taxonomy}, Figure~\ref{fig:taxonomy}) of structured and unstructured approaches, analyzing their distinctions and associated challenges. Additionally, we offer insights into future research directions for both structured and unstructured techniques, as well as perspectives from point cloud and NeRF compression, which are missing in the prior survey.  Furthermore, our work is the first comprehensive study to provide an in-depth discussion on the efficiency (rendering speed) of different compression methods.  

\noindent
\textbf{Highlighted Features: } The major features of this survey include (1) Highlighting Compression Potential: This survey underscores the potential of 3DGS to be highly compressed with minimal quality loss. We demonstrate that 3DGS is compatible with various compression methodologies and their combinations. For the first time, we provide a comprehensive understanding of different 3DGS compression methodologies from a topological perspective. (2) Key Concepts and Improvements: We discuss the essential concepts involved in compressing 3DGS and outline strategies for further improvements. This includes an in-depth examination of current techniques and their impact on compression efficiency and quality preservation. (3) Guidelines for Future Research: Drawing from recent advancements, we extract pivotal ideas from state-of-the-art compression methodologies. Based on these insights, we propose a set of guidelines for future research to enhance 3DGS compression pipelines. This includes best practices and innovative approaches to be incorporated in future studies.

% \textbf{Contributions: } The main contributions of our survey are as follows.
As the pioneering attempt to present a comprehensive survey on 3DGS compression, our survey aims to help the readers of interest quickly grasp its development. Overall, the contributions of our survey are summarized as follows:

(1) The paper offers a detailed taxonomy of 3DGS compression methods, categorizing them into structured and unstructured approaches, and provides an in-depth analysis of their methodologies, performance trade-offs, and limitations. % (Section~\ref{sec:compression}).

(2) We systematically analyze existing compression techniques, highlighting key challenges such as scalability constraints in large-scale scenes, dependence on vector quantization, suboptimal loss function designs, and limitations in deploying 3DGS on resource-constrained hardware.

(3) We provide insights for advancements including scalar quantization for efficient hardware compatibility, hybrid frameworks that integrate structured and unstructured compression strategies, and leveraging insights from NeRF and point cloud compression to enhance performance and adaptability.


% (2) In Section~\ref{sec:struc_vs_unstruc} we highlight critical challenges in 3DGS compression, such as scalability for large scenes, reliance on vector quantization, and underexplored loss function optimization, while emphasizing the need for solutions suitable for resource-constrained environments.

% (3) The survey paper also proposes future directions, including scalar quantization for efficient hardware deployment, hybrid frameworks that integrate structured and unstructured methods, and optimizing loss functions to balance compression and quality. It also emphasizes real-world deployment challenges and applications, such as VR/AR and edge computing, while drawing inspiration from advancements in NeRFs and point cloud compression techniques to inform future innovations (Section~\ref{sec:future_works}).
%For a detailed background and explanation of 3DGS, please refer to the supplementary material. The remainder of the paper is structured as follows: Section II introduces the problem statement for 3DGS compression. Sections III and IV discuss unstructured and structured compression methods, respectively. Section V provides a comparative analysis of both techniques. Section VI explores future directions inspired by NeRFs and point cloud compression. Finally, Section VII concludes the paper.  

The remainder of the paper is structured as follows: Section II presents a detailed background on 3DGS and its working principles. Section III introduces the problem statement for 3DGS compression. Sections IV and V discuss unstructured and structured compression methods, respectively. Section VI provides a comparative analysis of both techniques. Section VII explores future directions inspired by NeRFs and point cloud compression. Finally, Section VIII concludes the paper.  
\section{Preliminaries}

\subsection{Differentiable Rendering }

% Point-based rendering techniques aim to produce realistic images by rendering a collection of discrete geometric primitives. Zwicker et al.~\cite{zwicker2001surface} introduced a method that uses ellipsoid-shaped splats, allowing them to cover multiple pixels and enhance image quality by reducing visible holes through overlapping splats. This approach is more efficient than traditional point-based representations. Kopanas et al.~\cite{kopanas2021point} developed a novel differentiable point-based rendering pipeline that incorporates bi-directional Elliptical Weighted Average splatting, a probabilistic depth test, and efficient camera selection. Recent advancements in this field have focused on improving the rendering process through techniques such as anti-aliasing texture filters~\cite{zwicker2001ewa}, enhancements in rendering efficiency~\cite{botsch2003high}, and addressing issues with discontinuous shading~\cite{rusinkiewicz2000qsplat}.

% Traditional point-based rendering methods are focused on generating high-quality outputs based on predefined geometries. However, recent advances in implicit representation techniques have led researchers to explore point-based rendering within the framework of neural implicit representations. This approach eliminates the need for predefined geometries in 3D reconstruction. A prominent example in this area is  NeRF~\cite{mildenhall2020nerf}, which uses an implicit density field to model 3D geometry and an appearance field to predict view-dependent colors. In NeRF, point-based rendering is used to aggregate the colors from all sample points along a camera ray to compute the final pixel color \( C \).

%Differentiable rendering (DR) enables end-to-end optimization by providing gradients of the rendering process, bridging 2D and 3D processing. This allows neural networks to optimize 3D entities through 2D projections. Gradients can be computed with respect to scene parameters, such as geometry, lighting, and camera pose which is essential for tasks like inverse graphics, 3D reconstruction, and neural rendering, where scene parameters are optimized using self supervision. Rendering approaches are broadly categorized into implicit and explicit methods. Implicit rendering uses continuous scene representations, such as neural networks or signed distance fields, and utilizes volume rendering for image generation. In contrast, explicit rendering defines scene geometry using discrete primitives like triangles, points, or splats. Explicit point-based rendering techniques aim to produce realistic images by rendering a collection of discrete geometric primitives. Zwicker et al.~\cite{zwicker2001surface} introduced a method that uses ellipsoid-shaped splats, allowing them to cover multiple pixels and enhance image quality by reducing visible holes through overlapping splats. This approach is more efficient than traditional point-based representations. Kopanas et al.~\cite{kopanas2021point} developed a novel differentiable point-based rendering pipeline incorporating bi-directional Elliptical Weighted Average splatting, a probabilistic depth test, and efficient camera selection. 

Differentiable rendering (DR) facilitates end-to-end optimization by computing gradients of the rendering process, bridging 2D and 3D processing~\cite{kato2020differentiable}. These gradients, with respect to scene parameters like geometry, lighting, and camera pose, are crucial for tasks such as human pose estimation~\cite{ge20193d,baek2019pushing}, 3D reconstruction~\cite{yan2016perspective,tulsiani2017multi}, and neural rendering~\cite{kato2018neural}, enabling self-supervised optimization. Rendering methods are categorized as implicit or explicit. Implicit approaches represent scenes using continuous functions, such as neural networks~\cite{sitzmann2020implicit} or signed distance fields~\cite{oleynikova2016signed}, and rely on volume rendering~\cite{drebin1988volume} for image generation. Explicit methods, in contrast, define geometry using discrete primitives like triangles, points, or splats. Explicit point-based rendering focuses on realistic image synthesis by rendering collections of discrete geometric primitives. Zwicker et al.\cite{kopanas2021point} advanced this with a differentiable point-based pipeline featuring bi-directional Elliptical Weighted Average splatting, probabilistic depth testing, and efficient camera selection.

Recent advances in implicit representation techniques have led researchers to explore point-based rendering within the framework of neural implicit representations. This approach eliminates the need for predefined geometries in 3D reconstruction. A prominent example in this area is  NeRF~\cite{mildenhall2020nerf}, which uses an implicit density field to model 3D geometry and an appearance field to predict view-dependent colors. In NeRF, point-based rendering is used to aggregate the colors from all sample points along a camera ray to compute the final pixel color

\[ C = \sum_{i=1}^{N} c_i \alpha_i T_i , \]

\noindent
where \( N \) represents the number of sample points along a ray. The view-dependent color and opacity value for the \( i \)-th point on the ray are given by:

\[ \alpha_i = \exp \left(- \sum_{j=1}^{i-1} \sigma_j \delta_j \right), \]
\noindent
where \( \sigma_j \) denotes the density value of the \( j \)-th point and \( \delta_j \) is the distance between consecutive sample points. The accumulated transmittance \( T_i \) is calculated as:

\[ T_i = \prod_{j=1}^{i-1} (1 - \alpha_j). \]

The rendering process in 3DGS shares similarities with NeRFs; however, there are two key differences. \newline
\textbf{Opacity Modeling:} 3DGS directly models opacity values, while NeRF first models density values and then converts these densities into opacity values.\newline
\textbf{Rendering Technique:} 3DGS utilizes rasterization-based rendering, which avoids the need for extensive point sampling, unlike NeRF, which relies on sampling points along rays for rendering.


% \subsection{Neural Implicit Field}
% %Neural implicit field representations have recently gained substantial attention~\cite{}. These methods conceptualize 2D or 3D signals as fields within corresponding Euclidean spaces, training neural networks on discrete samples to approximate these fields. This approach supports tasks such as reconstructing, interpolating, and extrapolating from the original discrete samples, facilitating applications like super-resolution of 2D images and novel view synthesis of 3D scenes. Specifically, in 3D reconstruction and novel view synthesis, NeRFs~\cite{} utilize neural networks to model 3D scenes as density and radiance fields. NeRF employs volumetric rendering to map these 3D fields to 2D images, enabling the reconstruction of 3D scenes from multiple 2D images and allowing for novel view rendering. Noteworthy methods in this area include Mip-NeRF 360~\cite{}, which excels in rendering quality, and Instant-NGP~\cite{}, renowned for its exceptional training efficiency.

% %In recent times, there has been a significant focus on NeRFs~\cite{ramirez2024deep,wu2021revisiting}. These techniques involve training neural networks on discrete samples to model 2D or 3D signals as fields within equivalent Euclidean spaces. This approach facilitates operations such as reconstruction, interpolation, and extrapolation, making it useful for applications like novel view synthesis of 3D scenes and 2D image super-resolution. 
% For applications like 3D reconstruction and novel view synthesis, NeRFs \cite{mildenhall2021nerf,ramirez2024deep,wu2021revisiting} use neural networks to implicitly represent 3D scenes as density and radiance fields. NeRF also employs volumetric rendering to project these 3D fields onto 2D images, enabling the generation of new views and the reconstruction of 3D scenes from multiple 2D images. Among the notable techniques in this domain are Instant-NGP~\cite{muller2022instant}, recognized for its exceptional training efficiency, and Mip-NeRF 360~\cite{barron2022mip}, which is distinguished by its superior rendering quality.

% NeRFs predominantly rely on volumetric ray marching for image rendering. This process involves sampling numerous points along each ray and passing them through the neural network to generate the final image. Consequently, rendering a single 1080p image can require up to \(10^8\) neural network forward passes, often leading to rendering delays of several seconds. Even though some approaches use explicit, discretized structures to represent continuous 3D fields—thereby reducing reliance on neural networks and accelerating the query process \cite{sun2022direct,chen2022tensorf}—the high number of sampling points still results in significant rendering costs. As a result, volumetric rendering-based techniques struggle to achieve real-time performance, limiting their applicability in interactive or real-time scenarios.

\subsection{3D Gaussian Splatting}
3DGS represents a significant advancement in real-time, high-resolution image rendering that does not rely on deep neural networks. %This section provides an overview of 3DGS. In Sec. 3.1, we explain how 3DGS synthesizes an image using optimized 3D Gaussians. This involves the rendering pipeline and techniques employed to achieve high-quality images efficiently. In Sec. 3.2, we discuss the process of generating well-optimized 3D Gaussians for a given scene. This section delves into the 3DGS optimization method, outlining the steps and algorithms used to refine the Gaussian parameters for optimal rendering performance.
%\subsubsection{Rendering with 3DGS}
%NeRFs utilize volumetric raymarching for rendering, a method that is both slow and computationally intensive. In contrast, 
A scene optimized with 3DGS achieves rendering by projecting the Gaussians onto a 2D image plane through splatting. After this projection, 3DGS sorts the Gaussians and computes the pixel values accordingly~\cite{kerbl20233d}. The following sections will first define the 3D Gaussian, the fundamental element of the 3DGS scene representation, then detail the differentiable rendering process used in 3DGS, and finally, describe the techniques employed to achieve high rendering speeds.
%Consider a scene with millions of optimized 3D Gaussians. The goal is to create an image using a specified camera pose. NeRFs address this task using computationally intensive volumetric raymarching, which involves sampling points in 3D space per pixel. This method struggles with high-resolution image synthesis and fails to achieve real-time rendering, particularly on platforms with limited computational resources. In contrast, 3DGS projects these 3D Gaussians onto a pixel-based image plane through a process called "splatting." Following this projection, 3DGS sorts the Gaussians and computes the pixel values. The following sections will start with defining a 3D Gaussian, the fundamental element in 3DGS scene representation. Next, we will explain how these 3D Gaussians are utilized for differentiable rendering. Finally, we will introduce the acceleration technique employed in 3DGS, which is essential for fast rendering.



\noindent
\textbf{3D Gaussian: }The characteristics of a 3D Gaussian include its color \( c \), 3D covariance matrix \( \Sigma \), opacity \( \alpha \), and center (position) \( \mu \). For view-dependent appearance, spherical harmonics (SH) are employed to represent color \( c \). Through back-propagation, all these attributes can be learned and optimized.

\noindent
\textbf{3D Gaussian Frustum Culling: } This stage identifies the 3D Gaussians that lie outside the camera's frustum for a given camera position. Excluding these Gaussians from subsequent computations conserves processing resources, as only the 3D Gaussians within the specified view are considered.

\noindent
\textbf{Splatting of Gaussians: }In this stage, 3D Gaussians (ellipsoids) are projected onto 2D image space (ellipses) for rendering. The projected 2D covariance matrix \( \Sigma' \) is computed using the viewing transformation \( W \) and the 3D covariance matrix \( \Sigma \) as
\noindent
\[ \Sigma' = JW\Sigma W^\top J ,\]
\noindent
where \( W \) is the viewing transformation matrix, \( \Sigma \) is the 3D covariance matrix, and \( J \) is the Jacobian of the projection transformation~\cite{kerbl20233d, zwicker2001ewa}.

\noindent
\textbf{Pixel Rendering: } The viewing transformation \( W \) is employed to compute the distance between each pixel \( x \) and all overlapping Gaussians, effectively determining their depths. This results in a sorted list of Gaussians \( N \). The final color of each pixel is subsequently determined using alpha compositing. The final color \( C \) is computed by summing the contributions of each Gaussian:

\[ C = \sum_{n=1}^{|N|} c_n \alpha'_n \prod_{j=1}^{n-1} (1 - \alpha'_j) ,\]

\noindent
where \( c_n \) is the learned color. The final opacity \( \alpha'_n \) is the product of the learned opacity \( \alpha_n \) and the Gaussian function:

\[ \alpha'_n = \alpha_n \times \exp \left( -\frac{1}{2} (x' - \mu'_n)^\top \Sigma'^{-1}_n (x' - \mu'_n) \right) ,\]

\noindent
where \( x' \) is the projected position, \( \mu'_n \) is the projected center, and \( \Sigma'_n \) is the projected 2D covariance matrix. There is a valid concern that the described rendering process could be slower than NeRFs due to the challenges in parallelizing the generation of the necessary sorted list. This concern is valid, as relying on a straightforward pixel-by-pixel approach could significantly impact rendering speeds. To achieve real-time rendering, 3DGS must consider several factors that facilitate parallel processing.
%Given the challenge of parallelizing the generation of the necessary sorted list, there is a legitimate concern that the described rendering process might be slower than NeRFs. This concern is valid, as rendering times can be significantly impacted by using such a basic pixel-by-pixel method. To enable real-time rendering, 3DGS must account for several factors that support parallel processing.

\section{Dataset}
\label{sec:dataset}

\subsection{Data Collection}

To analyze political discussions on Discord, we followed the methodology in \cite{singh2024Cross-Platform}, collecting messages from politically-oriented public servers in compliance with Discord's platform policies.

Using Discord's Discovery feature, we employed a web scraper to extract server invitation links, names, and descriptions, focusing on public servers accessible without participation. Invitation links were used to access data via the Discord API. To ensure relevance, we filtered servers using keywords related to the 2024 U.S. elections (e.g., Trump, Kamala, MAGA), as outlined in \cite{balasubramanian2024publicdatasettrackingsocial}. This resulted in 302 server links, further narrowed to 81 English-speaking, politics-focused servers based on their names and descriptions.

Public messages were retrieved from these servers using the Discord API, collecting metadata such as \textit{content}, \textit{user ID}, \textit{username}, \textit{timestamp}, \textit{bot flag}, \textit{mentions}, and \textit{interactions}. Through this process, we gathered \textbf{33,373,229 messages} from \textbf{82,109 users} across \textbf{81 servers}, including \textbf{1,912,750 messages} from \textbf{633 bots}. Data collection occurred between November 13th and 15th, covering messages sent from January 1st to November 12th, just after the 2024 U.S. election.

\subsection{Characterizing the Political Spectrum}
\label{sec:timeline}

A key aspect of our research is distinguishing between Republican- and Democratic-aligned Discord servers. To categorize their political alignment, we relied on server names and self-descriptions, which often include rules, community guidelines, and references to key ideologies or figures. Each server's name and description were manually reviewed based on predefined, objective criteria, focusing on explicit political themes or mentions of prominent figures. This process allowed us to classify servers into three categories, ensuring a systematic and unbiased alignment determination.

\begin{itemize}
    \item \textbf{Republican-aligned}: Servers referencing Republican and right-wing and ideologies, movements, or figures (e.g., MAGA, Conservative, Traditional, Trump).  
    \item \textbf{Democratic-aligned}: Servers mentioning Democratic and left-wing ideologies, movements, or figures (e.g., Progressive, Liberal, Socialist, Biden, Kamala).  
    \item \textbf{Unaligned}: Servers with no defined spectrum and ideologies or opened to general political debate from all orientations.
\end{itemize}

To ensure the reliability and consistency of our classification, three independent reviewers assessed the classification following the specified set of criteria. The inter-rater agreement of their classifications was evaluated using Fleiss' Kappa \cite{fleiss1971measuring}, with a resulting Kappa value of \( 0.8191 \), indicating an almost perfect agreement among the reviewers. Disagreements were resolved by adopting the majority classification, as there were no instances where a server received different classifications from all three reviewers. This process guaranteed the consistency and accuracy of the final categorization.

Through this process, we identified \textbf{7 Republican-aligned servers}, \textbf{9 Democratic-aligned servers}, and \textbf{65 unaligned servers}.

Table \ref{tab:statistics} shows the statistics of the collected data. Notably, while Democratic- and Republican-aligned servers had a comparable number of user messages, users in the latter servers were significantly more active, posting more than double the number of messages per user compared to their Democratic counterparts. 
This suggests that, in our sample, Democratic-aligned servers attract more users, but these users were less engaged in text-based discussions. Additionally, around 10\% of the messages across all server categories were posted by bots. 

\subsection{Temporal Data} 

Throughout this paper, we refer to the election candidates using the names adopted by their respective campaigns: \textit{Kamala}, \textit{Biden}, and \textit{Trump}. To examine how the content of text messages evolves based on the political alignment of servers, we divided the 2024 election year into three periods: \textbf{Biden vs Trump} (January 1 to July 21), \textbf{Kamala vs Trump} (July 21 to September 20), and the \textbf{Voting Period} (after September 20). These periods reflect key phases of the election: the early campaign dominated by Biden and Trump, the shift in dynamics with Kamala Harris replacing Joe Biden as the Democratic candidate, and the final voting stage focused on electoral outcomes and their implications. This segmentation enables an analysis of how discourse responds to pivotal electoral moments.

Figure \ref{fig:line-plot} illustrates the distribution of messages over time, highlighting trends in total messages volume and mentions of each candidate. Prior to Biden's withdrawal on July 21, mentions of Biden and Trump were relatively balanced. However, following Kamala's entry into the race, mentions of Trump surged significantly, a trend further amplified by an assassination attempt on him, solidifying his dominance in the discourse. The only instance where Trump’s mentions were exceeded occurred during the first debate, as concerns about Biden’s age and cognitive abilities temporarily shifted the focus. In the final stages of the election, mentions of all three candidates rose, with Trump’s mentions peaking as he emerged as the victor.
\begin{figure*}[t]
    \centering
    \includegraphics[width=\linewidth]{Figures/Opacity_gradient_pruning_comparison.pdf}
    \caption{Comparison between opacity based pruning and gradient + opacity based pruning. The number of Gaussians are in millions (M).}
    \label{fig:opacity_gradient_pruning_comparison}
\end{figure*}

\begin{figure}[t]
    \centering
    \includegraphics[width=0.8\linewidth]{Figures/b_p_opacity_1.pdf}
    \caption{Opacity distribution before and after pruning for the \texttt{truck} scene. Figure taken from~\cite{salman2024trimming}.}
    \label{fig:opacity_distribution}
\end{figure}
\noindent
\textbf{Tiling: }
3DGS reduces computational costs by shifting rendering accuracy from fine-grained, pixel-level detail to a more coarse, patch-level approach. The image is first divided into non-overlapping patches or tiles, and then 3DGS identifies which tiles intersect with the projected Gaussians. If a single projected Gaussian spans multiple tiles, it duplicates the Gaussian, assigning each copy a unique tile ID corresponding to the relevant tile as shown in Figure~\ref{fig:working}.

\noindent
\textbf{Rendering Parallelization: }Following the Gaussians' replication, 3DGS associates each corresponding tile ID with the depth values obtained from the view transformation. This process generates an unsorted list of bytes where the higher bits represent the tile ID and the lower bits encode the depth information. This list can then be sorted for rendering (alpha compositing). This method is particularly well-suited for parallel processing because it allows each tile and pixel to be handled independently as shown in the Figure~\ref{fig:working}. Additionally, the advantage of allowing each pixel of the tile to access shared memory while maintaining a unified read pattern enables more efficient parallel processing during the alpha compositing stage. The framework essentially treats tile and pixel processing similarly to blocks and threads in CUDA programming, as demonstrated in the original study's implementation.

%After replicating the Gaussians, 3DGS associates the depth values obtained from the view transformation with each corresponding tile ID. This results in an unsorted list of bytes where the lower bits represent the depth and the upper bits denote the tile ID. This list can then be sorted for rendering (alpha compositing). This method is particularly well-suited for parallel processing because it allows each tile and pixel to be handled independently. Additionally, the advantage of each tile's pixels being able to access a shared memory while maintaining a consistent read sequence facilitates more efficient parallel execution of alpha compositing. The framework essentially treats tile and pixel processing similarly to blocks and threads in CUDA programming, as demonstrated in the original study's implementation.

In summary, 3DGS enhances computational efficiency while preserving high image reconstruction quality by incorporating several approximations during the rendering process.

\noindent
\textbf{3D Gaussian Splatting Optimization: }
%The core of 3DGS involves an optimization process aimed at generating a substantial number of 3D Gaussians that effectively capture the scene's essence, enabling free-viewpoint rendering. 
The key component of 3DGS lies in an optimization process aimed at generating a substantial number of Gaussians that effectively encapsulate the key features of the scene, thereby enabling real-time 3D scene rendering. Differentiable rendering is employed to fine-tune the parameters of these 3D Gaussians to align with the scene's textures. However, the optimal number of 3D Gaussians required to accurately represent a scene is not predetermined. %In the following sections, we will address the optimization of each Gaussian's attributes and the regulation of Gaussian density. 
The optimization workflow encompasses a series of interrelated procedures.

\noindent
\textbf{Loss Function: }
%After completing the image synthesis, the discrepancy between the rendered image and the ground truth can be assessed. 
The disparity between the ground truth and the rendered output can be evaluated upon completion of the image synthesis. To achieve this, the \( \ell_1 \) loss and D-SSIM (Differentiable Structural Similarity Index)~\cite{bakurov2022structural} loss functions are utilized. Stochastic gradient descent (SGD) is then employed to optimize all the learnable parameters. The loss function used for 3DGS is given by~\cite{kerbl20233d}:

\[ L = (1 - \lambda)L_1 + \lambda L_{\text{D-SSIM}}, \]
\noindent
where \( \lambda \in [0, 1] \) is a weighting factor.





\noindent
\textbf{Parameter Optimization:}
Back-propagation can be utilized to directly optimize most attributes of a 3D Gaussian. 
It is crucial to note that optimization of the covariance matrix directly could yield a non-positive semi-definite matrix, thereby contradicting the physical restrictions typically imposed on covariance matrices.
%However, a key consideration is that directly optimizing the covariance matrix \(\Sigma\) might result in a non-positive semi-definite matrix, which conflicts with the physical constraints typically associated with covariance matrices. 
To address this issue, 3DGS employs an alternative approach by optimizing a quaternion \(q\) and a 3D vector \(s\), where \(q\) represents rotation and \(s\) represents scale. This method ensures that the resulting covariance matrix adheres to the required physical properties. This method enables the covariance matrix \(\Sigma\) to be reconstructed using the following formula:

\[
\Sigma = R S S^\top R^\top,
\]
\noindent
where \(S\) and \(R\) represent the scaling and rotation matrices derived from \(s\) and \(q\), respectively. The process of calculating opacity \(\alpha\) involves a complex computational graph: \(s\) and \(q \to \Sigma\), \(\Sigma \to \Sigma'\), and \(\Sigma' \to \alpha\). To avoid the computational expense of automatic differentiation, 3DGS directly derives and computes the gradients for \(q\) and \(s\) during optimization.

%\begin{figure}[t]
    \centering
    \includegraphics[width=0.8\linewidth]{Figures/b_p_opacity_1.pdf}
    \caption{Opacity distribution before and after pruning for the \texttt{truck} scene. Figure taken from~\cite{salman2024trimming}.}
    \label{fig:opacity_distribution}
\end{figure}

\noindent
\textbf{Densification Process:}
In 3DGS, optimization parameters are initialized using sparse points from Structure-from-Motion (SfM)~\cite{snavely2006photo}. Densification and pruning techniques are then applied to optimize the density of 3D Gaussians.
During densification, 3DGS adaptively increases Gaussian density in regions with sparse coverage or missing geometric features. After specific training intervals, Gaussians with gradients above a threshold are densified: large Gaussians are split into smaller ones, and small Gaussians are duplicated and shifted along gradient directions. This ensures an optimal distribution for improved 3D scene synthesis.
In the pruning stage, redundant or insignificant Gaussians are removed to regularize the representation. Gaussians with excessive view-space size or opacity \( \alpha \) below a threshold are discarded, ensuring efficiency while preserving accuracy in scene representation.


%However, neural implicit field methods heavily depend on volumetric rendering to generate rendered images. This process involves sampling numerous points along each ray and passing them through the neural network to produce the final image. As a result, rendering a single 1080p image can require approximately \(10^8\) neural network forward passes, often resulting in rendering times of several seconds. Although some approaches utilize explicit, discretized structures to represent continuous 3D fields, which reduces reliance on neural networks and speeds up the query process [9], [10], [11], the extensive number of sampling points still results in high rendering costs. Consequently, methods reliant on volumetric rendering cannot achieve real-time performance, limiting their applicability for real-time or interactive applications.
%\section{Preliminaries}\label{subs:preliminaries}

For a comprehensive introduction to feedback linearization, the interested reader is referred to \cite{Isidori1995}.

Consider a multivariable nonlinear system
    \begin{equation}
    \left\{
\begin{split}
     \dot{\xv}&= \fv(\xv)+\Gm(\xv)\uv,\\
    \yv &= \hv(\xv),
    \label{eq:sys}
    \end{split}
    \right.
    \end{equation}
where \mbox{$\xv\in \mathbb{R}^n$} is the state, the input matrix is \mbox{$\Gm(\xv)=\begin{bmatrix}
    \gv_1(\xv)  \;\cdots \; \gv_{p}(\xv)
\end{bmatrix}\in \mathbb{R}^{n \times p}$}, $\fv(\xv)$,  $\gv_1(\xv), \ldots, \gv_{p}(\xv)$ are smooth vector fields, and $\hv(\xv)=\begin{bmatrix} h_1(\xv)  \cdots h_{p}(\xv)\end{bmatrix}^\top$ is a smooth function defined on an open set of $\mathbb{R}^n$.
The  system (\ref{eq:sys}) is said to have \emph{(vector) relative degree} $\rv = \{
    r_1, \ldots, r_{p}
\}$ at a point $\xv^\circ$ w.r.t. the input-output pair  $(\uv,\yv)$ if  
\begin{flalign}
&\text{\textrm{(i)}}    & L_{\gv_j}L^{k}_{\fv} h_i(\xv) &=0,&
\end{flalign}
for all $1\leq j \leq p$, for all $k\leq r_i-1$, for all $1\leq i \leq p$ and for all $\xv$ in a neighborhood of $\xv^\circ$, and\\
\textrm{(ii)}\;\;  the $p\times p$ matrix 
   \begin{align}
    \Am(\xv) &:= 
         \begin{pmatrix}
            L_{\gv_1}L^{r_1-1}_{\fv}h_1(\xv) & \cdots&  L_{\gv_{p}}L^{r_1-1}_{\fv}h_1(\xv) \\ 
            L_{\gv_1}L^{r_2-1}_{\fv}h_2(\xv) & \cdots&  L_{\gv_{p}}L^{r_2-1}_{\fv}h_2(\xv) \\  
            \vdots & & \vdots \\
            L_{\gv_1}L^{r_{p}-1}_{\fv}h_{p}(\xv) & \cdots&  L_{\gv_{p}}L^{r_{p}-1}_{\fv}h_{p}(\xv) 
        \end{pmatrix}
    \label{eq:intbmatrix}
\end{align}  
is nonsingular at $\xv = \xv^\circ$. 
The output array at the $\rv$-th derivative may then be written as an affine system of the form
\begin{equation}
{\yv}^{(\rv)} :=\begin{bmatrix}
    y_1^{(r_1)} \;\cdots\;  y_{p}^{(r_{p})}
\end{bmatrix}^\top =\bv(\xv)+\Am(\xv)\uv,
\label{eq:y_r_now}
\end{equation}
with 
\begin{equation}
 \bv(\xv):=\begin{bmatrix}
L_{\boldsymbol{f}}^{(r_1)}{h_1(\xv)} \; \cdots \; 
L_{\boldsymbol{f}}^{(r_{p})}{h_{p}(\xv)}
\end{bmatrix}^\top.
\label{eq:b}
\end{equation}


Suppose the system \eqref{eq:sys} has some \emph{(vector) relative degree} $\rv:=\{r_1,\ldots,r_p\}$ at $\xv^\circ$ and that the matrix $\Gm(\xv^\circ)$ has rank $p$ in a  neighborhood $\mathcal{U}$ of $\xv^\circ$. Suppose also that  \mbox{$r_1+r_2+\ldots+r_p=n$}, and choose the  control input to be $$\uv = \Am^{-1}(\xv)[-\bv(\xv)+\vv],
$$
where $\vv \in \mathbb{R}^{p}$ can be assigned freely and $\Am(\xv), \bv(\xv)$ are defined as in~(\ref{eq:intbmatrix}) and~(\ref{eq:b}).
Then the output dynamics \eqref{eq:y_r_now} become
$$
\yv^{(\rv)} = \vv.
$$
We refer to \(\yv\) as a \textit{linearizing output array}, which possesses the property that the entire state and input of the system can be expressed in terms of \(\yv\) and its time derivatives. 


 \section{3DGS Compression}
 \label{sec:compression}
%3DGS has rapidly gained prominence in the vision and graphics community due to its simplicity and effectiveness, finding applications in areas such as Simultaneous Localization and Mapping (SLAM)~\cite{tosi2024nerfs,yan2024gs,keetha2024splatam,matsuki2024gaussian}, Dynamic Scene Reconstruction~\cite{yang2023real,yang2024deformable,wu20244d,lin2024gaussian}, AI-Generated Content (AIGC)~\cite{tang2023dreamgaussian,yi2024gaussiandreamer,tang2024lgm,zhou2024dreamscene360}, Endoscopic Scene Reconstruction~\cite{huang2024endo, liu2024endogaussian,zhu2024deformable,zhao2024hfgs}, and Large-scale Scene Reconstruction~\cite{kerbl2024hierarchical,liu2024citygaussian,lin2024vastgaussian,ren2024octree}. For detailed information on these applications, please refer to the relevant articles~\cite{DBLP:journals/corr/abs-2401-03890,bao20243d}. Despite its advantages, 3DGS encounters significant scalability challenges compared to NeRFs. While NeRFs require only the storage of weight parameters for a multi-layer perceptron (MLP), 3DGS necessitates storing millions of Gaussians per scene. This issue becomes more critical in large, complex scenes, where computational and memory demands increase exponentially.  Consequently, optimizing both memory usage and computational efficiency for storage and training is crucial.

3DGS faces significant scalability challenges compared to NeRFs. While NeRFs require only the storage of weight parameters for a multilayer perceptron (MLP), 3DGS necessitates storing the parameters of millions of Gaussians per scene. This issue becomes especially critical in large, complex scenes, where computational and memory demands increase significantly. The number of Gaussians is directly proportional to storage and computational complexity and inversely proportional to rendering efficiency. Therefore, optimizing memory usage and computational efficiency for both storage and rendering is essential to improve the scalability of 3DGS-based methods.

\noindent
\textbf{3DGS Attributes: }The input signal to be compressed consists of $N$ Gaussians, each characterized by multiple attributes: $3\times1$ position vectors ($\mu$), $3\times1$  scale vector and $4\times1$  rotation quaternion vector, scalar opacity values and spherical harmonics (SH) coefficients for view-dependent RGB color modeling. For degree $3$ SH, this requires $48$ coefficients per Gaussian ($16$ per RGB channel). Each parameter has a bit depth of 32-bit floating point. 


\noindent
\textbf{Evaluation Metrics:} The compression cost is quantified either as the bit count of the compressed representation or its compression ratio compared to the baseline 3DGS-30k (trained for $30,000$ iterations). The performance evaluation of 3DGS compression relies on image-based fidelity metrics such as PSNR (Peak Signal-to-Noise Ratio), SSIM (Structural Similarity Index Measure), and LPIPS (Learned Perceptual Image Patch Similarity), and its computational efficiency is measured in terms of Frames per second (FPS). 

\noindent
\textbf{Datasets:} Similar to NeRF-based rendering methods, 3DGS-based methods are commonly evaluated on 9 scenes from Mip-NeRF360~\cite{barron2022mip}, which includes both indoor and outdoor scenes, two scenes from Tanks\&Temples~\cite{knapitsch2017tanks}, and the Deep Blending~\cite{hedman2018deep} dataset. Figure~\ref{fig:dataset} shows a sample image of each scene from all the datasets. To ensure consistent benchmarking studies typically adhere to the train-test split used in Mip-NeRF360~\cite{barron2022mip} and 3DGS where every 8th image must be selected for testing.


\begin{table*}[ht]
    \caption{Performance and compression comparison of the 3DGS baseline with unstructured compression methods on the Mip-NeRF360 and Tanks\&Temple datasets. All values are sourced from their respective papers. Memory size is reported in megabytes (MB). The best-performing results are highlighted in red, followed by yellow and green.}
    \label{tab:unstruc_mip_tandt}
    \centering
    \small
    \resizebox{0.95\textwidth}{!}{
    \begin{tabular}{l|ccccc|ccccc}
    \toprule
    &  \multicolumn{5}{c}{\textbf{Mip-NeRF360~\cite{barron2022mip}}} & \multicolumn{5}{c}{\textbf{Tanks\&Temples~\cite{knapitsch2017tanks}}} \\
    \textbf{Model} & \textbf{SSIM$^\uparrow$} & \textbf{PSNR$^\uparrow$} & \textbf{LPIPS$^\downarrow$} & \textbf{Mem.}$^\downarrow$ & \textbf{Comp.$^\uparrow$} & \textbf{SSIM$^\uparrow$} & \textbf{PSNR$^\uparrow$} & \textbf{LPIPS$^\downarrow$} & \textbf{Mem.}$^\downarrow$ & \textbf{Comp.$^\uparrow$}\\
    \midrule

3DGS-30k~\cite{kerbl20233d} & \cellcolor{yellow!25}0.815 & 27.21 & \cellcolor{red!25}0.214 & 734.0 & 1$\times$ & \cellcolor{red!25}0.841 & 23.14 & \cellcolor{red!25}0.183 & 411.0 & 1$\times$ \\
\midrule
LightGaussian~\cite{fan2023lightgaussian} & 0.805 & \cellcolor{green!25}27.28 & 0.243 & 42.0 & 18$\times$ & 0.817 & 23.11 & 0.231 & 22.0 & 19$\times$ \\
Compact3D~\cite{lee2023compact} & 0.798 & 27.08 & 0.247 & 48.8 & 15$\times$ & 0.831 & 23.32 & 0.201 & 39.4 & 10$\times$ \\
CompGS-16K~\cite{navaneet2023compact3d} & 0.804 & 27.03 & 0.243 & \cellcolor{red!25}18.0 & \cellcolor{red!25}41$\times$ & 0.836 & 23.39 & 0.200 & \cellcolor{green!25}12.0 & \cellcolor{green!25}34$\times$ \\
CompGS-32K~\cite{navaneet2023compact3d} & 0.806 & 27.12 & 0.240 & \cellcolor{yellow!25}19.0 & \cellcolor{yellow!25}39$\times$ & \cellcolor{green!25}0.838 & 23.44 & 0.198 & 13.0 & 32$\times$ \\
Niedermayr et al.~\cite{niedermayr2024compressed} & 0.801 & 26.98 & 0.238 & 28.8 & 26$\times$ & 0.832 & 23.32 & \cellcolor{green!25}0.194 & 17.3 & 24$\times$ \\
EAGLES~\cite{girish2023eagles} & \cellcolor{green!25}0.810 & 27.23 & 0.240 & 54.0 & 14$\times$ & \cellcolor{yellow!25}0.840 & 23.37 & 0.200 & 29.0 & 14$\times$ \\
Papantonakis et al.~\cite{papantonakis2024reducing} & 0.809 & 27.10 & \cellcolor{green!25}0.226 & 29.0 & 25$\times$ & \cellcolor{yellow!25}0.840 & 23.57 & \cellcolor{yellow!25}0.188 & 14.0 & 29$\times$ \\
Trimming the Fat~\cite{salman2024trimming} & 0.798 & 27.13 & 0.248 & \cellcolor{green!25}20.1 & \cellcolor{green!25}37$\times$ & 0.831 & \cellcolor{green!25}23.69 & 0.210 & \cellcolor{red!25}8.6 & \cellcolor{red!25}48$\times$ \\
Efficientgs~\cite{liu2024efficientgs} & \cellcolor{red!25}0.817 & \cellcolor{red!25}27.38 & \cellcolor{yellow!25}0.216 & 98.0 & 8$\times$ & 0.837 & 23.45 & 0.197 & 33.0 & 13$\times$ \\
RDO-Gaussian~\cite{wang2024end} & 0.802 & 27.05 & 0.239 & 23.5 & 31$\times$ & 0.835 & 23.34 & 0.195 & \cellcolor{green!25}12.0 & \cellcolor{green!25}34$\times$ \\
PUP 3D-GS~\cite{hanson2024pup} & 0.792 & 26.83 & 0.268 & 86.3 & 9$\times$ & 0.807 & 23.03 & 0.245 & 50.1 & 8$\times$ \\
Kim et al.~\cite{kim2024color} & 0.797 & 27.07 & 0.249 & 73.0 & 10$\times$ & 0.830 & 23.18 & 0.198 & 42.0 & 10$\times$ \\
ELMGS-medium~\cite{ali2024elmgs} & 0.792 & \cellcolor{yellow!25}27.31 & 0.264 & 38.6 & 19$\times$ & 0.838 & \cellcolor{red!25}24.08 & 0.191 & 18.8 & 22$\times$\\
ELMGS-small~\cite{ali2024elmgs} & 0.779 & 27.00 & 0.286 & 25.8 & 28$\times$ & 0.825 & \cellcolor{yellow!25}23.90 & 0.233 & \cellcolor{yellow!25}11.6 & \cellcolor{yellow!25}35$\times$\\


    \bottomrule
    \end{tabular}
    }
\end{table*}


\begin{table}[ht]
    \caption{Performance and compression comparison of the 3DGS baseline with unstructured compression methods on the Deep Blending dataset. All values are sourced from their respective papers. Memory size is reported in megabytes (MB). The best-performing results are highlighted in red, followed by yellow and green.}
    \label{tab:unstruc_db}
    \centering
    \small
    \resizebox{0.49\textwidth}{!}{

    \begin{tabular}{l|ccccc}
    \toprule
    &  \multicolumn{4}{c}{\textbf{Deep Blending~\cite{hedman2018deep}}}  \\
    \textbf{Model} & \textbf{SSIM$^\uparrow$} & \textbf{PSNR$^\uparrow$} & \textbf{LPIPS$^\downarrow$} & \textbf{Mem.}$^\downarrow$ & \textbf{Comp.$^\uparrow$} \\
    \midrule
3DGS-30k~\cite{kerbl20233d} & 0.903 & 29.41 & \cellcolor{red!25}0.243 & 676.0 & 1$\times$ \\
\midrule
Compact3D~\cite{lee2023compact} & 0.901 & \cellcolor{green!25}29.79 & 0.258 & 43.2 & 16$\times$ \\
CompGS-16K~\cite{navaneet2023compact3d} & \cellcolor{green!25}0.906 & \cellcolor{red!25}29.90 & 0.252 & \cellcolor{red!25}12.0 & \cellcolor{red!25}56$\times$ \\
CompGS-32K~\cite{navaneet2023compact3d} & \cellcolor{yellow!25}0.907 & \cellcolor{red!25}29.90 & 0.251 & 13.0 & 52$\times$ \\
Niedermayr et al.~\cite{niedermayr2024compressed} & 0.898 & 29.38 & 0.253 & 25.3 & 27$\times$ \\
EAGLES~\cite{girish2023eagles} & \cellcolor{red!25}0.910 & \cellcolor{yellow!25}29.86 & \cellcolor{green!25}0.250 & 52.0 & 13$\times$ \\
Papantonakis et al.~\cite{papantonakis2024reducing} & 0.902 & 29.63 & \cellcolor{yellow!25}0.249 & 18.0 & 38$\times$ \\
Trimming the Fat~\cite{salman2024trimming} & 0.897 & 29.43 & 0.267 & \cellcolor{green!25}12.5 & \cellcolor{green!25}54$\times$ \\
Efficientgs~\cite{liu2024efficientgs} & 0.903 & 29.63 & 0.251 & 40.0 & 17$\times$ \\
RDO-Gaussian~\cite{wang2024end} & 0.902 & 29.63 & 0.252 & 18.0 & 38$\times$ \\
PUP 3D-GS~\cite{hanson2024pup} & 0.881 & 28.61 & 0.305 & 80.8 & 8$\times$ \\
Kim et al.~\cite{kim2024color} & 0.902 & 29.71 & 0.255 & 72.0 & 9$\times$ \\
ELMGS-medium~\cite{ali2024elmgs} & 0.897 & 29.48 & 0.261 & 23.5 & 29$\times$\\
ELMGS-small~\cite{ali2024elmgs} & 0.894 & 29.24 & 0.273 & \cellcolor{yellow!25}12.3 & \cellcolor{yellow!25}55$\times$\\
    \bottomrule
    \end{tabular}
}
\vspace{-0.5cm}
\end{table}






\noindent
 \textbf{Categorization of 3DGS Compression Methods:}
There has been a plethora of works focusing on the compression of 3DGS. In contrast to the structured feature grids used in NeRF-based methods, the 3D Gaussians employed in 3DGS are sparse and lack organization, which causes significant difficulties in establishing structural relations~\cite{chen2024hac}. Consequently, compression strategies for 3DGS can be categorized into two groups: \newline i) \textbf{Unstructured Compression:} Those primarily concerned with compressing the "value" of model parameters $N$, utilizing methods like pruning~\cite{fan2023lightgaussian,lee2023compact}, quantization~\cite{fan2023lightgaussian,lee2023compact,navaneet2023compact3d,niedermayr2024compressed}, and entropy constraints~\cite{girish2023eagles} without considering the relationship between the Gaussians. \newline ii) \textbf{Structured Compression:} those exploring compression techniques that consider the relationships between Gaussians~\cite{lu2023scaffold,chen2024hac}.

Figure~\ref{fig:taxonomy} presents a detailed taxonomy of 3DGS compression methods, while Table~\ref{tab:Taxonomy} lists representative publications categorized according to their taxonomy classification. The following sections provide an overview of unstructured and structured compression methods, discussing their variations and the challenges associated with each approach.

\vspace{-0.15cm}
\section{Unstructured 3DGS Compression} 
Compression methods built on top of the baseline 3DGS that exploit the sparse nature of Gaussians without altering the fundamental structure of 3D Gaussians or considering their interrelationships fall into this category. These approaches apply techniques such as pruning, quantization, and entropy coding within the existing 3DGS framework, making minimal modifications to the underlying architecture.  The primary objective of these methods is to reduce memory and computational costs while preserving the core advantages of 3DGS, ensuring efficient storage and faster processing without degrading scene representation quality. In this section, based on the taxonomy presented in Figure~\ref{fig:taxonomy} and Table~\ref{tab:Taxonomy}, we provide an in-depth analysis of unstructured compression methods, discussing their performance, challenges, and potential future directions.  
%Compression methods built on top of the baseline 3DGS that leverage the sparse nature of Gaussians, without altering the fundamental structure of 3D Gaussians, fall into this category. These approaches employ techniques such as pruning, quantization, and entropy coding on the existing 3DGS framework with minimal modifications to the underlying architecture. The primary goal of these methods is to reduce memory and computational requirements while preserving the core benefits of 3DGS, ensuring efficient storage and faster processing without compromising the quality of the scene representation. In this section based on the taxonomy presented in Figure~\ref{fig:taxonomy}, we first discuss unstructured compression methods utilizing pruning, followed by quantization and entropy encoding strategies.


\subsection{Pruning}
%Pruning techniques in 3DGS focus on removing redundant Gaussians which results in storage optimization and also improves rendering efficiency. Different pruning strategies utilize specific attributes of Gaussians, enabling a more compact representation while preserving visual fidelity.
Pruning techniques in 3DGS aim to reduce the number of Gaussians resulting in storage optimization and rendering efficiency while maintaining rendering fidelity. These approaches target different Gaussian attributes, utilizing structural, statistical, and learned information to optimize scene representation. The pruning strategies can be broadly classified into size-based, gradient-based, opacity-based, spatial-based, and significance-scoring-based techniques.


\noindent
\textbf{Size-based Pruning:} This method eliminates Gaussians that are structurally redundant due to their small size, as they contribute minimally to scene reconstruction. CompGS~\cite{navaneet2023compact3d} and Papantonakis et al.~\cite{papantonakis2024reducing} apply size-based pruning to remove such Gaussians, improving efficiency while maintaining reconstruction quality.

\noindent
%\textbf{Gradient-based Pruning: }This method removes Gaussians that contribute minimally to optimization by evaluating the magnitude of gradients associated with each Gaussian during training. EfficientGS utilizes cumulative gradient sum analysis to halt unnecessary densification and applies pruning to eliminate redundant Gaussians. GDGS reduces Gaussian density by modeling scene gradients, enhancing compactness without explicit Gaussian removal. Trimming the Fat applies gradient-based pruning, removing 75\% of Gaussians while maintaining high visual quality. Kim et al. extend this approach by incorporating SH gradients alongside positional gradients, further refining the pruning process.
\textbf{Gradient-based Pruning:} This method removes Gaussians that contribute minimally to optimization by evaluating the magnitude of gradients associated with each Gaussian during training. EfficientGS~\cite{liu2024efficientgs} utilizes cumulative gradient sum analysis to halt unnecessary densification and applies pruning to eliminate redundant Gaussians. GDGS~\cite{gong2024gdgs} reduces Gaussian density by modeling scene gradients, enhancing compactness. Trimming the Fat~\cite{salman2024trimming} and ELMGS~\cite{ali2024elmgs} applies gradient-based pruning, removing 75\% of Gaussians while maintaining high visual quality. Kim et al.~\cite{kim2024color} extend this approach by incorporating SH gradients alongside positional gradients, further refining the pruning process.

\begin{figure*}[t]
    \centering
    \includegraphics[width=\linewidth]{Figures/unstruc_mem_psnr.pdf}
    \caption{Comparison of PSNR and storage size across unstructured compression techniques for the Mip-NeRF360, Tanks \& Temple, and Deep Blending datasets.}
    \label{fig:unstruc_memvspsnr}
    \vspace{-0.5cm}
\end{figure*}



\noindent
%\textit{Opacity-based Pruning:} This method removes Gaussians with low opacity values, as they contribute minimally to the final rendered image. Trimming the Fat and ELMGS employs opacity-based pruning in combination with gradient-based pruning to optimize Gaussian selection.
\textbf{Opacity-based Pruning:} This method eliminates Gaussians with low opacity values, as they contribute minimally to scene reconstruction. Trimming the Fat~\cite{salman2024trimming} and ELMGS~\cite{ali2024elmgs} apply opacity-based pruning alongside gradient-based pruning, effectively removing floaters and redundant Gaussians. CompGS~\cite{navaneet2023compact3d}, in combination with size-based pruning, employs a learnable opacity masking approach, dynamically removing Gaussians with persistently low opacity throughout training. Figure~\ref{fig:opacity_gradient_pruning_comparison} presents a qualitative comparison between pruning based solely on opacity and pruning that incorporates both opacity and gradient information. The results highlight the effectiveness of opacity and gradient-informed pruning in preserving scene details while achieving better compression efficiency.

\noindent
\textbf{Significance-based pruning:} This approach incorporates explicit scoring functions to regulate pruning decisions, preventing excessive removal of Gaussians that could degrade scene fidelity. LightGaussian~\cite{fan2023lightgaussian} employs a significance-driven pruning approach that evaluates each Gaussian’s contribution to rendering based on its projection onto camera viewpoints, ensuring minimal perceptual degradation. The significance score is computed from the frequency of Gaussian intersections with rays across all training views. EAGLES~\cite{girish2023eagles} adopts a coarse-to-fine pruning strategy, eliminating Gaussians with the least contribution to reconstruction quality, thereby enhancing training and inference speeds. Papantonakis et al.~\cite{papantonakis2024reducing} combine size-based and significance-based pruning, dynamically adapting SH coefficients to remove structurally redundant Gaussians. SafeguardGS~\cite{lee2024safeguardgs} introduces a pruning score function to ensure optimal Gaussian selection, mitigating the risk of catastrophic scene degradation. LP-3DGS~\cite{zhang2024lp} adopts a trainable binary mask approach that automatically determines the optimal pruning ratio, leveraging Gumbel-Sigmoid-based gradient approximation to maintain compatibility with existing 3DGS training pipelines.

\begin{figure*}[t]
    \centering
    \includegraphics[width=\linewidth]{Figures/unstruc_fps_psnr.pdf}
    \caption{Comparison of PSNR and FPS across unstructured compression techniques. The FPS for all the techniques is evaluated on a single NVIDIA A40 GPU.}
    \label{fig:unstruc_fpsvspsnr}
    \vspace{-0.5cm}    
\end{figure*}


\noindent
\textbf{Spatial-based Pruning:} This method removes Gaussians based on scene location, preserving detail in important regions while reducing redundancy elsewhere. PUP 3D-GS~\cite{hanson2024pup} employs a second-order reconstruction error approximation to selectively prune Gaussians with minimal impact on scene reconstruction. RTGS~\cite{lin2024rtgs}, using a foveated rendering (FR)~\cite{guenter2012foveated,patney2016towards} approach for Point-Based Neural Rendering (PBNR)~\cite{kerbl20233d}, prunes Gaussians based on pixel eccentricity~\cite{wandell1995foundations}, maintaining high-density in critical regions while sparsifying peripheral areas. This approach optimizes memory usage and rendering speed while preserving perceptual consistency, making it particularly effective for real-time rendering, VR, and AR applications.


%\textbf{Spatial-based pruning} selectively removes Gaussians based on their spatial importance within the scene. PUP 3D-GS implements a second-order reconstruction error approximation to guide pruning decisions, ensuring that Gaussians in critical regions are retained while those in less significant regions are removed.

By integrating these diverse pruning methodologies, 3DGS achieves substantial memory reduction and enhanced rendering speeds, making possible real-time Gaussian rendering across various applications, including VR/AR~\cite{zhai2024splatloc}, mobile deployment~\cite{lin2024rtgs}, and autonomous systems~\cite{zhou2024drivinggaussian}. 
However, even after removing redundant Gaussians through pruning, millions of Gaussians are still required for accurate scene reconstruction. To further reduce the memory and storage complexity of these essential Gaussians, quantization techniques are employed.

\subsection{Quantization}
%To further minimize the memory and computational footprint of 3DGS, quantization strategies are employed.
Quantization plays a crucial role in 3DGS compression by reducing the bit precision of Gaussian attributes while maintaining rendering fidelity. By encoding Gaussian parameters more compactly, quantization significantly decreases storage requirements and computational costs. Several works integrate quantization within their compression pipelines, either in isolation or combined with pruning, to achieve higher efficiency. Quantization can be broadly categorized into scalar quantization and vector quantization. Scalar quantization compresses individual attributes independently, while vector quantization groups multiple attributes into a shared codebook, enabling more efficient compression and reduced storage overhead.

\noindent
\textbf{Vector Quantization (VQ):} VQ is widely used in 3DGS compression due to its higher compression efficiency, achieved by clustering similar Gaussians and encoding them through compact indices. LightGaussian~\cite{fan2023lightgaussian} and CompGS~\cite{navaneet2023compact3d} employ codebook-based vector quantization to identify shared Gaussian parameters, further compressing indices via run-length encoding. Niedermayr et al.~\cite{niedermayr2024compressed} introduce sensitivity-aware vector clustering combined with quantization-aware training, optimizing directional color and Gaussian parameters. EAGLES~\cite{girish2023eagles} integrates quantized embeddings, significantly reducing per-point memory requirements while accelerating training and inference.


\noindent
\textbf{Scalar Quantization (SQ):} Although VQ is widely used in 3DGS-based methods, recent advances suggest that SQ offers better hardware efficiency, particularly for deployment in edge devices~\cite{gholami2022survey}. ELMGS~\cite{ali2024elmgs} used learned step-size-based uniform quantization~\cite{esser2019learned} for quantization. However, uniform SQ suffers from a performance drop at lower bit-depths due to the non-uniform distribution of 3DGS attributes such as opacity as seen in Figure~\ref{fig:opacity_distribution}~\cite{esser2019learned}. The distribution of opacity in 3DGS is highly non-uniform, with peaks at both lower and higher opacity levels. This underscores the necessity for non-uniform SQ methods specifically tailored for 3DGS quantization, ensuring better preservation of scene fidelity even at extremely low bit-depths.



\subsection{Entropy Encoding}
After quantization, entropy encoding further compresses Gaussian attributes by eliminating redundancy at the storage level, ensuring a compact representation. CompGS, ELMGS, and Niedermayr et al. apply run-length encoding and entropy-based compression to minimize storage overhead. RDO-Gaussian~\cite{wang2024end} is among the first to introduce an end-to-end rate-distortion framework, dynamically adjusting compression based on a quality-loss trade-off. Morgenstern et al.~\cite{morgenstern2023compact} propose a 2D-grid-based entropy encoding that efficiently organizes Gaussian attributes, enabling parallelized encoding and decoding during rendering.

Entropy encoding serves as a crucial final step in 3DGS compression, significantly enhancing compression efficiency by further reducing storage redundancy. However, it introduces additional computational complexity during decoding. Methods such as RDO-Gaussian, which integrate entropy encoding with quantization and pruning, offer the most efficient end-to-end compression pipelines, balancing compression ratio, and computational complexity.


% \subsubsection{State of the Art}
% The subsequent works highlight prominent contributions in unstructured compression, focusing on their unique methodologies, performance trade-offs, and impact on real-world deployment. These works represent the state-of-the-art in addressing the challenges of compressing 3DGS.

% \textit{LightGaussian~\cite{fan2023lightgaussian}} utilizes pruning, quantization, and knowledge distillation to compress the Gaussians effectively. LightGaussian~\cite{fan2023lightgaussian} identifies Gaussians that have minimal impact on scene reconstruction and applies a pruning and recovery process to effectively reduce redundancy in the number of Gaussians while maintaining visual quality. Additionally, LightGaussian~\cite{fan2023lightgaussian} uses knowledge distillation to transfer spherical harmonics coefficients to a lower degree, facilitating the conversion of knowledge into more compact representations without compromising the appearance of the scene. Moreover, they introduce Gaussian Vector Quantization based on the global significance of Gaussians to quantize all redundant attributes, leading to lower bit-width representations with minimal accuracy loss. In summary, LightGaussian~\cite{fan2023lightgaussian} achieves an average compression rate of over 15$\times$ while increasing the frames per second (FPS) from 119 to 209 on Mip-NeRF360 and Tanks\&Temples dataset. 

% \textit{Compact 3D~\cite{lee2023compact}} introduces a learnable mask strategy that significantly decreases the number of Gaussians while maintaining high performance. They also developed a compact yet effective method for representing view-dependent colors by using a grid-based neural field instead of spherical harmonics. Additionally, they employ codebooks to represent the geometric attributes of Gaussians through vector quantization efficiently. Utilizing model compression techniques such as quantization and entropy coding, they achieved over 25× reduced storage and enhanced rendering speed, while preserving the quality of the scene representation compared to the 3DGS baseline.

% \textit{Navaneet \textit{et al.}~\cite{navaneet2023compact3d}} observed that a large number of Gaussians might share similar parameters. To address this, they introduced a straightforward vector quantization method based on the K-means algorithm to quantize the Gaussian parameters. They stored a small codebook along with the index of the code for each Gaussian and further compressed the indices by sorting them and using a method similar to run-length encoding. This simple strategy could reduce the storage cost of the original 3DGS method by nearly 20$\times$, with only a minimal decrease in the quality of rendered images.

% \textit{Niedermayr \textit{et al.}~\cite{niedermayr2024compressed}} proposed a compressed 3DGS representation that employs sensitivity-aware vector clustering for pruning, combined with quantization-aware training to compress directional colors and Gaussian parameters. The learned codebooks feature low bitrates and achieve a compression rate of up to 31$\times$ on real-world scenes with minimal visual quality degradation. This compressed splat representation can be efficiently rendered using hardware rasterization on lightweight GPUs, resulting in up to 4× higher framerates compared to the baseline 3DGS.

% \textit{EAGLES~\cite{girish2023eagles}} introduced a technique that uses quantized embeddings to substantially decrease per-point memory storage requirements, along with a coarse-to-fine training strategy to enable faster and more stable optimization of Gaussian point clouds. Their approach includes a pruning stage that accelerates training times and rendering speeds for real-time rendering of high-resolution scenes while significantly reducing memory storage by more than an order of magnitude. This method maintains high reconstruction quality while consuming 10-20× less memory and achieving faster training and inference speeds.

% \textit{Papantonakis et al.~\cite{papantonakis2024reducing}} identified three critical factors that could significantly reduce the memory complexity of 3DGS: (1) the number of Gaussians, (2) the number of spherical harmonics used to represent color, and (3) the optimal precision bits required to store 3D Gaussian features. To address these issues, they proposed a comprehensive solution. First, they introduced a pruning method that effectively reduces the number of Gaussians by half. Second, they developed a technique to adaptively adjust the number of coefficients for spherical harmonic (SH) features for each 3D Gaussian. Lastly, they implemented codebook quantization combined with a 16-bit float representation to further decrease memory complexity. These modifications collectively achieved a compression ratio of approximately 27$\times$ compared to the original 3DGS, along with a frame-per-second (FPS) gain of about 1.7$\times$.

% \textit{Trimming the Fat~\cite{salman2024trimming}} is another pruning-based approach designed to remove redundant Gaussians from 3DGS. This method prunes Gaussians based on their opacity and gradient information, eliminating those below a certain threshold. Remarkably, Trimming the Fat can remove about 75\% of the Gaussians while maintaining performance on benchmark datasets and can achieve rendering speeds of up to 600 FPS. When combined with the approach by Neidermayr~\cite{niedermayr2024compressed}, Trimming the Fat can result in approximately 50$\times$ compression gains compared to the baseline 3DGS, significantly enhancing memory efficiency and computational performance.

% \textit{SafeguardGS~\cite{lee2024safeguardgs}} offers a comprehensive analysis of various pruning methods applied to 3DGS and introduces a score function designed to regulate the pruning process. This score function is crucial in preventing excessive pruning that could lead to the destruction of the entire scene. Additionally, SafeguardGS analyzes the performance differences among various pruning functions and examines how these methods impact Gaussian features. 

% \textit{GDGS~\cite{gong2024gdgs}} introduced an innovative approach to reducing the density of Gaussians in 3DGS. Instead of directly modeling the scene, GDGS focuses on modeling the gradients of the scene. This method significantly decreases storage requirements and reduces both computational and memory complexity. The original scene can be accurately reconstructed by solving a Poisson equation, which operates with linear complexity~\cite{farbman2011convolution,gong2015spectrally}. 

% \begin{figure*}[t]
%     \centering
%     % Placeholder for a figure
%     \rule{6cm}{6cm}  % Width x Height of the placeholder
%     \caption{Figure: Big Picture of the Survey Figure}
%     \label{fig:placeholder}
% \end{figure*}

% \begin{table*}[]
%     \caption{Performance and compression comparison of 3DGS baseline with unstructured compression methods on benchmark datasets.}
%     \label{tab:stru_compression_comp}
%     \centering
%     \small
%     \resizebox{0.60\textwidth}{!}{
%     \begin{tabular}{l|ccccc}
%     \toprule
%      \multicolumn{6}{c}{\textbf{Mip-NeRF360~\cite{barron2022mip}}} \\
%      \midrule \midrule
%     \textbf{Model} & \textbf{SSIM$^\uparrow$} & \textbf{PSNR$^\uparrow$} & \textbf{LPIPS$^\downarrow$} & \textbf{FPS}$^\uparrow$ & \textbf{Mem.}$^\downarrow$ \\
%     \midrule




% 3DGS-7k~\cite{kerbl20233d} & 0.770 & 25.60 & 0.279 & 160 & 523 \\
% 3DGS-30k~\cite{kerbl20233d} & 0.815 & 27.21 & 0.214 & 134 & 734 \\
% LightGaussian~\cite{fan2023lightgaussian} & 0.805 & 27.28 & 0.243 & 209 & 42 \\
% Compact3D~\cite{lee2023compact} & 0.798 & 27.08 & 0.247 & 128 & 49 \\
% CompGS 16K~\cite{navaneet2023compact3d} & 0.804 & 27.03 & 0.243 & 346 & 18 \\
% CompGS 32K~\cite{navaneet2023compact3d} & 0.806 & 27.12 & 0.240 & 344 & 19 \\
% Niedermayr et al.~\cite{niedermayr2024compressed}  & 0.801 & 26.98 & 0.238 & - & 29 \\
% EAGLES~\cite{girish2023eagles}  & 0.810 & 27.23 & 0.240 & 131 & 54 \\
% Papantonakis et al.~\cite{papantonakis2024reducing}   & 0.809 & 27.10 & 0.226 & 284 & 29 \\
% Trimming the Fat~\cite{salman2024trimming}  & 0.798 & 27.13 & 0.248 & 210 & 20 \\
% Efficientgs~\cite{liu2024efficientgs}  & 0.817 & 27.38 & 0.216 & 218 & 98 \\
% RDO-Gaussian~\cite{wang2024end}  & 0.802 & 27.05 & 0.239 & 191 & 24 \\
% PUP 3D-GS~\cite{hanson2024pup}  & 0.792 & 26.83 & 0.268 & 244 & 86 \\
% Kim et al~\cite{kim2024color}  & 0.797 & 27.07 & 0.249 & 166 & 73 \\

%    \midrule    \midrule

%    \multicolumn{6}{c}{\textbf{Tanks\&Temples~\cite{knapitsch2017tanks}}}\\
%    \midrule    \midrule

% 3DGS-7k~\cite{kerbl20233d} & 0.767 & 21.20 & 0.280 & 197 & 270 \\
% 3DGS-30k~\cite{kerbl20233d} & 0.841 & 23.14 & 0.183 & 154 & 411 \\
% LightGaussian~\cite{fan2023lightgaussian} & 0.817 & 23.11 & 0.231 & 209 & 22 \\
% Compact3D~\cite{lee2023compact} & 0.831 & 23.32 & 0.201 & 185 & 39 \\
% CompGS 16K~\cite{navaneet2023compact3d} & 0.836 & 23.39 & 0.200 & 479 & 12 \\
% CompGS 32K~\cite{navaneet2023compact3d} & 0.838 & 23.44 & 0.198 & 475 & 13 \\
% Niedermayr et al.~\cite{niedermayr2024compressed} & 0.832 & 23.32 & 0.194 & - & 17 \\
% EAGLES~\cite{girish2023eagles} & 0.840 & 23.37 & 0.200 & 227 & 29 \\
% Papantonakis et al.~\cite{papantonakis2024reducing} & 0.840 & 23.57 & 0.188 & 433 & 14 \\
% Trimming the Fat~\cite{salman2024trimming} & 0.831 & 23.69 & 0.210 & 510 & 9 \\
% Efficientgs~\cite{liu2024efficientgs} & 0.837 & 23.45 & 0.197 & 439 & 33 \\
% RDO-Gaussian~\cite{wang2024end} & 0.835 & 23.34 & 0.195 & 269 & 12 \\
% PUP 3D-GS~\cite{hanson2024pup} & 0.807 & 23.03 & 0.245 & 418 & 50 \\
% Kim et al~\cite{kim2024color} & 0.830 & 23.18 & 0.198 & 231 & 42 \\

%     \midrule    \midrule

%     \multicolumn{6}{c}{\textbf{Deep Blending}}\\
%     \midrule    \midrule

% 3DGS-7k~\cite{kerbl20233d} & 0.875 & 27.78 & 0.317 & 172 & 386 \\
% 3DGS-30k~\cite{kerbl20233d} & 0.903 & 29.41 & 0.243 & 137 & 676 \\

% Compact3D~\cite{lee2023compact} & 0.901 & 29.79 & 0.258 & 181 & 43 \\
% CompGS 16K~\cite{navaneet2023compact3d} & 0.906 & 29.90 & 0.252 & 485 & 12 \\
% CompGS 32K~\cite{navaneet2023compact3d} & 0.907 & 29.90 & 0.251 & 484 & 13 \\
% Niedermayr et al.~\cite{niedermayr2024compressed} & 0.898 & 29.38 & 0.253 & - & 25 \\
% EAGLES~\cite{girish2023eagles} & 0.910 & 29.86 & 0.250 & 130 & 52 \\
% Papantonakis et al.~\cite{papantonakis2024reducing} & 0.902 & 29.63 & 0.249 & 360 & 18 \\
% Trimming the Fat~\cite{salman2024trimming} & 0.897 & 29.43 & 0.267 & 440 & 13 \\
% Efficientgs~\cite{liu2024efficientgs} & 0.903 & 29.63 & 0.251 & 401 & 40 \\
% RDO-Gaussian~\cite{wang2024end} & 0.902 & 29.63 & 0.252 & 207 & 18 \\
% PUP 3D-GS~\cite{hanson2024pup} & 0.881 & 28.61 & 0.305 & 296 & 81 \\
% Kim et al~\cite{kim2024color} & 0.902 & 29.71 & 0.255 & 208 & 72 \\




%     \bottomrule
%     \end{tabular}
%     }
% \end{table*}


% \begin{table*}[]
%     \caption{Performance and compression comparison of 3DGS baseline with unstructured compression methods on Mip-NeRF360 and Tanks\&Temples datasets.}
%     \label{tab:unstruc_mip_tandt}
%     \centering
%     \small
%     \resizebox{0.95\textwidth}{!}{
%     \begin{tabular}{l|ccccc|ccccc}
%     \toprule
%     &  \multicolumn{5}{c}{\textbf{Mip-NeRF360~\cite{barron2022mip}}} & \multicolumn{5}{c}{\textbf{Tanks\&Temples~\cite{knapitsch2017tanks}}} \\
%     \textbf{Model} & \textbf{SSIM$^\uparrow$} & \textbf{PSNR$^\uparrow$} & \textbf{LPIPS$^\downarrow$} & \textbf{FPS}$^\uparrow$ & \textbf{Mem.}$^\downarrow$ & \textbf{SSIM$^\uparrow$} & \textbf{PSNR$^\uparrow$} & \textbf{LPIPS$^\downarrow$} & \textbf{FPS}$^\uparrow$ & \textbf{Mem.}$^\downarrow$\\
%     \midrule

% 3DGS-7k~\cite{kerbl20233d} & 0.770 & 25.60 & 0.279 & 160 & 523.0 & 0.767 & 21.20 & 0.280 & 197 & 270.0 \\ 
% 3DGS-30k~\cite{kerbl20233d} & \cellcolor{yellow!25}0.815 & 27.21 & \cellcolor{red!25}0.214 & 134 & 734.0 & \cellcolor{red!25}0.841 & 23.14 & \cellcolor{red!25} 0.183 & 154 & 411.0 \\
% \midrule
% LightGaussian~\cite{fan2023lightgaussian} & 0.805 & \cellcolor{yellow!25} 27.28 & 0.243 & 209 & 42.0 & 0.817 & 23.11 & 0.231 & 209 & 22.0 \\
% Compact3D~\cite{lee2023compact} & 0.798 & 27.08 & 0.247 & 128 & 48.8 & 0.831 & 23.32 & 0.201 & 185 & 39.4 \\ 
% CompGS 16K~\cite{navaneet2023compact3d} & 0.804 & 27.03 & 0.243 & \cellcolor{yellow!25}346 & \cellcolor{red!25}18.0 & 0.836 & 23.39 & 0.200 & \cellcolor{yellow!25} 479 & \cellcolor{yellow!25} 12.0 \\ 
% CompGS 32K~\cite{navaneet2023compact3d} & 0.806 & 27.12 & 0.240 & \cellcolor{red!25}344 & \cellcolor{yellow!25}19.0 & 0.838 & 23.44 & 0.198 & \cellcolor{green!25} 475 & \cellcolor{green!25}13.0 \\ 
% Niedermayr et al.~\cite{niedermayr2024compressed} & 0.801 & 26.98 & 0.238 & 113 & 28.8 & 0.832 & 23.32 & \cellcolor{green!25} 0.194 & 149 & 17.3 \\ 
% EAGLES~\cite{girish2023eagles} & \cellcolor{green!25}0.810 & \cellcolor{green!25}27.23 & 0.240 & 131 & 54.0 & \cellcolor{green!25}0.840 & 23.37 & 0.200 & 227 & 29.0 \\ 
% Papantonakis et al.~\cite{papantonakis2024reducing} & 0.809 & 27.10 & \cellcolor{green!25}0.226 & \cellcolor{green!25}284 & 29.0 & \cellcolor{yellow!25}0.840 & \cellcolor{yellow!25}23.57 & \cellcolor{yellow!25} 0.188 & 433 & 14.0 \\ 
% Trimming the Fat~\cite{salman2024trimming} & 0.798 & 27.13 & 0.248 & 210 & \cellcolor{green!25}20.1 & 0.831 & \cellcolor{red!25} 23.69 & 0.210 & \cellcolor{red!25}510 & \cellcolor{red!25} 8.6 \\ 
% Efficientgs~\cite{liu2024efficientgs} & \cellcolor{red!25}0.817 & \cellcolor{red!25}27.38 & \cellcolor{yellow!25}0.216 & 218 & 98.0 & 0.837 & \cellcolor{green!25}23.45 & 0.197 & 439 & 33.0 \\ 
% RDO-Gaussian~\cite{wang2024end} & 0.802 & 27.05 & 0.239 & 191 & 23.5 & 0.835 & 23.34 & 0.195 & 269 & \cellcolor{yellow!25} 12.0 \\ 
% PUP 3D-GS~\cite{hanson2024pup} & 0.792 & 26.83 & 0.268 & 244 & 86.3 & 0.807 & 23.03 & 0.245 & 418 & 50.1 \\ 
% Kim et al.~\cite{kim2024color} & 0.797 & 27.07 & 0.249 & 166 & 73.0 & 0.83 & 23.18 & 0.198 & 231 & 42.0 \\ 




%     \bottomrule
%     \end{tabular}
%     }
% \end{table*}





% \begin{table}[]
%     \caption{Performance and compression comparison of 3DGS baseline with unstructured compression methods on Deep Blending dataset.}
%     \label{tab:unstruc_db}
%     \centering
%     \small
%     \resizebox{0.49\textwidth}{!}{

%     \begin{tabular}{l|ccccc}
%     \toprule
%     &  \multicolumn{5}{c}{\textbf{Deep Blending~\cite{hedman2018deep}}}  \\
%     \textbf{Model} & \textbf{SSIM$^\uparrow$} & \textbf{PSNR$^\uparrow$} & \textbf{LPIPS$^\downarrow$} & \textbf{FPS}$^\uparrow$ & \textbf{Mem.}$^\downarrow$ \\
%     \midrule
% 3DGS-7k~\cite{kerbl20233d} & 0.875 & 27.78 & 0.317 & 172 & 386.0 \\
% 3DGS-30k~\cite{kerbl20233d} & 0.903 & 29.41 & \cellcolor{red!25} 0.243 & 137 & 676.0 \\
% \midrule
% Compact3D~\cite{lee2023compact} & 0.901 & \cellcolor{green!25} 29.79 & 0.258 & 181 & 43.2 \\
% CompGS 16K~\cite{navaneet2023compact3d} & \cellcolor{green!25} 0.906 & \cellcolor{red!25}29.90 & 0.252 & \cellcolor{red!25} 485 & \cellcolor{red!25}12.0 \\
% CompGS 32K~\cite{navaneet2023compact3d} & \cellcolor{yellow!25} 0.907 & \cellcolor{red!25} 29.90 & 0.251 & \cellcolor{yellow!25} 484 & \cellcolor{green!25} 13.0 \\
% Niedermayr et al.~\cite{niedermayr2024compressed} & 0.898 & 29.38 & 0.253 &128 & 25.3 \\
% EAGLES~\cite{girish2023eagles} & \cellcolor{red!25} 0.910 & \cellcolor{yellow!25} 29.86 & \cellcolor{green!25} 0.250 & 130 & 52.0 \\
% Papantonakis et al.~\cite{papantonakis2024reducing} & 0.902 & 29.63 & \cellcolor{yellow!25} 0.249 & 360 & 18.0 \\
% Trimming the Fat~\cite{salman2024trimming} & 0.897 & 29.43 & 0.267 & \cellcolor{green!25} 440 & \cellcolor{yellow!25} 12.5 \\
% Efficientgs~\cite{liu2024efficientgs} & 0.903 & 29.63 & 0.251 & 401 & 40.0 \\
% RDO-Gaussian~\cite{wang2024end} & 0.902 & 29.63 & 0.252 & 207 & 18.0 \\
% PUP 3D-GS~\cite{hanson2024pup} & 0.881 & 28.61 & 0.305 & 296 & 80.8 \\
% Kim et al.~\cite{kim2024color} & 0.902 & 29.71 & 0.255 & 208 & 72.0 \\ 


%     \bottomrule
%     \end{tabular}
% }
% \end{table}






% \textit{EfficientGS~\cite{liu2024efficientgs}} introduced a pruning method and an enhanced densification strategy to address the overexpansion of Gaussians seen in the baseline 3DGS model. The densification process was refined by halting unnecessary cloning and splitting of Gaussians based on the cumulative gradient sum, thereby preventing the creation of redundant Gaussians. This was followed by pruning to remove excess Gaussians and reduce redundancy in the SH features of the remaining Gaussians. Additionally, EfficientGS proposed a technique to lower the SH order for redundant Gaussians, further optimizing the model. These improvements led to a significant reduction in training time and a faster rendering rate, achieving approximately a 10$\times$ compression ratio compared to the baseline 3DGS.

% \textit{RDO-Gaussian~\cite{wang2024end}}, inspired by advancements in learned image compression (LIC)~\cite{balle2016end,ali2024towards}, was among the first to introduce an end-to-end compression method with dynamic rate control specifically for 3DGS. This approach implemented a dynamic pruning method coupled with an entropy-constrained vector quantization (ECVQ) module, which jointly optimized both the rate (compression efficiency) and distortion (quality loss). Additionally, RDO-Gaussian employed learnable parameters to model the colors of different regions, enhancing the overall fidelity of the compressed representation. The proposed method achieved a substantial compression rate of approximately 40× compared to the baseline 3DGS, significantly reducing storage requirements while maintaining high-quality rendering.

% \textit{RTGS~\cite{lin2024rtgs}} is a 3DGS based Point-Based Neural Rendering (PBNR) system, pioneering the delivery of neural rendering on edge devices while maintaining high fidelity. RTGS incorporates efficiency-aware pruning techniques to enhance rendering speed, making it suitable for resource-constrained environments. Additionally, it introduces a Foveated Rendering (FR)~\cite{guenter2012foveated, patney2016towards} method, which focuses computational resources on areas of the image that are within the viewer's direct gaze, further boosting FPS. According to human evaluations, RTGS was able to achieve over 100 FPS on the NVIDIA Jetson Xavier without any perceptual loss in video quality, demonstrating its effectiveness for real-time applications on edge devices.

% \textit{Kheradmand et al.~\cite{kheradmand20243d}} reinterpreted 3D Gaussians as samples from a probability distribution representing the underlying scene, akin to a Markov Chain Monte Carlo (MCMC) process. With this perspective, the update of Gaussians is performed using Stochastic Gradient Langevin Dynamics (SGLD)~\cite{kheradmand2024accelerating,brosse2018promises}, which introduces noise to the optimization process. In this framework, the traditional pruning and densification steps are reimagined: pruning becomes a regularization process that removes less important Gaussians, while densification is replaced by a relocalization scheme, adjusting the Gaussians' positions. This approach enhances rendering quality by offering better control over both the initialization and the total number of Gaussians.

% \textit{PUP 3D-GS~\cite{hanson2024pup}} introduced a score-based post-hoc pruning method that leverages spatial sensitivity to optimize Gaussian pruning. This method calculates a second-order approximation of the reconstruction error concerning the spatial parameters of each Gaussian on training views. Additionally, PUP 3D-GS proposed a multi-round prune-refine pipeline that operates without modifying the training process, making it compatible with any pre-trained 3DGS model. Using this approach, PUP 3D-GS was able to prune 88.44\% of Gaussians, resulting in a 2.65$\times$ increase in rendering speed.

% \textit{LP-3DGS~\cite{zhang2024lp}} introduced a learning-to-prune technique that utilizes a trainable binary mask applied to the importance score to automatically determine the optimal pruning ratio for 3DGS. Unlike the traditional straight-through estimator (STE) method, LP-3DGS modifies the masking function to leverage the Gumbel-Sigmoid approach, allowing for a more accurate gradient approximation while maintaining compatibility with the existing 3DGS training process. This approach allows LP-3DGS to determine the optimal Gaussian point size for each scene in a single training session, rather than relying on a fixed model size.

% \textit{Morgenstern et al.~\cite{morgenstern2023compact}} introduced a compact scene representation technique that arranges the 3DGS parameters into a 2D grid based on local homogeneity, significantly reducing storage requirements without compromising rendering quality. Their method capitalizes on the perceptual overlaps inherent in natural scenes, where various Gaussian parameter combinations can represent the scene similarly. To maintain the neighborhood structure and organize the high-dimensional Gaussian parameters into a 2D grid, they proposed a novel highly parallel method called Parallel Linear Assignment Sorting. This approach reinforces local smoothness between the grid's sorted parameters during training. The uncompressed Gaussians maintain compatibility with standard renderers, as they adhere to the original 3DGS structure. For complex scenes, their method achieves a reduction in size by 17$\times$ to 42$\times$ while maintaining the same training time.

% \textit{Kim et al.~\cite{kim2024color}} addressed the over-densification issue in 3DGS by developing a more compact Gaussian model that maintains image quality while reducing data size and GPU memory usage. Their approach introduces the use of color cues by incorporating the SH gradient alongside the 2D position gradient, which was the focus of the original method. By leveraging both positional and color information, the proposed method resolves the inefficiencies found in the original densification process. This strategy aligns with an expanded densification scheme, enabling a reduction in data size by at least 9$\times$ while preserving perceptual quality. Additionally, it cuts GPU memory utilization by 1.5$\times$ and enhances both training time efficiency and rendering performance.

% \textit{Mini-Splatting~\cite{fang2024mini}} identifies two critical issues in the vanilla 3DGS: ``overlapping'' and ``under-reconstruction''. These problems result in an inefficient distribution of Gaussians, which ultimately limits both the rendering quality and speed. The challenge lies in achieving an optimal, minimal Gaussian representation without compromising the quality of rendering. Instead of relying on explicit pruning, Mini-Splatting introduces a novel approach that reorganizes the spatial positions of Gaussians through densification and simplification techniques. The densification process includes ``blur split'' and ``depth reinitialization'', which enhance the density of Gaussians by adjusting their distribution based on screen-space and world-space information. This ensures that the Gaussians are more effectively spread around the object, addressing under-reconstruction issues. The simplification process involves ``intersection preservation'' and ``Gaussian sampling'', which help to maintain a balance between the number of Gaussians and the quality of the rendered scene. By combining these densification and simplification strategies with related additional processing, Mini-Splatting achieves a balanced trade-off between rendering quality, resource usage, and storage. This approach leads to a more efficient and effective Gaussian representation, improving both rendering speed and quality while optimizing resource consumption.


\subsection{Discussion}
We analyze the performance of various unstructured compression techniques for 3DGS across the Mip-NeRF360, Tanks\&Temples, and Deep Blending datasets, focusing on three key aspects: compression ratio, fidelity, and FPS.

\noindent
\textbf{Fidelity: }EfficientGS~\cite{liu2024efficientgs}, Papantonakis et al.~\cite{papantonakis2024reducing}, and EAGLES~\cite{girish2023eagles} across all datasets exhibit strong performance, maintaining high SSIM, PSNR, and low LPIPS scores, indicating minimal perceptual degradation as shown in Tables~\ref{tab:unstruc_mip_tandt} and~\ref{tab:unstruc_db}. ELMGS-medium~\cite{ali2024elmgs} provides a compelling balance between compression and quality, notably achieving 24.08dB in PSNR on Tanks\&Temples outperforming all the other methods. On the other hand, highly compressed techniques such as Trimming the Fat~\cite{salman2024trimming}, CompGS~\cite{navaneet2023compact3d}, and ELMGS-small~\cite{ali2024elmgs} exhibit slightly elevated LPIPS scores, suggesting perceptual quality loss despite their strong numerical PSNR performance. These results emphasize that while EfficientGS,  Papantonakis et al~\cite{papantonakis2024reducing}, and EAGLES provide the best fidelity, ELMGS and Trimming the Fat offer promising trade-offs between compression efficiency and reconstruction quality.


\begin{figure*}
    \centering
    \includegraphics[width=0.9\linewidth]{Figures/scaffoldgs.pdf}
    \caption{\textbf{Overview of the ScaffoldGS pipeline.} (a) A sparse voxel grid is initialized from Structure-from-Motion (SfM)-derived points, with each \textbf{anchor} placed at the voxel center and assigned a learnable scale, capturing the scene’s occupancy. (b) Within the view frustum, \textbf{k Gaussians} are generated from each \textit{visible anchor} with offsets \(\{O_k\}\), and their attributes—including opacity, color, scale, and quaternion—are decoded from the anchor feature. (c) To enhance efficiency and reduce redundancy, only significant Gaussians (\(\alpha \geq \tau_\alpha\)) are rasterized. The final rendered image is optimized using reconstruction loss. Figure taken from~\cite{lu2023scaffold}.}
    \label{fig:scaffoldgs}
    \vspace{-0.5cm}
\end{figure*}

\noindent
\textbf{Compression Ratio: } %Compression is crucial for reducing storage and bandwidth requirements. 
The results in Tables~\ref{tab:unstruc_mip_tandt} and~\ref{tab:unstruc_db} reveal significant variability in compression performance across methods and Figure~\ref{fig:unstruc_memvspsnr} shows rate-distortion tradeoff. ELMGS-small demonstrates the highest compression rates, achieving a 55$\times$ reduction on the Deep Blending dataset and 35$\times$  on Tanks\&Temples, making it one of the most storage-efficient methods. Similarly, CompGS-16K and CompGS-32K provide compression factors of 39$\times$ –56$\times$ , albeit at a minor fidelity trade-off. Notably, Trimming the Fat achieves a 48$\times$  compression on Tanks\&Temples while maintaining a competitive balance between storage complexity and reconstruction quality. 

\noindent
\textbf{FPS: }Frame rate is a crucial factor in real-time rendering applications, ensuring whether a method is viable for interactive environments such as virtual reality and gaming or low-power edge devices. The rendering speeds are calculated on a single NVIDIA A40 GPU. The FPS vs. PSNR plots in Figure~\ref{fig:unstruc_fpsvspsnr} highlight a clear trade-off between rendering speed and reconstruction accuracy. ELMGS, Trimming the Fat, and Papantonakis et al.~\cite{papantonakis2024reducing} achieve the highest FPS values, with ELMGS surpassing 400 FPS on Deep Blending and 600 FPS on Tanks\&Temples, making it particularly well-suited for real-time applications with limited computational complexity. Compared to the 3DGS-30k baseline all the unstructured compression methods offer a better trade-off between rendering speed and fidelity, making them strong candidates for real-world applications.

EfficientGS on Mip-NeRF360 and EAGLES on deep blending provide the highest fidelity. However, ELMGS offers the most aggressive compression while retaining competitive reconstruction quality. CompGS and Trimming the Fat achieve optimal trade-offs between compression, fidelity, and real-time rendering speed. These findings indicate that no single method dominates all aspects, highlighting the importance of application-specific selection when choosing an appropriate unstructured compression technique for 3DGS.
% \subsection{Discussion} 
% Tables~\ref{tab:unstruc_mip_tandt} and~\ref{tab:unstruc_db} compare unstructured compression methods across various datasets. EfficientGS and EAGLES consistently achieve high SSIM and PSNR scores across all benchmarks. On Mip-NeRF360, EfficientGS attains the highest PSNR (27.38) with an SSIM of 0.817, while EAGLES achieves a PSNR of 27.23 and an SSIM of 0.810. LightGaussian also demonstrates strong fidelity (SSIM = 0.805, PSNR = 27.28), indicating that its compression strategy effectively preserves scene details.
% In the Tanks\&Temples dataset, which features complex real-world scenes, EfficientGS achieves the highest PSNR (23.45), closely followed by Papantonakis et al. (PSNR = 23.69) and EAGLES (PSNR = 23.37). SSIM values remain high across these methods, with EAGLES achieving the best perceptual similarity (SSIM = 0.840). On the Deep Blending dataset, EAGLES (SSIM = 0.910, PSNR = 29.63) and CompGS 16K (SSIM = 0.906, PSNR = 29.90) outperform most other approaches, effectively balancing scene fidelity and perceptual distortion through vector quantization and entropy-aware pruning.
% For LPIPS Compact3D and LightGaussian demonstrated lower scores than other methods. EAGLES, EfficientGS, and CompGS variants maintain competitive LPIPS values, ensuring minimal perceptual artifacts. In contrast, methods that aggressively prune Gaussians, such as Trimming the Fat, exhibit slightly higher LPIPS values, highlighting the trade-off between compression ratio and perceptual fidelity.
% Memory efficiency is a critical factor in 3DGS compression, particularly for large-scale scene rendering. EfficientGS (8.6MB), Trimming the Fat (10.8MB), and Papantonakis et al. (20.1MB) achieve the highest compression rates on the Tanks\&Temples dataset, significantly reducing memory usage compared to 3DGS baselines (270MB for 3DGS-7k and 411MB for 3DGS-30k). Similarly, in the Deep Blending dataset, CompGS 16K (12MB) and Trimming the Fat (10.8MB) achieve the lowest memory footprints, making them well-suited for resource-constrained environments. These findings highlight the importance of entropy-constrained quantization and aggressive pruning in reducing storage requirements while maintaining rendering quality.
% Overall, EfficientGS and EAGLES strike the best balance between reconstruction fidelity and compression efficiency, maintaining high SSIM and PSNR values while minimizing memory consumption. CompGS 16K emerges as a strong performer on the Deep Blending dataset, while Trimming the Fat achieves the most aggressive compression, albeit with slightly higher perceptual distortion. These results underscore the effectiveness of hybrid pruning-quantization approaches in optimizing 3DGS representations for memory-efficient real-time rendering.



% Most of the works discussed above in unstructured compression have followed a similar pipeline involving pruning, quantization, and entropy encoding, albeit with some minor variations. Despite being unstructured, these compression techniques have achieved substantial scene compression (up to 30$\times$) and significantly enhanced rendering speeds as seen from the results in the Table~\ref{tab:unstruc_mip_tandt} and~\ref{tab:unstruc_db} on Mip-Nerf360~\cite{barron2021mip}, Tanks\&Temples~\cite{knapitsch2017tanks} and Deep Blending~\cite{hedman2018deep} datasets.

% \noindent
%\textbf{Identifying Redundancy in 3DGS: }
% Papantonakis et al.~\cite{papantonakis2024reducing} identified the primary causes of redundancy in 3DGS as the excessive number of Gaussians, SH coefficients, and the use of full precision for Gaussian features. To address these issues, various methods have been proposed, including pruning, quantization, entropy encoding, and knowledge distillation. Among these, pruning has emerged as the most common technique for Gaussian removal, with many approaches being closely related in their methodology.

% \noindent
% \textbf{Gaussian Pruning Techniques:} Several methods, such as LightGaussian~\cite{fan2023lightgaussian}, EAGLES~\cite{girish2023eagles}, EfficientGS~\cite{liu2024efficientgs}, and RDO-Gaussian~\cite{wang2024end}, employ a weight-based metric to assess the importance of individual Gaussians. These methods prune Gaussians based on their calculated weight scores, which consider factors like contribution to each pixel, opacity, scale, and transmittance. 

% Other approaches differ slightly in their pruning criteria. For example, CompGS~\cite{navaneet2023compact3d} prunes Gaussians solely based on opacity values, while Trimming the Fat integrates gradient information with opacity values to make more informed pruning decisions. On the other hand, Compact3D~\cite{lee2023compact} combines opacity and scale values, operating on the observation that Gaussians with smaller scales tend to be redundant. Niedermayr et al.~\cite{niedermayr2024compressed} introduce a sensitivity-aware clustering approach to prune less important Gaussians, though their pruning ratio remains relatively modest.

% Papantonakis et al.~\cite{papantonakis2024reducing} propose a distinct approach, focusing on the redundancy score of Gaussians. Instead of relying solely on weight-based metrics, they calculate the redundancy score based on the number of overlaps between Gaussians. If a Gaussian overlap exceeds a specific threshold, it is pruned, reducing redundancy more effectively while maintaining scene fidelity. This method represents a novel direction to address the inherent redundancy in 3DGS.

% \noindent
% \textbf{Impact of Pruning on Rendering Speed: }
% The reduction in the number of Gaussians directly impacts rendering speed, as fewer Gaussians require less computational overhead during rendering. Methods that efficiently remove a large number of Gaussians tend to achieve significantly higher rendering speeds. This correlation is evident from the results in Table~\ref{tab:unstruc_mip_tandt} and~\ref{tab:unstruc_db}, where CompGS~\cite{navaneet2023compact3d}, Trimming the Fat~\cite{salman2024trimming}, Papantonakis et al.~\cite{papantonakis2024reducing}, and EfficientGS~\cite{liu2024efficientgs} demonstrate markedly higher rendering speeds compared to other methods. These approaches excel in pruning redundant or less significant Gaussians, resulting in both optimized memory usage and faster rendering performance.




% \noindent
% \textbf{Integrating Pruning and Quantization: }Integrating various pruning methodologies can significantly improve the efficiency and effectiveness of 3DGS. By combining the post-hoc pruning strategies used in Trimming the Fat~\cite{salman2024trimming}, CompGS~\cite{navaneet2023compact3d}, and Compact3D~\cite{lee2023compact} which focus on opacity, gradient, and scale—one can develop a more nuanced approach that removes Gaussians redundant across multiple dimensions. This can be further refined with sensitivity-aware clustering methods like those proposed by EAGLES~\cite{girish2023eagles}  and Niedermayr et al.~\cite{niedermayr2024compressed}, which identify and prune clusters of minimally contributing Gaussians. Furthermore, incorporating the redundancy score method from Papantonakis et al.~\cite{papantonakis2024reducing}, which evaluates Gaussian overlaps, provides an additional layer of pruning that targets excessive redundancy. This comprehensive approach, starting with post-hoc pruning, followed by sensitivity-aware clustering, and finalized with redundancy scoring, can lead to a more substantial reduction in Gaussian numbers, significantly improving the rendering speed and memory efficiency of 3DGS while maintaining high-quality scene representation.


% To reduce the bit widths of Gaussian features, many of the aforementioned methods have not only employed pruning but also applied codebook vector quantization~\cite{gray1984vector}. The primary advantage of vector quantization over scalar quantization lies in its ability to significantly reduce the size of the quantized models. Vector quantization groups multiple values into vectors before quantizing them, which allows for more efficient compression and thus a smaller memory footprint. Given the objective of minimizing the memory usage of Gaussians, all the discussed approaches have opted for vector codebook quantization to achieve this goal effectively~\cite{fan2023lightgaussian,lee2023compact,navaneet2023compact3d,niedermayr2024compressed,girish2023eagles,papantonakis2024reducing,wang2024end}. The VQ approach ensures that the memory requirements are kept as low as possible without sacrificing the quality of the 3D scene representation.

% \noindent
% \textbf{Rate-Distortion Optimization: } Aside from RDO-Gaussian~\cite{wang2024end}, most of the other works follow a conventional pipeline of pruning, quantizing, and entropy encoding based on widely used compression algorithms, such as the LZ77 algorithm. This traditional approach focuses on reducing the number of Gaussians, lowering bit widths through vector quantization, and then applying entropy encoding to compress the data further.

% However, RDO-Gaussian~\cite{wang2024end} takes a different approach by designing its entire framework around the rate-distortion optimization (RDO) principle. In this method, they do not just prune Gaussians but also introduce an entropy-focused quantization module that allows them to optimize the Gaussians specifically for different bit rates~\cite{balle2018variational}. 

% Given the advantages of such a rate-distortion framework, it is likely that rate-distortion optimized Gaussians will play a significant role in the future of 3DGS compression. This approach offers a more sophisticated and adaptive means of balancing the compression rate with the quality of the rendered scenes, making it a promising direction for future research and development in the field.

\noindent
\textbf{Challenges and Future Direction}
Despite substantial progress in compressing Gaussian splats, the majority of research has primarily concentrated on reducing the storage footprint of 3DGS. A critical challenge that remains unaddressed is the densification of 3DGS, particularly for large scenes like those in the Mip-NeRF dataset~\cite{barron2022mip}, such as the \texttt{garden} and \texttt{bicycle} scenes, which can demand up to 15GB of GPU memory for training and rendering. This level of scaling is unsustainable for large-scale scenes, indicating a pressing need for unstructured 3DGS compression techniques to tackle this issue.

Furthermore, most of the existing compression methods rely on vector quantization. However, recent advances in model compression for deep learning have demonstrated that scalar quantization is more favorable and easier to implement in hardware, especially for low-powered edge devices~\cite{gholami2022survey,nascimento2023hyperblock}. Therefore, developing specialized scalar quantization techniques for efficiently rendering Gaussian splats on such devices should be a priority.

Additionally, none of the current compression works have investigated the behavior of loss functions in relation to different compression methods. In image compression, it has been shown that altering the loss function can significantly impact compression performance~\cite{ali2024towards}. Thus, future research should explore the effects of various loss functions, such as perceptual loss and edge-based loss, on the performance of 3DGS compression. This could lead to more efficient and perceptually optimized compression techniques for 3DGS.

% \begin{table*}[]
%     \caption{Performance and compression comparison of 3DGS baseline with structured compression methods on Mip-NeRF360 and Tanks\&Temples datasets.}
%     \label{tab:structured_mip_tandt}
%     \centering
%     \small
%     \resizebox{0.95\textwidth}{!}{
%     \begin{tabular}{l|ccccc|ccccc}
%     \toprule
%     &  \multicolumn{5}{c}{\textbf{Mip-NeRF360~\cite{barron2022mip}}} & \multicolumn{5}{c}{\textbf{Tanks\&Temples~\cite{knapitsch2017tanks}}} \\
%     \textbf{Model} & \textbf{SSIM$^\uparrow$} & \textbf{PSNR$^\uparrow$} & \textbf{LPIPS$^\downarrow$} & \textbf{FPS}$^\uparrow$ & \textbf{Mem.}$^\downarrow$ & \textbf{SSIM$^\uparrow$} & \textbf{PSNR$^\uparrow$} & \textbf{LPIPS$^\downarrow$} & \textbf{FPS}$^\uparrow$ & \textbf{Mem.}$^\downarrow$\\
%     \midrule
    
% 3DGS-7k~\cite{kerbl20233d} & 0.770 & 25.60 & 0.279 & \cellcolor{yellow!25} 160 & 523.0 & 0.767 & 21.20 & 0.280 & \cellcolor{red!25}197 & 270.0 \\
% 3DGS-30k~\cite{kerbl20233d} & 0.815 & 27.21 & \cellcolor{yellow!25}0.214 & 134 & 734.0 & 0.841 & 23.14 & \cellcolor{green!25} 0.183 & 154 & 411.0 \\
% ScaffoldGS~\cite{lu2023scaffold}            & \cellcolor{red!25}0.848 & \cellcolor{red!25}28.84 & 0.220 & \cellcolor{green!25}156 & 102.0  & \cellcolor{yellow!25}0.853 & 23.96 & \cellcolor{yellow!25}0.177 & 110 & 87.0  \\
% GF\_Large~\cite{zhang2024gaussian}   & 0.803 & 27.45 & \cellcolor{red!25}0.212 & 105 & 85.0 & 0.839 & 23.67 & 0.188 & \cellcolor{green!25}164 & 45.0 \\ 
% GF\_Small~\cite{zhang2024gaussian}   & 0.797 & 27.33 & \cellcolor{green!25}0.219 & 121 & 50.0 & 0.836 & 23.56 & 0.194 & \cellcolor{yellow!25}175 & 38.0 \\
% SUNDAE (30\%)~\cite{yang2024spectrally}            & \cellcolor{yellow!25} 0.826 & 27.24 & 0.228 & 109 & 279.0  & 0.817 & 23.46 & 0.242 & 116 & 148.0  \\
% SUNDAE (1\%)~\cite{yang2024spectrally}             & 0.716 & 24.70 & 0.375 & \cellcolor{red!25}171 & 38.0   & 0.703 & 20.49 & 0.375 & 127 & 33.0   \\
% HAC-lowrate~\cite{chen2024hac}             & 0.807 & 27.53 & 0.238 & - & \cellcolor{green!25} 15.3  & 0.846 & \cellcolor{green!25}24.04 & 0.187 & - & \cellcolor{green!25}8.1  \\
% HAC-highrate~\cite{chen2024hac}            & \cellcolor{green!25}0.811 & \cellcolor{yellow!25}27.77 & 0.230 & - & 21.9  & \cellcolor{yellow!25}0.853 & \cellcolor{red!25}24.40 & \cellcolor{yellow!25}0.177 & - & 11.2 \\
% ContextGS (lowrate)~\cite{wang2024contextgs}     & 0.808 & 27.62 & 0.237 & - & \cellcolor{yellow!25} 12.7  & \cellcolor{green!25}0.852 & 24.20 & 0.184 & - & \cellcolor{yellow!25}7.1  \\
% ContextGS (highrate)~\cite{wang2024contextgs}    & \cellcolor{green!25}0.811 & \cellcolor{green!25}27.75 & 0.231 & - & 18.4 & \cellcolor{red!25}0.855 & \cellcolor{yellow!25}24.29 & \cellcolor{red!25}0.176 & - & 11.8  \\
% CompGS (highrate)~\cite{liu2024compgs}       & 0.800 & 27.26 & 0.240  & - & 16.5  & 0.840 & 23.70 & 0.210  & - & 9.6   \\
% CompGS (lowrate)~\cite{liu2024compgs}        & 0.780 & 26.37 & 0.280  & - & \cellcolor{red!25} 8.8   & 0.810 & 23.11 & 0.240  & - & \cellcolor{red!25}5.9   \\




%     \bottomrule
%     \end{tabular}
%     }
% \end{table*}

\begin{table*}[]
    \caption{Performance and compression comparison of the 3DGS baseline with structured compression methods on the Mip-NeRF360 and Tanks\&Temples datasets. All values are sourced from their respective papers. Memory size is reported in megabytes (MB). The best-performing results are highlighted in red, followed by yellow and green.}
    \label{tab:structured_mip_tandt}
    \centering
    \small
    \resizebox{0.95\textwidth}{!}{
    \begin{tabular}{l|ccccc|ccccc}
    \toprule
    &  \multicolumn{5}{c}{\textbf{Mip-NeRF360~\cite{barron2022mip}}} & \multicolumn{5}{c}{\textbf{Tanks\&Temples~\cite{knapitsch2017tanks}}} \\
    \textbf{Model} & \textbf{SSIM$^\uparrow$} & \textbf{PSNR$^\uparrow$} & \textbf{LPIPS$^\downarrow$} & \textbf{Mem.}$^\downarrow$ & \textbf{Comp.$^\uparrow$} & \textbf{SSIM$^\uparrow$} & \textbf{PSNR$^\uparrow$} & \textbf{LPIPS$^\downarrow$} & \textbf{Mem.}$^\downarrow$ & \textbf{Comp.$^\uparrow$} \\
    \midrule
    
%3DGS-7k~\cite{kerbl20233d} & 0.770 & 25.60 & 0.279 & 523.0 & 1.40 & 0.767 & 21.20 & 0.280 & 270.0 & 1.52 \\
3DGS-30k~\cite{kerbl20233d} & 0.815 & 27.21 & \cellcolor{yellow!25}0.214 & 734.0 & 1$\times$ & 0.841 & 23.14 & \cellcolor{green!25}0.183 & 411.0 & 1$\times$\\
\midrule
ScaffoldGS~\cite{lu2023scaffold}            & \cellcolor{red!25}0.848 & \cellcolor{red!25}28.84 & 0.220 & 102.0  & 7$\times$ & \cellcolor{yellow!25}0.853 & 23.96 & \cellcolor{yellow!25}0.177 & 87.0  & 5$\times$ \\
GF\_Large~\cite{zhang2024gaussian}   & 0.803 & 27.45 & \cellcolor{red!25}0.212 & 85.0 & 9$\times$ & 0.839 & 23.67 & 0.188 & 45.0 & 9$\times$ \\ 
GF\_Small~\cite{zhang2024gaussian}   & 0.797 & 27.33 & \cellcolor{green!25}0.219 & 50.0 & 15$\times$ & 0.836 & 23.56 & 0.194 & 38.0 & 11$\times$ \\
SUNDAE (30\%)~\cite{yang2024spectrally}            & \cellcolor{yellow!25} 0.826 & 27.24 & 0.228 & 279.0  & 3$\times$ & 0.817 & 23.46 & 0.242 & 148.0  & 3$\times$ \\
SUNDAE (1\%)~\cite{yang2024spectrally}             & 0.716 & 24.70 & 0.375 & 38.0   & 19$\times$ & 0.703 & 20.49 & 0.375 & 33.0   & 13$\times$ \\
HAC-lowrate~\cite{chen2024hac}             & 0.807 & 27.53 & 0.238 & \cellcolor{green!25}15.3  & \cellcolor{green!25}48$\times$ & 0.846 & \cellcolor{green!25}24.04 & 0.187 & \cellcolor{green!25}8.1  & \cellcolor{green!25}51$\times$ \\
HAC-highrate~\cite{chen2024hac}            & \cellcolor{green!25}0.811 & \cellcolor{yellow!25}27.77 & 0.230 & 21.9  & 34$\times$ & \cellcolor{yellow!25}0.853 & \cellcolor{red!25}24.40 & \cellcolor{yellow!25}0.177 & 11.2 & 37$\times$ \\
ContextGS (lowrate)~\cite{wang2024contextgs}     & 0.808 & 27.62 & 0.237 & \cellcolor{yellow!25}12.7  & \cellcolor{yellow!25}58$\times$ & \cellcolor{green!25}0.852 & 24.20 & 0.184 & \cellcolor{yellow!25}7.1  & \cellcolor{yellow!25}58$\times$ \\
ContextGS (highrate)~\cite{wang2024contextgs}    & \cellcolor{green!25}0.811 & \cellcolor{green!25}27.75 & 0.231 & 18.4 & 40$\times$ & \cellcolor{red!25}0.855 & \cellcolor{yellow!25}24.29 & \cellcolor{red!25}0.176 & 11.8  & 35$\times$ \\
CompGS (highrate)~\cite{liu2024compgs}       & 0.800 & 27.26 & 0.240  & 16.5  & 45$\times$ & 0.840 & 23.70 & 0.210  & 9.6   & 43$\times$ \\
CompGS (lowrate)~\cite{liu2024compgs}        & 0.780 & 26.37 & 0.280  & \cellcolor{red!25}8.8   & \cellcolor{red!25}83$\times$ & 0.810 & 23.11 & 0.240  & \cellcolor{red!25}5.9   & \cellcolor{red!25}70$\times$ \\

    \bottomrule
    \end{tabular} % Ensure column alignment matches the tabular preamble
    }
    \vspace{-0.2cm}
\end{table*}



\begin{table}[ht]
    \caption{Performance and compression comparison of the 3DGS baseline with structured compression methods on the Deep Blending dataset. All values are sourced from their respective papers. Memory size is reported in megabytes (MB). The best-performing results are highlighted in red, followed by yellow and green.}
    \label{tab:struct_db}
    \centering
    \small
    \resizebox{0.49\textwidth}{!}{

    \begin{tabular}{l|ccccc}
    \toprule
    &  \multicolumn{5}{c}{\textbf{Deep Blending~\cite{hedman2018deep}}}  \\
    \textbf{Model} & \textbf{SSIM$^\uparrow$} & \textbf{PSNR$^\uparrow$} & \textbf{LPIPS$^\downarrow$} & \textbf{Mem.}$^\downarrow$ & \textbf{Comp.$^\uparrow$} \\
    \midrule
%3DGS-7k~\cite{kerbl20233d} & 0.875 & 27.78 & 0.317 & 386.0 & 2  \\
3DGS-30k~\cite{kerbl20233d} & 0.903 & 29.41 & \cellcolor{green!25}0.243 & 676.0 & 1$\times$  \\
\midrule
ScaffoldGS~\cite{lu2023scaffold} & 0.906 & \cellcolor{green!25}30.21 & 0.254 & 66.0 & 10$\times$  \\
GF\_Large~\cite{zhang2024gaussian} & \cellcolor{yellow!25}0.908 & 30.18 & \cellcolor{red!25}0.215 & 98.0 & 7$\times$  \\
GF\_Small~\cite{zhang2024gaussian} & 0.905 & 30.11 & \cellcolor{yellow!25}0.223 & 64.0 & 11$\times$  \\
SUNDAE (30\%)~\cite{yang2024spectrally} & 0.899 & 29.40 & 0.248 & 203.0 & 3$\times$  \\
SUNDAE (1\%)~\cite{yang2024spectrally} & 0.861 & 26.57 & 0.355 & 36.0 & 19$\times$  \\
HAC-lowrate~\cite{chen2024hac} & 0.902 & 29.98 & 0.269 & \cellcolor{yellow!25}4.4 & \cellcolor{yellow!25}154$\times$  \\
HAC-highrate~\cite{chen2024hac} & 0.906 & \cellcolor{yellow!25}30.34 & 0.258 & 6.4 & 106$\times$  \\
ContextGS (lowrate)~\cite{wang2024contextgs} & \cellcolor{green!25}0.907 & 30.11 & 0.265 & \cellcolor{red!25}3.6 & \cellcolor{red!25}188$\times$  \\
ContextGS (highrate)~\cite{wang2024contextgs} & \cellcolor{red!25}0.909 & \cellcolor{red!25}30.39 & 0.258 & 6.6 & 102$\times$  \\
CompGS (highrate)~\cite{liu2024compgs}  & 0.900 & 29.69 & 0.280 & 8.8 & 77$\times$  \\
CompGS (lowrate)~\cite{liu2024compgs}  & 0.900 & 29.30 & 0.290 & \cellcolor{green!25}6.0 & \cellcolor{green!25}113$\times$  \\

    \bottomrule
    \end{tabular}
}
\vspace{-0.6cm}
\end{table}


% \begin{table}[]
%     \caption{Performance and compression comparison of 3DGS baseline with structured compression methods on Deep Blending dataset.}
%     \label{tab:struct_db}
%     \centering
%     \small
%     \resizebox{0.49\textwidth}{!}{

%     \begin{tabular}{l|ccccc}
%     \toprule
%     &  \multicolumn{5}{c}{\textbf{Deep Blending~\cite{hedman2018deep}}}  \\
%     \textbf{Model} & \textbf{SSIM$^\uparrow$} & \textbf{PSNR$^\uparrow$} & \textbf{LPIPS$^\downarrow$} & \textbf{FPS}$^\uparrow$ & \textbf{Mem.}$^\downarrow$ \\
%     \midrule
% 3DGS-7k~\cite{kerbl20233d} & 0.875 & 27.78 & 0.317 & \cellcolor{red!25}172 & 386.0 \\
% 3DGS-30k~\cite{kerbl20233d} & 0.903 & 29.41 & \cellcolor{green!25}0.243 & 137 & 676.0 \\
% ScaffoldGS~\cite{lu2023scaffold} & 0.906 & \cellcolor{green!25}30.21 & 0.254 & 139 & 66.0 \\
% GF\_Large~\cite{zhang2024gaussian} & \cellcolor{yellow!25}0.908 & 30.18 & \cellcolor{red!25}0.215 & 96 & 98.0 \\
% GF\_Small~\cite{zhang2024gaussian} & 0.905 & 30.11 & \cellcolor{yellow!25}0.223 & 107 & 64.0 \\
% SUNDAE (30\%)~\cite{yang2024spectrally} & 0.899 & 29.40 & 0.248 & \cellcolor{green!25}156 & 203.0 \\
% SUNDAE (1\%)~\cite{yang2024spectrally} & 0.861 & 26.57 & 0.355 & \cellcolor{yellow!25}165 & 36.0 \\
% HAC-lowrate~\cite{chen2024hac} & 0.902 & 29.98 & 0.269 & - & \cellcolor{yellow!25}4.4 \\
% HAC-highrate~\cite{chen2024hac} & 0.906 & \cellcolor{yellow!25}30.34 & 0.258 & - & 6.4 \\
% ContextGS (lowrate)~\cite{wang2024contextgs} & \cellcolor{green!25}0.907 & 30.11 & 0.265 & - & \cellcolor{red!25}3.6 \\
% ContextGS (highrate)~\cite{wang2024contextgs} & \cellcolor{red!25}0.909 & \cellcolor{red!25}30.39 & 0.258 & - & 6.6 \\

% CompGS (highrate)~\cite{liu2024compgs}  & 0.900 & 29.69 & 0.280 & - & 8.8 \\
% CompGS (lowrate)~\cite{liu2024compgs}  & 0.900 & 29.30 & 0.290 & - & \cellcolor{green!25}6.0 \\

%     \bottomrule
%     \end{tabular}
% }
% \end{table}

%

\section{Structured Compression}
Structured compression techniques for 3DGS introduce organization into the otherwise sparse Gaussian representations, enabling more efficient storage and processing. These approaches leverage spatial relationships between Gaussians through anchors, graphs, trees, hierarchical structures, and predictive models. Unlike unstructured methods, which primarily focus on pruning, quantization, and entropy encoding without incorporating the underlying structure of Gaussians, structured approaches introduce architectural constraints to optimize compression efficiency while maintaining reconstruction quality.

\noindent
\subsection{Anchor Based}
Scaffold-GS~\cite{lu2023scaffold} introduces anchor points to structure Gaussians within a scene, ensuring more compact representations while maintaining fidelity (Figure~\ref{fig:scaffoldgs}). Anchors serve as structured reference points, guiding the hierarchical organization and spatial distribution of Gaussians. Unlike traditional 3DGS, where Gaussians freely drift or split, Scaffold-GS employs a grid of anchors, initially derived from SfM points, to provide a region-aware and constrained representation. Each anchor acts as a fixed control center that tethers multiple Gaussians with learnable offsets, enabling local adaptations while preserving structural coherence. Scaffold-GS dynamically predicts Gaussian attributes such as opacity, color, rotation, and scale based on local feature representations and viewing positions. It also employs anchor growing and pruning strategies to refine scene representations—introducing new anchors in underrepresented regions and removing redundant ones. As a foundational method, Scaffold-GS serves as a backbone for subsequent structured compression techniques, providing a reliable framework for further optimization.
HAC (Hash-Grid Assisted Context Modeling)~\cite{chen2024hac}  builds upon Scaffold-GS by incorporating hash grid features to capture spatial dependencies among Gaussians. A hash grid is a structured data representation to efficiently store and query spatial data. HAC uses a binary hash grid to capture spatial consistencies among unorganized 3D Gaussians (or anchors). HAC introduces an Adaptive Scalar Quantization Module (AQM) that dynamically adjusts quantization step sizes for different anchor attributes. HAC queries the hash grid by anchor location to obtain interpolated hash features, which are then used to predict the distribution of anchor attribute values, enabling efficient entropy coding, such as Arithmetic Coding (AE)~\cite{witten1987arithmetic}, for a highly compact model representation. However, HAC encodes all anchors simultaneously, leaving room for further optimization to reduce spatial redundancy.
% Hash-Grid Assisted Context Modeling (HAC)~\cite{chen2024hac}  builds upon Scaffold-GS by incorporating hash grid features to capture spatial dependencies among Gaussians. HAC introduces an Adaptive Scalar Quantization Module (AQM) that dynamically adjusts quantization step sizes for different anchor attributes. HAC queries interpolated hash features to predict anchor attributes, facilitating efficient entropy coding via Arithmetic Coding (AE)~\cite{witten1987arithmetic}. However, HAC encodes all anchors simultaneously, leaving room for further optimization to reduce spatial redundancy.
%HAC (Hash-Grid Assisted Context Modeling)~\cite{} builds upon Scaffold-GS by incorporating hash grid features to capture spatial dependencies among Gaussians. HAC queries interpolated hash features to predict anchor attributes, facilitating efficient entropy coding via Arithmetic Coding (AE)~\cite{witten1987arithmetic}. To further improve representation, HAC introduces an Adaptive Scalar Quantization Module (AQM) that dynamically adjusts quantization step sizes for different anchor attributes, ensuring high fidelity while maintaining compression efficiency. However, HAC encodes all anchors simultaneously, leaving room for further optimization in reducing spatial redundancy.



\noindent
\subsection{Contextual/ Autoregressive (AR) Modeling}
Context models~\cite{minnen2018joint}, commonly used in image compression (LIC) to enhance coding efficiency by predicting the distribution of latent pixels based on already coded ones, inspired ContextGS~\cite{wang2024contextgs}. This approach encodes anchor features autoregressively, predicting anchor points from those already coded at coarser levels. However, compared to LIC methods, the AR-based approach used in ContextGS is significantly faster. ContextGS employs a three-level hierarchical encoding scheme for anchors, where higher-level anchors rely on context from coarser-level anchors for both encoding and decoding. This approach closely mirrors the ChARM~\cite{minnen2020channel} method, making the process more efficient while utilizing the context for compression. Another approach CompGS (Compressed Gaussian Splatting)~\cite{li2017efficient} also utilizes context by introducing a hybrid primitive structure, where sparse anchor primitives serve as reference points for predicting attributes of other Gaussians. 
%ContextGS utilizes AR encoding, a technique commonly used in learned image compression (LIC) \cite{minnen2018joint}. By hierarchically predicting anchor points, ContextGS organizes them into three levels, encoding coarser-level anchors first and using them to predict finer-level anchors. This progressive refinement improves coding efficiency and reduces storage overhead.
%CompGS (Compressed Gaussian Splatting)~\cite{li2017efficient} also utilizes context by introducing a hybrid primitive structure, where sparse anchor primitives serve as reference points for predicting attributes of other Gaussians. This enables a highly compact representation, optimizing the trade-off between compression efficiency and reconstruction quality.

\noindent
\subsection{Graph Based}
Unlike anchor-based methods, graph-based compression treats Gaussians as a graph, leveraging their spatial relationships to optimize compression. Spectrally Pruned Gaussian Fields with Neural Compensation (SUNDAE)~\cite{yang2024spectrally} prunes redundant Gaussians while preserving key scene details. A distinct feature of SUNDAE is its neural compensation head, which shifts from direct GS to feature splatting, enabling lightweight neural network-based feature interpolation. %This approach significantly reduces memory footprint while maintaining high reconstruction quality.
GaussianForest (GF)~\cite{zhang2024gaussian} introduces a tree-structured hierarchical representation for 3D scenes. Unlike traditional unstructured Gaussians, GF assigns explicit attributes (e.g., position, opacity) to leaf nodes, while implicit attributes (e.g., covariance matrix, view-dependent color) are shared across hierarchical levels. Additionally, GF employs adaptive growth and pruning, dynamically adjusting the representation to scene complexities.

\noindent
\subsection{Factorization Approach}
Factorization-based compression techniques introduce mathematical approximations to efficiently represent dense clusters of Gaussians. F-3DGS (Factorized 3D Gaussian Splatting)~\cite{sun2024f} employs matrix and tensor factorization to represent high-dimensional Gaussian parameters using a limited number of basis components per axis. Instead of storing all parameters directly, F-3DGS factorizes attributes such as color, scale, and rotation substantially reducing storage requirements while preserving essential visual attributes. 

\subsection{Discussion}
We also analyze the performance of structured compression techniques for 3DGS on the Mip-NeRF360, Tanks\&Temples, and Deep Blending datasets, focusing on fidelity, compression ratio, and FPS.% Unlike unstructured compression methods, structured approaches impose systematic constraints on the representation, potentially leading to improved storage efficiency while maintaining high reconstruction fidelity.

\begin{figure*}[t]
    \centering
    \includegraphics[width=\linewidth]{Figures/memvspsnr.pdf}
    \caption{Comparison of PSNR and size across structured and unstructured compression techniques. The box represents unstructured compression methods, while the downward triangle represents structured compression methods.}
    \label{fig:memvspsnr}
\end{figure*}

\noindent
\textbf{Fidelity: }
Structured compression methods such as ScaffoldGS, HAC-highrate, and ContextGS (highrate) exhibit strong fidelity across all datasets as shown in Table~\ref{tab:structured_mip_tandt} and Table~\ref{tab:struct_db}. ContextGS-highrate achieves the highest PSNR of 30.39 dB and SSIM of 0.909 on Deep Blending, while HAC-highrate attains a PSNR of 30.34 dB and SSIM of 0.906. ScaffoldGS maintains a competitive PSNR of 30.21 dB and SSIM of 0.906 on Mip-NeRF360.
Interestingly, CompGS-highrate, despite achieving high PSNR, exhibits slightly higher LPIPS score, indicating some perceptual degradation. These results suggest that ScaffoldGS, ContextGS, and HAC-highrate offer the best fidelity, while CompGS provides an optimal balance between compression efficiency and perceptual quality.

\noindent
\textbf{Compression Ratio: }Tables~\ref{tab:structured_mip_tandt} and~\ref{tab:struct_db} show CompGS (lowrate) achieves the highest compression ratio, reducing memory by 83$\times$ on Mip-NeRF360, 70$\times$ on Tanks\&Temples, and 113$\times$ on Deep Blending, while ContextGS (lowrate) reaches 58$\times$ on Mip-NeRF360 and Tanks\&Temples and 188$\times$ on Deep Blending, making them the most storage-efficient techniques with lowest memory footprint compared to the baseline. HAC-lowrate also demonstrates strong compression performance, achieving 48$\times$ reduction on Mip-NeRF360 and 154$\times$ on Deep Blending, with a favorable balance between storage efficiency and fidelity. In contrast, ScaffoldGS and GF, though still reducing memory footprint, operate at relatively lower compression rates (7$\times$–10$\times$).

\noindent
\textbf{FPS: }Structured compression methods, while achieving high compression rates, often maintain FPS similar to or lower than the 3DGS-30k baseline, as shown in Figure~\ref{fig:fpsvspsnr}. GF and HAC, despite their efficient compression, exhibit slower rendering speeds due to high encoding/decoding complexity, which improves compression but reduces rendering efficiency. CompGS, however, demonstrates better rendering speeds than 3DGS-30k, HAC, and GF across all benchmark datasets, making it a more balanced approach between compression efficiency and rendering speed.

Our analysis reveals that HAC-lowrate, CompGS (lowrate), and ContextGS (lowrate) achieve the most aggressive compression ratios, while ScaffoldGS, ContextGS (highrate), and HAC-highrate attain the highest fidelity. Although structured compression techniques offer superior storage efficiency compared to unstructured methods, there remains a trade-off between extreme compression and rendering speeds. 


\noindent
\textbf{Challenges and Future Direction: }
Structured compression organizes sparse Gaussians, offering significant memory footprint reduction as discussed in earlier sections. However, the added complexity from hashing~\cite{chen2024hac}, offsetting~\cite{lu2023scaffold}, encoding, and decoding~\cite{liu2024compgs} introduces computational overhead, which impacts rendering efficiency. Unlike unstructured compression, where compression gains typically lead to faster rendering, structured methods do not inherently translate compression efficiency into rendering efficiency. %Future research should focus on reducing the computational complexity of structured approaches to ensure that compression gains also improve rendering performance.

%Structured compression introduces organization to the sparse Gaussians in the baseline 3DGS framework, leading to notable benefits, as demonstrated by HAC, ContextGS, and CompGS. These methods achieve significantly higher compression rates, as shown in Table~\ref{tab:structured_mip_tandt} and~\ref{tab:struct_db}. However, unlike unstructured compression, where compression gains typically result in faster rendering speeds, structured compression methods maintain rendering speeds similar to baseline 3DGS.

One challenge with structured compression lies in the diversity of methodologies employed by different approaches, which complicates their scalability and standardization. Unlike unstructured compression methods, structured techniques cannot be easily integrated into the baseline 3DGS as plug-and-play solutions.

To advance the field, future research should aim to merge insights from unstructured compression into structured approaches, particularly to enhance rendering speed. Given that techniques inspired by LIC have already influenced 3DGS, it is likely that future work will delve deeper into exploring relationships among Gaussians, developing strategies for effective grouping and combining of Gaussians, and optimizing their encoding and decoding processes for greater efficiency and effectiveness.

\begin{figure*}[t]
    \centering
    \includegraphics[width=\linewidth]{Figures/fps_unstruc_vs_struc.pdf}
    \caption{Comparison of PSNR and FPS across structured and unstructured compression techniques benchmarked against the baseline 3DGS. The box represents unstructured compression methods, while the upward triangle denotes the baseline 3DGS-30k. The downward triangle indicates structured compression methods. FPS values are measured on a single NVIDIA A40 GPU.}
    \label{fig:fpsvspsnr}
\end{figure*}

\section{Structured Compression vs Unstructured Compression}
\label{sec:struc_vs_unstruc}
%Both structured and unstructured compression approaches, as discussed earlier, have effectively reduced the memory footprint of 3DGS. However, the choice between these approaches may vary depending on specific requirements. Structured compression methods introduce organization to the sparse Gaussians in the baseline 3DGS framework, resulting in higher compression rates as shown in Figure~\ref{fig:memvspsnr}. However, this added structure may also introduce additional parameters, which can reduce rendering speed as shown in Figure~\ref{fig:fpsvspsnr}. In contrast, unstructured compression methods build directly on top of the baseline 3DGS, making them more adaptable as plug-and-play solutions. Since they do not add extra parameters, unstructured methods are able to significantly increase rendering speed compared to structured compression. 




% Both structured and unstructured compression techniques, as previously discussed, have demonstrated significant reductions in the memory footprint of 3DGS. The selection of a specific approach, however, depends on the application's unique requirements. Structured compression methods introduce the organization to the sparse Gaussians within the baseline 3DGS framework, achieving higher compression rates, as illustrated in Figure~\ref{fig:memvspsnr}. This structural enhancement, however, often comes at the cost of additional parameters, which can adversely impact rendering speed, as shown in Figure~\ref{fig:fpsvspsnr}.
% Conversely, unstructured compression techniques build directly on the baseline 3DGS, making them versatile and straightforward plug-and-play solutions. These methods avoid introducing extra parameters, thereby significantly improving rendering speed, as highlighted in Figure~\ref{fig:fpsvspsnr}. Figure~\ref{fig:fpsvspsnr} further illustrates that while structured compression methods maintain rendering speeds comparable to baseline 3DGS, unstructured methods boost FPS by approximately 4$\times$ to 5$\times$ on average. This higher efficiency and reduced complexity make unstructured compression methods particularly suitable for low-power edge devices.

% Figure~\ref{fig:memvspsnr} presents a comparison of memory usage and PSNR for structured and unstructured methods. Structured approaches, with their added organization, offer better compressibility and superior performance. Therefore, in scenarios demanding lower memory complexity alongside higher performance, structured compression methods are generally more favorable than their unstructured counterparts.

% Despite the significant gains in compression performance, both structured and unstructured approaches still face common challenges. A majority of compression methods rely on vector quantization, which, although providing superior compression rates, is slower compared to scalar quantization. There is also a notable lack of research on optimizing these methods for low-power edge devices. For instance, the rasterization process could be further refined based on the compression technique used. Niedermayr et al.~\cite{niedermayr2024compressed} modified the rasterization method to better suit vector quantization, and similarly, structured compression methods could enhance the rasterization algorithm to boost rendering speed.

% Additionally, there is limited investigation into the impact of different loss functions on the quality of 3DGS. Studies in image compression have shown that the choice of loss function can significantly affect the visual quality of generated images or scenes, a factor that is likely even more pronounced in structured compression approaches. Addressing these areas could lead to more efficient and higher-quality compression methods for 3DGS.

Both structured and unstructured compression techniques significantly reduce the memory footprint of 3DGS, with the choice of method depending on application-specific needs. 

\noindent
\textbf{Fidelity:} Structured compression methods enhance PSNR and SSIM by imposing organization within the sparse Gaussian framework, as seen in ScaffoldGS and HAC-highrate, which achieve PSNR values of 30.21 on Deep Blending and 27.77 on Mip-NeRF360. Unstructured methods, in contrast, preserve the flexibility of baseline 3DGS-30K, offering a more direct adaptation while maintaining competitive fidelity.

\noindent
\textbf{Compression Ratio: } Structured techniques outperform unstructured methods by achieving higher compression rates while retaining visual quality. CompGS (lowrate) and ContextGS (lowrate) achieve 70$\times$ and 113$\times$  compression on Mip-NeRF360 and Deep Blending, respectively, making them highly efficient for memory-constrained settings. Unstructured methods, while not matching these extreme ratios, still achieve significant memory reductions without additional constraints, offering a better balance between compression and flexibility. Figure~\ref{fig:memvspsnr} shows the compression-performance tradeoff of structured and unstructured compression methods.

\noindent
\textbf{FPS: }Structured methods remain comparable to the 3DGS-30K baseline, whereas unstructured techniques achieve 4$\times$  to 5$\times$  FPS improvements, as shown in Figure~\ref{fig:fpsvspsnr}. This makes unstructured compression preferable for real-time and low-power applications, where efficiency is critical. While structured methods introduce additional parameters that can slightly affect rendering speed, they still maintain a high FPS suitable for interactive applications.

\noindent
\textbf{Challenges and Future Direction: }Despite these advancements, challenges remain. Vector quantization, while effective for compression, is computationally expensive compared to scalar quantization~\cite{ali2024elmgs}. Additionally, optimizing structured methods for low-power edge devices remains underexplored. Adaptive rasterization strategies, such as those proposed by Niedermayr et al.~\cite{niedermayr2024compressed}, could enhance rendering efficiency while refining loss functions could further improve 3DGS quality~\cite{ali2024towards}. Addressing these issues will enable more efficient and high-quality 3DGS compression techniques, benefiting both structured and unstructured approaches.




\section{Future Works and Direction}
\label{sec:future_works}
\subsection{Inspiration from NeRFs}
In the years between 2020 and 2023, we have witnessed rapid advancements in NeRFs, as moving through a very steep trajectory, multiple compression techniques have significantly impacted the design of newer generation NeRFs. Recent work on NeRF compression, such as~\cite{tancik2022block} demonstrated how neural network-based approaches can efficiently represent large-scale 3D scenes with compact latent representations. This paradigm suggests that 3DGS compression could similarly benefit from end-to-end deep learning techniques that directly optimize 3D Gaussian representations, rather than relying merely on traditional data compression methods. By learning more efficient latent space representations of 3D objects, neural network-based models like NeRFs have shown the potential to dramatically reduce the storage and transmission requirements for high-fidelity 3D scenes, which could inspire similar methods for compressing 3D Gaussian components.

One of the key contributions of NeRFs to the field of compression is the use of hierarchical representations~\cite{tang2022compressible,zheng2024hpc}. In NeRFs, the scene is encoded in multiple levels of resolution, progressively refining the details of the 3D structure and lighting~\cite{di2025boost}. This hierarchical approach can be applied to 3D Gaussian Splitting by allowing for multi-level Gaussian components to be represented at varying degrees of detail, depending on the importance of different regions of the 3D space. The concept of progressive compression, where data is encoded in multiple stages to maintain high fidelity in key regions while aggressively compressing less critical data, could offer a path forward for improving 3DGS techniques, especially in complex 3D scenes where different regions (eg., background vs. foreground) require different levels of precision.

Additionally, NeRF compression methods such as \cite{tang2022compressible,deng2023compressing} have highlighted the potential of sparsity in 3D data, where significant portions of the scene can be represented with minimal data points. For 3DGS, leveraging sparse representations could lead to substantial gains in compression efficiency, as Gaussian components that are not central to the scene’s overall structure could be pruned or encoded with fewer parameters. This sparsity, combined with efficient encoding methods inspired by NeRFs, could enable next-generation 3DGS algorithms to handle large-scale 3D data more effectively.

Furthermore, techniques like adaptive quantization and neural compression employed in NeRFs open the door for utilizing machine learning models to optimize the encoding of Gaussian components. Such models can automatically adjust the precision of 3D Gaussian parameters based on the local geometric complexity and importance of the regions being encoded, or simply depth, concept explored in~\cite{deng2022depth}. By adopting similar principles, 3DGS compression could be enhanced to dynamically adjust encoding strategies based on the structure of 3D data.


\subsection{Inspiration from Point Cloud Compression}
Point clouds are among the most important and widely used 3D representations in applications such as autonomous driving, robotics, and physics simulation. The output of 3DGS is also a point cloud, making point cloud compression essential for the efficient storage and transmission of 3DGS scenes. To achieve a favorable compression ratio, it is critical to focus on lossy compression methods and address a key question: what properties of point clouds should be preserved within a limited bitrate budget?

Despite the fact that the output of 3DGS is a point cloud, there has been little focus on combining 3DGS compression with dedicated point cloud compression techniques. While the compression methods used in 3DGS result in a compressed point cloud, this is more of a byproduct of those methods rather than a direct effort to compress the point cloud itself. Given the growing importance of 3D representations in various applications, point cloud compression is a popular research area. Inspired by recent advancements in point cloud compression~\cite{he2022density,song2023efficient}, it is worth exploring how these techniques can be integrated into the end-to-end training process of 3DGS or developed as a separate module specifically tailored for compressing the output of 3DGS.

\section{Conclusion}

This survey provides a comprehensive overview of 3DGS compression methods within the framework of the proposed taxonomy. Compression techniques are categorized into unstructured and structured methods, highlighting significant advancements in memory efficiency and computational performance. We evaluate these techniques in terms of fidelity, compression ratios, and rendering speeds.  Both approaches offer distinct advantages and trade-offs. Structured compression methods leverage the relationships between Gaussians to achieve compression ratios of up to 100$\times$ compared to baseline 3DGS while maintaining high fidelity and perceptual quality. However, this comes at the cost of rendering speed, with FPS comparable to baseline 3DGS-30k. In contrast, unstructured compression methods achieve compression ratios of up to 50$\times$, preserving fidelity similar to baseline 3DGS-30k, while significantly improving rendering speeds by up to 7$\times$. These techniques address critical challenges such as storage constraints and computational overhead, further establishing 3DGS as a scalable and efficient representation for applications ranging from virtual reality to autonomous systems.  

Despite these advances, the scalability of 3DGS for large and complex scenes remains a significant challenge, especially for resource-constrained environments like mobile AR/VR devices. The field also lacks a unified framework that integrates the strengths of structured and unstructured compression methods, limiting the adaptability and standardization of 3DGS models. Furthermore, existing research has not sufficiently explored the potential of novel scalar quantization techniques or the optimization of loss functions tailored specifically for 3DGS. Addressing these gaps could unlock higher compression efficiencies, improved rendering speeds, and enhanced fidelity, making 3DGS more practical for real-world deployment. Additionally, interdisciplinary approaches combining insights from NeRFs, point-based rendering, and emerging deep learning paradigms hold promise for overcoming these limitations.

Looking forward, the future of 3DGS lies in developing hybrid compression frameworks, optimizing models for edge devices, and expanding its applications across diverse domains. Scalar quantization techniques, loss function innovations, and hardware-aware optimizations will be critical to enabling real-time rendering on low-power devices. The impact of 3DGS extends beyond academic research to industries such as healthcare, where it can enhance simulation fidelity, and entertainment, where it enables immersive experiences. By addressing current challenges, the transformative potential of 3DGS can be fully realized, redefining how machines and humans interact with 3D environments. Through continued innovation and exploration, 3DGS can become a cornerstone technology in the next generation of 3D scene rendering.

% This survey has provided a comprehensive exploration of 3DGS, emphasizing its transformative impact on 3D scene rendering and representation. Unlike traditional NeRFs, which rely on implicit models, 3DGS utilizes millions of learnable 3D Gaussians for explicit scene representation, enabling real-time rendering with remarkable editability. The review of current compression techniques, categorized into unstructured and structured methods, highlights significant advancements in memory and computational efficiency. Methods such as pruning, quantization, and entropy coding have achieved impressive compression ratios of up to 40$\times$ without substantial degradation in visual fidelity. These techniques address critical bottlenecks, including storage limitations and computational overhead, and have positioned 3DGS as a versatile tool for various applications ranging from virtual reality to autonomous systems.

% Despite these advances, the scalability of 3DGS for large and complex scenes remains a significant challenge, especially for resource-constrained environments like mobile AR/VR devices. The field also lacks a unified framework that integrates the strengths of structured and unstructured compression methods, limiting the adaptability and standardization of 3DGS models. Furthermore, existing research has not sufficiently explored the potential of novel scalar quantization techniques or the optimization of loss functions tailored specifically for 3DGS. Addressing these gaps could unlock higher compression efficiencies, improved rendering speeds, and enhanced quality preservation, making 3DGS more practical for real-world deployment. Additionally, interdisciplinary approaches combining insights from NeRFs, point-based rendering, and emerging deep learning paradigms hold promise for overcoming these limitations.

% Looking forward, the future of 3DGS lies in developing hybrid compression frameworks, optimizing models for edge devices, and expanding its applications across diverse domains. Scalar quantization techniques, loss function innovations, and hardware-aware optimizations will be critical to enabling real-time rendering on low-power devices. The impact of 3DGS extends beyond academic research to industries such as healthcare, where it can enhance simulation fidelity, and entertainment, where it enables immersive experiences. By addressing current challenges, the transformative potential of 3DGS can be fully realized, redefining how machines and humans interact with 3D environments. Through continued innovation and exploration, 3DGS can become a cornerstone technology in the next generation of 3D scene rendering.

%This survey provides a comprehensive overview of the trends, methods, and future directions for 3DGS compression. Given the rapid growth in popularity and the numerous follow-up works focusing on compressing 3D Gaussian Splatting (3DGS), this survey categorizes various compression methodologies, highlighting their advantages, limitations, and areas for improvement. In addition to analyzing the current state of 3DGS compression techniques, we identify key challenges and propose future research directions to address these issues, aiming to guide the development of more efficient and effective compression strategies.


%\textcolor{green}{\section{Figures and Tables to Add}}
% \begin{enumerate}
%     \item \textcolor{green}{Highlight Figure (highlighting number of works in 3DGS in last year)}
%     %\item \textcolor{green}{Main Figure of 3DGS}
%     %\item \textcolor{green}{Working of 3DGS (details of rasterization algorithm)}
%     %\item \textcolor{green}{Taxonomy Figure}
%     %\item \textcolor{green}{Table for Categorizing the Methods}
%     %\item \textcolor{green}{Figure for visual explanation of categorization of 3DGS compression and techniques involved}
%     %\item \textcolor{green}{Qualitative Comparison of Different Figures}
% \end{enumerate}





% \section{Discussion}
% NeRFs revolutionized the field of 3D reconstruction despite being initially slow and challenging to train and render. However, since the pioneering NeRF work in 2020, subsequent research has significantly addressed these issues, with efficient NeRFs now capable of rendering at up to 200 FPS~\cite{garbin2021fastnerf}. Nevertheless, the advent of 3DGS has quickly surpassed the challenges faced by NeRFs and has rapidly become a transformative force in 3D reconstruction. The unparalleled flexibility offered by 3DGS already surpasses the performance of state-of-the-art NeRFs. Considering the multitude of works in 3DGS, we can anticipate its widespread adoption in 3D reconstruction applications in the coming years~\cite{DBLP:journals/corr/abs-2401-03890}. However, a critical factor for its success is dependent upon reducing the memory footprint of 3DGS, making compression of 3DGS an essential topic for ongoing research and development.

% This work aims to provide an excellent starting point for the community aiming to address the challenges of 3DGS, particularly in compression. This paper shall provide a detailed overview of 3DGS, their challenges, and problems, and shall offer an overview of recent compression techniques, and suggests how future research can draw inspiration from advancements in the efficient NeRFs domain for efficient 3DGS compression.
 
 
%The goal of this paper is to inform readers about recent preliminary works focusing on 3DGS compression and to provide insights from efficient NeRF compression techniques and information theory-based compression methods. This foundation aims to support further advancements in the compression of 3DGS models.





\bibliographystyle{ieeetr}
\bibliography{ref}

\end{document}
