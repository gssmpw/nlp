% \newpage
\section{Appendix~--~System Prompts, Query Templates, Example Methods}\label{sec:appendix-a-system-prompts}

In this appendix, we present the \emph{System Prompts} used in our experiments as defined in Section~\ref{sec:system_prompts}. We also supply examples of the \emph{Query Templates} and \emph{Example Method Variations} outlined in Sections~\ref{sec:query_templates} and~\ref{sec:example_method_variations}. 
Listings~\ref{lst:sp1},~\ref{lst:sp2}, and~\ref{lst:sp3} show the system prompts used for SciCite. The prompts for ACL-ARC follow the same structure, with the appropriate changes to the class definitions and can be found in the supplementary materials. Listings~\ref{lst:qt1} and~\ref{lst:qt2} show examples of the query templates we defined and Listings~\ref{lst:em1} and~\ref{lst:em2} show the different ways that we supplied the samples in our prompts.

% \newpage

\begin{lstlisting}[caption=SP1 for SciCite, label=lst:sp1]
You are an expert researcher tasked with classifying the intent of a citation in a scientific publication.

The three (3) possible classes are the following: "background information", "method", "results comparison".

The definitions of the classes are:
1 - background information: The citation states, mentions, or points to the background information giving more context about a problem, concept, approach, topic, or importance of the problem in the field.
2 - method: Making use of a method, tool, approach or dataset.
3 - results comparison: Comparison of the paper's results/findings with the results/findings of other work.

For each given sentence, you must analyse only the citation with the @@CITATION@@. You must assign only one class to each citation. Only return the class name, with no elaboration.
\end{lstlisting}

% \newpage

\begin{lstlisting}[caption=SP2 for SciCite, label=lst:sp2]
# CONTEXT # 
You are an expert researcher tasked with classifying the intent of a citation in a scientific publication.

########

# OBJECTIVE # 
You will be given a sentence containing a citation, you must output the appropriate class as an answer.

########

# CLASS DEFINITIONS #

The three (3) possible classes are the following: "background information", "method", "results comparison".

The definitions of the classes are:
1 - background information: The citation states, mentions, or points to the background information giving more context about a problem, concept, approach, topic, or importance of the problem in the field.
2 - method: Making use of a method, tool, approach or dataset.
3 - results comparison: Comparison of the paper's results/findings with the results/findings of other work.

########

# RESPONSE RULES #
You must strictly adhere to the following rules for your response:
- For each sentence, you must analyse only the citation with the @@CITATION@@ tag. 
- You must assign only one class to each citation. 
- Only respond with the class name, with no explanation or elaboration.
- Only answer with one or two words.
- Always be very brief.
\end{lstlisting}

\begin{lstlisting}[caption=SP3 for SciCite, label=lst:sp3]
# CONTEXT # 
You are an expert researcher tasked with classifying the intent of a citation in a scientific publication.

########

# OBJECTIVE # 
You will be given a sentence containing a citation. You must classify the intent of the citation by assigning it to one of three predefined classes.

########

# CLASS DEFINITIONS #

The three (3) possible classes are the following: "background information", "method", "results comparison".

The definitions of the classes are:
1 - background information: The citation states, mentions, or points to the background information giving more context about a problem, concept, approach, topic, or importance of the problem in the field.
2 - method: Making use of a method, tool, approach or dataset.
3 - results comparison: Comparison of the paper's results/findings with the results/findings of other work.

########

# RESPONSE RULES #
- Analyze only the citation marked with the @@CITATION@@ tag.
- Assign exactly one class to each citation.
- Respond only with the exact name of one of the following classes: "background information", "method", or "results comparison".
- Do not provide any explanation or elaboration.
\end{lstlisting}



\begin{lstlisting}[caption=Example of a Simple Query Template (QT1), label=lst:qt1]
These results are in contrast with the findings of Santos et al.@@CITATION@@, who reported a significant association between low sedentary time and healthy CVF among Portuguese.
Class: results comparison
\end{lstlisting}

\begin{lstlisting}[caption=Example of a Multiple-Choice Query Template (QT2), label=lst:qt2]
These results are in contrast with the findings of Santos et al.@@CITATION@@, who reported a significant association between low sedentary time and healthy CVF among Portuguese.
### Question: Which is the most likely intent for this citation?
a) background information
b) method 
c) results comparison
### Answer: results comparison
\end{lstlisting}

\newpage

\begin{lstlisting}[caption=Example of the ``Inline'' Examples Method (EM1), label=lst:em1]
[Start of system prompt]

[system prompt content]

########

# EXAMPLES #

These results are in contrast with the findings of Santos et al.@@CITATION@@, who reported a significant association between low sedentary time and healthy CVF among Portuguese.
Class: results comparison

Two or fewer cancer-containing cores has been shown to correlate with tumour volume in US series (@@CITATION@@).
Class: background information

...

[end of system prompt]

{
    role: user
    message: The regions of dhfr and dhps genes containing the mutations for antifolate resistance were amplified as described elsewhere (@@CITATION@@) and sequenced for detection of the mutations..
},
{
    role: assistant
    message: [prediction]
}
\end{lstlisting}

% \newpage

\begin{lstlisting}[caption=Example of the ``Roles'' Examples Method (EM2), label=lst:em2]
[Start of system prompt]

[system prompt content]

[end of system prompt]

{
    role: user
    message: These results are in contrast with the findings of Santos et al.@@CITATION@@, who reported a significant association between low sedentary time and healthy CVF among Portuguese.
},
{
    role: assistant
    message: results comparison
},
{
    role: user
    message: Two or fewer cancer-containing cores has been shown to correlate with tumour volume in US series (@@CITATION@@).
},
{
    role: assistant
    message: background information
},
...
{
    role: user
    message: The regions of dhfr and dhps genes containing the mutations for antifolate resistance were amplified as described elsewhere (@@CITATION@@) and sequenced for detection of the mutations..
},
{
    role: assistant
    message: [prediction]
}
\end{lstlisting}
