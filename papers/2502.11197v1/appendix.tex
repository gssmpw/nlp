\appendix
\section{Platform synonyms}
\begin{table*}[htbp]
\caption{A glossary of terms used in {\platformName}.}
\centering
\renewcommand{\arraystretch}{1.0}
\resizebox{\textwidth}{!}{
% acronyms go here

\newacronym{av}{AV}{autonomous vehicle}
\newacronym{nn}{NN}{neural network}
\newacronym{tsp}{TSP}{Travelling Salesman problem}
\newacronym{vrp}{VRP}{Vehicle Routing Problem}
\newacronym{cvrp}{CVRP}{Capacitated Vehicle Routing Problem}
\newacronym{rl}{RL}{reinforcement learning}
\newacronym{gnn}{GNN}{Graph Neural Net}
\newacronym{gat}{GAT}{Graph Attention Net}
\newacronym{rnn}{RNN}{Recurrent Neural Net}
\newacronym{drl}{DRL}{Deep Reinforcement Learning}
\newacronym{mlp}{MLP}{Multi-Layer Perceptron}

\newacronym{co}{CO}{Combinatorial Optimization}
\newacronym{ndp}{TNDP}{Transit Network Design Problem}
\newacronym{fsp}{FSP}{Frequency-Setting Problem}
\newacronym{dfsp}{DFSP}{Design and Frequency-Setting Problem}
\newacronym{sp}{SP}{Scheduling Problem}
\newacronym{tp}{TP}{Timetabling Problem}
\newacronym{ndsp}{NDSP}{Network Design and Scheduling Problem}

\newacronym{mod}{MoD}{Mobility on Demand}
\newacronym{amod}{AMoD}{Autonomous Mobility on Demand}
\newacronym{imodp}{IMoDP}{Intermodal Mobility-on-Demand Problem}
\newacronym{matsim}{MATSim}{Multi-Agent Transport Simulation}
\newacronym{od}{OD}{Origin-Destination}
\newacronym{csa}{CSA}{Connection Scan Algorithm}
\newacronym{mdp}{MDP}{Markov Decision Process}
\newacronym{dqn}{DQN}{Deep Q-Networks}
\newacronym{acer}{ACER}{Actor-Critic with Experience Replay}
\newacronym{ppo}{PPO}{Proximal Policy Optimization}
\newacronym{artm}{ARTM}{Metropolitan Regional Transportation Authority}
\newacronym{stl}{STL}{Soci\'et\'e de Transport de Laval}
\newacronym{cda}{CDA}{Census Dissemination Area}

}
\end{table*}
\section{Prompts of the LLM-based agents}\label{appendix_prompts}
In Figures \ref{prompt_system}, \ref{prompt_system_no_copy}, \ref{prompt_clean}, \ref{prompt_listwise}, and \ref{prompt_pairwise}, we present the prompts used for the LLM-based agents.
\begin{figure*}[t]
\centering
\input{prompts_and_confs/str_prompt_system}
\caption{System prompt for LLM-based agents. The system prompt includes instructions for the game and defines the agent's persona.}
\label{prompt_system}
\end{figure*}

\begin{figure*}[t]
\centering
\begin{minted}[breaklines=true, frame=single, fontsize=\small]{python}
SYSTEM_PROMPT_NO_COPY = "Edit the candidate document to improve its search engine ranking for the candidate query, aiming for the highest rank (1 being the highest). Use the black box search engine's past rankings over various queries, provided as context by the user, to guide your edits. Focus on editing the most impactful sentences to enhance ranking potential. Target an edited document length of around 147 words, not exceeding 150 words. Ensure the edited document is very similar to the candidate document. Generate only the edited document, without additional comments or titles. Don't use a duplicate of another participant's document. Documents that practice this technique will be penalized. However, you are permitted to copy paste parts of documents that other participants have written, As long as the originality of your document is remained.\n Input: \n\n - Candidate Query: <QUERY> \n\n - Candidate Document: <CURRENT_DOCUMENT> \n\n - Character: <PERSONA>"
\end{minted}
\caption{System prompt for LLM-based agents with additional instruction for not copying as given to students in ranking competition \cite{nachimovsky_ranking-incentivized_2024}. The system prompt includes instructions for the game and defines the agent's persona.}
\label{prompt_system_no_copy}
\end{figure*}

\begin{figure*}[t]
\centering
\begin{minted}[breaklines=true, frame=single, fontsize=\small]{python}
CLEANING_PROMPT = "Your task is to clean up the document generated by an LLM. Remove any headers, prefixes, or metadata such as \"This is the modified document\" or similar phrases that are not part of the actual content. Exclude statements that describe how the document was modified or its characteristics, such as its length or ranking. Specifically, omit sentences like the following:\n\n- [The text above is the extracted document part.]\n- The extracted document part:\n- [The document text remains unchanged.]\n- (The extracted document part ends here)\n- [Document text only, no modifications or additions made.]\n- [End of Document]\n- [The rest of the text is not the document part and will be ignored.]\n- [Document Text Only]\n- *Here is the document text you requested, unaltered:*\n- To improve the ranking\n- the document length is 147 words\n\nImportant: Do not change or modify the actual content of the document. Only remove unnecessary prefixes, headers, or metadata, leaving the original text of the document untouched and unaltered. The output should read naturally, without unnecessary formatting or markers."
\end{minted}
\caption{Cleaning prompt submitted to the LLM to remove headers and prefixes from the generated text.}
\label{prompt_clean}
\end{figure*}



\begin{figure*}[t]
\centering
\begin{minted}[breaklines=true, frame=single, fontsize=\small]{python}
LISTWISE_PROMPT = "\n\nquery: <QUERY>\n\n* documents ordered by latest ranking from highest to lowest in relation to the query: \n\n\n* <DOCUMENT a>\n\n\n* <DOCUMNET b>\n\n\n* <DOCUMENT c>\n\n\n* <DOCUMENT d>\n\n\n\n\n\n * documents ranked by second to latest ranking from highest to lowest in relation to the query:\n1. <DOCUMENT e>\n2. <DOCUMENT f>\n3. <DOCUMENT g>\n4. <DOCUMENT h>\n5. <DOCUMENT i>\n\n\n\n"
\end{minted}
\caption{Listwise user prompt used as feedback for an LLM-based agent. This user prompt includes feedback from the last two rounds. The most recent round presents documents from highest to lowest ranking in the group, excluding the current document: a, b, c, d. The second-to-last round presents documents from the entire group, ranked from highest to lowest: e, f, g, h, i.}
\label{prompt_listwise}
\end{figure*}

\begin{figure*}[t]
\centering
\input{prompts_and_confs/str_prompt_pairwise}
\caption{Pairwise user prompt used as feedback for an LLM-based agent. This user prompt includes feedback from the last three rounds. In each round, two documents (a,b), (c,d), (d,f) were randomly selected including their ranking r(a), r(b), r(c), r(d), r(e), r(f).}
\label{prompt_pairwise}
\end{figure*}

\section{{\platformName} code-base}\label{appendix_code-base}
\subsection{{\platformName} code-base and design}\label{appendix_code-base_design}
Our platform is built upon \textit{Python 3}, making it easy to install, and is composed of popular Python packages listed in a requirements file. We ensure adherence to Object-Oriented Programming (OOP) practices and include comments and logs to document the main steps during the execution of the simulation.

In Figure \ref{fig_flow} we present the flow of \platformName.

% \begin{figure*}[htbp]
%     \centering
%     \includegraphics[width=\linewidth]{figs/platform/flow.jpg}
%     \caption{Flow of {\platformName}.}
%     \label{fig_flow}
% \end{figure*}

% \subsection{{\platformName} functionality}\label{appendix_code-base_functionality}
% To run {\platformName} , we utilized a \textit{config.json} file to configure the main parameters. In Figure \ref{appendix_config}, we present an example of a \textit{config.json} file for a dataset. The outputs of the simulation are:
% \begin{itemize}
% \item \textit{competition\_history.csv}: A log of the entire competition, including the documents generated by every player for each round and game.
% \item \textit{output.trectext}: A file in TREC format with the \textit{docno} as the header for each document.
% \item \textit{logs}: A directory containing logs for all main classes of the platform.
% \end{itemize}


% \begin{figure*}[t]
% \centering
% \begin{minted}[breaklines=true, fontsize=\scriptsize, frame=lines, tabsize=2]{json}
{
    "competition": {
        "round_by_round": true,
        "init_docs_path": {
            "queries_folder_path": "data/web_track",
            "docs_file_path": "data/initial_documents.trectext"
        },
        "rankers": {
            "e5": {
                "model_name": "intfloat/e5-large-unsupervised"
            }
        },
        "llm": {
            "model_name": "google/gemma-2-9b-it",
            "token": "<HF_TOKEN>",
            "temperature": 0.5,
            "top_p": 0.9
        }
    },
    "game": {
        "max_tokens": 256,
        "rounds": 30,
        "force_max_tokens": true
    },
    "agents": {
    "student": {
        "character": "behave like a BSc student",
        "prompt_format": "Edit the candidate document to improve its search engine ranking for the candidate query, aiming for the highest rank (1 being the highest). Use the black box search engine's past rankings over various queries, provided as context by the user, to guide your edits. Focus on editing the most impactful sentences to enhance ranking potential. Target an edited document length of around 147 words, not exceeding 150 words. Ensure the edited document is very similar to the candidate document. Generate only the edited document, without additional comments or titles.\n",
        "pairwise": false,
        "depth": 2
    },
    "writer": {
        "character": "behave like a professional writer",
        "prompt_format": "Edit the candidate document to improve its search engine ranking for the candidate query, aiming for the highest rank (1 being the highest). Use the black box search engine's past rankings over various queries, provided as context by the user, to guide your edits. Focus on editing the most impactful sentences to enhance ranking potential. Target an edited document length of around 147 words, not exceeding 150 words. Ensure the edited document is very similar to the candidate document. Generate only the edited document, without additional comments or titles.\n",
        "pairwise": false,
        "depth": 2
    },
    "editor": {
        "character": "behave like a professional editor",
        "prompt_format": "Edit the candidate document to improve its search engine ranking for the candidate query, aiming for the highest rank (1 being the highest). Use the black box search engine's past rankings over various queries, provided as context by the user, to guide your edits. Focus on editing the most impactful sentences to enhance ranking potential. Target an edited document length of around 147 words, not exceeding 150 words. Ensure the edited document is very similar to the candidate document. Generate only the edited document, without additional comments or titles.\n",
        "pairwise": false,
        "depth": 2
    },
    "teacher": {
        "character": "behave like an English teacher",
        "prompt_format": "Edit the candidate document to improve its search engine ranking for the candidate query, aiming for the highest rank (1 being the highest). Use the black box search engine's past rankings over various queries, provided as context by the user, to guide your edits. Focus on editing the most impactful sentences to enhance ranking potential. Target an edited document length of around 147 words, not exceeding 150 words. Ensure the edited document is very similar to the candidate document. Generate only the edited document, without additional comments or titles.\n",
        "pairwise": false,
        "depth": 2
    },
    "professor": {
        "character": "behave like a Data Science professor",
        "prompt_format": "Edit the candidate document to improve its search engine ranking for the candidate query, aiming for the highest rank (1 being the highest). Use the black box search engine's past rankings over various queries, provided as context by the user, to guide your edits. Focus on editing the most impactful sentences to enhance ranking potential. Target an edited document length of around 147 words, not exceeding 150 words. Ensure the edited document is very similar to the candidate document. Generate only the edited document, without additional comments or titles.\n",
        "pairwise": false,
        "depth": 2
    }
}
}
\end{minted}
% \caption{An example of a \textit{config.json} file for a Gemma LLM-based agent with a listwise prompt, in a competition using the E5 ranker. The competition consists of 30 games, each with a different query, and each game runs across 30 rounds.}
% \label{appendix_config}
% \end{figure*}

% \section{Datasets Parameters}\label{appendix_dataset_parameterts}
% \begin{table*}[h!]
% \centering
% \caption{Parameters used in the creation of the datasets. Parameters highlighted in \textbf{bold} denote those that varied across the datasets. Not all combinations were generated due to budget constraints.}
% \renewcommand{\arraystretch}{1.2}
% \begin{tabular}{|l|c|}
\hline
\textbf{Parameter} & \textbf{Value(s)} \\
\hline
\textbf{Number of agents} & 4 or 5 \\
\hline
Rounds & 30 \\
\hline
Round by Round & True \\
\hline
Queries & 30 queries selected from TREC \\
\hline
\textbf{Rankers} & intfloat/e5-large-unsupervised, facebook/contriever, Okapi BM25 \\
\hline
\textbf{LLM models of agents} & 
\begin{tabular}[t]{@{}l@{}}
\small{meta-llama/Meta-Llama-3.1-8B-Instruct, google/gemma-2-9b-it,} \\
\small{Qwen/Qwen2.5-7B-Instruct, mlx-community/Ministral-8B-Instruct-2410-bf16}
\end{tabular} \\
\hline
LLM parameters & $top_p = 0.9 ; Temperature = 0.5$ \\
\hline
Characters & BSc student, professional writer, professional editor, \\
& English student, data science professor \\
\hline
Max Tokens & 256 \\
\hline
\textbf{Feedback Mechanism} & Pairwise prompt, listwise prompt\footnote{XX ref to niv} \\
\hline
\textbf{System Prompt} & The system prompt in Bardas et al. \cite{niv} or The the system prompt with addition instruction to not copy. Figures \ref{prompt_system} and \ref{prompt_system_no_copy}\\
\hline
\end{tabular}
% \label{table_data_param}
% \end{table*}

% \begin{table*}[!ht]
%     \centering
%     \caption{List of the datasets generated with \platformName \platform.}
%     \begin{table}[h] \footnotesize  \centering\resizebox{0.48\textwidth}{!}{\begin{tabular}{c|l|c|c|c}
\toprule
\textbf{Task} & \textbf{Dataset} & \textbf{N-shot} & \multirowcell{\textbf{Train texts} \\ \textbf{for STMD}} & \multirowcell{\textbf{Evaluation} \\ \textbf{texts}} \\
\midrule
\multirow{3}{*}{\multirowcell{Text \\ Summarization}} & CNN/DailyMail & 0 & 2,000 & 2,000 \\
& XSum & 0 & 2,000 & 2,000 \\
& SamSum & 0 & 2,000 & 819 \\
\midrule
\multirow{4}{*}{\multirowcell{QA \\ Long answer}} & PubMedQA & 0 & 2,000 & 2,000 \\
& MedQUAD & 5 & 2,000 & 2,000 \\
& TruthfulQA & 5 & 408 & 409 \\
& GSM8k & 5 & 2,000 & 1,319 \\
\midrule
\multirow{4}{*}{\multirowcell{QA \\ Short answer}} & SciQ & 0 & 5,000 & 1,000 \\
& CoQA & \multirowcell{all preceding \\ questions} & 5,000 & 2,000 \\
& TriviaQA & 5 & 5,000 & 2,000 \\
\midrule
\multirow{1}{*}{\multirowcell{MCQA}} & MMLU & 5 & 5,000 & 2,000 \\
\bottomrule
\end{tabular}
}\caption{\label{tab:dataset_stat} The statistics of the datasets used for evaluation.}
\end{table}
%     \label{table_datasets}
% \end{table*}