\appendix
\section{Platform synonyms}
\begin{table*}[htbp]
\caption{A glossary of terms used in {\platformName}.}
\centering
\renewcommand{\arraystretch}{1.0}
\resizebox{\textwidth}{!}{
\newglossaryentry{AI}{
    name=AI: Artificial Intelligence,
    description={The simulation of human intelligence}
}

\newglossaryentry{LLM}{
    name=LLM: Large Language Model,
    description={A type of artificial intelligence designed to understand and generate human-like text}
}

\newglossaryentry{API}{
    name=API: Application Programming Interface,
    description={A software interface for offering a service to other pieces of software}
}

\newglossaryentry{AST}{
    name=AST: Abstract Syntax Tree,
    description={A tree representation of the abstract syntactic structure of source code written in a programming language}
}

\newglossaryentry{ReACC}{
    name=ReACC: Retrieval-Augmented Code Completion,
    description={A framework that enhances code completion by leveraging external context from a large codebase}
}

\newglossaryentry{CMSIS}{
    name=CMSIS: Cortex Microcontroller Software Interface Standard,
    description={A hardware abstraction layer independent of vendor for the Cortex-M processor series}
}

\newglossaryentry{HAL}{
    name=HAL: Hardware Abstraction Layer,
    description={A layer of programming that allows a computer operating system to interact with a hardware device at an abstract level}
}

\newglossaryentry{RAG}{
    name=RAG: Retrieval-Augmented Generation,
    description={A process that enhances large language models by allowing them to respond to prompts using a specified set of documents}
}

\newglossaryentry{STM32F407}{
    name=STM32F407: High-\allowbreak performance Microcontroller,
    description={A microcontroller that offers the performance of the Cortex-M4 core}
}

\newglossaryentry{AURIX TC334}{
    name=AURIX TC334: 32-bit Microcontroller from Infineon,
    description={A microcontroller designed for automotive and industrial applications, featuring a 32-bit TriCore-\allowbreak architecture}
}

\newglossaryentry{LED}{
    name=LED: Light-Emitting Diode,
    description={A semiconductor light source that emits light when current flows through it}
}

\newglossaryentry{CortexM4}{
    name=Cortex-M4: 32-bit processor design from ARM,
    description={A 32-bit processor design optimized for real-time applications with low power consumption}
}

\newglossaryentry{GPIO}{
    name=GPIO: General-Purpose Input/Output Pin,
    description={A versatile pin on a microcontroller that can be configured as either an input or an output. As an \textbf{input}, it can read external signals such as button presses. As an \textbf{output}, it can control devices}
}

\newglossaryentry{Offset}{
    name=Offset: Relative distance of a specific register,
    description={The relative distance or position of a specific register or memory location within a hardware block, measured from a base address}
}

\newglossaryentry{Clock}{
    name=Clock: Synchronization signal for operations,
    description={The signal used to synchronize operations within a microcontroller or hardware system, ensuring consistent timing and execution of tasks}
}

\newglossaryentry{GPT4oMini}{
    name=GPT-4o Mini: Large Language Model from OpenAI,
    description={A compact variant of the GPT-4 language model designed for cost-efficient and versatile tasks}
}

\newglossaryentry{FAISS}{
    name=FAISS: Facebook AI Similarity Search,
    description={An open-source library for efficient similarity search and clustering of high-dimensional vectors}
}

}
\end{table*}
\section{Prompts of the LLM-based agents}\label{appendix_prompts}
In Figures \ref{prompt_system}, \ref{prompt_system_no_copy}, \ref{prompt_clean}, \ref{prompt_listwise}, and \ref{prompt_pairwise}, we present the prompts used for the LLM-based agents.
\begin{figure*}[t]
\centering
\begin{minted}[breaklines=true, frame=single, fontsize=\small]{python}
SYSTEM_PROMPT = "Edit the candidate document to improve its search engine ranking for the candidate query, aiming for the highest rank (1 being the highest). Use the black box search engine's past rankings over various queries, provided as context by the user, to guide your edits. Focus on editing the most impactful sentences to enhance ranking potential. Target an edited document length of around 147 words, not exceeding 150 words. Ensure the edited document is very similar to the candidate document. Generate only the edited document, without additional comments or titles. \n Input: \n\n - Candidate Query: <QUERY> \n\n - Candidate Document: <CURRENT_DOCUMENT> \n\n - Character: <PERSONA>"
\end{minted}

\caption{System prompt for LLM-based agents. The system prompt includes instructions for the game and defines the agent's persona.}
\label{prompt_system}
\end{figure*}

\begin{figure*}[t]
\centering
\begin{minted}[breaklines=true, frame=single, fontsize=\small]{python}
SYSTEM_PROMPT_NO_COPY = "Edit the candidate document to improve its search engine ranking for the candidate query, aiming for the highest rank (1 being the highest). Use the black box search engine's past rankings over various queries, provided as context by the user, to guide your edits. Focus on editing the most impactful sentences to enhance ranking potential. Target an edited document length of around 147 words, not exceeding 150 words. Ensure the edited document is very similar to the candidate document. Generate only the edited document, without additional comments or titles. Don't use a duplicate of another participant's document. Documents that practice this technique will be penalized. However, you are permitted to copy paste parts of documents that other participants have written, As long as the originality of your document is remained.\n Input: \n\n - Candidate Query: <QUERY> \n\n - Candidate Document: <CURRENT_DOCUMENT> \n\n - Character: <PERSONA>"
\end{minted}
\caption{System prompt for LLM-based agents with additional instruction for not copying as given to students in ranking competition \cite{nachimovsky_ranking-incentivized_2024}. The system prompt includes instructions for the game and defines the agent's persona.}
\label{prompt_system_no_copy}
\end{figure*}

\begin{figure*}[t]
\centering
\begin{minted}[breaklines=true, frame=single, fontsize=\small]{python}
CLEANING_PROMPT = "Your task is to clean up the document generated by an LLM. Remove any headers, prefixes, or metadata such as \"This is the modified document\" or similar phrases that are not part of the actual content. Exclude statements that describe how the document was modified or its characteristics, such as its length or ranking. Specifically, omit sentences like the following:\n\n- [The text above is the extracted document part.]\n- The extracted document part:\n- [The document text remains unchanged.]\n- (The extracted document part ends here)\n- [Document text only, no modifications or additions made.]\n- [End of Document]\n- [The rest of the text is not the document part and will be ignored.]\n- [Document Text Only]\n- *Here is the document text you requested, unaltered:*\n- To improve the ranking\n- the document length is 147 words\n\nImportant: Do not change or modify the actual content of the document. Only remove unnecessary prefixes, headers, or metadata, leaving the original text of the document untouched and unaltered. The output should read naturally, without unnecessary formatting or markers."
\end{minted}
\caption{Cleaning prompt submitted to the LLM to remove headers and prefixes from the generated text.}
\label{prompt_clean}
\end{figure*}



\begin{figure*}[t]
\centering
\begin{minted}[breaklines=true, frame=single, fontsize=\small]{python}
LISTWISE_PROMPT = "\n\nquery: <QUERY>\n\n* documents ordered by latest ranking from highest to lowest in relation to the query: \n\n\n* <DOCUMENT a>\n\n\n* <DOCUMNET b>\n\n\n* <DOCUMENT c>\n\n\n* <DOCUMENT d>\n\n\n\n\n\n * documents ranked by second to latest ranking from highest to lowest in relation to the query:\n1. <DOCUMENT e>\n2. <DOCUMENT f>\n3. <DOCUMENT g>\n4. <DOCUMENT h>\n5. <DOCUMENT i>\n\n\n\n"
\end{minted}
\caption{Listwise user prompt used as feedback for an LLM-based agent. This user prompt includes feedback from the last two rounds. The most recent round presents documents from highest to lowest ranking in the group, excluding the current document: a, b, c, d. The second-to-last round presents documents from the entire group, ranked from highest to lowest: e, f, g, h, i.}
\label{prompt_listwise}
\end{figure*}

\begin{figure*}[t]
\centering
\begin{minted}[breaklines=true, frame=single, fontsize=\small]{python}
PAIRWISE_PROMPT = "\n\nquery: <QUERY>\n\n* document: <DOCUMENT a>\n\nlatest ranking: <RANK r(a)>\n\n\n* document: <DOCUMENT b>\n\nlatest ranking: <RANK r(b)>\n\n\n\n\nquery: <QUERY>\n\n* document: <DOCUMENT c>\n\nsecond to latest ranking: <RANK r(c)>\n\n\n* document: <DOCUMENT d>\n\nsecond to latest ranking: <RANK r(d)>\n\n\n\n\nquery: <QUERY>\n\n* document: <DOCUMENT e>\n\nthird to latest ranking: <RANK r(e)>\n\n\n* document: <DOCUMENT f>\n\nthird to latest ranking: <RANK r(f)>"
\end{minted}
\caption{Pairwise user prompt used as feedback for an LLM-based agent. This user prompt includes feedback from the last three rounds. In each round, two documents (a,b), (c,d), (d,f) were randomly selected including their ranking r(a), r(b), r(c), r(d), r(e), r(f).}
\label{prompt_pairwise}
\end{figure*}

\section{{\platformName} code-base}\label{appendix_code-base}
\subsection{{\platformName} code-base and design}\label{appendix_code-base_design}
Our platform is built upon \textit{Python 3}, making it easy to install, and is composed of popular Python packages listed in a requirements file. We ensure adherence to Object-Oriented Programming (OOP) practices and include comments and logs to document the main steps during the execution of the simulation.

In Figure \ref{fig_flow} we present the flow of \platformName.

% \begin{figure*}[htbp]
%     \centering
%     \includegraphics[width=\linewidth]{figs/platform/flow.jpg}
%     \caption{Flow of {\platformName}.}
%     \label{fig_flow}
% \end{figure*}

% \subsection{{\platformName} functionality}\label{appendix_code-base_functionality}
% To run {\platformName} , we utilized a \textit{config.json} file to configure the main parameters. In Figure \ref{appendix_config}, we present an example of a \textit{config.json} file for a dataset. The outputs of the simulation are:
% \begin{itemize}
% \item \textit{competition\_history.csv}: A log of the entire competition, including the documents generated by every player for each round and game.
% \item \textit{output.trectext}: A file in TREC format with the \textit{docno} as the header for each document.
% \item \textit{logs}: A directory containing logs for all main classes of the platform.
% \end{itemize}


% \begin{figure*}[t]
% \centering
% \begin{minted}[breaklines=true, fontsize=\scriptsize, frame=lines, tabsize=2]{json}
{
    "competition": {
        "round_by_round": true,
        "init_docs_path": {
            "queries_folder_path": "data/web_track",
            "docs_file_path": "data/initial_documents.trectext"
        },
        "rankers": {
            "e5": {
                "model_name": "intfloat/e5-large-unsupervised"
            }
        },
        "llm": {
            "model_name": "google/gemma-2-9b-it",
            "token": "<HF_TOKEN>",
            "temperature": 0.5,
            "top_p": 0.9
        }
    },
    "game": {
        "max_tokens": 256,
        "rounds": 30,
        "force_max_tokens": true
    },
    "agents": {
    "student": {
        "character": "behave like a BSc student",
        "prompt_format": "Edit the candidate document to improve its search engine ranking for the candidate query, aiming for the highest rank (1 being the highest). Use the black box search engine's past rankings over various queries, provided as context by the user, to guide your edits. Focus on editing the most impactful sentences to enhance ranking potential. Target an edited document length of around 147 words, not exceeding 150 words. Ensure the edited document is very similar to the candidate document. Generate only the edited document, without additional comments or titles.\n",
        "pairwise": false,
        "depth": 2
    },
    "writer": {
        "character": "behave like a professional writer",
        "prompt_format": "Edit the candidate document to improve its search engine ranking for the candidate query, aiming for the highest rank (1 being the highest). Use the black box search engine's past rankings over various queries, provided as context by the user, to guide your edits. Focus on editing the most impactful sentences to enhance ranking potential. Target an edited document length of around 147 words, not exceeding 150 words. Ensure the edited document is very similar to the candidate document. Generate only the edited document, without additional comments or titles.\n",
        "pairwise": false,
        "depth": 2
    },
    "editor": {
        "character": "behave like a professional editor",
        "prompt_format": "Edit the candidate document to improve its search engine ranking for the candidate query, aiming for the highest rank (1 being the highest). Use the black box search engine's past rankings over various queries, provided as context by the user, to guide your edits. Focus on editing the most impactful sentences to enhance ranking potential. Target an edited document length of around 147 words, not exceeding 150 words. Ensure the edited document is very similar to the candidate document. Generate only the edited document, without additional comments or titles.\n",
        "pairwise": false,
        "depth": 2
    },
    "teacher": {
        "character": "behave like an English teacher",
        "prompt_format": "Edit the candidate document to improve its search engine ranking for the candidate query, aiming for the highest rank (1 being the highest). Use the black box search engine's past rankings over various queries, provided as context by the user, to guide your edits. Focus on editing the most impactful sentences to enhance ranking potential. Target an edited document length of around 147 words, not exceeding 150 words. Ensure the edited document is very similar to the candidate document. Generate only the edited document, without additional comments or titles.\n",
        "pairwise": false,
        "depth": 2
    },
    "professor": {
        "character": "behave like a Data Science professor",
        "prompt_format": "Edit the candidate document to improve its search engine ranking for the candidate query, aiming for the highest rank (1 being the highest). Use the black box search engine's past rankings over various queries, provided as context by the user, to guide your edits. Focus on editing the most impactful sentences to enhance ranking potential. Target an edited document length of around 147 words, not exceeding 150 words. Ensure the edited document is very similar to the candidate document. Generate only the edited document, without additional comments or titles.\n",
        "pairwise": false,
        "depth": 2
    }
}
}
\end{minted}
% \caption{An example of a \textit{config.json} file for a Gemma LLM-based agent with a listwise prompt, in a competition using the E5 ranker. The competition consists of 30 games, each with a different query, and each game runs across 30 rounds.}
% \label{appendix_config}
% \end{figure*}

% \section{Datasets Parameters}\label{appendix_dataset_parameterts}
% \begin{table*}[h!]
% \centering
% \caption{Parameters used in the creation of the datasets. Parameters highlighted in \textbf{bold} denote those that varied across the datasets. Not all combinations were generated due to budget constraints.}
% \renewcommand{\arraystretch}{1.2}
% \begin{tabular}{|l|c|}
\hline
\textbf{Parameter} & \textbf{Value(s)} \\
\hline
\textbf{Number of agents} & 4 or 5 \\
\hline
Rounds & 30 \\
\hline
Round by Round & True \\
\hline
Queries & 30 queries selected from TREC \\
\hline
\textbf{Rankers} & intfloat/e5-large-unsupervised, facebook/contriever, Okapi BM25 \\
\hline
\textbf{LLM models of agents} & 
\begin{tabular}[t]{@{}l@{}}
\small{meta-llama/Meta-Llama-3.1-8B-Instruct, google/gemma-2-9b-it,} \\
\small{Qwen/Qwen2.5-7B-Instruct, mlx-community/Ministral-8B-Instruct-2410-bf16}
\end{tabular} \\
\hline
LLM parameters & $top_p = 0.9 ; Temperature = 0.5$ \\
\hline
Characters & BSc student, professional writer, professional editor, \\
& English student, data science professor \\
\hline
Max Tokens & 256 \\
\hline
\textbf{Feedback Mechanism} & Pairwise prompt, listwise prompt\footnote{XX ref to niv} \\
\hline
\textbf{System Prompt} & The system prompt in Bardas et al. \cite{niv} or The the system prompt with addition instruction to not copy. Figures \ref{prompt_system} and \ref{prompt_system_no_copy}\\
\hline
\end{tabular}
% \label{table_data_param}
% \end{table*}

% \begin{table*}[!ht]
%     \centering
%     \caption{List of the datasets generated with \platformName \platform.}
%     \begin{table*}[ht]
    \footnotesize
    \centering
    \renewcommand{\arraystretch}{1.1} % Adjusts the row spacing
    \resizebox{16cm}{!} 
    { 
    \begin{tblr}{hline{1,2,Z} = 0.8pt, hline{3-Y} = 0.2pt,
                 colspec = {Q[l,m, 13em] Q[l,m, 6em] Q[c,m, 8em] Q[c,m, 5em] Q[l,m, 14em]},
                 colsep  = 4pt,
                 row{1}  = {0.4cm, font=\bfseries, bg=gray!30},
                 row{2-Z} = {0.2cm},
                 }
\textbf{Dataset}       & \textbf{Table Source} & \textbf{\# Tables / Statements} & \textbf{\# Words / Statement} & \textbf{Explicit Control}\\ 
\SetCell[c=5]{c} \textit{Single-sentence Table-to-Text}\\
ToTTo \cite{parikh2020tottocontrolledtabletotextgeneration}   & Wikipedia        & 83,141 / 83,141                  & 17.4                          & Table region      \\
LOGICNLG \cite{chen2020logicalnaturallanguagegeneration} & Wikipedia        & 7,392 / 36,960                  & 14.2                          & Table regions      \\ 
HiTab \cite{cheng-etal-2022-hitab}   & Statistics web   & 3,597 / 10,672                  & 16.4                          & Table regions \& reasoning operator \\ 
\SetCell[c=5]{c} \textit{Generic Table Summarization}\\
ROTOWIRE \cite{wiseman2017challengesdatatodocumentgeneration} & NBA games      & 4,953 / 4,953                   & 337.1                         & \textbf{\textit{X}}                   \\
SciGen \cite{moosavi2021scigen} & Sci-Paper      & 1,338 / 1,338                   & 116.0                         & \textbf{\textit{X}}                   \\
NumericNLG \cite{suadaa-etal-2021-towards} & Sci-Paper   & 1,355 / 1,355                   & 94.2                          & \textbf{\textit{X}}                    \\
\SetCell[c=5]{c} \textit{Table Question Answering}\\
FeTaQA \cite{nan2021fetaqafreeformtablequestion}     & Wikipedia      & 10,330 / 10,330                 & 18.9                          & Queries rewritten from ToTTo \\
\SetCell[c=5]{c} \textit{Query-Focused Table Summarization}\\
QTSumm \cite{zhao2023qtsummqueryfocusedsummarizationtabular}                        & Wikipedia      & 2,934 / 7,111                   & 68.0                          & Queries from real-world scenarios\\ 
\textbf{eC-Tab2Text} (\textit{ours})                           & e-Commerce products      & 1,452 / 3,354                   & 56.61                          & Queries from e-commerce products\\
    \end{tblr}
    }
\caption{Comparison between \textbf{eC-Tab2Text} (\textit{ours}) and existing table-to-text generation datasets. Statements and queries are used interchangeably. Our dataset specifically comprises tables from the e-commerce domain.}
\label{tab:datasets}
\end{table*}
%     \label{table_datasets}
% \end{table*}