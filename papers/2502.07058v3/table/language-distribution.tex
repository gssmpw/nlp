% Please add the following required packages to your document preamble:
% \usepackage{booktabs}
\begin{table}[]
\centering
\footnotesize
\addtolength{\tabcolsep}{-1.2mm}
\begin{tabular}{@{}l@{\kern0pt}rrrr@{}}
\toprule
\multirow{2}{*}{\textbf{Category}} & \multicolumn{2}{c}{\textbf{CN}} & \multicolumn{2}{c}{\textbf{TW}} \\ \cmidrule(lr){2-3} \cmidrule(lr){4-5}
 & \multicolumn{1}{c}{\textbf{Count}} & \multicolumn{1}{c}{\textbf{Ratio}} & \multicolumn{1}{c}{\textbf{Count}} & \multicolumn{1}{c}{\textbf{Ratio}} \\ \midrule
\textbf{Only Traditional} & 2,000 & 8.73\% & 17,130 & 74.74\% \\
\textbf{Only Simplified} & 15,816 & 69.01\% & 90 & 0.39\% \\
\textbf{Only English }& 107 & 0.47\% & 119 & 0.52\% \\
\textbf{Only Emoji }& 1 & 0.00\% & 4 & 0.02\% \\
\textbf{Only Symbol }& 1 & 0.00\% & 5 & 0.02\% \\
\textbf{Only Bopomofo} & 4 & 0.02\% & 35 & 0.15\% \\
\textbf{Only JP/KR} & 0 & 0.00\% & 0 & 0.00\% \\
\textbf{Only Punctuation} & 4 & 0.02\% & 5 & 0.02\% \\
\textbf{Only Unknown} & 0 & 0.00\% & 0 & 0.00\% \\ \midrule
% \textbf{Non-Chinese} & 868 & 3.79\% & 457 & 1.99\% \\  \midrule
\textbf{Traditional + English} & 251 & 1.10\% & 3,022 & 13.19\% \\
\textbf{Traditional + Emoji} & 75 & 0.33\% & 666 & 2.91\% \\
\textbf{Traditional + Symbol} & 79 & 0.34\% & 894 & 3.90\% \\
\textbf{Traditional + Bopomofo} & 8 & 0.03\% & 66 & 0.29\% \\
\textbf{Traditional + JP/KR} & 0 & 0.00\% & 9 & 0.04\% \\
\textbf{Traditional + Unknown} & 30 & 0.13\% & 246 & 1.07\% \\ \midrule
\textbf{Simplified + English} & 2,681 & 11.70\% & 12 & 0.05\% \\
\textbf{Simplified + Emoji} & 383 & 1.67\% & 1 & 0.00\% \\
\textbf{Simplified + Symbol} & 323 & 1.41\% & 0 & 0.00\% \\
\textbf{Simplified + Bopomofo} & 0 & 0.00\% & 0 & 0.00\% \\
\textbf{Simplified + JP/KR} & 22 & 0.10\% & 0 & 0.00\% \\
\textbf{Simplified + Unknown} & 90 & 0.39\% & 1 & 0.00\% \\ \bottomrule
\end{tabular}
\addtolength{\tabcolsep}{+1.2mm}
\caption{Language distribution. CN and TW users similarly mix non-Chinese elements with their primary writing systems (Simplified or Traditional Chinese). However, CN users incorporate Traditional characters more frequently than TW users use Simplified ones.}
\label{tab:language-distribution}
\end{table}



\begin{comment}
\begin{table}[t]
\centering
\small
\begin{tabular}{@{}ccccc@{}}
\toprule
\multicolumn{1}{l}{} & \textbf{Traditional} & \textbf{Simplified} & \textbf{Both} & \textbf{Others} \\ \midrule
\textbf{TW} & 22,033 & 104 & 351 & 430 \\
\textbf{CN} & 2,443 & 19,315 & 742 & 418 \\ \bottomrule
\end{tabular}
\caption{Language distribution. Both means that both the traditional and simplified Chinese appear in the text. Others mean that there are other languages involved or the language is unknown.}
\label{tab:language-distribution}
\end{table}
\end{comment}