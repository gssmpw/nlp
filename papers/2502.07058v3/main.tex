% This must be in the first 5 lines to tell arXiv to use pdfLaTeX, which is strongly recommended.
\pdfoutput=1
% In particular, the hyperref package requires pdfLaTeX in order to break URLs across lines.

\documentclass[11pt]{article}

% Change "review" to "final" to generate the final (sometimes called camera-ready) version.
% Change to "preprint" to generate a non-anonymous version with page numbers.
\usepackage[final]{acl}

% Standard package includes
\usepackage{times}
\usepackage{latexsym}

% For proper rendering and hyphenation of words containing Latin characters (including in bib files)
\usepackage[T1]{fontenc}
% For Vietnamese characters
% \usepackage[T5]{fontenc}
% See https://www.latex-project.org/help/documentation/encguide.pdf for other character sets

% This assumes your files are encoded as UTF8
\usepackage[utf8]{inputenc}

% This is not strictly necessary, and may be commented out,
% but it will improve the layout of the manuscript,
% and will typically save some space.
\usepackage{microtype}

% This is also not strictly necessary, and may be commented out.
% However, it will improve the aesthetics of text in
% the typewriter font.
\usepackage{inconsolata}

%Including images in your LaTeX document requires adding
%additional package(s)
\usepackage{graphicx}

\author{
Zixin Tang\textsuperscript{1}~~~Chieh-Yang Huang\textsuperscript{2}~~~Tsung-Che Li\textsuperscript{3}~~~Ho Yin Sam Ng\textsuperscript{1}\\
\textbf{Hen-Hsen Huang\textsuperscript{3}~~~Ting-Hao `Kenneth' Huang\textsuperscript{1}}\\
\textsuperscript{1}College of Information Sciences and Technology, The Pennsylvania State University\\
\textsuperscript{2}MetaMetrics Inc.~~~\textsuperscript{3}Institute of Information Science, Academia Sinica\\
\textsuperscript{1}\texttt{\{zxtang,sam.ng,txh710\}@psu.edu}~~~\textsuperscript{2}\texttt{cyhuang@lexile.com} \\
\textsuperscript{3}\texttt{\{george,hhhuang\}@iis.sinica.edu.tw}
}

%%%%%%%%%%%%%%%%%%%%%%%%%%%%%%%%%%%%%%%%%%%%%%%%%%%%%%%
%%%%%%%%%%%%%%%    theorems %%%%%%%%%%%%%%%%%%%%%%%%%%%
%%%%%%%%%%%%%%%%%%%%%%%%%%%%%%%%%%%%%%%%%%%%%%%%%%%%%%%
% \usepackage{mdframed}
\usepackage{kantlipsum}

%%%%%%%%%%%%%%%%%%%%%%%%%%%%%%%%%%%%%%%%%%%%%%%%%%%%%%%
%%%%%%%%%%%%%%%    theorems %%%%%%%%%%%%%%%%%%%%%%%%%%%
%%%%%%%%%%%%%%%%%%%%%%%%%%%%%%%%%%%%%%%%%%%%%%%%%%%%%%%
\theoremstyle{plain}
\newtheorem{theorem}{Theorem}[section]
\newtheorem{proposition}[theorem]{Proposition}
\newtheorem{lemma}[theorem]{Lemma}
\newtheorem{example}[theorem]{Example}
\newtheorem{corollary}[theorem]{Corollary}
\theoremstyle{definition}
\newtheorem{definition}[theorem]{Definition}
\newtheorem{assumption}[theorem]{Assumption}
\theoremstyle{remark}
\newtheorem{remark}[theorem]{Remark}


% \titleformat{\subsection}[runin]% runin puts it in the same paragraph
%        {\normalfont\bfseries}% formatting commands to apply to the whole heading
%        {\thesubsection}% the label and number
%        {0.5em}% space between label/number and subsection title
%        {}% formatting commands applied just to subsection title
%        [.]% punctuation or other commands following subsection title


%%%%%%%%%%%%%%%%%%%%%%%%%%%%%%%%%%%%%%%%%%%%%%%%%%%%%%%
%%%%%%%%%%%%%%%  mathematical notations%%%%%%%%%%%%%%%%
% \usepackage[english]{babel}
% \usepackage[utf8]{inputenc}
% \usepackage[T1]{fontenc}

%% Figures, tables and lists
\usepackage[dvipsnames]{xcolor}
\usepackage{paralist}
\usepackage{graphicx}
\usepackage{subcaption}
\usepackage{longtable} 
\usepackage{multirow}
\usepackage{listings}
\usepackage{makecell}
\usepackage{array}
\usepackage{float}
\usepackage{dsfont}
\usepackage{rotating}
\usepackage{booktabs}
\usepackage{enumerate}
\usepackage{tikz}
\usepackage{pgf}
\usepackage{enumitem}
\usepackage{lipsum} % for generating filler text
\usepackage{titlesec}

%% Math
% \usepackage{amssymb, amsthm,bbm}
\usepackage{mathtools}
\usepackage{mathrsfs}
%% References and author info 
\mathtoolsset{showonlyrefs}
\usepackage{natbib}
\usepackage{authblk}
\usepackage{todonotes}
\usepackage{xr-hyper}


%%%%%%%%%%%%%%%%%%%%%%%%%%%%%%%%%%%%%%%%%%%%%%%%%%%%%%%
\newcommand{\R}{\mathbb R}
\newcommand{\EE}{\mathbb{E}}

\DeclareMathOperator{\Tr}{Tr}
\DeclareMathOperator*{\argmin}{argmin}
\DeclareMathOperator*{\argmax}{argmax}

\newcommand{\bs}[1]{\ensuremath{\boldsymbol{#1}}}
\newcommand{\mc}{\mathcal}
\newcommand{\opt}{^\star}


\newcommand{\diff}{\textnormal{d}}


\def \iid {\stackrel{\textnormal{i.i.d.}}{\sim}}
\def \iidtext {\textnormal{i.i.d.}}





%%%%%%%%%%%%%%%%%%%%%%%%%%%%%%%%%%%%%%%%%%%%%%%%%%%%%%%
%%%%%%%%%%%%%%%%%%%%% colors     %%%%%%%%%%%%%%%%%%%%%%
%%%%%%%%%%%%%%%%%%%%%%%%%%%%%%%%%%%%%%%%%%%%%%%%%%%%%%%
\definecolor{myblue}{rgb}{.8, .8, 1}
\definecolor{mathblue}{rgb}{0.2472, 0.24, 0.6} % mathematica's Color[1, 1--3]
\definecolor{mathred}{rgb}{0.6, 0.24, 0.442893}
\definecolor{mathyellow}{rgb}{0.6, 0.547014, 0.24}


% May add more in future.







\begin{document}

\maketitle

%\kenneth{Editorial note 1: Please use these custom commands: \twChinese, \cnChinese, \twChineseFull, \cnChineseFull}

%\kenneth{Editorial note 2: For consistency, we use ``language variety'' instead of ``language variation''. Janet suggested we stick with ``language variety''.}


\begin{abstract}
\begin{abstract}
Out-of-distribution (OOD) detection and OOD generalization are widely studied in Deep Neural Networks (DNNs), yet their relationship remains poorly understood. We empirically show that the degree of Neural Collapse (NC) in a network layer is inversely related with these objectives: stronger NC improves OOD detection but degrades generalization, while weaker NC enhances generalization at the cost of detection. This trade-off suggests that a single feature space cannot simultaneously achieve both tasks. To address this, we develop a theoretical framework linking NC to OOD detection and generalization. We show that entropy regularization mitigates NC to improve generalization, while a fixed Simplex Equiangular Tight Frame (ETF) projector enforces NC for better detection. Based on these insights, we propose a method to control NC at different DNN layers. In experiments, our method excels at both tasks across OOD datasets and DNN architectures. 

\begin{comment}   

Out-of-distribution (OOD) detection and OOD generalization are critical for deploying machine learning models in real-world scenarios. While substantial progress has been made in addressing these problems independently, few works have attempted to tackle them jointly. However, existing methods often rely on auxiliary OOD training data and primarily focus on covariate-shifted OOD data that share labels with in-distribution (ID) data. In contrast, we tackle the more realistic and challenging task of jointly detecting and generalizing to semantic OOD data with disjoint labels from the ID data, without auxiliary OOD training data.
Achieving both objectives simultaneously is inherently difficult due to a fundamental conflict — OOD generalization requires enhanced transferability, while OOD detection necessitates the inhibition of transfer.
To address this, we leverage insights from neural collapse (NC) — a phenomenon in deep networks where top-layer representations suppress feature variability and adopt a Simplex Equiangular Tight Frame (ETF) structure, impairing transferability. By controlling NC, we unify OOD detection and generalization: preventing NC enhances OOD transfer while inducing NC improves OOD detection.
Our proposed method excels at both tasks across various OOD datasets and architectures. 

\end{comment}


\end{abstract}
\end{abstract}


%\kenneth{Again, we can use custom commands defined in micro.tex to add comments.}
%\zixin{Like this!}

%\kenneth{I think the high-level, central theme of our paper is this: How do you "fairly" compare a model's performance across language variations? An (unspoken) underlying argument we want to push back is this: "When you have big enough data---of popular, mainstream variation---and a big enough model, NLP tasks in less popular variations will be automatically solved altogether." In order to push back this argument, we want to construct a robust, fair, and measurable evaluation to pinpoint the performance gap.\\\\
%Although we might be able to learn some "methods" in machine translation papers on how they compared a model's performance across different languages, there's a fundamental difference between benchmarking across variations and benchmarking across languages: People just KNOW English and Spanish are different. Machine-translation people do not need to care as much (in some cases, they do care!) about having a very very "fair" comparison across different languages because... people just know different languages are different.}

\section{Introduction}
\section{Introduction}

% State of the world (robots for creative activites)
The term ``robot,'' originally signifying `forced labor,' has long been associated with labor and work. Robots have demonstrated their utility in various automated productive and social contexts, where the primary goals are improving productivity, safety, and fostering social interactions with humans~\cite{simoes2022designing, weidemann2021role, honig2018understanding}. However, an increasing number of cases feature using of robots in creative settings. Unlike productive contexts, where the focus is on efficiency and task completion~\cite{arents2022smart}, or social contexts, where communication and trust are prioritized~\cite{nam2020trust, saunderson2019robots}, creative environments prioritize artistic innovation and expression~\cite{hsueh2024counts}. This shift fundamentally alters the dynamics of human-robot interaction, redefining the roles and expectations for both humans and robots.

For instance, robots’ social behaviors are leveraged to support the generation and expression of creative ideas~\cite{hu2021exploring, sandoval2022human, alves2020creativity}, and programmable robotic movements and trajectories are employed to inspire artistic activities such as sketching~\cite{lin2020your}. These studies often engage participants from creative fields who possess limited prior experience with robotics, and are typically conducted in short-term, experimental settings. Consequently, the findings from these studies remain constrained since much can be learned from professional practitioners' experiences to inform system design such as digital fabrication~\cite{hirsch2023nothing}. There is a notable gap in research examining the long-term, active, and practical experience of integrating robotic systems into the creative processes. As a result, the deeper insights into how robots facilitate and shape creative processes, beyond simply augmenting human creativity, remain underexplored. In this study, we aim to better understand the impacts of robots on creative processes and outcomes.

As early as Leonardo da Vinci's 16th century ``Automaton,'' artists have explored the creative affordances of robotic systems~\cite{shanken2002cybernetics, pagliarini2009development, jeon2017robotic}. The artistic creation process typically encompasses various stages, including the exploration of materials and techniques, ongoing experimentation and iteration, and the continual refinement of the artists' insights into their creative subjects~\cite{lewis2023art, sturdee2022state}. Therefore, investigating the artistic process involving robots offers an opportunity to gain deeper insights into robots' creative potential. Robotic art, in particular, provides a compelling case for this exploration.

We define robotic art as artworks that utilize robotic or automated machines to create artistic experiences and tangible artifacts. One example is robotic installation art, in which robots are programmed to follow specific rules that embody the artist’s expression (\autoref{fig:teaser} (a)). Another example is responsive art, in which robots react to their environment, with behaviors that change over time or in response to spectators (\autoref{fig:teaser} (b)). Additionally, there are robotic creators, which possess a degree of agency, allowing them to collaborate with human artists and produce works that extend beyond mere replication of human-created art (\autoref{fig:teaser} (c) and (d)). As such, robotic art becomes a rich case for exploring human-machine interactions in creative contexts. Gaining a deeper understanding of how robots facilitate artistic expression can provide insights for designing computing systems to support creative activities~\cite{gomez2021robot}.

% Therefore, we did...
We draw on semi-structured, in-depth interviews with renowned professional robotic artists to investigate the use of robots in artistic practice. Specifically, our goal is to understand how artistic exploration of robotic systems challenges conventional assumptions about the functions of robots, such as their roles in automating repetitive tasks or serving human needs. We also explore the implications of robots in the artistic process and examine how creativity may emerge within robotic art. To address these interrelated inquiries, our study focuses on the practice of robotic art, posing the research question: \textit{How do robotic artists utilize robots in their artistic practice?} We approach this inquiry through the perspectives and experiences of robotic artists, who creatively design, modify, and repurpose robotic systems for artistic expression and exploration.

% The key findings are...
Our findings highlight the social, material, and temporal dimensions of artists' practices that shape their creativity and artistic outcomes. The creation of robotic art is largely a social process, as artists receive both explicit and implicit feedback through the audience's reactions and reception of their work. Simultaneously, the embodiment and malfunctions inherent to robotic systems drive artistic experimentation. The temporal processes of creation and exhibition, beyond just the final product, further enhance the creative value. Our empirical analysis presents how creativity emerges through the interplay of social, material, and temporal interactions among artists, robots, audiences, and the environment.

% The contributions of this work are...
We make two main contributions to HCI in this study. 
First, we elucidate the interactive mechanisms among key actors---human creators, machines, audiences, and environments---within the practice of robotic art, a topic that remains underexplored in HCI. Our findings reveal the significance of sociality (e.g., interactions between artists and audiences), materiality (e.g., the embodiment and malfunctions of robots), and temporality (e.g., the processes of creation and exhibition) in shaping creative values. We propose that these three facets are central to the creative process and facilitate the emergence of creativity in robotic art.
Second, drawing from the findings, we offer implications for \textit{socially informed}, \textit{material-attentive}, and \textit{process-oriented} creation with computing systems. We suggest leveraging these three aspects to enhance creativity and the creative experience. Specifically, we discuss the value of incorporating implicit audience feedback, designing with technical malfunctions, and focusing on the post-creation process to foster alternative creative experiences with machines~\cite{alter2010designing, juarez2022glitch}.




\section{Related Work}
%\subsection{End-User Programming}

%\kenneth{The way I like to think about Related Work is that this section should (sometimes subtly, not explicitly, but effectively!) answer some underlying questions that reviewers might want to ask. So, here we go:}\steven{sounds good!}

\subsection{Ways of Optimizing Prompts for LLMs}
%\subsection{Prompt Engineering and How Good Humans Are at It}
Prompts are the primary means by which users interact with, utilize, and instruct LLMs. 
Since the emergence of these models, researchers and developers have invested significant effort into understanding how to craft better prompts for more effective use. 

\paragraph{Automatic Prompt Optimization.}
Much of the prior work has focused on automatically optimizing prompts. 
A common theme across these studies is the use of gold-standard labels to guide the optimization process.
For example, \citet{pryzant2023automatic} introduced an automatic prompt optimization method inspired by gradient descent; 
\citet{manas2024improving} presented an approach that begins with a user prompt and iteratively generates revised prompts to maximize consistency between the generated image and prompt, without human intervention; 
\citet{wan2024teach} explored two types of prompt optimization, instruction and exemplar, and suggested that combining both can yield optimal results; 
\citet{sun2023autohint} combined zero-shot and few-shot learning to optimize prompts automatically; %eliminating the need for manual prompt engineering; 
and \citet{levi2024intent} improved prompt optimization through synthetic data generation and iterative refinement, focusing on aligning prompts with user intent by creating challenging boundary cases for iterative prompt refinement.
While these studies were interesting and relevant, they generally assumed the availability of gold-standard labels and did not address situations where labels are absent or where standards are constantly evolving.

\paragraph{User-Driven Prompt Optimization.}
In addition to automatic prompt optimization, some research has focused on human capabilities in optimizing prompts. 
\citet{zhou2023revisiting} found that manual prompting often outperforms automated methods in various scenarios; 
\citet{10.1145/3544548.3581388} discovered that people tend to design prompts opportunistically rather than systematically, which often leads to lower success rates. 
To the best of our knowledge, the most relevant prior work is by \citet{wang2024end}, who developed an iterative refinement system that enables users to prompt LLMs to build a personalized classifier for social media content. 
Their study explored three user strategies for improving prompts and measured their effectiveness. 
While conceptually related to our work, their focus was not on how users evolve their understanding and expectations when interacting with LLMs. 
Instead, participants labeled ground truth at the beginning of the study, prior to using the system.



%--------------------- dead kitten --------------
\begin{comment}
 





The most relevant prior work is by \citet{wang2024end}, who developed an iterative refinement system allowing users to prompt LLMs to build a personalized classifier for social media content.
While their work is closely related to ours in concept, their study did not focus on how users evolve their understanding and expectations while working with LLMs. 
Instead, participants labeled ground truth at the outset before using the system.


\kenneth{The key question for our paper is this: Did prior work try to measure users' prompt engineering performance *over multiple iterations*? (What do we know about human performance in prompt engineering?) I think you can maybe find some papers, especially papers for automatic prompt optimization like DSPy, measuring users' individual prompt's output accuracy (or MSE) or performance (e.g., BLEU in generation task), but it might be hard to find papers capture and measure *multiple iterations* from the same user for the same prompt.--This is the main argument for our paper: we did something that was hard and thus has not been done.}

\kenneth{Take a look at this survey paper:~\cite{chen2023unleashing}}



\steven{iterative tool involve human}
PromptIDE is an interactive tool that helps the experts to iteratively refine tools by providing various prompts, visualizing their performance on small validation datasets, and iterative optimizing them based on quantitative feedback~\cite{strobelt2022interactive}. \steven{gold label exists}

PromptAID is a visual analytics system that helps non-experts iteratively improve prompts through exploration, perturbation, testing, and refinement. It supports prompts through keyword adjustment, paraphrasing, and adding few-shot examples. \steven{has test dataset, it is a complex system}

\steven{automate prompting}
\citet{pryzant2023automatic} introduces an automatic prompt optimization prompt inspired by gradient descent. \steven{this fell into software designing, involve gold labels}

The study starts from a user prompt and iteratively generates revised prompts with the goal of maximizing a consistency score between the generated image and prompt without a human in the loop\cite{manas2024improving}\steven{without human involvement in the loop, gold labels}

\citet{zhou2023revisiting} found that manual prompting often performed better than automated methods in various steps. 

\cite{wan2024teach} explores the distinction between two types of prompt optimization: instruction optimizer and exemplar. This study suggested combining both approaches could lead to optimal results.

\cite{sun2023autohint} combines zero-shot and few-shot learning to optimize prompts automatically, without manual efforts in prompt engineering.

\cite{levi2024intent} improve prompt engineering optimization by synthetic data generation and iterative refinement, focusing on aligning prompts with user intent by generating challenging boundary cases and using these to refine the prompt iteratively.





\paragraph{Prompt Engineering Tools.}
\kenneth{After making the first point, we can have a follow-up paragraph to say that many tools were created to help people do prompt engineering (list a few and name their focuses), but again, they did not focus on measuring how good humans are in prompt engineering--- Of course, there could be an argument that suggests: no matter how good you are, you will always need some tool. It is true---for example, ChainForge basically create a easy-to-use UI that make things easier, which is not really about accuracy---But for annotation tasks, performance is still critical and it is always good to know how well human did, almost like many AI leaderboard has various "human" performance for comparison.}
PromptMaker, a platform for rapidly prototyping new ML models using prompt-based programming, was difficult to evaluate their prompts systematically~\cite{10.1145/3491101.3503564}.

\cite{arawjo2024chainforge}  is an Open-source visual toolkit for prompt engineering and on-demand hypothesis testing of text-generation LLMs.

 promptfoo is test-driven LLM development, not trial-and-error, producing matrix views that let you quickly evaluate outputs across many prompts~\cite{webster2023promptfoo}.

\cite{madaan2024self} introduces a method that LLM iterative improve their output by using their own feedback, without external supervision. 

\saniya{austin etal points:
1. used only COPRO, evaluation criteria utilized a custom LLM-as-a-judge metric. The paper showed that their automated prompt optimizer worked better tha DSPy }
   
\end{comment}


\subsection{Tools for Prompt Engineering}
With the advances in LLMs, numerous tools have been developed to assist with prompt engineering. 
Most of these tools follow a software-engineering paradigm, where testing (such as unit tests or integration tests) is a central concept, and thus often assume the existence of gold-standard labels.
For example, PromptIDE is an interactive tool that helps experts iteratively refine prompts by providing various inputs, visualizing their performance on small validation datasets, and optimizing them based on quantitative feedback~\cite{strobelt2022interactive}; 
PromptAid is a visual analytics system for interactively creating, refining, testing, and iterating prompts while tracking accuracy changes~\cite{mishra2023promptaid};
%It allows users to adjust prompts through keyword modifications, paraphrasing, and adding few-shot examples; 
ChainForge is an open-source visual toolkit for prompt engineering and on-demand hypothesis testing of text-generation LLMs~\cite{arawjo2024chainforge};
and, promptfoo applies a test-driven approach to LLM development, producing matrix views that enable quick evaluation of outputs across multiple prompts~\cite{webster2023promptfoo}.
While these tools are inspiring and valuable, the scenarios we focus on do not rely on the constant availability of gold labels.

%\cite{mishra2023promptaid}


\begin{comment}






\kenneth{In here, we want to answer this questions: Why do we need to built \system? Can't we just use some existing tools??? The underlying answer could be: all the tools, including the one we mentioned in previous subsection, were not really aiming for ``general users'' and only thing general users can reliably use is probably chat interface come with ChatGPT etc.}

\citet{10.1145/3544548.3581388} mentioned that people tended to design prompts opportunistically, not systematically, which resulted in less success. \system provides a systematic process for composing and refining prompts, allowing non-expert users to adapt to the prompt creation process effortlessly.

\saniya{Amy Zhang points:
\newline 1. Accuracy didnot improve; reported improvements in recall
\newline 2. Observed that humans are pretty bad at being consistent
\newline 3. Quoted  Miles Turpin, Julian Michael, Ethan Perez, and Samuel Bowman. 2024. Language models don't always say what they think: unfaithful explanations
in chain-of-thought prompting. Advances in Neural Information Processing Systems 36 (2024).
Han Wang, Ming Shan Hee, Md Rabiul Awal, Kenny Tsu Wei Choo, and Roy Ka-Wei Lee. 2023. Evaluating GPT-3 Generated Explanations for
Hateful Content Moderation. arXiv:2305.17680 [cs.CL] for not using LLM prompt explanations
\newline 4. They had a bigger training set of around 700 examples: paper excerpt: "This process resulted in a balanced dataset of 800 comments. We randomly divided our dataset into a training dataset and a test dataset of 100 examples for each participant. The training dataset was used to help participants create their classiiers, whereas the test dataset was labeled by participants and used to evaluate their created classiiers."
}
    
\end{comment}

\subsection{Human-LLM Collaborative Data Annotation}
%Another relevant area of research involves using LLMs for data annotation. 
Beyond simply treating LLMs as automatic labelers---common in countless NLP projects~\cite{tan2024large}---a growing body of work explores how to combine human and LLM efforts to achieve better annotation outcomes, such as improved accuracy or speed.
Even as LLMs outperform humans in many labeling tasks, human-AI collaboration often produces better results than either alone~\cite{vaccaro2024combinations}.
For example, \citet{kim2024meganno+} introduced a human-LLM collaborative annotation system where LLMs handle bulk annotation tasks, while humans selectively verify and refine the annotations. 
%\steven{However, this system was limited to deployment within Jupyter Notebook, lacking an end-to-end solution. This design imposed significant barriers, as it required users to possess technical expertise for system setup before using the tool, limiting accessibility and scalability in non-technical domains.}
\citet{goel2023llms} proposed an approach that combines LLMs with human expertise to efficiently generate ground truth labels for medical text annotation.
Additionally, \citet{10.1145/3613904.3642834} demonstrated how aggregating crowd workers' labels with GPT-4's output can achieve higher labeling accuracy than either source alone.
These studies generally aim to split the workflow of data labeling between humans and LLMs in a smart way, making the task more effective or efficient. 

In contrast, our work does not focus on dividing or combining the workload, but on how humans can teach LLMs---through prompt refinement---to better label the specific type of data.
Few prior studies have emphasized iterative prompt refinement in human-LLM collaborative data annotation.
For example, \citet{liu2024harnessing} developed a workflow for video content analysis, refining prompts to improve LLM-generated annotations and align them with human judgment.
Additionally, \citet{zhang2023llmaaa} proposed LLMAAA, which uses LLMs as annotators in a feedback loop to label data efficiently.
Their study shows that poorly designed prompts result in subpar performance, especially in complex tasks. %while incorporating demonstrations and aligning label descriptions with natural language significantly enhances accuracy and reliability.
Our work advances this relatively understudied area of human-LLM collaborative annotation research.

%----------------------------- dead kitten --------------------------------

\begin{comment}








\steven{\citet{vaccaro2024combinations} emphaized that designing innovative processes for integrating humans and AI is as critical as developing advanced AI technologies. This aligns with the need for LLM-powered systems that iteratively guide AI outputs to meet user-specific standards, prioritizing effective collaboration between users and AI systems.}

\steven{\citet{liyanage2024gpt} found that GPT-4, using few-shot, zero-shot, and Chain-of-Thoughts (CoT) prompting techniques, could not outperform models fine-tuned on human-labeled data. Among these, the few-shot approach exhibited the highest degree of similarity to human annotations. However, in scenarios where gold labels are unavailable, fine-tuning is not applicable, and alternative methods must be explored.}

\steven{\citet{liu2024harnessing} developed a workflow for video content analysis, iteratively crafting prompts to enhance LLMs' ability to generate structured annotations and comprehensive explanations that aligned with human judgment. }

\steven{\citet{zamfirescu2023herding} found that while prompts can effectively address most UX goals, they struggle with nuanced, edge-case, or spontaneous interactions. The study highlights that the effectiveness of each instruction in the prompt is highly sensitive to its phrasing and location. Additionally, highly prescriptive prompts, though reliable, limited the spontaneity and flexibility of GPT responses.
In our system, users are only required to provide task information—such as task descriptions, rules, and examples—to construct instructions, allowing for greater flexibility in accommodating diverse task requirements..}

\steven{\citet{guyre2024prompt} illustrates how prompt engineering can empower non-experts to design tailored conversational agents by iteratively refining prompts and infusing domain-specific knowledge. Their study emphasizes democratizing chatbot development, allowing users to align AI behavior with their specific goals and values.}

\steven{\citet{zhang2023llmaaa} proposes LLMAAA that leverages LLMs as Active Annotators in a feedback loop to efficiently annotate data. The study highlights that poorly designed prompts lead to suboptimal performance by LLM annotators, particularly in complex or domain-specific tasks. However, incorporating demonstrations and aligning label descriptions with natural language significantly enhances annotation accuracy and reliability.}

%\kenneth{Here, we then answer this question: Did people create ANYTHING to support LLM-powered data annotation? There are two parts of the answer to this: 1) Many or even most papers, including our CHI paper last year, focus on the labeling performance of LLMs, for example, as compared to crowdsourcing. They did not focus on the UI aspect of it. 2) Some prompt chaining tools, like ChainForge, can support workflow like this, but (a) hey do not focus on data annotation in particular so some functions are missing, like data resampling, and (b) more importantly, they do not aim to support general users. Most of them expect you to know some programming, e.g., ChainForge clearly say it's a visual programming tool. They're not really aiming for generic users.}


\cite{kim2024meganno+} introduced a human-LLM collaborative annotation system that allows LLM to handle bulk annotation tasks while humans verify selectively to refine annotation. 

\cite{goel2023llms} introduced an approach that combines LLM wth human expertise to create an efficient method for generating
ground truth labels for medical text annotation.


\cite{shankar2024validates} introduced a tool, EvalGen, to address the challenge of validating LLM. 
EvalGen helps users design evaluation criteria for LLM outputs and align that evaluation with human preferences through a mixed-initiative system.
A key finding is the concept of criteria drift, where users modify their evaluation standards while grading outputs. 


\cite{brade2023promptify} Promptify utilizes an LLM-powered suggestion engine to help users quickly explore and craft diverse prompts for text-to-image generation tasks.

    
\end{comment}


%\subsection{Survey Study in Data Annotation}
%\steven{
We conducted a survey study to investigate how individuals interact with LLMs and utilize gold-standard labels in the data annotation process. 
The participants primarily represent roles in research, machine learning engineering, and software development. \\
\textbf{Workflows: }Participants described diverse workflows for integrating LLMs into data annotation process, highlighting a common iterative and human-in-the-loop approach. \textbf{Most workflows begin with manual annotation of a small subset of data to establish a baseline.} Participants then employ prompt engineering, iteratively refining LLM prompts by evaluating their performance against the manually annotated subset. \\
Once refined, the prompts are used to label larger datasets, with participants using tools or manual checks to review the LLM's annotations and identify any invalid labels. The process is typically concluded with a thorough manual verification of the dataset. \\
One participant mentioned they manually tabulate data points along with their descriptions. \\
\textbf{Initialize Prompting: }Most participants use their pre-defined prompts to initialized the annotation on their known tasks. 
For new tasks, one participant mentioned that they initialize the annotation process with LLMs by starting with a clear problem definition and iteratively refining a classification-based approach. For less familiar tasks, some participants may seek suggestions from the LLM to guide the initial setup.
\textbf{Revising Prompt: } Participants use a small dataset to finetune the prompt. They address issues by adding rules or context examples to tackle failure cases. When inconsistencies or error arise, they revisit and recheck the manually tagged dataset to improve performance. Some participants also engage the LLM by asking questions about data points and their descriptions, retraining to against inconsistencies to minimize hallucinations and enhance annotation reliability.
}

\subsection{Gold-Standard Labels in Annotation Tasks}\label{sec:related-work-gold-label}
Decades of research have shown that gold-standard labels play a critical role in quality control for data annotation pipelines~\cite{han2020crowd,gadiraju2015training,le2010ensuring,doroudi2016toward,hettiachchi2021challenge}.
Embedding items with known labels into the data annotation process allows requesters to reliably capture quality signals, 
such as workers' level of expertise~\cite{abraham2016many, abassi2019worker, yang2018improving} %\kenneth{TODO: Add refs about using gold labels to decide workers' expertise level}\steven{added}
or attentiveness to tasks~\cite{hettiachchi2021challenge, oleson2011programmatic}. %\kenneth{TODO: Add refs about using gold labels to do attention checks for workers}\steven{added}
These insights enable requesters to take appropriate actions, such as 
retraining annotators~\cite{le2010ensuring, doroudi2016toward,hettiachchi2021challenge}, %\kenneth{TODO: Add refs about retraining workers}\steven{added}
removing low-performing workers~\cite{10.1145/3613904.3642834, snow2008cheap,downs2010your,le2010ensuring}, %\kenneth{TODO: Add refs about removing or blocking low-performing workers}\steven{added}
or identifying potential issues in the annotation interfaces~\cite{toomim2011utility,10.1145/3613904.3642834, rahmanian2014user, komarov2013crowdsourcing}. %\kenneth{TODO: Add refs for crowd worker interfaces. At least cite: Toomim, M., Kriplean, T., Pörtner, C., \& Landay, J. (2011, May). Utility of human-computer interactions: Toward a science of preference measurement. In Proceedings of the SIGCHI Conference on Human Factors in Computing Systems (pp. 2275-2284).}\steven{added}
Gold labels are also beneficial for requesters during post-annotation data processing. 
They can be used to weight labels from different workers in label aggregation~\cite{abassi2017gold,abassi2019worker}, %\kenneth{TODO: Add label aggregation methods that use gold labels particularly to weight different workers}\steven{added}
improve label aggregation strategies~\cite{khattak2011quality, snow2008cheap},  %\kenneth{TODO: Add label aggregation methods that learn whatever from gold labels}\steven{added}
or 
exclude unreliable workers' outputs entirely~\cite{abassi2019worker}. %\kenneth{TODO: Cite ref using gold labels to remove workers from label aggregation}\steven{added}
Beyond requesters, gold labels are also beneficial for data labelers like crowd workers. 
Gold labels can be used to train workers~\cite{doroudi2016toward, le2010ensuring, gadiraju2015training,han2020crowd}, %\kenneth{TODO: Cite ref that uses gold labels for worker training}\steven{added}
provide real-time feedback to help them recalibrate their understanding of the task~\cite{le2010ensuring,hettiachchi2021challenge}, %\kenneth{TODO: Cite the visible gold paper from Amazon}\steven{added}
or remind them to pay more attention~\cite{ hettiachchi2021challenge,oleson2011programmatic}. %\kenneth{TODO: Cite attention check papers}\steven{amazon paper also warn workers in real time}

While gold labels are useful for quality control, as stated in the Introduction (Section~\ref{sec:intro}), %\kenneth{TODO: Update references}\steven{done}
they are not always available in real-world scenarios due to constraints such as data privacy or the cost of gathering gold labels~\cite{liu2019deep, yang2019evaluating, oikarinen2021detecting, slote2024unlocking}.
To address these challenges, researchers have developed methods to generate (approximations of) quality signals without gold labels. 
In the realm of LLM-powered data annotation, for instance, CoPrompter evaluates how well an LLM's output aligns with user-specified requirements as a feedback mechanism~\cite{joshi2024coprompter}. %\kenneth{TODO: Cite: Joshi, I., Shahid, S., Venneti, S., Vasu, M., Zheng, Y., Li, Y., ... \& Chan, G. Y. Y. (2024). CoPrompter: User-Centric Evaluation of LLM Instruction Alignment for Improved Prompt Engineering. arXiv preprint arXiv:2411.06099.}\steven{added}
Other studies also leverage the stability~\cite{chiang2023can} %\kenneth{TODO: Add ref}\steven{added}
%chiang2023can found LLM evaluation are stable over different formatting
or confidence~\cite{gligoric2024can} %\kenneth{TODO: Add ref}\steven{added}
%gligoric2024can introduce CONFIDENCEDRIVEN INFERENCE: a method that combines LLM annotations and LLM confidence indicators to strategically select which human annotations should be collected
of LLM outputs to infer quality signals.
%Our research investigates how effectively humans can iteratively refine prompts to guide LLMs in labeling data when gold-standard labels are unavailable, providing alternative quality signals.
Our research examines how effectively humans can refine prompts to guide LLMs in labeling data without gold-standard labels, providing insights into human prompting capabilities in the absence of reliable guidance signals.










%------------- dead kitten -------------
\begin{comment}




\kenneth{------------------------KENNETH IS WORKING HERE----------------------}



Gold-standard labels are widely used for quality control and crowd worker training~\cite{doroudi2016toward, gadiraju2015training,le2010ensuring,hettiachchi2021challenge}. For example, \citet{hettiachchi2021challenge} demonstrated that incorporating visible gold questions -- where annotators receive periodic feedback based on pre-labeled gold-standard examples -- could improve their work quality. 
Similarly, \citet{doroudi2016toward} found that providing expert examples was the most effective method of training for crowd workers and can help workers avoid specific types of incorrect responses. 
Additionally, \citet{le2010ensuring} employed dynamic learning systems that leveraged gold-standard labels to deliver real-time feedback and improve worker outcomes.
These studies, however, predominantly address the annotators' perspective -- workers who adhere to predefined guidelines and follow established standards.
While annotators are crucial components of the task pipeline, our study shifts focus to the requesters' perspective, those responsible for task design and pipeline management.
For requesters, gold-standard labels serve as signals to assess worker performance and refine training processes, thereby improving the overall quality of the entire pipeline.
Critically, the aforementioned studies assume the availability of gold-standard labels, typically under controlled experimental settings. 
In real-world scenarios, this assumption often does not hold due to constraints such as data privacy, security concerns, or the absence of labeled data~\cite{liu2019deep, yang2019evaluating, oikarinen2021detecting, slote2024unlocking}. 
To address this gap, our research explores settings where predefined gold-standard labels are unavailable. 
We designed a novel framework for requesters to iteratively develop and evolve their labeling standards through interactions with LLMs. 
By bridging the divide between controlled experiments and real-world challenges, our work highlights the potential of adaptive, LLM-driven approaches for dynamic task management without reliance on predefined gold-standard labels.

\steven{\citet{hettiachchi2021challenge} demonstrated that incorporating visible gold questions -- where annotators receive periodic feedback based on pre-labeled gold-standard examples -- could improve their work quality. 
Their study leveraged gold-standard labels to train crowd workers to align with pre-defined standards, effectively guiding annotators thorugh examples and feedback. 
While this approach focues on improving labeling quality at the annotator level, our work shifts the focus to requester and researcher perspective. Instead solely training labelers to meet pre-existing standards, we emphasize the broader implications of designing system in the entire labeling process, particularly in context involving dynamic or subjective tasks. \citet{gadiraju2015training} showed that training workers with gold labels can enhance accuracy and decrease response times. \citet{han2020crowd} used gold standard labels to guide crowd workers in revising incorrect judgments to align with predefined standards. 
}

\steven{
\citet{doroudi2016toward} found that providing expert examples was the most effective method of training for crowd workers. In our study, however, each participant was treated as an individual researcher rather than a crowd worker. While this finding underscores the value of providing gold labels to improve language model performance, it does not directly highlight their significance for researchers. Furthermore, \citet{doroudi2016toward} observed that gold standard labels help workers avoid specific types of incorrect responses. 
In contrast, our task is subjective, with participants’ standards potentially shifting across iterations. Introducing pre-set gold standard labels could enforce a uniform standard across each participant, which might not align with the iterative and subjective nature of our study
}

\steven{\citet{gadiraju2015training} showed that training workers with gold labels can enhance accuracy and decrease response times. [They were still focusing on crowd worker level.] }

\steven{\citet{han2020crowd} used gold standard labels for quality control and to guide crowd workers in revising incorrect judgments to align with predefined standards.}

\steven{\citet{le2010ensuring} employed gold standard labels within a dynamic learning environment that provided real-time feedback to train workers. However, the selection of specific examples for training could influence worker responses, potentially introducing bias in their judgments. [This is why we implemented a random sample in our system]}


\steven{\citet{liu2019deep} developed a HITL system that kept model upgrading with progressively collected data without having a pre-labeled data. [\textbf{they used 30 samples per iteration.} -add to justification for 10 and 50 instances.]}

\steven{\citet{wall2019using} found that end-users could build models without using expert patterns that have comparable performance to those who built by expert. This approach was required more effort and more mental demand than those who received guidance.}

\kenneth{TODO: Add references to every part of this paragraph.}
Decades of research have established that gold-standard labels are highly effective for quality control in data annotation~\cite{han2020crowd,gadiraju2015training,le2010ensuring,doroudi2016toward,hettiachchi2021challenge}. 
Embedding items with known labels into the annotation process enables requesters to monitor annotator or data quality and take actions such as retraining annotators, removing them from the pipeline, or reducing their weight in label aggregation. 
Beyond requesters, gold labels also allow for real-time feedback to workers, helping them recalibrate their understanding of the task or focus more carefully.
While gold labels are widely recognized as useful for quality control, most research assumes their availability.
However, as discussed in our Introduction (Section X), this assumption does not necessarily hold in real-world scenarios due to constraints such as data privacy or the cost of gathering gold labels~\cite{liu2019deep, yang2019evaluating, oikarinen2021detecting, slote2024unlocking}. 
To address these challenges, researchers have developed systems to provide proxy quality signals without gold labels. 
For instance, CoPrompter evaluates how well an LLM's output aligns with user-specified requirements as a feedback mechanism. 
Other studies leverage the stability or confidence of LLM outputs to infer quality signals.
Our research investigates how effectively humans can refine prompts to guide LLMs when gold-standard labels are unavailable.
    
\end{comment}

%\subsection{Explanations in AI-Assisted Tools}


%\subsection{Variables in System}
%There are lots of variables in a system could impact user's performance. 
\citet{kulesza2012tell} suggested that the more users understand the underlying system, the more effectively they can control it. 

\steven{\citet{lee2024clarify} introduces a system that allows non-expert users to train and correct models by directly interact with model using natural languages. In each iteration, the system will use similarity score between user description and image and display images above a threshold. The system will also provide 0-1 score indicating how well description separates the error cases from the correct prediction. Basically using metrics to guide user.
It does not mentioned about the sample size selection.}

\steven{[Data Instance:] In active learning, the goal is to minimize the amount of interaction required by users by querying the most important information~\cite{bernard2018vial}. [This can be used to justify why we increase to 50, to ensure the diversity. We cannot deploy algorithms to find most representative data sample because of the technical limitation of Google App Script]}

\steven{[Data Instance:] \citet{vermetten2022analyzing} investigated how the number of sample size affects the reliability of algorithm comparisons in iterative optimization. The study found that small sample sizes lead to high variability in performance estimates and larger sample sizes could decrease the impact of outliers. The performance could loss due to small samples and increasing sample size consistently improves reliability. }

\steven{\citet{purohit2018ranking} suggested capping the maximum number of annotation tasks assigned per unit of time to manage workload effectively to mitigate annotator burnout.}

\steven{\citet{pandey2022modeling} mentioned annotator can develop a mental representation of a concept by seeing a sufficient number of examples.}

\steven{\citet{wang2016human} limited users to verify the top-50 in each round, where users did binary classification on whether image was match or not.}

\steven{[explanation]\citet{kulesza2015principles} presents a system that explains the reason behind each prediction for users to better understand the system's logic to tailor the system toward their needs. In the system, users will modify feature weights within the model. n our LLM-powered system, users need to use natural language to guide the system. However, this can be more challenging because large models are less responsive to prompt variations compared to smaller models~\cite{zhuo2024prosa}.}

\steven{The more users understand the underlying system, the more effectively they can control it~\cite{kulesza2012tell}.}

\steven{\citet{teso2023leveraging} discusses a general framework for incorporating explanations into interactive machine learning. Users can get a better understanding of the machine's logic by observing the machine's explanations. [In LLM system, the explanation is the supporting argument for selecting a label.] Once understanding the bugs and limitation, users could modify the algorithm to correct flaws~\cite{kulesza2015principles}. [In our case, user cannot directly modify LLMs but only provide natural language to guide them. Also, subjective tasks does not have universal correct answers, where users need to provide their own standards to steer LLMs. ] }

\steven{[Task Difficulty:] 
A task being too difficult can frustrate users~\cite{zheng2022virtual}, particularly when exceeding their skill level, and a task being to easy can lead to boredom~\cite{zhang2021personalized}.
  These study focused on the impacts of difficulty on users' performance on a pre-defined task. However, in our study, our work prioritizes the dynamics of human-LLM interaction, emphasizing how effective humans could guide LLMs to align with their standard. In this context, the difficulty level of the task itself is less critical, as our primary objective is to assess the effectiveness of human guidance, regardless of the inherent complexity of the task.}


\steven{[task type:]\citet{cayir2016study} found the complexity and definition of a task significantly influence user performance. }

\steven{[task type:] \citet{hettiachchi2022survey} discusses different task assignment methods, including the modeling of worker performance and the impact of task heterogeneity on assignment strategies.
\citet{zhen2021crowdsourcing} provides a detailed exploration of task assignment challenges, task types, and their effects on worker performance and task outcomes. 
}
% \steven{ending of related word}We wanted to design a system to bridge the gap of xxxx: a graphical interface implemented on Google Sheet add-on, generalizing to single-class data annotation tasks, without requiring extensive knowledge of programming and system configuration. By combining the widespread familiarity and advanced features of Google Sheets with large-scale data annotation and iteration tracking, we aimed to make it easier for people to experiment with and benefit from LLMs.


% Please add the following required packages to your document preamble:
% \usepackage{booktabs}
% \usepackage{multirow}
\begin{table*}[t]
\centering
\footnotesize
\begin{tabular}{@{}ccccr@{\kern-4mm}lccr@{\kern-4mm}lccr@{\kern-4mm}l@{}}
\toprule
 &  & \multicolumn{12}{c}{\textbf{Accuracy (Acc)$\uparrow$}} \\ \cmidrule{3-14}
 &  & \multicolumn{4}{c}{\textbf{structured}} & \multicolumn{4}{c}{\textbf{plain}} & \multicolumn{4}{c}{\textbf{shuffled}} \\ \cmidrule(lr){3-6} \cmidrule(lr){7-10} \cmidrule(lr){11-14}
\multirow{-4}{*}{\textbf{\begin{tabular}[c]{@{}c@{}}Text\\ Length\\ (\#Character)\end{tabular}}} & \multirow{-4}{*}{\textbf{Model}} & \textbf{tw} & \textbf{cn} & \multicolumn{2}{c}{\textbf{\begin{tabular}[c]{@{}c@{}}$\Delta$Acc\\ (cn-tw)\end{tabular}}} & \textbf{tw} & \textbf{cn} & \multicolumn{2}{c}{\textbf{\begin{tabular}[c]{@{}c@{}}$\Delta$Acc\\ (cn-tw)\end{tabular}}} & \textbf{tw} & \textbf{cn} & \multicolumn{2}{c}{\textbf{\begin{tabular}[c]{@{}c@{}}$\Delta$Acc\\ (cn-tw)\end{tabular}}} \\ \midrule
 & \textbf{GPT-4o} & 26.52 & 27.43 & \cellcolor[HTML]{FBE8E6}0.91\hspace{4mm} & \multicolumn{1}{l}{} & 19.16 & 20.78 & \cellcolor[HTML]{F7D5D2}1.62\hspace{4mm} & *** & 18.57 & 20.16 & \cellcolor[HTML]{F8D6D3}1.60\hspace{4mm} & *** \\
 & \textbf{Llama3 8b} & 27.40 & 26.39 & \cellcolor[HTML]{D9F1F3}-1.01\hspace{4mm} & \multicolumn{1}{l}{} & 19.21 & 19.08 & \cellcolor[HTML]{FAFDFD}-0.13\hspace{4mm} & \multicolumn{1}{l}{} & 17.43 & 17.71 & \cellcolor[HTML]{FEF8F8}0.28\hspace{4mm} & \multicolumn{1}{l}{} \\
 & \textbf{Llama3 70b} & 35.43 & 35.00 & \cellcolor[HTML]{EFF9FA}-0.43\hspace{4mm} & \multicolumn{1}{l}{} & 28.21 & 29.60 & \cellcolor[HTML]{F9DBD9}1.39\hspace{4mm} & ** & 27.54 & 29.51 & \cellcolor[HTML]{F6CCC8}1.97\hspace{4mm} & *** \\
 & \textbf{Llama3 405b} & 37.96 & 40.51 & \cellcolor[HTML]{F3BDB8}2.55\hspace{4mm} & *** & 27.42 & 30.12 & \cellcolor[HTML]{F2B9B4}2.70\hspace{4mm} & *** & 27.59 & 30.17 & \cellcolor[HTML]{F3BCB7}2.58\hspace{4mm} & *** \\
 & \textbf{Gemma2 9b} & 15.69 & 14.45 & \cellcolor[HTML]{D1EEF0}-1.24\hspace{4mm} & ** & 17.01 & 17.26 & \cellcolor[HTML]{FEF9F9}0.25\hspace{4mm} & \multicolumn{1}{l}{} & 15.81 & 16.35 & \cellcolor[HTML]{FDF1F0}0.54\hspace{4mm} & \multicolumn{1}{l}{} \\
\multirow{-6}{*}{\textbf{\begin{tabular}[c]{@{}c@{}}Short\\ (1-49)\end{tabular}}} & \textbf{Gemma2 27b} & 15.34 & 14.27 & \cellcolor[HTML]{D7F0F2}-1.07\hspace{4mm} & ** & 13.94 & 14.03 & \cellcolor[HTML]{FFFDFD}0.09\hspace{4mm} & \multicolumn{1}{l}{} & 13.91 & 14.29 & \cellcolor[HTML]{FEF6F5}0.37\hspace{4mm} & \multicolumn{1}{l}{} \\ \midrule
 & \textbf{GPT-4o} & 35.59 & 38.39 & \cellcolor[HTML]{F2B7B2}2.79\hspace{4mm} & *** & 28.15 & 33.16 & \cellcolor[HTML]{E67C73}5.01\hspace{4mm} & *** & 26.73 & 31.36 & \cellcolor[HTML]{E8867E}4.64\hspace{4mm} & *** \\
 & \textbf{Llama3 8b} & 25.31 & 27.01 & \cellcolor[HTML]{F7D3D0}1.70\hspace{4mm} & * & 19.53 & 21.24 & \cellcolor[HTML]{F7D3D0}1.71\hspace{4mm} & ** & 18.92 & 21.11 & \cellcolor[HTML]{F5C6C2}2.19\hspace{4mm} & *** \\
 & \textbf{Llama3 70b} & 34.66 & 38.24 & \cellcolor[HTML]{EEA29B}3.59\hspace{4mm} & *** & 35.02 & 37.45 & \cellcolor[HTML]{F3C0BC}2.43\hspace{4mm} & ** & 33.66 & 36.43 & \cellcolor[HTML]{F2B7B2}2.77\hspace{4mm} & *** \\
 & \textbf{Llama3 405b} & 37.20 & 40.52 & \cellcolor[HTML]{EFA9A3}3.31\hspace{4mm} & *** & 36.09 & 38.00 & \cellcolor[HTML]{F6CECA}1.91\hspace{4mm} & * & 34.38 & 36.60 & \cellcolor[HTML]{F4C5C1}2.22\hspace{4mm} & ** \\
 & \textbf{Gemma2 9b} & 14.84 & 15.66 & \cellcolor[HTML]{FBEAE9}0.82\hspace{4mm} & \multicolumn{1}{l}{} & 18.22 & 20.00 & \cellcolor[HTML]{F7D1CE}1.78\hspace{4mm} & ** & 16.59 & 17.98 & \cellcolor[HTML]{F9DBD9}1.38\hspace{4mm} & * \\
\multirow{-6}{*}{\textbf{\begin{tabular}[c]{@{}c@{}}Long\\ (50+)\end{tabular}}} & \textbf{Gemma2 27b} & 13.44 & 14.52 & \cellcolor[HTML]{FAE3E1}1.08\hspace{4mm} & \multicolumn{1}{l}{} & 15.48 & 16.99 & \cellcolor[HTML]{F8D8D5}1.51\hspace{4mm} & * & 15.16 & 17.16 & \cellcolor[HTML]{F6CBC8}2.00\hspace{4mm} & *** \\ \midrule
 & \textbf{GPT-4o} & 29.61 & 31.16 & \cellcolor[HTML]{F8D7D4}1.55\hspace{4mm} & *** & 22.22 & 24.99 & \cellcolor[HTML]{F2B7B2}2.78\hspace{4mm} & *** & 21.35 & 23.98 & \cellcolor[HTML]{F2BBB6}2.63\hspace{4mm} & *** \\
 & \textbf{Llama3 8b} & 26.69 & 26.61 & \cellcolor[HTML]{FCFDFE}-0.08\hspace{4mm} & \multicolumn{1}{l}{} & 19.32 & 19.82 & \cellcolor[HTML]{FDF2F2}0.50\hspace{4mm} & \multicolumn{1}{l}{} & 17.94 & 18.88 & \cellcolor[HTML]{FBE7E5}0.94\hspace{4mm} & * \\
 & \textbf{Llama3 70b} & 35.16 & 36.10 & \cellcolor[HTML]{FBE7E5}0.94\hspace{4mm} & * & 30.53 & 32.27 & \cellcolor[HTML]{F7D2CF}1.75\hspace{4mm} & *** & 29.62 & 31.87 & \cellcolor[HTML]{F4C5C1}2.24\hspace{4mm} & *** \\
 & \textbf{Llama3 405b} & 37.70 & 40.51 & \cellcolor[HTML]{F1B6B1}2.81\hspace{4mm} & *** & 30.39 & 32.82 & \cellcolor[HTML]{F3C0BC}2.43\hspace{4mm} & *** & 29.92 & 32.38 & \cellcolor[HTML]{F3BFBB}2.46\hspace{4mm} & *** \\
 & \textbf{Gemma2 9b} & 15.40 & 14.86 & \cellcolor[HTML]{EBF7F8}-0.54\hspace{4mm} & \multicolumn{1}{l}{} & 17.42 & 18.19 & \cellcolor[HTML]{FCEBEA}0.77\hspace{4mm} & * & 16.07 & 16.90 & \cellcolor[HTML]{FBEAE8}0.83\hspace{4mm} & * \\
\multirow{-6}{*}{\textbf{Overall}} & \textbf{Gemma2 27b} & 14.69 & 14.35 & \cellcolor[HTML]{F2FAFB}-0.34\hspace{4mm} & \multicolumn{1}{l}{} & 14.47 & 15.04 & \cellcolor[HTML]{FDF1F0}0.57\hspace{4mm} & \multicolumn{1}{l}{} & 14.34 & 15.27 & \cellcolor[HTML]{FBE7E6}0.93\hspace{4mm} & ** \\ \bottomrule
\end{tabular}
\vspace{-.5pc}
\caption{Accuracy (Acc~$\uparrow$) by length for GPT-4o, Llama3 (8b, 70b, 405b), and Gemma2 (9b, 27b) models. Red (green) indicates better (worse) performance in CN, with darker shades representing larger gaps. 
(Statistical group differences are indicated as ${^{*}}$  (p<.05), ${^{**}}$ (p<.01), and ${^{***}}$ (p<.001) regarding the model performance.)}
\vspace{-1pc}
\label{tab:new-acc-results}
\end{table*}

%\kenneth{TODO CY: Figure out the color code}

% \begin{tabular}{@{}llccccccccc@{}}
% \toprule
% \multirow{3}{*}{\textbf{\begin{tabular}[c]{@{}l@{}}Text\\ Length\\ (\#word)\end{tabular}}} & \multirow{3}{*}{\textbf{Model}} & \multicolumn{9}{c}{\textbf{Accuracy (Acc)~$\uparrow$}} \\ \cmidrule(l){3-11} 
%  &  & \multicolumn{3}{c}{\textbf{structured}} & \multicolumn{3}{c}{\textbf{plain}} & \multicolumn{3}{c}{\textbf{shuffled}} \\ \cmidrule(l){3-11} 
%  &  & \textbf{tw} & \textbf{cn} & \textbf{\begin{tabular}[c]{@{}c@{}}$\Delta$Acc\\ (cn-tw)\end{tabular}} & \textbf{tw} & \textbf{cn} & \textbf{\begin{tabular}[c]{@{}c@{}}$\Delta$Acc\\ (cn-tw)\end{tabular}} & \textbf{tw} & \textbf{cn} & \textbf{\begin{tabular}[c]{@{}c@{}}$\Delta$Acc\\ (cn-tw)\end{tabular}} \\ \midrule
% \multirow{6}{*}{\textbf{\begin{tabular}[c]{@{}l@{}}Short\\ (1-49)\end{tabular}}} & \textbf{GPT-4o} & 26.52 & 27.43 & 0.91 & 19.16 & 20.78 & 1.62*** & 18.57 & 20.16 & 1.60*** \\
%  & \textbf{Llama3 8b} & 27.40 & 26.39 & -1.01 & 19.21 & 19.08 & -0.13 & 17.43 & 17.71 & 0.28 \\
%  & \textbf{Llama3 70b} & 35.43 & 35.00 & -0.43 & 28.21 & 29.60 & 1.39** & 27.54 & 29.51 & 1.97*** \\
%  & \textbf{Llama3 405b} & 37.96 & 40.51 & 2.55*** & 27.42 & 30.12 & 2.70*** & 27.59 & 30.17 & 2.58*** \\
%  & \textbf{Gemma2 9b} & 15.69 & 14.45 & -1.24** & 17.01 & 17.26 & 0.25 & 15.81 & 16.35 & 0.54 \\
%  & \textbf{Gemma2 27b} & 15.34 & 14.27 & -1.07** & 13.94 & 14.03 & 0.09 & 13.91 & 14.29 & 0.37 \\ \midrule
% \multirow{6}{*}{\textbf{\begin{tabular}[c]{@{}l@{}}Long\\ (50+)\end{tabular}}} & \textbf{GPT-4o} & 35.59 & 38.39 & 2.79*** & 28.15 & 33.16 & 5.01*** & 26.73 & 31.36 & 4.64*** \\
%  & \textbf{Llama3 8b} & 25.31 & 27.01 & 1.70* & 19.53 & 21.24 & 1.71** & 18.92 & 21.11 & 2.19*** \\
%  & \textbf{Llama3 70b} & 34.66 & 38.24 & 3.59*** & 35.02 & 37.45 & 2.43** & 33.66 & 36.43 & 2.77*** \\
%  & \textbf{Llama3 405b} & 37.20 & 40.52 & 3.31*** & 36.09 & 38.00 & 1.91* & 34.38 & 36.60 & 2.22** \\
%  & \textbf{Gemma2 9b} & 14.84 & 15.66 & 0.82 & 18.22 & 20.00 & 1.78** & 16.59 & 17.98 & 1.38* \\
%  & \textbf{Gemma2 27b} & 13.44 & 14.52 & 1.08 & 15.48 & 16.99 & 1.51* & 15.16 & 17.16 & 2.00*** \\ \midrule
% \multirow{6}{*}{\textbf{Overall}} & \textbf{GPT-4o} & 29.61 & 31.16 & 1.55*** & 22.22 & 24.99 & 2.78*** & 21.35 & 23.98 & 2.63*** \\
%  & \textbf{Llama3 8b} & 26.69 & 26.61 & -0.08 & 19.32 & 19.82 & 0.50 & 17.94 & 18.88 & 0.94* \\
%  & \textbf{Llama3 70b} & 35.16 & 36.10 & 0.94* & 30.53 & 32.27 & 1.75*** & 29.62 & 31.87 & 2.24*** \\
%  & \textbf{Llama3 405b} & 37.70 & 40.51 & 2.81*** & 30.39 & 32.82 & 2.43*** & 29.92 & 32.38 & 2.46*** \\
%  & \textbf{Gemma2 9b} & 15.40 & 14.86 & -0.54 & 17.42 & 18.19 & 0.77* & 16.07 & 16.90 & 0.83* \\
%  & \textbf{Gemma2 27b} & 14.69 & 14.35 & -0.34 & 14.47 & 15.04 & 0.57 & 14.34 & 15.27 & 0.93** \\ \bottomrule
% \end{tabular}
% Please add the following required packages to your document preamble:
% \usepackage{booktabs}
% \usepackage{multirow}
\begin{table*}[t]
\centering
\footnotesize
\begin{tabular}{c@{\kern2mm}c@{\kern2mm}ccr@{\kern-4mm}lccr@{\kern-4mm}lccr@{\kern-4mm}l}
\toprule
 &  & \multicolumn{12}{c}{\textbf{Mean Squared Error (MSE)~$\downarrow$}} \\ \cmidrule(l){3-14} 
 &  & \multicolumn{4}{c}{\textbf{structured}} & \multicolumn{4}{c}{\textbf{plain}} & \multicolumn{4}{c}{\textbf{shuffled}} \\ \cmidrule(lr){3-6} \cmidrule(lr){7-10} \cmidrule(lr){11-14}
\multirow{-4}{*}{\textbf{\begin{tabular}[c]{@{}c@{}}Text\\ Length\\ (\#Character)\end{tabular}}} & \multirow{-4}{*}{\textbf{Model}} & \textbf{tw} & \textbf{cn} & \textbf{\begin{tabular}[c]{@{}c@{}}$\Delta$MSE\\ (cn-tw)\end{tabular}} & \multicolumn{1}{l}{} & \textbf{tw} & \textbf{cn} & \textbf{\begin{tabular}[c]{@{}c@{}}$\Delta$MSE\\ (cn-tw)\end{tabular}} & \multicolumn{1}{l}{} & \textbf{tw} & \textbf{cn} & \textbf{\begin{tabular}[c]{@{}c@{}}$\Delta$MSE\\ (cn-tw)\end{tabular}} & \multicolumn{1}{l}{} \\ \midrule
 & \textbf{GPT-4o} & 3.563 & 3.769 & \cellcolor[HTML]{DBF3F4}0.206\hspace{4mm} & *** & 4.091 & 3.385 & \cellcolor[HTML]{EEA7A1}-0.706\hspace{4mm} & *** & 4.347 & 3.561 & \cellcolor[HTML]{EC9D96}-0.786\hspace{4mm} & *** \\
 & \textbf{Llama3 8b} & 2.187 & 2.268 & \cellcolor[HTML]{F1FAFB}0.082\hspace{4mm} & \multicolumn{1}{l}{} & 2.999 & 2.801 & \cellcolor[HTML]{FAE6E4}-0.199\hspace{4mm} & *** & 3.377 & 3.016 & \cellcolor[HTML]{F6D2CF}-0.361\hspace{4mm} & *** \\
 & \textbf{Llama3 70b} & 1.732 & 1.626 & \cellcolor[HTML]{FCF1F0}-0.107\hspace{4mm} & ** & 2.977 & 2.534 & \cellcolor[HTML]{F4C8C4}-0.443\hspace{4mm} & *** & 3.006 & 2.605 & \cellcolor[HTML]{F5CDC9}-0.401\hspace{4mm} & *** \\
 & \textbf{Llama3 405b} & 2.782 & 2.635 & \cellcolor[HTML]{FBECEB}-0.147\hspace{4mm} & \multicolumn{1}{l}{} & 4.624 & 3.685 & \cellcolor[HTML]{E88A82}-0.939\hspace{4mm} & *** & 4.620 & 3.740 & \cellcolor[HTML]{EA918A}-0.880\hspace{4mm} & *** \\
 & \textbf{Gemma2 9b} & 3.026 & 3.164 & \cellcolor[HTML]{E7F7F8}0.138\hspace{4mm} & * & 4.483 & 3.828 & \cellcolor[HTML]{EFADA8}-0.655\hspace{4mm} & *** & 4.928 & 4.131 & \cellcolor[HTML]{EC9C95}-0.797\hspace{4mm} & *** \\
\multirow{-6}{*}{\textbf{\begin{tabular}[c]{@{}c@{}}Short\\ (1-49)\end{tabular}}} & \textbf{Gemma2 27b} & 2.945 & 3.028 & \cellcolor[HTML]{F1FAFB}0.083\hspace{4mm} & \multicolumn{1}{l}{} & 4.888 & 4.191 & \cellcolor[HTML]{EEA8A2}-0.697\hspace{4mm} & *** & 4.944 & 4.250 & \cellcolor[HTML]{EEA9A3}-0.693\hspace{4mm} & *** \\ \midrule
 & \textbf{GPT-4o} & 1.846 & 1.577 & \cellcolor[HTML]{F8DDDB}-0.269\hspace{4mm} & *** & 1.834 & 1.57 & \cellcolor[HTML]{F8DEDC}-0.264\hspace{4mm} & *** & 2.070 & 1.743 & \cellcolor[HTML]{F7D6D3}-0.327\hspace{4mm} & *** \\
 & \textbf{Llama3 8b} & 1.674 & 1.548 & \cellcolor[HTML]{FBEFEE}-0.127\hspace{4mm} & *** & 2.046 & 1.895 & \cellcolor[HTML]{FBECEA}-0.152\hspace{4mm} & *** & 2.127 & 1.906 & \cellcolor[HTML]{F9E3E1}-0.220\hspace{4mm} & *** \\
 & \textbf{Llama3 70b} & 1.473 & 1.302 & \cellcolor[HTML]{FAE9E8}-0.171\hspace{4mm} & *** & 1.534 & 1.406 & \cellcolor[HTML]{FBEFEE}-0.128\hspace{4mm} & ** & 1.671 & 1.495 & \cellcolor[HTML]{FAE9E7}-0.176\hspace{4mm} & *** \\
 & \textbf{Llama3 405b} & 1.910 & 1.674 & \cellcolor[HTML]{F9E1DF}-0.236\hspace{4mm} & *** & 1.909 & 1.766 & \cellcolor[HTML]{FBEDEC}-0.143\hspace{4mm} & * & 2.085 & 1.892 & \cellcolor[HTML]{FAE6E5}-0.194\hspace{4mm} & ** \\
 & \textbf{Gemma2 9b} & 2.479 & 2.337 & \cellcolor[HTML]{FBEDEC}-0.142\hspace{4mm} & ** & 2.199 & 2.024 & \cellcolor[HTML]{FAE9E7}-0.175\hspace{4mm} & *** & 2.511 & 2.294 & \cellcolor[HTML]{F9E4E2}-0.217\hspace{4mm} & *** \\
\multirow{-6}{*}{\textbf{\begin{tabular}[c]{@{}c@{}}Long\\ (50+)\end{tabular}}} & \textbf{Gemma2 27b} & 2.703 & 2.519 & \cellcolor[HTML]{FAE8E6}-0.184\hspace{4mm} & *** & 2.680 & 2.500 & \cellcolor[HTML]{FAE8E7}-0.180\hspace{4mm} & *** & 2.649 & 2.496 & \cellcolor[HTML]{FBECEA}-0.153\hspace{4mm} & ** \\ \midrule
 & \textbf{GPT-4o} & 2.978 & 3.022 & \cellcolor[HTML]{F8FDFD}0.044\hspace{4mm} & \multicolumn{1}{l}{} & 3.323 & 2.767 & \cellcolor[HTML]{F1BAB5}-0.555\hspace{4mm} & *** & 3.571 & 2.942 & \cellcolor[HTML]{F0B0AB}-0.630\hspace{4mm} & *** \\
 & \textbf{Llama3 8b} & 2.011 & 2.021 & \cellcolor[HTML]{FEFFFF}0.010\hspace{4mm} & \multicolumn{1}{l}{} & 2.672 & 2.490 & \cellcolor[HTML]{FAE8E6}-0.182\hspace{4mm} & *** & 2.948 & 2.635 & \cellcolor[HTML]{F7D8D5}-0.313\hspace{4mm} & *** \\
 & \textbf{Llama3 70b} & 1.644 & 1.515 & \cellcolor[HTML]{FBEEED}-0.129\hspace{4mm} & *** & 2.486 & 2.150 & \cellcolor[HTML]{F7D5D2}-0.335\hspace{4mm} & *** & 2.551 & 2.227 & \cellcolor[HTML]{F7D6D4}-0.324\hspace{4mm} & *** \\
 & \textbf{Llama3 405b} & 2.483 & 2.306 & \cellcolor[HTML]{FAE9E7}-0.177\hspace{4mm} & *** & 3.695 & 3.028 & \cellcolor[HTML]{EFACA6}-0.667\hspace{4mm} & *** & 3.752 & 3.107 & \cellcolor[HTML]{EFAEA9}-0.645\hspace{4mm} & *** \\
 & \textbf{Gemma2 9b} & 2.840 & 2.882 & \cellcolor[HTML]{F8FDFD}0.043\hspace{4mm} & \multicolumn{1}{l}{} & 3.705 & 3.213 & \cellcolor[HTML]{F3C2BD}-0.491\hspace{4mm} & *** & 4.105 & 3.505 & \cellcolor[HTML]{F0B4AF}-0.600\hspace{4mm} & *** \\
\multirow{-6}{*}{\textbf{Overall}} & \textbf{Gemma2 27b} & 2.863 & 2.855 & \cellcolor[HTML]{FEFEFD}-0.008\hspace{4mm} & \multicolumn{1}{l}{} & 4.136 & 3.615 & \cellcolor[HTML]{F2BEB9}-0.521\hspace{4mm} & *** & 4.162 & 3.653 & \cellcolor[HTML]{F2BFBB}-0.509\hspace{4mm} & *** \\ \bottomrule
\end{tabular}
\vspace{-.5pc}
\caption{Mean squared error (MSE~$\downarrow$) by length for GPT-4o, Llama3 (8b, 70b, 405b), and Gemma2 (9b, 27b) models. Statistical significance notations and color coding follow the same conventions as in Table~\ref{tab:new-mse-results}.}
\vspace{-1pc}
\label{tab:new-mse-results}
\end{table*}
%(Statistical group differences are indicated as ${^{*}}$  (p<.05), ${^{**}}$ (p<.01), and ${^{***}}$ (p<.001) regarding the model performance.)
%\kenneth{TODO CY: Figure out the color code}

\begin{comment}
\begin{tabular}{@{}llccccccccc@{}}
\toprule
\multirow{3}{*}{\textbf{\begin{tabular}[c]{@{}l@{}}Text\\ Length\\ (\#word)\end{tabular}}} & \multirow{3}{*}{\textbf{Model}} & \multicolumn{9}{c}{\textbf{Mean Squared Error (MSE)~$\downarrow$}} \\ \cmidrule(l){3-11} 
 &  & \multicolumn{3}{c}{\textbf{structured}} & \multicolumn{3}{c}{\textbf{plain}} & \multicolumn{3}{c}{\textbf{shuffled}} \\ \cmidrule(l){3-11} 
 &  & \textbf{tw} & \textbf{cn} & \textbf{\begin{tabular}[c]{@{}c@{}}$\Delta$MSE\\ (cn-tw)\end{tabular}} & \textbf{tw} & \textbf{cn} & \textbf{\begin{tabular}[c]{@{}c@{}}$\Delta$MSE\\ (cn-tw)\end{tabular}} & \textbf{tw} & \textbf{cn} & \textbf{\begin{tabular}[c]{@{}c@{}}$\Delta$MSE\\ (cn-tw)\end{tabular}} \\ \midrule
\multirow{6}{*}{\textbf{\begin{tabular}[c]{@{}l@{}}Short\\ (1-49)\end{tabular}}} & \textbf{GPT-4o} & 3.563 & 3.769 & 0.206*** & 4.091 & 3.385 & -0.706*** & 4.347 & 3.561 & -0.786*** \\
 & \textbf{Llama3 8b} & 2.187 & 2.268 & 0.082 & 2.999 & 2.801 & -0.199*** & 3.377 & 3.016 & -0.361*** \\
 & \textbf{Llama3 70b} & 1.732 & 1.626 & -0.107** & 2.977 & 2.534 & -0.443*** & 3.006 & 2.605 & -0.401*** \\
 & \textbf{Llama3 405b} & 2.782 & 2.635 & -0.147 & 4.624 & 3.685 & -0.939*** & 4.620 & 3.740 & -0.880*** \\
 & \textbf{Gemma2 9b} & 3.026 & 3.164 & 0.138* & 4.483 & 3.828 & -0.655*** & 4.928 & 4.131 & -0.797*** \\
 & \textbf{Gemma2 27b} & 2.945 & 3.028 & 0.083 & 4.888 & 4.191 & -0.697*** & 4.944 & 4.250 & -0.693*** \\ \midrule
\multirow{6}{*}{\textbf{\begin{tabular}[c]{@{}l@{}}Long\\ (50+)\end{tabular}}} & \textbf{GPT-4o} & 1.846 & 1.577 & -0.269*** & 1.834 & 1.570 & -0.264*** & 2.070 & 1.743 & -0.327*** \\
 & \textbf{Llama3 8b} & 1.674 & 1.548 & -0.127*** & 2.046 & 1.895 & -0.152*** & 2.127 & 1.906 & -0.220*** \\
 & \textbf{Llama3 70b} & 1.473 & 1.302 & -0.171*** & 1.534 & 1.406 & -0.128** & 1.671 & 1.495 & -0.176*** \\
 & \textbf{Llama3 405b} & 1.910 & 1.674 & -0.236*** & 1.909 & 1.766 & -0.143* & 2.085 & 1.892 & -0.194** \\
 & \textbf{Gemma2 9b} & 2.479 & 2.337 & -0.142** & 2.199 & 2.024 & -0.175*** & 2.511 & 2.294 & -0.217*** \\
 & \textbf{Gemma2 27b} & 2.703 & 2.519 & -0.184*** & 2.680 & 2.500 & -0.180*** & 2.649 & 2.496 & -0.153** \\ \midrule
\multirow{6}{*}{\textbf{Overall}} & \textbf{GPT-4o} & 2.978 & 3.022 & 0.044 & 3.323 & 2.767 & -0.555*** & 3.571 & 2.942 & -0.630*** \\
 & \textbf{Llama3 8b} & 2.011 & 2.021 & 0.010 & 2.672 & 2.490 & -0.182*** & 2.948 & 2.635 & -0.313*** \\
 & \textbf{Llama3 70b} & 1.644 & 1.515 & -0.129*** & 2.486 & 2.150 & -0.335*** & 2.551 & 2.227 & -0.324*** \\
 & \textbf{Llama3 405b} & 2.483 & 2.306 & -0.177*** & 3.695 & 3.028 & -0.667*** & 3.752 & 3.107 & -0.645*** \\
 & \textbf{Gemma2 9b} & 2.840 & 2.882 & 0.043 & 3.705 & 3.213 & -0.491*** & 4.105 & 3.505 & -0.600*** \\
 & \textbf{Gemma2 27b} & 2.863 & 2.855 & -0.008 & 4.136 & 3.615 & -0.521*** & 4.162 & 3.653 & -0.509*** \\ \bottomrule
\end{tabular}
\end{comment}


%

\begin{table*}
    \centering
    \caption{Evaluations on AlpacaEval 2 and MixEval. LC WR and WR denote length-controlled win rate and win rate respectively. Offline baseline performances on AlpacaEval 2 are from \citep{meng2024simpo}. We use LLM-blender \citep{jiang2023llm} as the reward model for a fair comparison with the baselines and also report the result with a stronger reward model FsfairX \citep{dong2024rlhf}}\label{tab:main-performance}
    \scalebox{0.87}{
    \begin{tabular}{lcccccccccccc}
        \toprule
        Model & \multicolumn{4}{c}{Mistral-Base (7B)} & \multicolumn{4}{c}{Mistral-Instruct (7B)} \\
        \cmidrule(lr){2-5} \cmidrule(lr){6-9}
        & \multicolumn{2}{c}{Alpaca Eval 2}  & \multirow{1}{*}{MixEval} & \multirow{1}{*}{MixEval-Hard} & \multicolumn{2}{c}{Alpaca Eval 2}  & \multirow{1}{*}{MixEval} & \multirow{1}{*}{MixEval-Hard} \\
        \cmidrule(lr){2-3} \cmidrule(lr){4-4} \cmidrule(lr){5-5} \cmidrule(lr){6-7} \cmidrule(lr){8-8} \cmidrule(lr){9-9}
        & LC WR & WR & Score & Score & LC WR & WR & Score & Score \\
        \midrule
        SFT    & 8.4  & 6.2    &  0.602  & 0.279  & 17.1 & 14.7  & 0.707 & 0.361 \\
        \midrule
        \multicolumn{9}{c}{Reward model: LLM-Blender \citep{jiang2023llm}}  \\
        \midrule
        RRHF   & 11.6 & 10.2  &  0.600  & 0.312  & 25.3 & 24.8  &   0.700    & 0.380 \\
        SLiC-HF & 10.9 & 8.9    & 0.679  &   0.334 & 24.1 & 24.6  &   0.700    & 0.381 \\
        DPO    & 15.1 & 12.5  &  0.686  &  0.341 & 26.8 & 24.9  & 0.702 & 0.355 \\
        IPO    & 11.8 & 9.4   &  0.673  & 0.326  & 20.3 & 20.3  & 0.695 & 0.376 \\
        CPO    & 9.8  & 8.9    & 0.632   &  0.307 & 23.8 & 28.8  & 0.699 & 0.405 \\
        KTO    & 13.1 & 9.1   & \textbf{0.704}  & 0.351   & 24.5 & 23.6  &   0.692    & 0.358 \\
        % ORPO   & 14.7 & 12.2 & 7.0  &   &    & 24.5 & 24.9 & 20.8 &  0.703     & 0.378 \\
        RDPO   & 17.4 & 12.8   & 0.693  & 0.355   & 27.3 & 24.5  &   0.695    & 0.364 \\
        SimPO  & 21.5 & 20.8 &  0.672  &  0.347 & 32.1 & 34.8  & 0.702  & 0.363 \\
        Iterative DPO  & 18.9  & 16.7  & 0.660   & 0.341  & 20.4 & 24.8  & 0.719  & 0.389 \\
        \midrule
        \Ours (Contrastive) & 31.6 & 30.8  &   0.703 & 0.409  & 32.7 & 38.6  &  0.718 & \textbf{0.418} \\
        \Ours (LambdaRank) &  \textbf{34.9} & \textbf{37.2} & 0.695 &  \textbf{0.452}  & \textbf{32.9} & \textbf{38.9}   & \textbf{0.720} & 0.417  \\
        \Ours (ListMLE) & 31.1  &  32.1   &  0.669  & 0.390  &  29.7 & 36.2    & 0.709  & 0.397 \\
        \midrule
        \multicolumn{9}{c}{Reward model: FsfairX \citep{dong2024rlhf}}  \\
        \midrule
        \Ours (Contrastive) & \textbf{41.5} & \textbf{42.9} & 0.718 & 0.417    & \textbf{43.0}  & \textbf{53.8} & 0.718 & 0.425   \\
        \Ours (LambdaRank) & 35.8 & 34.1 & 0.717 & 0.431   & 41.9  & 48.1 & \textbf{0.740} & \textbf{0.440}  \\
        \Ours (ListMLE) & 36.6 & 37.8 & \textbf{0.730} & \textbf{0.423}   & 39.6  & 48.1 & 0.717 & 0.397   \\
        \bottomrule
    \end{tabular}}
    % \vspace{-0.1in}
\end{table*}


\section{Constructing a Contextually-Aligned Review Dataset for Language Varieties}
%\subsection{Data}
%Some prior work also used Booking.com data~\cite{ALDERIGHI2022769}

\vspace{-.5pc}
\paragraph{Data.}
We constructed a dataset of hotel reviews sourced from \texttt{Booking.com},\footnote{Data processing code: \href{https://github.com/Crowd-AI-Lab/Contextually-Aligned-Online-Reviews}{https://github.com/Crowd-AI-Lab/Contextually-Aligned-Online-Reviews}} which has been used in prior research studies~\cite{ALDERIGHI2022769,barnes-etal-2018-multibooked}.
%This dataset included 4,447,853 reviews in both \twChinese and \cnChinese, written by users from Taiwan and Mainland China. 
%This dataset consists of 4,447,853 reviews labeled as written in Chinese by the platform, written by users who self-identified as being from Taiwan and Mainland China. 
This dataset consists of 4,447,853 reviews labeled by the platform as written in Chinese.
%, authored by users who self-identified as being from Taiwan and Mainland China.
%The reviews encompass 149,879 hotels across Japan, Mainland China, South Korea, Taiwan, Thailand, and Vietnam, and were gathered between August 2021 and August 2024.
The reviews cover 149,879 hotels located in Japan, Mainland China, South Korea, Taiwan, Thailand, and Vietnam, and were collected from August 2021 to August 2024. 
These locations were selected to ensure a substantial volume of data, as they are popular destinations for Mandarin-speaking travelers.
%\kenneth{TODO Zixin: Update the locations and time span} 
%The dataset comprises both positive and negative feedback, ratings, room types, and traveler labels. 
%We constructed a dataset that consists of the hotel reviews from \texttt{booking.com}, which is a data source for a series of prior studies~\cite{ALDERIGHI2022769,barnes-etal-2018-multibooked}. \kenneth{TODO Sam: Find one or two more papers from NLP conference that also used Booking.com data.}
%We deliberately selected locations outside Taiwan and Mainland China to avoid potential biases associated with users posting reviews from a particular language variety's region.
Each review comprises three main components: the review title, positive feedback, and negative feedback.
Additionally, it includes review ratings (ranging from 1 to 10 stars) and metadata such as hotel ID, posting time, and more (see \Cref{app:booking-data-sample} for an actual sample). 
Booking.com claims to invest significant effort in ensuring that reviews are posted by real users and in maintaining review quality. 
%Based on their assertion, we chose not to conduct extensive data cleaning. 
We included only non-empty reviews, meaning reviewers provided input in at least one of the following: 
the review title, positive feedback, or negative feedback. 
In total, we collected 1,513,056 reviews written in Chinese.
%\kenneth{TODO Zixin: Update the numbers}


%\kenneth{TODO GG and HH: Please describe how you scraped Booking.com. What region (and why)? What time span (and why)? How many raw reviews were collected? How did you select hotels (and why?}

%\paragraph{Data Content and Pre-Processing.}

% \kenneth{TODO Zixin and CY: Describe how you cleaned and pre-processed the data. Mostly (1) We believe Booking.com already made sure all the comments are real. (2) Each review has what information: title, pos, neg, and all the meta data. (3) we had three settings (and why): (i) original, (ii) plain, and (iii) shuffled. Say we use original to calculate the length.}

%\begin{itemize}
    %\item \textbf{Language Identification:}\kenneth{TODO: CY will figure out}
    %\item \textbf{Short Comments:}\kenneth{What is the criteria? (Chinese word seg?)---CY will figure out, maybe use LLMs to plot accuracy drop (will it drop?)--- Pay attention to bias}
    %\item \textbf{Bot-like/Dup Comments:}\kenneth{Zixin will look into is to figure out the model}
    %\item \textbf{User Accounts Acting Like Bots:} (1) Always post the same content\kenneth{Zixin--Find papers used Booking.com data and see how they cleaned it}; 
    %(2) Unreasonable stay schedule\kenneth{Can random people leave reviews? Or do you need to stay in the hotel? YES??????}
    %\item \textbf{Lack of user id.....}
%\end{itemize}
%\paragraph{Aligning Reviews Using Contexts.}

%\kenneth{TODO Zixin and CY: Describe how you paired the reviews.}

%+Zixin should just use nationality
%+Chieh-Yang will run lang detection tool

\vspace{-.6pc}
\subsection{Contextually Aligning Reviews}
%We only included reviews with textual input, where reviewers have input in either the review title, the positive feedback, or the negative feedback. 
%The users' self-specified ``nationality/region'' labels, which are required by the Booking.com platform, defined the users' language varieties in this study. 
%We then aligned and formed review pairs following the rules below:
We used users' self-specified ``nationality/region'' labels from Booking.com to determine the reviews' language varieties. %in this study. 
In total, we collected 1,403,669 reviews written in \twChinese and \cnChinese, where 95.591\% of them come from \twChinese users.
To ensure a balanced representation between \textbf{\twChinese (TW)} and \textbf{\cnChinese (CN)} reviews, we paired them based on the following criteria:
%we employed a sampling method that pairs reviews in two varieties using the following criteria:
%We then paired \twChinese reviews with \cnChinese reviews according to the following criteria:


\vspace{-.8pc}

\begin{itemize}[leftmargin=*]
%\item \textbf{Both reviews contribute to the same hotel:} Both reviews in the same pair are from the same hotel. Such a pairing rule ensures that reviewers comment on highly similar scenarios or objects, which is the target hotel in our study.
\item
\textbf{Same hotel for both reviews:} 
Both reviews in each pair are from the same hotel, ensuring that the reviewers are commenting on similar scenarios or objects---the hotel itself.



\vspace{-.5pc}

    
%\item \textbf{Both reviews share similar ratings:} To maximize the size of paired reviews while maintaining similar sentiment components in each pair of reviews, we selected pairing candidates with a 3-class label scheme ( 1-3 as negative reviews, 4-7 as neutral reviews, and 8-10 as positive reviews). This process helps align reviews with similar ratings to form comparable pairs.
\item
\textbf{Similar ratings for both reviews:} 
To form comparable pairs with similar sentiments, we used a 3-class rating scheme (1-3 as negative, 4-7 as neutral, and 8-10 as positive) and paired reviews based on this classification. 
This approach maximizes the number of review pairs while maintaining comparable sentiment.



\vspace{-.5pc}

%\item \textbf{Both reviews have a similar amount of text input:} To minimize the effect of input size, we classified all reviews into 10-token wide bins before the pairing process. This ensures the paired reviews will share a similar size of text input. We remove reviews longer than 500 tokens to enhance the overall quality of the data set (see \cref{app:length-exp} for more information regarding potential impacts of text length).\zixin{we need a further explanation on this.}
\item
\textbf{Similar text length for both reviews:} 
%To minimize the effect of input text length, 
To ensure paired reviews have similar text lengths, we grouped reviews into 10-token bins before pairing and required both reviews in each pair to fall within the same length bin.
%This ensures paired reviews have similar text lengths.
% 22 r
Reviews longer than 500 tokens were excluded (see \Cref{app:length-exp}.)
%\kenneth{TODO Zixin: Update numbers}


\end{itemize}



\vspace{-.5pc}


The final dataset contained 22,918 review pairs, each with one TW and one CN user review.








%Following the criteria, the final paired dataset contained 22,918 pairs of reviews, with each pair containing one review from a \twChinese user and one from a \cnChinese user.
%from \twChinese users and \cnChinese users. 
%Each pair 
%making 45,836 data entries in total.

\subsection{Data Quality Validation\label{sec:data-quality-validation}}
%To validate the data quality, we recruited 10 participants, evenly divided between native speakers of \twChinese (TW) and \cnChinese (CN), to rate 200 randomly selected reviews of each language variety.
%To validate the data quality, 
%We recruited five native speakers of \twChinese (TW) to review 200 randomly selected reviews written in \twChinese. 
%We followed the same process for reviews in \cnChinese (CN).
Five native speakers of \twChinese reviewed 200 random \twChinese reviews; the same process applied to \cnChinese.
The focus was on two key aspects: {\em (i)} \textbf{writing quality} and {\em (ii)} \textbf{content-rating agreement}, evaluated on a 5-point Likert scale (see Appendix~\ref{app:human-validation}.) 
%Evaluations were conducted using a 5-point Likert scale, with 1 indicating strong disagreement and 5 indicating strong agreement.
%Appendix~\ref{app:human-validation} shows the material used. 
Each participant was paid \$10.
%\kenneth{TODO Sam: Describe human validation process and results}
%To validate the data quality,
%To construct a contextually-aligned review dataset for language varieties,
%we recruited 10 participants, evenly divided between native speakers of Taiwan Mandarin (TW) and Mainland Mandarin (CN), to rate 200 hotel reviews. The focus was on two key aspects: writing quality and content-rating agreement. Evaluations were conducted using a 5-point Likert scale, with 1 indicating strong disagreement and 5 indicating strong agreement. 
%The 200 comments from Taiwan and China were paired based on the same hotel, similar length, and category ratings. \sam{seems to be mentioned in the previous paragraph}
%\kenneth{TODO Sam: Report the mean and SD for each group here.}
%\sam{done}
As a result, for the writing quality ratings, the TW group had a mean of 4.18 (SD=0.44), and the CN group had a mean of 3.94 (SD=0.49). 
Regarding the rating-content agreement, the TW group had a mean of 4.00 (SD=0.46), and the CN group had a mean of 3.56 (SD=0.55).
%\kenneth{Displaying two decimal (instead of 3) places is sufficient}
%The average rating correlation between participants for writing quality was 0.22 for the TW group and 0.14 for the CN group.
%For content-rating agreement, the average rating correlation between participants was 0.21 for the TW group and 0.17 for the CN group.
%\kenneth{TODO Sam: (1) Displaying two decimal places is sufficient; our sample size isn't large enough to justify showing four decimal places. (2) Calculate (a) the avg correlation between two TW raters, and (b) the avg correlation between two CN raters. This is to show the inner-annotator agreement (IAA) in each group.}
%To validate the dataset's quality, 
%The correlation between the CN and TW groups was 0.258 for writing quality and 0.194 for content-rating agreement. These low positive correlations suggest limited agreement between the groups, indicating potential differences in evaluation criteria or perspectives.


%We calculated the Mean Squared Error (MSE) between the evaluations of TW and CN groups for both  writing quality and content-rating agreement, resulting in values of 0.3766 and 0.5738, respectively.
%\kenneth{I'd report avg corelation between two participants in each group. MSE is not very straighforward.} \sam{done!}

% These MSE values indicate a moderate level of agreement between the two groups. The relatively low MSE for writing quality suggests consistent evaluations across locales, enhancing the dataset's reliability for further analysis. However, significant differences were found in the Two Paired Samples T-Test between the CN and TW groups for both writing quality and content-rating agreement.\sam{singificant found which means CN vs TW groups are not aligned, probably find some interesting examples to address --> may delete if not enough space}
% Regarding this, we tried to study the example with the highest MSE within top 10\% of the example. In content-rating agreement, it is interesting that CN groups are consistently rating lower compared to TW group. And nearly all the comments are from the "positive" category (only 1 from neutral, and no from negative).
% In writing quality, it is also the similar case that most of all comments deviated are from positive comment (only 2 from neutral, and also no from negative)

%\kenneth{I don't feel we have space for expert vs. non-expert...}
%Considering the evaluation of an expert with a Master's degree in Hospitality Management from the CN group, we conducted a Wilcoxon Signed-Rank Test for paired samples between expert and non-expert evaluations. Significant differences were observed in both writing quality $(p\text{-value} = 7.08 \times 10^{-10})$ and content-rating agreement $(p\text{-value} = 7.76 \times 10^{-21})$. Additionally, the expert's ratings showed higher variability (SD = 0.91, 0.86) compared to non-experts (SD = 0.47, 0.57), suggesting that the expert's background may lead to more nuanced evaluations, reflecting deeper insights or different criteria compared to non-experts. 
%\sam{BUT if I do SD for individual participants one by one for each aspect, P1 (expert) rating is not that greatly deviated compared}
%We acknowledge that the sample size for the expert is small, indicating that further evaluation from more experts could be valuable in future studies.

%\begin{itemize}
%   \item The writing quality is reasonable.
%  \item The rating reflects the content of the review.
%\end{itemize}

%\subsection{Data Quality Validation}


%\kenneth{TODO: Update Gemma2 27b result}



\section{Experimental Results\label{sec:experiment}}

%\kenneth{TODO CY: Re-run Llama3 (x3) and Gemma2 (x2)---Will finish on Sunday Feb 2}

%\kenneth{TODO Zixin: Re-run the stats test after Feb 2}



\section{Empirical Evaluation}
\begin{table*}[!ht]
    \centering
    \resizebox{0.88\textwidth}{!}{    
    \begin{tabular}{r|cccccc|cccccc}
        \toprule 
        & \multicolumn{6}{c}{\textbf{LLaVA-1.5-7B}} & \multicolumn{6}{c}{\textbf{LLaVA-1.5-13B}} \\ 
        \cmidrule(lr){2-7}\cmidrule(lr){8-13}
        & \multicolumn{3}{c}{\textbf{MM-SafetyBench}} & \multicolumn{3}{c|}{\textbf{MOSSBench}} & \multicolumn{3}{c}{\textbf{MM-SafetyBench}} & \multicolumn{3}{c}{\textbf{MOSSBench}} \\
        \textbf{Method} & \textbf{DSR}$\uparrow$ & \textbf{RR}$\uparrow$ & \textbf{Avg}$\uparrow$ & \textbf{DSR}$\uparrow$ & \textbf{RR}$\uparrow$ & \textbf{Avg}$\uparrow$ & \textbf{DSR}$\uparrow$ & \textbf{RR}$\uparrow$ & \textbf{Avg}$\uparrow$ & \textbf{DSR}$\uparrow$ & \textbf{RR}$\uparrow$ & \textbf{Avg}$\uparrow$\\
        \midrule
        w/o Defense          & 0.06  & 0.98  & 0.52  & 0.14  & 0.97  & 0.55  & 0.10  & 0.97  & 0.53  & 0.30  & 0.96  & 0.63  \\
        \midrule
        \multicolumn{13}{c}{Baseline} \\
        \midrule
        Responsible          & 0.12  & 0.96  & 0.54  & 0.32  & 0.96  & 0.64  & 0.18  & 0.96  & 0.57  & 0.47  & 0.92  & 0.70  \\
        Policy               & 0.08  & 0.96  & 0.52  & 0.18  & 0.98  & 0.58  & 0.12  & 0.97  & 0.55  & 0.34  & 0.97  & 0.65  \\
        Demonstration        & 0.15  & 0.97  & 0.56  & 0.37  & 0.95  & 0.66  & 0.25  & 0.96  & 0.60  & 0.52  & 0.92  & \textbf{0.72}  \\
        SFT                  & 0.20  & 0.95  & 0.58  & 0.50  & 0.88  & 0.69  & 0.13  & 0.98  & 0.55  & 0.49  & 0.88  & 0.68 \\
        SafeDecoding         & 0.08  & 0.97  & 0.53  & 0.31  & 0.94  & 0.62  & 0.12  & 0.96  & 0.54  & 0.42  & 0.93  & 0.68  \\
        Caption              & 0.09  & 0.98  & 0.53  & 0.21  & 0.98  & 0.60  & 0.12  & 0.97  & 0.55  & 0.27  & 0.94  & 0.60  \\
        Caption (w/o image)  & 0.16  & 0.95  & 0.55  & 0.34  & 0.94  & 0.64  & 0.22  & 0.93  & 0.57  & 0.45  & 0.89  & 0.67 \\
        Intention            & 0.07  & 0.98  & 0.53  & 0.20  & 0.99  & 0.59  & 0.11  & 0.96  & 0.54  & 0.26  & 0.97  & 0.61  \\
        \midrule
        % \multicolumn{13}{c}{} \\
        % \midrule
        \midrule
        \multicolumn{13}{c}{SR++} \\
        \midrule        
        Responsible-Demonstration & 0.18 & 0.95 & 0.57 & 0.40 & 0.94 & 0.67 & 0.29 & 0.96 & 0.62 & 0.58 & 0.85 & \textbf{0.72} \\
        Responsible-Policy & 0.12 & 0.96 & 0.54 & 0.27 & 0.97 & 0.62 & 0.18 & 0.96 & 0.57 & 0.46 & 0.94 & 0.70 \\
        Policy-Demonstration & 0.13 & 0.96 & 0.55 & 0.37 & 0.97 & 0.67 & 0.20 & 0.96 & 0.58 &0.51 & 0.93 & \textbf{0.72}\\
        Responsible-Policy-Demonstration & 0.15 & 0.96 & 0.55 & 0.38 & 0.95 & 0.66 & 0.25 & 0.97 & 0.61 & 0.53 & 0.88 & 0.70\\
        \midrule
        \multicolumn{13}{c}{SR+MO} \\
        \midrule     
        Responsible-SFT & 0.56 & 0.93 & \textbf{0.75} & 0.61 & 0.72 & 0.67 & 0.35 & 0.96 & 0.65 & 0.74 & 0.62 & 0.68 \\
        Responsible-SafeDecoding & 0.30 & 0.96 & 0.63 & 0.54 & 0.87 & \underline{0.70} & 0.23 & 0.96 & 0.59 & 0.63 & 0.79 & 0.71\\
        Demonstration-SFT & 0.60 & 0.90 & \textbf{0.75} & 0.65 & 0.77 & \textbf{0.71} & 0.56 & 0.92 & \textbf{0.74} & 0.67 & 0.70 & 0.68\\
        Demonstration-SafeDecoding & 0.38 & 0.96 & \underline{0.67} & 0.55 & 0.87 & \textbf{0.71} & 0.40 & 0.96 & \underline{0.68} & 0.62 & 0.78 & 0.70\\
        \midrule
        \multicolumn{13}{c}{QR++} \\
        \midrule   
        Caption-Intention & 0.09 & 0.97 & 0.53 & 0.20 & 0.98 & 0.59 & 0.14 & 0.95 & 0.55 & 0.26 & 0.96 & 0.61\\
        % Caption-Intention (w/o image) & 0.18 & 0.96 & 0.57 & 0.32 & 0.95 & 0.64 & 0.25 & 0.92 & 0.59 & 0.45 & 0.92 & 0.68\\
        \midrule
        % \multicolumn{13}{c}{} \\
        % \midrule
        \midrule
        \multicolumn{13}{c}{QR\textbar{}SR} \\
        \midrule   
        Caption-Responsible & 0.34 & 0.96 & 0.65 & 0.53 & 0.79 & 0.66 & 0.33 & 0.96 & 0.65 & 0.50 & 0.82 & 0.66\\
        Intention-Responsible & 0.36 & 0.97 & \underline{0.67} & 0.51 & 0.86 & 0.68 & 0.27 & 0.96 & 0.61 & 0.49 & 0.90 & 0.70\\
        Caption-Responsible (w/o image) & 0.96 & 0.25 & 0.60 & 0.93 & 0.16 & 0.55 & 0.60 & 0.80 & \underline{0.70} & 0.72 & 0.72 & \textbf{0.72}\\
        % Responsible-Intention (w/o image) & 0.99 & 0.06 & 0.52 & 0.95 & 0.17 & 0.56 & 0.61 & 0.81 & 0.71 & 0.68 & 0.77 & 0.72\\
        \midrule
        \multicolumn{13}{c}{QR\textbar{}MO} \\
        \midrule
        Caption-SafeDecoding & 0.20 & 0.96 & 0.58 & 0.39 & 0.88 & 0.64 & 0.33 & 0.94 & 0.63 & 0.40 & 0.90 & 0.65 \\
        Intention-SFT & 0.28 & 0.97 & 0.62 & 0.43 & 0.78 & 0.61 & 0.25 & 0.96 & 0.60 & 0.50 & 0.88 & 0.69\\
        Caption-SafeDecoding (w/o image) & 0.24 & 0.95 & 0.60 & 0.41 & 0.89 & 0.65 & 0.36 & 0.85 & 0.61 & 0.56 & 0.84 & 0.70\\
        \bottomrule
    \end{tabular}}
    \caption{Comparison results of ensemble strategies with the corresponding individual defenses. \textbf{Bold} indicates the best overall performance, while \underline{underlined} highlights the top three methods.} % and the full score is 100\%
    \label{tab:en_inter_results}
\end{table*}


\subsection{Experimental Setup}
We empirically evaluate various defense methods and their ensemble strategies on LLaVA-1.5-7B and LLaVA-1.5-13B~\cite{liu2024visual} to validate their effectiveness in standard settings. Using MM-SafetyBench and MOSSBench datasets, we assess safety and helpfulness by measuring defense success rate (DSR) on harmful queries and response rate (RR) on benign queries. We evaluate 28 defense methods, including system reminders, optimization techniques, query refactoring, and noise injection, as well as inter- and intra-mechanism ensembles. Detailed descriptions of defense methods and experimental setups are provided in Appendix~\ref{sec:defense strategies} and~\ref{sec:experiment_detail}. 
For a broader evaluation, we add more experiments in Appendix~\ref{sec:utility}, ~\ref{sec:diverse_attacks} and~\ref{sec:time}, including evaluation with the MM-Vet dataset for testing the quality of model's response on general queries, tests on JailbreakV-28K for more diverse and complex attack scenarios, and a comparison of inference time for different defense methods.

\subsection{Individual Defense Results}

Table~\ref{tab:indi_results} shows results of individual defense methods across four categories. Most methods, except for noise injection, effectively improve model safety across different models and datasets, as evidenced by increased defense success rates. This aligns with our analysis in Figure~\ref{fig:analysis results} where system reminder, model optimization and query refactoring lead to an overall increase in refusal probabilities. 

\paragraph{Safety shift defenses compromise helpfulness.} System reminder and model optimization methods generally reduce response rates on the benign subset while increasing defense success rates on the harmful subset. This confirms that safety shift tend to compromise helpfulness. This is more pronounced in MOSSBench than MM-SafetyBench due to the more apparent harmfulness and concealed harmlessness in MOSSBench queries.

\paragraph{Harmfulness discrimination defenses mitigate over-defense.} Query refactoring methods, except for Caption (w/o image), generally achieve the highest response rates on the benign subset, particularly for MOSSBench with misleadingly benign queries. This validates that harmfulness discrimination improves the model's ability to distinguish between truly harmful and benign queries. Notably, the removal of images in the Caption (w/o image) significantly reduces response rates for both harmful and benign queries, highlighting the crucial role images play in jailbreaking LVLMs.
% \paragraph{Image matters.} The removal of images in the Caption (w/o image) and Intention (w/o image) defenses leads to significant improvements in DSR compared to their image-included counterparts, underscoring the crucial role that images play in jailbreaking LVLMs.

\paragraph{Multimodal defense is challenging.}
However, all individual defense methods still exhibit limited defense success rates. While larger-scale LVLMs (i.e., LLaVA-1.5-13B) tend to achieve slightly higher success rates, they are also more susceptible to over-defense. This underscores the inherent challenges of jailbreak defense for LVLMs, especially when relying on individual defense methods. 

\subsection{Ensemble Defense Results}
Table~\ref{tab:en_inter_results} provides the empirical evaluation of both inter-mechanism and intra-mechanism ensemble strategies, leading to the following insights:

\paragraph{Ensembles improve safety.} Compared to individual methods, most ensemble strategies effectively enhance safety across both datasets and model sizes, showing increased defense success rates, especially in \textit{SR+MO} and \textit{QR\textbar{}SR} methods.

\paragraph{Inter-mechanism ensembles amplify.} Our evaluation shows most \textit{SR++} and \textit{SR+MO} ensembles improve defense success rates while reducing responses rates, whereas the \textit{QR++} ensemble better maintain responses rates. This confirms that inter-mechanism ensembles can amplify a single defense mechanism. Specifically, safety shift ensembles would further enhance model safety at the expense of helpfulness, while harmfulness discrimination ensemble better preserves helpfulness. Among inter-mechanism ensembles, those combining different types of specific methods (e.g., SR+MO) show a more pronounced amplification effect than those combining the same type (e.g., SR++). 
Notably, the Demonstration-SFT method excels in defense strength, utility, and response rate. Its success comes from combining two strong safety shift defenses, Demonstration and SFT, which complement each other and boost overall performance.

\paragraph{Intra-mechanism ensembles complement.} Compared to inter-mechanism ensembles, most \textit{QR\textbar{}SR} and \textit{QR\textbar{}MO} methods—except those without input images—can simultaneously maintain decent defense success rates and stable response rates,
compared to the undefended model and individual defense methods. This demonstrates that intra-mechanism ensemble can complement each other to achieve a more balanced trade-off. Additionally, the removal of input images offering a most conservative ensemble for multimodal defense while still maintaining certain helpfulness.
% In contrast, the defenses in intra-mechanism ensemble complement each other, strengthening safety while maintaining a stable level of helpfulness.
% In contrast, intra-mechanism ensembles combine the strengths of both mechanisms to achieve a more balanced trade-off. Specifically, \textit{QR\textbar{}SR} and \textit{QR\textbar{}MO} increase the refusal probability for harmful queries, while maintaining or even decreasing the refusal probability for benign queries, thereby improving the model's ability to distinguish between benign and harmful queries. This makes them a better choice for general scenarios where balancing safety and helpfulness is essential. 


\subsection{How Do Fine-tuning Affect Model Safety?}
We examine how different fine-tuning methods impact the safety of LVLMs by training LLaVA-1.5-7B using DPO and SFT with two datasets: SPA-VL~\cite{zhang2024spa} and VLGuard~\cite{zong2024safety}. SPA-VL focuses on safety discussions, while VLGuard emphasizes query rejection. We also test the effect of adding 5000 general instruction-following data from LLaVA.  

Table~\ref{tab:training_dataset_results} shows that DPO with SPA-VL and LLaVA provides a slight safety boost without significantly changing response behavior. In contrast, SFT has a stronger impact, but its effectiveness depends on the dataset. SPA-VL improves safety while maintaining helpfulness, though it may miss some harmful cases. VLGuard, however, makes the model overly defensive, rejecting too many queries. Adding LLaVA data helps balance safety and helpfulness, reducing excessive refusals.  


\begin{table}[ht]
    \centering
    \resizebox{0.49\textwidth}{!}{
    \begin{tabular}{r|cccccc}
        \toprule 
        & \multicolumn{3}{c}{\textbf{MM-SafetyBench}} & \multicolumn{3}{c}{\textbf{MOSSBench}} \\
        \textbf{Method} & \textbf{DSR}$\uparrow$ & \textbf{RR}$\uparrow$ & \textbf{Avg}$\uparrow$ & \textbf{DSR}$\uparrow$ & \textbf{RR}$\uparrow$ & \textbf{Avg}$\uparrow$ \\
        \midrule
        w/o Defense          & 0.06  & 0.98  & 0.52  & 0.14  & 0.97  & 0.55 \\
        \midrule
        \multicolumn{7}{c}{DPO} \\
        \midrule
        \multicolumn{1}{l|}{SPA-VL + LLaVA}          & 0.06  & 0.97  & 0.52  & 0.28  & 0.97  & 0.63  \\
        \midrule
        \multicolumn{7}{c}{SFT} \\
        \midrule
        \multicolumn{1}{l|}{SPA-VL}          & 0.24  & 0.96  & 0.60  & 0.58  & 0.78  & 0.68  \\
        + LLaVA     & 0.20  & 0.95  & 0.58  & 0.50  & 0.88  & 0.69  \\
        \midrule
        \multicolumn{1}{l|}{VLGuard}          & 1.00  & 0.09  & 0.55  & 0.90  & 0.21  & 0.55  \\
        + LLaVA     & 0.97  & 0.43  & 0.70  & 0.76  & 0.58  & 0.67  \\
        \bottomrule
    \end{tabular}}
    \caption{Comparison of varying fine-tuning settings.} % and the full score is 100\%
    \label{tab:training_dataset_results}
\end{table}



%\section{Examining  of More Potential Confounding Factors}
\section{Examining Confounding Variables}
%Contributing to the Performance Gap

\begin{comment}
    

Relaxing strict semantic alignment between paired text items inevitably introduces confounding variables.
%We believe that 
This trade-off is worth exploring because it enables researchers to compare model behaviors across language varieties in new ways. 
%is worthwhile, as it enables comparisons of model behavior across language varieties in new ways. %that were previously impossible.
This section examines a few potential confounding variables.
%This section examines a few key factors.
\end{comment}

\paragraph{Could the performance gap be due to \cnChinese reviews having better \ul{writing quality} or better \ul{alignment between content and ratings}?}
\textit{Rationale:} Better writing quality or better content-rating alignment could make it easier for LLMs to predict ratings.
\textit{Analysis \& Findings:} \textbf{No.} 
Our human validation (Section~\ref{sec:data-quality-validation}) shows that \cnChinese reviews had slightly worse writing quality and content-rating alignment. 









\paragraph{Could the performance gap be due to more \ul{code-mixed usage} in \twChinese?}
\textit{Rationale:} NLP models often struggle with code-mixed data~\cite{zhang-etal-2023-multilingual, ochieng2024beyond}. 
%\kenneth{TODO Zixin: Add citations}
\textit{Analysis \& Findings:}
\textbf{No.}
%\twChinese reviews contained less code-mixed usage.
The \cnChinese reviews contain more mixed-language input (30.99\%) than the \twChinese reviews (25.26\%, see Appendix~\ref{appendix:language-analysis} and Table~\ref{tab:language-distribution}).
%\kenneth{TODO CY: I fixed the ref. Are these (1) ref and (2) numbers correct?}\cy{number is correct and based on our latest table}
%Our findings aligned with other studies investigating LLMs' performance between simplified Mandarin (used in Mainland China) and traditional Mandarin (used in Taiwan), which showed potential performance disparity across multiple NLP tasks in Taiwan Mandarin~\cite{chen2024measuring}.

\paragraph{Could the performance gap be due to \cnChinese \ul{users} systematically \ul{giving higher scores}, which align better with LLM-generated scores?}
\textit{Rationale:} LLMs tend to assign higher scores~\cite{stureborg2024large,kobayashi-etal-2024-large,golchin-etal-2025-grading}.
%wang-etal-2024-large-language-models-fair,
%stureborg2024large,wang-etal-2024-large-language-models-fair,
%kobayashi-etal-2024-large
% \kenneth{TODO CY and Zixin: Is this true? If so, add citations.}
\textit{Analysis \& Findings:}
\textbf{Unlikely.}
%Even though we had fewer restrictions on the review pairing process (using similar rating reviews rather than reviews with the same scores) to maximize pairing amounts, this process didn't introduce biases in score rating. 
In our dataset, \twChinese and \cnChinese reviews show no significant difference in scores (\textit{t}(22917) = .160, \textit{p} = .873).
%In our dataset, there is no statistical difference in actual scores between \twChinese and \cnChinese reviews (\textit{t}(22917)=.160, \textit{p}=.873). 
%This result showed that in our data set, 
%Namely, reviews written in \cnChinese and \twChinese share a similar distribution of rating scores; there is no systematic scoring difference between the two groups.

\paragraph{Are \cnChinese reviews \ul{easier for humans to guess ratings}?}
\textit{Rationale:} Human performance is sometimes used as an indicator of a task's difficulty for LLMs~\cite{sakamoto-etal-2025-development,ding2024easyhardbench}.
% \kenneth{TODO CY and Zixin: Add citations.}
\textit{Analysis \& Findings:}
\textbf{Plausible.}
We conducted a user study with 10 participants (5 native speakers from each variety) who reviewed 50 random CN-TW review pairs (100 total reviews) and predicted their rating scores.
%Participants also rated the readability and nativeness of each review.
%More details are provided in \Cref{app:human-prediction}.
%Although there was no significant difference in readability or nativeness between CN and TW reviews, 
%In the study, 
Participants performed significantly better at predicting ratings for reviews in \cnChinese.
After excluding two TW native speakers whose accuracy was more than two standard deviations below the mean, 6 out of the 8 participants had better accuracy on CN reviews than TW reviews, and 7 had better (lower) MSE on CN reviews than TW reviews (see \Cref{app:human-prediction} for more details).
%\kenneth{TODO Zixin: (1) Update these numbers, (2) Move all other results of this study to Appendix: ``For overall content quality, our results show no significant differences in score predictions among the data pairs, indicating raters have no biases in reading and understanding reviews from either group of speakers/writers. However, results showed statistical significance in both ACC (37.00\% vs. 28.75\%, \textit{p}=.016) and MSE (2.795 vs. 3.510, \textit{p}=.036), showing that native speakers might have more difficulties in correctly guessing the review scores for reviews in \twChinese.''}
%We want to emphasize that 

These results should be interpreted with caution.
Unlike question-answering, predicting hundreds of review scores from content is not a typical human task, and most NLP papers on sentiment analysis do not compare model performance to human performance.
Thus, it is unclear whether human performance gaps in such tasks reliably indicate task difficulty for LLMs, especially given the small differences between the two varieties.
%As a result, it is unclear whether our findings indicate task difficulty for LLMs, especially given the relatively small differences in human performance between the two varieties.
Additionally, our participants may not represent the average Mandarin speaker's ability in sentiment analysis, as the two participants performed notably poorly. 
Finally, despite our efforts to examine confounding variables such as text length, code-mixing, and writing quality, we still \textbf{lack a clear understanding of what causes the observed LLMs' performance gaps across language varieties}.
%Further research is needed to understand LLM behavior across language varieties.
%Poor human performance does not necessarily mean LLMs will struggle in the same way.











%After excluding two TW native speakers whose accuracy was more than two standard deviations below the mean, 6 of the remaining 8 participants had higher accuracy on CN reviews than on Tw reivews, and 6 had lower MSE on CN reviews. 











%\zixin{1. no significant differences in naiveness and readability between two sets of reviews; 2. within raters, there are no significant differences in score predictions across \cnChinese and \twChinese review pairs - native speakers perceive similarly between \cnChinese and \twChinese reviews; \textbf{3. there are stat. significances in terms of ACC and MSE in prediction tasks with better performance towards \cnChinese reviews.}} 



%\kenneth{TODO Zixin: Add text here}

%, meaning this factor does not account for the performance gap.


%\kenneth{TODO: Human study will be done by Feb 3 (Mon)}



\section{Discussion}
%\kenneth{TODO: One more subsection under Discussion to debunk the common alternative interpretations, including (1) writing quality, (2) code mix, (3) extra human study}

%\kenneth{Encoding? How often Google Translation changed the content?}

\section{Discussion}
\label{sec:discussion}
In this work, we propose to leverage few-shot learning to enable users to self-define personal undesirable actions for personalized intervention on smartwatches.
We developed a three-stage pipeline that began with a self-supervised, pre-trained IMU model for robust feature extraction, then fine-tuned it for accurate human activity recognition, and finally enhanced it with data augmentation and synthesis that enabled rapid customization of new user-defined actions using only a small number of examples. 
We implemented this pipeline on a smartwatch as a real-time intervention system, \projectname, and demonstrated its effectiveness and advantages over the rule-based method through a multi-hour user study.
In this section, we discuss some interesting takeaways from our study, together with our vision of how \projectname can be generally applied to other health domains. We also briefly summarize the limitations of our work.


\subsection{Distorted Perception with AI-powered Intervention}
\label{sub:discussion:distorted}
During the study, we observed an interesting phenomenon where some participants developed a distorted perception towards their own actions or the intervention (see Sec.~\ref{sub:intervention_evaluation:qualitative_results}).
For instance, several participants felt \projectname's vibrations were stronger than the baseline (yet the actual strength of vibration remained constant), and some felt they did the target actions more frequently with \projectname (yet the objective data indicated otherwise).
There are several potential interpretations of such interesting observations.
The distorted perception might be caused by participants' heightened awareness of the AI-guided interventions: because \projectname more accurately and promptly caught the target actions, users started to pay extra and prolonged attention to any intervention. This could leave a stronger impression on them, and subsequently, they found it stronger or more frequent.
Another potential explanation is that the participants, often associating their personal and idiosyncratic undesirable actions with ``wrong-doing'' and thus responding with negative emotions, might have subconsciously perceived their undesirable actions as being more frequent due to the \projectname's more precise and timely feedback eliciting stronger negative emotions. This, combined with an emotional interpretation of being 'corrected', may have amplified their perception of the intervention's intensity (vibration strength) and created the mistaken impression of performing these actions excessively.

Meanwhile, it is an interesting open question of how long such perception will last from a longitudinal intervention perspective. Depending on the cases, the growing self-awareness and/or reliability of AI could potentially assist users in building a long-term habit to reduce the target action, or on the contrary, the effect may fade away due to the AI intervention method no longer being novel or enticing.
Future work can explore the lasting effect of the intervention, the corresponding perception, as well as user engagement in a long-term, field-based intervention study.~\cite{middleton2013long, short2018measuring, wei2020design}.


\subsection{Towards Human-AI Collaborative Interventions}
\label{sub:discussion:collaboration}
Users' mental models of \projectname varied significantly. Some viewed it as a passive watchdog, and some viewed it as a playful interactive system, while others sought to take greater agency in the moment of intervention delivery.
Our findings show the potential for and benefit of developing a collaborative relationship between humans and AI for behavioral intervention.
An AI system can provide appropriate support to users and help them achieve effective intervention outcomes.
Such collaboration is closely relevant to the vision of just-in-time adaptive interventions (JITAIs)~\cite{nahum-shani_translating_2021, nahum2018just}, where the delivery timing and methods of intervention are designed to be dynamically adapting to an individual's internal state and surrounding context.

For instance, for users who see the system as a passive monitor, the system can provide the option for them to configure the frequency and style of intervention (\eg higher/lower-intensity vibrations or consolidated notifications), ensuring the AI remains in the background but still provides supportive nudges.
Taking one step further, the AI system may analyze user behavior over time and suggest new setups or goals for users with transparency (\eg transitioning from nail-biting to managing stress). Users can accept, modify, or reject these suggestions, creating a dialogue where AI acts as a coach or collaborator rather than a rigid enforcer of predefined behaviors.
Meanwhile, for those who see AI as a proactive system, one promising avenue is to incorporate user feedback into the AI's learning process~\cite{orzikulova2024time2stop}. Users can label the AI's predictions as accurate or not, which could serve as input for the model to further adapt to the user and improve performance over time (\eg through reinforcement learning).
Combined with contextual information that can potentially be inferred from sensors~\cite{xu2023globem}, such feedback can enable more precise, context-sensitive and personalized JITIs.
In addition, the system would periodically prompt users to reassess their goals and update intervention targets, ensuring long-term relevance and efficacy.

It is noteworthy that such a human-AI collaboration paradigm needs to follow the principles of transparency and ethical design.
Other than the options mentioned above, namely custom configurations and continuous feedback, users should have visibility into the system's functionality and action logic regardless of the specific collaboration setup. This is important to provide users with agency and build their trust in the system.

\subsection{Beyond Smartwatch and Broader Customization}
In this work, our real-time intervention was implemented on a smartwatch. However, our proposed idea of empowering users to define any personal action and achieve a personalized intervention system can be more broadly applied to other domains.
Instead of relying solely on a watch-based IMU, we can explore other body-based sensor arrays (\eg headbands, rings, or joint sensors) to capture a more diverse range of behaviors in real time.
This would enable the system to accommodate a wide variety of undesirable actions or habits, such as posture corrections and fidgeting management.
In addition, beyond physical interventions, future customization can also delve into psychological or mental health support.
For instance, individuals dealing with obsessive-compulsive disorder (OCD) or other habitual thought/action patterns could define personal triggers (\eg a particular repetitive motion or behavioral cue) and receive timely AI-driven interventions.
Such holistic approaches highlight the flexibility and scalability of our pipeline.
By enabling user-defined actions, we open up possibilities for long-term and effective management of both physical and psychological well-being using a multitude of wearable and sensor-based platforms.

\subsection{Limitations}

Despite \projectname's positive outcome and the promising insights generated, we recognize some limitations in our study design.
As mentioned above, our current model relies solely on accelerometer data for action recognition, which may limit its ability to capture the full range of motion characteristics or other physiology. Future work can explore additional sensing modalities, such as gyroscope, photoplethysmography (PPG), joint locations, to enhance the accuracy and robustness of action recognition. 
Besides, the study was conducted with a relatively small number of participants and a limited set of actions, which may not fully capture the variability and diversity of human activities in real-world scenarios \cite{trapp2015individual, narayanan2013behavioral}.
Additionally, although we tried to simulate real-life scenarios, our intervention study was conducted over a limited duration and in controlled experimental settings, which may not fully reflect the complexities and dynamics of real-life environments. 
Real-world contexts introduce factors such as environmental noise, varying sensor placements, and user behavior changes over time \cite{trapp2015individual,truong2015deployment,mejia2023enhancing,mills2022development}, which were not thoroughly simulated in this study. Future research should conduct longitudinal field experiments with real-world deployment of the system.





\section{Conclusion and Future Work}
\section{Conclusion and Future Work}

This study introduces \model{}, a model-agnostic and prior-knowledge-free targeted poisoning attack for practical federated recommender systems (FRS). 
We propose \modelI{} and \modelII{} as two diverse solutions to increase the exposure of target items based on effectively mining popular items during FRS training. Existing federated defenses have been found ineffective against \model{}, leading us to propose a new defense method with two well-designed regularization terms. Extensive experiments across model types, datasets, attacks, and defenses have validated the efficacy of our proposals.

For future work, it is interesting to explore collaborative defense methods that combine both server-side and client-side strategies. 
% \changeone{Moreover, our defense is effective against existing targeted\rone{R1.O2} model poisoning attacks, but the future exploration of more general defenses that adapt to untargeted model poisoning attacks is warranted.} 
It is also interesting to extend our attack and defense methods to content-based federated recommendations.

% \vfill
\section*{Acknowledgment}
This work is supported by the National Key R\&D Program of China (No.~2022YFB3304100), the Pioneer R\&D Program of Zhejiang (No.~2024C01021), and the Major Research Program of Zhejiang Provincial Natural Science Foundation (No.~LD24F020015).
% This work is supported by the National Key R\&D Program of China (No.2022YFB3304100) and Fundamental Research Funds for the Central Universities.

\section{Limitations}
\section{Limitations}

The limitations of this work can be summarized in two main aspects.

First, while {\name} demonstrates lower accuracy degradation compared to existing methods while accessing a comparable proportion of KV cache,
it still exhibits non-negligible performance degradation.
This suggests opportunities for future work to investigate adaptive attention sparsity allocation strategies that dynamically optimize the sparsity ratios across layers and attention heads, based on their contextual importance.

Second, while {\name} increases long context serving throughput by up to \(2.7 \times\), our current implementation is limited to single-GPU deployment.
Future research could further explore (1) distributed multi-GPU system designs for scaled deployment, (2) integration with disaggregated LLM serving architectures like MoonCake~\citep{mooncake}.

\section{Ethics Statement}
%\cy{Still missing}
We assess that the general risks and ethical concerns of our work are no greater than those involved in using user-generated reviews to test sentiment analysis models.








\section*{Acknowledgement}
We thank the anonymous reviewers for their feedback and the participants for their contributions to our human studies. 
This work was partially supported by the 2024-2025 Seed Grant from the College of Information Sciences and Technology at Pennsylvania State University. 
We also acknowledge Dr. Janet G. van Hell, Co-PI of the seed grant, for her support and valuable input. 
Additionally, this work was partially supported by the National Science and Technology Council (NSTC), Taiwan, under the project ``\textit{Taiwan's 113th Year Endeavoring in the Promotion of a Trustworthy Generative AI Large Language Model and the Cultivation of Literacy Capabilities (Trustworthy AI Dialog Engine, TAIDE)}.''




% Bibliography entries for the entire Anthology, followed by custom entries
\bibliography{bib/custom}
% Custom bibliography entries only
%\bibliography{custom}

\appendix

\clearpage
\begin{appendices}

\section{Production Fault Trace}
\label{appendix:production-fault-trace}
The production fault trace was collected from an 8-GPU node pretrain cluster with 2880 GPUs over a period of 160 days. The trace includes details such as fault start time, fault end time, and the ID of the faulty node. \figref{fig:simulation:trace:timetrace} and \figref{fig:simulation:trace:cdf} provide a macro-level overview of the production fault trace. On average, the ratio of faulty 8-GPU nodes at any given time is $3.83\%$, with a p99 value of $7.22\%$.

\begin{figure}[h!t]
    \centering
    \begin{subfigure}[b]{0.23\textwidth}
        \centering
        \includegraphics[width=\textwidth]{figs/evaluation/fault_server_ratio.pdf}
        \caption{Fault Node Ratio Trace.}
        \label{fig:simulation:trace:timetrace}
    \end{subfigure}
    \hspace{2pt}
    \begin{subfigure}[b]{0.23\textwidth}
        \centering
        \includegraphics[width=\textwidth]{figs/evaluation/fault_server_cdf.pdf}
        \caption{Cumulative Distribution.}
        \label{fig:simulation:trace:cdf}
    \end{subfigure}
    \vspace{-2ex}
    \caption{Fault node trace in the production AI DC.}
    \label{fig:simulation:trace}
\end{figure}

Since most of failure events are GPU faults, we normalized the trace of 8-GPU nodes to generate 4-GPU nodes trace. Assuming that the fault rates of GPUs are i.i.d. with a fault probability of $p$ for each GPU, and considering that a node is deemed faulty if any GPU within it fails, the fault rate of an 8-GPU node is calculated as:  

\vspace{-1em}
$$
P_{fault}(8\text{-GPU}) = 1 - (1-p)^8 = 3.83\%.
$$  

From this, we derive $p = 0.49\%$. The fault rate for a 4-GPU node is then:  
$$
P_{fault}(4\text{-GPU}) = 1 - (1-p)^4 = 1.93\%.
$$  

The fault event of 4-GPU node is generate with Bayesian Equation, as:


\begin{align*}\label{eq:convert-trace}
& P_{fault}( \text{4-GPU} \mid  \text{8-GPU})\\ 
    &=\frac{P_{fault}(\text{8-GPU} \mid \text{4-GPU}) P_{fault}(\text{4-GPU})}{P_{fault}(\text{8-GPU})} \\ 
    & =  \frac{1 \times 1.93\%}{3.83\%} = 50.39\% \\
\end{align*}

Thus, whenever a fault occurs in an 8-GPU node in the original trace, each of the two corresponding 4-GPU nodes at the same location has a $50.39\%$ probability of fault. This method is used to convert the traces.

As node faults are i.i.d., the simulator linearly maps the fault trace to different network architectures.

\section{GPT-MoE Architecture}
\label{appendix:gpt-moe}
This model is a mixture-of-experts (MoE) model with the following configuration:

\para{Model Configuration:}
\begin{itemize}
    \item \textbf{Number of Layers:} 192
    \item \textbf{Inner Layer Dimension:} 49152
    \item \textbf{Embedding Dimension:} 12288
    \item \textbf{Hidden Dimension:} 12288
    \item \textbf{Vocabulary Size:} 64000
    \item \textbf{Number of Attention Heads:} 128
    \item \textbf{Maximum Sequence Length:} 2048
    \item \textbf{Number of Experts:} 8
    \item \textbf{MoE Layer Ratio:} 0.5
    \item \textbf{Top-K Experts:} 2
\end{itemize}

\para{Runtime Configuration:}
\begin{itemize}
    \item \textbf{Virtual Pipeline Parallelism:} 3
    \item \textbf{Micro Batch Size:} 1
    \item \textbf{Global Batch Size:} 1536
    \item \textbf{Max Sequence Length:} 2048
\end{itemize}




\section{Theoretical analysis of wasted GPU ratio for \sys}
\label{appendix:ft-anay}

The count of backup lines as $2K - 2$ will significantly influence the fault tolerance of \sys. We use the expectation of waste ratio caused by GPU failure and fragmentation problem to evaluate this design, the result is shown in \tabref{table:design:1.5ratio}.

For one single working server in the middle of line, the count of breakpoints $B$ on its two sides has the expectation as:

\vspace{-1em}
\begin{equation*}
E_B(\eta = 1,middle) = 2(P_s^K + P_s^{2K})
\end{equation*}

Where $P_s$ is the fail probability of GPU server, and $\eta$ is count of servers. The expectation of breakpoints count is:

Once the distance between one server and the tail of line is $\alpha < K$, it will connect to all servers between itself and the last one, so there will be no breakpoints on this side, and the expectation of breakpoints count is less than servers in the middle of line.
Then, for any server in the line topology:

\vspace{-1em}
$$
E_B(\eta = 1) \leq E_B(\eta = 1,middle) 
$$

When the distance between two servers is $\beta \geq K$, the breakpoints among them can be calculated as independent.
Once the distance $\beta < K$, as all servers in this range are connected to these two servers, there will be no breakpoints between them. So, the expectation is less than two independent servers. Then,



\vspace{-1em}
\begin{align*}
E_B(\eta =& 2) < E_B(\eta = 2, \beta \geq K) =  2E(\eta = 1)   \\ 
 E_B(\eta =& N_s) \leq N_s E_B(\eta = 1) 
\end{align*}

For a LLM job which require a ring communication size (TP .etc) as $N_t$, \sys   will cut the whole line topology into several sub lines with the length of $N_t/R$.
Once \sys is cutting a new sub line from the remaining servers in the line, 
all $N_t$ GPU will be wasted when one break point exist in the middle of this sub line required, shown in \fig{fig:subline-waste}. 
Then the expectation for waste GPU caused by one single break point is:

\vspace{-1em}
$$
E_W(B=1) = N_t R\cdot (1 - (N_t/R)^{-1} ) = R(N_t -R)
$$

\begin{figure}[h!t]
    \centering
    \includegraphics[width=0.8\linewidth]{figs/design/intra-topo/break-topo.drawio.pdf}
    \caption{Break point can cause server waste compare to ideal situation.}
    \vspace{-1em}
    \label{fig:subline-waste}
\end{figure}

As the influence between two break points only reduce the expectation of wasted GPUs, we can have this for $X$ break points:

\vspace{-1em}
\begin{equation*}
E_W(B = X) \leq XE_W(B=1) = XR(N_t-R)
\end{equation*}

So the expectation of wasted GPU for a servers cluster with $N_s$ GPU servers is:

\vspace{-1em}
\begin{align*}
E_W(\eta = N_s) &\leq \sum P(B=X ,\eta = N_s) \cdot X\cdot  E_W(B=1)\\
&= E_B(\eta = N_s)\cdot E_W(B=1)\\
&\leq  \lim_{P_s\rightarrow 0}2N_s\cdot R \cdot (N_t-R)P_s^K
\end{align*}



The final expectation of GPUs waste ratio is \eqref{eq:design:ratio}:

\begin{equation}
E_{WR}(\eta = N_s) = \frac{E_W(\eta = N_s)}{N_g} \leq 2(N_t-R)(P_s)^K
\label{eq:design:ratio}
\end{equation}

In our trace for a 160 days long pre-train job on 10K-GPU, the p99 failure rate for 8-card machines is 7\%. If a TP32 jobs is running on \sys, we can get the upper bond for waste ratio expectation for various configuration in \tabref{table:design:1.5ratio}.

\begin{table}[h!t]
\centering
\begin{tabular}{cccc}
    \toprule
        & $K=2$&$K=3$&$K=4$\\
    \midrule
     R=4& $7.35\%$ & $0.26\%$ & $9.00\times 10^{-4}$ \\
     R=8& $27.4\%$ & $1.92\%$ & $0.13\%$ \\
     \bottomrule
\end{tabular}
\caption{Upper bond for waste ratio expectation of GPU, where GPU failure rate is 0.875\% and X is 32}
\vspace{-2em}
\label{table:design:1.5ratio}
\end{table}

As shown in the table, for 4 GPU server ($R=4$) 3 bundles ($K = 3$) design, the additional waste of GPU is less than 0.26\%, while the waste ratio for $R=8,K=4$ is less than 0.13\%. This is sufficient for production clusters. 

\section{Orchestration For Fat-Tree}
\label{appendix:orch-algo}
In this section, we introduce the orchestration algorithm under Fat-Tree DCN in detail.

\para{Notations}
\label{appendix:orch-algo:notation}
To ensure rigorous mathematical reasoning, we introduce the following notations:

\begin{itemize}
    \item {
        $n$: number of nodes in the data-center.
    }
    \item {
        $K$: \docs{} bundle (see \S\ref{section:design:topology}).
    }
    \item {
        $S_{all}$: ordered set, represents all nodes numbered from 1 according to their physical connection order in DCN fabric. $|S_{all}|=n$.
    }
    \item {
        $S$: ordered subset, represents nodes, $\forall u \in S, u \in S_{all}$. Adjacent elements in $S$ are also adjacent from the perspective of the \SYS{} topology. 
    }
    \item{
        $E$: The set of edges across $S$, should be equal to $\{ (S_i, S_j) \mid 1 \leq i < j \leq n, j - i \leq K \} $, representing the connections between nodes, including both primary and backup links, and $O(|E|) = O(K|S|)$.
    }
    \item {
        $InfHBD=<S,E>$: the topology of \SYS{} as an undirected graph.
    }
    \item {
        $F$: faulty nodes.
    }
    \item {
        $HealthyHBD=<H,HE>$: healthy node subgraph where the set of healthy nodes $H = S - F$ and the edge set $HE = \{ (u, v) \mid u \in H \text{ and } v \in H \text{ and } (u, v) \in E \}$.
    }
    \item{
        $t$: TP size, number of GPUs in one TP Group.
    }
    \item{
        $r$: GPU ranks per node.
    }
    \item{
        $m=t/r$: number of nodes in a TP group.
    }
    % \item{
    %     $k$: number of rails in rail-optimized network.
    % }
    \item{
        $s$: job scale, number of GPUs required for the job.
    }
    \item{
        $d$: Aggregation-Switches Domain size. Number of nodes under coverage of one group of Aggregation-Switches.
    }
    \item{
        $n_{constrains}$: number of applied constraints in binary-search-based orchestration algorithm.
    }
    \item{
        $p$: number of nodes under each ToR.
    }
    \item{
        $l$: shortest sub-line length under fat-tree orchestration.
    }
    \item{
        $n_{maxsubline}=\lfloor \frac{nd}{p} \rfloor$: max number of sub-lines.
    }
    \item{
        $G_{deploy}=<S_{deploy},E_{deploy}>$: deployed topology. After applying the deployment strategy, the topology from the perspective of \SYS{} is described as follows: $S_{\text{deploy}}$ is an ordered set where adjacent elements correspond to adjacent nodes in \SYS{}, and $E_{\text{deploy}}$ represents the connections between nodes.
    }
    
\end{itemize}


% For the \SYS{} the orchestration algorithm in ideal conditions is relatively straightforward. The detailed steps of the algorithm are outlined in \algref{alg:orchestration-ideal}.

% Assume that the \SYS{}(with \docs{} direction $K$) is represented as an undirected graph $ \text{InfHBD} = \langle S, E \rangle $, where the ordered set of nodes $ S $ represents nodes. Adjacent elements in $S$ are also adjacent from the perspective of the \SYS{} topology. The set of edges $E$ should be equal to $\{ (S_i, S_j) \mid 1 \leq i < j \leq n, j - i \leq K \} $, representing the connections between nodes, including both primary and backup links, and $O(|E|) = O(K|S|)$. The set of faulty nodes is denoted as $ F \subseteq S $.

% The algorithm proceeds as follows:

% \begin{enumerate}
%     \item {\textbf{Extract the Healthy Node Subgraph:} First, extract the subgraph $\text{HealthyHBD} = \langle H, HE \rangle$ where the set of healthy nodes $H = S - F$ and the edge set $HE = \{ (u, v) \mid u \in H \text{ and } v \in H \text{ and } (u, v) \in E \}$. See \algref{alg:orchestration-ideal}.
%     }
%     \item {\textbf{Identify Connected Components:} Next, identify all connected components in the graph $\text{HealthyHBD}$. Faulty nodes may cause disconnections in the \SYS{} fabric, splitting the original cluster into multiple sub-HBDs. These sub-HBDs are the connected components, and TP Groups cannot span across these disconnected sub-HBDs. We use a simple Depth-First Search (DFS) algorithm here. See \algref{alg:dfs}.}
%     \item {\textbf{Generate Placement Scheme:} Given the excellent physical properties of the \SYS{}, TP Groups can be arranged sequentially within each connected component to generate placement scheme maximizing GPU utilization. See \algref{alg:orchestration-ideal}.
%     }
% \end{enumerate}

% Since each of the three steps involves traversing the entire graph's edges and nodes only once, 
The orchestration algorithm (\algref{alg:orchestration-ideal}) without considering DCN has the overall time complexity $3\cdot O(|H| + |HE|) = O(|S| + |E|) = O((K+1)|S|) = O(|S|)$.

% \begin{algorithm}[!h]
% \small
% \caption{Connected-Component-DFS}
% \label{alg:dfs}
% \SetAlgoNlRelativeSize{-1}
% \SetAlgoNlRelativeSize{1}
%  \KwIn{ $node$, $HealthyHBD$, $visited$}
%  \KwOut{ $component$}

%  Initialize $stack = [node]$ \;
%  Initialize $component = []$\;

% \While{ stack is not empty}
% {
%      $current = stack.pop()$\;
%     \If{$current$ not in $visited$}
%     {
%          Add $current$ to $visited$\;
%          Add $current$ to $component$\;
%         \For{ each neighbor in $HealthyHBD.neighbors(current)$}
%         {
%              $stack.push(neighbor)$\;
%         }
%     }
% }
        
% \KwRet{$component$}
% \end{algorithm}

\begin{algorithm}[!h]
\small
\caption{Orchestration-DCN-Free}
\label{alg:orchestration-ideal}
\SetAlgoNlRelativeSize{-1}
\SetAlgoNlRelativeSize{1}
\KwIn{$\text{InfHBD}=\langle S, E \rangle$, $F$, $m$}
\KwOut{ Placement scheme maximizing GPU utilization}

 Initialize $H = S - F$\;
 Initialize $HE = \{ (u, v) \mid u \in H \text{ and } v \in H \text{ and } (u, v) \in E \}$\;
 Create subgraph $HealthyHBD = \langle H, HE \rangle$\;
 Initialize $component\_list = []$\;
 Initialize $visited = \{\}$\;
 Initialize $placement\_scheme= \{\}$\;

\For{ each node $s$ in $H$}
{
    \uIf{ $s$ not in $visited$}
    {
         $component = Connected-Component-DFS(s, HealthyHBD, visited)$\;
         Add $component.sortedinHBD()$ to $component\_list$\;
    }
}
\For{ each $component$ in $component\_list$}
{
    \While{ $component.size()\geq m$}
    {
         Add $component.pop(m)$ to $placement\_scheme$\;
    }
}
        
 \KwRet{$placement\_scheme$}
 \end{algorithm}
 
% \subsection{Algorithms under Rail-Optimized Network}
% \label{appendix:orch-algo:rail-optimized}

% This subsection provides a detailed description of the orchestration algorithm for Rail-Optimized network.  

% The rail-optimized network topology is specifically designed for highly regular machine learning workload traffic patterns, making it a commonly used and effective architecture. As illustrated in \fig{fig:rail-topo}, Rail Switch $i$ connects to GPU $i$ in node, dividing the network into multiple rails. Let $r$ denote the GPU ranks per node, and $k$ the number of rails. In traditional rail-optimized networks, $k = r$, and a typical training strategy involves running TP $r$ within the single-node HBD, while DP operates between HBDs. Since in DP, GPUs only communicate with GPUs of the same rank in different TP groups, in other words, DP traffic is confined to the rail itself. Therefore, the Rail-Optimized topology perfectly meets this requirement.

% % \begin{figure}[!h]
% %     \centering
% %     \includegraphics[width=\linewidth]{figs/design/Orchestration/rail-optimized.drawio.pdf}
% %     \caption{Rail-Optimized Network: GPU ranks per node $r=4$, Number of rails $k=8$, Aggregation-Switches Domain size $d$, Number of Aggregation-Switches Domain $nd$, Node IDs from 1 to $nd\cdot d$. }
% %     \label{fig:rail-topo}
% % \end{figure}

% \para{Orchestration Constraints. }To minimize the cross-rail traffic which can lead to congestion and latency, the rail-optimized network introduces two key constraints for orchestration algorithms:


% \begin{itemize}
%     \item {
%         \textbf{Aggregation-Switches Domain Coverage Constraint. }
%         The coverage domian of a group of Aggregation-Switches is limited, meaning that TP groups spanning across Aggregation-Switches domains would result in cross-rail traffic, which should be avoided as much as possible.
%     }
%     \item {
%         \textbf{Node Rail State Constraint. }When$ k = r$, this constraint does not apply, as there is no cross-rail traffic.However, as HBDs extend beyond single nodes and the need for larger DP scales due to the expansion of LLM scale, scenarios with $k = p \cdot r$ may arise. This results in $p$ different node states within the data center, with each state occupying $r$ rails, and inter-state communication leads to cross-rail traffic. The specific form of this constraint depends on the deployment strategy.
%     }
% \end{itemize}

% \para{Deployment Strategy. }If the \SYS{} connections continue to follow the physical layout of nodes on the DCN Fabric, avoiding cross-rail traffic would require each TP Group to have an equal number of nodes from each state, making the algorithm to maximize GPU Utilization NP-Complete (see Appendix.\ref{appendix:np-hard-orchestration}). However, by altering the physical connection sequence of \SYS{}, this NP-Complete problem can be reduced to polynomial time. As shown in \fig{fig:parallel-line}, nodes of each state are arranged into $p$ parallel sub-lines, which are then connected end-to-end to form a single line. By restricting DP to operate within sub-lines, all DP traffic remains within the rails, effectively reducing the $k = p * r$ scenario to $k = r$. 

% % \begin{figure}[!h]
% %     \centering
% %     \includegraphics[width=\linewidth]{figs/design/Orchestration/parallel-line.drawio.pdf}
% %     \caption{The deployment strategy example with $p=4$ and Aggregation-Switches Domain size $K=8$. Node IDs from 1 to n are arranged according to their connection order in the DCN Fabric.}
% %     \label{fig:parallel-line}
% % \end{figure}

% \para{The binary search-based Orchestration algorithm.} Based on the above-mentioned constraints and the deployment strategy, we developed an orchestration algorithm that maximizes the number of constraints satisfied while meeting the job scale requirements. This is achieved using a binary search approach with the number of satisfied constraints as the variable. Both types of constraints essentially involve splitting the Line into sub-lines. Therefore, controlling the number of constraints translates to managing the number of sub-lines: fewer sub-lines mean longer sub-lines, leading to higher GPU Utilization. Since the Ideal orchestration algorithm with complexity $O(n)$ can be applied within sub-lines.

% \algref{alg:orchestration-fat-tree} is the main binary-search-based orchestration algorithm. It begins by generating the topology from the perspective of \SYS{} based on the hardware deployment strategy (\algref{alg:deployment-strategy}). Using the number of satisfied constraints as a variable, the algorithm performs a binary search to identify the placement scheme that maximizes the number of satisfied constraints while meeting the job scale requirements.  

% \algref{alg:placement-rail-optimized} calculates the placement scheme for a given number of constraints. It divides the topology into multiple ideal sub-lines and applies the ideal-case orchestration algorithm (\algref{alg:orchestration-ideal}) to each sub-line.  

% Since the time complexity of \algref{alg:orchestration-ideal} is $O(|S|)$, the complexity of \algref{alg:placement-rail-optimized} is 

% \begin{align*}
% &\sum_{i=1}^{n_{constraints}} O(|S_{subline}|) \\
% &= O(\sum_{i=1}^{n_{constraints}} |S_{subline}|) \\
% &= O(|S_{all}|) = O(n)
% \end{align*}

% Thus, the overall time complexity of \algref{alg:orchestration-rail-optimized} is $O(n \log n)$.

\begin{algorithm}[!h]
\small
\caption{Deployment-Strategy}
\label{alg:deployment-strategy}
\SetAlgoNlRelativeSize{-1}
\SetAlgoNlRelativeSize{1}
 \KwIn{Node ordered set $S$, \docs{} direction $K$, parallel factor $p$}
 \KwOut{Deployment topology $G_{deploy}=<S_{deploy},E_{deploy}>$}
 Initialize ordered set $S_{deploy}=[]$\;
 Initialize $l=\lfloor \frac{|S|}{p}\rfloor$\;
\For{$i$ in $0...p-1$}
{
    \For{$j$ in $0...l-1$}{
         Add $i+j\cdot p$ to $S_{deploy}$\;}
}
 Create $E_{deploy}=\{(S_{deploy}^i,S_{deploy}^j)|1\leq i\le j\leq |S_{deploy}|, j-i\leq K \}$\;
 \KwRet{$G_{deploy}=<S_{deploy},E_{deploy}>$}
\end{algorithm}


% \begin{algorithm}[!h]
% \small
% \caption{Placement-Rail-Optimized}
% \label{alg:placement-rail-optimized}
% \SetAlgoNlRelativeSize{-1}
% \SetAlgoNlRelativeSize{1}
%  \KwIn{Deployment topology $G_{deploy}=<S_{deploy},E_{deploy}>$, Number of applied constraints $n_{constraints}$, Faulty node $F$, Sub-line length $l$, Number of node in one TP group $m$}
%  \KwOut{Placement scheme}
%  Initialize $placement\_scheme=\{\}$\;
% \For{$i$ in $1..n_{constraints}$}
% {
%      $S_{subline}=S_{deploy}.pop(l)$\;
%      $E_{subline}=\{(u,v)\mid u\in S_{subline} \text{ and } v\in S_{subline} \text{ and } (u,v)\in E_{subline}\}$\;
%      $F_{subline}=F\cap S_{subline}$\;
%      $placement\_scheme=placement\_scheme\cup \text{Orchestration-Ideal}(<S_{subline},E_{subline}>, F_{subline}, m)$\;
% }
%  $E_{res}=\{(u,v)\mid u \in S_{deploy} \text{ and } v \in S_{deploy} \text{ and } (u,v) \in E_{deploy}\}$\;
%  $F_{res}=F\cap S_{deploy}$\;
%  $placement\_scheme=placement\_scheme\cup \text{Orchestration-Ideal}(<S_{deploy},E_{res}>, F_{res},m)$\;
%  \KwRet{$placement\_scheme$}
% \end{algorithm}


% \begin{algorithm}[!h]
% \small
% \caption{Orchestration-Rail-Optimized}
% \label{alg:orchestration-rail-optimized}
% \SetAlgoNlRelativeSize{-1}
% \SetAlgoNlRelativeSize{1}
%  \KwIn{Node ordered set $S$ (from 1 to n in DCN Fabric), GPU ranks per node $r$, Number of rails $k$, Faulty set $F$, TP size $t$, Job scale $s$ (number of GPUs required for the job), Aggregation-Switches Domain size $d$, \docs{} directions $K$.}
%  \KwOut{Placement scheme that satisfies job scale and minimizes cross-rail traffic.}
%  Initialize $p=k/r$, $m=t/r$, $n=|S|$, $l=\lfloor \frac{d}{p}\rfloor$\;
%  Create graph $G_{deploy}=<S_{deploy},E_{deploy}>=\text{Deployment-Strategy}(S,K,p)$\;
%  Initialize $high=\lfloor\frac{nd}{p}\rfloor$\;
%  Initialize $low=0$\;
%  Initialize $placement\_scheme=\{\}$\;
% \While{ $low \leq$ high}
% {
%      $mid=\lfloor \frac{low+high}{2} \rfloor$\;
%      $placement\_scheme=\text{Placement-Rail-Optimized}(G_{deploy},mid,F,l,m)$\;
%     \eIf {$|placement\_scheme|\cdot m\cdot r\ge s$}
%     {
%          $low=mid+1$\;
%     }
%     {
%          $high=mid-1$\;
%     }
% }
    
% \eIf{$|placement\_scheme|\cdot m\cdot r\ge s$}
% {
%   \KwRet {$placement\_scheme$}
% }
% {
%     \KwRet {None}
% }
% \end{algorithm}
  

Fat-Tree topology is another common data center topology. A typical training strategy for this topology aims to maximize the bandwidth utilization under ToR (Top of Rack) Switches. Using Meta's two-stage clos topology\cite{sigcomm2024meta} as a reference, it can be observed that there is an attempt to run CP under ToR.

\para{Deployment Strategy:} Assuming there are $p$ nodes under each ToR, nodes with the same index under each ToR are deployed along the same parallel sub-line, and the $p$ sub-lines are connected end-to-end, as shown in \fig{fig:fat-tree-topo}. The training strategy involves running CP $p$ across the sub-lines and running TP within them.

\para{Orchestration Constraints. }To maximize the utilization of ToR bandwidth and minimize cross-ToR traffic, the fat-tree topology introduces two constraints:

\begin{packeditemize}
    \item {
        \textbf{Aggregation-Switches Domain Constraint: }The coverage domian of a group of Aggregation Switches is limited, meaning that TP groups spanning across Aggregation Switches domains would result in cross-rail traffic, which should be avoided as much as possible.
    }
    \item {
        \textbf{TP Group Alignment Constraint: } A CP Group consists of TP Groups across parallel sub-lines. To keep CP traffic within the ToR, the TP Groups must be aligned. If a node fails under one ToR, all nodes under that ToR are considered failed, expanding the failure radius by a factor of $p$. 
    }
\end{packeditemize}

\para{Binary-Search-Based Orchestration Algorithm.} Based on the constraints and deployment strategy, we develop a binary search orchestration algorithm (see \algref{alg:orchestration-fat-tree}) that adjusts the number of satisfied constraints. The binary search first relaxes the TP Group alignment constraints within the Aggregation-Switches Domain and then relaxes the TP Group crossing constraints between Aggregation-Switch domains (see \algref{alg:placement-fat-tree}). This process is monotonic.


% \begin{figure}[!h]
%     \centering
%     \includegraphics[width=\linewidth]{figs/design/Orchestration/meta-topo.drawio.pdf}
%     \caption{Orchestration example for Fat-Tree Topology under single Aggregation-Switches Domain with $p=2$. Green indicates active node, red indicates faulty node and yellow indicates idle nodes}
%     \label{fig:meta-topo}
% \end{figure}


The time complexity of \algref{alg:orchestration-ideal} is $O(|S|)$, and the complexity of \algref{alg:placement-fat-tree} is 

$$\sum_{i=1}^{n_{subline}} O(|S_{subline}|) = O(\sum_{i=1}^{n_{subline}} |S_{subline}|) = O(|S_{all}|) = O(n)$$  

Thus, the overall time complexity of \algref{alg:orchestration-fat-tree} is $O(n \log n)$.

\begin{algorithm}[!h]
\small
\caption{Placement-Fat-Tree}
\label{alg:placement-fat-tree}
\SetAlgoNlRelativeSize{-1}
\SetAlgoNlRelativeSize{1}
 \KwIn{$G_{deploy}=<S_{deploy},E_{deploy}>$, $n_{constraints}$, $F$, $l$, $m$, $n_{maxsubline}$, $d$, $p$}
 \KwOut{Placement scheme}
 Initialize $placement\_scheme=\{\}$\;
 Initialize $n_{align}=max(0,n_{constraints}-n_{maxsubline})$, $n_{subline}=min(n_{maxsubline},n_{constraints})$\;
 
\For{$i$ in $0..n_{align}-1$}
{
    \For{$j$ in $1..d$}
    {
        $sid=i*d+j$\;
        \If{$sid \in F$}
        {
            $F\cup \{\lfloor \frac{sid-1}{p}\rfloor\cdot p+1..(\lfloor \frac{sid-1}{p}\rfloor+1)\cdot p \}$\;
        }
    }
}
\For{$i$ in $1..n_{subline}$}
{
     $S_{subline}=S_{deploy}.pop(l)$\;
     $E_{subline}=\{(u,v)\mid u\in S_{subline} \text{ and } v\in S_{subline} \text{ and } (u,v)\in E_{subline}\}$\;
     $F_{subline}=F\cap S_{subline}$\;
     $placement\_scheme=placement\_scheme\cup \text{Orchestration-Ideal}(<S_{subline},E_{subline}>, F_{subline}, m)$\;
}
 $E_{res}=\{(u,v)\mid u \in S_{deploy} \text{ and } v \in S_{deploy} \text{ and } (u,v) \in E_{deploy}\}$\;
 $F_{res}=F\cap S_{deploy}$\;
 $placement\_scheme=placement\_scheme\cup \text{Orchestration-Ideal}(<S_{deploy},E_{res}>, F_{res},m)$\;
 \KwRet{$placement\_scheme$}
\end{algorithm}

\begin{algorithm}[!h]
\small
\caption{Orchestration-Fat-Tree}
\label{alg:orchestration-fat-tree}
\SetAlgoNlRelativeSize{-1}
\SetAlgoNlRelativeSize{1}
 \KwIn{$S$, $r$, $p$, $F$, $t$, $s$, $d$, $K$.}
 \KwOut{Placement scheme that satisfies job scale and minimizes cross-rail traffic.}
 Initialize $m=t/r$, $n=|S|$, $l=\lfloor\frac{d}{p}\rfloor$\, $n_{domain}=\lfloor\frac{n}{d}\rfloor$, $n_{maxsubline}=\lfloor\frac{nd}{p}\rfloor$\;
 Create graph $G_{deploy}=<S_{deploy},E_{deploy}>=\text{Deployment-Strategy}(S,K,p)$\;
 Initialize $high=n_{domain}+n_{maxsubline}$\;
 Initialize $low=0$\;
 Initialize $placement\_scheme=\{\}$\;
\While{ $low \leq$ high}
{
     $mid=\lfloor \frac{low+high}{2} \rfloor$\;
     $placement\_scheme=\text{Placement-Fat-Tree}(G_{deploy},mid,F,l,m,n_{maxsubline},d,p)$\;
    \eIf {$|placement\_scheme|\cdot m\cdot r\ge s$}
    {
         $low=mid+1$\;
    }
    {
         $high=mid-1$\;
    }
}
    
\eIf{$|placement\_scheme|\cdot m\cdot r\ge s$}
{
    \KwRet {$placement\_scheme$}
}
{
    \KwRet {None}
}
\end{algorithm}





\section{Additional Simulation Results for Fault Resilience}
\label{appendix:wasted-GPUs-ratio}
This section presents additional simulation results related to \S\ref{sec:simulation:fault}. \figref{fig:simulation:wasted-trace} shows the variation of the GPU waste ratio over time under the production fault trace. \figref{fig:simulation:waste-cdf:gr4:supple} presents the CDF data for the GPU waste ratio. \figref{fig:simulation:model:wasted-gr4} illustrates the waste GPU ratio for different HBD architectures under various node failure rates, including the results for TP-8 to TP-64. \figref{fig:simulation:breakdown-duration-supple} shows the proportion of job-fault waiting time relative to total time for different job scales. All the aforementioned experiments include results for TP-8, TP-16, TP-32, and TP-64 configurations.








\begin{figure*}[h!t]
    \centering
    \begin{subfigure}[b]{0.23\linewidth}
        \centering
        \includegraphics[width=\linewidth]{figs/evaluation/fault_trace_based/frag_trace_tp8_gr4.pdf}
        \caption{TP-8.}
        \label{fig:simulation:wasted-trace:tp8-4gpu}
    \end{subfigure}
    \hspace{2pt}
    \begin{subfigure}[b]{0.23\linewidth}
        \centering
        \includegraphics[width=\linewidth]{figs/evaluation/fault_trace_based/frag_trace_tp16_gr4.pdf}
        \caption{TP-16.}
        \label{fig:simulation:wasted-trace:tp16-4gpu}
    \end{subfigure}
    \hspace{2pt}
    \begin{subfigure}[b]{0.23\linewidth}
        \centering
        \includegraphics[width=\linewidth]{figs/evaluation/fault_trace_based/frag_trace_tp32_gr4.pdf}
        \caption{TP-32.}
        \label{fig:simulation:wasted-trace:tp32-4gpu}
    \end{subfigure}
    \hspace{2pt}
    \begin{subfigure}[b]{0.23\linewidth}
        \centering
        \includegraphics[width=\linewidth]{figs/evaluation/fault_trace_based/frag_trace_tp64_gr4.pdf}
        \caption{TP-64.}
        \label{fig:simulation:wasted-trace:tp64-4gpu}
    \end{subfigure}

    \vspace{-1ex}
    \caption{GPU waste ratio over production fault trace, 4 GPU node.}
    \label{fig:simulation:wasted-trace}
\end{figure*}


\begin{figure*}[h!t]
    \centering
    \begin{subfigure}[b]{0.23\linewidth}
        \centering
        \includegraphics[width=\linewidth]{figs/evaluation/fault_trace_based/cdf_trace_waste_tp8_gr4.pdf}
        \caption{TP-8.}
        \label{fig:simulation:waste-cdf:tp8-gr4}
    \end{subfigure}
    \hspace{2pt}
    \begin{subfigure}[b]{0.23\linewidth}
        \centering
        \includegraphics[width=\linewidth]{figs/evaluation/fault_trace_based/cdf_trace_waste_tp16_gr4.pdf}
        \caption{TP-16.}
        \label{fig:simulation:waste-cdf:tp16-gr4}
    \end{subfigure}
    \hspace{2pt}
    \begin{subfigure}[b]{0.23\linewidth}
        \centering
        \includegraphics[width=\linewidth]{figs/evaluation/fault_trace_based/cdf_trace_waste_tp32_gr4.pdf}
        \caption{TP-32.}
        \label{fig:simulation:waste-cdf:tp32-gr4}
    \end{subfigure}
    \hspace{2pt}
    \begin{subfigure}[b]{0.23\linewidth}
        \centering
        \includegraphics[width=\linewidth]{figs/evaluation/fault_trace_based/cdf_trace_waste_tp64_gr4.pdf}
        \caption{TP-64.}
        \label{fig:simulation:waste-cdf:tp64-gr4}
    \end{subfigure}
    \vspace{-1ex}
    \caption{CDF of GPU waste ratio over production fault trace, 4 GPU node.}
    \label{fig:simulation:waste-cdf:gr4:supple}
\end{figure*}


\begin{figure*}[h!t]
    \centering
    \begin{subfigure}[b]{0.23\linewidth}
        \centering
        \includegraphics[width=\linewidth]{figs/evaluation/fault_model_based/frag_ratio_tp8_gr4.pdf}
        \caption{TP-8.}
        \label{fig:simulation:model:wasted:tp8}
    \end{subfigure}
    \hspace{2pt}
    \begin{subfigure}[b]{0.23\linewidth}
        \centering
        \includegraphics[width=\linewidth]{figs/evaluation/fault_model_based/frag_ratio_tp16_gr4.pdf}
        \caption{TP-16.}
        \label{fig:simulation:model:wasted:tp16}
    \end{subfigure}
    \hspace{2pt}
    \begin{subfigure}[b]{0.23\linewidth}
        \centering
        \includegraphics[width=\linewidth]{figs/evaluation/fault_model_based/frag_ratio_tp32_gr4.pdf}
        \caption{TP-32.}
        \label{fig:simulation:model:wasted:tp32}
    \end{subfigure}
    \hspace{2pt}
    \begin{subfigure}[b]{0.23\linewidth}
        \centering
        \includegraphics[width=\linewidth]{figs/evaluation/fault_model_based/frag_ratio_tp64_gr4.pdf}
        \caption{TP-64.}
        \label{fig:simulation:model:wasted:tp64}
    \end{subfigure}
    \vspace{-1ex}
    \caption{GPU wastes ratio with different GPU fault ratio, 4-GPU node.}
    \label{fig:simulation:model:wasted-gr4}
\end{figure*}



\begin{figure*}[h!t]
    \centering
    \begin{subfigure}[b]{0.23\linewidth}
        \centering
        \includegraphics[width=\linewidth]{figs/evaluation/fault_trace_based/breakdown_ratio_tp8_gr4.pdf}
        \caption{TP-8.}
        \label{fig:simulation:breakdown-duration:tp8-4gpu}
    \end{subfigure}
    \hspace{2pt}
    \begin{subfigure}[b]{0.23\linewidth}
        \centering
        \includegraphics[width=\linewidth]{figs/evaluation/fault_trace_based/breakdown_ratio_tp16_gr4.pdf}
        \caption{TP-16.}
        \label{fig:simulation:breakdown-duration:tp16-4gpu}
    \end{subfigure}
    \hspace{2pt}
    \begin{subfigure}[b]{0.23\linewidth}
        \centering
        \includegraphics[width=\linewidth]{figs/evaluation/fault_trace_based/breakdown_ratio_tp32_gr4.pdf}
        \caption{TP-32.}
        \label{fig:simulation:breakdown-duration:tp32-4gpu}
    \end{subfigure}
    \hspace{2pt}
    \begin{subfigure}[b]{0.23\linewidth}
        \centering
        \includegraphics[width=\linewidth]{figs/evaluation/fault_trace_based/breakdown_ratio_tp64_gr4.pdf}
        \caption{TP-64.}
        \label{fig:simulation:breakdown-duration:tp64-4gpu}
    \end{subfigure}
    \vspace{-1ex}
    \caption{Job fault-waiting duration with different levels of job-scale, 4 GPU node}
    \label{fig:simulation:breakdown-duration-supple}
\end{figure*}





\vspace{-12em}
\section{Detailed Cost and power consumption Analysis}
\label{appendix:cost}
In this section, \tabref{tab:eval:components} provides a detailed description of the quantity, cost, bandwidth, and power consumption of the interconnect components in various network architectures, including Google TPUv4~\cite{isca2023tpu}, NVIDIA GB200 NVL series~\cite{nvl72}, Alibaba HPN\cite{sigcomm2024hpn}, and \sys{}.


\begin{table*}[h!t] \small
    \centering
    \begin{tabular}{lllll}
    \toprule
    
    \textbf{Component} & \textbf{Quantity} & \textbf{Unit Cost (\$)}  & \textbf{Unit Bandwidth (GBps)} & \textbf{Unit Power (W)} \\

    \midrule
    \multicolumn{5}{c}{\textbf{Google TPUv4\cite{isca2023tpu} with 4096 GPU, bandwidth 300GBps/GPU}} \\
    
    \midrule
    OCS\cite{sigcomm2023lightwave} & 48 & 80000 & 6400 & 108 \\
    DAC Cable\cite{400G_DAC} & 5120 & 63.60 & 50 & 0.1 \\
    Optical Module\cite{400G_OPTICAL_MODULE} & 6144 & 360 & 50 & 12  \\
    Fiber\cite{FIBER}& 6144 & 6.80 & 50 & 0 \\
    
    \midrule
    \multicolumn{5}{c}{\textbf{NVIDIA GB200 NVL-36\cite{SEMIANALYSIS_GB200} with 36 GPU, bandwidth 900GBps/GPU}}\\
    \midrule
    NVLink Switch\cite{SEMIANALYSIS_Power} & 9 & 28000 & 3600 & 275 \\
    DAC Cable\cite{200G_DAC} & 2592 & 35.60 & 25 & 0.1 \\
    
    \midrule
    \multicolumn{5}{c}{\textbf{NVIDIA GB200 NVL-72\cite{nvl72}\cite{SEMIANALYSIS_GB200} with 72 GPU, bandwidth 900GBps/GPU}}\\
    \midrule
    NVLink Switch\cite{SEMIANALYSIS_Power} & 18 & 28000 & 3600 & 275 \\
    DAC Cable\cite{200G_DAC} & 5184 & 35.60 & 25 & 0.1 \\
    \midrule
    \multicolumn{5}{c}{\textbf{NVIDIA GB200 NVL-36x2\cite{SEMIANALYSIS_GB200} with 72 GPU, bandwidth 900GBps/GPU}}\\
    \midrule
    NVLink Switch\cite{SEMIANALYSIS_Power} & 36 & 28000 & 3600 &  275\\
    DAC Cable\cite{200G_DAC} & 6480 & 35.60 & 25 & 0.1 \\
    ACC Cable\cite{SEMIANALYSIS_Power} & 162 & 320 & 200 & 2.5 \\

    \midrule
    \multicolumn{5}{c}{\textbf{NVIDIA GB200 NVL-576\cite{SEMIANALYSIS_GB200} with 576 GPU, bandwidth 900GBps/GPU}}\\
    \midrule
    NVLink Switch\cite{SEMIANALYSIS_Power} & 432 & 28000 & 3600 & 275 \\
    DAC Cable\cite{200G_DAC} & 41472 & 35.60 & 25 & 0.1 \\
    Optical Module\cite{OSFPXD} & 4608 & 850 & 200 & 25 \\
    Fiber\cite{FIBER} & 4608 & 6.80 & 200 & 0 \\

    \midrule
    \multicolumn{5}{c}{\textbf{Alibaba HPN\cite{sigcomm2024hpn} with 16320 GPU, bandwidth 50GBps/GPU}}\\
    \midrule
    EPS\cite{51.2T_EPS} & 360 & 14960 & 6400 & 3145 \\
    DAC Cable\cite{200G_DAC} & 32640 & 35.60 & 25 & 0.1\\
    Optical Module\cite{400G_OPTICAL_MODULE} & 28800 & 360 & 50 & 12 \\
    Fiber\cite{FIBER} & 14400 & 6.80 & 50 & 0 \\

    \midrule
    \multicolumn{5}{c}{\textbf{\SYS{}($K=2$)  with 4 GPU, bandwidth 800GBps/GPU}}\\
    \midrule
    DAC Cable\cite{1.6T_DAC}& 4 & 199.60 & 200 & 0.1\\
    dOCS Module & 16 & 600 & 100 & 12 \\
    Fiber\cite{FIBER} & 16 & 6.80 & 100 & 0 \\

    \midrule
    \multicolumn{5}{c}{\textbf{\SYS{}($K=3$)  with 4 GPU, bandwidth 800GBps/GPU}}\\
    \midrule
    DAC Cable\cite{1.6T_DAC} & 2 & 199.60 & 200 & 0.1\\
    dOCS Module & 24 & 600 & 100 & 12 \\
    Fiber\cite{FIBER} & 24 & 6.80 & 100 & 0 \\
    \bottomrule
    \end{tabular}
    \caption{Interconnect cost and power consumption of components used in different network architectures.}
    \label{tab:eval:components}
\end{table*}


\end{appendices}







\end{document}
