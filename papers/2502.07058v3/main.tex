% This must be in the first 5 lines to tell arXiv to use pdfLaTeX, which is strongly recommended.
\pdfoutput=1
% In particular, the hyperref package requires pdfLaTeX in order to break URLs across lines.

\documentclass[11pt]{article}

% Change "review" to "final" to generate the final (sometimes called camera-ready) version.
% Change to "preprint" to generate a non-anonymous version with page numbers.
\usepackage[final]{acl}

% Standard package includes
\usepackage{times}
\usepackage{latexsym}

% For proper rendering and hyphenation of words containing Latin characters (including in bib files)
\usepackage[T1]{fontenc}
% For Vietnamese characters
% \usepackage[T5]{fontenc}
% See https://www.latex-project.org/help/documentation/encguide.pdf for other character sets

% This assumes your files are encoded as UTF8
\usepackage[utf8]{inputenc}

% This is not strictly necessary, and may be commented out,
% but it will improve the layout of the manuscript,
% and will typically save some space.
\usepackage{microtype}

% This is also not strictly necessary, and may be commented out.
% However, it will improve the aesthetics of text in
% the typewriter font.
\usepackage{inconsolata}

%Including images in your LaTeX document requires adding
%additional package(s)
\usepackage{graphicx}

\author{
Zixin Tang\textsuperscript{1}~~~Chieh-Yang Huang\textsuperscript{2}~~~Tsung-Che Li\textsuperscript{3}~~~Ho Yin Sam Ng\textsuperscript{1}\\
\textbf{Hen-Hsen Huang\textsuperscript{3}~~~Ting-Hao `Kenneth' Huang\textsuperscript{1}}\\
\textsuperscript{1}College of Information Sciences and Technology, The Pennsylvania State University\\
\textsuperscript{2}MetaMetrics Inc.~~~\textsuperscript{3}Institute of Information Science, Academia Sinica\\
\textsuperscript{1}\texttt{\{zxtang,sam.ng,txh710\}@psu.edu}~~~\textsuperscript{2}\texttt{cyhuang@lexile.com} \\
\textsuperscript{3}\texttt{\{george,hhhuang\}@iis.sinica.edu.tw}
}

\def\method{\text MixMin~}
\def\methodnospace{\text MixMin}
\def\genmethod{$\mathbb{R}$\text Min~}
\def\genmethodnospace{ $\mathbb{R}$\text Min}


\begin{document}

\maketitle

%\kenneth{Editorial note 1: Please use these custom commands: \twChinese, \cnChinese, \twChineseFull, \cnChineseFull}

%\kenneth{Editorial note 2: For consistency, we use ``language variety'' instead of ``language variation''. Janet suggested we stick with ``language variety''.}


\begin{abstract}
Large language model (LLM)-based agents have shown promise in tackling complex tasks by interacting dynamically with the environment. 
Existing work primarily focuses on behavior cloning from expert demonstrations and preference learning through exploratory trajectory sampling. However, these methods often struggle in long-horizon tasks, where suboptimal actions accumulate step by step, causing agents to deviate from correct task trajectories.
To address this, we highlight the importance of \textit{timely calibration} and the need to automatically construct calibration trajectories for training agents. We propose \textbf{S}tep-Level \textbf{T}raj\textbf{e}ctory \textbf{Ca}libration (\textbf{\model}), a novel framework for LLM agent learning. 
Specifically, \model identifies suboptimal actions through a step-level reward comparison during exploration. It constructs calibrated trajectories using LLM-driven reflection, enabling agents to learn from improved decision-making processes. These calibrated trajectories, together with successful trajectory data, are utilized for reinforced training.
Extensive experiments demonstrate that \model significantly outperforms existing methods. Further analysis highlights that step-level calibration enables agents to complete tasks with greater robustness. 
Our code and data are available at \url{https://github.com/WangHanLinHenry/STeCa}.
\end{abstract}


%\kenneth{Again, we can use custom commands defined in micro.tex to add comments.}
%\zixin{Like this!}

%\kenneth{I think the high-level, central theme of our paper is this: How do you "fairly" compare a model's performance across language variations? An (unspoken) underlying argument we want to push back is this: "When you have big enough data---of popular, mainstream variation---and a big enough model, NLP tasks in less popular variations will be automatically solved altogether." In order to push back this argument, we want to construct a robust, fair, and measurable evaluation to pinpoint the performance gap.\\\\
%Although we might be able to learn some "methods" in machine translation papers on how they compared a model's performance across different languages, there's a fundamental difference between benchmarking across variations and benchmarking across languages: People just KNOW English and Spanish are different. Machine-translation people do not need to care as much (in some cases, they do care!) about having a very very "fair" comparison across different languages because... people just know different languages are different.}

\section{Introduction}
\section{Introduction}

Despite the remarkable capabilities of large language models (LLMs)~\cite{DBLP:conf/emnlp/QinZ0CYY23,DBLP:journals/corr/abs-2307-09288}, they often inevitably exhibit hallucinations due to incorrect or outdated knowledge embedded in their parameters~\cite{DBLP:journals/corr/abs-2309-01219, DBLP:journals/corr/abs-2302-12813, DBLP:journals/csur/JiLFYSXIBMF23}.
Given the significant time and expense required to retrain LLMs, there has been growing interest in \emph{model editing} (a.k.a., \emph{knowledge editing})~\cite{DBLP:conf/iclr/SinitsinPPPB20, DBLP:journals/corr/abs-2012-00363, DBLP:conf/acl/DaiDHSCW22, DBLP:conf/icml/MitchellLBMF22, DBLP:conf/nips/MengBAB22, DBLP:conf/iclr/MengSABB23, DBLP:conf/emnlp/YaoWT0LDC023, DBLP:conf/emnlp/ZhongWMPC23, DBLP:conf/icml/MaL0G24, DBLP:journals/corr/abs-2401-04700}, 
which aims to update the knowledge of LLMs cost-effectively.
Some existing methods of model editing achieve this by modifying model parameters, which can be generally divided into two categories~\cite{DBLP:journals/corr/abs-2308-07269, DBLP:conf/emnlp/YaoWT0LDC023}.
Specifically, one type is based on \emph{Meta-Learning}~\cite{DBLP:conf/emnlp/CaoAT21, DBLP:conf/acl/DaiDHSCW22}, while the other is based on \emph{Locate-then-Edit}~\cite{DBLP:conf/acl/DaiDHSCW22, DBLP:conf/nips/MengBAB22, DBLP:conf/iclr/MengSABB23}. This paper primarily focuses on the latter.

\begin{figure}[t]
  \centering
  \includegraphics[width=0.48\textwidth]{figures/demonstration.pdf}
  \vspace{-4mm}
  \caption{(a) Comparison of regular model editing and EAC. EAC compresses the editing information into the dimensions where the editing anchors are located. Here, we utilize the gradients generated during training and the magnitude of the updated knowledge vector to identify anchors. (b) Comparison of general downstream task performance before editing, after regular editing, and after constrained editing by EAC.}
  \vspace{-3mm}
  \label{demo}
\end{figure}

\emph{Sequential} model editing~\cite{DBLP:conf/emnlp/YaoWT0LDC023} can expedite the continual learning of LLMs where a series of consecutive edits are conducted.
This is very important in real-world scenarios because new knowledge continually appears, requiring the model to retain previous knowledge while conducting new edits. 
Some studies have experimentally revealed that in sequential editing, existing methods lead to a decrease in the general abilities of the model across downstream tasks~\cite{DBLP:journals/corr/abs-2401-04700, DBLP:conf/acl/GuptaRA24, DBLP:conf/acl/Yang0MLYC24, DBLP:conf/acl/HuC00024}. 
Besides, \citet{ma2024perturbation} have performed a theoretical analysis to elucidate the bottleneck of the general abilities during sequential editing.
However, previous work has not introduced an effective method that maintains editing performance while preserving general abilities in sequential editing.
This impacts model scalability and presents major challenges for continuous learning in LLMs.

In this paper, a statistical analysis is first conducted to help understand how the model is affected during sequential editing using two popular editing methods, including ROME~\cite{DBLP:conf/nips/MengBAB22} and MEMIT~\cite{DBLP:conf/iclr/MengSABB23}.
Matrix norms, particularly the L1 norm, have been shown to be effective indicators of matrix properties such as sparsity, stability, and conditioning, as evidenced by several theoretical works~\cite{kahan2013tutorial}. In our analysis of matrix norms, we observe significant deviations in the parameter matrix after sequential editing.
Besides, the semantic differences between the facts before and after editing are also visualized, and we find that the differences become larger as the deviation of the parameter matrix after editing increases.
Therefore, we assume that each edit during sequential editing not only updates the editing fact as expected but also unintentionally introduces non-trivial noise that can cause the edited model to deviate from its original semantics space.
Furthermore, the accumulation of non-trivial noise can amplify the negative impact on the general abilities of LLMs.

Inspired by these findings, a framework termed \textbf{E}diting \textbf{A}nchor \textbf{C}ompression (EAC) is proposed to constrain the deviation of the parameter matrix during sequential editing by reducing the norm of the update matrix at each step. 
As shown in Figure~\ref{demo}, EAC first selects a subset of dimension with a high product of gradient and magnitude values, namely editing anchors, that are considered crucial for encoding the new relation through a weighted gradient saliency map.
Retraining is then performed on the dimensions where these important editing anchors are located, effectively compressing the editing information.
By compressing information only in certain dimensions and leaving other dimensions unmodified, the deviation of the parameter matrix after editing is constrained. 
To further regulate changes in the L1 norm of the edited matrix to constrain the deviation, we incorporate a scored elastic net ~\cite{zou2005regularization} into the retraining process, optimizing the previously selected editing anchors.

To validate the effectiveness of the proposed EAC, experiments of applying EAC to \textbf{two popular editing methods} including ROME and MEMIT are conducted.
In addition, \textbf{three LLMs of varying sizes} including GPT2-XL~\cite{radford2019language}, LLaMA-3 (8B)~\cite{llama3} and LLaMA-2 (13B)~\cite{DBLP:journals/corr/abs-2307-09288} and \textbf{four representative tasks} including 
natural language inference~\cite{DBLP:conf/mlcw/DaganGM05}, 
summarization~\cite{gliwa-etal-2019-samsum},
open-domain question-answering~\cite{DBLP:journals/tacl/KwiatkowskiPRCP19},  
and sentiment analysis~\cite{DBLP:conf/emnlp/SocherPWCMNP13} are selected to extensively demonstrate the impact of model editing on the general abilities of LLMs. 
Experimental results demonstrate that in sequential editing, EAC can effectively preserve over 70\% of the general abilities of the model across downstream tasks and better retain the edited knowledge.

In summary, our contributions to this paper are three-fold:
(1) This paper statistically elucidates how deviations in the parameter matrix after editing are responsible for the decreased general abilities of the model across downstream tasks after sequential editing.
(2) A framework termed EAC is proposed, which ultimately aims to constrain the deviation of the parameter matrix after editing by compressing the editing information into editing anchors. 
(3) It is discovered that on models like GPT2-XL and LLaMA-3 (8B), EAC significantly preserves over 70\% of the general abilities across downstream tasks and retains the edited knowledge better.

\section{Related Work}


\section{Related Work}
Our work draws on and contributes to research in mobility aids and the built environment, online image-based survey for urban assessment, personalized routing applications and accessibility maps.

\subsection{Mobility Aids and the Built Environment}
People who use mobility aids (\textit{e.g.,} canes, walkers, mobility scooters, manual wheelchairs and motorized wheelchairs) face significant challenges navigating their communities.
Studies have repeatedly found that sidewalk conditions can significantly impede mobility among these users~\cite{bigonnesse_role_2018,fomiatti_experience_2014,f_bromley_city_2007,rosenberg_outdoor_2013, harris_physical_2015,korotchenko_power_2014}. 
In a review of the physical environment's role in mobility, \citet{bigonnesse_role_2018} summarized factors affecting mobility aid users, including uneven or narrow sidewalks (\textit{e.g.,}~\cite{fomiatti_experience_2014,f_bromley_city_2007}), rough pavements (\textit{e.g.,}~\cite{fomiatti_experience_2014,f_bromley_city_2007}), absent or poorly designed curb ramps (\textit{e.g.,}~\cite{rosenberg_outdoor_2013, f_bromley_city_2007, korotchenko_power_2014}), lack of crosswalks (\textit{e.g.,}~\cite{harris_physical_2015}), and various temporary obstacles (\textit{e.g.,}~\cite{harris_physical_2015}).

Though most research on mobility disability and the built environment has focused on wheelchair users~\cite{bigonnesse_role_2018}, mobility challenges are not experienced uniformly across different user populations~\cite{prescott_factors_2020, bigonnesse_role_2018}. 
For example, crutch users could overcome a specific physical barrier (such as two stairs down to a street), whereas motorized wheelchair users could not (without a ramp)~\cite{bigonnesse_role_2018}. 
Such variability demonstrates how person-environment interaction can differ based on mobility aids and environmental factors~\cite{sakakibara_rasch_2018,smith_review_2016}.
Further, mobility aids such as canes, crutches, or walkers are more commonly used than wheelchairs in the U.S.~\cite{taylor_americans_2014, firestine_travel_2024}: in 2022, approximately 4.7 million adults used a cane, crutches, or a walker, compared to 1.7 million who used a wheelchair~\cite{firestine_travel_2024}.
This underscores the importance of considering a diverse range of mobility aid users in urban accessibility research.
For example, \citet{prescott_factors_2020} explored the daily path areas of users of manual wheelchairs, motorized wheelchairs, scooters, walkers, canes, and crutches and found that the type of mobility device had a strong association with users' daily path area size.
Our study aims to further advance knowledge of how different mobility aid users perceive sidewalk barriers, with a more inclusive understanding of urban accessibility.

\begin{figure*}
    \centering
    \includegraphics[width=1\linewidth]{figures/figure-tutorial.png}
    \caption{Survey Part 2.1 showed all 52 images and asked participants to rate their passability based on their lived experience and use of their mobility aid. Above is the interactive tutorial we showed at the beginning of this part.}
    \Description{This figure shows a screenshot from the online survey. In survey part 2.1, participants were presented with 52 images and were asked to rate their passibility based on their lived experience and use of their mobility aid. The screenshot shows the interactive tutorial shown before this section.}
    \label{fig:survey-part2-instructions}
\end{figure*}

\subsection{Online Image-Based Survey for Urban Assessment}
Sidewalk barriers hinder individuals with mobility impairments not just by preventing particular travel paths but also by reducing confidence in self-navigating and decreasing one's willingness to travel to areas that might be physically challenging or unsafe~\cite{vasudevan_exploration_2016,clarke_mobility_2008}.
Prior work in this area traditionally uses three main study methods: in-person interviews (\textit{e.g}.~\cite{rosenberg_outdoor_2013,castrodale_mobilizing_2018}), GPS-based activity studies (\textit{e.g.,}~\cite{prescott_exploration_2021, prescott_factors_2020,rosenberg_outdoor_2013}), and online-questionnaires (\textit{e.g.,}~\cite{carlson_wheelchair_2002}). 
In-person interviews, while providing detailed and nuanced information, are limited by small sample sizes~\cite{rosenberg_outdoor_2013}. GPS-based activity studies involve tracking mobility aids user activity over a period of time, offering insights into movement patterns and activity space; however, these studies are constrained by geographical location~\cite{prescott_exploration_2021}. In contrast, online questionnaires can reach much larger populations and cover broader geographical regions, but they often yield high-level information that lacks the depth and nuance of the other approaches~\cite{carlson_wheelchair_2002}.
Our study aims to strike a balance between these approaches, capturing nuanced perspectives of mobility aid users about the built environment while maintaining a sufficiently large enough sample size for robust statistical analysis. 
Building on~\citet{bigonnesse_role_2018}'s work, we explore not only the types of factors considered to be barriers, but the \textit{intensity} of these barriers and their differential impacts.

Visual assessment of environmental features has long been employed by researchers across diverse fields, including human well-being~\cite{humpel_environmental_2002}, ecosystem sustainability~\cite{gobster_shared_2007}, and public policy~\cite{dobbie_public_2013}. 
These studies examine the relationship between images and the reactions they provoke in respondents or compare differences in reactions between groups.
Over the past decade, online visual preference surveys have gained popularity (\textit{e.g.,}~\cite{evans-cowley_streetseen_2014, salesses_collaborative_2013, goodspeed_research_2017}), where respondents are asked to make pairwise comparisons between randomly selected images.
Using this approach has two advantages: it adheres to the law of comparative judgment~\cite{thurstone_law_2017} by allowing respondents to make direct comparisons, and it prevents inter-rater inconsistency possible with scale ratings~\cite{goodspeed_research_2017}.
Additionally, online surveys generally offer advantages of increased sample sizes, reduced costs, and greater flexibility~\cite{wherrett_issues_1999}.
For people with disabilities, online surveys can be particularly beneficial. They help reach hidden or difficult-to-access populations~\cite{cook_challenges_2007,wright_researching_2005} and are believed to encourage more honest answers to sensitive questions~\cite{eckhardt_research_2007} by providing a higher level of anonymity and confidentiality~\cite{cook_challenges_2007, wright_researching_2005}.

\begin{figure*}
    \centering
    \includegraphics[width=1\linewidth]{figures/figure-comaprison-screenshot.png}
    \caption{In survey Part 2.2, participants were asked to perform a series of pairwise comparisons based on their 2.1 responses.}
    \Description{This figure shows a screenshot from the online survey. In Survey Part 2.2, participants were asked to perform a series of pairwise comparisons based on their 2.1 responses.}
    \label{fig:survey-part2b-pairwise}
\end{figure*}

\subsection{Personalized Routing Applications and Accessibility Maps}
Navigation challenges faced by mobility aid users can be mitigated through the provision of routes and directions that guide them to destinations safely, accurately, and efficiently~\cite{kasemsuppakorn_understanding_2015}. However, current commercial routing applications (\textit{e.g.}, \textit{Google Maps}) do not provide sufficient guidance for mobility aid users.
To address this gap, significant research has focused on routing systems for this population over the past two decades~\cite{barczyszyn_collaborative_2018, karimanzira_application_2006, matthews_modelling_2003, kasemsuppakorn_understanding_2015, volkel_routecheckr_2008, holone_people_2008, wheeler_personalized_2020, gharebaghi_user-specific_2021, ding_design_2007}.
One early, well-known prototype system is \textit{MAGUS}~\cite{matthews_modelling_2003}, which computes optimal routes for wheelchair users based on shortest distance, minimum barriers, fewest slopes, and limits on road crossings and challenging surfaces.
\textit{U-Access}~\cite{sobek_u-access_2006} provides the shortest route for people with three accessibility levels: unaided mobility, aided mobility (using crutch, cane, or walker), and wheelchair users.
However, U-Access only considers distance and ignores other
important factors for mobility aid users~\cite{barczyszyn_collaborative_2018}.
A series of projects by Kasemsuppakorn \textit{et al}.~\cite{kasemsuppakorn_personalised_2009, kasemsuppakorn_understanding_2015} attempted to create personalized routes for wheelchair users using fuzzy logic and \textit{Analytic Hierarchy Process} (AHP).

While influential, many personalized routing prototypes face limited adoption due to a scarcity of accessibility data for the built environment. 
Geo-crowdsourcing~\cite{karimi_personalized_2014}, a.k.a. volunteered geographic information (VGI)~\cite{goodchild_citizens_2007}, has emerged as an effective solution~\cite{karimi_personalized_2014, wheeler_personalized_2020}.
In this approach, users annotate maps with specific criteria or share personal experiences of locations, typically using web applications based on Google Maps or \textit{OpenStreetMap} (OSM)~\cite{karimi_personalized_2014}.
Examples include \textit{Wheelmap}~\cite{mobasheri_wheelmap_2017}, \textit{CAP4Access}~\cite{cap4access_cap4access_2014}, \textit{AXS Map}~\cite{axs_map_axs_2012}, and \textit{Project Sidewalk}~\cite{saha_project_2019}.
Recent research demonstrated the potential of using crowdsourced geodata for personalized routing~\cite{goldberg_interactive_2016, bolten_accessmap_2019,menkens_easywheel_2011, neis_measuring_2015}.
For example, \textit{EasyWheel}~\cite{menkens_easywheel_2011}, a mobile social navigation system based on OSM, provides wheelchair users with optimized routing, accessibility information for points of interest, and a social community for reporting barriers. 
\textit{AccessMap}~\cite{bolten_accessmap_2019} offers routing information tailored to users of canes, manual wheelchairs, or powered wheelchairs, calculating routes based on OSM data that includes slope, curbs, stairs and landmarks. 
Our work builds on the above by gathering perceptions of sidewalk obstacles from different mobility aid users to create generalizable profiles based on mobility aid type. We envision that these profiles can provide starting points in tools like Google Maps for personalized routing but can be further customized by the end user to specify additional needs (\textit{e.g.}, ability to navigate hills, \textit{etc.})

Beyond routing applications, our study data can contribute to modeling and visualizing higher-level abstractions of accessibility. 
Similar to \textit{AccessScore}~\cite{li_interactively_2018}, data from our survey can provide personalizable and interactive visual analytics of city-wide accessibility. By identifying both differences between mobility groups and common barriers within groups, we can develop analytical tools to prioritize barriers and assess the impact of their mitigation or removal, potentially benefiting the broadest range of mobility group users. Incorporating perceptions of passibility into urban planning processes provides a new dimension for urban planners' toolkits, which are often narrowly focused on compliance with ADA standards.






% Please add the following required packages to your document preamble:
% \usepackage{booktabs}
% \usepackage{multirow}
\begin{table*}[t]
\centering
\footnotesize
\begin{tabular}{@{}ccccr@{\kern-4mm}lccr@{\kern-4mm}lccr@{\kern-4mm}l@{}}
\toprule
 &  & \multicolumn{12}{c}{\textbf{Accuracy (Acc)$\uparrow$}} \\ \cmidrule{3-14}
 &  & \multicolumn{4}{c}{\textbf{structured}} & \multicolumn{4}{c}{\textbf{plain}} & \multicolumn{4}{c}{\textbf{shuffled}} \\ \cmidrule(lr){3-6} \cmidrule(lr){7-10} \cmidrule(lr){11-14}
\multirow{-4}{*}{\textbf{\begin{tabular}[c]{@{}c@{}}Text\\ Length\\ (\#Character)\end{tabular}}} & \multirow{-4}{*}{\textbf{Model}} & \textbf{tw} & \textbf{cn} & \multicolumn{2}{c}{\textbf{\begin{tabular}[c]{@{}c@{}}$\Delta$Acc\\ (cn-tw)\end{tabular}}} & \textbf{tw} & \textbf{cn} & \multicolumn{2}{c}{\textbf{\begin{tabular}[c]{@{}c@{}}$\Delta$Acc\\ (cn-tw)\end{tabular}}} & \textbf{tw} & \textbf{cn} & \multicolumn{2}{c}{\textbf{\begin{tabular}[c]{@{}c@{}}$\Delta$Acc\\ (cn-tw)\end{tabular}}} \\ \midrule
 & \textbf{GPT-4o} & 26.52 & 27.43 & \cellcolor[HTML]{FBE8E6}0.91\hspace{4mm} & \multicolumn{1}{l}{} & 19.16 & 20.78 & \cellcolor[HTML]{F7D5D2}1.62\hspace{4mm} & *** & 18.57 & 20.16 & \cellcolor[HTML]{F8D6D3}1.60\hspace{4mm} & *** \\
 & \textbf{Llama3 8b} & 27.40 & 26.39 & \cellcolor[HTML]{D9F1F3}-1.01\hspace{4mm} & \multicolumn{1}{l}{} & 19.21 & 19.08 & \cellcolor[HTML]{FAFDFD}-0.13\hspace{4mm} & \multicolumn{1}{l}{} & 17.43 & 17.71 & \cellcolor[HTML]{FEF8F8}0.28\hspace{4mm} & \multicolumn{1}{l}{} \\
 & \textbf{Llama3 70b} & 35.43 & 35.00 & \cellcolor[HTML]{EFF9FA}-0.43\hspace{4mm} & \multicolumn{1}{l}{} & 28.21 & 29.60 & \cellcolor[HTML]{F9DBD9}1.39\hspace{4mm} & ** & 27.54 & 29.51 & \cellcolor[HTML]{F6CCC8}1.97\hspace{4mm} & *** \\
 & \textbf{Llama3 405b} & 37.96 & 40.51 & \cellcolor[HTML]{F3BDB8}2.55\hspace{4mm} & *** & 27.42 & 30.12 & \cellcolor[HTML]{F2B9B4}2.70\hspace{4mm} & *** & 27.59 & 30.17 & \cellcolor[HTML]{F3BCB7}2.58\hspace{4mm} & *** \\
 & \textbf{Gemma2 9b} & 15.69 & 14.45 & \cellcolor[HTML]{D1EEF0}-1.24\hspace{4mm} & ** & 17.01 & 17.26 & \cellcolor[HTML]{FEF9F9}0.25\hspace{4mm} & \multicolumn{1}{l}{} & 15.81 & 16.35 & \cellcolor[HTML]{FDF1F0}0.54\hspace{4mm} & \multicolumn{1}{l}{} \\
\multirow{-6}{*}{\textbf{\begin{tabular}[c]{@{}c@{}}Short\\ (1-49)\end{tabular}}} & \textbf{Gemma2 27b} & 15.34 & 14.27 & \cellcolor[HTML]{D7F0F2}-1.07\hspace{4mm} & ** & 13.94 & 14.03 & \cellcolor[HTML]{FFFDFD}0.09\hspace{4mm} & \multicolumn{1}{l}{} & 13.91 & 14.29 & \cellcolor[HTML]{FEF6F5}0.37\hspace{4mm} & \multicolumn{1}{l}{} \\ \midrule
 & \textbf{GPT-4o} & 35.59 & 38.39 & \cellcolor[HTML]{F2B7B2}2.79\hspace{4mm} & *** & 28.15 & 33.16 & \cellcolor[HTML]{E67C73}5.01\hspace{4mm} & *** & 26.73 & 31.36 & \cellcolor[HTML]{E8867E}4.64\hspace{4mm} & *** \\
 & \textbf{Llama3 8b} & 25.31 & 27.01 & \cellcolor[HTML]{F7D3D0}1.70\hspace{4mm} & * & 19.53 & 21.24 & \cellcolor[HTML]{F7D3D0}1.71\hspace{4mm} & ** & 18.92 & 21.11 & \cellcolor[HTML]{F5C6C2}2.19\hspace{4mm} & *** \\
 & \textbf{Llama3 70b} & 34.66 & 38.24 & \cellcolor[HTML]{EEA29B}3.59\hspace{4mm} & *** & 35.02 & 37.45 & \cellcolor[HTML]{F3C0BC}2.43\hspace{4mm} & ** & 33.66 & 36.43 & \cellcolor[HTML]{F2B7B2}2.77\hspace{4mm} & *** \\
 & \textbf{Llama3 405b} & 37.20 & 40.52 & \cellcolor[HTML]{EFA9A3}3.31\hspace{4mm} & *** & 36.09 & 38.00 & \cellcolor[HTML]{F6CECA}1.91\hspace{4mm} & * & 34.38 & 36.60 & \cellcolor[HTML]{F4C5C1}2.22\hspace{4mm} & ** \\
 & \textbf{Gemma2 9b} & 14.84 & 15.66 & \cellcolor[HTML]{FBEAE9}0.82\hspace{4mm} & \multicolumn{1}{l}{} & 18.22 & 20.00 & \cellcolor[HTML]{F7D1CE}1.78\hspace{4mm} & ** & 16.59 & 17.98 & \cellcolor[HTML]{F9DBD9}1.38\hspace{4mm} & * \\
\multirow{-6}{*}{\textbf{\begin{tabular}[c]{@{}c@{}}Long\\ (50+)\end{tabular}}} & \textbf{Gemma2 27b} & 13.44 & 14.52 & \cellcolor[HTML]{FAE3E1}1.08\hspace{4mm} & \multicolumn{1}{l}{} & 15.48 & 16.99 & \cellcolor[HTML]{F8D8D5}1.51\hspace{4mm} & * & 15.16 & 17.16 & \cellcolor[HTML]{F6CBC8}2.00\hspace{4mm} & *** \\ \midrule
 & \textbf{GPT-4o} & 29.61 & 31.16 & \cellcolor[HTML]{F8D7D4}1.55\hspace{4mm} & *** & 22.22 & 24.99 & \cellcolor[HTML]{F2B7B2}2.78\hspace{4mm} & *** & 21.35 & 23.98 & \cellcolor[HTML]{F2BBB6}2.63\hspace{4mm} & *** \\
 & \textbf{Llama3 8b} & 26.69 & 26.61 & \cellcolor[HTML]{FCFDFE}-0.08\hspace{4mm} & \multicolumn{1}{l}{} & 19.32 & 19.82 & \cellcolor[HTML]{FDF2F2}0.50\hspace{4mm} & \multicolumn{1}{l}{} & 17.94 & 18.88 & \cellcolor[HTML]{FBE7E5}0.94\hspace{4mm} & * \\
 & \textbf{Llama3 70b} & 35.16 & 36.10 & \cellcolor[HTML]{FBE7E5}0.94\hspace{4mm} & * & 30.53 & 32.27 & \cellcolor[HTML]{F7D2CF}1.75\hspace{4mm} & *** & 29.62 & 31.87 & \cellcolor[HTML]{F4C5C1}2.24\hspace{4mm} & *** \\
 & \textbf{Llama3 405b} & 37.70 & 40.51 & \cellcolor[HTML]{F1B6B1}2.81\hspace{4mm} & *** & 30.39 & 32.82 & \cellcolor[HTML]{F3C0BC}2.43\hspace{4mm} & *** & 29.92 & 32.38 & \cellcolor[HTML]{F3BFBB}2.46\hspace{4mm} & *** \\
 & \textbf{Gemma2 9b} & 15.40 & 14.86 & \cellcolor[HTML]{EBF7F8}-0.54\hspace{4mm} & \multicolumn{1}{l}{} & 17.42 & 18.19 & \cellcolor[HTML]{FCEBEA}0.77\hspace{4mm} & * & 16.07 & 16.90 & \cellcolor[HTML]{FBEAE8}0.83\hspace{4mm} & * \\
\multirow{-6}{*}{\textbf{Overall}} & \textbf{Gemma2 27b} & 14.69 & 14.35 & \cellcolor[HTML]{F2FAFB}-0.34\hspace{4mm} & \multicolumn{1}{l}{} & 14.47 & 15.04 & \cellcolor[HTML]{FDF1F0}0.57\hspace{4mm} & \multicolumn{1}{l}{} & 14.34 & 15.27 & \cellcolor[HTML]{FBE7E6}0.93\hspace{4mm} & ** \\ \bottomrule
\end{tabular}
\vspace{-.5pc}
\caption{Accuracy (Acc~$\uparrow$) by length for GPT-4o, Llama3 (8b, 70b, 405b), and Gemma2 (9b, 27b) models. Red (green) indicates better (worse) performance in CN, with darker shades representing larger gaps. 
(Statistical group differences are indicated as ${^{*}}$  (p<.05), ${^{**}}$ (p<.01), and ${^{***}}$ (p<.001) regarding the model performance.)}
\vspace{-1pc}
\label{tab:new-acc-results}
\end{table*}

%\kenneth{TODO CY: Figure out the color code}

% \begin{tabular}{@{}llccccccccc@{}}
% \toprule
% \multirow{3}{*}{\textbf{\begin{tabular}[c]{@{}l@{}}Text\\ Length\\ (\#word)\end{tabular}}} & \multirow{3}{*}{\textbf{Model}} & \multicolumn{9}{c}{\textbf{Accuracy (Acc)~$\uparrow$}} \\ \cmidrule(l){3-11} 
%  &  & \multicolumn{3}{c}{\textbf{structured}} & \multicolumn{3}{c}{\textbf{plain}} & \multicolumn{3}{c}{\textbf{shuffled}} \\ \cmidrule(l){3-11} 
%  &  & \textbf{tw} & \textbf{cn} & \textbf{\begin{tabular}[c]{@{}c@{}}$\Delta$Acc\\ (cn-tw)\end{tabular}} & \textbf{tw} & \textbf{cn} & \textbf{\begin{tabular}[c]{@{}c@{}}$\Delta$Acc\\ (cn-tw)\end{tabular}} & \textbf{tw} & \textbf{cn} & \textbf{\begin{tabular}[c]{@{}c@{}}$\Delta$Acc\\ (cn-tw)\end{tabular}} \\ \midrule
% \multirow{6}{*}{\textbf{\begin{tabular}[c]{@{}l@{}}Short\\ (1-49)\end{tabular}}} & \textbf{GPT-4o} & 26.52 & 27.43 & 0.91 & 19.16 & 20.78 & 1.62*** & 18.57 & 20.16 & 1.60*** \\
%  & \textbf{Llama3 8b} & 27.40 & 26.39 & -1.01 & 19.21 & 19.08 & -0.13 & 17.43 & 17.71 & 0.28 \\
%  & \textbf{Llama3 70b} & 35.43 & 35.00 & -0.43 & 28.21 & 29.60 & 1.39** & 27.54 & 29.51 & 1.97*** \\
%  & \textbf{Llama3 405b} & 37.96 & 40.51 & 2.55*** & 27.42 & 30.12 & 2.70*** & 27.59 & 30.17 & 2.58*** \\
%  & \textbf{Gemma2 9b} & 15.69 & 14.45 & -1.24** & 17.01 & 17.26 & 0.25 & 15.81 & 16.35 & 0.54 \\
%  & \textbf{Gemma2 27b} & 15.34 & 14.27 & -1.07** & 13.94 & 14.03 & 0.09 & 13.91 & 14.29 & 0.37 \\ \midrule
% \multirow{6}{*}{\textbf{\begin{tabular}[c]{@{}l@{}}Long\\ (50+)\end{tabular}}} & \textbf{GPT-4o} & 35.59 & 38.39 & 2.79*** & 28.15 & 33.16 & 5.01*** & 26.73 & 31.36 & 4.64*** \\
%  & \textbf{Llama3 8b} & 25.31 & 27.01 & 1.70* & 19.53 & 21.24 & 1.71** & 18.92 & 21.11 & 2.19*** \\
%  & \textbf{Llama3 70b} & 34.66 & 38.24 & 3.59*** & 35.02 & 37.45 & 2.43** & 33.66 & 36.43 & 2.77*** \\
%  & \textbf{Llama3 405b} & 37.20 & 40.52 & 3.31*** & 36.09 & 38.00 & 1.91* & 34.38 & 36.60 & 2.22** \\
%  & \textbf{Gemma2 9b} & 14.84 & 15.66 & 0.82 & 18.22 & 20.00 & 1.78** & 16.59 & 17.98 & 1.38* \\
%  & \textbf{Gemma2 27b} & 13.44 & 14.52 & 1.08 & 15.48 & 16.99 & 1.51* & 15.16 & 17.16 & 2.00*** \\ \midrule
% \multirow{6}{*}{\textbf{Overall}} & \textbf{GPT-4o} & 29.61 & 31.16 & 1.55*** & 22.22 & 24.99 & 2.78*** & 21.35 & 23.98 & 2.63*** \\
%  & \textbf{Llama3 8b} & 26.69 & 26.61 & -0.08 & 19.32 & 19.82 & 0.50 & 17.94 & 18.88 & 0.94* \\
%  & \textbf{Llama3 70b} & 35.16 & 36.10 & 0.94* & 30.53 & 32.27 & 1.75*** & 29.62 & 31.87 & 2.24*** \\
%  & \textbf{Llama3 405b} & 37.70 & 40.51 & 2.81*** & 30.39 & 32.82 & 2.43*** & 29.92 & 32.38 & 2.46*** \\
%  & \textbf{Gemma2 9b} & 15.40 & 14.86 & -0.54 & 17.42 & 18.19 & 0.77* & 16.07 & 16.90 & 0.83* \\
%  & \textbf{Gemma2 27b} & 14.69 & 14.35 & -0.34 & 14.47 & 15.04 & 0.57 & 14.34 & 15.27 & 0.93** \\ \bottomrule
% \end{tabular}
% Please add the following required packages to your document preamble:
% \usepackage{booktabs}
% \usepackage{multirow}
\begin{table*}[t]
\centering
\footnotesize
\begin{tabular}{c@{\kern2mm}c@{\kern2mm}ccr@{\kern-4mm}lccr@{\kern-4mm}lccr@{\kern-4mm}l}
\toprule
 &  & \multicolumn{12}{c}{\textbf{Mean Squared Error (MSE)~$\downarrow$}} \\ \cmidrule(l){3-14} 
 &  & \multicolumn{4}{c}{\textbf{structured}} & \multicolumn{4}{c}{\textbf{plain}} & \multicolumn{4}{c}{\textbf{shuffled}} \\ \cmidrule(lr){3-6} \cmidrule(lr){7-10} \cmidrule(lr){11-14}
\multirow{-4}{*}{\textbf{\begin{tabular}[c]{@{}c@{}}Text\\ Length\\ (\#Character)\end{tabular}}} & \multirow{-4}{*}{\textbf{Model}} & \textbf{tw} & \textbf{cn} & \textbf{\begin{tabular}[c]{@{}c@{}}$\Delta$MSE\\ (cn-tw)\end{tabular}} & \multicolumn{1}{l}{} & \textbf{tw} & \textbf{cn} & \textbf{\begin{tabular}[c]{@{}c@{}}$\Delta$MSE\\ (cn-tw)\end{tabular}} & \multicolumn{1}{l}{} & \textbf{tw} & \textbf{cn} & \textbf{\begin{tabular}[c]{@{}c@{}}$\Delta$MSE\\ (cn-tw)\end{tabular}} & \multicolumn{1}{l}{} \\ \midrule
 & \textbf{GPT-4o} & 3.563 & 3.769 & \cellcolor[HTML]{DBF3F4}0.206\hspace{4mm} & *** & 4.091 & 3.385 & \cellcolor[HTML]{EEA7A1}-0.706\hspace{4mm} & *** & 4.347 & 3.561 & \cellcolor[HTML]{EC9D96}-0.786\hspace{4mm} & *** \\
 & \textbf{Llama3 8b} & 2.187 & 2.268 & \cellcolor[HTML]{F1FAFB}0.082\hspace{4mm} & \multicolumn{1}{l}{} & 2.999 & 2.801 & \cellcolor[HTML]{FAE6E4}-0.199\hspace{4mm} & *** & 3.377 & 3.016 & \cellcolor[HTML]{F6D2CF}-0.361\hspace{4mm} & *** \\
 & \textbf{Llama3 70b} & 1.732 & 1.626 & \cellcolor[HTML]{FCF1F0}-0.107\hspace{4mm} & ** & 2.977 & 2.534 & \cellcolor[HTML]{F4C8C4}-0.443\hspace{4mm} & *** & 3.006 & 2.605 & \cellcolor[HTML]{F5CDC9}-0.401\hspace{4mm} & *** \\
 & \textbf{Llama3 405b} & 2.782 & 2.635 & \cellcolor[HTML]{FBECEB}-0.147\hspace{4mm} & \multicolumn{1}{l}{} & 4.624 & 3.685 & \cellcolor[HTML]{E88A82}-0.939\hspace{4mm} & *** & 4.620 & 3.740 & \cellcolor[HTML]{EA918A}-0.880\hspace{4mm} & *** \\
 & \textbf{Gemma2 9b} & 3.026 & 3.164 & \cellcolor[HTML]{E7F7F8}0.138\hspace{4mm} & * & 4.483 & 3.828 & \cellcolor[HTML]{EFADA8}-0.655\hspace{4mm} & *** & 4.928 & 4.131 & \cellcolor[HTML]{EC9C95}-0.797\hspace{4mm} & *** \\
\multirow{-6}{*}{\textbf{\begin{tabular}[c]{@{}c@{}}Short\\ (1-49)\end{tabular}}} & \textbf{Gemma2 27b} & 2.945 & 3.028 & \cellcolor[HTML]{F1FAFB}0.083\hspace{4mm} & \multicolumn{1}{l}{} & 4.888 & 4.191 & \cellcolor[HTML]{EEA8A2}-0.697\hspace{4mm} & *** & 4.944 & 4.250 & \cellcolor[HTML]{EEA9A3}-0.693\hspace{4mm} & *** \\ \midrule
 & \textbf{GPT-4o} & 1.846 & 1.577 & \cellcolor[HTML]{F8DDDB}-0.269\hspace{4mm} & *** & 1.834 & 1.57 & \cellcolor[HTML]{F8DEDC}-0.264\hspace{4mm} & *** & 2.070 & 1.743 & \cellcolor[HTML]{F7D6D3}-0.327\hspace{4mm} & *** \\
 & \textbf{Llama3 8b} & 1.674 & 1.548 & \cellcolor[HTML]{FBEFEE}-0.127\hspace{4mm} & *** & 2.046 & 1.895 & \cellcolor[HTML]{FBECEA}-0.152\hspace{4mm} & *** & 2.127 & 1.906 & \cellcolor[HTML]{F9E3E1}-0.220\hspace{4mm} & *** \\
 & \textbf{Llama3 70b} & 1.473 & 1.302 & \cellcolor[HTML]{FAE9E8}-0.171\hspace{4mm} & *** & 1.534 & 1.406 & \cellcolor[HTML]{FBEFEE}-0.128\hspace{4mm} & ** & 1.671 & 1.495 & \cellcolor[HTML]{FAE9E7}-0.176\hspace{4mm} & *** \\
 & \textbf{Llama3 405b} & 1.910 & 1.674 & \cellcolor[HTML]{F9E1DF}-0.236\hspace{4mm} & *** & 1.909 & 1.766 & \cellcolor[HTML]{FBEDEC}-0.143\hspace{4mm} & * & 2.085 & 1.892 & \cellcolor[HTML]{FAE6E5}-0.194\hspace{4mm} & ** \\
 & \textbf{Gemma2 9b} & 2.479 & 2.337 & \cellcolor[HTML]{FBEDEC}-0.142\hspace{4mm} & ** & 2.199 & 2.024 & \cellcolor[HTML]{FAE9E7}-0.175\hspace{4mm} & *** & 2.511 & 2.294 & \cellcolor[HTML]{F9E4E2}-0.217\hspace{4mm} & *** \\
\multirow{-6}{*}{\textbf{\begin{tabular}[c]{@{}c@{}}Long\\ (50+)\end{tabular}}} & \textbf{Gemma2 27b} & 2.703 & 2.519 & \cellcolor[HTML]{FAE8E6}-0.184\hspace{4mm} & *** & 2.680 & 2.500 & \cellcolor[HTML]{FAE8E7}-0.180\hspace{4mm} & *** & 2.649 & 2.496 & \cellcolor[HTML]{FBECEA}-0.153\hspace{4mm} & ** \\ \midrule
 & \textbf{GPT-4o} & 2.978 & 3.022 & \cellcolor[HTML]{F8FDFD}0.044\hspace{4mm} & \multicolumn{1}{l}{} & 3.323 & 2.767 & \cellcolor[HTML]{F1BAB5}-0.555\hspace{4mm} & *** & 3.571 & 2.942 & \cellcolor[HTML]{F0B0AB}-0.630\hspace{4mm} & *** \\
 & \textbf{Llama3 8b} & 2.011 & 2.021 & \cellcolor[HTML]{FEFFFF}0.010\hspace{4mm} & \multicolumn{1}{l}{} & 2.672 & 2.490 & \cellcolor[HTML]{FAE8E6}-0.182\hspace{4mm} & *** & 2.948 & 2.635 & \cellcolor[HTML]{F7D8D5}-0.313\hspace{4mm} & *** \\
 & \textbf{Llama3 70b} & 1.644 & 1.515 & \cellcolor[HTML]{FBEEED}-0.129\hspace{4mm} & *** & 2.486 & 2.150 & \cellcolor[HTML]{F7D5D2}-0.335\hspace{4mm} & *** & 2.551 & 2.227 & \cellcolor[HTML]{F7D6D4}-0.324\hspace{4mm} & *** \\
 & \textbf{Llama3 405b} & 2.483 & 2.306 & \cellcolor[HTML]{FAE9E7}-0.177\hspace{4mm} & *** & 3.695 & 3.028 & \cellcolor[HTML]{EFACA6}-0.667\hspace{4mm} & *** & 3.752 & 3.107 & \cellcolor[HTML]{EFAEA9}-0.645\hspace{4mm} & *** \\
 & \textbf{Gemma2 9b} & 2.840 & 2.882 & \cellcolor[HTML]{F8FDFD}0.043\hspace{4mm} & \multicolumn{1}{l}{} & 3.705 & 3.213 & \cellcolor[HTML]{F3C2BD}-0.491\hspace{4mm} & *** & 4.105 & 3.505 & \cellcolor[HTML]{F0B4AF}-0.600\hspace{4mm} & *** \\
\multirow{-6}{*}{\textbf{Overall}} & \textbf{Gemma2 27b} & 2.863 & 2.855 & \cellcolor[HTML]{FEFEFD}-0.008\hspace{4mm} & \multicolumn{1}{l}{} & 4.136 & 3.615 & \cellcolor[HTML]{F2BEB9}-0.521\hspace{4mm} & *** & 4.162 & 3.653 & \cellcolor[HTML]{F2BFBB}-0.509\hspace{4mm} & *** \\ \bottomrule
\end{tabular}
\vspace{-.5pc}
\caption{Mean squared error (MSE~$\downarrow$) by length for GPT-4o, Llama3 (8b, 70b, 405b), and Gemma2 (9b, 27b) models. Statistical significance notations and color coding follow the same conventions as in Table~\ref{tab:new-mse-results}.}
\vspace{-1pc}
\label{tab:new-mse-results}
\end{table*}
%(Statistical group differences are indicated as ${^{*}}$  (p<.05), ${^{**}}$ (p<.01), and ${^{***}}$ (p<.001) regarding the model performance.)
%\kenneth{TODO CY: Figure out the color code}

\begin{comment}
\begin{tabular}{@{}llccccccccc@{}}
\toprule
\multirow{3}{*}{\textbf{\begin{tabular}[c]{@{}l@{}}Text\\ Length\\ (\#word)\end{tabular}}} & \multirow{3}{*}{\textbf{Model}} & \multicolumn{9}{c}{\textbf{Mean Squared Error (MSE)~$\downarrow$}} \\ \cmidrule(l){3-11} 
 &  & \multicolumn{3}{c}{\textbf{structured}} & \multicolumn{3}{c}{\textbf{plain}} & \multicolumn{3}{c}{\textbf{shuffled}} \\ \cmidrule(l){3-11} 
 &  & \textbf{tw} & \textbf{cn} & \textbf{\begin{tabular}[c]{@{}c@{}}$\Delta$MSE\\ (cn-tw)\end{tabular}} & \textbf{tw} & \textbf{cn} & \textbf{\begin{tabular}[c]{@{}c@{}}$\Delta$MSE\\ (cn-tw)\end{tabular}} & \textbf{tw} & \textbf{cn} & \textbf{\begin{tabular}[c]{@{}c@{}}$\Delta$MSE\\ (cn-tw)\end{tabular}} \\ \midrule
\multirow{6}{*}{\textbf{\begin{tabular}[c]{@{}l@{}}Short\\ (1-49)\end{tabular}}} & \textbf{GPT-4o} & 3.563 & 3.769 & 0.206*** & 4.091 & 3.385 & -0.706*** & 4.347 & 3.561 & -0.786*** \\
 & \textbf{Llama3 8b} & 2.187 & 2.268 & 0.082 & 2.999 & 2.801 & -0.199*** & 3.377 & 3.016 & -0.361*** \\
 & \textbf{Llama3 70b} & 1.732 & 1.626 & -0.107** & 2.977 & 2.534 & -0.443*** & 3.006 & 2.605 & -0.401*** \\
 & \textbf{Llama3 405b} & 2.782 & 2.635 & -0.147 & 4.624 & 3.685 & -0.939*** & 4.620 & 3.740 & -0.880*** \\
 & \textbf{Gemma2 9b} & 3.026 & 3.164 & 0.138* & 4.483 & 3.828 & -0.655*** & 4.928 & 4.131 & -0.797*** \\
 & \textbf{Gemma2 27b} & 2.945 & 3.028 & 0.083 & 4.888 & 4.191 & -0.697*** & 4.944 & 4.250 & -0.693*** \\ \midrule
\multirow{6}{*}{\textbf{\begin{tabular}[c]{@{}l@{}}Long\\ (50+)\end{tabular}}} & \textbf{GPT-4o} & 1.846 & 1.577 & -0.269*** & 1.834 & 1.570 & -0.264*** & 2.070 & 1.743 & -0.327*** \\
 & \textbf{Llama3 8b} & 1.674 & 1.548 & -0.127*** & 2.046 & 1.895 & -0.152*** & 2.127 & 1.906 & -0.220*** \\
 & \textbf{Llama3 70b} & 1.473 & 1.302 & -0.171*** & 1.534 & 1.406 & -0.128** & 1.671 & 1.495 & -0.176*** \\
 & \textbf{Llama3 405b} & 1.910 & 1.674 & -0.236*** & 1.909 & 1.766 & -0.143* & 2.085 & 1.892 & -0.194** \\
 & \textbf{Gemma2 9b} & 2.479 & 2.337 & -0.142** & 2.199 & 2.024 & -0.175*** & 2.511 & 2.294 & -0.217*** \\
 & \textbf{Gemma2 27b} & 2.703 & 2.519 & -0.184*** & 2.680 & 2.500 & -0.180*** & 2.649 & 2.496 & -0.153** \\ \midrule
\multirow{6}{*}{\textbf{Overall}} & \textbf{GPT-4o} & 2.978 & 3.022 & 0.044 & 3.323 & 2.767 & -0.555*** & 3.571 & 2.942 & -0.630*** \\
 & \textbf{Llama3 8b} & 2.011 & 2.021 & 0.010 & 2.672 & 2.490 & -0.182*** & 2.948 & 2.635 & -0.313*** \\
 & \textbf{Llama3 70b} & 1.644 & 1.515 & -0.129*** & 2.486 & 2.150 & -0.335*** & 2.551 & 2.227 & -0.324*** \\
 & \textbf{Llama3 405b} & 2.483 & 2.306 & -0.177*** & 3.695 & 3.028 & -0.667*** & 3.752 & 3.107 & -0.645*** \\
 & \textbf{Gemma2 9b} & 2.840 & 2.882 & 0.043 & 3.705 & 3.213 & -0.491*** & 4.105 & 3.505 & -0.600*** \\
 & \textbf{Gemma2 27b} & 2.863 & 2.855 & -0.008 & 4.136 & 3.615 & -0.521*** & 4.162 & 3.653 & -0.509*** \\ \bottomrule
\end{tabular}
\end{comment}


%
\sisetup{print-zero-integer=false}

\begin{table*}[t]
\centering
\small
\addtolength{\tabcolsep}{-1.2mm}
\begin{tabular}{
    @{}ll
    S[table-format = .3, table-space-text-post = $^{***}$] S[table-format = 0.3]
    S[table-format = .3, table-space-text-post = $^{***}$] S[table-format = 0.3]
    S[table-format = .3, table-space-text-post = $^{***}$] S[table-format = 0.3]
    S[table-format = 1.3, table-space-text-post = $^{***}$] S[table-format = 1.3]
    S[table-format = 1.3, table-space-text-post = $^{***}$] S[table-format = 1.3]
    S[table-format = 1.3, table-space-text-post = $^{***}$] S[table-format = 1.3]@{}
    % ccccccccccc@{}
}
% S[table-format = 2.3, table-space-text-post = $^{***}$]
\toprule
\multirow{3}{*}{\textbf{\begin{tabular}[c]{@{}l@{}}Text\\ Length\\ (\#word)\end{tabular}}} & \multirow{3}{*}{\textbf{Model}} & \multicolumn{6}{c}{\textbf{ACC} $\uparrow$} & \multicolumn{6}{c}{\textbf{MSE} $\downarrow$} \\ \cmidrule(lr){3-8} \cmidrule(lr){9-14}
 &  & \multicolumn{2}{c}{\textbf{structured}} & \multicolumn{2}{c}{\textbf{plain}} & \multicolumn{2}{c}{\textbf{shuffled}} & \multicolumn{2}{c}{\textbf{structured}} & \multicolumn{2}{c}{\textbf{plain}} & \multicolumn{2}{c}{\textbf{shuffled}} \\ \cmidrule(lr){3-4} \cmidrule(lr){5-6} \cmidrule(lr){7-8} \cmidrule(lr){9-10} \cmidrule(lr){11-12} \cmidrule(lr){13-14}
 &  & \textbf{tw} & \textbf{cn} & \textbf{tw} & \textbf{cn} & \textbf{tw} & \textbf{cn} & \textbf{tw} & \textbf{cn} & \textbf{tw} & \textbf{cn} & \textbf{tw} & \textbf{cn} \\ \midrule

 
\multirow{6}{*}{\textbf{\begin{tabular}[c]{@{}l@{}}Short\\ (1-49)\end{tabular}}} 
 & \textbf{GPT-4o} & \cellcolor{lightred}.265 & .274 & \cellcolor{lightred}.192$^{***}$ & .208 & \cellcolor{lightred}.186$^{***}$ & .202 & 3.563$^{*}$ & \cellcolor{lightred}3.769 & \cellcolor{lightred}4.092$^{***}$ & 3.385 & \cellcolor{lightred}4.347$^{***}$ & 3.561 \\
 % & \textbf{GPT-4o +T} & .291 & \cellcolor{lightred}.286 & \cellcolor{lightred}.213 & .217 & \cellcolor{lightred}.206 & .214 & \cellcolor{lightred}2.280 & 2.252 & \cellcolor{lightred}3.035$^{*}$ & 2.618 & \cellcolor{lightred}3.130$^{*}$ & 2.749 \\
 & \textbf{Llama3 8b} & .234$^{***}$ &\cellcolor{lightred}.216 & \cellcolor{lightred}.176 & .184 & \cellcolor{lightred}.162 & .169 & 3.402$^{***}$ & \cellcolor{lightred}3.931 & \cellcolor{lightred}3.059$^{***}$ & 2.851 & \cellcolor{lightred}3.220$^{**}$ & 3.071 \\
 & \textbf{Llama3 70b} & \cellcolor{lightred}.392 & .399 & \cellcolor{lightred}.297$^{**}$ & .310 & \cellcolor{lightred}.288$^{***}$ & .310 & \cellcolor{lightred}1.785$^{**}$ & 1.661 & \cellcolor{lightred}3.465$^{***}$ & 3.005 & \cellcolor{lightred}3.427$^{***}$ & 2.941 \\
 & \textbf{Llama3 405b} & \cellcolor{lightred}.386$^{***}$ & .412 & \cellcolor{lightred}.283$^{***}$ & .315 & \cellcolor{lightred}.290$^{***}$ & .322 & \cellcolor{lightred}2.605$^{**}$ & 2.430 & \cellcolor{lightred}4.365$^{***}$ & 3.598 & \cellcolor{lightred}4.355$^{***}$ & 3.612 \\
 & \textbf{Gemma2 9b} & .150$^{**}$ & \cellcolor{lightred}.138 & \cellcolor{lightred}.182$^{*}$ & .193 & \cellcolor{lightred}.171$^{**}$ & .184 & 2.987$^{***}$ & \cellcolor{lightred}3.168 & \cellcolor{lightred}4.089$^{***}$ & 3.367 & \cellcolor{lightred}4.462$^{***}$ & 3.646 \\
 & \textbf{Gemma2 27b} & .109 & .109 & \cellcolor{lightred}.122 & .124 & \cellcolor{lightred}.122 & .127 & \cellcolor{lightred}5.399$^{***}$ & 5.120 & \cellcolor{lightred}6.150$^{***}$ & 5.292 & \cellcolor{lightred}5.880$^{***}$ & 5.183 \\ \midrule

 
\multirow{6}{*}{\textbf{\begin{tabular}[c]{@{}l@{}}Long\\ (50+)\end{tabular}}} 
 & \textbf{GPT-4o} & \cellcolor{lightred}.356$^{***}$ & .384 & \cellcolor{lightred}.281$^{***}$ & .332 & \cellcolor{lightred}.267$^{***}$ & .314 & \cellcolor{lightred}1.846$^{***}$ & 1.577 & \cellcolor{lightred}1.834$^{***}$ & 1.570 & \cellcolor{lightred}2.070$^{***}$ & 1.743 \\
 % & \textbf{GPT-4o +T} & \cellcolor{lightred}.321$^{**}$ & .336 & \cellcolor{lightred}.262$^{*}$ & .283 & \cellcolor{lightred}.248$^{*}$ & .273 & \cellcolor{lightred}1.595$^{*}$ & 1.453 & \cellcolor{lightred}1.741$^{*}$ & 1.621 & \cellcolor{lightred}1.881$^{*}$ & 1.743 \\
 & \textbf{Llama3 8b} & \cellcolor{lightred}.224 & .228 & \cellcolor{lightred}.189$^{***}$ & .214 & \cellcolor{lightred}.179$^{***}$ & .200 & 1.895 & \cellcolor{lightred}1.901 & \cellcolor{lightred}2.024$^{***}$ & 1.835 & \cellcolor{lightred}2.111$^{***}$ & 1.933 \\
 & \textbf{Llama3 70b} & \cellcolor{lightred}.409$^{***}$ & .435 & \cellcolor{lightred}.373$^{*}$ & .391 & \cellcolor{lightred}.360$^{***}$ & .388 & \cellcolor{lightred}1.424$^{**}$ & 1.284 & \cellcolor{lightred}1.667$^{*}$ & 1.557 & \cellcolor{lightred}1.771$^{**}$ & 1.608 \\
 & \textbf{Llama3 405b} & \cellcolor{lightred}.380$^{***}$ & .410 & \cellcolor{lightred}.371$^{**}$ & .396 & \cellcolor{lightred}.357$^{***}$ & .384 & \cellcolor{lightred}1.866$^{***}$ & 1.627 & \cellcolor{lightred}1.906$^{**}$ & 1.719 & \cellcolor{lightred}2.052$^{***}$ & 1.844 \\
 & \textbf{Gemma2 9b} & \cellcolor{lightred}.142 & .151 & \cellcolor{lightred}.192$^{**}$ & .209 & \cellcolor{lightred}.174$^{*}$ & .188 & \cellcolor{lightred}2.515$^{***}$ & 2.350 & \cellcolor{lightred}2.144$^{***}$ & 1.979 & \cellcolor{lightred}2.399$^{***}$ & 2.193 \\
 & \textbf{Gemma2 27b} & .096 & \cellcolor{lightred}.092 & .077 & \cellcolor{lightred}.075 & \cellcolor{lightred}.079 & .083 & \cellcolor{lightred}5.135$^{**}$ & 4.919 & \cellcolor{lightred}5.913$^{**}$ & 5.578 & \cellcolor{lightred}5.606$^{***}$ & 5.160 \\ \midrule

 
\multirow{6}{*}{\textbf{Overall}} 
 & \textbf{GPT-4o} & \cellcolor{lightred}.296$^{***}$ & .312 & \cellcolor{lightred}.222$^{***}$ & .250 & \cellcolor{lightred}.213$^{***}$ & .240 & 2.978 & \cellcolor{lightred}3.022 & \cellcolor{lightred}3.323$^{***}$ & 2.767 & \cellcolor{lightred}3.571$^{***}$ & 2.942 \\
 % & \textbf{GPT-4o +T} & \cellcolor{lightred}.302 & .303 & \cellcolor{lightred}.231$^{**}$ & .240 & \cellcolor{lightred}.221$^{*}$ & .235 & \cellcolor{lightred}2.036$^{**}$ & 1.969 & \cellcolor{lightred}2.574$^{*}$ & 2.264 & \cellcolor{lightred}2.685$^{*}$ & 2.390 \\
 & \textbf{Llama3 8b} & .230$^{*}$ & \cellcolor{lightred}.221 & \cellcolor{lightred}.181$^{***}$ & .195 & \cellcolor{lightred}.168$^{***}$ & .181 & 2.838$^{***}$ & \cellcolor{lightred}3.170 & \cellcolor{lightred}2.672$^{***}$ & 2.470 & \cellcolor{lightred}2.806$^{***}$ & 2.644 \\
 & \textbf{Llama3 70b} & \cellcolor{lightred}.398$^{**}$ & .412 & \cellcolor{lightred}.324$^{***}$ & .339 & \cellcolor{lightred}.313$^{***}$ & .337 & \cellcolor{lightred}1.658$^{***}$ & 1.529 & \cellcolor{lightred}2.835$^{***}$ & 2.496 & \cellcolor{lightred}2.847$^{***}$ & 2.473 \\
 & \textbf{Llama3 405b} & \cellcolor{lightred}.384$^{***}$ & .411 & \cellcolor{lightred}.314$^{***}$ & .344 & \cellcolor{lightred}.313$^{***}$ & .344 & \cellcolor{lightred}2.347$^{***}$ & 2.148 & \cellcolor{lightred}3.505$^{***}$ & 2.939 & \cellcolor{lightred}3.550$^{***}$ & 2.992 \\
 & \textbf{Gemma2 9b} & .147 & \cellcolor{lightred}.143 & \cellcolor{lightred}.185$^{***}$ & .199 & \cellcolor{lightred}.172$^{***}$ & .186 & 2.823 & \cellcolor{lightred}2.882 & \cellcolor{lightred}3.410$^{***}$ & 2.881 & \cellcolor{lightred}3.742$^{***}$ & 3.137 \\
 & \textbf{Gemma2 27b} & .105 & \cellcolor{lightred}.103 & \cellcolor{lightred}.106 & .107 & \cellcolor{lightred}.107 & .111 & \cellcolor{lightred}5.307$^{***}$ & 5.049 & \cellcolor{lightred}6.067$^{***}$ & 5.392 & \cellcolor{lightred}5.785$^{***}$ & 5.175 \\ \bottomrule
 %\multicolumn{14}{l}{\textit{Statistical group differences are indicated as ${^{*}}$  (p<.05), ${^{**}}$ (p<.01), and ${^{***}}$ (p<.001) regarding the model performance.}}
\end{tabular}

\addtolength{\tabcolsep}{+1.2mm}
\vspace{-.5pc}
\caption{Results by length for GPT-4o, Llama3 (8b, 70b, 405b), and Gemma2 (9b, 27b) models. Red cells indicate a worse performance than the other group. (Statistical group differences are indicated as ${^{*}}$  (p<.05), ${^{**}}$ (p<.01), and ${^{***}}$ (p<.001) regarding the model performance.)\kenneth{TODO: (1) This table needs redo (2) Update the numbers}}
\vspace{-1pc}
\label{tab:results-by-length}
\end{table*}


%%%%%%%%%%%%%%%%%%%%%%%%%%%%%%%%%%%%%%%%%%%%%%%%%%%%%%%%%%%%%%%%%%%%%%%%%%%%%%%%%%%%%%%%%%%%%%%%%%%%%%%%%%%%%%%%%%%%%%%%%%%%%%%%%%%%%%%%

% Please add the following required packages to your document preamble:
% \usepackage{booktabs}
% \usepackage{multirow}

\iffalse
\begin{table*}[t]
\centering
\small
\addtolength{\tabcolsep}{-0.8mm}
\begin{tabular}{@{}llcccccccccccc@{}}
\toprule
\multirow{3}{*}{\textbf{\begin{tabular}[c]{@{}l@{}}Text\\ Length\\ (\#word)\end{tabular}}} & \multirow{3}{*}{\textbf{Model}} & \multicolumn{6}{c}{\textbf{ACC} $\uparrow$} & \multicolumn{6}{c}{\textbf{MSE} $\downarrow$} \\ \cmidrule(lr){3-8} \cmidrule(lr){9-14}
 &  & \multicolumn{2}{c}{\textbf{structured}} & \multicolumn{2}{c}{\textbf{plain}} & \multicolumn{2}{c}{\textbf{shuffled}} & \multicolumn{2}{c}{\textbf{structured}} & \multicolumn{2}{c}{\textbf{plain}} & \multicolumn{2}{c}{\textbf{shuffled}} \\ \cmidrule(lr){3-4} \cmidrule(lr){5-6} \cmidrule(lr){7-8} \cmidrule(lr){9-10} \cmidrule(lr){11-12} \cmidrule(lr){13-14}
 &  & \textbf{tw} & \textbf{cn} & \textbf{tw} & \textbf{cn} & \textbf{tw} & \textbf{cn} & \textbf{tw} & \textbf{cn} & \textbf{tw} & \textbf{cn} & \textbf{tw} & \textbf{cn} \\ \midrule
\multirow{6}{*}{\textbf{\begin{tabular}[c]{@{}l@{}}Short\\ (1-49)\end{tabular}}} 
 & \textbf{GPT-4o} & \cellcolor{lightred}0.265 & 0.274 & 0.192 & \cellcolor{lightblue}0.208 & 0.186 & 0.202 & 3.563 & 3.769 & \cellcolor{lightgreen}4.092 & 3.856 & 4.347 & 3.561 \\
 & \textbf{Llama3 8b} & 0.234 & 0.216 & \textbf{0.176} & 0.184 & \textbf{0.162} &0.169. & \textbf{3.402} & 3.931 & \textbf{3.059} & 2.851 & \textbf{3.220} & 3.071 \\
 & \textbf{Llama3 70b} & 0.392 & 0.399 & 0.297 &0.310 & 0.288 & 0.310 & 1.785 & 1.661 & 3.465 & 3.005 & 3.427 & 2.941 \\
 & \textbf{Llama3 405b} & \textbf{0.386} & 0.412 & \textbf{0.283} & 0.315 & \textbf{0.290} & 0.322 & \textbf{2.605} & 2.430 & \textbf{4.365} & 3.598 & \textbf{4.355} & 3.612 \\
 & \textbf{Gemma2 9b} &0.150 & \textbf{0.138} &\textbf{0.182} & 0.193 & \textbf{0.171} & 0.184 & 2.987 & \textbf{3.168} & \textbf{4.089} & 3.367 & \textbf{4.462} & 3.646 \\
 & \textbf{Gemma2 27b} & 0.109 & 0.109 & 0.122 & 0.124 & 0.122 & 0.127 & \textbf{5.399} & 5.120 & \textbf{6.150} & 5.292 & \textbf{5.880} & 5.183 \\ \midrule
\multirow{6}{*}{\textbf{\begin{tabular}[c]{@{}l@{}}Long\\ (50+)\end{tabular}}} 
 & \textbf{GPT-4o} & 0.356 & \cellcolor{lightyellow}0.384 & 0.281 & 0.332 & \cellcolor{lightgreen}0.267 & 0.314 & 1.846 & 1.577 & 1.834 & \cellcolor{lightred}1.570 & 2.070 & 1.743 \\
 & \textbf{Llama3 8b} & 0.224 & 0.228 & 0.189 & 0.214 & 0.179 & 0.200 & 1.895 & 1.901 & 2.024 & 1.835 & 2.111 & 1.933 \\
 & \textbf{Llama3 70b} & \textbf{0.409} & 0.435 & \textbf{0.373} & 0.391 & \textbf{0.360} & 0.388 & \textbf{1.424} & 1.284 & \textbf{1.667} & 1.557 & \textbf{1.771} & 1.608 \\
 & \textbf{Llama3 405b} & \textbf{0.380} & 0.410 & \textbf{0.371} & 0.396 & \textbf{0.357} & 0.384 & \textbf{1.866} & 1.627 & \textbf{1.906} & 1.719 & \textbf{2.052} & 1.844 \\
 & \textbf{Gemma2 9b} & 0.142 & 0.151 & \textbf{0.192} & 0.209 & \textbf{0.174} & 0.188 & \textbf{2.515} & 2.350 & \textbf{2.144} & 1.979 & \textbf{2.399} & 2.193 \\
 & \textbf{Gemma2 27b} & 0.096 & 0.092 & 0.077 & 0.075 & 0.079 & 0.083 & \textbf{5.135} & 4.919 & \textbf{5.913} & 5.578 & \textbf{5.606} & 5.160 \\ \midrule
\multirow{6}{*}{\textbf{Overall}} 
 & \textbf{GPT-4o} & 0.296 & 0.312 & \cellcolor{lightblue}0.222 & 0.250 & 0.213 & \cellcolor{lightyellow}0.240 & 2.978 & 3.022 & 3.323 & 2.767 & \cellcolor{lightred}3.571 & 2.942 \\
 & \textbf{Llama3 8b} &0.230 & 0.221 & 0.181 & 0.195 & 0.168 & 0.181 & 2.838 & 3.170 &2.672 & 2.470 & 2.806 & 2.644 \\
 & \textbf{Llama3 70b} & \textbf{0.398} & 0.412 & \textbf{0.324} & 0.339 & \textbf{0.313} & 0.337 & \textbf{1.658} & 1.529 & \textbf{2.835} & 2.496 & \textbf{2.847} & 2.473 \\
 & \textbf{Llama3 405b} & \textbf{0.384} & 0.411 & \textbf{0.314} & 0.344 & \textbf{0.313} & 0.344 & \textbf{2.347} & 2.148 & \textbf{3.505} & 2.939 & \textbf{3.550} & 2.992 \\
 & \textbf{Gemma2 9b} & 0.147 & 0.143 & \textbf{0.185} & 0.199 & \textbf{0.172} & 0.186 & 2.823 & 2.882 & \textbf{3.410} & 2.881 & \textbf{3.742} & 3.137 \\
 & \textbf{Gemma2 27b} & 0.105 & 0.103 & 0.106 & 0.107 & 0.107 & 0.111 & \textbf{5.307} & 5.049 & \textbf{6.067} & 5.392 & \textbf{5.785} & 5.175 \\ \bottomrule

\end{tabular}
\addtolength{\tabcolsep}{+0.8mm}
\caption{Results by length for GPT-4o, Llama3 (8b, 70b, 405b), and Gemma2 (9b, 27b) models. Bold numbers indicate a significant difference between the groups. \cy{I would prefer to us * to represent the significance. The number of stars is also important to report.}}
\label{tab:results-by-length}
\end{table*}
\fi

% 
\iffalse
\begin{table*}[t]
\centering
\small
\addtolength{\tabcolsep}{-0.5mm}
\begin{tabular}{@{}llcccccccccccc@{}}
\toprule
\multirow{3}{*}{\textbf{\begin{tabular}[c]{@{}l@{}}Text\\ Length\\ (\#word)\end{tabular}}} & \multirow{3}{*}{\textbf{Model}} & \multicolumn{6}{c}{\textbf{ACC}} & \multicolumn{6}{c}{\textbf{MSE}} \\ \cmidrule(lr){3-8} \cmidrule(lr){9-14}
 &  & \multicolumn{2}{c}{\textbf{structured}} & \multicolumn{2}{c}{\textbf{plain}} & \multicolumn{2}{c}{\textbf{shuffled}} & \multicolumn{2}{c}{\textbf{structured}} & \multicolumn{2}{c}{\textbf{plain}} & \multicolumn{2}{c}{\textbf{shuffled}} \\ \cmidrule(lr){3-4} \cmidrule(lr){5-6} \cmidrule(lr){7-8} \cmidrule(lr){9-10} \cmidrule(lr){11-12} \cmidrule(lr){13-14}
 &  & \textbf{tw} & \textbf{cn} & \textbf{tw} & \textbf{cn} & \textbf{tw} & \textbf{cn} & \textbf{tw} & \textbf{cn} & \textbf{tw} & \textbf{cn} & \textbf{tw} & \textbf{cn} \\ \midrule
\multirow{6}{*}{\textbf{\begin{tabular}[c]{@{}l@{}}Short\\ (1-50)\end{tabular}}} 
 & \textbf{GPT-4o} & 0.265 & 0.274 & 0.192 & 0.208 & 0.186 & 0.202 & 3.563 & 3.769 & 4.092 & 3.856 & 4.347 & 3.561 \\
 & \textbf{Llama3 8b} & . & . & . & . & . & . & . & . & . & . & . & . \\
 & \textbf{Llama3 70b} & . & . & . & . & . & . & . & . & . & . & . & . \\
 & \textbf{Llama3 405b} & . & . & . & . & . & . & . & . & . & . & . & . \\
 & \textbf{Gemma2 9b} & . & . & . & . & . & . & . & . & . & . & . & . \\
 & \textbf{Gemma2 27b} & . & . & . & . & . & . & . & . & . & . & . & . \\ \midrule
\multirow{6}{*}{\textbf{\begin{tabular}[c]{@{}l@{}}Long\\ (51+)\end{tabular}}} 
 & \textbf{GPT-4o} & 0.356 & 0.384 & 0.281 & 0.332 & 0.267 & 0.314 & 1.846 & 1.577 & 1.834 & 1.570 & 2.070 & 1.743 \\
 & \textbf{Llama3 8b} & . & . & . & . & . & . & . & . & . & . & . & . \\
 & \textbf{Llama3 70b} & . & . & . & . & . & . & . & . & . & . & . & . \\
 & \textbf{Llama3 405b} & . & . & . & . & . & . & . & . & . & . & . & . \\
 & \textbf{Gemma2 9b} & . & . & . & . & . & . & . & . & . & . & . & . \\
 & \textbf{Gemma2 27b} & . & . & . & . & . & . & . & . & . & . & . & . \\ \midrule
\multirow{6}{*}{\textbf{Overall}} 
 & \textbf{GPT-4o} & 0.296 & 0.312 & 0.222 & 0.250 & 0.213 & 0.240 & 2.978 & 3.022 & 3.323 & 2.767 & 3.571 & 2.942 \\
 & \textbf{Llama3 8b} & . & . & . & . & . & . & . & . & . & . & . & . \\
 & \textbf{Llama3 70b} & . & . & . & . & . & . & . & . & . & . & . & . \\
 & \textbf{Llama3 405b} & . & . & . & . & . & . & . & . & . & . & . & . \\
 & \textbf{Gemma2 9b} & . & . & . & . & . & . & . & . & . & . & . & . \\
 & \textbf{Gemma2 27b} & . & . & . & . & . & . & . & . & . & . & . & . \\ \bottomrule
\end{tabular}
\addtolength{\tabcolsep}{+0.5mm}
\caption{Results by length for GPT-4o, Llama3 (8b, 70b, 405b), and Gemma2 (9b, 27b) models.}
\label{tab:results-by-length}
\end{table*}
\fi

\begin{comment}
% Please add the following required packages to your document preamble:
% \usepackage{booktabs}
% \usepackage{multirow}
\begin{table*}[t]
\centering
\footnotesize
\begin{tabular}{@{}llcccccccccccc@{}}
\toprule
\multirow{3}{*}{\textbf{\begin{tabular}[c]{@{}l@{}}Text\\ Length\\ (\#word)\end{tabular}}} & \multirow{3}{*}{\textbf{Model}} & \multicolumn{6}{c}{\textbf{ACC}} & \multicolumn{6}{c}{\textbf{MSE}} \\ \cmidrule(l){3-14} 
 &  & \multicolumn{2}{c}{\textbf{structured}} & \multicolumn{2}{c}{\textbf{plain}} & \multicolumn{2}{c}{\textbf{shuffled}} & \multicolumn{2}{c}{\textbf{structured}} & \multicolumn{2}{c}{\textbf{plain}} & \multicolumn{2}{c}{\textbf{shuffled}} \\ \cmidrule(l){3-14} 
 &  & \textbf{tw} & \textbf{cn} & \textbf{tw} & \textbf{cn} & \textbf{tw} & \textbf{cn} & \textbf{tw} & \textbf{cn} & \textbf{tw} & \textbf{cn} & \textbf{tw} & \textbf{cn} \\ \midrule
\multirow{3}{*}{\textbf{\begin{tabular}[c]{@{}l@{}}Short\\ (1-49)\end{tabular}}} & \textbf{GPT4o} & .265 & .274 & .192 & .208 & .186 & .202 & 3.563 & 3.769 & 4.092 & 3.856 & 4.347 & 3.561 \\
 & \textbf{Llama3} & .MMM & .NNN & .OOO & .PPP & .QQQ & .RRR & .MMM & .NNN & .OOO & .PPP & .QQQ & .RRR \\
 & \textbf{Gemma2} & .SSS & .TTT & .UUU & .VVV & .XXX & .YYY & .SSS & .TTT & .UUU & .VVV & .XXX & .YYY \\ \midrule
\multirow{3}{*}{\textbf{\begin{tabular}[c]{@{}l@{}}Long\\ (50+)\end{tabular}}} & \textbf{GPT} & .356 & .384 & .281 & .332 & .267 & .314 & 1.846 & 1.577 & 1.834 & 1.570 & 2.070 & 1.743 \\
 & \textbf{Llama} & .MMM & .NNN & .OOO & .PPP & .QQQ & .RRR & .MMM & .NNN & .OOO & .PPP & .QQQ & .RRR \\
 & \textbf{Gemma2} & .SSS & .TTT & .UUU & .VVV & .XXX & .YYY & .SSS & .TTT & .UUU & .VVV & .XXX & .YYY \\ \midrule
\multirow{3}{*}{\textbf{Overall}} & \textbf{GPT} & .296 & .312 & .222 & .250 & .213 & .240 & 2.978 & 3.022 & 3.323 & 2.767 & 3.571 & 2.942 \\
 & \textbf{Llama} & .MMM & .NNN & .OOO & .PPP & .QQQ & .RRR & .MMM & .NNN & .OOO & .PPP & .QQQ & .RRR \\
 & \textbf{Gemma2} & .SSS & .TTT & .UUU & .VVV & .XXX & .YYY & .SSS & .TTT & .UUU & .VVV & .XXX & .YYY \\ \bottomrule
\end{tabular}
\caption{Results by length.\kenneth{TODO: List (1) GPT-4o, (2) Llama3 8b, (3) Llama3 70b, (4) Gemma2 9b, (5) Llama3 405b, (6) Gemma2 27b}}
\label{tab:results-by-length}
\end{table*}
\end{comment}

\begin{comment}
% Please add the following required packages to your document preamble:
% \usepackage{booktabs}
% \usepackage{multirow}
\begin{table*}[t]
\centering
%\small

\iffalse
\begin{tabular}{@{}llcccc@{}}
\toprule
\multirow{2}{*}{\textbf{\begin{tabular}[c]{@{}l@{}}Text\\ Length\end{tabular}}} & \multirow{2}{*}{\textbf{Model}} & \multicolumn{2}{c}{\textbf{ACC}} & \multicolumn{2}{c}{\textbf{MSE}} \\ \cmidrule(l){3-6} 
 &  & \textbf{tw} & \textbf{ch} & \textbf{tw} & \textbf{ch} \\ \midrule
\multirow{2}{*}{\textbf{\begin{tabular}[c]{@{}l@{}}Short (0-50 words)\\ n=A,AA\end{tabular}}} & \textbf{GPT-4o} & .264, .191, .185 & .273, .206, .200 & E.EE & F.FF \\
 & \textbf{LLaMA 3.1} & .AAA & .BBB & E.EE & F.FF \\ \midrule
\multirow{2}{*}{\textbf{\begin{tabular}[c]{@{}l@{}}Medium (50-270 words)\\ n=B,BB\end{tabular}}} & \textbf{GPT-4o} & .355, .280, .266 & .383, .331, .313 & E.EE & F.FF \\
 & \textbf{LLaMA 3.1} & .AAA & .BBB & E.EE & F.FF \\ \midrule
\multirow{2}{*}{\textbf{\begin{tabular}[c]{@{}l@{}}Long (270+ words)\\ n=C,CC\end{tabular}}} & \textbf{GPT-4o} & .002, .001, .001 & .001, .001, .001 & E.EE & F.FF \\
 & \textbf{LLaMA 3.1} & .AAA & .BBB & E.EE & F.FF \\ \midrule
\multirow{2}{*}{\textbf{\begin{tabular}[c]{@{}l@{}}Overall\\ n=D,DD\end{tabular}}} & \textbf{GPT-4o} & .296, .222, .213 & .312, .250, .240 & E.EE & F.FF \\
 & \textbf{LLaMA 3.1} & .AAA & .BBB & E.EE & F.FF \\ \bottomrule
\end{tabular}
\fi

\end{table*}
\end{comment}



\begin{comment}



% Please add the following required packages to your document preamble:
% \usepackage{booktabs}
% \usepackage{multirow}
\begin{table*}[t]
\centering
\small
\begin{tabular}{@{}llcccccccc@{}}
\toprule
\multirow{2}{*}{\textbf{\begin{tabular}[c]{@{}l@{}}Text\\ Length\end{tabular}}} & \multirow{2}{*}{\textbf{LLMs}} & \multicolumn{4}{c}{\textbf{ACC}} & \multicolumn{4}{c}{\textbf{MSE}} \\ \cmidrule(l){3-10} 
 &  & \textbf{tw} & \textbf{ch} & \textbf{tw-\textgreater{}ch} & \textbf{ch-\textgreater{}tw} & \textbf{tw} & \textbf{ch} & \textbf{tw-\textgreater{}ch} & \textbf{ch-\textgreater{}tw} \\ \midrule
\multirow{3}{*}{\textbf{\begin{tabular}[c]{@{}l@{}}Short\\ (0-50 words)\\ n=A,AA\end{tabular}}} & \textbf{GPT-4o} & .AAA & .BBB & .CCC & .DDD & E.EE & F.FF & G.GG & H.HH \\
 & \textbf{OpenAI o1} & .AAA & .BBB & .CCC & .DDD & E.EE & F.FF & G.GG & H.HH \\
 & \textbf{LLaMA 3.1} & .AAA & .BBB & .CCC & .DDD & E.EE & F.FF & G.GG & H.HH \\ \midrule
\multirow{3}{*}{\textbf{\begin{tabular}[c]{@{}l@{}}Medium\\ (50-300 words)\\ n=B,BB\end{tabular}}} & \textbf{GPT-4o} & .AAA & .BBB & .CCC & .DDD & E.EE & F.FF & G.GG & H.HH \\
 & \textbf{OpenAI o1} & .AAA & .BBB & .CCC & .DDD & E.EE & F.FF & G.GG & H.HH \\
 & \textbf{LLaMA 3.1} & .AAA & .BBB & .CCC & .DDD & E.EE & F.FF & G.GG & H.HH \\ \midrule
\multirow{3}{*}{\textbf{\begin{tabular}[c]{@{}l@{}}Long\\ (300+ words)\\ n=C,CC\end{tabular}}} & \textbf{GPT-4o} & .AAA & .BBB & .CCC & .DDD & E.EE & F.FF & G.GG & H.HH \\
 & \textbf{OpenAI o1} & .AAA & .BBB & .CCC & .DDD & E.EE & F.FF & G.GG & H.HH \\
 & \textbf{LLaMA 3.1} & .AAA & .BBB & .CCC & .DDD & E.EE & F.FF & G.GG & H.HH \\ \midrule
\multirow{3}{*}{\textbf{\begin{tabular}[c]{@{}l@{}}Overall\\ n=D,DD\end{tabular}}} & \textbf{GPT-4o} & .AAA & .BBB & .CCC & .DDD & E.EE & F.FF & G.GG & H.HH \\
 & \textbf{OpenAI o1} & .AAA & .BBB & .CCC & .DDD & E.EE & F.FF & G.GG & H.HH \\
 & \textbf{LLaMA 3.1} & .AAA & .BBB & .CCC & .DDD & E.EE & F.FF & G.GG & H.HH \\ \bottomrule
\end{tabular}
\caption{Results by length.}
\label{tab:results-by-length}
\end{table*}
    
\end{comment}

\section{Constructing a Contextually-Aligned Review Dataset for Language Varieties}
Effective follow-up question generation (QG) requires models to infer and target gaps between the provided answer and the broader context of a conversation. We follow the task definition presented by the \fd{} \cite{meng-etal-2023-followupqg}: ``to generate follow-up questions that seek new information given the initial question and answer''. For simplicity, we denote the ``initial question'' as IQ, ``initial answer'' as IA, and the ``follow-up question'' as FQ. Critical limitations in the training dataset are identified, including quality issues, which are addressed through dataset cleaning (Sec~\ref{sec:data:cleaning}). The small scale (2,790 instances) and low diversity of the dataset are tackled by a novel data augmentation pipeline introduced in this paper (Sec~\ref{sec:data:augmentation}). Specifically, we augment the \fd{} training set with synthetic data generated by a pipeline that mimics human-like gap-driven questioning. This approach utilizes LLM-generated comprehensive answers to systematically identify missing information in initial answers and generate follow-up questions targeting those gaps. We demonstrate that the augmented data retains high quality (Sec~\ref{sec:data:quality}).

\subsection{Data Cleaning}
\label{sec:data:cleaning}

The \fd{} dataset is limited by its small scale, comprising 3,790 samples: 2,790 for training, 500 for validation, and 500 for testing. Within the 2,790 training instances, there are only 2,651 unique (IQ, IA, FQ) triplets, indicating duplication. Additionally, the number of 2,648 unique (IQ, IA) pairs suggest minimal follow-up question diversity, as 99.8\% of pairs have only one reference FQ. Further analysis also uncovered data quality issues, likely stemming from automated data collection (see Appendix~\ref{app:problematic_sample}). To improve the data quality, we did the following:

\begin{itemize}[leftmargin=10pt]
    \item \textbf{Deduplication.} We removed 139 duplicate (IQ, IA, FQ) triplet instances.
    \item \textbf{Reference quality check.} We manually filtered out 84 instances where the reference FQ diverged entirely from the initial question. 
    \item \textbf{Sensitive content removal.} We excluded 24 instances involving topics like self-harm or crime, which LLMs are likely to refuse to answer.
\end{itemize}

% \paragraph{Deduplication.} We removed 139 duplicate (IQ, IA, FQ) triplet instances. 

% \paragraph{Reference quality check.} We manually filtered out 84 instances where the reference FQ diverged entirely from the initial question. %This step ensures coherence in the augmented dataset.   

% \paragraph{Sensitive content removal.} We excluded 24 instances involving topics like self-harm or crime, which modern LLMs are likely to refuse to answer. %, avoiding pipeline failures during augmentation.

The cleaned dataset (2,543 instances) retained broad topic coverage (2,533 unique question-answer pairs).

\subsection{Augmentation Pipeline}
\label{sec:data:augmentation}

As discussed in Section~\ref{sec:data:cleaning}, the limited scale of the dataset and the lack of follow-up question diversity hinder the coverage of diverse questioning strategies, restricting model generalization. To address this, we design a GPT-4-based pipeline that augments the original dataset by generating additional follow-up questions. Our pipeline simulates human reasoning through three interconnected stages: comprehensive answer generation, information gap identification, and follow-up question generation\footnote{Please refer to Appendix~\ref{app:prompts} for the LLM prompts used for the following stages.}. %Each stage aims to enhance diversity, improve quality, and align with human strategies for effective information-seeking dialogues.

\paragraph{Comprehensive answer generation.} To identify gaps in the IA, we generate a comprehensive answer (CA) that represents a complete and thorough response to the IQ. As shown in Figure~\ref{fig:data_aug_pipeline}, we prompt GPT-4 iteratively to generate answers to IQ that target different perspectives, such as technical, ethical, and practical, and synthesize a unified CA.
% Prompts such as ``Generate a concise answer focused on a single perspective'' and ``Synthesize prior answers into a comprehensive explanation'' ensure the CA is broad yet cohesive. This process mirrors how humans consider multiple dimensions to intuit missing information. 

\paragraph{Information gap identification.} The next step is to identify key concepts or details discussed in the comprehensive answer (CA) but not covered in the initial answer (IA). This is done by prompting GPT-4. For example, in the example shown in Figure~\ref{fig:data_aug_pipeline}, the initial answer covers the topic of privacy issues but does not cover areas of cyber security (i.e. an information gap).
%The LLM uses the prompt: ``Identify specific explanations or concepts in the comprehensive answer absent from the initial answer.'' 

\paragraph{Follow-up question generation.} Using the identified information gaps, we prompt GPT-4 to generate follow-up questions that address those gaps while maintaining contextual relevance to the IQ and IA. The generated questions must meet three criteria: be (1) answerable by the CA, (2) unanswerable by the IA, and (3) grounded in terminology and context from the IQ. % This ensures that the follow-up questions are focused, relevant, and factually grounded, emulating human curiosity and reasoning.

% \paragraph{Dataset Reformation and Merging}
To match the format of the original follow-up questions in the \fd{} dataset, we automatically reformat the generated FQs to remove artifacts such as bullets or numbering. Each (IQ, IA) pair is enriched with multiple follow-up questions generated by our pipeline, averaging 10.95 questions per pair. The restructured dataset merges synthetic questions with cleaned human-generated examples, resulting in 27,874 samples---a 10$\times$ increase from the original dataset size. This expanded dataset captures the open-ended nature of human questioning, providing models with diverse and explicit signals to learn strategies for addressing information gaps effectively.

\subsection{Augmented Data Validation}
\label{sec:data:quality}

To assess the quality of the generated follow-up questions, we conducted a human evaluation study using Cloud Connect. To ensure high-quality annotations, we restricted participation to native English-speaking annotators with a minimum of 1,000 completed annotation tasks and an approval rating exceeding 90\%. A randomly sampled subset of 100 (IQ, IA, FQ) triplets was evaluated based on three key criteria: (1) whether the follow-up question was a valid question\footnote{A valid question must be in a question format and ask meaningful information, including Wh-questions (what/why/where/etc.), open-ended questions, probing questions and etc. \cite{meng-etal-2023-followupqg}}, (2) whether any component of the triplet contained sensitive information, and (3) the degree of relatedness between the follow-up question and the initial question-answer pair. The full survey format, including example annotations, is provided in Appendix~\ref{sec:augmented_data_annotation_guideline}. 
The results show that 94\% of the follow-up questions are labeled as valid, 92\% as not sensitive, and 91\% are related to the original (IQ, IA) pair. Inter-annotator agreement was moderate, with a Cohen's Kappa score of $\kappa = 0.73$ \cite{mchugh2012interrater}.
%, demonstrating strong inter-rater reliability in the evaluation process.

% \label{human_GPT}


%\kenneth{TODO: Update Gemma2 27b result}



\section{Experimental Results\label{sec:experiment}}

%\kenneth{TODO CY: Re-run Llama3 (x3) and Gemma2 (x2)---Will finish on Sunday Feb 2}

%\kenneth{TODO Zixin: Re-run the stats test after Feb 2}






\section{Experiments}
\seclabel{experiments}
Our experiments are designed to test a) the extent to which open loop execution is an issue for precise mobile manipulation tasks, b) how effective are blind proprioceptive correction techniques, c) do object detectors and point trackers perform reliably enough in wrist camera images for reliable control, d) is occlusion by the end-effector an issue and how effectively can it be mitigated through the use of video in-painting models, and e) how does our proposed \name methodology compare to large-scale imitation learning? 


\subsection{Tasks and Experimental Setup}
We work with the Stretch RE2 robot. Stretch RE2 is a commodity mobile manipulator with a 5DOF arm mounted on top of a non-holomonic base. We upgrade the robot to use the Dex Wrist 3, which has an eye-in-hand RGB-D camera (Intel D405). 
We consider 3 task families for a total
of 6 different tasks: a) holding a knob to pull open a cabinet or drawer, b) holding a
handle to pull open a cabinet, and c) pushing on objects (light buttons, books
in a book shelf, and light switches). Our focus is on generalization. {\it
Therefore, we exclusively test on previously unseen instances, not used during
development in any way.} 
\figref{tasks} shows the instances that we test on. 

All tasks involve some precise manipulation, followed by execution of a motion
primitive. {\bf For the pushing tasks}, the precise motion is to get the
end-effector exactly at the indicated point and the motion primitive is to push
in the direction perpendicular to the surface and retract the end-effector 
upon contact. The robot is positioned such
that the target position is within the field of view of the wrist camera. A user
selects the point of pushing via a mouse click on the wrist camera image. The
goal is to push at the indicated location. Success is determined by whether the
push results in the desired outcome (light turns on / off or book gets pushed in). 
The original rubber gripper bends upon contact, we use a rigid known tool
that sticks out a bit. We take the geometry of the tool into account while servoing.

{\bf For the opening articulated object tasks}, the precise manipulation is grasping the
knob / handle, while the motion primitive is the whole-body motion that opens
the cupboard. Computing and executing this full body motion is difficult. We
adopt the modular approach to opening articulated objects (MOSART) from Gupta \etal~\cite{gupta2024opening} and invoke it
after the gripper has been placed around the knob / handle. The whole tasks 
starts out with the robot about 1.5m way from the target object, with the 
target object in view
from robot's head mounted camera. We use MOSART to compute articulation
parameters and convey the robot to a pre-grasp
location with the target handle in view of the wrist camera. At this point,
\name (or baseline) is used to center the gripper around the knob / handle, 
before resuming MOSART: extending the gripper till contact, close the gripper, and play rest of the predicted motion plan. Success is 
determined by whether the cabinet opens by more than $60^\circ$
or the drawer is pulled out by more than $24cm$, similar to the criteria used in \cite{gupta2024opening}.


For the precise manipulation part, all baselines consume the current and
previous RGB-D images from the wrist camera and output full body motor
commands.

% % Please add the following required packages to your document preamble:
% % \usepackage{graphicx}
% \begin{table*}[!ht]
% \centering
% \caption{}
% \label{tab:my-table}
% \resizebox{\textwidth}{!}{%
% \begin{tabular}{lcccccc}
% \toprule
%  & \multicolumn{2}{c}{ours} & \multicolumn{2}{c}{Gurobi} & \multicolumn{2}{c}{MOSEK} \\
%  & \multicolumn{1}{l}{time (s)} & \multicolumn{1}{l}{optimality gap (\%)} & \multicolumn{1}{l}{time (s)} & \multicolumn{1}{l}{optimality gap (\%)} & \multicolumn{1}{l}{time (s)} & \multicolumn{1}{l}{optimality gap (\%)} \\ \hline
% \begin{tabular}[c]{@{}l@{}}Linear Regression\\ Synthetic \\ (n=16000, p=16000)\end{tabular} & 57 & 0.0 & 3351 & - & 2148 & - \\ \hline
% \begin{tabular}[c]{@{}l@{}}Linear Regression\\ Cancer Drug Response\\ (n=822, p=2300)\end{tabular} & 47 & 0.0 & 1800 & 0.31 & 212 & 0.0 \\ \hline
% \begin{tabular}[c]{@{}l@{}}Logistic Regression\\ Synthetic\\ (n=16000, p=16000)\end{tabular} & 271 & 0.0 & N/A & N/A & 1800 & - \\ \hline
% \begin{tabular}[c]{@{}l@{}}Logistic Regression\\ Dorothea\\ (n=1150, p=91598)\end{tabular} & 62 & 0.0 & N/A & N/A & 600 & 0.0 \\
% \bottomrule
% \end{tabular}%
% }
% \end{table*}

% Please add the following required packages to your document preamble:
% \usepackage{multirow}
% \usepackage{graphicx}
\begin{table*}[]
\centering
\caption{Certifying optimality on large-scale and real-world datasets.}
\vspace{2mm}
\label{tab:my-table}
\resizebox{\textwidth}{!}{%
\begin{tabular}{llcccccc}
\toprule
 &  & \multicolumn{2}{c}{ours} & \multicolumn{2}{c}{Gurobi} & \multicolumn{2}{c}{MOSEK} \\
 &  & time (s) & opt. gap (\%) & time (s) & opt. gap (\%) & time (s) & opt. gap (\%) \\ \hline
\multirow{2}{*}{Linear Regression} & \begin{tabular}[c]{@{}l@{}}synthetic ($k=10, M=2$)\\ (n=16k, p=16k, seed=0)\end{tabular} & 79 & 0.0 & 1800 & - & 1915 & - \\ \cline{2-8}
 & \begin{tabular}[c]{@{}l@{}}Cancer Drug Response ($k=5, M=5$)\\ (n=822, p=2300)\end{tabular} & 41 & 0.0 & 1800 & 0.89 & 188 & 0.0 \\ \hline
\multirow{2}{*}{Logistic Regression} & \begin{tabular}[c]{@{}l@{}}Synthetic ($k=10, M=2$)\\ (n=16k, p=16k, seed=0)\end{tabular} & 626 & 0.0 & N/A & N/A & 2446 & - \\ \cline{2-8}
 & \begin{tabular}[c]{@{}l@{}}DOROTHEA ($k=15, M=2$)\\ (n=1150, p=91598)\end{tabular} & 91 & 0.0 & N/A & N/A & 634 & 0.0 \\
 \bottomrule
\end{tabular}%
}
% \vspace{-3mm}
\end{table*}

\begin{figure*}
\insertW{1.0}{figures/figure_6_cropped_brighten.pdf}
\caption{{\bf Comparison of \name with the open loop (eye-in-hand) baseline} for opening a cabinet with a knob. Slight errors in getting to the target cause the end-effector to slip off, leading to failure for the baseline, where as our method is able to successfully complete the task.}
\figlabel{rollout}
\end{figure*}

\begin{table}
\setlength{\tabcolsep}{8pt}
  \centering
  \resizebox{\linewidth}{!}{
  \begin{tabular}{lcccg}
  \toprule
                              & \multicolumn{2}{c}{\bf Knobs} & \bf Handle & \bf \multirow{2}{*}{\bf Total} \\
                              \cmidrule(lr){2-3} \cmidrule(lr){4-4}
                              & \bf Cabinets & \bf Drawer & \bf Cabinets & \\
  \midrule
  RUM~\cite{etukuru2024robot}  & 0/3    & 1/4         & 1/3         & 2/10 \\
  \name (Ours) & 2/3    & 2/4         & 3/3     &  7/10 \\
  \bottomrule
  \end{tabular}}
  \caption{Comparison of \name \vs RUM~\cite{etukuru2024robot}, a recent large-scale end-to-end imitation learning method trained on 1200 demos for opening cabinets and 525 demos for opening drawers across 40 different environments. Our evaluation spans objects from three environments across two buildings.}
  \tablelabel{rum}
\end{table}

\subsection{Baselines}
We compare against three other methods for the precise manipulation part of
these tasks. 
\subsubsection{Open Loop (Eye-in-Hand)} To assess the precision requirements of
the tasks and to set it in context with the manipulation capabilities of the
robot platform, this baseline uses open loop execution starting from estimates
for the 3D target position from the first wrist camera image.
\subsubsection{MOSART~\cite{gupta2024opening}}
The recent modular system for opening cabinets and drawers~\cite{gupta2024opening}
reports impressive performance with open-loop control (using the head camera from 1.5m away), combined with proprioception-based feedback to 
compensate for errors in perception and control when interacting with handles. 
We test if such correction is also sufficient for interacting with knobs. Note 
that such correction is not possible for the smaller buttons and pliable books.

\subsubsection{\name (no inpainting)} To understand how much of an issue
occlusion due to the end-effector is during manipulation, we ablate the use of
inpainting. %

\subsubsection{Robot Utility Models (RUM)~\cite{etukuru2024robot}}
For the opening articulated object tasks, we also compare to Robot Utility Models (RUM), 
a closed-loop imitation learning method recently proposed by Etukuru et al. \cite{etukuru2024robot}.
RUM is trained on a substantial dataset comprising expert demonstrations, including 
1,200 instances of cabinet opening and 525 of drawer opening, gathered from roughly 
40 different environments.
This dataset stands as the most extensive imitation 
learning dataset for articulated object manipulation to date, establishing RUM as a 
strong baseline for our evaluation.

Similar to our method, we use MOSART to compute articulation
parameters and convey the robot to a pre-grasp location
with the target handle in view of the wrist camera.
One of the assumptions of RUM is a good view of the handle.
To benefit RUM, we try out three different heights of the wrist camera,
and \textit{report the best result for RUM.}

\begin{figure*}
\insertW{1.0}{figures/figure_9_cropped_brighten.pdf}
\caption{{\bf \name \vs open loop (eye-in-hand) baseline for pushing on user-clicked points}. Slight errors in getting to the target cause failure, where as \name successfully turns the lights off. Note the quality of CoTracker's track ({\color{blue} blue dot}).}
\figlabel{rollout_v2}
\end{figure*}

\begin{figure*}
\insertW{1.0}{figures/figure_5_v2_cropped_brighten.pdf}
\caption{{\bf Comparison of \name with and without inpainting}. Erroneous detection without inpainting causes execution to fail, where as with inpainting the target is correctly detected leading to a successful grasp and a successful execution.}
\figlabel{rollouts2}
\end{figure*}


\subsection{Results}
\tableref{results} presents results from our experiments. 
Our training-free approach \name successfully 
solves over 85\% of task instances that we test on.
As noted, all these
tests were conducted on unseen object instances in unseen
environments that were not used for development in any way. We discuss our key
experimental findings below.

\subsubsection{Closing the loop is necessary for these precise tasks} 
While the proprioception-based strategies proposed in MOSART~\cite{gupta2024opening}
work out for handles, they are inadequate for targets like knobs and just
don't work for tasks like pushing buttons. Using estimates from the wrist
camera is better, but open loop execution still fails for knobs and pushing
buttons. 

\subsubsection{Vision models work reasonably well even on wrist camera images}
Inpainting works well on wrist camera images (see \figref{occlusion} and \figref{inpainting}).
Closing the loop using feedback from vision detectors and point trackers on
wrist camera images also work well, particularly when we use in-painted images.
See some examples detections and point tracks in \figref{rollout} and \figref{rollout_v2}. 
Detic~\cite{zhou2022detecting} was able to reliably detect the knobs and
handles and CoTracker~\cite{karaev2023cotracker} was able to successfully track
the point of interaction letting us solve 24/28 task instances.

\subsubsection{Erroneous detections without inpainting hamper performance on 
handles and our end-effector out-painting strategy effectively mitigates it} 
As shown in \figref{rollouts2}, presence of the end-effector caused the object
detector to miss fire leading to failed execution. Our out painting approach
mitigates this issue leading to a higher success rate than the 
approach without out-painting. Interestingly, CoTracker~\cite{karaev2023cotracker} is quite robust
to occlusion (possibly because it tracks multiple points) and doesn't benefit
from in-painting. 


\subsubsection{Closed-loop imitation learning struggles on novel objects}
As presented in \tableref{rum}, \name significantly outperforms RUM in a paired evaluation on unseen objects across three novel environments. A common failure mode of RUM is its inability to grasp the object's handle, even when it approaches it closely.
Another failure mode we observe is RUM misidentifying keyholes or cabinet edges as handles, also resulting in failed grasp attempts.
These result demonstrate that a modular approach that leverages the broad generalization capabilities of vision foundation models is able to generalize much better than an end-to-end imitation learning approach trained on 1000+ demonstrations, which must learn all aspects of the task from scratch.





%\section{Examining  of More Potential Confounding Factors}
\section{Examining Confounding Variables}
%Contributing to the Performance Gap

\begin{comment}
    

Relaxing strict semantic alignment between paired text items inevitably introduces confounding variables.
%We believe that 
This trade-off is worth exploring because it enables researchers to compare model behaviors across language varieties in new ways. 
%is worthwhile, as it enables comparisons of model behavior across language varieties in new ways. %that were previously impossible.
This section examines a few potential confounding variables.
%This section examines a few key factors.
\end{comment}

\paragraph{Could the performance gap be due to \cnChinese reviews having better \ul{writing quality} or better \ul{alignment between content and ratings}?}
\textit{Rationale:} Better writing quality or better content-rating alignment could make it easier for LLMs to predict ratings.
\textit{Analysis \& Findings:} \textbf{No.} 
Our human validation (Section~\ref{sec:data-quality-validation}) shows that \cnChinese reviews had slightly worse writing quality and content-rating alignment. 









\paragraph{Could the performance gap be due to more \ul{code-mixed usage} in \twChinese?}
\textit{Rationale:} NLP models often struggle with code-mixed data~\cite{zhang-etal-2023-multilingual, ochieng2024beyond}. 
%\kenneth{TODO Zixin: Add citations}
\textit{Analysis \& Findings:}
\textbf{No.}
%\twChinese reviews contained less code-mixed usage.
The \cnChinese reviews contain more mixed-language input (30.99\%) than the \twChinese reviews (25.26\%, see Appendix~\ref{appendix:language-analysis} and Table~\ref{tab:language-distribution}).
%\kenneth{TODO CY: I fixed the ref. Are these (1) ref and (2) numbers correct?}\cy{number is correct and based on our latest table}
%Our findings aligned with other studies investigating LLMs' performance between simplified Mandarin (used in Mainland China) and traditional Mandarin (used in Taiwan), which showed potential performance disparity across multiple NLP tasks in Taiwan Mandarin~\cite{chen2024measuring}.

\paragraph{Could the performance gap be due to \cnChinese \ul{users} systematically \ul{giving higher scores}, which align better with LLM-generated scores?}
\textit{Rationale:} LLMs tend to assign higher scores~\cite{stureborg2024large,kobayashi-etal-2024-large,golchin-etal-2025-grading}.
%wang-etal-2024-large-language-models-fair,
%stureborg2024large,wang-etal-2024-large-language-models-fair,
%kobayashi-etal-2024-large
% \kenneth{TODO CY and Zixin: Is this true? If so, add citations.}
\textit{Analysis \& Findings:}
\textbf{Unlikely.}
%Even though we had fewer restrictions on the review pairing process (using similar rating reviews rather than reviews with the same scores) to maximize pairing amounts, this process didn't introduce biases in score rating. 
In our dataset, \twChinese and \cnChinese reviews show no significant difference in scores (\textit{t}(22917) = .160, \textit{p} = .873).
%In our dataset, there is no statistical difference in actual scores between \twChinese and \cnChinese reviews (\textit{t}(22917)=.160, \textit{p}=.873). 
%This result showed that in our data set, 
%Namely, reviews written in \cnChinese and \twChinese share a similar distribution of rating scores; there is no systematic scoring difference between the two groups.

\paragraph{Are \cnChinese reviews \ul{easier for humans to guess ratings}?}
\textit{Rationale:} Human performance is sometimes used as an indicator of a task's difficulty for LLMs~\cite{sakamoto-etal-2025-development,ding2024easyhardbench}.
% \kenneth{TODO CY and Zixin: Add citations.}
\textit{Analysis \& Findings:}
\textbf{Plausible.}
We conducted a user study with 10 participants (5 native speakers from each variety) who reviewed 50 random CN-TW review pairs (100 total reviews) and predicted their rating scores.
%Participants also rated the readability and nativeness of each review.
%More details are provided in \Cref{app:human-prediction}.
%Although there was no significant difference in readability or nativeness between CN and TW reviews, 
%In the study, 
Participants performed significantly better at predicting ratings for reviews in \cnChinese.
After excluding two TW native speakers whose accuracy was more than two standard deviations below the mean, 6 out of the 8 participants had better accuracy on CN reviews than TW reviews, and 7 had better (lower) MSE on CN reviews than TW reviews (see \Cref{app:human-prediction} for more details).
%\kenneth{TODO Zixin: (1) Update these numbers, (2) Move all other results of this study to Appendix: ``For overall content quality, our results show no significant differences in score predictions among the data pairs, indicating raters have no biases in reading and understanding reviews from either group of speakers/writers. However, results showed statistical significance in both ACC (37.00\% vs. 28.75\%, \textit{p}=.016) and MSE (2.795 vs. 3.510, \textit{p}=.036), showing that native speakers might have more difficulties in correctly guessing the review scores for reviews in \twChinese.''}
%We want to emphasize that 

These results should be interpreted with caution.
Unlike question-answering, predicting hundreds of review scores from content is not a typical human task, and most NLP papers on sentiment analysis do not compare model performance to human performance.
Thus, it is unclear whether human performance gaps in such tasks reliably indicate task difficulty for LLMs, especially given the small differences between the two varieties.
%As a result, it is unclear whether our findings indicate task difficulty for LLMs, especially given the relatively small differences in human performance between the two varieties.
Additionally, our participants may not represent the average Mandarin speaker's ability in sentiment analysis, as the two participants performed notably poorly. 
Finally, despite our efforts to examine confounding variables such as text length, code-mixing, and writing quality, we still \textbf{lack a clear understanding of what causes the observed LLMs' performance gaps across language varieties}.
%Further research is needed to understand LLM behavior across language varieties.
%Poor human performance does not necessarily mean LLMs will struggle in the same way.











%After excluding two TW native speakers whose accuracy was more than two standard deviations below the mean, 6 of the remaining 8 participants had higher accuracy on CN reviews than on Tw reivews, and 6 had lower MSE on CN reviews. 











%\zixin{1. no significant differences in naiveness and readability between two sets of reviews; 2. within raters, there are no significant differences in score predictions across \cnChinese and \twChinese review pairs - native speakers perceive similarly between \cnChinese and \twChinese reviews; \textbf{3. there are stat. significances in terms of ACC and MSE in prediction tasks with better performance towards \cnChinese reviews.}} 



%\kenneth{TODO Zixin: Add text here}

%, meaning this factor does not account for the performance gap.


%\kenneth{TODO: Human study will be done by Feb 3 (Mon)}



\section{Discussion}
%\kenneth{TODO: One more subsection under Discussion to debunk the common alternative interpretations, including (1) writing quality, (2) code mix, (3) extra human study}

%\kenneth{Encoding? How often Google Translation changed the content?}

\section{Discussion}
Through our application of personalized accessibility maps and routing applications, we showed how data and insights from our survey findings can help inform the development of more accurate navigation and analytical tools. 
We now situate our findings in related work, highlight how this survey contributes to personalized routing and accessibility mapping for mobility disability groups as well as present directions for future research.

\subsection{Online Image Survey Method}
In this study, we conducted a large-scale image survey (\textit{N=}190) to gather perceptions of sidewalk barriers from different mobility aid user groups. 
This approach helped us to collect insights on the differences between mobility aid user groups as well as shared challenges.
Previous research exploring the relationship between mobility aids and physical environment have mainly employed methods including in-person interviews~\cite{rosenberg_outdoor_2013}, GPS tracking~\cite{prescott_exploration_2021, prescott_factors_2020,rosenberg_outdoor_2013}, and online questionnaires~\cite{carlson_wheelchair_2002}. While interviews and tracking studies typically yield rich detailed information, they are limited to a small sample size. Online text based questionnaires often achieve larger sample sizes but at a cost of depth and nuance. Our image survey method struck a balance between sample size and detail. We collected a large sample within a relatively short time frame, enabling us to gather valuable insights and synthesize patterns across user groups.

Despite advantages, our approach has some limitations. Although street view images help situate and ground a participant's response---as one pilot participant said ``\textit{You're triggering a similar response to a real-life scenario''}, they cannot fully replicate the experience of evaluating a sidewalk \textit{in situ}. The lack of physical interaction with the environment limits the assessment of certain factors. For instance, one of our pilot participants noted that determining whether they could navigate past an obstacle like a trash can varies depending on \sayit{whether the trash can is light enough so I can push it away.} Using our findings as a backdrop, future work should conduct follow-up interviews and in-person evaluations. Such approaches would complement the quantitative data with richer qualitative insights, allowing researchers to better understand the patterns observed in quantitative data as well as the reasoning behind mobility aids users’ assessment.

\subsection{Personalized Accessibility Maps}
Our approach to infuse accessibility maps and routing algorithms with personalized information contributes to the field of accessible urban navigation and analytics. 
Based on our findings, we implemented two accessibility-oriented mapping prototypes, which demonstrate how our data can be used in urban accessibility analytics and personalized routing algorithms. While our current implementation serves as a proof of concept, future research could explore using our findings with more advanced modeling methods such as fuzzy logic~\cite{kasemsuppakorn_personalised_2009, gharebaghi_user-specific_2021, hashemi_collaborative_2017} and AHP~\cite{kasemsuppakorn_personalised_2009,kasemsuppakorn_understanding_2015, hashemi_collaborative_2017}. 

For our current map applications, we used a single set of open-source sidewalk data from Project Sidewalk. However, we acknowledge that other important factors are not included, such as sidewalk topography, width, stairs, crossing conditions, paving material, lighting conditions, weather, and pedestrian traffic~\cite{rosenberg_outdoor_2013,kasemsuppakorn_personalised_2009,darko_adaptive_2022,hashemi_collaborative_2017,sobek_u-access_2006,bigonnesse_role_2018}. 
Future work should build upon our foundation by incorporating more crowdsourced and government official datasets.

While mobility aids play a crucial role in determining accessibility needs, we must recognize that individuals using the same type of mobility aid may have diverse preferences. As one of our pilot participants stated, \sayit{your wheelchair has to be shaped and fitted to your body similar to how you need shoes specifically for your feet.} This insight underscores the need for personalization beyond broad mobility aid categories. Other factors including age~\cite{rosenberg_outdoor_2013}, disability type~\cite{prescott_factors_2020}, body strength~\cite{prescott_factors_2020}, and route familiarity~\cite{kasemsuppakorn_understanding_2015} should be explored in the future. Our attempt in creating personalized maps is not to provide a one-size-fits-all solution for generalized mobility aid groups, but rather to leverage the power of defaults~\cite{nielsen_power_2005} and offer users an improved baseline from which they can easily customize based on their individual needs.

\subsection{Limitations and Future Work}
Due to the visual nature of our survey—images were the primary stimuli—we specifically excluded people who are blind or have low vision\footnote{That said, the custom online survey was made fully screen reader accessible; see \href{https://sidewalk-survey.github.io/}{https://sidewalk-survey.github.io/} for the images and alt text.}. However, as noted previously, many different disabilities can impact mobility, including sensory, physical, and cognitive. Prior research has explored the incorporation of visually impaired or blind individuals into route generation~\cite{volkel_routecheckr_2008}, recognizing shared barriers and the prevalence of multiple disabilities among users. Building upon this foundation, future work should expand the participant pool to include a broader range of disabilities, thereby providing a more comprehensive understanding of diverse accessibility needs.

While we demonstrated two basic scenario applications, our survey findings and personalized mapping approach have potential for broader implementation. One promising direction is in developing barrier removal strategies for policymakers~\cite{eisenberg_barrier-removal_2022}. Current government plans often rely on simple metrics, such as population density or proximity to public buildings~\cite{seattle_department_of_transportation_seattle_2021}. Our methodology could enhance these efforts by identifying sidewalk barriers whose removal would yield the greatest overall benefit to the largest percentage of mobility aid users in the form of connected, safe, accessible routes.

\section{Conclusion and Future Work}

%This paper presents a novel and cost-effective method for benchmarking model performance across language varieties. 
%We propose using international online review platforms as data sources to capture comments in different language varieties from similar real-world contexts.
This paper introduces a cost-effective method for benchmarking model performance across language varieties using international online reviews from similar contexts.
To validate this, we built a contextually aligned dataset of \twChinese and \cnChinese reviews and tested six LLMs on sentiment analysis, finding that LLMs consistently underperform in \twChinese. 
%Our approach demonstrates the potential for evaluating performance gaps across language varieties, with future work aiming to extend this method to more varieties for reliable benchmarking and addressing performance disparities. The ultimate goal is to develop models that perform equally well across diverse language varieties and tasks.
%\kenneth{TODO: A few sentences here to conclude the work.}
We aim to extend this approach to more language varieties, with the ultimate goal of creating LLMs that perform equally well across them.







%In the future, we aim to extend the proposed approach to more language varieties.
%, enabling reliable benchmarking and helping address performance disparities. 
%Our ultimate goal is to create LLMs that perform equally well across diverse language varieties.

\section{Limitations}
We acknowledge several limitations in our work. First, while our CA-based pipeline is effective in knowledge-driven contexts, its applicability to non-knowledge-based conversations, such as opinion-based questions (e.g., ``What would you do in such a scenario?''), remains unclear, as the subjective judgment required in these conversations can be difficult for a generated CA to capture. Additionally, although our pipeline prioritizes informativeness, follow-up questions do not always need to introduce new information \cite{kurkul2018question}---for example, requests for simpler explanations (e.g., ``Can you explain this in an easier-to-understand way?''). In the future, we hope to extend this method to support various types of follow-up questions and integrate it into downstream dialogue-based applications. 
%Moreover, in Section~\ref{sec:analysis:infogain}, we preliminarily explored using generated CAs to assess the informativeness of follow-up questions, highlighting a potential application beyond question generation. However, this approach requires further refinement and rigorous validation to ensure generalizability across domains and question types. We encourage future work to explore hybrid methods that combine human evaluation with automated CA-based informativeness assessment to reduce human effort while maintaining reliability.

% Conversational Agents (CAs) can facilitate information elicitation in various scenarios, such as semi-structured interviews. Current CAs can ask predetermined questions but lack skills for asking follow-up questions. Thus, we designed three approaches for CAs to automatically ask follow-up questions, i.e., follow-ups on concepts, follow-ups on related concepts, and general follow-ups. To investigate their effects, we conducted a user study (N=26) in which a CA interviewer asked follow-up questions generated by algorithms and crafted by human wizards. Our results showed that the CA's follow-up questions were readable and effective in information elicitation. The follow-ups on concepts and related concepts achieved a lower drop rate and better relevance, while the general follow-ups elicited more informative responses. Further qualitative analysis of the human-CA interview data revealed algorithm drawbacks and identified follow-up question techniques used by the human wizards. We provided design implications for improving information elicitation of future CAs based on the results.

% \cite{hu2024designing}

% Inquisitive questions -- open-ended, curiosity-driven questions people ask as they read -- are an integral part of discourse processing (Kehler and Rohde, 2017; Onea, 2016) and comprehension (Prince, 2004). Recent work in NLP has taken advantage of question generation capabilities of LLMs to enhance a wide range of applications. But the space of inquisitive questions is vast: many questions can be evoked from a given context. So which of those should be prioritized to find answers? Linguistic theories, unfortunately, have not yet provided an answer to this question. This paper presents QSALIENCE, a salience predictor of inquisitive questions. QSALIENCE is instruction-tuned over our dataset of linguist-annotated salience scores of 1,766 (context, question) pairs. A question scores high on salience if answering it would greatly enhance the understanding of the text (Van Rooy, 2003). We show that highly salient questions are empirically more likely to be answered in the same article, bridging potential questions (Onea, 2016) with Questions Under Discussion (Roberts, 2012). We further validate our findings by showing that answering salient questions is an indicator of summarization quality in news.

% \cite{wu2024questions}

\section{Ethics Statement}
%\cy{Still missing}
We assess that the general risks and ethical concerns of our work are no greater than those involved in using user-generated reviews to test sentiment analysis models.








\section*{Acknowledgement}
We thank the anonymous reviewers for their feedback and the participants for their contributions to our human studies. 
This work was partially supported by the 2024-2025 Seed Grant from the College of Information Sciences and Technology at Pennsylvania State University. 
We also acknowledge Dr. Janet G. van Hell, Co-PI of the seed grant, for her support and valuable input. 
Additionally, this work was partially supported by the National Science and Technology Council (NSTC), Taiwan, under the project ``\textit{Taiwan's 113th Year Endeavoring in the Promotion of a Trustworthy Generative AI Large Language Model and the Cultivation of Literacy Capabilities (Trustworthy AI Dialog Engine, TAIDE)}.''




% Bibliography entries for the entire Anthology, followed by custom entries
\bibliography{bib/custom}
% Custom bibliography entries only
%\bibliography{custom}

\appendix

\section{Dataset Examples}
\label{app:dataset-eg}
Figure \ref{fig:dataset-eg} illustrates example data instances from MemeCap, NewYorker, and YesBut.

\begin{figure*}[t]
  \includegraphics[width=\linewidth]{figures/dataset-eg.pdf} \hfill
  \caption {Dataset Examples on MemeCap, NewYorker, and YesBut.}
  \label{fig:dataset-eg}
\end{figure*}


\section{SentenceSHAP}
\label{app:sentence-shap}
In this section, we introduce SentenceSHAP, an adaptation of TokenSHAP \cite{horovicz-goldshmidt-2024-tokenshap}. While TokenSHAP calculates the importance of individual tokens, SentenceSHAP estimates the importance of individual sentences in the input prompt. The importance score is calculated using Monte Carlo Shapley Estimation, following the same principles as TokenSHAP.

Given an input prompt \( X = \{x_1, x_2, \dots, x_n\} \), where \( x_i \) represents a sentence, we generate all possible combinations of \( X \) by excluding each sentence \( x_i \) (i.e., \( X - \{x_i\} \)). Let \( Z \) represent the set of all combinations where each \( x_i \) is removed. To estimate Shapley values efficiently, we randomly sample from \( Z \) with a specified sampling ratio, resulting in a subset \( Z_s = \{X_1, X_2, \dots, X_s\} \), where each \( X_i = X - \{x_i\} \).

Next, we generate a base response \( r_0 \) using a VLM (or LLM) with the original prompt \( X \), and a set of responses \( R_s = \{r_1, r_2, \dots, r_s\} \), each generated by a prompt from one of the sampled combinations in \( Z_s \).

We then compute the cosine similarity between the base response \( r_0 \) and each response in \( R_s \) using Sentence Transformer (\texttt{BAAI/bge-large-en-v1.5}). The average similarity between combinations with and without \( x_i \) is computed, and the difference between these averages gives the Shapley value for sentence \( x_i \). This is expressed as:
\begin{align}
\notag
\phi(x_i) = \\ \notag
&\frac{1}{s} \sum_{j=1}^{s} \left( \text{cos}(r_0, r_j \mid x_i) - \text{cos}(r_0, r_j \mid \neg x_i) \right)
\end{align}
where \( \phi(x_i) \) represents the Shapley value for sentence \( x_i \), $\text{cos}(r_0, r_j \mid x_i)$ is the cosine similarity between the base response and the response that includes sentence $x_i$, $\text{cos}(r_0, r_j \mid \neg x_i)$ is the cosine similarity between the base response and the response that excludes sentence $x_i$, and $s$ is the number of sampled combinations in $Z_s$.

\section{Error Analysis Based on SentenceSHAP}
Figure \ref{fig:error-analysis} presents two examples of negative impacts from implications: dilution of focus and the introduction of irrelevant information.
\label{app:error-analysis-shap}
\begin{figure*}[t]
  \includegraphics[width=\linewidth]{figures/error-analysis.pdf} \hfill
  \caption {Examples of negative impact from implications from Phi (top) and GPT4o (bottom).}
  \label{fig:error-analysis}
\end{figure*}

\section{Details on human anntations}
\label{app:cloudresearch}
We present the annotation interface on CloudResearch used for human evaluation to validate our evaluation metric in Figure \ref{fig:cloud-research}. Refer to Sec.~\ref{sec:ethics} for details on annotator selection criteria and compensation.

\begin{figure*}[t]
  \includegraphics[width=\linewidth]{figures/cloud-research.pdf} \hfill
  \caption {Annotation interface on CloudResearch used for human evaluation to validate our evaluation metric.}
  \label{fig:cloud-research}
\end{figure*}



\section{Generation Prompts for Selection and Refinement}
\label{app:gen-prompts}
Figures \ref{fig:desc-prompt}, \ref{fig:seed-imp-prompt}, and \ref{fig:nonseed-imp-prompt} show the prompts used for generating image descriptions, seed implications (1st hop), and non-seed implications (2nd hop onward). Figure \ref{fig:cand-prompt} displays the prompt used to generate candidate and final explanations. Image descriptions are used for candidate explanations when existing data is insufficient but are not used for final explanations. For calculating Cross Entropy values (used as a relevance term), we use the prompt in Figure \ref{fig:cand-prompt}, substituting the image with image descriptions, as LLM is used to calculate the cross entropies.

\begin{figure*}[h]
\small
\begin{tcolorbox}[
    title=Prompt for Image Descriptions,
    colback=white,
    colframe=CadetBlue,
    arc=0pt,        % Remove rounded corners
    outer arc=0pt   % Remove outer rounded corners (important for some styles)    
]

Describe the image by focusing on the noun phrases that highlight the actions, expressions, and interactions of the main visible objects, facial expressions, and people.\\
\\
Here are some guidelines when generating image descriptions:\\
* Provide specific and detailed references to the objects, their actions, and expressions. Avoid using pronouns in the description.\\
* Do not include trivial details such as artist signatures, autographs, copyright marks, or any unrelated background information.\\
* Focus only on elements that directly contribute to the meaning, context, or main action of the scene.\\
* If you are unsure about any object, action, or expression, do not make guesses or generate made-up elements.\\
* Write each sentence on a new line.\\
* Limit the description to a maximum of 5 sentences, with each focusing on a distinct and relevant aspect that directly contribute to the meaning, context, or main action of the scene.\\
\\
Here are some examples of desired output:
---\\
\text{[Description]} (example of newyorker cartoon image):\\
Through a window, two women with surprised expressions gaze at a snowman with human arms.\\
---\\
\text{[Description]} (example of newyorker cartoon image):\\
A man and a woman are in a room with a regular looking bookshelf and regular sized books on the wall.\\
In the middle of the room the man is pointing to text written on a giant open book which covers the entire floor.\\
He is talking while the woman with worried expression watches from the doorway.\\
---\\
\text{[Description]} (example of meme):\\
The left side shows a woman angrily pointing with a distressed expression, yelling ``You said memes would work!''.\\
The right side shows a white cat sitting at a table with a plate of food in front of it, looking indifferent or smug with the text above the cat reads, ``I said good memes would work''.\\
---\\
\text{[Description]} (example of yesbut image):\\
The left side shows a hand holding a blue plane ticket marked with a price of ``\$50'', featuring an airplane icon and a barcode, indicating it's a flight ticket.\\
The right side shows a hand holding a smartphone displaying a taxi app, showing a route map labeled ``Airport'' and a price of ``\$65''.\\
---\\

Proceed to generate the description.\\
\text{[Description]}:

\end{tcolorbox}
\caption{A prompt used to generate image descriptions.} % Add a caption to the figure
\label{fig:desc-prompt}
\end{figure*}


%%%%%%%%%%%%%%%%%%%%%%%%%%% Prompt for implications %%%%%%%%%%%%%%%%%%%%%%%%%%%
\begin{figure*}[t]
\small
\begin{tcolorbox}[
    title=Prompt for Seed Implications,
    colback=white,
    colframe=Green,
    arc=0pt,        % Remove rounded corners
    outer arc=0pt,  % Remove outer rounded corners (important for some styles)    
    % breakable,
]

You are provided with the following inputs:\\
- \text{[}Image\text{]}: An image (e.g. meme, new yorker cartoon, yes-but image)\\
- \text{[}Caption\text{]}: A caption written by a human.\\
- \text{[}Descriptions\text{]}: Literal descriptions that detail the image.\\
\\
\#\#\# Your Task:\\
\texttt{[ One-sentence description of the ultimate goal of your task. Customize based on the task. ]}\\
Infer implicit meanings, cultural references, commonsense knowledge, social norms, or contrasts that connect the caption to the described objects, concepts, situations, or facial expressions.\\
\\
\#\#\# Guidelines:\\
- If you are unsure about any details in the caption, description, or implication, refer to the original image for clarification.\\
- Identify connections between the objects, actions, or concepts described in the inputs.\\
- Explore possible interpretations, contrasts, or relationships that arise naturally from the scene, while staying grounded in the provided details.\\
- Avoid repeating or rephrasing existing implications. Ensure each new implication introduces fresh insights or perspectives.\\
- Each implication should be concise (one sentence) and avoid being overly generic or vague.\\
- Be specific in making connections, ensuring they align with the details provided in the caption and descriptions.\\
- Generate up to 3 meaningful implications.\\
\\
\#\#\# Example Outputs:\\
\#\#\#\# Example 1 (example of newyorker cartoon image):\\
\text{[}Caption\text{]}: ``This is the most advanced case of Surrealism I've seen.''\\
\text{[}Descriptions\text{]}: A body in three parts is on an exam table in a doctor's office with the body's arms crossed as though annoyed.\\
\text{[}Connections\text{]}:\\
1. The dismembered body is illogical and impossible, much like Surrealist art, which often explores the absurd.\\
2. The body’s angry posture adds a human emotion to an otherwise bizarre scenario, highlighting the strange contrast.\\
\\
\#\#\#\# Example 2 (example of newyorker cartoon image):\\
\text{[}Caption\text{]}: ``He has a summer job as a scarecrow.''\\
\text{[}Descriptions\text{]}: A snowman with human arms stands in a field.\\
\text{[}Connections\text{]}:\\
1. The snowman, an emblem of winter, represents something out of place in a summer setting, much like a scarecrow's seasonal function.\\
2. The human arms on the snowman suggest that the role of a scarecrow is being played by something unexpected and seasonal.\\
\\
\#\#\#\# Example 3 (example of yesbut image):\\
\text{[}Caption\text{]}: ``The left side shows a hand holding a blue plane ticket marked with a price of `\$50'.''\\
\text{[}Descriptions\text{]}: The screen on the right side shows a route map labeled ``Airport'' and a price of `\$65'.\\
\text{[}Connections\text{]}:\\
1. The discrepancy between the ticket price and the taxi fare highlights the often-overlooked costs of travel beyond just booking a flight.\\
2. The image shows the hidden costs of air travel, with the extra fare representing the added complexity of budgeting for transportation.\\
\\
\#\#\#\# Example 4 (example of meme):\\
\text{[}Caption\text{]}: ``You said memes would work!''\\
\text{[}Descriptions\text{]}: A cat smirks with the text ``I said good memes would work.''\\
\text{[}Connections\text{]}:\\
1. The woman's frustration reflects a common tendency to blame concepts (memes) instead of the quality of execution, as implied by the cat’s response.\\
2. The contrast between the angry human and the smug cat highlights how people often misinterpret success as simple, rather than a matter of quality.\\
\\
\#\#\# Now, proceed to generate output:\\
\text{[}Caption\text{]}: \texttt{[ Caption ]}\\
\\
\text{[}Descriptions\text{]}:\\
\texttt{[ Descriptions ]}\\
\\
\text{[}Connections\text{]}:

\end{tcolorbox}
\caption{A prompt used to generate seed implications.} % Add a caption to the figure
\label{fig:seed-imp-prompt}
\end{figure*}


%%%%%%%%%%%%%%%%%%%%%%%%%%% Prompt for nonseed implications %%%%%%%%%%%%%%%%%%%%%%%%%%%
\begin{figure*}[t]
\small
%  \begin{tcolorbox}[
%  width=\textwidth,
%  colback={white},
%  title={Title},
%  colbacktitle={DarkGreen},
%  coltitle=white,
%  colframe={DarkGreen},
%  breakable
% ]
 % \parskip=5pt

\begin{tcolorbox}[
    % breakable,
    title=Prompt for Non-Seed Implications (2nd hop onward),
    colback=white,
    colframe=Green,
    arc=0pt,        % Remove rounded corners
    outer arc=0pt,  % Remove outer rounded corners (important for some styles)    
    % breakable,
]

You are provided with the following inputs:\\
- \text{[}Image\text{]}: An image (e.g. meme, new yorker cartoon, yes-but image)\\
- \text{[}Caption\text{]}: A caption written by a human.\\
- \text{[}Descriptions\text{]}: Literal descriptions that detail the image.\\
- \text{[}Implication\text{]}: A previously generated implication that suggests a possible connection between the objects or concepts in the caption and description.\\
\\
\#\#\# Your Task:\\
\texttt{[ One-sentence description of the ultimate goal of your task. Customize based on the task. ]}\\
Infer implicit meanings across the objects, concepts, situations, or facial expressions found in the caption, description, and implication. Focus on identifying relevant commonsense knowledge, social norms, or underlying connections.\\
\\
\#\#\# Guidelines:\\
- If you are unsure about any details in the caption, description, or implication, refer to the original image for clarification.\\
- Identify potential connections between the objects, actions, or concepts described in the inputs.\\
- Explore interpretations, contrasts, or relationships that naturally arise from the scene while remaining grounded in the inputs.\\
- Avoid repeating or rephrasing existing implications. Ensure each new implication provides fresh insights or perspectives.\\
- Each implication should be concise (one sentence) and avoid overly generic or vague statements.\\
- Be specific in the connections you make, ensuring they align closely with the details provided.\\
- Generate up to 3 meaningful implications that expand on the implicit meaning of the scene.\\
\\
\#\#\# Example Outputs:\\
\#\#\#\# Example 1 (example of newyorker cartoon image):\\
\text{[}Caption\text{]}: "This is the most advanced case of Surrealism I've seen."\\
\text{[}Descriptions\text{]}: A body in three parts is on an exam table in a doctor's office with the body's arms crossed as though annoyed.\\
\text{[}Implication\text{]}: Surrealism is an art style that emphasizes strange, impossible, or unsettling scenes.\\
\text{[}Connections\text{]}:\\
1. A body in three parts creates an unsettling juxtaposition with the clinical setting, which aligns with Surrealist themes.\\
2. The body’s crossed arms add humor by assigning human emotion to an impossible scenario, reflecting Surrealist absurdity.\\
... \\
\texttt{[ We used sample examples from the prompt for generating seed implications (see Figure \ref{fig:seed-imp-prompt}), following the above format, which includes [Implication]:. ]}
\\
---\\
\\
\#\#\# Proceed to Generate Output:\\
\text{[}Caption\text{]}: \texttt{[ Caption ]}\\
\\
\text{[}Descriptions\text{]}:\\
\texttt{[ Descriptions ]}\\
\\
\text{[}Implication\text{]}:\\
\texttt{[ Implication ]}\\
\\
\text{[}Connections\text{]}:
\end{tcolorbox}
\caption{A prompt used to generate non-seed implications.} % Add a caption to the figure
\label{fig:nonseed-imp-prompt}
\end{figure*}


%%%%%%%%%%%%%%%%%%%%%%%%%%% Prompt for nonseed implications %%%%%%%%%%%%%%%%%%%%%%%%%%%
\begin{figure*}[t]
\small
%  \begin{tcolorbox}[
%  width=\textwidth,
%  colback={white},
%  title={Title},
%  colbacktitle={DarkGreen},
%  coltitle=white,
%  colframe={DarkGreen},
%  breakable
% ]
 % \parskip=5pt

\begin{tcolorbox}[
    % breakable,
    title=Prompt for Candidate and Final Explanations,
    colback=white,
    colframe=RedViolet,
    arc=0pt,        % Remove rounded corners
    outer arc=0pt,  % Remove outer rounded corners (important for some styles)    
    % breakable,
]

You are provided with the following inputs:\\
- **\text{[}Image\text{]}:** A New Yorker cartoon image.\\
- **\text{[}Caption\text{]}:** A caption written by a human to accompany the image.\\
- **\text{[}Image Descriptions\text{]}:** Literal descriptions of the visual elements in the image.\\
- **\text{[}Implications\text{]}:** Possible connections or relationships between objects, concepts, or the caption and the image.\\
- **\text{[}Candidate Answers\text{]}:** Example answers generated in a previous step to provide guidance and context.\\
\\
\#\#\# Your Task:\\
Generate **one concise, specific explanation** that clearly captures why the caption is funny in the context of the image. Your explanation must provide detailed justification and address how the humor arises from the interplay of the caption, image, and associated norms or expectations.\\
\\
\#\#\# Guidelines for Generating Your Explanation:\\
1. **Clarity and Specificity:**  \\
   - Avoid generic or ambiguous phrases.  \\
   - Provide specific details that connect the roles, contexts, or expectations associated with the elements in the image and its caption.  \\
\\
2. **Explain the Humor:**  \\
- Clearly connect the humor to the caption, image, and any cultural, social, or situational norms being subverted or referenced.  \\
- Highlight why the combination of these elements creates an unexpected or amusing contrast.\\
\\
3. **Prioritize Clarity Over Brevity:**  \\
- Justify the humor by explaining all important components clearly and in detail.  \\
- Aim to keep your response concise and under 150 words while ensuring no critical details are omitted.  \\
\\
4. **Use Additional Inputs Effectively:**\\
- **\text{[}Image Descriptions\text{]}:** Provide a foundation for understanding the visual elements."   \\
- **\text{[}Implications\text{]}:** Assist in understanding relationships and connections but do not allow them to dominate or significantly alter the central idea.\\
- **\text{[}Candidate Answers\text{]}:** Adapt your reasoning by leveraging strengths or improving upon weaknesses in the candidate answers.\\
\\
Now, proceed to generate your response based on the provided inputs.\\
\\
\#\#\# Inputs:\\
\text{[}Caption\text{]}: \texttt{\text{[} Caption \text{]}}\\
\\
\text{[}Descriptions\text{]}:\\
\texttt{\text{[} Top-K Implications \text{]}}\\
\\
\text{[}Implications\text{]}:\\
\texttt{\text{[} Top-K Implications \text{]}}\\
\\
\text{[}Candidate Anwers\text{]}:\\
\texttt{\text{[} Top-K Candidate Explanations \text{]}}\\
\\
\text{[}Output\text{]}:\\

\end{tcolorbox}
\caption{A prompt used to generate candidate and final explanations.} % Add a caption to the figure
\label{fig:cand-prompt}
\end{figure*}


\section{Evaluation Prompts}
\label{app:eval-prompts}
Figures \ref{fig:recall-prompt} and \ref{fig:precision-prompt} present the prompts used to calculate recall and precision scores in our LLM-based evaluation, respectively.

%%%%%%%%%%%%%%%%%%%%%%%%%%% Prompt for nonseed implications %%%%%%%%%%%%%%%%%%%%%%%%%%%
\begin{figure*}[t]
\small
\begin{tcolorbox}[
    % breakable,
    title=Prompt for Evaluating Recall Score,
    colback=white,
    colframe=MidnightBlue,
    arc=0pt,        % Remove rounded corners
    outer arc=0pt,  % Remove outer rounded corners (important for some styles)    
    % breakable,
]

Your task is to assess whether \text{[}Sentence1\text{]} is conveyed in \text{[}Sentence2\text{]}. \text{[}Sentence2\text{]} may consist of multiple sentences.\\
\\
Here are the evaluation guidelines:\\
1. Mark 'Yes' if \text{[}Sentence1\text{]} is conveyed in \text{[}Sentence2\text{]}.\\
2. Mark 'No' if \text{[}Sentence2\text{]} does not convey the information in \text{[}Sentence1\text{]}.\\
\\
Proceed to evaluate. \\
\\
\text{[}Sentence1\text{]}: \texttt{[ One Atomic Sentence from Decomposed Reference Explanation ]} \\
\\
\text{[}Sentence2\text{]}: \texttt{[ Predicted Explanation ]}\\
\\
\text{[}Output\text{]}:

\end{tcolorbox}
\caption{Prompt for evaluating recall score.} % Add a caption to the figure
\label{fig:recall-prompt}
\end{figure*}


\begin{figure*}[t]
\small
\begin{tcolorbox}[
    % breakable,
    title=Prompt for Evaluating Precision Score,
    colback=white,
    colframe=MidnightBlue,
    arc=0pt,        % Remove rounded corners
    outer arc=0pt,  % Remove outer rounded corners (important for some styles)    
    % breakable,
]

Your task is to assess whether \text{[}Sentence1\text{]} is inferable from \text{[}Sentence2\text{]}. \text{[}Sentence2\text{]} may consist of multiple sentences.\\
\\
Here are the evaluation guidelines:\\
1. Mark "Yes" if \text{[}Sentence1\text{]} can be inferred from \text{[}Sentence2\text{]} — whether explicitly stated, implicitly conveyed, reworded, or serving as supporting information.\\
2. Mark 'No' if \text{[}Sentence1\text{]} is absent from \text{[}Sentence2\text{]}, cannot be inferred, or contradicts it.\\
\\
Proceed to evaluate. \\
\\
\text{[}Sentence1\text{]}: \texttt{[ One Atomic Sentence from Decomposed Predicted Explanation ]}\\
\\
\text{[}Sentence2\text{]}: \texttt{[ Reference Explanation ]}\\
\\
\text{[}Output\text{]}:


\end{tcolorbox}
\caption{Prompt for evaluating precision score.} % Add a caption to the figure
\label{fig:precision-prompt}
\end{figure*}

\section{Prompts for Baselines}
\label{app:base-prompts}

Figure \ref{fig:base-prompt} presents the prompt used for the ZS, CoT, and SR Generator methods. While the format remains largely the same, we adjust it based on the baseline being tested (e.g., CoT requires generating intermediate reasoning, so we add extra instructions for that).
Figure \ref{fig:critic-prompt} shows the prompt used in the SR critic model. The critic's criteria include: (1) \textit{correctness}, measuring whether the explanation directly addresses why the caption is humorous in relation to the image and its caption; (2) \textit{soundness}, evaluating whether the explanation provides a well-reasoned interpretation of the humor; (3) \textit{completeness}, ensuring all important aspects in the caption and image contributing to the humor are considered; (4) \textit{faithfulness}, verifying that the explanation is factually consistency with the image and caption; and (5) \textit{clarity}, ensuring the explanation is clear, concise, and free from unnecessary ambiguity.
\begin{figure*}
\small
\begin{tcolorbox}[
    % breakable,
    title=Prompt for Baselines,
    colback=white,
    colframe=Black,
    arc=0pt,        % Remove rounded corners
    outer arc=0pt,  % Remove outer rounded corners (important for some styles)    
    % breakable,
]

You are provided with the following inputs:\\
- **\text{[}Image\text{]}:** A New Yorker cartoon image.\\
- **\text{[}Caption\text{]}:** A caption written by a human to accompany the image.\\
\texttt{[ if Self-Refine with Critic is True: ]} \\
- **\text{[}Feedback for Candidate Answer\text{]}:** Feedback that points out some weakness in the current candidate responses.\\
\texttt{[ if Self-Refine is True: ]} \\
- **\text{[}Candidate Answers\text{]}:** Example answers generated in a previous step to provide guidance and context.\\
\\
\#\#\# Your Task:\\
Generate **one concise, specific explanation** that clearly captures why the caption is funny in the context of the image. Your explanation must provide detailed justification and address how the humor arises from the interplay of the caption, image, and associated norms or expectations.\\
\\
\#\#\# Guidelines for Generating Your Explanation:\\
1. **Clarity and Specificity:**  \\
   - Avoid generic or ambiguous phrases.  \\
   - Provide specific details that connect the roles, contexts, or expectations associated with the elements in the image and its caption.  \\
\\
2. **Explain the Humor:**  \\
- Clearly connect the humor to the caption, image, and any cultural, social, or situational norms being subverted or referenced.  \\
- Highlight why the combination of these elements creates an unexpected or amusing contrast.\\
\\
3. **Prioritize Clarity Over Brevity:**  \\
- Justify the humor by explaining all important components clearly and in detail.  \\
- Aim to keep your response concise and under 150 words while ensuring no critical details are omitted.  \\
\\
\texttt{[ if Self-Refine is True: ]}\\
4. **Use Additional Inputs Effectively:**\\
- **[Candidate Answers]:** Adapt your reasoning by leveraging strengths or improving upon weaknesses in candidate answers. \\
\texttt{[ if Self-Refine with Critic is True: ]}\\
- **[Feedback for Candidate Answer]:** Feedback that points out some weaknesses in the current candidate responses.\\
\\
\texttt{ [ if CoT is True: ]} \\
Begin by analyzing the image and the given context, and explain your reasoning briefly before generating your final response. \\
\\
Here is an example format of the output: \\
\{\{ \\
    "Reasoning": "...", \\
    "Explanation": "..."   \\
\}\} \\

Now, proceed to generate your response based on the provided inputs.\\
\\
\#\#\# Inputs:\\
\text{[}Caption\text{]}: \texttt{\text{[} Caption \text{]}}\\
\\
\text{[}Candidate Answers\text{]}: \texttt{\text{[} Candidate Explanations \text{]}}\\
\\
\text{[}[Feedback for Candidate Answer]:\text{]}: \texttt{\text{[} Feedback for Candidate Explanations \text{]}}\\
\\
\text{[}Output\text{]}:\\

\end{tcolorbox}
\caption{A prompt used for baseline methods, with conditions added based on the specific baseline being experimented with.} % Add a caption to the figure
\label{fig:base-prompt}
\end{figure*}


\begin{figure*}
\small
\begin{tcolorbox}[
    % breakable,
    title=Prompt for Self-Refine Critic,
    colback=white,
    colframe=Black,
    arc=0pt,        % Remove rounded corners
    outer arc=0pt,  % Remove outer rounded corners (important for some styles)    
    % breakable,
]
\texttt{[ Customize goal text here: ]} \\
\texttt{MemeCap:} You will be given a meme along with its caption, and a candidate response that describes what meme poster is trying to convey. \\
\texttt{NewYorker:} You will be given an image along with its caption, and a candidate response that explains why the caption is funny for the given image. \\
\texttt{YesBut:} You will be given an image and a candidate response that describes why the image is funny or satirical. \\
\\
Your task is to criticize the candidate response based on the following evaluation criteria: \\
- Correctness: Does the explanation directly address why the caption is funny, considering both the image and its caption? \\
- Soundness: Does the explanation provide a meaningful and well-reasoned interpretation of the humor? \\
- Completeness: Does the explanation address all relevant aspects of the caption and image (e.g., visual details, text) that contribute to the humor? \\
- Faithfulness: Is the explanation factually consistent with the details in the image and caption? \\
- Clarity: Is the explanation clear, concise, and free from unnecessary ambiguity? \\
 \\
Proceed to criticize the candidate response ideally using less than 5 sentences:\\
\\
\text{[}Caption\text{]}: \texttt{[ caption ]}\\
\\
\text{[}Candidate Response\text{]}: \\
 \texttt{\text{[} Candidate Response \text{]}}\\
\\
\text{[}Output\text{]}: \\
\end{tcolorbox}
\caption{A prompt used in SR critic model.} % Add a caption to the figure
\label{fig:critic-prompt}
\end{figure*}

% \begin{figure*}[t]
%   \includegraphics[width=\linewidth]{figures/error-analysis.pdf} \hfill
%   \vspace{-20pt}
%   \caption {Examples of negative impact from implications from Phi (top) and GPT4o (bottom).}
%   \label{fig:error-analysis}
% \end{figure*}



\end{document}
