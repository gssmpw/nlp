

\paragraph{Do users who self-label as being from Taiwan always use \twChinese?}
In this study, we use users' self-reported nationality/region to infer whether they are speakers of \twChinese or \cnChinese. 
The convention is that \twChinese employs traditional Chinese characters, while \cnChinese uses simplified characters. 
However, analysis using predefined character sets revealed that 30.99\% of samples in the CN group contained characters beyond simplified Chinese, and 25.26\% of samples in TW group included characters not limited to traditional Chinese. 
This suggests that the relationship between self-reported nationality/region, language variety, and character usage is more complex in real-world data.
In \Cref{appendix:language-analysis}, \Cref{tab:language-distribution} shows the distribution of Chinese script variants among users.
%\footnote{\texttt{hanzidentifier}: \url{https://github.com/tsroten/hanzidentifier}}

%----------------- dead kitten -------------

\begin{comment}
 
%\kenneth{TODO CY: Add some discussion here.}
%By analyzing sentiment prediction of hotel reviews among two different varieties of Mandarin users (\twChinese vs. \cnChinese), we observed a systematic lower performance toward minor language users, which is \twChinese in our case. 
%This result aligned with several previous studies towards minor language varieties \cite{valuebench, multivaluebench}, showing that LLMs exhibit performance deficits when dealing with "non-standard" language inputs, even in some high-resourced languages.

%Compared to original users' generated reviews, combined data inputs such as plain or shuffled text reviews exhibit more robust and consistent performance differences between language varieties. One possibility is that several reviews, especially short reviews, contain all the information in one of the input fields rather than organized in all three fields. Such input patterns may introduce difficulties for LLMs in parsing and capturing correct sentiment elements.

In this study, we use the users' self-reported nationality/region to infer whether they are speakers of \twChinese or \cnChinese.
However, upon applying a Chinese variation detection tool\footnote{We used \texttt{hanzidentifier}: \url{https://github.com/tsroten/hanzidentifier}},
we discovered that the assumption that Taiwan users exclusively use traditional Chinese and Mainland China users use simplified Chinese does not always hold true.

\Cref{tab:language-distribution} shows the distribution of Chinese script variants among users.
Notably, 15.72\% of samples in the cn group do not consist purely of simplified Chinese, while only 3.86\% of samples in the tw group deviate from traditional Chinese.
While the precise impact of this variation is yet to be determined, our main findings suggest that if we were to refine the data by enforcing strict language constraints,
the performance gap observed in \Cref{tab:results-by-length} might actually widen.
This is because 10.66\% of samples in the cn group use traditional Chinese, for which our model demonstrates lower performance.

   
\end{comment}
