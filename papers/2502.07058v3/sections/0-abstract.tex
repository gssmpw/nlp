A language can have different varieties. 
These varieties can affect the performance of natural language processing (NLP) models, including large language models (LLMs), which are often trained on data from widely spoken varieties.
This paper introduces a novel and cost-effective approach to benchmark model performance across language varieties.
We argue that international online review platforms, such as Booking.com, can serve as effective data sources for constructing datasets that capture \textbf{comments in different language varieties from similar real-world scenarios}, like reviews for the same hotel with the same rating using the same language (\eg, Mandarin Chinese) but different language varieties (\eg, \twChinese, \cnChinese).  
To prove this concept, we constructed a \textbf{contextually aligned} dataset comprising reviews in \twChinese and \cnChinese and tested six LLMs in a sentiment analysis task. 
Our results show that LLMs consistently underperform in \twChinese. %demonstrating the feasibility of our approach. %for evaluating performance gaps across language varieties. 