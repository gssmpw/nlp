% Google Sheet is here: https://docs.google.com/spreadsheets/d/1kLe5e_I4hCsyASmnrVgtOnTQlCswiDwc-JOLCiXNoWY/edit?usp=sharing

% Should also cite these: https://x.com/zraytam/status/1842514965821087853
Beyond machine translation~\cite{kantharuban2023quantifying},
researchers tried to benchmark NLP models across language varieties~\cite{Zampieri_Nakov_Scherrer_2020,joshi2024natural,blodgett2020biasreview,hovy-johannsen-2016-exploring,vardial2019report}, but the focus on identifying gaps between these varieties varies widely.
Some prior work focused solely on a single less-representative variety, such as \twChinese~\cite{tamtmmlu+,chen2024measuring}, without measuring performance gaps across multiple varieties.
Other studies that measured these gaps employed different levels of granularity.
The most common approach, \textbf{task-level comparison}, benchmarks the same NLP task across language varieties~\cite{dialectbench}, such as sentiment analysis, but datasets often differ in source or genre across varieties, making the reported performance numbers not directly comparable. 
For instance, sentiment analysis datasets for \cnChinese and \twChinese often used different sources~\cite{seki2007overview}. %leading to incomparable results despite using the same label scheme.
A more refined approach, \textbf{scenario-level comparison}, evaluates performance within the same dataset or scenario, 
such as essay grading~\cite{liang2023gptesl} or speech rating~\cite{kwako2023bertbias},
across data partitions of different language varieties~\cite{lwowski2022measuring,AAE2017racial}. 
While this method eliminates biases caused by differing data sources, it cannot fully address biases introduced during dataset construction.
%, such as cultural or population differences among language varieties. 
The most rigorous method, \textbf{instance-level comparison}, involves constructing parallel datasets with an item-by-item alignment between varieties~\cite{valuebench,multivaluebench,groenwold2020AAENLP,macucocorpus}, where each instance is converted between language varieties. 
However, creating such comparisons is very costly, requiring native speakers and language experts to ensure accuracy. 
Our approach achieves instance-level comparability with lower costs.






%\cite{kantharuban2023quantifying}: This study investigates the performance difference regarding multiple dialectal language inputs on two sets of tasks: machine translation and automatic speech recognition. In the MT task, most LLMs experienced a performance drop when dealing with dialectal input compared to the "standard" variation of language input, indicating that the dialect gap exists in the current LLM/NLP mechanism. This paper further found that training data's impact is significant but inconsistent across models and dialect variations, indicating that the solution to mitigating such a gap is not universal.
%\textbf{Methods}: parallel corpus of each language for MT tasks. The performance gap is measured using BLEU score on the generation product.


%\cite{joshi2024natural}

%\subsection{Measuring NLP models' performance gaps across Language Variations}
%%\kenneth{Did people compare NLP models' performances across language variations? How? What did they find? Why can (or can not) their methods be used in our case?}

%\zixin{Here are several existing benchmarks specifically for language variation/diversity}
%\kenneth{Cool. Say a few words about what these benchmarks did? How do they compare model performance across variations?}
%\zixin{What about tasks directly focus on detecting language variation? i.e. dialect detection and identification? I just separated those papers from other benches so we can discuss them later.}\kenneth{In our setting, we kinda KNOW the variation of each data item, as Booking.com has users' self-reported nationality/region. So, variation detection is a separate problem-- it might be useful when, say, your data don't tell you the variations, so you need to detect them yourself.}

%\zixin{I found this 2020 ACL paper discussing several "missing parts" NLP studies on "biases", including related work outside of NLP tasks and communities. It might be worth taking a look}
%\cite{blodgett2020biasreview}


%General NLP benchmarks for language variations/diversity: DIALECTBENCH~\cite{dialectbench}, VALUE (Specific for African American English~\cite{valuebench}), Macuco corpus \cite{macucocorpus}


%\paragraph{CAMP 1: Compared or aligned using NLP tasks.}
%DIALECTBENCH~\cite{dialectbench}:
%Collect a set of languages and put them in the same leaderboard





%\kenneth{Bad: It's not really a fair or even comparable comparison.}

%\paragraph{CAMP 2: Compared or aligned using data (direct conversion at the instance level).}

%VALUE~\cite{valuebench}:
%Use rule-based approaches to convert MAE to AAE and eval on AAE. The conversation was validated by humans!

%Multi-VALUE~\cite{multivaluebench}:
%Extended version of VALUE to 50 dialects with 189 linguistic features, and synthetic mapping to SAE. Native speakers verified several dialects.

%\kenneth{Bad: Conversion is hard or expensive}








%\cite{hovy-johannsen-2016-exploring}

%\kenneth{This is one paper I found, which has a whole ``Language Variation'' subsection. Maybe take a look:}
%\cite{lwowski2022measuring}

%\kenneth{Did people compare NLP models' performances across different languages? How? What did they find? Why can (or can not) their methods be used in our case?}


%\kenneth{Now when I think about it, we can also learn some methods from people who work on biases in AI models on how they compare AI models' performances gap across different user populations}

%\paragraph{scenario-based studies for biases in AI models regarding language variations}

%Case study for AAE tweets~\cite{AAE2017racial}: This case study investigated models' performance gap on language identification tasks regarding AAE and SAE tweets. Results showed gaps in identifying AAE short tweets as English, indicating potential biases against minority language users in large-scale models training on major language variation.

%\textbf{Method:} Language identification tasks (whether it is an English tweet) on AAE and SAE tweets

%\citeauthor{groenwold2020AAENLP}: Using parallel AAVE/SAE tweet prompts and GPT-2, researchers found a consistent disparity in AAVE-like generation, including the sentiment analysis scores, benchmark scores on BLUE and ROUGE, and human-based evaluation on generated text quality. 

%\textbf{Method:} stat. analyses on benchmark scores (sentiment, BLUE, ROUGE) and human evaluations regarding NLG output.

%\citeauthor{liang2023gptesl}: Compared to essays from 8th-grade native speakers, non-native English essays have a higher chance of being classified as "AI-generated" by GPT-based detectors (ChatGPT in this case). Such issues can be reduced by enhancing word choices in the essays, indicating that such a biased pattern is related to linguistic diversity in language production.

%\textbf{Method:} Ratio of "AI-generated" labels regarding non-native essays and essays from 8th-grade students.

%\citeauthor{kwako2023bertbias}: This study investigated BERT's performance of rating transcripts of non-native speakers' speeches, and whether BERT can perform similar patterns as humans when conducting the same tasks. Results showed that BERT may exacerbate the difference against different non-native speakers groups, indicating potentially stronger implicit biases in BERT. Such rating differences are larger if the speaker is older or speaks longer. 

%\textbf{Method:} auto-grading system based on off-the-shelf BERT(so zero-shot I guess?), and the differences between humans' and BERT's scores are represented as differential item functioning (DIF)


%Macuco corpus~\cite{macucocorpus}:
%A parallel data for multiple English variations in Europe. Includes only variation identification tasks.


%\subsection{Chinese Sentiment Analysis Datasets}
%%\kenneth{Maybe need to mention the efforts of creating localized LLMs for China and Taiwan, respectively.}

%\kenneth{Discuss how some shared tasks, e.g., NTCIR-6, sometimes have BOTH China and Taiwan Mandarin subset. Like this: https://research.nii.ac.jp/ntcir/permission/ntcir-6/perm-en-OPINION.html}

%\kenneth{Maybe discuss the shared tasks in SIGHAN: https://dimabsa2024.github.io/}

%\paragraph{Related work on Chinese NLP: Fine-tuning dataset}

%COIG-CQIA ~\cite{bai2024coig}:
%A large fine-tuning instruction dataset for LLMs, contains various types of data including Wikis, examinations, professional documents, and online Q\&A community threads. However, most of the data comes from mainland China, which might introduce biases against other great Chinese region users.


%\paragraph{Direct NLP tasks/datasets for identifying and detecting language variation}

%2019 VarDial Evaluation Campaign~\cite{vardial2019report}:
%Dataset for various types of language variation NLP tasks, including dialect identification and cross-lingual morphological analysis tasks. It also includes a dataset for discriminating Taiwan and Mainland variations of Chinese under the same scripts/orthography system. 
%\zixin{Can we borrow similar ideas of using the same writing script system for both language variations?}\kenneth{When we justify our method of comparing TW Mandarin and China Mandarin, yes? I'd move this to Chinese NLP subsection though.}






