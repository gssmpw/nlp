%\kenneth{Maybe need to mention the efforts of creating localized LLMs for China and Taiwan, respectively.}

%\kenneth{Discuss how some shared tasks, e.g., NTCIR-6, sometimes have BOTH China and Taiwan Mandarin subset. Like this: https://research.nii.ac.jp/ntcir/permission/ntcir-6/perm-en-OPINION.html}

%\kenneth{Maybe discuss the shared tasks in SIGHAN: https://dimabsa2024.github.io/}

%\paragraph{Related work on Chinese NLP: Fine-tuning dataset}

%COIG-CQIA ~\cite{bai2024coig}:
%A large fine-tuning instruction dataset for LLMs, contains various types of data including Wikis, examinations, professional documents, and online Q\&A community threads. However, most of the data comes from mainland China, which might introduce biases against other great Chinese region users.


%\paragraph{Direct NLP tasks/datasets for identifying and detecting language variation}

%2019 VarDial Evaluation Campaign~\cite{vardial2019report}:
%Dataset for various types of language variation NLP tasks, including dialect identification and cross-lingual morphological analysis tasks. It also includes a dataset for discriminating Taiwan and Mainland variations of Chinese under the same scripts/orthography system. 
%\zixin{Can we borrow similar ideas of using the same writing script system for both language variations?}\kenneth{When we justify our method of comparing TW Mandarin and China Mandarin, yes? I'd move this to Chinese NLP subsection though.}


