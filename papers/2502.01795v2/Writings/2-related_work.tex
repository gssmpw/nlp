\section{Related Work}
As we study practices of artists making robotic art with the intent to inform ways of facilitating creativity, we review literature about robotic art as a case of creative practice to probe the factors that contribute to creativity in the artwork. To justify our contributions as learning from the art to inform design, we review artistic practices in HCI and their unique benefits to the field.

\subsection{Robots in Creative Contexts}

Introducing robots into creative domains provides a new research medium through robots' embodied materiality and interactions \cite{Nitsche2019craft}. Robots applied in creative tasks have been treated not only as executors following a set of predetermined rules and executing commands but also as social actors that can respond to dynamic human behaviors and environments~\cite{nam2023dreams}. For example, robot collaborative drawing highlights robot's role as a tool, to mediate and engage the spatial arrangement, physical activities, and social interactions of people in the co-creation process through robotic motions and creations~\cite{hoggenmueller2020woodie}.  The collaboration between humans and robots in choreography provides a lens to explore how human movement can be reinterpreted and fused by robots, thus inspiring more creative expression~\cite{saviano2024cage, hsueh2019understanding}. In this context, robots function not merely as tools but as partners that contribute to improvisation~\cite{rogel2022music}. Th{\"o}rn et al.~\cite{thorn2020human} examined the potential of robots as dance partners, emphasizing the importance of synchronization of robotic movement and emotional expression in creating a compelling dance experience~\cite{nam2014interactive}. The participation of social robots in storytelling underscored the potential of robots to enhance creative processes through their presence, body language, facial expressions, and vocal modulations~\cite{jeon2017robotic}. \citet{antunes2022inclusive} demonstrated how robots can act as co-narrators or audience members in storytelling sessions, thereby enriching the narrative experience and fostering children's creativity through facilitating multisensory feedback and emotional behaviors.

These examples illuminate the multifaceted roles that robots can play in creative processes, from facilitating creative processes as tools and partners to acting as resources~\cite{chung2021intersection}. To fully grasp the potential of robots in the creative domain, it is essential to explore how their material features and interactive capabilities shape and support creative expression. Beyond the roles that robots play in creative processes, we are also interested in what is unique about robots (e.g., material features and interaction processes) that accounts for their continual advancement and application in creative contexts in a more integrated way. The artistic appropriation of robots may provide part of the answer. Notably, robotic artists engage proactively with robots when creating artwork, exploiting the artistic potential of robots, rather than passively accommodating the robots' inherent limitations. They break down robots, redefine them, and repurpose them as instruments for artistic goals. We propose that robots' specific manifested qualities embed the potential to stimulate creativity and drive such interest and engagement by artists as in their practice. Thus, we aim to better explore the above questions by examining the practices and characteristics of artists' voluntary utilization of robots in achieving artistic ends. Answering these questions can contribute to knowledge about the unique capabilities and roles of robots in creative contexts.

\subsection{Robotic Art}

Robotic art involves the use of robots as both mediums and subjects of artistic expression~\cite{penny2016robotics, kac1997foundation}. Robotic art covers a group of practices that is usually related to computer-automated cultural artifacts (CACA)~\cite{penny2013art}, although some authors refer to the broader field of mechanically driven kinetic art in an attempt to frame it within media art practices~\cite{hoetzlein2009new}. From a general perspective, a collection of works with a certain level of machine autonomy is often addressed as robotic art ~\cite{kac2001origin}, belonging to a broader category of machinic art, which includes `telerobotic art' ~\cite{reichle2009art}, `postbiological life forms' ~\cite{reichle2009art}, `robotic installation art'~\cite{shi2023what}, and so on. Artworks featuring robots not only create more engaging experiences but also challenge the boundaries of intelligent agency and human creativity. For instance, robotic artists prompt us to question whether a robot that creates art can be considered an artist in its own right~\cite{shi2023what, mika2022can}.

Robots have been adopted in artistic expression and creation due to their anthropomorphic qualities, creating a metaphor as social components and providing a degree of sociality~\cite{jeon2017robotic,kahn2014creative}. In this context, artists can leverage robots' social behaviors to stimulate artistic expression. For instance, natural language dialogue can establish social connections with creative producers and guide creative imagination through questioning~\cite{kahn2016human, hu2021exploring}. Artists have utilized robots' facial expressions, body postures, movements, and speech intonation to provide emotional communication during creative theater performances, thereby encouraging the generation of creative human-human/robot interaction~\cite{knight2011eight, li2023nice}. These characteristics distinguish robots from traditional artistic materials such as plaster or fabric, allowing artists to create artwork and express themselves more dynamically through interactions with living materials.

Beyond responding to human social interactions, robotic art attracts artists through its non-static artifact characteristics and adaptive environmental responsiveness, enabling the creation of new forms of interactive artwork~\cite{nelkin1984science}. While traditional art forms are often judged by their fixed final products, much of robotic art's appeal lies in the process—the trajectory of robot movement, reactions to input, and the way outputs evolve over time provide a spectacle that, along with the final result, constitutes the artwork's complete expression. A common example is drawing robots, whose movement trajectories can be identified and preserved by artists as inspiration for artistic imagination or incorporated into the final design~\cite{lin2020your}. By changing drawing tools and materials, robots can transform drawing styles and patterns~\cite{jeon2017robotic, tresset2013portrait, lin2020your}. Mobile drawing robots can also recognize boundaries with creators or the environment to adjust creation trajectories~\cite{moura2007new, lin2020your}.

Thanks to robots' interactivity, robotic artworks often incorporate audiences into the creation process, allowing artists to harness more collective creativity and enabling greater possibilities in artistic expression~\cite{vavara2022, mikalauskas2018improvising}. Examples include Tresset's performance installations with \textit{Paul}, a custom robot art and computer program for real-time portrait drawing~\cite{tresset2014artistically}. In these works, audiences not only observe a robot creating art but are invited to witness the creative process of a technological ensemble at work, observing how human artists and robotic systems continuously influence each other~\cite{chung2022sketching, gomez2021robot, tresset2014artistically}.

While these researches have demonstrated the potential of robots in artistic creation and their capacity to stimulate creativity, they discuss discretely how creativity is supported and unleashed by single robotic systems, revealing limited approaches toward creativity. There remains a need for research that explores multiple cases involving diverse forms of robots to deepen our understanding of how robots can contribute meaningfully to the creative process. Our study addresses this gap by analyzing the creative practices of multiple robotic art practitioners working with various mediums, such as drawing, dance, and installation. This approach allows us to uncover shared experiences and insights across different artistic forms, potentially revealing broader principles for how robots can effectively facilitate and shape creative practices.

\subsection{Artistic Practices in HCI}

The relationship between art and technology is becoming increasingly intertwined, with artistic practices being extensively integrated into the HCI community to study human interactions with machines in specific contexts~\cite{jeon2019rituals}. By adopting exploratory approaches, these practices challenge traditional usability-study research design. For instance, some studies involve participants in spatial interactive sound installations and use sound theory and micro-phenomenological interviews to analyze the interplay between sensory perception, bodily intent, and sound~\cite{Robson2024sound, Frid2019sound} capturing the subtleties of how users perceive and engage with sound in immersive environments that are often overlooked by conventional measurement techniques. ~\citet{Rajcic2020mirror} introduced an emotion-responsive mirror that generates poetry in real-time, exploring the contrasts in emotional cognition, processing, and expression between machines and users. Similarly, \textit{Being the Machine} by \citet{devendorf2015being} invites artists to perform 3D printing manually in a machine-like manner, probing questions of agency and control in human-machine interactions. Furthermore, collaborations with choreographers have led to the creation of movement-sonification scarves as probes in dance performances, investigating dancers' engagement with music and aiding HCI researchers in exploring how to design body-centric interaction technologies~\cite{Bang2023dance}. The creation of artistic works, with their inherent ambiguity and potential for unforeseen human interactions, parallels the challenges in studying human interactions with emerging technologies~\cite{coessens2009artistic}. Interactive artworks empower HCI researchers to expand their understanding of emergent human behaviors in relation to machines.

However, the role of artistic practice in HCI research has been a subject of ongoing debate with some researchers questioning integrating artistic knowledge into HCI~\cite{frankjaer2020knowledge, candy2011research}. Our review suggests that examining the relationship between art and technology serves as both inspiration and a means to understand the interaction and the broader social world~\cite{taylor2017performing}, leading to novel design research methods such as cultural probes~\cite{gorichanaz2020public,gaver1999probe}, participatory/co-design~\cite{clark2016situated, disalvo2007mapover,soden2020disaster}, and arts-based ethnography~\cite{kang2022electronicists}. Some studies incorporate interdisciplinary teams in the creation and performance of artworks to explore nonlinear collaboration across disciplines, highlighting the intersection of technological and aesthetic thinking
~\cite{kang2022electronicists,kang2018intermodulation}. Research on self-situated performance practices expands our understanding of new relationships and interactions between researchers and participants in HCI, allowing for ongoing refinement of research practices~\cite{taylor2017performing}. In these cases, artistic practice is not merely an object for investigation but something that implies broader issues of research, offering art creators, participants, and researchers opportunities for reflection and knowledge transformation.

Besides, artistic practices also bring diverse aesthetic experiences to the HCI community, helping researchers focus more on the sensory dimensions of interactions with technology. The concept of aesthetic computing, exploring how art and aesthetics are applied in different areas of computer science, introduces a sensuous aesthetic framework to the exploration of technology in HCI research~\cite{fishwick2006aesthetic, jeon2017robotic}. Artistic practices bring embodiment and interaction to the forefront, placing human sensory experience, judgment, and contemplation at the core of aesthetic experiences~\cite{taylor2017performing}. For example, soma design focuses on elevating the sensory experiences of bodily activities to consciousness, utilizing these experiences as resources in the design process. Designers engage deeply with materials—whether physical, digital, cultural, or movement-based—through all their senses, cultivating their aesthetic sensibilities to create innovative designs~\cite{hook2018designing, hook2016soma, luft2023boards}. Consequently, various artistic practices, including choreography~\cite{allen2022choreo,eriksson2019drone}, sound art~\cite{kilic2023corsetto}, and media art~\cite{branch2021tele}, are increasingly incorporated in HCI to shift the focus to bodily sensations, prompting a reevaluation of our judgments about technology.

Integrating artistic practices into HCI research has shown significant potential in broadening our understanding of user experience and technology design. This study explores how robotic art can help the HCI community deepen its understanding of the relationship between human creative behaviors and robots as facilitative materials. We aim to bridge the gap between artistic practice and technology design for creativity by proposing a framework explicating how computing systems like robots serve as the tools and sites for creative endeavors and offering subsequent design implications for creative activities.
