\section{Conclusion}

This study empirically examines the unique practice of using robots in artistic production. We uncover the social, material, and temporal dimensions of this practice that shape perceptions of creativity in robotic art. Our analysis highlights the role robots play in driving artistic inquiry and fostering creative thinking, expression, and experimentation. Insights from robotic art offer implications for understanding how creativity emerges through interactions among humans, technological systems, and material actants over time. These findings also suggest ways to facilitate creative activities through deeper engagement with computing systems. As a first step, we propose a framework to advance creative practices with computing systems by promoting \textit{socially informed}, \textit{material-attentive}, and \textit{process-oriented} designs that better support human creators.

