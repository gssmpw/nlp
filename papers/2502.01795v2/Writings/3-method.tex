\section{Method}

\subsection{Participant Recruitment}

In this study, we conducted in-depth, semi-structured interviews with nine artists working on robotic art. They are renowned professional artists who use or build physical robots for their artistic work, namely robotic art. Designers and amateurs are out of this criterion. We identified potential participants by browsing individual and artistic organizations' websites during the initial recruitment. Subsequently, we employed snowball sampling to recruit more artists who met our criteria. The recruitment ended when iterative data analysis reached saturation. The vast majority of the participants have had formal art training in drawing, painting, installation, media art, music, sculpture, or dance, and all of them incorporated robots into their particular genres. Some have technical backgrounds; many have a doctoral degree in the fields of art or technology. All of them build their robots or at least modify out-of-shell robots. \autoref{tab:demographics} shows more demographic information. 
%We refrain from disclosing more specific information about the artists, e.g., art forms, and job titles, which reasonably increases the likelihood of identifying the individuals. 
The study is approved by the Institutional Review Board of the lead author's institution. The artists provided oral consent to participate in this study voluntarily.

\begin{table*}[htbp]
\centering
\begin{tabular}{c c c c c c}
\toprule
Participant & Gender & Age & Geographical Base & Occupation & Years as Artist \\
\midrule
Alex & Man & 30+ & East Asia & University \& art & 10+ \\
Daniel & Man & 30+ & Australia & University \& art & 10+ \\
David & Man & 50+ & Europe & Full-time art & 30+ \\
Evelyn & Woman & 50+ & North America & Full-time art & 10+ \\
Linda & Woman & 30+ & North America & Non-profit \& art & 10+ \\
Mark & Man & 40+ & Europe & Full-time art & 10+ \\
Robert & Man & 60+ & North America & Full-time art & 30+ \\
Samuel & Man & 60+ & East Asia & University \& art & 20+ \\
Sophie & Woman & 30+ & Europe & Full-time art & 10+ \\
\bottomrule
\end{tabular}
\caption{Participant demographics in alphabetical order of the pseudonyms.}
\label{tab:demographics}
\end{table*}

\subsection{Semi-Structured Interviews}

Before the interviews, the first author reviewed the media press on the artists' websites, gaining background information to build rapport, develop interview questions, and inform analysis. He conducted the interviews via Zoom with video recording, which lasted an average of 67 minutes. The in-depth interviews enabled us to look at the layers of meaning that artists relate with their work---unraveling feelings, attitudes, values, and motivations crucial for addressing our research question~\cite{pugh2013good}. The initial interview questions were structured around the following topics: the workflow, motivation, and rewards and challenges of working on robotic art; the relationship between artists and robots; and the source of creativity. These inquiries were driven by our curiosity in making sense of the artists’ work practices. We aimed to understand their motivations for engaging with robotic art, unpacking the values embedded in their practices and informing broader topics around human-machine creation. 

As the interviews began, the first author asked the artists to \textit{introduce themselves and their artwork}. Then he asked about \textit{their inspiration source for their artwork and motivations for working on robotic art}, steering the conversation toward creativity and the characteristics of robotic art. Following that, he solicited their reflections and anecdotes on the \textit{opportunities and challenges of working on robotic art} to understand the nuances of robotic art practices. He then guided the conversation to explore the \textit{relationships between artists and robots}, focusing on their respective agencies and contributions to creative outcomes in art practices. Additionally, he posed questions regarding the artistic practices' \textit{embodied/material engagement}. Concluding the session, he asked the artists to elaborate on any aspects they wished to add. To comply with the semi-structured nature, he prompted follow-up questions throughout the interviews and remained open to new ideas introduced by the artists, allowing some emergent ideas to lead the conversation. After each interview, the artist received a \$40 Amazon or Visa e-gift card if they chose to accept compensation.

\subsection{Qualitative Analysis}

Data collection and analysis proceeded in parallel and iterative ways throughout the study. We began the analysis shortly after conducting each interview. Emerging insights from the analysis led us to refine our interview protocol several times, focusing on the most salient story in the data. The analysis also informed theoretical sampling~\cite{charmaz2006constructing} at the later stage, purposefully recruiting participants who are more likely to provide missing components or revise the emergent themes to construct robust findings.

The analysis mainly followed Charmaz's constructivist grounded theory approach~\cite{charmaz2006constructing} and used Atlas.ti. The first author began with line-by-line open coding on the interview transcripts, followed by focused coding to synthesize and organize the initial open codes into categories. He compiled analytical memos after each coding phase to document emerging categories, reflect on assumptions, and enable constant comparisons across participants, codes, and categories. The process was iterative and collaborative with other authors, leading to 101 codes in 9 categories. For example, the code \textit{VALUE: process} within the \textit{VALUE} category marks discussions around artistic values inherent in creating and presenting robotic art. The code \textit{RELATIONSHIP: instrument} within the \textit{RELATIONSHIP} category illustrates how the artists perceive robots as instruments for their artistic pursuit. The numbers of codes and definitions of all categories are listed in \autoref{app: coding}. Subsequently, we assigned high-level theoretical codes to the categories to delineate the relationships among the categories. This theoretical coding resulted in four themes explaining the generation of creative values in robotic art practice: \textit{Embodied and material presence}, \textit{Technological malfunction}, \textit{Audience reception and feedback}, and \textit{Creation and exhibition processes}. The iterative process forged us to construct a coherent narrative that well-represents the data. The analysis reasonably reached saturation as participants recruited later to a large extent repeated previous artists when answering questions regarding our key themes.

When the analysis was complete, we conducted a member check---shared the Findings section with the artists, asked for comments, and revised the findings when appropriate. Nevertheless, our positionalities inevitably shaped the analysis. The first and third authors have limited experience in professional art practices, which might constrain their ability to fully grasp the nuances of the artistic process. The second author, with an educational background in design and arts and publications in art exhibitions, brings complementary artistic expertise to the analysis.