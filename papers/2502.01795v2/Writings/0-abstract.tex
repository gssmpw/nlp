\begin{abstract}

Robots extend beyond the tools of productivity; they also contribute to creativity. While typically defined as utility-driven technologies designed for productive or social settings, the role of robots in creative settings remains underexplored. This paper examines how robots participate in artistic creation. Through semi-structured interviews with robotic artists, we analyze the impact of robots on artistic processes and outcomes. We identify the critical roles of social interaction, material properties, and temporal dynamics in facilitating creativity. Our findings reveal that creativity emerges from the co-constitution of artists, robots, and audiences within spatial-temporal dimensions. Based on these insights, we propose several implications for \textit{socially informed}, \textit{material-attentive}, and \textit{process-oriented} approaches to creation with computing systems. These approaches can inform the domains of HCI, including media and art creation, craft, digital fabrication, and tangible computing.

\end{abstract}