% This is samplepaper.tex, a sample chapter demonstrating the
% LLNCS macro package for Springer Computer Science proceedings;
% Version 2.20 of 2017/10/04
%
\documentclass[runningheads]{llncs}
%
\usepackage{hyperref}
\usepackage{graphicx}
\usepackage{multirow}
\usepackage{amsmath}

% Used for displaying a sample figure. If possible, figure files should
% be included in EPS format.
%
% If you use the hyperref package, please uncomment the following line
% to display URLs in blue roman font according to Springer's eBook style:
% \renewcommand\UrlFont{\color{blue}\rmfamily}

\begin{document}
%
\title{ApplE: An Applied Ethics Ontology\\with Event Context}
%
%\titlerunning{Abbreviated paper title}
% If the paper title is too long for the running head, you can set
% an abbreviated paper title here
%
\author{Aisha Aijaz\inst{1}\orcidID{0000-0002-8137-106X} \and
Raghava Mutharaju\inst{1}\orcidID{0000-0003-2421-3935} \and
Manohar Kumar\inst{2}\orcidID{0000-0003-2540-1356}}
%
\authorrunning{Aijaz et al.}
\titlerunning{ApplE: An Applied Ethics Ontology with Event Context}
% First names are abbreviated in the running head.
% If there are more than two authors, 'et al.' is used.

\institute{Knowledgeable Computing and Reasoning (KRaCR) Lab, IIIT-Delhi, India\\ 
\and
Department of Social Sciences and Humanities, IIIT-Delhi, India\\
\email{\{aishaa,raghava.mutharaju,manohar.kumar\}@iiitd.ac.in}}


\maketitle  % typeset the header of the contribution
\begin{abstract}
Applied ethics is ubiquitous in most domains, requiring much deliberation due to its philosophical nature. Varying views often lead to conflicting courses of action where ethical dilemmas become challenging to resolve. Although many factors contribute to such a decision, the major driving forces can be discretized and thus simplified to provide an indicative answer. Knowledge representation and reasoning offer a way to explicitly translate abstract ethical concepts into applicable principles within the context of an event. To achieve this, we propose ApplE, an Applied Ethics ontology that captures philosophical theory and event context to holistically describe the morality of an action. The development process adheres to a modified version of the Simplified Agile Methodology for Ontology Development (SAMOD) and utilizes standard design and publication practices. Using ApplE, we model a use case from the bioethics domain that demonstrates our ontology's social and scientific value. Apart from the ontological reasoning and quality checks, ApplE is also evaluated using the three-fold testing process of SAMOD. ApplE follows FAIR principles and aims to be a viable resource for applied ethicists and ontology engineers.

\keywords{Ontology Development \and Ethical Reasoning \and Ethics Philosophy \and Applied Ethics \and Knowledge Representation.}

\end{abstract}

\noindent\textbf{Resource Type:} Ontology\\
\textbf{License:}\href{https://www.apache.org/licenses/LICENSE-2.0}{Apache License 2.0}\\
\textbf{Resource Link:}\href{https://purl.org/appliedethicsontology}{https://purl.org/appliedethicsontology}\\
\textbf{Documentation:}\href{https://purl.org/appliedethicsontology/documentation}{https://purl.org/appliedethicsontology/documentation}\\
\textbf{Github Repository:}\href{https://github.com/kracr/applied-ethics-ontology}{https://github.com/kracr/applied-ethics-ontology}

\section{Introduction}
\label{Intro}
Significant efforts in the past few decades have been made to integrate intellectual, emotional, and even physical capabilities into artificial systems through powerful learning algorithms, natural language processing techniques, and advanced robotics \cite{cominelli2018seai,abdullah2022chatgpt,sheetz2020trends}. However, even the state-of-the-art intelligent systems of today may not be considered inherently moral agents \cite{brozek2019can}. Considering how commonplace these systems are in society,  especially in critical decision-making settings, developing moral machines becomes an imperative.

Determining an appropriate ethical course of action within a given context is governed by complex reasoning. It involves understanding the intention of the agent, the consequences that result from an action, the ethical principles violated, and other factors that are sensitive to the context. This aspect is further complicated by the disagreement among different ethical theories about the morality of an action, given a situation \cite{bergmann2014challenges}. For example, Kant would consider killing a punishable offense regardless of the context \cite{hill2013kantianism}, whereas utilitarianism would allow it if doing so saves more lives than are being sacrificed \cite{shallow2011trolley}. Virtue ethics considers honesty to be a trait of a virtuous, hence, ethical human being \cite{wilson2018honesty}. However, deontology pushes agents to follow the law and not spill state secrets \cite{mokrosinska2023necessary}. 

The development of the ApplE ontology aims to capture the most pertinent information about the morality of an action that would indicate its rightness to facilitate ethical decision-making. With this motivation in mind, the scope of the vast field of applied ethics is narrowed to a few major factors that would contribute most to making such a decision. 

Applied ethics identifies the methods to incorporate general ethical principles and theories in real-world applications. It requires certain contextual information, such as the agents involved, the action, and time and place, along with some normative principles, such as consequences, virtues, and associated rules.

This paper makes the following contributions:
\begin{enumerate}
    \item A taxonomy for applied ethics with ethics theory and event context.
    \item A modified methodology that takes from SAMOD and Spiral development to best suit the development of the ApplE ontology.
    \item An Applied Ethics (ApplE) ontology that models the created taxonomy.
    \item A real-world motivation scenario from bioethics modeled using ApplE.
    \item  SWRL rules to indicate the moral nature of an action.
\end{enumerate}

The paper is organized as follows. In Section~\ref{Related work}, we discuss the gap in the current literature, along with our novel contributions. The modified SAMOD methodology \cite{peroni2016samod} and the ontology development process are described in Section \ref{Methodology}. A description of ApplE's major modules has been outlined in Section \ref{OntologyDescription}, followed by the resource specifications in Section \ref{Resource Specifications}. We also evaluate the ontology using a set of varied competency questions and formalization techniques in Section \ref{Evaluation}. Finally, in Section \ref{limit}, we outline the limitations and the plan for future work.

\section{Background}
\label{Related work}
Inclination towards predisposition results in most people making predictable decisions that may be backed by moral reasoning but often not ethics theory \cite{reynolds2006moral}. In order to understand applied ethics, one must first understand the many ethical theories and schools of thought that justify various actions in certain contexts. 

Although philosophers generally do not agree on which theories may be superior or relevant or ideal \cite{bergmann2014challenges,archard2011moral}, there is some general understanding of how to approach certain ethical issues in particular domains \cite{holmes2018introduction}. As proposed by Demarco in his work, a coherent outlook that considers ethical principles, rules, and judgments to determine which aspect of morality would take precedence in a particular context when making an ethical decision would be at the core of applied ethics \cite{demarco1997coherence}. Due to this general understanding, applied ethics becomes more comprehensible than normative ethics, which is concerned with the fundamental principles of how people should live \cite{kagan1992structure}. 

Most applications of modeling ethics in computer systems take from the normative theories: utilitarianism (maximizes utility), virtue ethics (prioritizes virtues and virtuous behavior), and deontology (characterized by duty and limitations). One or more of these theories are applied to specific use cases in these computer applications, such as autonomous vehicles \cite{awad2018moral}, resource allocation \cite{10.1093/jlb/lsac012}, and racial discrimination \cite{briggs2020mitigating}.

Various techniques have also been used to model ethics in computer systems, such as logic, empirical studies, and machine learning algorithms. GenEth \cite{anderson2018geneth}, the Moral Machine Project \cite{awad2018moral}, MoralDM \cite{dehghani2008integrated}, and the consequence engine \cite{vanderelst2016architecture} are some examples. Although these valuable contributions tackle imperative concerns in today's society, there are certain issues they do not resolve. These systems rely on data and frameworks that reflect acute moral reasoning rather than a deeper understanding of ethical theory and its application. For example, the Moral Machine Project by MIT uses crowd-sourced data from which a powerful model learns the choices made to resolve the Trolley Problem \cite{awad2018moral}. However, this data may be unreliable and inaccurate, resulting in a system that has learned merely what humans \textit{say} would do when facing the trolley problem rather than what humans \textit{should} do in that situation. 

For this reason, an explicit taxonomy of applied ethics is required that may serve as an invariable source of ethics theory for computer systems to refer to and infer from when given a certain event context. This requirement may be fulfilled by a symbolic representation or an ontology. 

There have been efforts in this space, such as Lewis et al.'s ontology, to standardize trustworthy AI by capturing international policies and regulations \cite{Lewis2021}. Although some of the classes of this ontology are similar to what we have used, the objective of the ontology is simply to provide structure to existing policy documents. Another ontology is presented by Vasquez et al. \cite{vasquez}, which is used to ethically judge emerging technologies based on the ethical issues, expectations they must fulfill, and social usability. Again, this ontology does not aid the objective of developing a general taxonomy for applied ethics.

The closest to our work is by Debellis in \cite{debellis2018universal}, which implements a Universal Moral Grammar (UMG) via ontology modeling. The framework is structured upon the work of Mikhail \cite{Mikhail2007}, which states that morality is universal amongst all moral agents. This, however, does not model ethics in a way that could be used for practical application, as it does not involve the contextual cues that are vital when applying ethics theory.

We identified two major gaps in the literature concerning the modeling of ethics in computer systems: (a) a fully coherent approach to modeling ethics regardless of the domain and (b) a general applied ethics ontology that includes both ethics theory as well as event context for application. The ApplE ontology aims to fill these gaps by contributing two separate modules: the Ethics theory module (which serves as the invariable taxonomy) and the Event Context module (which provides relevant information regarding the situation at hand).

\section{Methodology}
\label{Methodology}

The ApplE ontology was developed primarily in accordance with the Simplified Agile Methodology for Ontology Development (SAMOD)\cite{peroni2016samod} and also takes from the Spiral Software Development Methodology \cite{boehm1988spiral}. This custom methodology borrows from agile development and supports quick, iterative cycles of development catered towards ontologies with fewer axioms. SAMOD describes the involvement of a domain expert (DE), in our case, an applied ethicist, who outlined the classes to be considered for the ontology in a taxonomy (See Section \ref{OntologyDescription}). The ontology engineers (OEs) implemented these entities. Although SAMOD recommends the involvement of the DE to be limited to the initial development phase, we deviated from this by involving the DE twice: once at the beginning to develop the initial modelet\footnote{A modelet is defined as a stand-alone model that describes a particular domain, which in our case would be applied ethics \cite{peroni2016samod}.} and then to model the motivation scenario described in Section \ref{Evaluation}.  

\begin{figure}[ht]
\includegraphics[width=\textwidth]{methodology.png}
\caption{A representation of the hybrid methodology inspired by the Spiral methodology at a macro-level and the SAMOD methodology at a micro-level.} \label{methodology diagram}
\end{figure}

Our methodology (Figure~\ref{methodology diagram}) shows the four major quadrants of development:
\begin{enumerate}
    \item Developing the domain taxonomy with the involvement of the DE. In the second pass, we developed the use cases.
    \item Formalizing the domain in a \textit{modelet}.
    \item Merge the modelet to any previous milestone.
    \item Refactor the new ontology to ensure semantic integrity.
\end{enumerate}

Steps 2, 3, and 4 end with an achieved milestone, and the three-fold testing is required at every step, as discussed in Section \ref{indevelopment eval}. We selected the most relevant information, keeping ApplE small yet specific. We have also utilized multiple ontology design patterns (ODPs) \cite{gangemi2009ontology} and entities from established ontologies wherever possible (see table \ref{tab1}). The ontology was built using middle entity expansion, and the naming scheme aims to be self-explanatory, thus aiding its value as a resource.

\paragraph{\textbf{Developing Domain Taxonomy.}} Two aspects are required when making an ethical decision -- theoretical and practical. To represent the theoretical aspects (colored purple in Fig. \ref{upper-level ontology diagram}), we used concepts from the field of applied ethics. For the practical factors, we considered the context of an event (colored green in Fig. \ref{upper-level ontology diagram}). The diagram shows the interacting concepts between the two categories, modeled separately in the ApplE ontology, to provide ease of modular development. The relationships between the theoretical and practical aspects of our ontology serve as bridge principles usually used in applied ethics to translate the theoretical normative philosophies into applicable concepts in real-world use cases \cite{nagel1990bridging}. This upper ontology is the first \textit{modelet} we created, from which we iteratively expanded more entities. 

\begin{figure}[!h]
    \centering
    \includegraphics[width=\textwidth]{upperlevelontology.png}
    \caption{An intermediate representation of the ApplE ontology with the two interacting modules. These models have been expanded in the following sections.}
    \label{upper-level ontology diagram}
\end{figure}

\subsection{Ontology Development}
The ontology was built in Protege~\cite{DBLP:journals/aimatters/Musen15} based on this modelet. Various ontology design patterns were also used for the Context module, such as the Participant, Co-participant, Trajectory, Event, and ObjectRole ODPs. Concepts from established ontologies were also reused. The prefixes for these ontologies and ODPs are shown in the Table~\ref{tab1}. 

\begin{table}[ht]
\centering
\caption{Namespaces of reused ODPs and ontologies in ApplE.}
\label{tab1}
\begin{tabular}{p{1.2in}l}
\hline
\textbf{Class/Property} & \textbf{Namespace} \\
\hline
Consequence, & airo: https://w3id.org/AIRO \\ 
hasConsequence, &  \\ 
Action, Domain &  \\
coParticipatesWith & copart: http://www.ontologydesignpatterns.org/cp/owl/ \\ &coparticipation.owl \\
subEventOf & event: http://w3id.org/daselab/onto/event \\
Context & ex: http://contextus.net/ontology/ontomedia/core/expression \\
Agent & foaf: http://xmlns.com/foaf/0.1/ \\
Bioethics & modsci: https://w3id.org/skgo/modsci \\
Role, hasRole & or: http://www.ontologydesignpatterns.org/cp/owl/ \\
 &objectrole.owl \\
hasParticipant, & part: http://www.ontologydesignpatterns.org/cp/owl/ \\ isParticipantIn & participation.owl\\
Event, Action & schema: http://schema.org/\\
hasBeginning, hasEnd & time: http://www.w3.org/2006/time\\
doesAction & traffic: http://www.sensormeasurement.appspot.com/ont/ \\
&transport/traffic\\
TimeEntity, & tj: http://w3id.org/daselab/onto/trajectory \\ Place, atPlace, & \\
atTime & \\
\hline
\end{tabular}
\end{table}

\paragraph{\textbf{Competency Questions.}} Competency questions have been used during the development of the ontology and for its evaluation. They help to discern the information expected from the ontology. As per SAMOD, these competency questions have been appropriately identified and marked. We have developed various CQs in collaboration with the DE, a few of which are tabulated in Section \ref{Evaluation}.

\paragraph{\textbf{Glossary of Terms.}} Once the motivating scenario and informal competency questions are outlined, OEs and DEs were to formalize a glossary of terms that may be useful to the readers of this ontology as they may or may not be familiar with the domain of applied ethics. A shortened version of the glossary is provided in table \ref{got table}. A full list of annotated entities is provided in the ontology documentation. 

\begin{table}[ht]\centering
\caption{Glossary of terms for ApplE Ontology.}\label{tab2}
\label{got table}
\begin{tabular}{|p{1.0in}|p{3.5in}|}
\hline
Term & Definition\\
\hline
\multicolumn{2}{|c|}{Applied Ethics Module}\\ 
\hline
Ethics & A discipline concerned with studying what is morally right, wrong, good, or bad.\\
Applied Ethics & A discipline concerned with the application of ethics in real-world events.\\
Applied Ethics Philosophy & A theory or ideology that stems from normative ethics and is applied to real-world events.\\
Ethical Principle & A concept that defends the morality of an action at an intrinsic level.\\
Ethical Issue & A situation where an ethical conflict may arise. \\
\hline
\multicolumn{2}{|c|}{Event Context Module}\\ 
\hline
Context & The circumstances in which an event occurs.\\
Event & Something that happens. \\
Agent & An entity capable of doing and receiving actions (eg: Person, Company, Animal, Government, etc.).\\
Action & Something that is done.\\
Consequence & The outcome of an action.\\
Moral Intention & The intention of an agent with regard to a moral action.\\
\hline
\end{tabular}
\end{table}

\paragraph{\textbf{Testing.}} The three-fold testing phase requires model, data, and query testing. These are expanded on in Section \ref{Evaluation} of this paper. 
The \textbf{model testing} phase requires OEs to test the model using available software to check for compatibility, coherence, and consistency.
\textbf{Data testing} requires the creation of an exemplar dataset that implements the natural language use cases mentioned in the motivating scenario. For these, we used the competency questions and determined the information required to be present in our ontology. Finally, for \textbf{query testing}, we formalized the informal queries using description logic (DL) and SPARQL to confirm the results are as expected.

\section{Description of the Ontology}
\label{OntologyDescription}
In this section, we describe the modules in the ApplE ontology in detail. When applying ethics to a real-world domain, certain factors remain consistent, while others are context-dependent. After considering various philosophical texts \cite{cohen2014contemporary,frey2008companion,sep-theory-bioethics} and collaborating with a domain expert, we have summarized the few most important entities that aid ethical decision-making. These may be categorized into two major modules: the Applied Ethics module and the Event Context module. 

\subsection{Applied Ethics Module}

\texttt{AppliedEthics} is a subclass of \texttt{Ethics}, alongside \texttt{MetaEthics} and \texttt{Normative Ethics}. It consists of subclasses divided based on \texttt{Domain}, such as \texttt{Bioethics} and \texttt{BusinessEthics}. Each of these has certain associated \texttt{Applied Ethics Philosophies}, which take into account prior cases to make ethical judgments. 

\begin{figure}[!ht]
\includegraphics[width=\textwidth]{ethicsmoduleupdated.png}
\caption{An intermediate diagram that shows the applied ethics module of the Applied Ethics ontology. Some instances and classes have been omitted from this diagram for the sake of brevity.} \label{fig3}
\end{figure}

\texttt{NormativeEthics} consists of three subclasses which are important ethical theories. \texttt{Consequentialism} favors \texttt{Consequence} in order to gauge the morality of an action. This is a popular method and inspires various applied ethics philosophies that may favor certain outcomes over others.

\texttt{Deontology} focuses on the intrinsic nature of such an \texttt{Action} via a set of rules that may determine if an action is morally right or wrong. \texttt{Action} is linked to certain \texttt{EthicalPrinciple}, for example, lying is linked to a violation of \texttt{Transparency}, and stealing is a violation of \texttt{nonmaleficence}. Various ethical principles exist, and it would be difficult to note down an accurate taxonomy, especially since individuals value principles differently. Considering ethical principles and moral intention in applied ethics allows us to move away from a purely consequentialist approach, which is not considered an ideal way to resolve ethical issues \cite{nye2015non}.    

\texttt{VirtueEthics} is a third subclass of ethics and describes an ethical action to be done by a virtuous agent. It is difficult to discretize such virtues in an agent, however, \texttt{MoralIntention} would give insight into the virtue of an agent \cite{fink2020acting}. Intention provides motivation towards action, and regardless of the consequences or the principles upheld or violated, moral intent is a significant indicator when judging an agent for morality.

The applied ethics module also defines a class to capture the \texttt{EthicalIssues} that may arise in a specific domain, which may then be resolved in accordance with the associated \texttt{applied ethics philosophies}. 

A few axioms from the Applied Ethics Module are as follows.

\begin{equation}
\texttt{AppliedEthics} \sqsubseteq \exists
\texttt{appliesPhilosophy}.\texttt{AppliedEthicsPhilosophy} 
\end{equation}
\begin{equation}
\texttt{AppliedEthics} \sqsubseteq \exists \texttt{appliesTo.Domain}
\end{equation}
\begin{equation}
\texttt{EthicalIssue} \sqsubseteq \exists \texttt{resolvedBy.AppliedEthicsPhilosophy}
\end{equation}
\begin{equation}
\texttt{EthicalPrinciple} \sqsubseteq \exists \texttt{violatedBy.Action}
\end{equation}
\begin{equation}
\texttt{Consequentialism} \sqsubseteq \exists \texttt{associatedWith.Consequence}
\end{equation}
\begin{equation}
\texttt{Consequentialism} \sqcap \texttt{Deontology} \sqcap \texttt{VirtueEthics} \sqsubseteq \bot
\end{equation}

\subsection{Event Context Module}
In this module, the ontology captures the important information about the \texttt{Context} in which an \texttt{Event} has occurred. The \texttt{Event} has various contextual information, such as 
\texttt{Domain}, \texttt{TimeEntity}, \texttt{Place}, and the \texttt{Agent} involved. The \texttt{Agent} is assigned a role using the \texttt{hasRole} predicate and a moral intention using \texttt{hasMoralIntention}.

An agent may be one of two types: an \texttt{ActiveAgent} and a \texttt{PassiveAgent}. The active agent does or causes an action (\texttt{ActiveAgent doesAction Action}) while the passive agent receives the repercussions of the action (\texttt{Action affects PassiveAgent}). An agent's \texttt{Role} may be important to consider in that particular context. For example, what may be considered ethical between two agents may be considered unethical between two different agents. 

The \texttt{Consequence} is amongst the most important considerations when deciding between possible courses of action. It may be characterized broadly as having a \texttt{SeverityOfConsequence}, \texttt{UtilityOfConsequence}, and \texttt{DurationOf Consequence}. Consequences may be mild, moderate, or severe, their utility may be good, bad, or neutral, and they may be temporally bound regarding their impact. It may not be clarified as to what duration would be considered short-term or what impact would be considered mild or moderate. However, given the situation at hand, it is easier to make decisions relatively between the courses of action available. Sometimes, an action is more severe than another, with a much longer impact. This has been made evident by the use cases discussed above. These terminologies, however vague, may be captured amongst these characteristics and discretized when confined to one or similar events. 

Finally, we consider the \texttt{Action} itself. The action may be anything that involves certain agents, leads to consequences, and reflects on some ethical principles. Based on the theoretical factors and contextual information, an action may be indicatively classified as a morally right, wrong, or grey action.

\begin{figure}[!ht]
\includegraphics[width=\textwidth]{contextmodule.png}
\caption{An intermediate diagram that shows the event context module of the Applied Ethics ontology. Some instances and classes have been omitted from this diagram for brevity.} \label{fig4}
\end{figure}

A few axioms from the Event Context Module are as follows.

\begin{equation}
\texttt{Context} \sqsubseteq \exists \texttt{describesEvent.Event}
\end{equation}
\begin{equation}
\texttt{Action} \sqsubseteq \exists \texttt{hasConsequence.Consequence}
\end{equation}
\begin{equation}
\texttt{Action} \sqsubseteq \exists \texttt{affects.PassiveAgent}
\end{equation}
\begin{equation}
\texttt{ActiveAgent} \sqsubseteq \exists \texttt{doesAction.Action}
\end{equation}
\begin{equation}
\texttt{Action} \sqsubseteq \exists \texttt{occursInEvent.Event}
\end{equation}
\begin{equation}
\texttt{MorallyRightAction} \sqsubseteq \exists \texttt{upholdsEthicalPrinciple.EthicalPrinciples}
\end{equation}
\begin{equation}
\texttt{Consequence} \sqsubseteq \exists\texttt{hasSeverityOfConsequence}.\texttt{SeverityOfConsequence}  
\end{equation}
\begin{equation}
\texttt{Event} \sqsubseteq \exists \texttt{hasParticipant}.\texttt{Agent}
\end{equation}
\begin{equation}
\texttt{ActiveAgent} \sqsubseteq \exists \texttt{hasMoralIntention.MoralIntention}
\end{equation}

\section{Resource Specifications}
\label{Resource Specifications}
The ontology has reused standard ontologies wherever relevant. It has applied a well-established methodology suited to its development. 

\paragraph{\textbf{FAIR Compliance.}} As per the FAIR principles \cite{wilkinson2016fair}, the ApplE ontology is findable via a permanent locator\footnote{\url{https://purl.org/appliedethicsontology}} and a GitHub repository\footnote{\url{https://github.com/kracr/applied-ethics-ontology/tree/main}}. It reuses entities from other ontologies and design patterns, which further facilitates interoperability. The naming scheme of the ontology is easily understandable while also being heavily annotated. The documentation of the ApplE ontology provides clarifications about the terms used through a tabular glossary as well. The ontology is open for access under the \href{https://www.apache.org/licenses/LICENSE-2.0}{Apache License 2.0}.

\paragraph{\textbf{Ontology Validation.}} We have used several tools to validate the ontology, apart from the three-fold evaluation conducted in accordance with the methodology opted. The OOPs Pitfall Scanner \cite{poveda2014oops} yielded no pitfalls in the ontology, and Protege's debugger HermiT (version 1.4.3.456) \cite{glimm2014hermit} deemed ApplE consistent and coherent. A series of over twenty competency questions were applied to the model to check the semantic validity of the hierarchy, properties, and axioms involved in the ontology. ApplE has proven to be clear, concise, and carefully constructed.

\paragraph{\textbf{Documentation.}} The technical documentation of this ontology was created using the online Live OWL Documentation Environment (LODE) \cite{lodeperoni} and is available for public access\footnote{\url{https://purl.org/appliedethicsontology/documentation}}. The documentation is also available on GitHub.  

\paragraph{\textbf{Resource Maintenance.}} We plan to maintain the ontology at a permanent location online where it will be available for public use. Furthermore, any future versions or iterations of the ontology will be updated at the same locations in order to remain consistently accessible.

\section{Evaluation}
\label{Evaluation}
The SAMOD methodology requires the use of a motivating scenario in order to evaluate the ontology using three testing methods: model, data, and query testing. A motivation scenario has been discussed in Section \ref{indevelopment eval}. This outlines the various classes and relationships used in the ontology using a sort of story that has been adapted to a real-world scenario. Of the various scenarios we considered, we have modeled one in detail to demonstrate the use of the ApplE ontology in representing pertinent ethical information in the Bioethics domain, but also to facilitate reasoning through extensive testing and rules. 

\subsection{In-development Evaluation}
\label{indevelopment eval}
During the development of the ontology, the SAMOD methodology encourages model, data, and query testing at every milestone. If the tests fail, it is recommended to go back to the previous milestone and start over. The iterative nature of this approach allowed us to get a clear understanding of how the entities interacted with one another, allowing for a simple but robust ontology. 

We also utilized SWRL rules \cite{swrl} with the Pellet reasoner \cite{SIRIN200751} to facilitate not only the modeling of the motivation scenario in the ontology but also use the captured ethical information to provide an indicative understanding of whether the action done is morally right, wrong, or grey. The bioethics use case modeled using the ontology also features the use of these SWRL rules and is demonstrated in figure \ref{use case modeling diagram}.

\paragraph{\textbf{Motivation Scenario:}} This case has been discussed in~\cite{liautaud2022power} and refers to the mass addiction to opioid-based painkillers in the US in the early 2000s~\cite{jones2019opioid}. Bioethics features an applied ethics philosophy called Principlism \cite{beauchamp2004principles}, which favors the principles of justice, nonmaleficence, beneficence, and autonomy. Consider this. A doctor, with good intentions, prescribed to a teenage patient thirty tablets of Oxycontin, a powerful painkiller containing oxycodone, a powerful opioid, for post-dental surgery pain. They were not fully aware of its addictive nature when taken over prolonged periods of time, and yet prescribed a large dose of the drug against policy so the patient would not have to come to the clinic multiple times. However, the continuous intake of the drug led to Opioid Use Disorder (OUD) \cite{strang2020opioid} in the patient. Given that the good, short-term consequence was mild (relieving pain), but the bad, long-term consequence was significant (addiction to drugs), the doctor was at fault regardless of his good intentions. The doctor violated the principles of responsibility and nonmaleficence while upholding the principle of beneficence. They were responsible for fully knowing the drug before prescribing it and following their duties without breaching any policy. 


\begin{figure}[!ht]
\includegraphics[width=\textwidth]{usecasemodeling.png}
\caption{A detailed diagram to show how the data exemplar is modeled using the ontology for the Bioethics domain. A similar modeling may be done for any other event of another domain. One of the SWRL rules is displayed to show that it found a match within the provided data and assigned PrescribeOpioidPainkiller as a Morally Wrong Action accordingly. A few axioms have been omitted for the sake of brevity.} \label{use case modeling diagram}
\end{figure}

\paragraph{\textbf{Model Testing.}} Model testing ensures that the classes, relationships, and instances are semantically intact and hold justified purposes with regard to both the domain and the ontology. 

We considered the OntoClean methodology \cite{ontocleanguarino2009overview} to \textit{clean up} the ontology wherever it seemed required. This leads us to a concise ontology with clarity on the rigidity, anti-rigidity, identity, and unity of its entities. Using this methodology, we were able to ask questions such as ``What is the difference between an event and an action?", ``Why is there a need for a neutral consequence?", and ``Is an active agent always active?".

Furthermore, we checked our ontology via the OOPS Pitfall Scanner \cite{poveda2014oops}, which yielded no pitfalls in the ontology. We tested our model via some DL queries relying on the HermiT reasoner \cite{hermit} to ensure that there were no unsatisfiable classes and the ontology was behaving the way we expected. Although the object properties have little complexity in terms of hierarchy, we also applied RBox compatibility checks to ensure that they are meaningful to avoid unwanted inferences \cite{rboxkeet2008representing}. We followed the basic model checks as outlined in \cite{keet2018introduction}. 

\paragraph{\textbf{Data Testing.}} The next step was to model a real-world scenario into the ontology. According to the methodology, these are known as data exemplars. We modeled the motivation scenario in the ontology to demonstrate how ethics theory and event context have been captured and can help with inferring the morality of the action \ref{use case modeling diagram}. This helps to evaluate the usefulness of ApplE in terms of real-world compatibility. After incorporating the use case data in the ontology, we created a few informal competency questions. We have tabulated a few of these in table \ref{informal cqs}.

\begin{table}[!h]\centering
\caption{Some informal competency questions for the ontology as per the exemplars.}\label{informal cqs}
\begin{tabular}{|p{0.5in}|p{2.4in}|p{1.8in}|}
\hline
Identifier &  Question & Expected Outcome\\
\hline
\multicolumn{3}{|c|}{Model-related CQs}\\
\hline
CQ1 & Are there any common issues that occur in both Bioethics and Business Ethics? & (Instance: Consent)\\
CQ2 &  What kinds of Applied Ethics adhere to the Professional Domain? & (Classes: AcademicEthics, BusinessEthics)\\
CQ3 & Which Applied Ethics fields apply the philosophy of Feminism? & (Classes: EnvironmentalEthics, Bioethics)\\
CQ4 & What kind of action may both uphold and violate some ethical principles? & (Class: MorallyGreyAction)\\
CQ5 & Which Applied ethics philosophy may help to resolve the issue of Deforestation? & (Instance: DeepEcology)\\
\hline
\multicolumn{3}{|c|}{Data-related CQs}\\
\hline
CQ6 & Who are the participants in the event of prescribing an addictive drug? & (Instances: Doctor, Patient)\\
CQ7 & What are the consequences of the action done by the Doctor? & (Instances: OpioidUseDisorder, PainRelief)\\
CQ8 & What are the characteristics of the consequence opioid use disorder (OUD)? & (Instances: BadConsequence, LongTermConsequence, SignificantConsequence)\\
CQ9 & What ethical principles are violated by the action of the doctor? & (Instances: Responsibility, Nonmaleficence)\\
CQ10 & What is the severity of the consequence of pain relief? & ModerateConsequence\\
\hline
\end{tabular}
\end{table}

\paragraph{\textbf{Query Testing.}} Finally, we tested these competency questions via queries. We used the DL tab in Protege to run the more basic queries and GraphDB's SPARQL-compliant tool for more complex queries. We have tabulated some of these queries in table \ref{query table}.

\begin{table}[!h]\centering
\caption{Queries in DL to formalize the informal CQs and their outcomes.}\label{query table}
\begin{tabular}{|p{0.5in}|p{2.4in}|p{1.8in}|}
\hline
Identifier & Associated Query & Outcome\\
\hline
CQ4 & \texttt{Action and (upholdsEthicalPrinciples some) and (violatesEthicalPrinciples some)}  & \texttt{MorallyGreyAction}\\
CQ7 & \texttt{Consequence and inverse hasConsequence some (Action and inverse doesAction some \{Doctor\})} & \texttt{OpioidUseDisorder, PainRelief}\\
CQ8 & \texttt{CharacteristicOfConsequence and inverse hasCharacteristicOf Consequence some \{OpioidUseDisorder\})} & \texttt{BadConsequence, LongTermConsequence, SignificantConsequence}\\
\hline
\end{tabular}
\end{table}

\subsection{Post-development Evaluation}
After creating the ApplE ontology, we also evaluated it via some established ontology quality metrics \cite{wilson2021analysis}. We chose the metrics suite as proposed by \cite{burton2005semiotic}, which requires manual evaluation based on four parameters: \textit{Syntactic, Semantic, Pragmatic, and Social Quality}.

The syntactic quality checks the correctness of syntax, which we have evaluated using previously mentioned pitfall scanners and debuggers. The semantic quality has to do with the meaningfulness and consistency of the terms that we have used. An advantage of using the SAMOD methodology was to ensure that the entities naming was understandable and clear, hence this checkbox was also ticked. The pragmatic quality ensures that the entities of the ontology are truly domain-specific and justified. Our only deviation from the SAMOD methodology was to seek confirmation from the DE in the second iteration of development to satisfy this condition. And finally, the social quality determines the extent of use of this ontology. Since this ontology is a novel contribution, we have yet to see other developers use our ontology for their own work. However, this is the first ontology of its kind and adequately captures various Applied Ethics concepts. We foresee much use for the ApplE ontology in the coming years. 

\section{Limitations and Future Work}
\label{limit}
Applied ethics requires capturing context as well as ethics theories. The authors of this ontology acknowledge the fact that the entities considered are not exhaustive of what ethicists consider. This is because when building an ontology like this, it would be imperative to consider the applicability of multiple domains. The ApplE ontology can be extrapolated to specific applied ethics domains, such as bioethics.

ApplE can adequately represent the ethics of action during an event. There is scope, however, to use the ontology along with some learning algorithms and more sophisticated reasoning methods to apply ethical dilemma resolution and case-based reasoning. Such algorithms might also facilitate the automatic selection of theories and philosophies to be applied and produce justifications. In the future, ApplE may be used to create modular ontologies for specific applied ethics fields and may facilitate ethical judgment in artificially intelligent systems.  

\section{Conclusion}
The ApplE ontology is an effort to capture the most imperative ethical information about an event to facilitate ethical reasoning. We have used the SAMOD ontology to ensure a simple yet rich ontology and have implemented various quality checks and measures as well. Various axioms have been asserted for both the Applied Ethics module and the Event context module, taking advantage of the many property characteristics, subsumptions, and cardinality. To demonstrate the use of the resource and its impact, we have modeled a real-world scenario from the Bioethics domain. 

Standard practices were employed to ensure the value of the resource. ApplE is FAIR-compliant and is easily accessible for reuse. We have evaluated the ontology using the prescribed methods of SAMOD, as well as a post-developmental evaluation that involved the semiotic metric suite to ensure validity. With this ontology, we aim to provide a novel way to apply ethics to any domain while also facilitating more complex reasoning for ethical issue resolution. 

The ApplE ontology is publicly available at \url{https://github.com/kracr/applied-ethics-ontology} under  \href{https://www.apache.org/licenses/LICENSE-2.0}{Apache License 2.0}. 

\bibliographystyle{splncs04}
\bibliography{ref}

\end{document}