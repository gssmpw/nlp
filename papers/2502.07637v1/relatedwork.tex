\section{Related Work}
In recent years, work has been done in the field of dataset creation for bias and hate speech in general, paying great attention to data coming from online sources, especially social media, such as Twitter, Facebook, Reddit, or blogs. This kind of work has been carried out in a multitude of languages, across several cultural contexts, and tends to cover various forms of sexism as it presents in written language. This is the case of \citet{chiril-etal-2020-annotated}, who present a corpus for detecting sexism in French tweets. Another example is the work of \citet{zeinert-etal-2021-annotating}, who sample their Bajer dataset from Twitter, Facebook and Reddit posts in Danish. However, research is also carried out with the target of more subtle, less explicit misogyny in mind; this is the case of the Biasly dataset by \citet{sheppard-etal-2024-biasly}, who gathered their data from scripts from North American movies, in English.

The most common method for data collection among the different existing datasets is using keywords \citep{chiril-etal-2020-annotated, zeinert-etal-2021-annotating, sheppard-etal-2024-biasly}. The degree of detail or the number of keywords varies from words that do not necessarily imply misogyny (e.g. ``she'') to ambiguous keywords, to keywords that are very highly related to misogyny and sexism (e.g. ``\#MeToo'').

Most of the existing misogyny detection datasets provide a taxonomy for different categories of misogyny in addition to the binary classification. Many regard the addition of a multi-label classification layer as necessary, given that ``binary detection […] disregards the diversity of sexist content, and fails to provide clear explanations for why something is sexist'' \citep{kirk-etal-2023-semeval}. There is no clear consensus regarding the types of misogyny to classify the sentences into, or even on the optimal level of detail regarding the categories.

To the best of our knowledge, this work is the first attempt to create resources for misogyny detection for the Swedish language.