\section{Dataset Details}
\label{appendix_dataset}

\subsection{American Trends Panel Datasets}
\label{appendix_atp_dataset}
Pew Research holds regular American Trends Panel (ATP) survey (called waves)~\cite{atp} covering various topics (\textit{e.g.} veterans, political priorities, gender and leadership) and releases result at an individual level.
For each anonymized individual, the following information is released: unique identification number, demographic details, survey responses, and weight.
Weights~\cite{mercer2018weighting} are the output of post-survey calibration process that helps adjusting survey results for response bias (e.g., non-response bias, sampling bias) correction and population representativeness.
As of January 2025, survey data until wave 132 has been released. About 20 surveys are conducted in each year.

\subsection{OpinionQA}
\label{appendix_opinionqa}
OpinionQA is a subset of ATP curated in \cite{santurkar2023whose}. This dataset consists of contentious 500 questions sampled from 14 ATP waves which have high intergroup disagreement (i.e. large Wasserstein distances among subpopulations' responses to a question). It also comes with hand-crafted ordinality information which provides structure to option lists. For example, options `Major reason', `Minor reason', and `Not a reason',
are assigned an ordinality mapping to 1, 2, and 3, respectively.
This ordinality allows a calculation of 1-dimensional Wasserstein distance.

Subpopulations we employ are listed in \Cref{table:opinionqa_detail}. This set of groups is adopted for several small-scale analysis \cite{santurkar2023whose, zhao2023group, kim2024few}.
We note that our approach is not limited to a specific number of groups and data is available for small or fine-grained demographic subpopulations.

\begin{table}[ht]
    \centering
    \scriptsize
    \captionsetup{font=small}
    \caption{
    A list of 22 subpopulations used throughout our fine-tuning and analysis.
    We provide the number of respondents in each subpopulation in American Trends Panel Wave 82 for reference.
    }
    \vspace{-5pt}
    \label{table:opinionqa_detail}
    \begin{tabular}{ccc}
        \toprule
        \textbf{Trait} & \textbf{Groups} & \textbf{Population \% in Wave 82} \\
        \midrule
        \multirow{2}{*}{Region} & Northeast & 17.2 \\
        & South & 37.8 \\
        \midrule
        \multirow{2}{*}{Education} & College grad+ & 24.2 \\
        & Less than high school & 5.2 \\
        \midrule
        \multirow{2}{*}{Gender} & Male & 44.3 \\
        & Female & 54.6 \\
        \midrule
        \multirow{4}{*}{Race / ethnicity} & Black & 9.6 \\
        & White & 66.1 \\
        & Asian & 4.8 \\
        & Hispanic & 15.2 \\
        \midrule
        \multirow{2}{*}{Income} & \$100,000 or more & 21.8 \\
        & Less than \$30,000 & 21.3 \\
        \midrule
        \multirow{2}{*}{Political Party} & Democrat & 35.1 \\
        & Republican & 29.1 \\
        \midrule
        \multirow{3}{*}{Political Ideology} & Liberal & 20.0 \\
        & Conservative & 22.6 \\
        & Moderate & 38.3 \\
        \midrule
        \multirow{5}{*}{Religion} & Protestant & 40.8 \\
        & Jewish & 2.0 \\
        & Hindu & 0.9 \\
        & Atheist & 0.6 \\
        & Muslim & 0.7 \\
        \bottomrule    
        \end{tabular}
\vspace{-5pt}
\end{table}
\newpage
\begin{table}[H]
    \centering
    \scriptsize
    \captionsetup{font=small}
    \caption{
    American Trends Panel (ATP) wave topics for waves included in \OURDATA-Train (top) and OpinionQA (bottom).
    Golden rows represent wave topics in \OURDATA-Train that are not present in OpinionQA, and blue rows represent wave topics in OpinionQA that are not present in \OURDATA-Train.
    For waves 68-79, survey questions related to COVID-19 (\textit{e.g.}, contact tracing, vaccines, and relocation) were included as part of a survey along with the main survey topic.
    }
    \label{table:subpop-train-detail}
    \begin{tabular}{ccm{4.5cm}}
    \toprule
    \textbf{Wave} & \textbf{\# questions} & \textbf{Wave Topic} \\
    \midrule
    68 & 90 & American News Pathways, George Floyd, Black Lives Matter \\
    69 & 92 & Politics, 2020 Census \\
    \highlightrow 70 & 56 & Religion in public life, social media’s role in politics and society \\
    71 & 84 & Voter attitudes \\
    \highlightrow 72 & 18 & New media \\
    \highlightrow 73 & 82 & American News Pathways, social media \\
    74 & 51 & Online harassment, race relations \\
    75 & 18 & 2020 pre-election survey \\
    \highlightrow 76 & 44 & American News Pathways \\
    \highlightrow 77 & 13 & Culture of work \\
    78 & 57 & 2020 post-election survey \\
    \highlightrow 79 & 93 & American News Pathways \\
    80 & 45 & Political priorities \\
    \highlightrow 81 & 52 & Economics, pandemic financial outlook \\
    \highlightrow 83 & 54 & Coronavirus vaccines and restrictions \\
    \highlightrow 84 & 50 & Religion in politics and tolerance \\
    85 & 93 & News coverage of the Biden administration’s first 100 days \\
    87 & 90 & Current political news and topics \\
    \highlightrow 88 & 37 & Tech companies and policy issues \\
    \highlightrow 90 & 79 & Twitter news attitudes \\
    91 & 64 & Benchmark study \\
    \highlightrow 93 & 19 & Social media update \\
    95 & 78 & Politics timely and topical \\
    \highlightrow 96 & 57 & Post-coronavirus pandemic spirituality \\
    \highlightrow 98 & 76 & Coronavirus impacts on communities, living arrangements and life decisions \\
    \highlightrow 99 & 20 & Artificial intelligence (AI) and human enhancement \\
    \highlightrow 103 & 12 & Economic well-being \\
    104 & 92 & Politics, Religion in Public Life \\
    105 & 38 & Global Attitudes US Survey 2022 \\
    \highlightrow 106 & 62 & Religion and the environment \\
    107 & 92 & Government and Parties \\
    \highlightrow 108 & 83 & COVID and Climate, Energy and the Environment \\
    109 & 51 & New Digital Platforms and Gender Identity \\
    110 & 90 & Politics timely and topical \\
    \highlightrow 111 & 23 & Online dating and E-commerce \\
    \highlightrow 112 & 31 & Social media update \\
    113 & 53 & 2022 National Survey of Latinos (NSL) \\
    \highlightrow 114 & 93 & Covid, scientists, and religion \\
    115 & 63 & Parents survey \\
    116 & 75 & Politics timely and topical \\
    117 & 16 & Religion and politics \\
    118 & 25 & Podcasts, news, and racial identity \\
    \highlightrow 119 & 70 & AI and human enhancement \\
    120 & 61 & Politics timely and topical \\
    \highlightrow 121 & 31 & Culture of work \\
    124 & 75 & Global Attitudes US Survey 2023 \\
    125 & 69 & Politics timely and topical \\
    126 & 93 & Racial attitudes, modern family \\
    127 & 59 & Americans and their data \\
    \highlightrow 128 & 89 & Americans and planet Earth \\
    129 & 107 & Politics timely and topical \\
    130 & 94 & Politics representation \\
    131 & 70 & Gender and leadership \\
    \midrule    
    \textbf{Wave} & \textbf{\# questions} & \textbf{Wave Topic} \\
    \midrule
    \highlightrowtwo 26 & 44 & Guns \\
    29 & 20 & Views on gender \\
    32 & 24 & Community types, Sexual harassment \\
    \highlightrowtwo 34 & 16 & Biomedical and food issues \\
    36 & 68 & Gender and leadership \\
    \highlightrowtwo 41 & 41 & Views of America in 2050 \\
    \highlightrowtwo 42 & 26 & Trust in science \\
    43 & 51 & Race in America \\
    45 & 13 & Misinformation \\
    49 & 19 & Privacy and surveillance \\
    50 & 43 & American families \\
    54 & 50 & Economic inequality \\
    82 & 56 & 2021 Global Attitudes Project U.S. survey \\
    92 & 23 & Political Typology \\
    \bottomrule   
    \end{tabular}
\end{table}

\subsection{\OURDATA-Train}
\label{appendix_ourdata_atp}
We gather additional data from the American Trends Panel, specifically collecting 53 waves from Wave 61 to 132.
There are 62 waves from Wave 61 - 132, however, some waves have missing demographic or ideology information (for example, wave 63 does not contain political ideology information) or the data is not available hence removed during the curation process.
To refine the dataset, we exclude questions that meet the following criteria:
those with more than 10 response options, redacted response data, or dependencies on prior questions (e.g., assessing political strength). 
For the remaining questions, we use GPT-4o to refine their wording, ensuring they are well-suited for prompting the language models while making minimal modifications.
In \Cref{fig:question_refine_ice} we provide a few-shot prompt for question refinement.

%%%%%%%%%%%%%%%%%%%%%%%%%%%%%%%%%%%%%%%%%%%%%%%%%%
\begin{figure}[ht]
    \captionsetup{font=small}
    \includegraphics[width=\linewidth]{figures/appendix_refine_ice.pdf}
    \vspace{-15pt}
    \caption{
    Few-shot prompt for refining the question to suit a language model prompting. An instruction is designed to make a minimal change to the original question, and in-context examples are provided.
    }
    \label{fig:question_refine_ice}
    \vspace{-5pt}
\end{figure}
%%%%%%%%%%%%%%%%%%%%%%%%%%%%%%%%%%%%%%%%%%%%%%%%%%

%%%%%%%%%%%%%%%%%%%%%%%%%%%%%%%%%%%%%%%%%%%%%%%%%%
\begin{figure}
    \centering
    \captionsetup{font=small}
    \includegraphics[width=0.8\linewidth]{figures/appendix_tsne.pdf}
    \caption{Embeddings of questions from OpinionQA, \OURDATA-Train, and \OURDATA-Eval.}
    \label{fig:question-embs}
\end{figure}
%%%%%%%%%%%%%%%%%%%%%%%%%%%%%%%%%%%%%%%%%%%%%%%%%%

In Figure \ref{fig:question-embs}, we visualize the embeddings of the question texts (projected to 2-dimensions using t-SNE) from OpinionQA compared to \OURDATA-Train and \OURDATA-Eval.
The visualization shows how much larger our dataset is than OpinionQA (6.5$\times$), along with the expanded coverage of our dataset into semantic areas untouched by OpinionQA. 
The embeddings also reveal the distribution shift from ATP questions to GSS questions: while the ATP and GSS questions overlap in embedding space, the GSS question appear as small clusters, not evenly distributed over the ATP questions. 
In Table \ref{table:subpop-train-detail}, we list each ATP wave in \OURDATA-Train and OpinionQA, along with its number of questions and wave topic(s), as defined by ATP.\footnote{ATP wave topics and time periods are defined at \url{https://www.pewresearch.org/american-trends-panel-datasets/}.} 
The table indicates which topics are new in \OURDATA-Train compared to OpinionQA, indicating the expanded coverage of our dataset, along with which topics remain unseen in OpinionQA, which we can use to test LLMs fine-tuned on \OURDATA-Train for generalization.

\subsection{\OURDATA-Eval}
\label{appendix_outdata_gss}
To further evaluate the out-of-distribution generalization ability of our fine-tuned models, we subsample 133 questions from the GSS 2022 dataset \cite{davern2024gss}.
We apply the same selection criteria as outlined in ~\Cref{appendix_ourdata_atp}, excluding questions that are redacted, conditioned on prior questions, inferable directly from the group information, derived from a set of questions, or those with more than 10 response options.

\subsection{Inspection of Identical Questions}
Distribution of cosine similarities between two text embeddings (an output of the embedding model OpenAI-text-embedding-3-large given a question text), one from a question in \OURDATA-Train and another from OpinionQA is shown in Figure~\ref{fig:cosine_sim_distribution}.
We observed a fraction of pairs having high cosine similarity,
and manually inspected question pairs with high relevance. We find that by setting a threshold cosine similarity of 0.87 we can detect all semantically identical pairs.
We took a conservative threshold of cosine similarity; this value was to maximize the recall at a cost of precision to ensure detection of overlapping questions.

%%%%%%%%%%%%%%%%%%%%%%%%%%%%%%%%%%%%%%%%%%%%%%%%%%
\begin{figure}[!t]
    \captionsetup{font=small}
    \includegraphics[width=\linewidth]{figures/appendix_cosine_dist.pdf}
    \vspace{-15pt}
    \caption{
    Distribution of cosine similarities between a question in \OURDATA-ATP and OpinionQA, having a long tail towards a high cosine similarity.
    We inspect the question pairs in the range of 0.8 to 1.0 (distribution shown in the magnified view) and use a similarity of 0.87 as a safe threshold to identify a semantically identical question pair.
    }
    \label{fig:cosine_sim_distribution}
    \vspace{-5pt}
\end{figure}
%%%%%%%%%%%%%%%%%%%%%%%%%%%%%%%%%%%%%%%%%%%%%%%%%%