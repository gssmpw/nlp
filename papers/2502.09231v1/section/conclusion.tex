\section{Some Results}
We implemented prototypes of both SharpASP, on top of the existing propositional model counter SharpSAT-TD (denoted as SharpASP (STD) and ApproxASP, on top of ASP solver Clingo. Finally, we empirically evaluate their performance against existing counting benchmarks used in answer set counting literature~\cite{FHMW2017,EHK2024,ACMS2015}.

\paragraph{SharpASP}
Our extensive empirical analysis of $1470$ benchmarks demonstrates significant performance gain over current state-of-the-art exact answer set counters. The result demonstrated is presented in Table~$1$ and the rightmost column presents the result of SharpASP. Specifically, by using SharpASP, we were able to solve $1023$ benchmarks with a PAR$2$ score of $3373$, whereas using prior state-of-the-art, we could only solve $869$ benchmarks with a PAR$2$ score of $4285$. A detailed experimental analysis revealed that the strength of SharpASP is that it spends less time in binary constraint propagation while making more decisions compared to off-the-shelf propositional model counters.

\begin{table}[t]
      \centering
      \begin{tabular}{m{4em} m{4em} m{4em} m{4em} m{4em} m{4em}}
      \toprule
      % & & \rotatebox{60}{\clingo} & \rotatebox{60}{DynASP} & \rotatebox{60}{Ganak}  & \rotatebox{60}{ApproxMC} & \rotatebox{60}{ApproxASP} \\ 
      & clingo & ASProb & aspmc+STD & lp2sat+STD & SharpASP (STD)\\
     \midrule 
      \#Solved  & 869 & 188 & 840 & 776 & \textbf{1023}\\
      \midrule
      PAR$2$ & 4285 & 8722 & 4572 & 5084 & \textbf{3373}\\
      \bottomrule
      \end{tabular}
      \caption{The performance comparison of SharpASP vis-a-vis other ASP counters in terms of the number of solved instances and PAR$2$ scores.
        %
      }
      % \label{table:comparison_counter_problem_wise}
\end{table}

\paragraph{ApproxASP}
Table~$2$ presents the result of ApproxASP with state-of-the-art answer set counters.
ApproxASP performs well in disjunctive logic programs.
ApproxASP solved $185$ instances among $200$ instances, while the
best ASP solver clingo solved a total of $177$ instances.  In
addition, on normal logic programs, ApproxASP performs on par with
state-of-the-art approximate model counter ApproxMC.

\begin{table}[t]
\centering
\begin{tabular}{m{1em} m{4em} m{4em}  m{4em}  m{4em}  m{4em} m{4em} } 
\toprule
% & & \rotatebox{60}{\clingo} & \rotatebox{60}{DynASP} & \rotatebox{60}{Ganak}  & \rotatebox{60}{ApproxMC} & \rotatebox{60}{ApproxASP} \\ 
&& Clingo & DynASP & Ganak & ApproxMC & ApproxASP\\
\midrule
\multirow{3}{*}{\rotatebox{90}{Normal}} 
& \#Instances & \multicolumn{5}{|c|}{1500}\\\cmidrule{2-7}
& \#Solved & 738 & 47 & 973 & \textbf{1325} & 1323\\
\cmidrule{2-7}
&PAR-$2$ & 5172 & 9705 & 3606 & \textbf{1200} & 1218\\
\midrule
\multirow{3}{*}{\rotatebox{90}{Disjunc.}} 
& \#Instances & \multicolumn{5}{|c|}{200}\\
\cmidrule{2-7}
& \#Solved & 177 & 0 & 0 & 0 & \textbf{185}\\
\cmidrule{2-7}
& PAR$2$ & 1372 & 10000 & 10000 & 10000 & \textbf{795}\\
\bottomrule
\end{tabular}
\caption{%
  The runtime performance comparison of Clingo, DynASP, Ganak, ApproxMC, and ApproxASP on all considered instances.
  %
}
% \label{tab:table_for_comparison}
\end{table}

% \noindent Further experimental analyses are detailed in their corresponding papers~\cite{KCM2024,KESHFM2022}.

