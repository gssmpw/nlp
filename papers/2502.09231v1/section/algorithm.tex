\section{Current Progress and Future Goals}
We have already engineered two ASP counters: SharpASP~\cite{KCM2024} and ApproxASP~\cite{KESHFM2022}. 
SharpASP\footnote{\url{https://github.com/meelgroup/SharpASP}} is an exact answer set counter and ApproxASP\footnote{\url{https://github.com/meelgroup/ApproxASP2}} is an approximate answer set counter. 

The principal contribution of SharpASP is to design a scalable answer set counter, without a substantial increase in the size of the transformed propositional formula, particularly when addressing circular dependencies. The key idea behind a substantial reduction in the size of the transformed formula is an alternative yet correlated perspective of defining answer sets. This alternative definition formulates the answer set counting problem on a pair of Boolean formulas $(F, G)$, where the formula $F$ over-approximates the search space of answer sets, while the formula $G$ exploits {\em justifications} to identify answer sets correctly. We set $F = \completion{P}$ since every answer set satisfies Clark completion. Note that $\completion{P}$ overapproximates answers sets of $P$. We propose another formula, named {\em copy formula}, denoted as $\copyoperation{P}$, which comprises a set of (implicitly conjoined) implications defined as follows:
\begin{enumerate}
\item \label{l1:type1} (type 1) for every $v \in \loopatoms{P}$, the implication $\copyatom{v} \rightarrow v$ is in $\copyoperation{P}$.
\item \label{l1:type2} (type 2) for every rule $x \leftarrow a_1, \ldots a_k, b_1, \ldots b_m, \sim c_1, \ldots \sim c_n$ in $P$, where
%  \begin{itemize}
  %    \item
  $x \in \loopatoms{P}$,
  %    \item
  $\{a_1, \ldots a_k\} \subseteq \loopatoms{P}$ and
  %  \item
  $\{b_1, \ldots b_m\} \cap \loopatoms{P} = \emptyset$,
%  \end{itemize}
  the implication $\copyatom{a_1} \wedge \ldots \copyatom{a_k} \wedge b_1 \wedge 
  \ldots b_m \wedge \neg{c_1} \wedge \ldots \neg{c_n} \rightarrow \copyatom{x}$ is in $\copyoperation{P}$.
\item No other implication is in $\copyoperation{P}$.
\end{enumerate}

For each satisfying assignment $M \models \completion{P}$, we have the following observations:
\begin{itemize}
    \item if $M \in \answer{P}$, then $\copyoperation{P}_{|M} = \emptyset$
    \item if $M \not\in \answer{P}$, then $\copyoperation{P}_{|M} \neq \emptyset$
\end{itemize}
where $\copyoperation{P}_{|M}$ denotes the {\em unit propagation} of $M$ on $\completion{P}$. We integrate these observations into propositional model counters to engineer an answer set counter.



% Additionally, the alternation definition of answer set admits to lift core ideas like decomposability and determinism in propositional model counters to facilitate answer set counting. While existing \#SAT-based answer set counters ensure a one-to-one correspondence between answer sets of $P$ and models of $F$ adapting heavy-weight propositional encodings, our alternative definition avoids such overhead by seeking justifications. Following ASP counter ASProblog~\cite{ACMS2015}, SharpASP adopts a {\em component-caching} based propositional model counting algorithms to ASP counting. From an implementation perspective, our proposed ASP counting framework requires minimal adjustments to existing propositional model counters.

Within ApproxASP, we present a scalable approach to approximate the number of answer sets.
  %
Inspired by approximate model counter ApproxMC~\cite{CMV2013}, our approach is based on systematically adding parity (XOR) constraints to
ASP programs to divide the search space uniformly and independently. We prove that adding random XOR constraints partitions the 
answer sets of an ASP program. 
When a randomly chosen partition is {\em quite small}, we can approximate the number of answer sets by simple multiplication. 
The XOR semantic in answer set programs was initiated by Everardo et al.~\cite{EJKS2019}.
In practice, we use a {\em Gaussian
elimination}-based approach by lifting ideas from SAT to ASP and
integrating them into a state-of-the-art ASP solver.


Our objective is to develop more efficient answer set counters by integrating specialized ASP counting techniques and advanced preprocessing methods. Furthermore, we are dedicated to enhancing the capabilities of SharpASP, currently limited to handling normal programs, to also support disjunctive answer set programs. In addition, we are eager to explore broader applications of ASP counting to demonstrate its versatility and potential in solving complex problems. We are also eager to extend the counting technique in other theories~\cite{KM2024}.