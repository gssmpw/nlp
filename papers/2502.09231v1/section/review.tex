\section{Related Works}
\label{section:relatedwork}
The decision version of normal logic programs is NP-complete; therefore, the ASP counting for normal logic programs is \#P-complete~\cite{valiant1979} via a polynomial reduction~\cite{JN2011}.  Given the \#P-completeness, a prominent line of work focused on ASP counting relies on translations from the ASP program to a CNF formula~\cite{LZ2004,Janhunen2004,Janhunen2006,JN2011}. Such translations often result in a large number of CNF clauses and thereby limit practical scalability for {\em non-tight} ASP programs. 
% Lin and Zhao~\cite{LZ2004},  Janhunen~\cite{Janhunen2006} and Janhunen and Niemel{\"a}~\cite{JN2011} offer straightforward
%thereby allowing for the use of a counter on the CNF formula to compute the answer set count.
%However, such translation results in a large number of clauses for programs that are not ``tight''.
%To address the blowup for {\em non-tight} programs, 
Eiter et al.~\cite{EHK2024} introduced T$_{\mathsf{P}}$-\textit{unfolding} to break cycles and produce a tight program. They proposed an ASP counter called aspmc, that performs a treewidth-aware Clark completion from a cycle-free program to a CNF formula. Jakl, Pichler, and Woltran~\cite{JPW09} extended the tree decomposition based approach for \#SAT due to Samer and Szeider~\cite{SS2010} to ASP and proposed a {\em fixed-parameter tractable} (FPT) algorithm for ASP counting. 
Fichte et al.~\cite{FHMW2017,FH2019} revisited the FPT algorithm due to Jakl et al.~\cite{JPW09}  and developed an exact model counter, called DynASP, that performs well on instances with {\em low treewidth}. 
Aziz et al.~\cite{ACMS2015} extended a propositional model counter to an answer set counter by integrating unfounded set detection. 
ASP solvers~\cite{GKS2012} can count answer set via enumeration, which is suitable for a sufficiently small number of answer sets.
Kabir et al.~\cite{KESHFM2022} focused on lifting hashing-based techniques to ASP counting, resulting in an approximate counter, called ApproxASP, with $(\varepsilon,\delta)$-guarantees. 
Kabir et al.~\cite{KCM2024} introduced an ASP counter that utilizes a sophisticated Boolean formula, termed the copy formula, which features a compact encoding.
