\newcommand{\rely}{\mathcal{R}}
\newcommand{\guarantee}{\mathcal{G}}
\newcommand{\stable}{\mathbf{stable}}
\section{Extension to Rely-Guarantee Reasoning}
The rely-guarantee (RG for short) reasoning~\cite{DBLP:phd/ethos/Jones81} is a compositional technique used to verify large systems, including those with concurrent threads. It extends Hoare logic~\cite{DBLP:journals/cacm/Hoare69} by incorporating assumptions about the behavior of the environment (rely conditions) and guarantees about the behavior of the executing (sub-)program, a.k.a, thread, (guarantee conditions).
%
Using this technique, conditions have the form $\rely$ (rely), $\guarantee$ (guarantee), $\alpha$ (precondition), $\beta$ (postcondition), and, we use $\Knows_{\tid}\ \alpha$ for the epistemic modality. 
The satisfaction condition is as expected:

\begin{proposition}
We say $\point \models_{\pi} \Knows_{\tid}\alpha$ holds, if for all points $\point' = \langle h',i \rangle$ for $\pi$ and $\tid$, if $\point= \langle h,i\rangle$ and $h\Restrict [0,i] = h'\Restrict [0,i]$ then $\point' \models \alpha$.
\end{proposition}

Moreover, to ensure that the thread $\tid$'s  knowledge of a given property $\phi$ persists under all environment actions allowed by $\rely$:
\begin{definition}
We say a formula $\phi$ is stable under $\rely$ if 
\[
\rely \vdash\ \Knows_{\tid}\ \phi\ \implies \Always(\Knows_{\tid}\ \phi)
\]
\end{definition}

In the rely-guarantee reasoning, an assertion is a judgment of the form $\rely,\guarantee \vdash \alpha\ \{\tid:c\}\ \beta$. The intention is that in the context of a given large and multithreaded program (giving the context that allows us to evaluate the epistemic modality) and under the assumption that the environment satisfies the rely condition $\rely$, thread $\tid$ sometimes starts executing command associated with $c$ in a state satisfying $\alpha$, throughout the execution the condition $\guarantee$ hold and the final state satisfy the property $\beta$.
%
%
To put this satisfaction condition formally,

\begin{proposition}
We say $\rely,\guarantee \vdash \alpha \{\tid:c\}\beta$ is satisfied, if for all runs $\pi$, and the given execution point $i$ such that $\Control(\pi(i),\tid) = c$ and $\History(\pi,\tid) = \langle h,f\rangle$ and
$\langle h,f(i) \rangle \models_{\pi,\tid} \alpha$:
\begin{enumerate}
    \item If the environment satisfies $\rely$ for all $j\geq f(i)$
    \item Then, $\tid$ maintains $\guarantee$ throughout the execution, i.e., $\langle h,j \rangle \models_{\pi,\tid} \guarantee$ for all $j\geq f(i)$
    \item If execution terminates, say in $k$, then  $\langle h,k \rangle \models_{\pi,\tid} \beta$
    \item And, $\forall \pi, i.\ \pi,i\models\ \alpha, \Always\ \Knows_{\tid}\rely\ \implies\ \pi,i\models\beta,\Always\ \Knows_{\tid}\ \guarantee$
\end{enumerate}
\end{proposition}




We can now define the inference rules, of which here we only present the parallel composition of threads, which is more interesting; defining other rules is straightforward.
%

% \[
% \begin{array}{c}
% \Always\ \Knows_{\tid_1}\rely_1, \Always\ \Knows_{\tid_1}\guarantee_1 \vdash \alpha_1\ \{\tid_1:c_1\}\  \beta_1 \hspace{0.4cm} \guarantee_1 \rightarrow \rely_2\\
% \Always\ \Knows_{\tid_2}\rely_2, \Always\ \Knows_{\tid_2}\guarantee_2 \vdash \alpha_2\ \{\tid_2: c_2\}\  \beta_2 \hspace{0.4cm} \guarantee_2 \rightarrow \rely_1\\ \hline
% \Always\ \Knows_{\tid_1}(\rely_1\! \wedge\! \guarantee_2), \Always\ \Knows_{\tid_2}(\rely_2\! \wedge\! \guarantee_1) \\ \hspace{1cm}\vdash\alpha_1\! \wedge\! \alpha_2 \{\{\tid_1\!:\!c_1\}\! \parallel\! \{\tid_2\!:\!c_2\}\}\ \Knows_{\tid_1} \beta_1 \wedge \Knows_{\tid_2}\beta_2
% \end{array}
% \]

% In a shared-variable concurrent setting, threads operate under partial observation of the global state. The epistemic postcondition $\Knows_{\tid_1} \beta_1 \wedge \Knows_{\tid_2} \beta_2$ explicitly captures the local knowledge each thread derives from its own execution history, which is critical for compositional reasoning. While a global postcondition $\beta_1 \wedge \beta_2$ might seem desirable, it would require verifying that both threads’ guarantees hold simultaneously in all global states, a task that inherently demands non-modular, whole-program analysis.

% By contrast, the epistemic formulation aligns with our proof system: leveraging thread-local knowledge to infer global correctness. The premises $\guarantee_1 \rightarrow \rely_2$ and $\guarantee_2 \rightarrow \rely_1$ ensure that each thread’s guarantee bounds the interference it imposes on the other, preserving the consistency between local knowledge and global truth. 



\[
\begin{array}{c}
\Always\ \Knows_{\tid_1}\rely_1, \Always\ \Knows_{\tid_1}\guarantee_1 \vdash \alpha_1\ \{\tid_1:c_1\}\  \beta_1 \hspace{0.4cm} \guarantee_1 \rightarrow \rely_2\\
\Always\ \Knows_{\tid_2}\rely_2, \Always\ \Knows_{\tid_2} \guarantee_2 \vdash \alpha_2\ \{\tid_2: c_2\}\  \beta_2 \hspace{0.4cm} \guarantee_2 \rightarrow \rely_1\\ \hline
\Always\ \Knows_{\tid_1}(\rely_1\! \wedge\! \stable(\guarantee_2)), \Always\ \Knows_{\tid_2}(\rely_2\! \wedge\! \stable(\guarantee_1)) \\ \hspace{-4cm}\vdash\alpha_1\! \wedge\! \alpha_2 \{\{\tid_1\!:\!c_1\}\! \parallel\! \{\tid_2\!:\!c_2\}\}\ \beta_1 \wedge \beta_2
\end{array}
\]

Where, $\stable(\guarantee_i)$ asserts that $\guarantee_i$ is epistemically stable under $\rely_j$.
In shared-variable concurrency, postconditions often depend on the global state, such as mutual exclusion or shared-variable invariants. While epistemic modalities ($\Knows_{\tid}$) are useful for local reasoning within each thread, the parallel composition rule must ensure that the combined behavior of threads satisfies global properties. To this end, we define the postcondition of the parallel composition rule as $\beta_1 \wedge \beta_2$ 
, where $\beta_1$ and $\beta_2$ are the postconditions of threads $\tid_1$
and $\tid_2$, respectively. This is justified by the rely-guarantee conditions ($\guarantee_1 \rightarrow \rely_2$ and $\guarantee_2 \rightarrow \rely_1$), which ensure that each thread's actions preserve the other's assertions, allowing us to lift local guarantees to global correctness. For example, in Peterson's algorithm, the mutual exclusion property is a global invariant that cannot be inferred solely from thread-local knowledge but follows from the compatibility of $\tid_1$'s and $\tid_2$'s rely-guarantee contracts.