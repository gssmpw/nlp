\section{Program Model}
\label{sec:ProgramModel}

Having introduced the basic proof system, we now move on to instantiate this general framework to a specific model, i.e., for a
statically parallel structured assembly-like language $\PLanOne$, defined by the following abstract syntax:
\begin{eqnarray*}
\Rn\in\Registers & ::= & R\{i\}\ \text{ for } 0 \leq i \leq 31\\
e\in\PExp & ::= & w\ |\ !\Rn\ |\ \Rn\ |\ \Rn\ \Op\ \Rn \ \\
\atm{}\in\AtomicCmd &::= & \Rn\ \ReadFrom\ e\ |\ \Rn\ \WriteTo\ \Rn \\
c\in\Cmd & ::= &  \atm{}\ |\ c ; c\ |\ \If\ \Rn\ c\ c\ |\ \While\ \Rn\ \{c\}
\end{eqnarray*}
where $w\in\Word$ is a word representing a constant value, $\Rn$ denotes the register \#$n$, $!$ is used for pointer dereferencing, and $\Op$ represents binary operations (left open).
%
Atomic commands in $\AtomicCmd$ correspond to move, load constant, and memory load and store.
The primary intention is to ensure that atomic commands correspond to, at most, one memory access.  
Branching tests on values in the register $\Rn$. Commands are labeled consecutively by labels $\Label\in\Labels$, and we write $\Label:c$ if $c$ is labeled $\Label$. The initial label for thread $\tid$ is $\Label_{\tid,0}$.

We give a rewrite semantics of static thread pools and map it into the model structure introduced in Section \ref{sec:ModelStructure}. 
The shared resource is now a store of the shape $\store : \Store = \Word \rightarrow \Word$ along with an event $\alpha$ recording the most recent memory access, if
any, and which thread performed it:
\[
\alpha \in \Events ::= \varepsilon\ |\ A : w_1 \ReadFrom w_2 \ |\ A : w_1 \WriteTo w_2\ 
\]

The events provide information needed for a specification that cannot be reliably inferred from other parts of the control state. For example, a transition by a thread $A$ may read $w_1$ from $w_2$ and assign $w_1$ to $R3$. There is no information in the control state outside the event itself to indicate that this control state change was due to a read and not some other computation step, and this information can be crucial for verification.

States now have the shape $\state = \config_{\tid_1}\parallel \cdots \parallel \config_{\tid_n}\parallel \store \parallel \alpha$.
Configurations have the shape $\config=\langle \tid,c,\regs\rangle$ where the control state now consists of a command $c$ with a register assignment $\regs:\Rn\mapsto w\in\Word$ of words to registers $\Rn$. With $\state$ and $\config$ as above we let $\Event(\state) = \alpha$, $\store(\state)=\store$, $\Command(\state,\tid)=c$, and $\Registers(\state,\tid)=\regs$.

The local transition structure is defined in Figure \ref{RewriteRelation}. It is rather straightforward, despite the symbol pushing, and assumes that ``;'' is associative, for simplicity.

\begin{figure}[t]
{\small
\[
\begin{array}{l}
\Next(\langle\tid,\Rn\ \ReadFrom\ w;c,\regs\rangle,\store\parallel\alpha)\;\; = \langle \langle \tid,c,\regs[\Rn\mapsto w]\rangle,\store\parallel\varepsilon\rangle \\[3pt]

\Next(\langle\tid,\Rn\ \ReadFrom\ !\Rm;c,\regs\rangle,\store\parallel\alpha) = \langle \langle \tid,\Rn \ReadFrom\store(\regs(\Rm));c,\regs\rangle,\store\parallel \tid:\Rn \ReadFrom \store(\regs(\Rm))\rangle \\[3pt]

\Next(\langle\tid,\Rn\ReadFrom \Rm;c,\regs\rangle,\store\parallel\alpha)\;\; = \langle \langle\tid,\Rn \ReadFrom\regs(\Rm);c,\regs\rangle,\store\parallel\varepsilon\rangle \\[3pt]

\Next(\langle\tid,\Rn\ReadFrom \Rm\ \Op\ \Rk;c,\regs\rangle,\store\parallel\alpha) = \langle \langle\tid,\Rn \ReadFrom\regs(\Rm)\ \Op\ \regs(\Rk));c,\regs\rangle,\store\parallel\varepsilon\rangle \\[3pt]

\Next(\langle\tid,\Rn\WriteTo \Rm;c,\regs\rangle,\store\parallel\alpha)\;\; = \langle \langle \tid,c,\regs\rangle,\store[\regs(\Rm) \mapsto \regs(\Rn)]\parallel\tid:\regs(\Rn)\ \WriteTo \regs(\Rm) \rangle 
\end{array}
\]
\[
\begin{array}{ll}
\mbox{If }\regs(\Rn) \neq 0\mbox{ then } \Next(\langle\tid,\If\ \Rn\ c_1\ c_2 ; c,\regs\rangle,\store\parallel\alpha) = \langle \langle \tid, c_1 ; c,\regs\rangle,\store\parallel\varepsilon \rangle \\[3pt]

\mbox{If }\regs(\Rn) = 0\mbox{ then } \Next(\langle\tid,\If\ \Rn\ c_1\ c_2 ; c,\regs\rangle,\store\parallel\alpha) = \langle \langle \tid,c_2 ; c,\regs\rangle,\store\parallel\varepsilon\rangle \\[3pt]

\mbox{If }\regs(\Rn) \neq 0\mbox{ then } \Next(\langle\tid,\While\ \Rn\ c' ;c,\regs\rangle,\store\parallel\alpha) = \langle \langle \tid,c';\While\ \Rn\ c' ; c,\regs\rangle,\store\parallel\varepsilon\rangle \\[3pt]

\mbox{If }\regs(\Rn) = 0\mbox{ then }\Next(\langle \tid,\While\ \Rn\ c' ; c,\regs\rangle,\store\parallel\alpha)  = \langle \langle \tid,c,\regs\rangle,\store\parallel\varepsilon \rangle
\end{array}
\]
}
%
\caption{Thread nextstate function}
\label{RewriteRelation}
\end{figure}
%
For the axiomatization, we will introduce a couple of helper functions later. First, a (partial) function that extracts the branching condition, when it exists:
\[
\Cond(c',c) =
\left\{
\begin{array}{ll}
\True & \mbox{if }c' = \alpha;c \\
\Rn\neq 0 & \mbox{if }c'=\If\ \Rn\ c_1\ c_2 ; c'' \mbox{ and }c=c_1;c'' \\
\Rn = 0 & \mbox{if }c'=\If\ \Rn\ c_1\ c_2 ; c'' \mbox{ and }c=c_2;c'' \\
\Rn\neq 0 & \mbox{if }c'=\While\ \Rn\ c_1 ; c'' \mbox{ and }c=c_1; \While\ \Rn\ c_1 ; c'' \\
\Rn = 0 & \mbox{if }c'=\While\ \Rn\ c_1 ; c'' \mbox{ and }c=c'' \\
\bot & \mbox{otherwise}
\end{array}
\right.
\]
%
Secondly, a symbolic version of the transition relation returning the prestate label and branching condition:
\[
\PrevState(\Label) = \{\langle \Label',\Cond(c',c) \rangle \mid \Label':c', \Label:c\}
\] 
%

\section{Logic Instantiation}
%
 We now instantiate the atomic propositions of $\LLanZero$ to reflect the structure of $\PLanOne$. The resulting language, $\LLanOne$, has thread variables $A,B$ ranging over $\tid$, and program variables $x,y$ ranging over words. 
%
Abstract formula syntax then becomes:
\begin{eqnarray*}
e\in\LExp & ::= & w\ \mid\ e\ \Op\ e\ |\ !e\ \mid\ x\ |\ \Rn \\
q\in\AProp & ::= & A\ \At\ \Label\ |\ A\ \Says\ e_1 = e_2\ |\ A: e_1\ \ReadFrom\ e_2\ |\ A:e_1\ \WriteTo\ e_2
\end{eqnarray*}
%
We refer to formulas of the shape $A: e_1\ \ReadFrom\ e_2$ or $A:e_1\ \WriteTo\ e_2$ as read or write statements, respectively. Also, we call an expression of the shape $!e$ a reference, and references of the shape $!w$ are fully evaluated.


The primary intention of these constructs is to enable reasoning about thread behavior and its interactions. For example, the proposition $A\ \At\ \Label$ indicates that thread $A$ is at a control state labeled $\Label$. The proposition $A\ \Says\ e_1 = e_2$ means that, when evaluated with respect to thread $A$'s local register assignment, $e_1$ is equal to $e_2$. If neither $e_1$ nor $e_2$ mentions registers, and so is dependent only on the store, we can abbreviate $A\ \Says\ e_1 = e_2$ by just $e_1 = e_2$. 
%
Instantaneous (strong) reading and writing events are captured by propositions such as $A: e_1\ \ReadFrom\ e_2$ and $A:e_1\ \WriteTo\ e_2$.

To enhance expressiveness, we introduce a few abbreviations. Weak versions of the reading and writing primitives are defined to express properties over past states. For instance, $A\ \Reads$ ($A\ \Writes$) indicates that there exists the memory location $x$ from which thread $A$ read (resp. writes into $x$) some value at some point in the past. Other reading and writing primitives include:

\begin{itemize}
\item $A\ \Reads\ e_1\ \From\ e_2 \equiv \Prev\ A\ (A:e_1\ \ReadFrom\ e_2)$
\item $A\ \Writes\ e_1\ \To\ e_2 \equiv \Prev\ A\ (A: e_1\ \WriteTo\ e_2)$
\item $A\ \Reads\ x \equiv \exists\ y.\ A\ \Reads\ y\ \From\ x$
\item $A\ \Writes\ x \equiv \exists\ y.\ A\ \Writes\ y\ \To\ x$
\item Similarly, 
the recently-wrote connective: 
$A\ \RecentlyWrote\ e\ \To\ x$ 
indicates that thread $A$ performed an assignment to $x$ in the past, and since then, has not written to $x$ again, i.e., $\neg(A\ \Writes\ x)\ \Since\ A\ \Writes\ e\ \To\ x\ .$
\end{itemize}

The extended semantics is shown in Figure \ref{LLanOneSemantics} and uses the function $\Sem{e}(\store ,\regs ,\rho)$ to evaluate expressions in
the obvious way.
%
\begin{figure}[t]
\begin{itemize}
\item $\pi,i\models_{\rho}A\ \At\ \Label$, if $\Label:\Command(\pi(i),\rho(A))$,
\item $\pi,i\models_\rho  A\ \Says\ e_1 = e_2$, if $w_1 = w_2$,
\item $\pi,i\models_\rho A: e_1\ \ReadFrom\ e_2$, if
$\Event(\pi,i+1) = \rho(A):w_1\ReadFrom w_2$,
\item $\pi,i\models_\rho A :\ e_1\WriteTo \ e_2$, if 
$\Event(\pi,i+1) = \rho(A):w_1\WriteTo w_2$,
\end{itemize}
where, in the above clauses, $w_j = \Sem{e_j}(\store(\pi(i)),\regs(\pi(i),\rho(A)),\rho)$, $j\in\{1,2\}$.
\caption{Satisfaction conditions for $\LLanOne$}
\label{SatisfactionDef}
\label{LLanOneSemantics}
\end{figure}
 

We then move on to get to the interesting part, which concerns the epistemic modalities. The epistemic modalities allow us to reason about the knowledge acquired by threads based on their observations. For example, suppose $A\ \Reads\ x$. In this case, if a thread ever wrote something to $x$ not equal to $!x$ then this must have happened before $!x$ was written,  by the same thread or a different one. And, this is known by $A$. In other words:
%
  \begin{eqnarray*}
  A\ \Reads\ y\ \From\ x & \Implies & \Sometime\ B\ \Writes\ z\ \To\ x\ \wedge\ z\neq y \\
   & \Implies & A\ \Knows\ (\exists\ C.\ B\ \Writes\ z\ \To\ x \HappensBefore C\ \Writes\ y\ \To\ x\ \\
   & &  \HappensBefore\ A\ \Reads\ y\ \From\ x)
  \end{eqnarray*}
Note that since we are working with a static and finite collection of threads, the existential quantifier is immediately eliminated in favour of a big disjunction.

\subsection{Example: Peterson Algorithm}

In this section, we illustrate the application of our logic and program model using the well-known Peterson's algorithm, a mutual exclusion algorithm designed for two threads. Peterson's algorithm ensures that two threads do not enter their critical sections simultaneously. This example helps demonstrate how our logic can be used to specify and verify concurrent programs.

As Figure \ref{PetersonFigOne} depicts, the Peterson's algorithm involves two threads, each trying to enter a critical section. This program can be easily translated into the program model of Section~\ref{sec:ProgramModel}. The threads communicate through shared variables, specifically an array \texttt{flag} and a variable \texttt{victim}. Each thread follows a sequence of steps to set its intention to enter the critical section and then waits until it is safe to proceed. The logic for each thread involves setting a \texttt{flag}, updating the \texttt{victim} variable, and then spinning in a loop until it is safe to enter the critical section.

To specify the desired properties of the algorithm, we focus on two main aspects: (1) the knowledge each thread has about its environment and (2) the specification of the global properties we wish to verify. For example, we want to ensure that if one thread has entered its critical section, the other thread cannot enter its critical section simultaneously.

We use the constructs of our logic to formalize these properties. For thread 
$A$, we define the conditions under which it knows certain facts about the shared variables. For instance, thread  $A$ knows the value of the \texttt{flag} and \texttt{victim} variables based on its own actions and the actions of thread $B$. We express this knowledge using epistemic modalities, allowing us to reason about what each thread can infer from its observations.
%
\begin{figure}[t]
\begin{lstlisting}[xleftmargin=1em, numbers=left, numberstyle=\color{red}]
Code for thread A:
work:      while True {
lock:      flag[0] = 1 ;
           victim = 0 ;
           while (flag[1] = 1 && victim = 0) {} ;
body:      noop ;
unlock:    flag[0] = 0 } ;

Code for thread B:
work:      while True {
lock:      flag[1] = 1 ;
           victim = 1 ;
           while (flag[0] = 1 && victim = 1) {} ;
body:      noop ;
unlock:    flag[1] = 0 } ;
 \end{lstlisting}
 \caption{The Peterson algorithm.}
 \label{PetersonFigOne}
 \end{figure}
%
% In Figure \ref{PetersonFigTwo} this algorithm is translated to the program syntax.
% \begin{figure}
% {\small
% \begin{lstlisting}[xleftmargin=1em, multicols=2]
% Thread A: 
% 1: noop   /*workA*/ 
% 2:  0 <- 1 
% 3:  1 <- 0 
% 4:  0 -> 1 /*flag[0]=1*/
% 5:  0 <- 0  
% 6:  1 <- 2  
% 7:  0 -> 1 /*victim=0*/
% 8:  0 <- 1 /*whileA*/
% 9:  1 <- !1  
% 10: 2 <- 0 = 1 /*flag[1]=1*/
% 11: 0 <- 0
% 12: 1 <- !2 /*victim*/
% 13: 3 <- 0 = 1 /*victim=0*/
% 14: 2 <- 2 and 3
% 15: jnz 2 8
% 16: noop   /*bodyA*/
% 17: 0 <- 0 /*unlockA*/
% 18: 1 <- 0
% 19: 0 -> 1
% 20: j 1
% Thread B:
% 21:  noop   /*workB*/
% 22:  0 <- 1 /*lockB*/
% 23:  1 <- 1 
% 24:  0 -> 1 /*flag[1]=1*/
% 25:  0 <- 1  
% 26:  1 <- 2  
% 27:  0 -> 1 /*victim=1*/
% 28:  0 <- 1 /*whileB*/
% 29:  1 <- !0  
% 30:  2 <- 0 = 1 /*flag[1]=1*/
% 31:  0 <- 1
% 32:  1 <- !2 /*victim*/
% 33:  3 <- 0 = 1 /*victim=0*/
% 34:  2 <- 2 and 3
% 35:  jnz 2 28
% 36:  noop   /*bodyB*/
% 37:  0 <- 0 /*unlockB*/
% 37:  1 <- 1
% 38:  0 -> 1
% 39:  j 21
% \end{lstlisting}
% }
%  \caption{Peterson. Locations: flag[0] = 0, flag[1] = 1, victim = 2. }
%  \label{PetersonFigTwo}
%  \end{figure}
% %
% As usual the low level representation is much more inscrutable than the source code so for the time being we refer to the latter.
%
For Peterson, there would seem to be two important pieces of knowledge $A$ has:
%
\begin{enumerate}
\item $!\FlagZero = x \Implies A\ \Knows\ (!\FlagZero = x\ \Since\ (A\ \Writes\ y\ \To\ \FlagZero \vee (\Init\ A \wedge x=0))$
\item $!\Victim = 0 \Implies A\ \Knows\ (!\Victim = 0\ \Since\ (A\ \Writes\ 0\ \To\ \Victim \vee (\Init\ A \wedge (!\Victim = 0))))$
\end{enumerate}
%
The intuition should be clear. For instance, for 1, due to the MRSW\footnote{Here by MRSW, we mean multiple reader single writer.} property of $\FlagZero$,
if $\FlagZero$ has the value $x$ then $x$ has remained the value of $\FlagZero$ since it was written to by $A$ (or else $x$ is 0 and
$A$ has not written to $\FlagZero$ yet).

We first try to prove mutual exclusion. For the global proof goal, define first:
%
\begin{eqnarray*}
A\ \EntersCS & \equiv & ((!\FlagOne \neq 1) \vee (!\Victim\ \neq 0)) \wedge \mbox{} \\
 & & A\ \RecentlyWrote\ 1\ \To\ \FlagZero \wedge \mbox{} \\
  & & A\ \RecentlyWrote\ 0\ \To\ \Victim \wedge \mbox{}  \\
  & & A\ \Writes\ 1\ \To\ \FlagZero \HappensBefore A\ \Writes\ 0\ \To\ \Victim \\
B\ \EntersCS & \equiv & ((!\FlagZero \neq 1) \vee (!\Victim\ \neq 1)) \wedge \mbox{} \\
 & & B\ \RecentlyWrote\ 1\ \To\ \FlagOne \wedge \mbox{} \\
  & & B\ \RecentlyWrote\ 1\ \To\ \Victim \wedge \mbox{}  \\
  & & B\ \Writes\ 1\ \To\ \FlagOne \HappensBefore B\ \Writes\ 1\ \To\ \Victim
\end{eqnarray*}
%
Observe that these two properties do not quite capture ``being in the critical region'' in the sense of the program counter having taken the spin loop
exit branch and not yet having started to unlock. For instance, $A\ \EntersCS$ also holds in a state where $A$ has completed its writes to
\texttt{flag[0]} and \texttt{victim} and where the spin loop exit conditions hold, but where the while loop has not yet been entered, which is not
normally regarded as part of the critical section. However, from such a state (where $A$ has completed its writes, etc.), it is possible for $A$ to
enter the critical section without any involvement by the environment, which is why we have to eliminate such states in an account such as here
where we are able to speak only about threads local view of their execution.

Now we can express mutual exclusion quite simply as follows:
\begin{equation}
\label{160222.1}
A\ \EntersCS \Implies \neg(B\ \EntersCS)\ .
\end{equation}

This statement is non-epistemic. In fact, we will prove, in Section~\ref{sec:petersoneg}, instead the epistemic property
$A\ \EntersCS \Implies A\ \Knows\ \neg(B\ \EntersCS)$ from which (\ref{160222.1}) follows by {\sc T}.
%
\section{Extended Inference System}
%
We next extend the proof system introduced in section \ref{sec:ProofSystem}. The class of models is now specialized to those supported by the program model of section \ref{sec:ProgramModel}. 

\subsection{Equations}
%
 First, we have logical omniscience, i.e. any universally valid equation is known:
 %
 \[
\begin{array}{cc}
\mbox{\sc =I}
&
 \begin{array}{c}
  \\ \hline
  \Gamma\vdash A\ \Says\ e_1 = e_2
 \end{array}
 \end{array}
 \]

 The side-condition here is that $e_1 = e_2$ is universally valid (valid for any assignment to variables, registers, memory locations). Similarly, we have that:
 %
 \[
\begin{array}{cc}
\mbox{\sc $\neq$I}
&
 \begin{array}{c}
  \\ \hline
  \Gamma\vdash A\ \Says\ e_1 \neq e_2
 \end{array}
 \end{array}
 \]
 with a side-condition that $e_1 \neq e_2$ is universally valid.
 
Second, we have some highly circumscribed abilities to substitute equals for equals. Let us call a formula $\phi(x)$ an \emph{$A$-context}, if $x$ occurs
only in the scope of an equality $A\ \Says\ e_1 = e_2$ or a read or write statement for thread $A$, and not in the scope of one of the modal operators $\Knows$, $\Prev$, or $\Since$. We obtain:
%
\[
\begin{array}{ccc}
\mbox{\sc =E}
&
\begin{array}{c}
\Gamma\vdash A\ \Says\ e = e' \hspace*{1em}  \Gamma \vdash \phi[e/x] \\ \hline
\Gamma \vdash \phi[e'/x]
\end{array}
&
(\phi(x)\ \mbox{is an $A$-context})
\end{array}
\]

Using  the {\sc K} operator, we also easily derive the following rule:
%
\[
\begin{array}{ccc}
\mbox{\sc =EK}
&
\begin{array}{c}
\Gamma\vdash A\ \Knows\ A\ \Says\ e = e' \hspace*{1em}  \Gamma \vdash A\ \Knows\ \phi[e/x] \\ \hline
\Gamma \vdash A\ \Knows\ \phi[e'/x]
\end{array}
&
(\phi(x)\ \mbox{is an $A$-context})
\end{array}
\]

Note that more general versions of {\sc =E} where $x$ is allowed to appear in a modal context are unsound. For instance, we may obtain that $\vdash A\ \Says\ 4=!3$ and
$\vdash A\ \Knows\ x=4[4/x]$, but $\vdash A\ \Knows\ !3 = 4$ is false (because some other thread might have written to location 3). Similar examples may be given
for $\Prev$ and $\Since$.
%


\subsection{Label Statements}
We then introduce inference rules related to label statements within our program model to reason about the control flow of threads by tracking their locations and transitions. 
%
\[
\begin{array}{ccc}
\begin{array}{cc}
\mbox{\sc LabelI1}
&
\begin{array}{c}
\Gamma \vdash \Init \\ \hline
\Gamma \vdash A\ \At\ \Label_{A,0}
\end{array}
\end{array}
&
\begin{array}{cc}
\mbox{\sc LabelI2}
&
\begin{array}{c}
- \\ \hline
\Gamma \vdash \bigvee\{A\ \At\ \Label\mid \Label\in\Labels\}
\end{array}
\end{array}
\end{array}
\]
%
\[
\begin{array}{cc}
\mbox{KAt}
&
\begin{array}{c}
\Gamma \vdash A\ \At\ \Label \\ \hline
\Gamma \vdash A\ \Knows\ A\ \At\ \Label
\end{array}
\end{array}
\]
\[
\begin{array}{cc}
\mbox{\sc PrAtI1} 
&
\begin{array}{c}
\Gamma \vdash A\ \At\ \Label \hspace*{1cm}
\Gamma \vdash \Prev\ A\ \Active \\ \hline
\Gamma \vdash \bigvee\{\Prev\ (A\ \At \ \Label'\ \wedge \phi) \mid \langle \Label',\phi\rangle \in \PrevState(\Label)\}
\end{array}
\end{array}
\]
\[
\begin{array}{cc}
\mbox{\sc PrAtI2} 
&
\begin{array}{c}
\Gamma \vdash A\ \At\ \Label \hspace*{1cm}
\Gamma \vdash \Prev\ \neg (A\ \Active) \\ \hline
\Gamma \vdash \Prev\ A\ \At\ \Label
\end{array}
\end{array}
\]
% 
\[
\begin{array}{ccc}
\mbox{\sc PrAAtE} 
&
\begin{array}{c}
\Gamma \vdash \Prev\ A\ (A\ \At\ \Label \wedge \Cond(c,c')) \\ \hline
\Gamma \vdash A\ \At \ \Label'
\end{array}
&
(\Label:c, \Label':c')
\end{array}
\]
%
\paragraph{Soundness.} Here we only show the soundness of {\sc PrAAtI1}; others rules can be  proved, similarly.  

Assume $\pi,i\models_\rho A\ \At\ \Label$, i.e. $\Label: \Command(\pi(i),\rho(A))$. Let $j$ be the previous thread state
for $A$ w.r.t. $\langle\pi,i\rangle$. There must be some $\langle\Label',\phi\rangle\in \PrevState(\Label)$ such that $\Label':\Command(\pi(j),\rho(A))$ and
$\pi,j\models_\rho A\ \Says\ \phi$. But then by Proposition \ref{160329.1},  the antecedent of {\sc PrAAtI1} holds.
The proof for rule {\sc PrAAtE} is equally simple.
%
\subsection{Activity Statements}
Additionally, we introduce a few rules to reason about the active state of threads.
%
\[
\begin{array}{ccc}
\begin{array}{cc}
\mbox{\sc ActiveI2}
&
\begin{array}{c}
\Gamma \vdash A: e_1\ \ReadFrom\ e_2 \\ \hline
\Gamma \vdash A\ \Active
\end{array}
\end{array}
&
\begin{array}{cc}
\mbox{\sc ActiveI3}
&
\begin{array}{c}
\Gamma \vdash A\ :\ e_1\ \WriteTo\ e_2 \\ \hline
\Gamma \vdash A\ \Active
\end{array}
\end{array}
\end{array}
\]
\[
\begin{array}{cc}
\mbox{\sc ActiveE}
&
\begin{array}{c}
\Gamma \vdash A\ \Active \hspace*{1cm}
\Gamma \vdash B\ \Active \hspace*{1cm} 
\Gamma \vdash  A \neq B \\ \hline
\Gamma \vdash \phi
\end{array}
\end{array}
\]
%
 

\subsection{Extend Proof System: Peterson Algorithm}
\label{sec:petersoneg}
%
We now turn to our running example and try to complete the proof using the inference rules we have derived in the previous sections. 

\begin{lemma} For any thread \( A \), the following holds:
\[
    \vdash A\ \EntersCS\ \Implies\ A\ \Knows\ \neg (B\ \EntersCS)
\]
\end{lemma}

\begin{proof}
We proceed through the following steps using the extended proof system:

\begin{enumerate}

\item \textbf{Assumption Decomposition}:
Assume \( A\ \EntersCS \). By conjunction elimination (\(\land\)E):
\begin{align*}
    &1.\ (\lnot!\FlagOne \neq 1) \lor (\lnot!\Victim \neq 0) \\
    &2.\ A\ \Writes\ 1\ \To\ \FlagZero\ \HappensBefore\ A\ \Writes\ 0\ \To\ \Victim \\
    &3.\ \neg(A\ \Writes\ \FlagZero)\ \Since\ A\ \Writes\ 1\ \To\ \FlagZero \\
    &4.\ \neg(A\ \Writes\ \Victim)\ \Since\ A\ \Writes\ 0\ \To\ \Victim
\end{align*}

\item \textbf{Epistemic Foundation}:
When \( A\ \At\ \Label_6 \) (critical section entry):
\begin{proposition}[Local Knowledge]
\label{prop:local}
\[
A\ \At\ \Label_6\ \vdash\ A\ \Knows\ (!\FlagZero = 1\ \land\ !\Victim \neq 1 )
\]
\end{proposition}
\begin{proof}
Apply \(\textsc{=E}\) rule to \( A \)'s write operations, using:
\begin{itemize}
\item From \( A\ \Writes\ 1\ \To\ \FlagZero\ \), we know \( A\ \Knows\ A\ \Says\ !\FlagZero = 1 \)
\item From \( A\ \Writes\ 0\ \To\ \Victim \), we know \(A\ \Knows\ A\ \Says\ !\Victim = 0 \)
\end{itemize}
Since $!\FlagZero = 1$ and $!\Victim = 0$ are universally valid in $A$'s local context (no interference after writes), combining via \(\land\)I and \(\textsc{=EK}\) we can get the final result.
\end{proof}

\item \textbf{Temporal Constraints}:
From \( A\ \Writes\ 1\ \To\ \FlagZero\ \HappensBefore\ A\ \Writes\ 0\ \To\ \Victim \):
\[
A\ \Knows\ \left( B\ \Writes\ 1\ \To\ \FlagOne\ \HappensBefore\ B\ \Writes\ 1\ \To\ \Victim \right)
\]
Because, by the definition of $\HappensBefore$, $B$'s actions cannot interleave between $A$'s writes (fair scheduling). Combined with Proposition~\ref{prop:local}, this precludes \( B\ \EntersCS \).

\item \textbf{Final Derivation}:
Combine results from the previous steps and \(\textsc{T}\) axiom:
\[
A\ \Knows\ \neg(B\ \EntersCS)
\]
By \(\Implies\)-introduction:
\[
\vdash A\ \EntersCS\ \Implies\ A\ \Knows\ \neg(B\ \EntersCS)
\]
\end{enumerate}
\end{proof}





