\section{Proposed Logic}
%
The logic we consider is discrete bounded past-time epistemic temporal logic and the language is $\LLanZero$. 
Thread variables $A,B$ ranging over $\tid$. 
%
Let $q\in\AProp$ be the atomic propositions. The abstract formula syntax of our logic includes the following constructs: 
\begin{eqnarray*}
\phi,\psi\in\Prop & ::= &  q\ |\ A\ \Active\ |\ \neg\phi\ |\ \phi\wedge\psi\ |\ \Prev\ \phi\ |\ \phi\ \Since\ \psi\ |\ A\ \Knows\ \phi
\end{eqnarray*}
%
In this context, $A\ \Active$ is a scheduling predicate indicating that the next transition is performed by thread A.
%
The construct $\Prev\ \phi$ denotes that the formula $\phi$ held at the previous global state.
%
The operator $\phi\ \Since\ \psi$ represents the past-time ``since'' relation, where $\psi$ held at some point in the past, and $\phi$ has held continuously since then, including the current state. The ``since'' operator quantifies over the global time axis to express global constraints, while the ``previous state'' operator quantifies over local time to reason about local thread behavior.
%
Finally, $A\ \Knows\ \phi$ is the labeled epistemic modality saying that $\phi$ should hold from the local observations of thread $A$.
%


Given the operators above, we can derive several others to enhance expressiveness. These include the logical constants $\True$, $\False$, $\vee$, $\Implies$, as well as, $\Init$ which is equivalent to $\neg \Prev\ \True$ and it denotes the initial state. 

Temporal operators can also be derived, such as always in the past $\Always\ \phi$, which signifies that $\phi$ has always held in the past and is defined as $\neg(\True\ \Since\ \neg\phi)$. Conversely, $\Sometime\ \phi\ (\equiv \True\ \Since\ \phi)$ indicates that $\phi$ held at some point in the past. The weak since operator, $\phi\ \Since_w\ \psi\ (\equiv (\phi\ \Since\ \psi) \vee (\Always\ \phi))$, conveys that either $\psi$ held some time in the past and $\phi$ held since, or else no past state exists not satisfying $\phi$.

We also define Lamport's ``happens-before'' relation, $\phi\ \HappensBefore\ \psi\ (\equiv (\neg\phi\ \Since\ (\psi \wedge \neg\phi)) \wedge \Sometime\ \phi$), to expresses that $\phi$ and $\psi$ both happened in the past, and $\phi$ has not held after the last occurrence of $\psi$. A weak version of this relation is given by $\phi\WeakHappensBefore \psi\ (\equiv \neg\phi\ \Since\ (\psi \wedge \neg\phi))$.

Additionally, we introduce the concept of a \textit{previous  $A$-state}, which is useful for expressing properties not of the previous global state but of the state preceding the last transition of thread  $A$. This is denoted by $\Prev\ A\ \phi\ (\equiv \Prev\ (\neg A\ \Active\ \Since\ (\phi \wedge A\ \Active)))$, indicating that sometime in the past, excluding the present state, 
$\phi$ held and thread $A$  performed a computation step, with no subsequent steps by thread $A$.

%
The weak happens-before expresses that $\phi$ happens before $\psi$ or else $\phi$ did not happen at all. 

\subsection{Semantics}
\label{sec:semantics}
The semantics of our discrete bounded past-time epistemic temporal logic is defined based on a satisfaction relation of the form $\pi,i\models_\rho \phi$, where $i\in\omega$ and $\rho$ is an interpretation mapping (i) thread variables $A,B$ to $\tid$'s and (ii) proposition variables $q$ to sets of states. The satisfaction conditions are given in Figure \ref{SatisfactionOneDef}. 

\begin{figure}[t]
\begin{itemize}
\item $\pi,i\models_{\rho}q$, if $\pi(i) \in \rho(q)$.
\item $\pi,i\models_\rho A\ \Active$, if $\Control(\pi(i+1),\rho(A))\neq \Control(\pi(i),\rho(A))$.
\item $\pi,i\models_\rho \Prev\ \phi$, if $i>0$ and $\pi,i-1\models_\rho \phi$.
\item $\pi,i\models_\rho \phi\ \Since\ \psi$, if for some $j:0\leq j\leq i$, $\pi,j\models_\rho \psi$ and for all $k:j < k\leq i$, $\pi,k\models_\rho \phi$.
\item $\pi,i\models_\rho A\ \Knows\ \phi$ if, and only if, for all $\pi',i'$, if $\History(\pi,\rho(A))=\langle h,f\rangle$,
$\History(\pi',\rho(A))=\langle h',f'\rangle$, and $ h\Restrict [0,f(i)] = h'\Restrict [0,f'(i')]$ then $\pi',i' \models_\rho \phi$.
\end{itemize}
\caption{Satisfaction conditions}
\label{SatisfactionOneDef}
\end{figure}
 %

\begin{proposition}
\mbox{}
\begin{enumerate}
\item Suppose $\History(\pi,\rho(A))=\langle h,f\rangle$ and $f(i_1) = f(i_2)$. If
$\pi,i_1\models_\rho A\ \Knows\ \phi$ then $\pi,i_2\models_\rho A\ \Knows\ \phi$.
\item $\pi,i\models_\rho A\ \Knows\ \phi$ if, and only if, for all $\pi',i'$, if $\History(\pi,\rho(A))=\langle h,f\rangle$,
$\History(\pi',\rho(A))=\langle h',f'\rangle$, $ h\Restrict [0,f(i)] = h'\Restrict [0,f'(i')]$, and $f'(i')\neq f'(i'+1)$ then $\pi',i' \models_\rho \phi$.
\end{enumerate}
\end{proposition}
\begin{proof}
For the proof of 1, suppose that $\pi,i_1\models_\rho A\ \Knows\ \phi$. Let $\pi'$, $i'$ be given such that 
$\History(\pi,\rho(A))=\langle h,f\rangle$, $\History(\pi',\rho(A))=\langle h',f'\rangle$, and $ h\Restrict [0,f(i_2)] = h'\Restrict [0,f'(i')]$.
But then $h\Restrict [0,f(i_1)] = h'\Restrict [0,f'(i')]$ and $\pi',i'\models_\rho \phi$.

Moreover, we prove part 2 as follows. Since $h$ is well-defined such an $i'$ exists, and $\pi',i' \models_\rho \phi$ by 1.
\end{proof}

Say that $A$ is active at point $\langle \pi,i\rangle$, if $\pi,i\models_\rho A\ \Active$. Also, let
$j$ be the \emph{previous thread state for} $A$ w.r.t. $\langle \pi,i\rangle$, if
$A$ is active in $\langle\pi,j\rangle$ and $A$ is not active in any point $\langle \pi,k \rangle$ for
which $j < k < i$. We obtain:

\begin{proposition}
\label{160329.1}
$\pi,i\models_\rho \Prev\ A\ \phi$ if, and only if, $\pi,j \models_\rho \phi$, where $j$ is the previous thread state
for $A$ w.r.t. $\langle\pi,i\rangle$ and for all $j<k<i$, $A$ was inactive.
\end{proposition}
\begin{proof}
Straightforward.
\end{proof}

 
\subsection{Inference System}
\label{sec:ProofSystem}
%
A central result of this paper is an inference system to reason about concurrent threads behavior. 
We propose a proof system that is both sound and expressive to capture the complexities of concurrent executions and the knowledge of individual threads. 
We achieve this by developing a Gentzen-style natural deduction proof system, which enables direct proof.

The inference system is built on sequents. In our system, sequents are of the form $\Gamma \vdash \phi$, where the antecedent $\Gamma$ denotes a set of propositions and $\vdash$ represents the deductive relation. We denote the semantics of the judgments by $\Gamma \models \phi$, which assert that if
for all models and all $\pi,i,\rho$, if $\pi,i\models_\rho \bigwedge\Gamma$ then $\pi,i\models_\rho \phi$.

Some of the rules in our proof system are standard, but we include them for completeness:
\[
\begin{array}{ccccc}
\mbox{\sc $\wedge$I}
&
\begin{array}{c}
 \Gamma \vdash \phi \hspace*{1em}  \Gamma \vdash \psi \\ \hline
 \Gamma\vdash \phi\wedge\psi
\end{array}
&
\begin{array}{cc}
\mbox{\sc $\wedge$El}
&
\begin{array}{c}
 \Gamma \vdash \phi_1 \wedge \phi_2 \\ \hline
 \Gamma \vdash \phi_1
\end{array}
\end{array} 
&
\begin{array}{cc}
\mbox{\sc $\wedge$Er}
&
\begin{array}{c}
 \Gamma \vdash \phi_1 \wedge \phi_2 \\ \hline
 \Gamma \vdash \phi_2
\end{array}
\end{array}
&
\begin{array}{cc}
\mbox{\sc $\vee$Il}
&
\begin{array}{c}
 \Gamma \vdash \phi_1\\ \hline
 \Gamma \vdash \phi_1  \vee \phi_2 
\end{array}
\end{array}
\end{array}
\]
\[
\begin{array}{ccc}
\begin{array}{cc}
\mbox{\sc $\vee$Ir}
&
\begin{array}{c}
 \Gamma \vdash \phi_2 \\ \hline
 \Gamma \vdash \phi_2  \vee \phi_2
\end{array}
\end{array}
&
\mbox{\sc $\vee$E}
&
\begin{array}{c}
\Gamma \vdash \phi_1 \vee \phi_2 \hspace*{0.5cm}
\Gamma, \phi_1 \vdash \psi \hspace*{0.5cm} \Gamma, \phi_2 \vdash \psi \\ \hline  
 \Gamma\vdash \psi
\end{array}
\end{array}
\]\[
\begin{array}{ccc}
\begin{array}{cc}
\mbox{\sc $\neg$I}
&
\begin{array}{c}
 \Gamma, \phi \vdash \bot \\ \hline
 \Gamma \vdash \neg\phi
\end{array}
\end{array}
&
\begin{array}{cc}
\mbox{\sc $\neg$E}
&
\begin{array}{c}
 \Gamma \vdash \phi \hspace*{0.5cm} 
 \Gamma \vdash \neg\phi \\ \hline
 \Gamma \vdash \psi
\end{array}
\end{array}
&
\begin{array}{cc}
\mbox{\sc $\neg\neg$}
&
\begin{array}{c}
 \Gamma \vdash \neg\neg\phi \\ \hline
 \Gamma \vdash \phi
\end{array}
\end{array}
\end{array}
\]

The more interesting rules are as follows:
\[
\begin{array}{cc}
\mbox{\sc ActiveI}
&
\begin{array}{c}
- \\ \hline
\Gamma \vdash \bigvee_{\mathit{thread}\ A} A\ \Active
\end{array}
\end{array}
\]\vspace{-1em}
\[
\begin{array}{ccc}
\mbox{\sc ActiveE}
&
\begin{array}{c}
\Gamma \vdash A\ \Active \hspace*{1cm} \Gamma \vdash B\ \Active \\ \hline
\Gamma \vdash \phi
\end{array}
&
(A\neq B \text{ and not mapped to same } \tid )
\end{array}
\]
\[
\begin{array}{cc}
\begin{array}{cc}
\mbox{\sc Prev}
&
\begin{array}{c}
\Gamma \vdash \phi \\ \hline
\Prev\ \Gamma \vdash \Prev\ \phi
\end{array}
\end{array}
&
\begin{array}{cc}
\mbox{\sc K}
&
\begin{array}{c}
\Gamma \vdash \phi \\ \hline
A\ \Knows\ \Gamma \vdash A\ \Knows\ \phi
\end{array}
\end{array}
\end{array}
\]\vspace{-0.5em}
\[
\begin{array}{cc}
\begin{array}{cc}
\mbox{\sc T}
&
\begin{array}{c}
\Gamma \vdash A\ \Knows\ \phi \\ \hline
\Gamma \vdash \phi 
\end{array}
\end{array}
&
\begin{array}{cc}
\mbox{\sc 4}
&
\begin{array}{c}
A\ \Knows\ \Gamma \vdash \phi \\ \hline
A\ \Knows\ \Gamma \vdash A\ \Knows\ \phi
\end{array}
\end{array}
\end{array}
\]
\[
\begin{array}{cc}
\begin{array}{cc}
\mbox{\sc SI1}
&
\begin{array}{c}
\Gamma \vdash \psi  \\ \hline
\Gamma \vdash \phi\ \Since\ \psi
\end{array}
\end{array}
&
\begin{array}{cc}
\mbox{\sc SI2}
&
\begin{array}{c}
\Gamma \vdash \phi \hspace*{1cm}
\Gamma \vdash \Prev\ (\phi\ \Since\ \psi)  \\ \hline
\Gamma \vdash \phi\ \Since\ \psi
\end{array}
\end{array}
\end{array}
\]\vspace{-0.5em}
%\[
%\begin{array}{cc}
%\mbox{\sc ???}
%&
%\begin{array}{c}
%\Gamma_1 \vdash \Delta_1 \hspace*{1cm} \Gamma_2 \vdash \Delta_2  \\ \hline
%\Gamma_1\ \Since\ \Gamma_2 \vdash \Delta_1\ \Since\ \Delta_2
%\end{array}
%\end{array}
%\]
\[
\begin{array}{cc}
\mbox{\sc SE}
&
\begin{array}{c}
\Gamma \vdash \phi_1\ \Since\ \phi_2 \hspace*{1cm}
\Gamma, \phi_2 \vdash \psi  \hspace*{1cm}
\Gamma, \Prev\ (\phi_1\ \Since\ \phi_2) \vdash \psi \\ \hline
\Gamma \vdash \psi
\end{array}
\end{array}
\]
%\[
%\begin{array}{cc}
%\mbox{\sc PK1}
%&
%\begin{array}{c}
%\Gamma \vdash A\ \Knows\ \Prev\ (A\ \Active \Implies \phi) \\ \hline
%\Gamma \vdash \Prev\ A\ \Knows\ (A\ \Active \Implies \phi) 
%\end{array}
%\end{array}
%\]
%\[
%\begin{array}{cc}
%\mbox{\sc PK2}
%&
%\begin{array}{c}
%\Gamma \vdash A\ \Knows\ \Prev\ (\neg A\ \Active \Implies \phi) \\ \hline
%\Gamma \vdash \Prev\ A\ \Knows\ (\neg A\ \Active \Implies \phi) 
%\end{array}
%\end{array}
%\]
\[
\begin{array}{cc}
\mbox{\sc KP1}
&
\begin{array}{c}
\Gamma \vdash \Prev\ A\ \Active \hspace*{1cm} \Gamma \vdash \Prev\ A\ \Knows\ \phi \\ \hline
\Gamma \vdash A\ \Knows\ \Prev\ (A\ \Active\ \Implies \phi)
\end{array}
\end{array}
\]\vspace{-0.5em}
\[
\begin{array}{cc}
\mbox{\sc KP2}
&
\begin{array}{c}
\Gamma \vdash \Prev\ \neg A\ \Active \hspace*{1cm} \Gamma \vdash \Prev\ A\ \Knows\ \phi \\ \hline
\Gamma \vdash A\ \Knows\ \Prev\ (\neg A\ \Active\ \Implies \phi)
\end{array}
\end{array}
\]\vspace{-0.5em}
%
\[
\begin{array}{cc}
\mbox{\sc KPA}
&
\begin{array}{c}
\Gamma \vdash \Prev\ A\ (A\ \Knows\ \phi) \\ \hline
\Gamma \vdash A\ \Knows\ \Prev\ A\ \phi
\end{array}
\end{array}
\]\vspace{-0.5em}
\[
\begin{array}{cc}
\mbox{\sc KSRA}
&
\begin{array}{c}
\Gamma \vdash \Prev\ A\ A\ \Knows\ (\phi\ \Since\ \psi) \hspace*{1cm} \Gamma \vdash A\ \Knows\ \phi \\ \hline
\Gamma \vdash A\ \Knows\ (\phi\ \Since\ \psi)
\end{array}
\end{array}
\]
%

Our proof system does not capture complete S5 properties of $K$, for practical purposes. We only focus on those, particularly $T$ and $K$, which are important for sound reasoning about knowledge.

\begin{theorem}[Soundness\footnote{We prove soundness under weak fairness assumption.}]
If $\Gamma \vdash \Delta$ is derivable then $\Gamma \models \Delta$.
\end{theorem}
\begin{proof} 

\noindent\Case{\underline{\bf Prev}}: If $\Gamma \models \phi$ and $\pi,i\models_\rho \Prev\ \psi$ for all $\psi\in\Gamma$ then for all $\psi\in\Gamma$, $\pi,i-1\models_\rho \psi$. We can conclude that $\pi,i-1\models_\rho \phi$ so $\pi,i \models_\rho \Prev\ \phi$.\\

\noindent\Case{\underline{\bf 4}}: (Standard) Suppose $A\ \Knows\ \Gamma \models \phi$ and $\pi,i \models A\ \Knows\ \Gamma$. Then for all $\pi',i'$ satisfying the appropriate conditions,
$\pi',i'\models_\rho A\ \Knows\ \Gamma$ as well, so $\pi',i'\models_\rho \phi$, hence $\pi,i\models_\rho A\ \Knows\ \phi$.\\

\noindent\Case{\underline{\bf SI1, SI2, SE}}: Standard.\\

\noindent\Case{\underline{\bf T}}: Standard. If $\pi,i\models_\rho A\ \Knows\ \phi$ then $\pi,i\models_\rho \phi$.\\

\noindent\Case{\underline{\bf KP1}}: Assume $\Gamma \models \Prev\ A\ \Active$, $\Gamma\models \Prev\ A\ \Knows\ \phi$ and that $\pi,i\models_\rho \bigwedge\Gamma$. 
Then for  $i>0$, pick $\pi',i'$ such that
$\History(\pi,\rho(A)) = \langle h,f\rangle$, $\History(\pi',\rho(A)) = \langle h',f'\rangle$ and $h\Restrict [0,f(i)] = h'\Restrict[0,f'(i')]$. Also assume 
$\pi',i'-1\models_\rho A\ \Active$. We know that $\pi,i\models_\rho A\ \Active$ as well. Then $h\Restrict[0,f(i-1)] = h'\Restrict[0,f'(i'-1)]$, so $\pi',i'-1\models_\rho \phi$,
concluding the case.\\

\noindent\Case{\underline{\bf KP2}}: Similar.\\

\noindent\Case{\underline{\bf KPA}}: Assume $\pi,i  \models_\rho \Prev\ A\ A\ \Knows\ \phi$. Let $\pi',i'$ be given such that $\History(\pi,\rho(A)) = \langle h,f\rangle$, 
$\History(\pi',\rho(A)) = \langle h',f'\rangle$ and $h\Restrict [0,f(i)] = h'\Restrict[0,f'(i')]$. By proposition \ref{160329.1}, $\pi,j\models_\rho A\ \Knows\ \phi$, where
$j<i$. Since $h\Restrict [0,f(i)] = h'\Restrict[0,f'(i')]$ we find $j'<i'$ such that $\pi',j' \models_\rho \phi$. Also, for all $k': j' < k' < i'$m, $\rho(A)$ is not active. Hence
$\pi',i'\models_\rho \Prev\ A\ \phi$, and hence $\pi,i \models_\rho A\ \Knows\ \Prev\ A\ \phi$.\\

\noindent\Case{\underline{\bf KSRA}}: Assume $\Gamma \models \Prev\ A\ A\ \Knows\ (\phi\ \Since\ \psi)$,
$ \Gamma\models  A\ \Knows\ \phi$,  and $\pi,i\models_\rho \bigwedge\Gamma$. Then 
$\pi,i \models_\rho  \Prev\ A\ A\ \Knows\ (\phi\ \Since\ \psi)$
and $\pi,i \models_\rho  A\ \Knows\ \phi$. Let $\pi',i'$ be given such that
$\History(\pi,\rho(A)) = \langle h,f\rangle$, $\History(\pi',\rho(A)) = \langle h',f'\rangle$, and $h\Restrict [0,f(i)] =
h'\Restrict[0,f'(i')]$. We can immediately conclude that $\pi',i'\models_\rho \phi$, and hence $\pi',i''\models_\rho \phi$
whenever $f'(i') = f'(i'')$. We can also conclude that $\pi,j \models_\rho A\ \Knows\ (\phi\ \Since\ \psi)$ where $j$ is the previous
thread state relative to $i$ for $\rho(A)$. Then $\pi',j'\models_\rho(A)  \phi\ \Since\ \psi$ for all $j'$ such that $f'(j')+1 = f(i')$.
But then it follows that $\pi',i'\models_\rho \phi\ \Since\ \psi$ as was to be proved.
\end{proof}
%
\subsection{Derived Rules}
We then derive additional rules for \textit{temporal operators} that extend our basic proof system. In particular, we use the basic axioms and inference rules from the previous section to derive rules for the ``always'' and ``sometime'' operators, as well as ``since'' and ``weak since''. 

\[
\begin{array}{ccc}
{\sc alwaysI}
&
\begin{array}{c}
\Gamma\vdash \phi \hspace*{0.6cm} \Gamma \vdash \Prev\ \Always\ \phi \\ \hline
\Gamma \vdash \Always\ \phi
\end{array}
&
\begin{array}{cc}
{\sc alwaysE1}
&
\begin{array}{c}
\Gamma\vdash \Always\ \phi \\ \hline
\Gamma \vdash \phi
\end{array}
\end{array}
\end{array}
\]
\[
\begin{array}{ccc}
{\sc alwaysE2}
&
\begin{array}{c}
\Gamma\vdash \Always\ \phi \\ \hline
\Gamma \vdash \Prev\ \Always\ \phi
\end{array}
&
\begin{array}{cc}
{\sc sometimeI1}
&
\begin{array}{c}
\Gamma\vdash \phi  \\ \hline
\Gamma \vdash \Sometime\ \phi
\end{array}
\end{array}
\end{array}
\]
\[
\begin{array}{cc}
{\sc sometimeI2}
&
\begin{array}{c}
\Gamma\vdash \Prev\ \Sometime\ \phi \\ \hline
\Gamma \vdash \Sometime\ \phi
\end{array}
\end{array}
\]
\[
\begin{array}{cc}
{\sc sometimeE}
&
\begin{array}{c}
\Gamma\vdash \Sometime\ \phi \hspace*{1cm} \Gamma,\phi \vdash \psi \hspace*{1cm} \Gamma,\Prev\ \Sometime\ \phi \vdash \psi \\ \hline
\Gamma \vdash \psi
\end{array}
\end{array}
\]
\[
\begin{array}{ccc}
{\sc weaksinceI1}
&
\begin{array}{c}
\Gamma\vdash \psi \hspace*{0.4cm} \Gamma\vdash \psi \Implies \phi \\ \hline
\Gamma \vdash \phi\ \Since_w\ \psi
\end{array}
&
\begin{array}{cc}
{\sc weaksinceI2}
&
\begin{array}{c}
\Gamma\vdash \phi \hspace*{0.4cm} \Gamma \vdash \Prev\ (\phi\ \Since_w\ \psi) \\ \hline
\Gamma \vdash \phi\ \Since_w\ \psi
\end{array}
\end{array}
\end{array}
\]
\[
\begin{array}{cc}
{\sc weaksinceE}
&
\begin{array}{c}
\Gamma\vdash \phi_1\ \Since_w\ \phi_2 \hspace*{0,5cm} \Gamma,\phi_2 \vdash \psi \hspace*{0,5cm} \Gamma,\Prev\ (\phi_1\ \Since_w\ \phi_2) \vdash \psi \\ \hline
\Gamma \vdash \psi
\end{array}
\end{array}
\]
\[
\begin{array}{cc}
{\sc hbI1}
&
\begin{array}{c}
\Gamma \vdash \phi_2 \hspace*{1cm}
\Gamma \vdash \neg\phi_1 \hspace*{1cm}
\Gamma \vdash \Sometime\ \phi_1 \\ \hline
\gamma \vdash \phi_1\ \HappensBefore\ \phi_2
\end{array}
\end{array}
\]
\[
\begin{array}{cc}
{\sc hbI2}
&
\begin{array}{c}
\Gamma \vdash \neg\phi_1 \hspace*{1cm}
\Gamma \vdash \Prev\ (\phi_1\ \HappensBefore\ \phi_2) \\ \hline
\gamma \vdash \phi_1\ \HappensBefore\ \phi_2
\end{array}
\end{array}
\]
\[
\begin{array}{cc}
{\sc hbE}
&
\begin{array}{c}
\Gamma \vdash \phi_1\ \HappensBefore\ \phi_2 \hspace*{0,3cm}
\Gamma,\phi_2,\neg\phi_1,\Sometime\ \phi_1 \vdash \psi \hspace*{0,3cm}
\Gamma,\neg\phi_1,\Prev\ (\phi_1\ \HappensBefore\ \phi_2) \vdash \psi
\\ \hline
\Gamma \vdash \psi
\end{array}
\end{array}
\]
%
