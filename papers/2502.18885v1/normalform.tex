\section{Normal Forms}
We introduce a \emph{normal form} representation for formulas in our logic. This normal form captures the essential properties of the original formula in a more structured representation and enables a more efficient verification.
Technically, the goal is to show that for each formula $\phi\in\LLanOne$, it is possible to find a formula $\NF(\phi)$ in a suitable normal form to be defined, such that $\NF(\phi)\models \phi$.
The reverse entailment $\phi\models \NF(\phi)$ does not and is not intended to hold by design---the normal form is intentionally less restrictive to ensure sufficiency for verification, not equivalence. This asymmetry arises from the inherent non-reversibility of epistemic modalities under rules like {\sc KPA} (knowledge propagation after activity) and {\sc KSRA} (knowledge stability across runs), particularly when combined with past-time operators like ``since''. Specifically, temporal-epistemic properties cannot always be decomposed into component formulas due to the monotonic growth of knowledge over time in concurrent executions.

While constructing some normal form is straightforward---for instance, 
$\top$ (truth) trivially entails any formula---such simplistic forms lack practical utility. The criterion are that the candidate normal forms should satisfy their formal requirements (i.e., they entail the original formula) and that they are actually useful in that they can be used to efficiently verify practical code in an understandable and efficient manner, like any other instrument for static analysis.

For now we restrict attention to what we call \emph{positive $A$ formulas}, i.e., formulas $\phi$ for which:

\begin{enumerate}
\item All epistemic connectives occur within an even number of negations and
\item The only thread variable used is $A$.
\end{enumerate}

\begin{definition}[Positive $A$ State Formula, Normal Form]
\mbox{}
\begin{enumerate}
\item
A positive $A$ state formula is a positive $A$ formula built using only formulas of the form $A\ \Says\ e_1 = e_2$, $A\ \Says\ e_1\neq e_2$, boolean
connectives, and the epistemic modality. 
\item A formula in normal form has the following shape 
\todo{activity and read write statements to talk about last transition, not next}
\begin{eqnarray}
 & \bigwedge\{\Prev\ A\ A\ \At\ \Label' \wedge A\ \At\ \Label \wedge \mbox{} \nonumber \\
 & (A\ \Active\ \Implies \bigvee_{j\in J_1}(\psi_{1,j,\Label',\Label} \wedge \Prev\ \psi_{2,j,\Label',\Label})) \wedge \mbox{} \nonumber \\ 
 & (\neg A\ \Active\ \Implies \bigvee_{j\in J_2}(\psi_{3,j,\Label',\Label} \wedge \Prev\ \psi_{4,j,\Label',\Label})) \mid \Label,\Label'\in\Labels\} \label{160418.0}
\end{eqnarray}

where the formulas $\psi_{i,j,\Label',\Label}$ satisfy the following requirements:
\begin{itemize}
\item $\psi_{1,j,\Label',\Label}$, $\psi_{3,j,\Label',\Label}$ are positive $A$ state formulas.
\item $\psi_{2,j,\Label',\Label}$, $\psi_{4,j,\Label',\Label}$ are positive $A$ formulas.
\end{itemize}
\end{enumerate}
%
\end{definition}
%
\todo{Need to add case for termination, see proof below}
%
In the sequel we refer to the formulas $\psi_{i,j,\Label',\Label}$ as the \emph{component formulas}, and to the index sets
$J_1$, $J_2$ as the \emph{disjunctive index sets}. 
%
%We prove the following theorem using the semantics. % later should do it using the proof system.
%
\begin{theorem}
For each positive $A$ formula $\phi$ exists a formula $\NF(\phi)$ in normal form such that $\NF(\phi) \models \phi$.
\end{theorem} 
\begin{proof}
First, we can rewrite each formula $\phi$ to a positive form with all subformulas of the forms
$A\ \Says\ e_1 = e_2$, $A\ \Active$, $\phi_1\wedge \phi_2$, $\Prev\ \phi$, $\phi_1\ \Since\ \phi_2$,
$A\ \Knows\ \phi$, $A\ \At\ \Label$, $A:e_1\ \ReadFrom e_2$, $A:e_1\ \WriteTo\ e_2$, or a negation thereof, with the exception of
the epistemic modality that only occurs positively.

Second, we note that it is trivial to rewrite each $\phi$ to the following form:
\begin{equation}
\label{160415.1}
\hat{\phi} == \bigwedge\{\Prev\ A\ A\ \At\ \Label' \wedge A\ \At\ \Label \Implies \phi_{\Label',\Label} \wedge \Prev\ \True \mid \Label,\Label'\in\Labels\}\ ,
\end{equation}
simply by choosing $\phi_{\Label',\Label} = \phi$ and noting that $\True \models \Prev\ \True$ and $\hat{\phi} \models \phi$.

The main proof is done by By induction on $\phi$:
\begin{itemize}
    \item Base case. For atomic propositions (e.g., $A\ \At\ \Label$), the normal form directly encodes the satisfaction condition.
    \item Inductive case.  For composite formulas (e.g., $A\ \Knows\ \phi$), the normal form's structure preserves the semantics of the original formula.
\end{itemize}
We now proceed by considering each subformula type in turn:\\

\noindent\Case{\underline{$\phi \equiv A\ \At\ \Label$}}:
Define \(\phi_{\Label',\Label''} = \top\) if \(\Label'' = \Label\), and \(\bot\) otherwise. By the satisfaction condition for \(A\ \At\ \Label\), \(\pi, i \models_\rho A\ \At\ \Label\) iff \(\Label : cmd(\pi(i), \rho(A))\).
The normal form ensures \(A\ \At\ \Label\) holds at the current state, and \(\textbf{prev } A \ (A\ \At\ \Label')\) captures the previous control state.
% Define
% \[
% \phi_{\Label',\Label''} == \left\{ \begin{array}{ll} \True, & \mbox{if } \Label'' = \Label \\ \False & \mbox{otherwise} \end{array}\right.
% \]

\noindent\Case{ \underline{$\phi \equiv \neg(A\ \At\ \Label)$}}: Dual of the previous case.

\noindent\Case{\underline{$\phi \equiv A\ \Says\ e_1 = e_2$, $\neg(A\ \Says\ e_1 = e_2$})}: Use (\ref{160415.1}), and a little boolean reasoning.

\noindent\Case{\underline{$\phi \equiv \phi_1\wedge \phi_2$}}: By the induction hypothesis each $\phi_i$ can be rewritten to normal form with component formulas
$\psi_{k,i,j,\Label',\Label}$, $k\in\{1,2\}$, $i\in\{1,2,3,4\}$, and $j$ in the disjunctive index sets $J_{k,1}$, $J_{k,2}$. 
Then, $\NF(\phi)$ can be constructed with disjunctive index sets $J_1 = J_{1,1}\times J_{2,1}$ and $J_2 = J_{1,2}\times J_{2,2}$ and with
component formulas 
$\psi_{i,(j_1,j_2),\Label',\Label}==\psi_{i,j_1,\Label',\Label} \wedge \psi_{i,j_2,\Label',\Label}$ we obtain $\NF(\phi)\models \phi$ as desired.

\noindent\Case{\underline{$\phi \equiv \neg(\phi_1\wedge \phi_2)$}}: In this case we can choose
$J_1 = J_1 \cup J_2$ (assuming the index sets are disjoint) and the $\psi_{i,(j_1,j_2),\Label',\Label}$ accordingly.

\noindent\Case{\underline{$\phi \equiv \Prev\ \phi'$}}: Trivial.

\noindent\Case{\underline{$\phi \equiv \neg(\Prev\ \phi')$}}: Use the equivalence $\neg(\Prev\ \phi') \equiv (\Prev\ \neg\phi') \vee \neg(\Prev\ \True)$.

\noindent\Case{\underline{$\phi \equiv \phi_1\ \Since\ \phi_2$}}: Use the equivalence 
\[
\phi_1\ \Since\ \phi_2 \equiv \phi_2 \vee (\phi_2 \wedge \Prev\ (\phi_1\ \Since\ \phi_2))\ .
\]

\noindent\Case{\underline{$\phi \equiv A\ \Knows\ \phi$}}: Assume that  $\phi$ has the form (\ref{160418.0}). First, use the equivalence
\begin{equation*}
A\ \Knows\ (\phi\wedge \psi) \equiv A\ \Knows\ \phi\wedge A\ \Knows\ \psi
\end{equation*}
and
\begin{equation*}
A\ \Knows\ A\ \At\ \Label \equiv A\ \At\ \Label
\end{equation*}
to rewrite $\phi$ to the form
%
\begin{eqnarray*}
 & \bigwedge\{\Prev\ A\ A\ \At\ \Label' \wedge A\ \At\ \Label \wedge \mbox{} \nonumber \\
 & A\ \Knows\ (A\ \Active\ \Implies \bigvee_{j\in J_1}(\psi_{1,j,\Label',\Label} \wedge \Prev\ \psi_{2,j,\Label',\Label})) \wedge \mbox{} \nonumber \\ 
 & A\ \Knows\ (\neg A\ \Active\ \Implies \bigvee_{j\in J_2}(\psi_{3,j,\Label',\Label} \wedge \Prev\ \psi_{4,j,\Label',\Label})) \mid \Label,\Label'\in\Labels\} \label{160418.1}
\end{eqnarray*}
%
We now use the entailments
\begin{equation*}
\label{160419.1}
A\ \Active\ \Implies\ A\ \Knows\ \phi \models A\ \Knows\ (A\ \Active\ \Implies\ \phi)
\end{equation*}
(cf. the rules {\sc KP1} and {\sc KP2}),
and, even more crudely:
\begin{equation*}
A\ \Knows\ \phi_1 \vee A\ \Knows\ \phi_2 \models A\ \Knows\ (\phi_1\vee \phi_2) 
\end{equation*}
Now rewrite to:
%
\begin{eqnarray*}
 & \bigwedge\{\Prev\ A\ A\ \At\ \Label' \wedge A\ \At\ \Label \wedge \mbox{} \nonumber \\
 & (A\ \Active\ \Implies \bigvee_{j\in J_1}(A\ \Knows\ \psi_{1,j,\Label',\Label} \wedge A\ \Knows\ \Prev\ \psi_{2,j,\Label',\Label})) \wedge \mbox{} \nonumber \\ 
 & (\neg A\ \Active\ \Implies \bigvee_{j\in J_2}(A\ \Knows\ \psi_{3,j,\Label',\Label} \wedge A\ \Knows\ \Prev\ \psi_{4,j,\Label',\Label})) \mid \Label,\Label'\in\Labels\} \label{160418.1}
\end{eqnarray*}
%
and then finally to
\begin{eqnarray*}
 & \bigwedge\{\Prev\ A\ A\ \At\ \Label' \wedge A\ \At\ \Label \wedge \mbox{} \nonumber \\
 & (A\ \Active\ \Implies \bigvee_{j\in J_1}(A\ \Knows\ \psi_{1,j,\Label',\Label} \wedge \Prev\ A\ \Knows\ \psi_{2,j,\Label',\Label})) \wedge \mbox{} \nonumber \\ 
 & (\neg A\ \Active\ \Implies \bigvee_{j\in J_2}(A\ \Knows\ \psi_{3,j,\Label',\Label} \wedge \Prev\ A\ \Knows\ \psi_{4,j,\Label',\Label})) \mid \Label,\Label'\in\Labels\} \label{160418.1}
\end{eqnarray*}
%
which is valid by suitably pushing around the activity statements.

\noindent\Case{\underline{$A:e_1\ \ReadFrom\ e_2$}}: Use the equivalence
%
\begin{equation}
\label{160419.2}
\Prev\ A\ A\ \At\ \Label' \wedge A\ \Active\ \wedge A\ \At\ \Label \wedge A\ \Says\ \Rn = e_1 \wedge A\ \Says\ \Rm = e_2 \equiv  A:e_1\ \ReadFrom\ e_2
\end{equation}
provided that $\PrevState(\Label) = \langle \Label',\True\rangle$ and $\Label':\Rn\ \ReadFrom\ !\Rm;c$.\todo{Is an update to the registers not needed here?}

\noindent\Case{\underline{$A:\neg(e_1\ \ReadFrom\ e_2)$}}: Use (\ref{160419.2}) again.

\noindent\Case{\underline{$A: e_1\ \WriteTo\ e_2$, $A:\neg(e_1\ \WriteTo\ e_2$)}}: Similar to the read case. 

\end{proof}
