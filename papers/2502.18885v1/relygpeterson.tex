\subsection{Applying the RG Method to the Peterson Algorithm}
We now try to apply the RG method to the Peterson algorithm and show how we can verify threads in isolation. As we discussed before the RG method involves specifying conditions under which each thread operates, known as rely and guarantee conditions. These conditions describe the assumptions a thread makes about its environment and the commitments it guarantees to the system. For the Peterson algorithm, these conditions are defined for two threads, $A$ and $B$, which use shared variables $\FlagZero$, $\FlagOne$, and $\Victim$ to manage access to the critical section.

To prove that the Peterson algorithm guarantees mutual exclusion, we define the following rely and guarantee conditions:
\begin{itemize}
    \item  Rely condition for $A$ (\(\textcolor{black}{ R_{1}}\)):
\[\Always\ \Knows_{\tid_1} (!\FlagZero = 1 \wedge !\Victim = 1) \Implies \neg(\tid_2\  \EntersCS)\]

    \item Guarantee condition for $A$ (\(\textcolor{black}{G_{1}}\)):
  %- \( G_A = (\text{\FlagZero} = 1 \Implies A \text{ is attempting }) \)
\[\Always\ \Knows_{\tid_1} (!\FlagZero = 0 \Implies \neg (\tid_1 \EntersCS) )\]
\end{itemize}

Moreover, since $\guarantee_1 \Implies \rely_2$, we know that $B$'s actions (under $\rely_2$) cannot write to $\FlagZero$, therefore, $A$'s knowledge of $\FlagZero = 0$ remains stable.
Similarly, for thread $B$ these conditions can be defined.
% \begin{itemize}
%     \item Rely condition for $B$ (\(\textcolor{blue}{R_{2}}\)):
% \[\Always\ \Knows_{\tid_2} (!\FlagOne = 1 \wedge !\Victim = 0) \Implies \neg(\tid_1\  \EntersCS)\]
%
%     \item Guarantee condition for $B$ (\(\textcolor{blue}{G_{2}}\)):
%   %- \( G_B = (\text{\FlagOne} = 1 \Implies B \text{ is attempting }) \)
% \[\Always\ \Knows_{\tid_2} ( !\FlagOne = 0 \Implies \neg (\tid_2  \EntersCS) )\]
% \end{itemize}
%
Then, each thread proves its guarantee using its rely condition. For example $A$ proves that $\vdash \alpha_1\ \{\tid_1:c_1\}\  \beta_1$ where the proof uses $\rely_1$ to assume $\tid_2$'s behavior and $\guarantee_1$ to ensure $\tid_1$'s actions $\beta_1 = \FlagZero = 0 \Implies \neg(\tid_1\ \EntersCS)$ preserve. Similarly, for the $B$, with $\beta_2 = \FlagOne = 0 \Implies \neg(\tid_2\ \EntersCS)$.

\begin{lemma}\label{stabilityLemma} If $\tid_i$ knows $\phi$ , and $\phi$ is stable under interference (ensured by $\guarantee_j \rightarrow \rely_i$, then $\phi$ holds globally.
\end{lemma}
\begin{proof}
\begin{itemize}
    \item $\tid_i$  knows $\phi$ means $\phi$ holds in all states consistent with $\tid_i$s observations.
    \item $\guarantee_j \rightarrow \rely_i$ ensures $\tid_j$'s actions preserve $\phi$.
    \item Since $\phi$ is stable and $\tid_i$ knows $\phi$, $\phi$ holds in all reachable global states.
\end{itemize}
\end{proof}

For thread $A$ this means:
\begin{itemize}
    \item $A$ knows $\beta_1$ locally, i.e., $\Knows_{\tid_1}\ beta_1$
    \item $\guarantee_2 \rightarrow \rely_1$ ensure that $B$'s actions do not invalidate $\beta_1$.
    \item Thus, $\beta_1$ holds globally.
\end{itemize}
Similar reasoning also holds for $\beta_2$.
%
Now we can apply the parallel composition rule, and by Lemma~\ref{stabilityLemma} we know the $\beta_1$ and $\beta_2$ hold globally, which in turn imply that $\neg (\tid_1\  \EntersCS\  \wedge\ \tid_2\  \EntersCS)$.

Finally, we prove mutual exclusion. Assume $\tid_1\  \EntersCS$ holds. We show $\neg(\tid_2\  \EntersCS)$:
\begin{enumerate}
    \item We first expand $\tid_1\  \EntersCS$:
     \[
     (!\FlagOne \neq 1 \vee !\Victim \neq 0) \wedge (\tid_1 \RecentlyWrote 1 to \FlagZero) \wedge ...
     \]
     \item From $\guarantee_1 \rightarrow \rely_2$, we know that $A$'s guarantee ensures that $!\FlagZero = 1$  persists until it exits, so $B$'s rely $rely_2$ (which assumes $!\FlagZero = 1 \Implies \neg \tid_2\  \EntersCS$) is triggered.
     \item From $B$'s perspective, $B$ observes $!\FlagZero = 1$ and $!\Victim = 1$, so by $\rely_2$, it cannot enter critical section.
\end{enumerate}

