\section{Related Work}
The verification of concurrent programs has been extensively studied, with approaches ranging from separation logic~\cite{DBLP:conf/concur/OHearn04} to rely-guarantee~\cite{DBLP:phd/ethos/Jones81} reasoning.
The PhD theses of Baumann and Bernhard~\cite{baumann2014ownership,kragl2020verifying} give
an overview of various techniques used in the verification of concurrent programs, including techniques beyond the scope of epistemic logic.  
Here, we only focus on those studies that have used epistemic logic for concurrent program verification.

\paragraph{Temporal Logic and Compositionality}
Temporal logic, introduced by Pnueli~\cite{DBLP:conf/focs/Pnueli77,DBLP:books/daglib/0080029} for reactive systems, has been widely adopted in concurrency. However, traditional linear-time (LTL) and branching-time (CTL)~\cite{DBLP:conf/lop/ClarkeE81} logics lack epistemic modalities, precluding assertions about what threads know based on their local states. 

\paragraph{Epistemic Logic in Concurrency Verification}
Epistemic logic~\cite{hintikka1962knowledge} originated as a framework for modeling knowledge in multi-agent systems, with applications in game theory, economics, distributed systems, and artificial intelligence. Halpern and Moses pioneered its use in distributed computing with their seminal work on common knowledge ~\cite{DBLP:conf/podc/HalpernM84,DBLP:journals/jacm/HalpernM90}, which formalizes conditions under which agreement protocols can achieve consensus. Their ideas inspired extensions to probabilistic settings~\cite{DBLP:conf/podc/HalpernT89}, zero-knowledge protocols~\cite{DBLP:conf/stoc/HalpernMT88}, and broader applications in multi-agent coordination~\cite{DBLP:conf/podc/NeigerT87,DBLP:conf/wdag/NeigerT90,DBLP:conf/podc/PanangadenT88}. Fagin et al.~\cite{DBLP:books/mit/FHMV1995} later systematized these concepts in a textbook, making epistemic logic a cornerstone of knowledge representation.

In the concurrency verification domain, there are only a few works that explored the usage of the epistemic logic for reasoning about the correctness of multi-threaded programs, e.g.,~\cite{DBLP:phd/hal/Knight13,DBLP:conf/forte/ChadhaDK09,DBLP:conf/lpar/DechesneMO07}. Notable contributions include the work by Chadha, Delaune, and Kremer~\cite{DBLP:conf/forte/ChadhaDK09} that proposed an epistemic logic for a variant of the $\pi$-calculus that is particularly tailored for modeling cryptographic protocols. Their work focuses on reasoning about epistemic knowledge, especially in the context of security properties such as secrecy and anonymity. Dechesne et al.~\cite{DBLP:conf/lpar/DechesneMO07} explored the relation of operational semantics and epistemic logic using labeled transition systems. Similarly, Knight~\cite{knight2013epistemic} studied the use of epistemic modalities as programming constructs within a process calculus, developed a dynamic epistemic logic for analyzing knowledge evolution in labeled transition systems, and introduced a game semantics for concurrent processes that allows for modeling agents with varying epistemic capabilities.

Also, Van der Hoek et al.~\cite{DBLP:conf/rex/HoekHM92} contributed to this discourse. Their work extends Halpern et al.~\cite{DBLP:conf/ijcai/HalpernM85,DBLP:journals/jacm/HalpernM90} work on distributed systems to facilitate the verification of concurrent computations using partially ordered sets of action labels. They employed a variant of Hoare's~\cite{DBLP:journals/cacm/Hoare78} communicating sequential processes (CSP) as a case study to show the application of their theoretical framework.