\section{Inference rules for FDs and MVDs}
\label{sec:monoidal}

\subsection{Inference Rules for FDs and Monoidal Category}


%Recall that in Section   \ref{sec:ClosureFD},  there are three inference rules of functional dependencies in Amstrong's axioms.    All three Amstrong's axioms can be applied to the category. In particular,  the first rule defines a projection morphism, and the second is about a composed morphism. Here we show that the second rule indeed defines a monoidal product.  



In Section \ref{sec:ClosureFD}, we introduced three inference rules governing functional dependencies within the framework of Armstrong's axioms. More specifically, the first rule establishes the concept of a projection morphism, while the third rule pertains to a composed morphism. In this appendix, we aim to demonstrate that the second rule in the following, in fact, delineates the properties of a monoidal product.


FD 2: If $f: X \to Y$, then $g: XZ \to YZ$;  Specifically, given an element $(x,z) \in XZ$, let $y = f(x)$, we define that $(y,z) \in YZ$ and $(y,z)$ is the image of $(x,z)$ under the function $g$. 


\begin{definition}(Monoidal Category) A monoidal category $\mathcal{C}$ consists of the following components:

\begin{itemize}
    \item A category $\mathcal{C}$ with objects denoted as $X, Y, Z, \ldots$ and morphisms between objects.
    \item A bifunctor $\otimes: \mathcal{C} \times \mathcal{C} \to \mathcal{C}$, called the monoidal product, which associates to each pair of objects $X$ and $Y$ an object $X \otimes Y$ in $\mathcal{C}$.
    \item An associator, which is a natural isomorphism:
    $ \alpha_{X, Y, Z}: (X \otimes Y) \otimes Z \xrightarrow{\sim} X \otimes (Y \otimes Z)$
    satisfying certain coherence conditions.
    \item A unit object $I$ and natural isomorphisms, called the left and right unitors: $ I \otimes X \xrightarrow{\sim} X$ and $  X \otimes I \xrightarrow{\sim} X$
    also satisfying coherence conditions.
\end{itemize}
\end{definition}

%Consider one category $\mathcal{C}_1$ with two objects $X$, $Y$ and a morphism $f: X \to Y$, and another category $\mathcal{C}_2$ with an object $Z$ and an identity morphism, then we define a monoidal category with $C_1$ and $C_2$, such that the bifunctor $\otimes: \mathcal{C}_1 \times \mathcal{C}_2 \to \mathcal{C}_3$, associate $X$, $Z$ into $XZ$, and $Y$, $Z$ into $YZ$. The morphism between $XZ$ and $YZ$ is defined as $k: \forall (x,z)\in XZ \to (y,z) \in YZ$, where $f(x)=y$.  A unit object $I$ in this monoidal category is an empty object $\epsilon$ with an identity morphism, such that $\forall x \in X$, $(x,\epsilon)=x$.  




Let us consider two distinct categories: $\mathcal{C}_1$, which consists of two objects, denoted as $X$ and $Y$, connected by a morphism $f: X \to Y$, and $\mathcal{C}_2$, featuring a single object $Z$ with an identity morphism. We aim to construct a monoidal category by combining $\mathcal{C}_1$ and $\mathcal{C}_2$, yielding $\mathcal{C}_3$. To achieve this, we introduce a bifunctor denoted as $\otimes: \mathcal{C}_1 \times \mathcal{C}_2 \to \mathcal{C}_3$. This bifunctor associates the pair $(X,Z)$ with the object $XZ$, and the pair $(Y,Z)$ with the object $YZ$. The morphism between objects $XZ$ and $YZ$ is defined as $k: \forall (x,z)\in XZ \to (y,z) \in YZ$, where $f(x)=y$.  Further,  a unit object $I$ is represented as an empty object denoted by $\epsilon$, accompanied by an identity morphism. This unit object $\epsilon$ satisfies the condition that for all elements $x$ within the object $X$, both pairs $(x,\epsilon)$ and $(\epsilon,x)$ correspond to $x$.


%\section{Connected components and pushout}
%\label{sec:pushout}

%The following example illustrates the connection between the pushout and the connected component of a graph.

%\begin{example} Given an undirected graph $G$, the \texttt{Edge} table includes two attributes \texttt{Node\_id1} and \texttt{Node\_id2}, which describes edges between any two nodes in $G$.   The pushout object \texttt{Component} computes the connected component in the graph $G$, as illustrated in the following commutative diagram.


%\[\xymatrix{Edge \ar[r]^{f} \ar[d]_{g} & Node\_id1 \ar[d]^{p_1} \\Node\_id2 \ar[r]_{p_2} & Component }\]\end{example}


%Consider that the pullback object corresponds to the join operator in relational databases, while the pushout object corresponds to the connected component in an undirected graph. As pullback and pushout are dual objects, an intriguing observation arises: the join operator in relational databases and the computation of connected components in graph databases demonstrate duality when examined through the framework of category theory.

\subsection{Inference Rules for MVDs} \label{sec:MVDInference}

Given a set of functional dependencies $F$ and a set of multivalued dependencies $M$, the inference rules to compute their closure can be found in literature, see e.g. \cite{10.5555/551350,10.1145/320613.320614,10.1145/509404.509414}. We provide those inference rules as follows. 
  
MVD 1: (Complement) If $X \to\to_U Y$, then $X \to\to_U (U-XY)$;

MVD 2: (Reflexivity) If $Y \subseteq X$ in a relation $U$, then $X \to\to_U Y$;

MVD 3: (Augmentation) If $Z \subseteq W$ and $X \to\to_U Y$, then $XW \to\to_U YZ$, where $U = X \cup Y \cup Z \cup W$;

MVD 4: (Transitivity) If $X \to\to_U Y$ and $Y \to\to_U Z$, then $X \to\to_U Z-Y$, where $U = X \cup Y \cup Z$.


There are two additional rules for both FD and MVD.

FD-MVD 1: If $X \to Y$ in a relation $U$, then $X \to\to_U Y$.

FD-MVD 2: If $X \to\to_U Z$ and $Y \to Z'$, ($Z' \subseteq Z$), where $U=X\cup Y \cup Z$, $Y$ and $Z$ are disjoint, then $X \to Z'$. 

For FD-MVD 2, given an element $x \in X$, we need to find a unique element $z' \in Z'$.  Without loss of generality, assume that $\exists y \in Y$ s.t.  $(x,y,z)\in U $, then since $Y \to Z'$, we can find this unique element $z' \in Z'$, s.t. $z'$ is the image of $y$ and $x$. Note that it is possible that there are multiple $y$'s associated with $x$ in $U$, but all those $y$'s should map to the same $z'$, otherwise, it contradicts with $X \to Z'$.

The detailed algorithm (by constructing the dependency basis) for computing the membership and closure of FDs and MVDs using the above rules can be found in previous work (e.g., \cite{10.1145/320613.320614,10.1145/322186.322190}).
 
 %one needs to compute the \textit{dependent basis} of an attribute $X$, denoted by $DEP(X)$. An MVD $X \to \to_U Y $ is in the closure if and only if $Y$ is a union of elements of $DEP(X)$. The algorithm to compute the dependent basis can be found in previous works, e.g.  \cite{10.1145/320613.320614}. Note that $X^+$  and $DEP(X)$ are indeed parallel concepts.  If we think of the collection of singleton sets, \{$(A)| A \in X^+$\}, then $X^+$ is just $DEP(X)$ that is functionally dependent on $X$ by FDs.
