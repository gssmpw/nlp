\section{Proof and Explanation of Lemmas and Theorems}
\label{sec:proofs}

\noindent \textit{Lemma \ref{lem:thincommutative}:  All diagrams in a thin category are commutative. }

\begin{proof} (Sketch) Without loss of generality, consider three objects $A$, $B$ and $C$ in a category.  Let $f: A \to B$,  $g: B \to C$, and   let $k = g \circ f$. Since there is only one arrow (i.e. $k$) between $A$ and $C$, then this diagram is commutative, otherwise, there are at least two arrows between $A$ and $C$, which contradicts the definition of a thin category. This commutative result can be proved for any two sequences of objects with the same starting and ending points, which concludes this proof.
\end{proof}




\noindent \textit{Theorem \ref{theo:BCNF}:  The relation schema $R$ output from the first reduced representation in Algorithm  \ref{alg:map2relationschema} is in the Boyce-Codd normal form.}

\begin{proof} 
(Sketch) We aim to prove that every non-trivial functional dependency (FD) in the relation $R$ is a key constraint. To establish this, we will employ a proof by contradiction. Assume the existence of an FD: $A \to B$ in the relation $R$, where $A$ is not a superkey, there are two cases:
(1)  $A$ has a single attribute.  Note that $R$ has at least one single attribute key $K$. This key, by definition, guarantees that $K \to A$ and $K \to B$. If there is an FD $A \to B$, then $K \to B$ should be removed in the minimal cover, which contradicts the fact $B \in R$. Note that the $K$ is possibly removed in \texttt{Clean} Function of Algorithm \ref{alg:map2relationschema}. But this does not change the above proof and the contradiction still occurs. (2) $A$ has multiple attributes, say $A_1 A_2... A_n \to B$. In the algorithm to compute the closure of a category (i.e. Algorithm \ref{alg:closureFD}, Lines 4 and 5), a new object $X$ with the associated projection arrows and  $X \to B$ has been added to the category. Then in the minimal cover, $K \to B$ would have been removed. This also contradicts that $B \in R$.

Furthermore,  note that in the output of Algorithm \ref{alg:map2relationschema}, it is possible for a relation $R$ to contain multiple candidate keys. This is due to the inclusion of bijective objects and their neighbors in $R$ (as indicated in Lines 15-17 of Algorithm \ref{alg:map2relationschema}). Each of these bijective objects corresponds to a candidate key within $R$.
\end{proof}

The inadequacy of BCNF (Boyce-Codd Normal Form) when applied to multiple relations has been demonstrated by Ling et al. \cite{journals/tods/LingTK81}. They identified that BCNF may contain ``\textit{superfluous}''  attributes. In response, the authors proposed an enhanced normal form in their paper, the following example illustrates the limitations of BCNF.

\begin{example} (\cite{journals/tods/LingTK81})
Let F =\{$AB \to CD$, $A \to E$, $B \to F$, $EF \to C$ \}, and consider the relational schema table $T_1(A,B,C,D)$ and key $AB$, $T_2(A,E)$ and key A, $T_3(B,F)$ and key B, and $T_4(E,F,C)$ and the key EF.  Although all tables satisfy the BCNF, a superfluous attribute is $C$ in $T_1$. This example shows the limit of BCNF to identify the redundant attributes across relations. \label{exp:improve3NF} 
\end{example}

\begin{definition} (\cite{journals/tods/LingTK81}) Given a set of relations $S = \{R_1,R_2,...,R_n\}$,  an attribute $B \in R_i$ is restorable in $S$ if $B \notin  K$ and  $K_i \to B$ is a functional dependency that can be derived by other FDs that do not involve $R_i$, where $K$ is a set of keys in $R_i$ and $K_i \in K$.
\end{definition}

In the above definition,  an attribute in a relation schema $R_i$ is \textit{restorable} means that the value of this attribute is derivable from the rest of $S$. In Example \ref{exp:improve3NF}, the attribute C in $T_1$ is restorable (superfluous), as its value is restorable from other tables. That is $AB \to C$ can be derived by $T_2$, $T_3$, and $T_4$, where $A \to E$, $B \to F$, and $EF \to C$.

\begin{definition} Given a set of relations $S = \{R_1,R_2,...,R_n\}$,  a relation schema $R_i \in S$ is in improved Boyce-Codd normal form if (i) $R_i$ is in BCNF; and (ii) no attribute of $R_i$ is restorable in $S$.    
\end{definition}

\noindent \textit{Theorem \ref{theo:improvedBCNF}   The relation schema $R$ output from the first reduced representation in Algorithm  \ref{alg:map2relationschema} is in the improved  Boyce-Codd normal form.}



\begin{proof} (Sketch) The proof of Theorem \ref{theo:BCNF} establishes that the output is guaranteed to be in BCNF. Building upon this result, we proceed to demonstrate that the output relations also do not possess any restorable attributes. In Algorithm \ref{alg:map2relationschema}, an attribute $B \in R_i$ only if there is an arrow from $R_i$ to $B$. In Algorithm \ref{alg:1RR}, note that Line 3 involves the computation of a minimal cover by considering all functional dependencies and arrows within the category. As a result, the minimal cover obtained is globally minimal, ensuring that each attribute $B$ in $R_i$ is not a superfluous attribute.   This concludes the proof.
\end{proof}

Following the exploration of the properties of relational data, we now shift towards examining the normal form for XML data.  We are able to establish proof that the resulting XML data, derived from the first reduced presentation, adheres to XML normal form as defined in the paper by Arenas and Libkin \cite{journals/tods/ArenasL04}. 


Let us introduce some concepts about XML normal form from \cite{journals/tods/ArenasL04}.  A tree tuple $t$ in a DTD $D$ is a function that assigns to each path in $D$ a value of an XML tree vertex or a string value. A tree tuple represents a finite tree with
paths from $D$ containing at most one occurrence of each path. For example, consider a DTD with expressions: $\epsilon \to student^+$ and $student \to @ID, BirthYear, Age$. An example of a tree tuple $t$ defines that: 
$t$($\epsilon.student$) = $v_1$, $t$($\epsilon.student.@ID$) = ``1'', $t$($\epsilon.student.BirthYear$.\textsc{\#P}) = ``2000'', etc, where the reserved symbols  \textsc{\#P} represent element type declarations \#PCDATA in DTD.


For a DTD D, a functional dependency (FD) over $D$ is an expression of the form $S_1 \to S_2$ where $S_1$ and  $S_2$ are nonempty sets of paths in $D$. For example, ``$\epsilon$.student.@ID $\to$ $\epsilon$.student.BirthYear.\#P" is an FD, meaning that each ID of a student can decide the birth year of this student. An XML tree $T$ satisfies $S_1 \to S_2$  if for every two tree tuples $t_1$, $t_2$, $t_1(S_1) = t_2(S_1)$ implies $t_1(S_2) = t_2(S_2)$ in $T$. 
For example,  see an XML tree of Figure \ref{fig:XMLNFExample}. It satisfies  ``$\epsilon$.student.BirthYear.\#P $\to$ $\epsilon$.student.Age.\#P''. But it does not satisfy ``$\epsilon$.student.BirthYear.\#P $\to$ $\epsilon$.student.Age", because the ``BirthYear'' cannot decide a unique ``Age'' vertex. As shown in Figure \ref{fig:XMLNFExample}, there are two ``Age'' vertexes associated with ``BirthYear 2000''. 

Given a DTD $D$, a set $\Sigma \subseteq FD(D)$ and $\varphi \in FD(D)$, we say that ($D, \Sigma$)
implies $\varphi$, written ($D, \Sigma$) $\vdash \varphi$ , if for any XML tree $T$ conforming to $D$ and satisfying $\Sigma$, then $T$ also satisfies $\varphi$. The set of all FDs implied by ($D, \Sigma$) is denoted by ($D, \Sigma$)$^+$.

\begin{definition} (\cite{journals/tods/ArenasL04})
Given a DTD $D$ and $\Sigma  \subseteq$  FD($D$), ($D$, $\Sigma$) satisfies XML normal form if and only if for every nontrivial FD: $\varphi \in$ ($D$, $\Sigma$)$^+$  of the form $X$ $\to$ $p$.@ID or $X$ $\to$ $p$.\#P, they must also have that $X$ $\to$ $p$ is in ($D$, $\Sigma$)$^+$, where $X$ is a finite nonempty subset of paths in $D$, and $p$ is one path in $D$. 
\end{definition}

\begin{figure}
\centering
\includegraphics[width=0.7\textwidth]{figures/XMLNFExample.png}
\caption{An example to illustrate the XML normal form} \label{fig:XMLNFExample}
\end{figure}

\begin{example} For example, recall a DTD with expressions: ``$\epsilon \to Student^+$'' and ``$Student \to @ID, BirthYear, Age$'', where XML FDs include that ``$\epsilon$.$Student.@ID$ $\to$ $\epsilon$.$Student.BirthYear.\#P$'',  ``$\epsilon$.$Student.@ID$ $\to$ $\epsilon$.$Student.Age.\#P$'' and ``$\epsilon.Student.BirthYear.\#P$ $\to$ $\epsilon.Student.Age.\#P$''. This DTD is not in an XML normal form. See an example XML tree in Figure \ref{fig:XMLNFExample}(a). Because it is not the case ``$\epsilon.Student.BirthYear.\#P$ $\to$ $\epsilon.Student.Age$'', as there are two $Student$ vertexes with the same $BirthYear$, but different $Age$ vertexes.  To remedy this issue, the $Age$ tag is removed from the children of the $Student$. Consider another DTD: ``$\epsilon \to Student^+ BirthYear^+$'', ``$Student \to @ID, BirthYear$'' and ``$BirthYear \to Age$''. This is in XML normal form, which avoids redundant storage. The corresponding XML  tree is shown in Figure \ref{fig:XMLNFExample}(b).
\end{example}


\begin{lemma}
Given an arrow $f: O_1 \to O_2$ in a representation of a category $\mathcal{C}$, by using Algorithm  \ref{alg:map2DTD} to convert $\mathcal{C}$ to a DTD schema $D$, the corresponding XML FD in $D$ is of the form $p.O_1 \to p.O_1.O_2$  where $p$ is a  path in $D$.
\label{lem:NFPrepare}\end{lemma}

\begin{proof} (Sketch)
We will consider two cases regarding the structural relationship between $O_1$ and $O_2$ in an XML tree that conforms to the output DTD $D$, following the conversion by Algorithm \ref{alg:map2DTD}.

Case (1): $O_1$ is not the parent of $O_2$ in the output XML tree. In this scenario, we have $p.O_1 \to p.O_2$, where both $O_1$ and $O_2$ share the same prefix path $p$, and they are sibling nodes in the XML tree. To demonstrate that this case is not possible, we will employ a proof by contradiction. Let us assume that $O_3$ is a node that represents the common parent of $O_1$ and $O_2$ in the tree. Consequently, we have $O_3 \to O_1$ and $O_3 \to O_2$ in the category. However, this contradicts the fact that the first reduced representation is a minimal cover of arrows, as the existence of $O_3 \to O_1$ and $O_1 \to O_2$ would imply $O_3 \to O_2$. Hence, this case is not possible.

Case (2): $O_1$ is the parent of $O_2$ in the XML tree.
In this situation, the output functional dependency  is of the form: $p.O_1 \to p.O_1.O_2$, which is the desired outcome of the mapping process.

\end{proof}

\noindent \textit{Theorem \ref{theo:DTDNormalform}. The DTD schema output D from the first reduced representation in Algorithm \ref{alg:map2DTD}  is in XML normal form.}


\begin{proof}
(Sketch) We aim to prove that $X \to p.\#P$ or $X \to p.@ID$ implies $X \to p$ in the output DTD.
Let us assume that the DTD schema $D$ is derived from a representation $G$ of a category using Algorithm \ref{alg:map2DTD}. We denote the last two objects in the paths of $X$ and $p$ as $O_1$ and $O_2$ in $G$, respectively. Since $G$ contains the arrow $O_1 \to O_2$, we can infer the existence of the corresponding XML functional dependency $p'.O_1 \to p'.O_1.O_2 \in (D, \Sigma)$ based on Lemma \ref{lem:NFPrepare}.

Now consider the following four cases about the path structure of $X$:


Case (1) $X= p'.O_1$: In this case, $X \to p'.O_1.O_2$. Note that  $p'.O_1.O_2 \to p'.O_1.O_2.@ID$ and  $p'.O_1.O_2 \to p'.O_1.O_2.\#P$. Thus, $X \to p$ belongs to the closure of ($D$, $\Sigma$)$^+$.

Case (2) $X= p'.O_1.@ID$: We have $p'.O_1.@ID \to p'.O_1$, as @ID is a global unique attribute for each element in $O_1$. By the composed rule of arrows,  ($X \to p) \in$ ($D$, $\Sigma$)$^+$.

Case (3) $X= p'.O_1.\#P$: Similarly, we have $ p'.O_1.\#P \to p'.O_1$, as $O_1$ is an attribute object and  different elements in $O_1$ have distinct \#$P$ values. Meanwhile, each $O_1$ is processed only once in Algorithm \ref{alg:map2DTD}. Thus, ($X \to p) \in$ ($D$, $\Sigma$)$^+$.

Case (4) $X= p'.O_1.@l$: where $@l$ is an attribute under $O_1$ except @ID.  We prove that this case is not valid. Without loss of generality, assume this $@l$ attribute represent an attribute object $O_3$ in $G$.  That is, $X= p'.O_1.@O_3$. In other words, $O_1 \to O_3$. Further, note that $O_3 \to O_2$, as $X \to p.@ID$ or $X \to p.\#P$ and $O_2 \in p$. This contradicts that $O_1 \to O_2$ is in 1RR, as it is a redundant arrow due to the existence of $O_1 \to O_3$ and $O_3 \to O_2$.   Thus, 
$ (p'.O_1.@l \to p'.O_1.O_2.@ID) \notin (D,\Sigma)^+$ and $ (p'.O_1.@l \to p'.O_1.O_2.\#P) \notin (D,\Sigma)^+$.


Hence, in all cases above,  we have demonstrated that $X \to p.\#P$ or $X \to p.@ID$ leads to $X \to p$ in the closure of ($D$, $\Sigma$), which concludes the proof.
\end{proof}

%\begin{example}Consider the input graph in Figure \ref{fig:1RRExample}(a) again.  If We use Algorithm \ref{alg:map2DTD} to convert it to XML DTD, there is an XML FD: $\epsilon.A.B_A.\#P \to \epsilon.A.C_A.\#P$. But in this case $\epsilon.A.B_A.\#P \to \epsilon.A.C_A$ does not hold, because it is possible that there are two tree tuples $t_1$ and $t_2$, s.t. $t_1(\epsilon.A.B_A.\#P) = t_2(\epsilon.A.B_A.\#P)$, but $t_1(\epsilon.A.C_A) \neq t_2(\epsilon.A.C_A)$. This, this output schema is not in XML normal form. However, if we convert Fig.  \ref{fig:1RRExample}(a) to (c) with 1RR, $C$ is not a child of $A$ and the XML FD becomes  $\epsilon.B.\#P \to \epsilon.B.C_B.\#P$. Note that $\epsilon.B.\#P \to \epsilon.B.C_B$ holds, as each value of $B$ is stored once as a child of the root. Therefore, the output schema with 1RR satisfies XML normal form. \end{example}


\begin{example} This example illustrates that 1RR results in the schema in XML normal form. Recall Figure \ref{fig:1RRExample}(a).  If we use Algorithm \ref{alg:map2DTD} to convert it to XML DTD, there is an XML FD: $\epsilon.A.B.\#P \to \epsilon.A.C.\#P$. But in this case $\epsilon.A.B.\#P \to \epsilon.A.C$ does not hold, because it is possible that there are two tree tuples $t_1$ and $t_2$, s.t. $t_1(\epsilon.A.B.\#P) = t_2(\epsilon.A.B.\#P)$, but $t_1(\epsilon.A.C) \neq t_2(\epsilon.A.C)$. This output schema is not in XML normal form. However, if we convert Fig.  \ref{fig:1RRExample}(a) to (c) with 1RR, $C$ is not a child of $A$ and the XML FD becomes  $\epsilon.B.\#P \to \epsilon.B.C.\#P$. Note that $\epsilon.B.\#P \to \epsilon.B.C$ holds, as each value of $B$ is stored once as a child of the root. Therefore, the output schema with 1RR satisfies XML normal form. 
\end{example}


Because of the absence of a well-established normal form theory specific to graph data (as far as our knowledge), there exists no analogous property that can be proven for the output of a property graph. Nevertheless, as the framework remains unified, we anticipate that the resulting graph data schemata exhibit favorable characteristics that help mitigate a certain level of redundancy, as illustrated in the following example.

\begin{figure}
\centering
\includegraphics[width=0.6\textwidth]{figures/GraphExample2.png}
\caption{An example to illustrate the graph normal form (1RR)} \label{fig:GraphExample}
\end{figure}

\begin{example}
Recall Figure \ref{fig:1RRExample}(a). When applying Algorithm \ref{alg:map2graphschema} to convert this category to a graph schema, we obtain the following schema: nodes $V= \{A,D\}$, edges $E= \{(D,A)\}$, properties $T=\{SK,B,C,E\}$, $P(D)=\{SK, E\}$ and $P(A)=\{SK, B, C\}$. In Figure \ref{fig:GraphExample}(a), we present an example of a graph data instance that adheres to this schema. Note that the properties of $E$ are omitted for simplicity in the figure. In this instance, the values of properties $B$ and $C$ are duplicated and stored for different nodes of type A. However, by adopting the 1RR approach in Figure \ref{fig:1RRExample}(c), we can avoid such redundancy. A graph instance based on the schema from 1RR is shown in Figure \ref{fig:GraphExample}(b), where a new node with type $B$ is created, and $b_1$ and $c_1$ occur only once. This example intuitively demonstrates the benefits of using 1RR in graph schema design, effectively reducing redundancy.
\end{example}

%\noindent \textit{Definition \ref{def:generalized4NF}: (Improved 4NF) A set of relation schemata $S$ is in the improved fourth normal form with respect to a set $F$ of functional, multivalued, and embedded binary join dependencies if, for every nontrivial embedded join dependency $EBJ_{x_1,\cdots,x_q}$($R_1$,$R_2$) specified on relation schema $R$ with $F^+$ (that is implied by $F$),  both $R_1$ and $R_2$ are superkeys of $R$. }


%Note that a multivalued dependency is a special case of embedded binary join dependency. That is, any multivalued dependency $X \to\to_U Y$ implies $EBJ_U(XY,X(U-Y))$. An embedded join dependency $EBJ_{x_1,\cdots,x_q}$($R_1$,$R_2$), specified on relation schema $R$, is \textbf{trivial} if one of the relation $R_1$ or $R_2$ is equal to $R$. Such a dependency is called trivial because it has the nonadditive join property for $R$ and thus does not specify meaningful constraints on $R$.   Therefore, the above improved 4NF is defined for the nontrivial embedded binary join dependency.



\noindent \textit{Theorem \ref{the:4NF}: Each relation schema output from the second reduced representation in Algorithm \ref{alg:map2relationschema} is in the  fourth normal form.}

\begin{proof} Let us first recall the definition of 4NF in the textbook \cite{elmasri2000fundamentals}.

\textbf{Definition}: A relation schema $R$ is in 4NF with respect to a set of dependencies $F$ (that includes functional dependencies and multivalued dependencies) if, for every nontrivial multivalued dependency $X \to \to Y$ in $F^+$, $X$ is a superkey for $R$.

Note that if $X \to Y$ in a relation $U$, then $X \to \to_U Y$. Thus any functional dependency can be also considered as a multivalued dependency. A multivalued dependency $X \to\to Y$ in $R$ is  trivial  if (a) $Y$ is a subset of $X$ or (b) $X \cup Y = R$.   By Theorem \ref{theo:BCNF}, we have already established that the  schema from 1RR follows BCNF.  That is for every functional dependency $X \to Y$ in $F^+$, $X$ is a superkey of $R$. Next we consider the case where the nontrivial multivalued dependencies that are not functional dependencies. In Line 2 of Algorithm \ref{alg:2RR} that computes the second reduced representation, we observe that all multivalued dependencies (MVDs) objects are decomposed.  This implies that there are no such nontrivial multivalued dependencies  $X \to \to_U Y$ in $F^+$ which can be output in a single relation $U$ in Algorithm \ref{alg:map2relationschema}. This concludes the proof.
\end{proof}


%\begin{proof} (Sketch) Our goal is to show that the output relation schema from Algorithm \ref{alg:map2relationschema} with 2RR has no constraints other than key constraints. By Theorem \ref{theo:BCNF}, we have already established that there are no non-trivial functional dependencies involving non-prime attributes. Next we further demonstrate the absence of embedded binary join dependencies.

%Recall that an embedded binary join dependency $EBJ(R_1, R_2)$ is equivalent to a multivalued dependency $X \to\to_U Y$ or $X \to\to_U Z$, where $X = R_1 \cap R_2$, $Y = R_1 - X$, $Z = R_2 - X$, and $U = R_1 \cup R_2$. Consequently, an embedded binary join dependency involves a subset of $U$. In Line 2 of Algorithm \ref{alg:2RR}, we observe that all derivable multivalued dependencies (MVDs) involving $X \to\to Y$ are decomposed, effectively destroying them. As a result, no derivable MVD objects are left in the output schema. This implies that there are no multivalued dependencies in the output schema, confirming its adherence to the fourth normal form. This concludes the proof.\end{proof}

%Currently, as far as we know, there is no well-established normal form theory for XML and graph data that directly aligns with the relational 4NF. Consequently, we are unable to show an analogous property for the output of DTD or graph schema for Algorithm \ref{alg:map2DTD} and \ref{alg:map2graphschema}. However, the unified framework proposed in this paper enables us to regard the output schema with the second reduced representation as achieving a similar normal form to 4NF for XML and graph data, as illustrated below:

%\begin{figure} \centering\includegraphics[width=0.7\textwidth]{figures/2RRGraphExample.png}\caption{An example to illustrate the graph normal form (2RR)} \label{fig:GraphExample2RR}\end{figure}

%\begin{example} This example illustrates a graph normal form with 2RR (corresponding to 4NF). See Figure \ref{fig:GraphExample2RR}. There are two MVDs: \textit{Course} $\to\to$ \textit{Teacher} and \textit{Course} $\to\to$ \textit{Student}. The schema in Figure \ref{fig:GraphExample2RR}(a) contains a $SCT$ node to interconnect three nodes, resulting in duplicated information due to the presence of the two MVDs. In contrast, Figure \ref{fig:GraphExample2RR} (b) is a normalized graph schema based on 2RR, wherein the $SCT$ object is decomposed into two distinct relationship objects ($SC$ and $TC$ nodes). This decomposition effectively eradicates redundancy in the schema.\end{example}

%\begin{figure}\centering\includegraphics[width=0.7\textwidth]{figures/NFScope.jpg}\caption{Reduced representations of categories and database normal forms}\label{fig:NFScope}\end{figure}

% Figure \ref{fig:NFScope} depicts the structures and relationships among various database normal forms and two reduced representations of categories discussed in this paper.