
\section{Introduction}

 

Managing data variety poses a significant challenge in modern database systems due to the diverse formats and models of data sources \cite{Lu:2019:MDN:3341324.3323214,journals/pvldb/KiehnSGPWWPR22}. Currently, relational data adheres to the relational model, XML and JSON data are typically represented using tree-structured models, and property-graph and RDF data are structured as graphs. A natural yet fundamental question is whether there exists a unified data model to unify those different types of data, which will bring together disparate types of data into a coherent framework and lead to a deeper understanding of the diverse data and potentially have practical applications for developing a new type of database.


%Handling such diversity exceeds the capabilities of a single relational database. As a result, two prominent solutions have emerged: polyglot persistence \cite{journals/pvldb/KiehnSGPWWPR22,conf/cikm/LuHC18} and multi-model databases \cite{journals/csur/LuH19,conf/er/HolubovaSL19}. Polyglot persistence involves using multiple databases to handle various types of data, while multi-model databases offer a unified backend for storing and querying multiple data models. 

%To illustrate this, let’s consider an example of a multi-model data environment of an E-commerce application that encompasses customers, social network connections, and order information, each utilizing distinct data models. The graph data captures the mutual relationships between customers, customer information is stored in relational tables, and purchase records are represented as JSON documents. In applications such as Customer-360-View \cite{journals/bpmj/Kotorov03}, where users demand a comprehensive analysis of customer behaviors, the challenge lies in combining and analyzing information from these disparate data sources.

%Humans are inherently curious and like to seek a unified theory in science that can simplify and make sense of complex phenomena by finding common principles or underlying connections. In the field of multi-model data management, we seek to develop a unified data model. 




%The management of data variety is a crucial concern in modern data management systems, as data sources often exhibit different formats and models. Handling such diversity exceeds the capabilities of a single relational database. As a result, two prominent solutions have emerged: polyglot persistence \cite{journals/pvldb/KiehnSGPWWPR22,conf/cikm/LuHC18} and multi-model databases \cite{journals/csur/LuH19,conf/er/HolubovaSL19}. Polyglot persistence involves using multiple databases to handle various types of data, while multi-model databases offer a unified backend for storing and querying multiple data models. Both approaches share a common challenge: \textit{providing a unified view and interface for different data models to facilitate integrated data processing}. Regardless of the number of databases in the backend, the system should present users with a unified and holistic perspective, enabling seamless queries and analyses across the entire dataset.



In this paper, we propose a unifying framework through the lens of category theory to address three types of structured data: relation, XML, and property graph data. Category theory's foundations were established by mathematicians \texttt{Samuel Eilenberg} and \texttt{Saunders Mac Lane} in the 1940s with the primary goal of categorizing and unifying topological objects and algebraic objects. Although category theory originated as an abstract mathematical theory, its applications have extended to various scientific fields, such as physics \cite{coecke2006introducing,zeng2019quantum}, philosophy \cite{peruzzi2006meaning,gangle2015diagrammatic}  and computer science \cite{pierce1991basic,fiadeiro2005categories,shiebler2021category}.


\begin{figure}\centering\includegraphics[width=0.6\textwidth]{figures/FirstExampleCategory.jpg}
\caption{This example shows a categorical representation of a toy database, where the object ``\texttt{Student}'' has two attribute objects, ``\texttt{First\_name}'' and ``\texttt{Last\_name}'' A function ``\texttt{First\_name}'' maps the surrogate key $S001$ to ``\texttt{John}'' and $S002$ to ``\texttt{Emily}''. `\texttt{Registration}'' is a relationship object (set) with two projection functions for ``\texttt{Student}'' and ``\texttt{Course}''.} \label{fig:firstexamplecategory}
\end{figure}


A category is composed of \textit{objects} and \textit{morphisms} that establish connections between pairs of objects, known as the domain and codomain of the morphism. Initially, morphisms in category theory were primarily conceptualized as homomorphisms between algebraic structures or continuous maps between topological spaces. However, as categories found applications in diverse domains, researchers began exploring objects and morphisms from domain-specific perspectives. For instance, in physics, objects are defined as states of systems, while morphisms represent transitions from one state to another \cite{coecke2006introducing}. In logic, objects can be represented as propositions, and morphisms serve as a means to demonstrate the inference of one proposition from another \cite{blute2004category}. In this paper, we adopt the category of \textbf{sets}, where each object corresponds to a set, and each morphism represents a function between two sets. Figure \ref{fig:firstexamplecategory} shows an example of a category representation of a toy database. 

%For example, the object ``\texttt{Student}" encompasses two attribute objects, ``\texttt{First\_name}" and ``\texttt{Last\_name}" A function $f$ from ``\texttt{Student}" to `\texttt{First\_name}" maps $S001$ to ``\texttt{John}" and $S002$ to ``\texttt{Emily}". `\texttt{Registration}" is a relationship object (set) with two projection functions for ``\texttt{Student}" and ``\texttt{Course}".


In the following, we explore several fundamental concepts in category theory, such as composed morphisms, pullback and limit, providing intuitive examples that demonstrate how these abstract mathematical concepts can be effectively applied to databases.



\noindent \textbf{Composed morphisms}: Let $\mathcal{C}$ be a category, and let $f: A \to B$ and $g: B \to C$ be two morphisms in $\mathcal{C}$. The composed morphism, denoted as $g \circ f$, represents the composite of $f$ and $g$. In our setting, morphisms are defined as functions between sets, so the composition $g \circ f$ is to be thought of as the composed functions.  When there are three composable morphisms, the associativity  $(h \circ g) \circ f$ = $h \circ (g \circ f)$ holds. A \textit{functional dependency} (FD) in databases defines how a given set of attributes determines another set of attributes.  Each morphism in a category naturally corresponds to a function dependency. The composition of morphisms in a category can be understood as the transitivity of functional dependencies. 
 
 %In addition, a morphism between two sets is similar to a functional dependency between two attributes in databases. The composed morphism follows the transitivity rule in functional dependency. 
 
 %Meanwhile, the functional dependency has more inference rules (i.e. Amsatrong axioms), which may not hold for a general category. But they hold for the database category discussed in this paper. We will discuss it later in more detail. 

 \smallskip
 
  \noindent \textbf{Pullback}:  The pullback object $P$ represents the ``\textit{most general}'' object that simultaneously satisfies the functions specified by $f$ and $g$. Given a category with objects $A$, $B$, and $C$, and morphisms $f: A \to C$ and $g: B \to C$, the pullback of $f$ and $g$ is an object $P$ along with two morphisms $p_1$: $P \to A$ and $p_2: P \to B$ that satisfy a universal property. Note that pullback objects can be used to define \textit{multivalued dependency} (MVD). To understand it, consider two MVDs: \texttt{Class} $\to\to$ \texttt{Text}, and \texttt{Class} $\to\to$ \texttt{Instructor}. Then the following diagram shows that the relation \texttt{(Class, Text, Instructor)} defines indeed a pullback from the two sub-relations \texttt{(Class, Text)} and \texttt{(Class, Instructor)}. 

\[\xymatrix{(Class, Text, Instructor) \ar[r]^{p_1} \ar[d]_{p_2} & (Class, Instructor) \ar[d]^f \\(Class, Text)  \ar[r]_g & Class}\]

%\begin{figure*}[t]\centering\includegraphics[width=0.8\textwidth]{figures/FirstExample.jpg}\caption{This example shows three types of data including relation, XML and property graph. They have the same unified categorical representation.} \label{fig:firstexample}\end{figure*}



 %To illustrate this concept, imagine   $f$ representing the child-mother relationship between two individuals  in a category. In this case, $f \circ f$ corresponds to the grandson-grandmother relationship. One fundamental property of morphism composition in category theory is associativity. This means that given three composable morphisms, the associativity property $(h \circ g) \circ f$ = $h \circ (g \circ f)$ holds. In other words, the order in which the morphisms are composed does not affect the final result.



 %This example highlights the connection between commutative diagrams and the theory of database normal forms, as the inclusion of $T4$ can lead to update anomalies across the tables.

%Universal morphism. Universal properties define objects uniquely up to a unique isomorphism. We may use this example to give the intuition of the categorical normal form. The weak entity in ER diagram can be explained with the universal morphism. See the example in Fig \ref{fig:commutativediagram}, where there is a unique morphsm from the dependent (the weak entity) to the employee (the own entity) so that the morphisms commmutes.

%\noindent  \textbf{Limits}:  A multivalued dependency is a special case of join dependency. Meanwhile, a pullback is a special type of limit. The concept of limits in category theory holds significant importance in the analysis and characterization of structures within categories. A limit can be viewed as a universal solution to a collection of objects and morphisms in a category.  While the precise mathematical definition of limits will be explored in subsequent sections, let us illustrate the notion of limits by considering a scenario with three tables \texttt{SP(Supplier, Product)}, \texttt{SJ(Supplier, Project)}, and \texttt{PJ(Project, Product)}. Surprisingly, the table \texttt{SPJ (Supplier, Project, Product)} = $SP \bowtie PJ \bowtie SJ$, which joins three tables, can be described using the concept of a limit. This is visually depicted as follows, which represents the corresponding limit diagram. 

\smallskip

%A multivalued dependency represents a specific case of a join dependency, while a pullback is a special type of limit. A limit serves as a universal solution to a collection of objects and morphisms in a given category.

\noindent  \textbf{Limit}:  In category theory, the concept of limit holds significant importance in the analysis and characterization of structures within categories.  While the precise mathematical definition of limits will be provided in subsequent sections, we can illustrate the notion of limit through the scenario involving three tables: \texttt{SP(Supplier, Product)}, \texttt{SJ(Supplier, Project)}, and \texttt{PJ(Project, Product)}. Interestingly, the table \texttt{SPJ (Supplier, Project, Product)} = $SP \bowtie PJ \bowtie SJ$, which joins these three tables, can be described using the concept of a limit. This description can be visually depicted through the corresponding limit diagram in Figure \ref{fig:SPJJoin}.


\begin{figure}\centering\[\xymatrix{&*+{SPJ (Limit)} \ar[dl] \ar[d] \ar[dr] \\SP \ar[dr] \ar[d]  & PJ  \ar[dl] \ar[dr] & SJ  \ar[d] \ar[dl]  \\  Product & Supplier & Project}\]
\caption{An example to illustrate join limit.} \label{fig:SPJJoin}
\end{figure}



%There is a correlation between the limits of category theory and the join dependency \cite{maier1979testing,10.1007/3-540-08921-7_102} employed in databases.

%\begin{figure}\centering\includegraphics[width=0.4\textwidth]{figures/graphpushout.jpg}\caption{Illustration to the pushout for computing the connected component in undirected graphs}\label{fig:graphpushout}\end{figure}

%\smallskip

%\noindent \textbf{Pushout}:  In category theory, pushouts and pullbacks are dual concepts that capture different aspects of relationships between objects and morphisms within a category. The pullback captures the concept of finding a common solution or shared property between two morphisms, and pushout captures the concept of combining two morphisms to form a new object. While pullback represents the join operator in relational data, we will show that pushout can represent the computation of the connected component in a graph.  


%For example, the following diagram shows the pushout objects which denote all connected components in the graph, where the \texttt{Edge} table includes two attributes \texttt{Node\_id1} and \texttt{Node\_id2}. 


%\[
%\xymatrix{
%    Edge \ar[r]^{f} \ar[d]_{g} & Node\_id1 \ar[d]^{i} \\
%    Node\_id2 \ar[r]_{j} & Component
%}
%\]

%Furthermore, the join dependency is linked to the fourth normal form (4NF) and the projection-join  normal form (PJ/NF) \cite{fagin1979normal}. In Section 5, we will delve into the advancement of normal form theory for multi-model databases through the lens of category theory.

%Consequently, this research presents a promising avenue for advancing normal form theory for multi-model databases through the lens of category theory.

%the fourth normal form \cite{journals/tods/Fagin77} and

%\begin{figure}\centering\includegraphics[width=0.4\textwidth]{figures/limitexample2.jpg}\caption{An example to illustrate the example of limit constraint for SPJ example.}\label{fig:limit}\end{figure}

%\begin{figure}
%\centering
%\includegraphics[width=0.4\textwidth]{figures/limitexample3.jpg}
%\caption{An example to illustrate the example of limit constraint for SPJ example.}
%\label{fig:limit}
%\end{figure}


%\begin{figure}\centering\includegraphics[width=0.5\textwidth]{figures/MVDExample.jpg}\caption{An example to illustrate the pullback and the multivalued dependencies.}\label{fig:MVD}\end{figure}

%such that whenever a supplier $s$ supplies product $p$, and a project $j$ uses product $p$, and the supplier $s$ supplies something for this project $j$, then the triple ($s, p,j$) appears in the table \texttt{SPJ}, meaning that the supplier $s$  will also be supplying product $p$ to project $j$. 



The true measure of a new framework lies in its ability to yield novel findings. In this paper, the proposed categorical framework makes the following  contributions:

(i)  The framework presented here offers a unique perspective on multi-model data by defining a database as a set category. We illustrate the connection between pullback and the binary join operator (as shown in multivalued dependency), as well as the connection between pushout and the connected components of an undirected graph. Furthermore, considering the duality between pullback (limit) and pushout (colimit) in category theory, an intriguing observation arises: the join operator in relational databases and the computation of connected components in graph databases manifest ``\textit{duality}''  when analyzed through the lens of category theory.

 

%We also describe algorithms to compute the closure of functional dependencies (FDs) and multivalued dependencies (MVDs) in categories.  

%incorporating category theory concepts like \textit{pullback}, \textit{pushout}, and \textit{limit} into the entity-relationship (ER) diagrams. Notably, there are two intriguing connections between the structural properties of categories and database operations. Firstly, the pullback object in category theory corresponds to the join operator in relational databases. Secondly, the pushout object corresponds to the connected component in a graph. Since pullback and pushout are dual objects, a captivating observation emerges: the join operator of relational databases and the computation of connected components of graph databases exhibit duality when viewed through the lens of category theory.


%By leveraging the dual notions of limit and colimit within categories, we demonstrate that the counterpart to join operations in relational databases is, in fact, the computation of connected components in undirected graphs. Notably, both operations exhibit computation complexities of PTIME-hard and LOGSPACE.

 (ii)  Existing research has proposed a rich set of normal forms for various data models, including relation (e.g. \cite{journals/tods/Fagin77,fagin1979normal,fagin1981normal}), XML (e.g. \cite{DBLP:conf/pods/ArenasL03,journals/tods/VincentLL04}) and object data model (e.g. \cite{journals/tods/TariSS97}). In this paper, our approach leverages the categorical framework to establish a coherent normal form theory across diverse data types, which avoids the necessity for format-specific definitions. This categorical framework presents a fresh perspective for comprehending the theory of normal forms in databases, where the process of devising database normal forms is essentially equivalent to eliminating redundant morphisms and objects that can be derived from others within a category's representation.  When viewed through the lens of category theory, schema normalization becomes ``\textit{representation reduction}'' of a category. In particular, the data schema output from the first reduced representation satisfies the Boyce-Codd Normal Form (BCNF) and XML Normal Form (XML NF) (\cite{DBLP:conf/pods/ArenasL03}), while that from the second reduced representation conforms to  Fourth Normal Form (4NF).


% (iii) Inspired by this categorical framework, the third contribution is of independent interest, deviating from the original research focus on multi-model data.  We study how to decompose a universal relation to satisfy Boyce-Codd normal form (BCNF). It is a widely acknowledged fact that finding a BCNF decomposition with the properties of both nonadditive join and dependency preservation is not always possible, depending on the set of functional dependencies (FDs) provided. However, in this paper,  we propose an alternative approach that demonstrates the feasibility of achieving BCNF decomposition with both nonadditive join and dependency preservation, by rewriting the set of FDs. The unexpected discovery of this method offers new insights into the field of relation schema design.
 
 %decomposition and provides valuable insights into [specific aspect]. We believe that this serendipitous contribution adds an exciting dimension to our study and 
 
 %enabling a more coherent and comprehensive treatment of multi-model data integrity.
 
 %By systematically removing such derivable elements, we can establish an advanced hierarchy of normal forms. 
 
 %This unified approach 


 %\noindent \textbf{Organization of the paper} ~~ 
 
 
 %In the rest of this paper, we first establish the categorical model of databases by extending the ER model (Section 2).  We then delve into the computation of the closure of a category with FDs and MVDs (Section 3).  The subsequent section (Section 4) elucidates the normalization process of a category representation. Section 5 introduces an algorithm tailored to facilitate the mapping of a category's representation to a relational data schema. In Section 6, we establish the theoretical correlation between the aforementioned reduced representations and existing relational and XML normal forms.

 The contributions and the organization of this paper are summarized as follows:

 \begin{itemize}
     \item We establish the categorical framework for multi-model databases by extending the ER model with a \textbf{thin set} category that includes rich categorical elements such as objects, morphisms, pullbacks, pushouts, and limits, among others. (Section 2).
     \item We show algorithms for the computation of the closure of function dependencies (FDs) and multivalued dependencies (MVDs) on categories  (Section 4).
     \item We build the framework of the normal form theory on categories by designing two levels of reduced representations of categories (Section 5).
     \item We introduce algorithms tailored to facilitate the mapping of a category's representation to a relational data schema, an XML DTD and/or a graph schema (Section 6 and Appendix).
     \item We demonstrate the correlations between the two levels of reduced representations of a category and BCNF, 4NF, and XML NF (Section 7).
     
 \end{itemize}
 
 
  %Section 5 develops an algorithm to map a representation of a category to relational data schema. and we establish the correspondence between the reduced representation and the other existing relational and XML normal forms in Section 6.


 %In Section 2, we establish the categorical model by extending the ER model.   Section 3 delves into the computation of the closure of a category empowered by functional dependencies and multivalued dependencies. Section 4 introduces two levels of reduced representation for a category. These concise representations involve the removal of redundant objects and morphisms. Transitioning to Section 5, we develop an algorithm to map a representation of a category to relational data schema. Section 6 focuses on establishing the correspondence between the reduced representation and the normal form theory.
 
 
 %We demonstrate that the first reduced representation guarantees that the output data satisfies the (improved) Boyce-Codd normal form (BCNF), and XML normal form. The second reduced representation satisfies the (generalized) fourth normal form. Lastly, Section 8 concludes this paper.
 
 %Lastly, the third reduced representation does not align with the traditional normal forms, but we establish its connection to previous terminologies on data dependency and integrity, such as project-join dependency in relational databases \cite{YANNAKAKIS19822} and local dependency in object databases \cite{journals/tods/TariSS97}.
 
 %We introduce objects, morphisms, limits, colimits and commutative diagrams. These concepts significantly enrich the semantics of ER model and can be used to design the novel formal theory later.  

 
 
 %We define three levels of reduced representation. In the eye of category theory, the essences of different levels of normal form correspond to the removal of the arrows and objects in the categorical framework. This observation captures the essence of the normal form theory and can be applied to different models of data simultaneously. 

  

 

%There is a plethora of research on relational data, and much less work on XML data, and as far as our knowledge almost no previous work on the normal form definitions for graph data. This categorical framework  opens up a new door towards understanding and designing of normal form theory over different models of data, as our proposed multi-model normal form theory satisfies known constraints while remaining applicable to various types of data simultaneously, eliminating the need for ad-hoc format-specific definitions. In addition, this new normal form can use commutative diagrams to capture the semantics between objects, which corresponds to interrelational constraints, which is not well studied even for the normal form of relational data. This new framework  expands the scope of traditional normal form theory.  


 
% The importance of the categorical framework of this paper for multi-model data can be understood in a roughly analogous way to the classical relation theory for relational data with two aspects.  First,  the relational model provides a theoretical foundation to organize and view data in relational databases. Analogously, categorical models describe the logical structure and semantics of multi-model data, which can be used as a sound foundation  for the next-generation multi-model databases. Second, the relational normal form defines the normalized schema which is useful for database design. Analogously, the categorical normal form (through three levels of reduced representation of categories) in this paper provides a unified approach to design the multi-model data, which can be considered as an extension of relational normal form theory.  

 This paper is not the first to apply category theory to databases, even in the more specific domain of multi-model databases (e.g. see the previous works \cite{video,journals/jbd/KoupilH22}). For an extensive review of the related literature, please refer to Appendix \ref{sec:relatedwork}. However, note that the contribution of this research lies not only in the development of a categorical model for multi-model databases but also in proposing a cross-model normal form theory for understanding the fundamental principles of multi-model database normal forms. 
 
 
 
 
 
% The novelty in this paper is not only to build a unified conceptual model for multi-model databases with rich semantic constraints, but also to show a fascinating connection between categorical structures and database normal form theory, which shed light on the cross-model normal form theory.  

 %

%The category theory has been introduced into the database field with many previous works (e.g.  \cite{islam1994categorical,journals/mscs/BaclawskiSW94,journals/cj/HofstedeLF96}). 

 

%The rest of the paper is organized as follows. Section 2 defines a unified categorical model for multi-model data and Section 3 is devoted to defining different types of representations of categories, which can be used to define different levels of normal forms. Section 4 presents algorithms to map a category to different structures of data and then Section 5 introduces the categorical normal forms. Finally, Section 5 introduces the related works, and Section 6 concludes this paper. 

%As a final remark in this section, we have taken a deliberate approach by providing fundamental mathematical definitions of category theory alongside intuitive examples drawn from database applications. This approach serves the purpose of making the content accessible and comprehensible to readers from the database community. Our intention is to bridge any knowledge gaps and facilitate a thorough understanding of the concepts presented in this paper.






