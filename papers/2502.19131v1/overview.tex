\section{An overview of categorical normal form}
\label{sec:overview}

\begin{figure}\centering\includegraphics[width=0.9\textwidth]{figures/reduced03.png}\caption{This figure intuitively illustrates the two reduced representations by removing redundant information. The 1RR removes the composed morphisms and the 2RR further removes the  limit (pullback) object. }\label{fig:reduced}\end{figure}

In this short section, we provide an overview of the categorical normal form theory, which will be described in detail in the subsequent sections.

%Figure \ref{fig:reduced} illustrates the main idea of the processing of normalization in categories. Given the graph representation of a category, generally speaking, the normalization process of the category is to repeatedly remove the redundancy morphisms or objects in the graph representation. In particular,  Figure \ref{fig:reduced} (b) shows the first reduced representation which eliminates the morphisms that can be composed from other morphisms. Figure \ref{fig:reduced} (c) shows the second reduced representation which eliminates the limit object that can be computed from other objects. We connect the relational normal form theory and the reduced representation of categories as follows:

Figure \ref{fig:reduced} illustrates the main idea behind the normalization process in categories. Given the graph representation of a category, the normalization process involves iteratively removing redundant morphisms and objects from the graph. Specifically, Figure \ref{fig:reduced}(b) shows the first reduced representation, which eliminates morphisms that can be composed from other morphisms, similar to removing transitive functional dependencies in third normal form. Figure \ref{fig:reduced}(c) presents the second reduced representation, which removes the pullback objects, analogous to decomposing a table that can be computed through a lossless binary join in the fourth normal form. We connect the relational normal form  with the reduced representation of categories as follows:



\begin{tcolorbox}[colback=white,colframe=black,width=\linewidth]  
%Informally speaking, through the lens of category theory, the essence of Third Normal Form (3NF) and Boyce-Codd Normal Form (BCNF) is to eliminate composed morphisms in the graph representation of a category and the essence of Fourth Normal Form (4NF) is to eliminate derived limit objects.

Informally speaking, through the lens of category theory, the essence of normalization for Third Normal Form (3NF) is to eliminate composed morphisms in the graph representation of a category, while the essence of  normalization for Fourth Normal Form (4NF) is to decompose pullback objects.

\end{tcolorbox}

In the following sections, we will establish the corresponding formal results for the above correlations. Specifically, Section 4 develops algorithms to compute the closure of Functional Dependencies (FDs) and Multivalued Dependencies (MVDs) in categories. Section 5 defines the two levels of reduced representations of categories. Section 6 demonstrates how to map a category into relational data. In Section 7, we present the main results, connecting the first and second reduced representations with relational normal form and XML normal form. In summary, the purpose of these sections is to establish a novel, unified normal form theory for databases.