\section{Conclusion}

Beyond the technical results, this paper can be regarded
also as arguing that a straightforward application of category theory to databases may not yield optimal results. The primary challenge lies in effectively integrating the language of category theory with existing database theories. For example, the normal form theory proposed in this paper leverages the inference rules of functional and multivalued dependencies in database theory. These rules extend well beyond the composition morphisms employed in a mathematical category.  By navigating the intersection of category theory and database theory, the purpose of our research is to merge the two domains, leveraging the strengths of each to establish a coherent framework for multi-model data management.



%The presented framework opens up numerous opportunities for future research:

  

 %Based on this unified framework, a great deal of further research remains to be done in the future. For example, how to design the multi-model query language based on the unified framework? Relational algebra and calculus are designed for relational data. Since a category can be used to unify different structures of data, it is natural to design categorical algebra and calculus to implement multi-model queries. Another significant challenge is how to design a unified query optimization plan for different models of data.  As each data model possesses distinct characteristics, designing a global optimization strategy that effectively handles multi-model queries becomes crucial.  
 
 
 
 
 
 %Further, this paper uses a thin category to model databases by following the Universal relational assumption (URA). However, the URA may not hold in many practical scenarios. How to model multi-model data with a non-thin category is also an interesting future direction. 

%Finally, it is worthy to mention that category theory is an abstract mathematical theory. The original purpose for Samuel Eilenberg and Saunders Mac Lane is not to develop a theory to apply it on compute science or databases. Therefore, it is impossible for a straightforward application of category theory on database will work very well. Here, the key challenge is how to combine the language of category theory with the existing database theories seamlessly. For example,  the inference rules of functional and multivalued dependency are important for database normal form theory, which is far beyond the composed rule in a category. Therefore, this paper heavily relies on those inference rules to compute the closure of dependencies to guarantee the output data can satisfy different normal forms. 

%Lastly, it is important to note that category theory originated as an abstract mathematical theory. The inventors of category theory initially developed it without the explicit intention of applying it to database domain. Thus a straightforward application of category theory to databases may not yield optimal results. The primary challenge lies in effectively integrating the language of category theory with existing database theories. For instance, the inference rules governing functional and multivalued dependencies play a crucial role in database normal form theory. These rules extend well beyond the composition morphisms employed in a mathematical category. Therefore, this paper utilizes these existing database inference rules to compute the closure of categories, ensuring that the resulting data satisfies various normal forms. By navigating the intersection of category theory and database theory, the purpose of our research is to merge the two domains, leveraging the strengths of each to establish a coherent framework for multi-model data management.




%While \texttt{Samuel Eilenberg} and \texttt{Saunders Mac Lane} developed category theory in the mid-20th century, their primary objective was to establish a cohesive language that could apply to two mathematics branches: \textit{algebra} and \textit{topology}. In this paper, our purpose is to construct a unified framework that encompasses three distinct models of data: relation, XML, and graph data. In both contexts, category theory assumes a fundamental role by providing a unified language and framework for describing various theories and structures. The discussion of future work can be found in Appendix \ref{sec:futurework}. 


%On the other hand, it is important to note that category theory originated as an abstract mathematical theory. The inventors of category theory initially developed it without the explicit intention of applying it to different domains. Thus a straightforward application of category theory to databases may not yield optimal results. The primary challenge lies in effectively integrating the language of category theory with existing database theories. For instance, the inference rules governing functional and multivalued dependencies play a crucial role in database normal form theory. These rules extend well beyond the composition morphisms employed in a general category. Therefore, this paper heavily relies on these inference rules to compute the closure of dependencies, ensuring that the resulting data satisfies various normal forms. By navigating the intersection of category theory and database theory, the purpose of our research is to harmoniously merge the two domains, leveraging the strengths of each to establish a coherent framework for multi-model data management.
