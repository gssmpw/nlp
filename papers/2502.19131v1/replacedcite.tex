\section{Related work}
\label{sec:relatedwork}








%\textbf{Part 1: General description on category theory for databases. }


The application of category theory to databases has a rich history, with its earliest instances dating back to the 1990s. Notable papers from this period include those by Tuijn and Gyssens (1992,1996) ____, Baclawski et al. (1994) ____, Libkin (1995)____,  and Hofstede et al. (1996) ____, etc. While the majority of previous papers focus on the application of category theory to relational databases, there are also works that apply category theory to object databases and graph databases. The existing literature in this field can be broadly categorized into three main areas: \textbf{category theory for conceptual modeling,  for query languages, and for data transformation}. To provide an overview of the related works, Table \ref{tab:relatedwork} presents a taxonomy of these contributions.

%As for previous works on using category theory to define a conceptual model.  Paper ____ proposes a category theory approach for conceptual modeling, by choosing appropriate instance categories and adding features such as missing values, multi-valued relations, and uncertainty into conceptual data models. Paper ____  defines a respective schema category and operations using standard categorical approaches (such as functors). Paper ____ defines Well-known conceptual data modelling concepts, such as relationship types, generalization, specialization, collection types, and constraint types.  Paper ____ shows how commutative diagrams have been used to develop new methodologies in ER modeling, constraint specification, and process modeling.


\begin{table*}
\caption{Applying category theory on databases}
\label{tab:relatedwork}
\begin{tabular}{ cccc } 
  \toprule
Databases & Conceptual Models & Data Transformation & Query Language  \\ [0.5ex] 
  \midrule
 \multirow{2}{*}{Relational DB} & ____ & ____ & ____ \\
  & ____ & ____ & ____ \\
  
 Object DB & ____  & None & ____  \\ 
 
 Multi-model DB & ____  & ____ &   None  \\ 
 \bottomrule 
\end{tabular}
\end{table*}


In order to apply category theory to conceptual models, previous works define concepts in the (extended) ER model using category theory to provide a general and formal framework. In particular, Lippe and Hofstede (1996) ____ define total role constraints as epimorphisms, unique constraints as monomorphisms, and generalization as colimits. Johnson and Dampney (1993) ____ demonstrate the application of commutative diagrams in developing novel methodologies for constraint specification and process modeling. More recent works by Spivak  (2014) ____  consider database schema as a notion of ``onto-logical logs" (olog), which applies the basic concepts of category theory for knowledge representation. 

%``\texttt{Last\_name}"

%connect Grothendieck construction with database schema transformation. 


%resent a category theory-based approach to conceptual modeling in their paper . They propose the selection of suitable instance categories and introduce features such as missing values, multi-valued relations, and uncertainty to enhance conceptual data models.  Hofstede et al. (1996) ____ focus on defining well-known concepts in conceptual data modeling using category theory. These include relationship types, generalization, specialization, collection types, and constraint types. Additionally,


%As for previous work on category theory for data mitigation: Paper ____  showed that morphisms of database schemas induce three "data migration functors", which translate instances from one schema to the other in canonical ways. For example, they Use push-forwards to construct joins of tables. In contrast, this paper proposes to use limit objects to model join operations. Paper ____ studied the data transformation capabilities associated with schemas that are presented by directed multi-graphs and path equations.  They also present an algebraic query language FQL based on these functors.




In the field of category theory applied to query languages, Moggi (1991) ____ proposes to use monads to organize the semantics of programming constructs. Wadler ____ (1990) shows that monads are also useful in organizing syntax, in particular, they explain the
"list-comprehension" syntax of functional programming. Libkin and Wong ____ exploit theoretical foundations for querying databases based on bags, showcasing how database operations on collections align with the categorical notion of a monad. Furthermore, Rosebrugh (1991) ____ presents a categorical terminology for describing relational databases, with a particular focus on studying database dynamics by considering updates as database objects within a suitable category indexed by a topos. Libkin (1995) ____ employs universality properties, a central concept in category theory, to propose syntax for query languages that incorporate approximations.  Gibbons et al. (2018) ____ delve into the role of adjunctions, which are categorical generalizations of Galois connections, in the processing of collections. These papers collectively contribute to the field by leveraging category theory to  deepen our understanding of database query languages.



Regarding previous work on the application of category theory to data transformation,  Spivak (2010) ____ demonstrates how morphisms of database schemata give rise to three distinct data migration functors, which serve as powerful tools for translating instances between different schemata in canonical ways. He employs push-forwards to construct joins of tables. In contrast, our paper propose the utilization of limit objects as a means to model join operations, offering an alternative perspective. Furthermore, Spivak and Wisnesky (2015) ____  also introduce an algebraic query language, FQL, which is built upon the aforementioned data migration functors.


%Paper ____  elaborated how adjunctions, the categorical generalisation of Galois connections, underlie collection processing, including the convenient notation based on comprehensions for expressing database queries over collections. Paper ____ exploits theoretical foundations for querying databases based on bags. Their operations on collections correspond to the categorical notion of a monad. Paper ____ uses Universality properties, a central categorical concept, to suggest syntax for query languages with approximations.  Paper ____ shows the description of relational databases in categorical terminology given here has as intended application the study of database dynamics by viewing updates as database objects in a suitable category indexed by a topos.
 

In the pursuit of applying category theory to object and graph databases, a series of papers have  introduced novel concepts. In particular, Tuijn and Gyssens (1996) ____ introduce typed graphs, where both scheme and
data are defined entirely in terms of categorical constructs; pattern matching of graphs
is realized by morphisms in a suitable graph category.  Ohori and Tajima  (1994) ____ show an operation for lifting record operations to objects and classes, which has a similarity with
certain operations associated with monads in category theory. Expanding on the utilization of category theory in database modeling, Tannen, Buneman, and Wong (1992) ____ show the ideas of category theory can be profitably used to organize semantics and syntax in nested relational and complex-object algebras: a cartesian category with a strong monad on it. 


%To apply category theory to  object and graph databases, paper ____ introduces typed graphs. Both scheme and data will be defined entirely in terms of categorical constructs; pattern matching of graphwill be realized by morphisms in a suitable graph category.   Paper ____ encapsulate approaches based on partitioning in a categorical framework in order to obtain a very general context to study views and decomposition that is also applicable to complex-object models. They use the categorical notion of presheaf. Paper ____ shows an operation for lifting record operations to objects and classes, which has an intriguing similarity withcertain operations associated with Monads in category theory. Paper ____  shows the ideas of category theory can be profitably used to organize semantics and syntax in nested relational and complex-object algebras: a cartesian category with a strong monad on it.  


In the context of multi-model databases, previous research has explored the application of category theory to  represent multi-model data. In particular, Thiry et al. (2018) ____ propose a categorical approach for modeling relational, document, and graph-oriented models. They demonstrate how to combine category theory with a functional programming language to offer techniques in the context of Big Data. Koupil and Holubova (2022) ____ introduce a mapping between multi-model data and the categorical representation, enabling mutual transformations between different data models. This mapping provides a tool for seamless integration and interoperability of multi-model databases. Furthermore, Liu et al. (2018) ____ envision the structural aspects of multi-model databases and advocate for category theory as a promising new theoretical foundation. Taking a practical perspective, Uotila et al. (2021) ____ present a demo system that utilizes category theory for multi-model query processing, employing the Haskell programming language. This demonstration shows the practical implementation and effectiveness of category theory in facilitating multi-model query operations.


%Paper ____ proposes a categorical approach for relational, document and graph-oriented models. They describe on how Category Theory combined with a functional programming language can be interesting in a Big Data context.  Paper ____ introduce a mapping between multi-model data and the categorical representation for mutual transformations between them.  Paper ____ envisions the structure of multi-model databases and promotes the category theory as a new theoretical foundation. Paper ____ shows a demo system with category theory for multi-model query processing with Haskell language.  


%While previous pioneering works leverage category theory on different aspects of databases, this paper makes a contribution to defining a categorical framework for multi-model data management with limits, colimits, and commutative diagrams, and reveals the connection between the reduced representation of a category and  multi-model normal form theory. 




%The history of the normal form theory in database systems dates back to the early days of the development of the relational model. It emerged as a fundamental concept to ensure data integrity and optimize database design. Over the years, researchers and database practitioners extended the normal form theory from the first normal form, the second normal form, the third normal form, Boyce-Codd normal form (BCNF), fourth normal form (4NF)  ____,  project-join normal form (PJ/NF) ____  and domain-key normal form (DK/NF) ____ which were introduced to tackle specific types of dependencies and further optimize database design. Additionally, the concept of normalization has been applied to other data models beyond the relational model. For example, in the context of XML ____ and object data models ____, researchers developed techniques such as path normalization and object normalization to ensure data consistency and efficiency. 


%Furthermore, the introduction of multi-model databases and the integration of various data models have raised new challenges and prompted the exploration of normalization techniques that can accommodate the complexities of diverse data structures. This paper addresses these challenges and proposed a framework to define a normal form representation of a category for different types of data. 



While previous pioneering works have explored the application of category theory to various aspects of databases, this paper makes a unique contribution by identifying a connection between the reduced representation of a category and database normal form theory. By establishing this connection, the paper sheds light on the principles for defining a unified theoretical framework for multi-model normal forms. 


    Furthermore, it is important to distinguish between the category data model and the object-oriented (OO) data model (e.g. ____). Although both models can be viewed as directed graphs, their semantic definitions differ significantly. The OO model defines objects and methods, while the category model defines objects and morphisms (functions). Unlike methods, morphisms have more rigorous mathematical definitions. Category theory offers a comprehensive set of precise mathematical terminologies and theories, which can lead to valuable insights for databases.


%This paper discussed object normal form: .

%\textbf{Part 4: Summarizing}

%While the existing works are theoretically interesting to show the broad application of the language of theory in different fields, from the perspective of the database community, category theory does not play an essential role in practical database theory and algorithms. The possible explanation is that it lacks convincing examples to show that category theory can be used to solve real database problems. If category theory can only provide another terminology or view to the known database theory and algorithm, then its impact on databases becomes limited. The purpose of this paper is to provide a unified framework for multi-model data which contributes to one important problem in databases.