

\section{Closure of FDs and MVDs in Categories}

%In the subsequent sections of this paper, our objective is to construct a framework of the normal form theory for multi-model data.

Given the reliance of relational normal form theory on the computation of closures for FDs and MVDs, this section is dedicated to the  presentation of algorithms designed to efficiently compute closures within categories. 


\subsection{Closure with FDs}
\label{sec:ClosureFD}

In a relational database, given a set $F$ of FDs, numerous other functional dependencies can be inferred or deduced from the FDs in $F$. In the context of category, if we treat arrows as functional dependencies, then existing arrows can imply new arrows. For example, if $X \to Y$, $Y \to Z$, then $X \to Z$ is an implied (composed) arrow. This is exactly the transitivity rule in Amstrong's axioms \cite{armstrong1974dependency}.  Note that there are three inference rules of functional dependencies in Amstrong's axioms. To compute a closure of a category,  we show how all three Amstrong's axioms can be applied to the category as follows:

\noindent FD 1: If $Y \subseteq X$, then $X \to Y$; 
In this case, $Y$ is an object which is a projection object with a relationship object $X$. We add
 a projection arrow between $X$ and $Y$ in the category.

\noindent FD 2: If $f: X \to Y$, then $g: XZ \to YZ$;  Specifically, given an element $(x,z) \in XZ$, let $y = f(x)$, we define that $(y,z) \in YZ$ and $(y,z)$ is the image of $(x,z)$ under the function $g$. we find an alternative interpretation within the framework of \textbf{monoidal category} for this rule.  A detailed discussion can be found in Appendix \ref{sec:monoidal}. 

\noindent FD 3: If $X \to Y$ and $Y \to Z$, then $X \to Z$. This is the composition rule in categories.

%It is important to note that these three rules are real axioms here. That is, they are correct by definition. We define the existence of objects and arrows with these rules.  


%Algorithm \ref{alg:closureFD} describes the main steps to compute the closure of a category with FDs. In Line  1, the graph $G$ of this category is transformed into a set of functional dependencies. For every relationship object $X$ in $G$, two new functional dependencies are included: $X \to A_1, \ldots, A_n$ and $A_1, \ldots, A_n \to X$, where $X$ is associated with  $A_1, \ldots, A_n$. Additionally, other arrows in $G$ are converted into corresponding functional dependencies. Thus, apart from the functional dependencies for relationship objects, all other arrows in $G$ have a single attribute on the LHS. Subsequently,  Line 4 computes the closure $X^+$ for each LHS $X$ by using Amstong's axioms. Finally,  the associated arrows and objects are added based on $X^+$. 

\begin{definition} Given a graph representation $G$ of a category and a set of functional dependencies $F$, any inferred functional dependency $X \to Y$ is considered \textbf{relevant} to $G$ and $F$ if $X$ is either an object in $G$ or appears on the left-hand side (LHS) of a functional dependency in $F$, and $Y$ is an object in $G$.
\end{definition} 

This criterion of relevant functional dependencies ensures that the closure consists solely of the pertinent functional dependencies with respect to $G$.


\begin{definition} 
Given a graph representation $G$ of a category and a set of functional dependencies $F$, the relevant closure representation of $F$ and $G$ denoted as $(G,F)^+$ is a graph where all arrows represent the set of all relevant functional dependencies that can be inferred from $F$ and $G$.

\end{definition}

A schema category $\mathcal{C}$ can be represented as a directed graph, which consists of vertices and arrows, where the vertices correspond to the objects and the arrows indicate the function dependency between these objects. Given a graph representation $G$ of a category and a set of functional dependencies $F$, Algorithm \ref{alg:closureFD} outlines the key steps to compute the closure representation $(G,F)^+$, where each relevant functional dependency is represented as an arrow.







%The analysis of time complexities of algorithms can be found in Appendix \ref{sec:complexity}.

In Line 1 of Algorithm \ref{alg:closureFD}, the graph $G$ is transformed into a set of functional dependencies. Any arrow from $X$ to $Y$ in $G$ is converted to the corresponding functional dependency $X \to Y$. In addition, for each relationship object $X$ in $G$, create two new functional dependencies: $X \to A_1, \ldots, A_n$ and $A_1, \ldots, A_n \to X$, where $A_1, \ldots, A_n$ are projection objects of $X$.   As a result, except for the functional dependencies associated with relationship objects, all other arrows have a single attribute on LHS.  In Line 2, the set $D$ contains all functional dependencies from the category and $F$. Subsequently, for each LHS $X$ of $D$, the closure attributes $X^+$ are computed using Amstong's axioms and the corresponding inferred function dependencies are added in $G$ as arrows (Lines 3-9).

%in Lines 3-4, the closure $X^+$ for each LHS $X$ is computed using Amstong's axioms.  Finally, based on $X^+$, the associated arrows and objects are added, completing the closure graph $(G,F)^+$, where each arrow denotes a valid function dependency.

%This step ensures that all FDs implied by the given set of FDs are identified.


%This process guarantees that all necessary arrows and objects are included to preserve the integrity and logical structure of the category. By following Algorithm \ref{alg:closureFD}, we can systematically compute the closure of a category, ensuring the preservation of functional dependencies and the accurate representation of its underlying relationships.

%Note that in the above algorithm, to compute a closure  $G^+$, not all FDs that can be inferred are present as arrows in $G^+$. Only those FDs $X \to Y$, where $X$ exists in $G$ or $F$ and $Y$ exists in $G$ should be inserted in $G^+$.

 

%preventing the inclusion of extraneous or unrelated information. 



\begin{algorithm}
\caption{Computing the Closure of Categories with FD}
\label{alg:closureFD}
\KwIn{A graph representation $G$ and  a set of functional dependencies $F$ } 
\KwOut{The relevant closure representation  $(G,F)^+$} 
\DontPrintSemicolon

 Convert $G$ to a set of functional dependencies denoted by $FD(G)$;
 
 $D = FD(G) \cup F$;
 
\ForEach{  $X \in LHS(D)$} 
{
    \If{$X \notin G$}
        {
        add $X$ in G; 
        }
    
    Compute the closure attributes of $X$: giving $X^+$;

   \ForEach{  $Y \in X^+$ and $Y \in G$} 
    {
     \If{ there is no arrow from $X$ to $Y$ in $G$}
        {
        add an arrow from $X$ to $Y$ in $G$;
        }
    }
    
        
}

\end{algorithm}


\begin{figure}\centering\includegraphics[width=0.7\textwidth]{figures/1RRExample2.png}\caption{This example illustrates the compuation of closure and  1RR.}\label{fig:1RRExample}\end{figure}

\begin{example} Figure \ref{fig:1RRExample}(a-b) shows an example to illustrate  Algorithm \ref{alg:closureFD}. Figure \ref{fig:1RRExample}(a) is the input of a category and a FD: $B \to C$.  In Line 1, $FD(G)$ includes four FDs:  $D$$\to$$E$, $D$$\to$$A$, $A$$\to$$B$ and $A$$\to$$C$.  In Line 2, $D = FD(G) \cup \{B \to C\}$. In Line 9, three new arrows are inserted, i.e. $D \to C$, $B \to C$ and $D \to B$ shown in Figure \ref{fig:1RRExample}(b). Note that Figure \ref{fig:1RRExample}(c) will be explained later in Section 
\ref{subsec:1RR}. \end{example}



%The time complexity of Algorithm \ref{alg:closureFD} is dominated by the cost to compute the closure of the functional dependencies (i.e. Line 4). Let $m$ and $n$ be the number of objects and arrows in $G$, and let $d$ be the number of FDs. The total time for computing the FD closure is $O((d+n) \cdot m)$ based on the implementations of the previous work (e.g. \cite{10.1145/320493.320489}). Using this procedure, one can implement Algorithm \ref{alg:closureFD} with the time $O((d+n) \cdot m)$.

\subsection{Closure with FDs and MVDs}

%Recall that the definition of MVD $X \to\to Y$ involves "a universal set U", as defined in Section 2.4. The reason is that the validity of  $X \to\to Y$ depends on all other attributes associated with this relation. In the setting of most database papers,  $U$ is fixed. However, in a category of this paper,  $U$ could be various relationship objects which contain $X$ and $Y$. Thus the specification of $U$ is an integral part of an MVD. We  use the notation $X \to\to_U Y$ to stress the fact that the MVD involves the set $U$.


In database theory, an MVD $X \to\to Y$ is defined in terms of  ``\textit{a universal set U}''. The reason is that the validity of  $X \to\to Y$ depends on all other attributes associated with this relation $U$.  Unlike previous relational database settings that maintain a fixed $U$, the category setting here allows for different relationship objects having both $X$ and $Y$. As a result, the specification of $U$ becomes an indispensable component of an MVD. To underscore the importance of including the set $U$ in an MVD, we adopt the notation $X \to\to_U Y$ throughout the rest of this paper. This notation serves as a reminder that an MVD is intrinsically tied to the object $U$ within the context of this category.


%Section \ref{sec:mvd} demonstrates the connection between multivalued dependencies and pullback in categories. Recall that the definition of MVD $X \to\to Y$ involves "\textit{a universal set U}". The reason behind incorporating this universal set $U$ lies in the fact that the validity of $X \to\to Y$ is contingent upon all other attributes associated with this relation. While most previous works maintain a fixed $U$, the category presented in this paper allows for various relationship objects that contain both $X$ and $Y$. Consequently, the specification of $U$ becomes an indispensable component of an MVD. To emphasize the inclusion of the set $U$ in an MVD, throughout the rest of this paper, we adopt the notation $X \to\to_U Y$. This notation serves as a reminder that the MVD is closely linked to the set $U$ within the context of this category. 

%While a functional dependency is presented as an arrow in a category, a multivalued dependency is manifested with an \textit{MVD object} in a category, which serves as the universal set $U$ of this dependency, as defined below:


 While a functional dependency is represented by an arrow, a multivalued dependency is manifested through an \textit{MVD object} in the context of categories. This MVD object works as the universal set $U$ for the corresponding dependency, as defined below:

\begin{definition} (\textbf{MVD objects}) Given a category $\mathcal{C}$, a relationship object $O \in C$ is called an MVD object if there is a multivalued dependency $X \to\to_O Y$ over $C$, such that  $X \cup Y \subset \pi(O) $,  where $\pi(O)$ denote the set of projection objects of $O$. \label{def:MVD}
\end{definition}

%Given a set of functional dependencies $F$ and a set of multivalued dependencies $M$, the inference rules to compute their closure can be found in literature, see e.g. \cite{10.5555/551350,10.1145/320613.320614,10.1145/509404.509414}. We provide those inference rules in Appendix \ref{sec:monoidal}. 

Given a set of functional dependencies \( F \) and a set of multivalued dependencies \( M \), the inference rules to compute their closure are well-documented in the literature (see, e.g., \cite{10.5555/551350,10.1145/320613.320614,10.1145/509404.509414}). For the convenience of readers, we have included these inference rules in Appendix \ref{sec:MVDInference}.

%In order to compute the closure of FDs and MVDs, one needs to compute the \textit{dependent basis} of an attribute $X$, denoted by $DEP(X)$. An MVD $X \to \to_U Y $ is in the closure if and only if $Y$ is a union of elements of $DEP(X)$. The algorithm to compute the dependent basis can be found in previous works, e.g.  \cite{10.1145/320613.320614}. Note that $X^+$  and $DEP(X)$ are indeed parallel concepts.  If we think of the collection of singleton sets, \{$(A)| A \in X^+$\}, then $X^+$ is just $DEP(X)$ that is functionally dependent on $X$ by FDs.
  


%\begin{definition} Given a graph representation $G$ of a category and a set of functional dependencies $F$ and  a set of multivalued dependencies $M$ whose RHS are objects in $G$, an MVD $X \to\to Y$ is considered \textbf{relevant} if $X$ is either an object in $G$ or appears on the LHS of a FD in $F$ or a MVD in $M$, and $Y$ is an object in $G$.\end{definition} 

\begin{definition} 
Given a graph representation $G$ of a category and a set of functional dependencies $F$ and  a set of multivalued dependencies $M$, the relevant closure representation of  $G, F, M$ denoted as $(G,F,M)^+$ is a graph where all arrows represent the relevant FDs that can be inferred from $F,G$ and $M$, and all MVD objects are identified based on the MVDs that can be inferred from $F,G$ and $M$.

\end{definition}

%Given a graph representation $G$ of a category $C$ with both a set of functional dependency $F$ and a set of multivalued dependency $M$, the computation of a closure of $G^+$ under $F$ and $M$ involves the iterative applying of inference rules as follows:

%In the context of a category $C$ represented by a graph $G$, where $(G,F,M)^+$ denotes the closure of $G$ under both a set of functional dependencies $F$ and a set of multivalued dependencies $M$, the process of computing the closure involves the iterative application of the inference rules as follows:



\begin{algorithm}
\caption{Computing the Closure of Categories with FD and MVD}
\label{alg:closureMVD}
\KwIn{A graph representation $G$, a set of functional dependencies $F$ and a set of multivalued dependencies $M$} 
\KwOut{The relevant closure representation of $(G,F,M)^+$} 
\DontPrintSemicolon

 Convert $G$ to a set of functional dependencies denoted by $FD(G)$;
 
$D$ = $FD(G) \cup F \cup M$;
 
\ForEach{  $X \in LHS(D)$} 
{
        \If{$X \notin G$}
        {
        add $X$ in $G$;
        }
        

        Compute the FD closure attributes of $X$ based on $D$: giving $X^+$;

        \ForEach{  $Y \in X^+$ and $Y \in G$} 
        {
            \If{$X \to Y $ is not an arrow in $G$}
            {
            add an arrow from $X$ to $Y$ in $G$;
            }
        }

       Identify all MVD objects in $G$ based on the inferred MVDs from $D$;
}

\end{algorithm}



%It seems that $X^+$  and $DEP(X)$ are different, as the former is a set of attributes, while the latter is a collection of sets of attributes. However, if we think of the collection of singleton sets, \{$(A)| A \in X^+$\}, then it is just the basis of the collection of sets that are functionally dependent on X by an FD. Thus,  $X^+$  and $DEP(X)$ are parallel concepts. 

%Let S be a collection of sets closed under union, intersection, and difference. The collection S contains a unique subcollection T of nonempty, pairwise disjoint sets such that every element of S is a union of some elements of T. The collection T is called the basis of S. We call the basis of this collection the dependent basis of X and denote it by DEP(X). By the definition, and an MVD $X \to \to Y $ is in G+ if and only if Y is a union of elements of DEP(X).  It seems that $X^+$  and $DEP(X)$ are different, as the former is a set of attributes, while the latter is a collection of sets of attributes. However, if we think of the collection of singleton sets, \{$(A)| A \in X^+$\}, then it is just the basis of the collection of sets that are functionally dependent on X by an FD. Thus,  $X^+$  and $DEP(X)$ are parallel concepts. To accomplish this, it is necessary to compute the dependent basis for each left-hand side (LHS) of multivalued dependencies (MVDs). The procedure for calculating the MVD dependency basis is provided in the appendix, drawing from the methodology presented in the paper \cite{10.1145/320613.320614}.





%\begin{definition} (Explicit pullback) Given a category C, a relationship object $O \in C$ is an explicit pullback if there are two objects $O_1, O_2 \in C$, such that  $O_1 = \{XY\}  $ and $O_2 = O -  \{XY\}$. \end{definition}

%In the above definition, $O$ is a universal object to witness the MVD $X \to\to Y$, so we require $\pi(O) - \{XY\} \neq \emptyset$.


%Algorithm \ref{alg:closureMVD} outlines the key steps involved in computing the closure of a category with FDs and MVDs. The purpose of this algorithm is not only to find all related inferred arrows $X \to Y$, but also to identify all MVD objects $O$ in $(G,F,M)^+$.

%When computing the closure $(G,F,M)^+$, not all inferred FDs and MVDs must be included in the returned graph representation. Instead, only those FDs and MVDs of the form $X \to Y$ are added to $(G,F,M)^+$, where the object $X$ exists in either $G$, the LHS of $F$, or $M$, and $Y$ exists in $G$.

   

%The main objective of this algorithm is not only to identify the inferred arrows, but also to pinpoint all MVD objects in $(G,F,M)^+$.


Algorithm \ref{alg:closureMVD} presents the essential steps for calculating the relevant closure representation with functional dependencies $F$ and multivalued dependencies $M$. The algorithm proceeds as follows: In Line 1, the graph $G$ is transformed into a set of FDs. Line 2 initializes a set $D$ to include all the FDs and MVDs. Lines 6 are responsible for computing $X^+$  for each $X \in LHS(D)$ by using $G, F$ and $M$. Subsequently, the new arrows are added, and the MVD objects are identified in Lines 9-10 based on the inferred FDs and MVDs. 

%It is important to emphasize that the closure computation does not involve inserting any new MVD objects, but rather focuses on identifying the relationship objects that satisfy the definition of MVD objects. 


%The sound and complete identification of MVD objects is crucial in obtaining a reduced representation for categories in the subsequent section.


\begin{figure}\centering\includegraphics[width=0.8\textwidth]{figures/2RRExample3.png}\caption{This example illustrates the closure and 2RR.}\label{fig:2RRExample}\end{figure}


\begin{example}  Figure \ref{fig:2RRExample} (a-b) show an example to illustrate Algorithm \ref{alg:closureMVD}. Fig \ref{fig:2RRExample}(a) is the input with one MVD $A \to\to_X B$. In Line 1, the graph is converted to five FDs. In Line 2, the set $D$ includes five FDs and one MVD.  In Line 9, the arrow $A \to C$ is added due to $A \to\to_X B$, $A \to\to_X CD$ and $B \to C$ (by the inference rule FD-MVD 2). In Line 10, $X$ is identified as an MVD object due to $A \to\to_X B$ and $A \to\to_X CD$, shown in Figure \ref{fig:2RRExample} (b). 
\label{exp:CloureMVDExample}
\end{example}

%$X$ is found as a derivable MVD object due to $A \to\to_X D$.



%The time complexity of the algorithm is essentially that of the computation of the closure of FDs and MVDs. Let $m$ be the number of objects in $G$, $n$ the number of arrows in $G$, and let $d_1$ and $d_2$ be the number of FDs and MVDs, respectively. The total time for computing the FD closure is $O((d_1+n) \cdot m)$ and MVD closure is $O((d_2+n) \cdot m^3)$ based on the implementations of papers  (\cite{10.1145/320493.320489, 10.1145/320613.320614}. Using these procedures, one can implement Algorithm \ref{alg:2RR} with the time $O(d_1+d_2+n) \cdot m^3)$.


%Algorithm \ref{alg:closureMVD} describes the main steps to compute the closure of a category with FDs and MVDs.  In Line  1, the graph $G$ is transformed into a set of functional dependencies.  Then all arrows, FDs and MVDs are included in $D$ in Line 2.  Lines 4 and 5 compute  $X^+$ and $DEP(X)$ for each $X$ respectively. Subsequently, the associated arrows are added using $X^+$ and the MVD objects are identified using $DEP(X)$.  Note that in this algorithm,  no new MVD objects are inserted, but found. The correct and complete  identification of MVD objects is necessary to find a reduced representation for categories in the next section.

%\begin{definition}A graph representation $G$ of a category  is said to cover another graph representation $G'$ if every arrow  in $G'$ is also in $G^+$; that is, if every arrow in G' can be inferred from G. \end{definition}

%In the above definition, every arrow in $G'$ is also in $G^+$, meaning that the two objects associated with this arrow should be also included in $G^+$.

%\begin{definition}Two graph representation $G$ and $G'$ are equivalent if $G$ covers $G'$ and $G'$ covers $G$. \end{definition}

%The equivalences of two graph representations of categories will be used in the next section to define  a reduced (compact) representation of categories that leads to normal form theory for categories. 