%\smallskip \smallskip

\newpage

\noindent \textbf{Organization of the Appendixes}  ~~ 

%Appendix \ref{sec:pushout} shows the connection between pushout and connected component of a graph and 

In Appendixes \ref{sec:XMLDTD} to \ref{sec:hybrid}, we show the algorithms used for mapping categories to XML DTD, graph, and hybrid schemata. In Appendix \ref{sec:monoidal}, we analyze the inference rules of FDs and MVDs.  Appendix \ref{sec:proofs} provides the proofs and explanations for the lemmas and theorems. Furthermore, Appendix \ref{sec:complexity} assesses the computing complexities associated with the various algorithms presented. 


%Finally, Appendix \ref{sec:relatedwork} conducts a review of related work in the field of category theory for databases and Appendix \ref{sec:futurework} offers insights into potential avenues for future research.

%\section{Mapping Categories to Relations}
%\label{sec:category2relation}

%Algorithm \ref{alg:map2relationschema} shows how to map schema category to relation schema. See one more example to illustrate this algorithm.


%\begin{example} Recall the 2RR graph in Figure \ref{fig:2RRExample}(c). If we run Algorithm \ref{alg:map2relationschema} to convert this category to relational schemes, the output comprises four relations:  $R_1(A,B)$, $R_2(A,D)$, $R_3(B,C)$ and $R_4(A,C)$. All those relations satisfy the fourth normal form. In contrast, if we convert the input graph in Figure \ref{fig:2RRExample}(a) into relations, they include one table $R(A,B,C,D)$, which fails to satisfy the fourth normal form due to $A \to\to B$ and $A \to\to CD$.  This example provides an intuitive demonstration of the benefits of the 2RR approach, which ensures that the resulting output schema satisfies the fourth normal form. \end{example}

\section{Mapping Categories to XML DTD}
\label{sec:XMLDTD}

%We shall use a somewhat simplified model of XML trees in order to keep the notation simple. We assume a countably infinite set of labels L, a countably infinite set of attributes A (we shall use the notation @$l_1$, @$l_2$, etc for attributes to distinguish them from labels), and a countably infinite set V of values of attributes. 



%XML is represented as a finite rooted directed tree, with each node in the tree associated with a tag. Within this tree structure, two special attributes are assigned to nodes: @ID and @IDREF. The @ID attribute serves as a unique identifier (key) for a particular node, while the @IDREF attribute is used to reference the ID of other nodes. Note that each node possesses only one @ID attribute, but it can have multiple @IDREF attributes to refer to different nodes within the XML structure.

%\begin{definition}(XML tree) An XML tree T is a tree with 5-tuple($V$, lab, elem, att, root) where\begin{itemize}\item $V$ is a finite set of node (tags);\item $lab$ map each node $v$ in $V$ to a type (tag);\item $elem$ map each node $v$ in a string or a set of other nodes  \item  att is a partial function $V \times Att \to str$. For each v, the set @a $\in Att $. There are two special attributes are assigned to nodes: @ID and @IDREF.   \item $root \in V$ is the root of T.\end{itemize}\end{definition}

 %Note that, in general XML trees, children of each node are ordered. However, this paper focuses on XML normal form theory, which does not use the ordering in the tree, we  disregard this ordering. 

%\begin{definition} A DTD is a 5-tuple($L$, A, P, R, r) where\begin{itemize}\item $L$ is a finite set of element types (tags);\item $A$ is a finite set of strings (attributes), starting with the symble @;\item P is a set of rules $a \rightarrow P_a$ for each $a \in L$, where  $P_a$ is a regular expression over $L$ - \{$r$\}\item  R assigns  each $e \in L $ a finite  subset of A (possibly empty; R(a) is the set of attributes of a)\item $r \in L$ is the root.\end{itemize}\end{definition}


%\begin{example} For example, Figure shows an XML tree and the corresponding XML DTD. \end{example}

XML is represented as a finite rooted directed tree, with each node in the tree associated with a tag. We use DTD to describe the schema of XML.

%Within this tree structure, two special attributes are assigned to nodes: @ID and @IDREF. The @ID attribute serves as a unique identifier (key) for a particular node, while the @IDREF attribute is used to reference the ID of other nodes. Note that each node possesses only one @ID attribute, but it can have multiple @IDREF attributes to refer to different nodes within the XML structure.

\begin{definition} A DTD is a 5-tuple($L, T, P, R, r$) where 
 \begin{itemize}[noitemsep,topsep=0pt]
  \item $L$ is a finite set of element types (tags);
  \item $T$ is a finite set of strings (attributes), starting with the symbols @. There is a special attribute @ID, which serves as a unique identifier (key) for a particular tag;
   %\item $T$ is a finite set of strings (attributes), starting with the symbols @. There are a  special attributes: @ID and @IDREF$_{A,B}$. The @ID attribute serves as a unique identifier (key) for a particular tag, while the @IDREF$_{A,B}$ attribute with tage $A$ is used to reference the ID attribute of other tag $B$;
  \item $P$ is a mapping from $L$ to element type definitions: Given any $a \in L$, $P_a = \#P$ or  $P_a$ is a regular expression over $L$ - \{$r$\}, where $\#P$ denotes $\#PCDATA$;
  \item $R$ assigns  each $e \in L $ a finite  subset of $T$ (possibly empty); 
   \item $r \in L$ is the root.
\end{itemize}
\end{definition}


%In the algorithm to map a schema to a category, each attribute object associated with an entity object O is constructed as the sub-element of O in XML DTD.

%If $O$ is an entity or relationship object, assign the ID attribute of the surrogate key.

%Algorithm \ref{alg:map2DTD}  outlines the mapping process from the reduced representation schema to a DTD schema.  First, $\epsilon$ is a root node, and process and add objects $O$  as children of $\epsilon$ (Lines 1-3). For an object $O$ with outgoing edges, create a new tag and append $O^+$ in the regular expression of $\epsilon$ (Lines 6-7). Subsequently, for each neighbor object $N$ of $O$, append N in the expression rule of $O$ (Lines 12-13). Further, if $N$ is a bidirectional attribute object of $O$, append all outgoing objects of $N$ in the expression rule of $O$ as well.  If $N$ is an entity neighbor of $O$, add the surrogate key of $N$ as an IDREF attribute to $O$ (Lines 17-18).

Algorithm \ref{alg:map2DTD} presents the mapping process from a categorical schema to a DTD schema. The algorithm proceeds as follows: Initially, $\epsilon$ is designated as the root node, and the algorithm processes and adds objects $O$ as children of $\epsilon$ (Lines 1-3).  A new tag $\lambda(O)$ is created for each object with outgoing edges, and $\lambda(O)^+$ is appended to the regular expression of $\epsilon$ (Lines 5-9), where $\lambda(O)$ denotes the label (name) of the object $O$. Subsequently, for each neighbor object $N$ of $O$, if $N$ has any outgoing edge, an attribute $@\lambda(N)$\_ID is added for $O$ (Lines 13-14), otherwise $\lambda(N)$ is appended to the regular expression rule of $\lambda(O)$ (Lines 16-17).  It is worth noting that the shape of the XML data that conforms to the output DTD is wide and shallow. This representation allows for a clear and concise XML structure, avoiding redundant storage.


%Further, if $N$ is an entity or relationship neighbor of $O$, the surrogate key of $N$ is added as an IDREF attribute to $O$ (Lines 20-22). This step establishes a linkage between $N$ and $O$ through the IDREF attribute.


%This mapping facilitates the representation and validation of XML data according to the structure and relationships defined in the reduced representation schema.


%The neighbor node $n$ of $O$ are output with the child node of v. The surrogate keys are defined as the ID attribute of the element.  The bi-directional node n is included in the subtrees of v, as n and v have a bijective function between them. 

%Figure \ref{fig:firstexample} shows the XML data based on the category. 

%$A=\{@ID,@IDREF\}$

\begin{example} Recall the 1RR graph in Figure \ref{fig:1RRExample}(c).   Algorithm \ref{alg:map2DTD} convert it to XML DTD as follows: $L=\{A,B,C,D,E\}$, $T$=\{@ID, @A\_ID, @B\_ID\}, $P=\{\epsilon \to D^+A^+B^+, D \to E,  B \to C$\}, $R(A)$=\{@ID, @B\_ID\}, $R(D)$=\{@ID, @A\_ID\}, $R(B)$=\{@ID\}, and $r = \epsilon$. $B$ has an outgoing edge and is appended into the regular expression following the root $\epsilon$. In contrast, $E$ and $C$ are attribute objects without outgoing edges, and thus are created as the sub-element of $D$ and $B$ respectively.  
\end{example}

\begin{example} Recall the 2RR graph in Figure \ref{fig:2RRExample}(c).
Algorithm \ref{alg:map2DTD} convert it to XML DTD as follows: $L=\{A,B,C$, $D,X_1,X_2\}$, $T=$\{@ID,@A\_ID, @B\_ID\}, $P=\{\epsilon \to A^+B^+X_1^+X_2^+$, $A \to  C$, $B \to C$, $X_2 \to D$\}, $R(X_1)$=\{@ID, @A\_ID, @B\_ID\}, $R(X_2)$=\{@ID,@A\_ID\}, $R(A)=R(B)=\{@ID\}$, and $r = \epsilon$.  
\end{example}

\begin{algorithm} \caption{Map  category schema to XML DTD}\label{alg:map2DTD}\KwIn{A graph representation $\mathcal{G(C)}$ of a schema of a category $\mathcal{C}$ }
    \KwOut{XML DTD $\mathcal{D}$=($L$, T, P, R, r)}
    \DontPrintSemicolon

Let $\epsilon$ be the root $r$ of $D$;

 $T= \{@ID\}$;

 Initialize $L$, $P$, and $R$ to be empty sets;

\ForEach{ unprocessed object $O$ in $V(\mathcal{G(C)})$ with outgoing edges}
{ 
     $L \leftarrow L \cup \{\lambda(O)\}$; \tcp*{$\lambda(O)$ is the label of $O$}

     $P_\epsilon \leftarrow P_\epsilon \cdot (\lambda(O))^+$;
    
     $R(\lambda(O))=\{@ID\}$;
    
    Add\_neighbours($O$);
      
    %Process\_entities\_relationships($O$)

     Mark $O$ as a processed node;
    
}


\SetKwFunction{func}{ $\mbox{Add\_neighbours}$}
\SetKwProg{Fn}{Function}{:}{}
\Fn{\func{$O$}}
{
    \ForEach{unprocessed object $N$ in $outNbr(O)$ }
    {    
     \If{$N$ has any outgoing edge}
     { 
        $T= T \cup \{@\lambda(N)$\_ID\};
        
        $R(O)=R(O) \cup \{@\lambda(N)$\_ID\};
     }
    \Else 
     {  
       $P_O \leftarrow P_O \cdot \lambda(N)$;

        $L \leftarrow L \cup \{\lambda(N)\}$;
      }   
    }
}   


\end{algorithm}


%\begin{algorithm} \caption{Map to XML tree data}\label{alg:map2tree2}\input{algorithms/map2tree2.tex}\end{algorithm}



 

%Readers might notice that the mapping algorithm above  generates the trees whose length are  always two, i.e. a shallow and wide tree. There are two reasons to explain this: (i) any tree can be reorganized as two levels tree with the schema shown in the figure (one object with all nodes, and the other objects with the edges), (ii) the normalization requires to reduce the redundancy. The tree with higher depth is more possible to incur the the redundancy. For example, see $root \to A$, $A \to B$, $B \to C$, R(A)=\{@SerialKey\_A\}, R(B)=\{@SerialKey\_B\} and R(C)=\{@SerialKey\_C\}.  If two different A has the same B, then the C nodes under the B has to be reduplicated. Therefore, it will cause the redundancy storage and violates the normalization. One way to avoid the redundancy is to use this schema: $root \to A,B,C$, R(A)= \{@SerialKey\_A, @SerialKey\_B\},   R(B)=\{@SerialKey\_B, @SerialKey\_C\}, R(C)=\{@SerialKey\_C\}.  


\begin{figure*}
\centering
\includegraphics[width=0.9\textwidth]{figures/Picture2.png}
\caption{The relation, XML and graph data converted from the category in Figure \ref{fig:firstexamplecategory}.} \label{fig:relationXML}
\end{figure*}



\section{Mapping Categories to Graphs}
\label{sec:PropertyGraph}

%For all objects and attributes in the category, we use the capital letter, such as A,B,O,..., and all vertexes and edges in the property graph, we use the small letter. such as v, w. (\cite{conf/icde/AlotaibiLQEO21})

%A property graph $PG (V, E)$  is an undirected multi-edge graph with vertex set $V$ and edge set $E$, where each vertex $v \in V$  and each edge $e \in E$ has data properties consisting of multiple attribute-value pairs \{$(a,u)$\}. $\lambda(v)$ and $\lambda(e)$ denote the labels of the node and edge.   

%A property graph, denoted as $PG(V,E)$, represents an undirected multi-edge graph. It consists of a vertex set $V$ and an edge set $E$. In this graph, each vertex $v \in V$ and each edge $e \in E$ possess data properties in the form of multiple attribute-value pairs \{$(a,u)$\}. The labels of the nodes and edges are denoted by $\lambda(v)$ and $\lambda(e)$ respectively.

%An edge connects two vertices: a source vertex and a target vertex. An edge type can be either directed or undirected. A directed edge has a clear semantic direction, from the source vertex to the target vertex. For simplicity of representation, all edges are undirected in this paper. But since an arrow in a category is directed, it is also feasible to convert a category into a directed property graph (and more precisely in some cases). 




%A property graph PG $(V, E)$  is a directed multi-graph with vertex set V and edge set E, where each node $v \in V$ and each edge $e \in  E$ has data properties consisting of multiple attribute-value pairs. The set of nodes of the graph G is denoted by V(G), and the set of edges of g is denoted by E(G). Note that an edge of E(G) is a set \{u, v\} $\subset$  V(G). For a node $v \in V(G)$, the label of v is denoted by $\lambda_G(v)$. For a node v of g, the set of neighbors of v is denoted by $nbr_G(v)$. Usually, the graph G is clear form the context, and then we may write just $\lambda(v)$, $\lambda(U)$ and $nbr(v)$ instead of $\lambda_G(v)$, $\lambda_G(u)$ and $nbr_G(v)$, respectively. Also, a node labeled with $\sigma$ is called a $\sigma$-node.


\begin{algorithm} \caption{Map to Property Graph Schema}\label{alg:map2graphschema}\KwIn{A graph representation $\mathcal{G(C)}$ of schema of a category $\mathcal{C}$ }
    \KwOut{A property graph  schema $\mathcal{PG}$= ($V, E, T, P$)}
    \DontPrintSemicolon

Initialize $V, E, T$, and $P$ to be empty sets;

\ForEach{ object $O$ in $V(\mathcal{G(C)})$ with outgoing edges}
{
     $V \leftarrow V \cup \{\lambda(O)\}$;

     \If{$O$ is an entity or attribute object  }
     {
        $P(O) \leftarrow P(O) \cup \{SK\}$;
     }
    
     Add\_neighbours($O$);
     
    %Process\_attributes($O$) 

    %Process\_entities\_relationships($O$)

    %Process\_relationships($O$)

    Mark $O$ as a processed object; 

}



\SetKwFunction{func}{ $\mbox{Add\_neighbours}$}
\SetKwProg{Fn}{Function}{:}{}
\Fn{\func{$O$}}
{
    \ForEach{ object $N$ in $outNbr(O)$ }
    {
        \If{$N$ is an attribute object and $N$ has no outgoing edges} 
        { 
            $P(O) \leftarrow P(O) \cup \{\lambda(N)\}$; 

            $T \leftarrow T \cup \{\lambda(N)\}$; 
        }
        \Else
        { 
            $E \leftarrow E \cup \{(O,N)\}$;

          %  $V \leftarrow V \cup \{N\}$
            
           % $P((O)) \leftarrow P((O)) \cup \{IDREF\}$
      
        }
         
        
        %\If{$N \in bin(O)$} 
        %{ 
        
        %    Add\_neighbours($N,S$)

        %    Mark $N$ as a processed object

        %}

        
        
    }
    
}   

%\SetKwFunction{func}{ %$\mbox{Process\_entities\_relationship}$}
%\SetKwProg{Fn}{Function}{:}{}
%\Fn{\func{$O$}}
%{
%    \ForEach{entity or relationship object $N$ in $outNbr(O)$ }
%    {
        
        %Create an edge $e$ between $N$ and $O$ and add $e$ into $E$ in schema

        %$E \leftarrow E \cup \{(N,O)\}$

        %$P((N)) \leftarrow P((N)) \cup \{ID\}$

        %$P((O)) \leftarrow P((O)) \cup \{IDREF\}$
     
        
    %}
%}

%\SetKwFunction{func}{ $\mbox{Process\_relationships}$}
%\SetKwProg{Fn}{Function}{:}{}
%\Fn{\func{$O$}}
%{
%    \ForEach{relationship object $R$ in $outNbr(O)$ }
%    {
            
%        \If{R is a binary relationship object} 
%        {
%            $P(R) \leftarrow P(R) \cup \{N \}$   
%        }
%        \Else 
%        {
            %create an edge $e$ between $R$ and $O$ into $E$ in the schema

%            $E \leftarrow E \cup \{(R,O)\}$
                
%        }
            
        
%    }
%}\end{algorithm}


%\begin{algorithm} \caption{Map to Property Graph}\label{alg:map2graph2}\input{algorithms/map2graph2}\end{algorithm}

%\item L is a finite set of labels (names) of $V \cup E$; 

\begin{definition} 
A (undirected) property graph schema is a 4-tuple ($V, E, T, P$) where 

\begin{itemize}[noitemsep,topsep=0pt]
  \item $V$ is a finite set of labels of vertices;
  \item $E \subseteq V \times V$, a finite set of labels of edges; 
  \item $T$ is a finite set of attributes;
  \item $P$ is a function, which maps a label in $V \cup E$ into a subset of $T$.
\end{itemize}
  
\end{definition} 


%\begin{figure}\centering\includegraphics[width=0.4\textwidth]{figures/outputgraph.jpg}\caption{The graph data converted from the unified category in Figure \ref{fig:firstexamplecategory}.} \label{fig:outputgraph}\end{figure}

%The algorithm follows the following steps, processing entity, relationship, and attribute objects in a specific order. If this entity object connects with another entity, the surrogate key of the connected entity is added as an IDREF attribute. 

%The associated attribute objects are included as attribute-value pairs for the corresponding node.
 
%Algorithm \ref{alg:map2graphschema} provides a mapping procedure for converting a category into a property graph schema.  Initially, convert each entity object in the category into a node in the property graph (Lines 2-3).  Then convert binary relationship objects into edges connecting two objects in the property graph (Lines 7-8). For the ternary or multi-way relationship objects, a node is created in the graph to represent the relationship (Lines 9-10). Assuming $N$ is a neighbor from  $O$. If $N$ is an attribute object, add $N$ as an attribute for $O$ (Lines 16-17). If $N$ is an entity (or relationship) object create a new edge in the graph for the object $N$ (Lines 18-19).   If the edge between $N$ and $O$ is bidirectional, add all outgoing neighbors of $N$ to an attribute value or an edge for $O$ (Line 20-21). 


Algorithm \ref{alg:map2graphschema} provides a mapping procedure designed to convert a categorical schema into a property graph schema.  The algorithm operates as follows: First, each object with outgoing edges in the category is transformed into a node within the property graph (Lines 2-7). In the case where the object represents an entity or a relationship, its surrogate key is added as a property, denoted by $SK$. Next, after any object $O$ is processed,  if its neighbor object $N$  is an attribute object without outgoing edges, it is added as an attribute for $O$ in the graph (Lines 10-12), otherwise, a new edge is created to connect $N$ and $O$ in the graph (Lines 14).






\begin{example}
Given the graph representation shown in Figure \ref{fig:1RRExample}(c), we use the above algorithm to convert it to a graph schema as follows: nodes $V= \{A,B,D\}$, edges $E= \{(A,B),(D,A)\}$, properties $T=\{SK, C, E\}$, $P(B)=\{SK, C\}$, $P(D)=\{SK, E\}$, and $P(A)=\{SK\}$.
\end{example}


\begin{example}
 Consider the graph representation of Figure \ref{fig:2RRExample}(c), we apply Algorithm \ref{alg:map2graphschema} again: nodes $V= \{A,B,X_1,X_2\}$, edges $E= \{(X_1,A),(X_1,B),(X_2,A)\}$, attributes $T=\{SK, C,D\}$, $P(A)=\{SK,C\}$,  $P(B)=\{SK,C\}$, $P(X_2)=\{SK,D\}$.
\end{example}

In addition, Figure \ref{fig:relationXML} shows another example for the output relations, XML and graph data based on the category from Figure \ref{fig:firstexamplecategory}.

%\noindent \textbf{Computation complexity}

%Figure \ref{fig:outputgraph} shows the output graph data based on the category from Figure \ref{fig:firstexamplecategory}.



%Here, $N$ represents the total number of elements across all objects, $d$ is the maximum in-degree observed among the objects, and $M$ corresponds to the maximum number of functions between any two objects. Importantly, the time complexity is bounded by the product of these factors due to the fact that each object is processed a maximum of $d$ times.


\section{Mapping to Hybrid Schema}
\label{sec:hybrid}

\begin{figure}
\centering
\includegraphics[width=0.7\textwidth]{figures/hybrid2.jpg}
\caption{An example to illustrate the output of two different models of data} \label{fig:hybrid}
\end{figure}

 %In the context of a multi-model database, it is often necessary to decompose a category into multiple components,  each of which may possess its own unique structure and model, serving specific aspects of the data. 
 
 
 In the context of a multi-model database, it is often necessary to decompose a categorical schema into multiple components,  each of which may possess its own unique structure and model, serving specific aspects of the data. To illustrate this concept, Figure \ref{fig:hybrid} demonstrates an example wherein customer data is divided into two distinct parts. The first part pertains to the information of customers, which is represented using the relational data model. The second part captures the friendship connections between customers and is best represented as a property graph model.
 
 
 Given a reduced representation $G$ of a category, we aim to decompose this representation $G$ into a set of subgraphs denoted as ${G_1, G_2, ..., G_n}$. The objective is to ensure that each node and edge in $G$ appear in at least one subgraph $G_i$, thereby preserving all information. Furthermore, when an edge $e$ belongs to a category $G_i$, the two nodes connected by $e$ must also be contained within $G_i$. This lossless decomposition approach guarantees that all nodes and edges can be retained after performing the decomposition. 
 
 
 %Consequently, each isolated component $G_i$ becomes amenable to individualized processing through the utilization of distinct algorithms, as shown in the preceding subsections.
 
 %Subsequently, each component $G_i$ can be processed individually using different algorithms, as explained in the previous subsections.
 




