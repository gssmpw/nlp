%%
%% This is file `sample-authordraft.tex',
%% generated with the docstrip utility.
%%
%% The original source files were:
%%
%% samples.dtx  (with options: `authordraft')
%% 
%% IMPORTANT NOTICE:
%% 
%% For the copyright see the source file.
%% 
%% Any modified versions of this file must be renamed
%% with new filenames distinct from sample-authordraft.tex.
%% 
%% For distribution of the original source see the terms
%% for copying and modification in the file samples.dtx.
%% 
%% This generated file may be distributed as long as the
%% original source files, as listed above, are part of the
%% same distribution. (The sources need not necessarily be
%% in the same archive or directory.)
%%
%% Commands for TeXCount
%TC:macro \cite [option:text,text]
%TC:macro \citep [option:text,text]
%TC:macro \citet [option:text,text]
%TC:envir table 0 1
%TC:envir table* 0 1
%TC:envir tabular [ignore] word
%TC:envir displaymath 0 word
%TC:envir math 0 word
%TC:envir comment 0 0
%%
%%
%% The first command in your LaTeX source must be the \documentclass command.
%\documentclass[sigconf,anonymous,review]{acmart}

\documentclass[acmsmall]{acmart}
%\documentclass[acmsmall]{acmart}
%\usepackage{hyperref}
%\usepackage{hyperxmp}


%% NOTE that a single column version may required for 
%% submission and peer review. This can be done by changing
%% the \doucmentclass[...]{acmart} in this template to 
%% \documentclass[manuscript,screen]{acmart}
%% 
%% To ensure 100% compatibility, please check the white list of
%% approved LaTeX packages to be used with the Master Article Template at
%% https://www.acm.org/publications/taps/whitelist-of-latex-packages 
%% before creating your document. The white list page provides 
%% information on how to submit additional LaTeX packages for 
%% review and adoption.
%% Fonts used in the template cannot be substituted; margin 
%% adjustments are not allowed.

%%
%% \BibTeX command to typeset BibTeX logo in the docs
\AtBeginDocument{%
  \providecommand\BibTeX{{%
    \normalfont B\kern-0.5em{\scshape i\kern-0.25em b}\kern-0.8em\TeX}}}

%% Rights management information.  This information is sent to you
%% when you complete the rights form.  These commands have SAMPLE
%% values in them; it is your responsibility as an author to replace
%% the commands and values with those provided to you when you
%% complete the rights form.
%%\setcopyright{acmcopyright}
\setcopyright{none}
%%\copyrightyear{2025}
%%\acmYear{2025}
%%\acmDOI{XXXXXXX.XXXXXXX}


%% These commands are for a PROCEEDINGS abstract or paper.
%%\acmConference[]{}{2025}{}
%
%  Uncomment \acmBooktitle if th title of the proceedings is different
%  from ``Proceedings of ...''!
%
%\acmBooktitle{Woodstock '18: ACM Symposium on Neural Gaze Detection,
%  June 03--05, 2018, Woodstock, NY} 
%\acmPrice{15.00}
%\acmISBN{978-1-4503-XXXX-X/18/06}


%%
%% Submission ID.
%% Use this when submitting an article to a sponsored event. You'll
%% receive a unique submission ID from the organizers
%% of the event, and this ID should be used as the parameter to this command.
%%\acmSubmissionID{123-A56-BU3}

%%
%% For managing citations, it is recommended to use bibliography
%% files in BibTeX format.
%%
%% You can then either use BibTeX with the ACM-Reference-Format style,
%% or BibLaTeX with the acmnumeric or acmauthoryear sytles, that include
%% support for advanced citation of software artefact from the
%% biblatex-software package, also separately available on CTAN.
%%
%% Look at the sample-*-biblatex.tex files for templates showcasing
%% the biblatex styles.
%%

%%
%% For managing citations, it is recommended to use bibliography
%% files in BibTeX format.
%%
%% You can then either use BibTeX with the ACM-Reference-Format style,
%% or BibLaTeX with the acmnumeric or acmauthoryear sytles, that include
%% support for advanced citation of software artefact from the
%% biblatex-software package, also separately available on CTAN.
%%
%% Look at the sample-*-biblatex.tex files for templates showcasing
%% the biblatex styles.
%%

%%
%% The majority of ACM publications use numbered citations and
%% references.  The command \citestyle{authoryear} switches to the
%% "author year" style.
%%
%% If you are preparing content for an event
%% sponsored by ACM SIGGRAPH, you must use the "author year" style of
%% citations and references.
%% Uncommenting
%% the next command will enable that style.
%%\citestyle{acmauthoryear}

%%
%% end of the preamble, start of the body of the document source.

\usepackage[all,cmtip]{xy}
\usepackage[linesnumbered,ruled,vlined]{algorithm2e} % For algorithms
%\usepackage{algorithmic}
\usepackage{enumitem}
\usepackage{multirow}
\usepackage[yyyymmdd,hhmmss]{datetime}
\usepackage{tcolorbox}
\settopmatter{printacmref=false}

\setlength{\textfloatsep}{5pt}

\begin{document}

%%
%% The "title" command has an optional parameter,
%% allowing the author to define a "short title" to be used in page headers.




\title{A Categorical Unification for Multi-Model Data: 
\\ Part I Categorical Model and Normal Forms}

%%
%% The "author" command and its associated commands are used to define
%% the authors and their affiliations.
%% Of note is the shared affiliation of the first two authors, and the
%% "authornote" and "authornotemark" commands
%% used to denote shared contribution to the research.
\author{Jiaheng Lu}
%\authornote{Both authors contributed equally to this research.}
\affiliation{%
  \institution{University of Helsinki}
  \country{Finland}
}
\email{Jiaheng.lu@helsinki.fi}
%\orcid{1234-5678-9012}






%%
%% By default, the full list of authors will be used in the page
%% headers. Often, this list is too long, and will overlap
%% other information printed in the page headers. This command allows
%% the author to define a more concise list
%% of authors' names for this purpose.
%\renewcommand{\shortauthors}{Trovato and Tobin, et al.}

%%
%% The abstract is a short summary of the work to be presented in the
%% article.
\begin{abstract}
%Modern database systems face a significant challenge in effectively handling the Variety of data. This paper presents a categorical framework that addresses this challenge by unifying three types of structured data: relation, XML, and graph-structured data. Leveraging the language of category theory, our framework offers a sound formal abstraction for representing these data types. The primary objective is to establish a unified data model that enables research and techniques for multi-model data management.

Modern database systems face a significant challenge in effectively handling the Variety of data. The primary objective of this paper is to establish a unified data model and theoretical framework for multi-model data management. To achieve this, we present a categorical framework to unify three types of structured or semi-structured data: relation, XML, and graph-structured data. Utilizing the language of category theory, our framework offers a sound formal abstraction for representing these diverse data types. We extend the Entity-Relationship (ER) diagram with enriched semantic constraints, incorporating categorical ingredients such as pullback, pushout and limit. Furthermore, we develop a categorical normal form theory which is applied to category data to reduce redundancy and facilitate data maintenance. Those normal forms are applicable to relation, XML and graph data simultaneously, thereby eliminating the need for ad-hoc, model-specific definitions as found in separated normal form theories before. Finally, we discuss the connections between this new normal form framework and  Boyce-Codd normal form, fourth normal form, and XML normal form.

%\today ~ \currenttime 

\end{abstract}

%%
%% The code below is generated by the tool at http://dl.acm.org/ccs.cfm.
%% Please copy and paste the code instead of the example below.
%%
\begin{CCSXML}
<ccs2012>
 <concept>
  <concept_id>10010520.10010553.10010562</concept_id>
  <concept_desc>Computer systems organization~Embedded systems</concept_desc>
  <concept_significance>500</concept_significance>
 </concept>
 <concept>
  <concept_id>10010520.10010575.10010755</concept_id>
  <concept_desc>Computer systems organization~Redundancy</concept_desc>
  <concept_significance>300</concept_significance>
 </concept>
 <concept>
\end{CCSXML}

%\ccsdesc[500]{Computer systems organization~Embedded systems}

%%
%% Keywords. The author(s) should pick words that accurately describe
%% the work being presented. Separate the keywords with commas.
%\keywords{datasets, neural networks, gaze detection, text tagging}

%% A "teaser" image appears between the author and affiliation
%% information and the body of the document, and typically spans the
%% page.
%\begin{teaserfigure}
%  \includegraphics[width=\textwidth]{sampleteaser}
%  \caption{Seattle Mariners at Spring Training, 2010.}
%  \Description{Enjoying the baseball game from the third-base seats. Ichiro Suzuki preparing to bat.}
%\label{fig:teaser}
%\end{teaserfigure}

%\received{20 February 2007}
%\received[revised]{12 March 2009}
%\received[accepted]{5 June 2009}

%%
%% This command processes the author and affiliation and title
%% information and builds the first part of the formatted document.
\maketitle


\section{Introduction}
\label{sec:introduction}
The business processes of organizations are experiencing ever-increasing complexity due to the large amount of data, high number of users, and high-tech devices involved \cite{martin2021pmopportunitieschallenges, beerepoot2023biggestbpmproblems}. This complexity may cause business processes to deviate from normal control flow due to unforeseen and disruptive anomalies \cite{adams2023proceddsriftdetection}. These control-flow anomalies manifest as unknown, skipped, and wrongly-ordered activities in the traces of event logs monitored from the execution of business processes \cite{ko2023adsystematicreview}. For the sake of clarity, let us consider an illustrative example of such anomalies. Figure \ref{FP_ANOMALIES} shows a so-called event log footprint, which captures the control flow relations of four activities of a hypothetical event log. In particular, this footprint captures the control-flow relations between activities \texttt{a}, \texttt{b}, \texttt{c} and \texttt{d}. These are the causal ($\rightarrow$) relation, concurrent ($\parallel$) relation, and other ($\#$) relations such as exclusivity or non-local dependency \cite{aalst2022pmhandbook}. In addition, on the right are six traces, of which five exhibit skipped, wrongly-ordered and unknown control-flow anomalies. For example, $\langle$\texttt{a b d}$\rangle$ has a skipped activity, which is \texttt{c}. Because of this skipped activity, the control-flow relation \texttt{b}$\,\#\,$\texttt{d} is violated, since \texttt{d} directly follows \texttt{b} in the anomalous trace.
\begin{figure}[!t]
\centering
\includegraphics[width=0.9\columnwidth]{images/FP_ANOMALIES.png}
\caption{An example event log footprint with six traces, of which five exhibit control-flow anomalies.}
\label{FP_ANOMALIES}
\end{figure}

\subsection{Control-flow anomaly detection}
Control-flow anomaly detection techniques aim to characterize the normal control flow from event logs and verify whether these deviations occur in new event logs \cite{ko2023adsystematicreview}. To develop control-flow anomaly detection techniques, \revision{process mining} has seen widespread adoption owing to process discovery and \revision{conformance checking}. On the one hand, process discovery is a set of algorithms that encode control-flow relations as a set of model elements and constraints according to a given modeling formalism \cite{aalst2022pmhandbook}; hereafter, we refer to the Petri net, a widespread modeling formalism. On the other hand, \revision{conformance checking} is an explainable set of algorithms that allows linking any deviations with the reference Petri net and providing the fitness measure, namely a measure of how much the Petri net fits the new event log \cite{aalst2022pmhandbook}. Many control-flow anomaly detection techniques based on \revision{conformance checking} (hereafter, \revision{conformance checking}-based techniques) use the fitness measure to determine whether an event log is anomalous \cite{bezerra2009pmad, bezerra2013adlogspais, myers2018icsadpm, pecchia2020applicationfailuresanalysispm}. 

The scientific literature also includes many \revision{conformance checking}-independent techniques for control-flow anomaly detection that combine specific types of trace encodings with machine/deep learning \cite{ko2023adsystematicreview, tavares2023pmtraceencoding}. Whereas these techniques are very effective, their explainability is challenging due to both the type of trace encoding employed and the machine/deep learning model used \cite{rawal2022trustworthyaiadvances,li2023explainablead}. Hence, in the following, we focus on the shortcomings of \revision{conformance checking}-based techniques to investigate whether it is possible to support the development of competitive control-flow anomaly detection techniques while maintaining the explainable nature of \revision{conformance checking}.
\begin{figure}[!t]
\centering
\includegraphics[width=\columnwidth]{images/HIGH_LEVEL_VIEW.png}
\caption{A high-level view of the proposed framework for combining \revision{process mining}-based feature extraction with dimensionality reduction for control-flow anomaly detection.}
\label{HIGH_LEVEL_VIEW}
\end{figure}

\subsection{Shortcomings of \revision{conformance checking}-based techniques}
Unfortunately, the detection effectiveness of \revision{conformance checking}-based techniques is affected by noisy data and low-quality Petri nets, which may be due to human errors in the modeling process or representational bias of process discovery algorithms \cite{bezerra2013adlogspais, pecchia2020applicationfailuresanalysispm, aalst2016pm}. Specifically, on the one hand, noisy data may introduce infrequent and deceptive control-flow relations that may result in inconsistent fitness measures, whereas, on the other hand, checking event logs against a low-quality Petri net could lead to an unreliable distribution of fitness measures. Nonetheless, such Petri nets can still be used as references to obtain insightful information for \revision{process mining}-based feature extraction, supporting the development of competitive and explainable \revision{conformance checking}-based techniques for control-flow anomaly detection despite the problems above. For example, a few works outline that token-based \revision{conformance checking} can be used for \revision{process mining}-based feature extraction to build tabular data and develop effective \revision{conformance checking}-based techniques for control-flow anomaly detection \cite{singh2022lapmsh, debenedictis2023dtadiiot}. However, to the best of our knowledge, the scientific literature lacks a structured proposal for \revision{process mining}-based feature extraction using the state-of-the-art \revision{conformance checking} variant, namely alignment-based \revision{conformance checking}.

\subsection{Contributions}
We propose a novel \revision{process mining}-based feature extraction approach with alignment-based \revision{conformance checking}. This variant aligns the deviating control flow with a reference Petri net; the resulting alignment can be inspected to extract additional statistics such as the number of times a given activity caused mismatches \cite{aalst2022pmhandbook}. We integrate this approach into a flexible and explainable framework for developing techniques for control-flow anomaly detection. The framework combines \revision{process mining}-based feature extraction and dimensionality reduction to handle high-dimensional feature sets, achieve detection effectiveness, and support explainability. Notably, in addition to our proposed \revision{process mining}-based feature extraction approach, the framework allows employing other approaches, enabling a fair comparison of multiple \revision{conformance checking}-based and \revision{conformance checking}-independent techniques for control-flow anomaly detection. Figure \ref{HIGH_LEVEL_VIEW} shows a high-level view of the framework. Business processes are monitored, and event logs obtained from the database of information systems. Subsequently, \revision{process mining}-based feature extraction is applied to these event logs and tabular data input to dimensionality reduction to identify control-flow anomalies. We apply several \revision{conformance checking}-based and \revision{conformance checking}-independent framework techniques to publicly available datasets, simulated data of a case study from railways, and real-world data of a case study from healthcare. We show that the framework techniques implementing our approach outperform the baseline \revision{conformance checking}-based techniques while maintaining the explainable nature of \revision{conformance checking}.

In summary, the contributions of this paper are as follows.
\begin{itemize}
    \item{
        A novel \revision{process mining}-based feature extraction approach to support the development of competitive and explainable \revision{conformance checking}-based techniques for control-flow anomaly detection.
    }
    \item{
        A flexible and explainable framework for developing techniques for control-flow anomaly detection using \revision{process mining}-based feature extraction and dimensionality reduction.
    }
    \item{
        Application to synthetic and real-world datasets of several \revision{conformance checking}-based and \revision{conformance checking}-independent framework techniques, evaluating their detection effectiveness and explainability.
    }
\end{itemize}

The rest of the paper is organized as follows.
\begin{itemize}
    \item Section \ref{sec:related_work} reviews the existing techniques for control-flow anomaly detection, categorizing them into \revision{conformance checking}-based and \revision{conformance checking}-independent techniques.
    \item Section \ref{sec:abccfe} provides the preliminaries of \revision{process mining} to establish the notation used throughout the paper, and delves into the details of the proposed \revision{process mining}-based feature extraction approach with alignment-based \revision{conformance checking}.
    \item Section \ref{sec:framework} describes the framework for developing \revision{conformance checking}-based and \revision{conformance checking}-independent techniques for control-flow anomaly detection that combine \revision{process mining}-based feature extraction and dimensionality reduction.
    \item Section \ref{sec:evaluation} presents the experiments conducted with multiple framework and baseline techniques using data from publicly available datasets and case studies.
    \item Section \ref{sec:conclusions} draws the conclusions and presents future work.
\end{itemize}



\section{Thin Set Category}

\nopagebreak

In this section, we describe a formal conceptual data model for multi-model data with a \textbf{thin set category}.

\subsection{Objects and Morphisms in Categories}


\begin{definition}\label{def:category} \cite{MacLane:205493} (Mathematical category) A category $\mathcal{C}$ consists of a collection of objects denoted by $Obj(\mathcal{C})$ and a collection of morphisms denoted by $Hom(\mathcal{C})$.  For each morphism $f \in Hom(\mathcal{C})$ there exists an object $A \in Obj(\mathcal{C})$ that is a domain of $f$ and an object $B \in Obj(\mathcal{C})$ that is a target of $f$. In this case we denote $f \colon A \to B$. We require that all the defined compositions of morphisms are included in $\mathcal{C}$: if $f\colon A \to B \in Hom(\mathcal{C})$ and $g \colon B \to C \in Hom(\mathcal{C})$, then $g \circ f \colon A \to C \in Hom(\mathcal{C})$. We assume that the composition operation is associative and for every object $A \in Obj(\mathcal{C})$ there exists an identity morphism $\text{id}_{A} \colon A \to A$ so that $f \circ \text{id}_{A} = f$ and $\text{id}_{A} \circ f = f$ whenever the composition is defined.
\end{definition}

%\vspace{-5mm}

Intuitively, a category can be viewed as a graph. A category consists of objects and morphisms, where the objects can be thought of as nodes in a graph, and the arrows can be seen as directed edges connecting the nodes. However, a category is a more structured and rich mathematical concept that incorporates additional properties and operations beyond those of a graph. In a category, there is a composition operation defined on the arrows, allowing for the composition of arrows along a path. This composition operation follows certain rules, such as associativity, which are not typically present in a general graph. 



%a category as a directed multi-graph.   Here, the objects are represented as nodes, and the morphisms as edges. However, a category extends beyond the concept of a graph in one key aspect. A category specifies the composition of any two morphisms, and this composition operation adheres to the principle of associativity. 

%Secondly, a category introduces the notion of an identity morphism for each object within the category. 

%It is worth noting that multiple edges can exist between two nodes. 

%Visually, an entity object is depicted as a square shape. Graphically, a relationship object is symbolized by a diamond shape. which is graphically represented as a circle.

A database can be viewed as a category, specifically a \textbf{Set} category. In this context, each object in the category represents a set, while each morphism corresponds to a function between two sets. Both objects and morphisms possess distinct names within a database category (see an example in Figure 1). The composition of morphisms aligns with the composition of functions. Drawing inspiration from the principles of entity-relationship (ER) diagrams, we identify three fundamental types of objects within this category: \textit{entity objects}, \textit{attribute objects}, and \textit{relationship objects}. An entity object can represent a physical object, an event, or a conceptual entity. Visually, an entity object is depicted as a square shape. The elements contained within an entity object consist of a set of surrogate keys, which are sequentially generated integer numbers.  A relationship object captures the connections between various objects. Graphically, a relationship object is symbolized by a diamond shape. Similar to entity objects, the elements within a relationship object are also represented by surrogate keys.  An attribute object is a particular property to describe an entity or a relationship object, which is graphically represented as a circle.  Each attribute is associated with the domain of values. The domain is similar to the basic data types in programming languages, such as string, integer, Boolean, etc.  

%\begin{definition} (Database category)A database category $\mathcal{C}$ is a special type of category, where each  object is a set and each morphism is a function.  There are three types of objects: entity objects, relationship objects, and attribute objects. There are three types of morphisms: monomorphism, epimorphism and isomorphism, which corresponds to three types of functions: surjective, injective and bijective functions. Each object and morphism in the category has a (distinct) label.\end{definition}


%In this paper, we use $Set$ category where each object is a set and  each morphism is a function.



%In addition, an attribute can be simple or composite. A simple attribute has a single domain (such as Char or integer), whereas a composite attribute object is an attribute  that is composed of multiple simple attributes. It represents a more complex piece of data that can be broken down into its constituent parts. For instance, address is a composite attribute that may include street, zip-code, and country etc simple attributes. The elements in each composite attribute are also a set of surrogate keys.

%A morphism  is considered an identity if it corresponds to an identity function. In other words, it maps each element to itself.  Every object within a category possesses an associated identity morphism. 

Morphisims are defined as functions in sets. A function maps elements from its domain to elements in its codomain. Let $X$ and $Y$ are two sets (objects), then $f$: $X \to Y $ is a function  from the domain $X$ to the codomain $Y$ such that for any element $\forall  x \in X$,  there is certainly one element $y \in Y$, $ y = f(x)$. A morphism is sometimes called an arrow. Thus, we say arrow, function, or morphism interchangeably, and they are equivalent. Given a relationship object $R$ which associates several other objects $X_1$,$X_2$,...,$X_n$, the morphism between $R$ and $X_i$ is called projection morphism, denoted as $\pi_i(R)=X_i$ 
 (1$\leq$$i$$\leq$$n$).   In database theory, a functional dependency (FD) is a constraint that specifies the relationship between sets of values.  In the context of category theory, each morphism, denoted as $X \to Y$, naturally corresponds to a functional dependency. This is due to the fact that there is a unique element in the set $Y$ for every element in the set $X$. The transitivity of functional dependencies can be understood as the composition of arrows in a category.

%An identity morphism maps each element to itself.  Every object within a category possesses an associated identity morphism. 

%However, note that there exists a subtle difference between morphisms in category theory and functional dependencies in relational tables. In a relational table, for every tuple belonging to attribute(s) $Y$, there will always be a corresponding value for the attribute(s) $X$. However, in a category, it is possible for an element in the set $Y$ to not have a preimage in the set $X$. This distinction arises from the fact that the definition of functions guarantees that each element in the domain has an image in the codomain, but not necessarily the other way around. However, for simplicity and ease of analysis, we intentionally overlook this distinction and assume that each function is surjective throughout this paper. This means that every element in the codomain is mapped to by at least one element in the domain.

%A functional dependency (FD) in DBMS is a constraint that specifies how a set of attributes determines another set of attributes in a relation. In a category, each morphism $X \to Y$ naturally defines a function dependency because there is only one element  in $Y$ for any given element in $X$. The transitivity of FD  corresponds to the composed arrows in a category. However, it is worth to note the subtle difference between morphism and FD. In a relational table, for each tuple from attribute(s) $Y$, there always exists a corresponding value for $X$. However, in a category, it is possible that an element in $Y$ cannot find its preimage from $X$. Because the definition of functions only guanrantee that each element in X has an image, not vice versa. For the purpose of this paper, we intentionally ignore this difference and assume that each function is surjective, where each element of the codomain is mapped to by at least one element of the domain. 

%A morphism is classified as atomic if it cannot be derived through the composition of other arrows. Otherwise, it is referred to as a composed arrow. For instance, the morphism connecting mothers and children is atomic, while that connecting grandmothers and children is a composed one. 





%A functional dependency (FD) in DBMS is a constraint that specifies how a set of attributes determines another set of attributes in a relation. In a database, each function $X \to Y$ defines a function dependency because there is only one element (tuple) in $Y$ for any given element (tuple) in $X$. The transitivity of FD  naturally corresponds to the composed arrows in a category. Meanwhile,  note the nuanced difference between them. In a relational table instance, for each tuple from attribute(s) $Y$, there exists a corresponding tuple for $X$. This ensures that the functional dependency is well-defined. However, in a category, it is possible that an element in an object $Y$ cannot find its preimage in $X$. But in this paper, when we discuss the category for database, we intentionally ignore this difference, We assume that each function is surjective. The function is surjective, or onto, if each element of the codomain is mapped to by at least one element of the domain. 




%In mathematics, injections, surjections, and bijections are three classes of functions.  A function is injective, or one-to-one, if each element of the codomain is mapped to by at most one element of the domain.  The function is surjective, or onto, if each element of the codomain is mapped to by at least one element of the domain. The function is bijective (one-to-one and onto) if each element of the codomain is mapped to exactly one element of the domain. 


%In the language of category theory,  these three functions are called monomorphism, epimorphism and isomorphism, which correspond to surjective, injective and bijective respectively. 



%It is important to understand a special type of morphism that is associated with any relationship object. Given objects $X_1$,$X_2$,...,$X_1$,  $Y$ within the category, the relationship object formed by their combination is denoted as $XY$. Naturally, two morphisms emerge from this relationship object: $\pi_1: XY \to X$ and $\pi_2: XY \to Y$.


%Furthermore, a specific type of morphism exists within the category, namely the projection morphism (denoted as $\pi$), which connects a relationship object with its associated object. To illustrate, 



%A morphism is atomic if it cannot be derived through the composition of other arrows, otherwise, it is called the composed arrow. For example, the  arrow between mothers and children is atomic, but the arrow between grandmothers and children is a composed arrow.  In addition, a morphism between a relationship object and its associated object is called projection morphism (denoted with $\pi$). That is, given two objects $X$ and $Y$ in a category, let $XY$ denote the relationship object of $X$ and $Y$. Naturally, two morphisms emerge from this relationship object:  $\pi_1: XY \to X$ and $\pi_2: XY \to Y$.



Similar to a schema for a database table, a category schema provides a descriptive representation of how data within the category is organized and structured. It encompasses various details, such as the category's name, the domains of elements within each object (set) and any constraints (e.g., pullback or pushout to be elaborated upon later) that are applicable to the category.  A schema does not store the actual data within the category but rather defines the overall shape and format of the data. Note that a categorical schema offers a unified view of multi-model data. In this paper, we show algorithms that facilitate the mapping of a categorical schema to different individual data schemas, such as relational schema, XML DTD, and property graph schema. 

%These algorithms will enable us to effectively translate the categorical schema into specific data representations, allowing for seamless integration and utilization across different data models.






%\begin{example} Assume that there is a relationship object (O) between Student (S) and Course (S), and $f_1$ denotes the function between O and S, and $f_2$ is the function between O and C:  (1) $f_1$ and $f_2$ are surjections, then each course has at most one student and each student takes at most one course; (2) $f_1$ and $f_2$ are injections, each course has at least one student, and each student takes at least one course; (3) $f_1$ is a surjection, and $f_2$ are injection, each student takes at most one course, and each course has at least one student; Total participation means surjection function. (4)  f1 is an injection, and f2 is a surjection, each student takes at least one course, and each course has at most one student. \end{example}





%Functional dependencies are important for designing good database schemas and normalizing relations. Given a set of functional dependencies, in the setting of relational databases, there are algorithms to generate a set of relational schema such that each relation satisfy a certain level of normal form to reduce data redundancy and inconsistency. Analogously, in this paper, we  design algorithms to map a set of functional dependencies to a category and then convert this unified category to different types of data schema, including relations, XML, and property graph schema. Those output schema can also guarantee  a certain level of normal form to avoid data redundancy and inconsistency.

%\noindent \textbf{Discussion on function dependencies and arrows}:







%One question is whether these three inference rules can be applied to infer in a category. 

%There are three inference rules on Amstrong Axims \cite{armstrong1974dependency}: FD 1: If $Y \subseteq X$, then $X \to Y$; FD 2: If $X \to Y$, then $XZ \to YZ$; FD 3: If $X \to Y$ and $Y \to Z$, then $X \to Z$.

%In the above three rules, FD 1 shows that relationship object X can derive its all projection objects Y. This is because by definition of relationship, it must refer to the element in its associated object, and thus the mapping always can be found. But FD 2 does not hold. Let us see an example. Assume that three objects: project, supplier, and product. If project $\to$ supplier, meaning each project has one specific supplier. However,   project,product $\to$ supplier, product may not be true. This is because it is possible there is some certain product, which is required by the project, but this product cannot be supplied by the supplier. The traditional database assumes that it is impossible, but in this paper, we do not make such an assumption. FD3 holds because of the composition rule.

%Given a functional dependency $X_1,X_2,...X_n \to Y$ where the left-hand side includes multiple attributes, in order to present this FD we need to create a relationship object  X=$X_1,X_2,...X_n$, then it becomes $X \to Y$. In addition, we also need ot add the projection arrow from $X$ to each $X_i$.


\subsection{Thin Category}

%The category discussed in this paper is not only a set category but also a thin category, defined as follows: 

In this paper, the category under discussion is not just any set category, but rather a thin category, precisely defined as follows:

%In this subsection, we define a thin category and connect it with Universal Relational Assumption (URA).

%For simplicity, this paper assumes that there is only one morphism between any two objects. This is called a thin category as defined follows:

%For example, if there are two morphisms $f$ and $g$ between $X$ and $Y$, then Y can be divided into Y1 and Y2, such that $f:X \to Y1$ and $g: X \to Y2$.

%Throughout this paper, we assume that  given a pair of objects x and y, and any two morphisms f, g: $x \to y$, the morphisms f and g are equal, which is called a 'thin category'.

\begin{definition}(Thin Category) \cite{roman2017introduction} Given a pair of objects $X$ and $Y$ in a category $\mathcal{C}$, and any two morphisms $f$, $g$: $X \to Y$, we say that $\mathcal{C}$ is a thin category if and only if the morphisms $f$ and $g$ are equal. 
\end{definition}

%For example, if there are two morphisms $f, g$: children $ \to $ parents, f is mother, and g father. Then to define a thin category, the parent can be divided into father and mother two objects.

%In practice, transforming a non-thin category into a thin category involves the process of partitioning the objects $X$ and $Y$ within the category. When given two distinct morphisms, $f$ and $g$, both mapping from $X$ to $Y$, we can partition the object $Y$ into two distinct subsets, denoted as $Y_1$ and $Y_2$. Consequently, $f$ and $g$ are associated with one of these subsets: $f$ maps to $Y_1$, and $g$ maps to $Y_2$. In addition, note that a thin category accommodates the presence of bijective morphisms. In other words, within a thin category, it is possible for both $f: Y \to X$ and its inverse $f^{-1}: Y \to X$ to coexist and be well-defined simultaneously.


%In practice, to convert a non-thin category to a thin category, given two objects $X$ and $Y$, if there are two different morphisms $f$ and $g$: $X \to Y$, then $Y$ can be divided into $Y_1$ and $Y_2$, such that $f:X \to Y_1$ and $g: X \to Y_2$.  Note that a thin category allows a bijective function, i.e.  $f: Y \to X$ and $f^{-1}: Y \to X$ can hold simultaneously.

%A commutative diagram is a diagram such that all directed paths in the diagram with the same start and endpoints lead to the same result. 

In category theory, a commutative diagram is a graphical representation that depicts the relationships between objects and morphisms.  A diagram $\mathcal{D}$ is commutative meaning that different paths through the diagram yield the same result. In other words, if there are multiple ways to get from one object to another by following a sequence of morphisms, all these paths lead to the same result. The following lemma (\cite{roman2017introduction}) connects thin categories with commutative diagrams.

%The proof of all lemmas and theorems in this paper can be found in Appendix \ref{sec:proofs}. 


 
%Commutative diagrams are usually composed of commutative triangles and commutative squares. A looped arrow indicates a map from a set to itself. 

 

 \begin{lemma}  [\cite{roman2017introduction}] All diagrams in a thin category are commutative.\label{lem:thincommutative} 
 \end{lemma}

%\begin{proof} (Sketch) Without loss of generality, consider three objects $A$, $B$ and $C$ in a category.  Let $f: A \to B$,  $g: B \to C$, and   let $k = g \circ f$. Since there is only one arrow (i.e. $k$) between $A$ and $C$, then this diagram is commutative, otherwise, there are at least two arrows between $A$ and $C$, which contradicts the definition of a thin category. This commutative result can be proved for any two sequences of objects with the same starting and ending points, which concludes this proof.\end{proof}

%\begin{example} Figure \ref{fig:commutativediagram}(a) illustrates a commutative diagram. Each group has one head and this head is affiliated with a school. Then this school is the same as the group's department's school. This is a commutative diagram. That is: $ \phi_2 \circ \phi_1 = \phi_4 \circ \phi_3$. Figure \ref{fig:commutativediagram}(b) shows the example of the weak entity object, which implies a commutative diagram for a dependent, the employer of each dependent is decided by the corresponding employer.    \end{example}

%A thin category ensures that different paths for constructing or transforming data are equivalent, resulting in consistent outcomes. It is interesting to note that a thin category is closely linked to the universal relational assumption (URA) in database theory, which asserts that each attribute name is unique, and all information can be consolidated within a single relation. The output of relation data from a thin category satisfies the URA.


 %A thin category ensures the equivalence of various paths, thereby guaranteeing commutativity for all diagrams. It is intriguing to note its link to the Universal Relational Assumption (URA) in database theory. The URA assumes that each attribute name must be unique, allowing the consolidation of all information within a single relation. The output of relational data from a thin category adheres to the URA.  Consequently, when examined through the lens of category theory, the URA in database theory can be seen as the different names to define a thin set category!


 A thin category ensures the equivalence of various paths, thereby guaranteeing commutativity for all diagrams. It is intriguing to note its link to the Universal Relational Assumption (URA) \cite{DBLP:conf/pods/KuckS82,DBLP:conf/pods/SteinM85} in database theory. The URA assumes that each attribute name must be unique, allowing the consolidation of all information from multiple relations within a single relation through join operators. The output of relational data from a thin category adheres to the URA. Consequently, when examined through the lens of category theory, the URA in database theory can be understood as an alternative way to define a thin category.
 
The following definition summarizes the main characteristics of the database category.
 
 
 
%Since all diagrams in a thin category are commutative, all information can be joined together to produce a single relation. 




\begin{definition}(Database Category) A database category is a special type of category defined as a thin set category. Each object is a set and each morphism is a function. There are three types of objects: \textit{entity objects}, \textit{attribute objects}, and \textit{relationship objects}. All diagrams in this category are \textbf{commutative}. \end{definition}



\subsection{Multivalued Dependencies and Pullback }
\label{sec:mvd}

%Functional dependencies (FDs) serve to describe the relationship between the values of one set of attributes and the values of another set. In contrast, multivalued dependencies (MVDs) capture non-trivial dependencies between sets of attributes, where each set can possess multiple values for the other. 

While a functional dependency, denoted as $X \to Y$, can be represented as a morphism between the sets $X$ and $Y$ within a category, defining multivalued dependencies (MVDs) requires additional conceptual tools. We employ the concept of \textbf{pullback}, which serves as a universal structure in categories, to define MVDs. 

%This allows us to precisely define and characterize the nature of multivalued dependencies in a systematic manner. 

%While functional dependencies (FDs) describe how the values of one set of attributes determine the values of another set, multivalued dependencies (MVDs) represent non-trivial dependencies between sets of attributes where each set can have multiple values for the other.  A functional dependency $X  \to Y$ can be interpreted as a morphism between X and Y in a category. On the other hand,  the definition of multivalued dependencies involves more machinery. We need to leverage pullback, a universal structure of categories to define MVD.


%Pullback describes the joining of two tables. As for the joining with multiple tables, we use a general definition called limits.



\begin{definition} (Pullback) Given a category $\mathcal{C}$ with three objects $A$, $B$, and $C$, and two morphisms: $f: A \to B$, $g: B \to C$. The pullback $P$ of $f$ and $g$ consists of morphisms $p_1: P \to A$  and $p_2: P \to B$ such that $f \circ p_1$=$g \circ p_2$  and $P$ has the universal property. That is, given any other object $P'$ with   two morphisms $p'_1: P' \to A$  and $p'_2: P' \to B$ such that $f \circ p'_1$=$g \circ p'_2$, there exists a unique morphism $u: P' \to  P$ with $p_1 \circ u = p'_1$ and  $p_2 \circ u = p'_2$.\end{definition}

%\[
%\xymatrix{
%P \ar[r]^{p_1} \ar[d]_{p_2} & A \ar[d]^f \\
%B \ar[r]_g & C
%}
%\]

%\[
%\xymatrix{
%P' \ar@/_/[ddr]_{p'_2} \ar@/^/[drr]^{p'_1} \ar@{.>}[dr]|-{u}\\
%& P \ar[d]^{p_2} \ar[r]_{p_1} & A\ar[d]_f \\
%&B \ar[r]^g &C}
%\]



%where $f,g$ are morphisms in $\mathcal{C}$. This means that for any object $Q$ and morphisms $q_1:Q\to X$ and $q_2:Q\to Y$ such that $f\circ q_1 = g\circ q_2$, there exists a unique morphism $u:Q\to P$ such that $p_1\circ u = q_1$ and $p_2\circ u = q_2$. 

In the category of sets, the pullback $P$ of functions $f : A \to C$ and $g : B \to C$ always exists and is given by the join of the sets $A$ and $B$ through $C$.
\[ P = A \bowtie_C B = \{ (a,b) \in A \times B | f(a)=g(b)=c, c \in C \} \]


%\begin{example} Assume that two tables T1(A,B), key is A, and T2(C,B), key is C. Then the join T1 and T2 based on A can be defined as the pullback object in the category:\[\xymatrix{{T_1 \bowtie  T_2} \ar[r]^{\pi} \ar[d]_{\pi} & A \ar[d] \\C \ar[r] & B}\]\end{example}

Given a relation $R(X,Y,Z)$ and two multivalued dependencies (MVDs) $ X \to\to Y$ and  $ X \to\to Z$.   The relation $R$ can be constructed by the following pullback diagram:
\[
\xymatrix{
{XYZ} \ar[r]^{\pi_1} \ar[d]_{\pi_2} & XY \ar[d]^{\pi_3} \\
XZ \ar[r]_{\pi_4} & X
}
\]


\[ XYZ = \{ (x,y,z) \in XY \times XZ |  (x,y) \in XY \wedge (x,z) \in XZ \wedge x \in X \} \]



%An MVD $X \to\to Y$ holds in a category C means that given any object $Z$, the relationship object XYZ is a pullback of XY and XZ, shown in the above diagram.  Since a pullback involves the values of object Z, a multivalued dependency $X \to\to Y$  cannot be checked through only X and Y alone. It follows that the specification of Z is an integral part of the MVD. It would be more appropriate, perhaps, to use the notation $X \to\to Y(Z)$ to stress the fact that the MVD (pullback) involves the set Z. However,  we (and most database papers) assume that the MVD is valid for all $Z$ in the database. For this reason, we still use  $X \to\to Y$. 

%Recall that the definition of MVD $X \to\to Y$ involves "a universal set U". The reason is that the validity of  $X \to\to Y$ depends on all other attributes associated with this relation. In the setting of most database papers,  $U$ is fixed. However, in a category of this paper,  $U$ could be various relationship objects which contain $X$ and $Y$. Thus the specification of $U$ is an integral part of an MVD. We sometimes use the notation $X \to\to_U Y$ to stress the fact that the MVD involves the set $U$.   


%It is possible that a relationship object UXY is valid, but U'XY is not valid. Thus, MVD constraints are sensitive to ''context'' while a functional dependency is not. We will see later that this global context gives rise to the inference rules of MVD are different from FDs. 

%In order to define an MVD $X \to\to Y$ in a category C, for any given object $Z$, then XYZ is a pullback relationship object with the above definition. 

%While a functional dependency $X  \to Y$ is defined in terms of the sets X and Y alone and can be interpreted as an arrow between X and Y in the category, the definition of multivalued dependencies involves more machinery.  On the other hand, the multivalued dependency $X \to\to Y$  depends on the values of all relationship objects which includes X, Y. It cannot be checked through only X and Y and their arrows. It is possible that a relationship object UXY is valid, but U'XY is not valid. Thus, MVD constraints are sensitive to ''context'' while a functional dependency is not. We will see later that this global context gives rise to the inference rules of MVD are different from FDs. It follows that the specification of U is an integral part of the MVD. It would be more appropriate, perhaps, to use the notation $X \to\to Y(U)$ to stress the fact that the MVD involves the set U. However, in this paper we assume that the MVD is valid for all $U \in C$. For this reason, we still use  $X \to\to Y$. In other words,  $X \to\to Y$ defines a set of limits. For any other object $XU$, there is a limit for $XYU$ in the category.



\subsection{Join Dependency and Limit }
\label{sec:joindependency}

 %Let $R$ be a relation and $S$ =\{$S_1,...,S_q$\} be a set of subsets of $R$, with the union of the $S_i$’s being $R$. If $R =  S_1(r) \bowtie  ... \bowtie  S_q(r)$, then we say $R$ satisfies the join dependency (JD) *[$S_1, ..., S_q$].  An MVD is a special case of a join dependency. If $U$ is a universal set of attributes, then a relation $I$ over $U$ satisfies an MVD: $X \to\to Y$ if $I$ satisfies JD:*[$XY,X(U-Y)$].  


 If $U$ is a universal set of attributes, a join dependency over $U$ is an expression of the form $\bowtie[X_1,...,X_n]$, where each of $X_1,...,X_n$ is a subset of $U$ with the union of the $X_i$’s being $U$. A relation $I$ over $U$ satisfy $\bowtie[X_1,...,X_n]$ if $I = ~\bowtie^n_{i=1}(\pi_{X_i}(I))$. A multivalued dependency (MVD) is a special case of a join dependency.  A relation $I$ over $U$ satisfies an MVD: $X \to\to Y$ if $I = ~\bowtie[XY,X(U-Y)]$.  
 
 %then a relation $I$ over $U$ satisfies an MVD: $X \to\to Y$ if $I$ satisfies JD:*[$XY,X(U-Y)$].  

%\[
%\xymatrix{
%x \ar@(ul,dl)[]|{id} \ar@/^/[rr]|f
%&& f(x) \ar@/^/[ll]|{f^{-1}} }
%\]

%Note that a limit is a generalization of a pullback in category theory, we use it to connect join dependency, which is a generalization of multivalued dependency.  

In category theory,  a limit is considered a broader concept, encompassing and generalizing the notion of a pullback. In database field, join dependency serves as a more inclusive concept compared to multivalued dependency. By recognizing these analogies, we can establish meaningful connections between join dependencies and limits, unveiling the parallels between these two concepts.






\begin{figure}\centering
\[\xymatrix{
& & {S'} \ar@{.>}[d]|u \ar[ddll]|{\psi'_1} \ar[ddl]|{\psi'_2} \ar[ddr]|{\cdots} \ar[ddrr]|{\psi'_n} \ar@/^1pc/[dd]|{\psi'_3} \\
 & & {S}  \ar[dll]|{\psi_1} \ar[dl]|{\psi_2} \ar[d]|{\psi_3} \ar[dr]|{\cdots} \ar[drr]|{\psi_n} \\
T_1 \ar[dr] \ar[d]|{\pi_{11}}  & T_2  \ar[dl] \ar[dr]  \ar[d]|{\pi_{22}} & T_3 \ar[d]|{\pi_{33}} \ar[dl] \ar[dr] & ... \ar[d]|{\pi_{ij}} \ar[dl] \ar[dr] & T_n \ar[d]|{\pi_{nn}} \ar[dl] \\  
A_1 & A_2 & A_3 & ... & A_m
}
\]
\caption{This commutative diagram serves to illustrate the concept of a join limit. For the sake of clarity, the composed morphisms from $S$ to $A_1, ..., A_m$ and from $S'$ to $A_1, ..., A_m$ have been omitted in the diagram.  } \label{fig:joinlimitdiagram}
 \end{figure}


 
 %because it can be expressed as a binary join dependency. For example, $X \to\to Y$ is the same as JD:*[$XY,X(U-Y)$], where U is the     Meanwhile, limits are the generalization of pullback. We use limits to define join dependency.

 %\begin{definition} (Universal arrow) A universal arrow from X to F is a unique pair ($A,u: X \to F(A)$) in D which has the following property, commonly referred to as a universal property. For any morphism of the form $f: X$ $\to$ $F(A')$ in D, there exists a unique morphism h: A $\to$ A' in C such that the following diagram commutes. \end{definition} 

 \begin{definition} (Cone) Let $\mathcal{J}$ and $\mathcal{C}$ be categories. A diagram of $\mathcal{J}$ in $\mathcal{C}$ is a functor $D:$$\mathcal{J}$ $\to$ $\mathcal{C}$. A cone to  $D$ consists of an object $S$ in $\mathcal{C}$ and a family of morphisms in $\mathcal{C}$, $\psi_X: S \to D(X) $ for each object $X$ in $\mathcal{J}$, such that for every morphism $f: X \to Y$, the triangle commutes $D(f) \circ \psi_X = \psi_Y$ in $\mathcal{C}$. $S$ is called the summit of the cone.
\end{definition}

\begin{definition} (Limit) A limit of the diagram $D:$$\mathcal{J}$ $\to$ $\mathcal{C}$ is a cone ($S, \psi_X$) to $D$ such that for every other cone ($S', \psi'_X$) there is a unique morphism $u: S' \to S$ such that $\psi_X \circ u = \psi'_X$ for all $X$ in $\mathcal{J}$. Thus, the limit is the ``\textit{closest}" cone to the diagram $D$. 
\end{definition}



%\begin{definition} \cite{MacLane:205493} (limits) Given categories C, J and the diagonal functor $\triangle: C \to C^J$, a limit for a functor $F: J \to C$ is a universal arrow (r,v) from $\triangle$ to F. It consists of an object r of C, called the limit object of the functor F, together with a natural transformation $v:\triangle r \to F$ which is universal among natural transformation $v:\triangle C \to F$. We call $\tau: c \to F$ a cone to the base F for the vertex c.\end{definition}


%\begin{definition} \cite{kan1958adjoint} (limits)Let $F \rightarrow J$ be a diagram of shape J in a category C. A cone to F is an object N of C together with a family $\psi_{X}: N \to F(X)$  of morphisms indexed by the objects X of J, such that for every morphism $f:X \to Y$ in J, we have  $F(f) \circ \psi _{X}=\psi_{Y}$.  A limit of the diagram $F:J\to C$ is a cone  $(L,\phi)$ to F such that for every other cone  $(N,\psi)$ to F there exists a unique morphism $u:N \to L$ such that  $\phi_{X} \circ u=\psi_{X}$ for all X in J. Limits are also referred to as universal cones, since they are characterized by a universal property.\end{definition}


%One says that the cone $(N,\psi)$ factors through the cone  $(L,\phi)$ with the unique factorization u.

%\begin{figure}
%\centering
%\includegraphics[width=0.5\textwidth]{figures/joinlimit.jpg}
%\caption{Illustration to the join limit}
%\label{fig:joinlimit}
%\end{figure}

%The following description connects the join dependency with limits. 

Based on the above general definition of limit, we describe a special type called \textbf{join limit} that is used in this paper to connect join dependency with limit. See the diagram in Figure \ref{fig:joinlimitdiagram}. A join limit is a limit of a diagram indexed by a  category $\mathcal{J}$. Consider a database category $\mathcal{C}$, $T_1,T_2,...,T_n$ are relationship objects, and $A_1,A_2,...,A_m$ are their associate objects. $\pi_{ij}$ is a projection morphisms from $T_i$ to $A_j$ defining a diagram of the shape in $\mathcal{C}$. A cone with the summit $S$ consists of  $m+n$ morphisms $\psi_X$, one for each object $X$ in the indexing category so that all triangles commute. The join limit $(S,\psi_X)$ is the closest cone over the diagram, such that for every other cone ($S', \psi'_X$) there is a unique morphism $u: S' \to S$ such that $\psi_X \circ u = \psi'_X$ for all $X$ in $\mathcal{J}$. In the context of the join limit, all morphisms $\psi_X$ and $\pi_{ij}$ are projection morphisms. The summit $S$ is a join limit, which satisfies $\bowtie[T_1,...,T_n]$. An example of join limit is shown in Introduction section (Figure \ref{fig:SPJJoin}) about three tables on \texttt{Supplier}, \texttt{Product} and \texttt{Project}.



%a commutative diagram with the following universal property (Figure \ref{fig:joinlimit}(b)): given any commutative diagram, there is a unique factorization of its legs through the summit of the limit cone. The limit object is called a join limit written $JD: *(T_1 \Join T_2 \Join ,..., \Join T_n)$.




%If J is the discrete category \{1,2,...$n$\}, a functor F: \{1,2,...$n$\} $\rightarrow$ C is a list of objects $<A_1,A_2,...,A_n>$ of C. The limit object is called a join limit of $a_1,a_2,...,a_n$, and is written $A_1 \Join A_2 \Join ,..., \Join A_n$; the limit diagram consists of $A_1 \Join A_2 \Join ,..., \Join A_n$ and $n$ arrows $f_1, f_2,...,f_n$, See Figure 9, called projection of the join limit. They constitute a cone from the vertex $A_1 \Join A_2 \Join ,..., \Join A_n$, so by the definition of a limit, there is a bijection of sets natural in c,\[ Hom(c,A_1 \Join A_2 \Join ,..., \Join A_n)   \cong Hom(c,A_1) \times  Hom(c,A_2) ... \times Hom(c,A_n)  \]which sends each $h$: $c \rightarrow A_1 \Join A_2 \Join ,..., \Join A_n$ to the set of composites $<f_1 h, f_2 h,... f_n h >$. Conversely, given arrows $g_i$:$c \rightarrow A_i $, there is a unique $h$: $c \rightarrow A_1 \Join A_2 \Join ,..., \Join A_n$ with $f_i h = g_i$. We write \[ h = (g_1.g_2,...g_n) : c \rightarrow A_1 \Join A_2 \Join ,..., \Join A_n  \]


%\begin{figure}\centering\[\xymatrix{&*+{SPJ (Limit)} \ar[dl] \ar[d] \ar[dr] \\SP \ar[dr] \ar[d]  & PJ  \ar[dl] \ar[dr] & SJ  \ar[d] \ar[dl]  \\  Product & Supplier & Project}\]\caption{An example to illustrate join limit.} \label{fig:SPJJoin}\end{figure}

 

%\begin{example} We illustrate the notion of join limit through an example involving three tables: \texttt{SP(Supplier, Product)},  \texttt{SJ(Supplier, Project)}, and \texttt{PJ(Project, Product)}. Assume the table \texttt{SPJ (Supplier, Project, Product)} = $SP \bowtie PJ \bowtie SJ$, which can be described using the concept of a join limit.   As shown in Figure \ref{fig:SPJJoin},  \texttt{SPJ} is the summit, \texttt{SP}, \texttt{PJ}, \texttt{SJ} are three relationship objects, and  \texttt{Product}, \texttt{Supplier} and \texttt{Project} are their projection objects.\end{example}

%satisfies a join dependency JD: *(\texttt{SP(Supplier, Product)}, \texttt{SJ(Supplier, Project)}, and 

 %Consider three tables \texttt{SP(Supplier, Product)}, \texttt{SJ(Supplier, Project)}, and \texttt{PJ(Project, Product)}.     The summit table \texttt{SPJ} shown in Figure \ref{fig:SPJJoin} satisfies a join dependency JD: *(\texttt{SP(Supplier, Product)}, \texttt{SJ(Supplier, Project)}, and \texttt{PJ(Project, Product)}). See Figure \ref{fig:joinlimitdiagram} for the commutative diagram of a join limit.


%This limit says that whenever a supplier s supplies part p, and a project j uses part p, and the supplier s supplies at least one part to project j, then the supplier will also be supplying part p to project j.

%\begin{figure}\centering\includegraphics[width=0.6\textwidth]{figures/LimitsGeneration.jpg}\caption{An example for the generation of limits from join dependency}\label{fig:limitsgeneration}\end{figure}

%Given a set of functional dependencies and join dependencies, Algorithm \ref{alg:JD2limit} shows how to generate a category.

%\begin{example} See Figure \ref{fig:limitsgeneration} for an example. In the first step, Bc, CD, and C are generated. Then in the second step, BC, CD are connected to C. In the last step, A, B D, E are generated and are connected to the corresponding set.\end{example}


%\begin{algorithm} \caption{Join dependency to limits }\label{alg:JD2limit}\input{algorithms/JD2limit}\end{algorithm}



%The following theorem shows that all limits can be boiled down to equalizer and product. Then this theorem can be explained that the join operator is considered an extended operator in relational algebra.  Any join operator can be boiled down to selection (which corresponds to equalizer) and cartesian product (which corresponds to product).

%\begin{definition} Let $\mathcal{C}$ be a category, and consider the diagram \[\begin{array}{ccc} A & \xRightarrow{f,g} & B \\ \end{array} \] The \textit{equalizer} of $f$ and $g$ consists of an object $E$ and a morphism $e: E \to A$ such that the following conditions hold: 1. $f \circ e = g \circ e$: This ensures that the diagram commutes. 2. For any object $Q$ and morphism $q: Q \to A$ satisfying $f \circ q = g \circ q$, there exists a unique morphism $\bar{q}: Q \to E$ such that $q = e \circ \bar{q}$: This condition states that for any other morphism $q$ satisfying the same property as $e$, there exists a unique morphism $\bar{q}$ that factors through $e$. The equalizer of $f$ and $g$ is denoted as $\text{eq}(f,g)$ or $\text{eq}(A,B)$. \end{definition}

%\begin{theorem} Any limit in a Set Category may be expressed as an equalizer of a pair of maps between products. Explicitly, for any small F: $J \to Set$, there is an equalizer diagram: \[ Lim_JF \to \prod_{j \in obj} F_j  \rightrightarrows  \prod_{f \in mor J} F(cod J) \]\end{theorem}The above theorem can be found in the book "category theory in context" \cite{riehl2017category}  page 87.

%The implication of join dependence is different from functional dependencies (fd's). It has been shown that there is no sound and complete set of inference rules for jd's analogous to those for fd's.  The absence of finite axiomatization for different types of dependencies, including join dependencies, is proved in paper \cite{petrov1989finite}. Note that The absence of a solution does not imply the nonenumerability of consequences. It implies only that a finite set of rules are not sufficient for obtaining all consequences. In addition, logical implication for jd's is decidable. The complexity of implication is polynomial for a fixed database schema but becomes NP-hard if the schema is considered part of the input \cite{10.5555/551350}.

%\begin{example}Figure shows an example to generate limits for a given set of JD dependencies.  \end{example}

%\begin{lemma}All join dependencies can be represented as limit objects in a category. \label{lem:join}\end{lemma}\begin{proof} See Algorithm \ref{alg:JD2limit} which shows how to convert a join dependency specification into a limit. Since the university property of limits, this algorithm can output results to find all join results. \end{proof}

%Meanwhile, it is important to note the difference between the three dependencies in the category. First functional dependency can be directly depicted in categories with an arrow. Second join dependency can be defined as a limit in the category. Third, multivalued dependency $X \to\to Y$ is more complicated. In fact, this MVD definition relies on the context. 



%\begin{figure}\centering\includegraphics[width=0.7\textwidth]{figures/JDImplication.jpg}\caption{This is an example to illustrate the necessity of implication for Join dependency implication. If we cannot find all JDs, then the output schema may not satisfy the generalized JD normal form. In this example, the table R6 can be computed from the join of R3, R4, and R5, and a projection on $x_1$ and $z$. }\label{fig:JDImplication}\end{figure}

%\begin{corollary}Any limit object in a database category may be computed through set intersection operations between caterisian products. \end{corollary}

%This theorem shows that all limits can be computed product and intersection operators in sets. 


\subsection{Pushout}

%The previous discussion highlights the correlation between pullbacks (limits) and multi-valued dependencies (join dependencies). This connection raises another question: how can we elucidate pushouts and colimits, which serve as the dual concepts to pullbacks and limits, respectively? 



In category theory, ``\textbf{duality}'' establishes a relationship between two concepts or operations by interchanging certain elements or properties. One such instance of duality exists between pushout and pullback. While a pullback captures the splitting of a morphism into two through projection, a pushout represents a way to merge two morphisms into a single entity. As pullbacks are associated with join operators, pushouts effectively define the equivalent class within a  database.


%Given that pullbacks are associated with multivalued dependencies, an intriguing inquiry arises as to whether pushouts give rise to another form of database constraint. In the subsequent discussion, we aim to demonstrate that pushouts indeed define the connected components within a graph database.

%The above discussion shows the connection between pullback (limit) and multi-valued dependency (join dependency). A natural question arises: how to explain pushout and colimit which are dual concepts of pullback and limit? In category theory,  duality is a notion that establishes a relationship between two concepts or operations by interchanging certain elements or properties. One such example is the duality between pushouts and pullbacks. while a pullback represents the operation that captures the splitting of a morphism into two by projection, a pushout represents a way to combine two morphisms into one. Given that pullbacks are associated with multivalued dependency, an intriguing inquiry arises as to whether pushouts give rise to another form of database constraint. In the subsequent discussion, we demonstrate that pushout indeed defines the connected components within an undirected graph database.



%Since pullback corresponds to multivalued dependency, one interesting question is whether pushout leads to another database constraint. In the following, we show that pushout actually defines the connected components in a (undirected) graph.

%A pushout is a quotient of the disjoint union of two sets. The ‘pushout’ of this diagram is the set X obtained by taking the disjoint union A+B and identifying $a \in A$ with $b \in B$ if there exists $x \in C$ such that $f(x)=a$ and $g(x)=b$ (and all identifications that follow to keep equality an equivalence relation).

%\begin{figure}\centering\includegraphics[width=0.7\textwidth]{figures/pushoutfamily2.jpg}\caption{Illustration to pushout with an example }\label{fig:pushoutfamily}\end{figure}

\begin{definition}  (Pushout) Given a category $\mathcal{C}$ with three objects $A$, $B$, and $C$, and two morphisms: $f: A \to B$, $g: A \to C$. The pushout $P$ of $f$ and $g$ consists of morphisms $p_1: B \to P$  and $p_2: C \to P$ such that $p_1 \circ f$=$p_2 \circ g$  and $P$ has the universal property. That is, given any other object $P'$ with two morphisms $p'_1: B \to P'$  and $p'_2: C \to P'$ such that $p'_1 \circ f$ = $p'_2 \circ g$, there exists a unique morphism $u: P \to  P'$ with $u \circ p_1 = p'_1$ and  $u \circ p_2 = p'_2$.




%\[
%  \xymatrix@=3pc{
%    & & P' \\
%    B \ar[r]^{p_1}\ar@/^/[rru]^{p'_1} & P \ar@{.>}[ru]_u  & \\
%    A \ar [u]^f \ar[r]_g & C \ar[u]_{p_2}\ar@/_/[ruu]_{p'_2} &
%  }
%\]



\end{definition}


%\begin{definition} (Pushout)  Let $\mathcal{C}$ be a category and $f: A \to B$, $g: A \to C$. The \textit{pushout} of $f$ and $g$ in $\mathcal{C}$ is a universal object denoted by $B \sqcup_A C$, along with the morphisms $i: B \to B \sqcup_A C$ and $j: C \to B \sqcup_A C$, satisfying the following conditions:1. The diagram below commutes:\[
%\xymatrix{
%    A \ar[r]^{f} \ar[d]_{g} & B \ar[d]^{i} \\
%    C \ar[r]_{j} & B \sqcup_A C
%}
%\]
%2. For any object $X$ in $\mathcal{C}$ and morphisms $h: B \to X$ and $k: C \to X$ such that $h \circ f = k \circ g$, there exists a unique morphism $m: B \sqcup_A C \to X$  such that $h = m \circ i$ and $k = m \circ j$. The pushout $B \sqcup_A C$ together with $i$ and $j$ is uniquely determined up to isomorphism.\end{definition}




%\begin{definition} (Pushout)Let $\mathcal{C}$ be a category, and consider the diagram\[\begin{array}{ccc}A & \xrightarrow{f} & B \\\downarrow{g} & & \downarrow{h} \\C & \xrightarrow{k} & D \\\end{array}\]The \textit{pushout} of the diagram is an object $P$ together with morphisms $i: B \to P$ and $j: D \to P$ satisfying the following conditions:1. $i \circ f = j \circ g$: This ensures that the diagram commutes.2. For any object $Q$ and morphisms $u: B \to Q$ and $v: D \to Q$ satisfying $u \circ f = v \circ g$, there exists a unique morphism $\bar{u}: P \to Q$ such that $u = \bar{u} \circ i$ and $v = \bar{u} \circ j$: This condition states that for any other morphisms $u$ and $v$ satisfying the same properties as $i$ and $j$ respectively, there exists a unique morphism $\bar{u}$ that factors through $i$ and $j$.The pushout of the diagram is denoted as $\text{pushout}(f,g,k)$ or $\text{pushout}(A,B,C,D)$.\end{definition} 

 In the context of set category, the pushout of $f$ and $g$ is the disjoint union of $B$ and $C$, where elements sharing a common preimage (in $A$) are identified, together with the morphisms $p_1, p_2$ from $B$ and $C$, i.e.  $P=(B\cup C)/ \sim $, where $\sim$ is the equivalence relation
such that $f(a) \sim g(a)$ for all $a \in A$ .


%\begin{example}  This example constructs a database instance to illustrate pushout. Consider a category with a  \texttt{Mother} object which has two elements \{$m_1$, $m_2$\}, a \texttt{Father} object with three elements:  \{$t_1$, $t_2$, $t_3$\}, and a \texttt{Child} object with three elements:  \{$s_1$, $s_2$, $s_3$\} such that $f(s_1)=m_1$, $f(s_2)=m_1$, $f(s_3)=m_2$, $g(s_1)=t_1$, $g(s_2)=t_2,  ~g(s_3)=t_3$. Then the pushout \texttt{Family} object contains two equivalence classes (elements): ($m_1$,$t_1$,$t_2$) and ($m_2$,$t_3$). The commutative diagram is illustrated below: 
%\[\xymatrix{Child \ar[r]^{f} \ar[d]_{g} & Mother \ar[d]^{p_1} \\Father \ar[r]_{p_2} & Family }\]
%\end{example}

%Further, pushout can be used to define connected components in an undirected graph, as each connected component constitute an equivalence class. More details for an example can be found in Appendix \ref{sec:pushout}.

%The pushout can be also employed to compute connected components within an undirected graph, as each connected component forms an equivalence class. For further elucidation and an example, please refer to Appendix \ref{sec:pushout}.

The following example illustrates the connection between the pushout and the connected component of an undirected graph.

\begin{example} Given an undirected graph $G$, the \texttt{Edge} table includes two attributes \texttt{Node\_id1} and \texttt{Node\_id2}, which describes edges between any two nodes in $G$.   The pushout object \texttt{Component} computes the connected component in the graph $G$, as illustrated in the following commutative diagram.


\[\xymatrix{Edge \ar[r]^{f} \ar[d]_{g} & Node\_id1 \ar[d]^{p_1} \\Node\_id2 \ar[r]_{p_2} & Component (Pushout) }\]\end{example}

%Note that the pullback object in category theory corresponds to the join operator in relational databases and the pushout object corresponds to the connected component in an undirected graph. Since pullback and pushout are dual objects, a captivating observation emerges: the join operator of relational databases and the computation of connected components of graph databases exhibit duality when viewed through the lens of category theory.

Consider that the pullback object corresponds to the join operator in relational databases, while the pushout object corresponds to the connected component in an undirected graph. As pullback and pushout are dual objects, an intriguing observation arises: \textit{the join operator in relational databases and the computation of connected components in graph databases demonstrate duality when examined through the lens of category theory}.

%For a general undirected graph, we may use the following coequalizer to compute the connected component.

%\begin{definition} (Coequalizer)Let $\mathcal{C}$ be a category and $f, g: A \to B$ be two morphisms in $\mathcal{C}$. A \textit{coequalizer} of $f$ and $g$ consists of an object $Q$ and a morphism $q: B \to Q$ such that the following conditions hold:1. $q \circ f = q \circ g$: This ensures that the diagram commutes. 2. For any object $R$ and morphism $r: B \to R$ satisfying $r \circ f = r \circ g$, there exists a unique morphism $\bar{r}: Q \to R$ such that $r = \bar{r} \circ q$: This condition states that for any other morphism $r$ satisfying the same property as $q$, there exists a unique morphism $\bar{r}$ that factors through $q$.The coequalizer of $f$ and $g$ is denoted as $\text{coeq}(f,g)$ or $\text{coeq}(A,B)$.\[  A  \rightrightarrows  B  \rightarrow Q \]\end{definition} 



%\begin{example} Consider another example about pushout with an undirected graph $G$. The \texttt{Edge} table includes two attributes \texttt{Node\_id1} and \texttt{Node\_id2}, which describes edges in $G$.   The pushout object \texttt{Component} computes the connected component in the graph $G$, as illustrated in the following commutative diagram.


%\[\xymatrix{Edge \ar[r]^{f} \ar[d]_{g} & Node\_id1 \ar[d]^{p_1} \\Node\_id2 \ar[r]_{p_2} & Component }\]\end{example}


%\begin{figure}\centering\includegraphics[width=0.4\textwidth]{figures/colimitSPJ.jpg}\caption{Illustration to colimit with SPJ}\label{fig:colimitSPJ}\end{figure}




%Pushout is a special type of colimit. It is straightforward to obtain the definition of colimit by inverting all morphisms in the limit.  The application of colimit includes the generalization of graph connect components as illustrated in the following example. 


%we will explicitly state them here and give an example of colimits in databases.

%\begin{definition} (colimits)Let $F: J \rightarrow C$ be a diagram of shape J in a category C. A co-cone to F is an object N of C together with a family $\psi_{X}:  F(X) \to N$  of morphisms indexed by the objects X of J, such that for every morphism $f:X \to Y$ in J, we have  $\psi_{Y} \circ  F(f) = \psi _{X}$.  A colimit of the diagram $F:J\to C$ is a co-cone  $(L,\phi)$ to F such that for every other cone  $(N,\psi)$ to F there exists a unique morphism $u:L \to N$ such that  $u \circ \phi_{X}  =\psi_{X}$ for all X in J.\end{definition}




%\begin{example} Consider an example of limits. We inverse the arrows on supplier, project and product. This is a generalized version of connected component, which shows the number of components fro supplier, project and product without any overlapping.  \end{example}




%Similarly, there is a general method to construct colimits in any category via coproducts and coequalizers. Let A and B be categories, and $T: A \to B$ be a functor. If B has coproducts of all families indexed by objects and morphisms of A and all binary coequalizers, then the colimit of T exists. It is the coequalizer of a pair of maps (F, G) from the coproduct of all domains of T to the coproduct of all values of T, such that F maps each morphism f in A to its domain and G maps each morphism f in A to its codomain composed with T(f). This means that any colimit in a category can be expressed as an coequalizer of a pair of maps between coproducts in that category.

%we prove the analogous formula for limits or colimits of diagrams valued in any category by reducing the proofs of these general results to the case of limits in Set. This strategy succeeds because limits and colimits in a general locally small category are defined representably in terms of limits in the category of sets. See around page 97 of that book.


% \begin{definition} (left restorable arrow) Let $A$, $B$ and $C$ are three objects in a category.  Let f: $A \to B$,  g: $B \to C$, and  k=   $g \circ f$. We say that $f$ is a left restorable arrow for $g$ and $k$ if and only if  $g$ is an injective function. That is, $g$ maps distinct elements of $B$ to distinct elements of $C$;\end{definition}

%\begin{lemma}Let $A$, $B$ and $C$ are three objects in a category.  Let f: $A \to B$,  g: $B \to C$, and  k=   $g \circ f$. If $f$ is a left unique arrow for $g$ and $k$, then the function $f$ can be derived from $g$ and $k$. \end{lemma} \begin{proof}We need to prove that given any element $a \in A$, the image of $a$ in $B$ can be uniquely decided by $g$ and $k$.  In fact, $f(a)=g^{-1}(k(a))$ and $g^{-1}$ exists for $k(a)$. Because the preimage of $k(a)$ always exists in $B$ and it is unique, as $g$ is an injective function and $k=   g \circ f$. \end{proof}

%\begin{definition} (right restorable Arrow) Let $A$, $B$ and $C$ are three objects in a category.  Let f: $A \to B$,  g: $B \to C$, and k =   $g \circ f$. That is, A, B C compose a commutative triangle. We say that $g$ is a right restorable arrow for $f$ and $k$ if and only if $f$ is a surjective function. That is, every element of $B$ is the image of at least one element of $A$.\end{definition}

%\begin{lemma} Let $A$, $B$ and $C$ are three objects in a category.  Let f: $A \to B$,  g: $B \to C$, and  k=   $g \circ f$. If $g$ is a right restorable arrow for $f$ and $k$, then the function $g$ can be derived from $f$ and $k$.\end{lemma}\begin{proof}We need to prove that given any element $b \in B$, the image of $b$ in $C$ can be uniquely decided by $f$ and $k$.   Because  $f$ is a surjective function, consider any $a$ in A and $f(a)=b$. We say that $g(b)=k(a)$, because $k=   g \circ f$. It is impossible that there is another element $a'$, s.t., $f(a')=b$, but $k(a) \neq k(a')$, as g is a single-image function.\end{proof}


%Functional dependencies play a crucial role in database management systems (DBMS) as they define the relationships between sets of attributes within a relation. These dependencies are essential for the design of efficient database schemas and the process of normalization. In the realm of relational databases, algorithms exist to derive a set of relational schemas from a given set of functional dependencies. These schemas adhere to specific levels of normal form, thereby minimizing data redundancy and ensuring data consistency. In this paper, we extend this concept further by proposing algorithms that convert sets of functional dependencies into a unified category. From this unified category, we generate different types of data representations, including relations, XML, and property graphs. Notably, these output data representations maintain the unified principles of normal form, thereby reducing redundancy and ensuring data consistency across the various data models.Algorithm 





%\begin{figure}\centering\includegraphics[width=0.6\textwidth]{figures/necessaryOfImplication.jpg}\caption{This example shows the necessity to compute the implication of FD. In this example, if we do not know the arrow from a to BC, then the output schema will not satisfy the 3NF. Thus, it is important to derive all related arrows (FDs) that are defined with the objects in the category.}\label{fig:implication}\end{figure}

%A set of constraints $C$ implies a constraint $c$, written $C \models c$, if $SAT(C)$ contains $SAT(c)$. A designer usually provides an explicit set of functional dependencies F whose closure F+ is the complete set of functional dependencies in the category representation. We say one functional dependency $X \to Y$ is proper  if there is no proper subset $X' \subset X$, such that  $X' \to Y$. In addition, the trivial FD $X \to Y$ means that $Y$ is a subset of $X$, otherwise, it is non-trivial.  Given any FD $X \to Y$, if $X$ includes two or multiple attributes, then we need to create a composite attribute which is an attribute object, but its elements are surrogate keys.  

%(1) Projection: for any two sets of objects $X$ and $Y$, then $XY \to X$ and  $XY \to Y$ are two projection morphisms.(2) Union: for any two sets of objects: if $X \to Y$ and $X \to Z$ then $X \to YZ$ is a composite morphism. Note that $YZ$ is not a Cartesian product object of $Y$ and $Z$. In particular, if $f(x)=y$ and $g(x)=z$, then $k: X \to YZ$ is defined as $k(x)=(y,z)$. (3) Composable: given three objects, if $X \to Y$, $Y \to Z$, then $X \to Z$. 



%\begin{algorithm} \caption{Complete a category given a set of functional dependencies}\label{alg:addarrows}\input{algorithms/addArrows}\end{algorithm}

%\begin{example}  This example illustrates the algorithm \ref{alg:addarrows}. Consider the FD $\Sigma$=\{$A \to B$, $A \to C$, $BC \to D$\}. $\Sigma^+$=$\Sigma$ $\cup$ \{$A \to D$\}, $\Sigma^\{++\}$=$\Sigma^+$ $\cup$ \{$A \to BC$\}. Finally, some projection arrows can be added such as $BC \to B$ and $BC \to C$. \end{example}


%\begin{lemma}Given a set of functional dependencies F and a category C which is constructed based on F with the above rules, all properly functional dependencies occur in $C$ as the arrows.    \label{lem:completeFDs}\end{lemma}

%Proof: The correctness holds based on the Armstrong axioms. All proper functional dependencies can be derived and constructed in C as arrows.  



%\subsection{Comparison between classic and categorical ER models}

%Table \ref{tab:ERmodel} provides a summary of the distinctions between the classic Entity-Relationship (ER) model and the categorical model proposed in this paper.


%\noindent \textbf{Comparison between the classic and categorical ER model}: The primary differentiation between the two models arises from three perspectives: 

%(1) The categorical ER model introduces the concepts of pullback, pushout, and limit, which encompass richer semantic constraints compared to the classic ER model.

%(2) The categorical ER model explicitly defines composable morphisms and the associative property, aspects that are not explicitly addressed in the classic ER model.

%(3) A fascinating connection emerges between the representation of the categorical model and the database normal form theory, which will be elaborated in the subsequent sections.

%It is noteworthy that related works (e.g., \cite{10.1145/111197.111200,10.1145/7474.7475} etc) have also extended the ER diagram to incorporate semantic notions and additional information, such as object-oriented features. An extensive discussion regarding related works on extended ER models with category theory can be found in Section \ref{sec:relatedwork}.






%(1) The categorical ER model in this paper introduces the concepts of pullback, pushout, and limits, which encompass richer semantic constraints compared to the classic ER model. (2) The categorical ER model explicitly defines composable morphisms and the associative property, which are not explicitly addressed in the classic ER model. (3) There is an intriguing connection between the representation of the categorical model and the database normal form theory, which will be described in the following sections.  Note that  related works (e.g. \cite{10.1145/111197.111200,10.1145/7474.7475} ) have also extended the ER diagram to incorporate semantic notions and additional information, such as object-oriented features.  See Section \ref{sec:relatedwork} for more discussion about related works about extended ER models with category theory. 



 

\begin{comment}
\begin{table}
\centering
\begin{tabular}{ |c|c| } 
 \hline
 Classic ER model & Categorical model\\ [0.5ex] 
  \hline  \hline 
 Entity & Entity object \\ 
 \hline 
 Relationship & Relationship object  \\ 
 \hline
 Attributes & Attribute object  \\ 
 \hline 
 No correspondence & Composition \\ 
  \hline
 Key attribute &  Bijective morphism \\ 
 \hline
Function dependency  &  Function morphism \\
\hline
Multivalued dependency  &  Pullback \\
 \hline
Join dependency  &  Limit \\
 \hline
No correspondence  &  Pushout and Colimit \\
 \hline
\end{tabular}
\caption{Comparison between classic ER and categorical ER models}
\label{tab:ERmodel}
\end{table}
\end{comment}

%\begin{figure}\centering\includegraphics[width=0.6\textwidth]{figures/commutativeexample.jpg}\caption{An example to illustrate the commutative diagram.}\label{fig:commutativediagram}\end{figure}





%\begin{figure}\centering\includegraphics[width=0.6\textwidth]{figures/colimitexample.jpg}\caption{An example to illustrate the example of colimit constraint.}\label{fig:colimitexample}\end{figure}


%\begin{figure}\centering\includegraphics[width=0.4\textwidth]{figures/universal.jpg}\caption{Illustration to the universal morphism. }\label{fig:universalmorphism}\end{figure}

%\begin{figure}\centering\includegraphics[width=0.4\textwidth]{figures/loopdiagram.jpg}\caption{A loop diagram to illustrate the commutative diagram. }\label{fig:loopdiagram}\end{figure}










\section{Overview}

\revision{In this section, we first explain the foundational concept of Hausdorff distance-based penetration depth algorithms, which are essential for understanding our method (Sec.~\ref{sec:preliminary}).
We then provide a brief overview of our proposed RT-based penetration depth algorithm (Sec.~\ref{subsec:algo_overview}).}



\section{Preliminaries }
\label{sec:Preliminaries}

% Before we introduce our method, we first overview the important basics of 3D dynamic human modeling with Gaussian splatting. Then, we discuss the diffusion-based 3d generation techniques, and how they can be applied to human modeling.
% \ZY{I stopp here. TBC.}
% \subsection{Dynamic human modeling with Gaussian splatting}
\subsection{3D Gaussian Splatting}
3D Gaussian splatting~\cite{kerbl3Dgaussians} is an explicit scene representation that allows high-quality real-time rendering. The given scene is represented by a set of static 3D Gaussians, which are parameterized as follows: Gaussian center $x\in {\mathbb{R}^3}$, color $c\in {\mathbb{R}^3}$, opacity $\alpha\in {\mathbb{R}}$, spatial rotation in the form of quaternion $q\in {\mathbb{R}^4}$, and scaling factor $s\in {\mathbb{R}^3}$. Given these properties, the rendering process is represented as:
\begin{equation}
  I = Splatting(x, c, s, \alpha, q, r),
  \label{eq:splattingGA}
\end{equation}
where $I$ is the rendered image, $r$ is a set of query rays crossing the scene, and $Splatting(\cdot)$ is a differentiable rendering process. We refer readers to Kerbl et al.'s paper~\cite{kerbl3Dgaussians} for the details of Gaussian splatting. 



% \ZY{I would suggest move this part to the method part.}
% GaissianAvatar is a dynamic human generation model based on Gaussian splitting. Given a sequence of RGB images, this method utilizes fitted SMPLs and sampled points on its surface to obtain a pose-dependent feature map by a pose encoder. The pose-dependent features and a geometry feature are fed in a Gaussian decoder, which is employed to establish a functional mapping from the underlying geometry of the human form to diverse attributes of 3D Gaussians on the canonical surfaces. The parameter prediction process is articulated as follows:
% \begin{equation}
%   (\Delta x,c,s)=G_{\theta}(S+P),
%   \label{eq:gaussiandecoder}
% \end{equation}
%  where $G_{\theta}$ represents the Gaussian decoder, and $(S+P)$ is the multiplication of geometry feature S and pose feature P. Instead of optimizing all attributes of Gaussian, this decoder predicts 3D positional offset $\Delta{x} \in {\mathbb{R}^3}$, color $c\in\mathbb{R}^3$, and 3D scaling factor $ s\in\mathbb{R}^3$. To enhance geometry reconstruction accuracy, the opacity $\alpha$ and 3D rotation $q$ are set to fixed values of $1$ and $(1,0,0,0)$ respectively.
 
%  To render the canonical avatar in observation space, we seamlessly combine the Linear Blend Skinning function with the Gaussian Splatting~\cite{kerbl3Dgaussians} rendering process: 
% \begin{equation}
%   I_{\theta}=Splatting(x_o,Q,d),
%   \label{eq:splatting}
% \end{equation}
% \begin{equation}
%   x_o = T_{lbs}(x_c,p,w),
%   \label{eq:LBS}
% \end{equation}
% where $I_{\theta}$ represents the final rendered image, and the canonical Gaussian position $x_c$ is the sum of the initial position $x$ and the predicted offset $\Delta x$. The LBS function $T_{lbs}$ applies the SMPL skeleton pose $p$ and blending weights $w$ to deform $x_c$ into observation space as $x_o$. $Q$ denotes the remaining attributes of the Gaussians. With the rendering process, they can now reposition these canonical 3D Gaussians into the observation space.



\subsection{Score Distillation Sampling}
Score Distillation Sampling (SDS)~\cite{poole2022dreamfusion} builds a bridge between diffusion models and 3D representations. In SDS, the noised input is denoised in one time-step, and the difference between added noise and predicted noise is considered SDS loss, expressed as:

% \begin{equation}
%   \mathcal{L}_{SDS}(I_{\Phi}) \triangleq E_{t,\epsilon}[w(t)(\epsilon_{\phi}(z_t,y,t)-\epsilon)\frac{\partial I_{\Phi}}{\partial\Phi}],
%   \label{eq:SDSObserv}
% \end{equation}
\begin{equation}
    \mathcal{L}_{\text{SDS}}(I_{\Phi}) \triangleq \mathbb{E}_{t,\epsilon} \left[ w(t) \left( \epsilon_{\phi}(z_t, y, t) - \epsilon \right) \frac{\partial I_{\Phi}}{\partial \Phi} \right],
  \label{eq:SDSObservGA}
\end{equation}
where the input $I_{\Phi}$ represents a rendered image from a 3D representation, such as 3D Gaussians, with optimizable parameters $\Phi$. $\epsilon_{\phi}$ corresponds to the predicted noise of diffusion networks, which is produced by incorporating the noise image $z_t$ as input and conditioning it with a text or image $y$ at timestep $t$. The noise image $z_t$ is derived by introducing noise $\epsilon$ into $I_{\Phi}$ at timestep $t$. The loss is weighted by the diffusion scheduler $w(t)$. 
% \vspace{-3mm}

\subsection{Overview of the RTPD Algorithm}\label{subsec:algo_overview}
Fig.~\ref{fig:Overview} presents an overview of our RTPD algorithm.
It is grounded in the Hausdorff distance-based penetration depth calculation method (Sec.~\ref{sec:preliminary}).
%, similar to that of Tang et al.~\shortcite{SIG09HIST}.
The process consists of two primary phases: penetration surface extraction and Hausdorff distance calculation.
We leverage the RTX platform's capabilities to accelerate both of these steps.

\begin{figure*}[t]
    \centering
    \includegraphics[width=0.8\textwidth]{Image/overview.pdf}
    \caption{The overview of RT-based penetration depth calculation algorithm overview}
    \label{fig:Overview}
\end{figure*}

The penetration surface extraction phase focuses on identifying the overlapped region between two objects.
\revision{The penetration surface is defined as a set of polygons from one object, where at least one of its vertices lies within the other object. 
Note that in our work, we focus on triangles rather than general polygons, as they are processed most efficiently on the RTX platform.}
To facilitate this extraction, we introduce a ray-tracing-based \revision{Point-in-Polyhedron} test (RT-PIP), significantly accelerated through the use of RT cores (Sec.~\ref{sec:RT-PIP}).
This test capitalizes on the ray-surface intersection capabilities of the RTX platform.
%
Initially, a Geometry Acceleration Structure (GAS) is generated for each object, as required by the RTX platform.
The RT-PIP module takes the GAS of one object (e.g., $GAS_{A}$) and the point set of the other object (e.g., $P_{B}$).
It outputs a set of points (e.g., $P_{\partial B}$) representing the penetration region, indicating their location inside the opposing object.
Subsequently, a penetration surface (e.g., $\partial B$) is constructed using this point set (e.g., $P_{\partial B}$) (Sec.~\ref{subsec:surfaceGen}).
%
The generated penetration surfaces (e.g., $\partial A$ and $\partial B$) are then forwarded to the next step. 

The Hausdorff distance calculation phase utilizes the ray-surface intersection test of the RTX platform (Sec.~\ref{sec:RT-Hausdorff}) to compute the Hausdorff distance between two objects.
We introduce a novel Ray-Tracing-based Hausdorff DISTance algorithm, RT-HDIST.
It begins by generating GAS for the two penetration surfaces, $P_{\partial A}$ and $P_{\partial B}$, derived from the preceding step.
RT-HDIST processes the GAS of a penetration surface (e.g., $GAS_{\partial A}$) alongside the point set of the other penetration surface (e.g., $P_{\partial B}$) to compute the penetration depth between them.
The algorithm operates bidirectionally, considering both directions ($\partial A \to \partial B$ and $\partial B \to \partial A$).
The final penetration depth between the two objects, A and B, is determined by selecting the larger value from these two directional computations.

%In the Hausdorff distance calculation step, we compute the Hausdorff distance between given two objects using a ray-surface-intersection test. (Sec.~\ref{sec:RT-Hausdorff}) Initially, we construct the GAS for both $\partial A$ and $\partial B$ to utilize the RT-core effectively. The RT-based Hausdorff distance algorithms then determine the Hausdorff distance by processing the GAS of one object (e.g. $GAS_{\partial A}$) and set of the vertices of the other (e.g. $P_{\partial B}$). Following the Hausdorff distance definition (Eq.~\ref{equation:hausdorff_definition}), we compute the Hausdorff distance to both directions ($\partial A \to \partial B$) and ($\partial B \to \partial A$). As a result, the bigger one is the final Hausdorff distance, and also it is the penetration depth between input object $A$ and $B$.


%the proposed RT-based penetration depth calculation pipeline.
%Our proposed methods adopt Tang's Hausdorff-based penetration depth methods~\cite{SIG09HIST}. The pipeline is divided into the penetration surface extraction step and the Hausdorff distance calculation between the penetration surface steps. However, since Tang's approach is not suitable for the RT platform in detail, we modified and applied it with appropriate methods.

%The penetration surface extraction step is extracting overlapped surfaces on other objects. To utilize the RT core, we use the ray-intersection-based PIP(Point-In-Polygon) algorithms instead of collision detection between two objects which Tang et al.~\cite{SIG09HIST} used. (Sec.~\ref{sec:RT-PIP})
%RT core-based PIP test uses a ray-surface intersection test. For purpose this, we generate the GAS(Geometry Acceleration Structure) for each object. RT core-based PIP test takes the GAS of one object (e.g. $GAS_{A}$) and a set of vertex of another one (e.g. $P_{B}$). Then this computes the penetrated vertex set of another one (e.g. $P_{\partial B}$). To calculate the Hausdorff distance, these vertex sets change to objects constructed by penetrated surface (e.g. $\partial B$). Finally, the two generated overlapped surface objects $\partial A$ and $\partial B$ are used in the Hausdorff distance calculation step.



\section{Closure of FDs and MVDs in Categories}

%In the subsequent sections of this paper, our objective is to construct a framework of the normal form theory for multi-model data.

Given the reliance of relational normal form theory on the computation of closures for FDs and MVDs, this section is dedicated to the  presentation of algorithms designed to efficiently compute closures within categories. 


\subsection{Closure with FDs}
\label{sec:ClosureFD}

In a relational database, given a set $F$ of FDs, numerous other functional dependencies can be inferred or deduced from the FDs in $F$. In the context of category, if we treat arrows as functional dependencies, then existing arrows can imply new arrows. For example, if $X \to Y$, $Y \to Z$, then $X \to Z$ is an implied (composed) arrow. This is exactly the transitivity rule in Amstrong's axioms \cite{armstrong1974dependency}.  Note that there are three inference rules of functional dependencies in Amstrong's axioms. To compute a closure of a category,  we show how all three Amstrong's axioms can be applied to the category as follows:

\noindent FD 1: If $Y \subseteq X$, then $X \to Y$; 
In this case, $Y$ is an object which is a projection object with a relationship object $X$. We add
 a projection arrow between $X$ and $Y$ in the category.

\noindent FD 2: If $f: X \to Y$, then $g: XZ \to YZ$;  Specifically, given an element $(x,z) \in XZ$, let $y = f(x)$, we define that $(y,z) \in YZ$ and $(y,z)$ is the image of $(x,z)$ under the function $g$. we find an alternative interpretation within the framework of \textbf{monoidal category} for this rule.  A detailed discussion can be found in Appendix \ref{sec:monoidal}. 

\noindent FD 3: If $X \to Y$ and $Y \to Z$, then $X \to Z$. This is the composition rule in categories.

%It is important to note that these three rules are real axioms here. That is, they are correct by definition. We define the existence of objects and arrows with these rules.  


%Algorithm \ref{alg:closureFD} describes the main steps to compute the closure of a category with FDs. In Line  1, the graph $G$ of this category is transformed into a set of functional dependencies. For every relationship object $X$ in $G$, two new functional dependencies are included: $X \to A_1, \ldots, A_n$ and $A_1, \ldots, A_n \to X$, where $X$ is associated with  $A_1, \ldots, A_n$. Additionally, other arrows in $G$ are converted into corresponding functional dependencies. Thus, apart from the functional dependencies for relationship objects, all other arrows in $G$ have a single attribute on the LHS. Subsequently,  Line 4 computes the closure $X^+$ for each LHS $X$ by using Amstong's axioms. Finally,  the associated arrows and objects are added based on $X^+$. 

\begin{definition} Given a graph representation $G$ of a category and a set of functional dependencies $F$, any inferred functional dependency $X \to Y$ is considered \textbf{relevant} to $G$ and $F$ if $X$ is either an object in $G$ or appears on the left-hand side (LHS) of a functional dependency in $F$, and $Y$ is an object in $G$.
\end{definition} 

This criterion of relevant functional dependencies ensures that the closure consists solely of the pertinent functional dependencies with respect to $G$.


\begin{definition} 
Given a graph representation $G$ of a category and a set of functional dependencies $F$, the relevant closure representation of $F$ and $G$ denoted as $(G,F)^+$ is a graph where all arrows represent the set of all relevant functional dependencies that can be inferred from $F$ and $G$.

\end{definition}

A schema category $\mathcal{C}$ can be represented as a directed graph, which consists of vertices and arrows, where the vertices correspond to the objects and the arrows indicate the function dependency between these objects. Given a graph representation $G$ of a category and a set of functional dependencies $F$, Algorithm \ref{alg:closureFD} outlines the key steps to compute the closure representation $(G,F)^+$, where each relevant functional dependency is represented as an arrow.







%The analysis of time complexities of algorithms can be found in Appendix \ref{sec:complexity}.

In Line 1 of Algorithm \ref{alg:closureFD}, the graph $G$ is transformed into a set of functional dependencies. Any arrow from $X$ to $Y$ in $G$ is converted to the corresponding functional dependency $X \to Y$. In addition, for each relationship object $X$ in $G$, create two new functional dependencies: $X \to A_1, \ldots, A_n$ and $A_1, \ldots, A_n \to X$, where $A_1, \ldots, A_n$ are projection objects of $X$.   As a result, except for the functional dependencies associated with relationship objects, all other arrows have a single attribute on LHS.  In Line 2, the set $D$ contains all functional dependencies from the category and $F$. Subsequently, for each LHS $X$ of $D$, the closure attributes $X^+$ are computed using Amstong's axioms and the corresponding inferred function dependencies are added in $G$ as arrows (Lines 3-9).

%in Lines 3-4, the closure $X^+$ for each LHS $X$ is computed using Amstong's axioms.  Finally, based on $X^+$, the associated arrows and objects are added, completing the closure graph $(G,F)^+$, where each arrow denotes a valid function dependency.

%This step ensures that all FDs implied by the given set of FDs are identified.


%This process guarantees that all necessary arrows and objects are included to preserve the integrity and logical structure of the category. By following Algorithm \ref{alg:closureFD}, we can systematically compute the closure of a category, ensuring the preservation of functional dependencies and the accurate representation of its underlying relationships.

%Note that in the above algorithm, to compute a closure  $G^+$, not all FDs that can be inferred are present as arrows in $G^+$. Only those FDs $X \to Y$, where $X$ exists in $G$ or $F$ and $Y$ exists in $G$ should be inserted in $G^+$.

 

%preventing the inclusion of extraneous or unrelated information. 



\begin{algorithm}
\caption{Computing the Closure of Categories with FD}
\label{alg:closureFD}
\KwIn{A graph representation $G$ and  a set of functional dependencies $F$ } 
\KwOut{The relevant closure representation  $(G,F)^+$} 
\DontPrintSemicolon

 Convert $G$ to a set of functional dependencies denoted by $FD(G)$;
 
 $D = FD(G) \cup F$;
 
\ForEach{  $X \in LHS(D)$} 
{
    \If{$X \notin G$}
        {
        add $X$ in G; 
        }
    
    Compute the closure attributes of $X$: giving $X^+$;

   \ForEach{  $Y \in X^+$ and $Y \in G$} 
    {
     \If{ there is no arrow from $X$ to $Y$ in $G$}
        {
        add an arrow from $X$ to $Y$ in $G$;
        }
    }
    
        
}

\end{algorithm}


\begin{figure}\centering\includegraphics[width=0.7\textwidth]{figures/1RRExample2.png}\caption{This example illustrates the compuation of closure and  1RR.}\label{fig:1RRExample}\end{figure}

\begin{example} Figure \ref{fig:1RRExample}(a-b) shows an example to illustrate  Algorithm \ref{alg:closureFD}. Figure \ref{fig:1RRExample}(a) is the input of a category and a FD: $B \to C$.  In Line 1, $FD(G)$ includes four FDs:  $D$$\to$$E$, $D$$\to$$A$, $A$$\to$$B$ and $A$$\to$$C$.  In Line 2, $D = FD(G) \cup \{B \to C\}$. In Line 9, three new arrows are inserted, i.e. $D \to C$, $B \to C$ and $D \to B$ shown in Figure \ref{fig:1RRExample}(b). Note that Figure \ref{fig:1RRExample}(c) will be explained later in Section 
\ref{subsec:1RR}. \end{example}



%The time complexity of Algorithm \ref{alg:closureFD} is dominated by the cost to compute the closure of the functional dependencies (i.e. Line 4). Let $m$ and $n$ be the number of objects and arrows in $G$, and let $d$ be the number of FDs. The total time for computing the FD closure is $O((d+n) \cdot m)$ based on the implementations of the previous work (e.g. \cite{10.1145/320493.320489}). Using this procedure, one can implement Algorithm \ref{alg:closureFD} with the time $O((d+n) \cdot m)$.

\subsection{Closure with FDs and MVDs}

%Recall that the definition of MVD $X \to\to Y$ involves "a universal set U", as defined in Section 2.4. The reason is that the validity of  $X \to\to Y$ depends on all other attributes associated with this relation. In the setting of most database papers,  $U$ is fixed. However, in a category of this paper,  $U$ could be various relationship objects which contain $X$ and $Y$. Thus the specification of $U$ is an integral part of an MVD. We  use the notation $X \to\to_U Y$ to stress the fact that the MVD involves the set $U$.


In database theory, an MVD $X \to\to Y$ is defined in terms of  ``\textit{a universal set U}''. The reason is that the validity of  $X \to\to Y$ depends on all other attributes associated with this relation $U$.  Unlike previous relational database settings that maintain a fixed $U$, the category setting here allows for different relationship objects having both $X$ and $Y$. As a result, the specification of $U$ becomes an indispensable component of an MVD. To underscore the importance of including the set $U$ in an MVD, we adopt the notation $X \to\to_U Y$ throughout the rest of this paper. This notation serves as a reminder that an MVD is intrinsically tied to the object $U$ within the context of this category.


%Section \ref{sec:mvd} demonstrates the connection between multivalued dependencies and pullback in categories. Recall that the definition of MVD $X \to\to Y$ involves "\textit{a universal set U}". The reason behind incorporating this universal set $U$ lies in the fact that the validity of $X \to\to Y$ is contingent upon all other attributes associated with this relation. While most previous works maintain a fixed $U$, the category presented in this paper allows for various relationship objects that contain both $X$ and $Y$. Consequently, the specification of $U$ becomes an indispensable component of an MVD. To emphasize the inclusion of the set $U$ in an MVD, throughout the rest of this paper, we adopt the notation $X \to\to_U Y$. This notation serves as a reminder that the MVD is closely linked to the set $U$ within the context of this category. 

%While a functional dependency is presented as an arrow in a category, a multivalued dependency is manifested with an \textit{MVD object} in a category, which serves as the universal set $U$ of this dependency, as defined below:


 While a functional dependency is represented by an arrow, a multivalued dependency is manifested through an \textit{MVD object} in the context of categories. This MVD object works as the universal set $U$ for the corresponding dependency, as defined below:

\begin{definition} (\textbf{MVD objects}) Given a category $\mathcal{C}$, a relationship object $O \in C$ is called an MVD object if there is a multivalued dependency $X \to\to_O Y$ over $C$, such that  $X \cup Y \subset \pi(O) $,  where $\pi(O)$ denote the set of projection objects of $O$. \label{def:MVD}
\end{definition}

%Given a set of functional dependencies $F$ and a set of multivalued dependencies $M$, the inference rules to compute their closure can be found in literature, see e.g. \cite{10.5555/551350,10.1145/320613.320614,10.1145/509404.509414}. We provide those inference rules in Appendix \ref{sec:monoidal}. 

Given a set of functional dependencies \( F \) and a set of multivalued dependencies \( M \), the inference rules to compute their closure are well-documented in the literature (see, e.g., \cite{10.5555/551350,10.1145/320613.320614,10.1145/509404.509414}). For the convenience of readers, we have included these inference rules in Appendix \ref{sec:MVDInference}.

%In order to compute the closure of FDs and MVDs, one needs to compute the \textit{dependent basis} of an attribute $X$, denoted by $DEP(X)$. An MVD $X \to \to_U Y $ is in the closure if and only if $Y$ is a union of elements of $DEP(X)$. The algorithm to compute the dependent basis can be found in previous works, e.g.  \cite{10.1145/320613.320614}. Note that $X^+$  and $DEP(X)$ are indeed parallel concepts.  If we think of the collection of singleton sets, \{$(A)| A \in X^+$\}, then $X^+$ is just $DEP(X)$ that is functionally dependent on $X$ by FDs.
  


%\begin{definition} Given a graph representation $G$ of a category and a set of functional dependencies $F$ and  a set of multivalued dependencies $M$ whose RHS are objects in $G$, an MVD $X \to\to Y$ is considered \textbf{relevant} if $X$ is either an object in $G$ or appears on the LHS of a FD in $F$ or a MVD in $M$, and $Y$ is an object in $G$.\end{definition} 

\begin{definition} 
Given a graph representation $G$ of a category and a set of functional dependencies $F$ and  a set of multivalued dependencies $M$, the relevant closure representation of  $G, F, M$ denoted as $(G,F,M)^+$ is a graph where all arrows represent the relevant FDs that can be inferred from $F,G$ and $M$, and all MVD objects are identified based on the MVDs that can be inferred from $F,G$ and $M$.

\end{definition}

%Given a graph representation $G$ of a category $C$ with both a set of functional dependency $F$ and a set of multivalued dependency $M$, the computation of a closure of $G^+$ under $F$ and $M$ involves the iterative applying of inference rules as follows:

%In the context of a category $C$ represented by a graph $G$, where $(G,F,M)^+$ denotes the closure of $G$ under both a set of functional dependencies $F$ and a set of multivalued dependencies $M$, the process of computing the closure involves the iterative application of the inference rules as follows:



\begin{algorithm}
\caption{Computing the Closure of Categories with FD and MVD}
\label{alg:closureMVD}
\KwIn{A graph representation $G$, a set of functional dependencies $F$ and a set of multivalued dependencies $M$} 
\KwOut{The relevant closure representation of $(G,F,M)^+$} 
\DontPrintSemicolon

 Convert $G$ to a set of functional dependencies denoted by $FD(G)$;
 
$D$ = $FD(G) \cup F \cup M$;
 
\ForEach{  $X \in LHS(D)$} 
{
        \If{$X \notin G$}
        {
        add $X$ in $G$;
        }
        

        Compute the FD closure attributes of $X$ based on $D$: giving $X^+$;

        \ForEach{  $Y \in X^+$ and $Y \in G$} 
        {
            \If{$X \to Y $ is not an arrow in $G$}
            {
            add an arrow from $X$ to $Y$ in $G$;
            }
        }

       Identify all MVD objects in $G$ based on the inferred MVDs from $D$;
}

\end{algorithm}



%It seems that $X^+$  and $DEP(X)$ are different, as the former is a set of attributes, while the latter is a collection of sets of attributes. However, if we think of the collection of singleton sets, \{$(A)| A \in X^+$\}, then it is just the basis of the collection of sets that are functionally dependent on X by an FD. Thus,  $X^+$  and $DEP(X)$ are parallel concepts. 

%Let S be a collection of sets closed under union, intersection, and difference. The collection S contains a unique subcollection T of nonempty, pairwise disjoint sets such that every element of S is a union of some elements of T. The collection T is called the basis of S. We call the basis of this collection the dependent basis of X and denote it by DEP(X). By the definition, and an MVD $X \to \to Y $ is in G+ if and only if Y is a union of elements of DEP(X).  It seems that $X^+$  and $DEP(X)$ are different, as the former is a set of attributes, while the latter is a collection of sets of attributes. However, if we think of the collection of singleton sets, \{$(A)| A \in X^+$\}, then it is just the basis of the collection of sets that are functionally dependent on X by an FD. Thus,  $X^+$  and $DEP(X)$ are parallel concepts. To accomplish this, it is necessary to compute the dependent basis for each left-hand side (LHS) of multivalued dependencies (MVDs). The procedure for calculating the MVD dependency basis is provided in the appendix, drawing from the methodology presented in the paper \cite{10.1145/320613.320614}.





%\begin{definition} (Explicit pullback) Given a category C, a relationship object $O \in C$ is an explicit pullback if there are two objects $O_1, O_2 \in C$, such that  $O_1 = \{XY\}  $ and $O_2 = O -  \{XY\}$. \end{definition}

%In the above definition, $O$ is a universal object to witness the MVD $X \to\to Y$, so we require $\pi(O) - \{XY\} \neq \emptyset$.


%Algorithm \ref{alg:closureMVD} outlines the key steps involved in computing the closure of a category with FDs and MVDs. The purpose of this algorithm is not only to find all related inferred arrows $X \to Y$, but also to identify all MVD objects $O$ in $(G,F,M)^+$.

%When computing the closure $(G,F,M)^+$, not all inferred FDs and MVDs must be included in the returned graph representation. Instead, only those FDs and MVDs of the form $X \to Y$ are added to $(G,F,M)^+$, where the object $X$ exists in either $G$, the LHS of $F$, or $M$, and $Y$ exists in $G$.

   

%The main objective of this algorithm is not only to identify the inferred arrows, but also to pinpoint all MVD objects in $(G,F,M)^+$.


Algorithm \ref{alg:closureMVD} presents the essential steps for calculating the relevant closure representation with functional dependencies $F$ and multivalued dependencies $M$. The algorithm proceeds as follows: In Line 1, the graph $G$ is transformed into a set of FDs. Line 2 initializes a set $D$ to include all the FDs and MVDs. Lines 6 are responsible for computing $X^+$  for each $X \in LHS(D)$ by using $G, F$ and $M$. Subsequently, the new arrows are added, and the MVD objects are identified in Lines 9-10 based on the inferred FDs and MVDs. 

%It is important to emphasize that the closure computation does not involve inserting any new MVD objects, but rather focuses on identifying the relationship objects that satisfy the definition of MVD objects. 


%The sound and complete identification of MVD objects is crucial in obtaining a reduced representation for categories in the subsequent section.


\begin{figure}\centering\includegraphics[width=0.8\textwidth]{figures/2RRExample3.png}\caption{This example illustrates the closure and 2RR.}\label{fig:2RRExample}\end{figure}


\begin{example}  Figure \ref{fig:2RRExample} (a-b) show an example to illustrate Algorithm \ref{alg:closureMVD}. Fig \ref{fig:2RRExample}(a) is the input with one MVD $A \to\to_X B$. In Line 1, the graph is converted to five FDs. In Line 2, the set $D$ includes five FDs and one MVD.  In Line 9, the arrow $A \to C$ is added due to $A \to\to_X B$, $A \to\to_X CD$ and $B \to C$ (by the inference rule FD-MVD 2). In Line 10, $X$ is identified as an MVD object due to $A \to\to_X B$ and $A \to\to_X CD$, shown in Figure \ref{fig:2RRExample} (b). 
\label{exp:CloureMVDExample}
\end{example}

%$X$ is found as a derivable MVD object due to $A \to\to_X D$.



%The time complexity of the algorithm is essentially that of the computation of the closure of FDs and MVDs. Let $m$ be the number of objects in $G$, $n$ the number of arrows in $G$, and let $d_1$ and $d_2$ be the number of FDs and MVDs, respectively. The total time for computing the FD closure is $O((d_1+n) \cdot m)$ and MVD closure is $O((d_2+n) \cdot m^3)$ based on the implementations of papers  (\cite{10.1145/320493.320489, 10.1145/320613.320614}. Using these procedures, one can implement Algorithm \ref{alg:2RR} with the time $O(d_1+d_2+n) \cdot m^3)$.


%Algorithm \ref{alg:closureMVD} describes the main steps to compute the closure of a category with FDs and MVDs.  In Line  1, the graph $G$ is transformed into a set of functional dependencies.  Then all arrows, FDs and MVDs are included in $D$ in Line 2.  Lines 4 and 5 compute  $X^+$ and $DEP(X)$ for each $X$ respectively. Subsequently, the associated arrows are added using $X^+$ and the MVD objects are identified using $DEP(X)$.  Note that in this algorithm,  no new MVD objects are inserted, but found. The correct and complete  identification of MVD objects is necessary to find a reduced representation for categories in the next section.

%\begin{definition}A graph representation $G$ of a category  is said to cover another graph representation $G'$ if every arrow  in $G'$ is also in $G^+$; that is, if every arrow in G' can be inferred from G. \end{definition}

%In the above definition, every arrow in $G'$ is also in $G^+$, meaning that the two objects associated with this arrow should be also included in $G^+$.

%\begin{definition}Two graph representation $G$ and $G'$ are equivalent if $G$ covers $G'$ and $G'$ covers $G$. \end{definition}

%The equivalences of two graph representations of categories will be used in the next section to define  a reduced (compact) representation of categories that leads to normal form theory for categories. 

\section{Representations of Categories}





%\begin{figure}\centering\includegraphics[width=0.8\textwidth]{figures/ex01.jpg}\caption{This example illustrates the main idea to produce three schemas of different data. Figure B illustrates that the transitive dependency will not be materialized in a graph.  Since student $\rightarrow$ Department and  Department  $\rightarrow$ DepartmentHead, student $\rightarrow$ DepartmentHead. Therefore, there is no edge between student and DepartmentHead.}\label{fig:exampleAB}\end{figure}

%\begin{figure}\centering\includegraphics[width=0.8\textwidth]{figures/composedarrows.jpg}\caption{Illustration to removing the composed arrows. }\label{fig:composedarrows}\end{figure}

%\begin{figure}\centering\includegraphics[width=0.6\textwidth]{figures/exa03.jpg}\caption{This example illustrates the fifth normal form. We do not need the SPJ table.  This example shows that SPJ is a join limit from three relation SP, PJ and SJ, which will generate the SPJ relation by three table joins.  This limit says that whenever a supplier s supplies part  p, and a project j uses part p, and the supplier s supplies at least one part to project j, then supplier will also be supplying part p to project j. }\label{fig:exampleD}\end{figure}

%\begin{figure}\centering\includegraphics[width=0.6\textwidth]{figures/SPJData.jpg}\caption{This figure shows the example data for the join of three tables, then the last table is the limit table.}\label{fig:SPJData}\end{figure}

%The categorical framework provides a unified view of multi-mode data. In this section, we demonstrate the usefulness of this framework to output the normalized schema for different types of data.


%including composed morphisms. In this paper, we only consider a category with a finite number of objects and arrows, but each object may contain an infinite number of elements \cite{hirst1993completeness}, which is connected with some application areas that involve streaming or temporal data.  since a complete representation is often unnecessary as some arrows (e.g. composed arrows) or objects are derivable from others. 

%\begin{figure}\centering\includegraphics[width=0.7\textwidth]{figures/ideas.jpg}\caption{Illustration to then main workflow for the optimization and output }\label{fig:ideas}\end{figure}

%Given a category $\mathcal{C}$, its representation is a directed graph, denoted as $\mathcal{G}(C)$, consisting of vertices and arrows, where the vertices correspond to the objects and the arrows indicate the functions between these objects (sets). A representation may not necessarily include  all the objects and morphisms in a given category. For instance, the composed arrows and identity arrows may be ignored in a representation without losing any information. In this section, we leverage the inference rules of functional dependencies and multivalued dependencies to derive two levels of reduced representation for categories. This reduced representation is a condensed representation that intentionally ignores some selected objects and morphisms with the properties of both sound and complete, meaning that it can accurately capture or derive all the morphisms and objects of $\mathcal{C}$. Later  we then can establish a connection between the reduced representation and the hierarchy of the database normal form theory, as discussed in Section 5.

%In this section, we will build the link between the reduced representation of categories and the normal form theory of databases, which presents an innovative  approach to define the  principles of normal forms in multi-model databases.

%This endeavor unfolds an innovative  approach to define the  principles of normal forms in multi-model databases.


%In the following three sections, we aim to establish a connection between the reduced representations of categories and the framework of normal forms in databases. To achieve this, 

In this section, we design two levels of reduced representations for categories. 


%as illustrated in Figure \ref{fig:reduced}, which serve as a condensed form by eliminating derivable objects and morphisms. 

%We will explicitly show the connection of these two reduced representations and normal form theory in databases in Section 6.  


%As mentioned in Section 2.1, strictly speaking, a category is beyond a graph with composed arrows. However, for the purpose of visual representation, we still can use a directed graph $G$ to represent a category $\mathcal{C}$ by omitting the composed arrows.  

%$G$ may not necessarily include all the objects and morphisms in $\mathcal{C}$. For example,

%Further,    identity arrows can also be omitted from the representation without any loss of information.

%A representation $G$ of a category $\mathcal{C}$ can be a directed graph. It is important to note that $G$ may not necessarily include all the objects and morphisms in $\mathcal{C}$. For example, composed arrows and identity arrows can be omitted from the representation without any loss of information. In this section, we derive two levels of reduced representations for categories, as illustrated in Figure \ref{fig:reduced}. These reduced representations serve as a condensed form intentionally ignoring redundant objects and morphisms. 

%We establish an intriguing connection between the reduced representations and the hierarchy of the database normal form theory, as discussed in Section 5. This connection sheds light on a new perspective of multi-model database normal form theory through the lens of category theory. 

%For the sake of simplicity, certain elements may be excluded from a representation without compromising the information it conveys. For example, composed arrows and identity arrows can be omitted from the representation without any loss of information. This selective omission allows for a more concise depiction of the category.




%Despite the omissions, it possesses the properties of being both sound and complete. This implies that the reduced representation accurately captures or derives all the morphisms and objects of $\mathcal{C}$.



%By developing a reduced representation for categories and exploring its correlation with database normal form theory, we advance our comprehension of category theory and its practical applications in the realm of databases.


%Given a category, we may convert it to a complete graph representation, where all objects and morphisms correspond to the nodes and vertexes in the graph. It is feasible as the limited number of objects and the single morphism between any two objects. 

%However, it is often not feasible to enumerate all the objects and morphisms in a given category. For instance, in a monoidal category, the presence of a loop morphism may lead to an infinite number of composed morphisms. To address this issue, this section focuses on studying a reduced representation $\mathcal{R}$ of a category $\mathcal{C}$.




%Furthermore, in relational databases, given a relational schema and a set of constraints (e.g. functional dependency and join dependency), there are different algorithms to decompose this relation to a good schema to satisfy different levels of normal forms. Analogously, given a category, we will introduce three levels of reduced representations, as illustrated in Figure \ref{fig:reduced}. These three levels enable the creation of a hierarchical structure of concise representations for categories. The significance of this hierarchical structure becomes evident when we establish a connection between it and the hierarchy of the database normal form theory, as discussed in Section 5. 

%This intriguing connection sheds light on the relationship between the reduced representations of categories and the normal form theory,

%that enable the creation of a hierarchy of concise representations of categories,   We will establish a fascinating connection between this hierarchy of reduced representations and the hierarchy of the normal form theory in Section 5.    


%A complete enumeration of  all objects and morphisms in a $\mathcal{C}$ is sometimes impossible. For example, consider the loop morphism in a category with an object (i.e. a monoidal category  ), there are possibly infinite numbers of composed morphisms.

%In this paper, we study a reduced (concise) representation $\mathcal{R}$ of a category $\mathcal{C}$, which is sound and complete, in the sense that all morphisms and objects of $\mathcal{C}$ can be correctly derived from $\mathcal{R}$. 


%Enhance with injective, surjective properties:FD 4: If $X \to Y$ and $X \to Z$, and $X \to Y$ is a subjective function, then $X \to Z$. FD 5: If $X \to Z$ and $Y \to Z$, and $Y \to Z$ is an injective function, then $X \to Y$.



\subsection{First Reduced Representation}
\label{subsec:1RR}

%In a relational database setting, given a set F of FDs, numerous other functional dependencies can be inferred or deduced from the FDs in F. We call them implied functional dependencies. There are three well-known inference rules for functional dependencies, called Amstrong's axioms: FD 1: If $Y \subseteq X$, then $X \to Y$; This says that every two objects which have containment relationship, then there is a projection arrow between them. FD 2: If $X \to Y$, then $XZ \to YZ$; In particular, given an element $(x,z) \in XZ$, we define that $(y,z) \in YZ$ and $(y,z)$ is the image of   $(x,z)$. FD 3: If $X \to Y$ and $Y \to Z$, then $X \to Z$. This is exactly the transitivity rule in categories.  It is important to note that these three rules are real axioms here. That is, they are correct by definition, not proof. We define the existence of objects and arrows with these rules.  


%There are three levels of cover between two graph representation G1 and G2

%1. Level 1 cover: A graph representation $G_1$ is said to cover another graph representation $G_2$ by level 1 if every arrows in $G_2$ can be inferred from $G_1^+$.

%2. Level 2 cover: A graph representation $G_1$ is said to cover another graph representation $G_2$ by level 2 if every arrows in $G_2$ can be inferred from  $(G_1 \cup L)^+$, where L denotes the restorable limit and colimits objects in $G_2$.

%3. Level 3 cover: A graph representation $G_1$ is said to cover another graph representation $G_2$ by level 3 if every arrows in $G_2$ can be inferred from  $(G_1 \cup L \cup A)^+$, where L denotes the restorable limit and colimits objects in $G_2$ and A denotes all the restorable arrows.

% which corresponds to Boyce-Codd normal form (BCNF) in relational normal form theory. 



\begin{definition}
Given a set of functional dependencies $F$,  a graph representation $G$ of a category $\mathcal{C}$ is said to cover another graph representation $G'$  if every arrow in $G'$ is also in $(G,F)^+$; that is, if every arrow in $G'$ can be inferred from $G$ and $F$. 
\end{definition}

\begin{definition}
    Given a set of functional dependencies $F$, two graph representations $G$ and $G'$ are equivalent if $G$ covers $G'$ and $G'$ covers $G$. 
\end{definition}

 Roughly speaking, the computation of a First Reduced Representation (1RR) is similar to computing a minimal (canonical) cover of functional dependencies,  wherein an equivalent representation with the minimum number of arrows is sought. Algorithm \ref{alg:1RR} describes the key steps involved in generating the 1RR. The inputs to this algorithm are a graph representation $G$ of a category and a set of functional dependencies $F$ in the canonical form. The output is the 1RR.

% The first step is to compute a closure category $(G,F)^+$, which is accomplished through Algorithm \ref{alg:closureFD} (Line 1). The closure category ensures that all relevant functional dependencies that can be inferred from the original ones are explicitly included. Next, we proceed to remove any redundant arrows from the graph while ensuring that the remaining graph remains equivalent (Lines 2-4). When an arrow $f: X \to Y$ is removed, two cases arise: (1) If the arrow $f$ is a projection arrow, the definition of the relationship object $X$ involving $f$ is altered by removing an extraneous object $Y$ from $X$, or (2) if the arrow $f$ is not a projection arrow, it is simply removed as it can be derived from other existing arrows.


 
 
 %Algorithm \ref{alg:1RR} describes the main steps to generate the 1RR. The input is a graph representation $G$ of a category and a set of functional dependencies with a canonical form. First, a closure category $G^+$ is computed by Algorithm \ref{alg:closureFD}. Second, any arrow is removed if the remaining graph after the removal is still equivalent. When an arrow $f: X \to Y$ is removed, there are two cases: (1) $X$ is a relationship object and this arrow is a projection arrow, then the definition of the relationship object $X$ is changed, an extraneous object $Y$ is removed from $X$, or (2) this arrow is not a projection arrow. This step removes a redundant arrow. 


 
%We will now proceed with a detailed walkthrough of the algorithm. In Step 1, a closure category $G_1$ is computed by Algorithm \ref{alg:closureFD}. In Step 2, the modified graph $G_1$ is transformed into a set of functional dependencies. For every relationship object $X$ in $G_1$, two new functional dependencies are included: $X \to A_1, \ldots, A_n$ and $A_1, \ldots, A_n \to X$, where $A_i \in \pi(X)$. Additionally, other arrows in $G_1$ are converted into corresponding functional dependencies. Consequently, apart from the functional dependencies for relationship objects, all other arrows in $G_1$ have a single attribute on the LHS. In Step 3, a minimal cover for all the functional dependencies is computed (\cite{elmasri2000fundamentals}). Finally, in Step 4, the computed minimal cover is transformed into a new graph representation.  Each functional dependency, represented as $f: X \to Y$, is translated into an arrow connecting the sets $X$ and $Y$ in the graph.  When the left-hand side (LHS) of a functional dependency $X$ consists of multiple attributes, e.g., $X = A_1, \ldots, A_n \to Y$, a new relationship object $X$ is introduced in $G_0$. Projection arrows represented as $X \to A_i$ for each $X_i$ are also added, along with the arrow $X \to Y$. 







%This process follows a similar approach to Step 1, where functional dependencies are translated into arrows connecting the appropriate objects.




%A canonical form of a functional dependency is obtained by decomposing it into a set of minimal cover, irreducible FDs and ensuring that the right-hand side of each FD contains only a single attribute. 

%In the sequel,  $X$  denotes a relationship object, which is associated with multiple attributes $A_1,...,A_n$.

 %We now go through the algorithm.  Step 1 adds all functional dependencies into $G_0$. Each FD $f: X \to Y$ is represented as an arrow between $X$ and $Y$. If $X$ or $Y$ does not occur in $G_0$, then add new objects and arrows correspondingly.   If the left-hand side (LHS) of $X$ has multiple attributes, say $ X=X_1,...,X_n \to Y$, then create a new relationship object $X$ in $G_0$, add all projection arrows $X \to X_i$ for each $X_i$, and add the arrow $X \to Y$. Step 2 converts the new $G_1$ into a set of FDs, For each relationship object $X$ in $G_1$, add two new FDs: $X \to X_1,...,X_n$ and $X_1,...,X_n \to X$.  And other arrows are converted to FDs correspondingly. Therefore, except for the FDs for relationship objects, all other arrows have a single attribute in LHS. Step 3 computes a minimal cover for all FDs.  An algorithm to compute minimal cover is offered in the appendix based on the approach in \cite{elmasri2000fundamentals}  for the sake of self-containment of this paper. Finally, in Step 4, this minimal cover is converted to a new graph representation.  The approach is similar to that of Step 1. 
 
 
 %For any rule with multiple attributes in LHS, say  $X_1,X_2,...X_n \to X$, then $X$ is defined as a relationship object for $X_1,X_2,...,X_n$. The following lemma shows that the definition of each relationship object $X$ is unique.    

%xcept for the FD involving the definition of a relationship object, the LHS of all other FDs has a single attribute, and those FDs are represented as arrows in the returned graph.  


 

% \begin{lemma} In the output of a set of FDs in Line 3 of Algorithm \ref{alg:1RR}, for each FD with multiple attributes in LHS, the corresponding RHS is different.  \end{lemma}

%\begin{proof} Because there is only one original rule whose RHS is X and LHS has multiple elements. It is possible that the LHS is reduced, but there is only one rule about X. Thus, we can define X uniquely.\end{proof}


 
 %In a relational database setting, given a set F of FDs, numerous other functional dependencies can be inferred or deduced from the FDs in F. We call them implied functional dependencies. There are three well-known inference rules for functional dependencies, called Amstrong's axioms: FD 1: If $Y \subseteq X$, then $X \to Y$; This says that every two objects which have containment relationship, then there is a projection arrow between them. FD 2: If $X \to Y$, then $XZ \to YZ$; In particular, given an element $(x,z) \in XZ$, we define that $(y,z) \in YZ$ and $(y,z)$ is the image of   $(x,z)$. FD 3: If $X \to Y$ and $Y \to Z$, then $X \to Z$. This is exactly the transitivity rule in categories.  It is important to note that these three rules are real axioms here. That is, they are correct by definition, not proof. We define the existence of objects and arrows with these rules.  A canonical cover is a set of functional dependencies that is minimal and equivalent to the original set of functional dependencies.




 

%Note that if RHS has multiple elements, then this rule is at most one as it is a minimal cover and the number of the original rules is at most one. 





%Given a set of functional dependencies, if the LHS X contains multiple attributes, if X appears in category C, replace X with a single attribute. Otherwise, create a new relationship object for X, and create the projection morphisms: $X \to X_i$ in C.


%Here are the steps to obtain a canonical cover: 1. Start with the original set of functional dependencies.2. Decompose any functional dependency that contains multiple attributes on the right-hand side into individual functional dependencies. For example, if you have $A  \to BC$, decompose it into $A \to B$ and $A \to C$.3. Eliminate any redundant dependencies. A dependency $X \to Y$ is redundant if Y can be derived from X and other existing dependencies. Remove the redundant dependencies while maintaining the original dependencies. 4. Repeat steps 2 and 3 until no further decomposition or elimination of dependencies is possible. The resulting set of dependencies is the canonical cover.


%\begin{algorithm}\caption{Level 1 cover}\label{alg:level1cover}\input{algorithms/level1cover}\end{algorithm}

%\begin{algorithm}\caption{Level 2 cover}\label{alg:level2cover}\input{algorithms/level2cover}\end{algorithm}


%\begin{algorithm}\caption{Level 3 cover}\label{alg:level3cover}\input{algorithms/level3cover}\end{algorithm}









%G1 is equivalent to G2 if and only if G1 covers G2 and G2 covers G1. We may also say G1 is equivalent to G2 by level 1, 2 or 3 inference.




%Given a representation G, if G' is the first reduced representation and G' is equivalent to G by level 2 inference, and G' has the minimal number of limit and colimit objects, then G' is the second reduced representation of G.


%A category without (co)limit and commutative diagram constraints can be represented with a set of functional dependencies. Each arrow in a schema category can be represented with a functional dependency. Each isolated object without morphism may be represented as a trivial FD: $X \to X$. Similarly a graph can be also represented as a set of FDs. A graph representation G of C is the first reduced representation if $FD(G)$ is a minmal cover of $FD(C)$.


%Intuitively, the first reduced representation of a category $\mathcal{C}$ determines a graph $\mathcal{G(C)}$  with the same objects and arrows, forgetting the identities and  all arrows (functions) that can be composed.
%Intuitively, the first reduced representation of a category $\mathcal{C}$ determines a graph $\mathcal{G(C)}$  with the same objects and arrows, forgetting the identities and  all arrows (functions) that can be composed.

%An algorithm to compute the minimal cover:https://www.inf.usi.ch/faculty/soule/teaching/2014-spring/cover.pdf


%\begin{definition} (First Reduced Representation) A graph representation $\mathcal{G}$ of a category $\mathcal{C}$ is in the first reduced representation (1RR) if and only if  all identity arrows and composed arrows are removed. In other words, 1RR keeps only atomic arrows and projection arrows. \label{def:1RR}\end{definition} 

%\begin{definition} (minimal cover) A graph representation $\mathcal{G}$ of a category $\mathcal{C}$ is a minimal cover of $\mathcal{C}$ if the moval of any arrows will not equivalent to $\mathcal{C}$.\label{def:mincover} \end{definition} 

%\begin{definition} (First Reduced Representation) A graph representation $\mathcal{G}$ of a category $\mathcal{C}$ is in the first reduced representation (1RR) if and only if  $\mathcal{G}$ is a minimal cover of $\mathcal{C}$. \label{def:1RR}\end{definition}

%\begin{definition} (First Reduced Representation) A graph representation $\mathcal{G}$ of a category $\mathcal{C}$ with functional dependencies is in the first reduced representation (1RR) if and only if it is a canonical cover of $\mathcal{C}$. We can use the following algorithm to find at least one first reduced representation.\label{def:1RR2}\end{definition} 

%Given a representation G, if G' is equivalent to G by level 1 inference, and G' has the minimal arrow cover and minimal relationship width, then G' is the first reduce representation of G.



%\begin{algorithm}\caption{Computing the complete representation}\label{alg:complete1}\input{algorithms/computecomplete1}\end{algorithm}

\begin{algorithm}
\caption{Computing the First Reduced Representation (1RR)}
\label{alg:1RR}
\KwIn{A graph representation $G$ and  a set of functional dependencies $F$} 
\KwOut{The first reduced representation of $G$} 
\DontPrintSemicolon

 Use Algorithm \ref{alg:closureFD} to compute a closure graph $(G,F)^+$;

\ForEach {arrow  $f: X \to Y$ in $(G,F)^+$} 
{
    \If{($(G,F)^+ - f)$ is equivalent to $(G,F)^+$}
    {
    Remove $f$ in $(G,F)^+$;
    }
}

Return the updated graph $(G,F)^+$;






\end{algorithm}








%\begin{example} See Figure \ref{fig:composedarrows}. In this example,Person has three attributes SSN, Yea and Age. $Year \to Age$. SSN is a bijective function with Person. In this case, the two composed arrows i.e. (SSN, Year) and (Person, Age) are removed. When we map the categories into relational data, based on the mapping algorithms which are introduced in Section 4, a complete category in Figure \ref{fig:composedarrows}(a) will output a table T1(Person,SSY,Age, Year). But  the reduce presentation will output two tables T2(Person,SSN,year) and T3(Year,Age). Note that T1 is not in the third normal form (3NF), but T2 and T3 are. This example intutively shows the connection between the reduced representation and the normal form theory, which will be elaborated on later.\end{example}

%\begin{example} See an example to illustrate 1RR \ref{fig:1RRExample}. Step 1, Convert FDs to G, and add the two objects X1=BC and X2=CD, and the corresponding projection arrows and other arrows. Step 2: all FDS: $A \to BCDE$, $X1 \to BCD$, $BC \to X1$, $X2 \to CDF$, $CD \to X2$, $E \to F$. Then compute the closures. For example, Closure(X1)= BCDF, Closure(X2)=CDF. Remove the projection and composed elements. Then the final category is shown in Figure (b).\end{example}

\vspace{-2mm}
\begin{example}  Recall Figure \ref{fig:1RRExample}, where the subfigure (c) illustrates the removal of three redundant arrows, namely $D \to B$, $D \to C$, and $A \to C$, resulting in the first reduced representation.

\label{exp:1RRExample} \end{example}

%The time complexity of this algorithm is dominated by the cost to compute the closure of the functional dependencies (i.e. Line 3). Let $m$ and $n$ be the number of objects and arrows in $G_0$, and let $d$ be the number of FDs. The total time for computing the FD closure is $O((d+n) \cdot m)$ based on the implementations of the papers (\cite{elmasri2000fundamentals,10.1145/320493.320489}. Using this procedure, one can implement Algorithm \ref{alg:1RR} with the time $O((d+n) \cdot m)$.


%This can be done with a polynomial algorithm, which runs in O(A*F), where A denotes the total number of attributes of all relationship objects, and F is the total number of arrows in G.

%Let $m$ be the number of objects in $G$, $n$ the number of arrows in $G$, and let $d_1$ and $d_2$ be the number of FDs and MVDs, respectively. The total time for computing the FD closure is $O((d_1+n) \cdot m)$ and MVD closure is $O((d_2+n) \cdot m^3)$ based on the implementations of papers  (\cite{10.1145/320493.320489, 10.1145/320613.320614}. Using these procedures, one can implement Algorithm \ref{alg:2RR} with the time $O(d_1+d_2+n) \cdot m^3)$.

%In the above example, the two arrows are removed.  We will show in the next section that this representation will produce two relational tables T1(Person,SSN,year) and T2(Year,Age). These two tables satisfy 3NF and BCNF. If we include all attributes in one table, then it has transitive function dependency and violates the 3NF. In the later section, we will the output relational data can satisfy both 3NF and BCNF, and further they can satisfy the improved 3NF and BCNF, which was proposed in paper \cite{journals/tods/LingTK81}. In essence, the first reduced representation can be viewed as a generalization of 3NF and BCNF for multi-model data. 

%Note that the first reduce representation of a category C is not unique in some cases, as the composed arrows may be defined differently. For example, consider three loop arrows for a set \{$a,b,c$\}. The function $f_0$ is an identity function, and the other two functions are defined as follows: $f_1(a)=b$, $f_1(b)=c$, $f_1(c)=a$, and $f_2(a)=c$, $f_2(b)=a$, and $f_2(c)=b$. Thus, $f_1 \circ f_1$ = $f_2$, and $f_2 \circ f_2$ = $f_1$. $f_0$ is an identity function that should be removed. However, either $f_1$ or $f_2$ can be considered as a composed arrow. Either one of them can be removed, but not both.

\subsection{Second Reduced Representation}

%The first reduced representation removes the selected arrows in a category, and the second reduced representation further removes selected objects called derivable relationship objects to further reduce redundancy. 

%The first reduced representation involves the removal of specific arrows within a category to reduce the redundancy, and the second reduced representation eliminates certain objects called derivable relationship objects to further minimize the redundancy.

While the first reduced representation condenses the graph representation by eliminating redundant arrows,  the second reduced representation (2RR) takes the optimization process a step further by eliminating (or decomposing) redundant objects, called \textit{derivable relationship objects}.


\begin{definition} (Derivable relationship objects) A  relationship object $O$ in a category $\mathcal{C}$ is derivable if the following conditions are satisfied: (i) $O$ is a limit or an MVD object, and (ii) there is no incoming arrow to $O$; and (iii) for each outgoing arrow $k: O \to X$, where $k$ is not a projection arrow, there exists a projection arrow $f: O \to Y$, and another arrow $g: Y \to X$, such that $k = g \circ f$ in $C$. \label{def:derivable_relation}\end{definition}

%A derivable relationship object can be removed without losing any information when the following three conditions are satisfied simultaneously, (i) they are universal objects (limit or pullback); (ii) they have no incoming arrows; and (iii) any outgoing edge can be derived through other arrows. In addition, if a derivable relationship object is an MVD object (see Definition \ref{def:MVD}), then we call it derivable MVD object, otherwise it is a derivable limit object.  

A derivable relationship object can be removed without losing any information when the following three conditions are simultaneously satisfied: (i) it is a universal object (limit or pullback); (ii) it has no incoming arrows; and (iii) any outgoing edge can be derived through other arrows. Additionally, if a derivable relationship object is an MVD object (defined in Definition \ref{def:MVD}), it is called a derivable MVD object; otherwise, it is a derivable limit object.

%Therefore, this (co)limit can be removed without losing extra information in the category.

%Intuitively, this algorithm removes the universal objects such as pure pullback, limits, and colimits.

%Algorithm \ref{alg:2RR}  generates the second reduced representation. The input is a graph representation and a set of functional dependencies and multivalued dependencies with the canonical form \cite{10.1145/27629.214286}. The output is the second reduced representation. First, similar to 1RR, all functional dependencies are added into $G$, and then the graph is converted into a set of FD in Lines 1 and 2. The key steps lie in Lines 3 and 4. In particular, Line 3 removes the implicit pullback objects. In order to compute the closure of MVDs, the dependent basis of each LHS is calculated. The appendix shows the algorithm to compute the MVD dependency basis. While there is any derivable relationship object $O$ containing both X and Y, where there is a non-trivial MVD $X \to\to Y$. If $O$ contains only X and Y, remove O, otherwise, decompose O with two subobjects $X \cup Y$ and $O-Y$, and the associated projection arrows are also separated. Line 4 removes the derivable limits and colimit objects, which can be recovered from other objects and morphisms.  Finally, lines 5 and 6 generate the corresponding category based on the minimal cover of functional dependencies as the approach for 1RR. 
%\cite{10.1145/27629.214286}

%Algorithm \ref{alg:2RR} is designed to generate the 2RR, with the input of a graph representation $G$, a set of FDs $F$, and a set of MVDs $M$.  The algorithm's steps are outlined as follows. It begins by utilizing Algorithm \ref{alg:closureMVD} to compute a closure category $(G,F,M)^+$ (Line 1). Next, the algorithm focuses on removing derivable MVD objects (Line 2), by iteratively examining MVD objects denoted as $O$ in the context of $X \to\to_O Y$. When such objects are found, the algorithm decomposes them into two subobjects: $X \cup Y$ and $O - Y$. Subsequently, the derivable limit objects are also eliminated. The final steps (Lines 3-5) involve removing the redundant arrows following a similar approach as 1RR.





Algorithm \ref{alg:2RR} is designed to generate the 2RR, using a graph representation $G$, a set of functional dependencies (FDs) $F$, and a set of multivalued dependencies (MVDs) $M$ as input. The steps of the algorithm are outlined as follows: The algorithm begins by utilizing Algorithm \ref{alg:closureMVD} to compute a closure category $(G,F,M)^+$ (Line 1). It then focuses on removing derivable MVD objects (Line 2). This is done by iteratively examining MVD objects, denoted as $O$, in the context of $X \to\to_O Y$. When such objects are found, the algorithm decomposes them into two subobjects: $X \cup Y$ and $O - Y$. Subsequently, it eliminates the derivable limit objects. The final steps (Lines 3-5) involve removing the redundant arrows, following a similar approach to the 1RR algorithm.



%In Line 2, the algorithm first proceeds to remove derivable MVD objects.  As long as there exist MVD objects denoted as $O$ with respect to $X \to\to_U Y$, the algorithm decomposes them into two subobjects: $X \cup Y$ and $O-Y$. Then it eliminates derivable limits objects. The final steps (Lines 3-5) involve removing the redundant arrows following a similar approach as 1RR.




%We now describe Algorithm \ref{alg:2RR}. First, in Step 1, similar to 1RR, all functional dependencies are initially added to the graph $G_0$. Step 2 removes implicit pullback objects. In particular, to infer all MVDs, the dependent basis of each left-hand side (LHS) of MVDs should be calculated. The algorithm to compute the MVD dependency basis is provided in the appendix based on the paper \cite{10.1145/320613.320614}. While there is any implicit pullback object $O$ with respect to $X \to\to Y$, it is decomposed into two subobjects: $X \cup Y$ and $O-Y$, while also separating the associated projection arrows. Step 4 handles the removal of derivable limits and colimit objects, which can be obtained from other objects and morphisms within the graph.  Finally, in Steps 5 and 6, the algorithm generates the corresponding category based on the minimal cover of functional dependencies, following the same approach as 1RR. By applying this algorithm, the second reduced representation is produced, resulting in a more optimized and streamlined category structure by removing redundant objects and morphisms.

%Given one MVD $X \to\to Y$, and any object $Z$, we assume that there is an object $XZ$ in C and $XYZ$ is a pullback for $XY$, $XZ$ and $X$.




%We say a category C is consistent with an MVD $X \to Y$ if there is at least one relationship object which associates with both $X$ and $Y$ in C.

 



%\begin{algorithm}\caption{Compute the complete category with MVDs}\label{alg:complete2}\input{algorithms/computecomplete2}\end{algorithm}


%If a restorable relationship object is associated with at least three objects, and two of them are $X, Y$, and there is a multivalued dependency $X \to\to Y$ in C, then we call this object is restorable pullback object withrespect to $X \to\to Y$.




\begin{algorithm}
\caption{Computing the second reduced representation (2RR)}
\label{alg:2RR}
\KwIn{A graph representation $G$, a set of multivalued dependencies $M$ and  a set of functional dependencies $F$} 
\KwOut{The second reduced representation (2RR) of $G$} 
\DontPrintSemicolon


 Use Algorithm \ref{alg:closureMVD} to compute a closure graph $(G,F,M)^+$;

% $D = FD(G') \cup M$ \tcp*{FD(G') is a set of FDs in G' }

$G'=$\texttt{Remove\_Objects}($(G,F,M)^+$);




\ForEach {arrow  $f: X \to Y$ in $G'$} 
{
    \If{($G' - f$) is equivalent to $G'$}
    {
    Remove $f$ in $G'$;
    }
}

Return the updated $G'$;

%Step 4: Compute a minimal cover V of FDs for $G_3$

%FOREACH object X remove all associated and composed FDs in ClosureFD(X)+ ENDFOR

%Step 5: Generate a graph representation $G_4$ based on  $V$ 



\SetKwFunction{func}{ $\mbox{Remove\_Objects}$}
\SetKwProg{Fn}{Function}{:}{}
\Fn{\func{$G$}}
{
%Convert $G$ to a set of FDs, denoted by $FD(G)$

%Compute the dependent basis of MVD in $FD(G) \cup M$ \tcp*{Appendix for details}

\While{$\exists$ any derivable MVD object $O \in G$ w.r.t $X \to\to_O Y$ }
 {
    decompose $O$ into two objects $O_1= X \cup Y$ and $O_2=O-Y$ and assign the corresponding projection arrows for $O_1$ and $O_2$;

    %\ForEach{projection arrow $f:O \to N$ }
    %    {
    %        \If{$\pi(N) \subseteq \pi(O_i)$ (i=1 or 2)}
     %        {
     %           add the projection arrow $f':O_i \to N$
     %       }
     %   }
    
  }

  Remove all derivable limit  objects in $G$; 

  
  Return the updated $G$;
}





%Compute the minimal cover C of $FD(G) \cup M \cup F$

%while there is a MVD in C but not in an object in G

%add one new object for this MVD, ENDHILE

%Continue to use the 1RR computation algorithm for G with functional dependencies and multivalued dependencies. 





\end{algorithm}


%\begin{definition} (Second Reduced Representation) A graph representation of a category $\mathcal{C}$ is in the second reduced representation (2RR) if and only if  it is in 1RR and it removes restorable pullback, limit and colimit objects. Algorithm \ref{alg:2RR} shows an algorithm that can find at least one second reduced representation.\end{definition}

%\begin{algorithm}\caption{Computing 2RR}\label{alg:2RR}\KwIn{A graph representation $G$, a set of multivalued dependencies $M$ and  a set of functional dependencies $F$} 
\KwOut{The second reduced representation (2RR) of $G$} 
\DontPrintSemicolon


 Use Algorithm \ref{alg:closureMVD} to compute a closure graph $(G,F,M)^+$;

% $D = FD(G') \cup M$ \tcp*{FD(G') is a set of FDs in G' }

$G'=$\texttt{Remove\_Objects}($(G,F,M)^+$);




\ForEach {arrow  $f: X \to Y$ in $G'$} 
{
    \If{($G' - f$) is equivalent to $G'$}
    {
    Remove $f$ in $G'$;
    }
}

Return the updated $G'$;

%Step 4: Compute a minimal cover V of FDs for $G_3$

%FOREACH object X remove all associated and composed FDs in ClosureFD(X)+ ENDFOR

%Step 5: Generate a graph representation $G_4$ based on  $V$ 



\SetKwFunction{func}{ $\mbox{Remove\_Objects}$}
\SetKwProg{Fn}{Function}{:}{}
\Fn{\func{$G$}}
{
%Convert $G$ to a set of FDs, denoted by $FD(G)$

%Compute the dependent basis of MVD in $FD(G) \cup M$ \tcp*{Appendix for details}

\While{$\exists$ any derivable MVD object $O \in G$ w.r.t $X \to\to_O Y$ }
 {
    decompose $O$ into two objects $O_1= X \cup Y$ and $O_2=O-Y$ and assign the corresponding projection arrows for $O_1$ and $O_2$;

    %\ForEach{projection arrow $f:O \to N$ }
    %    {
    %        \If{$\pi(N) \subseteq \pi(O_i)$ (i=1 or 2)}
     %        {
     %           add the projection arrow $f':O_i \to N$
     %       }
     %   }
    
  }

  Remove all derivable limit  objects in $G$; 

  
  Return the updated $G$;
}





%Compute the minimal cover C of $FD(G) \cup M \cup F$

%while there is a MVD in C but not in an object in G

%add one new object for this MVD, ENDHILE

%Continue to use the 1RR computation algorithm for G with functional dependencies and multivalued dependencies. 




\end{algorithm}




\begin{example}  This example illustrates the 2RR algorithm. Figure  \ref{fig:2RRExample}(a) is the input. In Line 1, a closure of the category is computed, which has been explained in Example \ref{exp:CloureMVDExample}.  In Line 2,  $X$  is decomposed to $X_1$ and $X_2$ due to MVDs $A \to\to_X B$ and $A \to\to_X CD$.  In Lines 3-5, the redundant arrows are removed. Finally, the 2RR is shown in Fig. \ref{fig:2RRExample}(c), where there is no derivable MVD object. 
\label{exp:2RRExample} \end{example}




%The time complexities of Algorithms  \ref{alg:1RR} and \ref{alg:2RR} are essentially that of the computation of the closure of FDs and MVDs. That is, they are the same as that of Algorithm \ref{alg:closureFD} and \ref{alg:closureMVD} respectively, which were analyzed before.


%A linear time algorithm for testing membership in the closure of a set of FDs is presented in \cite{10.1145/320493.320489}. Using this procedure, one can implement Algorithm 1 with a time bound of $O(m^2)$, . 



%We now analyze the time complexity of the algorithm. The membership problem for MVDs can be decided in time $O(L(D)^4)$, where $L(D)$ is the size of the description of the given set of dependencies and the category.

%It is important to note that in 2RR algorithm, all restorable pullback objects are decomposed and removed. Because we use the inference rules for FD and MVDs. But we do not apply the inference rule. 

%As shown in Section \ref{sec:limits},  limits have their connection with join dependency. Recall that the 4th normal form and Project-Join normal form (PJ/NF)  is defined with the join dependency. Later we will connect the second reduced representation with PJ/NF.

%Figure \ref{fig:reduced} (b) shows the main idea for the second reduced representation: eliminating all pure limits and colimits objects.  The relational data after this reduction will satisfy the fifth normal form. A relation schema R is in fifth normal form (5NF) (or project-join normal form (PJNF)) with respect to a set F of functional, multivalued, and join dependencies if, for every nontrivial join dependency JD($R_1$,$R_2$,...$R_n$) in $F^+$ (that is, implied by F), every $R_i$ is a superkey of R.


%\subsection{Third Reduced Representation}3RR further removes unique arrows in  commutative triangles to avoid redundant storage. Let us see the following example to understand it.  

%\begin{figure} \centering \includegraphics[width=0.6\textwidth]{figures/UniqueArrow2.jpg} \caption{This example illustrates that the left and right unique arrow. For Fig (a), assume that each school has exactly one dean, then staff can decide department, and department can decide faculty. Dean can decide the school. Then there is one unique arrow from staff to dean.  For Fig (b), each research group belong to one faculty. If two groups has the same GroupHead, then these two groups should belong to the same faculty. All GroupHead should have one group. That is $\forall groupHead, \exists group, s.t. group \rightarrow groupHead $ } \label{fig:universal}  \end{figure}


%\begin{example} See Fig. \ref{fig:universal}.This example illustrates that the left and right unique arrows. See Fig. \ref{fig:universal}(a). Assume that each faculty has exactly one dean, then staff can decide department, and department can decide faculty. Dean can decide the faculty. Then there is one unique arrow from staff to dean. For example, if one staff change the department, or one faculty change the dean, there is no need to update the relation between the staff and the dean. And thus it can avoid the issue of data anomalies. See Fig. \ref{fig:universal} (b). This example illustrates that  we can avoid the implementation of the arrow $f$ from GroupHead to School. Here we assume that this $f$ is unique. Each group belongs to one faculty. If two groups has the same GroupHead, then these two groups should belong to the same faculty. All GroupHead should have one group. That is $\forall groupHead, \exists group, s.t. group \rightarrow groupHead $.\label{exp:3RR} \end{example}


%\begin{figure}\centering\includegraphics[width=0.2\textwidth]{figures/restorablearrow.jpg}\caption{Illustration to the left and right restorable arrows in  the commutative triangle. }\label{fig:restorablearrows}\end{figure}



%\begin{definition} (Third Reduced representation) A graph representation $\mathcal{G}$ of a category $\mathcal{C}$ is in the third reduced representation (3RR) if and only if $\mathcal{G}$ is a 2RR and a left or right unique arrow is removed in any commutative triangle of $\mathcal{G}$.\end{definition}



%\begin{algorithm}\caption{Computing 3RR}\label{alg:3RR}\input{algorithms/compute3RR}\end{algorithm}

%Recall Example \ref{exp:3RR} which illustrates the left unique arrow (between staff and Dean) and the right unique arrow (between GHead and School).  Note that this normalized representation cannot be defined by the classic normalized form.  The algorithm for the third reduced representation outputs a specification of relation schema that is beyond the fifth normal form.  The existing normal form theory decomposes a table into several smaller pieces. But the third representation amps to schema without the redundant attributes. That is, the third avoids redundancy by removing some attributes. Although the relational normal form has no similar idea, By considering the object normal and XML normal form, we can find some related ideas which will be discussed later in Section 5.   




\section{Mapping to Relation Data Schema}

%In the view of this paper, the purpose of a database is to store all information on objects and morphisms in a (set) category. For example,  a relational database stores data using relational tables. An XML database uses XML documents and a  multi-model database leverages the combination of multiple formats of data to store the information in a category. 



%In this section, we introduce an algorithm for mapping a category schema to a relation schema. Due to space limitations, other algorithms which map a category schema to XML and property graph schemas have to be put in Appendix.  Before delving into the details, we provide an overview of the notations that will be used throughout this discussion.

\begin{algorithm} \caption{Map  category schema to relational Schema}
\label{alg:map2relationschema}
\KwIn{A reduced graph representation $G(C)$ of  a schema category $\mathcal{C}$ }
\KwOut{Relation schema   $\mathcal{R}$=\{$R_1,R_2,...,R_p$\}}
\DontPrintSemicolon

\ForEach{ unprocessed object $O_i$ in $V(G\mathcal{(C)})$ with outgoing edges } 
{
    Create a table $R_i$ for $O_i$; \tcp*{$i$=1,...,$p$}

    \If{ $O_i$ is a relational or entity object} 
    { 
        $sort(R_i) = \{SK\}$; \tcp*{Surrogate Key}
    }
    \Else{
        $sort(R_i) = \{\lambda(O_i)\}$; \tcp*{$\lambda(O_i)$ is the name of the object $O_i$}
    }

    Add\_neighbours($R_i,O_i$);

    Mark $O_i$ as a processed object;

}

 Clean($\mathcal{R}$);


\SetKwFunction{func}{ $\mbox{Add\_neighbours}$}
\SetKwProg{Fn}{Function}{:}{}
\Fn{\func{$R,O$}}
{
    \ForEach{object $N$ in $outNbr(O)$ }
    {
         
        %Add an attribute $\lambda(f)$ in relation $R_i$ 

        $sort(R) \leftarrow sort(R) \cup \{\lambda(N)\}$;

        \If{$N$ is a relational or entity object} 
        { 
        Let $\lambda(N)$ be a foreign key in $R$ to reference to $SK$ in $N$; 
        }
        
        \If{ $N \in bin(O)$} 
        { 
            Add\_neighbours($R,N$);   \tcp*{Add the outgoing neighbours of $N$ into $R$}

            Mark $N$ as a processed object;
        }
    }
}   


\SetKwFunction{func}{ $\mbox{Clean}$}
\SetKwProg{Fn}{Function}{:}{}
\Fn{\func{$\mathcal{R}$}}
{
    \ForEach{relation $R_i$ in $\mathcal{R}$ }
        {\If{$K$ is the surrogate key without any referential from other relations} 
            { 
              Remove $K$ in $R_i$; 
         } 
    }

    \If{$sort(R_i) \subseteq sort(R_j)$ in $\mathcal{R}$} 
            { 
                Remove $R_i$ in $\mathcal{R}$; 
            } 
        
}
\end{algorithm}

In this section, we present an algorithm that converts a category schema into a relation schema. Due to space constraints, the algorithms that converts a category schema to XML, graph and hybrid schemata are included in Appendixes \ref{sec:XMLDTD} to \ref{sec:hybrid}.

%Prior to delving into the algorithm, we offer an overview of the notations that will be employed.

%\subsection{Notations of Categories}

%Following the relational database notation, the symbol $\pi$ represents a projection edge in  $E(G(\mathcal{C}$ between two objects.

Let us consider a category denoted as $\mathcal{C}$ and its corresponding reduced representation, $G(\mathcal{C})$.  The set of nodes in the representation is referred to as $V(G(\mathcal{C}$)), while the set of edges is denoted as $E(G(\mathcal{C}))$.  For an object $O$ in $V(G(\mathcal{C}$)), the label associated with node $O$ is denoted as $\lambda_G(O)$.  Furthermore, the set of outgoing objects from  $O$ is denoted as $outNbr_{G(C)}(O)$, and the set of bi-directional neighbors is denoted as $bin_G(O)$. To simplify the notation, when the graph $G(\mathcal{C})$ is clear from the context, we may simply use $\lambda(O)$,  $outNbr(O)$, and $bin(O)$ respectively. In addition, a database schema on universe $U$ is a set $R$ of relation schemes \{$R_1,R_2,...,R_p$\},
where  $\bigcup_{i=1}^{p} R_{i} = U$.
The set of attributes of $R_i$ is $sort(R_i)$ and the surrogate key of $R_i$ is denoted by $SK(R_i)$ (if any).

%If $O$ is an entity or relationship object, then $SK_G(O)$ refers to the surrogate keys of $O$.


%Table \ref{tab:notation} provides a summary of the notations used in the algorithms.


%A database schema on universe $U$ is a set $R$ of relation schemes \{$R_1,R_2,...,R_p$\}, where  $\bigcup_{i=1}^{p} R_{i} = U$. The label (name) of a table $R_i$ is $\lambda(R_i)$. Given a tuple $t$ in a relation, the surrogate key of $t$ (if any) is denoted as $SK(t)$.


%XML is represented as a finite rooted directed tree, with each node in the tree associated with a tag. Within this tree structure, two special attributes are assigned to nodes: @ID and @IDREF. The @ID attribute serves as a unique identifier (key) for a particular node, while the @IDREF attribute is used to reference the ID of other nodes. Note that each node possesses only one @ID attribute, but it can have multiple @IDREF attributes to refer to different nodes within the XML structure.

%An XML tree T is a tree with 5-tuple($V$, lab, elem, att, root) where $V$ is a finite set of node (tags); $lab$ map each node $v$ in $V$ to a type (tag); $elem$ map each node $v$ in a string or a set of other nodes;  att is a partial function $V \times Att \to str$. For each v, the set @a $\in Att $. There are two special attributes are assigned to nodes: @ID and @IDREF, $root \in V$ is the root of T. Note that, in general XML trees, children of each node are ordered. However, this paper focuses on XML normal form theory, which does not use the ordering in the tree, we  disregard this ordering. 

%\begin{definition}(XML tree) An XML tree T is a tree with 5-tuple($V$, lab, elem, att, root) where\begin{itemize}\item $V$ is a finite set of node (tags);\item $lab$ map each node $v$ in $V$ to a type (tag);\item $elem$ map each node $v$ in a string or a set of other nodes  \item  att is a partial function $V \times Att \to str$. For each v, the set @a $\in Att $. There are two special attributes are assigned to nodes: @ID and @IDREF.   \item $root \in V$ is the root of T.\end{itemize}\end{definition}

 

%\begin{definition} A DTD is a 5-tuple($L$, A, P, R, r) where\begin{itemize}\item $L$ is a finite set of element types (tags);\item $A$ is a finite set of strings (attributes), starting with the symble @;\item P is a set of rules $a \rightarrow P_a$ for each $a \in L$, where  $P_a$ is a regular expression over $L$ - \{$r$\}\item  R assigns  each $e \in L $ a finite  subset of A (possibly empty; R(a) is the set of attributes of a)\item $r \in L$ is the root.\end{itemize}\end{definition}



% A DTD is a 5-tuple($L$, A, P, R, r) where $L$ is a finite set of element types (tags); $A$ is a finite set of strings (attributes), starting with the symble @; P is a set of rules $a \rightarrow P_a$ for each $a \in L$, where  $P_a$ is a regular expression over $L$ - \{$r$\}.  R assigns  each $e \in L $ a finite  subset of A (possibly empty; R(a) is the set of attributes of a, $r \in L$ is the root.

%A property graph, denoted as $PG(V,E)$, represents an undirected multi-edge graph. It consists of a vertex set $V$ and an edge set $E$. In this graph, each vertex $v \in V$ and each edge $e \in E$ possess data properties in the form of multiple attribute-value pairs \{$(a,u)$\}. The labels of the nodes and edges are denoted by $\lambda(v)$ and $\lambda(e)$ respectively.

%A Property graph schema is a 5-tuple(V, E, A, L, P) where V is a finite set of vertices; E is a finite set of edges; A is a finite set of attributes; L is a finite set of labels (names) of vertices and edges; P is a function $V \cup E \rightarrow \{(a,str)\}$, mapping vertices and edges to a set of attribute-value pairs.

%\begin{definition}(Property graph schema) A Property graph schema is a 5-tuple(V, E, A, L, P) where\begin{itemize}\item V is a finite set of vertices;\item E is a finite set of edges;\item A is a finite set of attributes; \item L is a finite set of labels (names) of vertices and edges;\item P is a function $V \cup E \rightarrow \{(a,str)\}$, mapping vertices and edges to a set of attribute-value pairs. \end{itemize}\end{definition}

%Given a unified categorical model (including schema category and instance category),  describe In this section, we describe algorithms to map the representations of categories to different types of data, including relation, hierarchical, graph, and hybrid data set. We show that the output data schemata satisfy different normalized forms with various reduced representations. 

\begin{comment}
\begin{table}
\centering
\begin{tabular}{ |c|c| } 
 \hline
 Category $\mathcal{C}$ & \\ [0.5ex] 
  \hline  \hline 
Objects & $O, A, B, C$ \\ 
 \hline 
 Functions (Arrows) & $f, g$   \\ 
 \hline
 Labels of objects and  arrows & $\lambda(O)$, $\lambda(f)$   \\ 
 \hline  
  Bi-directional neighbours of $O$ & $Bin(O)$ \\
  \hline
 Outgoing neighbours of $O$ &  $outNbrAdj(O)$ \\ 
 \hline
 The set of the associated objects of $O$   &   $\pi(O)$ \\
 \hline
 the surrogate key of $O$  &   $SK(O)$ \\
 \hline
XML tree data  &   \\
 \hline
ID and IDREF attributes  & @ID, @IDREF   \\
 \hline
Property Graph $\mathcal{PG}$ &   \\
 \hline
Vertices and edges  &  $V, E$ \\
 \hline
 Types of vertices and edges  &  $\lambda(v)$, $\lambda(e)$ \\
 \hline
 Attribute-value pairs of vertices and edges  &  $\rho(v)$, $\rho(e)$ \\
 \hline
\end{tabular}
\caption{Notation}
\label{tab:notation}
\end{table}
\end{comment}




%The algorithm processes each object $O$ in a specific order: entity, relationship, and attribute objects (Line 1).

%Algorithm \ref{alg:map2relationschema} outlines the framework to map a reduced representation $G(\mathcal{C})$ to a  relational schema.  For each object $O$, a new relation $R_i$ is generated (Lines 2-5), and the neighbor objects $N$ of $O$ are inserted into $R_i$ as attributes (Lines 6-10). If $N$ is an entity or relationship object, treat the attribute $N$ as a foreign key in $R_i$, referring to the relation generated for $N$ (Lines 12). If $N$ is a bidirectional neighbor of $O$, then all outgoing edges of $N$ are also inserted into $R_i$, as $N$ is a key of $R_i$ because of the bijective function between $N$ and $O$ (Line 13-15).  Finally, clean the output data by removing surrogate keys if they are not referenced by any other table, and removing the subsumed relations (Lines 17-21).

Algorithm \ref{alg:map2relationschema} outlines an approach for the conversion of a reduced representation $G(\mathcal{C})$ into a relational schema.  The algorithm proceeds as follows: For each object $O_i$ having outgoing edges, a corresponding relation $R_i$ is instantiated (Lines 1-8).  The algorithm proceeds to include every outgoing neighbor $N$ of $O_i$ as an attribute within relation $R_i$ (Lines 10-17). If $N$ is an entity or relationship object, the surrogate key of $N$ is treated as a foreign key in $R_i$, referencing the relation generated for $N$ (Lines 13-14). Additionally, if $N$ is a bidirectional neighbor of $O_i$, the algorithm augments $R_i$ with the outgoing neighbors of $N$, since $N$  serves as a key in $R_i$ due to the bijective relationship between $O_i$ and $N$ (Lines 15-17). Finally, the algorithm eliminates subsumed relations and removes any surrogate key that is not referred to by any other relation (Lines 18-23). 

%Appendix \ref{sec:category2relation} show several examples to illustrate this algorithm. 




\begin{example} Recall the 1RR graph in Fig \ref{fig:1RRExample}(c). If we run Algorithm \ref{alg:map2relationschema}, the output comprises three relations:  $R_1(A,E)$  (key is $AE$) ($D$ is removed in Line 21), $R_2(A,B)$  (key is $A$) and $R_3(B,C)$  (key is $B$). All those relations satisfy  BCNF. In contrast, if we convert the input graph in Figure \ref{fig:1RRExample}(a) into relations, they include a table $R(A,B,C)$ (with $A$ as the key), which fails to satisfy BCNF (or 3NF) due to $B \to C$.  This example provides an intuitive demonstration of the benefits of the 1RR approach, which ensures that the resulting output schema satisfies BCNF. 

%A more formal discussion will follow in the next section.
\end{example}

%\begin{example} Example \ref{exp:2RRExample} shows a 2RR graph in Figure \ref{fig:2RRExample}(c). If we run the above algorithm, the output schema comprises four relations:  $R_1(A,E)$ (key is $AE$), $R_2(A,C,F)$, (key is $A$), $R_3(F,B)$ (key is $F$), and $R_4(B,D)$ (key is $B$). All those relations satisfy the 4NF.  However, if we convert the input in Figure \ref{fig:2RRExample}(a) into a relation schema, wherein one output table  $R_5(A,E,D)$ is not in 4NF due to the MVD $A \to\to_{R_5} D$. This example provides an intuitive demonstration of the benefits of the 2RR approach, which ensures that the resulting output schema satisfies 4NF. The forthcoming section will provide proof of this result. \end{example}

%In contrast, if we convert the input graph in Figure \ref{fig:1RRExample}(a) to relational schema, then $R(A,B,C)$ (key is $A$) is output, but it not in BCNF (or 3NF) due to the FD $B \to C$. This example intuitively shows that the 1RR can guarantee that the output schema satisfies BCNF. We will give proof of this result in the next section. 


%A database schema S is a finite set of relation names, with a set of attributes, denoted by attr(R), associated with each R $\in$ S. We shall identify attr(R) of cardinality m with \{1, . . ., m\}.  An instance I of schema S assigns to each symbol R $\in$ S with m = $attr(R)$ a relation I(R) which is a finite set of m-tuples over N+.



%\begin{algorithm} \caption{Map to Relational Data}\label{alg:map2relation2}\input{algorithms/map2relation2.tex}\end{algorithm}

%\begin{figure}\centering\includegraphics[width=0.8\textwidth]{figures/ex02.jpg}\caption{ Two FDs: Student, Course $\to$ Instructor and Instructor $\to$ Course. Figure C illustrates that the base diagram can avoid the edge between SC and Course, and generate three tables to satisfy the  BCNF. R1(SC, student, Instructor), R2(Instructor,Course). Then The relation between SC and course can be derived from R1 and R2. Note that in relation tables R(student, instructor, course), we cannot find a lossless decomposition to obtain tables satisfying BCNF. }\label{fig:exampleC}\end{figure}






%Further, if $N$ is an entity or relationship neighbor of $O$, the surrogate key of $N$ is added as an IDREF attribute to $O$ (Lines 20-22). This step establishes a linkage between $N$ and $O$ through the IDREF attribute.


%This mapping facilitates the representation and validation of XML data according to the structure and relationships defined in the reduced representation schema.


%The neighbor node $n$ of $O$ are output with the child node of v. The surrogate keys are defined as the ID attribute of the element.  The bi-directional node n is included in the subtrees of v, as n and v have a bijective function between them. 




 

%Readers might notice that the mapping algorithm above  generates the trees whose length are  always two, i.e. a shallow and wide tree. There are two reasons to explain this: (i) any tree can be reorganized as two levels tree with the schema shown in the figure (one object with all nodes, and the other objects with the edges), (ii) the normalization requires to reduce the redundancy. The tree with higher depth is more possible to incur the the redundancy. For example, see $root \to A$, $A \to B$, $B \to C$, R(A)=\{@SerialKey\_A\}, R(B)=\{@SerialKey\_B\} and R(C)=\{@SerialKey\_C\}.  If two different A has the same B, then the C nodes under the B has to be reduplicated. Therefore, it will cause the redundancy storage and violates the normalization. One way to avoid the redundancy is to use this schema: $root \to A,B,C$, R(A)= \{@SerialKey\_A, @SerialKey\_B\},   R(B)=\{@SerialKey\_B, @SerialKey\_C\}, R(C)=\{@SerialKey\_C\}.  










%\begin{algorithm} \caption{Map to Property Graph}\label{alg:map2graph2}\input{algorithms/map2graph2}\end{algorithm}

%\item L is a finite set of labels (names) of $V \cup E$; 




%The associated attribute objects are included as attribute-value pairs for the corresponding node.
 
%Algorithm \ref{alg:map2graphschema} provides a mapping procedure for converting a category into a property graph schema.  Initially, convert each entity object in the category into a node in the property graph (Lines 2-3).  Then convert binary relationship objects into edges connecting two objects in the property graph (Lines 7-8). For the ternary or multi-way relationship objects, a node is created in the graph to represent the relationship (Lines 9-10). Assuming $N$ is a neighbor from  $O$. If $N$ is an attribute object, add $N$ as an attribute for $O$ (Lines 16-17). If $N$ is an entity (or relationship) object create a new edge in the graph for the object $N$ (Lines 18-19).   If the edge between $N$ and $O$ is bidirectional, add all outgoing neighbors of $N$ to an attribute value or an edge for $O$ (Line 20-21). 



%By following this algorithm, a category can be effectively mapped to a property graph schema. This mapping process facilitates the categorical representation in a graph-based database environment.
 
 %See an example in Figure \ref{fig:composedarrows}, assume that SSN has another attribute  object N, then this attribute will be added to Person object as an attribute-value pair, instead of a pair for SSN. We will not create a separate node for SSN, as it is the key for the person.   

%\begin{definition} A graph schema is a structure (V,E,$\eta$,$\lambda$,$\kappa$) where\begin{itemize}\item $V \subseteq O$ is a finite set of objects, called vertices;\item $E \subseteq O$ is a finite set of objects, called edges;\item $\eta: E \to  V  \times V$ is a function assigning to each edge an ordered pair of vertices;\item $\lambda: V \cup E \to L$ is a function assigning to each object a finite set of labels.\item $\kappa: V \cup E \to P$ is a function assigning to each object a finite set of properties.\end{itemize}\end{definition}



%The following rules can be considered when we map a category to a graph. Entity types become vertex types.  Binary relationship types become edge types. N-ary relationship types become vertex types




%\begin{algorithm} \caption{Map to a graph schema}\label{alg:map2graph}\input{algorithms/map2graph.tex}\end{algorithm}


%Figure \ref{fig:firstexample} illustrates an example of three types of data derived from category data. 









%Here, $N$ represents the total number of elements across all objects, $d$ is the maximum in-degree observed among the objects, and $M$ corresponds to the maximum number of functions between any two objects. Importantly, the time complexity is bounded by the product of these factors due to the fact that each object is processed a maximum of $d$ times.
 
 %To illustrate this concept, Figure \ref{fig:hybrid} demonstrates an example wherein customer data is divided into two distinct parts. The first part pertains to the information of customers, which is represented using relational data structures. The second part captures the friendship connections between customers and is best represented as a graph.

%\begin{algorithm} \caption{Map to a hybrid schema}\label{alg:map2hybrid}\input{algorithms/map2hybrid.tex}\end{algorithm}

%\begin{figure}
%\centering
%\includegraphics[width=0.7\textwidth]{figures/hybrid.jpg}
%\caption{An example to illustrate the output of two different models of data} \label{fig:hybrid}
%\end{figure}





\renewcommand{\thefootnote}{\fnsymbol{footnote}}
\footnotetext[1]{Correspondence to: J. Mayer $<$\texttt{research@jmayer.ai}$>$}
\renewcommand{\thefootnote}{\arabic{footnote}}

\section{Related Work}
The landscape of large language model vulnerabilities has been extensively studied in recent literature \cite{crothers2023machinegeneratedtextcomprehensive,shayegani2023surveyvulnerabilitieslargelanguage,Yao_2024,Huang2023ASO}, that propose detailed taxonomies of threats. These works categorize LLM attacks into distinct types, such as adversarial attacks, data poisoning, and specific vulnerabilities related to prompt engineering. Among these, prompt injection attacks have emerged as a significant and distinct category, underscoring their relevance to LLM security.

The following high-level overview of the collected taxonomy of LLM vulnerabilities is defined in \cite{Yao_2024}:
\begin{itemize}
    \item Adversarial Attacks: Data Poisoning, Backdoor Attacks
    \item Inference Attacks: Attribute Inference, Membership Inferences
    \item Extraction Attacks
    \item Bias and Unfairness
Exploitation
    \item Instruction Tuning Attacks: Jailbreaking, Prompt Injection.
\end{itemize}
Prompt injection attacks are further classified in \cite{shayegani2023surveyvulnerabilitieslargelanguage} into the following: Goal hijacking and \textbf{Prompt leakage}.

The reviewed taxonomies underscore the need for comprehensive frameworks to evaluate LLM security. The agentic approach introduced in this paper builds on these insights, automating adversarial testing to address a wide range of scenarios, including those involving prompt leakage and role-specific vulnerabilities.

\subsection{Prompt Injection and Prompt Leakage}

Prompt injection attacks exploit the blending of instructional and data inputs, manipulating LLMs into deviating from their intended behavior. Prompt injection attacks encompass techniques that override initial instructions, expose private prompts, or generate malicious outputs \cite{Huang2023ASO}. A subset of these attacks, known as prompt leakage, aims specifically at extracting sensitive system prompts embedded within LLM configurations. In \cite{shayegani2023surveyvulnerabilitieslargelanguage}, authors differentiate between prompt leakage and related methods such as goal hijacking, further refining the taxonomy of LLM-specific vulnerabilities.

\subsection{Defense Mechanisms}

Various defense mechanisms have been proposed to address LLM vulnerabilities, particularly prompt injection and leakage \cite{shayegani2023surveyvulnerabilitieslargelanguage,Yao_2024}. We focused on cost-effective methods like instruction postprocessing and prompt engineering, which are viable for proprietary models that cannot be retrained. Instruction preprocessing sanitizes inputs, while postprocessing removes harmful outputs, forming a dual-layer defense. Preprocessing methods include perplexity-based filtering \cite{Jain2023BaselineDF,Xu2022ExploringTU} and token-level analysis \cite{Kumar2023CertifyingLS}. Postprocessing employs another set of techniques, such as censorship by LLMs \cite{Helbling2023LLMSD,Inan2023LlamaGL}, and use of canary tokens and pattern matching \cite{vigil-llm,rebuff}, although their fundamental limitations are noted \cite{Glukhov2023LLMCA}. Prompt engineering employs carefully designed instructions \cite{Schulhoff2024ThePR} and advanced techniques like spotlighting \cite{Hines2024DefendingAI} to mitigate vulnerabilities, though no method is foolproof \cite{schulhoff-etal-2023-ignore}. Adversarial training, by incorporating adversarial examples into the training process, strengthens models against attacks \cite{Bespalov2024TowardsBA,Shaham2015UnderstandingAT}.

\subsection{Security Testing for Prompt Injection Attacks}

Manual testing, such as red teaming \cite{ganguli2022redteaminglanguagemodels} and handcrafted "Ignore Previous Prompt" attacks \cite{Perez2022IgnorePP}, highlights vulnerabilities but is limited in scale. Automated approaches like PAIR \cite{chao2024jailbreakingblackboxlarge} and GPTFUZZER \cite{Yu2023GPTFUZZERRT} achieve higher success rates by refining prompts iteratively or via automated fuzzing. Red teaming with LLMs \cite{Perez2022RedTL} and reinforcement learning \cite{anonymous2024diverse} uncovers diverse vulnerabilities, including data leakage and offensive outputs. Indirect Prompt Injection (IPI) manipulates external data to compromise applications \cite{Greshake2023NotWY}, adapting techniques like SQL injection to LLMs \cite{Liu2023PromptIA}. Prompt secrecy remains fragile, with studies showing reliable prompt extraction \cite{Zhang2023EffectivePE}. Advanced frameworks like Token Space Projection \cite{Maus2023AdversarialPF} and Weak-to-Strong Jailbreaking Attacks \cite{zhao2024weaktostrongjailbreakinglargelanguage} exploit token-space relationships, achieving high success rates for prompt extraction and jailbreaking.

\subsection{Agentic Frameworks for Evaluating LLM Security}

The development of multi-agent systems leveraging large language models (LLMs) has shown promising results in enhancing task-solving capabilities \cite{Hong2023MetaGPTMP, Wang2023UnleashingTE, Talebirad2023MultiAgentCH, Wu2023AutoGenEN, Du2023ImprovingFA}. A key aspect across various frameworks is the specialization of roles among agents \cite{Hong2023MetaGPTMP, Wu2023AutoGenEN}, which mimics human collaboration and improves task decomposition.

Agentic frameworks and the multi-agent debate approach benefit from agent interaction, where agents engage in conversations or debates to refine outputs and correct errors \cite{Wu2023AutoGenEN}. For example, debate systems improve factual accuracy and reasoning by iteratively refining responses through collaborative reasoning \cite{Du2023ImprovingFA}, while AG2 allows agents to autonomously interact and execute tasks with minimal human input.

These frameworks highlight the viability of agentic systems, showing how specialized roles and collaborative mechanisms lead to improved performance, whether in factuality, reasoning, or task execution. By leveraging the strengths of diverse agents, these systems demonstrate a scalable approach to problem-solving.

Recent research on testing LLMs using other LLMs has shown that this approach can be highly effective \cite{chao2024jailbreakingblackboxlarge, Yu2023GPTFUZZERRT, Perez2022RedTL}. Although the papers do not explicitly employ agentic frameworks they inherently reflect a pattern similar to that of an "attacker" and a "judge". \cite{chao2024jailbreakingblackboxlarge}  This pattern became a focal point for our work, where we put the judge into a more direct dialogue, enabling it to generate attacks based on the tested agent response in an active conversation.

A particularly influential paper in shaping our approach is Jailbreaking Black Box Large Language Models in Twenty Queries \cite{chao2024jailbreakingblackboxlarge}. This paper not only introduced the attacker/judge architecture but also provided the initial system prompts used for a judge.

\section{Future Works}
In future research, several avenues could be explored to enhance the performance and applicability of question-answering models for Amharic. Firstly, various adjustments of weights (w1 and w2) for cosine similarity and LCS for Algorithm 1 to locate the translated answer from the translated context should be investigated for the extraction process. Human evaluations should also provide feedback on the quality of the AmSQuAd dataset. Additionally, there is a need to address the limitations of current datasets, such as the absence of unanswerable questions and the relatively small size of datasets like AmQA. Incorporating unanswerable questions and expanding dataset sizes would provide a more realistic evaluation of model capabilities. Furthermore, future studies could investigate the impact of pre-training on larger Amharic corpora using diverse pre-training strategies, as demonstrated in recent work. These efforts could lead to significant advancements in question-answering systems.

%\section{Related Work}
The landscape of large language model vulnerabilities has been extensively studied in recent literature \cite{crothers2023machinegeneratedtextcomprehensive,shayegani2023surveyvulnerabilitieslargelanguage,Yao_2024,Huang2023ASO}, that propose detailed taxonomies of threats. These works categorize LLM attacks into distinct types, such as adversarial attacks, data poisoning, and specific vulnerabilities related to prompt engineering. Among these, prompt injection attacks have emerged as a significant and distinct category, underscoring their relevance to LLM security.

The following high-level overview of the collected taxonomy of LLM vulnerabilities is defined in \cite{Yao_2024}:
\begin{itemize}
    \item Adversarial Attacks: Data Poisoning, Backdoor Attacks
    \item Inference Attacks: Attribute Inference, Membership Inferences
    \item Extraction Attacks
    \item Bias and Unfairness
Exploitation
    \item Instruction Tuning Attacks: Jailbreaking, Prompt Injection.
\end{itemize}
Prompt injection attacks are further classified in \cite{shayegani2023surveyvulnerabilitieslargelanguage} into the following: Goal hijacking and \textbf{Prompt leakage}.

The reviewed taxonomies underscore the need for comprehensive frameworks to evaluate LLM security. The agentic approach introduced in this paper builds on these insights, automating adversarial testing to address a wide range of scenarios, including those involving prompt leakage and role-specific vulnerabilities.

\subsection{Prompt Injection and Prompt Leakage}

Prompt injection attacks exploit the blending of instructional and data inputs, manipulating LLMs into deviating from their intended behavior. Prompt injection attacks encompass techniques that override initial instructions, expose private prompts, or generate malicious outputs \cite{Huang2023ASO}. A subset of these attacks, known as prompt leakage, aims specifically at extracting sensitive system prompts embedded within LLM configurations. In \cite{shayegani2023surveyvulnerabilitieslargelanguage}, authors differentiate between prompt leakage and related methods such as goal hijacking, further refining the taxonomy of LLM-specific vulnerabilities.

\subsection{Defense Mechanisms}

Various defense mechanisms have been proposed to address LLM vulnerabilities, particularly prompt injection and leakage \cite{shayegani2023surveyvulnerabilitieslargelanguage,Yao_2024}. We focused on cost-effective methods like instruction postprocessing and prompt engineering, which are viable for proprietary models that cannot be retrained. Instruction preprocessing sanitizes inputs, while postprocessing removes harmful outputs, forming a dual-layer defense. Preprocessing methods include perplexity-based filtering \cite{Jain2023BaselineDF,Xu2022ExploringTU} and token-level analysis \cite{Kumar2023CertifyingLS}. Postprocessing employs another set of techniques, such as censorship by LLMs \cite{Helbling2023LLMSD,Inan2023LlamaGL}, and use of canary tokens and pattern matching \cite{vigil-llm,rebuff}, although their fundamental limitations are noted \cite{Glukhov2023LLMCA}. Prompt engineering employs carefully designed instructions \cite{Schulhoff2024ThePR} and advanced techniques like spotlighting \cite{Hines2024DefendingAI} to mitigate vulnerabilities, though no method is foolproof \cite{schulhoff-etal-2023-ignore}. Adversarial training, by incorporating adversarial examples into the training process, strengthens models against attacks \cite{Bespalov2024TowardsBA,Shaham2015UnderstandingAT}.

\subsection{Security Testing for Prompt Injection Attacks}

Manual testing, such as red teaming \cite{ganguli2022redteaminglanguagemodels} and handcrafted "Ignore Previous Prompt" attacks \cite{Perez2022IgnorePP}, highlights vulnerabilities but is limited in scale. Automated approaches like PAIR \cite{chao2024jailbreakingblackboxlarge} and GPTFUZZER \cite{Yu2023GPTFUZZERRT} achieve higher success rates by refining prompts iteratively or via automated fuzzing. Red teaming with LLMs \cite{Perez2022RedTL} and reinforcement learning \cite{anonymous2024diverse} uncovers diverse vulnerabilities, including data leakage and offensive outputs. Indirect Prompt Injection (IPI) manipulates external data to compromise applications \cite{Greshake2023NotWY}, adapting techniques like SQL injection to LLMs \cite{Liu2023PromptIA}. Prompt secrecy remains fragile, with studies showing reliable prompt extraction \cite{Zhang2023EffectivePE}. Advanced frameworks like Token Space Projection \cite{Maus2023AdversarialPF} and Weak-to-Strong Jailbreaking Attacks \cite{zhao2024weaktostrongjailbreakinglargelanguage} exploit token-space relationships, achieving high success rates for prompt extraction and jailbreaking.

\subsection{Agentic Frameworks for Evaluating LLM Security}

The development of multi-agent systems leveraging large language models (LLMs) has shown promising results in enhancing task-solving capabilities \cite{Hong2023MetaGPTMP, Wang2023UnleashingTE, Talebirad2023MultiAgentCH, Wu2023AutoGenEN, Du2023ImprovingFA}. A key aspect across various frameworks is the specialization of roles among agents \cite{Hong2023MetaGPTMP, Wu2023AutoGenEN}, which mimics human collaboration and improves task decomposition.

Agentic frameworks and the multi-agent debate approach benefit from agent interaction, where agents engage in conversations or debates to refine outputs and correct errors \cite{Wu2023AutoGenEN}. For example, debate systems improve factual accuracy and reasoning by iteratively refining responses through collaborative reasoning \cite{Du2023ImprovingFA}, while AG2 allows agents to autonomously interact and execute tasks with minimal human input.

These frameworks highlight the viability of agentic systems, showing how specialized roles and collaborative mechanisms lead to improved performance, whether in factuality, reasoning, or task execution. By leveraging the strengths of diverse agents, these systems demonstrate a scalable approach to problem-solving.

Recent research on testing LLMs using other LLMs has shown that this approach can be highly effective \cite{chao2024jailbreakingblackboxlarge, Yu2023GPTFUZZERRT, Perez2022RedTL}. Although the papers do not explicitly employ agentic frameworks they inherently reflect a pattern similar to that of an "attacker" and a "judge". \cite{chao2024jailbreakingblackboxlarge}  This pattern became a focal point for our work, where we put the judge into a more direct dialogue, enabling it to generate attacks based on the tested agent response in an active conversation.

A particularly influential paper in shaping our approach is Jailbreaking Black Box Large Language Models in Twenty Queries \cite{chao2024jailbreakingblackboxlarge}. This paper not only introduced the attacker/judge architecture but also provided the initial system prompts used for a judge.

%\section{Conclusion}
In this work, we propose a simple yet effective approach, called SMILE, for graph few-shot learning with fewer tasks. Specifically, we introduce a novel dual-level mixup strategy, including within-task and across-task mixup, for enriching the diversity of nodes within each task and the diversity of tasks. Also, we incorporate the degree-based prior information to learn expressive node embeddings. Theoretically, we prove that SMILE effectively enhances the model's generalization performance. Empirically, we conduct extensive experiments on multiple benchmarks and the results suggest that SMILE significantly outperforms other baselines, including both in-domain and cross-domain few-shot settings.


%%
%% The next two lines define the bibliography style to be used, and
%% the bibliography file.
\bibliographystyle{ACM-Reference-Format}
\bibliography{sample-base}




\appendix

%\smallskip \smallskip

\newpage

\noindent \textbf{Organization of the Appendixes}  ~~ 

%Appendix \ref{sec:pushout} shows the connection between pushout and connected component of a graph and 

In Appendixes \ref{sec:XMLDTD} to \ref{sec:hybrid}, we show the algorithms used for mapping categories to XML DTD, graph, and hybrid schemata. In Appendix \ref{sec:monoidal}, we analyze the inference rules of FDs and MVDs.  Appendix \ref{sec:proofs} provides the proofs and explanations for the lemmas and theorems. Furthermore, Appendix \ref{sec:complexity} assesses the computing complexities associated with the various algorithms presented. 


%Finally, Appendix \ref{sec:relatedwork} conducts a review of related work in the field of category theory for databases and Appendix \ref{sec:futurework} offers insights into potential avenues for future research.

%\section{Mapping Categories to Relations}
%\label{sec:category2relation}

%Algorithm \ref{alg:map2relationschema} shows how to map schema category to relation schema. See one more example to illustrate this algorithm.


%\begin{example} Recall the 2RR graph in Figure \ref{fig:2RRExample}(c). If we run Algorithm \ref{alg:map2relationschema} to convert this category to relational schemes, the output comprises four relations:  $R_1(A,B)$, $R_2(A,D)$, $R_3(B,C)$ and $R_4(A,C)$. All those relations satisfy the fourth normal form. In contrast, if we convert the input graph in Figure \ref{fig:2RRExample}(a) into relations, they include one table $R(A,B,C,D)$, which fails to satisfy the fourth normal form due to $A \to\to B$ and $A \to\to CD$.  This example provides an intuitive demonstration of the benefits of the 2RR approach, which ensures that the resulting output schema satisfies the fourth normal form. \end{example}

\section{Mapping Categories to XML DTD}
\label{sec:XMLDTD}

%We shall use a somewhat simplified model of XML trees in order to keep the notation simple. We assume a countably infinite set of labels L, a countably infinite set of attributes A (we shall use the notation @$l_1$, @$l_2$, etc for attributes to distinguish them from labels), and a countably infinite set V of values of attributes. 



%XML is represented as a finite rooted directed tree, with each node in the tree associated with a tag. Within this tree structure, two special attributes are assigned to nodes: @ID and @IDREF. The @ID attribute serves as a unique identifier (key) for a particular node, while the @IDREF attribute is used to reference the ID of other nodes. Note that each node possesses only one @ID attribute, but it can have multiple @IDREF attributes to refer to different nodes within the XML structure.

%\begin{definition}(XML tree) An XML tree T is a tree with 5-tuple($V$, lab, elem, att, root) where\begin{itemize}\item $V$ is a finite set of node (tags);\item $lab$ map each node $v$ in $V$ to a type (tag);\item $elem$ map each node $v$ in a string or a set of other nodes  \item  att is a partial function $V \times Att \to str$. For each v, the set @a $\in Att $. There are two special attributes are assigned to nodes: @ID and @IDREF.   \item $root \in V$ is the root of T.\end{itemize}\end{definition}

 %Note that, in general XML trees, children of each node are ordered. However, this paper focuses on XML normal form theory, which does not use the ordering in the tree, we  disregard this ordering. 

%\begin{definition} A DTD is a 5-tuple($L$, A, P, R, r) where\begin{itemize}\item $L$ is a finite set of element types (tags);\item $A$ is a finite set of strings (attributes), starting with the symble @;\item P is a set of rules $a \rightarrow P_a$ for each $a \in L$, where  $P_a$ is a regular expression over $L$ - \{$r$\}\item  R assigns  each $e \in L $ a finite  subset of A (possibly empty; R(a) is the set of attributes of a)\item $r \in L$ is the root.\end{itemize}\end{definition}


%\begin{example} For example, Figure shows an XML tree and the corresponding XML DTD. \end{example}

XML is represented as a finite rooted directed tree, with each node in the tree associated with a tag. We use DTD to describe the schema of XML.

%Within this tree structure, two special attributes are assigned to nodes: @ID and @IDREF. The @ID attribute serves as a unique identifier (key) for a particular node, while the @IDREF attribute is used to reference the ID of other nodes. Note that each node possesses only one @ID attribute, but it can have multiple @IDREF attributes to refer to different nodes within the XML structure.

\begin{definition} A DTD is a 5-tuple($L, T, P, R, r$) where 
 \begin{itemize}[noitemsep,topsep=0pt]
  \item $L$ is a finite set of element types (tags);
  \item $T$ is a finite set of strings (attributes), starting with the symbols @. There is a special attribute @ID, which serves as a unique identifier (key) for a particular tag;
   %\item $T$ is a finite set of strings (attributes), starting with the symbols @. There are a  special attributes: @ID and @IDREF$_{A,B}$. The @ID attribute serves as a unique identifier (key) for a particular tag, while the @IDREF$_{A,B}$ attribute with tage $A$ is used to reference the ID attribute of other tag $B$;
  \item $P$ is a mapping from $L$ to element type definitions: Given any $a \in L$, $P_a = \#P$ or  $P_a$ is a regular expression over $L$ - \{$r$\}, where $\#P$ denotes $\#PCDATA$;
  \item $R$ assigns  each $e \in L $ a finite  subset of $T$ (possibly empty); 
   \item $r \in L$ is the root.
\end{itemize}
\end{definition}


%In the algorithm to map a schema to a category, each attribute object associated with an entity object O is constructed as the sub-element of O in XML DTD.

%If $O$ is an entity or relationship object, assign the ID attribute of the surrogate key.

%Algorithm \ref{alg:map2DTD}  outlines the mapping process from the reduced representation schema to a DTD schema.  First, $\epsilon$ is a root node, and process and add objects $O$  as children of $\epsilon$ (Lines 1-3). For an object $O$ with outgoing edges, create a new tag and append $O^+$ in the regular expression of $\epsilon$ (Lines 6-7). Subsequently, for each neighbor object $N$ of $O$, append N in the expression rule of $O$ (Lines 12-13). Further, if $N$ is a bidirectional attribute object of $O$, append all outgoing objects of $N$ in the expression rule of $O$ as well.  If $N$ is an entity neighbor of $O$, add the surrogate key of $N$ as an IDREF attribute to $O$ (Lines 17-18).

Algorithm \ref{alg:map2DTD} presents the mapping process from a categorical schema to a DTD schema. The algorithm proceeds as follows: Initially, $\epsilon$ is designated as the root node, and the algorithm processes and adds objects $O$ as children of $\epsilon$ (Lines 1-3).  A new tag $\lambda(O)$ is created for each object with outgoing edges, and $\lambda(O)^+$ is appended to the regular expression of $\epsilon$ (Lines 5-9), where $\lambda(O)$ denotes the label (name) of the object $O$. Subsequently, for each neighbor object $N$ of $O$, if $N$ has any outgoing edge, an attribute $@\lambda(N)$\_ID is added for $O$ (Lines 13-14), otherwise $\lambda(N)$ is appended to the regular expression rule of $\lambda(O)$ (Lines 16-17).  It is worth noting that the shape of the XML data that conforms to the output DTD is wide and shallow. This representation allows for a clear and concise XML structure, avoiding redundant storage.


%Further, if $N$ is an entity or relationship neighbor of $O$, the surrogate key of $N$ is added as an IDREF attribute to $O$ (Lines 20-22). This step establishes a linkage between $N$ and $O$ through the IDREF attribute.


%This mapping facilitates the representation and validation of XML data according to the structure and relationships defined in the reduced representation schema.


%The neighbor node $n$ of $O$ are output with the child node of v. The surrogate keys are defined as the ID attribute of the element.  The bi-directional node n is included in the subtrees of v, as n and v have a bijective function between them. 

%Figure \ref{fig:firstexample} shows the XML data based on the category. 

%$A=\{@ID,@IDREF\}$

\begin{example} Recall the 1RR graph in Figure \ref{fig:1RRExample}(c).   Algorithm \ref{alg:map2DTD} convert it to XML DTD as follows: $L=\{A,B,C,D,E\}$, $T$=\{@ID, @A\_ID, @B\_ID\}, $P=\{\epsilon \to D^+A^+B^+, D \to E,  B \to C$\}, $R(A)$=\{@ID, @B\_ID\}, $R(D)$=\{@ID, @A\_ID\}, $R(B)$=\{@ID\}, and $r = \epsilon$. $B$ has an outgoing edge and is appended into the regular expression following the root $\epsilon$. In contrast, $E$ and $C$ are attribute objects without outgoing edges, and thus are created as the sub-element of $D$ and $B$ respectively.  
\end{example}

\begin{example} Recall the 2RR graph in Figure \ref{fig:2RRExample}(c).
Algorithm \ref{alg:map2DTD} convert it to XML DTD as follows: $L=\{A,B,C$, $D,X_1,X_2\}$, $T=$\{@ID,@A\_ID, @B\_ID\}, $P=\{\epsilon \to A^+B^+X_1^+X_2^+$, $A \to  C$, $B \to C$, $X_2 \to D$\}, $R(X_1)$=\{@ID, @A\_ID, @B\_ID\}, $R(X_2)$=\{@ID,@A\_ID\}, $R(A)=R(B)=\{@ID\}$, and $r = \epsilon$.  
\end{example}

\begin{algorithm} \caption{Map  category schema to XML DTD}\label{alg:map2DTD}\KwIn{A graph representation $\mathcal{G(C)}$ of a schema of a category $\mathcal{C}$ }
    \KwOut{XML DTD $\mathcal{D}$=($L$, T, P, R, r)}
    \DontPrintSemicolon

Let $\epsilon$ be the root $r$ of $D$;

 $T= \{@ID\}$;

 Initialize $L$, $P$, and $R$ to be empty sets;

\ForEach{ unprocessed object $O$ in $V(\mathcal{G(C)})$ with outgoing edges}
{ 
     $L \leftarrow L \cup \{\lambda(O)\}$; \tcp*{$\lambda(O)$ is the label of $O$}

     $P_\epsilon \leftarrow P_\epsilon \cdot (\lambda(O))^+$;
    
     $R(\lambda(O))=\{@ID\}$;
    
    Add\_neighbours($O$);
      
    %Process\_entities\_relationships($O$)

     Mark $O$ as a processed node;
    
}


\SetKwFunction{func}{ $\mbox{Add\_neighbours}$}
\SetKwProg{Fn}{Function}{:}{}
\Fn{\func{$O$}}
{
    \ForEach{unprocessed object $N$ in $outNbr(O)$ }
    {    
     \If{$N$ has any outgoing edge}
     { 
        $T= T \cup \{@\lambda(N)$\_ID\};
        
        $R(O)=R(O) \cup \{@\lambda(N)$\_ID\};
     }
    \Else 
     {  
       $P_O \leftarrow P_O \cdot \lambda(N)$;

        $L \leftarrow L \cup \{\lambda(N)\}$;
      }   
    }
}   


\end{algorithm}


%\begin{algorithm} \caption{Map to XML tree data}\label{alg:map2tree2}\input{algorithms/map2tree2.tex}\end{algorithm}



 

%Readers might notice that the mapping algorithm above  generates the trees whose length are  always two, i.e. a shallow and wide tree. There are two reasons to explain this: (i) any tree can be reorganized as two levels tree with the schema shown in the figure (one object with all nodes, and the other objects with the edges), (ii) the normalization requires to reduce the redundancy. The tree with higher depth is more possible to incur the the redundancy. For example, see $root \to A$, $A \to B$, $B \to C$, R(A)=\{@SerialKey\_A\}, R(B)=\{@SerialKey\_B\} and R(C)=\{@SerialKey\_C\}.  If two different A has the same B, then the C nodes under the B has to be reduplicated. Therefore, it will cause the redundancy storage and violates the normalization. One way to avoid the redundancy is to use this schema: $root \to A,B,C$, R(A)= \{@SerialKey\_A, @SerialKey\_B\},   R(B)=\{@SerialKey\_B, @SerialKey\_C\}, R(C)=\{@SerialKey\_C\}.  


\begin{figure*}
\centering
\includegraphics[width=0.9\textwidth]{figures/Picture2.png}
\caption{The relation, XML and graph data converted from the category in Figure \ref{fig:firstexamplecategory}.} \label{fig:relationXML}
\end{figure*}



\section{Mapping Categories to Graphs}
\label{sec:PropertyGraph}

%For all objects and attributes in the category, we use the capital letter, such as A,B,O,..., and all vertexes and edges in the property graph, we use the small letter. such as v, w. (\cite{conf/icde/AlotaibiLQEO21})

%A property graph $PG (V, E)$  is an undirected multi-edge graph with vertex set $V$ and edge set $E$, where each vertex $v \in V$  and each edge $e \in E$ has data properties consisting of multiple attribute-value pairs \{$(a,u)$\}. $\lambda(v)$ and $\lambda(e)$ denote the labels of the node and edge.   

%A property graph, denoted as $PG(V,E)$, represents an undirected multi-edge graph. It consists of a vertex set $V$ and an edge set $E$. In this graph, each vertex $v \in V$ and each edge $e \in E$ possess data properties in the form of multiple attribute-value pairs \{$(a,u)$\}. The labels of the nodes and edges are denoted by $\lambda(v)$ and $\lambda(e)$ respectively.

%An edge connects two vertices: a source vertex and a target vertex. An edge type can be either directed or undirected. A directed edge has a clear semantic direction, from the source vertex to the target vertex. For simplicity of representation, all edges are undirected in this paper. But since an arrow in a category is directed, it is also feasible to convert a category into a directed property graph (and more precisely in some cases). 




%A property graph PG $(V, E)$  is a directed multi-graph with vertex set V and edge set E, where each node $v \in V$ and each edge $e \in  E$ has data properties consisting of multiple attribute-value pairs. The set of nodes of the graph G is denoted by V(G), and the set of edges of g is denoted by E(G). Note that an edge of E(G) is a set \{u, v\} $\subset$  V(G). For a node $v \in V(G)$, the label of v is denoted by $\lambda_G(v)$. For a node v of g, the set of neighbors of v is denoted by $nbr_G(v)$. Usually, the graph G is clear form the context, and then we may write just $\lambda(v)$, $\lambda(U)$ and $nbr(v)$ instead of $\lambda_G(v)$, $\lambda_G(u)$ and $nbr_G(v)$, respectively. Also, a node labeled with $\sigma$ is called a $\sigma$-node.


\begin{algorithm} \caption{Map to Property Graph Schema}\label{alg:map2graphschema}\KwIn{A graph representation $\mathcal{G(C)}$ of schema of a category $\mathcal{C}$ }
    \KwOut{A property graph  schema $\mathcal{PG}$= ($V, E, T, P$)}
    \DontPrintSemicolon

Initialize $V, E, T$, and $P$ to be empty sets;

\ForEach{ object $O$ in $V(\mathcal{G(C)})$ with outgoing edges}
{
     $V \leftarrow V \cup \{\lambda(O)\}$;

     \If{$O$ is an entity or attribute object  }
     {
        $P(O) \leftarrow P(O) \cup \{SK\}$;
     }
    
     Add\_neighbours($O$);
     
    %Process\_attributes($O$) 

    %Process\_entities\_relationships($O$)

    %Process\_relationships($O$)

    Mark $O$ as a processed object; 

}



\SetKwFunction{func}{ $\mbox{Add\_neighbours}$}
\SetKwProg{Fn}{Function}{:}{}
\Fn{\func{$O$}}
{
    \ForEach{ object $N$ in $outNbr(O)$ }
    {
        \If{$N$ is an attribute object and $N$ has no outgoing edges} 
        { 
            $P(O) \leftarrow P(O) \cup \{\lambda(N)\}$; 

            $T \leftarrow T \cup \{\lambda(N)\}$; 
        }
        \Else
        { 
            $E \leftarrow E \cup \{(O,N)\}$;

          %  $V \leftarrow V \cup \{N\}$
            
           % $P((O)) \leftarrow P((O)) \cup \{IDREF\}$
      
        }
         
        
        %\If{$N \in bin(O)$} 
        %{ 
        
        %    Add\_neighbours($N,S$)

        %    Mark $N$ as a processed object

        %}

        
        
    }
    
}   

%\SetKwFunction{func}{ %$\mbox{Process\_entities\_relationship}$}
%\SetKwProg{Fn}{Function}{:}{}
%\Fn{\func{$O$}}
%{
%    \ForEach{entity or relationship object $N$ in $outNbr(O)$ }
%    {
        
        %Create an edge $e$ between $N$ and $O$ and add $e$ into $E$ in schema

        %$E \leftarrow E \cup \{(N,O)\}$

        %$P((N)) \leftarrow P((N)) \cup \{ID\}$

        %$P((O)) \leftarrow P((O)) \cup \{IDREF\}$
     
        
    %}
%}

%\SetKwFunction{func}{ $\mbox{Process\_relationships}$}
%\SetKwProg{Fn}{Function}{:}{}
%\Fn{\func{$O$}}
%{
%    \ForEach{relationship object $R$ in $outNbr(O)$ }
%    {
            
%        \If{R is a binary relationship object} 
%        {
%            $P(R) \leftarrow P(R) \cup \{N \}$   
%        }
%        \Else 
%        {
            %create an edge $e$ between $R$ and $O$ into $E$ in the schema

%            $E \leftarrow E \cup \{(R,O)\}$
                
%        }
            
        
%    }
%}\end{algorithm}


%\begin{algorithm} \caption{Map to Property Graph}\label{alg:map2graph2}\input{algorithms/map2graph2}\end{algorithm}

%\item L is a finite set of labels (names) of $V \cup E$; 

\begin{definition} 
A (undirected) property graph schema is a 4-tuple ($V, E, T, P$) where 

\begin{itemize}[noitemsep,topsep=0pt]
  \item $V$ is a finite set of labels of vertices;
  \item $E \subseteq V \times V$, a finite set of labels of edges; 
  \item $T$ is a finite set of attributes;
  \item $P$ is a function, which maps a label in $V \cup E$ into a subset of $T$.
\end{itemize}
  
\end{definition} 


%\begin{figure}\centering\includegraphics[width=0.4\textwidth]{figures/outputgraph.jpg}\caption{The graph data converted from the unified category in Figure \ref{fig:firstexamplecategory}.} \label{fig:outputgraph}\end{figure}

%The algorithm follows the following steps, processing entity, relationship, and attribute objects in a specific order. If this entity object connects with another entity, the surrogate key of the connected entity is added as an IDREF attribute. 

%The associated attribute objects are included as attribute-value pairs for the corresponding node.
 
%Algorithm \ref{alg:map2graphschema} provides a mapping procedure for converting a category into a property graph schema.  Initially, convert each entity object in the category into a node in the property graph (Lines 2-3).  Then convert binary relationship objects into edges connecting two objects in the property graph (Lines 7-8). For the ternary or multi-way relationship objects, a node is created in the graph to represent the relationship (Lines 9-10). Assuming $N$ is a neighbor from  $O$. If $N$ is an attribute object, add $N$ as an attribute for $O$ (Lines 16-17). If $N$ is an entity (or relationship) object create a new edge in the graph for the object $N$ (Lines 18-19).   If the edge between $N$ and $O$ is bidirectional, add all outgoing neighbors of $N$ to an attribute value or an edge for $O$ (Line 20-21). 


Algorithm \ref{alg:map2graphschema} provides a mapping procedure designed to convert a categorical schema into a property graph schema.  The algorithm operates as follows: First, each object with outgoing edges in the category is transformed into a node within the property graph (Lines 2-7). In the case where the object represents an entity or a relationship, its surrogate key is added as a property, denoted by $SK$. Next, after any object $O$ is processed,  if its neighbor object $N$  is an attribute object without outgoing edges, it is added as an attribute for $O$ in the graph (Lines 10-12), otherwise, a new edge is created to connect $N$ and $O$ in the graph (Lines 14).






\begin{example}
Given the graph representation shown in Figure \ref{fig:1RRExample}(c), we use the above algorithm to convert it to a graph schema as follows: nodes $V= \{A,B,D\}$, edges $E= \{(A,B),(D,A)\}$, properties $T=\{SK, C, E\}$, $P(B)=\{SK, C\}$, $P(D)=\{SK, E\}$, and $P(A)=\{SK\}$.
\end{example}


\begin{example}
 Consider the graph representation of Figure \ref{fig:2RRExample}(c), we apply Algorithm \ref{alg:map2graphschema} again: nodes $V= \{A,B,X_1,X_2\}$, edges $E= \{(X_1,A),(X_1,B),(X_2,A)\}$, attributes $T=\{SK, C,D\}$, $P(A)=\{SK,C\}$,  $P(B)=\{SK,C\}$, $P(X_2)=\{SK,D\}$.
\end{example}

In addition, Figure \ref{fig:relationXML} shows another example for the output relations, XML and graph data based on the category from Figure \ref{fig:firstexamplecategory}.

%\noindent \textbf{Computation complexity}

%Figure \ref{fig:outputgraph} shows the output graph data based on the category from Figure \ref{fig:firstexamplecategory}.



%Here, $N$ represents the total number of elements across all objects, $d$ is the maximum in-degree observed among the objects, and $M$ corresponds to the maximum number of functions between any two objects. Importantly, the time complexity is bounded by the product of these factors due to the fact that each object is processed a maximum of $d$ times.


\section{Mapping to Hybrid Schema}
\label{sec:hybrid}

\begin{figure}
\centering
\includegraphics[width=0.7\textwidth]{figures/hybrid2.jpg}
\caption{An example to illustrate the output of two different models of data} \label{fig:hybrid}
\end{figure}

 %In the context of a multi-model database, it is often necessary to decompose a category into multiple components,  each of which may possess its own unique structure and model, serving specific aspects of the data. 
 
 
 In the context of a multi-model database, it is often necessary to decompose a categorical schema into multiple components,  each of which may possess its own unique structure and model, serving specific aspects of the data. To illustrate this concept, Figure \ref{fig:hybrid} demonstrates an example wherein customer data is divided into two distinct parts. The first part pertains to the information of customers, which is represented using the relational data model. The second part captures the friendship connections between customers and is best represented as a property graph model.
 
 
 Given a reduced representation $G$ of a category, we aim to decompose this representation $G$ into a set of subgraphs denoted as ${G_1, G_2, ..., G_n}$. The objective is to ensure that each node and edge in $G$ appear in at least one subgraph $G_i$, thereby preserving all information. Furthermore, when an edge $e$ belongs to a category $G_i$, the two nodes connected by $e$ must also be contained within $G_i$. This lossless decomposition approach guarantees that all nodes and edges can be retained after performing the decomposition. 
 
 
 %Consequently, each isolated component $G_i$ becomes amenable to individualized processing through the utilization of distinct algorithms, as shown in the preceding subsections.
 
 %Subsequently, each component $G_i$ can be processed individually using different algorithms, as explained in the previous subsections.
 






\section{Inference rules for FDs and MVDs}
\label{sec:monoidal}

\subsection{Inference Rules for FDs and Monoidal Category}


%Recall that in Section   \ref{sec:ClosureFD},  there are three inference rules of functional dependencies in Amstrong's axioms.    All three Amstrong's axioms can be applied to the category. In particular,  the first rule defines a projection morphism, and the second is about a composed morphism. Here we show that the second rule indeed defines a monoidal product.  



In Section \ref{sec:ClosureFD}, we introduced three inference rules governing functional dependencies within the framework of Armstrong's axioms. More specifically, the first rule establishes the concept of a projection morphism, while the third rule pertains to a composed morphism. In this appendix, we aim to demonstrate that the second rule in the following, in fact, delineates the properties of a monoidal product.


FD 2: If $f: X \to Y$, then $g: XZ \to YZ$;  Specifically, given an element $(x,z) \in XZ$, let $y = f(x)$, we define that $(y,z) \in YZ$ and $(y,z)$ is the image of $(x,z)$ under the function $g$. 


\begin{definition}(Monoidal Category) A monoidal category $\mathcal{C}$ consists of the following components:

\begin{itemize}
    \item A category $\mathcal{C}$ with objects denoted as $X, Y, Z, \ldots$ and morphisms between objects.
    \item A bifunctor $\otimes: \mathcal{C} \times \mathcal{C} \to \mathcal{C}$, called the monoidal product, which associates to each pair of objects $X$ and $Y$ an object $X \otimes Y$ in $\mathcal{C}$.
    \item An associator, which is a natural isomorphism:
    $ \alpha_{X, Y, Z}: (X \otimes Y) \otimes Z \xrightarrow{\sim} X \otimes (Y \otimes Z)$
    satisfying certain coherence conditions.
    \item A unit object $I$ and natural isomorphisms, called the left and right unitors: $ I \otimes X \xrightarrow{\sim} X$ and $  X \otimes I \xrightarrow{\sim} X$
    also satisfying coherence conditions.
\end{itemize}
\end{definition}

%Consider one category $\mathcal{C}_1$ with two objects $X$, $Y$ and a morphism $f: X \to Y$, and another category $\mathcal{C}_2$ with an object $Z$ and an identity morphism, then we define a monoidal category with $C_1$ and $C_2$, such that the bifunctor $\otimes: \mathcal{C}_1 \times \mathcal{C}_2 \to \mathcal{C}_3$, associate $X$, $Z$ into $XZ$, and $Y$, $Z$ into $YZ$. The morphism between $XZ$ and $YZ$ is defined as $k: \forall (x,z)\in XZ \to (y,z) \in YZ$, where $f(x)=y$.  A unit object $I$ in this monoidal category is an empty object $\epsilon$ with an identity morphism, such that $\forall x \in X$, $(x,\epsilon)=x$.  




Let us consider two distinct categories: $\mathcal{C}_1$, which consists of two objects, denoted as $X$ and $Y$, connected by a morphism $f: X \to Y$, and $\mathcal{C}_2$, featuring a single object $Z$ with an identity morphism. We aim to construct a monoidal category by combining $\mathcal{C}_1$ and $\mathcal{C}_2$, yielding $\mathcal{C}_3$. To achieve this, we introduce a bifunctor denoted as $\otimes: \mathcal{C}_1 \times \mathcal{C}_2 \to \mathcal{C}_3$. This bifunctor associates the pair $(X,Z)$ with the object $XZ$, and the pair $(Y,Z)$ with the object $YZ$. The morphism between objects $XZ$ and $YZ$ is defined as $k: \forall (x,z)\in XZ \to (y,z) \in YZ$, where $f(x)=y$.  Further,  a unit object $I$ is represented as an empty object denoted by $\epsilon$, accompanied by an identity morphism. This unit object $\epsilon$ satisfies the condition that for all elements $x$ within the object $X$, both pairs $(x,\epsilon)$ and $(\epsilon,x)$ correspond to $x$.


%\section{Connected components and pushout}
%\label{sec:pushout}

%The following example illustrates the connection between the pushout and the connected component of a graph.

%\begin{example} Given an undirected graph $G$, the \texttt{Edge} table includes two attributes \texttt{Node\_id1} and \texttt{Node\_id2}, which describes edges between any two nodes in $G$.   The pushout object \texttt{Component} computes the connected component in the graph $G$, as illustrated in the following commutative diagram.


%\[\xymatrix{Edge \ar[r]^{f} \ar[d]_{g} & Node\_id1 \ar[d]^{p_1} \\Node\_id2 \ar[r]_{p_2} & Component }\]\end{example}


%Consider that the pullback object corresponds to the join operator in relational databases, while the pushout object corresponds to the connected component in an undirected graph. As pullback and pushout are dual objects, an intriguing observation arises: the join operator in relational databases and the computation of connected components in graph databases demonstrate duality when examined through the framework of category theory.

\subsection{Inference Rules for MVDs} \label{sec:MVDInference}

Given a set of functional dependencies $F$ and a set of multivalued dependencies $M$, the inference rules to compute their closure can be found in literature, see e.g. \cite{10.5555/551350,10.1145/320613.320614,10.1145/509404.509414}. We provide those inference rules as follows. 
  
MVD 1: (Complement) If $X \to\to_U Y$, then $X \to\to_U (U-XY)$;

MVD 2: (Reflexivity) If $Y \subseteq X$ in a relation $U$, then $X \to\to_U Y$;

MVD 3: (Augmentation) If $Z \subseteq W$ and $X \to\to_U Y$, then $XW \to\to_U YZ$, where $U = X \cup Y \cup Z \cup W$;

MVD 4: (Transitivity) If $X \to\to_U Y$ and $Y \to\to_U Z$, then $X \to\to_U Z-Y$, where $U = X \cup Y \cup Z$.


There are two additional rules for both FD and MVD.

FD-MVD 1: If $X \to Y$ in a relation $U$, then $X \to\to_U Y$.

FD-MVD 2: If $X \to\to_U Z$ and $Y \to Z'$, ($Z' \subseteq Z$), where $U=X\cup Y \cup Z$, $Y$ and $Z$ are disjoint, then $X \to Z'$. 

For FD-MVD 2, given an element $x \in X$, we need to find a unique element $z' \in Z'$.  Without loss of generality, assume that $\exists y \in Y$ s.t.  $(x,y,z)\in U $, then since $Y \to Z'$, we can find this unique element $z' \in Z'$, s.t. $z'$ is the image of $y$ and $x$. Note that it is possible that there are multiple $y$'s associated with $x$ in $U$, but all those $y$'s should map to the same $z'$, otherwise, it contradicts with $X \to Z'$.

The detailed algorithm (by constructing the dependency basis) for computing the membership and closure of FDs and MVDs using the above rules can be found in previous work (e.g., \cite{10.1145/320613.320614,10.1145/322186.322190}).
 
 %one needs to compute the \textit{dependent basis} of an attribute $X$, denoted by $DEP(X)$. An MVD $X \to \to_U Y $ is in the closure if and only if $Y$ is a union of elements of $DEP(X)$. The algorithm to compute the dependent basis can be found in previous works, e.g.  \cite{10.1145/320613.320614}. Note that $X^+$  and $DEP(X)$ are indeed parallel concepts.  If we think of the collection of singleton sets, \{$(A)| A \in X^+$\}, then $X^+$ is just $DEP(X)$ that is functionally dependent on $X$ by FDs.


\section{Proofs}
\label{sec:appendix}



\section{Computational complexity}\label{sec:compl}

In this section, we show that both of our problems are \np-hard.
We saw in the previous section that if
we set $\alpha  = \infty$, then \problemcdcsm reduces to \problemdts, which can be solved exactly in
polynomial time. However, \problemcdcsm is \np-hard when $\alpha = 0$.
%However, we show next that the complexity of fair densestsubgraph $\problemcdcsm$ which allows $0 < \alpha \leq 1$ is $\np$-hard.


\begin{proposition} 
\label{prop:np}
\problemcdcsm is \np-hard.
\end{proposition} 
\begin{proof}
We prove the hardness from $k$-\prbclique, a problem where, given a graph $H$, we are asked if there is a clique of size at least $k$.

Assume that we are given a graph $H = (V, E)$ with $n$ nodes, $n \geq k$. We set $\alpha = 0$.
The graph snapshot $G_1$ consists of the
graph $H$ and an additional set of $k$ singleton vertices $U$. $G_2$ consists of a $k$-clique connecting the vertices in $U$. 

We claim that there is a subset $S$ yielding $\dens{S, \calG} =  (k - 1)/2$ if and only if there is an $k$-clique in $H$.

Assume that  there is a subset $S$ yielding $\dens{S, \calG} =  (k - 1)/2$.
Since the value of objective is $(k - 1)/2$, we have $\dens{S, G_1} = \dens{S, G_2} = (k - 1)/4$. 
Let $S = W \cup T$ where $W \subseteq V$ and denotes the subset of vertices from $H$ and $T \subseteq U$ is the subset of vertices from  $U$ in $S$. 

Assume that $\abs{W} < \abs{T}$. Since $\abs{T} \leq k$, $\abs{W} < k$.  The density induced on $G_1$ is bounded by $\dens{S, G_1} \leq \frac{{\abs{W} \choose 2}}{\abs{T} + \abs{W}} < \frac{{\abs{W} \choose 2}}{2\abs{W}} < \frac{k - 1}{4}$, which is a contradiction.
Assume that  $\abs{W} > \abs{T}$. Then the density induced on $G_2$ is bounded by $\dens{S, G_2} = \frac{{\abs{T} \choose 2}}{\abs{T} + \abs{W}} < \frac{{\abs{T} \choose 2}}{2\abs{T}} = \frac{\abs{T} - 1}{4} \leq \frac{k - 1}{4}$, again a contradiction. Therefore, $\abs{W} =  \abs{T}$. 

Consequently, $\dens{S, G_2} = (\abs{T} - 1)/4$, implying that $\abs{T} = k$. Finally, $\frac{k - 1}{4} = \dens{S, G_1} = \frac{\abs{E(S)}}{2k}$ implies that $\abs{E} = {k \choose 2}$, that is, $W$ is a $k$-clique in $H$.

%Let  $d_1$ and $d_2$ are the densities induced on $G_1$ and $G_2$ respectively by $S$.
%Based on our assumption, $\dens{S, \calG} =  (k - 1)/2$ yields and 
%the only way to obtain a density value  of $f(S, G_2) = (k - 1)/4$  which satisfies marginal density constraint is that  $S = W \cup U$ and thus there should be an $k$-clique in $H$.
%Therefore, if the density is $  (k - 1)/4$ there is an $k$-clique in $H$.


On the other hand, assume there is a clique $C$ of size $k$ in $H$.
%Let clique $C$ is formed by the set of vertices in $W$.
% Let $d_1$ and $d_2$ are the optimal densities induced on $G_1$ and $G_2$.
% Since the density constraint should be satisfied $d_1 \geq (d_1 + d_2)/2$ and $d_2 \geq (d_1 + d_2)/2$ which implies $d_1 \geq d_2$ and $d_2 \geq d_1$. Therefore, $d_1 = d_2$.  
% Let $k_1$ and $k_2$ number of non-singleton nodes contribute for $d_1$ and $d_2$ respectively.
% Since $d_1 = d_2$,  $\frac{ k_1(k_1 - 1)}{ 2(k_1 + k_2)} = \frac{k_2 (k_2 - 1)}{ 2(k_1 + k_2)}$ which implies $k_1 =  k_2$. Then  the sum of the densities is given by $\frac{ (k_1 - 1)}{ 2} $ where the optimal is  when $k_1  = k =  k_2$.
Set $S = C \cup U$. 
Immediately, $\dens{S, \calG} =   (k - 1)/2$ proving the claim.
\qed
\end{proof}

\iffalse
\begin{proposition}
\label{prop:inapproximability}
\problemcdcsm does not have any polynomial time approximation algorithm with an approximation ratio better than $n^{1-\epsilon}$ for any constant $\epsilon > 0$, unless $\np = \zpp$.
\end{proposition}
\fi

A similar proof will show that \problemcdcsdiff is \np-hard, and inapproximable.

\begin{proposition} 
\label{prop:np2}
\problemcdcsdiff is \np-hard.
Unless $\poly = \np$, there is no polynomial-time algorithm with multiplicative approximation guarantee for \problemcdcsdiff. 
\end{proposition} 

\begin{proof}
We use the same reduction from $k$-\prbclique as in the proof of Proposition~\ref{prop:np}.
We also set $\sigma = (k - 1)/2$. If there is a clique $C$ in $H$, then selecting $S = C \cup U$ yields $\diff{S, \calG} = 0$.
On the other hand, if $\diff{S, \calG} = 0$, then $\dens{S, G_1} = \dens{S, G_2} = (k - 1)/4$, and the argument in the proof of Proposition~\ref{prop:np} shows that there must be a $k$-clique in $H$.
In summary, the difference $\diff{S, \calG} = 0$ for a solution $S$ if and only if there is a $k$-clique in $H$.

This also immediately implies that there is no polynomial-time algorithm with multiplicative approximation guarantee since this algorithm can be then used to test whether there is a set $S$ with $\diff{S, \calG} = 0$.
\qed
\end{proof}



\end{document}
\endinput
%%
%% End of file `sample-authordraft.tex'.
