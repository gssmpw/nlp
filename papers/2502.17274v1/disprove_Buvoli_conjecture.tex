
% !TEX TS-program = pdflatex
% !TEX encoding = UTF-8 Unicode
\documentclass[12pt, leqno]{article}
\usepackage[utf8]{inputenc}
\usepackage{geometry}
\linespread{1.2}
\geometry{a4paper, scale=0.8}

%\usepackage{listings}
%\lstset{language=Matlab}      
% \usepackage[framed,numbered,autolinebreaks,useliterate]{mcode}

\usepackage{graphicx} % support the \includegraphics command and options
\graphicspath{{fig/}}

% \usepackage[parfill]{parskip} % Activate to begin paragraphs with an empty line rather than an indent

%%% PACKAGES
\usepackage{booktabs} % for much better looking tables
\usepackage{multirow}




\usepackage{array} % for better arrays (eg matrices) in maths
\usepackage{paralist} % very flexible & customisable lists (eg. enumerate/itemize, etc.)
\usepackage{verbatim} % adds environment for commenting out blocks of text & for better verbatim
\usepackage{subfig} % make it possible to include more than one captioned figure/table in a single float

\usepackage{mathtools}
\usepackage{bm}
\everymath{\displaystyle}
\allowdisplaybreaks

%%% HEADERS & FOOTERS
%\usepackage{fancyhdr} % This should be set AFTER setting up the page geometry
%\pagestyle{fancy} % options: empty , plain , fancy
%\renewcommand{\headrulewidth}{0pt} % customise the layout...
%\lhead{}\chead{}\rhead{}
%\lfoot{}\cfoot{\thepage}\rfoot{}

%%% SECTION TITLE APPEARANCE
%\usepackage{sectsty}
%\allsectionsfont{\sffamily\mdseries\upshape} % (See the fntguide.pdf for font help)
% (This matches ConTeXt defaults)

%%% ToC (table of contents) APPEARANCE
\usepackage[nottoc,notlof,notlot]{tocbibind} % Put the bibliography in the ToC
\usepackage[titles,subfigure]{tocloft} % Alter the style of the Table of Contents
\renewcommand{\cftsecfont}{\rmfamily\mdseries\upshape}
\renewcommand{\cftsecpagefont}{\rmfamily\mdseries\upshape} % No bold!

\usepackage{extarrows}
\usepackage{amsmath, amsthm, amsfonts, amssymb}
%\newtheorem*{remark}{Remark}

\usepackage{bbm}
\newtheorem{theorem}{Theorem}[section]
\newtheorem{lemma}[theorem]{Lemma}
\newtheorem{proposition}{proposition}
\theoremstyle{remark}
\newtheorem*{remark}{Remark}
\newtheorem{corollary}{Corollary}[section] 

\theoremstyle{definition}
\newtheorem{example}{Example}[section]
\newtheorem{definition}{Definition}[section]

\usepackage{xcolor}

\usepackage[colorlinks,
linkcolor=blue,       
anchorcolor=blue,  
citecolor=blue,        
]{hyperref}
\usepackage{cleveref}

\numberwithin{equation}{section}

\title{The Stability and Accuracy of The Adams-Bashforth-type Integrator}
\usepackage{authblk}

\author[a]{Daopeng Yin}
\author[a]{Liquan Mei \thanks{Corresponding author: lqmei@mail.xjtu.edu.cn}}
\affil[a]{School of Mathematics and Statistics. Xi'an Jiaotong University. No.28, Xianning Xi Lu, Xi'an, Shaanxi, 710049, P.R. China}
\renewcommand*{\Affilfont}{\small\it} % 修改机构名称的字体与大小
\renewcommand\Authands{ and } % 去掉 and 前的逗号

\newcommand\keywords[1]{\textbf{Keywords}: #1}
\date{}
\begin{document}
\maketitle
\begin{abstract}
This paper presents stability and accuracy analysis of a high-order explicit time stepping scheme introduced by \cite[Section 2.2]{Buvoli2019}, which exhibits superior stability compared to classical Adams-Bashforth. A conjecture that is supported by several numerical phenomena in \cite[Figure 2.5]{Buvoli2018}, the method appears to remain stable when the accuracy approaches infinity, although it is not yet proven. It is regrettable that this hypothesis has been refuted from a fundamental perspective in harmonic analysis. Notwithstanding the aforementioned, this method displays considerably enhanced stability in comparison to conventional explicit schemes.  Furthermore, we present a criterion for ascertaining the maximum permissible accuracy for a given specific parabolic stability radius. Conversely, the original method will lose one order associated with the expected accuracy, which can be recovered with a slight modification. Consequently, a unified analysis strategy for the \( L^2 \)-stability will be presented for extensional PDEs under the CFL condition. Finally, a selection of representative numerical examples will be shown in order to substantiate the theoretical analysis.  
\end{abstract}

\keywords{Adams-Bashforth-type integrator, region of absolute stability, limiting stability conjecture,  \( L^2 \)-stability, CFL condition.} \\
\par 
\textbf{AMS subject classifications(2020): 65L20, 65M15, 65E99}
\section{Introduction}
When dealing with time evolution equations, explicit time-discretization schemes offer advantages such as high computational efficiency and clearer uniqueness of numerical solutions. Particularly, theses algorithm have remarkable advantage for the non-linear evolutionary equations.  However, their algorithmic stability has consistently hindered the promotion of higher-order explicit schemes. 
\par 
In literature \cite{Dahlquist1963}, the concept of Dahlquist’s stability barrier be introduced, which says  that there exists no A-stable explicit \( k \)-step method. Historically, numerous scholars have devoted considerable effort to enhancing the stability of explicit schemes. M. Ghrist, et al  consider a variations of the Adams–Bashforth (AB) via staggered grid techniques which improve the range of the stability area on the imaginary axis shown by \cite[Table 4.2, Table 4.3]{Ghrist2000}.   Although there are some methods called explicit ways to achieve A-stability \cite{Norsett1969, Chollom2012}, there are some discrepancies in the essence of the classic A-stability description.   Recently, a review article \cite{Givoli2023} illustrates  two types of barrier breakers.  The existing articles essentially only improve the stability of explicit schemes and cannot strictly achieve A-stability. Therefore, it is also necessary to examine the conditionally stable behavior of explicit schemes.

\par
A recently published article \cite{Buvoli2019} has proposed an intriguing algorithm that employs the Taylor expansion, the Cauchy integral formula on the complex time plane \cite{Henrici1993}, and its discretization using the trapezoidal rule \cite{Austin2014, Trefethen2014}. This algorithm is outlined in detail in reference \cite[Section 2]{Buvoli2019} of the cited article. 
In section \ref{sec:Methodological review}, we will elaborate on the connection between this scheme and the AB method. For this reason, this method will be referred to in our description as the Adams-Bashforth-type integrator (ABTI). 
Prior to this, there have been successful cases where traditional time discretization methods were extended to the complex plane for the construction and solution of differential equations \cite{Corliss1980a,Fornberg2011,Orendt2009,Hansen2009}. Numerical differentiation methods based on interpolation at the unit roots of the complex plane have effectively overcome the ill-conditioning of classical difference schemes \cite{Lyness1967,Fornberg1981}.
\par 
Considering the initial value problem \( \partial_t u(t) = f(t,u(t)) \) with given initial data \( u(0) = u^0 \), the ABTI can be expressed in matrix form 
\begin{equation}
 \mathbf{u}^{[n+1]}=\mathbf{A}\mathbf{u}^{[n]} + r \mathbf{B}(\alpha)\mathbf{f}^{[n]},  \text{ when } n \ge 1, 
 \label{eq:main_integrator}
\end{equation} 
where \( \mathbf{u}^{[n+1]}, \mathbf{u}^{[n]} \) represent the solution vector, \( \mathbf{A}, \mathbf{B}(\alpha) \) are discrete-time matrix and \( \mathbf{f}^{[n]} = f(\mathbf{t}^{[n]}, \mathbf{u}^{[n]}) \).  Here, the parameter \( \alpha = (t^{n+1} - t^{n})/r \) under the uniform temporal discrete grid. The initial solution vector \( \mathbf{u}^{[0]} \) obtained using so-called iterator \( \mathbf{u}^{[0]} = u^0 \mathbbm{1} + r\mathbf{B}(0)\mathbf{f}^{[0]} \). Finally, the classical numerical solution, represented by the variable \( u^{n+1} \), is reconstructed at discrete time level \( n+1 \) by the component of the solution vector, denoted by \( \mathbf{u}^{[n+1]} \). In the context of a ABTI situation, \( u^{n+1} \) is equal to the arithmetic mean value of all entries in \( \mathbf{u}^{[n+1]} \).
Indeed, if we were to take the arithmetic mean of both sides of equation \eqref{eq:main_integrator}, we would be able to discern the relationship between ABTI and classical ABs. This is due to the fact that the AB method is designed to perform polynomial approximation with previous multiple nodes for the function \( f(t, u(t)) \) in the integral form of the ordinary differential equation (ODE)
\begin{equation}
u(t_{n+1}) = u(t_n) + \int_{t_n}^{t_{n+1}}f(t,u(t)) \mathrm{d}t. \nonumber
\end{equation}
The distinction between the ABTI and ABs can be attributed to the manner of approximation employed for the function \( f(t, u(t)) \). Notwithstanding, there are notable enhancements in the stability of the ABTI for high-order schemes when compared to ABs and other explicit schemes.
\par
In order to gain insight into the stability boundaries of ABTI, it is first necessary to review some of the fundamental concepts associated with the linear stability analysis \cite[Chapter 5]{Hairer1996}. The Dalhquist problem, given by the differential equation \( \partial_t u(t) = \lambda u(t),  \lambda \in \{z\in \mathbb{C}; \mathtt{Re}(z) \le 0 \} \) with initial data \( u(0) = 1 \), can be discretized using the first-order Adams-Bashforth  (AB1) method. The numerical solution can then be rewritten as  \( u^{n+1} = R(z)u^n\)  where \( R(z) = 1+z \) be called stability function with \( z:= \lambda \tau \),  and \( \mathcal{S} := \{z \in \mathbb{C}; \lvert R(z) \rvert < 1\} \) the region of absolute stability correspondingly. Similarly, these concepts can be extended to encompass ABTI, as defined below.
\begin{definition}
\label{def:absolute_stability_region}
Under the consideration of ABTI for Dalhquist problem, the matrix-value stability function defined by  \( \mathbf{R}(z) = \mathbf{A} + z \mathbf{B}(\alpha)/\alpha \) with \( z:= \tau \lambda \) such that the numerical solution vector \( \mathbf{u}^{[n+1]} = \mathbf{R}(z)\mathbf{u}^{[n]} \) and the absolute stability region described by spectral radius \( \mathcal{S}_{ q } := \{z \in \mathbb{C}; \rho(\mathbf{R}(z)) < 1 \}  \) where \( \mathbf{R}(z) \) is a \( s \)-dimensional square matrix where \( s \ge q \).
\end{definition}
Regarding the ABTI algorithm \cite[Section 2.2]{Buvoli2019}, Buvoli gave a guess to quantify the degree of stability improvement compared to existed explicit schemes. For the AB method, we have a general understanding that the region of absolute stability will shrink to origin rapidly as the order of accuracy increases. This means that stronger time-step constraints are required when calculating stiffness problems. At the same time, when the scheme is applied to a parabolic problem, the number of Courant-Friedrichs-Lewy (CFL) conditions is smaller, i.e., strong step size limit is also required. In contrast, using the definition \ref{def:absolute_stability_region} to characterize stability, Buvoli conjecture that the region of absolute stability of the ABTI will tend to a limiting region of absolute stability (centered on the origin and \( 1/e \) as the left semicircle of radius) as the approximation order increases, viz,  
\begin{equation}
\label{eq:uniform_stability_domain}
\lim\limits_{q\to \infty} \mathcal{S}_{q} \to \mathcal{S}_{\infty} := \left\{z\in \mathbb{C}, \mathtt{Re}(z) < 0 \text{ and } |z|< \frac{1}{e} \right\}, \nonumber
\end{equation} 
where \( e \) is the Euler's number. 
This signifies that the issue of stability anxiety can be effectively addressed through the utilisation of a higher-order explicit scheme. In other words, when a uniform step size constraint is applied, ABTI is still capable of ensuring stability for arbitrary higher-order discrete calculations. Furthermore, given the nature of component parallel computing in ABTI, concerns regarding the computational efficiency of this method are unwarranted, provided that the requisite computing resources are available. To the best of my knowledge, this result has only been numerically illustrated \cite[Figure 2.5]{Buvoli2019}, and has not yet been mathematically rigorously proven. This is one of the tasks that should be completed in this work.
%{\color{gray}While the ABTI also exhibits the capacity to converge towards the origin as the approximation order increases, the rate of shrinkage gradually diminishes and ultimately evolves into a semicircular region}
\par
In this paper, we disprove the conjecture put forth by T. Buvoli through a series of methods and provide a criterion for determining the relationship between the parabolic stability radius and the maximum approximation accuracy. On the other hand, the accuracy of original ABTI is one order of magnitude lower than the ideal order. We identify the underlying cause of this discrepancy and restore the accuracy of ABTI through a straightforward correction. By applying the tensor matrix eigenvalue formula, we extend the results of linear stability analysis to linear parabolic equations, identifying the optimal parabolic CFL condition for \( L^2 \)-stability. Finally, a series of illustrative numerical examples are presented.
\par
Throughout this paper, we denote by \( C \) a generic positive constant that is independent of the temporal and spatial size of discretization \( \tau, h \). Let \( \mathbf{B} \) as the abbreviations of \( \mathbf{B}(\alpha)/\alpha \) without causing ambiguity. \( \mathtt{BV}(\Omega) \) means the bounded variation space equided with norm \( \operatorname{TV}( \mathbf{u}^{[n]} ):= \Sigma_j \lvert u_{j}^{[n]} - u_{j-1}^{[n]} \rvert \) and \( L^2( \Omega) \) the classical  square integrable space. $\gamma_{n}(z):= z^n/n! $ be denoted as Gelfand-Shilov function, for which the exponential function \( e^z \) can be represented by \( \Sigma_{ \nu = 0}^{\infty}\gamma_n(z) \) in sense of series expansion. $\mathtt{Re}(z)$ and $\mathtt{Im}(z)$ stand for the real and imaginary part of complex number $z$ respectively. The   $s$ distinct $s$-th roots of unity $\omega_{j}:=e^{\frac{i 2\pi j}{s}}$ for all $j=1,2, \ldots, s$ whose $i = \sqrt{-1}$. The discrete complex time nodes \( t_j^{[n]} := t_{n} + r \omega_{j} \) for all \( j = 1, \ldots, s \) with \( t_{n} \) is traditional time node and \( r  \) is arbitrary non-negative real number stand for radius. Correspondingly, \( u_{j}^{[n]} \) or \( f_{j}^{[n]} \) represent the approximation value or sampling value at the complex time discrete node \( t_j^{[n]} \). \( \delta_{q} \) is a selective variable such that \( \delta_q = 1 \) whenever \( s = q \) and vanishing for all \( s > q \).
%The discrete Fourier transform $\hat{f}_{\nu}^{[n]} := \tfrac{1}{s}\Sigma_{k=1}^{s}f_k^{[n]}e^{-i \nu \theta_k}, 0 \le \nu \le q - 1 < s$.  In the matrix-vector presentation of temporal discretization, we refer \( \mathbf{u}^{[n]} \) to the solution vector. The traditional numerical solution at $n$-th time level is $u^{n} = \tfrac{1}{s}\Sigma_{j=1}^{s}u_j^{[n]} $ which also equals to $ \hat{u}_{0}^{[n]}$. 
\par
The rest of this paper is organized as follows. In Section \ref{sec:Methodological review},  we review the construction of the ABTI algorithm and identify a certain structure present in its matrix expression. At the same time, we compare it with the classical first-order Adams-Bashforth method to facilitate our understanding of such algorithms. In Section \ref{sec:The Main Results}, we presents the main results of the paper, including the mathematical analysis of the Buvoli's conjecture and subtle modifications to the original scheme to achieve the optimal convergence order. For the sake of readability, some of the more tedious derivations are placed in Appendix \ref{sec:Derivation of Characteristic Polynomial}. In Section \ref{sec:Applications}, 
we will present the application of the ABTI method to parabolic equations, deriving the \( L^2 \)-stability and \( L^2 \)-error under conditions compatible with ODE problems. In Section \ref{sec:Numerical Verifications} 
we will numerically validate our analytical results.  In Appendix \ref{sec:Derivation of Characteristic Polynomial}, we derives the characteristic polynomial of the amplification matrix, and it will be placed at the end of the paper as preparatory work for Section \ref{sec:The Main Results}.
\section{Methodological review}
\label{sec:Methodological review}
In this section, we restatement the formulation of ABTI \cite[Section 2]{Buvoli2019}.
 As a starting, we consider a typical ODE problem
\begin{equation}
\label{eq:ODE}
\partial_t u = f(t,u), \text{ with } u(0) = u^0, 
\end{equation}
and its numerical treatment by ABTI. 
\par 
We outline the main characteristics of the methods by comparing them with the traditional AB1 method, specifically the explicit Euler method. The comparison ranges from the AB1 scheme to higher-order time-stepping methods in the complex plane. The AB1 method requires the initial value at \( t_0 \) and obtains the numerical solution for all \( n \ge 0 \) through repeated iterations from \( t_n \) to \( t_{n+1} \). This process approximates the values at subsequent time steps. Since the time-stepping scheme constructed from \( t_n \) to \( t_{n+1} \) is a linear approximation, the optimal approximation, in the usual sense, can only achieve first-order accuracy. To construct higher-order AB methods, values from previous time nodes are typically needed. It is well-known that derivatives describe the local behavior at a specific moment. However, the involvement of non-local nodes in higher-order numerical schemes and the values they approximate inherently lead to a contradiction, similar to the overfitting phenomenon of higher-order polynomial approximation in data fitting. This contradiction can only be resolved when the time step approaches zero. Consequently, the stability of the scheme requires more stringent constraints on the time step size.  
%我们将时间变量视为一个复变量,并将原来的时间节点tn,tn+1为中心的s个单位根视为一个整体,并分别记为tn, tn+1。这个算法的实现包含三个主要的环节:
%1. 预处理(启动向量) - 在传统的AB方法中,格式的启动值u^{0}严格的等于方程的初始条件。在DCI中,格式的启动值实质上是一个向量,然而我们仅仅知道实时间节点t0处的值。为此,我们需要通过某种方式将u^{0} 和 u^{0} 建立联系,实质上这种联系可以通过方程所提供的信息进行建立。(升维)
%
%2. 解向量步进-通过格式的构造得到从体n到体n+1处的时间步进。以tn+1为中心的s个单位根处的近似值。
%3. 后处理(数值解重构) - 由启动向量和解向量步进我们得到所有n\ge0的解向量。然而,我们最希望得到的是实时间节点tn处的值。将tn+1处的信息重构出实时间节点处的近似值tn+1.
%总体来说,从时间步进过程上看,DCI与AB1具有很高的相似性并且当$r \to 0 $时DCI能够退化成为AB1方法。
\par 
We treat the time as a complex variable and consider the original time nodes \( t_n \) as vector,  centered at \( t_n \) around \( s \) unit roots, and denote them as \( \mathbf{t}^{[n]} \). The implementation of this algorithm involves three main steps:
\begin{enumerate}
\item
\textbf{Preprocessing} (Initialization Vector) – In the traditional AB method, the initial value \( u^{0} \) is strictly equal to the initial condition of the equation. In the ABTI method, the initialization value is essentially a vector, but we only know the value at the real time node \( t_0 \). Therefore, we need to somehow establish a connection between \( u^{0} \) and \( \mathbf{u}^{[0]} \), which can be done using the information provided by the differential equation. This is exactly what we will introduce next: the iterators.
\item
\textbf{Time-stepping} (Solution Vector Updating)– From the constructed scheme, we compute the numerical solution vector \( \mathbf{u}^{[n+1]} \) from the given \( \mathbf{u}^{[n]} \) and the right-hand-side function \( f \). This is exactly what we will introduce next: the propagator.
\item
\textbf{Postprocessing} (Numerical Solution Reconstruction) – From the previous two steps \textbf{preprocessing} and \textbf{time-stepping}, we obtain the solution vectors \( \mathbf{u}^{[n]} \) for all \( n \ge 0 \). However, the primary goal is to obtain the values at the real time node \( t_{n} \). The information at \(  \mathbf{t}^{[n]} \) is then used to reconstruct the approximate value at the real time node \( t_n \). According to Schwarz's reflection principle, we know that the function values of an analytic function at conjugate nodes are complex conjugates of each other. Thus, the numerical solution at the real time node \( t_n \) can be expressed as the arithmetic mean of all components of the solution vector \( \mathbf{u}^{[n]} \), namely \( u^{n}=\frac{1}{s}\sum_{j=1}^{s}u_{j}^{[n]}\).
\end{enumerate}
\par 
Overall, from the perspective of the \textbf{time stepping} process, ABTI has a high similarity to the AB1 method and, as \( r \to 0 \), ABTI can degenerate into the AB1. The locality of the ABTI method aligns with the locality of the derivative, and this consistency ensures that the ABTI method maintains good numerical stability. In other words, we can also consider preprocessing and postprocessing as processes of dimension extension and dimension reduction, respectively.
\par
 Next, we will review the details of the scheme construction. Since the Cauchy–Kowalevski theorem guarantees analyticity of solution of original ODE \eqref{eq:ODE} inside the proper circular domain, the \( u(t_j^{[n+1]}) = u(t_{n+1}+r \omega_j) = u(t_{n}+\tau+r \omega_{j}) \) can be approximated by the Laurent expansion  at \( t_{n+1} \) or \( t_{n} \) that 
\begin{subequations}
\begin{align}
 u(t_j^{[n+1]})  \approx u^{n+1} + \sum\limits_{\nu=1}^{q}  \frac{u^{(\nu)}(t_{n+1})}{\nu!} (r \omega_{j})^{\nu}, \label{eq:pre_iterator}\\
 u(t_j^{[n+1]})  \approx u^{n} + \sum\limits_{\nu=1}^{q}  \frac{u^{(\nu)}(t_{n})}{\nu!} (r \omega_{j}+\tau)^{\nu}. \label{eq:pre_propagator}
\end{align}
Due to  the similarity of treatment about expansion coefficients in \eqref{eq:pre_iterator} and \eqref{eq:pre_propagator}, the statement focus on the second one. The approximation \eqref{eq:pre_propagator} whose coefficients of the Laurent series can be represented by the Cauchy integral formula \cite{Bornemann2010}
\begin{align*}
\displaystyle
\frac{u^{(\nu)}(t_n)}{\nu !} = 
\begin{cases}
\dfrac{1}{2 \pi i}\oint_{\Gamma}\dfrac{u(z)}{(z-t_{n})}\mathrm{d}z, & \nu = 0, \\
\dfrac{1}{2 \pi \nu i} \oint_{\Gamma}\dfrac{u^{\prime}(z)}{(z-t_n)^{\nu}}\mathrm{d}z = \dfrac{1}{2 \pi \nu i} \oint_{\Gamma}\dfrac{f(z, u(z))}{(z-t_n)^{\nu}}\mathrm{d}z , & \nu \ge 1.
\end{cases}
\end{align*}
Equivalently, choosing a relatively simple \( \Gamma \) to be a circular of contour radius \( r \) centered at \( t_n \) and utilizing the change of variables  \( z = t_n + re^{i \theta} \), then the above expression is equivalent to 
\begin{align*}
\frac{u^{(\nu)}(t_n)}{\nu !} = 
\begin{cases}
\dfrac{1}{2\pi} \int_{0}^{2\pi}u(r e^{i \theta} + t_n) \mathrm{d}\theta, & \nu = 0, \\
\dfrac{1}{2 \pi \nu r^{\nu-1}}\int_{0}^{2\pi}\dfrac{f(re^{i \theta} + t_n, u(re^{i \theta} + t_n))}{e^{i(\nu-1)\theta}}\mathrm{d}\theta,  & \nu \ge 1.
\end{cases}
\end{align*}
\end{subequations}
The processing of integrals in the continuous sense remains challenging for computer programs and necessitates additional discretization. The trapezoidal rule \cite{Trefethen2014}
% , whose error bounded by \(  M \frac{( \rho^j + \rho^{-j})}{ \rho^{s}-1} \text{ with  } \left| u(z) \right| \le M
% \), viz, an exponential accuracy approximation\cite{Henrici1979}, 
 discrete the above formula consequently that 
\begin{equation}
\label{eq:trapezoidal_rule}
\frac{u^{(\nu)}(t_n)}{\nu !} \approx
\begin{cases}
\frac{1}{s}\sum\limits_{k=1}^{s}u_k^{[n]}, & \nu = 0, \\
\frac{1}{\nu r^{\nu-1}}\cdot \frac{1}{s}\sum\limits_{k=1}^{s}f_k^{[n]}e^{-i(\nu-1)\theta_k} = \frac{\hat{f}_{\nu-1}^{[n]}}{\nu r^{\nu-1}}, & q \ge \nu \ge 1,
\end{cases}
\end{equation}
where \( u_k^{[n]}, f_k^{[n]} \) stand for the value of functions \( u, f \) at sample point \( t_n + r\omega_k \) and $ s $ means number of sampling points. 
%{\color{blue}\par The derivation of \eqref{eq:Taylor_expansion}, \eqref{eq:trapezoidal_rule}, \eqref{eq:component_form} and \eqref{eq:component_form_0} should use the Laurant expansion of analytic function with the Cauchy integral representation of whose coefficients ,  and the trapezoidal sums for integrand disretization. }
Thence, \eqref{eq:pre_iterator} and \eqref{eq:pre_propagator} have 
\begin{subequations}
\begin{align}
u(t_j^{[n+1]}) \approx & u^{n+1} + r \sum\limits_{\nu=1}^{q}\frac{\omega_{j}^{\nu}}{\nu}\hat{f}_{\nu-1}^{[n+1]} =\frac{1}{s}\sum\limits_{k=1}^{s}u_k^{[n+1]} + r \sum_{\nu=1}^{q}\int_{s=0}^{\omega_{j}}s^{\nu-1} \mathrm{d}s \cdot  \hat{f}_{\nu-1}^{[n+1]} =  u_j^{[n+1]}, \label{eq:component_iterator} \\
\text{ and   ~~~~~~~~~~~~~~} &  \nonumber\\
u(t_j^{[n+1]}) \approx & \frac{1}{s}\sum\limits_{k=1}^{s}u_k^{[n]} + \sum_{\nu=1}^{q} \left(\frac{\hat{f}_{\nu-1}^{[n]}}{\nu r^{\nu-1}}\right) (r \omega_{j} + \tau)^{\nu} \xlongequal{\tau  = \alpha r }
\frac{1}{s}\sum\limits_{k=1}^{s}u_k^{[n]} + r \sum\limits_{\nu=1}^{q}\frac{(\alpha+\omega_{j})^{\nu}}{\nu}\hat{f}_{\nu-1}^{[n]} \nonumber \\
= & \frac{1}{s}\sum\limits_{k=1}^{s}u_k^{[n]} + r \sum\limits_{\nu=1}^{q}\int_{s=0}^{\alpha+\omega_j}s^{\nu-1}\mathrm{d}s \cdot \hat{f}_{\nu-1}^{[n]}  = u_j^{[n+1]}, \label{eq:component_propagator}
\end{align}
\end{subequations}
where \( 1 \le j \le s \).
 \begin{figure}[!t]
\centering
\includegraphics[scale=.35]{dci.eps}
\caption{Obtain the initial solution vector $\mathbf{u}^{[0]}$ by iterator \eqref{eq:iterator}(Left) and update each of component \( u_j^{[n+1]} \) of solution vector \( \mathbf{u}^{[n+1]} \) by propagator \eqref{eq:propagator} (Right). }
\label{fig:fig_dci}
\end{figure}
Summarily, the component form of time discrete scheme \eqref{eq:component_iterator} and \eqref{eq:component_propagator} can be represented by matrix-vector form
\begin{subequations}\label{eq:time scheme}
\begin{align}
\text{iterator: }  &  \mathbf{u}^{[n+1]} = \mathbf{A}\mathbf{u}^{[n+1]} + r\mathbf{B}(0)\mathbf{f}^{[n+1]}, \label{eq:iterator}\\
\text{propagator: }  &  \mathbf{u}^{[n+1]}=\mathbf{A}\mathbf{u}^{[n]} + r \mathbf{B}(\alpha)\mathbf{f}^{[n]},  \text{ when } n \ge 1, \label{eq:propagator}
\end{align} 
\end{subequations}
where 
\begin{itemize}
\item
\( \mathbf{u}^{[n]}:=[u_j^{[n]}]_{j=1}^s \) and \( \mathbf{f}^{[n]}:=[f_j^{[n]}]_{j=1}^s \) stand for solution vector and right-hand side vector respectively;
\item 
\(  \mathbf{A} \)  is an \( s \)-dimensional all-one square matrix multiplied by \( 1/s \);
\item $\mathbf{B}(\alpha) := \mathbf{S}(\alpha)\mathbf{F}$ whose
\begin{align*}
\mathbf{S}(\alpha) =
\begin{bmatrix}
\sigma_{1,1} & \ldots & \sigma_{1,q} \\
\vdots & \vdots & \vdots \\
\sigma_{s,1} & \ldots & \sigma_{s,q} 
\end{bmatrix} 
\text{ and }
\mathbf{F} = 
\frac{1}{s} 
\begin{bmatrix}
\omega_1^0 & \ldots & \omega_s^0 \\
\vdots & \ddots & \vdots \\
\omega_{1}^{1-q} & \ldots & \omega_s^{1-q} \\
\end{bmatrix}_{q\times s}
% = \{\omega_{k}^{-(j-1)}\}_{j=1,k=1}^{q, s}
\end{align*}
where \( \sigma_{j,k} = \sigma_{j,k}(\alpha) :=  \int_{x=0}^{\alpha + \omega_j}x^{k-1} \mathrm{d}x\). Here, \( \mathbf{S}(\alpha) \) can be regarded as a generalized Vandermonde matrix under the integral sense and \( \mathbf{F} \) is a non-equiv row-column discrete Fourier matrix.
\end{itemize}
One of the main utilities of the iterator is to obtain the initial solution vector \( \mathbf{u}^{[0]} \) by replacing \( n+1 \) as \( 0 \) directly, namely
\begin{equation}
\mathbf{u}^{[0]} = u^0 \mathbbm{1} + r\mathbf{B}(0)\mathbf{f}^{[0]}
\end{equation}
where \( \mathbbm{1} \) is \( s \)-dimensional all-one column vector. 
\par 
In the original paper \cite{Buvoli2018}, the number of sampling points \( s \) and the number of terms in the Taylor expansion \( q \) are equal. Here, we distinguish between \( s \) and \( q \) because we find that when \( s = q \), the accuracy of the scheme is one order lower than the ideal approximation of order \( q \). The \( q \)-order accuracy is referred to as the ideal approximation because the construction of the scheme involves both the truncation error of the Taylor expansion and the error in approximating the Cauchy integral formula, with the latter achieving exponential convergence, thus not dominating the overall error.
\par
In addition to its good stability, ABTI also possesses several other characteristics. From equation \eqref{eq:propagator}, it is not difficult to observe that once the matrix is determined, the time-stepping procedure is very similar to the explicit Euler method, making the programming of ABTI very straightforward. Since it is an explicit scheme, we can see that each component of the unknown vector on the left-hand side of \eqref{eq:propagator} can be computed independently. This characteristic becomes even more evident when the matrix form of the scheme is reverted to the component form as presented in the original paper. Upon examining the matrix representation of the scheme, we observe that the matrix exhibits a certain structure, which also opens up possibilities for algorithm acceleration or GPU programming. For example, there are many acceleration algorithms for the multiplication of Fourier matrices and vectors e.g. fast Fourier transform (FFT)\cite{Cooley1965}.
\section{The Main Results}
\label{sec:The Main Results}
This section includes two parts. First, there is the theoretical analysis regarding the conjecture that ABTI possesses limiting stability. Secondly, an analysis of the reasons why the original ABTI cannot achieve the desired convergence order is provided, along with potential solutions.
%首先是关于DCI具备极限稳定性猜测的理论分析。其次是原格式无法达到理想收敛阶的原因分析并提供解决方案。
\subsection{Disproval  of a conjecture}
At the outset of this subsection, we will present a concise overview of the proof methodology that we have devised to evaluate Buvoli's conjecture. A relatively straightforward approach is to identify the characteristic polynomial of the amplification matrix and examine the relationship between the control parameter $ z \in \mathbb{C} $ and the Schur stability of the polynomial. This involves locating $z$ the complex plane in a way that ensures all the zeros of the polynomial are situated within the unit disk. In the event that the amplification matrix is expressed explicitly, the characteristic polynomial is not readily apparent. Initially, it is determined that the matrix $ \mathbf{B}(\alpha) $ in the scheme can be expressed as the product of two matrices \( \mathbf{B}(\alpha) = \mathbf{S}(\alpha)\mathbf{F} \) with a particular structure, using the aforementioned matrix structure and the properties of the block matrix determinant, it was determined that a particularly simple band matrix exhibits an identical characteristic polynomial to that of the matrix $ \mathbf{A} + z \mathbf{B}(\alpha) $. The difference equation is established with the value of the determinant, and the explicit expression the characteristic polynomial is obtained through two algebraic matrix inversions. Subsequently, the analysis of the absolute stability region is transformed into an analysis of the zeros of the polynomial. 
\par 
By employing the analytical approach of \textit{the root locus curve} \cite[V.I]{Hairer1996}, a straightforward variable substitution is employed to derive the parabolic radius of the stable region\footnote{Here, the parabolic radius refers to the length of the segment along the real axis, starting from the origin and extending toward \( - \infty \), that reaches the boundary of the stability region. Essentially, this means that the spectral points of the Laplace operator \( \Delta \) in the parabolic equation lie on the negative real axis and can extend to \( - \infty \). The discretization of the Laplace operator, or more specifically, the spectrum of the discretized Laplace operator, must fall within the allowable range of the time scheme's parabolic radius to ensure the stability of the scheme.}, which is a crucial prerequisite for validating the conjecture. Consequently, our focus can be narrowed to the distribution of zeros of a real-coefficient polynomial with solely real zeros. These polynomials can be regarded as a polynomial sequence with respect to the index $q$, and the generating function of this polynomial sequence is obtained by simple derivation. Subsequent value of the polynomial at a specified real point is represented as the Fourier transform value of a function derived from a generating function through the Cauchy integral formula and variable substitution. Ultimately, in consideration of the relationship between the function's smoothness and the decay of the Fourier coefficient, it is demonstrated that the conjecture cannot be guaranteed to hold when $q$ tends to infinity. 
\par 
Nevertheless, this scheme demonstrates superior stability compared to that of ABs of equivalent accuracy, particularly in the case of high-order approximation. The stability of the scheme can be expressed as follows: as the approximation accuracy increases, the stable region will gradually diminish, and the rate of this diminution will become increasingly slow. Consequently, we also provide the quantization of the maximum  approximation accuracy that can guarantee the stability of the scheme for a given parabolic radius. In other words, this is a necessary condition for the stability of the scheme for arbitrary approximation accuracy when solving parabolic problems. Conversely, for a given precision, the parabolic radius can be determined by solving for the smallest modulus zero of the polynomial.
\par
To highlight the core ideas of the proof and enhance the readability of the article, we have placed the derivation of the characteristic polynomial of \(  \mathbf{A} + z \mathbf{B} \) in the appendix \ref{sec:Derivation of Characteristic Polynomial}. In this section, we only present the final expression of the characteristic polynomial
\begin{equation}
\label{eq:characteristic polynomial}
 p_q(\lambda; \mathbf{A} + z \mathbf{B}) = f_q( \lambda; z) + f _{q-1}( \lambda; z) - \gamma_{q}(-z/\alpha)\delta_{q}, 
\end{equation} 
where \( f_q( \lambda; z) :=  \sum_{j=0}^{q} \gamma_{q-j}((j+1)z)(-\lambda)^{j} \).

\par
From the Figure \ref{fig:zeta_z_plane}, we can observe that the stability region of ABTI is the common area enclosed by a series of \textit{roots locus curve} of the polynomial \( p_q( z; \theta) := p_q(e^{i \theta}; \mathbf{A} + z \mathbf{B}) \) with respect to \( z = \lambda \tau \in \mathbb{C} \), parameterized by \( \theta \in [-\pi, \pi) \). These \textit{roots locus curve} can be divided into two types: one is a pseudo-semicircle \( \Gamma^{(1)} \) passing through the origin, and the other consists of multiple quasi-concentric circles \( \Gamma^{(2)}_j \) that do not pass through the origin. We can order the set \( \{\Gamma^{(2)}_j \}_{j \ge 1} \) by their distance from the origin, such that \( \Gamma^{(2)}_1 \) is the quasi-circle closest to the origin that does not pass through it. 
\par
For convenience in handling and observation, we divide the original polynomial by a factor \( (-e^{i \theta})^q \), which does not change the zeros of the original polynomial, and introduce the variable \( \zeta = - e^{-i \theta} \). This results in the following variant polynomial,
\begin{equation}
\tilde{p}_{q}(\zeta; \theta) := \sum_{j=0}^{q} \gamma_{q-j}((j+1) \zeta) - e^{-i \theta}\sum_{j=0}^{q-1} \gamma_{q-1-j}((j+1)\zeta) - \gamma_{q}(-\zeta/\alpha)\delta_q, 
\end{equation}
whose zeros have the same modulus as those of \( p_{q}(z; \theta) \).
\par 
{The existence of a limiting parabolic radius is a necessary condition for the conjecture to hold. Considering the parabolic radius is not merely to restrict the characteristic polynomial to real coefficients, but rather because for certain polynomials \( p \) with \( q \geq 1 \), the \textit{roots locus curve} \( \Gamma^{(1)} \) may not necessarily include the imaginary axis. These pseudo-semicircles, which do not contain the imaginary axis, will slightly deviate away from the imaginary axis as they approach it. This characteristic limits the ability of the ABTI scheme to handle hyperbolic problems and provides one of the reasons for considering the parabolic radius instead of the hyperbolic radius. If we construct a circle \( \Gamma^* \) with the origin as the center and the parabolic radius as the radius, it represents the largest semicircular region that can be contained within the stability region. This property requires examination through the deformed polynomial. Notably, there is a one-to-one correspondence between \( \Gamma^{(2)}_1 \) and \( \gamma^{(2)}_1 \), where \( \gamma^{(2)}_1 \) is a convex small circle on \( \zeta \)-plane with similar definition of \( \Gamma^{(1)} \) or \( \Gamma^{(2)}_{j} \). Therefore, we know that the distance between \( \Gamma^{(2)}_{1} \) and the origin increases gradually from \( \theta = \pm \pi \) to \( \theta = 0 \). In other words, the region enclosed by \( \Gamma^* \) can be completely contained within the region enclosed by \( \Gamma^{(2)}_1 \). Through experiments, we further observe that the radius of \( \gamma^{(2)}_1 \) decreases as \( q \) increases, causing \( \Gamma^{(2)}_1 \) to approach a more regular circular shape.}
Another reason is that considering the equivalent polynomial allows for a more concise analysis and expression.

%\begin{itemize}
%\item
%存在极限抛物半径是猜想成立的必要条件。考虑抛物半径不仅仅只是为了将特征多项式限制到实系数情况,而是在 \( q\ge1 \) 的个别多项式 \( p \) 的根曲线中的 \( \Gamma^{(1)} \)未必能够包含虚轴。这些不包含虚轴的拟半圆周在靠近虚轴部分会稍微地远离虚轴。这一点限制了DCI格式处理双曲问题以及成为我们考虑抛物半径而不是双曲半径的其中一个理由。如果以原点为圆心,以抛物半径为半径构成画圆C^*,恰好是稳定区域能够包含的最大半圆形区域。这个特质就需要从变形多项式进行考察,注意到 \( \Gamma^{(2)}_1 \) 与 \( \gamma^{(2)}_1 \)的一一对应关系以及\( \gamma^{(2)}_1 \)为一个凸的小圆周,所以我们知道\( \lvert \Gamma^{(2)}_1(\theta) \rvert \)的值从	\( \theta = \pm \pi \) 到 \( \theta = 0 \)逐渐增大。换言之,\( C^{*} \) 所围成的区域能够完全地由 \( \Gamma^{(2)}_1 \) 所围成的区域所包含。通过实验我们进一步注意到 \( \gamma^{(2)}_1 \) 的半径会随着q的增大而越来越小,这样使得 \( \Gamma^{(2)}_1 \) 越来越趋向于规则的圆周。
%\item
%另一个理由是,考察等价多项式能够使得多项式零点的分析过程和表达式更加地简洁。
%\end{itemize}
\begin{figure}[!h]
\centering
\includegraphics[scale=.45]{f3_z.eps}
\includegraphics[scale=.45]{f3_zeta.eps}
\caption{The zeros of \( p_{4}(z; \theta) \) on the \( z \)-plane (Left) and  the simple transformation \( \zeta(\theta) = - e^{-i \theta} z(\theta)  \) on the \( \zeta \)-plane (Right). The zeros $\zeta(\theta)$ and $z(\theta)$ be marked by blue whenever \( \theta \in [-\tfrac{\pi}{2}, \tfrac{\pi}{2}] \).}
\label{fig:zeta_z_plane}
\end{figure}
%{\color{red}
%If we directly prove the all zeros of polynomial sequence \( p_n( \lambda; r e ^{i \theta}) \) lie in the interval \( (-1, 1) \) with \( \theta = \pi, r = 1/e \), whose all zeros must lie in the real axis from the similar technique in lemma \ref{lem:real zeros}, the proof will be more complex comparing with the adopting the \textit{roots locus curve}\cite{Haire2006}.  
%Let \( p_n(\lambda) = p_n(\lambda, z = - 1/e ) \), then the Schur stability of \( p_n( \lambda ) \) is equivalent to that the Dudan-Fourier count subjects to 
%\begin{align*}
%N_{p_n}(-1, 1] = V_{p_n}(-1) - V_{p_n}(1) = n.
%\end{align*}
%Introduce a similar polynomial which have same zeros with \( p_n \),
%\begin{align*}
%\hat{p}_n(e\lambda) := 
%(-e)^{n}\cdot p_n(\lambda) =  & \sum_{j=0}^{n}\frac{(j+1)^{n-j}}{(n-j)!}(e \lambda)^{j} - e\sum_{j=0}^{n-1} \frac{(j+1)^{n-1-j}}{(n-1-j)!}(e \lambda)^{j}.
%\end{align*}
%%namely
%%\begin{align*}
%%\hat{p}_n^{(k)}(x) = \sum_{j=0}^{n-k}\frac{(j+1)^{n-k-j}(j+1)_k}{(n-k-j)!}x^j - e\sum_{j=0}^{n-k-1} \frac{(j+1)^{n-1-j}(j+1)_k}{(n-1-j)!}x^j
%%\end{align*} 
%%where \( 0 \le k \le n-1 \) stand for the degree of derivative about polynomial \( \hat{p}_n(x) \).
%It is easy to notice that 
%\begin{align*}
%N_{p_n}(-1, 1] = N_{\hat{p}_n}(-e, e] = V_{\hat{p}_n}(-e) - V_{\hat{p}_n}(e).
%\end{align*}
%Simillar to the proof in the lemma \ref{lem:zeros-free}, \( \widehat{ \mathcal{P}}_x ^{(k)} (x,t) \) is the generating function of the polynomial sequences \( \hat{p}_n^{(k)}(x) \) obtained by similar ways with \( \tilde{p}_n ^{(k)}(x) \).
%\begin{equation}
%\begin{array}{lll}
%\operatorname{sgn} \hat{p}_n^{(k)}(-e) = (-1)^{n-k}   & \Leftrightarrow V_{\hat{p}_n}(-e) = n  &  \Leftrightarrow \text{completely monotonicity of } \widehat{ \mathcal{P}}_x ^{(k)}(-e, t), \nonumber \\
%\hat{p}_n^{(k)}(e) > 0  &  \Leftrightarrow V_{\hat{p}_n}(e) \equiv 0  &  \Leftrightarrow \text{positive definiteness of } \widehat{ \mathcal{P}}_x ^{(k)}(e, t).
%\end{array} 
%\end{equation}
%The first column relation can verify numerically by the Matlab code\footnote{ \url{} }. However, the rigorous analysis have some difficulties possibly.}
\begin{lemma}
\label{lem:real zeros}
All zeros of  polynomial \( \tilde{p}_n( \zeta ; \theta ) \) lie in the real axis when \( \theta= 0 \) or \( \pi \). 
\end{lemma}
\begin{proof}
%From the Figure \ref{fig:zeta_z_plane} of the $\zeta$-plane, and the generation process of the \textit{roots locus curve} long with the varying $\theta$, we intuitively feel that when $\theta = 0, \pi$, all zeros of the polynomial \( \tilde{p}_n(\zeta; \theta) \) are real. Theoretically, 
$\tilde{p}_n(\zeta; \theta) -  \tilde{p}_{n}(\bar{\zeta}; - \theta) = 2 \mathtt{Im}(\tilde{p}_n(\zeta; \theta)) \nonumber$ implies that the zeros of polynomial \( \tilde{p}_n(\zeta; \theta) \) must be zeros of \( \tilde{p}_{n}(\bar{\zeta}; - \theta)  \). In other word, the distribution of  the zeros associated with \( \theta \) are strictly symmetric about real axis in the Figure \ref{fig:zeta_z_plane} from the continuous dependence of zero distribution of polynomials on coefficients. Therefore, we can conclude this lemma. 
\end{proof}
\begin{remark}
From another rigorous deduction, we can also judge that \( \tilde{p}_n(\zeta; \theta = 0 \text{ or } \pi) \) is the log-concave polynomial \cite{Huh2022} which has only real zeros, for which the polynomial coefficient are a unimodal sequences \cite{Liu2007}. This kind of polynomial is widely used in combinatorial mathematics.  
\end{remark}
When \( \theta = 0 \), the zeros of the polynomial are non-negative. When \( \theta = \pi \), the zeros of the polynomial are strictly negative. At this point, the absolute value of the zero with the smallest modulus is exactly the parabolic radius of the stability region of the ABTI method. This is the content that the following theorem needs to examine.
\begin{theorem}\label{lem:disprove}
For all \( 1 \le n < \infty \), the zeros of the polynomial \( \tilde{p}_n( \zeta; \pi) \) are real and distinct. Furthermore, for arbitray finite \( N \) exists  non-negative \( r_{N} \) such that $( - r_{N}, 0]$ is a \textit{zero-free interval} of \( \tilde{p}_{n}(\zeta; \pi) \) for all \( n < N \).
\end{theorem}
\begin{proof}
Denote the polynomial sequences \( \tilde{f}_n( \zeta ) := \sum_{j=0}^{n}\gamma_{n-j}((j+1) \zeta) \) and its generating function denoted by
\begin{align*}
\widetilde{ \mathsf{F}}(\zeta,t)  & := \sum_{n=0}^{\infty}\tilde{f}_{n}(\zeta)t^{n} = \sum_{n=0}^{\infty}\left(\sum_{j=0}^{n}\frac{((j+1)\zeta)^{n-j}}{(n-j)!}\right)t^{n} \\
= & \sum_{j=0}^{\infty}\sum_{n=j}^{\infty}\frac{((j+1)\zeta)^{n-j}}{(n-j)!}t^{n} =  \sum_{j=0}^{\infty}\sum_{n=0}^{\infty}\frac{((j+1)\zeta)^{n}}{n!}t^{n+j}\\
 = & \sum_{j=0}^{\infty}t^{j} e^{(j+1)\zeta t} = e^{\zeta t}\sum_{j=0}^{\infty}(t e^{\zeta t})^{j} = \frac{1}{e^{- \zeta t} - t} \text{ with \( | t e^{\zeta t} | < 1 \).}
\end{align*}
Where we interchange the order of summation and recalling the range of index \( n \).
%\begin{figure}[!h]
%\centering
%\includegraphics[scale=.5]{extrema.eps}
%\end{figure}
Thence, 
\begin{equation}
\widetilde{ \mathsf{P} }(t; \zeta)  = \frac{ 1 + t }{e^{-\zeta t}-t} - e^{-\zeta t}\delta_q \nonumber
\end{equation}
is the generating function of \( \tilde{p}_{n}(\zeta; \pi) \) which can be obtained by similar way. It is easy to check that the convergent radius of the series \(  | t e^{\zeta t} | < 1\) leads to \( \left|\zeta\right| > 1/e \) for all \( t, \zeta \in \mathbb{R} \). 
In order to confirm the range of the \( \tilde{p}_{n}(\zeta) \), we turn it to a form of Fourier transfromation. 
\begin{equation}
\label{eq:to_Fourier_coef}
\tilde{p}_{n}(\zeta) = \tilde{p}_{n}(\zeta; \pi) = \frac{1}{2 \pi i} \oint_{\left|t\right|=1} \frac{\widetilde{\mathsf{P} }(t; \zeta)}{t^{n+1}}  \mathrm{d}t = \frac{1}{2 \pi}\int_{0}^{2 \pi}\tilde{\mathcal{P}}(\varphi; \zeta)\cdot e^{-i  n \varphi} \mathrm{d}\varphi ,
\end{equation} 
% \begin{align*}
% \tilde{p}(\xi) = \tilde{p}(\xi; \pi) = & \frac{1}{2 \pi i} \oint_{\left|t\right|=1} \frac{\widetilde{\mathsf{P} }_{ \zeta } ^{(k)} (t; \xi)}{t^{n+1}}  \mathrm{d}t \\ 
% =  & \frac{1}{2 \pi}\int_{0}^{2 \pi}\widetilde{\mathsf{P}}_{ \zeta } ^{(k)} (e^{i \varphi}; \xi)\cdot e^{-i  n \varphi} \mathrm{d}\varphi = \frac{1}{2 \pi}\int_{0}^{2 \pi}\tilde{\mathfrak{p}}_k(\varphi; \xi)\cdot e^{-i  n \varphi} \mathrm{d}\varphi,
% \end{align*}
where \( \tilde{\mathcal{P}}(\varphi; \zeta) := \widetilde{ \mathsf{P} }( e^{i \varphi}; \zeta)  \) is a function with period \( 2 \pi \) about \( \varphi \in \mathbb{R} \).
Thus, we have a bridge between the value of the sequences \( \tilde{p}_{n}(\zeta) \) and the spectrum of the function \( \tilde{\mathcal{P}}(\varphi; \zeta) \). 
\par  
If Buvoli's conjecture valid, then \( \tilde{p}_{n \to \infty }(\zeta) > 0 \) for all \( \zeta \in (-1/e,0] \). However, we can disprove this result from the view of harmonic analysis.
We notice that  \( \tilde{{\mathcal{P}}}(\varphi; \zeta) \in \mathtt{BV}(\mathbb{R}\times \mathbb{R}) \) which also can be observed from the Figure~\ref{fig:disprove} directly.  
\cite[Table 3.1]{Grafakos2014} has shown the relation between the decay rate of Fourier coefficients of a function \( \tilde{\mathcal{P}}(\varphi; \zeta) \) associated with variable \( \varphi \) and whose smooth properties, viz, $\tilde{p}_{n} ( \zeta )  = \mathcal{O}(n ^{-1})$ since \( \tilde{\mathcal{P}}(\varphi; \zeta) \in \mathtt{BV}(\mathbb{R}\times \mathbb{R}) \)  for all \( \zeta \in (-1/e, 0] \).  Therefore, for arbitrary \( \zeta \in (-1/e, 0] \), \( \tilde{p}_{n \to \infty }(\zeta) \to 0 \). 
\par 
Since the continuity of polynomial \( \tilde{p}_{n}(\zeta) \) and it across the common point \( (0,2) \)  for all \( n \ge 1 \), there are \( N \)  for all \( r_{N} \in [-1, 0) \) such that \( \tilde{p}_{n}(\zeta) > 0 \) whenever \( n \le N \).
\end{proof}
\begin{remark}
Even so, this method has significant advantages compared to traditional AB methods of order 4 and above. By presetting an appropriate parabolic radius, we can determine the maximum permissible accuracy by calculating the integral expression on the right side of the equation \eqref{eq:to_Fourier_coef}. Taking Buvoli's conjectured radius \( 1/e \) as an example, using Trefethen's numerical integration toolbox \footnote{\url{https://www.chebfun.org}} and setting the tolerance of integrand more than \(  10 ^{-12} \), we can determine that the maximum permissible accuracy is $n = 31$. When the radius equals to AB4, namely \( r_n = 0.3 \), the maximum permissible accuracy is $n = 57$.
\begin{table}[h]
\centering
\begin{tabular}{ c|c|c|c|c|c } 
\hline
\( r_n =  \) & \( 0.6 \) & \( 0.5 \) & \( 0.4 \) & \( \mathbf{1/e} \) & \( 0.3 \) \\
\hline
\( n <  \) & 2 & 3 & 7 & 31 & 57  \\ 
\hline
\end{tabular}
\caption{The  maximum permissible accuracy for given parabolic radius.}
\end{table}
\begin{figure}[!h]
	\centering
	\includegraphics[scale=.4]{close_origin.eps}
	\includegraphics[scale=.4]{disprove_BV.eps}
	\caption{The local plot of polynomial sequence \( \tilde{p}_{n}(\zeta; \pi) \) (Left) and the graph of \( \lvert \tilde{\mathcal{P}}(\varphi; \zeta) \rvert \) with \( \varphi \in [0, 2 \pi] \) and \( \zeta = - \tfrac{1}{e}, - \tfrac{1}{2e}, - \tfrac{1}{4e} \)(Right)}
	\label{fig:disprove}
\end{figure} 
From the left subfigure of Figure~\ref{fig:disprove} of the polynomial sequence, it can be observed that the points \(  (0, 2) \) represent a common intersection. Furthermore, the direction of the arrows in the graph indicates the direction of change in the polynomial graph as the value of \( n \) increases. If Buvoli's hypothesis is accurate, then the black marker point \( (-1/e, f_n(-1/e) ) \) in the graph will approach the real axis as \( n \) increases, but will never cross it.
%\par In fact, as the parameter \( \zeta \) moves away from \( - 1/e \) along with the \( - \infty \) direction of real axis, it can be seen from the Figure \ref{fig:Asymptotic_diagram} that \( | \tilde{\mathfrak{p}}(x; \zeta)| \) gradually violates \( \left| \tilde{\mathfrak{p}}(x; \zeta)\right| < \tilde{\mathfrak{p}}(0; \zeta) \) the necessary condition for positive definite functions.
\end{remark}
Rephrasing the above theorem, we obtain the following criterion for determining the conditions that ensure the stability of the scheme based on the given parabolic radius.
\begin{corollary}
Given the parabolic radius \( r_N < r2 \) and  denoted \( \tilde{\mathcal{P}}(\varphi; \zeta) := \widetilde{ \mathsf{P} }( e^{i \varphi}; \zeta) \) with 
\begin{align*}
\widetilde{ \mathsf{P} }(t; \zeta)  = \frac{ 1 + t }{e^{-\zeta t}-t} - e^{-\zeta t}\delta_{q} \text{ where }
\delta_{q} = 
\begin{cases}
1, \text{ when \( s = q \)}, \\
0, \text{when \( s>q \)},
\end{cases} 
\end{align*} 
if for all \( \zeta \in (r_N, 0] \) it holds that
\begin{equation}
 \frac{1}{2 \pi}\int_{0}^{2 \pi}\tilde{\mathcal{P}}(\varphi; \zeta)\cdot e^{-i  N \varphi} \mathrm{d}\varphi > 0,  
 \label{eq:discriminant}
\end{equation}
then ABTI can maintain stability for the approximation accuracy up to \( N \).
\end{corollary}
\begin{corollary}
\label{coro:the smallest modulus zero of the polynomial}
For any given order of approximation \(n\), the reciprocal of its parabolic radius \(r_n\) is the smallest modulus zero of the polynomial $\tilde{p}_{n}(\zeta; \pi)$ when $s=q+1$. For the case where \( s = q \), we only need to consider the zeros of the polynomial \( \tilde{p}_{n+1}(\zeta; \pi) \), and this will not be elaborated further here. The table below lists the parabolic radii for some specific cases
\begin{table}[h]
\centering
\begin{tabular}{ c|c|c|c|c|c } 
\hline
\( n  \) & \( 2 \) & \( 4 \) & \( 6 \) & \( 8 \) & \( 10 \) \\
\hline
\( r_{n}  \) & 0.7639 & 0.4658 &  0.4124 & 0.3934 &  0.3845 \\ 
\hline
\end{tabular}
\caption{The parabolic radius for given order of approximation.}
\end{table}
\end{corollary}

\subsection{Accuracy lose and recovery}
In the previous section, we mentioned that the original ABTI method \cite[section 2.2]{Buvoli2019} fails to achieve the accuracy we envisioned. In this section, we will explore the reasons behind this and provide solutions. First, we examine how ABTI affects the accuracy. If we assume that \( \mathbf{u}^{[n]} \) is computed accurately, it is evident that \( u^{n}\mathbbm{1} \) is also equally accurate through the arithmetic averaging of both sides of the scheme. Thus, \( \mathbf{B}(\alpha)\mathbf{f}^{[n]} \) is the main factor influencing the accuracy of the scheme. The following lemma will characterize the approximation property of the ABTI.
\begin{lemma} For a function \( f(t) \in \mathcal{C} ^{q} (0, T]  \), then exists approximation
\label{lem:approximation}
\begin{equation}
\frac{r}{s}\mathbbm{1}^{\mathrm{T}}\mathbf{B}(\alpha)\mathbf{f}^{[n]} - \int_{t_n}^{t_{n+1}}f(t) \mathrm{d}t = 
\begin{cases}
\mathcal{O}(\tau^{q}),  &  \text{ when } s = 1, \\
 \mathcal{O}(\tau^{q+1}),   &  \text{ when } s > q.
\end{cases}
\end{equation}

\end{lemma}

\begin{proof}
For all \( s \ge q \), 
\begin{align*}
\frac{1}{s}\sum_{j=1}^{s}\frac{(\alpha + \omega_j)^q}{q} = \frac{1}{s}\sum_{j=1}^{s}\frac{\sum_{m=0}^{q}\binom{q}{m}\alpha^{q-m}\omega_j^{m}}{q} = \frac{1}{q}\sum_{m=0}^{q}\binom{q}{m}\alpha^{q-m}\left(\frac{1}{s}\sum_{j=0}^{s}\omega_j^m\right) =
 \frac{\alpha^{q}}{q} + \frac{\delta_q}{q} 
\end{align*}
where \( \delta_q = 1 \) whenever \( s = q \) and vanishing for all \( s > q \).
Thence, 
\begin{align*}
\frac{r}{s}\mathbbm{1}^{\mathrm{T}}\mathbf{B}(\alpha)\mathbf{f}^{[n]} = & \sum_{\nu=0}^{q-1}\frac{\alpha^{\nu+1}}{\nu+1}  +  r \cdot\frac{\delta_q}{q}\cdot\hat{f}_{q-1} = \int_0^\tau \sum_{\nu=0}^{q-1} \left(\frac{t}{r}\right)^{\nu}\cdot \hat{f}_{\nu} \mathrm{d}t +  r \cdot\frac{\delta_q}{q}\cdot\hat{f}_{q-1} 
\end{align*}
where \( \hat{f}_{\nu} = \left \langle \mathbf{F}\mathbf{f}^{[n]}  \right \rangle _{\nu+1} \).
\par 
By Taylor expansion and its remainder term of an integral, we know that 
\begin{align*}
\hat{f}_{\nu} =  & \frac{1}{s}\sum_{j=1}^{s}\omega_j^{-\nu}f(t_n + r \omega_j) = \frac{1}{s}\sum_{j=1}^{s}\omega_j^{-\nu}\left(\sum_{\mu=0}^{q-1}\frac{f^{(\mu)}(t_n)}{\mu!}(r \omega_j)^{\mu} + 	\int_{t_{n}}^{t_{j}^{[n]}}\frac{(t-t_j^{[n]})^{q-1}}{(q-1)!}\cdot f^{(q)}(t)\mathrm{d}t \right) \\
 =  & \sum_{\mu=0}^{q-1}\frac{f^{(\mu)}(t_n)}{\mu!}r^{\mu} \left(\frac{1}{s}\sum_{j=1}^{s}  \omega_j ^{\mu - \nu}\right) + \frac{1}{s}\sum_{j=1}^{s}\omega_j^{-\nu}\left(\int_{t_{n}}^{t_{j}^{[n]}}\frac{(t-t_j^{[n]})^{q-1}}{(q-1)!}\cdot f^{(q)}(t)\mathrm{d}t \right) \\
 =  & \frac{f^{(\nu)}(t_n)}{\nu!}r^{\nu}  + \frac{1}{s}\sum_{j=1}^{s}\omega_j^{-\nu}\int_{t_{n}}^{t_{j}^{[n]}}\frac{(t-t_j^{[n]})^{q-1}}{(q-1)!}\cdot f^{(q)}(t)\mathrm{d}t, \text{ for all \( 0 \le \nu \le q-1 \)}.
\end{align*}
Thence,
\begin{align*}
 &  \frac{r}{s}\mathbbm{1}^{\mathrm{T}}\mathbf{B}(\alpha)\mathbf{f}^{[n]} = \int_{0}^{\tau}\left[\sum_{\nu=1}^{q-1}\frac{f^{(\nu)(t_{n})}}{\nu!}t^{\nu} + \mathfrak{r}_{n,1}(t)\right] \mathrm{d}t \\
 & + r \cdot \frac{\delta_{q}}{q}\cdot\left(\frac{f^{(q-1)}(t_{n})}{(q-1)!}r^{q-1} + \frac{1}{s}\sum_{j=1}^{s}\omega_{j}\int_{0}^{r}\frac{(r-t)^{q-1}}{(q-1)!}\cdot f^{(q)}(\omega_{j}t+t_{n})\mathrm{d}t\right)
\end{align*}
where
\begin{equation}
\mathfrak{r}_{n,1} := \sum_{\nu=0}^{q-1}\left(\frac{t}{r}\right)^{\nu}\cdot \frac{1}{s}\sum_{j=1}^{s}\omega_{j}^{q-\nu}\int_{0}^{r}\frac{(r-t)^{q-1}}{(q-1)!}\cdot f^{(q)}(\omega_{j}t + t_{n})\mathrm{d}t. \nonumber
\end{equation}
By similar way, 
\begin{align*}
\int_{t_{n}}^{t_{n+1}}f(t)\mathrm{d}t = \int_{0}^{\tau}f(t+t_{n})\mathrm{d}t = \int_{0}^{\tau}\left(
\sum_{\nu=0}^{q-1}\frac{f^{\nu}(t_{n})}{\nu!}t^{\nu} + \mathfrak{r}_{n,2}(t) \right)\mathrm{d}t
\end{align*}
where
\begin{equation}
\mathfrak{r}_{n,2}(t) := \int_{0}^{t}\frac{(t-\sigma)^{q-1}}{(q-1)!}f^{(q)}(\sigma + t_{n})\mathrm{d}\sigma. \nonumber 
\end{equation}
Therefore, we know 
\begin{align*}
& \frac{r}{s}\mathbbm{1}^{\mathrm{T}}\mathbf{B}(\alpha)\mathbf{f}^{[n]} - \int_{t_n}^{t_{n+1}}f(t)\mathrm{d}t \\
= & \int_{0}^{\tau}\mathfrak{r}_{n,1}(t) - \mathfrak{r}_{n,2}(t) \mathrm{d}t + \left(\delta_{q}\cdot \frac{f^{(q-1)}(t_{n})}{q!}r^{q} + r \cdot \frac{\delta_{q}}{q}\cdot \frac{1}{s}\sum_{j=1}^{s}\omega_{j}\int_{0}^{r}\frac{(r-t)^{q-1}}{(q-1)!}\cdot f^{(q)}(\omega_{j}t+t_{n})\mathrm{d}t\right) \\
= & \delta_{q}\cdot \frac{f^{(q-1)}(t_{n})}{\alpha^{q} \cdot q!}\tau^{q} + \mathcal{O}(\tau^{q+1}).
\end{align*}
By this way, we attain a reliable explanation for order-lose phenomenon which completes the proof.
\end{proof}
The above lemma tells us that \( \delta_q \ne 0 \) allows the lower-order terms in the truncation error, specifically \( \delta_q \cdot \frac{f^{(q-1)}(t_n)}{\alpha^q \cdot q!} \tau^q \), to be retained. Therefore, to achieve the ideal approximation accuracy, we only need to choose the number of sampling points \( s \) to be greater than the number of terms \( q \) in the Taylor expansion. However, increasing the number of sampling points too much will increase the dimensions of the matrices \( \mathbf{A} \) and \( \mathbf{B}(\alpha) \), thus raising the computational cost. Therefore, the optimal accuracy recovery strategy is to set \( s = q + 1 \).
\section{Applications}
\label{sec:Applications}
In this section, we will give applications in  partial differential equations (PDEs).  We consider the linear parabolic equation and its $L^2$-stability fully discrete scheme
\begin{equation}\label{eq:heat_equation}
\begin{cases}
\partial_t u(\mathbf{x}, t) = \mathcal{L} u(\mathbf{x}, t) + f(t, \mathbf{x}), & \mathbf{x} \in \Omega \times [0,T],  \\
u(\mathbf{x}, 0) = u_0(x), & \mathbf{x} \in \Omega, \\
u(\mathbf{x}, t) = 0, & \partial \Omega \times [0, T],
\end{cases}
\end{equation}
where \( \mathcal{L} \) is unbounded self-adjoint linear operator in Hilbert space.
\par
It is worth mentioning that when using ABTI for time discretization of the parabolic equation, there exists an issue with the order of space-time discretization. Since the Laplace operator is an unbounded self-adjoint negative semi-definite operator, its spectrum lies on the negative real axis and extends toward negative infinity, we need to first perform spatial discretization of the original parabolic equation, aside from time discretization schemes that are \( A \)-stable or \( A(\theta) \)-stable. Essentially, the discretization of the Laplace operator is equivalent to truncating an unbounded operator into a bounded one. As ABTI is a conditionally stable explicit scheme, spatial discretization must be performed first, and the spatial semi-discretization can be described as a stiffness ODE system. The time step constraint, dependent on the CFL condition, is closely related to the spectral radius of the discrete matrix of the Laplace operator. Therefore, the spatial semi-discretized equation can also be generalized as an abstract ODE problem of the form \eqref{eq:ODE}.
\par
The fully discrete scheme of the equation \eqref{eq:heat_equation} gives 
\begin{subequations}\label{eq:fully_discrete}
\begin{align}
\left(\mathbf{E}_{ \tau }\otimes\mathbf{M}\right) \mathbf{u}^{[0]}_{h} =  & \mathbbm{1}\otimes\mathbf{v}_h + r\left(\mathbf{B}(\alpha)\otimes\mathbf{K}\right) \mathbf{u}^{[0]}_{h} + r \left(\mathbf{B}(\alpha) \otimes \mathbf{E}_{h}\right) \mathbf{f}_{h}^{[0]}, \\ 
\left(\mathbf{E}_{ \tau }\otimes\mathbf{M}\right) \mathbf{u}^{[n+1]}_{h} =  & \left(\mathbf{A}\otimes\mathbf{M}\right)\mathbf{u}_{h}^{[n]} + r \left(\mathbf{B}(\alpha) \otimes \mathbf{K}\right) \mathbf{u}_{h}^{[n]} 
+ r \left(\mathbf{B}(\alpha) \otimes \mathbf{E}_{h}\right) \mathbf{f}_{h}^{[n]},
\end{align}
\end{subequations}
where \( \mathbf{M}\) and \( \mathbf{K} \) are mass and stiffness matrix respectively  associate with finite element methods (FEMs) or another spatial discretization methods.  \( \mathbf{u}_{h}^{[n]}:= [\mathbf{u}_{1,h}^{n}, \mathbf{u}_{2,h}^{n}, \ldots, \mathbf{u}_{s,h}^{n}]^{\mathrm{T}} \) is a \( (s \times \mathrm{N}_h) \)-dimensional column solution vector whose \( N_s \) stand for the spatially discrete degree of freedom.
\subsection{\( L^2 \)-stability}
\begin{theorem}[ \( L^{2} \)-stability]
\label{thm:l2_stability}
For arbitrary \( q \ge 1 \), under the time scale restriction with respect to the parabolic CFL condition \( \rho(\mathbf{G}) < 1 \) were \( \mathbf{G}:= \mathbf{A}\otimes\mathbf{E}_h  + r (\mathbf{B}\otimes\mathbf{K})(\mathbf{E}_\tau\otimes\mathbf{M})^{-1} \), then the fully discrete scheme \eqref{eq:fully_discrete} is \( L^2 \) stable in the sense that   
\begin{equation}
\left\| u_h^{n+1} \right\| \le \left\| u_h^0 \right\| + \tau \sum_{\nu=0}^{n}\left\| f_{h}^{\nu} \right\|. 
\end{equation}

\end{theorem}
\begin{proof}
%Find \( u,v \in \mathbf{V}_{h} \subset \mathbf{V}:= H^2(\Omega) \cap H_0^1(\Omega) \), such that 
%\begin{equation}
%\left(\partial_t u, v \right) = - \left( \nabla u, \nabla v \right) + \left(f, v\right) .\nonumber
%\end{equation}
We review the spatial semi-discretization
\begin{align*}
\partial_t \mathbf{M}\mathbf{u}_h(t) = \mathbf{K}\mathbf{u}_h(t) + \mathbf{f}_{h}(t),
\end{align*}
where \( \mathbf{M} := \int_{\Omega}\bm{\varphi}\cdot \bm{\varphi}^{\mathrm{T}}\mathrm{d}x,  \mathbf{K} := \int_{\Omega} \bm{\varphi}\cdot \Delta\bm{\varphi}^{\mathrm{T}}\mathrm{d}x = - \int_{\Omega} \nabla\bm{\varphi}\cdot (\nabla\bm{\varphi})^{\mathrm{T}}\mathrm{d}x, \mathbf{f}_h(t) := \int_{\Omega}\bm{\varphi}\cdot f \mathrm{d}x \) are mass and  stiffness matrix and load vector respectively with base function vector \( \bm{\varphi} =[\ldots, \varphi_j, \ldots]^{\mathrm{T}} \). 
\par 
The fully discretization, denote \( \mathbf{v}_h^{[n]} := (\mathbf{E}_\tau\otimes\mathbf{M})\mathbf{u}_h^{[n]} \) and \( \mathbf{G}:= \mathbf{A}\otimes\mathbf{E}_h  + r (\mathbf{B}\otimes\mathbf{K})(\mathbf{E}_\tau\otimes\mathbf{M})^{-1} \) whose the size of identity matrix \( \mathbf{E}_{ \tau } \) and \(  \mathbf{E}_{ h } \) keep up with time and space discrete matrix repsectively, 
\begin{align*}
\mathbf{v}_h^{[n+1]} = \mathbf{G}\mathbf{v}_h^{[n]} + r\tilde{\mathbf{f}}_h^{[n]}, \text{ where } 
\tilde{\mathbf{f}}_h^{[n]} := \left(\mathbf{B} \otimes \mathbf{E}_h\right) \mathbf{f}_{h}^{[n]}. 
\end{align*}
By the recursion, 
\begin{equation}
\mathbf{v}_h^{[n+1]} = \mathbf{G}^{n+1}\mathbf{v}_h^{0} + r \sum_{j=0}^{n}\mathbf{G}^{n-j}\tilde{\mathbf{f}}_h^{j}, \nonumber
\end{equation}
Notice that \( (\mathbf{u}_h^{[m]})^{\mathrm{T}}\mathbf{v}_h^{[n]} = (\mathbf{u}_h^{[m]})^{\mathrm{T}}(\mathbf{E}_\tau\otimes\mathbf{M})\mathbf{u}_h^{[n]} = \sum_{j=1}^{s}\left \langle \mathbf{u}_{h,j}^{[m]}, \mathbf{u}_{h,j}^{[n]} \right \rangle = s \left \langle u_h^{m}, u_h^{n} \right \rangle \). Then, we take the product \( (\mathbf{u}_h^{[n+1]})^{\mathrm{T}} \) on the both side that 
\begin{align*}
%\sum_{j=1}^{s}\lVert \mathbf{u}_{h,j}^{[n+1]} \rVert ^2 = \sum_{j=1}^{s}\left \langle \tilde{\mathbf{u}}_{h,j}^{[n+1], (n+1)}, \mathbf{u}_{h,j}^{[0]} \right \rangle  + r \sum_{j=1}^{s}\sum_{\nu=0}^{n}\left \langle \tilde{\mathbf{u}}_{h,j}^{[n+1], (n-\nu)}, \tilde{\mathbf{f}}_{h,j}^{[\nu]}   \right \rangle \\
\lVert u_h^{n+1} \rVert ^2 \le  &  \left \langle \tilde{u}_h^{n+1,(n+1)}, u_h^0 \right \rangle + r \sum_{\nu=0}^{n}\left \langle \tilde{u}_h^{n+1, (n-\nu)}, \tilde{f}_h^{\nu}  \right \rangle  \\
\le  &  \lVert \tilde{u}_h^{n+1, (n+1)} \rVert \cdot \lVert u_h^0 \rVert + r \sum_{\nu=0}^{n}\lVert \tilde{u}_h^{n+1, (n-\nu)}, \tilde{f}_h^{\nu} \rVert,
\end{align*}
where \( \tilde{u}_h^{n+1, (n-\nu)} = \frac{1}{s}\sum_{j=1}^{s} \tilde{\mathbf{u}}_{h, j}^{[n+1],(n-\nu)} := \frac{1}{s}\sum_{j=1}^{s}\sum_{\mu=1}^{N}\left \langle (\mathbf{G}^{n-\nu})^{\mathrm{T}}\mathbf{u}_{h,j}^{[n+1]} \right \rangle_\mu \cdot \varphi_\mu  \) whose coefficients are \( \mu \)-th component of the column vector \(  (\mathbf{G}^{n-\nu})^{\mathrm{T}}\mathbf{u}_{h,j}^{[n+1]} \).
%Since \( \lVert u_h^{n} \rVert^2 = \frac{1}{s}\sum_{j=1}^{s}\lVert \mathbf{u}_{h,j}^{[n]} \rVert^2  \), using the Cauchy-Schwartz inequality, 
The CFL condition associated with the spectral radius \( \rho( \mathbf{G} ) \le 1 \) yields 
\begin{align*}
\lVert \tilde{u}_{h}^{n+1,(n-\nu)} \rVert \le \lVert u_h^{n+1} \rVert \text{ for all }  \nu \le n. 
\end{align*}
When \( j = n \), then \( \tilde{u}_{h}^{n+1,(n-\nu)} = u_h^{n+1} \), the inequality holds obviously. Notice that 
\begin{align*}
\lVert \tilde{u}_{h, n-j}^{[n+1]} \rVert \sim \lVert \mathbf{G}\mathbf{v} \rVert \le \lVert \mathbf{v} \rVert \sim \lVert u_{h}^{n+1} \rVert \text{ and } \lVert \tilde{f}_h^{\nu} \rVert \sim \alpha \lVert f_h^{\nu} \rVert.
\end{align*}
Therefore, 
\begin{align*}
\lVert u_h^{n+1} \rVert \le  \lVert u_h^0 \rVert + r \sum_{\nu=0}^{n}\lVert \tilde{f}_h^\nu \rVert \le  \lVert u_h^0 \rVert + \tau \sum_{\nu=0}^{n}\lVert f_h^\nu \rVert.
\end{align*}
The proof of the theorem concludes at this point.
\end{proof}
The above \( L^2 \)-stability analysis framework can expand to block Adams-Bashforth (BAB), block Adams-Moulton (BAM) or block backward differentiation formula (BBDF) which refers to \cite{Buvoli2018} for temporal discretizations within general form as 
\begin{equation}
\mathbf{u}^{[n+1]} = \mathbf{Au}^{[n]} + r \mathbf{B}(\alpha)\left(\mathcal{L}_h\mathbf{u}^{[n]} + \mathbf{f}^{[n]} \right) \nonumber
\end{equation}
where \( \mathbf{u}^{[n]}, \mathbf{A} \) and \( \mathbf{B}(\alpha) \) deduced by specific strategy and \( \mathcal{L}_h \) is the discretization of \( \mathcal{L} \).  Particularly, our analysis results have two following reduced version. 
\begin{itemize}
\item 
When \( \mathbf{A}\to 1, \mathbf{B}(\alpha) \to \alpha \) or \( r \to 0 \), the above method method reduce to the classical AB1 method. 
\item
When \( \mathbf{M}\to 1, \mathbf{K} \to \lambda \), the above method meas that \( \Delta u \to \lambda u \) that is the heat equation reduces to the Dahlquist test problem.  
At this point, the CFL condition for \( L^2 \)-stability is consistent with the stability condition derived from the stability region of the ODE problem.
\end{itemize}
\subsection{\( L^{2} \)-error}
The error bound of fully discrete scheme, combined with the spatial finite element discretization, essentially only requires the analysis of the standard spatial semi-discretization scheme, the \( L^2 \)-stability analysis results Theorem \ref{thm:l2_stability}, and the approximation of the ABTI method  Lemma \ref{lem:approximation}.
\begin{theorem}[\( L^2 \)-error]
\label{thm:p1_time_error}
Denote the error \( e^{n} := \lVert u(t_n) - u^n \rVert \) associate with varying discrete time nodal, then fully discrete scheme \eqref{eq:fully_discrete} has the error bound 
\begin{equation}
e^{N} \le C(\tau^{q- \delta_q} + h^{k}), \nonumber
\end{equation}
where \( \delta_q = 1 \) with \( s = q \) and \( \delta_q = 0 \) whenever \( s > q \). 

\end{theorem}

\begin{proof}
Denote 
\begin{align*}
\text{ \( L^2 \) projection } P_h & : V \to V_h \text{ such that } \left(P_h u - u, \chi \right) = 0, \\
\text{ Ritz projection } R_h & : H_0^1 \to V_h \text{ such that } A\left(R_h u - u, \chi \right) = 0,
\end{align*}
where  \( \chi \in \mathcal{V}_h \subset \mathcal{V} \).
The discrete Laplacian operator defined by \( \left(\Delta_h u_h , \chi \right) = A\left(u_h, \chi\right) \) for all \( \chi \in \mathcal{V}_h \). 
Noticing the \( L^2 \) projection acting on the original equation, we can compare with
\begin{align*}
\partial_t P_h u - \Delta_h R_h u = P_h f, \text{ since \(  P_h \Delta = \Delta_h R_h \)} 
\end{align*}
with the spatial semi-discretization
\begin{equation}\label{eq:spatial semi-discrete}
\partial_t u_h(t) -\Delta_h u_h(t) = P_h f.
\end{equation}
and obtain  the error equation of spatial semi-discretization that 
\begin{equation}\label{eq:spatial semi-discrete}
\partial_t e_h(t) - \Delta_h e_h(t) = - \partial_t (P_h u - u)  +  \Delta_h (R_h u - u) , \text{ with  initial value } e_h(0) = 0,
\end{equation} 
where \( e_h(t) := u(t) - u_h(t) \). 
\par 
By standard analysis techniques \cite[Theorem 1.2]{Thomee2006}, the error bound of the spatial semi-discretization \eqref{eq:spatial semi-discrete} yields 
\begin{equation}
\lVert e_h(t) \rVert = \lVert u_h(t_n) - u(t_n) \rVert  \le C _{h} h^{k} . \nonumber
\end{equation}
Next, we consider the temporal discretization  about \eqref{eq:spatial semi-discrete} as
\begin{align*}
\mathbf{u}_h^{[n+1]} = \mathbf{A}\mathbf{u}_h^{[n]} + r \mathbf{B}(\alpha)\left(\Delta_h \mathbf{u}_h^{[n]} + \mathbf{f}_h^{[n]} \right),
\end{align*}
where \( \mathbf{f}_h^{[n]} := P_h \mathbf{f}^{[n]}  \). Take \( \frac{1}{s}\mathbbm{1}^{\mathrm{T}} \) on the both side,
we know
\begin{align*}
u_h^{n+1} = u_h^{n} + \frac{r}{s}\mathbbm{1}^{\mathrm{T}} \mathbf{B}(\alpha)\left(\Delta_h \mathbf{u}_h^{[n]} + \mathbf{f}_h^{[n]} \right).
\end{align*}
Take \( \int_{t_n}^{t_{n+1}}\cdot ~ \mathrm{d}t  \) for \( \partial_t u_h(t) = \Delta_h u_h(t) + f_h \) with notation \(
f_h := P_h f \), we notice
\begin{align*}
u_h(t_{n+1}) = u_h(t_{n}) + \int_{t_n}^{t_{n+1}}  \left(\Delta_h u_h(t) + f_h \right) \mathrm{d}s.
\end{align*}
Thence,  we know
\begin{align*}
e_{h, \tau}^{n+1} = e_{h, \tau}^{n} +  \frac{r}{s}\mathbbm{1}^{\mathrm{T}} \mathbf{B}(\alpha)\left(\Delta_h \mathbf{u}_h^{[n]} + \mathbf{f}_h^{[n]} \right) - \int_{t_n}^{t_{n+1}}  \left(\Delta_h u_h(t) + f_h \right) \mathrm{d}s =: e_{h, \tau}^{n} + T_n,
\end{align*}
where	\( e_{h, \tau}^{n}:= u_h^n - u_h(t_n) \).  The approximation lemma \ref{lem:approximation} tell us that \( e_{h, \tau}^{n+1} \le e_{h, \tau}^{n} + \mathcal{O}(\tau^{q-\delta_q}) \)
then \( e_{h, \tau}^{n+1} \le e_{h, \tau}^{0} + \sum_{j=0}^{n} T_n \).
\par 
Therefore, we know
\begin{align*}
\lVert u_h^n - u(t_n) \rVert \le  & \lVert u_h^n - u_h(t_n) \rVert + \lVert u_h(t_n) - u(t_n) \rVert  \\
\le  & \left( \lVert e_{h, \tau}^{0}  \rVert  + \sum_{j=0}^{n} \lVert T_n  \rVert \right) +  \lVert u_h(t_n) - u(t_n) \rVert  \\
\le  & C(\tau^{q- \delta_{q}} + h^{k}),
\end{align*}
where \( C = \max \left\{C _{ \tau }, C _{ h } \right\} \).
\end{proof}


%{
%\begin{lemma}
%
%\end{lemma}
%\begin{proof}
%\begin{align*}
%\mathbf{AB} =  &  \mathbf{A S F} = \mathbbm{1} \left[\omega_1 \sum_{\nu=1}^{q}\left(\frac{\alpha}{\omega_1}\right)^{\nu}\frac{1}{\nu}, \ldots, \omega_s \sum_{\nu=1}^{q}\left(\frac{\alpha}{\omega_s}\right)^{\nu}\frac{1}{\nu} + \frac{\delta_q}{q} \right]
%\end{align*}
%where \( 1\le j \le s \).
%\begin{align*}
%\lambda_1(\mathbf{AB}) = \sum_{j=1}^{s}\omega_j \sum_{\nu=1}^{q}\left(\frac{\alpha}{\omega_j}\right)^{\nu}\frac{1}{\nu}  + \frac{\delta_q}{q} = 1 + \frac{\delta_q}{sq}
%\end{align*}
%and \( \lambda_{j} \equiv 0, 2\le j \le s \).
%
%\begin{equation}
%\lambda_{1}\left(A + z AB \right) = 2 +  z\left(1  + \frac{\delta_q}{sq}\right)
%\end{equation}
%
%
%\end{proof}\begin{theorem}
%Under the CFL condition \( \rho(\mathbf{A} + z \lambda(\mathbf{K}) \mathbf{B})  < 1 \), 
%\begin{enumerate}
%\item
%then the fully discrete scheme \eqref{eq:fully_discrete} is \( L^2 \)-stable for linear parabolic equation in the sense that   
%\begin{equation}
%\left\| u_h^N \right\| \le \left\| u_h^0 \right\| + \tau \sum_{n=0}^{N}\left\| f_{h}^{n} \right\|. 
%\end{equation}
%
%\item
%then the fully discrete scheme \eqref{eq:fully_discrete} is modified energy stable for Allen-Cahn equation in the sense that   
%\begin{equation}
%\tilde{E}(u_h^{n+1}) \le \tilde{E}(u_h^{n}),
%\end{equation}
%where
%\begin{equation}
%\tilde{E}(u_h^n) := \frac{1}{s}\int\left((\mathbf{u}_h^{[n]})^{\mathrm{T}}\mathbf{AB}\mathbf{u}_h^{[n]} + \mathbbm{1} \mathbf{B}\mathcal{N}(\mathbf{u}_h^{[n]})\right)\mathrm{d}\Omega
%\end{equation}
%
%Furthermore, the modified energy have approximation \( \lvert \tilde{E}(u_h^{n}) - E(u_h^{n}) \rvert < \mathcal{O}(\tau^{q-\varepsilon}) \) comparing with original energy.
%\begin{equation}
%E(u_h^n) := \frac{1}{s}\int\left((\mathbf{u}_h^{[n]})^{\mathrm{T}}\mathbf{A}\mathbf{u}_h^{[n]} + \mathbbm{1}_s F(\mathbf{u}_h^{[n]})\right)\mathrm{d}\Omega
%\end{equation}
%
%\end{enumerate}
%\end{theorem}
%\begin{proof}[Proof of \( L^2 \)-stability]
%This proof includes two parts:
%\begin{enumerate}
%\item
%\textbf{\( L^2 \)-norm:}
%
%
%
%
%Testing \( \frac{1}{s}(\mathbf{u}_{h}^{[n+1]})^{\mathrm{T}} \mathbf{A} \) on the both side.   
%On the left-hand side, we can easily attain 
%\begin{align*}
%\frac{1}{s}(\mathbf{u}_{h}^{[n+1]})^{\mathrm{T}} \mathbf{A} (\mathbf{u}_{h}^{[n+1]} - \mathbf{A}\mathbf{u}_{h}^{[n]} ) = \frac{1}{2}\left(\lVert u_h^{n+1} \rVert^{2} - \lVert u_h^{n} \rVert^{2} +  \frac{1}{s}(\delta\mathbf{u}_{h}^{[n+1]})^{\mathrm{T}} \mathbf{A}\delta\mathbf{u}_{h}^{[n+1]}\right),
%\end{align*}
%where \( \mathbf{A} = \mathbf{A}^{2} \) has been used.
%On the right-hand side,  we focus on the dissipation term, in which the \emph{Green's formula} and \emph{polarization identity} will be used, 
%\begin{align*}
% &  \tfrac{1}{s}(\mathbf{u}_{h}^{[n+1]})^{\mathrm{T}} \mathbf{A}\cdot r \mathbf{B}\left(\Delta \mathbf{u}_h^{[n]}\right) \\
%=  &  - \frac{r}{4}\left[\nabla(\mathbf{u}^{[n+1]}+\mathbf{u}^{[n]})^{\mathrm{T}}\left(\frac{\mathbf{AB}}{s}\right)\nabla(\mathbf{u}^{[n+1]}+\mathbf{u}^{[n]}) - (\delta\nabla\mathbf{u}^{[n+1]})^{\mathrm{T}}\left(\frac{\mathbf{AB}}{s}\right)(\delta \nabla \mathbf{u}^{[n+1]})\right] \\
% =  &  - r(\nabla\mathbf{u}^{[n+\tfrac{1}{2}]})^{\mathrm{T}}\left(\frac{\mathbf{AB}}{s}\right)(\nabla\mathbf{u}^{[n+\tfrac{1}{2}]}) + \frac{r}{4}(\delta\nabla\mathbf{u}^{[n+1]})^{\mathrm{T}}\left(\frac{\mathbf{AB}}{s}\right)(\delta\nabla\mathbf{u}^{[n+1]}),
%\end{align*}
%	where the vector difference operator \( \delta \mathbf{u}_h^{[n+1]} := \mathbf{u}_h^{[n+1]} - \mathbf{u}_h^{[n]} \) and the vector mean-value operator \( \mathbf{u}_h^{[n+\tfrac{1}{2}]} := \frac{1}{2}(\mathbf{u}_h^{[n+1]} + \mathbf{u}_h^{[n]})  \). Notice that all of eigenvalue of matrix \( \mathbf{AB} \) are non-negative real number, the right-hand side has upper bound consequently
%\begin{align*}
%% \frac{1}{s}(\mathbf{u}_{h}^{[n+1]})^{\mathrm{T}} \mathbf{A}\cdot r \mathbf{B}\left(\Delta \mathbf{u}_h^{[n]} + \mathbf{f}^{[n]}\right) \le 
%\frac{r}{4}(\delta\nabla\mathbf{u}^{[n+1]})^{\mathrm{T}}\left(\frac{\mathbf{AB}}{s}\right)(\delta\nabla\mathbf{u}^{[n+1]})  +  r(\mathbf{u}_{h}^{[n+1]})^{\mathrm{T}}\left(\frac{\mathbf{AB}}{s}\right)\mathbf{f}^{[n]}.
%\end{align*}
%{\color{red}when \( \tau\rho(K) < r_{n} \)}
%According to the assumption of CFL condition, we know 
%\begin{align*}
%\mathbf{v}_h^{\mathrm{T}} \mathbf{A}\mathbf{v}_h
%\ge
%\frac{r}{2}\rho(\mathbf{K})\mathbf{v}_h^{\mathrm{T}}\mathbf{AB}\mathbf{v}_h
% \ge \frac{r}{2}(\nabla\mathbf{v}_h)^{\mathrm{T}}\mathbf{AB}(\nabla\mathbf{v}_h)
%\end{align*}
%where the each component of \( \mathbf{v}_h := \delta \mathbf{u}_h^{[n+1]} \) belongs to \( V_h \).
%Thence 
%\begin{align*}
% \frac{1}{2}\left(\lVert u_h^{n+1} \rVert^{2} - \lVert u_h^{n} \rVert^{2} \right) \le  r(\mathbf{u}_{h}^{[n+1]})^{\mathrm{T}}\left(\frac{\mathbf{AB}}{s}\right)\mathbf{f}^{[n]} = r \left(u_h^{n+1}, f^{n} + \mathcal{O}(\tau^{q-\varepsilon}) \right)
%\end{align*}
%
%{\color{red}
%\( \lambda(A) > 0, {\color{gray}\lambda(B) > 0} \) and \( \rho(A + z B)  < \rho(A) \) then \( - 2 \lambda(A) \le \lambda(z B) \le 0 \)?} 
%
%For \( z \in \mathbb{R} \), 
%\begin{align*}
%\lvert 1 + z \rvert < 1  & \Leftrightarrow 2 + z > 0 \\
%\rho(A + z B)  < 1  &  \Leftrightarrow \lambda\left(2A + z AB \right) > 0   \\
%\rho(A \otimes M + \tau B \otimes K)  < 1  &  \Leftrightarrow \lambda\left(2A\otimes M + \tau B \otimes K \right) > 0
%\end{align*}
%
%\item
%\textbf{\( H^1 \)-semi-norm:}
%Recalling the discretization as 
%\begin{equation}
%\mathbf{u}_h^{[n+1]} - \mathbf{A}\mathbf{u}_h^{[n]} = r \mathbf{B}\left(\Delta \mathbf{u}_h^{[n]} - f(\mathbf{u}_h^{[n]})\right)
%\end{equation}
%and testing \( \frac{1}{s}(\delta\mathbf{u}_{h}^{[n+1]})^{\mathrm{T}} \mathbf{A} \) on the both side.   
%
%
%\begin{align*}
% \frac{1}{s}(\delta\mathbf{u}_{h}^{[n+1]})^{\mathrm{T}} \mathbf{A}(\delta\mathbf{u}_{h}^{[n+1]})
%  = \lVert \delta u_h^{n+1} \rVert^2.
%\end{align*}
%and \( {\color{red} (b-a)a = \frac{1}{2}\left(b^{2} - a^2 - (b-a)^2\right)} \).
%\begin{align*}
% \frac{r}{s}(\delta\mathbf{u}_{h}^{[n+1]})^{\mathrm{T}} \mathbf{A} \mathbf{B}(\Delta \mathbf{u}_h^{[n]})
% =  &  -  \frac{r}{s} \left((\delta \nabla\mathbf{u}_{h}^{[n+1]})^{\mathrm{T}} \mathbf{A} \mathbf{B}(\nabla \mathbf{u}_h^{[n]})\right) \\
%=  &   -  \frac{1}{2} \cdot \frac{r}{s} \left(\delta(\nabla\mathbf{u}_{h}^{[n+1]})^{\mathrm{T}} \mathbf{A} \mathbf{B}(\nabla \mathbf{u}_h^{[n+1]}) - (\nabla \delta \mathbf{u}_{h}^{[n+1]})^{\mathrm{T}} \mathbf{A} \mathbf{B}(\nabla \delta \mathbf{u}_h^{[n+1]})\right)
%\end{align*}
%
%
%
%
%\end{enumerate}
%
%
%\end{proof}}
%
%\begin{remark}
%Introducing a notation \( \mathbf{G} := \mathbf{A}\otimes \mathbf{M} + r \mathbf{B}(\alpha)\otimes \mathbf{K} \) for amplification matrix, we rewrite 
%\begin{equation}
%\left(\mathbf{E}_{q}\otimes\mathbf{M}\right) \mathbf{u}^{n+1}_{h} = \mathbf{G}\mathbf{u}_{h}^{n} 
%+ r \left(\mathbf{B}(\alpha) \otimes \mathbf{E}_{h} \right) \mathbf{f}_{h}^{n}.
%\end{equation}
%For the relation about the matrix tensor product, viz \( \lambda(A\otimes B) = \{\lambda_i(A) \cdot \lambda_j(B) | 1\le i \le m,  1\le j \le n \} \) where \( \lambda_i(A), \lambda_j(B) \) are eigenvalue of matrix	\( A,B \) respectively, a equivalent relationship yields 
%\begin{equation}
%\rho(\mathbf{G}) <1 \Leftrightarrow \rho(\mathbf{A} + z \lambda(\mathbf{K}) \mathbf{B})  < 1. \nonumber
%\end{equation}
%
%
%
%\end{remark}

%{
%\begin{proof}[Proof of Modified Energy Stability]
%Recalling the discretization as 
%\begin{equation}
%\mathbf{u}_h^{[n+1]} - \mathbf{A}\mathbf{u}_h^{[n]} = r \mathbf{B}\left(\Delta \mathbf{u}_h^{[n]} - f(\mathbf{u}_h^{[n]})\right)
%\end{equation}
%and testing \( \frac{1}{s}(\delta\mathbf{u}_{h}^{[n+1]})^{\mathrm{T}} \mathbf{A} \) on the both side.   
%
%
%\begin{align*}
% \frac{1}{s}(\delta\mathbf{u}_{h}^{[n+1]})^{\mathrm{T}} \mathbf{A}(\delta\mathbf{u}_{h}^{[n+1]})
%  = \lVert \delta u_h^{n+1} \rVert^2.
%\end{align*}
%and \( {\color{red} (b-a)a = \frac{1}{2}\left(b^{2} - a^2 - (b-a)^2\right)} \).
%\begin{align*}
% \frac{r}{s}(\delta\mathbf{u}_{h}^{[n+1]})^{\mathrm{T}} \mathbf{A} \mathbf{B}(\Delta \mathbf{u}_h^{[n]})
% =  &  -  \frac{r}{s} \left((\delta \nabla\mathbf{u}_{h}^{[n+1]})^{\mathrm{T}} \mathbf{A} \mathbf{B}(\nabla \mathbf{u}_h^{[n]})\right) \\
%=  &   -  \frac{1}{2} \cdot \frac{r}{s} \left(\delta(\nabla\mathbf{u}_{h}^{[n+1]})^{\mathrm{T}} \mathbf{A} \mathbf{B}(\nabla \mathbf{u}_h^{[n+1]}) - (\nabla \delta \mathbf{u}_{h}^{[n+1]})^{\mathrm{T}} \mathbf{A} \mathbf{B}(\nabla \delta \mathbf{u}_h^{[n+1]})\right)
%\end{align*}
%
%\begin{align*}
% r  \frac{1}{s}(\delta\mathbf{u}_{h}^{[n+1]})^{\mathrm{T}} \mathbf{A} \mathbf{B}\left(\Delta \mathbf{u}_h^{[n]} - f(\mathbf{u}_h^{[n]})\right)
%\end{align*}
%
%{\color{red}
%\begin{align*}
%F(u) - F(v) = f(\xi)(u-v) \ge f(v)(u-v)
%\end{align*}
%
%}
%
%\end{proof}}

\section{Numerical Verifications}
\label{sec:Numerical Verifications}
This section will present several typical numerical examples to validate the analysis results mentioned earlier.
\subsection{ODE}
To avoid the influence of spatial errors, we first consider a stiff nonlinear ODE problem to testing the convergence of ABTI. In general, such problems do not have an analytical solution. However, the Allen-Cahn ODE problem falls into this category and has an exact solution \cite{Stuart1998}
\begin{equation}
u(t) = \frac{u^0}{ \sqrt{e^{- \frac{2}{ \varepsilon^2}t} + u_0^2(1-e^{- \frac{2}{ \varepsilon^2 }t})}}. \nonumber
\end{equation}
The equation is expressed as follows
\begin{equation}
\partial_t u + \frac{1}{ \varepsilon^2 }f(u) = 0,   t \in [0,T] \text{ with initial data } u(0) = u^0, \nonumber
\end{equation}
where \( f(u) = u^3 - u \) equals to the first derivative of double-well potential function \( F(u) = \tfrac{1}{4}(u^2-u)^2 \). 
\par
Theo. c.o. is the abbreviation for theoretical convergence order. \( \tau \) represents the length of discrete time step, and \( \mathcal{E}(\tau) \) denotes error at the terminal time $T = 1$ under this time discretization. In the entire comparative experiment, we fix the small parameter in the equation as \( \epsilon = 0.5 \), which reflects the width of the interface and the initial data $u^{0} = 0.01$. 
For the parameter \( \alpha \) in the scheme, since different values of \( \alpha \) do not affect the step size constraint, we without loss of generality choose \( \alpha = 1 \). Since we are dealing with a nonlinear problem, the initial solution vector requires solving the following nonlinear algebraic equation
\begin{equation}
\mathbf{u}^{[0]} = u^{0}\mathbbm{1} - \frac{r}{\epsilon^{2}} \mathbf{B}(\alpha)f(\mathbf{u}^{[0]}). \nonumber
\end{equation}
In our experiment, we use the Newton's iteration for numerically solving the above equation, with the initial guess for the iteration set as \( u^{0}\mathbbm{1} \).
\par 
The  Table \ref{tab:AC_ODE_qs} and \ref{tab:AC_ODE_qs1} compare the errors and convergence for the cases \( s = q \) and \( s = q+1 \). From the experimental results, we achieved the optimal convergence of the ABTI method with lower computational cost. Due to the limitations of computer machine precision, it is recommended to use the high-precision toolbox\footnote{\url{https://www.advanpix.com/}} in MATLAB for schemes of order four and above.
\begin{table}[h]
\centering
\begin{tabular}{ccccccc}
\toprule
\multirow{2}{*}{$1/\tau$} &
\multicolumn{2}{c}{$q=s=1$} &
\multicolumn{2}{c}{$q=s=2$} &
\multicolumn{2}{c}{$q=s=3$} \\
\cline{2-7} 
& {$\mathcal{E}(\tau)$} & {c.o.} & {$\mathcal{E}(\tau)$} & {c.o.} & {$\mathcal{E}(\tau)$} & {c.o.} \\
\midrule
  128  &  2.299e-02 & - &  2.011e-02 & - &  1.548e-04 & - \\ 
  256  &  2.669e-02 & -0.215 &  1.024e-02 & 0.974 &  3.941e-05 & 1.974 \\ 
  512  &  2.861e-02 & -0.100 &  5.162e-03 & 0.988 &  9.939e-06 & 1.987 \\ 
 1024  &  2.958e-02 & -0.048 &  2.592e-03 & 0.994 &  2.496e-06 & 1.994 \\ 
Theo. c.o. & - &  0.000 & - & 1.000  &  - & 2.000 \\
\bottomrule
\end{tabular}
\caption{The error and convergence of the original ABTI $s=q$ in solving the Allen-Cahn ODE.}
\label{tab:AC_ODE_qs}
\end{table}

\begin{table}[h]
\centering
\begin{tabular}{ccccccc}
\toprule
\multirow{2}{*}{$N _{\tau}$} &
\multicolumn{2}{c}{$q=s-1=1$} &
\multicolumn{2}{c}{$q=s-1=2$} &
\multicolumn{2}{c}{$q=s-1=3$} \\
\cline{2-7} 
& {$\mathcal{E}(N _{\tau})$} & {c.o.} & {$\mathcal{E}(N _{\tau})$} & {c.o.} & {$\mathcal{E}(N _{\tau})$} & {c.o.} \\
\midrule
  128  &  2.054e-02 & - &  1.687e-04 & - &  4.885e-07 & - \\ 
  256  &  1.034e-02 & 0.990 &  4.115e-05 & 2.035 &  5.287e-08 & 3.208 \\ 
  512  &  5.188e-03 & 0.995 &  1.016e-05 & 2.018 &  6.113e-09 & 3.112 \\ 
 1024  &  2.598e-03 & 0.998 &  2.523e-06 & 2.009 &  7.337e-10 & 3.059 \\ 
Theo. c.o. & - &  1.00 & - & 2.00  &  - & 3.00 \\
\bottomrule
\end{tabular}
\caption{The error and convergence of the modified ABTI $s=q+1$ in solving the Allen-Cahn ODE.}
\label{tab:AC_ODE_qs1}
\end{table}
\subsection{PDE}
Consider classical heat equation 
\begin{equation}
\begin{cases}
\partial_t u = \Delta u, & u \in [0, T] \times [-\pi, \pi], \\
u(0, x) = \cos(x), & x \in [ - \pi, \pi] \text{ and periodical boundary condition}.
\end{cases}
\end{equation}
 As we well known that its exact solution \( u = e^{-t}\cos(x) \) and  obviously \( u \in H_0^{1}([-\pi, \pi]) \).
\par 
In high-dimensional complex regions, the mass and stiffness matrices obtained from high-order finite element methods are relatively complicated. Therefore, to better understand the optimality of the stability conditions derived from the amplification matrix, we first examine the content in the case of a one-dimensional homogeneous boundary condition. Considering the relation between FEM with the linear elements approximation on \( 1 \)-dimensional case and  the center difference methods, the stiffness matrix  for the discretization of Laplacian operator can represented by 
\begin{align*}
\mathbf{K} = 
\frac{1}{h^2}
\begin{bmatrix}
-2 & 1 & & \\
1 & \ddots & \ddots & \\
 & \ddots & \ddots & 1 \\
& & 1 & -2
\end{bmatrix}
\end{align*}
and the mass matrix $\mathbf{M} = \mathbf{E}_h $ is a identity matrix. 
By simple evaluation, we claim \( \mathbf{K} \) is a negative defined matrix and the spectral radius  \( \rho(\mathbf{K}) \le 4/h^2 \). Meanwhile, a key tensor identity will be used, namely  \( \rho(\mathbf{G}) =  \rho(\mathbf{A}\otimes \mathbf{E}_{h} + \tau \mathbf{B}\otimes \mathbf{K}) = \rho(\mathbf{A} - \tau \rho(\mathbf{K}) \mathbf{B}) \).
It is easy to check that \( \rho(\mathbf{G}) < 1 \) whenever \( \tau/h^2 \le r_{n}/4 \). 
Thus, for a given target time approximation accuracy, the determination of the parabolic CFL condition only requires the discriminant \ref{eq:discriminant}  to determine the parabolic radius \( r_n \).
\par
For example, to illustrate with a fixed second-order time approximation accuracy, we have \( r_2 \approx 0.7639 \). At this point, we fix the spatial step size \( h = \pi/32 \) and choose \( \tau = r_2/4\times h^2 \) and \( \tau = r_2/4 \times h^2 (1\pm 0.1) \), and compare their numerical solutions to test the stability of the scheme. Set the parameter of ABTI \( \alpha = 1 \) and \( s = q + 1 = 3 \). 
The table below shows the spectral radius of the amplification matrix under different conditions
%\begin{table}[h]
%\centering
%\begin{tabular}{ccccccc}
%\toprule
%\multirow{2}{*}{$N _{\tau}$} &
%\multicolumn{2}{c}{$\tau = 0.6/4\times h^2 - 0.5 $} &
%\multicolumn{2}{c}{$0.6/4\times h^2$} &
%\multicolumn{2}{c}{$0.6/4\times h^2 + 0.5$} \\
%\cline{2-7} 
%& {$\mathcal{E}(\tau, h)$} & {c.o.} & {$\mathcal{E}(\tau, h)$} & {c.o.} & {$\mathcal{E}(\tau, h)$} & {c.o.} \\
%\midrule
%128  &  5.724e-03 & 0.000 &  2.588e-05 & 0.000 &  8.795e-08 & 0.000 \\
%256  &  3.258e-03 & 0.813 &  8.470e-06 & 1.611 &  1.654e-08 & 2.411 \\ 
%512  &  1.620e-03 & 1.008 &  2.107e-06 & 2.007 &  2.057e-09 & 3.007 \\ 
%1024  &  8.074e-04 & 1.004 &  5.254e-07 & 2.004 &  2.565e-10 & 3.003 \\ 
%%Theo. c.o. & - &  2.00 & - & 2.00  &  - & 2.00 \\
%\bottomrule
%\end{tabular}
%\caption{The \( L^2 \) error and convergence order with fixed parameters $\alpha=1$.}
%\label{tab:ending_time}
%\end{table}
and Figure \ref{fig:blowup} shows the corresponding numerical solution. It is easy to see that when the time step size is outside the range constrained by the CFL condition, the numerical solution will exhibit a blow-up phenomenon.
\begin{center}
\begin{tabular}{ c|c|c|c  } 
\hline
\( \tau  \) & \( r_2/4 \times h^2\times 0.9 \) & \( r_2/4 \times h^2 \) & \( r_2/4 \times h^2\times 1.1 \)  \\
\hline
\( \rho(\mathbf{G})  \) & 0.9996 & 0.9995 &1.1161  \\ 
\hline
\end{tabular}
\end{center}
\begin{figure}[h]
\centering
\includegraphics[scale=.33]{fig/Num_Solu_0.eps}
\includegraphics[scale=.33]{fig/Num_Solu_1.eps}
\includegraphics[scale=.33]{fig/Num_Solu_2.eps}
\caption{\( \tau=r_2/4 \times h^2\times 0.9 \) (Left), \( \tau=r_2/4 \times h^2 \) (Middle) and \( \tau = r_2/4 \times h^2\times 1.1 \) (Right). }
\label{fig:blowup}
\end{figure}

\section{Conclusion}
In this paper, we address a conjecture regarding the uniform stability of Adams-Bashforth-type integrator with arbitrary-order accuracy. We provide the discriminant conditions for maintaining scheme stability at a given parabolic radius and the allowable accuracy for this stability. We also identify the reasons why the original scheme fails to achieve the ideal accuracy and propose solutions. As an application of the algorithm to parabolic equations, we derive the CFL condition compatible with ODE problems, along with its \( L^2 \)-stability and error bounds. Finally, some numerical examples are presented at the end of the paper to validate our analytical results.
\section{Acknowledgment}
This work is partially supported by the National Natural Science Foundation of China No.12171385. 


\appendix
\section{Derivation of Characteristic Polynomial}
\label{sec:Derivation of Characteristic Polynomial}
Derivation of characteristic polynomial of matrix \( \mathbf{A}+ z \mathbf{B} \) includes two steps. First, a trick of the determinant of the block matrix deduces the equivalent matrix in the sense of characteristic polynomial. Second, the virtue of a constructed matrix helps us to obtain a difference equation about the determinant, and its written matrix form can carry out whose specific expression. In addition, the notation \( z \) used in the lemmas and proofs in this section simply represents a complex variable, which is distinct from \( z = \lambda \tau \in \mathbb{C} \).
\begin{lemma}\label{lem:same_characteristic_polynomial}
For all \( \alpha \in \mathbbm{R}^{+} \) and \( z \in \mathbb{C} \), matrices $\mathbf{A}+z\mathbf{S}(\alpha)\mathbf{F}$ and $\mathbf{e}_{1}\mathbf{e}_{1}^{\mathrm{T}} + z \mathbf{F}\mathbf{S}(\alpha)$ have the same characteristic polynomial. Here, \( \mathbf{e}_1 \) stand for a vector whose only the first element equals one and vanish else, and length keep up with the square matrix \( \mathbf{FS}(\alpha) \).
\end{lemma}
\begin{proof}
%[Proof of Lemma \ref{lem:same_characteristic_polynomial}]
By the standard operation, we consider the characteristic polynomial in start
\begin{equation}
p_q(\lambda;\mathbf{A}+z\mathbf{B}(\alpha)) := 
\begin{vmatrix}
\mathbf{A} + z\mathbf{S}(\alpha)\mathbf{F} -\lambda \mathbf{E}_{q}
\end{vmatrix}\nonumber
\end{equation}
and then effort to obtain its expression. By the block determinant method, the right-hand side of the above formula equals to 
\begin{align*}
\begin{vmatrix}
\mathbf{E}_{q} & \mathbf{F} \\
-z\mathbf{S}(\alpha) & \mathbf{A} - \lambda \mathbf{E}_{q}
\end{vmatrix}
= 
\begin{vmatrix}
1 & \bm{0}^{\mathrm{T}} & \bm{0}^{\mathrm{T}} \\
\bm{0} & \mathbf{E}_{q} & \mathbf{F} \\
-\mathbbm{1} & -z\mathbf{S}(\alpha) & (\mathbf{A} - \lambda \mathbf{E}_{q})
\end{vmatrix}
=
\begin{vmatrix}
1 & \bm{0}^{\mathrm{T}} & \frac{1}{q}\mathbbm{1}^{\mathrm{T}} \\
\bm{0} & \mathbf{E}_{q} & \mathbf{F} \\
-\mathbbm{1} & -z\mathbf{S}(\alpha) & - \lambda \mathbf{E}_{q}
\end{vmatrix} 
\end{align*}
where $\mathbf{E}_{q}$ is an identity matrix and obviously reversible. Add a last row and a column for the last determination on the above in order to ensure the block determinant is a square one, we consequently obtain that
\begin{align*}
-\frac{1}{\lambda} \cdot 
\begin{vmatrix}
1 & \bm{0}^{\mathrm{T}} & \frac{1}{q}\mathbbm{1}^{\mathrm{T}} & 0 \\
0 & \mathbf{E}_{q} & \mathbf{F} & \bm{0} \\
- \mathbbm{1} & -z\mathbf{S}(\alpha) & - \lambda \mathbf{E}_{q} & \bm{0} \\
0 & \bm{0}^{\mathrm{T}} & \bm{0}^{\mathrm{T}} & -\lambda \\
\end{vmatrix} 
= - \frac{1}{\lambda}\cdot \left|-\lambda \mathbf{E}_{q+1} - 
\begin{bmatrix}
\frac{1}{q}\mathbbm{1}^{\mathrm{T}} & 0 \\
\mathbf{F} & \bm{0}
\end{bmatrix}
\begin{bmatrix}
-\mathbbm{1} & -z\mathbf{S}(\alpha) \\
0 & \bm{0}^{\mathrm{T}} 
\end{bmatrix}  \right| \\
= 
-\frac{1}{\lambda}
\begin{vmatrix}
-\lambda\mathbf{E}_{q+1} 
-
\begin{bmatrix}
- 1 & -z\frac{1}{q}\mathbbm{1}^{\mathrm{T}}\mathbf{S}(\alpha) \\
- \mathbf{F}\mathbbm{1} & - z \mathbf{F}\mathbf{S}(\alpha)
\end{bmatrix}
\end{vmatrix}
= 
-\frac{1}{\lambda}(-1)^{q+1}
\begin{vmatrix}
\lambda -1 & -z \frac{1}{q}\mathbbm{1}^{\mathrm{T}} \mathbf{S}(\alpha) \\
- \mathbf{F}\mathbbm{1} &  \lambda\mathbf{E}_{q} - z \mathbf{F}\mathbf{S}(\alpha)
\end{vmatrix}.
\end{align*}
Here, obviously the communicative property of certain matrix such that 
\begin{align*}
\begin{bmatrix}
- \lambda \mathbf{E}_q & \bm{0} \\
\bm{0}^{\mathrm{T}} & - \lambda
\end{bmatrix}
\begin{bmatrix}
- \mathbbm{1} & -z\mathbf{S}(\alpha) \\
0 & \bm{0}^{\mathrm{T}} 
\end{bmatrix}
= 
\begin{bmatrix}
- \mathbbm{1} & -z\mathbf{S}(\alpha) \\
0 & \bm{0}^{\mathrm{T}} 
\end{bmatrix}
\begin{bmatrix}
- \lambda \mathbf{E}_q & \bm{0} \\
\bm{0}^{\mathrm{T}} & - \lambda
\end{bmatrix}
\end{align*}
which means the requirement of the determination of the block matrix has been guaranteed. 
\par Notice that  \( \mathbf{F}\mathbbm{1} = [1,0, \ldots, 0]=:\mathbf{e}_{1}^{\mathrm{T}} \) and \( \tfrac{1}{q}\mathbbm{1}^{\mathrm{T}} \mathbf{S}(\alpha) \) equals the first row of \( \mathbf{F}\mathbf{S}(\alpha) \), subtract the first row from the second row and add the second column to the first column after that, that is 
\begin{align*}
& 
\begin{vmatrix}
\lambda -1 & -z \frac{1}{q}\mathbbm{1}^{\mathrm{T}} \mathbf{S}(\alpha) \\
- \mathbf{F}\mathbbm{1} &  \lambda\mathbf{E}_{q} - z \mathbf{F}\mathbf{S}(\alpha)
\end{vmatrix}
=
\begin{vmatrix}
\lambda & -\lambda\mathbf{e}_{1}^{\mathrm{T}} \\
-\mathbf{e}_{1} & \lambda\mathbf{E}_{q} -z \mathbf{F}\mathbf{S}(\alpha)
\end{vmatrix}
= 
\begin{vmatrix}
\lambda & \bm{0}^{\mathrm{T}} \\
-\mathbf{e}_1 & \lambda\mathbf{E}_{q}-z \mathbf{F}\mathbf{S}(\alpha) - \mathbf{e}_{1}\mathbf{e}_{1}^{\mathrm{T}} 
\end{vmatrix} \\
=  & 
\lambda \left|\lambda\mathbf{E}_{q}-z \mathbf{F}\mathbf{S}(\alpha) - \mathbf{e}_{1}\mathbf{e}_{1}^{\mathrm{T}} \right| = \lambda (-1)^q \left|z\mathbf{F}\mathbf{S}(\alpha) + \mathbf{e}_{1}\mathbf{e}_{1}^{\mathrm{T}} - \lambda\mathbf{E}_{q} \right|.
\end{align*}
Therefore,
\begin{equation}
%\label{eq:same_characteristic_polynomial}
p_q(\lambda;\mathbf{A}+z\mathbf{S}(\alpha)\mathbf{F}) = p_q(\lambda;\mathbf{e}_{1}\mathbf{e}_{1}^{\mathrm{T}} + z \mathbf{F}\mathbf{S}(\alpha)) \nonumber
\end{equation}
finished this proof.
\end{proof}
 By this operation, we can get two benefits that $\mathbf{e}_{1}\mathbf{e}_{1}^{\mathrm{T}}$ is extremely sparse matrix with only one non-trivial element at first-column first-row and $\mathbf{F}\mathbf{S}(\alpha)$ is a real quasi-upper triangle matrix with non-zero lower sub-diagonal elements which will be investigate below. By the way, while both of $\mathbf{F}$ and $\mathbf{S}(\alpha)$ are square matrix, we can straightforwardly use the unitary property of matrix $\mathbf{F}$, that is $\mathbf{F}\mathbf{F}^{*}=\mathbf{E}=\mathbf{F}^*\mathbf{F}$. 
\begin{lemma}\label{lem:two_inverse_matrix}
For all $0\le j \le q$,
\begin{enumerate}[1)]
\item 
\begin{equation}\label{eq:equlity_one}
\begin{bmatrix}
\beta_{0} & 0 & \ldots & 0 \\ 
\beta_{1} & \beta_{0} & \ldots & 0 \\ 
\vdots & \ddots & \ddots & \vdots \\
\beta_{q} & \ldots & \beta_{1} & \beta_{0}
\end{bmatrix}^{-1}
=
\begin{bmatrix}
\beta_{0}^{(-1)} & 0 & \ldots & 0 \\ 
\beta_{1}^{(-1)} & \beta_{0}^{(-1)} & \ldots & 0 \\ 
\vdots & \ddots & \ddots & \vdots \\
\beta_{q}^{(-1)} & \ldots & \beta_{1}^{(-1)} & \beta_{0}^{(-1)}
\end{bmatrix}
\end{equation}
where $ \beta_{j} = \gamma_{j}(-\alpha z) $ and $ \beta_{j}^{(-1)} = \gamma_{j}(\alpha z) $.
\item 
\begin{equation}
\begin{bmatrix}
1 && &&\\ 
\lambda\beta_{0}^{(-1)} & 1 &&&\\ 
\vdots & \ddots & \ddots &  &  \\
\lambda\beta_{q-1}^{(-1)} & \ldots & \lambda\beta_{0}^{(-1)} & 1
\end{bmatrix}^{-1}
= 
\begin{bmatrix}
\eta_{0} && &&\\ 
\eta_{1} & \eta_{0} & &&\\ 
\vdots & \ddots & \ddots &  &  \\
\eta_{q} & \ldots & \eta_{1} & \eta_{0}
\end{bmatrix}
\end{equation}
where $ \eta_{j} = \Sigma_{k=0}^{j-1}\gamma_{k}(j-k)(-\lambda)^{j-k}(\alpha z)^{k} $ for all $0\le j \le q$.

\end{enumerate}
\end{lemma}
\begin{proof}
The inverse matrix derived from the identity of a lower triangular matrix is always invertible and remains a lower triangular matrix with an identity. The correctness of the lemma can be verified by confirming that the product of a matrix and its inverse yields the identity matrix. In fact, the proof of both relations ultimately comes down to the binomial theorem.
\begin{enumerate}[1)]
\item 
It is easy to check for the equality \eqref{eq:equlity_one} that
\begin{align*}
\sum_{\nu=0}^{i-j}\beta_{\nu}\beta_{i-j-\nu}^{(-1)} = & \sum_{\nu=0}^{i-j}\gamma_{\nu}(-\alpha z)\gamma_{i-j-\nu}(\alpha z) = \gamma_{i-j}(\alpha)\sum_{\nu=0}^{i-j}\binom{i-j}{\nu}(-1)^{\nu}(1)^{i-j-\nu} \\
= & \gamma_{i-j}(\alpha \cdot 0) =
\begin{cases}
1, \text{ when \( j = i \),}\\
0, \text{ otherwise. }
\end{cases}
\end{align*}
\item 
Let $\beta_{-1}^{(-1)}:=1/\lambda$ for the compatible which need to be considered. 
While $i=j$, $\lambda \beta_{-1}^{(-1)}\eta_{0} = 1$ obviously. The same result can be achieved by stating that the inverse of the identity lower triangular matrix must be the identity lower triangular matrix. Next, we consider $i-j\ge 1$,
\begin{align*}
& \sum_{\nu=0}^{i-j}\lambda\beta_{\nu-1}^{(-1)}\eta_{i-j-\nu} = \eta_{i-j} + \sum_{\nu=1}^{i-j-1}\lambda\beta_{\nu-1}^{(-1)}\eta_{i-j-\nu} + \lambda \beta_{i-j-1}^{(-1)}\eta_{0}   \\
 = & 
\sum_{k=0}^{i-j-1}\gamma_{k}(i-j-k)(-\lambda)^{i-j-k}(\alpha z)^{k} 
+ \\
& \lambda\sum_{\nu=1}^{i-j-1}\gamma_{\nu-1}(\alpha z) \sum_{k=0}^{i-j-\nu-1}\gamma_{k}(i-j-\nu-k)(-\lambda)^{i-j-\nu-k}(\alpha z)^{k} \\
& + \lambda \gamma_{i-j-1}(\alpha z) \\
% = & 
%\sum_{k=0}^{i-j-1}\gamma_{k}(i-j-k)(-\lambda)^{i-j-k}(\alpha z)^{k} 
%+ \\
%& \lambda\sum_{\nu=1}^{i-j-1}\gamma_{\nu-1}(\alpha z) \sum_{k=1}^{i-j-\nu}\gamma_{k-1}(i-j-\nu-k+1)(-\lambda)^{i-j-\nu-k+1}(\alpha z)^{k-1} \\
%& + \lambda \gamma_{i-j-1}(\alpha z) \\
 = & (-\lambda)^{i-j}\sum_{k=0}^{i-j-2}\gamma_{k}\left((i-j-k) \frac{-\alpha z}{\lambda} \right)  \\
& - (-\lambda)^{i-j}\sum_{\nu=1}^{i-j-1}\gamma_{\nu-1}\left(\frac{-\alpha z}{\lambda}\right) \sum_{k=0}^{i-j-\nu-1}\gamma_{k}\left((i-j-\nu-k)\frac{-\alpha z}{\lambda} \right). 
\end{align*}
While $i=j+1$, the above equality equals zeros for which the summation will be empty.
\par While \( i \ge j +2 \), let \( H:=i-j-2 \ge 0 \) and \( x:= - \tfrac{z}{\lambda} \) as brief notations,
\begin{align*}
 & \sum_{k=0}^H \gamma_{k}\left((H-k+2)x \right)  -  \sum_{\nu=0}^H \gamma_{\nu}(x) \sum_{k=0}^{H-\nu}\gamma_{k}\left((H-\nu-k+1)x \right) \\
    =  & \sum_{k=0}^H \gamma_{k}(H-k+2)x^{k} -  \sum_{\nu=0}^H \sum_{k=0}^{H-\nu}\frac{(H-\nu-k+1)^k}{k!\cdot \nu!} x^{\nu+k} \\
    =  & \sum_{\mu=0}^H \gamma_{\mu}(H-\mu+2)x^{\mu} -  \sum_{\mu=0}^H \left(\sum_{k=0}^{\mu} \frac{(H-\mu+1)^k}{k!\cdot (\mu-k)!}\right) x^{\mu}, \text{ let $\mu = \nu+k$} \\
    = & \sum_{\mu=0}^{H}\left(\gamma_{\mu}(H-\mu+2)  - \sum_{k=0}^{\mu} \frac{(H-\mu+1)^k}{k!\cdot (\mu-k)!}\right)x^{\mu} \\
= & \sum_{\mu=0}^{H}\left(\gamma_{\mu}(H-\mu+2)  - \frac{1}{\mu!}\sum_{k=0}^{\mu}\binom{\mu}{k}(H-\mu+1)^{k}1^{\mu-k}\right)x^{\mu} \\
= & {\sum_{\mu=0}^{H}\left(\gamma_{\mu}(H-\mu+2)  - \frac{(H-\mu+2)^{\mu}}{\mu!}\right)x^{\mu} = \sum_{\mu=0}^{H}0\cdot x^{\mu} = 0}. 
\end{align*}
\end{enumerate}
At this point, the proof of this lemma is complete.
\end{proof}
\begin{lemma}\label{lem:simplify_eig_polynomial}
For all \( \alpha \in \mathbb{R} \) and \( z, \lambda \in \mathbb{C} \), then 
\begin{equation}
\sum_{j=0}^{q}\gamma_{q-j}(\alpha z)\sum_{k=0}^{j-1}\gamma_{k}((j-k)\alpha z)(-\lambda)^{j-k} = \sum_{j=0}^{q}\gamma_{q-j}((j+1)\alpha z)(-\lambda)^{j}. 
\end{equation}
If $q=0$, then $\Sigma_{k=0}^{-1} = 1$ on the left-hand side of the above equality for setting.
\end{lemma}
\begin{proof}
To improve readability, we use the matrix language to represent two different summation methods and their relationship visually. Denote \(A_j := \gamma_{q-j}(\alpha z), C_{j-k}^{(j)} := \gamma_{k}((j-k)\alpha z) \) 
\begin{equation}\label{eq:matrix_to_sum}
\mathbbm{1}^{\mathrm{T}}
\begin{bmatrix}
A_1 &&& \\
& A_2 && \\
&& \ddots & \\
&&& A_q
\end{bmatrix}
\cdot
\begin{bmatrix}
C_{1}^{(1)} & &&& \\
C_{1}^{(2)} & C_{2}^{(2)} & &&\\
\vdots & \vdots & \ddots \\
C_{1}^{(q)} & C_{2}^{(q)} & \ldots & C_{q}^{(q)} 
\end{bmatrix}
\begin{bmatrix}
(-\lambda)^{1} \\
(-\lambda)^{2} \\
\vdots \\
(-\lambda)^{q} \\
\end{bmatrix}.
\end{equation}
In the following, $\mathbf{LHS}$ and $\mathbf{LHS}$ abbreviate the left-hand side and right-hand side respectively. 
\begin{align*}
\mathbf{LHS} = & \sum_{j=1}^{q}A_j \sum_{k=0}^{j-1}C_{j-k}^{(j)}(- \lambda)^{j-k} + \gamma_{q}(\alpha z),  \text{ evaluate from right to left in eq. \eqref{eq:matrix_to_sum}}\\ 
= & \sum_{j=1}^{q}\left(\sum_{k=j}^{q}A_k C_{j}^{(k)}\right)(-\lambda)^{j}  + \gamma_{q}(\alpha z), \text{ evaluate from left to right in eq. \eqref{eq:matrix_to_sum}} \\
= & \sum_{j=1}^{q}\left(\sum_{k=j}^{q}\gamma_{q-k}(\alpha z)\cdot \gamma_{k-j}(j \alpha z)\right)(-\lambda)^{j}  + \gamma_{q}(\alpha z) \\
%= & \sum_{j=1}^{q}\left(\sum_{k=j}^{q}\frac{j^{k-j}(\alpha z)^{q-j}}{(q-k)!\cdot (k-j)!}\right)(-\lambda)^{j}  + \omega_{q}(\alpha z)  \\
= & \sum_{j=1}^{q}\left(\sum_{k=j}^{q} j^{k-j}\binom{q-j}{q-k}\right)\cdot \gamma_{q-j}(\alpha z)(-\lambda)^{j}  + \gamma_{q}(\alpha z)  \\
= & \sum_{j=0}^{q}\gamma_{q-j}((j+1)\alpha z)(-\lambda)^{j} = \mathbf{RHS},
\end{align*}
where the last line use a fact that \( \Sigma_{k=j}^{q}j^{k-j}\tbinom{q-j}{q-k} = (j+1)^{q-j} \text{ for all \( 1 \le j \le q \)}  \) which is easy to check by binomial theorem and \( j=0 \) can include the term \( \gamma_{q}(\alpha z) \).
\end{proof}
\begin{lemma}\label{lem:characteristic_polynomial_evaluation}
For all \( z \in \mathbb{C} \) and \( q \ge 1 \), 
\begin{equation}
 p_q(\lambda; \mathbf{e}_1 \mathbf{e}_1 ^{\mathrm{T}} + z \mathbf{FS}(\alpha)) = f_q( \lambda; \alpha z) + f _{q-1}( \lambda; \alpha z) - \gamma_{q}(-z)\delta_{q}, \nonumber
\end{equation} 
where \( f_q( \lambda; \alpha z) :=  \sum_{j=0}^{q} \gamma_{q-j}((j+1)\alpha z)(-\lambda)^{j} \).
\end{lemma}
\begin{proof}
 By simple evaluation, we can know that  
\begin{align*}
\langle \mathbf{FS}(\alpha)\rangle_{j,k} = & \sum_{n=1}^{q}\mathbf{F}_{j,n}\mathbf{S}_{n,k}(\alpha) =  \frac{1}{q}\sum_{n=1}^{q}\omega_n^{1-j}\cdot\frac{(\alpha+\omega_n)^k}{k}
= \frac{1}{q}\sum_{n=1}^{q}\omega_n^{1-j}\cdot\frac{1}{k}\sum_{m=0}^{k}\binom{k}{m}\alpha^{k-m}\omega_{n}^{m} \\
= & \sum_{m=0}^{k}\frac{1}{k}\binom{k}{m}\alpha^{k-m}\cdot\frac{1}{q} \sum_{n=1}^{q} \omega_n^{1-j+m} \\
= & 
\begin{cases}
\frac{1}{k}\binom{k}{j-1}\alpha^{k-j+1} + \frac{1}{q}\delta_{\{(j,k),(1,q)\}} &\text{ if } j \le k+1,\\
0 , &\text{ if } j > k+1,
\end{cases}
\end{align*}
where should be used a fact that $\tfrac{1}{q}\Sigma_{n=1}^{q} \omega_n^{1-j+m} = 1 $ whenever $1-j+m = 0 ~ (\mod q)$ and varnishing otherwise.
\begin{align*}
\begin{cases}
0 \le m = j -1 \le k \le q & \Rightarrow 1 \le j \le q, k \ge j -1. \\
0 \le m = j -1 + q \le k \le q & \Rightarrow j = 0, k = q.
\end{cases}
\end{align*}
Particularly, the term $\tfrac{1}{q}\delta_{\{(j,k),(1,q)\}}$ will vanish when $s>q$. Then, 
$\mathbf{FS}(\alpha)$ is a real quasi-upper triangle matrix with a non-zero lower sub-diagonal elements.
Denote 
\begin{equation}
m_{j,k} = 
\begin{cases}
z\mu_{j,k} & \text{ when } 1 \le j < k \le q, \\
z\mu_{j,k} - \lambda & \text{ when } 1 \le j = k \le q. \\
% z\mu_{j,k} - \lambda + 1 & \text{ when } 1 = j = k.
\end{cases}
\text{ with } \mu_{j,k} := \frac{1}{k}\binom{k}{j-1}\alpha^{k-j+1}, \nonumber 
\end{equation}
then \( p_q(\lambda; \mathbf{e}_1 \mathbf{e}_1 ^{\mathrm{T}} + z \mathbf{FS}(\alpha)) \) equals to
\begin{align*}
 & 
\begin{vmatrix}
m_{1,1} + 1 & m_{1,2} & \ldots & m_{1,q-1} & m_{1,q} + \frac{z}{q} \\
z & m_{2,2}  & \ldots & m_{2,q-1} & m_{2,q} \\
  & z/2 & \ddots & m_{3,q-1} & m_{3,q} \\
  &   & \ddots & \vdots & \vdots \\
&  &  & z/(q-1) & m_{q,q}
\end{vmatrix} \\
= & 
\begin{vmatrix}
m_{1,1} & m_{1,2} & m_{1,3} & \ldots & m_{1,q} \\
z & m_{2,2} & m_{2,3} & \ldots & m_{2,q} \\
 & z/2 & m_{3,3} & \ldots & m_{3,q} \\
 &  & \ddots & \ddots & \vdots \\
 &  &  & & m_{q,q}
\end{vmatrix}
+ 
\begin{vmatrix}
m_{2,2} & m_{2,3} & m_{2,4} & \ldots & m_{2,q} \\
z/2 & m_{3,3} & m_{3,4} & \ldots & m_{3,q} \\
 & z/3 & m_{4,4} & \ldots & m_{4,q} \\
 & & \ddots & \ddots & \vdots \\
 & &  &  & m_{q,q}
\end{vmatrix}
- \gamma_{q}(-z) \\
=: & D_{q} + \tilde{D}_{q} - \gamma_{q}(-z).
\end{align*}
The blank part of the matrix is filled with zero elements. Here, we just give the operation of determinant $D_q$ while $\tilde{D}_q$ can be calculated by similar way. 
Expand $D_q$ along with the last line and then get a recursion relation
\begin{align*}
D_q = & m_{q,q}D_{q-1} - \frac{z}{q-1}\left(m_{q-1,q}D_{q-2} - \frac{z}{q-2}\left(\ldots\right)\right) \\
= & m_{q,q}D_{q-1} + \frac{-zm_{q-1,q}}{q-1}D_{q-2} + \frac{z^2m_{q-2,q}}{(q-2)_{2}}D_{q-3} + \ldots  + \frac{(-z)^{q-2} m_{2,q}}{(2)_{q-2}}D_1 + \frac{(-z)^{q-1} m_{1,q}}{(1)_{q-1}} \\
= & \sum_{j=0}^{q-1}\frac{(-z)^{j}}{(q-j)_{j}}m_{q-j,q}D_{q-j-1} = - \sum_{j=0}^{q-1}\frac{(-z)^{j+1}}{(q-j)_{j}} \frac{1}{q}\binom{q}{q-j-1}\alpha^{j+1} D_{q-j-1}  - \lambda D_{q-1} \\
 = & - \sum_{j=0}^{q-1}\gamma_{j+1}(-\alpha z)D_{q-j-1} - \lambda D_{q-1}, 
\end{align*}
where $(a)_n := a(a+1)\ldots(a+n-1), (a)_0 := 1 $ are the rising factorial (or called Pochhammer symbol) and  \( D_{0} = 1 \) for setting. 
Alternatively, we can also obtain the same result directly using the related formula about Hessenberg matrix.
\par 
Thus, we can obtain a difference equation associated with the determinant \( D_{j} \), 
\begin{equation}
\lambda D_{q-1}  + \sum_{j=0}^{q}\gamma_{j}(-\alpha z)D_{q-j} = 0, \nonumber
\end{equation}
denote \( \beta_{j}:= \gamma_{j}(-\alpha z) \), which formulate a matrix equation about unknown vector \( \mathbf{d} \) 
\begin{equation}
\bm{\Lambda}\mathbf{d} + \mathbf{B}\mathbf{d}=\mathbf{e}_{1}, \nonumber
\end{equation}
where 
\begin{equation}
\bm{\Lambda} := 
\begin{bmatrix}
0 &  & & \\
\lambda & 0  & & \\
 & \ddots  & \ddots &  \\
& & \lambda & 0
\end{bmatrix}, 
\mathbf{B} := 
\begin{bmatrix}
\beta_{0} & & & & \\ 
\beta_{1} & \beta_{0} & & & \\ 
\vdots & \ddots & \ddots & & \\
\beta_{q} & \ldots & \beta_{1} & \beta_{0}
\end{bmatrix}
\text{ and }
\mathbf{d}:=
\begin{bmatrix}
D_0 \\
D_1 \\
\vdots \\
D_q
\end{bmatrix}. \nonumber
\end{equation}
%\begin{align*}
%\begin{bmatrix}
%0 &  & & \\
%\lambda & 0  & & \\
% & \ddots  & \ddots &  \\
%& & \lambda & 0
%\end{bmatrix}
%\begin{bmatrix}
%D_0 \\
%D_1 \\
%\vdots \\
%D_q
%\end{bmatrix}
%+ 
%\begin{bmatrix}
%\beta_{0} & & & & \\ 
%\beta_{1} & \beta_{0} & & & \\ 
%\vdots & \ddots & \ddots & & \\
%\beta_{q} & \ldots & \beta_{1} & \beta_{0}
%\end{bmatrix}
%\begin{bmatrix}
%D_0 \\
%D_1 \\
%\vdots \\
%D_q
%\end{bmatrix}
%=
%\begin{bmatrix}
%1 \\
%0 \\
%\vdots \\
%0 
%\end{bmatrix}.
%\end{align*}
Thence, 
\begin{equation}
\bm{\Lambda}\mathbf{d} + \mathbf{B}\mathbf{d}=\mathbf{e}_{1}
\Rightarrow 
(\mathbf{B}^{-1}\bm{\Lambda} + \mathbf{E})\mathbf{d} = \mathbf{B}^{-1} \mathbf{e}_{1} 
\Rightarrow 
\mathbf{d} = (\mathbf{B}^{-1}\bm{\Lambda} + \mathbf{E})^{-1} \mathbf{B}^{-1} \mathbf{e}_{1}.
 \nonumber
\end{equation}
Firstly, thank to the property of Toeplitz matrix with  \( \beta_{j} = \gamma_{j}(-\alpha z) \), \( \mathbf{B}^{-1} \) also possess Toeplitz structure and each entries have expression 
\begin{equation}
\mathbf{B}^{-1} = 
\begin{bmatrix}
\beta_{0}^{(-1)} & &&\\ 
\beta_{1}^{(-1)} & \beta_{0}^{(-1)} &&\\ 
\vdots & \ddots & \ddots & & \\
\beta_{q}^{(-1)} & \ldots & \beta_{1}^{(-1)} & \beta_{0}^{(-1)}
\end{bmatrix}
\text{ where \( \beta_{j}^{(-1)} = \gamma_{j}(\alpha z) \) } \text{( by Lemma \ref{lem:two_inverse_matrix})}.  \nonumber
\end{equation}
%Consequently, 
%\begin{equation}
%\mathbf{d} = (\mathbf{B}^{-1}\bm{\Lambda} + \mathbf{E})^{-1} \mathbf{B}^{-1} \mathbf{e}_{1}, \nonumber
%\end{equation}
%
%Thence, our goal determinants solve the following classical algebra system
%\begin{align*}
%\begin{bmatrix}
%1 && &&\\ 
%\lambda\beta_{0}^{(-1)} & 1 &&&\\ 
%\vdots & \ddots & \ddots &  &  \\
%\lambda\beta_{q-1}^{(-1)} & \ldots & \lambda\beta_{0}^{(-1)} & 1
%\end{bmatrix}
%\begin{bmatrix}
%D_0 \\
%D_1 \\
%\vdots \\
%%D_{q-1} \\
%D_q
%\end{bmatrix}
%= 
%\begin{bmatrix}
%\beta_{0}^{(-1)} \\ 
%\beta_{1}^{(-1)} \\ 
%\vdots   \\
%\beta_{q}^{(-1)} 
%\end{bmatrix}.
%\end{align*}
Secondly, \( \mathbf{H}=(\mathbf{B}^{-1}\bm{\Lambda} + \mathbf{E})^{-1} \) is a Teoplitz matrix similarly, 
\begin{equation}
\mathbf{H} := 
\begin{bmatrix}
\eta_{0} && &&\\ 
\eta_{1} & \eta_{0} & &&\\ 
\vdots & \ddots & \ddots &  &  \\
\eta_{q} & \ldots & \eta_{1} & \eta_{0}
\end{bmatrix}
\text{ where } \eta_{n} = \sum_{j=0}^{n-1}\gamma_{j}(n-j)(-\lambda)^{n-j}(\alpha z)^{j} \text{ ( by Lemma \ref{lem:two_inverse_matrix})}\nonumber
\end{equation}
where \( 0 \le n \le q \) and let \( \Sigma_{j=0}^{-1} := 1 \).
Since  \( \beta_{j}^{(-1)} \) and \( \eta_{j} \) for \( 0 \le j \le q \) are specific, immediately we can obtain the explicit expression of  \( D_{q} = \Sigma_{j=0}^{q}\beta_{q-j}^{(-1)}\cdot\eta_{j} \), that is, 
\begin{align}
D_q = \sum_{j=0}^{q}\gamma_{q-j}(\alpha z) \sum_{k=0}^{j-1}\gamma_{k}(j-k)(-\lambda)^{j-k}(\alpha z)^{k} = \sum_{j=0}^{q}
\gamma_{q-j}((j+1)\alpha z)(-\lambda)^{j}, \nonumber
\end{align}
where the Lemma \ref{lem:simplify_eig_polynomial} should be used. By similar operation, 
\begin{equation}
\tilde{D}_{q}= \sum_{j=0}^{q-1}\gamma_{q-1-j}((j+1)\alpha z)(-\lambda)^{j}. \nonumber
\end{equation}
Therefore, we obtain the explicit expression of the \(  p_q(\lambda; \mathbf{e}_1 \mathbf{e}_1 ^{\mathrm{T}} + z \mathbf{FS}(\alpha)) \).
\end{proof}
Based on Lemmas \ref{lem:same_characteristic_polynomial} and \ref{lem:characteristic_polynomial_evaluation}, it is not difficult to conclude that
\begin{equation}
p_q(\lambda;\mathbf{A}+z\mathbf{B}) = p_q(\lambda;\mathbf{A}+z\mathbf{B}(\alpha)/\alpha) = f_q( \lambda; z) + f _{q-1}( \lambda; z) - \gamma_{q}(-z/\alpha)\delta_{q}, \nonumber
\end{equation}
where \( f_q( \lambda; z) :=  \sum_{j=0}^{q} \gamma_{q-j}((j+1)z)(-\lambda)^{j} \).
\par 
To intuitively demonstrate the accuracy of the above derivation, we can plot the stability region using the original matrix \( \mathbf{A} + z \mathbf{B}(\alpha)/\alpha \), the equivalent matrix \( \mathbf{e}_1 \mathbf{e}_1 ^{\mathrm{T}} + z \mathbf{FS}(\alpha))/\alpha \), and the characteristic polynomial in Figure \ref{fig:uniform_stability_area}.
\begin{figure}[t]
\centering
\includegraphics[scale=.4]{fig/stability_area_fig1.eps}
\includegraphics[scale=.4]{fig/stability_area_fig2.eps}
\includegraphics[scale=.4]{fig/stability_area_fig3.eps}
\caption{The stability domain described by the spectral radius of the original matrix $\mathbf{A} + z \mathbf{B}(\alpha)/\alpha = \mathbf{A} + z \mathbf{S}(\alpha)\mathbf{F}/\alpha$ (Left), the spectral radius of the equivalent matrix $\mathbf{e}_{1}\mathbf{e}_{1}^{\mathrm{T}} + z \mathbf{F}\mathbf{S}(\alpha)/\alpha$ (Middle) and the maximum module of the zeros of the derived characteristic polynomial (Right). The circle centered at the origin with a radius of \( 1/e \) marked with red line and the contour of stability region for $q = 2,\ldots, 16$ with blue one. }
\label{fig:uniform_stability_area}
\end{figure}


\newpage
\bibliography{ref_AB.bib}
\bibliographystyle{plain}
\end{document}  