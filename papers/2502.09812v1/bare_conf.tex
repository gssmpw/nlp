
%% bare_conf.tex
%% V1.4b
%% 2015/08/26
%% by Michael Shell
%% See:
%% http://www.michaelshell.org/
%% for current contact information.
%%
%% This is a skeleton file demonstrating the use of IEEEtran.cls
%% (requires IEEEtran.cls version 1.8b or later) with an IEEE
%% conference paper.
%%
%% Support sites:
%% http://www.michaelshell.org/tex/ieeetran/
%% http://www.ctan.org/pkg/ieeetran
%% and
%% http://www.ieee.org/

%%*************************************************************************
%% Legal Notice:
%% This code is offered as-is without any warranty either expressed or
%% implied; without even the implied warranty of MERCHANTABILITY or
%% FITNESS FOR A PARTICULAR PURPOSE! 
%% User assumes all risk.
%% In no event shall the IEEE or any contributor to this code be liable for
%% any damages or losses, including, but not limited to, incidental,
%% consequential, or any other damages, resulting from the use or misuse
%% of any information contained here.
%%
%% All comments are the opinions of their respective authors and are not
%% necessarily endorsed by the IEEE.
%%
%% This work is distributed under the LaTeX Project Public License (LPPL)
%% ( http://www.latex-project.org/ ) version 1.3, and may be freely used,
%% distributed and modified. A copy of the LPPL, version 1.3, is included
%% in the base LaTeX documentation of all distributions of LaTeX released
%% 2003/12/01 or later.
%% Retain all contribution notices and credits.
%% ** Modified files should be clearly indicated as such, including  **
%% ** renaming them and changing author support contact information. **
%%*************************************************************************


% *** Authors should verify (and, if needed, correct) their LaTeX system  ***
% *** with the testflow diagnostic prior to trusting their LaTeX platform ***
% *** with production work. The IEEE's font choices and paper sizes can   ***
% *** trigger bugs that do not appear when using other class files.       ***                          ***
% The testflow support page is at:
% http://www.michaelshell.org/tex/testflow/



\documentclass[conference]{IEEEtran}
% Some Computer Society conferences also require the compsoc mode option,
% but others use the standard conference format.
%
% If IEEEtran.cls has not been installed into the LaTeX system files,
% manually specify the path to it like:
% \documentclass[conference]{../sty/IEEEtran}

\usepackage{multirow}
\usepackage{amsmath}
\usepackage{amssymb}
\usepackage{xcolor}
\usepackage{tikz}
\usepackage{makecell}
\usepackage{lscape}
\usepackage{pifont}
\newcommand{\cmark}{\ding{51}}%
\newcommand{\xmark}{\ding{55}}%
\usepackage[graphicx]{realboxes}
\usepackage{adjustbox}



\newcommand{\specialcell}[2][c]{%
  \begin{tabular}[#1]{@{}c@{}}#2\end{tabular}}
  
\def\checkmark{\tikz\fill[scale=0.4](0,.35) -- (.25,0) -- (1,.7) -- (.25,.15) -- cycle;}
\newcommand{\crossmark}{%
\tikz[scale=0.23] {
    \draw[line width=0.7,line cap=round] (0,0) to [bend left=6] (1,1);
    \draw[line width=0.7,line cap=round] (0.2,0.95) to [bend right=3] (0.8,0.05);
}}
\usepackage{hyperref}

\usepackage{array}
\newcolumntype{P}[1]{>{\centering\arraybackslash}p{#1}}
\setlength\extrarowheight{3pt}
\newcolumntype{C}[1]{>{\centering\arraybackslash}p{#1}}

% Some very useful LaTeX packages include:
% (uncomment the ones you want to load)


% *** MISC UTILITY PACKAGES ***
%
%\usepackage{ifpdf}
% Heiko Oberdiek's ifpdf.sty is very useful if you need conditional
% compilation based on whether the output is pdf or dvi.
% usage:
% \ifpdf
%   % pdf code
% \else
%   % dvi code
% \fi
% The latest version of ifpdf.sty can be obtained from:
% http://www.ctan.org/pkg/ifpdf
% Also, note that IEEEtran.cls V1.7 and later provides a builtin
% \ifCLASSINFOpdf conditional that works the same way.
% When switching from latex to pdflatex and vice-versa, the compiler may
% have to be run twice to clear warning/error messages.






% *** CITATION PACKAGES ***
%
%\usepackage{cite}
% cite.sty was written by Donald Arseneau
% V1.6 and later of IEEEtran pre-defines the format of the cite.sty package
% \cite{} output to follow that of the IEEE. Loading the cite package will
% result in citation numbers being automatically sorted and properly
% "compressed/ranged". e.g., [1], [9], [2], [7], [5], [6] without using
% cite.sty will become [1], [2], [5]--[7], [9] using cite.sty. cite.sty's
% \cite will automatically add leading space, if needed. Use cite.sty's
% noadjust option (cite.sty V3.8 and later) if you want to turn this off
% such as if a citation ever needs to be enclosed in parenthesis.
% cite.sty is already installed on most LaTeX systems. Be sure and use
% version 5.0 (2009-03-20) and later if using hyperref.sty.
% The latest version can be obtained at:
% http://www.ctan.org/pkg/cite
% The documentation is contained in the cite.sty file itself.






% *** GRAPHICS RELATED PACKAGES ***
%
\ifCLASSINFOpdf
  % \usepackage[pdftex]{graphicx}
  % declare the path(s) where your graphic files are
  % \graphicspath{{../pdf/}{../jpeg/}}
  % and their extensions so you won't have to specify these with
  % every instance of \includegraphics
  % \DeclareGraphicsExtensions{.pdf,.jpeg,.png}
\else
  % or other class option (dvipsone, dvipdf, if not using dvips). graphicx
  % will default to the driver specified in the system graphics.cfg if no
  % driver is specified.
  % \usepackage[dvips]{graphicx}
  % declare the path(s) where your graphic files are
  % \graphicspath{{../eps/}}
  % and their extensions so you won't have to specify these with
  % every instance of \includegraphics
  % \DeclareGraphicsExtensions{.eps}
\fi
% graphicx was written by David Carlisle and Sebastian Rahtz. It is
% required if you want graphics, photos, etc. graphicx.sty is already
% installed on most LaTeX systems. The latest version and documentation
% can be obtained at: 
% http://www.ctan.org/pkg/graphicx
% Another good source of documentation is "Using Imported Graphics in
% LaTeX2e" by Keith Reckdahl which can be found at:
% http://www.ctan.org/pkg/epslatex
%
% latex, and pdflatex in dvi mode, support graphics in encapsulated
% postscript (.eps) format. pdflatex in pdf mode supports graphics
% in .pdf, .jpeg, .png and .mps (metapost) formats. Users should ensure
% that all non-photo figures use a vector format (.eps, .pdf, .mps) and
% not a bitmapped formats (.jpeg, .png). The IEEE frowns on bitmapped formats
% which can result in "jaggedy"/blurry rendering of lines and letters as
% well as large increases in file sizes.
%
% You can find documentation about the pdfTeX application at:
% http://www.tug.org/applications/pdftex





% *** MATH PACKAGES ***
%
%\usepackage{amsmath}
% A popular package from the American Mathematical Society that provides
% many useful and powerful commands for dealing with mathematics.
%
% Note that the amsmath package sets \interdisplaylinepenalty to 10000
% thus preventing page breaks from occurring within multiline equations. Use:
%\interdisplaylinepenalty=2500
% after loading amsmath to restore such page breaks as IEEEtran.cls normally
% does. amsmath.sty is already installed on most LaTeX systems. The latest
% version and documentation can be obtained at:
% http://www.ctan.org/pkg/amsmath





% *** SPECIALIZED LIST PACKAGES ***
%
%\usepackage{algorithmic}
% algorithmic.sty was written by Peter Williams and Rogerio Brito.
% This package provides an algorithmic environment fo describing algorithms.
% You can use the algorithmic environment in-text or within a figure
% environment to provide for a floating algorithm. Do NOT use the algorithm
% floating environment provided by algorithm.sty (by the same authors) or
% algorithm2e.sty (by Christophe Fiorio) as the IEEE does not use dedicated
% algorithm float types and packages that provide these will not provide
% correct IEEE style captions. The latest version and documentation of
% algorithmic.sty can be obtained at:
% http://www.ctan.org/pkg/algorithms
% Also of interest may be the (relatively newer and more customizable)
% algorithmicx.sty package by Szasz Janos:
% http://www.ctan.org/pkg/algorithmicx




% *** ALIGNMENT PACKAGES ***
%
%\usepackage{array}
% Frank Mittelbach's and David Carlisle's array.sty patches and improves
% the standard LaTeX2e array and tabular environments to provide better
% appearance and additional user controls. As the default LaTeX2e table
% generation code is lacking to the point of almost being broken with
% respect to the quality of the end results, all users are strongly
% advised to use an enhanced (at the very least that provided by array.sty)
% set of table tools. array.sty is already installed on most systems. The
% latest version and documentation can be obtained at:
% http://www.ctan.org/pkg/array


% IEEEtran contains the IEEEeqnarray family of commands that can be used to
% generate multiline equations as well as matrices, tables, etc., of high
% quality.




% *** SUBFIGURE PACKAGES ***
%\ifCLASSOPTIONcompsoc
%  \usepackage[caption=false,font=normalsize,labelfont=sf,textfont=sf]{subfig}
%\else
%  \usepackage[caption=false,font=footnotesize]{subfig}
%\fi
% subfig.sty, written by Steven Douglas Cochran, is the modern replacement
% for subfigure.sty, the latter of which is no longer maintained and is
% incompatible with some LaTeX packages including fixltx2e. However,
% subfig.sty requires and automatically loads Axel Sommerfeldt's caption.sty
% which will override IEEEtran.cls' handling of captions and this will result
% in non-IEEE style figure/table captions. To prevent this problem, be sure
% and invoke subfig.sty's "caption=false" package option (available since
% subfig.sty version 1.3, 2005/06/28) as this is will preserve IEEEtran.cls
% handling of captions.
% Note that the Computer Society format requires a larger sans serif font
% than the serif footnote size font used in traditional IEEE formatting
% and thus the need to invoke different subfig.sty package options depending
% on whether compsoc mode has been enabled.
%
% The latest version and documentation of subfig.sty can be obtained at:
% http://www.ctan.org/pkg/subfig




% *** FLOAT PACKAGES ***
%
%\usepackage{fixltx2e}
% fixltx2e, the successor to the earlier fix2col.sty, was written by
% Frank Mittelbach and David Carlisle. This package corrects a few problems
% in the LaTeX2e kernel, the most notable of which is that in current
% LaTeX2e releases, the ordering of single and double column floats is not
% guaranteed to be preserved. Thus, an unpatched LaTeX2e can allow a
% single column figure to be placed prior to an earlier double column
% figure.
% Be aware that LaTeX2e kernels dated 2015 and later have fixltx2e.sty's
% corrections already built into the system in which case a warning will
% be issued if an attempt is made to load fixltx2e.sty as it is no longer
% needed.
% The latest version and documentation can be found at:
% http://www.ctan.org/pkg/fixltx2e


%\usepackage{stfloats}
% stfloats.sty was written by Sigitas Tolusis. This package gives LaTeX2e
% the ability to do double column floats at the bottom of the page as well
% as the top. (e.g., "\begin{figure*}[!b]" is not normally possible in
% LaTeX2e). It also provides a command:
%\fnbelowfloat
% to enable the placement of footnotes below bottom floats (the standard
% LaTeX2e kernel puts them above bottom floats). This is an invasive package
% which rewrites many portions of the LaTeX2e float routines. It may not work
% with other packages that modify the LaTeX2e float routines. The latest
% version and documentation can be obtained at:
% http://www.ctan.org/pkg/stfloats
% Do not use the stfloats baselinefloat ability as the IEEE does not allow
% \baselineskip to stretch. Authors submitting work to the IEEE should note
% that the IEEE rarely uses double column equations and that authors should try
% to avoid such use. Do not be tempted to use the cuted.sty or midfloat.sty
% packages (also by Sigitas Tolusis) as the IEEE does not format its papers in
% such ways.
% Do not attempt to use stfloats with fixltx2e as they are incompatible.
% Instead, use Morten Hogholm'a dblfloatfix which combines the features
% of both fixltx2e and stfloats:
%
% \usepackage{dblfloatfix}
% The latest version can be found at:
% http://www.ctan.org/pkg/dblfloatfix




% *** PDF, URL AND HYPERLINK PACKAGES ***
%
%\usepackage{url}
% url.sty was written by Donald Arseneau. It provides better support for
% handling and breaking URLs. url.sty is already installed on most LaTeX
% systems. The latest version and documentation can be obtained at:
% http://www.ctan.org/pkg/url
% Basically, \url{my_url_here}.




% *** Do not adjust lengths that control margins, column widths, etc. ***
% *** Do not use packages that alter fonts (such as pslatex).         ***
% There should be no need to do such things with IEEEtran.cls V1.6 and later.
% (Unless specifically asked to do so by the journal or conference you plan
% to submit to, of course. )


% correct bad hyphenation here
\hyphenation{op-tical net-works semi-conduc-tor}
\usepackage{comment}
\usepackage{amsmath}
\usepackage{amsfonts}
\usepackage{graphicx}
\usepackage{caption}
\usepackage{marvosym}
\begin{document}
%
% paper title
% Titles are generally capitalized except for words such as a, an, and, as,
% at, but, by, for, in, nor, of, on, or, the, to and up, which are usually
% not capitalized unless they are the first or last word of the title.
% Linebreaks \\ can be used within to get better formatting as desired.
% Do not put math or special symbols in the title.

\title{Face Deepfakes - A Comprehensive Review}

%\title{Audio deepfakes: Taxonomy, Generation, Detection, Multimodality, Implications, and way forward}

\author{\IEEEauthorblockN{Tharindu Fernando\IEEEauthorrefmark{1}, Darshana Priyasad\IEEEauthorrefmark{1},
Sridha Sridharan\IEEEauthorrefmark{1},  Arun Ross\IEEEauthorrefmark{2}, and
Clinton Fookes\IEEEauthorrefmark{1}.}\\

\IEEEauthorblockA{\IEEEauthorrefmark{1}Signal Processing, Artificial Intelligence \& Vision Technologies, Queensland University of Technology, Australia.
}\\

\IEEEauthorblockA{\IEEEauthorrefmark{2}Michigan State University, Department of Computer Science and Engineering, United States.}\\
\thanks{Corresponding author: T. Fernando (email: t.warnakulasuriya@qut.edu.au).}}

% author names and affiliations
% use a multiple column layout for up to three different
% affiliations
% \author{\IEEEauthorblockN{Michael Shell}
% \IEEEauthorblockA{School of Electrical and\\Computer Engineering\\
% Georgia Institute of Technology\\
% Atlanta, Georgia 30332--0250\\
% Email: http://www.michaelshell.org/contact.html}
% \and
% \IEEEauthorblockN{Homer Simpson}
% \IEEEauthorblockA{Twentieth Century Fox\\
% Springfield, USA\\
% Email: homer@thesimpsons.com}
% \and
% \IEEEauthorblockN{James Kirk\\ and Montgomery Scott}
% \IEEEauthorblockA{Starfleet Academy\\
% San Francisco, California 96678--2391\\
% Telephone: (800) 555--1212\\
% Fax: (888) 555--1212}}

% conference papers do not typically use \thanks and this command
% is locked out in conference mode. If really needed, such as for
% the acknowledgment of grants, issue a \IEEEoverridecommandlockouts
% after \documentclass

% for over three affiliations, or if they all won't fit within the width
% of the page, use this alternative format:
% 
%\author{\IEEEauthorblockN{Michael Shell\IEEEauthorrefmark{1},
%Homer Simpson\IEEEauthorrefmark{2},
%James Kirk\IEEEauthorrefmark{3}, 
%Montgomery Scott\IEEEauthorrefmark{3} and
%Eldon Tyrell\IEEEauthorrefmark{4}}
%\IEEEauthorblockA{\IEEEauthorrefmark{1}School of Electrical and Computer Engineering\\
%Georgia Institute of Technology,
%Atlanta, Georgia 30332--0250\\ Email: see http://www.michaelshell.org/contact.html}
%\IEEEauthorblockA{\IEEEauthorrefmark{2}Twentieth Century Fox, Springfield, USA\\
%Email: homer@thesimpsons.com}
%\IEEEauthorblockA{\IEEEauthorrefmark{3}Starfleet Academy, San Francisco, California 96678-2391\\
%Telephone: (800) 555--1212, Fax: (888) 555--1212}
%\IEEEauthorblockA{\IEEEauthorrefmark{4}Tyrell Inc., 123 Replicant Street, Los Angeles, California 90210--4321}}




% use for special paper notices
%\IEEEspecialpapernotice{(Invited Paper)}




% make the title area
\maketitle

% As a general rule, do not put math, special symbols or citations
% in the abstract
\begin{abstract}
In recent years, remarkable advancements in deepfake generation technology have led to unprecedented leaps in its realism and capabilities. Despite these advances, we observe a notable lack of structured and deep analysis deepfake technology. The principal aim of this survey is to contribute a
thorough theoretical analysis of state-of-the-art face deepfake generation and detection methods. Furthermore, we provide a coherent and systematic evaluation of the implications of deepfakes on face biometric recognition approaches. In addition, we outline key applications of face deepfake technology, elucidating both positive and negative applications of the technology, provide a detailed discussion regarding the gaps in existing research, and propose key research directions for further investigation.
\end{abstract}

% no keywords




% For peer review papers, you can put extra information on the cover
% page as needed:
% \ifCLASSOPTIONpeerreview
% \begin{center} \bfseries EDICS Category: 3-BBND \end{center}
% \fi
%
% For peerreview papers, this IEEEtran command inserts a page break and
% creates the second title. It will be ignored for other modes.
\IEEEpeerreviewmaketitle


\section{Introduction}
\label{sec:introduction}
The business processes of organizations are experiencing ever-increasing complexity due to the large amount of data, high number of users, and high-tech devices involved \cite{martin2021pmopportunitieschallenges, beerepoot2023biggestbpmproblems}. This complexity may cause business processes to deviate from normal control flow due to unforeseen and disruptive anomalies \cite{adams2023proceddsriftdetection}. These control-flow anomalies manifest as unknown, skipped, and wrongly-ordered activities in the traces of event logs monitored from the execution of business processes \cite{ko2023adsystematicreview}. For the sake of clarity, let us consider an illustrative example of such anomalies. Figure \ref{FP_ANOMALIES} shows a so-called event log footprint, which captures the control flow relations of four activities of a hypothetical event log. In particular, this footprint captures the control-flow relations between activities \texttt{a}, \texttt{b}, \texttt{c} and \texttt{d}. These are the causal ($\rightarrow$) relation, concurrent ($\parallel$) relation, and other ($\#$) relations such as exclusivity or non-local dependency \cite{aalst2022pmhandbook}. In addition, on the right are six traces, of which five exhibit skipped, wrongly-ordered and unknown control-flow anomalies. For example, $\langle$\texttt{a b d}$\rangle$ has a skipped activity, which is \texttt{c}. Because of this skipped activity, the control-flow relation \texttt{b}$\,\#\,$\texttt{d} is violated, since \texttt{d} directly follows \texttt{b} in the anomalous trace.
\begin{figure}[!t]
\centering
\includegraphics[width=0.9\columnwidth]{images/FP_ANOMALIES.png}
\caption{An example event log footprint with six traces, of which five exhibit control-flow anomalies.}
\label{FP_ANOMALIES}
\end{figure}

\subsection{Control-flow anomaly detection}
Control-flow anomaly detection techniques aim to characterize the normal control flow from event logs and verify whether these deviations occur in new event logs \cite{ko2023adsystematicreview}. To develop control-flow anomaly detection techniques, \revision{process mining} has seen widespread adoption owing to process discovery and \revision{conformance checking}. On the one hand, process discovery is a set of algorithms that encode control-flow relations as a set of model elements and constraints according to a given modeling formalism \cite{aalst2022pmhandbook}; hereafter, we refer to the Petri net, a widespread modeling formalism. On the other hand, \revision{conformance checking} is an explainable set of algorithms that allows linking any deviations with the reference Petri net and providing the fitness measure, namely a measure of how much the Petri net fits the new event log \cite{aalst2022pmhandbook}. Many control-flow anomaly detection techniques based on \revision{conformance checking} (hereafter, \revision{conformance checking}-based techniques) use the fitness measure to determine whether an event log is anomalous \cite{bezerra2009pmad, bezerra2013adlogspais, myers2018icsadpm, pecchia2020applicationfailuresanalysispm}. 

The scientific literature also includes many \revision{conformance checking}-independent techniques for control-flow anomaly detection that combine specific types of trace encodings with machine/deep learning \cite{ko2023adsystematicreview, tavares2023pmtraceencoding}. Whereas these techniques are very effective, their explainability is challenging due to both the type of trace encoding employed and the machine/deep learning model used \cite{rawal2022trustworthyaiadvances,li2023explainablead}. Hence, in the following, we focus on the shortcomings of \revision{conformance checking}-based techniques to investigate whether it is possible to support the development of competitive control-flow anomaly detection techniques while maintaining the explainable nature of \revision{conformance checking}.
\begin{figure}[!t]
\centering
\includegraphics[width=\columnwidth]{images/HIGH_LEVEL_VIEW.png}
\caption{A high-level view of the proposed framework for combining \revision{process mining}-based feature extraction with dimensionality reduction for control-flow anomaly detection.}
\label{HIGH_LEVEL_VIEW}
\end{figure}

\subsection{Shortcomings of \revision{conformance checking}-based techniques}
Unfortunately, the detection effectiveness of \revision{conformance checking}-based techniques is affected by noisy data and low-quality Petri nets, which may be due to human errors in the modeling process or representational bias of process discovery algorithms \cite{bezerra2013adlogspais, pecchia2020applicationfailuresanalysispm, aalst2016pm}. Specifically, on the one hand, noisy data may introduce infrequent and deceptive control-flow relations that may result in inconsistent fitness measures, whereas, on the other hand, checking event logs against a low-quality Petri net could lead to an unreliable distribution of fitness measures. Nonetheless, such Petri nets can still be used as references to obtain insightful information for \revision{process mining}-based feature extraction, supporting the development of competitive and explainable \revision{conformance checking}-based techniques for control-flow anomaly detection despite the problems above. For example, a few works outline that token-based \revision{conformance checking} can be used for \revision{process mining}-based feature extraction to build tabular data and develop effective \revision{conformance checking}-based techniques for control-flow anomaly detection \cite{singh2022lapmsh, debenedictis2023dtadiiot}. However, to the best of our knowledge, the scientific literature lacks a structured proposal for \revision{process mining}-based feature extraction using the state-of-the-art \revision{conformance checking} variant, namely alignment-based \revision{conformance checking}.

\subsection{Contributions}
We propose a novel \revision{process mining}-based feature extraction approach with alignment-based \revision{conformance checking}. This variant aligns the deviating control flow with a reference Petri net; the resulting alignment can be inspected to extract additional statistics such as the number of times a given activity caused mismatches \cite{aalst2022pmhandbook}. We integrate this approach into a flexible and explainable framework for developing techniques for control-flow anomaly detection. The framework combines \revision{process mining}-based feature extraction and dimensionality reduction to handle high-dimensional feature sets, achieve detection effectiveness, and support explainability. Notably, in addition to our proposed \revision{process mining}-based feature extraction approach, the framework allows employing other approaches, enabling a fair comparison of multiple \revision{conformance checking}-based and \revision{conformance checking}-independent techniques for control-flow anomaly detection. Figure \ref{HIGH_LEVEL_VIEW} shows a high-level view of the framework. Business processes are monitored, and event logs obtained from the database of information systems. Subsequently, \revision{process mining}-based feature extraction is applied to these event logs and tabular data input to dimensionality reduction to identify control-flow anomalies. We apply several \revision{conformance checking}-based and \revision{conformance checking}-independent framework techniques to publicly available datasets, simulated data of a case study from railways, and real-world data of a case study from healthcare. We show that the framework techniques implementing our approach outperform the baseline \revision{conformance checking}-based techniques while maintaining the explainable nature of \revision{conformance checking}.

In summary, the contributions of this paper are as follows.
\begin{itemize}
    \item{
        A novel \revision{process mining}-based feature extraction approach to support the development of competitive and explainable \revision{conformance checking}-based techniques for control-flow anomaly detection.
    }
    \item{
        A flexible and explainable framework for developing techniques for control-flow anomaly detection using \revision{process mining}-based feature extraction and dimensionality reduction.
    }
    \item{
        Application to synthetic and real-world datasets of several \revision{conformance checking}-based and \revision{conformance checking}-independent framework techniques, evaluating their detection effectiveness and explainability.
    }
\end{itemize}

The rest of the paper is organized as follows.
\begin{itemize}
    \item Section \ref{sec:related_work} reviews the existing techniques for control-flow anomaly detection, categorizing them into \revision{conformance checking}-based and \revision{conformance checking}-independent techniques.
    \item Section \ref{sec:abccfe} provides the preliminaries of \revision{process mining} to establish the notation used throughout the paper, and delves into the details of the proposed \revision{process mining}-based feature extraction approach with alignment-based \revision{conformance checking}.
    \item Section \ref{sec:framework} describes the framework for developing \revision{conformance checking}-based and \revision{conformance checking}-independent techniques for control-flow anomaly detection that combine \revision{process mining}-based feature extraction and dimensionality reduction.
    \item Section \ref{sec:evaluation} presents the experiments conducted with multiple framework and baseline techniques using data from publicly available datasets and case studies.
    \item Section \ref{sec:conclusions} draws the conclusions and presents future work.
\end{itemize}
%\section{Voice Deepfakes}\label{sec:voice_deepfakes}
%\textcolor{red}{RESPONSIBILITY - Ivan \& Darshana}

\subsection{Generation}\label{sec:voice_generation}

Voice deepfake or synthetic voice is a voice that closely resembles a real person’s voice. It refers to any voice that is generated using artificial intelligence (AI) technology and can accurately capture tonality, accents, cadence, and other unique characteristics of a real person’s voice. Fake voices can be used to create speech sentences that sound like real people saying things they did not say. Human voice synthesis can be broadly classified into text-to-speech (TTS) and voice conversion (VC). Figure~\ref{fig:voice_deepfakes} shows two popular pipelines for creating voice deepfakes.

\subsubsection{Voice Deepfake Types}

\begin{figure*}
    \centering
    \includegraphics[width=\linewidth]{figures_new/voice_deepfakes.pdf}
    \caption{Two popular methods for creating voice deepfakes: (a) Text-to-speech and (b) Voice conversion. [REDONE]}
    \label{fig:voice_deepfakes}
\end{figure*}

\paragraph{Text-to-Speech (TTS)}

A text-to-speech (TTS) system aims to create speech from a provided input text. Thanks to recent advances in deep learning and artificial intelligence, TTS systems can now produce realistic and natural-sounding speech using any speaker’s voice. Generally, modern TTS systems consist of three basic components \cite{surveytts}: a text analysis module, an acoustic model, and a vocoder. When the user inputs text into the TTS system, a text analysis module processes the text and extracts linguistic features. The acoustic model then generates acoustic features based on the linguistic features. Finally, the vocoder synthesizes speech from the predicted acoustic features. To better capture the characteristics of voices from various speakers, speaker embeddings extracted from the speaker encoder network are commonly used in the acoustic modelling process \cite{jia2018transfer}. In recent years, a transformer with a self-attention mechanism \cite{transformer} has gained widespread popularity as an architecture for acoustic models in neural TTS. Moreover, fully end-to-end models like FastSpeech 2s \cite{ren2021fastspeech}, VITS (Conditional Variational Autoencoder with Adversarial Learning for End-to-End Text-to-Speech) \cite {vits}, and NaturalSpeech \cite{naturalspeech} have been specifically designed to directly generate speech waveforms from character or phoneme sequences, aiming to enhance optimization and reduce development time for various TTS components.

In recent times, Large Language Models (LLMs) have been introduced to improve the quality and speaker adaptation of TTS models even further. LLMs are fundamental machine learning models that are trained with massive amounts of text to understand, predict, and generate human language \cite {surveyllm}. 
For instance, several studies on TTS have utilized Bidirectional Encoder Representations from Transformers (BERT) embeddings to generate speech with improved prosody prediction, better pronunciation and expressiveness \cite{chen2021, kenter2020, xiao2020, xu2021}. Since BERT or BERT-like models learn contextual information from a large-scale text corpus, they can provide syntactic and semantic information to help in predicting the fundamental frequency \cite{yasuda2022}, pauses \cite{xiao2020}, and other prosodic characteristics \cite{moya2023a} of natural speech. The representations from large-scale Self-Supervised Learning (SSL) models such as HuBERT \cite{hsu2021hubert}, WavLM \cite{chen2022wavlm}, and Wav2vec 2.0 \cite{baevski2020wav2vec} are highly beneficial for speech generation tasks. They can be easily adapted to various non-automatic speech recognition tasks, including speaker identification, spoken language understanding, and speech emotion recognition \cite{ftwav2vec}. In the field of TTS, SSL features have been explored in prosody modelling \cite{zhang2023}, improving pronunciation accuracy in cross-lingual TTS \cite{cong2022}, noise-robust synthesis \cite{Siuzdak2022}, and zero-shot TTS \cite{hierspeech,fujita2023}. To achieve superior performance, recent TTS systems such as $\text{M}^{2}$-CTTS \cite{xue2023}, WavThruVec \cite{siuzdak22} and HierSpeech++ \cite{hierspeech} have used SSL representations as an additional semantic representation. StyleTTS2 \cite{styletts2} has also benefited from the SSL model through adversarial training.

The development of Generative Pre-trained Transformers (GPT) technology \cite{gpt} has inspired recent efforts to harness the capabilities of GPT-like auto-regressive models for building large-scale TTS systems. For instance, VALL-E \cite{valle} is the initial neural codec language model designed based on the architecture of GPT-3, featuring robust in-context learning abilities. VALL-E generates the discrete audio codec from phonemes representing the target content and acoustic code prompts extracted from the speaker's voice. It also can produce high-quality personalized voices and emotions based on short speech prompts. Its multilingual variant, VALL-E X \cite{vallex} introduces language ID for synthesizing cross-lingual speech, which supports speech-to-speech translation. LauraGPT (Listen, Attend, Understand, and Regenerate Audio with GPT) \cite{lauratts} adopts a decoder-only Transformer as a GPT backbone to perform unified audio and text modelling and can generate outputs in either modality. Furthermore, Cosyvoice \cite{cosyvoice} and FireRedTTS \cite{fireredtts} introduce semantic speech tokens for TTS modelling and formulate TTS as a next-token prediction task with text as prompts using the GPT-like decoder transformer.

While the auto-regressive model-based architecture has achieved promising results, inference latency is higher compared to non-autoregressive models because the codec token needs to be sampled sequentially for speech generation. Using the text-guided speech-infilling task \cite{le2023voicebox},  E2-TTS \cite{e2tts} and F5-TTS \cite{f5tts} predict masked speech given its surrounding audio and the text transcript in a non-autoregressive manner. During training, the text input is padded with filler tokens of the same length as input speech, simplifying the TTS systems without complex designs such as duration model, text encoder, and phone alignment.  

In voice generation scenarios, providing a text description to describe the intended voice attributes such as speaker identity, tone, speaking rate, and emotion, would be more user-friendly and offer greater voice customization \cite{guo2023, instruct_tts, prompt_tts2}. One notable example is Parler-TTS \cite{lacombe-etal-2024-parler-tts} which utilizes the MusicGen \textcolor{red}{\cite{copet2024simple}} architecture conditioned on embeddings extracted from the Flan-T5 text encoder through a cross-attention mechanism to control different speaker attributes using natural language descriptions \cite{lyth2024natural}.

\paragraph{Voice Conversion (VC)}

Voice Conversion (VC) is the process of transforming a source speaker's voice to sound like that of another speaker without changing the linguistic content. This change alters the speaker identity of the converted speech \cite{hamid2017}. There is a growing demand for various practical applications of VC technology, which can benefit consumers in many ways, such as personalized text-to-speech, voice dubbing, voice mimicry, and speaking aids for speech-impaired persons \cite{kain2007, biadsy2019}. However, VC can also be misused by malicious actors to mimic voices and manipulate what we hear for their advantage.

The VC system pipeline typically consists of three main phases \cite{Sisman2020}: (1) speech analysis and feature extraction phase where the speech from a source speaker is converted into features containing linguistic contents and disentangled from its speaker's characteristics, (2) feature mapping phase where the decomposed information is transferred into intermediate features that match the target speaker's characteristics, and (3) reconstruction phase where the modified intermediate features are transformed back into a speech waveform using a vocoder.
In the parallel voice conversion (VC) technique, the VC system is trained to learn a mapping function from the source speech to the target speech using a training set that contains parallel data. This means that the training set includes speech data of the same linguistic content from both the source and the target speakers \cite{Sisman2020}. However, the requirement for parallel data has often hindered the progress of parallel VC techniques because it is usually assumed that there is a limited amount of data available from both the source and target speakers.  In recent years, deep learning techniques have become increasingly effective in solving conversion problems using a mapping function with powerful regression capabilities, without the need for a parallel corpus (non-parallel VC). VC frameworks such as CyleGAN-VC \cite{cycleganvc}, StarGAN-VC \cite{starganvc}, and VAW-GAN \cite{vawgan} have been developed by incorporating generative adversarial networks to map various voice attributes. These techniques can achieve better voice quality and similarity to the target speaker when a large number of training examples are available.

The use of linguistic-related features to capture spoken content has been a popular approach for non-parallel VC \cite{flytek2018, tian2019}. These features can be derived from an Automatic Speech Recognition (ASR) system, such as bottleneck features or Phonetic PosteriorGrams (PPGs) \cite{ppg2016}. A PPG is a matrix that represents the posterior probabilities of the phonetic class corresponding to each time frame. It is common to employ speaker-independent ASR (SI-ASR) trained using a large multi-speaker corpus to obtain PPGs to accommodate for differences in speakers. Recently, large-scale self-supervised speech representations such as wav2vec 2.0, vq-wav2vec \cite{baevski2019vq}, HuBERT and WavLM have been used to replace the PPGs in voice conversion tasks \cite{polyak2021, huang2022, li2023freevc}. For instance, Free-VC \cite{li2023freevc}, an end-to-end model based on the VITS architecture designed for voice conversion, has used WavLM to encode linguistic content. Speaker information is acquired from a speaker encoder trained jointly with the rest of the components from scratch. Additionally, SR-based data augmentation is employed to enhance the quality of waveform generation.

Although continuous features extracted from self-supervised pre-trained models can capture linguistic, speaker, prosodic, and semantic information of speech, it remains a challenge to completely separate the content from the speaker identity, resulting in reduced speaker similarity of the converted speech. There are multiple methods in the literature that can be applied to SSL continuous features such as discretization \cite{polyak2021, softvc}, and speaker disentanglement mechanisms \cite{contentvec}. For example, ContentVec \cite{contentvec} introduces disentanglement modules in the mask prediction task of HuBERT without significant loss of content information. The content encoder in \cite{softvc} produces soft speech units, called HuBERT-Soft, which are learned by predicting a distribution over the discrete units extracted from HuBERT to achieve speaker disentanglement. These HuBERT-Soft features have been widely employed as content features in voice conversion tasks \cite{quickvc, rythmvc, streamvc}, including singing voice conversion \cite{singingvc, diffsvc}, and have notably improved speaker similarity.

%The method in~\cite{softvc} is proposed to apply k-means on the continuous features to generate discrete speech units as a mean to improve the speaker similarity.

\subsubsection{Deepfake Generation Process} 

\paragraph{Autoregressive Models}

The autoregressive model predicts the model behaviour of the current step using the data in the previous time steps. For example, in a linear autoregressive model of order $n_i$, the output $y[t]$ can be estimated by:

\begin{equation}
y[t] = \sum^{n_i}_{i=1} a_{i}x[t-i] + e[t]
\end{equation}

where $a_{i}$ is the model parameters, $e[t]$ is white noise or residual error with zero mean. In deep autoregressive models, the model parameters are estimated using deep neural networks. The dependency modelling of an input data sequence in an autoregressive manner allows the complex structure of data to be learned in a temporally more coherent way. However, an autoregressive model suffers from slow generation because it depends on the previous observations to predict the value at the current time step. 

Autoregressive structures have been adopted in many neural TTS systems to achieve state-of-the-art performance such as Tacotron 1/2 \cite{tacotron, tacotron2}, DurIAN \cite{durian}, and DeepVoice 3 \cite{deepvoice3}. One of the first neural-based vocoders for speech synthesis is WaveNet \cite{wavenet}. The autoregressive structure of WaveNet improves the continuity of the generated waveform because it generates one sample at a time.

\paragraph{Variational Autoencoders}

Variational Autoencoders (VAE) consist of an encoder that encodes input data $x$ into a regularized multivariate latent distribution \textcolor{red}{$q(z|x)$}, and a decoder that reconstructs the data as accurately as possible by sampling a point from this distribution \textcolor{red}{$z \sim q(z|x)$}. The decoder and encoder in VAE are trained to minimize the construction error, which also adds a regularization term in the loss function. This term ensures that the latent distribution is regularized to some prior distribution, usually multivariate Gaussian, during training.

In the neural network-based TTS model, the VAE is used to compress the high-dimensional speech $x$ into frame-level representations $z$, where $z$ is sampled from the posterior distribution $q(z|x)$. It is assumed that the prior $p(z)$ is the standard isotropic multivariate Gaussian. Given $q(z|x)$, the speech waveform is reconstructed from $p(x|z)$ using a decoder. In VITS \cite{vits, naturalspeech}, the $z$ is predicted from phoneme sequence $y$, where $z$ is sampled the predicted prior distribution, $p(z|y)$. To enable the speech synthesis, the VAE and the prior prediction are jointly optimized with the loss function consists of a waveform reconstruction loss -log $p(x|z)$ and a Kullback-Leibler divergence loss between the posterior $q(z|x)$ and the prior $p(z|y)$, $KL[q(z|x)||p(z|y)]$.

%The VAE has been used as the key component in many neural network based TTS models.

\paragraph{Generative Adversarial Networks}

Generative Adversarial Networks (GANs) are generative models that create new data instances with the same underlying data distribution of the training samples. GANs generate an N-dimensional vector from an N-dimensional random vector sampled from a simple distribution (standard Gaussian). Because of the complexity of data distribution, GANs formulate this problem into a game between two neural networks that compete against each other to reach a zero-sum Nash equilibrium profile~\cite{Goodfellow2014}. The generator network aims to produce synthetic output based on the target's data distribution to fool the discriminator. Meanwhile, the discriminator network aims to differentiate whether the generated sample is real or fake by comparing the output with the true data.

GAN has been widely used to train a neural vocoder for synthesizing high-fidelity speech audio. Typical non-autoregessive neural vocoders such as~\cite{melgan, hifigan, parallelgan} employ a generator and a sets of discriminators which are trained adversarially using GAN losses to model raw waveforms.

\paragraph{Normalizing Flows}

Normalizing flows employs a sequence of invertible and differentiable mappings, referred as the flow, that transform a simple probability distribution (e.g., standard Normal) into a more complex distribution.

To generate data points $\boldsymbol{x}$ that follow the distribution we want to generate, one can sample $\boldsymbol{z}$ from the base distribution (the “noise”) $p_{\theta}(\boldsymbol{z})$ of a zero mean spherical Gaussian, $p_{\theta}(\boldsymbol{z}) = \mathcal{N}(\boldsymbol{z};0,\boldsymbol{I})$. This process can be written as,
\begin{equation}
   \boldsymbol{z} \sim \mathcal{N}(\boldsymbol{z};0,\boldsymbol{I})
\end{equation}
\begin{equation}
   \boldsymbol{x} = \boldsymbol{f}_{1} \circ \boldsymbol{f}_{2} \circ \ldots \boldsymbol{f}_{K} %(\bm{z})
\end{equation}
where $\boldsymbol{x}$ can be generated by progressively transforming a random variable ($\boldsymbol{z}$) using a set of K invertible  functions, i.e., bijective. The flow-based model is trained by maximizing the log likelihood of the model parameters given the data $\boldsymbol{x}$ based on a change of variable rules,
\begin{equation}
    \log p_{\theta}(\boldsymbol{x}) = \log p_{\theta}(\boldsymbol{z}) + \sum_{i=1}^{K} \log |\det(\boldsymbol{J}(\boldsymbol{f}_{i}^{-1}(\boldsymbol{x})))|
    \end{equation}
\begin{equation}
    \boldsymbol{z} = \boldsymbol{f}_{K}^{-1} \circ \boldsymbol{f}_{k-1}^{-1} \circ \ldots \boldsymbol{f}_{0}^{-1}(\boldsymbol{x})
\end{equation}
where $\boldsymbol{J}$ denotes the Jacobian matrix of $\boldsymbol{f}_{i}^{-1}$, and $\det(.)$ is the the determinant of the matrix.

In a flow-based vocoder such as WaveGlow~\cite{waveglow}, the invertible mapping is constructed using an affine coupling layer~\cite{realnvp}. Each flow operation consists of invertible $1 \times 1$ convolution followed by an affine coupling layer. %Normalizing flow has been adopted for speech synthesis, but the implementation usually require large number of model parameters to improve the quality of synthesis~\cite{}. 
In TTS, Flowtron model~\cite{flowtron} is based on autoregressive flow that generates a sequence of mel spectrogram frames conditioned on text and speaker embedding. Flowtron provides control over speech variation and expressiveness and allows for style transfer.


%Some vocoders such as WaveGlow has been using GLOW ..


\paragraph{Denoising Diffusion Probabilistic Models}

Denoising Diffusion Probabilistic Models (DDPMs) encompass two interconnected processes: a forward process that maps the data distribution to a simple prior distribution, typically a Gaussian, and a reverse process that gradually reverse the effect of forward process using neural networks by simulating Ordinary or Stochastic Differential Equations.

Recently, DDPMs have become popular approach for audio generation including TTS. In TTS tasks, the diffusion model has been applied to the acoustic models and also to the vocoders. For example, the acoustic model in Diff-TTS~\cite{diff-tts} and Grad-TTS~\cite{grad-tts} generate Mel-spectrogram from the text with DDPM. In diffusion-based vocoders, DDPMs turn random noise into a high-fidelity speech waveform through an iterative sampling process after hundreds of iterations. The quality of neural vocoders based on DDPMs has been shown to be comparable with AR models~\cite{wavefit}. WaveGrad is one of the earliest work for conditional model waveform generation by combining the score matching and diffusion probabilistic models~\cite{wavegrad}. Another neural vocoder based on diffusion model is DiffWave~\cite{diffwave}. DiffWave is non-autoregressive model, conditioned on Mel-spectrogram. 

% https://arxiv.org/pdf/2303.13336.pdf

\subsubsection{Performance Evaluation}

% https://www.mdpi.com/2079-9292/9/2/267
% https://www.sciencedirect.com/science/article/pii/S0885230803000676

The quality of synthetic speech is typically evaluated using subjective and objective assessments. Subjective tests involve human listeners rating audio segments based on naturalness and intelligibility~\cite{king2014, electronics20}. Naturalness includes pleasantness, ease of listening, and audio flow, while intelligibility covers listening effort, pronunciation, articulation, comprehension, and speaking rate~\cite{articlemos}. These features are usually evaluated using a mean opinion score (MOS) on a Likert scale from 1 to 5, defined in the recommended standard ITU-T P.85, 1994 for the evaluation of voice output systems~\cite{itu_p85}.

The process of conducting a listening test can be both expensive and time-consuming. It requires a large number of participants who speak the native language or reside in specific geographical areas in order to obtain credible results. In recent years, deep learning models like AutoMOS~\cite{automos}, MOSNet~\cite{mosnet}, and LDNet~\cite{ldnet} have been developed to automatically predict the MOS of synthetic speech, which is highly correlated with human subjective ratings. For the MOS evaluation of multilingual speech synthesis, a Speech Quality Identification (SQuID) system based on multilingual Speech and Language Model (mSLAM) has been designed to predict speech naturalness ratings from 65 different locales~\cite{squid}. Additionally, objective metrics such as Mel-Cepstral distortion (MCD)~\cite{mcd} and word error rate (WER) are often used in combination with subjective tests to measure the quality and intelligibility of synthetic speech.

%The latest work is using regression method: 
%https://arxiv.org/pdf/2210.06324.pdf

\subsubsection{Summary of voice deepfake generation types and methods}

Text-to-speech and voice conversion are the two popular methods for generating fake speech samples. The progress has advanced rapidly with the introduction of an autoregressive GPT-like models, and when self-supervised speech representation learning models are utilised in the training process. Moreover, high-quality samples achieving human parity speech are produced when diffusion probabilistic models are used in the TTS's acoustic modeling. Flow matching~\cite{lipman2023flow}, a recent paradigm in generative modeling that combine aspects from continuous normalizing flows (CNFs) and diffusion models (DMs) to transform between noise and data samples by learning the probability path via ordinary differential equation (ODE) has shown superior modeling performance with stable and faster training. For example, Macha-TTS~\cite{matcha2024}  use optimal-transport conditional flow matching to train an encoder-decoder TTS model. Meanwhile, VoiceFlow~\cite{voiceflow} employs a rectified flow matching in TTS to achieve superior sample quality with a few number of sampling steps. %which is a new way to synthesize faster and higher quality samples by learning ordinary differential equations to sample from the data distribution.

\hspace{2mm}
\subsection{Detection}\label{sec:voice_detection}
\subsubsection{Features used for deepfake detection} %-> THEY ARE COMBINED/PART OF the literature review on voice deepfake detection


An audio spoofing attack occurs when an attacker creates a fake recording of someone's voice to bypass an automatic speaker verification (ASV) system. Although the fake recording may sound like a genuine human person to a human ear, it may still contain certain peculiarities or artifacts that can help in distinguishing it from genuine recordings. To protect the ASV system, voice spoofing countermeasure solutions are usually employed as a first defense mechanism to discriminate between bonafide (genuine) speech and spoofed speech generated using voice deepfake tools. Only bonafide speech will be fed into the ASV system.

A countermeasure system consists of two main modules, a feature extractor and a classifier. Feature selection and extraction play integral parts in spoofing detection algorithms. The features are engineered to be more compact and less redundant compared to the original audio signal. The detection model can also combine model outputs from different types of features to create a fusion model for improved spoofing detection performance~\cite{balamurali2019}.

\paragraph{Techniques based on hand-crafted features}

\textbf{Short-term Cepstral features:}
Mel-Frequency Cepstral Coefficients (MFCCs)~\cite{mfcc} are arguably the most commonly used features as a baseline in speech spoofing detection~\cite{sahidullah2015}. The extraction process of MFCCs begins with computing the power spectrum on each frame of audio signal using a short-time Fourier transform (STFT). Next, triangular filter bank, on a Mel-scale are applied to the power spectrum to extract frequency bands. Mel-scale aims to mimic the perceptually relevant aspects of human hearing, by being more discriminative at lower frequencies and less at higher frequencies. The logarithm of magnitude spectrum is then computed and followed by applying discrete cosine transform (DCT) to decorrelate the filter bank energies. Typically in speech recognition, 13 coefficients of Cepstral values, from the second to the fourteenth, are taken to represent the information regarding the vocal tract features. In practice, it is often to calculate the dynamic values such as delta and acceleration coefficients from the static Cepstral values to take into account the temporal changes in speech from frame to frame.

\textbf{Cepstral features related to MFCCs:}
Different types of filter bank can be used when integrating the power spectrum to compute the Cepstral coefficients. For example, in rectangular filter cepstral coefficients (RFCCs)~\cite{alegre2013}, the integration is performed using a rectangular window, spaced in linear scale. While the linear frequency cepstral coefficients (LFCCs) uses triangular filter bank, spaced in linear frequency scale~\cite{hasan2013}. On the other hand, the IMFCC is produced using the filters that are linearly spaced on the "inverted-mel" scale to put emphasis on the high frequency region~\cite{chakroborty2007}.

\textbf{Constant Q Cepstral Coefficients (CQCCs):}
The constant Q cepstral coefficients (CQCCs) are introduced for the first time as a spoofing countermeasure for automatic speaker verification~\cite{Todisco2017}.  
The CQCCs are obtained by applying time-frequency analysis on speech signal using the constant Q transform (CQT)~\cite{Youngberg1978} followed by the Cepstral analysis.
The CQT provides a greater frequency resolution for lower frequencies and a higher time resolution at higher frequencies, unlike the short-time Fourier transform which
gives constant time and frequency resolution. The resolution of the CQT features captures small variations in lower frequencies which are more useful to the task of spoofing detection.

\textbf{Short-term Phase Features:}
Other researchers have also investigated the use of phase information for speech spoofing detection. For example, some speech synthesis methods cause the loss or phase distortion during the analysis-synthesis steps that can be detected from the phase structure of the speech signals. Phase-based features such as relative phase shift (RPS)~\cite{sanchez2015} which contain purely phase information, modified group delay (MGD)~\cite{wu2012}, and modified relative phase (MRP)~\cite{wang2017} have been shown to be used successfully in the detection of synthetic speech and voice conversion based attacks~\cite{saratxaga2016}.

\subsubsection{Literature review on voice deepfake detection}

%The RPS representation uncovers the phase structure of
%the speech, which is not apparent from the instantaneous
%phase.

\paragraph{Techniques based on learned feature representation}
The conventional method for discriminating bonafide speech from spoofed speech is to train binary classifiers using hand-crafted features. Integrating multiple feature parametrisation is often necessary to improve anti-spoofing performance, for example via score fusion of subsystems based on different features to capture complementary information~\cite{zwu2017}.
Instead of using handcrafted features which are fixed and not learnable, relevant features for discrimination can be learned directly from raw waveform using deep learning techniques. The learnable front-ends can be categorized into two~\cite{fastaudio}: Short-Time Fourier Transform (STFT) based front-ends and First-order Scattering Transform (FST) based front-ends. In STFT based front-ends such as DNN-FBCC~\cite{dnn-fbank} and FastAudio~\cite{fastaudio}, the learned filterbank matrix are multiplied with the a spectrogram to create compressed representations.  For spoof detection task, FastAudio is shown to perform better than handcrafted CQT features and FTS-based front-ends on the ASVspoof 2019 dataset~\cite{fastaudio}.

The FST-based front-end approaches use a convolutional neural network to learn filtering process on raw audio waveform. 
%For example, ~\cite{} uses 
One example of FST-based front-ends is SincNet~\cite{sincnet}, an end-to-end neural network architecture comprises %parameterized Sinc functions as filters in the first layer followed by convolutional layers. 
learnable a bank of band-pass filters parameterized in the form of Sinc functions. SincNet encourages the network to discover more meaningful filters from temporal signal, learning a customized filter bank structure specifically tuned for the desired application. 
RawNet2 combines SincNet and RawNet1 architectures for end-to-end antispoofing~\cite{tak2021}. The first layer of RawNet2 follows the SincNet architecture and the upper layers consist of the same residual networks and GRU layer as RawNet1. RawNet1 is aimed to directly model raw waveforms for the extraction of speaker embedding using convolutional and recurrent neural networks~\cite{jung2019RawNet}. 

To take advantage of the vocoder artifacts in fake audio signals, ~\cite{sun2023} incorporates a vocoder identification module with the feature extractor of the RawNet2. The proposed multi-task learning framework combines multi-class classification loss to classify the type of vocoders used to synthesize fake audio, and a binary classifier to discriminate real from fake. 

\paragraph{Techniques based on self-supervised learning models for speech representations}

Feature extractors using self-supervised learning models for speech representations such as wav2vec and HuBERT have also been investigated for spoofing detection~\cite{xie2021, lantingli2023, kang2024}. The aim is to improve the generalization of the model to unknown attacks rather than using the traditional acoustic features. For example in~\cite{kang2024}, the pretrained wav2vec 2.0 is utilized as a feature extractor, in which several transformer layers are fine-tuned for spoofing countermeasure systems. It is shown that the fine-tuning of the SSL models can result in better generalised spoofing detection performance. Moreover,~\cite{hemlata2023} report the lowest EER for the ASVspoof 2021 logical access (LA) database when using wav2vec 2.0 front-end with the self-attentive aggregation layer and data augmentation strategy.   


%
%\subsubsection{Learning objective and loss functions}
\subsubsection{Performance evaluation}

Voice deepfake detection is primarily treated as a binary classification problem where the task of a countermeasure is to discriminate between bonafide and spoofed utterances. There are two types the binary classifier makes: \textit{false alarms} (false positive), where the classifier accepting spoofed speech that should have been rejected, and \textit{misses} (false negative), where the classifier rejects bonafide speech that should have been accepted. The \textit{false alarm rate} $P_{fa}$ and \textit{miss rate} $P_{miss}$ can be estimated by dividing the number of misses and false alarms by the number of spoof trials and bonafide cases, respectively~\cite{bonastre2021}. 
The most common metric adopted for binary classification is \textit{equal error rate} (EER), a single number in which the \textit{false alarms rate} and the \textit{miss rate} are equal. Lower EER points to a more reliable classifier. However, the EER metric does not consider the level of importance between the $P_{fa}$ and $P_{miss}$, for example when false alarms can be more tolerated than misses.

To suit real-world applications, different cost and class prior parameters are introduced in ASVSpoof 2019 and 2021 to take into account both countermeasure and ASV subsystems as a tandem system, called tandem detection cost function (t-DCF)~\cite{yamagishi:hal-03360794}. The t-DCF is defined as:
\begin{equation}
\text{t-DCF} = 
    \min\limits_{\tau}  \{C_o + C_1P_{miss}(\tau) + C_2 P_{fa}(\tau)\}
\end{equation}
where $P_{miss}$ and $P_{fa}$ are miss and false alarm rates of a countermeasures using a threshold $\tau$, and $C_{0}$, $C_{1}$ and $C_{2}$ are pre-defined cost function parameters that depend on ASV performance as well.
%\subsubsection{Universal deepfake detectors ?}

\hspace{2mm}
\subsection{Combating voice deepfakes in voice biometrics}

Deepfakes pose a significant threat to voice biometrics systems. Using commercial tools, attackers can synthesize the speech of any person to fool speaker recognition systems. For example~\cite{BILIKA2024} found that voice assistants such as Google’s Assistant and Apple’s Siri can be easily tricked using synthesized samples of victims' speech commands,  in which they used an open source "Real Time Voice Cloning" project~\cite{jeremy2019} to clone the voice of the authorized users. As such, it is necessary to evaluate the efficacy of the state-of-the-art voice deepfake generation methods to fool advanced voice biometric systems. In the following subsection, we provide a summary of our evaluation results. For a detailed discussion please refer to Sec. I of supplementary material.

\subsubsection{Efficacy of voice deepfakes to fool voice biometrics systems}
The evaluations presented in Tab. 1 of the supplementary material illustrate that the state-of-the-art speaker recognition system can be vulnerable to off-the-shelf,  publicly available voice deepfakes generation software. As such, there is a pertinent need to upgrade the safety of the existing voice biometrics models against voice deepfakes. 

\subsubsection{Measures for revealing true identity} To the best of our knowledge, currently, there is no method to extract the true identity of a subject when the speech is manipulated using the voice deepfakes method.

% Please add the following required packages to your document preamble:
% \usepackage{multirow}
\begin{table*}[htbp]
\caption{Top ten AI voice cloning tools to create voice deepfakes.}
\resizebox{\textwidth}{!}{%
\begin{tabular}{c|c|c|c|c}
\multicolumn{1}{c}{\multirow{2}{*}{Tools}} & \multicolumn{3}{c}{Features}                                                                                   & \multicolumn{1}{c}{\multirow{2}{*}{URL}} \\
\multicolumn{1}{c}{}              & Used a cloned voice for TTS & Customize pitch/style & Open-source & \multicolumn{1}{c}{}                     \\ \hline \hline
MetaVoice-1B & \cmark & \xmark & \cmark & https://github.com/metavoiceio/metavoice-src \\ \hline
XTTS-v2 & \cmark & \xmark & \cmark & https://github.com/coqui-ai/tts \\ \hline
Soft-VC VITS & \xmark & \cmark & \cmark & https://github.com/svc-develop-team/so-vits-svc \\ \hline
Tortoise-TTS & \cmark & \xmark & \cmark & https://github.com/neonbjb/tortoise-tts \\ \hline
OpenVoice & \cmark & \cmark & \cmark & https://github.com/myshell-ai/OpenVoice \\ \hline
VoiceCraft~\cite{peng2024voicecraft} & \cmark & \xmark & \cmark & https://github.com/jasonppy/VoiceCraft \\ \hline
ElevenLabs     &    \cmark    &    \xmark   & \xmark   & https://elevenlabs.io/                   \\ \hline
Descript       &     \cmark                                   &     \xmark                       & \xmark   &https://www.descript.com/                \\ \hline
Respeecher     &   \cmark                                     &           \cmark                 & \xmark   &https://www.respeecher.com/              \\ \hline
Play.ht   &    \cmark                                    &      \xmark                      & \xmark   &https://play.ht/                         \\ \hline
%Resemble AI  &   \cmark                                     &     \xmark                       & \xmark   &https://www.resemble.ai/       \\ \hline
\end{tabular}}\hspace{2mm}

\end{table*}

\section{Face deepfakes}\label{sec:face_deepfakes} %\textcolor{red}{RESPONSIBILITY - Tharindu}
\subsection{Generation}

Face manipulation techniques within deepfakes can be broadly categorised into four groups based on the level of manipulation. (i) synthesising the entire face: creating a non-existent face using generative AI technology, (ii) identity swap: replacing the face of one person in a video with another one, (iii) attribute manipulation: modifying some facial attributes such as eyeglasses, hair color, etc., and (iv) expression swap: modifying facial expressions in an image or video. Fig. \ref{fig:face_generation_types} visually illustrates differences between these generation types. 

% \begin{figure*}[htbp]
%     \centering
%     \includegraphics[width=\textwidth]{figures/face_generation_types.pdf}
%     \caption{Illustration of different face generation techniques within deepfakes. Sub-figures (i)-(iv) have been sourced from \tiny \footnotemark, \footnotemark, \footnotemark, and \footnotemark, respectively. }
%     \label{fig:face_generation_types}
% \end{figure*}

% \footnotetext[2]{https://thispersondoesnotexist.com/}
% \footnotetext[3]{https://www.mdpi.com/2076-3417/13/11/6711}
% \footnotetext[4]{https://link.springer.com/chapter/10.1007/978-3-031-19778-9\_41}
% \footnotetext[5]{https://ieeexplore.ieee.org/abstract/document/10285057/}

\begin{figure*}[htbp]
    \centering
    \includegraphics[width=\textwidth]{figures/face_generation_types.pdf}
    \caption{Illustration of different face generation techniques within deepfakes. Sub-figures (i)-(iv) have been sourced from {\tiny\protect\footnotemark[2]}, {\tiny\protect\footnotemark[3]}, {\tiny\protect\footnotemark[4]}, and {\tiny\protect\footnotemark[5]}, respectively.}
    \label{fig:face_generation_types}
\end{figure*}

\footnotetext[2]{https://thispersondoesnotexist.com/}
\footnotetext[3]{https://www.mdpi.com/2076-3417/13/11/6711}
\footnotetext[4]{https://link.springer.com/chapter/10.1007/978-3-031-19778-9\_41}
\footnotetext[5]{https://ieeexplore.ieee.org/abstract/document/10285057/}

Deep learning-based face swap models are capable of replacing one person's face in an image or video with another person's face, maintaining the overall structure and movement of the original face \cite{mirsky2021creation}. As such they can seamlessly attain identity swap face manipulation. The word reenactment implies acting out a past event or bringing something to life. Similarly, facial reenactment refers to bringing the source image to life by modifying it based on the movement of the head, lips, and facial expression in the target video (also called as driving video). Therefore, facial reenactment methods fall under the expression swap category and are not intended to directly alter a person's identity in a video. However, the recent advances that facial reenactment technology attained have enabled it to achieve far-reaching flexibility, that extends beyond simple expression manipulations and is of significant concern. As such review both face swap and face reenactment literature in this section.

Despite its negative implications such as misinformation and fake news, impersonation, identity theft, deepfake pornography, face swap technology has its faithful applications such as therapeutic and psychological applications and entertainment. For instance, face swap technology is already being used in exposure therapy and empathy-building exercises \cite{yang2022can}. Furthermore, off-the-shelf face swap apps such as Deepswap \footnote{https://deepgram.com/ai-apps/deepswap} and Faceswapper \footnote{https://faceswapper.ai/} are readily being used in social media for creating humorous or creative content by swapping faces with celebrities or other popular characters. As such research on improving the quality and realism of the face swap content has been at the forefront of machine learning research.

Along the same lines face reenactment or in other words talking face generation becoming increasingly popular with each passing day as it opens up a multitude of novel applications in this digital age. Video conferencing by animating a well-dressed image of ourselves without the need to transmit a live video stream \cite{deepfakevideocall} or commercials in which human actors are replaced by the faces generated by deepfakes \cite{lomnitz2020multimodal} which would have seemed a fantasy a few years a go has now become a reality thanks to advances of face reenactment technology.

\subsubsection{Face deepfake types}

%In face reenactment, the goal is to animate a source image using a driving video’s motion while preserving the source identity \cite{agarwal2023audio}. 
The face swap models can be generally categorised based on the algorithm that they utlise in the face swap process. For instance, landmark-based methods where facial landmarks are first detected in both the source and target faces to guide the swapping process, auto-encoder-based end-to-end learning models, Generative Adversarial Networks (GANs) based approaches, etc. The technical details of these approaches are discussed in detail in Sec. \ref{sec:DeepfakeGenerationProcess}.

Existing works on face reenactment can be broadly categorised based on the type of modality used to drive the reenactment into three groups: (i) video-driven face reenactment methods, (ii) audio-driven face reenactment methods, and (iii) text-driven face reenactment methods. 

Video-driven methods where information from the driving video is used to extract the features required to reenact a source image are considered the most powerful compared to audio-driven and text-driven methods. Despite the fact that the identity of the person in the source image and the driving video could be different, motion, expression, and other facial features could be extracted from the driving video, allowing the video-driven method to extract a richer feature. However, audio and text-driven methods offer more practicality in real-world applications such as video conferencing, film production, or augmented reality where obtaining a driving video is impractical. However, obtaining driving audio or text is more feasible \cite{agarwal2023audio}.


\subsubsection{deepfake generation process}\label{sec:DeepfakeGenerationProcess}
In this section we provide an overview of the techniques utilised in literature for face swapping and face reenactment.

\begin{figure*}
    \centering
    \includegraphics[width=\textwidth]{figures/face_swap_vs_reenactment.pdf}
    \caption{A comparison between face swapping and face reenactment processes}
    \label{fig:DeepfakeGenerationProcess}
\end{figure*}

Fig. \ref{fig:DeepfakeGenerationProcess} provides a comparison between the processes involved in face swapping and face reenactment. In general, face swap algorithms have three main steps \cite{waseem2023deepfake}. First, they detect the faces in both the source and target video, and the main attributes of the face in the target video such as nose, mouth, and eyes are replaced by the corresponding features of the source face. The next step involves the blending of the manipulated attributes to match the target video's colour and lighting. 

In contrast, in the face reenactment process, both the source image and the driving video are encoded into a latent space. Lower dimensional motion representations that capture head pose, expression, etc. are extracted from the latent space of the target video, and the identity information from the latent space of the source image. The decoder leverages this information and animates the source image using a driving video’s motion while preserving the source identity \cite{agarwal2023audio}.  
 
Autoencoders, Generative Adversarial Networks, Latent Space Decomposition approaches, and Diffusion Models are among the most popular techniques utilised in literature for face swapping and face reenactment, and the rest of the section provides an overview of these techniques and how they have been leveraged in the deepfake generation process

\textbf{Autoencoders: } Based on an encoder-decoder architecture, autoencoders are driven by the principle of learning a compressed latent representation of the input data that captures the salient attributes. The encoder encodes the input data into this learned latent space and the decoder should be able to use these latent embeddings and reconstruct the input without any information loss. The training process is guided by a loss function that minimises the reconstruction error and among the loss functions leveraged in the literature Mean Squared Error (MSE) Loss and Kullback-Leibler (KL) Divergence Loss are popular.

MSE loss can be written as,

\begin{equation}
    L_{MSE}(x, \hat{x}) = \frac{1}{N} \sum_{i=1}^{N} (x_i - \hat{x}_i)^2,
\end{equation}
where $x$ denotes input data, $\hat{x}$ denotes the reconstruction of the input data using the latent embeddings and $N$ is the number of samples in input data.

KL divergence loss measures the discrepancy between the distribution of the encoded latent representations and  predefined prior distribution and can be calculated as,

\begin{equation}
L_{KL}(q(z|x) \| p(z)) = -\frac{1}{2} \sum_{i=1}^{N} \left(1 + \log(\sigma_i^2) - \mu_i^2 - \sigma_i^2\right),
\end{equation}
$q(z|x)$ is the distribution of the encoded latent representations, $p(z)$ denotes predefined prior distribution, and $\mu_i$, and $\sigma_i$ are the mean and standard deviation of the latent representation of the $i$th input, respectively.

Numerous architectures \cite{perov2020deepfacelab, dfaker, tewari2017mofa} have been proposed that make use of the autoencoder technique for face swapping and face reenactment. While they have intricate differences they are based on the principle of learning a latent space that captures salient facial characteristics and identity information and decoding that information onto the target video. 

\textbf{Generative Adversarial Network (GAN)}: Inspired by the recent success of GANs for generating photo-realistic synthetic content, numerous works have leveraged GANs for generating face deepfakes. GANs also operate under the same principle of encoder-decoder architecture, however, additional supervision is provided via a discriminator, $D$. Specifically, the encoder of the generator $G$ maps the source data $x$ into a latent embedding $\phi$ i.e. $x \rightarrow \phi$, and the decoder portion of $G$ utilises this latent embedding for decoding the target representation $\hat{y}$ i.e $\phi \rightarrow y$. To augment the learning of mapping $x \rightarrow \phi \rightarrow \hat{y}$ an adversarial objective is proposed where the goal is to make the generated synthetic content look realistic such that the discriminator cannot differentiate between real and generated content. This objective can be written as,

\begin{equation}
\begin{split}
L_{GAN} & = \min_G \max_D \mathbb{E}_{x \sim p_{{data}}(x)} [\log D(x)] \\
& + \mathbb{E}_{z \sim p_z(z)} [\log (1 - D(G(z)))]
\end{split}
\end{equation}

where $y$ is the ground truth target and $z$ is random noise. 

In addition to the adversarial objective, several works have utilised additional loss terms such as L1 reconstruction loss between $\hat{y}$ and $y$ \cite{wu2018reenactgan} or perceptual loss which calculates the differences between high-level feature representations extracted from pre-trained networks for $\hat{y}$ and $y$ \cite{yu2020multimodal}, to provide additional supervision to the generator. In face-swapping methods the generator of the GAN framework receives the frames of the target video and a representation (i.e image or video) of the source identity. Then the generator learns to map the attributes of the source face onto the target face. Similarly, in most literature that leverages the GAN framework for face reenactment, the source image and the driving modality (i.e. audio, video, and text) are encoded and fused in the encoder of the generator. The decoder learns to map this encoded representation into the target frames. 

\textit{Cycle GAN:} The success of the Cycle GAN model proposed by Zhu et al. \cite{zhu2017unpaired} in unpaired Image-to-Image Translation has also been leveraged in several face reenactment literature \cite{wu2018reenactgan, xu2017face}. Specifically, the Cycle GAN removes the need for paired inputs and targets from input, $X$, and target, $Y$ domains using a cycle consistency, which ensures that translating from one domain to another and then back results in an output close to the original. Formally, the cycle consistency loss can be written as,
\begin{equation}
\begin{split}
  L_{Cycle-GAN} & = \min_{G_X, G_Y} \max_{D_X, D_Y} \mathcal{L}_{\text{GAN}}(G_X, D_X, Y, X) \\ & + \mathcal{L}_{\text{GAN}}(G_Y, D_Y, X, Y) + \lambda \mathcal{L}_{\text{cycle}}(G_X, G_Y),
\end{split}
\end{equation}

where $G_X$ and $G_Y$ denote generators for domains X and Y, while 
$D_X$ and $D_Y$ are the discriminators for domains X and Y, respectively. $\lambda$ is a hyperparameter controlling the importance of cycle-consistency. The works that utilises the Cycle GAN framework for deepfake face generation follow a similar approach to GAN-based architectures. These works treat this encoded representation as domain X and the target video as domain Y. The main difference between GAN-based approaches and Cycle GAN-based approaches is the requirement for paired inputs and targets requirement in GAN-based approaches while Cycle GAN allows many input source images to be mapped into one target video.

\textit{Recurrent GANs:} Motivated by the need to capture temporal information in the generation process, authors of several works have based their frameworks on recurrent GAN network structure.

Within the structure of the generator and discriminator, the Recurrent GAN utilises recurrent neural networks which allow it to model the temporal relationships within the data. Recurrent GAN possesses two additional losses, in addition to the typical GAN loss, namely, temporal loss and recurrent loss. 

The temporal loss can be written as, 
\begin{equation}
\mathcal{L}_{\text{temporal}}(\textbf{G}) = \frac{1}{T-1}\sum_{t=1}^{T-1} \| \textbf{G}(\textbf{z})_t - \textbf{G}(\textbf{z})_{t+1} \|_2^2
\end{equation}
is designed to promote smoothness in the penalise generated sequence by minimising the difference between consecutive frames. The coherence of the generated sequence is encouraged by the recurrent loss where,
\begin{equation}
\mathcal{L}_{\text{recurrent}}(\textbf{G}) = \frac{1}{T}\sum_{t=1}^{T} \| \textbf{G}(\textbf{z})_t - \textbf{G}(\textbf{z})_{t-1} \|_2^2
\end{equation}
natural temporal progression in a sequence is encouraged.

%Chen et al. [10], [28] propose a cascade GAN approach ATVGnet to generate realistic talking face videos, which consists of a multi-modal convolutional-RNN-based (MMCRNN) generator and a regression-based discriminator structure. 

\textit{Multimodal GANs:} Especially, in audio and text-driven face reenactment scenarios multimodal fusion strategies are employed due to the distinct modality representation of the input space capture. Some of the popular modalities that are used in literature are, audio features \cite{yu2020multimodal,agarwal2023audio}, linguistic features \cite{yu2020multimodal, fan2022faceformer}, categorical representation of the emotion \cite{goyal2023emotionally}, and expression, pose, and gaze-related features \cite{agarwal2023audio}. 

Numerous fusion strategies ranging from direct concatenation \cite{goyal2023emotionally} to attention-based fusion \cite{yu2020multimodal} have been leveraged in the generator to combine the multimodal inputs.


\textbf{Latent Space Decomposition:} A latent space is a learned lower-dimensional representation of the data that captures only the essential features. Several architectures, including, Variational Autoencoders, and GANs utilise the concept of latent space transfer which involves manipulating this learned latent space such that the decoded representation is of another representation of the input. For instance, \cite{wu2018reenactgan} proposes to map the input visual representations into a boundary latent space that represents a face with respect to its boundaries instead of raw pixels-based values. Several works \cite{wang2021one,jang2023s} within the face deepfakes generation literature have extended the concept of latent space transfer such that the encoded features in the latent space are decomposed into sub-attributes or sub-regions in which the features are disentangled. 
For example, in \cite{jang2023s} the authors propose a decomposition of the audio and visual inputs into canonical space and multimodal motion Space. In canonical space, every face has the same motion patterns but different identities while in the multimodal motion space only represents motion-related features irrespective of identities. 

\textbf{Diffusion Models:} Diffusion models also operate based on the concept of manipulating the latent space. They extend this concept by learning the underlying probability distribution of the data in the lower-dimensional space. Furthermore, they often employ hierarchical structures to capture the latent space in multiple levels of abstraction in the data, facilitating learning of both local and global structures. Diffusion models are newly emerging within the landscape of deepfake generation \cite{chen2023text, xu2023multimodal, wang2022diffusion}. They are preferred over the GAN-based counterparts due to the stability of training of diffusion models over GANs. Generators of the GAN often suffer from issues such as mode collapse \cite{xu2023multimodal}. On the other hand diffusion models exhibit more stable training dynamics due to their well-defined learning objectives.

Among different variants of diffusion models denoising diffusion models are most commonly used. Specifically, a diffusion process gradually adds noise to the data sampled from the target distribution as a Markov chain and the objective is to estimate the clean version of the input by leveraging the reverse diffusion process. This objective could be written as,
\begin{equation}
 \mathcal{L}_{\text{denoise}} = \mathbb{E}_{\mathbf{x} \sim p_{\text{data}}} \left[ \frac{1}{2\sigma^2} \| \mathbf{x} - \tilde{\mathbf{x}} \|_2^2 + \frac{1}{2} \log(2\pi\sigma^2) \right],   
\end{equation}
where $\mathbf{x}$ is the ground truth clean data sample, $\tilde{\mathbf{x}}$ is the denoised data sample generated by the diffusion model. $\sigma^2$ denotes the variance of the noise added to input during the forward diffusion process and $\|.,.\|_2^2$ is the Euclidean distance between two vectors $\mathbf{x}$ and $\tilde{\mathbf{x}}$. Once the training of the diffusion model completes the learned latent space is used for facial manipulation.


\subsubsection{Performance evaluation} 

Several different metrics have been used to measure the quality of the generated video. Among these methods L1, LMD, AED, ID, PSNR, SSIM, EFD and Sync are popular. \textbf{L1} measures the average L1 distance between the ground truth and generated video considering all the pixels while \textbf{LMD} measures the distance with respect to facial landmarks using a pre-trained facial landmark detector \cite{siarohin2019first}. \textbf{AED}  Average Euclidean Distance measures the distance of the ground truth and generated behavioural information using a pre-trained facial feature extractor such as Openface \cite{amos2016openface}. Among the extracted behavioural features expression, face angle, and eye gaze are popular. Several works have also used distance in terms of identity \textbf{(ID)} features extracted from pre-trained face recognition models Curricularface \cite{huang2020curricularface} and Arcface \cite{deng2019arcface} to measure the quality of the synthesised faces. Typically this distance is measured as cosine similarity between the ground truth and generated identity features. Peak Signal to Noise Ratio \textbf{(PSNR)} evaluates the reconstruction quality of the generated image sequence compared to the ground truth image sequence. Similarly, the Structural Similarity Index 
 \textbf{(SSIM)} has been used to evaluate the changes with reference to the structural information of the ground truth and generated images. To measure the ability of the deepfake generation methods to synthesise natural human emotions some works \cite{xu2023multimodal, goyal2023emotionally} employ pre-trained emotion recognition models such as Affectnet \cite{mollahosseini2017affectnet} and measure Emotion/Expression Feature Distance (EFD). To measure the lip synchronisation \textbf{(Sync)} several works \cite{fan2022faceformer, xu2023multimodal, jang2023s} have used the Syncnet \cite{chung2017out} or Meshtalk \cite{richard2021meshtalk} confidence score. In addition, we would like to point out that a few studies have used human trials to validate the perceptual realism of the generated synthetic media. For instance, in \cite{wu2018reenactgan} 30 volunteers are used to compare the quality of the images generated by \cite{wu2018reenactgan} and baseline methods using the protocol presented in \cite{isola2017image}

\subsubsection{Literature review on deepfake generation}
This section presents the literature review of deepfake generation technology under two categories, face swapping and face reenactment. \newline \newline
\textbf{Face Swapping:} One of the first approaches for successfully generating face swap videos is from 2017 when a Reddit user utilised an autoencoder-based framework to generate face deepfakes \cite{waseem2023deepfake}. During the training phase of this architecture, which is illustrated in Fig. \ref{fig:ae_face_swapping}, it receives images of two separate identities and a shared encoder generates latent representations for those two inputs. The motivation for using a shared encoder is to learn a shared latent space to represent both source and target individuals. Two decoder networks are used to recreate the inputs of their respective identities. Once the training completes the decoder of the source face is leveraged to reconstruct the source face on the input frames of the target video. Due to the simplicity and powerfulness of this framework numerous applications, including, DeepFaceLab and DFaker, have been proposed which are based on the principles of this autoencoder architecture.

\begin{figure}
    \centering
    \includegraphics[width=.8\linewidth]{figures_new/face_swap_vs_reenactment.pdf}
    \caption{Illustration of the autoencoder-based framework introduced for face swapping. A shared encoder generates latent representations for source and target faces and the two decoder networks recreate the inputs of their respective identities. In the face-swapping stage, the decoder of the source face is used to reconstruct the source face on the target video.}
    \label{fig:ae_face_swapping}
\end{figure}

A landmark detection-based approach is proposed in \cite{nirkin2018face} where the authors propose to first detect 2D facial landmarks in both source and target faces to compute the 3D pose which accounts for both viewpoint and expression of the respective faces. Then the face regions are segmented using a pre-trained fully convolution neural network (FCN) to remove background and occlusions. During the face transfer stage, the source face is warped onto the target face using the alignment priors computed based on the 3D face poses. As the final step, the authors propose to blend the overlayed source face with the target face's background using an off-the-shelf algorithm \cite{perez2003poisson}. Despite the significant advances made by these methods, their perceptual quality was poor as they left a lot of artifacts in the generated faces. To account for those limitations and to improve the quality of face swapping GAN-based approaches were proposed. 

Early GAN-based methods such as Face-swap GAN (FS-GAN) \cite{deepswapgan} and DeepFaceLab \cite{perov2020deepfacelab} require subject-specific training. For instance, in the FS-GAN method, the encoder-decoder networks of the autoencoder architecture are used as the generator of the GAN framework, and a discriminator is added to provide additional supervision. In addition to this adversarial loss, the mean square error between the reconstructed and the ground truth face and perceptual loss which is computed using the features extracted using the VGGFace model are used to guide the generator training. Due to the subject specificity of the training, the capabilities models are restricted for swapping faces between specific identities \cite{waseem2023deepfake}, hence offering limited generalistion ability. 

\begin{figure*}[htbp]
    \centering
    \includegraphics[width=\textwidth]{figures_new/Deep_fake_face_generation_2.pdf}
    \caption{Illustration of the architecture of FSNet model \cite{natsume2019fsnet} which is composed of VAE-based encoder-decoder architecture, and a GAN based generator network.}
    \label{fig:FSNet}
\end{figure*}

Subject-agnostic methods have emerged to overcome the limitations of subject-specific approaches. FSNet \cite{natsume2019fsnet} and RSGAN \cite{natsume2018rsgan} are two popular subject-agnostic methods that are based on VAE. Specifically, FSNet employs a VAE-based encoder-decoder architecture and obtains a latent representation of the face that is independent of the face geometry and appearance of the non-face region in the image. Then the generator of the GAN framework leverages this latent variable and synthesizes a face-swapped image. Fig. \ref{fig:FSNet} visually illustrates this framework. When training this framework the authors have used two separate losses for training the VAE and the GAN. The VAE generates three outputs, namely, face mask, face-part image, and landmark image. The authors propose to use cross entropy losses for the face masks and landmark images and an L1 loss for the face-part images. In addition, identity loss which is calculated as triplet loss is also utilised in training. The proposed GAN framework has two discriminator networks, a global discriminator that distinguishes real and synthesised images, and a patch discriminator that classifies whether a local patch of the image is from a real or a synthesized image. These two discriminators generate two adversarial losses to govern the generator network. In contrast, the RSGAN framework comprises three sub-networks, two separator networks, and a composer network. The separator networks generate latent space representations for face and hair regions of the input image and the composer is trained to reconstruct the input face image using these latent space representations. Similar to FSNet global and patch discriminator networks are used in this architecture. For training the VAE, three reconstruction losses are defined, representing the reconstruction of the face region, hair region, and the full image. For GAN training, similar to FSNet, two adversarial losses based on the two discriminators are incorporated. To enforce the composer network to learn the visual attributes, the authors have added a classifier to the composer which classifies the visual attributes of the input.

Version 2 of the Face Swapping GAN (FS-GANv2) \cite{nirkin2022fsganv2} is a GAN-based architecture that is capable of face swapping and face reenactment. The iterative architecture of FSGAN enables it to handle occluded face regions using interpolation. This architecture is composed of three main components, a reenactment generator and the segmentation CNN module, a face inpainting network, and a blending module. The reenactment generator and the segmentation CNN module receives facial landmarks of the target face such that the pose and expression of the source face can be augmented to match the target face. Then the segmentation CNN computes the segmentation masks for the hair and face of this augmented source face. The face inpainting network inpaints the missing parts due to occlusions and the blending network blends the swapped face region to match the illumination and skin tone of the target face. When training this GAN framework perceptual loss computed based on VGG-19 \cite{simonyan2014very}, reconstruction loss computed using L1 loss, and adversarial loss computed using the multi-scale discriminator architecture of pix2pixHD \cite{wang2018high} is used.

In a different line of work, the authors of the FaceShifter \cite{li2019faceshifter} model propose to extensively utilise target face information in the face-swapping process. Specifically, an attributes encoder for extracting multi-level target face attributes in various spatial resolutions is leveraged. Therefore, the identity encoder of this framework encodes the identity information of the source image in latent space while the attributes encoder receives the target face and extracts attributes of the target. Leveraging these two latent embeddings the proposed Adaptive Attentional Denormalization (AAD) Generator generates the swapped face image. For training this framework the authors have utilised a series of losses, including, adversarial loss computed using the multi-scale discriminator architecture,  identity preservation loss computed using cosine similarity loss, attributes preservation loss at the embedding level, and reconstruction loss as pixel level computed as L2 distance. 

Another notable GAN-based architecture in face swapping is the SimSwap \cite{chen2020simswap} model which proposes an architecture that is generalisable to arbitrary faces and preserves the facial expression and gaze direction of the target face during the swapping process. One of the pivotal contributions of this work is the introduction of the ID injection module which transfers the identity information of the source face into a feature representation that the decoder uses in the decoding process. In addition, an identity loss that encourages this translation to have a similar identity as the source face and a weak feature matching Loss that preserves the attributes of the target face are proposed to train the network. Therefore in total the training process of SimSwap leverages, adversarial loss, reconstruction loss, identity loss, and weak feature matching loss. 

The latent space disentanglement approach of StyleGAN has inspired a few face-swapping architectures. For instance, Xu et al. \cite{xu2022high} proposed an approach that disentangles the texture and appearance features of the source and target faces. During the face-swapping process, the source identity and texture characteristics are mapped to the target appearance features. To maintain the face structure facial landmarks of the source and target faces are also encoded. The authors have utilised a series of loss functions for training this architecture which includes, adversarial loss,  identity-preservation loss, landmark-alignment loss, and style-transfer loss which is based on BeautyGAN \cite{li2018beautygan}. The authors of MegaFS \cite{zhu2021one} utilises the latent space of StyleGAN2 for high-resolution face swapping with different identities. The proposed
Hierarchical Representation Face Encoder (HieRFE) \cite{xu2022high} encodes the facial attributes in a hierarchical manner to maintain more facial details. This is achieved via a ResNet50 backbone-based multiple residual blocks for the extraction of salient features and representing these in a feature pyramid structure based on Feature Pyramid Network \cite{lin2017feature}. In the next stage, the Face Transfer Module (FTM) which controls the mixing of the latent spaces of the source and target faces. For training the HieRFE pixel-wise reconstruction loss, perceptual loss, identity loss, and landmarks loss are used. For FTM the authors propose to use the above four losses as well as a stabilisation loss to stabilise the training process. 

More recently, \cite{rosberg2023facedancer} proposes the FaceDancer architecture to overcome the challenges that the existing methods face due to the lighting, occlusions, and pose variations in the source and target faces. This paper introduces two major modules: Adaptive Feature Fusion Attention (AFFA) and Interpreted Feature Similarity Regularization (IFSR). The AFFA produces attention masks to gate the incoming features that have been conditioned on the source identity information and the unconditioned target face information selectively. Specifically, during the training process AFFA learns which conditioned features (e.g. identity information of the source face) to discard and which unconditioned features (e.g. background information) to keep in the target face. In contrast, the IFSR is proposed for preservation of the attributes such as facial expression, head pose, and lighting while still transferring the identity. The authors employ a series of loss functions during the training, including, identity loss, reconstruction loss, perceptual loss, cycle consistence loss, and adversarial loss. 

% \begin{table*}[htbp]
% \label{tab:face_swap_summary}
% \caption{Summary of state-of-the-art face swap deepfake generation approaches. NA indicates that quantitative comparisons are not available.}
% \resizebox{\textwidth}{!}{%
% \begin{tabular}{|p{2cm}|p{2cm}|p{2cm}|p{2cm}|p{5cm}|p{5cm}|}
% \hline
% Method                  & Input Features                                                 & Architecture                    & Performance & Strengths                                                                                                           & Weaknesses                                                                                                                                                                                      \\ \hline
% FCN \cite{nirkin2018face}           & Facial images and landmarks                                    & CNN                             &      FaceForensics++ \cite{rossler2019faceforensics++} 42.1 (LMD), 0.45  (ID)         & Introduces a semi-supervised pipeline for training the face segmentation model                                      & Requires extensive training data to train and the quality of the face swapping depends on the training data.                                                                                    \\ \hline
% Face-swap GAN \cite{deepswapgan}  & Facial images and landmarks                                    & GAN                             &      NA       & Simple encoder-decoder-discriminator architecture. Reasonable handling of occlusion. Generate realistic eye regions & Can only swap faces between specific identities. Generates overly smoothed faces, face alignment is required.                                                                                   \\ \hline
% DeepFaceLab \cite{perov2020deepfacelab}   & Facial images and landmarks                                    & GAN                             &    FaceForensics++ \cite{rossler2019faceforensics++}  0.73 (LMD)         & High resolution generations. Mature open access toolkit                                                             & The quality of the synthesisation is tightly coupled with the face segmentation quality.                                                                                                        \\ \hline
% FSNet \cite{natsume2019fsnet}        & Facial images and landmarks                                    & VAE-based encoder-decode +  GAN &    FaceForensics++ \cite{rossler2019faceforensics++} 30.8 (LMD), 0.36 (ID)          & Subject-agnostic                                                                                                    & Cannot handle occlusions. Synthesised faces have poor resolution compared to resent state-of-the-art methods                                                                                    \\ \hline
% RSGAN \cite{natsume2018rsgan}         & Facial images and landmarks                                    & Facial Region separator + GAN   &  CelebA \cite{liu2015deep} 1.127 (AED)         & Subject-agnostic. Can also be used to edit facial attributes                                                        & Synthesised faces have poor resolution compared to resent state-of-the-art methods                                                                                                              \\ \hline
% FS-GANv2 \cite{nirkin2022fsganv2}     & Facial images and landmarks                                    & GAN                             &    FaceForensics++ \cite{rossler2019faceforensics++} 21.6 (LMD), 0.37 (ID)          & Can handle occluded faces. Capable of face swapping and face reenactment.                                           & Reliant on facial landmark detection which can sometimes produce erroneous landmarks. Iterative architecture that uses only one frame at a time which does not utilise any temporal information \\ \hline

% Faceshifter \cite{li2019faceshifter}   & Facial images and facial attributes extracted from these faces & GAN                             &    FaceForensics++ \cite{rossler2019faceforensics++}  45.51 (LMD), 0.60 (ID)           & Visually appealing results with consistency in pose, expression, and lighting.                                      & Cannot generate high-resolution images. Iterative frame-by-frame processing which does not incorporate temporal information                                                                     \\ \hline
% SimSwap \cite{chen2020simswap}       & Facial images                                                  & GAN                             &    FaceForensics++ \cite{rossler2019faceforensics++} 8.04 (EFD), 11.76 (FID)           & Effective injection of source identity,                                                                             & Cannot handle occlusions. Limited resolution in the synthesised faces                                                                                                                           \\ \hline
% HiRFS \cite{xu2022high}         & Facial images and landmarks                                    & GAN                             &  FaceForensics++ \cite{rossler2019faceforensics++} 2.79 (EFD)           & Disentanglement of semantics within the latent space. Introduces specialised losses to enable temporal cohenerency. & The quality of the synthesised results depends on the latent codes produced by the  StyleGAN model. Therefore, not guaranteed to preserve the identity attributes of the source face            \\ \hline
% MegaFS \cite{zhu2021one}       & Facial images                                                  & GAN                             &   FaceForensics++ \cite{rossler2019faceforensics++} 2.96 (EFD)          & High resolution face swapping. Can manipulate multiple latent codes concurrently.                                   & The quality of the synthesised results depends on the latent codes produced by the  StyleGAN model                                                                                              \\ \hline
% FaceDancer \cite{rosberg2023facedancer}    & Facial images                                                  & GAN                             &    FaceForensics++ \cite{rossler2019faceforensics++} 7.97 (EFD), 16.30 (FID)          & Better preservation of facial attributes of the source face.                                                        & Limited resolution in the synthesised faces. Limited robustness in occlusions and in poor lighting conditions.                                                                                 \\ \hline
% \end{tabular}}
% \end{table*}

\noindent \textbf{Face Reenactment:} One of the pioneer works in the domain of face reenactment is the Face2Face project \cite{thies2016face2face}.  The facial expressions of both source and target video are tracked. The mouth interior that best matches the re-targeted expression is retrieved from the target sequence and warped to produce an accurate fit. Finally, a blending process is conducted to seamlessly blend the new expression on the target face. However, it should be noted that the coarse 3D facial reconstructions of the target face make the reconstructed target face not accurately follow the source person’s head and eye movements \cite{waseem2023deepfake}. When training this framework the authors have utilised photo-metric alignment loss at a pixel level which measures how well the synthesised image represents input data, a feature alignment loss that enforces feature similarity between a set of salient facial feature points, regularisation loss to make the synthesised faces follow a normal distribution are used.

In a different line of work, ReenactGAN \cite{wu2018reenactgan} utilises the concept of facial boundary transfer for the task of face reenactment. The authors show that the direct transfer of facial movements and expressions at the pixel level is suboptimal and could result in structural artifacts. As such, the authors propose to map the source face into a boundary latent space, and a transformer is subsequently used to adapt the source face’s boundary to the target’s boundary in this latent space. The authors show that learning in boundary space allows them to perform model training without paired data, enabling them to perform many-to-one mapping such that they can reenact a target face based on images or videos from arbitrary sources. For training this framework the authors have used cycle consistency loss, adversarial loss, and a shape constraint loss that encourages a transformed boundary to better follow its source.

An approach that utilises emotion Action Units (AU) for generating diverse facial expressions is proposed in GANimation \cite{pumarola2018ganimation}. The authors show that popular GAN-based approaches suffer from the inability to generate diverse expressions and are limited to generating a discrete number of expressions, determined by the content of the dataset. In contrast, the AU-based approach allows conditioning the GAN synthesis using a continuous manifold allowing them to control the magnitude of activation of each AU. For training this pipeline adversarial loss, total variation loss that is computed as the sum of the squared differences for neighboring pixel values, conditional expression loss that guides the generator to generate target expression, and identity loss are used. 

One-shot and few-shot learning methods have emerged to overcome the need for large-scale datasets of source and target identities for training the existing models which makes them ineffective for reenacting unknown identities. Among the few-shot learning models, \cite{siarohin2019first} and \cite{zakharov2019few} are notable considering the robustness they achieve in diverse settings. Specifically, the First Order Motion Model (FOMM) \cite{siarohin2019first} proposes a few-short learning architecture that decouples appearance and motion information using a self-supervised learning objective.  To account for complex motions, the motion components around the learned keypoints are represented with their local affine transformations. This few-slot learning framework is capable of generating re-enactments with just a few training examples. Furthermore, the generator of this network is capable of handling occlusions using an occlusion mask for regions not visible in the source image and anticipating their appearance. To train this framework the authors have used a reconstruction loss based on the perceptual loss, a relative motion transfer loss, and a keypoint localisation loss. 


Zakharov et al. \cite{zakharov2019few} proposed a few-shot learning architecture that is based on meta-learning. This architecture is composed of three main components, namely the embedder network maps the input images to the embedding vectors, a generator network maps input face landmarks into output frames, and a discriminator to determine the realism of the syntheised images. During the meta-learning stage, the authors trained all three subcomponents of their framework using content loss, adversarial loss, and embedding match loss. The content loss measures the distance between the ground truth image and the reconstruction using the perceptual similarity measure. The embedding match loss encourages the similarity of the two types of ground truth and the encoded image. 

Audio-driven facial reenactment has recently attained significant traction within the research community due to its numerous applications ranging from virtual assistants to dubbing. One of the pioneering works within this domain is in \cite{jang2023s} where the authors propose a framework that is highly flexible in terms of accounting for full target motion including head pose, eyebrows, eye blinks, and eye gaze movements. The authors achieve it by manipulating the motion-related latent space of the face while preserving semantically meaningful features associated with the identity. Specifically, this framework disentangles the latent space of StyleGAN into two distinct subspaces, (i) canonical space that captures different facial identities irrespective of facial attributes, and (ii) multimodal motion space that contains motion features irrespective of modality. The disentanglement of the two subspaces is achieved via introducing an orthogonality constraint between the canonical space and the multimodal motion space. To train this framework the authors have utilised a series of loss functions, including, adversarial loss, identity loss, L1 reconstruction loss, perceptual loss, synchronisation loss which is formulated using SyncNet \cite{chung2017out}, and orthogonality loss implemented by extending \cite{yang2021l2m}. This architecture is visually depicted in Fig. \ref{fig:jang2023s}. 

\begin{figure}[htbp]
    \centering
    \includegraphics[width=\linewidth]{figures_new/deep_fake_face_generation_1.pdf}
    \caption{Disentanglement of the canonical and multimodal motion latent spaces in \cite{jang2023s} which allowed them to manipulate only the motion-related features and preserve identity features.}
    \label{fig:jang2023s}
\end{figure}

The authors of \cite{yu2020multimodal} illustrated the importance of learning the correlation between speech and the movement of the face region around the mouth (lips, cheeks, and chin) and proposed a novel speech-driven face reenactment architecture named Face2Vid. In this architecture, a time-delayed LSTM that receives both text and audio inputs is adopted to predict mouth landmarks. Leveraging these landmarks in the next module generates optical flow frames such that smooth transitions in both lips and facial movements can be achieved throughout the entire synthesised video clip. Finally, the Face2Vid module translates these optical flow images into video frames. To train this framework the authors have employed, a temporal adversarial loss, a feature mapping loss based on the discriminators \cite{wang2018high}, and the perceptual loss. 

In a different line of work, the authors of \cite{goyal2023emotionally} argue that little focus has been paid to people's expressions and emotions when synthesising deepfake faces and proposes a conditional GAN framework that is capable of generating more realistic and convincing videos with a broad range of six emotions, including, happiness, sadness, fear, anger, disgust, and neutral. This architecture extends the Wav2Lip \cite{chung2017out} framework by conditioning the synthesis on the categorical one-hot vector representation of the emotion and by using an additional emotion encoder and an emotion discriminator. Similar to the prior work the authors have utilised L1 reconstruction loss, and perceptual loss for training this framework. In addition, lip sync-loss loss and emotion discriminator loss are used to provide further guidance. 


More recently, Agarwal et al. \cite{agarwal2023audio} have proposed an Audio-Visual Face Re-enactment GAN named AVFR-GAN. In contrast to purely video or audio-driven architecture, the AVFR-GAN uses both audio and visual cues to generate highly realistic face reenactments. When encoding face information the authors propose to provide additional priors about the structure of the face in the form of a face segmentation mask and face mesh. Melspectrogram representation of the speech is also provided to an audio encoder to help reenact the mouth region. The extracted audio and visual feature maps are combined, warped, and passed to the identity-aware generator along with the source face which generates the reenacted frame. L1 reconstruction loss, perceptual similarity loss, and the equivarience constraints of \cite{siarohin2019first} are the loss functions used for training AVFR-GAN.

% \begin{table*}[htbp]
% \label{tab:face_reenact_summary}
% \caption{Summary of state-of-the-art face reenactment deepfake generation approaches. NA indicates that quantitative comparisons are not available. }
%  \resizebox{\textwidth}{!}{%
% \begin{tabular}{|p{2cm}|p{5cm}|p{2cm}|p{2cm}|p{5cm}|p{5cm}|}
% \hline
% Method                    & Input Features                         & Architecture & Performance & Strengths & Weaknesses \\ \hline
% Face2Face \cite{thies2016face2face}       &    Facial images and landmarks                                    &      3D Morphable Face Models         &           NA  &   A semi-supervised architecture        & Cannot handle occlusions and different head poses           \\ \hline
% ReenactGAN \cite{wu2018reenactgan}      &        Facial images                                 &    GAN          &   DISFA \cite{mavadati2013disfa} 58.4 \% (Facial Action Units Accuracy)       &   robust to different poses, expressions and lighting conditions        &     Output resolution is poor       \\ \hline
% GANimation \cite{pumarola2018ganimation}      & Facial images and emotion action units &        GAN      &       NA      & Can handle complex backgrounds and illumination conditions          &     Unable to adapt to gaze variations       \\ \hline
% FOMM \cite{siarohin2019first}            &    Sparse key-points                                    &   GAN           &  VoxCeleb2 \cite{nagrani2020voxceleb}   0.043 (L1)       &    Can handle complex motions and can animate diverse object types       &    Complexities when handling dynamic backgrounds        \\ \hline
% Talking Heads \cite{zakharov2019few} &    Facial images and landmarks                                    &          GAN    &     VoxCeleb2 \cite{nagrani2020voxceleb} 30.6 (FID)        &   A few-shot learning architecture        &   Cannot manipulate gaze         \\ \hline
% FC-TFG \cite{jang2023s}                 &    Facial images and audio                                    &           GAN   &    VoxCeleb2 \cite{nagrani2020voxceleb} 1.58 (LMD)         &  controllable head pose, eyebrows, eye blinks, eye gaze, and lip movements         &  Requires multimodal inputs          \\ \hline
% Multimodal Talking Faces \cite{yu2020multimodal}                &   Source audio and target face video                                     &  GAN            &      In house Trump Dataset 0.889 (SSIM)       &     Audio driven reenactment      &  Only generates faces with limited pose variations          \\ \hline
% EmoGen \cite{goyal2023emotionally}                 &        Audio, video and emotions                                &    GAN          &    CREMA-D \cite{cao2014crema} 6.04 (FID)         &  can generate faces with diverse emotions         &    has been evaluated with straight head poses        \\ \hline
% AVFR-GAN \cite{agarwal2023audio}        &   Facial images and audio                                     &     GAN         &   VoxCeleb \cite{nagrani2017voxceleb} 8.48  (FID)        &  Generalises well to unseen faces         &     cannot handle occlusions       \\ \hline
% PNCC GAN  \cite{xue2023high}                &    3D face                                    & GAN             &    VoxCeleb \cite{nagrani2017voxceleb} 17.21 (FID)         &       preserves target face identity    &  cannot handle extreme head poses          \\ \hline
% \end{tabular}}
% \end{table*}

\noindent\textbf{Summary of face deepfake generation methods and open research questions:} GAN-based technologies have been the dominant approach for the generation of deepfakes within both face swap and face reenactment categories. Preservation of facial expressions, head-poses, etc. of the target face has been one of the main challenges within the face swapping task. MegaFS \cite{zhu2021one}, FaceDancer \cite{rosberg2023facedancer} architectures take an important step towards the preservation of target facial attributes, however, still there exist limitations with respect to handling occlusions and the resolution of the synthesised faces. On the other hand, the recent trends within the face reenactment literature have been on audio-driven facial reenactment and few-slot learning frameworks that could be trained with a few training examples. While these attempts are commendable and could elevate the utility of face deepfake generation technology with numerous novel applications, the synthesised faces lack realism. For instance, most of the recent state-of-the-art methods such as FC-TFG \cite{jang2023s}, AVFR-GAN \cite{agarwal2023audio}, and GANimation \cite{pumarola2018ganimation} within the face reenactment category are only capable of generating face deepfakes at $\leqq 256 \times 256$ resolution. This is significantly lower when compared with the output resolution of the recent face swap technology such as MegaFS \cite{zhu2021one} and HiRFS \cite{xu2022high} which can achieve resolution up to $1024 \times 1024$. Furthermore, most of the existing state-of-the-art methods within the face reenactment domain lack gaze adaptation, cannot handle extreme poses, and fail to preserve source facial features. In Tabs. 1 and 2 in supplementary material we summarise the state-of-the-art face-swap and face-reenact deepfake generation methodologies, respectively, and discuss their strengths and weaknesses. 

\noindent\textbf{Top-ten tools for generating face deepfakes:}
In Tab. \ref{tab:face_get_tools} we summarise top-ten tools, including free and open source tools, that are available for the generation of face deepfakes. When comparing the available tools for ranking we consider the quality of the generated faces, customizability, types of media that they can manipulate and the availability of the source codes. 
\begin{table*}[htbp]
\caption{Top ten tools to create face deepfakes.}
\resizebox{\linewidth}{!}{%
\begin{tabular}{|c|c|c|c|c|c|c|}
\hline
deepfake Type                    & Method                   & Images & Audio & Video & Open-source & URL                                                       \\ \hline
\multirow{6}{*}{Face Swap}        & Face Swap Live  & \xmark      & \xmark     & \cmark     & \xmark           & https://apps.apple.com/us/app/face-swap-live/id1042987645 \\ \cline{2-7} 
                                  & Deepfakes Web            & \xmark      & \xmark     & \cmark     & \xmark           & https://deepfakesweb.com/                                 \\ \cline{2-7} 
                                  & FaceMagic                & \cmark      & \xmark     & \cmark     & \xmark           & https://www.facemagic.net/faceswap                        \\ \cline{2-7} 
                                  & DeepFaceLab              & \cmark      & \xmark     & \cmark     & \cmark           & https://github.com/iperov/DeepFaceLab                     \\ \cline{2-7} 
                                  & ReFace                   & \cmark      & \xmark     & \cmark     & \xmark           & https://www.reflect.tech/                                 \\ \cline{2-7} 
                                  & Faceswap                 & \cmark      & \xmark     & \cmark     & \cmark           & https://github.com/deepfakes/faceswap                     \\ \hline
\multirow{4}{*}{Face Reenactment} & Avatarify                & \cmark      & \xmark     & \cmark     & \xmark           & https://avatarify.ai/                                     \\ \cline{2-7} 
                                  & Wav2Lip                  & \xmark      & \xmark     & \cmark     & \cmark           & https://github.com/Rudrabha/Wav2Lip                       \\ \cline{2-7} 
                                  & Myheritage               & \xmark      & \xmark     & \cmark     & \xmark           & https://www.myheritage.com/deep-nostalgia                 \\ \cline{2-7} 
                                  & First Order Motion Model & \cmark      & \xmark     & \cmark     & \cmark           & https://github.com/AliaksandrSiarohin/first-order-model   \\ \hline
\end{tabular}}
\label{tab:face_get_tools}
\end{table*}

We would also like to note popular generative AI tools such as Deepbrain
\footnote{https://www.deepbrain.io/}, Midjourney \footnote{http://www.midjourney.com/home} and DALL-E 2 \footnote{https://openai.com/index/dall-e-2/} which are models that are capable of generating realistic images and video from text descriptions. However, these methodologies do not directly fall within the scope of this paper as they are not deepfake tools. Therefore, we do not include them in the comparison in Tab. \ref{tab:face_get_tools}. 


\hspace{2mm}
\subsection{Detection}
\subsubsection{Features used for deepfake detection}
\noindent\textbf{Hand-crafted features based approaches:} Early works of deepfake detection leveraged statistical features that have been hand-crafted by analysing the image's pixel values. For instance, \cite{koopman2018detection} proposed the use of Photo Response Non-Uniformity (PRNU) analysis to detect unique noise patterns that are left in the image due to manufacturing defects in the camera sensor. The authors show that when performing the face swap it alters the PRNU patterns of the original video. In another work \cite{kharbat2019image} edge features such as Histogram of Gradient (HoG) features were extracted to illustrate that the edge features from a real video are more correlated than the edge features from the fake video. Xia et al. \cite{xia2022towards} proposed to do a statistical analysis of the colour space of the image to determine the differences between real and fake video frames in various color channels. The input RGB frames are converted to HSV and YCbCr color spaces and first-order differential operators are employed to extract the texture difference features from the colour channels. Wang et al. \cite{wang2022ffr_fd} argued that due to the smoothing of the face region during the blending stage of the face-swapping process, fake videos have fewer feature points than real videos. The authors proposed to use feature point descriptors such as Speeded Up Robust Features (SURF), Scale-Invariant Feature Transform (SIFT), and Oriented Fast and Rotated Brief (ORB) from eight different facial regions such as mouth, inner mouth, eyebrows, eye, and nose. 

\noindent\textbf{Artefacts based approaches:} Majority of the works within the face deepfake detection domain are developed based on detecting the artifacts that are left by the face deepfake generation methods. These artifacts could be broadly categorised into (i) visual artifacts such as facial artifacts, texture artifacts, and boundary artifacts; and (ii) biological artifacts such as changes in heart rate, eye movements, facial movements, and facial expressions.

\noindent\textit{Visual artifacts:} Yang et al. \cite{yang2019exposing} proposed to detect deepfakes via detecting inconsistencies in 3D head pose. In another work \cite{li2021exploiting} facial symmetry is used as the feature for detecting deepfakes. The authors show the existence of inconsistencies and unnatural traces in facial symmetry in face deepfakes. Xu \cite{xu2021deepfake} proposed to use Gray-Level Co-occurrence Matrix to extract texture features with the hypothesis that deepfake generation models produce blurred and irregular textures. In another study \cite{kingra2022lbpnet}, the Local Binary Pattern (LBP) texture feature is used as a descriptor for authenticity. Several methods have also emerged that utilises features from the frequency domain. For instance, in \cite{durall2019unmasking, liu2021spatial} Discrete Fourier Transform (DFT) is used to examine the spectral distributions in both real and fake videos, and in \cite{kohli2021detecting} the inputs are converted into the frequency domain using a 2D global Discrete Cosine Transform and analysed using a CNN. Deep learning models have also been used in some works for extracting artifacts. For instance, Yuezun et al. \cite{li2018exposing} used pre-trained  VGG16, ResNet50, ResNet101, and ResNet152 feature extractors and Kim et al. \cite{kim2021exposing} proposed the use of two deep feature extractors to simultaneously extract content features and trace features from a face image. In \cite{agarwal2020detecting} the authors propose the use of appearance and motion features of the face extracted from pre-trained VGG-19 and Facial Attributes-Net \cite{wiles2018self} models, respectively.

\noindent\textit{Biological artifacts:} Agarwal et al. \cite{agarwal2019protecting} argued that the face deepfakes lack the expressiveness of real videos and proposed a method that is based on analysing the movement of facial landmarks. This approach is extended in \cite{agarwal2020detecting} to capture head poses, facial landmarks, and expressions. In a different line of work \cite{nguyen2020eyebrow} inconsistencies in the eyebrow region are extracted and analysed using CNN-based feature extractors including ResNet and SqueezeNet \cite{iandola2016squeezenet}. Lip reading models have been leveraged by Haliassos et al. \cite{haliassos2021lips} to map irregularities in mouth movement and \cite{liao2023famm} proposes to examine facial muscle motion features. In addition, the lack of eye blinks has also been leveraged as a biological artifact in face deepfakes. Heart rate estimation-based features have also been utilised in the literature. For instance, irregularities in heartbeat rhythms from blood flow in the face \cite{qi2020deeprhythm} and inconsistencies in colour changes in the face caused by variations in oxygen concentration in the blood \cite{fernandes2019predicting, hernandez2020deepfakeson} have been used. Several remote PhotoPlethysmoGraphy (rPPG) based methods \cite{ciftci2020fakecatcher, wu2024local} have also emerged to detect face deepfakes by analysing the irregularities in light absorption in facial skin tissues. 

\noindent\textbf{Deep Learning based approaches:} Under deep learning-based approaches we categorise the neural network architectures that have been proposed to learn features that are indicative of face deepfakes automatically, instead of focusing on a particular feature as the methods mentioned earlier in this section. MesoNet \cite{afchar2018mesonet} and Capsule Network architecture of \cite{nguyen2019capsule} are some of the early works within this domain. However, these models poorly generalise to unseen deepfake generation methods \cite{waseem2023deepfake}. Motivated by the fact that real images have consistent source characteristics throughout the image while manipulated images have inconsistent source characteristics Zhao et al.  \cite{zhao2021learning} proposed a model that is trained using pair-wise self-consistency learning paradigm. An autoencoder-based approach is proposed in \cite{cozzolino2018forensictransfer} which is capable of generalising different yet related manipulation methods. This is achieved via learning a compact embedding that could be translated between different manipulation domains by activating specific regions of the latent space. Kumar et al. \cite{kumar2020detecting} proposed training separate CNN feature extractors to specific facial regions and proposed a framework consisting of five ResNet-18 models. An ensemble of deep learning networks that incorporates numerous state-of-the-art classification models, including XceptionNet, MobileNet, ResNet101, InceptionV3, DensNet121, InceptionReseNetV2, and DenseNet169, into a single pipeline is proposed in \cite{rana2020deepfakestack}. 

Spatio-temporal deep learning approaches are also popular within the face deepfakes detection literature as they possess the ability to analyse spatial and temporal consistency across video frames. Guera \cite{guera2018deepfake} proposed to analyse the framewise feature of a video extracted from a CNN using an LSTM network. A joint learning framework where the CNN architecture is jointly trained with an RNN is proposed in \cite{sabir2019recurrent}. A 3DCNN is proposed in \cite{nguyen2021learning} to simultaneously process spatial and temporal dimensions. A novel architecture named Interpretable Spatial-Temporal Video Transformer (ISTVT) is proposed in \cite{zhao2023istvt} which leverages  Xception blocks to extract spatial features and the authors propose to map spatial and temporal correlations using self-attention modules. In a similar line of work, vision transformers for face deepfake detection are introduced in \cite{wodajo2021deepfake} where the authors propose a network named  Convolutional Vision-Transformer (CVT). Graph neural networks have also been used within the face deepfakes detection literature where the authors of \cite{shang2023constructing} propose the Spatial Relation Graph Unit (SRGU). This architecture can capture local and global spatial inconsistencies through graph convolution. Features of the same spatial location across different frames are modeled into a fully connected graph and a cosine distance-based similarity matrix to detect temporal incoherencies. 

Contrastive learning-based approaches are also popular within the deep learning-based face deepfakes detection approaches. For instance, Xu et al. \cite{xu2022supervised} supervised contrastive (SupCon) learning to discriminate between real and fake images while \cite{dong2023contrastive} proposes a framework to combine intra-domain and cross-domain formation to improve generalisation. 

When considering the multimodal approaches, Mittal et al. \cite{mittal2020emotions} proposed the utilisation of emotion features extracted from audio and video modalities and analysing the inconsistencies. The architecture of \cite{chugh2020not} is a two-stream architecture where the visual stream uses a 3D- ResNet architecture to extract features, and Mel-Frequency Cepstral Coefficients (MFCC) are extracted as audio features. This framework is trained to detect irregularities within the audio and video modalities in a contrastive manner. The framework of Zhou et al. \cite{zhou2021joint} leverages the concept of temporal alignment between audio and video streams. Two separate subnetworks are used to model video and audio streams separately and a synchronisation stream is used to learn the synchronisation patterns between modalities. 

\noindent\textbf{Anomaly detection based approaches: } The main difference between the deep learning based approaches mentioned above and the anomaly detection based approaches is that deep learning based approaches treat deepfake detection as a classification problem where they classify the input into real or fake classes. In contrast, the anomaly detection based approaches formulate deepfake detection as the task of learning normality and detection of fake media with respect to the deviation from this normality. 

One of the pioneering works within anomaly detection based approaches is in \cite{khodabakhsh2020generalizable} in which the authors propose a probabilistic approach that predicts the logarithmic probability of observing a particular pixel's intensity by considering the relationship between previous pixels. In a different line of work local motion patterns of real videos are analysed in Wang et al. \cite{wang2020exposing} to detect anomalies in fake videos. \cite{khalid2020oc} uses a VAE to reconstruct real images and the fake images are detected by considering the root mean square error between the input and reconstructed image. Audio-visual features of authentic videos learned using the large-scale Voxceleb dataset \cite{nagrani2017voxceleb} are used in \cite{cozzolino2023audio} for detecting anomalies caused by deepfakes.


\subsubsection{Literature review on deepfake detection} 

In this section, we summarise the state-of-the-art face deepfake detection methods under artefact-based approaches, deep learning-based approaches, and anomaly detection-based approaches. Note that we do not include a detailed discussion regarding the hand-crafted feature-based approaches due to their inferior performance in current state-of-the-art benchmarks. For instance, the methods such as \cite{kharbat2019image} and \cite{xia2022towards} methods struggle to detect deepfakes in highly compressed videos, and \cite{wang2022ffr_fd} detecting deepfakes in complex backgrounds \cite{waseem2023deepfake}.

\noindent\textbf{Artefacts based approaches:} The face deepfake detection method of Yang et al. \cite{yang2019exposing} leverages artifacts in 3D head pose. The authors observe that the face-swap algorithms only swap faces in the central face region while keeping the outer contour of the face intact. Due to this mismatch of the landmarks in fake faces, there exist inconsistencies in 3D head pose estimation when it is estimated from central and whole facial landmarks. 68 3D facial landmarks are estimated using the OpenFace2 \cite{baltruvsaitis2016openface} library and the head poses from the central face region and whole face are estimated. The differences between the obtained rotation matrices and translation vectors are used as the features to train a Support Vector Machine (SVM) classifier. In a similar line of work, the movement of facial action units is leveraged in \cite{agarwal2019protecting} for detecting face deepfakes. 16 different facial action units (AU) are extracted using the OpenFace2 library and four additional features, including, pitch and roll of head rotation, the 3D horizontal distance between the corners of the mouth, and the 3D vertical distance between the lower and upper lip are extracted. The authors extract this 20-dimensional feature vector for each frame in a 10-second video and apply Pearson correlation to measure the linearity between these features, yielding a 190-dimensional feature vector which is subsequently fed to an SVM for classification. Nguyen et al. \cite{nguyen2020eyebrow} proposed a biometric matching pipeline for the eyebrow region for the task of detecting deepfakes. Specifically, this framework assumes that a bonafide image of the subject is available and this image is used for biometric enrollment. They evaluated four state-of-the-art deep learning models, including, LightCNN \cite{wu2015lightened}, Resnet, DenseNet \cite{huang2017densely}, and SquezeNet for extracting features for biometric matching.  The cosine distance metric is used to measure the similarity between eyebrow features from the enrolled face and the probe face. 

Physiological measurements such as remote
visual PhotoPlethysmoGraphy (PPG) have also been popular among the assessments for identifying artefacts. Specifically, the DeepRythm architecture of Qi et al. \cite{qi2020deeprhythm} leverages the power of remote PPG which could detect and track the minuscule periodic changes in skin color due to the blood flow through the face from a video. The authors introduce a Motion-Magnified SpatialTemporal Representation (MMSTR) that could capture heart rhythm signals and generate motion-magnified
spatial-temporal map which highlights salient motion regions. The authors also introduce dual spatiotemporal attention to adapt to changing head poses, illumination variations, and different deepfake types. This method has been trained using cross-entropy calculated based on the model's deepfake detection performance. In a similar line of work \cite{ciftci2020fakecatcher} extract G channel-based \cite{zhao2018novel} chrominance-based \cite{de2013robust} remote PPG signals from the left cheek, right cheek, and mid-region of the face. Maps representing the spatiotemporal variations of these signals are constructed which are then used to train a CNN to classify the authenticity. More recently Wu et al. \cite{wu2024local} proposed a two-stage network architecture that could detect the inconsistencies in both spatial and temporal domains of the PPG. This architecture is also illustrated in Fig. \ref{fig:rppg_model}. With the motivation that different video manipulation techniques affect distinct facial regions, the authors first divide the input video into $T$ frame video clips and for each clip face alignment is performed and facial landmarks are obtained. Based on the obtained landmarks sub-regions that encompass cheeks, forehead, and jaw are selected. Average pixel values for each sub-region is computed and the min–max normalisation is applied. These sub-regions are then utilised for the generation of PPG maps. A temporal transformer is employed to capture long-term dependencies between adjacent clips. In addition, a MaskGuided Local Attention module (MLA) is used to highlight the position in the PPG that corresponds to the
modified regions of the face image. To train this network a combination of cross-entropy loss and attention mask loss is leveraged. 

\begin{figure*}[htbp]
    \centering
    \includegraphics[width=\textwidth]{figures_new/face_deep_fake_detection_Artefact.pdf}
    \caption{Two-stage network architecture proposed in \cite{wu2024local} which analyses rPPG signals extracted from face, and analyses irregularities in light absorption in facial skin tissues.}
    \label{fig:rppg_model}
\end{figure*}

Eyeblink patterns have also been utilised as biological signals for the detection of face deepfakes. For instance, in \cite{li2018exposing} the authors propose a framework to capture the phenomenological and temporal irregularities in eye-blinking that are left by the deepfake generation methods. Specifically, the authors argue that real videos possess periodic eye blinking patterns while the fake videos do not have such blinking patterns. In this pipeline face detection is performed and the face is aligned to a unified coordinate space using facial landmarks. From the aligned face, an area that surrounds the eye is extracted. This region of interest is passed through a CNN to extract features and the temporal relationships across the frames are mapped using an LSTM which predicts the probability of eye blinking. This framework is trained using cross entropy loss. Another work that leverages eye blinking patterns for deepfake detection is in \cite{jung2020deepvision}. This algorithm, named DeepVision takes age, gender, activity and time of the day information in addition to the video for the detection of deepfakes. The eye blink patterns are identified using the Fast-HyperFace \cite{ranjan2017hyperface} for face detection and the Eye Aspect Ratio algorithm \cite{cech2016real} to detect and track the eye. The deepfakes are detected considering the number of eye blinks and the period of blinks. 

\noindent\textbf{Deep Learning Based Approaches: } MesoNet is among the early works that leveraged deep learned features that have been learned end-to-end for the task of detecting face deepfakes. The Meso-4 architecture proposed by the authors is composed of four layers of successive convolutions and pooling followed by two layers of fully connected layers with dropout. Sigmoid activation is used to generate the binary classification. The authors also propose the MesoInception-4 architecture which is generated by replacing the first two convolutional layers of Meso4 by the inception module of \cite{szegedy2015going}. These frameworks were trained using mean squared error loss. The goal of the ForensicTransfer model proposed in \cite{cozzolino2018forensictransfer} is to ensure the generalisation of the model across different but related manipulation types. The authors propose to train an autoencoder-based deep neural network architecture to disentangle real and fake images in the latent space. This training is done on the source domain data. The tuning of the model on the target domain is done using a few target training samples. To train this framework the authors have used both reconstruction loss and the activation loss. The reconstruction loss measures the difference between the input image and the reconstructed image in the pixel space using L1 distance. The activation loss is used to avoid intra-class variations so that there is a clear separation between the latent space of real images and the latent space that corresponds to the images of all manipulation types, including novel manipulation types. 

Ensemble learning approaches have also been leveraged in literature. For instance, Kumar et al. \cite{kumar2020detecting} proposed to use five ResNet18 models to extract local and global features. Specifically, one ResNet architecture learns overall facial attributes and the remaining four
are dedicated to learning the local, regional attributes. The outputs from these five parallel ResNet-18s, which represent the classifications from the models only considering their respective inputs are concatenated to form a 10-dimensional vector. Then the weighted fusion of these individual scores is performed to generate the final binary classification. This architecture is visually illustrated in Fig. \ref{fig:ensemble_model}.

\begin{figure}[htbp]
    \centering
    \includegraphics[width=\linewidth]{figures_new/face_deep_fake_detection_DL.pdf}
    \caption{Ensemble learning approach of \cite{kumar2020detecting} which incorporates numerous state-of-the-art classification models.}
    \label{fig:ensemble_model}
\end{figure}

When training this framework the authors have utilised a series of cross-entropy losses. The total loss is composed of the sum of the cross-entropy loss of the full-face model, the entropy loss of local regional models, and the cross-entropy loss after the final fusion.  

Another ensemble learning architecture is in \cite{rana2020deepfakestack} in which the authors propose to leverage seven state-of-the-art deep learning models to extract representations. Specifically, XceptionNet, MobileNet, ResNet101, InceptionV3, DensNet121, InceptionReseNetV2, and DenseNet169 models initialised with ImageNet weights have been used as base learners. The authors have replaced the last two layers with a layer with softmax activation. Greedy Layer-wise Pretraining is used to finetune these base learners. Given an input image, these base learners generate true or fake class predictions and the authors propose a stack generalisation model which learns to pick the best combination of the prediction considering the outputs of individual base learners. This framework is trained using the categorical cross-entropy loss. 

In \cite{sabir2019recurrent} the authors argue that less attention has been paid to the temporal features for the detection of deepfakes and propose a recurrent convolutional framework. In their proposed approach the first step is to detect, crop, and align the faces to a reference coordinate system such that any rigid motion of the face is compensated. In the next stage face manipulation detection is conducted using a recurrent convolutional network where the encoding of the frame-wise features is done using the CNN backbone and the final prediction is generated by a recurrent neural network via analysing those sequences of features. Both ResNet and DenseNet have been experimented as the backbone architecture. Moreover, the authors propose to extract features at multiple levels from the backbone CNNs, and these features are processed by individual recurrent networks. This framework is trained end-to-end using cross-entropy loss for binary classification. 

More recently, the success of transformer networks in modelling spatiotemporal features has seeped into the face deepfakes detection domain. In \cite{wodajo2021deepfake} a Convolutional Vision Transformer model is proposed for the detection of deepfakes. Specifically, the CNN architecture is capable of extracting discriminative features from the individual frames and the transformer module learns to analyse the correlation across the sequence of these features and classify them
using an attention mechanism. The authors named the feature extraction CNN as the Feature Learning (FL) component which is a stack of 17 convolutional blocks. The transformer block which receives the feature map of FL is identical to the ViT architecture in \cite{dosovitskiy2020image}. This framework is trained using the binary cross-entropy loss function. In a similar line of work \cite{zhao2023istvt} proposes an Interpretable Spatial-Temporal Video Transformer (ISTVT) for deepfake detection. This architecture leverages a feature extractor constructed using Xception blocks to extract salient textures from the input face. These feature maps are decomposed into tokens. spatial and temporal self-attention modules are used to attend to both dimensions. Specifically, in temporal self-attention the attention heads attend to patches of the same location across the frames while in spatial attention all the patches in each frame are considered. This decomposition of self-attention enabled the authors to interpret the model across both dimensions. ISTVT framework is trained by the binary cross-entropy (BCE) classification loss.

In \cite{xu2022supervised} the authors motivate the need for the deepfake detection algorithm to be agnostic across generation type, quality, and appearance. Furthermore, they argue the need for contrastive learning as the appearance characteristics of the fake video is highly indistinguishable. The authors first train an encoder network which learns to generate normalised embedding from augmented data.  A projection network then uses these embeddings and computes the supervised contrastive loss. Finally, a linear classifier is trained using cross-entropy loss to discriminate between real and fake faces. In another work \cite{dong2023contrastive} multiple views of the same image are used as the augmentation for contrastive learning. The authors observe that the deeper feature representation tends to focus on semantic information while the artifacts left by deepfake generation algorithms exist in shallow feature maps and propose a multi-scale feature enhancement module to combine both local and global features. Furthermore, the authors argue that the deepfake generation artifacts can be found in the frequency feature domain and propose a Steganalysis Rich Model (SRM) \cite{fridrich2012rich} to extract local noise features from neighboring pixels. The authors propose a combination of cross-entropy loss and consistency loss to train this framework. Specifically, the consistency loss minimises the cosine distance in feature space for different augmentations of the same image and cross-entropy loss supports the deepfake detection. 

When considering the multimodal approaches for face deepfake detection, \cite{mittal2020emotions} and \cite{zhou2021joint} are notable considering their effectiveness. In \cite{mittal2020emotions} a two-branch architecture is adopted to process features from both real and fake videos. Specifically, the authors propose to extract facial features using OpenFace \cite{baltruvsaitis2016openface} and speech features using pyAudioAnalysis \cite{giannakopoulos2015pyaudioanalysis} from the raw videos. The extracted features are passed through the two-branch neural network architecture where a separate branch is used for processing each modality separately. Within each branch, there are two separate networks for extracting features representing the modalities and perceived emotions. These feature vectors, each with 250-dimensions, is used to compute a triplet loss function that  minimise the similarity between the modalities from the fake video and maximise the similarity between modalities for the real video. Another multi-stream network architecture is in \cite{zhou2021joint} which uses audio and video streams for the detection of deepfakes. To fuse the video and audio streams the authors propose to apply central connections. At each layer, the audio and visual representation will be fused with the current layer of sync-stream and used as input to the fusion at the next layer. This is achieve through, (i) inter-attention: which computes attention across visual and audio representations, (ii) Inter+intra-attention: which is the video or audio modality-specific self-attention, and (iii) Joint-attention: where the authors have applied same attention weights on both visual and audio representations. During the inference stage, preliminary predictions are obtained through the sync-stream and if it is a positive prediction video and audio branches will be individually analysed to generate the final prediction.

\noindent\textbf{Anomaly detection-based approaches: } In contrast to the deep learning-based approaches which learn discriminative features that could differentiate real faces from fake ones, the anomaly detection-based approaches are designed to learn the distribution of real faces. An input face that significantly deviates from the learned real distribution is identified as a fake face. 

In \cite{khalid2020oc} the authors propose the OC-FakeDect framework which is formulated as a VAE-based approach. Two versions of the OC-FakeDect network architecture are proposed. In the first version of the model the authors propose to compare the input and reconstructed images directly in the image space using Root Mean Square Error (RMSE). In the next version, an additional encoder is appended to map the reconstructed image back to the latent space and the authors propose to compare the input and reconstructed images in the latent space. This architecture is visually illustrated in Fig. \ref{fig:OC-FakeDect-2}. For training this framework the authors have used the KL divergence loss to force the network to approximate a Gaussian distribution in the latent space and mean square error to help minimise the error between the input and reconstructed images. 

\begin{figure}
    \centering
    \includegraphics[width=.6\linewidth]{figures_new/deep_fake_face_anomaly.pdf}
    \caption{OC-FakeDect-2 architecture proposed in \cite{khalid2020oc}which is based on VAE and detects which compares the reconstruction error of the input image and the reconstructed image in the latent space for deepfake detection.}
    \label{fig:OC-FakeDect-2}
\end{figure}

A two-step approach that learns the probability of an occurrence of a certain pixel based on its neighbourhood is proposed in \cite{khodabakhsh2020generalizable}. The authors condition each pixel on pixels before (in raster order) and extend the PixelRNN model \cite{van2016pixel} to learn this distribution for real images. This learned model, named PixelCNN++, predicts the probability distribution that denotes the likelihood of observing a specific pixel value at a given location considering all pixel values before it. Using this approach the authors propose to calculate a probability matrix for the entire image which denotes the likelihood of observing the input image. A Universal Background Model (UBM) is trained with the PixelCNN++ to further refine the features.  A simple classifier based on LeNet-5 \cite{lecun1998gradient} is trained on the output of the UBM model to generate the real/fake classification. 

We would also like to note the contrastive learning approach proposed in \cite{cozzolino2023audio} that exploits the audio-visual features that are exhibited in real videos. Specifically, they propose to extract audio and face embedding vectors, and at each training iteration, these features from $N$ input videos are extracted. By comparing only-video, only-audio, and audio-video feature vectors of $N$ inputs three $N \times N$ similarity matrices are computed. The authors have then utilsed three contrastive losses, one for each similarity matrix, to push the embedded vectors of the same individual closer and move those of different individuals farther apart. For training this framework the overall loss is defined by aggregating the three contrastive losses. During the test time, the authors assume that they have at least 10 real videos of the person of interest and calculate a similarity matrix between the features of the test video and this set of reference videos. the mean and standard deviation of the similarity index are calculated which is used to make a decision regarding the authenticity of the test video. 

\noindent\textbf{Summary of face deepfake detection methods and open research questions:} When reviewing the literature it could be seen that variety of features have been proposed to date for the task of face deepfakes detection. They range from blinking patterns, biological signals such as PPG signals, and 3D head poses to facial behavioral features. Despite these advances to date, there is no universal face deepfake detection methodology that could withstand the current and future advances of the face deepfake generation technology. For instance, most of the existing state-of-the-art face deepfakes detection methods such as \cite{mittal2020emotions} and \cite{zhou2021joint} are not robust against external face deepfake generation methodologies that haven't been seen during the training. Furthermore, they poorly generalise to unseen datasets. As such, the current research achievements within face deepfake detection are far from producing a universal face deepfake detector and warrant further research efforts. Moreover, the lack of interpretability of the face deepfake detection methods has been a major limitation in order to build trust in the general public regarding their decisions. In Tab. 3 of supplementary material we provide a summary of different face deepfake detection methods, highlighting their strengths and weaknesses.
% %In Tab. \ref{tab:detection_summary} we provide a summary of different face deepfake detection methods, highlighting their strengths and weaknesses.



% \begin{landscape}
% \begin{table}[htbp]
% \label{tab:detection_summary}
% \caption{Summary of face deepfake detection approaches}
% \begin{tabular}{|p{2cm}|p{.5cm}|p{5cm}|p{3cm}|p{5cm}|p{5cm}|}
% \hline
% Approach Category                  & Method                                               & Main Features                                                                                                          & Best Performance                                                            & Strengths                                                                                                                                                                                                                                   & Weaknesses                                                                                            \\ \hline
% \multirow{4}{*}{Hand-crafted}      & \cite{koopman2018detection}         & Photo Response Non-Uniformity                                                                                          & High correlation for bonafide images than deepfakes in a self-build dataset & A simple feature that can be efficiently extracted                                                                                                                                                                                          & Evaluations have been conducted using a self-build dataset. Cannot handle unseen deepfake categories.  \\ \cline{2-6} 
%                                    & \cite{kharbat2019image}             & Histogram of Gradient                                                                                                  & ACC=0.94 in UADFV dataset \cite{xie2020deepfake}                          & A simple feature that can be efficiently extracted                                                                                                                                                                                          & Can only handle face-swap deepfakes. Cannot handle unseen deepfake categories.                        \\ \cline{2-6} 
%                                    & \cite{xia2022towards}               & texture difference from the colour channels                                                                            & AUC=0.99 in FF++ dataset \cite{rossler2019faceforensics++}                           & The framework is interpretable.                                                                                                                                                                                                             & Can only handle face-swap deepfakes. Cannot handle unseen deepfake categories.                        \\ \cline{2-6} 
%                                    & \cite{wang2022ffr_fd}              & Speeded Up Robust Features (SURF), Scale-Invariant Feature Transform (SIFT), and Oriented Fast and Rotated Brief (ORB) & AUC=0.99 in DF-TIMIT(LQ) dataset \cite{korshunov2018deepfakes}                   & Can generalise to unseen datasets and different face deepfake generation methods                                                                                                                                                            & Can only handle face-swap deepfakes.                                                                  \\ \hline
% \multirow{9}{*}{Artefacts}         & \cite{yang2019exposing}             & 3D head pose                                                                                                           & AUC=0.89 in UADFV dataset \cite{xie2020deepfake}                          & The framework is interpretable.                                                                                                                                                                                                             & Can only handle face-swap deepfakes. Cannot handle unseen deepfake categories.                        \\ \cline{2-6} 
%                                    & \cite{xu2021deepfake}               & Gray-Level Co-occurrence Matrix                                                                                        & ACC=0.94 in DF-TIMIT(HQ) dataset \cite{korshunov2018deepfakes}                   & A simple feature that can be efficiently extracted                                                                                                                                                                                          & Can only handle face-swap deepfakes. Cannot generalise to unseen datasets.                            \\ \cline{2-6} 
%                                    & \cite{kingra2022lbpnet}             & Local Binary Pattern (LBP)                                                                                             & AUC=0.99 in FF++ \cite{rossler2019faceforensics++} dataset                           & A simple feature that can be efficiently extracted. Can be used to detect both face swap and face reenactment categories. Can generalise to unseen datasets and different face deepfake generation methods. The framework is interpretable. & Cannot handle unseen deepfake categories.                                                             \\ \cline{2-6} 
%                                    & \cite{agarwal2019protecting}        & head poses, facial landmarks, and expression                                                                           & AUC=0.96 in a self-build dataset                                            & The framework is interpretable. Can be used to detect both face swap and face reenactment categories.                                                                                                                                       & Evaluations have been conducted using a self-build dataset. Cannot handle unseen deepfake categories. \\ \cline{2-6} 
%                                    & \cite{nguyen2020eyebrow}            & eyebrow                                                                                                                & AUC 0.88 in Celeb-DF dataset \cite{Celeb_DF_cvpr20}                        & Consistent performance using this feature as the input to different backbone feature extractors                                                                                                                                             & Can only handle face-swap deepfakes. Cannot handle unseen deepfake categories.                        \\ \cline{2-6} 
%                                    & \cite{haliassos2021lips}            & mouth movement                                                                                                         & AUC=0.97 in DF1.0 \cite{jiang2020deeperforensics} dataset                          & Can be used to detect both face swap and face reenactment categories. Can generalise to unseen datasets and different face deepfake generation methods                                                                                      & The detection process is not interpretable                                                            \\ \cline{2-6} 
%                                    & \cite{qi2020deeprhythm}             & Skin colour                                                                                                            & Acc=0.98 in FF++ \cite{rossler2019faceforensics++}                                   & Can be used to detect high-resolution face deepfakes                                                                                                                                                                                        & Can only handle face reenactment deepfakes. Cannot handle unseen deepfake categories.                 \\ \cline{2-6} 
%                                    & \cite{fernandes2019predicting}      & oxygen concentration in the blood                                                                                      & Classification Loss of 0.0215 in a self-build dataset                       & Can be used to detect high-resolution face deepfakes                                                                                                                                                                                        & Can only handle face-swap deepfakes. Cannot handle unseen deepfake categories.                        \\ \cline{2-6} 
%                                    & \cite{ciftci2020fakecatcher}        & remote PhotoPlethysmoGraphy                                                                                            & Acc=0.97 in UADFV \cite{xie2020deepfake} dataset                          & Can be used to detect both face swap and face reenactment categories.                                                                                                                                                                       & Cannot generalise to unseen datasets.                                                                 \\ \hline

% \end{tabular}
% \end{table}
% \end{landscape}

% \begin{landscape}
% \begin{table}[htbp]
% \ContinuedFloat 
% \caption{(Continued.) Summary of face deepfake detection approaches}
% \begin{tabular}{|p{2cm}|p{.5cm}|p{5cm}|p{3cm}|p{5cm}|p{5cm}|}
% \hline
% Approach Category                  & Method                                               & Main Features                                                                                                          & Best Performance                                                            & Strengths                                                                                                                                                                                                                                   & Weaknesses                                                                                            \\ \hline

% \multirow{9}{*}{Deep Learning}     & \cite{afchar2018mesonet}            & Deep features extracted from a Capsule Network architecture                                                            & AUC=0.91 in a self-build dataset                                            & Can be used to detect both face swap and face reenactment categories.                                                                                                                                                                       & Cannot generalise to unseen datasets.                                                                 \\ \cline{2-6} 
%                                    & \cite{kumar2020detecting}           & Deep features from ResNet-18                                                                                           & Acc=0.99 in FF++ dataset \cite{rossler2019faceforensics++}                           & A simplified framework for deepfake detection                                                                                                                                                                                               & Can only handle face reenactment deepfakes. Cannot handle unseen deepfake categories.                 \\ \cline{2-6} 
%                                    & \cite{rana2020deepfakestack}        & Deep features from XceptionNet, MobileNet, ResNet101, InceptionV3, DensNet121, InceptionReseNetV2, and DenseNet169     & Acc=0.99 in a self-build dataset                                            & Can be used to detect both face swap and face reenactment categories.                                                                                                                                                                       & Cannot generalise to unseen datasets.                                                                 \\ \cline{2-6} 
%                                    & \cite{guera2018deepfake}            & Deep features from a CNN + LSTM framework                                                                              & Acc=0.97 in a self-build dataset                                            & A simplified framework for deepfake detection in videos                                                                                                                                                                                     & Can only handle face-swap deepfakes. Cannot handle unseen deepfake categories.                        \\ \cline{2-6} 
%                                    & \cite{nguyen2021learning}           & Deep features from a 3DCNN                                                                                             & Acc=0.99 in VidTIMID(HQ) \cite{sanderson2009multi} dataset                   & A simplified framework for deepfake detection in videos                                                                                                                                                                                     & Can only handle face-swap deepfakes. Cannot handle unseen deepfake categories.                        \\ \cline{2-6} 
%                                    & \cite{wodajo2021deepfake}           & Deep features from a Convolutional Vision-Transformer                                                                  & Acc=0.93 in FF++ dataset \cite{rossler2019faceforensics++}                           & Can be used to detect both face swap and face reenactment categories.                                                                                                                                                                       & Cannot generalise to unseen datasets.                                                                 \\ \cline{2-6} 
%                                    & \cite{mittal2020emotions}           & Deep learned emotion features extracted from audio and video                                                           & ACC=0.96 in DF-TIMIT(LQ) dataset \cite{korshunov2018deepfakes}                   & Can be used to detect both face swap and face reenactment categories. The framework is interpretable.                                                                                                                                       & Cannot generalise to unseen datasets.                                                                 \\ \cline{2-6} 
%                                    & \cite{chugh2020not}                 & Deep features from 3D- ResNetand audio features from Mel-Frequency Cepstral Coefficients                               & ACC=0.97 in DF-TIMIT(LQ) dataset \cite{korshunov2018deepfakes}                   & The framework is interpretable.                                                                                                                                                                                                             & Can only handle face-swap deepfakes. Cannot handle unseen deepfake categories.                        \\ \cline{2-6} 
%                                    & \cite{zhou2021joint}                & Deep Learned synchronisation features extracted from audio and video streams                                           & Acc=0.99 in FF++ dataset \cite{rossler2019faceforensics++}                           & Can be used to detect both face swap and face reenactment categories. Can generalise to unseen datasets. The framework is interpretable.                                                                                                    & Cannot handle unseen deepfake categories.                                                             \\ \hline
% \multirow{4}{*}{Anomaly Detection} & \cite{khodabakhsh2020generalizable} & logarithmic probability of observing a particular pixel's intensity                                                    & Acc=0.98 in FF++ dataset \cite{rossler2019faceforensics++}                           & Can be used to detect both face swap and face reenactment categories. Can generalise to unseen datasets. The framework is interpretable.                                                                                                    & Cannot handle unseen deepfake categories.                                                             \\ \cline{2-6} 
%                                    & \cite{wang2020exposing}             & local motion patterns                                                                                                  & Acc=0.98 in FF++ dataset \cite{rossler2019faceforensics++}                           & Can be used to detect both face swap and face reenactment categories. The framework is interpretable.                                                                                                                                       & Cannot generalise to unseen datasets.                                                                 \\ \cline{2-6} 
%                                    & \cite{khalid2020oc}                 & Video reconstruction                                                                                                   & F1=0.98 in DFD dataset \cite{bhat2024dfda}                             & Can generalise to unseen datasets.                                                                                                                                                                                                          & Can only handle face-swap deepfakes. Cannot handle unseen deepfake categories.                        \\ \cline{2-6} 
%                                    & \cite{cozzolino2023audio}           & Deep learned audio-visual features of authentic videos &  AUC=0.99 in DF-TIMIT dataset \cite{korshunov2018deepfakes}  & Can be used to detect both face swap and face reenactment categories.                                                                                                                                                                       & Cannot generalise to unseen datasets.                                                                 \\ \hline
% \end{tabular}
% \end{table}
% \end{landscape}


%\subsubsection{Learning objective and loss functions}
\subsubsection{Performance evaluation of deepfake detection methods}

For evaluating the efficacy of the face deepfake detection algorithms several different metrics have been used in the literature. Among them, Accuracy, Precision, Recall, F1-Score, Area Under the ROC Curve (AUC), and Error Rate are the most commonly used. For details please refer to the supplementary material section 4.4 on ``Introducing standard evaluation protocols''.


 

\begin{table*}[htbp]
\caption{Top 10 Tools to Detect Face Deepfakes}
 \resizebox{\textwidth}{!}{%
\begin{tabular}{|c|cccc|c|c|c|}
\hline
\multirow{2}{*}{Method}          & \multicolumn{4}{c|}{Type of Deepfake that it Can Detect}                                          & \multirow{2}{*}{Free} & \multirow{2}{*}{Open-source} & \multirow{2}{*}{URL}                              \\ \cline{2-5}
                                 & \multicolumn{1}{c|}{Image} & \multicolumn{1}{c|}{Audio} & \multicolumn{1}{c|}{Video} & Multimodal &                       &                              &                                                   \\ \hline
Sentinel                         & \multicolumn{1}{c|}{\cmark}     & \multicolumn{1}{c|}{\cmark}     & \multicolumn{1}{c|}{\cmark}     & \cmark          & \xmark                     & \xmark                            & https://thesentinel.ai/                           \\ \hline
Sensity                          & \multicolumn{1}{c|}{\cmark}     & \multicolumn{1}{c|}{\cmark}     & \multicolumn{1}{c|}{\cmark}     & \cmark          & \xmark                     & \xmark                            & https://sensity.ai/                               \\ \hline
Microsoft Video AI Authenticator & \multicolumn{1}{c|}{\cmark}     & \multicolumn{1}{c|}{\xmark }     & \multicolumn{1}{c|}{\cmark}     & \xmark          & \cmark                     & \xmark                            & https://blogs.microsoft.com                       \\ \hline
Deepware                         & \multicolumn{1}{c|}{\cmark}     & \multicolumn{1}{c|}{\xmark }     & \multicolumn{1}{c|}{\cmark}     & \xmark          & \cmark                     & \xmark                            & https://deepware.ai/                              \\ \hline
Intel's FakeCatcher              & \multicolumn{1}{c|}{\cmark}     & \multicolumn{1}{c|}{\xmark }     & \multicolumn{1}{c|}{\cmark}     & \xmark          & \cmark                     & \xmark                            & https://www.intel.com                             \\ \hline
DeepReal                         & \multicolumn{1}{c|}{\cmark}     & \multicolumn{1}{c|}{\xmark }     & \multicolumn{1}{c|}{\cmark}     & \xmark          & \cmark                     & \xmark                            & https://deepfakes.real-ai.cn/                     \\ \hline
CADDM                            & \multicolumn{1}{c|}{\cmark}     & \multicolumn{1}{c|}{\xmark }     & \multicolumn{1}{c|}{\cmark}     & \xmark          & \cmark                     & \cmark                            & https://github.com/megvii-research/CADDM          \\ \hline
ID-Reveal                        & \multicolumn{1}{c|}{\xmark }     & \multicolumn{1}{c|}{\xmark }     & \multicolumn{1}{c|}{\cmark}     & \xmark          & \cmark                     & \cmark                            & https://github.com/grip-unina/id-reveal           \\ \hline
Audio Visual Forensics           & \multicolumn{1}{c|}{\cmark}     & \multicolumn{1}{c|}{\cmark}     & \multicolumn{1}{c|}{\cmark}     & \cmark          & \cmark                     & \cmark                            & https://github.com/cfeng16/audio-visual-forensics \\ \hline
DuckDuckGoose                    & \multicolumn{1}{c|}{\cmark}     & \multicolumn{1}{c|}{\xmark }     & \multicolumn{1}{c|}{\cmark}     & \xmark          & \xmark                     & \xmark                            & https://www.duckduckgoose.ai/                     \\ \hline
\end{tabular}}
\label{tab:tool_det_face_deepfakes}
\end{table*}

\noindent\textbf{Top-ten tools for detecting face deepfakes:} Tab. \ref{tab:tool_det_face_deepfakes} summarises top-ten tools, including free and open-source tools, that can be leveraged for the detection of face deepfakes. It should be noted that for the ranking of these methods, we consider the accuracy of the detection, efficiency, the modalities that the detection algorithm can consider, and the availability of the source codes.


\hspace{2mm}
\subsection{Combating face deepfakes in face biometrics}
While it is difficult to fool physical biometric authentication systems using face deepfakes, online authentication systems such as mobile-based personal authentication systems can be fooled using state-of-the-art deepfake technology as they can counter liveliness detection method such as micro-muscle movements, eye-blinking patterns. For instance, a recent Gartner report \footnote{https://www.gartner.com/en/newsroom/press-releases/2024-02-01-gartner-predicts-30-percent-of-enterprises-will-consider-identity-verification-and-authentication-solutions-unreliable-in-isolation-due-to-deepfakes-by-2026} predicts that by 2026, due to face deepfakes, 30\% of enterprises will no longer be able to consider face biometric and authentication solutions to be reliable in isolation.
As such, it is important to investigate the ability of off-the-shelf face deepfakes to thwart state-of-the-art biometric recognition models. 

The following subsection provides a summary of the results of this investigation and we refer the reader to Sec. 3 of supplementary material for detailed comparisons.


\subsubsection{Summary of the efficacy of face deepfakes to fool face biometrics systems}
Evaluation results of face deepfakes on face biometrics are presented in Tab. 4 of the supplementary material and provide alarming evidence demonstrating the ability of both face swap and face reenactment methods to thwart state-of-the-art face recognition models. Our evaluations demonstrate that deepfake methods such as Wav2Lip, SimSwap, and first-order model are capable of fooling biometric recognition systems, especially the lightweight systems such as MobileNet. This vulnerability is of significant concern due to the vast utilization of lightweight face verification methods for authentication in applications such as mobile device unlocking, app login and payment gateways, and in social media apps for photo tagging.


\subsubsection{Measures for revealing true identity: } While significant research has been conducted in the area of deepface detection, solutions for reversing a faked face, resulting from manipulations, to recover the original real face,  are yet to be developed. \textbf{To date there has been only a single work on this topic.} This framework, introduced by Chang et. al \cite{chang2023cyber}, works using a pair of neural network modules, named, vaccinator and neutraliser for manipulation reversal. Within their conceptual framework the deepfake generator sits in between the vaccinator and neutraliser and these two models jointly attack the model in the middle. Specifically, the vacinator learns to synthesize the face region of an original image based on the mask and the neutraliser leverages this mask and the deepfake image in which the face region has been masked to reconstruct the original face. Due to the joint training of both vaccinator and the neutraliser, during the vaccination stage the vacinator injects identity-specific features which the neutraliser could leverage to recover the true identity. We discuss the area of revealing true identity of a manipulated face in Section 4 on Future Research Directions in the supplementary material.
% % \newpage
\section{Multimodal Deepfakes}
\label{sec:multimodal_deepfakes_intro} 

The concept of multi-modality in deep learning involves integrating and processing data from various sources simultaneously. These sources can encompass text, images, audio, video, and sensor data. By leveraging different data types, multi-modal deep learning models can capture more comprehensive and diverse information, resulting in enhanced performance for tasks that require understanding the relationships between different data types \cite{gao2020survey, summaira2021recent, jabeen2023review}. In the realm of deepfakes, multi-modality entails using various types of data, such as images, audio, and video, to create highly realistic synthetic media that convincingly mimics real-world content, including visuals and sounds \cite{khalid2021fakeavceleb, hou2024polyglotfake}. Through aligning and synchronizing these modalities, deepfakes can produce seamless and coherent fake content, such as matching a person's lip movements to a different audio track or convincingly cloning their voice \cite{pei2024deepfake, prajwal2020lip, cheng2022videoretalking, lomnitz2020multimodal}. While this technology offers positive applications in entertainment, media, and education, such as creating special effects and developing realistic training simulations, it also poses significant ethical and security challenges \cite{pandey2021deepfakes}. These include the potential for misuse in misinformation, impersonation, and fraud. Detecting and preventing malicious deepfakes is a burgeoning area of research aimed at ensuring the responsible use of this powerful technology. As multi-modal deepfakes continue to evolve, it is crucial to balance innovation with ethical considerations to mitigate risks and maximize benefits \cite{khalid2021evaluation, liu2023magnifying, cheng2023voice}. In this section, we will investigate state-of-the-art methodologies for the generation (refer Section \ref{subsec:multimodal_generation}) and the detection of multi-modal deepfakes (refer Section \ref{subsec:multimodal_detection}). We will analyze the advanced and innovative techniques outlined in the existing literature, alongside the datasets utilized for deepfake detection and generation.

\subsection{Multimodal Deepfake Generation}
\label{subsec:multimodal_generation}

Combining audio and video deepfakes involves a sophisticated process of synchronizing lip movements with synthetic speech to produce seamless and coherent content \cite{liz2024generation}. This multi-modal approach, which integrates both visual and auditory elements, significantly enhances the realism of the generated media. By ensuring that the audio matches the lip movements and facial expressions perfectly, these deepfakes become more lifelike and convincing, making detection increasingly challenging \cite{hou2024polyglotfake}. Audio-visual multimodal deepfakes can be categorized into three main types based on the modality being faked \cite{khalid2021fakeavceleb}. This categorization helps in understanding the different ways in which deepfakes can manipulate audio and visual components to create convincing forgeries. Understanding these categories is important for recognizing and combating the potential misuse of deepfake technology.

\paragraph{Fake Video and Real Audio}
\label{para:multimodal_fakeV_realA}

Fake video and real audio deepfakes involve manipulating visual content while retaining the original audio, creating a synthetic video that depicts events or actions that never actually occurred. By keeping the original audio, which includes the true voice, tone, and speech patterns of the person, these deepfakes gain an added layer of credibility and can be particularly persuasive. The process of creating fake videos with real audio often involves altering the appearance, expressions, or actions of individuals in the video \cite{karras2019style, nirkin2019fsgan, korshunova2017fast}. For instance, face swapping \cite{korshunova2017fast} can replace the subject’s face with someone else's, or visual effects can be used to create entirely new scenes that seem authentic. The combination of real audio and fake video poses significant challenges for detection, as the genuine audio can make the fabricated visuals appear more believable. Detecting these deepfakes involves analyzing inconsistencies between the audio and video elements. Techniques such as temporal analysis of facial expressions, detecting unnatural movements, and scrutinizing visual artifacts are crucial \cite{kaur2020deepfakes, liu2023ti2net}.

\paragraph{Real Video and Fake Audio}
\label{para:multimodal_realV_fakeA}

In this type of deepfake, the video remains unaltered, while the audio is synthetically generated to mislead the audience about what is being said. This method generally involves using text-to-speech models and voice cloning techniques to create synthetic speech that closely mimics the vocal characteristics of a specific person \cite{deng2020unsupervised, kinnunen2017asvspoof, polyak2019tts}. By manipulating the audio, these deepfakes can make it seem like the person in the video is saying something they never actually said. This technique is particularly dangerous because the genuine video lends credibility to the fake audio, making it more convincing and harder to detect. Examples of using real video with fake audio include creating false news reports, tampering with evidence in legal cases, and producing deceitful content for political propaganda \cite{sankaranarayanan2021presidential}. Detecting such deepfakes requires a comprehensive analysis of audio-visual synchronization to identify discrepancies between lip movements and speech \cite{agarwal2020detecting}. Additionally, it demands the development of robust audio forensic techniques to scrutinize voice patterns and identify synthetic anomalies \cite{almutairi2022review}.

\paragraph{Fake Video and Fake Audio}
\label{para:multimodal_fakeV_fakeA}

These deepfakes manipulate visual content to depict events or actions that never occurred, while also synthesizing audio to accompany the fabricated visuals.  This allows for a wider range of possible sample alterations and a variety of manipulation techniques. For example, a deepfake could depict a person giving a speech they never delivered, with the voice and the video being entirely fabricated. Achieving realism in both modalities requires training models on large datasets of real audio and video to learn the nuances of human speech, facial expressions, and body movements. The challenge lies in creating seamless synchronization between the audio and video components to make the deepfake indistinguishable from genuine content. Furthermore, standardized multimodal deepfake datasets serve as benchmarks for evaluating the performance of detection algorithms \cite{dolhansky2020deepfake, khalid2021fakeavceleb, hou2024polyglotfake}. They offer a common ground for researchers to compare different approaches, facilitating the identification of the most effective methods for detecting deepfakes \cite{liu2023magnifying, cheng2023voice, feng2023self}. This benchmarking is vital for pushing the boundaries of deepfake detection technology, ensuring that the algorithms can generalize well across different types of deepfakes and are not limited to specific scenarios or formats.

\subsection{Multimodal Deepfake Detection Datasets}
\label{subsec:multimodal_datasets}

Multimodal deepfake datasets are crucial for advancing the understanding and detection of deepfakes, which involve the manipulation of multiple types of data such as audio, video, and text to create convincingly fake content. These datasets offer diverse examples of synthetic media that combine various modalities, accurately reflecting the complex, real-world scenarios where deepfakes are likely to be encountered. This diversity is indispensable for training sophisticated detection techniques, primarily deep learning-based networks, to recognize and detect deepfakes across various scenarios and formats. By including examples that span different combinations of audio, video, and text manipulations, multimodal datasets allow researchers to develop and refine algorithms that can analyze the consistency and coherence between these modalities.

Multimodal deepfake datasets are essential for advancing both the understanding and detection of deepfakes, which involve manipulating multiple types of data, such as audio and video, to create compelling fake content. These provide diverse examples of synthetic media, combining various modalities like audio, video, and text reflecting real-world scenarios where deepfakes are likely to be used. This diversity is essential for training sophisticated detection techniques (mainly deep learning-based networks) to recognize and detect deepfakes across different scenarios and formats. These multimodal datasets enable researchers to develop algorithms that analyze the consistency and coherence between modalities. Furthermore, standardized multimodal deepfake datasets offer benchmarks for evaluating the performance of detection algorithms. They provide a common ground for comparing different approaches and identifying the most effective methods for detecting multimodal deepfakes.

\begin{table*}[htbp]
\caption{\textcolor{black}{Multimodal Deepfake Datasets}}
\centering
\resizebox{\textwidth}{!}{%
\begin{tabular}{|P{67pt}|P{42pt}|P{40pt}|P{35pt}|P{35pt}|P{100pt}|P{100pt}|}
\hline
\multirow{2}{*}{Dataset} & \multirow{2}{*}{Multilingual} & \multirow{2}{*}{Subjects} & \multicolumn{2}{c|}{Samples} & \multicolumn{2}{c|}{Manipulation Techniques} \\\cline{4-7}
&  &  & Real & Fake & Video & Audio\\\hline\hline
DFDC \cite{dolhansky2020deepfake} & No  & 960 & 104,500 & 23,654 & DFAE, MM/NN face swap, NTH, FSGAN \cite{nirkin2019fsgan}, StyleGAN \cite{karras2019style} & TTS-Skins \cite{polyak2019tts}\\\hline
FakeAVCeleb \cite{khalid2021fakeavceleb} & No  & 500 & 500 & 19,500 & FaceSwap \cite{korshunova2017fast}, Wav2Lip \cite{prajwal2020lip}, FSGAN & SV2TTS \cite{jia2018transfer}\\\hline
VideoSham \cite{mittal2023video} & No & - & \textcolor{black}{413} & \textcolor{black}{413} & \multicolumn{2}{c|}{Manual manipulations by professional video editors (6 types)} \\\hline
PDD \cite{sankaranarayanan2021presidential} & No & 2 & 16 & 16 & Wav2Lip & Manipulated content recorded by voice actors \\\hline
LAV-DF \cite{cai2022you} & No & 153 & 36,431 & 99,873 & Wav2Lip & SV2TTS \\\hline
MMDFD \cite{asha2023mmdfd} & No  & 50 & 1,500 & 5,000 & FaceSwap, FSGAN, Wav2Lip, DeepFaceLab \cite{perov2020deepfacelab} & SV2TTS (AurisAI \cite{aurisaiAurisFree} for Text)\\\hline
DefakeAVMiT \cite{10081373} & No  & 43 & \multicolumn{2}{c|}{6,480} & FaceSwap, DeepFaceLab, EVP \cite{ji2021audio}, Wav2Lip, PC-AVS \cite{zhou2021pose}  & SV2TTS, Voice Replay \cite{kinnunen2017asvspoof}, AV exemplar autoencoders \cite{deng2020unsupervised} \\\hline
PolyGlotFake \cite{hou2024polyglotfake} & Yes  & 766 & 766 & 14,472 & VideoRetalking \cite{cheng2022videoretalking}, Wav2Lip & XTTS \cite{Gölge2024coqui}, Bark \cite{Kucsko2024suno} + FreeVC \cite{li2023freevc}, Tacotron \cite{wang2017tacotron} + FreeVC, MicrosoftTTS \cite{microsoftTextSpeech} +FreeVC, Vall-E-X \cite{wang2023neural}\\\hline
\end{tabular}}
\label{tab:multimodal_datasets}
\end{table*}

As illustrated in Table \ref{tab:multimodal_datasets}, the manipulation of modalities to generate multimodal deepfakes was accomplished using state-of-the-art deepfake generation techniques available at the time of the dataset release. The quality of the synthetic content is contingent upon the strengths and limitations of these techniques. Advances in sophisticated deep learning methodologies over time have yielded increasingly realistic fake content. Consequently, the techniques employed for multimodal deepfake generation are further elaborated in Section \ref{subsec:multimodal_generation}.

\subsection{Multimodal Deepfake Generation}
\label{subsec:multimodal_generation}

In the following section, we will delve into the latest and most widely utilized techniques for producing synthetic audio-visual content by synchronizing lip and/or face movements, drawing from state-of-the-art research in the field.

\paragraph{LipGAN}
\label{para:multimodal_lipgan}

Prajwal \textit{et al.} \cite{kr2019towards} proposed a GAN-based lip synchronization model called LipGAN. This model is capable of handling faces in random poses without the need for realignment to a template pose. Furthermore, LipGAN enables the generation of realistic talking face videos through an automated pipeline for face-to-face translation from any audio, without dependence on language.

\paragraph{Wav2Lip}
\label{para:multimodal_wav2lip}

Most state-of-the-art methods excel at generating accurate lip movements for static images or videos of specific individuals seen during the training phase. However, these methods often fail to accurately morph lip movements for arbitrary identities in dynamic, unconstrained talking face videos, resulting in significant portions of the video being out-of-sync with the new audio. As discussed by Prajwal et al. in \cite{\cite{prajwal2020lip}}, this failure can be primarily attributed to limitations in both training objectives (i.e., loss functions used) and lip-sync discriminators. Typically, the face reconstruction loss is computed for the detected facial region to ensure correct pose generation and identity preservation. However, the lip region constitutes a very small proportion of the face region (or the entire frame), causing the total reconstruction loss (often L1 distance) to be less impacted by the lip region due to its limited spatial extent. This can adversely affect the learning process, where the reconstruction of the surrounding image is prioritized over the lip region. To address this issue, specific discriminators, such as those used in LipGAN \cite{kr2019towards}, are employed to evaluate lip-sync accuracy. Prajwal et al. \cite{prajwal2020lip} emphasized the importance of short temporal context in detecting lip-sync discrepancies and incorporated this concept into the development of the Wav2Lip framework. They demonstrated that considering a short temporal context significantly improves lip-sync accuracy. Additionally, they noted that introducing artifacts such as pose variations during GAN training can negatively impact lip-sync performance. This degradation occurs because the discriminator may fail to focus on the correspondence between the video and lip movements, underscoring the need for discriminators specifically designed to evaluate lip-sync quality. Prajwal \textit{et al.} utilises a customised SyncNet-based \cite{chung2017out} discriminator to mitigate the limitations mentioned above, which is substantially more accurate than the previous methodologies. Furthermore, to increase the quality of the morphed regions in the reconstructed images, they have also utilised a visual-quality discriminator. This technique has been used by most of the above multimodal datasets to generate faked audio-visual content.

\paragraph{FaceRetalking}
\label{para:multimodal_faceretalking}

Cheng \textit{et al.} \cite{cheng2022videoretalking} highlighted that the state-of-the-art techniques often omit the original lip motion changes or retiming the background to avoid unnatural movements between the head pose and lip \cite{prajwal2020lip, song2022everybody}. However, in the FaceRatalking method, the lower half face (not only the lip) Cheng \textit{et al.} emphasized that existing state-of-the-art approaches often overlook authentic lip motion alterations or adjust the background timing to prevent unnatural movements between head pose and lip motion. In contrast, the FaceRatalking technique not only modifies the lip region but also encompasses editing of the entire lower half of the face, incorporating facial movements through an innovative face reenactment process. Additionally, they identified an information leakage in conditional in-painting-based methods when the original frame was utilized as the conditional image for lip synchronization. To remedy this, they introduced a semantic-guided reenactment network to alter the expression of the entire lower half of the face, producing an enriched frame with a consistent expression, which then served as the basis for subsequent lip synthesis. The lip synthesis network in the FaceRetalking approach incorporates a conditional inpainting-inspired network \cite{prajwal2020lip}. This network leverages pre-processed frames from the face reenactment network as the identity and structure reference, along with the audio and the masked original frames as the condition resulting in a highly effective method for synthesizing a lip-syncing video based on the input audio. They argued that even though the synthesized videos accurately depict lip movements, the visual quality is limited due to low-resolution training data. To tackle this issue, they developed an identity-preserving face enhancement network to improve output quality through progressive training. When compared to Wav2Lip \cite{prajwal2020lip} which mainly focuses only on lip synchronization, FaceRetalking provides a broader facial synthesis that includes expressions and head movements, enabling more realistic fake content.

\paragraph{Diff2Lip}
\label{para:multimodal_diff2lip}

In their work, Mukhopadhyay \textit{et al.} \cite{mukhopadhyay2024diff2lip} proposed Diff2Lip, which is an audio-conditioned diffusion model used for inpainting producing precise and natural lip sync by focusing on the fine details of lip movements. This model can achieve lip synchronization in real-world scenarios while preserving identity, pose, emotions, and image quality. The Diff2Lip model takes three inputs: a masked input frame (providing pose context), a reference frame (containing identity and mouth region textures), and an audio frame (used to drive the lip shape). It then outputs the lip-synced mouth region. The multimodal conditional diffusion implemented in the model allows for a fine balance between all contextual input information, effectively avoiding lip-sync problems. However, when compared with the FaceRetalking model, the FaceRetalking may provide a better integrated facial performance, enhancing overall expressiveness and realism beyond just the lips. However, through the leverage of the diffusion process, Diff2Lip can enhance the naturalness and fluidity of lip movements over time compared to similar lip synthesis methods such as Wav2Lip.


\subsection{Performance Evaluation in Deepfake Generation}
\label{subsec:multimodal_lossfunc}


Performance evaluation is a critical step in multimodal deepfake generation, providing a comprehensive framework to ensure that the generated content is realistic, high-quality, coherent, and consistent across different modalities. Various performance evaluation metrics are employed in the literature, encompassing both application-specific and generalized measures. In this section, we discuss some of the most widely considered performance metrics and their roles in assessing the quality and consistency of multimodal deepfake generation models.

\paragraph{The Fréchet Inception Distance (FID)}

The Fréchet Inception Distance (FID) serves as a crucial metric for evaluating the quality of generated images \cite{nunn2021compound, singh2020using, yu2021artificial}. By comparing the feature distribution of generated images with real images, it provides valuable insights into both the fidelity and diversity of the generated images. Lower FID values signify higher similarity to real images, indicating superior quality. An FID of 0 implies that the generated images are indistinguishable from real images in terms of their feature distributions. However, the FID score is sensitive to the choice of feature extractor. The FID score can be calculated as in Equation \ref{eq:multimodal_fid} where $\mu_r$, $\mu_g$, $\Sigma_r$, $\Sigma_g$ and $T_r$ represent the mean of real image features, mean of generated image features, covariance or real image features, covariance of generated image features and Trace of the matrix respectively \cite{nunn2021compound}. 

\begin{equation}
    FID = ||\mu_r - \mu_g||^2 + Tr(\Sigma_r+\Sigma_g - 2(\Sigma_r\Sigma_g)^{0.5})
    \label{eq:multimodal_fid}
\end{equation}

\paragraph{Structural Similarity Index (SSIM)}

SSIM compares two images by analyzing their structure, luminance, and contrast. Its goal is to provide a more accurate measure of perceptual image quality compared to simpler metrics like Mean Squared Error (MSE). SSIM is specifically designed to take human perception into account when assessing image quality, making it a more reliable method \cite{sun2020landmark, dagar2022literature, husseini2023comprehensive}. It does not consider higher-order image statistics, which may be important for certain aspects of image quality. However, it's important to note that SSIM is a pixel-wise image similarity metric that compares two images and may not be the best choice for capturing variability in video generation \cite{shrivastava2021diverse}. The SSIM can be calculated as in Equation \ref{eq:multimodal_ssim} where $c_1$, and $c_2$ represent constants to stabilize the division with a weak denominator \cite{wang2004image}.

\begin{equation}
    SSIM(x,y) = \frac{(2\mu_x \mu_y + c_1)(2\sigma_{xy}+c_2}{(\mu_x^2+\mu_y^2+c1)(\sigma_x^2+\sigma_y^2+c_2)}
    \label{eq:multimodal_ssim}
\end{equation}

\paragraph{Lip Movement Distance (LMD)}

Lip Movement Distance (LMD) is an important metric used to assess how well the movement of lips matches the corresponding audio, especially in the context of creating deepfakes. It measures the spatial difference between the lip positions in the generated frames and the actual frames, providing a quantitative evaluation of the accuracy of lip movements to the spoken audio \cite{chen2018lip}. However, the accuracy of LMD depends on the reliability of the facial landmark detection model used to extract lip positions. Additionally, since LMD focuses on spatial alignment, it may not fully capture the temporal dynamics and smoothness of lip movements over time. The LMD over $N$ frames can be calculated as in Equation \ref{eq:multimodal_lmd} where $L_t^{gen}$, and $L_t^{gt}$ represent lip landmarks for the generated and ground truth images respectively \cite{chen2018lip}.

\begin{equation}
    LMD = \frac{1}{N} \Sigma_{t=1}^n ||L_t^{gen} - L_t^{gt}||
    \label{eq:multimodal_lmd}
\end{equation}

\paragraph{LSE-C and LSE-D}

LSE-C and LSE-D are two important metrics used to assess the performance of the Wav2Lip model \cite{prajwal2020lip}, which is used to achieve synchronisation of audio-manipulated talking face videos. This model ensures that the lip movements in a video align with the corresponding audio. The LSE-D error (Lip-Sync Error-Distance), measures the misalignment between audio and visual streams in terms of lip synchronization, and a lower LSE-D denotes a higher audio-visual match \cite{prajwal2020lip}. LSE-C (Lip-Sync Error-Confidence) is the confidence score. A higher score suggests a better audio-visual correlation and more accurate alignment of lip movements with audio. Prajwal \textit{et al.} calculated these error metrics based on SyncNet \cite{chung2017out} extracted features from both the audio and visual inputs where the visual features are typically derived from the region around the lips, while audio features are extracted from the corresponding audio segment. However, poor feature extraction can lead to inaccurate error measurement, and may not capture the naturalness of continuous speech synchronization. These losses have been widely used in later research to validate the performance of lip-speech synchronisation \cite{wang2022attention, zhang2022meta, lu2022visualtts}. 

\paragraph{Peak Signal-to-Noise Ratio (PSNR)}

PSNR, which stands for Peak Signal-to-Noise Ratio, is a metric used to measure the quality of a reconstructed or generated image in comparison to a reference image \cite{huang2020fakeretouch, huang2020fakepolisher, wang2021faketagger}. It quantifies the level of distortion or noise introduced during the generation process, with higher PSNR values indicating better quality and less distortion. PSNR is easy to compute and understand, unlike most other metrics, providing a straightforward measure of image and video quality. However, it's important to note that PSNR does not always align well with human visual perception, and high PSNR values do not guarantee that the image will look good to human observers. PSNR can be calculated as in Equation \ref{eq:multimodal_psnr} where $MAX$ and $MSE$ refer to the maximum pixel value (can be either 1 or 255 depending on whether the input image is in double-precision floating-point or 8-bit unsigned integer format) and the Mean Squared Error (MSE) between the reference video frame (or image) and the generated frame.

\begin{equation}
    MSE = \frac{\Sigma_{M,N}[I_1(m,n)-I_2(m,n)]^2}{M*N}
    \label{eq:multimodal_psnr}
\end{equation}

\begin{equation}
    PSNR = 10.log_{10} (\frac{MAX^2}{MSE})
    \label{eq:multimodal_psnr}
\end{equation}

% \paragraph{Lip Sync Error Rate (LSER)}

% Lip sync error rate is the measure of how often or to what extent the audio track of a video does not match up correctly with the visual track of the speaker's lip movements. Various techniques can be used to calculate this metric, all of which assess how well the created lip movements synchronize with the accompanying audio, ensuring that the visual and audio components are convincingly synchronized. 

The performance metrics mentioned above can be divided into two main categories: lip-sync rate-related measures and realism-related measures. Lip-sync is essential for applications that require precise audio-visual alignment, focusing on synchronizing audio and lip movements. Realism is important for applications that require high visual fidelity and overall believability of the generated content. It takes into account not only lip synchronization, but also other factors such as facial expressions, eye movements, skin texture, and lighting. Lip-sync rate can be measured using metrics like LSE-C, LSE-D, and LMD, while realism can be measured through PSNR, SSIM, FID, and perceptual human evaluations (such as mean opinion score (MOS) \cite{lu2022visualtts, hou2024polyglotfake}).

\paragraph{BRISQUE}

BRISQUE is an image quality assessment (IQA) model that can evaluate the quality of an image without needing a reference image \cite{mittal2012no}. It works by analyzing natural scene statistics in the spatial domain, directly examining pixel values without transforming the image into a different domain, such as the frequency domain. The model employs statistical features derived from natural scene statistics to capture deviations from natural image properties, indicating distortions and quality degradation. This makes it useful for detecting deepfakes, as it does not rely on a real image for comparison once it's deployed in the world \cite{hou2024polyglotfake, yang2020deepfake}. 


\subsection{Multimodal Deepfake Generation Tools}

\textcolor{black}{A Deepfake Generation Tool is a sophisticated software or system designed to synthesize and create realistic media content across multiple modalities—primarily video and audio. These tools leverage advanced artificial intelligence (AI) and deep learning techniques to generate highly convincing synthetic content, often using generative models like GANs (Generative Adversarial Networks) or diffusion models. Deepfake generation tools can create content that closely mimics real media, making it challenging to distinguish between authentic and synthetic.}

\begin{table*}[htbp]
\caption{Top Tools to Detect Multimodal Deepfakes}
\centering
\resizebox{\textwidth}{!}{%
\begin{tabular}{|c|c|c|c|c|}
\hline
Method & Free & Open-source & URL\\\hline\hline
Synthesia  & \xmark & \xmark & https://www.synthesia.io/ \\\hline
AI Studios by DeepBrain AI  & \xmark & \xmark & https://www.aistudios.com/ \\\hline
Creative Reality Studio by D-ID  & \xmark & \xmark & https://www.d-id.com/creative-reality-studio/ \\\hline
Elai.io  & \xmark & \xmark & https://elai.io/ \\\hline
\end{tabular}}
\label{tab:tool_det_multimodal_deepfakes}
\end{table*}

\textcolor{black}{The rise of accessible deepfake generation tools has revolutionized content creation in several areas, particularly in Training & Education, E-commerce, Social Media, and Customer Support. These platforms use advanced AI to generate realistic videos based on text prompts, featuring avatars that can speak, express emotions, and perform subtle facial gestures. Designed mainly for professional use, these tools allow users to create customized video content at scale without the need for complex equipment, technical expertise, or high production costs. One of the most notable features of these tools is their user-friendly interface, which enables non-technical users and beginners to quickly produce sophisticated video content. Most of these platforms utilize deep learning and computer vision techniques to animate pre-existing avatars according to user-provided text prompts. The avatars are often photorealistic digital representations of real individuals, providing a high level of realism.}

\subsection{Multimodal Deepfake Detection}
\label{subsec:multimodal_detection}

Multimodal deepfake detection is an important area of research that focuses on identifying synthetic media manipulating multiple modalities, such as video, audio, and text, to create highly realistic and deceptive content \cite{raza2023multimodaltrace, katamneni2023mis, 10081373}. Unlike traditional deepfakes that target a single modality, multimodal deepfakes integrate alterations across several types of data, making them more sophisticated and harder to detect. Detection approaches typically use machine learning and deep learning algorithms, which are trained to recognize subtle anomalies in the synchronized behaviour of different modalities. For example, discrepancies between lip movements and speech, unnatural facial expressions, or inconsistencies in lighting and shadows can signal the presence of a deepfake \cite{lewis2020deepfake}. As deepfake technology continues to evolve, detection methods must also evolve to ensure the integrity and authenticity of digital media in an increasingly digital and interconnected world. In this section, we highlight the latest advancements in multi-modal deepfake detection technology.

\Rotatebox{90}{
\centering
\renewcommand{\arraystretch}{1}
\caption{\textcolor{black}{Multimodal Deepfake Detection Approaches}}
\begin{tabular}{|P{70pt}|P{120pt}|P{120pt}|P{130pt}|P{85pt}|}
\hline
\multirow{2}{*}{Method} & \multicolumn{2}{c|}{Feature Extractors} & \multirow{2}{*}{Technical Novelty} & \multirow{2}{*}{Performance} \\\cline{2-3}
& Audio & Video & & \\\hline\hline
Multimodaltrace \cite{raza2023multimodaltrace} & Resnet-1D on positive frequencies of FFT features & Multilayered 3D ResNet on normalised stacked video frames & Independent and joint feature learning through IntrAmodality Mixer Layer (IAML) and IntErModality Mixer Layer (IEML) & FakeAVCeleb: 92.9\% PDD: 70\% \\\hline
MIS-AVoiDD \cite{katamneni2023mis} & MFCC & MTCNN for face detection and Xception-based for feature extraction & Joint use of modality-invariant and specific representations to ensure both common and unique patterns of real or fake content are preserved and fused &  FakeAVCeleb: 96.2\%(Accuracy) 0.973(AUC) \\\hline
AVoiD-DF \cite{10081373} & transformer network on Mel-spectrograms of audio & transformer network on visual frames & Multimodal temporal \& spatial encoder (TSE) with multimodal joint decoder (MMD)  &  FakeAVCeleb: 83.7\%(Accuracy) 0.892(AUC) \\\hline
PVASS-MDD \cite{yu2023pvass} & VGGish network on log mel-spectrograms of audio \cite{hershey2017cnn} & MTCNN for face extraction and a Swin-Transformer \cite{liu2022video} & Cross-modal predictive VA alignment module  &  FakeAVCeleb: 84.3\%(Accuracy) 0.875(AUC) \\\hline
Emotions Don't Lie \cite{mittal2020emotions} & 13 MFCC features from pyAudioAnalysis \cite{giannakopoulos2015pyaudioanalysis} + DCNNs for modality encoding and perceived emotion encoding & facial features from OpnFace (430-D) \cite{amos2016openface} + DCNNs for modality encoding and perceived emotion encoding & Comparison of affective cues corresponding to perceived emotion to infer whether the video is manipulated & DFDC: 84.4\% \\\hline
AVAD \cite{feng2023self} & \multicolumn{2}{C{160pt}|}{Audio-visual synchronisation model  as in \cite{chen2021audio}. CNN-based feature encoders for visual frames and audio spectrograms and transformer as synchronisation module.} & Video forensics posed as an audio-visual anomaly detection problem and learning only on real videos  &  FakeAVCeleb: 87.9\%(AP) 0.900(AUC) \\\hline
VFD \cite{cheng2023voice} &  \multicolumn{2}{C{160pt}|}{Deep forward convolutional projection on the spectrogram and visual frames + transformer-like network to learn identity-related features} & A face-voice matching technique that measures homogeneity between the audio and video to identify deepfakes  &  FakeAVCeleb: 81.52\%(Accuracy) 0.8611(AUC) DFDC: 80.96\%(Accuracy) 0.8513(AUC)  \\\hline
Capsule Forensics (score fusion) \cite{muppalla2023integrating} & Capsule network on Mel-spectrograms of audio & Capsule network on MTCNN extracted face regions &  Multimodal score-fusion capable of identifying inconsistencies across
various deepfake types and artifacts within each modality  &  FakeAVCeleb: 99.2\%(Accuracy) 0.993(AUC) \\\hline
FCMT + DDIC \cite{liu2023magnifying} &  Audio Forgery Clues Magnification Transformer (FCMT) & Video FCMT &  FCMT module to capture intra-modal artifacts from different modalities by magnifying forgery clues + image spatial artifacts magnification with DDIC  &  FakeAVCeleb: 99.13\%(Accuracy) 0.9927(AUC) DFDC: 98.45\%(Accuracy) 0.9903(AUC)  \\\hline
\end{tabular}
\label{tab:multimodal_detectiondatasets}
}

To effectively detect realistic deepfakes, it is crucial to address both audio and video manipulation. This can be achieved either by independently detecting audio and video cues in deepfakes or through a combined approach that leverages joint audiovisual representation learning. Raza \textit{et al.} \cite{raza2023multimodaltrace} introduced a unified multimodal framework called "Multimodaltrace" which extracts learned feature representations from both audio and visual data, processing these elements separately before integrating them using an innovative multimodal fusion technique. Furthermore, they proposed a novel reformulation of the audiovisual deepfake detection problem, framing it as a multi-label classification task. This new approach predicts confidence levels across both audio and visual modalities, offering a more nuanced and effective method for identifying deepfakes. In their study, Katamneni et al.\cite{katamneni2023mis} focused on fusing modality invariant and specific feature representations for audio and visual streams. This method is similar to previous approaches but employs a different combination of regularization and learning objectives (modality invariant loss, modality-specific loss, and orthogonal loss), leading to improved results. Yang \textit{et al.} \cite{10081373} have proposed an innovative approach for detecting deepfakes by using audio-visual joint learning (AVoiD-DF) which leverages audio-visual inconsistencies for multi-modal forgery detection. The process begins by embedding temporal-spatial information in a Temporal-Spatial Encoder (TSE) to obtain temporal-spatial inconsistency between audio-visual signals (real and fake can exist across frames along the temporal dimension). It is then followed by a Multi-Modal Joint Decoder (MMD) to fuse multi-modal features and learn inherent relationships concurrently. Finally, a Cross-Modal Classifier is developed to detect manipulation by detecting inter-modal and intra-modal disharmony. Furthermore, to test the effectiveness of the proposed deepfake detection model in real-world scenarios, the researchers introduced DefakeAVMiT \cite{10081373}, a multimodal deepfake dataset where various forgery techniques have been applied to different modalities.

In the PVASS-MDD framework proposed by Yu \textit{et al.} \cite{yu2023pvass}, there are two main modules: an auxiliary PVASS stage that focuses on exploring common correspondences between video and audio (AV) and a cross-modal predictive VA alignment module (MDD). The PVASS module works by iteratively predicting audio features using visual features and then reconstructing visual features based on audio features and prediction errors to eliminate discrepancies between video and audio. In the MDD stage, the frozen PVASS network from the first stage is used to align the VA features of real videos, enabling the detection network to better learn the inconsistencies between video and audio in deepfake videos. This MDD stage, with the assistance of PVASS, can extract more accurate VA inconsistencies for multimodal deepfake detection. When it comes to detecting deepfakes, Feng \textit{et al.} \cite{feng2023self} took a different approach compared to other techniques. Instead of treating it as a classification problem, they looked at it as an anomaly detection problem. They analyzed the distribution of audio-visual examples and flagged those with low probability. They focused on subtle properties that manipulated videos are unlikely to accurately capture. They used three unique features for audio-visual anomaly detection: discrete time delay, time-delay distribution, and audio-visual network activations. They found that time-delay distribution is more meaningful for anomaly detection than time-delay alone. They also studied the effect of feature activations within the audio-visual synchronization network on anomaly detection. The results showed that manipulated videos can be detected by identifying unlikely sequences of these features based on a learned distribution. Mittal \textit{et al.} \cite{mittal2020emotions} proposed an innovative method for detecting alterations in videos, such as deepfakes. This approach utilizes both audio (speech) and video (face) data and extracts emotional features from both modalities. The method uses a Siamese network-based architecture (triplet learning) to process real and deepfake videos at the same time during training. It generates modality and perceived emotion embedding vectors for the subject's face and speech, which are then used to distinguish between real and fake content. Through experiments, the study demonstrated that the perceived emotion cues from both modalities play a crucial role in detecting deepfake content by assessing the similarity between modality signals. 

Cheng \textit{et al.} \cite{cheng2023voice} investigated using voice-face matching to detect deepfake videos. Their empirical results indicated that the identities behind voices and faces are often mismatched in deepfake videos and that voices and faces have some level of homogeneity. They detected deepfakes by examining the intrinsic correlation of facial and audio information, without using any additional auxiliary data such as more modalities or visual features. Muppalla \textit{et al.} \cite{muppalla2023integrating} utilised capsule networks to extract robust features from audio spectrograms and face visuals followed by multimodal fusion and classification for deepfake detection. They utilised both score-fusion and feature-fusion approaches, which substantially improved over the state-of-the-art methods. In their recent work, Liu \textit{et al.} \cite{liu2023magnifying} introduced an innovative multimodal Deepfake detection framework that enhances intra-modal and cross-modal forgery clues. The framework consists of several key modules. Firstly, the Forgery Clues Magnification Transformer (FCMT) module is proposed to capture temporal intra-modal defects by magnifying forgery clues based on sequence-level relationships. Additionally, a Distribution Difference Inconsistency Computing (DDIC) module, based on Jensen–Shannon divergence, is used to adaptively align multimodal information for further magnifying the cross-modal inconsistency. The framework also explores spatial artifacts by connecting multi-scale feature representation to provide comprehensive information. Finally, a feature fusion module is designed to adaptively fuse features to generate a more discriminative feature. Experimental results showed that the proposed framework outperforms independently trained models and demonstrated a superior generalisation on unseen types of Deepfake. The overall performance of the selected deepfake detection techniques on FakeAVCeleb dataset \cite{khalid2021fakeavceleb} is illustrated in Figure \ref{fig:multimodal_summary} in terms of accuracy and area under the curve (AUC) score.

\begin{figure}
    \centering
    \includegraphics[width=.5\linewidth]{figures_new/FakeAVCeleb.pdf}
    \caption{\textcolor{black}{Performance variation of recent state-of-the-art methods in detecting deepfakes on FakeAVCeleb dataset in terms of Accuracy and AUC.}}
    \label{fig:multimodal_summary}
\end{figure}

\textcolor{black}{Detecting deepfakes that involve manipulation across multiple types of media is a crucial task, and it is challenging to detect these using single-modal approaches highlighting the importance of addressing both audio and video manipulation in detection. While many methods have been proposed in the literature using different feature extraction and fusion techniques, most of them share similar architectural patterns. However, some state-of-the-art techniques have introduced novel approaches for deepfake detection. These advanced methods analyze audiovisual features iteratively, treating detection as an anomaly detection problem, and identifying inconsistencies using unique features like time-delay distribution and network activations. In addition, leveraging emotional features from both audio and visual sources, along with effective voice and face-matching techniques, greatly improves the chances of identifying deepfakes. More refined techniques also enhance the ability to detect forgery by emphasizing signs of manipulation and aligning the information from different media types. This leads to better performance and accuracy when detecting deepfakes that haven't been seen before. These developments in multimodal deepfake detection provide a more thorough and reliable way to spot manipulated content. Such improvements are essential for maintaining the trustworthiness of digital media and reducing the risks posed by misinformation in our increasingly complex online world.}

\subsection{Multimodal Deepfake Detection Tools}

\textcolor{black}{A Deepfake Detection Tool is an advanced software solution designed to identify and flag manipulated or synthetically generated digital content, including videos, audio, images, and increasingly, text. As deepfake generation techniques have become more sophisticated, detection tools have had to evolve as well, often utilizing state-of-the-art artificial intelligence (AI) and deep learning (DL) methods to keep pace with the capabilities of AI-generated forgeries. These detection tools analyze various features within digital media, employing algorithms that can identify subtle anomalies in pixel patterns, audio signals, or inconsistencies in images and videos that often occur during the deepfake creation process. Such tools not only serve as protective measures for individuals and organizations but are also gaining traction in sectors like law enforcement, cybersecurity, and media integrity, where they help maintain trustworthy sources of information. Table \ref{tab:tool_det_multimodal_deepfakes} provides a comparative overview of several currently available deepfake detection tools.}

\begin{table*}[htbp]
\caption{Top Tools to Detect Multimodal Deepfakes}
\centering
\resizebox{\textwidth}{!}{%
\begin{tabular}{|c|c|c|c|c|}
\hline
Method & Free & Open-source & URL\\\hline\hline
Sentinel  & \xmark & \xmark & https://thesentinel.ai/ \\\hline
Sensity  & \xmark & \xmark & https://sensity.ai/  \\\hline
Audio Visual Forensics & \cmark & \cmark & https://github.com/cfeng16/audio-visual-forensics \\\hline
Deepware  & \xmark & \xmark & https://deepware.ai/  \\\hline
Reality-Defender  & \xmark & \xmark & https://www.realitydefender.com/ \\\hline
Phoneme-Viseme Mismatch Detector  & \xmark & article & https://ieeexplore.ieee.org/document/9151013 \\\hline
\end{tabular}}
\label{tab:tool_det_multimodal_deepfakes}
\end{table*}


\subsection{Future Trends in Multimodal Deepfake Generation and Detection}

\textcolor{black}{In recent years, the creation of multimodal deepfakes has progressed rapidly, creating an ongoing challenge between making realistic deepfakes and developing methods to detect them. Diffusion models are one recent approach in image generation, which can be adapted for multimedia deepfake creation. These models can manipulate visual features and, when combined with audio signal manipulation techniques, allow for the generation of synchronized, multimodal content that includes both audio and visual modifications \cite{du2024dfadd, firc2024diffuse, av2024latent, bhattacharyya2024diffusion}. Despite the advances in state-of-the-art deepfake generation methods, issues with synchronization across different modalities (such as audio and video) persist. These synchronization inconsistencies are detectable by advanced detection systems and can help identify manipulated content \cite{liz2024generation, ivanovska2024vulnerability, mubarak2023survey}. To improve both deepfake generation and detection methods, high-quality datasets with a wide range of audio and video manipulations are essential. Adding datasets with content from multiple languages (most datasets are currently English-only) and diverse demographic representation would also be beneficial. This would support the development of more robust and generalized generation and detection models. Another key area in deepfake detection is explainability \cite{haq2024multimodal, tsigos2024towards}. Currently, most detection methods rely on deep learning, but what these models specifically learn to distinguish real from fake samples remains unclear. Future research could focus on understanding and interpreting these learning patterns, which could enhance both the effectiveness and trustworthiness of detection systems. In the future, multimodal deepfakes are likely to become increasingly realistic and harder to detect as techniques evolve. This makes it critical for research to keep pace, continually improving both generation and detection methods to address these advancements.}


\subsection{Combating Multimodal Deepfakes in Multimodal Biometrics}
\label{subsec:multimodal_biometrics}
With the advances in biometric evaluation techniques multimodal biometric recognition has also been recently introduced. Therefore, it is important to investigate the ability of multimodal deepfakes, especially the voice and face multimodal systems to thwart multimodal biometric recognition. In the following subsection, we discuss the summary of the findings of our evaluation, and a detailed discussion is provided in Sec. III of supplementary material
\subsubsection{Efficacy of multimodal deepfakes to fool multimodal biometrics systems}
Table III in Sec. III of supplementary material discusses the effectiveness of state-of-the-art multimodal deepfakes to thwart multimodal biometric recognition. Please note that this evaluation was conducted using voice and face multimodal systems and we considered a framework where voice modality and face modality are individually validated biometrically and the final decision is generated by fusing the individual decisions. While the current state-of-the-art multimodal deepfake generation methods failed to thwart the overall framework, especially due to their poor performance in manipulating the voice modality, the advances in multimodal deepfake technology could soon surpass the multimodal biometric reconnection and become a threat to multimodal authentication systems.


\subsubsection{Measures for revealing true identity:} To the best of our knowledge, there is no method to recover true identity from multimodal deepfakes. 



% \newpage

% \subsubsection{Universal multimodal deepfake detectors ? \textcolor{black}{I think this must come under unimodal techniques. Couldn't find any one detector that works on both audio and video. But there are many scenarios with identifying multiple types of attacks using single modality with one network}}

% \hspace{2mm}
% \subsection{Combating multimodal deepfakes in multimodal biometrics}
% \subsubsection{Efficacy of multimodal deepfakes to fool multimodal biometrics systems}
% \subsubsection{Measures for revealing true identity}
% \section{Full-body Deepfakes}
\label{sec:fullbody_deepfakes}

The term "deepfake" refers to synthetic media in which a person in an existing image or video is replaced with someone else's likeness using artificial intelligence. Full-body deepfakes are advanced synthetic media creations that use deep learning algorithms to generate realistic videos of a person's entire body. Unlike traditional deepfakes that focus on facial manipulation, full-body deepfakes replicate gestures, movements, and postures, creating an illusion of a person performing actions they never did. Even though full-body deepfakes are not extensively explored as a separate field of study, techniques such as image animation, which uses deep learning to bring static images to life by inferring realistic motions, can be considered part of deepfakes. This technology uses models such as GANs and motion transfer algorithms to create a single image based on the motion patterns derived from a driving video. This process can make a person in a still photo appear to move, speak, or perform various actions, creating highly engaging and dynamic visuals resulting in deepfakes. 

\subsection{Performance Evaluation in Full-body Animation}
\label{sub:fullbody_metrics}

In evaluating motion transfer from the driving video to the source video in most state-of-the-art methods, both quantitative and qualitative analyses are employed, allowing a comprehensive assessment of the effectiveness of the motion transfer process. Quantitative analysis involves the use of various metrics to assess performance, and such metrics are listed below.  

\paragraph{L1 Distance ($L_1$)}

The L1 distance can be used to quantify the difference between the pixel values of the two videos frame by frame and can be calculated as in Equation \ref{eq:fullbody_l1} where $I_1$, $I_2$, $m$, $n$, and $T$ refer to ground-truth frame, generated frame, width of the frame, height of the frame image and number of frames in the video respectively.

\begin{equation}
    L_1 = \Sigma_{t=1}^T \Sigma_{i=1}^m \Sigma_{j=1}^n | I_1 (i,j) - I_2 (i,j) |
    \label{eq:fullbody_l1}
\end{equation}


\paragraph{Average Keypoint Distance (AKD)}

The average key point distance in animation generation is a measure used to evaluate how accurately the generated animation matches the ground truth regarding specific key points, such as facial landmarks, joint positions, or other significant points. First, the key points which are represented as coordinates in the image space are extracted from both the ground truth and the generated animation frame. The Euclidean distance is calculated between the corresponding key points of ground truth and generated images, followed by averaging over all the pairs. This metric provides a meaningful way to assess the quality of the generated animation by focusing on how well it replicates the key structural features of the ground truth. $AKD$ can be calculated as illustrated in Equation \ref{eq:fullbody_akd} where $T$, $N$ and $K$ refer to the number of frames in the video, the number of key points in a frame and the key points ($K(x,y)$) respectively.

\begin{equation}
    AKD = \frac{1}{T} \Sigma_{t=1}^T \frac{1}{N} \Sigma_{i=1}^N || K_1^t(i) - K_2^t(i)||
    \label{eq:fullbody_akd}
\end{equation}

\paragraph{Missing Keypoint Rate (MKR)}

The Missing keypoint rate (MKR) in animation generation is a metric that quantifies the frequency at which key points are not detected or generated in the animation compared to the ground truth. First, the key points are detected in both ground truth and generated images, and a threshold $d_{th}$ is used to determine whether a key point is detected or not. The key point is considered missing if the distance between the ground truth key point and the generated key point exceeds $d_{th}$. Then, the number of key points in the ground truth that do not have a corresponding generated key point within the threshold distance is counted over all the frames and averaged to obtain $MKR$ (refer to Equation \ref{eq:fullbody_mkr}). 

\begin{equation}
    MKR = \frac{1}{N \times T} (\Sigma_{t=1}^T \Sigma_{i=1}^N ||K_1^t(i) - K_1^t(i) || > d_{th})
    \label{eq:fullbody_mkr}
\end{equation}

\paragraph{Average Euclidean Distance (AED)}

The Average Euclidean Distance (AED) calculates the average Euclidean distance in feature embedding between the ground truth representation and the generated video. The chosen feature embedding assesses how well the identity is maintained. Public re-identification networks for bodies extract identity from reconstructed and ground truth frame pairs and then compute the mean L2 norm of their difference across all pairs. \cite{siarohin2021motion}.

\subsubsection{Datasets for Full-body Deepfake (Video Animation) Generation}

Most state-of-the-art methods in the next section are evaluated using the Tai-Chi-HD and TED-Talks datasets. The Tai-Chi-HD dataset comprises 3048 (2884 in \cite{guo2024human}) and 285 video chunks for training and testing, extracted from 252 training and 28 testing videos, respectively \cite{hong2022qs, siarohin2019first}. The video length varies from 128 to 1024 frames \cite{siarohin2019first}. The TED-Talks dataset consists of videos from TED talks available on YouTube, with 1132 training videos and 131 test videos (1255 in \cite{tao2022motion}) \cite{guo2024human}. The number of frames per video ranges from 64 to 1024 \cite{guo2024human, zhao2022thin, siarohin2021motion}. The slight variations in dataset distributions mentioned above might have caused differences in the performances of different state-of-the-art models, as shown in Table \ref{tab:fullbodydatasets}. Furthermore, a low-resolution Tai-Chi dataset \cite{tulyakov2018mocogan} was also used in earlier evaluations which consisted of 4500 tai-chi video clips split into 3288 and 822 videos for training and testing sets respectively. The video length of the Tai-Chi dataset varied from 32 to 100 frames \cite{siarohin2019animating}. The Tai-Chi dataset consisted of low-resolution frames (64 $\times$ 64), while Tai-Chi-HD included frames with resolutions of 256 $\times$ 256 and 512 $\times$ 512. The TedTalks dataset mainly utilized frames with a resolution of 512 $\times$ 512.

\subsection{Full-body Deepfake (Video Animation) Generation}
\label{subsec:fullbody_generation}

Motion transfer is a key technique in creating full-body deepfake videos. It involves taking the movements or poses from one person (the "driving source") and applying them to another person (the "target" or "source subject") so that the target subject mimics the exact movements of the driving subject. This is done while keeping the appearance of the target subject, essentially allowing one person’s body movements to be applied to another person. Recent advancements in deep learning, particularly in motion transfer, pose estimation, and video synthesis, have led to the development of models capable of generating high-quality full-body deepfake videos. Two commonly used datasets for full-body deepfake generation and motion transfer are the TaiChiHD dataset and the TED Talks dataset. The TaiChiHD dataset is primarily used for generating full-body animations that capture precise, fluid human motions, such as Tai Chi movements. It focuses on complex poses and transitions between them. Models trained on TaiChiHD can learn to transfer precise body movements from one individual to another, ensuring that synthesized videos maintain the accuracy of the poses and gestures while adapting the appearance of the subject. The TEDTalks dataset can be used to extract and model the dynamic body language and physical gestures exhibited by speakers. These movements can then be transferred to another target subject, creating a video where the target mimics the gestures and body language of the TEDTalks speaker. Some state-of-the-art techniques in motion transfer methods are given in Table \ref{tab:fullbodydatasets} (summarised in Figure \ref{fig:taichihd} and \ref{fig:tedtalks}) and are discussed in Section \ref{subsubsec:fullbody_gen}.

\subsubsection{Literature Review on Full Body Deepfake Generation}
\label{subsubsec:fullbody_gen}

Monkey-Net (MOviNg KEYpoints) \cite{siarohin2019animating} is a groundbreaking model used for motion transfer and creating realistic deepfake animations. It transfers motion from a driving video to a target image to create animations using deep learning. The network included a keypoint detector trained to extract object keypoints, a dense motion prediction network for generating dense heatmaps from sparse keypoints to better encode motion, and a motion transfer network that uses the motion heatmaps and appearance information from the input image to create the output frames. Siarohin \textit{et al} \cite{siarohin2021motion} introduced a new approach for motion transfer (MRAA), which does not rely on keypoint detectors. They developed a network that can identify object parts, track them in a video, and infer their motions based on their principal axes. Unlike previous methods that used keypoint detectors, this approach focuses on extracting meaningful and consistent regions corresponding to semantically relevant and distinct object parts easily detected in the video frames. To separate foreground from background, they incorporated an additional affine transformation to model non-object-related global motion. They also disentangled the shape and pose of objects in the region space to facilitate animation and prevent the leakage of the driving object's shape. Since MRAA uses local affine transformations near the keypoints to estimate the motion, the temporal continuity of the reconstructed video depends on the smoothness of the
keypoints change. If the location of the key points in two adjacent frames changes greatly, it can cause pixel jitter \cite{zhao2022thin}. In the TPS-MM method proposed by Zhao \textit{et al.}, the use of Thin-Plate-Spline (TPS) motion estimation alleviated previous issues by generating each transformation using multiple keypoints, thus increasing the robustness of motion estimation. Before the TPS-MM method \cite{zhao2022thin}, motion transfer on arbitrary objects was mainly conducted using unsupervised methods without prior knowledge. However, these unsupervised methods often failed when there was a large pose gap between the objects in the source and driving images. As a solution, TPS-MM proposes an end-to-end unsupervised motion transfer framework that utilizes thin-plate spline motion estimation to create a more flexible optical flow, which warps the feature maps of the source image to the feature domain of the driving image. Additionally, to restore the missing regions, they propose multi-resolution occlusion masks to achieve more effective feature fusion. Despite significant advancements in the field of motion transfer, challenges such as distorted limbs and missing semantics persist due to the complex representation of motion and the unknown correspondence between human bodies. To address these challenges, Guo \textit{et al.} \cite{guo2024human} proposed a novel, semantically guided, unsupervised method of motion transfer that utilizes semantic information to model motion and appearance. This approach involved utilizing a pre-trained human parsing network to encode rich foreground semantic information, enabling the generation of fine details. Additionally, an attention-based network layer was proposed to learn the semantic region's correspondence between human bodies, guiding the network in selecting appropriate input features and ultimately leading to more accurate results.

\begin{table*}[htbp]
\caption{\textcolor{black}{State-of-the-art Video Animation Generation Methods (Video reconstruction). We have highlighted the performance of each method as given in the original paper. However, a few evaluation metrics on the TedTalks dataset and all FID values were referred from HIA-SG \cite{guo2024human}, and referred through $^\dagger$. $^{\ast}$ represents performance on Tai-Chi dataset.}}
\centering
\resizebox{\textwidth}{!}{%
\begin{tabular}{|P{95pt}|P{27pt}|P{27pt}|P{27pt}|P{27pt}|P{27pt}||P{27pt}|P{27pt}|P{27pt}|P{27pt}|P{27pt}|P{27pt}|}
\hline
\multirow{2}{*}{Method} & \multicolumn{5}{c||}{TaiChiHD} & \multicolumn{5}{c|}{TedTalk}\\\cline{2-11}
 & $L_1$ & $AKD$ & $MKR$ & $AED$ & $FID$ & $L_1$ & $AKD$ & $MKR$ & $AED$ & $FID$ \\\hline\hline
HIA-SG \cite{guo2024human} & 0.049 & 4.34 & 0.016 &  0.154 & 25.57$^\dagger$ & 0.030 & 3.38 &  0.007 & 0.138 & 25.78$^\dagger$\\\hline
TPS-MM \cite{zhao2022thin} & 0.045 & 4.57 & 0.018 &  0.151 & 24.29$^\dagger$ & 0.027 & 3.39 &  0.007 & 0.124 &  23.28$^\dagger$ \\\hline
Motion Transformer \cite{tao2022motion} & 0.045 & 4.670 & 0.021 & 0.148 & - & 0.026 & 3.456 & 0.007 & 0.113 & - \\\hline
QS-Craft \cite{hong2022qs} & - & 4.61 & 0.017 & - & 25.064 & - & - & - & - & - \\\hline
FOMM \cite{siarohin2019first} & 0.063 & 6.862 & 0.036 & 0.179 & 28.08$^\dagger$ & 0.033$^\dagger$ & 7.07$^\dagger$ &  0.014$^\dagger$ &  0.163$^\dagger$ & 29.87$^\dagger$\\\hline
MRAA \cite{siarohin2021motion} &  0.047 & 5.58 & 0.027 & 0.152 & 25.74$^\dagger$ & 0.026 & 4.01 & 0.012 &  0.116 & 22.54$^\dagger$\\\hline
Monkey-Net \cite{siarohin2021motion} & 0.077 & 10.80 & 0.059 & 0.228 & - & - & - & - & - & - \\\hline
Monkey-Net$^{\ast}$ \cite{siarohin2019animating} & 0.050$^{\ast}$ & 2.53$^{\ast}$ & \textcolor{black}{17.4}$^{\ast}$ & 0.21$^{\ast}$ & - & - & - & - & - & - \\\hline
\end{tabular}}
\label{tab:fullbodydatasets}
\end{table*}

\begin{figure}
    \centering
    \includegraphics[width=\linewidth]{figures_new/TaiChi.pdf}
    \caption{\textcolor{black}{Performance variation of recent state-of-the-art methods in motion transfer to create full body deepfakes on Tai-Chi-HD dataset}}
    \label{fig:taichihd}
\end{figure}

\begin{figure}
    \centering
    \includegraphics[width=\linewidth]{figures_new/TedTalk.pdf}
    \caption{\textcolor{black}{Performance variation of recent state-of-the-art methods in motion transfer to create full body deepfakes on TedTalks dataset}}
    \label{fig:tedtalks}
\end{figure}


\textcolor{black}{Motion transfer technology is rapidly advancing, enabling the creation of highly realistic full-body deepfake animations by mapping motion from a video onto a static image. At its core, motion transfer involves generating representations of movement that can capture complex human motions and expressions, allowing animations to synchronize seamlessly with the source video. This technology synthesizes both appearance and motion in increasingly smooth and lifelike ways, facilitating the production of convincing animated content. By identifying key body parts and tracking their movements, these systems replicate subtle details in a subject’s posture, gestures, and interactions while maintaining coherence across frames. However, this level of realism raises important ethical questions regarding deepfake content. The ability to produce highly believable full-body deepfakes that mimic specific individuals can blur the line between reality and artificial content in media. As a result, the future of motion transfer technology in deepfakes will likely involve balancing creative applications with ethical considerations. Society must navigate the implications of increasingly realistic virtual representations on privacy, trust, and the authenticity of media.}

\subsubsection{Literature Review on Full Body Deepfake Detection} 

To the best of our knowledge, currently there is no method to detect full-body deepfakes.

\subsection{Combating Full-body Deepfakes in Gait Biometrics}
Gait biometrics is recently being adapted into security surveillance applications such as identifying and tracking individuals from CCTV footage. Therefore, it is important to investigate whether full-body deepfakes are capable of maintaining a subject's gait biometric features when animating a still image of that particular subject. In the following subsection, we discuss the summary of the findings
of our evaluation, and a detailed discussion is provided in Sec.
IV of supplementary material.

\subsubsection{Efficacy of full-body deepfakes to fool gait biometrics systems} Table IV in Sec. IV of supplementary material demonstrates the viability of full-body deepfakes to fool gait recognition models. Specifically, we observe substantial vulnerability when synthesising normal walking patterns of the target subject. Therefore, it is important to address this limitation of gait recognition methods. lease refer to Sec. IV of supplementary
material for additional details regarding this evaluation.

\subsubsection{Measures for revealing true identity: }To the best of our knowledge, currently there is no method to reveal true identity from full-body deepfakes.

\section{Applications of deepfakes} \label{sec:applications}
In this section, we present a spectrum of positive applications of deepfakes ranging from fashion to the entertainment industry. Moreover, the ethical, psychological, and security implications of deepfakes are also discussed in this section. 

\subsection{Positive commercial applications}

Despite its controversial reputation, the deepfake technology holds great potential for positive commercial applications across a myriad of industries. In addition to the reenactment of passed actors with hyper-realistic visual quality, which we see being readily used in the entertainment industry, there are various current and future use cases of deepfake technology across a diverse set of applicational areas if this technology is used ethically and responsibly. We use this section to introduce such sample use cases.  

\subsubsection{Fashion and beauty industry:}
With the rise of deepfake technology, the way that leading brands engage with consumers has been revolutionised. In 2021 Kati Chitrakorn, the Vogue business technology expert predicted that deepfake technology would transform the landscape of the fashion industry \cite{deepfakevoguebusiness}. 

For instance, \textbf{digital fashion shows and influencer collaborations} has now become a reality \cite{deepfakeglamour}. The onset of the COVID-19 pandemic only accelerated the use of deepfake technology enabling the collaborations and interactions between people in a setting where in-person activities are restricted. Specifically, Demna Gvasalia's `deepfake' Spring 2022 Balenciaga fashion show illustrated how deepfake technology can be used to model garments by celebrities or influencers with just a few sets of images without the need to even be there in person.  

In the 2019 London Fashion Week, some selected sets of participants were able to watch themselves wearing HANGER’s latest collection using deepfake technology. These \textbf{virtual try-ons} were being projected behind the live models enabling the members of the audience to see their appearance if they were actually wearing the garments that were being modelled. Superpersonal \footnote{https://www.producthunt.com/products/superpersonal} is an app that has been specifically designed to allow users to try on clothes virtually. This allows the consumers to visualize products before purchase and understand which styles fit their taste better. Several other innovations have also emerged in this direction. Fxgear's FXMirror \footnote{https://www.fxgear.net/vr-fashion} which provides an augmented-reality fitting room experience such that the shoppers can try on clothes virtually and YourFit solution \footnote{https://3dlook.ai/yourfit/} by 3DLOOK are only a few examples. 

\subsubsection{Marketing industry}
 
Furthermore, the flexibility and customisability of the synthesised media have no limits. As such \textbf{hyper-personalised advertisements} could be generated targeting consumers in different demographics by customising the clothing, voice, and location of characters in the advertisement. Unethical and fraudulent use of this technology was experienced in Taylor Swift's Deepfake Campaign where her image was used without permission for the creation of advertising media that promoted products she did not endorse. However, this technology can be ethically used with the permission of the celebrity or the influencer to create highly influential advertisements that could reach a far greater audience with minimal cost. The 2019 malaria awareness deepfake advertisement that featured David Beckham \footnote{https://youtu.be/QiiSAvKJIHo?si=RuNnN5hE1R78JTCQ} is a prime example that demonstrates this marketing potential. In this video, David Beckham speaks in nine different languages appealing to end malaria and multimodal deepfake technology has made Beckham appear multilingual. German online retail giant Zalando has also been readily using deepfake technology in its marketing campaigns. For example, the deepfake technology enabled supermodel Cara Delevingne to appear in 290,000 localised advertisements \cite{deepfakevoguebusiness}. deepfake technology in marketing can be further extended to reenact historical figures and bring them to life with contemporary public figures. This will \textbf{enhance the storytelling} of the marketing materials and better capture the attention of the targetted consumers. 

The deepfake technology is transforming the marketing industry as a whole. The deepfake technology reduces production costs by cutting down the costs associated with the hiring of the production crews, including, videographers, camera operators, media editors, casting assistants, and directors. Furthermore, it doesn't need a location to shoot the videos or equipment to record them. Hour One \footnote{https://hourone.ai/} is a company that readily uses deepfakes to create commercial media. In one of their advertisement campaigns human actors are replaced by animated digital clones of real humans generated by deepfakes \cite{deepfaketechnologyreview}. The impracticality of using real actors and production crews to create thousands of videos in different languages has led Hour One to opt for deepfakes.

\subsubsection{Corporate training, simulation, and virtual assistants}

The British multinational company WPP has created training videos together with Synthesia, a synthetic media generation company, targeting its employees \cite{deepfakewired}. These videos have been sent to thousands of employees that WPP has worldwide and address the employee by name and explain some basic concepts in AI. This example shows how deepfake technology can be used to create \textbf{corporate training} material in a personalised and cost-effective manner, which is otherwise impractical for a multinational company. However, the merits of deepfake technology in corporate training and simulation settings surpass this simple application. For instance, it can be used to create highly realistic customer interactions or crisis management situations for training purposes. The AI models are adaptable and they can dynamically change their responses based on the responses of the trainee, providing a personalised training experience. Moreover, the deepfake technology provides a risk-free training environment for areas such as law enforcement and healthcare, therefore, trainees can practice decision-making in critical settings.  

The Live Interactive Customer Experience (ALICE) receptionist \footnote{https://www.alicereceptionist.com/} is an ideal example of how \textbf{virtual receptionist kiosk} can be used to handle visitors' queries, replacing the role of a human receptionist. We incorporate the virtual receptionist application under the deepfake category considering the use of deepfake technology to create human-like avatars that represent the virtual receptionist. Furthermore, this virtual assistant is capable of conducting a range of visitor management tasks, including, pre-visit check-ins, visitor screening, driver's license scanning, and body temperature check. These virtual receptionist kiosks are currently being used in various American International Group and ING Group branches.


\subsubsection{ Entertainment industry: }

deepfake technology is widely being used for \textbf{dubbing or revoicing media in the entertainment industry}. This allows the synchronise facial expressions, lip movements, and expression of emotions after the dubbing process. An AI-driven startup company named Flawless \footnote{https://www.flawlessai.com/} is generating deepfake dubs which is cost-effective and efficient, and help the media content reach new audiences. Compared to traditional dubbing methods which have mistimed mouth movement, Flawless utilises deepfake technology to artificially synthesise lip movements that match the translated speech. The result is a much smoother revoicing of the media. 

The movie industry is heavily utilising Computer-Generated Imagery (CGI) to create visual effects. This is a meticulous process done by visual effects artists. The deepfake technology has the potential to automate this process by generating different renders of the chosen character automatically. The gaming industry has started adapting the deepfake technology. For instance, in the video game Cyberpunk 2077 \footnote{https://www.cyberpunk.net/au/en/} where celebrities play roles in the game. We believe this technology will soon seep into the film industry as well giving more potential to the filmmakers, enabling scenarios such as reenacting historical events or bringing characters to life.  

\hspace{2mm}

\subsection{Negative implications}
This subsection summarises the primary negative implications of deepfake technology and highlights the need for the urgent need for countermeasures. 

\subsubsection{Misinformation and disinformation propaganda}

In mid-March 2022 a deepfake video of Ukrainian President Volodymyr Zelenskyy appeared on social media calling on the Ukrainians to stop fighting and to surrender their weapons \cite{deepfakenortheastern}. This video was even broadcast on the Ukrainian television channel Ukraine 24 by a team of hackers. 

The harmful effects of mis/disinformation generated through deepfakes do not stop from fake news. It can be used to impact elections, perform corporate sabotage with well-times and articulated falsifying evidence, and harm the image of public figures. Most importantly these mis/disinformation campaigns could deteriorate public trust regarding the authenticity of genuine material in mainstream media. For instance, when the Princess of Wales, released a video statement in March 2024 sharing that she had been diagnosed with cancer a fresh round of conspiracies regarding deepfakes reappeared in social media \cite{deepfakewashingtonpost}. However, this time it was people disbelieving a real video. These real-world examples clearly elaborate the growing threat of multimodal deepfake to society in an era where seeing is not believing.

Apart from these recent examples, there are other evidence of deepfake being used to spread mis/disinformation. It has been suggested that a deepfake story could have sparked the diplomatic confrontation between Saudi Arabia and Qatar \cite{deepfakebrookings1}. The unnatural speech of President Ali Bongo sparked a military coup in 2018 in Gabon claiming that the video was a deepfake and the president was no longer healthy enough for the office or even had died \cite{deepfakeforbes}. Furthermore, there was an unsuccessful attempt to discredit and overthrow Malaysia’s economic affairs minister using deepfake-based fabricated media \cite{deepfakewii}. 

Moreover, deepfakes can be used to create fake influences or endorsements. For instance, fraudulent Taylor Swift advertisements that promoted a cookware brand on social media are a prime example of such fraudulent narratives. Based on our review of deepfake detection methods, there is no universal deepfake detector that could suffice and withstand all the advances of current and future deepfake generation technology, and until such robustness is met our society faces ongoing threats due to the malicious use of deepfake technology. 


\subsubsection{Psychological impact}
The psychological impact of deepfakes is quite concerning as it affects not only individuals on a personal level but also at social and societal levels. 

Deepfakes disrupt our ability to believe what we perceive. Therefore, it could lead to deterioration of trust of people regarding news and media in general. This effect occurs even if the deepfake is unsuccessful in misleading a particular individual. The sense of deception leads to increased skepticism and uncertainty in our daily online and offline interactions. For instance, a study by Vaccari and Chadwick \cite{vaccari2020deepfakes} found that even if a person is not completely misled by a deepfake the exposure to it reduces their trust in news.

Another study found alarming evidence of deepfakes modifying our memories and even implanting false memories. For instance, in \cite{dobber2021microtargeted} the authors found that watching deepfake videos could result in participants falsely remembering nonexistent films. Furthermore, it could lead to a change in one's attitude. Specifically, the authors of \cite{dobber2021microtargeted} constructed multimodal deepfakes and have shown them to a selected group of individuals to see if there is any change in their attitudes toward the politician and the attitudes toward his or her political party. This study revealed that microtargeting the deepfake to groups that are most likely to be offended could amplify its negative implications.   Furthermore, we should consider the profound emotional impact of deepfakes on the individuals whose identities have been maliciously depicted in the video. The fabricated media could be embarrassing, offensive, and damaging to their reputation, leading to anxiety, and even altering their beliefs and behaviour.

\subsubsection{National security threat of deepfakes} 

Lt. Gen. Jack Weinstein who is the deputy chief of staff for strategic deterrence and nuclear integration at the United States (US) Air Force Pentagon headquarters stated ``The greatest existential threat to the United States of America is the fracturing of our democracy and the intentional misleading of facts to support political agendas''. Therefore, it is clear that false or misleading information that is deliberately spread to deceive a population could cripple the world's largest economy and the second-largest democracy. Furthermore, based on the press release of The US National Security Agency on September 12, 2023 \cite{deepfakeNSA}, synthetic media can cause public unrest through the spread of false information about political, social, military, or economic issues. This report states that public availability of the implementation of deepfake generation algorithm has made mass production of fake media easier and less expensive, which has broadened their impact to a larger scale.  

The national security threat of deepfakes also includes cyber espionage through impersonation where deepfake technology can be used to fabricate fake communications of high-ranking officials. While to the best of our knowledge, this has not occurred to date, there exists evidence to the use of audio deepfakes has been used to steal personally identifiable information during fake online interviews of potential applicants \cite{deepfakeFBI}. This information can be used to create fake credentials to gain access to sensitive information or critical infrastructure systems. 

Moreover, one should consider the economic impact of deepfake as the economy is closely associated with national security. This is clearly evident by the findings in the 2024 global risk report of the  World Economic Forum. This report states that misinformation and disinformation are the biggest short-term risks to the world economy. For instance, meticulously targeted false information about large corporations of a certain country could be used to disrupt markets or manipulate stock prices leading to economic instability in that particular country. 

Government agencies, researchers, and policymakers should continue to collaborate together to minimise the threat of deepfakes to national and global security. Furthermore, mainstream media has a major role in promoting the awareness and literacy of the general public regarding the threat of deepfakes and how to spot them.

\subsubsection{ Privacy violation} 
A popular example of deepfakes in privacy violation is the creation and distribution of pornographic material by swapping an individual's face, voice, and body into real pornography. For instance, the Reddit user made deepfake sex videos of female celebrities using their images and videos. However, the potential victims are not limited to public figures. It can be used to generate revenue or targeted attacks against specific individuals, such as ex-partners, or rivals which is an invasion of sexual privacy \cite{citron2018sexual}.

The impact of such activities is not limited to emotional distress, reputational damage, and personal trauma. Blackmailers could use deepfakes to extortion. The victims may be forced to provide money or even business secrets to prevent the release of the deepfakes. Several legislative changes have been proposed to protect victims from privacy-related issues caused by deepfakes, however, it has been identified that there exist several issues when dealing with deepfakes in litigation and governments need to take more actions to protect victims \cite{deepprivacy}.


\hspace{3mm}
\section{Future research directions} \label{sec:future_research}
 
We refer the readers to Sec. 4 in the supplementary material, where we discuss future research directions, including the design of universal deepfake detection methods, recovering the true identity, explainable deepfakes detection methods, introducing standard evaluation protocols, and design of a regulatory framework for the governance of deepfake research. 

\section{Conclusion}\label{sec:conclusions}
In this survey paper, we have discussed existing state-of-the-art methods for the generation and detection of face deepfakes. Our analysis emphasises an algorithmic perspective, providing an in-depth discussion of the architectures, and include details such as training paradigms, loss functions and evaluation metics. 
In addition, we have discussed the biometric implications of the generated face deepfakes and we have provided in-depth discussion regarding their positive and negative applications. As concluding remarks, we outlined key research gaps and proposed possible future research directions for further investigation. 
\appendix
\section{Reconstruction Attack when  $\nexamples$ is Unknown}\label{appendix:cp_model_n_unknown}

\subsection{Confidence Interval on $\nexamples$}

While reconstruction attacks targeting a complete dataset recovery in the literature generally assume knowledge of its size $\nexamples$~\cite{dwork2017exposed}, our approach can be slightly modified to be applicable without this information. In this section, we provide and empirically evaluate a CP formulation derived from the one presented in Section~\ref{sec:attack} that does not rely on knowing $\nexamples$. Note that while this setup is both more generic and more realistic, it is also more challenging.

When $\nexamples$ is unknown, it becomes a variable to infer. To bound the search, we will first calculate a confidence interval $\llbracket \nexamples_{\textsc{min}}, \nexamples_{\textsc{max}} \rrbracket$.
According to the DP RF algorithm in Section~\ref{sec:dp_rf}, each training example is used exactly once per tree. For a tree $\tree \in \forest$, the sum of the non-noisy leaves' counts equals $\nexamples$.
In our case, since the Laplace noise is centered at zero, for a tree $\tree \in \forest$, we can expect the sum $\nexamples_\tree^* = \smash{\sum\limits_{\node,\class} \nodesupport[\tree,\node,\class]^*}$ to lie close to $\nexamples$.

We know that the noisy counts are computed by the Laplace mechanism as follows:
\[
\nodesupport[\tree,\node,\class]^* = \textrm{int}(\nodesupport[\tree,\node,\class] + \noisevar_{\tree\node\class}) = \nodesupport[\tree,\node,\class]  +  \textrm{int}( \noisevar_{\tree\node\class}) ,
\]
in which $\noisevar_{\tree\node\class}$ is a random variable sampled from a Laplace distribution \( \text{Lap}(1 / \varepsilon_\node)\) and
$\textrm{int}(.)$ is the integer part function. Recall that the per-tree DP budget $\varepsilon_\node$ is computed by dividing the total privacy budget among the trees: $\varepsilon_\node = \varepsilon/\lvert \forest \rvert$.
The variance of each $\noisevar_{\tree\node\class}$ is $\text{Var}(\noisevar_{\tree\node\class}) = 2 / \varepsilon_\node^2$. Since the $\noisevar_{\tree\node\class}$ random variables are i.i.d., the variance of their sum $\sumnoisevars_\tree$ (whose $\nexamples_\tree^*$ is a realization) is:
\begin{align*}
     \text{Var}(\sumnoisevars_\tree) &= \text{Var}\left(\sum\limits_{\node \in \leaves[\tree]} \sum\limits_{\class \in \classes} \noisevar_{\tree\node\class}\right) 
     = \sum\limits_{\node \in \leaves[\tree]} \sum\limits_{\class \in \classes} \text{Var}(\noisevar_{\tree\node\class})    
    = \vert \leaves[\tree] \rvert \cdot \lvert \classes \rvert \cdot \frac{2}{\varepsilon_\node^2},
\end{align*} 
with a standard deviation:
\[
\sigma_{\nexamples_\tree^*} = \frac{\sqrt{2 \cdot \lvert \leaves[\tree] \rvert \cdot \lvert \classes \rvert}}{\varepsilon_\node}.
\]

Let $\nexamples^* = \frac{1}{\lvert \forest \rvert} \sum_{\tree \in \forest} \nexamples_\tree^*$ be the average noisy count across all trees. By the Central Limit Theorem, $\nexamples^*$ approximately follows a normal distribution for large $\lvert \forest \rvert$. Thus, the 95\% confidence interval $\llbracket \nexamples_{\textsc{min}}, \nexamples_{\textsc{max}} \rrbracket$ can be defined as:
\[
    \nexamples_{\textsc{max}} = \nexamples^* + t_{95} \frac{\sigma_{\sumnoisevars_\tree}}{\sqrt{\lvert \forest \rvert}} \quad \text{and} \quad \nexamples_{\textsc{min}} = \max\left(1,\nexamples^* - t_{95} \frac{\sigma_{\sumnoisevars_\tree}}{\sqrt{\lvert \forest \rvert}}\right),
\]
where $t_{95}$ is the Student's t-coefficient for a 95\% confidence level according to $\lvert \forest \rvert$.


\subsection{Adapted Constraint Programming Formulation}

As in the formulation presented in Section~\ref{sec:attack}, for each leaf $\node \in \leaves[\tree]$ of each tree $\tree \in \forest$, we let variables $\nodesupport[\tree,\node,\class] \in \{\max(0,\nodesupport[\tree,\node,\class]^*-\delta), \dots, \nodesupport[\tree,\node,\class]^* +\delta \}$ and $\Delta_{\tree\node\class } \in \{ -\delta, \dots, \delta \}$ respectively model the (guessed) number of examples of class $\class$ within this leaf, and the amount of noise added to it.
We also use variable $\smash{\widetilde{\nexamples}}\in\{\nexamples_{\textsc{min}},\dots, \nexamples_{\textsc{max}}\}$ encode the (guessed) number of reconstructed examples.

Since $\nexamples$ is unknown, we define variables for up to $\nexamples_{\textsc{max}}$ examples in the worst-case. For each example $\example \in \{1, \dots, \nexamples_{\textsc{max}}\}$, $\varx[\example,\feature] \in \{0,1\}$ represents the value of its attribute $\feature \in \{1, \dots, \nattributes\}$ in the reconstruction, while $\varz[\example,\class] \in \{0,1\}$ equals $1$ if and only if it belongs to class $\class$. Although these reconstruction variables are defined for $\nexamples_{\textsc{max}}$ examples, the attack's output is obtained from the first $\widetilde{\nexamples}$ examples.

Finally, we define a set of variables modeling the path of each example $\example \in \{1,\dots,\nexamples_{\textsc{max}}\}$ through each tree $\tree \in \forest$: variable $\varyb[\tree,\node,\example,\class] \in \{0,1\}$ indicates if it is classified in class $\class \in \classes$ by leaf $\node \in \leaves[\tree]$.
Given these variables, as previously, we define constraints to compute class counts and connect them to the assignments of examples to leaves:
\begin{align}
    &\sum\limits_{\node \in \leaves[\tree]} \sum\limits_{\class \in \classes} \nodesupport[\tree,\node,\class] = \widetilde{\nexamples} &\tree \in \forest \label{constr_count_sum_N_bis}\\
     &\sum\limits_{\class \in \classes}\varz[\example,\class]=1 &\example \in \{1,\dots,\nexamples_{\textsc{max}}\}\label{constr_exactly_one_class_bis}\\
    &\varz[\example,\class] = 0 \Rightarrow\sum\limits_{\tree \in \forest} \sum\limits_{\node \in \leaves[\tree]} \varyb[\tree,\node,\example,\class] = 0 &\example \in \{1,\dots,\nexamples_{\textsc{max}}\},~\class \in \classes\label{constr_not_in_class_counts_bis}\\
    &\example > \widetilde{\nexamples}  \Rightarrow\sum\limits_{\tree \in \forest} \sum\limits_{\node \in \leaves[\tree]}\sum\limits_{\class \in \classes} \varyb[\tree,\node,\example,\class] = 0 &\example \in \{1,\dots,\nexamples_{\textsc{max}}\}\label{constr_N_examples_in_reco}\\
    & \sum_{\example=1}^{\smash{\nexamples_{\textsc{max}}}} \varyb[\tree,\node,\example,\class] =\nodesupport[\tree,\node,\class] &\tree \in \forest,~\node \in \leaves[\tree],~\class \in \classes\label{constr_counts_flow_bis}
\end{align}
Constraint~\eqref{constr_count_sum_N_bis} ensures that the examples' counts sum up to $\widetilde{\nexamples}$ within each tree. Constraint~\eqref{constr_exactly_one_class_bis} guarantees that each example is associated to exactly one class. Constraint~\eqref{constr_not_in_class_counts_bis} enforces that each example can only contribute to the counts of its class. Note that this constraint differs from the formulation presented in Section~\ref{sec:attack}. Indeed, if $\varz[\example,\class] = 1$ but $\example > \widetilde{\nexamples}$, example $\example$ is actually not part of the reconstructed dataset, and hence is not assigned to any leaf of any tree, as stated in Constraint~\eqref{constr_N_examples_in_reco}.
Constraint~\eqref{constr_counts_flow_bis} ensures consistency between the examples' assignments to the leaves, and the (guessed) counts.

Finally, the rest of the model remains unchanged, the following constraint enforces consistency between the examples' assignments to the leaves, and the value of their attributes:
\begin{align}
\sum\limits_{\class \in \classes} \varyb[\tree,\node,\example,\class] = 1 \Rightarrow &\left(\displaystyle\bigwedge_{i\in\Phi_v^+} \varx[k,i] = 1 \right) \wedge
\left( \displaystyle\bigwedge_{i\in \Phi_v^-} \varx[k,i] = 0 \right) & \tree \in \forest,~\node \in \leaves[\tree],~\example \in \{1,\dots,\nexamples_{\textsc{max}}\},\label{constr_flow_attr_values_bis}
\end{align}
and the original objective is retained with $\Delta_{\tree\node\class} = \nodesupport[\tree,\node,\class]^{*} - \nodesupport[\tree,\node,\class]$:
\begin{align}
    \max \ \sum\limits_{\tree \in \forest}\sum\limits_{\node \in \leaves[\tree]} \sum\limits_{\class \in \classes} \sum\limits_{\noiseval = -\delta}^{\delta} \log(p_\noiseval) \mathds{1}_{\Delta_{\tree\node\class} = \noiseval}.
\end{align}

\subsection{Experimental Evaluation} 

To assess the effectiveness of our reconstruction attack in a context where the number of training examples $\nexamples$ is unknown, we reproduce the experimental setup as described in Section~\ref{subsec:setup}, using our modified attack formulation introduced in the previous subsection. The computation of the reconstruction error must be slightly adapted. When $\nexamples$ is known, the attack's performance is assessed by determining the correspondence between reconstructed and original examples using a minimum-cost matching. The average distance between these matched pairs defines the reconstruction error.
However, when $\nexamples$ is unknown, it is possible that the inferred reconstruction contains more (or less) than $\nexamples$ examples, making this pairwise matching not directly applicable. In the case where the number of reconstructed examples exceeds the actual one ($\widetilde{\nexamples} > \nexamples$), we randomly duplicate some (true) training examples. In contrast, if the number of reconstructed examples is smaller than the actual one ($\widetilde{\nexamples} < \nexamples$), we randomly subsample the same number of examples from the original training set. The same minimum-cost matching is then performed, and the average distance between pairs of matched examples defines the reconstruction error. 

The results of these experiments are presented in Table~\ref{tab:error-values_N_var} for all datasets, forest sizes $\lvert \forest \rvert$, and tree depths $\depth$. Comparing these results with those in Table~\ref{tab:error-values} reveals two key trends. First, a few runs failed to retrieve a feasible reconstruction within the given time frame ($2$ hours), indicating that the attack is computationally more expensive in this setup. 
This is expected, as the modified CP model introduces additional variables and constraints, leading to a larger search space and calling for longer solution times. Still, in most cases, the attack finds a feasible reconstruction. 
Second, and more importantly, the reconstruction error remains comparable to the case where $\nexamples$ is known. Despite the increased computational cost, the attack is still able to retrieve meaningful information encoded by the forest regarding its training data in most cases, thereby confirming our \textbf{Result 5}.

\begin{table*}[htbp]
\centering
\caption{Experimental results without assuming the training set size $\nexamples$ is known.
The table reports the average reconstruction error across varying numbers of trees $\lvert \forest \rvert$, tree depths $d$, and privacy budgets $\varepsilon$ for the three datasets. Cases where the solver did not find a feasible reconstruction within the $2$-hour time limit are indicated by ``--''.}
\label{tab:error-values_N_var}
\resizebox{.71\textwidth}{!}{%
\begin{tabular}{llccccccccc}
\toprule
                                &                     & \multicolumn{3}{c}{Adult} & \multicolumn{3}{c}{COMPAS} & \multicolumn{3}{c}{Default Credit} \\
\cmidrule(lr){3-5} \cmidrule(lr){6-8} \cmidrule(lr){9-11}
                                &                     & $d=3$   & $d=5$  & $d=7$  & $d=3$   & $d=5$   & $d=7$  & $d=3$      & $d=5$      & $d=7$     \\
\midrule
\multirow{6}{*}{$\lvert \forest \rvert = 1$}  & $\varepsilon = 0.1$ & 0.26    & 0.24   & 0.20   & 0.27    & 0.24    & 0.16   & 0.34       & 0.31       & 0.26      \\
                                & $\varepsilon = 1$   & 0.26    & 0.24   & 0.17   & 0.26    & 0.22    & 0.15   & 0.34       & 0.30       & 0.22      \\
                                & $\varepsilon = 5$   & 0.25    & 0.15   & 0.22   & 0.25    & 0.28    & 0.22   & 0.32       & 0.31       & 0.27      \\
                                & $\varepsilon = 10$  & 0.25    & 0.13   & 0.24   & 0.25    & 0.20    & 0.20   & 0.32       & 0.29       & 0.30      \\
                                & $\varepsilon = 20$  & 0.25    & 0.13   & 0.28   & 0.25    & 0.19    & 0.19   & 0.32       & 0.28       & 0.28      \\
                                & $\varepsilon = 30$  & 0.25    & 0.12   & 0.28   & 0.25    & 0.20    & 0.20   & 0.32       & 0.30       & 0.15      \\
\midrule
\multirow{6}{*}{$\lvert \forest \rvert = 5$}  & $\varepsilon = 0.1$ & 0.24    & 0.24   & 0.26   & 0.23    & 0.20    & 0.33   & 0.30       & 0.25       & 0.44      \\
                                & $\varepsilon = 1$   & 0.19    & 0.24   & 0.20   & 0.15    & 0.20    & 0.14   & 0.24       & 0.23       & 0.23      \\
                                & $\varepsilon = 5$   & 0.17    & 0.15   & 0.25   & 0.10    & 0.08    & 0.08   & 0.22       & 0.17       & 0.18      \\
                                & $\varepsilon = 10$  & 0.17    & 0.13   & 0.11   & 0.10    & 0.06    & 0.05   & 0.22       & 0.15       & 0.14      \\
                                & $\varepsilon = 20$  & 0.16    & 0.13   & 0.09   & 0.10    & 0.07    & 0.03   & 0.21       & 0.16       & 0.12      \\
                                & $\varepsilon = 30$  & 0.16    & 0.12   & 0.09   & 0.10    & 0.06    & 0.03   & 0.21       & 0.17       & 0.12      \\
\midrule
\multirow{6}{*}{$\lvert \forest \rvert = 10$} & $\varepsilon = 0.1$ & 0.28    & 0.31   & 0.56   & 0.26    & 0.24    & --   & 0.32       & 0.31       & 0.47      \\
                                & $\varepsilon = 1$   & 0.24    & 0.35   & 0.25   & 0.24    & 0.20    & 0.18   & 0.28       & 0.29       & 0.26      \\
                                & $\varepsilon = 5$   & 0.14    & 0.15   & 0.19   & 0.10    & 0.09    & 0.09   & 0.16       & 0.19       & 0.21      \\
                                & $\varepsilon = 10$  & 0.13    & 0.12   & 0.15   & 0.07    & 0.06    & 0.07   & 0.14       & 0.16       & 0.17      \\
                                & $\varepsilon = 20$  & 0.12    & 0.10   & 0.11   & 0.06    & 0.03    & 0.04   & 0.14       & 0.13       & 0.13      \\
                                & $\varepsilon = 30$  & 0.12    & 0.09   & 0.09   & 0.06    & 0.03    & 0.03   & 0.14       & 0.12       & 0.11      \\
\midrule
\multirow{6}{*}{$\lvert \forest \rvert = 20$} & $\varepsilon = 0.1$ & 0.30    & 0.49   & --   & 0.29    & 0.38    & --   & 0.33       & 0.49       & --      \\
                                & $\varepsilon = 1$   & 0.31    & 0.26   & 0.48   & 0.33    & 0.26    & 0.28   & 0.30       & 0.29       & 0.49      \\
                                & $\varepsilon = 5$   & 0.15    & 0.20   & 0.29   & 0.14    & 0.19    & 0.20   & 0.18       & 0.22       & 0.35      \\
                                & $\varepsilon = 10$  & 0.12    & 0.15   & 0.20   & 0.08    & 0.08    & 0.11   & 0.15       & 0.18       & 0.23      \\
                                & $\varepsilon = 20$  & 0.10    & 0.12   & 0.15   & 0.07    & 0.06    & 0.09   & 0.13       & 0.15       & 0.19      \\
                                & $\varepsilon = 30$  & 0.10    & 0.10   & 0.12   & 0.06    & 0.05    & 0.07   & 0.13       & 0.13       & 0.15      \\
\midrule
\multirow{6}{*}{$\lvert \forest \rvert = 30$} & $\varepsilon = 0.1$ & 0.38    & 0.35   & --   & 0.37    & 0.29    & --   & 0.37       & 0.49       & --      \\
                                & $\varepsilon = 1$   & 0.38    & 0.27   & 0.53   & 0.38    & 0.28    & 0.32   & 0.34       & 0.30       & 0.48      \\
                                & $\varepsilon = 5$   & 0.19    & 0.26   & 0.25   & 0.17    & 0.27    & 0.23   & 0.21       & 0.27       & 0.30      \\
                                & $\varepsilon = 10$  & 0.13    & 0.15   & 0.32   & 0.09    & 0.10    & 0.18   & 0.18       & 0.19       & 0.28      \\
                                & $\varepsilon = 20$  & 0.11    & 0.12   & 0.21   & 0.08    & 0.08    & 0.10   & 0.15       & 0.14       & 0.24      \\
                                & $\varepsilon = 30$  & 0.10    & 0.11   & 0.19   & 0.07    & 0.07    & 0.09   & 0.14       & 0.13       & 0.21      \\
\bottomrule
\end{tabular}%
}
\end{table*}


\clearpage 

\section{Determining a Search Interval for Noise Values}\label{appendix:width_delta_search_interval}

Rather than modeling all possible (integer) random noise values $\noisevar_{\tree\node\class}$ within $\mathbb{Z}$, our CP model focuses on a (sufficient) predefined range $\{ -\delta,\dots, \delta \}$ such that $\mathbb{P}(\textrm{int}(\noisevar_{\tree\node\class}) \in \{ -\delta,\dots, \delta \} ) \geq 0.999$. In practice, we use $\delta = \lceil 12 / \varepsilon_\node \rceil$. 
As displayed in Figure~\ref{intervalle_recherche}, this value is suitable for all considered privacy budgets $\varepsilon_\node$. Indeed, it always leads to the definition of slightly wider search intervals for the variables modelling the noise, theoretically guaranteeing that the probability of the actual noise value being within it is greater than $0.999$.

\begin{figure}[htbp]
    \vskip 0.2in
    \begin{center}
    \centerline{\includegraphics[width=0.6\columnwidth]{epsilon_delta.pdf}}
    \caption{Width of $\Delta_{\tree\node\class }$ search interval as a function of $\varepsilon_v$. 
    As expected, the magnitude of the noise added decreases when the privacy budget increases, resulting in a smaller search interval.}
    \label{intervalle_recherche}
    \end{center}
    \vskip -0.2in
\end{figure}


\section{Trade-offs between Predictive Performance and Reconstruction Error -- Detailed Results}\label{appendix:complete_results_tradeoffs_perf_reconstr_error}

Figures~\ref{fig:results_all_tradeoffs_predictive_perf_vs_recontr_error} and \ref{fig:results_all_tradeoffs_predictive_perf_vs_recontr_error_bis} present the detailed trade-offs between the RFs' predictive performance and the empirical reconstruction error of our attack across all experiments. These figures confirm the key trends discussed in the main paper, particularly \textbf{Results 3 \& 4}, and are consistent across all datasets, privacy budgets, tree depths, and forest sizes. In particular, most RFs are located either in the top-left or bottom-right corners of the plots. The top-left region corresponds to RFs with non-trivial predictive performance, giving some advantage to the reconstruction attack, while the bottom-right region includes RFs that provide no significant advantage for the reconstruction attack but perform worse than a majority classifier. Interestingly, the few RFs that exhibit non-trivial predictive performance while remaining resilient to reconstruction attacks (top-right corner) are either the sparsest ($\lvert \forest \rvert = 1$) or the largest ($\lvert \forest \rvert = 20$ or $\lvert \forest \rvert = 30$). 

\begin{figure*}[htb]
\begin{subfigure}{0.5\textwidth}
    \centering
    \includegraphics[width=1.0\linewidth]{N_fixed_adult_train_accuracy_vs_error_depth_3.pdf}
    \caption{UCI Adult Income dataset, depth 3 decision trees.} \label{fig:results_tradeoffs_acc_reconstr_adult_depth_3}
\end{subfigure}%
\hfill
 \begin{subfigure}{0.5\textwidth}
    \centering
    \includegraphics[width=1.0\linewidth]{N_fixed_adult_train_accuracy_vs_error_depth_5.pdf}
    \caption{UCI Adult Income dataset, depth 5 decision trees.}
\label{fig:results_tradeoffs_acc_reconstr_adult_depth_5}
\end{subfigure}%

\vspace{10pt}

\begin{subfigure}{0.5\textwidth}
    \centering
    \includegraphics[width=1.0\linewidth]{N_fixed_adult_train_accuracy_vs_error_depth_7.pdf}
    \caption{UCI Adult Income dataset, depth 7 decision trees.}
    \label{fig:results_tradeoffs_acc_reconstr_adult_depth_7}
\end{subfigure}
\hfill
\begin{subfigure}{0.5\textwidth}
    \centering
    \includegraphics[width=1.0\linewidth]{N_fixed_compas_train_accuracy_vs_error_depth_3.pdf}
    \caption{COMPAS dataset, depth 3 decision trees.}
    \label{fig:results_tradeoffs_acc_reconstr_COMPAS_depth_3}
\end{subfigure}

\vspace{10pt}

\begin{subfigure}{0.5\textwidth}
    \centering
    \includegraphics[width=1.0\linewidth]{N_fixed_compas_train_accuracy_vs_error_depth_5.pdf}
    \caption{COMPAS dataset, depth 5 decision trees.}
    \label{fig:results_tradeoffs_acc_reconstr_COMPAS_depth_5}
\end{subfigure}
\hfill
\begin{subfigure}{0.5\textwidth}
    \centering
    \includegraphics[width=1.0\linewidth]{N_fixed_compas_train_accuracy_vs_error_depth_7.pdf}
    \caption{COMPAS dataset, depth 7 decision trees.}
    \label{fig:results_tradeoffs_acc_reconstr_COMPAS_depth_7}
\end{subfigure}

\vspace{10pt}

\begin{subfigure}{\textwidth}
    \centering
    \includegraphics[width=0.6\linewidth]{legend_tradeoffs.pdf}
\end{subfigure}
\caption{Average training accuracy of $\varepsilon$-DP RFs as a function of the reconstruction error of our attack for different numbers of trees $\vert \forest \rvert$ and privacy budgets $\varepsilon$, for the considered datasets and tree depths values.}\label{fig:results_all_tradeoffs_predictive_perf_vs_recontr_error}
\end{figure*}

\begin{figure*}[t!] %\ContinuedFloat
\begin{subfigure}{0.5\textwidth}
    \centering
    \includegraphics[width=1.0\linewidth]{N_fixed_default_credit_train_accuracy_vs_error_depth_3.pdf}
    \caption{Default of Credit Card Clients dataset, depth 3 decision trees.}
    \label{fig:results_tradeoffs_acc_reconstr_default_credit_depth_3}
\end{subfigure}
\hfill
\begin{subfigure}{0.5\textwidth}
    \centering
    \includegraphics[width=1.0\linewidth]{N_fixed_default_credit_train_accuracy_vs_error_depth_5.pdf}
    \caption{Default of Credit Card Clients dataset, depth 5 decision trees.}
    \label{fig:results_tradeoffs_acc_reconstr_default_credit_depth_5}
\end{subfigure}

\vspace{10pt}

\hfill
\begin{subfigure}{0.5\textwidth}
    \centering
    \includegraphics[width=1.0\linewidth]{N_fixed_default_credit_train_accuracy_vs_error_depth_7.pdf}
    \caption{Default of Credit Card Clients dataset, depth 7 decision trees.}
    \label{fig:results_tradeoffs_acc_reconstr_default_credit_depth_7}
\end{subfigure}
\hfill
\vspace{10pt}

\begin{subfigure}{\textwidth}
    \centering
    \includegraphics[width=0.6\linewidth]{legend_tradeoffs.pdf}
\end{subfigure}

\caption{Average training accuracy of $\varepsilon$-DP RFs as a function of the reconstruction error of our attack for different numbers of trees $\vert \forest \rvert$ and privacy budgets $\varepsilon$, for the considered datasets and tree depths values (continued).}\label{fig:results_all_tradeoffs_predictive_perf_vs_recontr_error_bis}
\end{figure*}


\bibliographystyle{IEEEtran}
\bibliography{bib/egdb, bib/deepfake, bib/fullbody}

% that's all folks
\end{document}


