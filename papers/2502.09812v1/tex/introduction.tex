\section{Introduction}%\textcolor{red}{RESPONSIBILITY - ALL}
On the 1st of June, 2019 artists Bill Posters and Daniel Howe released a deepfake video \footnote{https://billposters.ch/the-zuckerberg-deepfake-heard-around-the-world/} that featured Mark Zuckerberg delivering a speech about the power of Facebook and its control over user data. This synthesised video was generated to raise awareness regarding deepfakes and their potential implication. Soon after its release, this video made it into the major global news platforms, including, the New York Times, ABC News, and BBC, and sparked discussions about the ethical and social implications of deepfake technology.

% \begin{figure}[htbp]
%     \centering
%     \includegraphics[width=.8\linewidth]{figures/mark_zuckerberg_deepfake.png}
%     \caption{``Imagine This…'' deepfake video of Mark Zuckerberg generated by artists Bill Posters and Daniel Howe in 2019. Image taken from {\tiny \footnotemark} }
%     \label{fig:mark_zuckerberg_deepfake}
% \end{figure}


It has been five years since the release of the Mark Zuckerberg deepfake video and during this period there have been significant technological advancements in deepfake generation technology On the other hand,  advances in deepfake detection have lagged behind and there has been very little response, over this 5-year duration from regulators and educators to educate and safeguard society from the malicious use of deepfake technology. The principal aim of this study is to provide up-to-date algorithmic insights regarding face deepfake generation and detection processes.  Our systematic analysis is not only beneficial to researchers and machine learning practitioners but is also of pertinent interest to the general public and policymakers and we expect this review to raise awareness among these groups regarding both positive and negative implications of deepfakes.

%\footnotetext{https://billposters.ch/the-zuckerberg-deepfake-heard-around-the-world/}

\subsection{What are deepfakes?}
The term deepfakes has been used to refer to any synthetic media that has been generated using deep learning techniques, including the media generated using generative AI technology. However, a clear distinction exists between generative AI and deepfakes generation when considering their purpose. The purpose of generative AI is mainly to generate synthetic content by analysing the patterns of the real-data while deepfakes are primarily designed to generate realistic-looking content with the main aim to fool the users of that media into thinking it is real. Deepfakes first came into focus in 2017 when a Reddit user released fake celebrity pornographic videos generated using the deepfake technology.

%============ Timeline goes here ===============%


\subsection{Why is it important to be educated about deepfakes?}
The 2024 global risk report \cite{wef2024} of the World Economic Forum identified that misinformation and disinformation are the biggest short-term risks to the world economy. A Boston University article \cite{Weinstein2021} suggested that international and domestic disinformation campaigns targeting Americans are America’s greatest national security threat, which is a greater danger than the nuclear capabilities of Russia, China, and North Korea. Deepfake technology plays a major role in generating highly convincing but entirely fabricated faces, voices, and text making it difficult for people to discern truth from fiction. Therefore, understanding the process of creation, the impact of deepfakes, and a study of the existing detection technologies in terms of their effectiveness is crucial in today's digital age due to the rapidly growing presence of deepfakes. 

While deepfakes can span various mediums of communication including text, audio, images, and video, in this article we limit our discussion to images, and video-based deepfakes of human faces as they are the most powerful mediums that the perpetrators can leverage to spread misinformation, manipulate public opinion, and deceive individuals. Understanding deepfakes would help organisations to analyse the potential vulnerabilities of the systems they utilise and apply mechanisms to mitigate these threats. Moreover, by staying informed about deepfake technology society can be better prepared for its negative implications and be vigilant by critically evaluating the media they consume. Furthermore, a better understanding of both the positive and negative impacts of deepfake technology is essential for recognising the potential for misuse of this technology, promoting responsible practices, and fostering ethical development and use of this technology. Our review paper takes an important step toward this by providing a comprehensive and systematic analysis of existing literature on face deepfake generation and detection with a special focus on their implications in biometric recognition. 


\subsection{How is our study different from existing surveys?}

While there exist several surveys that discuss the generation and detection of face deepfakes, we observe a lack of studies that provide an in-depth algorithmic overview including a discussion regarding training paradigms, the loss functions, and evaluation metrics. In Tab. \ref{tab:comparison_to_other_surveys} we summarise the main topics that our paper covers and compare it with the coverage of existing works.

\begin{table*}[htbp]
\caption{Comparison of our survey to other related studies. Note: $*$ indicates a comprehensive discussion, and $+$ indicates just an overview.}
 \resizebox{\textwidth}{!}{%
\begin{tabular}{|c|l|c|c|c|c|c|c|}
\hline
Paper  &Year& Face Deepfakes Generation &Face Deepfakes Detection&Algorithimic Discussion& Applications of Deepfakes & Impact on Biometric Recongition  &Future Research Directions\\ \hline


    
    Dagar et. al \cite{dagar2022literature}  &2022&  \cmark ($*$)&\cmark($*$)&\cmark ($+$)&                          \cmark ($+$)&    \xmark                               &\cmark ($+$)\\ \hline
   Nguyen \cite{nguyen2022deep}    &2022&  \cmark($*$)&\cmark($*$)&\cmark($+$)&     \xmark                      & \xmark                                 &\cmark ($+$)\\ \hline
  Waseem et. al \cite{waseem2023deepfake} &2023&  \cmark ($*$)&\cmark ($*$)&\cmark($+$)&  \xmark                          &      \xmark                              &\cmark($*$)\\ \hline
 Masood et. al \cite{masood2023deepfakes}     &2023&  \cmark($*$)&\cmark($*$)&\xmark  &          \xmark                 &       \xmark                           &\cmark ($+$)\\ \hline
 Mubarak et. al \cite{mubarak2023survey}     &2023&  \xmark  &\cmark($*$)&\xmark  &          \xmark                 &       \cmark ($+$)&\xmark                           \\ \hline
  
   Patel et. al \cite{patel2023deepfake}     &2023&  \cmark($*$)&\cmark($*$)&\cmark ($*$) &          \xmark                 &       \xmark                           &\xmark                           \\ \hline
    Wang et. al \cite{wang2024deepfake}     &2024&  \xmark  &\cmark($*$)&\cmark ($+$)&          \xmark                 &       \xmark                           &\xmark                           \\ \hline
    Heidari et. al \cite{heidari2024deepfake}     &2024&  \xmark  &\cmark($*$)&\xmark&          \xmark                 &       \xmark                           &\cmark ($+$)\\ \hline
   Passos et. al \cite{passos2024review}     &2024& \xmark  & \cmark($*$)&\cmark ($+$)&          \xmark                 &       \xmark                           &\cmark ($+$)\\ \hline
   Sharma et. al \cite{sharma2024systematic}     &2024&  \xmark & \cmark($*$)& \cmark ($*$) &  \cmark ($+$)        &       \xmark                           &\cmark ($*$)\\\hline
    Ours    &2025&  \cmark($*$) & \cmark($*$)& \cmark ($*$) &  \cmark($*$)       &      \cmark($*$)                      &\cmark($*$)\\\hline
 
\end{tabular}}
\label{tab:comparison_to_other_surveys}
\end{table*}



\subsection{Organisation}
The rest of our paper has the structure illustrated in Fig. \ref{fig:organisation}. Sec.  \ref{sec:face_deepfakes} discusses face deepfakes, with separates subsection devoted for both the generation and detection technologies. Under generation of face deepfakes, we first introduce different deepfake types, provide an overview of the deepfake generation process, illustrate popular evaluation metrics used to evaluate the quality of the generated deepfakes, and review the state-of-the-art approaches for the generation of deepfakes. Under the detection subsection of face deepfakes, we first analyse the features that have been leveraged to detect those deepfakes. Then we conduct a review of state-of-the-art methodologies for deepfake detection and introduce the evaluation metrics that have been introduced to evaluate the detection performance. Moreover, we quantitatively analyse the efficacy of the generated deepfake media to fool the state-of-the-art biometric recognition models. In addition, we also discuss the methodologies that have been introduced to uncover the true identity from the manipulated media. Furthermore, Sec. \ref{sec:applications} of our paper dicusses the applications of deepfakes, including both positive and negative applications. Sec. \ref{sec:future_research}  illustrates some of the challenges and limitations of existing literature on deepfakes frameworks, and provides future directions to pursue. Sec. \ref{sec:conclusions} contains concluding remarks.

\begin{figure*}[htbp]
    \centering
    \includegraphics[width=\linewidth]{figures_new/overview_figure.pdf}
    \caption{The Organization of this Survey}
    \label{fig:organisation}
\end{figure*}