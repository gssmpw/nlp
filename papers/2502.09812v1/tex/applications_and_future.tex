\section{Applications of deepfakes} \label{sec:applications}
In this section, we present a spectrum of positive applications of deepfakes ranging from fashion to the entertainment industry. Moreover, the ethical, psychological, and security implications of deepfakes are also discussed in this section. 

\subsection{Positive commercial applications}

Despite its controversial reputation, the deepfake technology holds great potential for positive commercial applications across a myriad of industries. In addition to the reenactment of passed actors with hyper-realistic visual quality, which we see being readily used in the entertainment industry, there are various current and future use cases of deepfake technology across a diverse set of applicational areas if this technology is used ethically and responsibly. We use this section to introduce such sample use cases.  

\subsubsection{Fashion and beauty industry:}
With the rise of deepfake technology, the way that leading brands engage with consumers has been revolutionised. In 2021 Kati Chitrakorn, the Vogue business technology expert predicted that deepfake technology would transform the landscape of the fashion industry \cite{deepfakevoguebusiness}. 

For instance, \textbf{digital fashion shows and influencer collaborations} has now become a reality \cite{deepfakeglamour}. The onset of the COVID-19 pandemic only accelerated the use of deepfake technology enabling the collaborations and interactions between people in a setting where in-person activities are restricted. Specifically, Demna Gvasalia's `deepfake' Spring 2022 Balenciaga fashion show illustrated how deepfake technology can be used to model garments by celebrities or influencers with just a few sets of images without the need to even be there in person.  

In the 2019 London Fashion Week, some selected sets of participants were able to watch themselves wearing HANGER’s latest collection using deepfake technology. These \textbf{virtual try-ons} were being projected behind the live models enabling the members of the audience to see their appearance if they were actually wearing the garments that were being modelled. Superpersonal \footnote{https://www.producthunt.com/products/superpersonal} is an app that has been specifically designed to allow users to try on clothes virtually. This allows the consumers to visualize products before purchase and understand which styles fit their taste better. Several other innovations have also emerged in this direction. Fxgear's FXMirror \footnote{https://www.fxgear.net/vr-fashion} which provides an augmented-reality fitting room experience such that the shoppers can try on clothes virtually and YourFit solution \footnote{https://3dlook.ai/yourfit/} by 3DLOOK are only a few examples. 

\subsubsection{Marketing industry}
 
Furthermore, the flexibility and customisability of the synthesised media have no limits. As such \textbf{hyper-personalised advertisements} could be generated targeting consumers in different demographics by customising the clothing, voice, and location of characters in the advertisement. Unethical and fraudulent use of this technology was experienced in Taylor Swift's Deepfake Campaign where her image was used without permission for the creation of advertising media that promoted products she did not endorse. However, this technology can be ethically used with the permission of the celebrity or the influencer to create highly influential advertisements that could reach a far greater audience with minimal cost. The 2019 malaria awareness deepfake advertisement that featured David Beckham \footnote{https://youtu.be/QiiSAvKJIHo?si=RuNnN5hE1R78JTCQ} is a prime example that demonstrates this marketing potential. In this video, David Beckham speaks in nine different languages appealing to end malaria and multimodal deepfake technology has made Beckham appear multilingual. German online retail giant Zalando has also been readily using deepfake technology in its marketing campaigns. For example, the deepfake technology enabled supermodel Cara Delevingne to appear in 290,000 localised advertisements \cite{deepfakevoguebusiness}. deepfake technology in marketing can be further extended to reenact historical figures and bring them to life with contemporary public figures. This will \textbf{enhance the storytelling} of the marketing materials and better capture the attention of the targetted consumers. 

The deepfake technology is transforming the marketing industry as a whole. The deepfake technology reduces production costs by cutting down the costs associated with the hiring of the production crews, including, videographers, camera operators, media editors, casting assistants, and directors. Furthermore, it doesn't need a location to shoot the videos or equipment to record them. Hour One \footnote{https://hourone.ai/} is a company that readily uses deepfakes to create commercial media. In one of their advertisement campaigns human actors are replaced by animated digital clones of real humans generated by deepfakes \cite{deepfaketechnologyreview}. The impracticality of using real actors and production crews to create thousands of videos in different languages has led Hour One to opt for deepfakes.

\subsubsection{Corporate training, simulation, and virtual assistants}

The British multinational company WPP has created training videos together with Synthesia, a synthetic media generation company, targeting its employees \cite{deepfakewired}. These videos have been sent to thousands of employees that WPP has worldwide and address the employee by name and explain some basic concepts in AI. This example shows how deepfake technology can be used to create \textbf{corporate training} material in a personalised and cost-effective manner, which is otherwise impractical for a multinational company. However, the merits of deepfake technology in corporate training and simulation settings surpass this simple application. For instance, it can be used to create highly realistic customer interactions or crisis management situations for training purposes. The AI models are adaptable and they can dynamically change their responses based on the responses of the trainee, providing a personalised training experience. Moreover, the deepfake technology provides a risk-free training environment for areas such as law enforcement and healthcare, therefore, trainees can practice decision-making in critical settings.  

The Live Interactive Customer Experience (ALICE) receptionist \footnote{https://www.alicereceptionist.com/} is an ideal example of how \textbf{virtual receptionist kiosk} can be used to handle visitors' queries, replacing the role of a human receptionist. We incorporate the virtual receptionist application under the deepfake category considering the use of deepfake technology to create human-like avatars that represent the virtual receptionist. Furthermore, this virtual assistant is capable of conducting a range of visitor management tasks, including, pre-visit check-ins, visitor screening, driver's license scanning, and body temperature check. These virtual receptionist kiosks are currently being used in various American International Group and ING Group branches.


\subsubsection{ Entertainment industry: }

deepfake technology is widely being used for \textbf{dubbing or revoicing media in the entertainment industry}. This allows the synchronise facial expressions, lip movements, and expression of emotions after the dubbing process. An AI-driven startup company named Flawless \footnote{https://www.flawlessai.com/} is generating deepfake dubs which is cost-effective and efficient, and help the media content reach new audiences. Compared to traditional dubbing methods which have mistimed mouth movement, Flawless utilises deepfake technology to artificially synthesise lip movements that match the translated speech. The result is a much smoother revoicing of the media. 

The movie industry is heavily utilising Computer-Generated Imagery (CGI) to create visual effects. This is a meticulous process done by visual effects artists. The deepfake technology has the potential to automate this process by generating different renders of the chosen character automatically. The gaming industry has started adapting the deepfake technology. For instance, in the video game Cyberpunk 2077 \footnote{https://www.cyberpunk.net/au/en/} where celebrities play roles in the game. We believe this technology will soon seep into the film industry as well giving more potential to the filmmakers, enabling scenarios such as reenacting historical events or bringing characters to life.  

\hspace{2mm}

\subsection{Negative implications}
This subsection summarises the primary negative implications of deepfake technology and highlights the need for the urgent need for countermeasures. 

\subsubsection{Misinformation and disinformation propaganda}

In mid-March 2022 a deepfake video of Ukrainian President Volodymyr Zelenskyy appeared on social media calling on the Ukrainians to stop fighting and to surrender their weapons \cite{deepfakenortheastern}. This video was even broadcast on the Ukrainian television channel Ukraine 24 by a team of hackers. 

The harmful effects of mis/disinformation generated through deepfakes do not stop from fake news. It can be used to impact elections, perform corporate sabotage with well-times and articulated falsifying evidence, and harm the image of public figures. Most importantly these mis/disinformation campaigns could deteriorate public trust regarding the authenticity of genuine material in mainstream media. For instance, when the Princess of Wales, released a video statement in March 2024 sharing that she had been diagnosed with cancer a fresh round of conspiracies regarding deepfakes reappeared in social media \cite{deepfakewashingtonpost}. However, this time it was people disbelieving a real video. These real-world examples clearly elaborate the growing threat of multimodal deepfake to society in an era where seeing is not believing.

Apart from these recent examples, there are other evidence of deepfake being used to spread mis/disinformation. It has been suggested that a deepfake story could have sparked the diplomatic confrontation between Saudi Arabia and Qatar \cite{deepfakebrookings1}. The unnatural speech of President Ali Bongo sparked a military coup in 2018 in Gabon claiming that the video was a deepfake and the president was no longer healthy enough for the office or even had died \cite{deepfakeforbes}. Furthermore, there was an unsuccessful attempt to discredit and overthrow Malaysia’s economic affairs minister using deepfake-based fabricated media \cite{deepfakewii}. 

Moreover, deepfakes can be used to create fake influences or endorsements. For instance, fraudulent Taylor Swift advertisements that promoted a cookware brand on social media are a prime example of such fraudulent narratives. Based on our review of deepfake detection methods, there is no universal deepfake detector that could suffice and withstand all the advances of current and future deepfake generation technology, and until such robustness is met our society faces ongoing threats due to the malicious use of deepfake technology. 


\subsubsection{Psychological impact}
The psychological impact of deepfakes is quite concerning as it affects not only individuals on a personal level but also at social and societal levels. 

Deepfakes disrupt our ability to believe what we perceive. Therefore, it could lead to deterioration of trust of people regarding news and media in general. This effect occurs even if the deepfake is unsuccessful in misleading a particular individual. The sense of deception leads to increased skepticism and uncertainty in our daily online and offline interactions. For instance, a study by Vaccari and Chadwick \cite{vaccari2020deepfakes} found that even if a person is not completely misled by a deepfake the exposure to it reduces their trust in news.

Another study found alarming evidence of deepfakes modifying our memories and even implanting false memories. For instance, in \cite{dobber2021microtargeted} the authors found that watching deepfake videos could result in participants falsely remembering nonexistent films. Furthermore, it could lead to a change in one's attitude. Specifically, the authors of \cite{dobber2021microtargeted} constructed multimodal deepfakes and have shown them to a selected group of individuals to see if there is any change in their attitudes toward the politician and the attitudes toward his or her political party. This study revealed that microtargeting the deepfake to groups that are most likely to be offended could amplify its negative implications.   Furthermore, we should consider the profound emotional impact of deepfakes on the individuals whose identities have been maliciously depicted in the video. The fabricated media could be embarrassing, offensive, and damaging to their reputation, leading to anxiety, and even altering their beliefs and behaviour.

\subsubsection{National security threat of deepfakes} 

Lt. Gen. Jack Weinstein who is the deputy chief of staff for strategic deterrence and nuclear integration at the United States (US) Air Force Pentagon headquarters stated ``The greatest existential threat to the United States of America is the fracturing of our democracy and the intentional misleading of facts to support political agendas''. Therefore, it is clear that false or misleading information that is deliberately spread to deceive a population could cripple the world's largest economy and the second-largest democracy. Furthermore, based on the press release of The US National Security Agency on September 12, 2023 \cite{deepfakeNSA}, synthetic media can cause public unrest through the spread of false information about political, social, military, or economic issues. This report states that public availability of the implementation of deepfake generation algorithm has made mass production of fake media easier and less expensive, which has broadened their impact to a larger scale.  

The national security threat of deepfakes also includes cyber espionage through impersonation where deepfake technology can be used to fabricate fake communications of high-ranking officials. While to the best of our knowledge, this has not occurred to date, there exists evidence to the use of audio deepfakes has been used to steal personally identifiable information during fake online interviews of potential applicants \cite{deepfakeFBI}. This information can be used to create fake credentials to gain access to sensitive information or critical infrastructure systems. 

Moreover, one should consider the economic impact of deepfake as the economy is closely associated with national security. This is clearly evident by the findings in the 2024 global risk report of the  World Economic Forum. This report states that misinformation and disinformation are the biggest short-term risks to the world economy. For instance, meticulously targeted false information about large corporations of a certain country could be used to disrupt markets or manipulate stock prices leading to economic instability in that particular country. 

Government agencies, researchers, and policymakers should continue to collaborate together to minimise the threat of deepfakes to national and global security. Furthermore, mainstream media has a major role in promoting the awareness and literacy of the general public regarding the threat of deepfakes and how to spot them.

\subsubsection{ Privacy violation} 
A popular example of deepfakes in privacy violation is the creation and distribution of pornographic material by swapping an individual's face, voice, and body into real pornography. For instance, the Reddit user made deepfake sex videos of female celebrities using their images and videos. However, the potential victims are not limited to public figures. It can be used to generate revenue or targeted attacks against specific individuals, such as ex-partners, or rivals which is an invasion of sexual privacy \cite{citron2018sexual}.

The impact of such activities is not limited to emotional distress, reputational damage, and personal trauma. Blackmailers could use deepfakes to extortion. The victims may be forced to provide money or even business secrets to prevent the release of the deepfakes. Several legislative changes have been proposed to protect victims from privacy-related issues caused by deepfakes, however, it has been identified that there exist several issues when dealing with deepfakes in litigation and governments need to take more actions to protect victims \cite{deepprivacy}.


\hspace{3mm}
\section{Future research directions} \label{sec:future_research}
 
We refer the readers to Sec. 4 in the supplementary material, where we discuss future research directions, including the design of universal deepfake detection methods, recovering the true identity, explainable deepfakes detection methods, introducing standard evaluation protocols, and design of a regulatory framework for the governance of deepfake research. 

\section{Conclusion}\label{sec:conclusions}
In this survey paper, we have discussed existing state-of-the-art methods for the generation and detection of face deepfakes. Our analysis emphasises an algorithmic perspective, providing an in-depth discussion of the architectures, and include details such as training paradigms, loss functions and evaluation metics. 
In addition, we have discussed the biometric implications of the generated face deepfakes and we have provided in-depth discussion regarding their positive and negative applications. As concluding remarks, we outlined key research gaps and proposed possible future research directions for further investigation. 