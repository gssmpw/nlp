\section{Related Work}\label{sec:related}

\paragraph{Noise Conditioning.} The seminal work of diffusion models \cite{sohl2015diffusion} proposes iteratively perturbing clean data and learning a model to reverse this process. In this pioneering work, the authors introduced a ``\textit{time dependent readout function}'', which is an early form of noise conditioning.

The modern implementation of noise conditioning is popularized by the introduction of \textit{Noise Conditional} Score Networks (NCSN) \cite{song2019ncsn}. NCSN is originally developed for score matching.
This architecture is adopted and improved in Denoising Diffusion Probabilistic Models (DDPM) \cite{ho2020denoising}, which explicitly formulate generation as an iterative denoising problem. The practice of noise conditioning has been inherited in iDDPM \cite{nichol2021iddpm}, ADM \cite{dhariwal2021diffusion}, and nearly all subsequent derivatives.

DDIM \cite{song2021ddim} and EDM \cite{karras2022edm} reformulate the reverse diffusion process into an ODE solver, enabling deterministic sampling from a single initial noise. Flow Matching (FM) models \cite{lipman2023flow,liu2023flow,albergo2023stochastic} reformulate and generalize the framework by learning flow fields that map one distribution to another. In all these methods, noise conditioning (also called time conditioning) is the \textit{de facto} choice.

Beyond diffusion models, Consistency Models \cite{song2023consistency} have emerged as a new family of generative models for non-iterative generation. It has been found \cite{song2024improved} that noise conditioning and its implementation details are critical for the success of consistency models, highlighting the central role of noise conditioning.


\paragraph{Blind Image Denoising.} In the field of image processing, blind image denoising has been studied for decades. It refers to the problem of denoising an image without any prior knowledge about the level, type, or other characteristics of the noise. Relevant studies include noise level estimation from noisy images \cite{stahl2000quantile,shin2005block,liu2013single,chen2015efficient}, as well as directly learning to perform blind denoising from data \cite{liu2007automatic,chen2018image,batson2019noise2self,zhang2023blind}. Modern neural networks, including the \mbox{U-Net} \cite{ronneberger2015u} commonly used in diffusion models, have been shown highly effective for these tasks.

Our research is closely related to classical work on blind denoising. However, the iterative nature of the generative process, where errors can accumulate, introduces new challenges. In addressing these challenges, our work opens up new research opportunities that extend classical approaches.






