\section{Discussion and Conclusion}\label{sec:conclusion}

We hope that rethinking the role of noise conditioning will open up new opportunities. 
Modern diffusion models are closely related to Score Matching \cite{hyvarinen2005estimation,song2019ncsn,song2021scorebased}, which provides an effective solution to Energy-Based Models (EBM) \cite{hopfield1982neural,ackley1985learning,lecun2006tutorial,song2021train}. The key idea of EBM is to represent a probability distribution $p(x)$ by $p(x) = e^{-E(x)} / Z$, where $E(x)$ is the energy function. With the score function of $p(x)$ (that is, $\nabla_x E(x)$), one can sample from the underlying $p(x)$ by Langevin dynamics. This classical formulation models the data distribution $p(x)$ by a \textit{single} energy function $E(x)$ that is solely dependent on $x$. Therefore, a classical EBM is inherently $t$-unconditional. However, with the presence of $t$-conditioning, the sampler becomes \textit{annealed} Langevin dynamics \cite{song2019ncsn}, which implies \textit{a sequence} of energy functions $\{E(x, t)\}_t$ indexed by $t$, with one sampling step performed on each energy. Our study suggests that it is possible to pursue a \textit{single} energy function $E(x)$, aligning with the goal of classical EBM.

Our study also reveals that certain families of models, \eg, Flow Matching \cite{lipman2023flow,liu2023flow,albergo2023stochastic}, can be more robust to the removal of $t$-conditioning. Although these models are closely related to diffusion, they can be formulated from a substantially different perspective---estimating a flow field between two distributions. While these models can inherit the $t$-conditioning design of diffusion models, their native formulation does not require the flow field to be dependent on $t$. Our study suggests that there exists a \textit{single} flow field for these methods to work effectively.

In summary, noise conditioning has been predominant in modern denoising-based generative models and related approaches. We encourage the community to explore new models that are not constrained by this design.

