\label{sec:method}

In this section, we introduce the method used to conduct the investigation on a set of \pc papers that discuss relevant bias issues.
Specifically, to construct the initial set of relevant work, we search the keywords ``bias" or ``fair" in the title of papers from NeurIPS, ICML, ICLR and FAccT published before February 2025. 
We include papers that discuss bias issues whose manifestation aligns with either Type I Bias or Type II Bias (we will detail the unification in~\cref{sec:unifying}).
We exclude papers that address other bias issues such as inductive bias~\cite{baxter2000model,zietlow2021demystifying}, implicit bias~\cite{fitzgerald2017implicit,camuto2021asymmetric}, selection bias~\cite{hernan2004structural,akbari2021recursive}, sampling bias~\cite{winship1992models,xu2022alleviating}, spectral bias~\cite{fang2024addressing}, exposure bias~\cite{li2024alleviating} or bias-variance~\cite{ha2024fine, chen2024on}.
Furthermore, to ensure we do not overlook any relevant papers without these keywords or from other prominent conferences such as CVPR, ICCV, and ECCV, we manually traversal the citation graph of the paper in the initial set and append the relevant papers that are either cited by or cite the papers in the initial set.






Once we identify the scope of the investigated papers, we read these papers to determine which type of bias they address by examining two aspects: problem statement and evaluation protocol.
We will elaborate on the criterion for categorizing papers into our definitions in~\cref{sec:unifying}.
To accommodate the recent emerging direction of addressing unlabeled and unknown bias, we enrich the taxonomy with an additional dimension about the status of attribute $A$.
As shown in~\cref{tab:taxonomy}, we count the number of papers in each category. 
Note that the total number is not equal to \pc since some papers address both types of biases.
We present the categorization list of all \pc investigated papers in Appendix.


\begin{table}[htbp]
\caption{The taxonomy of bias issues based on \pc papers.}
\label{tab:taxonomy}
\centering
\resizebox{0.45\textwidth}{!}{%

\begin{tabular}{lcccc}
\toprule
\multirow{2}{*}{Type of Bias} & \multicolumn{2}{c}{Attribute $A$} & \multirow{2}{*}{Papers} & \multirow{2}{*}{Examples}                                                   \\
\cmidrule(lr){2-3} 
                              & Known           & Labeled         &                         &                                                                             \\
                              \midrule
\multirow{3}{*}{Type I Bias}  & \cmark          & \cmark          & 253                     & \cite{DebFace,GAC,RL_RBN}                                                   \\
                              & \cmark          & \xmark          & -                       & -                                                                           \\
                              & \xmark          & \xmark          & -                       & -                                                                           \\
                              \midrule
\multirow{3}{*}{Type II Bias} & \cmark          & \cmark          & 246                     & \cite{learn_not_to_learn_Colored_MNIST,CSAD,End}                            \\
                              & \cmark          & \xmark          & 8                       & \cite{HEX_texture_bias1, ReBias_texture_bias2,rubi} \\
                              & \xmark          & \xmark          & 30                      & \cite{LfF_CelebA_Bias_conflicting,ECS,UBNet}                               \\
                              \midrule
Survey                        & -               & -               & 25                       & \cite{MLbias_survey,prediciton_quality_disparity,discussion_on_DP_EO}      \\
\bottomrule
\end{tabular}
}

\end{table}

