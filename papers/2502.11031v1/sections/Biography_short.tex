\vspace{-12mm}

\begin{IEEEbiography}[{\includegraphics[width=1in,height=1.25in,clip,keepaspectratio]{sections/figures/Jiazhi.jpeg}}]{Jiazhi Li}
received his Bachelor's degree in Electrical Engineering from Beijing Institution of Technology in Beijing, China in 2018 and his Master's degree in Electrical Engineering from University of Southern California, USA in 2020. Currently, he is a Ph.D. student at the Department of Electrical and Computer Engineering, and a Graduate Research Assistant at Information Sciences Institute, both being units of USC Viterbi School of Engineering, under the supervision of Prof. Jieyu Zhao and Prof. Wael AbdAlmageed. His research interests include machine learning fairness and generative models.
\end{IEEEbiography}

\vspace{-12mm}

\begin{IEEEbiography}[{\includegraphics[width=1in,height=1.25in,clip,keepaspectratio]{sections/figures/mahyar.jpg}}]{Mahyar Khayatkhoei}
is a Computer Scientist at USC Information Sciences Institute. He received his B.Sc. in electrical engineering from the University of Tehran, and his M.Sc. and Ph.D. in computer science from Rutgers University. His research focuses on identifying and measuring the biases and limitations of deep neural networks in general, and the theory and application of deep generative models in particular.
\end{IEEEbiography}

\vspace{-14mm}

\begin{IEEEbiography}[{\includegraphics[width=1in,height=1.25in,clip,keepaspectratio]{sections/figures/Jiageng.jpeg}}]{Jiageng Zhu}
is a Ph.D. student at USC Ming Hsieh Department of Electrical and Computer Engineering and a Graduate Research Assistant at USC Information Sciences Institute. His current research interests focus on Causal Representation Learning, Disentanglement and Invariant Representation Learning, and Dynamics Prediction.
\end{IEEEbiography}

\vspace{-18mm}

\begin{IEEEbiography}[{\includegraphics[width=1in,height=1.25in,clip,keepaspectratio]{sections/figures/Hanchen.jpeg}}]{Hanchen Xie}
is a Ph.D. candidate at USC Thomas Lord Department of Computer Science and a Graduate Research Assistant at USC Information Sciences Institute; both are units of USC Viterbi School of Engineering. His research interests include representation learning under less labeled data scenarios (\eg Semi-Supervised Learning, Few-Shot Learning, and Zero-Shot Learning), generative networks, and Dynamics Prediction.
\end{IEEEbiography}

\vspace{-15mm}

\begin{IEEEbiography}[{\includegraphics[width=1in,height=1.25in,clip,keepaspectratio]{sections/figures/Mohamed.jpeg}}]{Mohamed E. Hussein}
is a Computer Scientist and Research Lead at USC ISI, and an Associate Professor (on leave) at Alexandria University, Egypt. He obtained his Ph.D. degree in Computer Science from the University of Maryland at College Park in 2009, specializing in computer vision and GPU computing.
His current research interest is in mitigating the vulnerabilities of AI models to spoofing attacks, adversarial attacks, and domain shifts.
\end{IEEEbiography}

\vspace{-15mm}

\begin{IEEEbiography}[{\includegraphics[width=1in,height=1.25in,clip,keepaspectratio]{sections/figures/Professor.jpeg}}]{Wael AbdAlmageed}
is a Tenured Full Professor at the Holcombe Department of Electrical and Computer Engineering at Clemson University. From 2013 to 2023, he was a Research Associate Professor at Department of Electrical and Computer Engineering, and a Research Director and Distinguished Principal Scientist at Information Sciences Institute. He is the Founding Director of the USC’s Visual Intelligence and Multimedia Analytics Laboratory (VIMAL). He received his B.S. in electrical engineering in 1994 and his M.S. in computer engineering in 1997 from Mansoura University in Egypt. 
He obtained his Ph.D. with Distinction from the University of New Mexico in 2003 where he was also awarded the Outstanding Graduate Student award. His research interests include representation learning, debiasing and fair representations, multimedia forensics and visual misinformation identification (such as deepfake and image manipulation detection), and face recognition and biometric anti-spoofing. He leads several multi-institution research efforts, including DARPA’s MediFor, GARD and LwLL and IARPA’s Janus, Odin and BRIAR. 

\end{IEEEbiography}
