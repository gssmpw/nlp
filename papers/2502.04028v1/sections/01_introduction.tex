
Developing cooperative policies in MARL is challenging due to the exponential growth of the joint action-observation space with an increasing number of agents, making fully centralized optimal policy search infeasible \cite{oroojlooy2023review}. Conversely, fully decentralized agents can scale well but suffer from non-stationarity due to the presence of multiple intelligent agents in the environment \cite{tan1993multi}. In many multi-agent domains, the outcome of an agent’s action often depends on a subset of other agents \cite{guestrin2003efficient,guestrin2002coordinated}. The approach of coordination graphs (CGs) \cite{guestrin2002coordinated,bohmer2020deep} encodes such localized interactions and improves efficiency in MARL. 

Despite their utility, prior research has represented only pairwise or direct dependencies between agents. However, direct relationships may not be enough in real-world scenarios, as they can oversimplify interactions, potentially leading to suboptimal coordination. \edits{MARL agents often also require capturing two other distinct types of complex interactions for effective cooperation:
\begin{figure}[!t]
    \centering
    \captionsetup[subfigure]{justification=centering}
    \subfloat[Coordination graphs]{\includegraphics[width=0.48\linewidth]{figures/diagrams/cg.png}} \hfill 
    \subfloat[Meta coordination graphs]{\includegraphics[width=0.48\linewidth]{figures/diagrams/meta_cg.png}} 
    \caption{\edits{Introducing meta coordination graphs. (a) Coordination graphs model static pairwise interactions among agents. (b) We propose meta coordination graphs that introduce dynamic edge types ($e_k^t$) where \( k \) denotes the type of interaction (color-coded) and evolves over time (\( t \)), enabling modeling of both higher-order and indirect interactions (dotted edges). This novel approach captures complex dependencies and cascading effects, offering a more adaptive and nuanced representation for multi-agent interactions.}} 
    \label{fig:metacg}
\end{figure}
\begin{itemize}
    \item \textbf{Higher-order interactions} which represent different types of dependencies that can emerge among agents due to varying goals, roles, or overlapping capabilities. For instance, these could include direct physical interactions (e.g., proximity-based effects), implicit communication links (e.g., signaling intentions or strategies through action choices), or influence-based dependencies (e.g., strategic or cooperative behavior). These can be captured by modeling agents connected through distinct types of relationships. 
    
    \item \textbf{Indirect interactions} which occur when the effect of an agent’s action propagates through intermediary agents to influence others. For example, an agent's decision may impact its neighbors, who in turn adapt their actions, indirectly affecting other agents further away in the graph. Modeling such cascading effects is essential for capturing emergent and multi-step dependencies.
\end{itemize}}

\edits{In this paper, we propose the novel concept of \textbf{Meta Coordination Graphs (MCGs)} to address these limitations. MCGs extend traditional CGs by introducing dynamic edge types ($e_k^t$), where $k$ denotes the type of interaction and $t$ represents time, enabling the graph structure to evolve dynamically. This allows MCGs to simultaneously capture higher-order relationships and indirect interactions. For instance, Figure~\ref{fig:metacg} illustrates how agents can exhibit various interaction types, while indirect interactions (dotted lines) capture cascading effects. We then propose a novel approach, \algosmall\ (\algoabb), designed to leverage MCGs as a building block for learning cooperative policies in MARL. By dynamically identifying relevant interaction patterns, \algoabb\ encodes the nuanced and emergent relationships essential for effective coordination in multi-agent systems.} The main contributions of this paper are:


\begin{itemize}[noitemsep,nolistsep]
    \item We introduce the concept of \textit{meta coordination graphs}, enabling the \edits{modeling of complex multi-agent interactions}, including higher-order and indirect relationships, for capturing intricate dependencies and multi-hop connections among agents. 
    \item We propose a novel MARL algorithm, \algoabb, which \edits{dynamically identifies and processes interaction patterns and MCGs} for improved coordination in complex environments. 
    \item We demonstrate \algoabb's superior performance on \edits{tasks that highlight miscoordination} challenges, such as \textit{relative overgeneralization}, and evaluate it on the StarCraft Multi-Agent Challenge (SMACv2) as well.
\end{itemize} 


