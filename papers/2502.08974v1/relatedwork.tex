\section{Related Work}
\textbf{Online Map Learning}
Recent advancements in online map learning focus on detecting map elements using onboard sensors to construct local high-definition (HD) maps. Traditionally, this task has been approached as a pixel-level semantic segmentation problem \cite{liu2023petrv2,liu2023vision}. To mitigate the time-consuming post-processing and shape ambiguity issues \cite{li2022hdmapnet}, recent approaches have shifted towards learning vectorized representations of map elements. For instance, VectorMapNet \cite{liu2023vectormapnet} introduces a vectorized HD map learning framework that predicts a sparse set of polylines from a bird's-eye view. The MapTR series \cite{liao2022maptr,liao2023maptrv2} leverage hierarchical query embedding to more effectively learn geometrical shapes at both point and shape levels. To further enhance positional embedding in queries, MapQR \cite{liu2024leveraging} introduces a scatter-and-gather query mechanism that encodes shared content across different positions. However, current online map learning methods still fall short in providing detailed lane information, such as lane direction and topological structure. While these methods are effective at detecting map elements, they are not equipped to deliver the detailed trajectories needed for planning in end-to-end autonomous driving.



\textbf{Topology Reasoning}
Topology Reasoning primarily focuses on centerline perception and connectivity relations. STSU \cite{can2021structured} is the first end-to-end framework to detect centerlines and objects using learnable queries from a BEV perspective. The centerline queries are processed by detection, control, and association heads to generate a lane graph, which is widely followed in most subsequent works. TopoNet \cite{li2023graph} represents lane connectivity as a lane graph and designs a scene graph neural network to refine the position and shape of lanes. LaneSegNet \cite{li2023lanesegnet} introduces lane attention and identical initialization to enhance long-range perception. RoadPainter \cite{ma2024roadpainter} generates centerline masks to better refine points with large curvature. To fully leverage higher recall 2D results, Topo2D \cite{li2024enhancing} updates 3D lane queries using 2D lane priors. TopoMLP \cite{wu2023topomlp} enhances topology results by incorporating lane point coordinates as positional embeddings. Most of these methods are built on a DETR-like framework \cite{zhu2020deformable}. However, the unordered nature and weak long-range perception of the DETR-like framework limit topology reasoning. In this paper, we address these challenges by introducing sequence-to-sequence learning.



\textbf{Visual Sequence-to-Sequence Learning}
Visual sequence-to-sequence learning relies on pixel-based observations, transforming downstream task objectives into a language-like format, i.e., sequences. Pix2Seq \cite{chen2021pix2seq} quantizes and serializes detected objects into sequences of discrete tokens, and by dequantizing the output sequences from the auto-regressive transformer, the detected bounding boxes are obtained. Pix2Seqv2 \cite{chen2022unified} introduces task-specific prompts at the beginning of sequences, enabling training for multiple vision tasks, such as detection, segmentation, keypoint detection, and captioning. However, the aforementioned tasks do not emphasize the order and relationships between instances, which are crucial for topology reasoning. To enable lane graph extraction in a sequence-to-sequence manner, RoadNet \cite{lu2023translating} integrates landmarks, curves, and topology into a unified sequence representation. LaneGraph2Seq \cite{peng2024lanegraph2seq} transforms lane graphs into a combination of vertex and edge sequences. Nevertheless, sequence-to-sequence learning, which relies on auto-regressive transformers, tends to be slow in inference. To fully leverage the strengths of sequence-to-sequence learning for long-range modeling and relationship extraction, we incorporate it into the training phase to enhance feature extraction, rather than using it directly for lane graph inference.