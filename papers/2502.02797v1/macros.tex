% For theorems and such
\usepackage{amsmath}
\usepackage{amssymb}
\usepackage{mathtools}
\usepackage{amsthm}
\usepackage{bm}
\usepackage{csquotes}
\usepackage{ragged2e}
% \usepackage{booktabs}
% \usepackage{siunitx}
% \usepackage{caption}

% if you use cleveref..
\usepackage[capitalize,noabbrev]{cleveref}

%%%%%%%%%%%%%%%%%%%%%%%%%%%%%%%%
% THEOREMS
%%%%%%%%%%%%%%%%%%%%%%%%%%%%%%%%
\theoremstyle{plain}
\newtheorem{theorem}{Theorem}[section]
\newtheorem{proposition}[theorem]{Proposition}
\newtheorem{lemma}[theorem]{Lemma}
\newtheorem{corollary}[theorem]{Corollary}
\newtheorem{remark}[theorem]{Remark}
\theoremstyle{definition}
\newtheorem{definition}[theorem]{Definition}
\newtheorem{assumption}[theorem]{Assumption}
%\theoremstyle{remark}
%\newtheorem{remark}[theorem]{Remark}

% Todonotes is useful during development; simply uncomment the next line
%    and comment out the line below the next line to turn off comments
%\usepackage[disable,textsize=tiny]{todonotes}
\usepackage[textsize=tiny]{todonotes}


%%%%%%%%%%%%%%%%%%%%%%%%%%%%%%%%
% COMMENTS AND DRAFTING
%%%%%%%%%%%%%%%%%%%%%%%%%%%%%%%%
\newcommand{\draft}[1]{\textcolor{red}{#1}}
\newcommand{\sujay}[1]{\textcolor{red}{Sujay: #1}}
\newcommand{\sunny}[2]{\textcolor{blue}{Sunny: #1}}
\newcommand{\hayden}[1]{\textcolor{brown}{Hayden: #1}}
\newcommand{\rd}[1]{\textcolor{violet}{Rudrajit: #1}}
\newcommand{\ali}[1]{\textcolor{cyan}{#1}}

%%%%%%%%%%%%%%%%%%%%%%%%%%%%%%%%
% PAPER-SPECIFIC PACKAGES
%%%%%%%%%%%%%%%%%%%%%%%%%%%%%%%%
\usepackage{multirow}       % For multirow spanning cells (MAYBE REMOVE)
\usepackage{xspace}         % spacing adjustments (for instance, when \method is used inline) and flexibility
\usepackage[inline,shortlabels]{enumitem}       % for inline itemization and margin control
% \setenumerate{itemsep=0pt,topsep=1pt,left=0pt}
% \setitemize{itemsep=0pt,topsep=1pt,left=0pt}

%%%%%%%%%%%%%%%%%%%%%%%%%%%%%%%%
% MATH OPERATORS AND EXPRESSIONS
%%%%%%%%%%%%%%%%%%%%%%%%%%%%%%%%
\DeclareMathOperator*{\argmin}{arg\,min}

\newcommand{\bW}{\mathbf{W}}
\newcommand{\bQ}{\mathbf{Q}}
\newcommand{\bP}{\mathbf{P}}
\newcommand{\bx}{\mathbf{x}}
\newcommand{\be}{\mathbf{e}}

\newcommand{\finetuning}{\text{forgetting-robust fine-tuning}\xspace} % knowledge-preserving fine-tuning
\newcommand{\method}{\textbf{$\mathsf{FLOW}$}\xspace}
\newcommand{\methodbold}{{$\boldsymbol{\mathsf{FLOW}}$}\xspace}
\newcommand{\methoditalic}{\textit{\textsf{FLOW}}\xspace}
\newcommand{\methodbolditalic}{{$\textit{\boldsymbol{\mathsf{FLOW}}}$}\xspace}
