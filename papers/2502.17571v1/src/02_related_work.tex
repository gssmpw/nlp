\section{Related Work}
In recent years, there has been increasing research into LLM-based clinical text generation, highlighting its potential in generating discharge summaries \citep{ando2022artificial, ellershaw2024automated, clough2024transforming, dubinski2024leveraging}, brief hospital courses~\citep{hartman2022day, hartman2023method, searle2023discharge}, and radiology reports \citep{alfarghaly2021automated, wang2023r2gengpt, yang2023radiology}. One study even shows that physicians often prefer AI-generated clinical texts over manually written ones~\citep{van2024adapted}. However, most of these approaches have treated clinical text generation as an end-to-end generation task, without offering user intervention and control. A recent example is the BioNLP ACL'24 Shared Task 'Discharge Me!'~\citep{xu-etal-2024-overview}, focusing on generating discharge summaries. However, the complexity of clinical texts, which often require external sources of information and are subject to individual guidelines and writing styles, makes end-to-end generation less feasible in practice. These challenges point to the need for more flexible generation allowing users to control specific aspects of the output, such as content and style.

Controlled text generation (CTG) is a growing area of research aimed at providing users with more influence over the generated content. This involves integrating specific control conditions, such as enforcing a professional tone or ensuring the use of domain-specific terminology, while maintaining fluency, coherence, and relevance in the generated text~\citep{zhang2023survey}. Prior work in this area has explored different control mechanisms that can be directly applied to clinical text generation, such as structure control~\citep{yang-klein-2021-fudge, zou2021controllable}, general style control~\citep{keskar2019ctrl}, and personal style control~\citep{tao2024cat}.

Moreover, recent studies have explored using Question-Answer (QA) pairs as a blueprint to guide the text generation process. This approach has been shown to reduce hallucinations and improve the factual consistency of generated content~\citep{narayan-etal-2023-conditional, huot-etal-2023-text}. It is based on the Question Under Discussion (QUD) theory~\citep{roberts:2012:information}, which states that all utterances in a discourse~\citep{van1995discourse} serve to answer either implicit or explicit questions. Building on these insights, we adapt the QUD framework for clinical document generation by framing clinical documents as responses to implicit questions arising from their intended purpose. These questions are typically addressed in a structured manner, even when the document appears unstructured. Using fine-grained topic segmentation, we aim to uncover this underlying structure by generating headings and QUDs, with corresponding text segments acting as their answers. This approach aligns topics with specific writing subtasks, simplifying the generation process while preserving the structured nature of clinical documentation.