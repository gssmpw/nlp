\section{Introduction}

% \deni{Change any reference to allocation with consumption. Make sure the distinction between good and commodity is correct throughout the paper}

% \deni{Note: Reason for system of equality formulation of eqm to be bad is because even for simple functions the system is never invertible, e.g. homogenous functions,}

% \deni{Has a very long anf confusing markets history\url{https://dornsife.usc.edu/assets/sites/1302/docs/Elgar_intro_Vol_2.pdf}}

% \deni{Point to fix here here: eqm is feasibility + walras' law, not market clearance and Walras' law}

\if 0
\sdeni{A \mydef{market} is a system for allocating resources
%\deni{Means is not precise, it suggest some weak anthopologic notion. We actually have a definition of what a market is in general equilibrium. It is a tuple consisting of a set of commodities, a demand function, and a supply function, s.t. the demand and supply functions depend on the prices of the set of commodities, fixing all other inputs of these functions. If you do not like the word system, I am happy to also use the terminology ``a model of resource allocation''.} 
governed 
%\sdeni{by prices, which in turn are based on supply and demand}
{by a supply and demand for resources which are determined by prices}
%\deni{the prices determine the supply and demand not vice versa! It is the whole point of general equilibrium!}. 
A \mydef{competitive} (or \mydef{Walrasian}) \mydef{equilibrium}, first described by Walras \cite{arrow-debreu, walras}, is a steady state of a market, which is \mydef{feasible}, i.e., the demand for each resource is less than or equal to its supply, and for which \mydef{Walras' law} holds, i.e., the value of the supply is equal to the value of the demand.}
\sdeni{In general, the supply and demand---and hence, prices---of resources in one market can depend on the supply and demand---and hence, prices---of resources in other markets.
Together such markets constitute an \mydef{economy};
%---a system of resource allocations governed by a supply and demand which are \emph{wholly} dependent on prices---and, 
\samy{as such, a market can be at a competitive equilibrium independent of the state of other markets.}{} \amy{perhaps what you mean here is that prices are sufficient to capture the state of a market, so these markets are not really independent of one another, but they are conditionally independent, given prices?} 
When every market in an economy is simultaneously at a competitive equilibrium, Walras' law holds for the economy as a whole, in which case the steady state is called a \mydef{general equilibrium}.}
\fi

In \citeyear{walras}, L\'eon Walras formulated a mathematical model of markets as a resource allocation system comprising supply and demand functions that map values for resources, called \mydef{prices}, to quantities of resources---\emph{ceteris paribus}, i.e., all else being equal.
Walras argued that any market would eventually settle into a steady state, which he called \mydef{competitive} (nowadays, also called \mydef{Walrasian}) \mydef{equilibrium}, as a collection of prices and associated supply and demand s.t.\@ 
%\amy{we are missing the fact that all consumers are optimizing!} \deni{Supply and demand directly implies utility/profit maximization. Either way, for Walras' theory this does not matter. Walras never actually thought optimization, he thought a system of equations, where demand was determined in closed form.} 
the demand is \mydef{feasible}, i.e., the demand for each resource is less than or equal to its supply, and \mydef{Walras' law} holds, i.e., the value of the supply is equal to the value of the demand.
Unlike in Walras' model, real-world markets do not exist in isolation but are part of an \mydef{economy}.
%(i.e., a collection of markets and a collection of mappings from quantities of supply and demand in each market to another \amy{cannot parse?}), so all else is \emph{not\/} equal.
Indeed, the supply and demand of resources in one market depend not only on prices in that market, but also on the supply and demand of resources in other markets.
%As such, fixing the supply, demand, and prices in other markets, one market can be at a competitive equilibrium, without the other markets in the economy necessarily being at a competitive equilibrium.
If every market in an economy is simultaneously at a competitive equilibrium, Walras' law holds for the economy as a whole; this steady state, now a property of the economy, is called a \mydef{general equilibrium}.

Beyond Walras' early forays into competitive equilibrium analysis, foremost to the development of the theory of general equilibrium was the introduction of a broad mathematical framework for modeling economies, which is known today as the \mydef{Arrow-Debreu competitive economy}
\cite{arrow-debreu}.
In this same paper, \citeauthor{arrow-debreu} developed their seminal game-theoretic model, namely (quasi)concave pseudo-games, and proved the existence of generalized Nash equilibrium in this model.
Since this game-theoretic model is sufficiently rich to capture Arrow-Debreu economies, they obtained as a corollary the existence of general equilibrium in these economies.



\if 0
\deni{Prices determine competitive equilibrium because they determine supply and demand, not vice-versa.} \amy{prices determine consumptions and productions, but it is an eqm, so consumptions and productions likewise determine prices---not in an optimal sense, but in a clearing sense} \deni{My statement is not about the equilibrium, it is about the markets. Also, even then, what you are saying is only maybe true in complete markets where a second welfare theorem holds so that you could say that any Pareto-optimal allocation determine prices (by the second welfare theorem). Either way the point is to bring the arrow-debreu model to Walras' framework. Walras says that an economy is a supply and demand function which depends on prices. Well, arrow-debreu build a model where supply and demand is indeed determined by prices. Also, while given prices there exists a well-defined demand, given any allocation, there does not necessarily exist some notion of prices for that allocation, except for equilibrium allocation (i.e. PO ones).} \amy{true! ok!!!}
\fi
% 
In their model, \citeauthor{arrow-debreu} posit a set of resources, modeled as commodities, each of which is assigned a price; a set of consumers, each choosing a quantity of each commodity to consume in exchange for their endowment; and a set of firms, each choosing a quantity of each commodity to produce, with prices determining \mydef{(aggregate) demand}, i.e., the sum of the consumptions across all consumers, and \mydef{(aggregate) supply}, i.e., the sum of endowments and productions across all consumers and firms, respectively.
This model is \mydef{static}, as it comprises only a single period model, but it is nonetheless rich, as commodities can be state and time contingent, with each one representing a good or service which can be bought or sold in a single time period, but that encodes delivery opportunities at a finite number of distinct 
%locations and 
points in time. Following \citeauthor{arrow-debreu}'s seminal existence result, the literature would slowly turn away from static economies, which do not \emph{explicitly\/} involve time and uncertainty, such as Arrow-Debreu competitive economies. 

\if 0
Following \citeauthor{arrow-debreu}'s seminal existence result, \sdeni{the literature turned its attention to questions of 
1.~(economic) \mydef{efficiency}, i.e., under what assumptions are competitive equilibria Pareto-optimal? \cite{arrow-welfare, arrow1951extension, arrow1958note, arrow-hurwicz, balasko1975some, debreu1951pareto};
2.~\mydef{uniqueness}, i.e., under what assumptions are competitive equilibria unique? \cite{dierker1982unique, pearce193unique};
3.~\mydef{stability},
%\amy{convergence?}\deni{Convergence is different than stability: Stability refers to whether if there are learning dynamics while convergence is just a property of the algorithm}
i.e., under what conditions can a competitive economy settle into a competitive equilibrium? \cite{hahn1958gross,balasko1975some, arrow-hurwicz, cole2008fast, cheung2018dynamics, fisher-tatonnement, goktas2023tatonn}, and
4.~\mydef{efficient computation}, i.e., under what conditions can a competitive equilibrium be computed efficiently? \cite{jain2005market, codenotti2005market, codenotti2006leontief, chen2009spending}.
While significant progress has been made in answering the first two questions in a wide variety of economic settings,
%\samy{}{(as per the aforementioned references)} \deni{I think the reader is smart enough to see that} \amy{of course! this is a comment on the old version, where there was a lot more going on b/n the mention of the earlier refs and this discussion!},
most progress on the latter two has been limited to
}{the literature would slowly turn away from} static economies, which do not \emph{explicitly\/} involve time and uncertainty, such as Arrow-Debreu competitive economies. 
\fi

\citeauthor{arrow-debreu}'s model fails to provide a comprehensive account of the economic activity observed in the real world, especially that which is designed to account for \emph{time\/} and \emph{uncertainty}. 
Chief among these activities are 
%\mydef{(financial)} \amy{can't you also use commodities to insure yourself against future bad states?} 
\mydef{asset markets}, 
%i.e., insurance markets, 
which allow consumers and firms to insure themselves against uncertainty about future states of the world. 
Indeed, while static economies with
%\citeauthor{arrow-debreu}'s 
state- and time-contingent commodities can \emph{implicitly\/} incorporate time and uncertainty, the assumption that a complete set of state- and time-contingent commodities are available at the time of trade is highly unrealistic. 
\citeauthor{arrow1964role} (\citeyear{arrow1964role}) thus proposed to 
%incorporate time and uncertainty explicitly
enhance the Arrow-Debreu competitive economy %\sdeni{which allowed consumers and firms to invest in a financial asset called a \mydef{\textit{num\'eraire} Arrow security},
with \mydef{assets} (or \mydef{securities} or \mydef{stocks}),%
\footnote{Some authors (e.g., \citet{geanakoplos1990introduction}) distinguish between assets, stocks, and securities, instead defining securities (resp. stocks) as those assets which are defined exogenously (resp.\@ endogenously), e.g., government bonds (resp.\@ company stocks). 
As this distinction makes no mathematical difference to our results, and is only relevant to stylized models, we make no such distinction.} 
i.e., contracts between two consumers which 
%can be traded \samy{and}{at an earlier time in return for the} 
promise the delivery of commodities by its seller to its buyer at a future date. 
In particular, \citeauthor{arrow1964role} introduced an asset type nowadays known as the \mydef{\textit{num\'eraire} Arrow security},
%%\deni{Arrow did not consider generalized securities, his securities only pay a unit of the numeraire at one state, not at multiple states.}
which transfers one unit of a designated commodity used as a unit of account---\mydef{the num\'eraire}---when a particular state of the world is observed, and nothing otherwise.
As the num\'eraire is often interpreted as money, assets which deliver only some amount of the num\'eraire, are called \mydef{financial assets}.
%and as such, num\'eraire Arrow securities are often referred to as financial assets.

% \amy{i think for this paper, stochastic implies dynamic} %\deni{Try to add somewhere. Difference between a static model vs. a stochastic model is one budget constraint vs. S budget constraints.} 

Formally, Arrow considered a \mydef{two-step stochastic exchange economy}.
In the initial state, consumers can buy or sell \textit{num\'eraire} Arrow securities in a \mydef{financial asset market}.
Following these trades, the economy stochastically transitions to one of finitely many other states in which consumers receive returns on their initial investment and participate in a \mydef{spot market}, i.e., a market for the immediate delivery of commodities, modeled as a static exchange economy---which, for our purposes, is better called an \mydef{exchange market}.%
\footnote{An (Arrow-Debreu) exchange economy is simply an (Arrow-Debreu) competitive economy without firms. Historically, for simplicity, 
%in establishing results for general equilibrium models, 
it has become standard practice \emph{not\/} to model firms, as most, if not all, results extend directly to settings with firms. 
In line with this practice, we do not model firms, but note that our results and methods also extend directly to settings that include firms.}
A general equilibrium of this economy is then simply prices for financial assets \emph{and\/} %\sdeni{spot markets}
commodities, which lead to a feasible allocation of all resources (i.e., financial assets and spot market commodities) that satisfies Walras' law. 
%\amy{again, we don't require that all mkt participants exhibit optimizing behavior?}

\citet{arrow1964role} demonstrated that the general equilibrium consumptions of an exchange economy with state- and time-contingent commodities can be implemented by the general equilibrium spot market consumptions of a two-step stochastic economy with a considerably smaller, yet \mydef{complete} set of \textit{num\'eraire} Arrow securities, i.e., a set of securities available for purchase in the first period that allow consumers to transfer wealth to \emph{all\/} possible states of the world that can be realized in the second period.
In conjunction with the welfare theorems \cite{debreu1951pareto, arrow1951extension}, this result implies that economies with \mydef{complete financial asset markets}, i.e., economies with such a complete set of securities, 
achieve a Pareto-optimal allocation of commodities by ensuring optimal risk-bearing via financial asset markets; and
conversely, any Pareto-optimal allocation of commodities in economies with time and uncertainty can be realized as a competitive equilibrium of a complete financial asset market.


Arrow's contributions led to the development of a new class of general equilibrium models, namely \mydef{stochastic economies} (or \mydef{dynamic stochastic general equilibrium---DSGE---models}) \cite{geanakoplos1990introduction}.%
\footnote{As these models incorporate both time and uncertainty, they are often referred to as dynamic stochastic general equilibrium models.
Nonetheless, we opt for the stochastic economy nomenclature, because, as we 
% \samy{painstakingly}{} \amy{haha!} \deni{Right!}
demonstrate in this paper, these economies can be seen as instances of (generalized) stochastic games.}
At a high-level, these models comprise a sequence of world states and spot markets,
%---markets in which commodities may be bought and sold on for a single time-period----
which are linked across time by
%\sdeni{(financial)}{} 
asset markets, with each next state of the world (resp.\@ spot market) determined by a stochastic process that is independent of market interactions (resp.\@ dependent only on their asset purchase) in the current state. 
Mathematically, the key difference between a static and a stochastic economy is that consumers in a stochastic economy face a collection of budget constraints, one per time-step, rather than only one.
Indeed, \citet{arrow1964role}'s proof that general equilibrium consumptions in stochastic complete economies are equivalent to general equilibrium consumptions in static state- and time-contingent commodity economies relies on proving that the many budget constraints in a complete stochastic economy can be reduced to a single one.

%In the years to come, pioneering work would develop the work started by Arrow further, amongst other things extending Arrow's model
Stochastic economies were introduced to model arbitrary finite time horizons \cite{radner1968competitive} and a variety of risky asset classes (e.g., stocks \cite{diamond1967incompletege}, risky assets \cite{lintner1975valuation}, derivatives \cite{black1973pricing}, capital assets \cite{mossin1966equilibrium}, debts \cite{modigliani1958cost} etc.), eventually leading to the emergence of \mydef{stochastic economies with incomplete asset markets} \cite{magill1991incomplete, magill2002theory, geanakoplos1990introduction}, or colloquially, \mydef{(incomplete) stochastic economies}.%
\footnote{While many authors have called these models incomplete economies \cite{geanakoplos1990introduction, magill2002theory, magill1991incomplete}, these models capture both incomplete and complete asset markets.
In contrast, we refer to stochastic economies with incomplete or complete asset markets as \mydef{stochastic economies}, adding the (in)complete epithet only when necessary to indicate that the asset market is (in)complete.}
%\amy{incomplete seems like a fine name to me. incomplete is more general. complete could be a special case with all the requisite assets.} \deni{We can switch and use incomplete economy as our language but I think we should only use "stochastic economy" or "incomplete economy" and not "stochastic incomplete economy". Given that our model is about Markov games, I am leaning towards using stochastic economy and keeping the incomplete stochastic economy to specifically indicate that the asset is strictly complete.}
Unlike in Arrow's stochastic economy, the asset market is not complete in such economies, so consumers cannot necessarily insure themselves against all future world states. 

The archetypal stochastic economy is the \mydef{Radner stochastic exchange economy}, deriving its name from Radner's proof of existence of a general equilibrium in his model \cite{radner1972existence}.
Radner's economy is a finite-horizon stochastic economy comprising a sequence of spot markets, linked across time by asset markets.
At each time period, a finite set of consumers observe a world state and trade in an asset market and a spot market, modeled as an exchange market. 
Each \mydef{asset market} comprises \mydef{assets}, modelled as time-contingent \mydef{generalized Arrow securities}, which specify quantities of the commodities the seller is obliged to transfer to its buyer, should the relevant state of the economy be realized at some specified future time.%
\footnote{Here, Arrow securities are ``generalized'' in the sense that they can deliver different quantities of \emph{many\/} commodities at different states of the world, rather than only one unit of a commodity at only one state of the world. Although \citet{arrow1964role} considered only \textit{num\'eraire} securities, 
%\sdeni{which are called \mydef{financial assets}}{}, 
his theory was subsequently generalized to %\sdeni{securities that can deliver any set of commodities, called \mydef{(real) assets}}
generalized Arrow securities \cite{geanakoplos1990introduction}.} 
Consumers can buy and sell assets, thereby transferring their wealth across time, all the while insuring themselves against uncertainty about the future.
%\amy{used to also say: ``thereby influencing the future state of the economy''. but i find this a bit strange, since in the usual model of CE, there are so many agents that each has little to no influence on price; everyone is a price-taker. but here we are saying that agents are not ``state''-takers, since they can influence future states. but maybe this ``price-taking'' assumption is the same point we were discussing earlier about Walras' economy, namely that prices are inputs (only) the supply and demand functions. if so, then this is a non-issue. \deni{Consumers are event takers, they do not influence future states, rather they get purchase commodities that will make then ``happier'' at future states, which we interpret as insurance.}}
% \amy{a bit strange, since in the usual model of CE, there are so many agents that each has little to no influence on price; everyone is a price-taker. but here we are saying that agents are not ``state''-takers, since they can influence future states.}
The canonical solution concept for stochastic economies, \mydef{Radner equilibrium}
(also called \mydef{sequential competitive equilibrium}%
\footnote{This terminology does not contradict the economy being at a competitive equilibrium, but rather indicates that at all times, the spot and asset markets are at a competitive equilibrium, hence implying the overall economy is at a general equilibrium.} 
\cite{mas-colell}, \mydef{rational expectations equilibrium} \cite{radner1979rational}, and \mydef{general equilibrium with incomplete markets} \cite{geanakoplos1990introduction}), is a collection of history-dependent prices for commodities and assets, as well as history-dependent consumptions of commodities and portfolios of assets, such that, for all histories, the aggregate demand for commodities and the \mydef{aggregate demand for assets} (i.e., the total number of assets bought) are feasible and satisfy Walras' law.
%\sdeni{1.~the aggregate demand for commodities is less than the aggregate supply at the last \amy{missing word?}; 2.~the \mydef{aggregate demand for Arrow securities}, i.e., the total number of Arrow securities bought, is less than or equal to the \mydef{aggregate supply of Arrow securities}, i.e., total number of Arrow securities sold; and 3.~Walras' law holds, i.e., the total value of the aggregate demand for commodities and Arrow securities is equal to the total value of the aggregate supply for commodities and Arrow securities.}{}

In spite of substantial interest in stochastic economies among microeconomists throughout the 1970s, the literature eventually trailed off, 
%for half a decade 
perhaps due to a seeming difficulty in proving existence of a general equilibrium in simple economies with incomplete asset markets that allow assets to be sold short \cite{geanakoplos1990introduction}, or to the lack of a second welfare theorem \cite{dreze1974investment, hart1975optimality}.
%\amy{don't you kind of need a GE existence result to have a first welfare theorem. so it seems like we might be able to say even in simpler models (no short selling) where GE exist, there is no 1st welfare them} \deni{You do not need a GE existence result to prove a 1st welfare theorem, in fact AD prove the welfare theorems in 1951 while they proved existed in 1954. If anything the welfare theorem indicate why one should prove the existence of GE, because they are efficient!} .
Financial and macroeconomists stepped up, however, with financial economists seeking to further develop the theoretical aspects of stochastic economies (see, for instance, \citet{magill2002theory}), and macroeconomists seeking practical methods by which to solve stochastic economies in order to determine the impact of various policy choices (via simulation; see, for instance, \citet{sargent2000recursive}).

\mydef{Infinite horizon stochastic economies} are one of the new and interesting directions in this more recent work on stochastic economies.
%Following \citet{radner1972existence}, 
Infinite horizon models come with one significant difficulty that has no counterpart in a finite horizon model, namely the possibility for agents to run a \mydef{Ponzi scheme} via asset markets, in which they borrow but then indefinitely postpone repaying their debts by refinancing them continually, from one period to the next.
%, in which case, an optimal consumption and portfolio for a consumer might not exist, even at fixed prices. 
From this perspective infinite horizon models represent very interesting objects of study, not only theoretically; it has also been argued that they are a better modeling paradigm for macroeconomists who employ simulations \cite{magill1994infinite}, because they facilitate the modeling of complex phenomena, such as asset bubbles \cite{huang2000asset}, which can be impacted by economic policy decisions.

\citet{magill1994infinite} introduced an extension of Radner's model to an infinite horizon setting, albeit with financial assets, 
%\amy{you deleted the distinction b/n financial and real assets?} \deni{Ok I now remember, it's because according to Geanokopoulos assets = real assets + financial assets, i.e., real assets are those assets that are not financial... previously real assets meant just all the assets which was incorrect.} 
and presented suitable assumptions under which a sequential competitive equilibrium is guaranteed to exist in this model.
Progress on the computational aspects of stochastic economies has been slow, however, and mostly confined to finite horizon settings
%spearheaded by only to a small number of computationally-astute macroeconomists 
(see, \citet{sargent2000recursive} and Volume 2 of \citet{taylor1999handbook} for a standard survey and \citet{FernandezVillaverde2023CompMethodsMacro}  for a more recent entry-level survey of computational methods used by macroeconomists).
Indeed, demands for novel computational methods for solving macroeconomic models, and theoretical frameworks in which to understand their computational complexity, have been repeatedly shared by macroeconomists \cite{taylor1999handbook}. 
This gap in the literature points to a novel research opportunity; however, it is challenging for non-macroeconomists 
%and non-financial economists 
to approach these problems with their computational tools. 

\if 0
Our proof relies on showing equality between the set of recursive competitive equilibria of an infinite horizon Markov exchange economy and the set of (generalized) Nash equilibria of an associated (generalized) Markov game, which in turn brings the problem of computing a general equilibrium, into a better understood domain of computer science, namely that of equilibrium computation in Markov games, opening the door for the use of computational complexity theory for theoretical advances and deep multiagent reinforcement learning for computational advances. 
As infinite horizon Markov exchange economies have as a special case static Leontief exchange economies, equilibrium computation in this model is PPAD-hard in general. 
Nevertheless, we reformulate the problem of computing a (generalized) Nash equilibrium in generalized Markov games as a generative adversarial learning problem which we show can be solved for a local solution in polynomial-time via multiagent reinforcement learning (i.e., a stochastic gradient descent ascent algorithm run on a suitable objective whose value and gradients are approximated histories by histories of play in generalized Markov games), and then solve this problem for infinite horizon Markov exchange economies using neural function classes with a novel generative adversarial neural network called \mydef{generative adversarial policy networks (GAPs)}.
\fi
