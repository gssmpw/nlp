\section{Incomplete Markov Economies}
\label{sec:infinite}

Having developed a mathematical formalism for Markov pseudo-games, along with a proof of existence of \MPGNE{} as well as an algorithm that computes them, we now move on to our main agenda, namely modeling incomplete stochastic economies in this formalism.
We establish the first proof, to our knowledge, of the existence of recursive competitive equilibria in standard incomplete stochastic economies, and we provide a polynomial-time algorithm for approximating them.

% \deni{Consider switching to profile notation for allocations, endowments etc...}
% \deni{Discuss Negishi's model somewhere!}
% \deni{Maybe add "Markov" to model nomenclature}
% \deni{Asset price notation clashes with q function definition.}

%%% WANTED TO ADD THE INTUITION BELOW, BUT RAN OUT OF TIME AND SPACE !!!

%\deni{Example to think about when you read this: the world state space could for instance be $\worldstates \doteq \{0, 1\}^\numbuyers$ which is a binary vector denoting whether if an agent is sick or not. Then, this world-state will naturally affect the types and endowments of the consumers (i.e., the spot market ) since a sick consumer might prefer medical commodities more than say sports commodities and might not be able to work physical jobs as much as when it is healthy (i.e., its endowment of physical labor hours goes down). However, can model more general shocks. In particular, in general there might not be a one-to-one mapping between world states and spot markets because while in the same example, while the consumer's physical labor endowment goes down when it is sick, it might now pick-up a coding bootcamp to work from home and increase its endowment of technical labor, which the world state does not directly determine since not all sick consumers will pick a coding bootcamp. Obviously, in the real world, there is a one-to-one mapping between world states and economic states since the world state captures every single characteristic of the world determining every possible decision of the consumer (e.g., characteristics that determine when the consumer picks up a coding bootcamp), but as it might not be possible to model the entire world state mathematically, this model allows you to capture shocks that are not explicitly modelled (e.g., picking up a bootcamp) as stochasticity over the spot markets (i.e., consumer endowments in bootcamp example). All this to say, in our model world-state stochastically determines spot markets, allowing us to model things more easily, while still giving us the choice of modelling everything in the real world if we could have!}



%\deni{TODO: Footnote on why this is called a market and not an economy}
%\deni{TODO: change any mention of commodity to commodity!! \amy{do you mean good?}}


\subsection{Static Exchange Economies} 

A \mydef{static exchange economy} (or \mydef{market%
\footnote{Although a static exchange ``market'' is an economy, we prefer the term ``market'' for the static components of an infinite horizon Markov exchange economy, a dynamic exchange economy in which each time-period comprises one static market among many.}})
$(\numbuyers, \numcommods, \numtypes, \consumptions, \consendowspace, \typespace, \util, \consendow, \type)$, abbreviated by $(\consendow, \type)$ when clear from context, comprises a finite set of $\numbuyers \in \N_+$ \mydef{consumers} and  $\numcommods \in \N_+$ \mydef{commodities}. 
Each consumer $\buyer \in \buyers$ arrives at the market with an \mydef{endowment} of commodities represented as vector $\consendow[\buyer] = \left(\consendow[\buyer][1], \dots, \consendow[\buyer][\numcommods] \right) \in \consendowspace[\buyer]$, where $\consendowspace[\buyer] \subset \R^\numcommods$ is called the \mydef{endowment space}.%
\footnote{Commodities are assumed to include labor services. 
Further, for any consumer $\buyer$ and endowment $\consendow[\buyer] \in \consendowspace[\buyer]$, $\consendow[\buyer][\commod] \geq 0$ denotes the quantity of commodity $\commod$ in consumer $\buyer$'s possession, while $\consendow[\buyer][\commod] < 0$ denotes consumer $\buyer$'s debt, in terms of commodity $\commod$.}
Any consumer $\buyer$ can sell its endowment $\consendow[\buyer] \in \consendowspace[\buyer]$ at \mydef{prices} $\price \in \simplex[\numcommods]$, where $\price[\commod] \geq 0$ represents the value (resp.\@ cost) of selling (resp.\@ buying) a unit of commodity $\commod \in \commods$, to purchase a consumption $\consumption[\buyer] \in \consumptions[\buyer]$ of commodities in its \mydef{consumption space} $\consumptions[\buyer] \subseteq \R^{\numcommods}$.%
\footnote{We note that, for any labor service $\commod$, consumer $\buyer$'s consumption $\consumption[\buyer][\commod]$ is negative and restricted by its consumption space to be lower bounded by the negative of $\buyer$'s endowment, i.e., $\consumption[\buyer][\commod] \in [- \consendow[\buyer][\commod], 0]$. 
This modeling choice allows us to model a consumer's preferences over the labor services she can provide. 
More generally, the consumption space models the constraints imposed on consumption by the ``physical properties'' of the world. 
That is, it rules out impossible combinations of commodities, such as strictly positive quantities of a commodity that is not available in the region where a consumer resides, or a supply of labor that amounts to more than 24 labor hours in a given day.}
Every consumer is constrained to buy a consumption with a cost weakly less than the value of its endowment, i.e., consumer $\buyer$'s \mydef{budget set}---the set of consumptions $\buyer$ can afford with its endowment $\consendow[\buyer] \in \consendowspace[\buyer]$ at prices $\price \in \simplex[\numcommods]$---is determined by its \mydef{budget correspondence} $\budgetset[\buyer] (\consendow[\buyer], \price) \doteq \{\consumption[\buyer] \in \consumptions[\buyer] \mid  \consumption[\buyer] \cdot \price \leq \consendow[\buyer] \cdot \price \}$.

Each consumer's consumption preferences are determined by its type-dependent preference relation $\succeq_{\buyer, \type[\buyer]}$ on $\consumptions[\buyer]$, represented by a type-dependent \mydef{utility function} $\consumption[\buyer] \mapsto \util[\buyer] (\consumption[\buyer]; \type[\buyer])$, for \mydef{type} $\type[\buyer] \in \typespace[\buyer]$ that characterizes consumer $\buyer$'s preferences within the \mydef{type space} $\typespace[\buyer] \subset \R^{\numtypes}$ of possible preferences.%
\footnote{In the sequel, we will be assuming, for any consumer $\buyer$ with any type $\type[\buyer] \in \typespace[\buyer]$, the type-dependent utility function $\consumption[\buyer] \mapsto \util[\buyer] (\consumption[\buyer]; \type[\buyer])$ is continuous, which implies that it can represent any type-dependent preference relation $\succeq_{\buyer, \type[\buyer]}$ on $\R^\numcommods$ that is complete, transitive, and continuous \cite{debreu1954representation}.}
The goal of each consumer $\buyer$ is thus to buy a consumption $\consumption[\buyer] \in \budgetset[\buyer] (\consendow[\buyer], \price)$ that maximizes its utility function $\consumption[\buyer] \mapsto \util[\buyer] (\consumption[\buyer]; \type[\buyer])$ over its budget set $\budgetset[\buyer] (\consendow[\buyer], \price)$.

% The goal of each consumer $\buyer$ is to buy a consumption $\consumption[\buyer] \in \consumptions[\buyer]$ that maximizes its \mydef{utility function}% $\consumption[\buyer] \mapsto \util[\buyer] (\consumption[\buyer]; \type[\buyer])$ parameterized by a \mydef{type} $\type[\buyer] \in \typespace[\buyer]$ from a \mydef{type space} $\typespace[\buyer]$ over a \mydef{budget set} 

%\amy{define generically in prelims, and delete.}
%We denote the space of consumptions, endowments, and types for all consumers, respectively, by $\consumptions = \bigtimes_{\buyer \in \buyers} \consumptions[\buyer]$, $\consendowspace \doteq \bigtimes_{\buyer \in \buyers} \consendowspace[\buyer]$, and $\typespace \doteq \bigtimes_{\buyer \in \buyers} \typespace[\buyer]$.

We denote any \mydef{endowment profile} (resp.\@ \mydef{type profile} and \mydef{consumption profile}) as $\consendow \doteq \left(\consendow[1], \hdots, \consendow[\numbuyers] \right)^T \in \consendowspace$ (resp.\@ $\type \doteq (\type[1], \hdots, \type[\numbuyers])^T \in \typespace$ and $\consumption[][][][] \doteq (\consumption[1][][][], \hdots, \consumption[\numbuyers][][][])^T \in \consumptions$).
% \deni{Maybe remove}
% Any matrix $\consumption[][][][]$ whose rows consists of consumer consumptions, i.e., $\consumption[][][][] \doteq (\consumption[1][][][], \hdots, \consumption[\numbuyers][][][])^T \in \consumptions$ is called a \mydef{consumption profile}. 
The \mydef{aggregate demand} (resp.\@ \mydef{aggregate supply}) of a \mydef{consumption profile} $\consumption \in \consumptions$ (resp.\@ \mydef{an endowment profile} $\consendow \in \consendowspace$) is defined as the sum of consumptions (resp.\@ endowments) across all consumers, i.e., $\sum_{\buyer \in \buyers} \consumption[\buyer]$ (resp.\@ $\sum_{\buyer \in \buyers} \consendow[\buyer])$.

\if 0
\begin{definition}[Arrow-Debreu Equilibrium]
    An \mydef{Arrow-Debreu (or Walrasian or competitive) equilibrium (ADE)} of an exchange economy $(\consendow, \type)$ is a tuple $(\consumption[][][][*], \price[][][*]) \in \consumptions \times \simplex[\numcommods]$, which consists respectively of a consumption profile and prices $\price \in \simplex[\numcommods]$ s.t.: 
    1.~each consumer $\buyer$'s equilibrium consumption maximizes its utility over its budget set:
    %\begin{align}
        $\consumption[\buyer][][][*] \in \argmax_{\consumption[\buyer] \in \budgetset[\buyer] (\consendow[\buyer], \price[][][*])} \util[\buyer] (\consumption[\buyer]; \type[\buyer])$;
    %\end{align}
    2.~the consumption profile is \mydef{feasible}, meaning aggregate demand is less than or equal to aggregate supply:,
    %\begin{align}
        $\sum_{\buyer = 1}^{\numagents} \consumption[\buyer][][][*] - \sum_{\buyer = 1}^{\numagents} \consendow[\buyer] \leq \zeros[\numcommods]$;
    %\end{align} 
    3.~\mydef{Walras' law} holds, so that the cost of the aggregate demand is equal to the value of the aggregate supply:
    %\begin{align}
        $\price[][][*] \cdot \left( \sum_{\buyer = 1}^{\numagents} \consumption[\buyer][][][*]  - \sum_{\buyer = 1}^{\numagents} \consendow[\buyer] \right) = 0$. %\enspace .
    %\end{align}
\end{definition}
\fi

\subsection{Infinite Horizon Markov Exchange Economies}
\label{sec:dsge}

An \mydef{infinite horizon Markov exchange economy} $\economy \doteq (\numbuyers, \numcommods, \numassets, \numtypes, \states, \consumptions, \portfoliospace, \consendowspace, \typespace,  \util, \discount, \trans, \returnset, \initstates)$, comprises $\numbuyers \in \N$ consumers who, over an infinite discrete time horizon $\numhorizon = 0, 1, 2, \hdots$, repeatedly encounter the opportunity to buy a consumption of $\numcommods \in \N$ commodities and a portfolio of $\numassets \in \N$ assets, with their collective decisions leading them through a \mydef{state space} $\states \doteq \worldstates \times (\consendowspace \times \typespace)$.
This state space comprises a \mydef{world state space} $\worldstates$ and 
% consisting \mydef{consumer-relevant world states} $\worldstates[\buyer]$ for all consumers $\buyer \in \buyers$ 
a \mydef{spot market space} $\consendowspace \times \typespace$. 
The spot market space is a collection of \mydef{spot markets}, each one a static exchange market $(\consendow, \type) \in \consendowspace \times \typespace \subseteq \R^\numcommods \times \R^\numtypes$. 
% is $(\numbuyers, \numcommods, \numtypes, \consumptions, \consendowspace, \typespace, \util, \consendow, \type)$

%\sdeni{For each consumer $\buyer \in \buyers$, we define the \mydef{consumer state space} $\states[\buyer] \doteq \worldstates \times (\consendowspace[\buyer] \times \typespace[\buyer]) = \states \setminus \bigtimes_{\buyer \neq \buyer^\prime} (\consendowspace[{\buyer^\prime}] \times \typespace[{\buyer^\prime}]) $ as those elements of the state space relevant to consumer $\buyer$.}{}

% , and a \mydef{financial market space} $\portfoliospace \doteq \bigtimes_{\buyer \in \buyers} \portfoliospace[\buyer]$ consisting of \mydef{financial states} $\portfoliospace[\buyer] \subseteq \assetpricespace$ s.t. a \mydef{financial state} $\portfolio[\buyer] \in \portfoliospace[\buyer]$ denotes the portfolio of the $\numassets$ financial assets for consumer $\buyer \in \buyers$.\footnote{For any financial asset $\asset \in \assets$, $\portfolio[\buyer][\asset] \geq 0$ denotes the units of asset $\asset$ purchased by the consumer, while $\portfolio[\buyer][\asset] < 0$ denotes the units of $\asset$ short-sold.} 

% For simplicity, without loss of generality, we will assume that the world state space is not redundant; that is, the cardinality of the worlds state space is less than or equal to that of the spot markets, i.e., $|\worldstates| \leq |\consendowspace \times \typespace|$ since any world state which cannot be associated with a spot market can be removed or merged with another world state without affecting the economy.
% A \mydef{consumer state} $\state[][\buyer] \doteq (\worldstate[][\buyer], \type[\buyer], \consendow[\buyer]) \in \states[\buyer]$ comprises of a  \mydef{consumer-relevant world state} $\worldstate[][\buyer] \in \worldstates[\buyer]$ for each consumer $\buyer \in \buyers$ 
% \mydef{world state} $\worldstate \doteq (\worldstate[][1], \hdots, \worldstate[][\numbuyers]) \in \worldstates \doteq \bigtimes_{\buyer \in \buyers} \worldstates$ consisting of. 

% a \mydef{type} $\type[\buyer] \in \typespace[\buyer]$ that parameterizes the consumer's \mydef{utility function}
% $\util[\buyer]: \R^\numcommods \times \typespace[\buyer] \to \mathbb{R}$, s.t. for any type $\type[\buyer] \in \typespace[\buyer]$, $\consumption[\buyer] \mapsto \util[\buyer] (\consumption[\buyer] ; \type[\buyer])$ defines a preference relation over commodities, and an \mydef{endowment} $\consendow[\buyer] \in \consendowspace[\buyer] \subseteq \R^\numcommods_+$ of commodities.
% its \mydef{consumption space} $\consumptions[\buyer] \subseteq \R^\numcommods_+$

Each \mydef{asset} $\asset \in \assets$ is a \mydef{generalized Arrow security}, i.e., a divisible contract that transfers to its owner a quantity 
of the $\commod$th commodity
%each of the $\numcommods$ commodities 
at any world state $\worldstate \in \worldstates$ determined by a matrix of asset returns $\returns[\worldstate] \doteq \left(\returns[\worldstate][1], \hdots, \returns[\worldstate][\numassets] \right)^T \in \R^{\numassets \times \numcommods}$
%vector of \mydef{returns} $\returns[\worldstate][\asset] \in \R^\numcommods$ 
s.t.\@ $\returns[\worldstate][\asset][\commod] \in \R$ denotes the quantity of commodity $\commod$ transferred at world state $\worldstate$ for one unit of asset $\asset$.
The collection of asset returns across all world states is given by $\returnset \doteq \{\returns[\worldstate] \}_{\worldstate \in \worldstates}$. 
At any time step $\numhorizon = 0, 1, 2, \hdots$, a consumer $\buyer \in \buyers$ can invest in an \mydef{asset portfolio} $\portfolio[\buyer] \in \portfoliospace[\buyer]$ from a \mydef{space of asset portfolios (or investments)} $\portfoliospace[\buyer] \subset \R^\numassets$ that define the \mydef{asset market}, where $\portfolio[\buyer][\asset] \geq 0$ denotes the units of asset $\asset$ bought (long) by consumer $\buyer$, while $\portfolio[\buyer][\asset] < 0$ denotes units that are sold (short).  
Assets are assumed to be \mydef{short-lived} \cite{magill1994infinite}, meaning that any asset purchased at time $\numhorizon$ pays its dividends in the subsequent time period $\numhorizon + 1$, and then expires.%
\footnote{While for ease of exposition we assume that assets are short-lived, our results generalize to infinitely-lived generalized Arrow securities \cite{huang2004implementing} (i.e., securities that never expire, so yield returns and can be resold in every subsequent time period following their purchase) with appropriate modifications to the definitions of the budget constraints and Walras' law. 
In contrast, our results do \emph{not\/} immediately generalize to $k$-\mydef{period-living generalized Arrow securities} (i.e., securities that yield returns and can be resold in the $k$ subsequent time periods following their purchase, until their expiration), as such securities introduce non-stationarities into the economy. 
To accommodate such securities would require that we generalize our Markov game model and methods to accommodate policies that depend on histories of length $k$.}
%\amy{i think our results should still go thru. it's just that we will need to learn $k$-Markov perfect equilibria, so policies with a history of length $k$, where $k$ is the lifetime of these options.} \deni{Ok agreed!}
 % 
Assets allow consumers to insure themselves against future realizations of 
%uncertainty regarding
the spot market (i.e., types and endowments), by allowing it to transfer wealth across world states.

% \footnote{Note that consumers only have to insure themselves against spot markets only, and not future financial market states since financial market states are deterministic! You get what you buy as determined by the returns! Additionally, consumers do not have to insure themselves against world states because we do not (as is standard in the literature) model the preferences (i.e., utility functions) and constraints (effectively their endowments) of the consumer as dependent on the world states). However, even if you are against this modelling choice, you could include the world-state into the preference and constraints of the consumer easily by capturing such dependencies in the types and endowments of consumers (which are associated with some world state). Going back to our example with a world state describing if a consumer is sick or healthy, indeed this is what happens, when we said that the preferences (i.e., types) of the consumer for medical commodities and its constraints (i.e., endowment of labor) changes.}

The economy starts at time period $\numhorizon = 0$ in an \mydef{initial state} $\staterv[0] \sim \initstates$ determined by an initial state distribution $\initstates \in \simplex(\states)$.
% \footnote{An alternative interpretation is that a mass of consumers start their life distributed across states according to $\initstates$. \deni{Double check if this matches the recCE def'n}} 
%
% , and has to repeatedly decide a consumption of $\numcommods \in \N$ divisible commodities, and a portfolio of $\numassets \in \N_+$ divisible financial assets,
% If there exists an asset that persists through all world-states, e.g., a risk-free bond, then one can take that asset as the num\'eraire, and if not one asset at each state of the asset market can be taken to be the num\'eraire.
% with every decision leading it to travel through the set of states $\states$.
% 
At each time step $\numhorizon = 0, 1, 2, \hdots$, the state of the economy is $\state[\numhorizon] \doteq (\worldstate[\numhorizon], \consendow[][][\numhorizon], \type[][][\numhorizon]) \in \states$. 
Each consumer $\buyer \in \buyers$, observes the world state $\worldstate[\numhorizon] \in \worldstates$, and participates in a spot market $(\consendow[][][\numhorizon], \type[][][\numhorizon])$, where it purchases \mydef{a consumption} $\consumption[\buyer][][\numhorizon] \in \consumptions[\buyer]$ at \mydef{commodity prices} $\price[][\numhorizon] \in \simplex[\numgoods]$, and an \mydef{asset market} where it invests in an \mydef{asset portfolio} $\portfolio[\buyer][][\numhorizon] \in \portfoliospace[\buyer]$ at \mydef{assets prices} $\assetprice[][\numhorizon] \in \R^\numassets$.\longversion{%
\footnote{In general, asset prices can be negative. 
This modeling assumption is in line with the real world: e.g., it is common for energy futures to see negative prices because of costs associated with overproduction and limited storage capacity \cite{sheppard2020us}.}} 
Every consumer is constrained to buy a consumption $\consumption[\buyer][][\numhorizon] \in \consumptions[\buyer]$ and invest in an asset portfolio $\portfolio[\buyer][][\numhorizon] \in \portfoliospace[\buyer]$ with a total cost weakly less than the value of its current endowment $\consendow[\buyer][][\numhorizon] \in \consendowspace[\buyer]$.
Formally, the set of consumptions and investment portfolios that a consumer $\buyer$ can afford with its current endowment $\consendow[\buyer][][\numhorizon] \in \consendowspace[\buyer]$ at current commodity prices $\price[][\numhorizon] \in \simplex[\numcommods]$ and current asset prices $\assetprice[][\numhorizon] \in \R^\numassets$, i.e., its \mydef{budget set} $\budgetset[\buyer] (\consendow[\buyer][][\numhorizon], \price[][\numhorizon], \assetprice[][\numhorizon])$, is determined by its \mydef{budget correspondence} $\budgetset[\buyer] (\consendow[\buyer], \price, \assetprice) \doteq \{(\consumption[\buyer], \portfolio[\buyer]) \in \consumptions[\buyer] \times \portfoliospace[\buyer] \mid  \consumption[\buyer] \cdot \price + \portfolio[\buyer] \cdot \assetprice  \leq \consendow[\buyer] \cdot \price \}$. 
%\amy{shouldn't they also profit from their investment decisions? that is, can't investments also lead to an increased budget? isn't that the point?} \deni{This is captured in the transition. They get a return in the next time period. the income does not appear in the budget constraint because it is already included in the endowments!} \amy{i see. this happens at the end of the next paragraph. maybe we need a forward pointer, or forward remark that endowments at each state include returns on investments.}

% of the consumer is subject to its \mydef{budget constraint}, i.e., $\consumption[\buyer][][\numhorizon] \cdot \price[][\numhorizon] + \portfolio[\buyer][][\numhorizon] \cdot \assetprice[][\numhorizon]  \leq  \consendow[\buyer][][\numhorizon] \cdot \price[][\numhorizon] + \portfolio[\buyer][][\numhorizon-1] \cdot \returns[{\worldstate[\numhorizon]}] $. This budget constraint requires the sum of the consumer's spending on commodities $\consumption[\buyer][][\numhorizon] \cdot \price[][\numhorizon]$, and financial assets $ \portfolio[\buyer][][\numhorizon] \cdot \assetprice[][\numhorizon]$, to be no more than its current income given by the sum the value of its endowment $\consendow[\buyer][][\numhorizon] \cdot \price[][\numhorizon]$, and its return on financial assets bought in the previous time-period $\portfolio[\buyer][][\numhorizon-1] \cdot \returns[{\worldstate[\numhorizon]}] $. 

% observes time-contingent , and time-contingent \mydef{financial assets prices} $\assetprice[][\numhorizon] \in \R^\numassets$ and then decides its \mydef{consumption of commodities} $\consumption[\buyer][][\numhorizon] \in \consumptions[\buyer]$ from its \mydef{consumption space} $\consumptions[\buyer]$, its \mydef{investment of financial assets} $\portfolio[\buyer][][\numhorizon] \in \portfoliospace[\buyer]$ from a \mydef{portfolios space} $\portfoliospace[\buyer] \subset \R^\numassets$. 
% As such, at each time step $\numhorizon \in \N$, the consumption and investment decision of the consumer is subject to its \mydef{budget constraint}, i.e.,  $\consumption[\buyer][][\numhorizon] \cdot \price[][\numhorizon] + \portfolio[\buyer][][\numhorizon] \cdot \assetprice[][\numhorizon]  \leq  \consendowrv[\buyer][][\numhorizon] \cdot \price[][\numhorizon] + \portfolio[\buyer][][\numhorizon-1] \cdot \returns[{\worldstaterv[\numhorizon]}] $. This budget constraint requires the sum of the consumer's spending on commodities $\consumption[\buyer][][\numhorizon] \cdot \price[][\numhorizon]$, and financial assets $ \portfolio[\buyer][][\numhorizon] \cdot \assetprice[][\numhorizon]$, to be no more than its current income given by the sum the value of its endowment $\consendowrv[\buyer][][\numhorizon] \cdot \price[][\numhorizon]$, and its return on financial assets bought in the previous time-period $\portfolio[\buyer][][\numhorizon-1] \cdot \returns[{\worldstaterv[\numhorizon]}] $. 

After the consumers make their consumption and investment decisions, they each receive \mydef{reward} $\util[\buyer] (\consumption[\buyer][][t]; \type[\buyer][][t])$ as a function of their consumption and type, and then
% it receives a(n instantaneous) utility $\util[\buyer] (\consumption[\buyer][][\numhorizon]; \type[\buyer])$ for its consumption of commodities,
the economy either collapses with probability $1 - \discount$, or survives with probability $\discount$, where $\discount \in (0, 1)$ is called the \mydef{discount rate}.
\footnote{While for ease of exposition we assume a single discount factor for all consumers, our results extend to a setting in which each consumer $\buyer \in \buyers$ has a potentially unique discount factor $\discount_\buyer \in (0, 1)$ by incorporating the discount rates into the consumers' payoffs in the Markov pseudo-game defined in \Cref{sec_app:stochastc_exchange_economy}, rather than the history distribution.} 
If the economy survives to see another day, then a new state is realized, namely 
%\ssadie{$(\worldstaterv[][][\prime], \consendowrv[][][][\prime], \typerv[][][][\prime]) \sim \trans(\cdot \mid \state[\numhorizon], \consumption[][][\numhorizon], \portfolio[][][\numhorizon])$}
$(\worldstaterv[][][\prime], \consendowrv[][][][\prime], \typerv[][][][\prime]) \sim \trans(\cdot \mid \state[\numhorizon], \portfolio[][][\numhorizon])$, according to a \mydef{transition probability function} 
%\ssadie{$\trans: \states \times \states \times \consumptions \times \portfoliospace \to [0, 1]$}
$\trans: \states \times \states \times \portfoliospace \to [0, 1]$ that depends on the consumers' 
%\ssadie{consumption profile $\consumption[][][\numhorizon] \doteq (\consumption[1][][\numhorizon], \hdots, \consumption[\numbuyers][][\numhorizon])^T$ and}
investment portfolio profile $\portfolio[][][\numhorizon] \doteq (\portfolio[1][][\numhorizon], \hdots, \portfolio[\numbuyers][][\numhorizon])^T \in \portfoliospace$, after which the economy transitions to a new state $\staterv[\numhorizon + 1] \doteq (\worldstaterv[][][\prime], \consendowrv[][][][\prime] + \portfolio[][][\numhorizon] \returns[{\worldstaterv[][][\prime]}], \typerv[][][][\prime])$, where the consumers' new endowments depends on their returns $\portfolio[][][\numhorizon] \returns[{\worldstaterv[][][\prime]}] \in \R^{\numbuyers \times \numcommods}$ on their investments.

%\amy{so Sadie, what you are saying eith these proposed edits is that transitions depend only on investments, not on consumptions? please confirm, and i will accept the changes.} \sadie{Yes! I talked with Deni, and we need this to prove Walras's law, which I think it's reasonable as you can think of our Fisher models where your future state only depends on your saving not consumption.}


% Intuitively, the transition probability function depends on the buyer's financial assets (including its saving of the num\'eraire commodity, since each financial asset will provide the consumer with a different return on its investment based on a future state of the world, hence determining the buyer's future state. 
% \deni{NEXT FINANCIAL STATE IS DETERMINISTIC}

\begin{remark}
If only one commodity is delivered in exchange for assets,
%at all world states where a delivery of commodities occurs, 
i.e., for all world states $\worldstate \in \worldstates$, $\rank(\returns[\worldstate]) \leq 1$, then the generalized Arrow securities are \mydef{num\'eraire generalized Arrow securities}, and the assets are called \mydef{financial assets}.%
\footnote{Recall that the num\'eraire is a fixed commodity that is used to standardize the value of other commodities, while a num\'eraire generalized Arrow security is a generalized Arrow security that delivers its returns in terms of the num\'erarire. 
If the assets deliver exactly one commodity, i.e., $\rank (\returns[\worldstate]) = 1$ at all world states $\worldstate$, we take that commodity to be the num\'eraire for the corresponding spot markets.
On the other hand, if the assets deliver no commodity, i.e., $\rank (\returns[\worldstate]) = 0$ at world state $\worldstate$, then we can take any arbitrary commodity to be the num\'eraire, in which case, the assets vacuously ``deliver'' zero units of the num\'erarire, and no units of any other commodities either.}
An infinite horizon Markov exchange economy is \mydef{world-state-contingent} iff the cardinality of the world state space is weakly greater than that of the spot market space, i.e., $|\worldstates| \, \geq |\consendowspace \times \typespace|$. 
Intuitively, when this condition holds,
%the cardinality of the world state space is weakly greater than that of the spot market space, 
there exists a surjection from world states to spot market states, which implies that spot market states are implicit in
%can endogenously be determined by
world states, so that the spot market states can be dropped from the state space, i.e., $\states \doteq \worldstates$.
An infinite horizon Markov exchange economy has \mydef{complete asset markets} if it is world-state-contingent, and assets can deliver some commodity at all world states, i.e., for all world states $\worldstate \in \worldstates$, $\rank(\returns[\worldstate]) \geq 1$.
Otherwise, it has \mydef{incomplete asset markets}.
Colloquially, we call an infinite horizon exchange economy with (in)complete asset markets an \mydef{(in)complete exchange economy}. 
Intuitively, in complete exchange economies, consumers can insure themselves against all future realizations of the spot market---uncertainty regarding their endowments and types---since a complete exchange economy is world-state contingent.
Further, when there is only a single commodity, s.t. $\numcommods = 1$, and only one financial asset which is a risk-free bond s.t. $\numassets = 1$, and the return matrix for all world states $\worldstate \in \worldstates$ (now a scalar since there is only one commodity and one financial asset) is given by $\returns[\worldstate][ ][ ] \doteq \alpha$, for some $\alpha \in \R$, we obtain the standard incomplete market model \amy{INCREDIBLE!!! NO WONDER YOU ARE PROUD!} \cite{blackwell1965discounted, lucas1971investment}. 
\end{remark}

A \mydef{history} $\hist[][][] \in \hists[\numhorizons] \doteq (\states \times \consumptions \times \portfoliospace \times \simplex[\numcommods]\times \R^\numassets)^\numhorizons \times \states$ 
%of length $\numhorizons \in \N$
is a sequence  $\hist[][][] = ((\state[\numhorizon], \consumption[][][\numhorizon], \portfolio[][][\numhorizon], \price[][\numhorizon], \assetprice[][\numhorizon])_{\numhorizon = 0}^{\numhorizons-1}, \state[\numhorizons])$  of tuples comprising states, consumption profiles, investment profiles, commodity price, and asset prices s.t.\@ a history of length $0$ corresponds only to the initial state of the economy. 
For any history $\hist[][][] \in \hists[\numhorizons]$, 
%of length $\numhorizons \in \N$,
we denote by $\hist[:p]$ the first $p \in [\numhorizons^{*}]$ steps of $\hist$, i.e., $\hist[:p]\doteq ((\state[\numhorizon], \consumption[][][\numhorizon], \portfolio[][][\numhorizon], \price[][\numhorizon], \assetprice[][\numhorizon])_{\numhorizon = 0}^{p-1}, \state[p])$.
Overloading notation, we define the \mydef{history space} $\hists \doteq \bigcup_{\numhorizons = 0}^\infty \hists[\numhorizons]$, and
then \mydef{consumption}, \mydef{investment}, \mydef{commodity price} and \mydef{asset price policies} as mappings  $\consumption[\buyer][][][]: \hists \to \consumptions[\buyer]$, $\portfolio[\buyer][][]: \hists \to \portfoliospace[\buyer]$, $\price: \hists \to \simplex[\numgoods]$, and $\assetprice: \hists \to \R^\numassets$ from histories to consumptions, investments, commodity prices, and asset prices, respectively, s.t.\@ $(\consumption[\buyer][][][], \portfolio[\buyer][][])(\hist)$ is the consumption-investment decision of consumer $\buyer \in \buyers$, and $(\price, \assetprice)(\hist)$ are commodity and asset prices, both at history $\hist \in \hists$. 
A \mydef{consumption policy profile} (resp.\@ \mydef{investment policy profile}) $\consumption(\hist) \doteq (\consumption[1], \hdots, \consumption[\numbuyers])(\hist)^T$ (resp.\@ $\portfolio(\hist) \doteq (\portfolio[1], \hdots, \portfolio[\numbuyers])(\hist)^T$) is a collection of consumption (resp.\@ investment) policies for all consumers.
A consumption policy $\consumption[\buyer]: \states \to \consumptions[\buyer]$ is  \mydef{Markov} if it depends only on the last state of the history, i.e., $\consumption[\buyer][][][] (\hist) = \consumption[\buyer][][][] (\state[\numhorizons])$, for all histories $\hist \in \hists[\numhorizons]$ of all lengths $\numhorizons \in \N$.
An analogous definition extends to investment, commodity price, and asset price policies.

%\amy{as below, $\histdistrib$ should depend on all consumers' policies. so we should be defining something more like $\histdistrib[\initstates][{(\allocation, \portfolio, \price, \assetprice)}][\numhorizons] (\hist)$, not $\histdistrib[\initstates][{(\allocation[\buyer][][][], \portfolio[\buyer][][])}][\numhorizons] (\hist)$} \sadie{Changed it and also the history definition above!}

Given $\policy \doteq (\consumption, \portfolio, \price, \assetprice)$ 
%\ssadie{consumer $\buyer$'s consumption and \samy{portfolio}{investment} policies $(\allocation[\buyer][][][], \portfolio[\buyer][][])$}{a consumption policy profile $\consumption$, \samy{a portfolio}{an investment} policy profile $\portfolio$, a commodity price policy $\price$, a asset price policy $\assetprice$} 
and a history $\hist[][][] \in \hists[\numhorizons]$, we define the \mydef{discounted history distribution} assuming initial state distribution $\initstates$ as
%
%\begin{align}
    $
    %\histdistrib[\initstates][{\consumption, \portfolio, \price, \assetprice}][\numhorizons] (\hist[][][]) 
    \histdistrib[\initstates][\policy][\numhorizons] (\hist[][][]) 
    = \initstates (\state[0]) \prod_{\numhorizon = 0}^{\numhorizons-1} \discount^\numhorizon \trans (\worldstaterv[\numhorizon +1], \consendow[][][\numhorizon + 1] + \portfolio[][][\numhorizon] \returns[{\worldstaterv[\numhorizon + 1][][]}], \type[][][\numhorizon + 1] \mid \state[\numhorizon], \portfolio[][][\numhorizon]) \setindic[{\{\portfolio[][][](\hist[:\numhorizon]) \}}]  (\portfolio[][][\numhorizon])
    $. \deni{$\worldstaterv$ should be not rv? Also how do we know this is not 0, what was our argument against the measurability problem.}
Overloading notation, we define the set of all realizable trajectories $\hists[\policy][\numhorizons]$ of length $\numhorizons$ under policy profile $\policy$ as $\hists[\policy][\numhorizons] \doteq \supp (\histdistrib[\initstates][\policy][\numhorizons])$, i.e., the set of all histories that occur with non-zero probability, 
and we let $\histrv[][] = \left((\staterv[\numhorizon], \actionrv[][][\numhorizon])_{\numhorizon = 0}^{\numhorizons - 1},
\staterv[\numhorizons]  \right)$ be any randomly sampled history from $\histdistrib[\initstates][\policy][\numhorizons]$.
Finally, we abbreviate $\histdistrib[\initstates][\policy] \doteq \histdistrib[\initstates][\policy][\infty]$.

% \deni{No Arbitrage condition!!! define no arbitrage which implies financial market clearance at state 0 and hence also for all other states as well}


\subsection{Solution Concepts and Existence}
\label{sec:inf_eqa}
\if 0
\deni{FINISH} 

Notice that one can reduce any infinite horizon \samy{}{Markov} exchange economy to a static exchange market with infinitely many goods $(\numbuyers, \infty, 0, \consumptions^{\N \times |\states|}, \consendowspace^{\N \times |\states| }, \emptyset, \cumulutil, (\consendow)_{\consendow \in \consendowspace}, \emptyset)$ by modeling the same commodities at different states and time periods as different commodities. 
Under this reduction, all consumption and pricing decisions are made at time $0$, and consumptions of commodities at subsequent time periods are interpreted as state-contingent claims on those commodities. 
%\deni{... etc...}
As this interpretation is \emph{unrealistic}---we do not all make a state-contingent plan for our lives (or a poker game) on the day we are born---and \emph{unwieldy}---we cannot ever hope to solve such a large market (or game) efficiently---we turn our attention to \mydef{closed-loop solutions} that incorporate feedback as time progresses, i.e., we seek equilibria in history-dependent policies. \fi

An \mydef{outcome} $(\consumption[][][][], \portfolio[][][][], \price, \assetprice): \hists \to \consumptions \times \portfoliospace \times \simplex[\numgoods] \times \R^\numassets$ of an infinite horizon Markov exchange economy is a tuple%
\footnote{Instead of expressing this tuple as $\consumptions^{\hists} \times \portfoliospace^{\hists} \times \simplex[\numgoods]^{\hists} \times {\R^\numassets}^{\hists}$, we sometimes write $(\consumption[][][][], \portfolio[][][][], \price, \assetprice): \hists \to \consumptions \times \portfoliospace \times \simplex[\numgoods] \times \R^\numassets$.}
% \sdeni{}{While an outcome is defined as a tuple of policies $\consumptions^{\hists} \times \portfoliospace^{\hists} \times \simplex[\numgoods]^{\hists} \times \R^\numassets^{\hists}$, for convenience, it will often be convenient to assume that $(\consumption[][][][], \portfolio[][][][], \price, \assetprice)$ is represented as a function from histories to consumption profiles $(\consumption[][][][], \portfolio[][][][], \price, \assetprice): \hists \to \consumptions \times \portfoliospace \times \simplex[\numgoods] \times \R^\numassets$, investment profiles, commodity price, and asset prices. The mathematical representation of $(\consumption[][][][], \portfolio[][][][], \price, \assetprice)$ will always be clear from context.}} \amy{is an outcome a tuple or a function from histories to tuples?}\sadie{I would say it's a tuple of functions/policies} \deni{Sadie is right.} \amy{BUT THE MATH IS OTHERWISE?!!? can we delete the $\hists \to$?} \deni{Wait I actually think here a footnote is best} 
consisting of a commodity prices policy, an asset prices policy, a consumption policy profile, and an investment policy profile. 
An outcome is \mydef{Markov} if all its constituent policies are Markov: i.e., if it depends only on the last state of the history, i.e., $(\consumption[][][][], \portfolio[][][][], \price, \assetprice)(\hist) = (\consumption[][][][], \portfolio[][][][], \price, \assetprice)(\state[\numhorizons])$, for all histories $\hist \in \hists[\numhorizons]$ of all lengths $\numhorizons \in \N$.
% \amy{i copied this language from below. but i'd prefer to say something like, ``the last state in a history suffices to represent a history.''}

%it consists of a Markov commodity prices policy, a Markov asset prices policy, a Markov consumption policy profile, and a Markov portfolio policy profile.

We now introduce a number of properties of infinite horizon Markov exchange economies outcomes, which we use to define our solution concepts. 
While these properties are defined broadly for (in general, history-dependent) outcomes, they also apply {in the special case of Markov outcomes. 

% where the expectation is taken w.r.t $\staterv[\iter + 1] \sim \trans( \cdot \mid \staterv[\iter], (\allocation[\buyer][][], \portfolio[\buyer][][])(\histrv[][][\iter]))$ and $\histrv[][][\iter]$ denotes the random history of length $\iter$.

%\amy{this value function does not seem to be defined accurately. these value functions need to depend on ALL players/buyers/consumers' policies, not just on $i$'s. they only have their own individual value functions, if the policy of all the other players' is fixed. so it has to be something more like this (but definitely double check carefully):} \sadie{I agree, and I rewrote everything (we also need pricing policies). Deni also agrees on this :)}

Given a consumption and investment profile $(\allocation, \portfolio)$, 
the \mydef{consumption state-value function} $\vfunc[\buyer][{(\allocation, \portfolio, \price, \assetprice)}]: \states \to \R$ is defined as:
%\begin{align}
    $\vfunc[\buyer][{(\allocation, \portfolio, \price, \assetprice)}] (\state) \doteq \Ex_{\histrv \sim \histdistrib[\initstates][{( \allocation, \portfolio, \price, \assetprice)}]} \left[ \sum_{\numhorizon = 0}^\infty \discount^\numhorizon \util[\buyer] \left( \allocation[\buyer] (\histrv[:\numhorizon][][]); \typerv[][][\numhorizon] \right) \mid \staterv[0] = \state \right]$.
%\enspace .
%\end{align}

%\amy{so utility does not also depend on investments $\allocation[\buyer] (\histrv[:\numhorizon][][])$? it depends only on consumption, again b/c the dependence on investments shows up sometime later in terms of the ensuing consumptions?}

\amy{ugh, we do take exp'ns wrt $\mu$ twice. ugh!}

An outcome $(\consumption[][][][*], \portfolio[][][][*], \price[][][*], \assetprice[][][*])$ is \mydef{optimal} for $\buyer$ 
% $(\allocation[\buyer][][][*], \portfolio[\buyer][][][*]): \hists \to \consumptions[\buyer] \times \portfoliospace[\buyer]$ s.t.
% $\portfolio[\buyer][][]: \hists \to \savings[\buyer]$ $(\allocationrv[][][\iter], \portfoliorv[][][\iter])_{\iter = 0}^\infty \subset \consumptions \times \portfoliospace$ of consumptions and portfolios
if $\buyer$'s \mydef{expected cumulative utility}   $\cumulutil[\buyer] (\allocation, \portfolio, \price, \assetprice) \doteq \Ex_{\state \sim \initstates} \left[ \vfunc[\buyer][{(\allocation, \portfolio, \price, \assetprice)}] (\state) \right]$ is maximized over all affordable consumption and investment policies, i.e.,
%\sdeni{}{those whose outputs are in $\player$'s budget set, for all states $\state \in \states$:}
\begin{align}
    (\allocation[\buyer][][][*], \portfolio[\buyer][][][*]) \in \argmax_{\substack{(\allocation[\buyer][][][], \portfolio[\buyer][][]): \hists \to \consumptions[\buyer] \times \portfoliospace[\buyer], \forall \numhorizon \in \N, \hist[][][] \in \hists[\numhorizon] \\(\allocation[\buyer], \portfolio[\buyer])(\hist[:\numhorizon]) \in \budgetset[\buyer] (\consendow[\buyer][][\numhorizon], \price[][][*] (\hist[:\numhorizon]), \assetprice[][][*] (\hist[:\numhorizon]))
    % \allocation[\buyer] (\hist[:\numhorizon]) \cdot \price(\hist[:\numhorizon]) + \portfolio[\buyer] (\hist[:\numhorizon]) \cdot \assetprice(\hist[:\numhorizon]) \leq \consendow[\buyer][][\iter] \cdot \price(\hist[:\numhorizon]) 
    }}\cumulutil[\buyer] (\allocation[\buyer][][][], \allocation[-\buyer][][][*], 
    \portfolio[\buyer][][], \portfolio[-\buyer][][][*], \price[][][*], \assetprice[][][*]) \enspace .
\label{eq:opt_consum}
\end{align}

\noindent
A Markov outcome $(\consumption[][][][*], \portfolio[][][][*], \price[][][*], \assetprice[][][*])$ is \mydef{Markov perfect} for $\buyer$ 
% $(\allocation[\buyer][][][*], \portfolio[\buyer][][][*]): \hists \to \consumptions[\buyer] \times \portfoliospace[\buyer]$ s.t.
% $\portfolio[\buyer][][]: \hists \to \savings[\buyer]$ $(\allocationrv[][][\iter], \portfoliorv[][][\iter])_{\iter = 0}^\infty \subset \consumptions \times \portfoliospace$ of consumptions and portfolios
if $\buyer$ maximizes its consumption state-value function over all affordable consumption and investment policies, i.e.,
%\sdeni{}{those whose outputs are in $\player$'s budget set, for all states $\state \in \states$:}
\begin{align}
    (\allocation[\buyer][][][*], \portfolio[\buyer][][][*]) \in \argmax_{\substack{(\allocation[\buyer], \portfolio[\buyer]): \states \to \consumptions[\buyer] \times \portfoliospace[\buyer]: \forall \state \in \states, \\(\allocation[\buyer], \portfolio[\buyer])(\state) \in \budgetset[\buyer] (\consendow[\buyer], \price[][][*] (\state), \assetprice[][][*] (\state))
    % \allocation[\buyer] (\state) \cdot \price(\state) + \portfolio[\buyer] (\state) \cdot \assetprice(\state)  \leq \consendow[\buyer] \cdot \price(\state) + \portfolio[\buyer][][] (\state) \cdot \returns[{\worldstate}]} 
    }} \left\{ \vfunc[\buyer][{(\allocation[\buyer][][][], \allocation[-\buyer][][][*], 
    \portfolio[\buyer][][], \portfolio[-\buyer][][][*], \price[][][*], \assetprice[][][*])}] (\state)  \right\} \enspace .
\label{eq:Markov_consum}
\end{align}
% Any solution to a stochastic Fisher market is called an \mydef{outcome} and described by a tuple $(\price, \assetprice, \consumption[][][][], \portfolio[][][][], \saving[][][]) \in \states \to \left(\simplex \times \R^\numassets \times \consumptions \times \savings \right)^\N$ of \mydef{spot price, asset price, consumption, portfolio, and saving policies} respectively, i.e. mappings from states to prices and consumption-saving decision for all time-steps, such that for all $\state \in \states$ and  $\iter \in \N$, $(\price[][\iter], \assetprice[][\iter], \consumption[][][\iter][], \portfolio[][][\iter][], \saving[][\iter][])(\state)$ denotes the prices and consumption-saving decisions made at time $\iter$ in state $\state$. A \mydef{stationary outcome} $(\price, \assetprice, \consumption[][][][], \portfolio[][][][], \saving[][][])$ is an outcome which is time-invariant i.e., $(\price, \assetprice, \consumption[][][][], \portfolio[][][][], \saving[][][]) = (\price[][\iter], \assetprice[][\iter], \consumption[][][\iter][], \portfolio[][][\iter][], \saving[][\iter][])$ for all time-steps $\iter \in \N_+$, in which case we omit all time indices.

A consumption policy $\consumption$ is said to be \mydef{feasible} iff for all time horizons $\numhorizons \in \N$ and histories $\hist \in \hists[\numhorizons]$ of length $\numhorizons$,
$\sum_{\buyer \in \buyers} \consumption[\buyer][][][] (\hist) - \sum_{\buyer \in \buyers} \consendow[\buyer][][\numhorizons] \leq \zeros[\numcommods]$,
\if 0
\begin{align}
    \begin{array}{c|c}
    \sum_{\buyer \in \buyers} \consumption[\buyer][][][] (\hist) - \sum_{\buyer \in \buyers} \consendow[\buyer][][\numhorizons] \leq \zeros[\numcommods] &\left(\text{resp.    } \sum_{\buyer \in \buyers} \portfolio[\buyer][][] (\hist) \leq \zeros[\numassets]  \right) \enspace .
    \end{array}
\label{eq:feasible}
\end{align} 
\fi
where $\consendow[\buyer][][\numhorizons] \in \consendowspace[\buyer]$ is consumer $\buyer$'s endowment at the end of history $\hist$, i.e., at state $\state[\numhorizons]$. 
Similarly, an investment policy is \mydef{feasible} if $\sum_{\buyer \in \buyers} \portfolio[\buyer][][] (\hist) \leq \zeros[\numassets]$.
If all the consumption and investment policies associated with an outcome are feasible, we will colloquially refer to the outcome as \mydef{feasible} as well.

An outcome $(\consumption[][][][], \portfolio[][][][], \price, \assetprice)$ is said to satisfy \mydef{Walras' law} iff for all time horizons $\numhorizons \in \N$ and histories $\hist \in \hists[\numhorizons]$ of length $\numhorizons$,
%\begin{align}
        $\price(\hist) \cdot \left( \sum_{\buyer \in \buyers} \consumption[\buyer][][][] (\hist) - \sum_{\buyer \in \buyers} \consendow[\buyer][][\numhorizons] \right)  +  \assetprice[][] (\hist) \cdot \left(\sum_{\buyer \in \buyers} \portfolio[\buyer][][] (\hist) \right) 
        % - \returns[{\worldstate[\numhorizons]}] \cdot \left(\sum_{\buyer \in \buyers} \portfolio[\buyer][][\numhorizons] \right) 
        = 0$,
%\enspace .
%\label{eq:walras}
%\end{align}
where, as above, $\consendow[\buyer][][\numhorizons] \in \consendowspace[\buyer]$ is consumer $\buyer$'s endowment at state $\state[\numhorizons]$. 
% An outcome $(\price, \assetprice, \consumption[][][][], \portfolio[][][][], \saving[][][])$ is said to be feasible if and only if:

The canonical solution concept for stochastic economies is the Radner equilibrium.

\begin{definition}[Radner Equilibrium]
    A \mydef{Radner (or sequential competitive) equilibrium (RE)} \cite{radner1972existence} of an infinite horizon Markov exchange economy $\economy$ is an outcome $(\consumption[][][][*], \portfolio[][][][*], \price[][][*], \assetprice[][][*])$
    %$: \hists \to \consumptions \times \portfoliospace \times \simplex[\numgoods] \times \R^\numassets$ 
    that is
    1.~optimal for all consumers, i.e., \Cref{eq:opt_consum} is satisfied, for all consumers $\buyer \in \buyers$; 
    2.~feasible; and 
    3.~satisfies Walras' law.
    %%% SPACE
    %(\Cref{eq:walras}).
\end{definition}

As a Radner equilibrium is in general infinite dimensional, we are often interested in a recursive Radner equilibrium which is a \emph{Markov\/} outcome, i.e., one that depends only on the last state of the history rather than the entire history \sdeni{}{, and as such better behaved.}.


\begin{definition}[Recursive Radner Equilibrium]\label{def:rre}
    A \mydef{recursive Radner (or Walrasian or competitive) equilibrium (RRE)} \cite{mehra1977recursive, prescott1980recursive} of an infinite horizon Markov exchange economy $\economy$ is a Markov outcome $(\consumption[][][][*], \portfolio[][][][*], \price[][][*], \assetprice[][][*])$
    %$: \states \to \consumptions \times \portfoliospace \times \simplex[\numgoods] \times \R^\numassets$ 
    that is 
    1.~Markov perfect for all consumers, i.e., \Cref{eq:Markov_consum} is satisfied, for all consumers $\buyer \in \buyers$; 
    2.~feasible; and 
    3.~satisfies Walras' law.
    %%% SPACE
    %(\Cref{eq:walras}).
\end{definition}



The following assumptions are standard in the equilibrium literature (see, for instance, \citet{geanakoplos1990introduction}).
We prove the existence of a recursive Radner equilibrium under these assumptions.
% The canonical solution concept for stochastic Fisher markets is the \mydef{sequential competitive (or Radner) equilibrium (SCE)} \cite{radner1972sequentualeqm}: an outcome $(\price[][][*], \assetprice[][][*], \consumption[][][][*], \portfolio[][][][*], \saving[][][*])$, s.t. 1) the buyers are expected utility maximizing fixing the price sequence (i.e., fixing $(\price[][][*], \assetprice[][*])$, for all $\buyer \in \buyers$, $(\consumption[\buyer][][][*], \portfolio[\buyer][][][*], \saving[\buyer][][*])$ satisfies \Cref{eq:eu_max}), and 2)  $(\consumption[\buyer][][][*], \portfolio[\buyer][][][*], \saving[\buyer][][*])$ are feasible, and the outcome satisfies Walras' law. Under the assumptions considered in this paper a SCE is guaranteed to exist in stochastic Fisher markets \cite{prescott1972note}. 

% A refinement of sequential competitive equilibrium is \mydef{recursive competitive equilibrium (RCE)} \cite{mehra1977recursive}: a stationary outcome $(\price[][][*], \assetprice[][*], \consumption[][][][*], \portfolio[][][][*], \saving[][][*])$, s.t. 1) the buyers are expected utility maximizing fixing the price sequence starting from any state (i.e., fixing $(\price[][][*], \assetprice[][*])$, for all $\buyer \in \buyers$, $(\consumption[\buyer][][][*], \portfolio[\buyer][][][*], \saving[\buyer][][*])$ satisfies \Cref{eq:vfunc_max}), and 2)  $(\consumption[\buyer][][][*], \portfolio[\buyer][][][*], \saving[\buyer][][*])$ are feasible, and the outcome satisfies Walras' law.

\begin{assumption}
\label{assum:existence_RRE}
Given an infinite horizon Markov exchange economy $\economy$,
%= (\numbuyers, \numcommods, \numassets, \numtypes, \states, \consumptions, \portfoliospace, \consendowspace, \typespace,  \util, \discount, \trans, \returnset, \initstates)$, 
assume for all 
%%% SPACE
%consumers 
$\buyer \in \buyers$,
\begin{enumerate}
    \item $\consumptions$,  $\portfoliospace$,
    $\consendowspace$, are non-empty, closed, convex, with $\consendowspace$ additionally bounded;
    \item $(\type[\buyer], \consumption[\buyer]) \mapsto\util[\buyer] (\consumption[\buyer]; \type[\buyer])$ is continuous and concave, and $(\state, \portfolio[\buyer]) \mapsto\trans(\state[][][\prime] \mid \state, \portfolio[\buyer], \portfolio[-\buyer])$ is continuous and stochastically concave, for all $\state[][\prime] \in \states$ and $\portfolio[-\buyer] \in \portfoliospace[-\buyer]$;
    \item for all $\consendow[\buyer] \in \consendowspace[\buyer]$, the correspondence $$(\price, \assetprice) \rightrightarrows \budgetset[\buyer] (\consendow[\buyer], \price, \assetprice) \cap \{(\consumption[\buyer], \portfolio[\buyer]) \mid \sum_{\buyer \in \buyers} \consumption[\buyer] \leq \sum_{\buyer \in \buyers} \consendow[\buyer], \sum_{\buyer \in \buyers} \portfolio[\buyer] \leq \zeros[\numcommods], (\consumption, \portfolio) \in \consumptions \times \portfoliospace\}$$ is continuous\footnote{One way to ensure that this condition holds is to assume that for all $\state=(\worldstate, \consendow, \type)\in \states$, returns from assets are positive $\returns[\worldstate] \geq \zeros[\numcommods\numassets]$, and for all consumers $\buyer \in \buyers$, there exists $(\consumption[\buyer], \portfolio[\buyer]) \in \consumptions[\buyer] \times \portfoliospace[\buyer]$, s.t. $\consumption[\buyer] < \consendow[\buyer]$, $\portfolio[\buyer] < 0$.};
    \item $\budgetset[\buyer] (\consendow[\buyer], \price, \assetprice) \cap \{(\consumption[\buyer], \portfolio[\buyer]) \mid \sum_{\buyer \in \buyers} \consumption[\buyer] \leq \sum_{\buyer \in \buyers} \consendow[\buyer], \sum_{\buyer \in \buyers} \portfolio[\buyer] \leq \zeros[\numcommods], (\consumption, \portfolio) \in \consumptions \times \portfoliospace\}$ is non-empty, convex, and compact, for all $\consendow[\buyer] \in \consendowspace[\buyer]$, $\price\in \simplex[\numgoods]$, and $\assetprice\in \R^\numassets$\footnote{One way to ensure that this condition holds is to assume that for all $\state=(\worldstate, \consendow, \type)\in \states$, returns from assets are positive, i.e., $\returns[\worldstate] \geq \zeros[\numcommods\numassets]$, and $\consumptions, \portfoliospace$ are bounded from below.};
    \item (no saturation) there exists an $\consumption[\buyer][][][+] \in \consumptions[\buyer]$ s.t. $\util[\buyer] (\consumption[\buyer][][][+]; \type[\buyer]) > \util[\buyer] (\consumption[\buyer]; \type[\buyer])$, for all $\consumption[\buyer] \in \consumptions[\buyer]$ and $\type[\buyer] \in \typespace[\buyer]$.
\end{enumerate}

\end{assumption}

Next we associate an \mydef{exchange economy Markov pseudo-game} $\mgame$ with a given infinite horizon Markov exchange economy $\economy$.

\begin{definition}[Exchange Economy Markov Pseudo-Game]
Let $\economy$
%= (\numbuyers, \numcommods, \numassets, \numtypes, \states, \consumptions, \portfoliospace, \assetpricespace, \consendowspace, \typespace, \samy{\util}{\reward}, \discount, \trans, \returnset, \initstates)$ 
be an infinite horizon Markov exchange economy. 
The corresponding \mydef{exchange economy Markov pseudo-game} $\mgame = (\numbuyers+1, %numplayer
\numcommods+\numassets, %numactions
\states, 
\bigtimes_{\player\in \players}(\consumptions[\player] \times \portfoliospace[\player]) \times (\pricespace \times \assetpricespace), %actionspace
\newbudgetset, %\actions, 
\newreward, 
\newtrans, 
\discount', 
\initstates')$ is defined as
%
\begin{itemize}
\setlength{\itemindent}{-5mm}
    \item The $\numbuyers+1$ players comprise $\numbuyers$ consumers, players $1, \ldots, \numbuyers$, and one auctioneer, player $\numbuyers+1$.
    
    \item The set of states $\states = \worldstates \times \consendowspace \times \typespace$.
    At each state $\state = (\worldstate, \consendow, \type) \in \states$, 
    \begin{itemize}
    \setlength{\itemindent}{-5mm}
        \item each consumer $\buyer \in \buyers$ chooses an action $\action[\buyer] = (\consumption[\buyer], \portfolio[\buyer]) \in \newbudgetset[\buyer] \left( \state, \action[-\buyer] \right) \subseteq \consumptions[\buyer] \times \portfoliospace[\buyer]$ from a set of feasible actions $\newbudgetset[\buyer] (\state, \action[-\buyer]) = \budgetset[\buyer] (\consendow[\buyer], \action[\numbuyers+1]) \cap \{(\consumption[\buyer], \portfolio[\buyer]) \mid \sum_{\buyer \in \buyers} \consumption[\buyer] \leq \sum_{\buyer \in \buyers} \consendow[\buyer], \sum_{\buyer \in \buyers} \portfolio[\buyer] \leq \zeros[\numcommods], (\consumption, \portfolio) \in \consumptions \times \portfoliospace\}$ and receives reward $\newreward[\buyer] (\state, \action) \doteq \util[\buyer] (\consumption[\buyer]; \type[\buyer])$; and
        
        \item  the auctioneer $\numbuyers+1$ chooses an action $\action[\numbuyers+1] = (\price, \assetprice) \in \newbudgetset[\numbuyers+1] \left( \state, \action[-(\numbuyers+1)] \right) \doteq \pricespace \times \assetpricespace$ 
        %from the unconstrained fixed set $\newbudgetset[\numbuyers+1] (\state, \action[-(\numbuyers+1)]) \doteq\pricespace[\numcommods] \times \R^{\numassets}$ 
        where $\pricespace
        \doteq \simplex[\numcommods]$ and  $\assetpricespace \subseteq [0, \max_{\consendow \in \consendowspace} \sum_{\buyer \in \buyers} \sum_{\commod \in \commods} \consendow[\buyer][\commod]]^\numassets$, and and receives reward $\newreward[\numbuyers+1] (\state, \action) \doteq \price\cdot \left(\sum_{\buyer\in \buyers}\consumption[\buyer]-\sum_{\buyer\in \buyers}\consendow[\buyer] \right) + \assetprice\cdot \left(\sum_{\buyer\in \buyers}\portfolio[\buyer] \right)$. 
    \end{itemize}
    
    \item The transition probability function 
    %$\newtrans: \states\times \states \times \actionspace \to [0,1]$ 
    is defined as $\newtrans(\state[][][\prime] \mid \state, \action) \doteq \trans(\state[][][\prime] \mid \state, \portfolio)$.

    \item The discount rate $\discount' = \discount$ and the initial state distribution $\initstates' = \initstates$.
\end{itemize}
\end{definition}


% \begin{definition}[Generalized Markov Exchange Economy Game]
% Let $\economy$
% %= (\numbuyers, \numcommods, \numassets, \numtypes, \states, \consumptions, \portfoliospace, \assetpricespace, \consendowspace, \typespace, \samy{\util}{\reward}, \discount, \trans, \returnset, \initstates)$ 
% be an infinite horizon Markov exchange economy. 
% The corresponding \mydef{auctioneer-consumer \samy{\samy{generalized Markov game}{Markov pseudo-game}}{Markov pseudo-game}} $\mgame = (\numbuyers+1, %numplayer
% \numcommods+\numassets, %numactions
% \states, 
% \bigtimes_{\player\in \players}(\consumptions[\player] \times \portfoliospace[\player]) \times (\pricespace[\numgoods] \times \assetpricespace)), %actionspace
% \newbudgetset, %\actions, 
% \newreward, 
% \newtrans, 
% \discount', 
% \initstates')$ is defined as
% %
% \amy{we should fix the indentation for the items in the list}
% \begin{itemize}
% \setlength{\itemindent}{-5mm}
%     \item The $\numbuyers+1$ players comprise $\numbuyers$ consumers, players $1, \ldots, \numbuyers$, and one auctioneer, player $\numbuyers+1$.
    
%     \item The set of states $\states = \worldstates \times \consendowspace \times \typespace$.
%     At each state $\state = (\worldstate, \consendow, \type) \in \states$, 
%     \begin{itemize}
%     \setlength{\itemindent}{-5mm}
%         \item each consumer $\buyer \in \buyers$ chooses an action $\action[\buyer]=(\consumption[\buyer], \portfolio[\buyer]) \in \newbudgetset[\buyer] (\state, \action[-\buyer]) \subseteq \consumptions[\buyer] \times \portfoliospace[\buyer]$ from a set of feasible actions $\newbudgetset[\buyer] (\state, \action[-\buyer]) = \budgetset[\buyer] (\consendow[\buyer], \action[\numbuyers+1])$ and receives reward $\newreward[\buyer] (\state, \action) \doteq \util[\buyer] (\consumption[\buyer]; \type[\buyer])$; and
        
%         \item  the auctioneer $\numbuyers+1$ chooses an action $\action[\numbuyers+1]=(\price, \assetprice) \in \newbudgetset[\numbuyers+1] (\state, \action[-(\numbuyers+1)]) \doteq \pricespace[\numcommods] \times \assetpricespace$ 
%         %from the unconstrained fixed set $\newbudgetset[\numbuyers+1] (\state, \action[-(\numbuyers+1)]) \doteq\pricespace[\numcommods] \times \R^{\numassets}$ 
%         and receives reward $\newreward[\numbuyers+1] (\state, \action) \doteq \price\cdot \left(\sum_{\buyer\in \buyers}\consumption[\buyer]-\sum_{\buyer\in \buyers}\consendow[\buyer] \right) + \assetprice\cdot \left(\sum_{\buyer\in \buyers}\portfolio[\buyer] \right)$. 
%     \end{itemize}
    
%     \item The transition probability function 
%     %$\newtrans: \states\times \states \times \actionspace \to [0,1]$ 
%     is defined as $\newtrans(\state[][][\prime] \mid \state, \action) \doteq \trans(\state[][][\prime] \mid \state, \bigtimes_{\buyer\in \buyers}\portfolio[\buyer])$.

%     \item The discount rate $\discount' = \discount$ and the initial state distribution $\initstates' = \initstates$.
% \end{itemize}
% \end{definition}

% \begin{restatable}{theorem}{thmexistrre}
% \label{thm:rre=mpgne}
% The set of recursive Radner equilibria (RRE) of any infinite horizon Markov exchange economy $\economy$ that satisfies \Cref{assum:existence_RRE} is equal to the set of Markov perfect GNE of the associated generalized Markov exchange economy game $\mgame$. 
% \end{restatable}

% \begin{proof}
%     Let $\policy[][][*]=(\consumption[][][][*], \portfolio[][][][*], \price[][][*], \assetprice[][][*]): \states \to \consumptions \times \portfoliospace \times \pricespace[\numgoods] \times \assetpricespace $ be an \MPGNE{} of the auctioneer-consumer \samy{\samy{generalized Markov game}{Markov pseudo-game}}{Markov pseudo-game} $\mgame$ associated with $\economy$. 
%     We want to show that it is also an RRE of $\economy$. 

%     First, we want to show that $\policy[][][*]$ is Markov perfect for all consumers. 
%     we can make some easy observations: the state value for consumer $\buyer\in \buyers$ at state $\state\in \states$ induced by the policy $\policy[][][*]$
%     \begin{align}
%         \vfunc[\buyer][{\policy[][][*]}] (\state)=
%         \Ex_{\histrv\sim \histdistrib[\initstates][{\policy[][][*]}]} \left[
%         \sum_{\numhorizon=0}^{\infty}\discount^\numhorizon \newreward(\staterv[\numhorizon], \actionrv[][][\numhorizon]) \mid \staterv[0]=\state)
%         \right]
%         &= 
%         \Ex_{\histrv\sim \histdistrib[\initstates][{\policy[][][*]}]} \left[
%         \sum_{\numhorizon=0}^{\infty}\discount^\numhorizon \util[\buyer] (\consumption[\buyer][][][*] (\staterv[\numhorizon]); \typerv[\buyer][][\numhorizon]) \mid \staterv[0]=\state)
%         \right]
%     \end{align}
%     is equal to the consumption state value induced by $(\consumption[][][][*], \portfolio[][][][*], \price[][][*], \assetprice[][][*])$
%     \begin{align}
%         \vfunc[\buyer][{(\consumption[][][][*], \portfolio[][][][*], \price[][][*], \assetprice[][][*])}] (\state) \doteq \Ex_{\histrv \sim \histdistrib[\initstates][{( \consumption[][][][*], \portfolio[][][][*], \price[][][*], \assetprice[][][*])}]} \left[ \sum_{\numhorizon = 0}^\infty \discount^\numhorizon \util[\buyer] \left( \allocation[\buyer][][][*] (\histrv[:\numhorizon][][]); \typerv[][][\numhorizon] \right) \mid \staterv[0] = \state \right]  \enspace .
%     \end{align}
%     as $\consumption[\player][][][*]$ is Markov. Since $\policy[][][*]$ is a \MPGNE, we know that for any $\buyer\in \buyers$,
%     $(\allocation[\buyer][][][*], \portfolio[\buyer][][][*]) \in \argmax_{\substack{(\allocation[\buyer], \portfolio[\buyer]): \states \to \consumptions[\buyer] \times \portfoliospace[\buyer]: \forall \state \in \states, \\(\allocation[\buyer], \portfolio[\buyer])(\state) \in \budgetset[\buyer] (\consendow[\buyer], \price[][][*] (\state), \assetprice[][][*] (\state))
%     % \allocation[\buyer] (\state) \cdot \price(\state) + \portfolio[\buyer] (\state) \cdot \assetprice(\state)  \leq \consendow[\buyer] \cdot \price(\state) + \portfolio[\buyer][][] (\state) \cdot \returns[{\worldstate}]} 
%     }} \left\{ \vfunc[\buyer][{(\allocation[\buyer][][][], \allocation[-\buyer][][][*], 
%     \portfolio[\buyer][][], \portfolio[-\buyer][][][*], \price[][][*], \assetprice[][][*])}] (\state)  \right\}$ for all $\state\in \states$, so $(\consumption[][][][*], \portfolio[][][][*], \price[][][*], \assetprice[][][*])$ is Markov perfect. 

%     Next, we want to show that $(\consumption[][][][*], \portfolio[][][][*], \price[][][*], \assetprice[][][*])$ satisfies the Walras's law. First, we show that for any $\buyer\in \buyers$, $\state\in \states$, $\consumption[\buyer][][][*] (\state) \cdot\price[][][*] (\state)+ \portfolio[\buyer][][][*] (\state) \cdot \assetprice[][][*] (\state)- \consendow[\buyer] \cdot \price[][][*] (\state)=0$. By way of contradiction, assume that there exists some $\buyer\in \buyers$, $\state\in \states$ such that $\consumption[\buyer][][][*] (\state) \cdot\price[][][*] (\state)+ \portfolio[\buyer][][][*] (\state) \cdot \assetprice[][][*] (\state)- \consendow[\buyer] \cdot \price[][][*] (\state) \neq 0$. Note that $(\consumption[\buyer][][][*] (\state), \portfolio[\buyer][][][*] (\state)) \in \newbudgetset(\state, \action[-\buyer])=\budgetset(\consendow[\buyer], \price[][][*] (\state), \assetprice[][][*] (\state))=\{(\consumption[\buyer], \portfolio[\buyer]) \in \consumptions[\buyer] \times \portfoliospace[\buyer] \mid  \consumption[\buyer] \cdot \price[][][*] (\state) + \portfolio[\buyer] \cdot \assetprice[][][*] (\state)  \leq \consendow[\buyer] \cdot \price[][][*] (\state) \}$, so we must have  $\consumption[\buyer][][][*] (\state) \cdot\price[][][*] (\state)+ \portfolio[\buyer][][][*] (\state) \cdot \assetprice[][][*] (\state)- \consendow[\buyer] \cdot \price[][][*] (\state)< 0$. By the (no saturation) condition of \Cref{assum:existence_RRE}, there exists $\consumption[\buyer][][][+] \in \consumptions[\buyer]$ s.t. $\util[\buyer] (\consumption[\buyer][][][+]; \type[\buyer])>\util[\buyer] (\consumption[\buyer][][][*] (\state); \type[\buyer])$. Moreover, since $\consumption[\buyer] \mapsto \util[\buyer] (\consumption[\buyer]; \type[\buyer])$ is concave, for any $0<t<1$, $\util[\buyer] (t\consumption[\buyer][][][+]+(1-t) \consumption[\buyer][][][*] (\state); \type[\buyer])>\util[\buyer] (\consumption[\buyer][][][*] (\state); \type[\buyer])$. Since $\consumption[\buyer][][][*] (\state) \cdot\price[][][*] (\state)+ \portfolio[\buyer][][][*] (\state) \cdot \assetprice[][][*] (\state)- \consendow[\buyer] \cdot \price[][][*] (\state)< 0$, we can pick $t$ small enough such that $\consumption[\buyer][][][\prime]=t\consumption[\buyer][][][+]+(1-t) \consumption[\buyer][][][*] (\state)$ satisfies $\consumption[\buyer][][]['] (\state) \cdot\price[][][*] (\state)+ \portfolio[\buyer][][][*] (\state) \cdot \assetprice[][][*] (\state)- \consendow[\buyer] \cdot \price[][][*] (\state) \leq 0$ but $\consumption[\buyer][][][\prime] \in \consumptions[\buyer]$ s.t. $\util[\buyer] (\consumption[\buyer][][][+]; \type[\buyer])>\util[\buyer] (\consumption[\buyer][][][*] (\state); \type[\buyer])$. Thus,
%     \begin{align}
%         &\qfunc[\buyer][{\policy[][][*]}](\state, \consumption[\buyer][][][\prime], \consumption[-\buyer][][][*](\state), \portfolio[][][][*](\state), \price[][][*](\state), \assetprice[][][*](\state))\\
%         &= \newreward[\buyer](\state, \consumption[\buyer][][][\prime], \consumption[-\buyer][][][*](\state), \portfolio[][][][*](\state), \price[][][*](\state), \assetprice[][][*](\state)) + \Ex_{\staterv[][][\prime] \sim \trans(\staterv[][][\prime] \mid \state, \portfolio[][][][*](\state))} [\discount \vfunc[\buyer][{\policy[][][*]}](\staterv[][][\prime])]\\
%         &=\util[\buyer](\consumption[\buyer][][][\prime];\type[\buyer]) + \Ex_{\staterv[][][\prime] \sim \trans(\staterv[][][\prime] \mid \state, \portfolio[][][][*](\state))} [\discount \vfunc[\buyer][{\policy[][][*]}](\staterv[][][\prime])]\\
%         &> \util[\buyer](\consumption[\buyer][][][*](\state);\type[\buyer]) + \Ex_{\staterv[][][\prime] \sim \trans(\staterv[][][\prime] \mid \state, \portfolio[][][][*](\state))} [\discount \vfunc[\buyer][{\policy[][][*]}](\staterv[][][\prime])]\\
%         &=\qfunc[\buyer][{\policy[][][*]}](\state, \consumption[][][][*](\state), \portfolio[][][][*](\state), \price[][][*](\state), \assetprice[][][*](\state))
%     \end{align}
%     This contradicts that fact that $\policy[][][*]$ is a \MPGNE{} since optimal policy supposed to be greedy respect to optimal action value function. Thus, we know that for all $\buyer\in \buyers$, $\state\in \states$, $\consumption[\buyer][][][*] (\state) \cdot\price[][][*] (\state)+ \portfolio[\buyer][][][*] (\state) \cdot \assetprice[][][*] (\state)- \consendow[\buyer] \cdot \price[][][*] (\state)=0$. Summing across the buyers, we get $\price[][][*](\state) \cdot \left( \sum_{\buyer \in \buyers} \consumption[\buyer][][][*] (\state) - \sum_{\buyer \in \buyers} \consendow[\buyer] \right)  +  \assetprice[][][*] (\state) \cdot \left(\sum_{\buyer \in \buyers} \portfolio[\buyer][][][*] (\state) \right)=0 $ for any $\state\in \states$, which is the Walras' law.
    

%     Finally, we want to show that $(\consumption[][][][*], \portfolio[][][][*], \price[][][*], \assetprice[][][*])$ is feasible. We first show that $\sum_{\buyer \in \buyers} \consumption[\buyer][][][*] (\state) - \sum_{\buyer \in \buyers} \consendow[\buyer][][] \leq \zeros[\numcommods]$ for any $\state\in \states$. 
%     We proved that for any state $\state\in \states$, $\newreward[\numbuyers+1](\state, \consumption[][][][*](\state), \portfolio[][][][*](\state), \price[][][*](\state), \assetprice[][][*](\state))=\price[][][*](\state) \cdot \left( \sum_{\buyer \in \buyers} \consumption[\buyer][][][*] (\state) - \sum_{\buyer \in \buyers} \consendow[\buyer] \right)  +  \assetprice[][][*] (\state) \cdot \left(\sum_{\buyer \in \buyers} \portfolio[\buyer][][][*] (\state) \right)=0$, which implies $\vfunc[\numbuyers+1][{\policy[][][*]}](\state)=0$. For any $\good\in \goods$, consider a $\price:\states\to \pricespace[\numcommods]$ defined by $\price(\state)=\bm{i}_\good$ \footnote{$\bm e_\good \in \R^{\numgoods}$ is the vector in which every component is 0, except the $\good$th, which is 1} for all $\state\in \states$ and a $\assetprice: \state\to \assetpricespace$ defined by $\assetprice(\state)=\zeros[\numassets]$ for all $\state\in \states$. Then, we know that
%     \begin{align}
%         &\qfunc[\numbuyers+1][{\policy[][][*]}](\state, \consumption[][][][*](\state), \portfolio[][][][*](\state), \price(\state), \assetprice(\state))\\
%         &=\newreward[\numbuyers+1](\state,  \consumption[][][][*](\state), \portfolio[][][][*](\state), \price(\state), \assetprice(\state)) + \Ex_{\staterv[][][\prime] \sim \trans(\staterv[][][\prime] \mid \state, \portfolio[][][][*](\state))} [\discount \vfunc[\buyer][{\policy[][][*]}](\staterv[][][\prime])]\\
%         &= \bm{i}_{\good}\cdot \left( \sum_{\buyer \in \buyers} \consumption[\buyer][][][*] (\state) - \sum_{\buyer \in \buyers} \consendow[\buyer] \right)\\
%         &= \sum_{\buyer \in \buyers} \consumption[\buyer][\good][][*] (\state) - \sum_{\buyer \in \buyers} \consendow[\buyer][\good]\\
%         &\leq \qfunc[\numbuyers+1][{\policy[][][*]}](\state, \consumption[][][][*](\state), \portfolio[][][][*](\state), \price[][][*](\state), \assetprice[][][*](\state))=\vfunc[\numbuyers+1][{\policy[][][*]}]=0
%     \end{align}
% Thus, we know that $\sum_{\buyer \in \buyers} \consumption[\buyer][][][*] (\state) - \sum_{\buyer \in \buyers} \consendow[\buyer][][] \leq \zeros[\numcommods]$ for any $\state\in \states$. 
% Finally, we show that $\sum_{\buyer\in \buyers}\portfolio[\buyer][][][*](\state)\leq \zeros[\numassets]$ for all $\state\in\states$. By way of contradiction, assume that $\sum_{\buyer\in \buyers}\portfolio[\buyer][][][*](\state)> \zeros[\numassets]$ for some state in $\states$. Then, the auctioneer can unlimitedly increase their cumulative payoff by increasing $\assetprice(\state)$, and this contradicts the (no arbitrage) condition of \Cref{assum:existence_RRE}. \sadie{Need to refine this part}
    
% Therefore, we can conclude that $\policy[][][*]=(\consumption[][][][*], \portfolio[][][][*], \price[][][*], \assetprice[][][*]): \states \to \consumptions \times \portfoliospace \times \pricespace[\numgoods] \times \assetpricespace $ is a RRE of $\economy$.
% \end{proof}

% By combining \Cref{thm:existence_of_mpgne} with \Cref{thm:rre=mpgne}, we establish the existence of \MPGNE{} in the generalized Markov exchange economy game, and thus the existence of RRE.


Our existence proof reformulates the set of recursive Radner equilibria of any infinite horizon Markov exchange economy as the set of \MPGNE{} of the exchange economy Markov pseudo-game.

\begin{restatable}{theorem}{thmexistRRE}
\label{thm:existence_RRE}
    Consider an infinite horizon Markov exchange economy $\economy$. 
    Under \Cref{assum:existence_RRE}, the set of recursive Radner equilibria of $\economy$ is equal to the set of \MPGNE{} of the associated exchange economy Markov pseudo-game $\mgame$.
\end{restatable}
    
\begin{restatable}{corollary}{corexistRRE}
\label{cor:existence_RRE}
    Under \Cref{assum:existence_RRE}, the set of recursive Radner equilibria of an infinite horizon Markov exchange economy is non-empty.
\end{restatable}


% Similarly, by combining \Cref{thm:convergence_GNE} with \Cref{thm:rre=mpgne}, we establish \samy{the computational results}{polynomial-time computability of} SRE and RRE \amy{RRE? i mean, not really... it's fine to say this, but let's explain in more detail what we mean by it. re: Lemmas 3+4.}\sadie{I don't think we can say polynomial-time computability of SRE and RRE since they are technically PPAD-hard.}\amy{ok -- but what do you mean by ``the computational results of SRE and RRE?''} in infinite horizon Markov exchange economies.
% \sadie{Hmmm I think I need some help to brainstorm the language here. Maybe I should discuss with Deni after his thesis defense. }


\subsection{Equilibrium Computation}
\label{sec:computation}
%In this section, we define exploitability and state exploitability, which measure the distance of any candidate equilibrium from a Radner equilibrium and a recursive Radner equilibrium, respectively.

% \sadie{The main issue is that the min-max problem is not only non-convex-non-concave, but also the exploitability may not be differentiable. The stationary points of the Moreau envelope correspond to stationary points of the subgradient of exploitability, but as exploitability is not neccessarily differentiable, the Moreau envelope is used to measure distance to a stationary point, see \cite{lin2020gradient}.}

Since a recursive Radner equilibrium 
%\sdeni{requires minimizing state exploitability for all states simultaneously, which is an impossible task assuming a continuous state space,}
is infinite-dimensional when the state space is continuous, its computation is FNP-hard \cite{murty1987some}. 
As such, it is generally believed that the best we can hope to find in polynomial time is an outcome that satisfies the necessary conditions of a stationary point of a recursive Radner equilibrium. 
Since the set of recursive Radner equilibria of any infinite horizon Markov exchange economy is equal to the set of \MPGNE{} of the associated exchange economy Markov pseudo-game (\Cref{thm:existence_RRE}), running \Cref{alg:two_time_sgda} on this exchange economy Markov pseudo-game will allow us to compute a policy profile that satisfies the necessary conditions of a stationary point of an \MPGNE, and hence a recursive Radner equilibrium.

Combining \Cref{thm:existence_RRE} and \Cref{thm:convergence_GNE}, we thus obtain the following computational complexity guarantees for \Cref{alg:two_time_sgda}, when run on the exchange economy Markov pseudo-game associated with an infinite horizon Markov exchange economy.\footnote{While for generality and ease of exposition we state Assumptions~\ref{assum:param_lipschitz} and \ref{assum:param_gradient_dominance} for the exchange economy Markov pseudo-game $\mgame$, we note that when the infinite horizon Markov exchange economy $\economy$ satisfies \Cref{assum:existence_RRE}, to ensure that the associated exchange economy Markov pseudo-game $\mgame$ satisfies \Cref{assum:param_lipschitz} and \ref{assum:param_gradient_dominance}, 
it suffices to assume that the parametric policy functions $(\policy, \depolicy)$ are affine; 
the policy parameter spaces $(\params, \deparams)$ are non-empty, compact, and convex; 
for all players $\player \in \players$ and types $\type[\buyer] \in \typespace[\buyer]$, the utility function $\consumption[\buyer] \mapsto\util[\buyer] (\consumption[\buyer]; \type[\buyer])$ is twice continuously differentiable;
and for all $\state, \state[][\prime] \in \states$, the transition function $\portfolio \mapsto\trans(\state[][][\prime] \mid \state, \portfolio)$ is twice continuously differentiable.}

\if 0
\sdeni{our goal is to find an approximate recursive Radner equilibrium by minimizing \mydef{exploitability}, a measure of distance from equilibrium (see \Cref{sec:gmg}). 
However, exploitability is non-convex and non-differentiable in general, so we turn our attention to computing a 
%suitable 
local surrogate for a global minimum, taking non-differentiability into account.
This surrogate is a \mydef{stationary point} of the \mydef{Moreau envelope} of the exploitability.
This notion of stationarity was first proposed by \citet{davis2019stochastic}, and then widely adopted as a tool in the weakly-convex optimization literature.

We now present an algorithm that converges in best iterates to an approximate stationary point of the Moreau envelope of the exploitability.
This result can be extended to stationary points of the state exploitability. 
We refer the interested reader to \Cref{sec:gmg} for additional background. \amy{additional background? or additional results?}. 
\amy{BUT WAIT?! we DO define state exploitability. we only do not define the Moreau envelope of the state exploitability. can we not just say it is defined analogously? and then, wouldn't we have room to include both theorems? offhand, i don't see why not.} \deni{Discuss!}
}{}
\fi

%\deni{I commented a big chunk below, there is not need to reintroduce definition of exploitability/Moreau for the economy is we introduce them in the pseudo-game section and we have also introduced the economy pseudo-game (which I now added to the prior subsection.}

\if 0
First, for notational convenience, 
% given any $\state\in \states$, $\consumption\in \consumptions$, $\portfolio\in \portfoliospace$,
% $\price\in \pricespace[\numgoods]$, and $\assetprice\in \assetpricespace$,
% we define $\reward[\numbuyers+1](\state, \consumption, \portfolio, \price, \assetprice)\doteq \price\cdot \left(\sum_{\buyer\in \buyers}\consumption[\buyer]-\sum_{\buyer\in \buyers}\consendow[\buyer] \right) + \assetprice\cdot \left(\sum_{\buyer\in \buyers}\portfolio[\buyer] \right)$. \textcolor{red}{Do we have a name for this term?}
% Moreoer,
given an outcome $(\consumption, \portfolio, \price, \assetprice)$,
we define $\vfunc[\numbuyers+1][{(\allocation, \portfolio, \price, \assetprice)}]:\states\to \R$
as the cumulative expected profit of a fictional auctioneer who buys all the consumers' endowments and resells them to other consumers starting at any state $\state \in \states$, i.e.,  $\vfunc[\numbuyers+1][{(\allocation, \portfolio, \price, \assetprice)}] (\state) 
\doteq$\\ $\Ex\limits_{\histrv \sim \histdistrib[\initstates][{( \allocation, \portfolio, \price, \assetprice)}]}$ 
$\bigg[
\sum_{\numhorizon = 0}^\infty \discount^\numhorizon 
\bigg(
\price(\histrv[:\numhorizon])\cdot \left(\sum_{\buyer\in \buyers}\consumption[\buyer](\histrv[:\numhorizon])
-\sum_{\buyer\in \buyers}\consendowrv[\buyer][][\numhorizon] \right) 
+ \assetprice(\histrv[:\numhorizon])\cdot \left(\sum_{\buyer\in \buyers}\portfolio[\buyer](\histrv[:\numhorizon]) \right)
\bigg)
\mid \staterv[0] = \state 
\bigg]$. 

\noindent
Similarly, we define the expected cumulative profit over the initial state distribution $\cumulutil[\numbuyers+1](\consumption, \portfolio, \price, \assetprice)\doteq \Ex_{\state\sim \initstates}\left[\vfunc[\numbuyers+1][{(\allocation, \portfolio, \price, \assetprice)}](\state) \right]$.

Next, given an infinite horizon Markov exchange economy $\economy$, 
% \if 0
we define the \mydef{state exploitability} of a Markov outcome $(\consumption[][][][], \portfolio[][][][], \price, \assetprice): \states \to \consumptions\times \portfoliospace \times \pricespace[\numgoods]\times \assetpricespace$ at state $\state\in \states$ as
%
\begin{align*}
    &\sexploit (\state, (\consumption[][][][], \portfolio[][][][], \price, \assetprice)) \doteq \sum_{\player \in \players} \nonumber \\
    &\max_{\substack{(\allocation[\buyer][][][\prime], \portfolio[\buyer][][][\prime]): 
    \states \to \consumptions[\buyer] \times \portfoliospace[\buyer]: \forall \state \in \states, \\(\allocation[\buyer][][][\prime], \portfolio[\buyer][][][\prime])(\state) \in \budgetset[\buyer] (\consendow[\buyer], \price[][][] (\state), \assetprice[][][] (\state))
    }}  
    \vfunc[\buyer][{(\allocation[\buyer][][][\prime], \allocation[-\buyer][][][], 
    \portfolio[\buyer][][][\prime], \portfolio[-\buyer][][][], \price[][][], \assetprice[][][])}] (\state)  \nonumber
    - \vfunc[\buyer][{(\allocation[][][][], 
    \portfolio[][][][], \price[][][], \assetprice[][][])}] (\state) \nonumber\\
    &+ \max_{\price[][][\prime], \assetprice[][][\prime]:\states\to \pricespace[\numgoods] \times \assetpricespace
    }
    \vfunc[\numbuyers+1][{(\allocation[][][][],
    \portfolio[][][][],\price[][][\prime], \assetprice[][][\prime])}] (\state) 
   -\vfunc[\numbuyers+1][{(\allocation[][][][],
    \portfolio[][][][],\price[][][], \assetprice[][][])}] (\state) 
\end{align*}

\noindent
Finally, 
% \fi
we define the \mydef{exploitability} of an outcome $(\consumption, \portfolio, \price, \assetprice): \hists \to \consumptions\times \portfoliospace \times \pricespace[\numgoods]\times \assetpricespace$ as
\begin{align*}
    &\exploit (\consumption[][][][], \portfolio[][][][], \price, \assetprice) \nonumber\\
    &\doteq \sum_{\player \in \players}
    \max_{\substack{(\allocation[\buyer][][][\prime], \portfolio[\buyer][][][\prime]): 
    \hists \to \consumptions[\buyer] \times \portfoliospace[\buyer]: \forall \hist \in \hists, \\(\allocation[\buyer][][][\prime], \portfolio[\buyer][][][\prime])(\hist) \in \budgetset[\buyer] (\consendow[\buyer], \price[][][] (\hist), \assetprice[][][] (\hist))
    }}  
    \cumulutil[\buyer](\allocation[\buyer][][][\prime], \allocation[-\buyer][][][], 
    \portfolio[\buyer][][][\prime], \portfolio[-\buyer][][][], \price[][][], \assetprice[][][])   \nonumber\\
    &- \cumulutil[\buyer](\allocation[][][][], 
    \portfolio[][][][],\price[][][], \assetprice[][][])  \nonumber\\
    &+ \max_{\price[][][\prime], \assetprice[][][\prime]:\states\to \pricespace[\numgoods] \times \assetpricespace
    }
    \cumulutil[\numbuyers+1](\allocation[][][][],
    \portfolio[][][][],\price[][][\prime], \assetprice[][][\prime]) 
   -\cumulutil[\numbuyers+1](\allocation[][][][],
    \portfolio[][][][],\price[][][], \assetprice[][][]) 
\end{align*}



In the next lemma, we show that exploitability and state exploitability can be viewed as measures of distance to RE and RRE, respectively. 
Consequently, we can approximate RE and RRE by minimizing exploitability and state exploitability, respectively.

\begin{restatable}{lemma}{lemmaexploitRRE}
\label{lemma:no_exploit_RRE}
    Given an infinite horizon exchange economy $\economy$, an outcome $(\consumption, \portfolio, \price, \assetprice): \hists \to \consumptions \times \portfoliospace \times \pricespace[\numgoods] \times \assetpricespace$ is a RE iff $\exploit (\consumption, \portfolio, \price, \assetprice) = 0$.
    Similarly, a Markov outcome $(\consumption, \portfolio, \price, \assetprice): \states \to \consumptions\times \portfoliospace \times \pricespace[\numgoods]\times \assetpricespace$ is a RRE iff $\sexploit (\state, (\consumption, \portfolio, \price, \assetprice)) = 0$, for all $\state\in \states$.
\end{restatable}

As we show in \Cref{sec:gmg}, exploitability minimization can then be restated as the following coupled min-max optimization problem:
%
\begin{align}
&\min_{(\consumption, \portfolio, \price, \assetprice): \hists \to \consumptions \times \portfoliospace \times \pricespace[\numgoods] \times \assetpricespace}
\max_{\substack{(\consumption[][][][\prime], \portfolio[][][][\prime], \price[][][\prime], \assetprice[][][\prime]): 
\hists \to \consumptions\times \portfoliospace \times \pricespace[\numgoods]\times \assetpricespace\\
(\allocation[\buyer][][][\prime], \portfolio[\buyer][][][\prime])(\hist) \in \budgetset[\buyer] (\consendow[\buyer], \price[][][] (\hist), \assetprice[][][] (\hist))\;\forall\buyer\in \buyers, \hist\in \hists}
} \notag\\
&\gcumulreg((\consumption, \portfolio, \price, \assetprice), (\consumption[][][][\prime], \portfolio[][][][\prime], \price[][][\prime], \assetprice[][][\prime]))\nonumber
    \\
   & \doteq \sum_{\player \in \players}
    \cumulutil[\buyer](\allocation[\buyer][][][\prime], \allocation[-\buyer][][][], 
    \portfolio[\buyer][][][\prime], \portfolio[-\buyer][][][], \price[][][], \assetprice[][][])  
    - \cumulutil[\buyer](\allocation[][][][], 
    \portfolio[][][][],\price[][][], \assetprice[][][])  \nonumber\\
    &+ 
    \cumulutil[\numbuyers+1](\allocation[][][][],
    \portfolio[][][][],\price[][][\prime], \assetprice[][][\prime]) 
   -\cumulutil[\numbuyers+1](\allocation[][][][],
    \portfolio[][][][],\price[][][], \assetprice[][][]) \label{eq:min_max_rre}
\end{align}

\noindent
\samy{T}{If exploitability were differentiable, t}his 
%coupled min-max optimization
problem \samy{can}{could} be solved by two-time scale simultaneous gradient descent ascent (TTSSGDA), which takes a step of gradient descent on $(\consumption, \portfolio, \price, \assetprice)$ and a \sdeni{\samy{}{faster}}{larger} step of gradient ascent on $(\consumption[][][][\prime], \portfolio[][][][\prime], \price[][][\prime], \assetprice[][][\prime])$, simultaneously (see, for instance, \Cref{alg:two_time_sgda} \cite{daskalakis2020independent}).
\amy{i think we want to add that all we can guarantee is convergence in best iterates. this makes the exp
'ts much more interesting!}
\sadie{The problem is that, even if the exploitability is differentiable, if it is not convex, what we can converge to is still a stationary point instead of a true solution. So maybe we can say that "if exploitability is convex and differentiable"}
%\samy{}{Note, however, that as policies are functions, in the complete theory provided in \Cref{sec:gmg}, we assume policies are represented by finitely many parameters.}
% In \Cref{sec:gmg_conv}, we represent any recursive Radner equilibrium as a parametric function (e.g., a neural network) and introduce \Cref{alg:two_time_sgda} to minimize state-exploitability over the parameter space.
% \amy{Rewrite intro paragraph first, and then this one:}
% As \samy{the computation of an RRE is in general intractable (see, for instance, \cite{chen2006settling}), and the}{} state-exploitability is non-convex, instead of seeking a minimum of this function, we relax our goal to computing a stationary point of state-exploitability, which we show in the following theorem can be done efficiently.
% \sadie{The statement in this paragraph is largely incorrect so I rewrite anything below.}

\amy{insert explanation again (sorry if i deleted something here?) about how the Moreau env. addresses the differentiability issue.}


\ssadie{}{However, as exploitability is non-convex and not differentiable in general. Therefore, we relax our goal to computing an alternative to exploitability, namely}
the \mydef{Moreau envelope} of the exploitability, defined as:
\begin{align}
 \regexploit(\consumption, \portfolio, \price, \assetprice)
= &\min_{\substack{(\consumption[][][][\prime], \portfolio[][][][\prime], \price[][][\prime], \assetprice[][][\prime]): \\
\states \to \consumptions\times \portfoliospace \times \pricespace[\numgoods]\times \assetpricespace} }  
\bigg\{
\exploit(\consumption[][][][\prime], \portfolio[][][][\prime], \price[][][\prime], \assetprice[][][\prime])
\nonumber \\
&+
\lipschitz[{\grad \gcumulreg}]\left\| (\consumption, \portfolio, \price, \assetprice) - (\consumption[][][][\prime], \portfolio[][][][\prime], \price[][][\prime], \assetprice[][][\prime])\right\|^2
\bigg\}
\end{align}

% where 
% \begin{align}
%     &\gcumulreg((\consumption, \portfolio, \price, \assetprice), (\consumption[][][][\prime], \portfolio[][][][\prime], \price[][][\prime], \assetprice[][][\prime]))\nonumber
%     \\
%     &= \sum_{\player \in \players}
%     \cumulutil[\buyer](\allocation[\buyer][][][\prime], \allocation[-\buyer][][][], 
%     \portfolio[\buyer][][][\prime], \portfolio[-\buyer][][][], \price[][][], \assetprice[][][])  
%     - \cumulutil[\buyer](\allocation[][][][], 
%     \portfolio[][][][],\price[][][], \assetprice[][][])  \nonumber\\
%     &+ 
%     \cumulutil[\numbuyers+1](\allocation[][][][],
%     \portfolio[][][][],\price[][][\prime], \assetprice[][][\prime]) 
%    -\cumulutil[\numbuyers+1](\allocation[][][][],
%     \portfolio[][][][],\price[][][], \assetprice[][][]) 
% \end{align}

\noindent
Moreover, we say that $(\consumption, \portfolio, \price, \assetprice)$ is an $\varepsilon$\mydef{-stationary point} of $\regexploit$ if $\|\grad \regexploit(\consumption, \portfolio, \price, \assetprice)\|\leq \varepsilon$.
\amy{we are missing the punchline here! we need to say that this $\epsilon$-stationary point is a Radner equilibrium!!!}\sadie{The problem is that it is not really the Radner equilibrium... We got a Radner equilibrium only if $\exploit=0$.}
%If $\varepsilon=0$, $(\consumption, \portfolio, \price, \assetprice)$ is a stationary point of the Moreau envelope of the exploitability. 
\fi 

%\begin{restatable}{definition}{moreauenvelope}
%     Given any function $f:\R^m\to \R$, its Moreau envelope $\Tilde{f}: \R^m \to \R$ is defined as
%     $\Tilde{f}(\x)=\min_{\x'\in \R^m} f(\x')+\frac{1}{2}$
% \end{restatable}
% In \Cref{sec:gmg_conv}, we represent any Markov outcome as a parametric function (e.g., a neural network), 
% and then adjusts those parameters to minimize the Moreau envelope of the exploitability based on its gradient. The algorithm we use is two time-scale stochastic gradient descent-ascent \cite{daskalakis2020independent}, for which we prove polynomial-time
% convergence in best iterates.


\begin{restatable}{theorem}{thmcomputeSRE}
\label{thm:compute_SRE}
    Consider an infinite horizon Markov exchange economy $\economy$ and the associated exchange economy Markov pseudo-games $\mgame$. 
    Let $(\policy, \depolicy, \params, \deparams)$ be a parametrization scheme for $\mgame$ and
    suppose Assumptions~\ref{assum:param_lipschitz}, \ref{assum:param_gradient_dominance}, and \ref{assum:existence_RRE} hold.
    Then, the convergence results in \Cref{thm:convergence_GNE} hold for $\mgame$.
    %let $\exploit$ and $\cumulregret$ be the exploitability and cumulative regret associated with the the exchange economy Markov pseudo-game $\mgame$.
\end{restatable}

\if 0
\amy{DELETE!!! the rest of this. just say the results of the theorem above hold for $\mgame$.}    
\deni{Then, for any $\delta > 0$, let $\varepsilon = \delta \|\brmismatch (\cdot, \initstates, \cdot)\|_\infty^{-1}$.
    If \Cref{alg:two_time_sgda} is run with inputs that satisfy,  
$\learnrate[\param][ ], \learnrate[\deparam][ ] \asymp  \poly(\varepsilon, \| \nicefrac{\partial \statedist[\initstates][{\policy[][][*]}]}{\partial \initstates} \|_\infty, \frac{1}{1-\discount}, \lipschitz[\grad \gcumulreg]^{-1}, \varconst[\deparam]^{-1}, \lipschitz[\gcumulreg]^{-1})$, then for some $\numiters \in \poly \left(\varepsilon^{-1}, (1-\discount)^{-1}, \| \nicefrac{\partial \statedist[\initstates][{\policy[][][*]}]}{\partial \initstates} \|_\infty, \lipschitz[{\grad \gcumulreg}], \lipschitz[\gcumulreg], \diam (\params \times \deparams), \learnrate[\param][ ]^{-1}\right)$, the best-iterate policy parameter $\bestiter[{\param}][\numiters] \in \argmin_{\numhorizon = 0, 1, \cdots, \numiters} \left\| \grad \regulexploit (\param[][][(\numhorizon)]) \right\|$ is a stationary point of the Moreau exploitability, i.e., $\| \grad \regulexploit (\bestiter[{\param}][\numiters])\| \leq \varepsilon$. 
Further, $\| \grad[\param] \regulexploit (\diststates, \bestiter[{\param}][\numiters]) \| \leq \delta$, for any arbitrary state distribution $\diststates \in \simplex(\states)$.}

    \if 0
     Then, for all $\varepsilon \geq 0$, an $\varepsilon$-stationary point of the Moreau envelope of the exploitability can be computed \amy{add best iterates somewhere?} in $O(\nicefrac{1}{\poly(\varepsilon)})$ operations.%
    \footnote{\sdeni{}{Once again, this result can be extended to stationary points of the state exploitability. We refer the interested reader to \Cref{thm:convergence_GNE} in \Cref{sec:gmg}.}}
    \fi 

\amy{!!!!! DELETE REMARK FOR SPACE !!!!!}
\fi 
% \begin{remark}

% \end{remark}


\if 0
\amy{here's some math that could be helpful:}
\samy{}{then the best-iterate policies $\bestiter[{\param}][\numiters] \in \argmin_{\numhorizon = 0, 1, \cdots, \numiters} \left\| \grad \regulexploit (\param[][][(\numhorizon)]) \right\|$ converge to a stationary point of the Moreau envelope of the exploitability, i.e., $\left\| \grad \regulexploit (\bestiter[{\param}][\numiters]) \right\| \leq \varepsilon$.} 

\amy{more fodder:}
\samy{}{Additionally, there exists $\param[][][*] \in \params$ s.t.\@ $\| \grad[\param] \sexploit (\diststates, \param) \| \leq \delta$, \amy{should this gradient be evaluated at $\param[][][*]$?} for any arbitrary \amy{initial???} state distribution $\diststates\in \Delta(\states)$.}

\amy{Note: $\sexploit (\diststates, \param)$ is undefined here, and in the appendix as well.}
\fi

% \begin{restatable}{corollary}{corollaryconvergerre}
% Consider the generalized Markov exchange economy game $\mgame$ associated with any infinite horizon Markov exchange economy $\economy$ that satisfies \Cref{assum:existence_RRE}.
% If the game's policy classes are parameterized such that they satisfy Assumptions~\ref{assum:param_lipschitz} and \ref{assum:param_gradient_dominance}, then \Cref{alg:two_time_sgda} converges to the stationary point of the Moreau envelope of the exploitability function of $\mgame$.


%\amy{but is exploitability defined for games? i thought it was defined for policies? so i'm not sure this makes sense.} \sadie{The exploitability function is defined as a function of policies, but when we say the stationary point of exploitability, we are talking about the stationary point of the function. But to be more precisely, maybe we should say "the Moreau envelope of the exploitability function defined by $\mgame$"?} in polynomial time.
% to an outcome that is close enough to a first-order RRE\footnote{Here, the first-order RRE corresponds to the first-order \MPGNE, implying that distance to a RRE can no more be minimized via first order deviations} in polynomial time.}



% if the \Cref{alg:two_time_sgda} is running on min-max optimization problem induced by $\mgame$
