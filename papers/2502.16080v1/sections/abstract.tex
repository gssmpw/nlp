
\begin{abstract} 
\deni{We need to change to "infinite horizon exchange economy Markov pseudo-game" in the appendix or whatever the terminology is. Also ask Sadie for deanonymized code repos.}
In this paper, we study a generalization of Markov games and pseudo-games that we call Markov pseudo-games, which like the former, captures time and uncertainty, and like the latter, allows for the players' actions to determine the set of actions available to the other players.
In the same vein as \citeauthor{arrow-debreu}, we intend for this model to be rich enough to encapsulate a broad mathematical framework for modeling economies.
We then prove the existence of a game-theoretic equilibrium in our model, which in turn implies the existence of a general equilibrium in the corresponding economies.
Finally, going beyond \citeauthor{arrow-debreu}, we introduce a solution method for Markov pseudo-games, and prove its polynomial-time convergence.
%and apply it to solve for general equilibria in infinite-horizon Markov exchange economies.

%We establish the existence of pure generalized Markov perfect equilibria (pure \MPGNE) in concave Markov pseudo-games. While computing such equilibria is generally PPAD-hard, we introduce a first-order solution method \samy{}{to find a stationary point of the exploitability that defines \MPGNE}, and prove its polynomial-time convergence.

We then provide an application of Markov pseudo-games to infinite-horizon Markov exchange economies, a stochastic economic model that extends \citeauthor{radner1972existence}'s stochastic exchange economy 
%\cite{radner1972existence} 
and \citeauthor{magill1994infinite}'s infinite horizon incomplete markets model.
%\cite{magill1994infinite}. 
We show that under suitable assumptions, the solutions of any infinite horizon Markov exchange economy (i.e., recursive Radner equilibria---RRE) can be formulated as the solution to a concave Markov pseudo-game, thus establishing the existence of RRE, and providing first-order methods for approximating RRE. 
Finally, we demonstrate the effectiveness of our approach in practice by building the corresponding generative adversarial policy neural network, and using it to compute RRE in a variety of infinite-horizon Markov exchange economies.
\end{abstract}

\if 0
    We introduce the infinite horizon Markov exchange economy, a general equilibrium model which explicitly incorporates time \& uncertainty, and generalizes the \citeauthor{radner1972existence} stochastic exchange economy \cite{radner1972existence} and \citeauthor{magill1994infinite}'s infinite horizon incomplete markets model \cite{magill1994infinite}. 
    We establish the existence of a recursive Radner (or competitive) equilibrium under standard economic assumptions on consumer preferences and endowments, as well as the structure of assets in the economy. As recursive Radner equilibria (RRE) are in general infinite dimensional, their computation is FNP-hard. As such, we turn our attention to their approximation via function approximation (i.e., deep learning). In particular, we introduce generative adversarial policy networks (GAPNets): a family of generative adversarial  networks that can learn RRE from samples of realized histories of the economy. We introduce a deep reinforcement learning method to train GAPNets, which we show converges to an approximate stationary point of the (regularized) exploitability (i.e., a measure of distance from a RRE). Finally, we demonstrate that GAPNets can more accurately represent a RRE than the state-of-the-art method for computing a RRE.
\fi
