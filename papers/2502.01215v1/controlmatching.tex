%

%

%
\documentclass{article}
\makeatletter
\providecommand{\@LN}[2]{}
\providecommand{\@LN@col}[1]{}
\makeatother
\pdfpagewidth=8.5in
\pdfpageheight=11in

%
%
\usepackage{ijcai25}
\usepackage{natbib}

%
\usepackage{times}
\usepackage{soul}
\usepackage{url}
%
\usepackage[utf8]{inputenc}
\usepackage[small]{caption}
\usepackage{graphicx}
\usepackage{amsmath}
\usepackage{amsthm}
\usepackage{amssymb}
\usepackage{paralist}
\usepackage{booktabs}
\usepackage{mathtools}



%
\usepackage{hyperref}
\hypersetup{%
%
  backref=true,
  pagebackref=true, 
  hypertexnames=true,
  urlcolor=blue,
  colorlinks=true,citecolor=green!60!black,linkcolor=red!60!black%
}

%
%

\usepackage{algorithm}
\usepackage{algorithmic}
\usepackage[switch]{lineno}
\usepackage{MnSymbol}	%

\usepackage{tabularx}
\usepackage{multirow}
\usepackage{colortbl}
\usepackage{tikz}
\usetikzlibrary{matrix,shapes,arrows,calc,arrows,fit}
\usepackage{rotating}
\usepackage{pifont}

\usepackage{mdframed}

\usepackage[textsize=tiny,textwidth=1.5cm,linecolor=green!70!black, backgroundcolor=green!10, bordercolor=black,disable]{todonotes}
%
 \newcommand{\todoH}[1]{\todo[linecolor=green!70!black, backgroundcolor=green!10]{H: #1}}
%
%
%
%
%
\newcommand{\todoHinline}[1]{}

%
%

\urlstyle{same}

%
%

\newcommand{\str}{\mathrm{str}}

%
\newcommand{\p}{\ensuremath{\mathrm{P}}}
\newcommand{\np}{\ensuremath{\mathrm{NP}}}
\newcommand{\npc}{\ensuremath{\mathrm{NPC}}}
\newcommand{\pp}{\ensuremath{\mathrm{PP}}}
\newcommand{\conp}{\ensuremath{\mathrm{coNP}}}
\newcommand{\classdp}{\ensuremath{\mathrm{DP}}}

\newcommand{\CSR}{\textsc{CSR}}
\newcommand{\CSM}{\textsc{CSM}}

%
%
%
%
\newtheorem{theorem}{Theorem}
\newtheorem{challenge}{Challenge}
\newtheorem{proposition}{Proposition}
\newtheorem{corollary}{Corollary}
\newtheorem{claim}{Claim}

\newcommand{\decprob}[3]{
  \begin{center}%
    \begin{minipage}{0.9\linewidth}%
      \textsc{#1}\\[0.2ex]
      \textbf{Input:} #2\\[0.2ex]
      \textbf{Question:} #3
    \end{minipage}%
  \end{center} }

%
%

\newenvironment{claimproof}[1]{\par\noindent\textit{Proof.}\space#1}{\hfill $\lhd$\bigskip}

%
\theoremstyle{definition}
\newtheorem{example}{Example}
\AtBeginEnvironment{example}{%
  \pushQED{\qed}\renewcommand{\qedsymbol}{$\righthalfcup$}%
%
%
}
\AtEndEnvironment{example}{\popQED\endexample}


%
%
%
%
%
%

%
%
%
%
%

%
%
%
%
%
%


%

%
%
\pdfinfo{
/TemplateVersion (IJCAI.2025.0)
}

\title{Control in {S}table {M}arriage and {S}table {R}oommates: {C}omplexity and Algorithms}


\author{
Jiehua Chen$^1$
 \and
Ildik\'{o} Schlotter$^2$\\
 \affiliations
 $^1$TU Vienna, Vienna, Austria\\
 $^2$HUN-REN Centre for Economic and Regional Studies,  Budapest, Hungary\\
 \emails
 jiehua.chen@ac.tuwien.ac.at,
 schlotterildi@gmail.com
}

%

\newcommand{\myemph}[1]{{\color{purple!50!black}\textit{#1}}}


\begin{document}

\maketitle

\begin{abstract}
  We study control problems in the context of matching under preferences: We examine how a central authority, called the \myemph{controller}, can manipulate an instance of the \textsc{Stable Marriage} or \textsc{Stable Roommates} problems in order to achieve certain goals. We investigate the computational complexity of the emerging problems, and provide both efficient algorithms and intractability results. 
\end{abstract}

\section{Introduction}


\def\CA{CA}
\def\A{\mathcal{A}}
\def\G{\mathcal{G}}
\def\AddAg{\mathsf{AddAg}}
\def\DelAg{\mathsf{DelAg}}
\def\DelAcc{\mathsf{DelAcc}}
\def\CI{\mathsf{CapInc}}
\def\CD{\mathsf{CapDec}}
\def\CM{\mathsf{CapMod}}
\def\MA{\mathsf{MA}}
\def\MP{\mathsf{MP}}
\def\SM{\mathsf{MS}}
\def\USM{\mathsf{USM}}
\def\ESM{\exists\mathsf{SM}}
\def\PSM{\exists\mathsf{PSM}}
\def\POSM{\exists\mathsf{POSM}}
\def\etal{\myemph{et al.}}
\def\SR{\textsc{Stable Roommates}}

%
%
%



\SR\ deals with the division of human or non-human agents into pairs,
where  agents have preference lists describing their desires about whom they want to be matched with.
More precisely, an instance of \textsc{Stable Roommates} consists of a set~$U$ of agents,
with each agent~$u \in U$ having a strict preference list over the subset
%
of the remaining agents that $u$ finds \myemph{acceptable}.
%
%
The goal is to decide whether there is a \myemph{stable matching}, that is, a set~$M$ of mutually disjoint pairs of agents for which \myemph{no} blocking pair exists. 
Herein, a \myemph{blocking pair} for~$M$ is a pair of two distinct agents $u$ and $w$ who are not matched together by~$M$ such that
\begin{compactenum}[(i)]
  \item  $u$ is either unmatched by $M$ or prefers $w$ to its partner in~$M$
  %
  and
  \item $w$ is either unmatched by $M$ or prefers $u$ to its partner in~$M$.
  %
\end{compactenum}

Note that each \textsc{Stable Roommates} instance induces an acceptability graph where the vertices correspond to the agents and there is an edge between two vertices if both consider each other acceptable. 
A well-known special case of \textsc{Stable Roommates} called \textsc{Stable Marriage}
is the \myemph{bipartite} variant where  agents are partitioned into two disjoint subsets such that each agent's preference list consists of agents of the opposite subset and can only be assigned a partner from that subset.
The acceptability graph of an instance of \textsc{Stable Marriage} is then bipartite.
Due to a celebrated result by~\citet{gal-sha:j:college-admissions-and-the-stability-of-marriage}, we know that in the marriage setting a stable matching always exists and can be found in time linear in the input size.
However, for the roommates setting, a stable matching is not guaranteed to exist and, more than two decades after the seminal paper by~\citet{gal-sha:j:college-admissions-and-the-stability-of-marriage}, 
\citet{irv:j:efficient-algorithm-for-stable-roommates-problem} provided a polynomial-time algorithm to decide whether a given instance has a stable matching.

\citet{bar-tov-tri:j:manipulating} introduced a framework for considering questions of manipulation and control in the context of elections, and their approach was adopted by
\citet{boe-bre-hee-nie:j:control-bribery-in-SM} to investigate issues of manipulation, control, and bribery in the context of \textsc{Stable Marriage}. 
Among these, we focus on variants of \myemph{control}, which assumes that an external agent called the \myemph{controller} aims to achieve a specific \myemph{goal} by applying as few \myemph{control actions} as possible. 
\citet{boe-bre-hee-nie:j:control-bribery-in-SM} studied three different goals and five actions, yielding 15 different computational problems;
however, in this work we only focus on those that fall under the category of \myemph{control}. In particular, our model assumes that the controller may influence which agents are eligible to participate in the matching, but is not able to change the preferences of the agents. 

The control actions we consider are 
\begin{compactitem}%
\item$\AddAg$: adding agents,
\item $\DelAg$: deleting agents, and 
\item $\DelAcc$: deleting acceptability.
\end{compactitem}
%
Given an instance of \SR, we focus on the following possible control goals:
\begin{compactitem}
\item $\MA$: a given agent is matched in some stable matching, 
\item $\MP$: a given pair is contained in some stable matching,	
\item $\SM$: a given matching becomes stable, 
\item $\ESM$: a stable matching exists, or 
\item $\PSM$: a perfect and stable matching exists.
\end{compactitem}
Note that \citet{boe-bre-hee-nie:j:control-bribery-in-SM} study the goals~$\MP$, $\SM$, and $\USM$,
where $\USM$ aims at making a given matching become the unique stable matching. 
\citet{ber-csa-kir:j:manipulation-SR} considered the goals~$\ESM$ and $\PSM$. 


For each control action~$\mathcal{A} \in \{\AddAg,\DelAg,\DelAcc\}$ and each goal~$\mathcal{G} \in \{\MA, \MP, \SM, \USM, \ESM, \PSM\}$ discussed above, the \textsc{Control for Stable Roommates}-$\A$-$\G$ (abbreviated as \CSR-$\A$-$\G$) problem asks
for a minimum number of control actions necessary to achieve the given goal in a given \SR\ instance.
When restricted to the bipartite marriage setting, we call these problems \textsc{Control for Stable Marriage}-$\A$-$\G$ (or \textsc{CSR}-$\A$-$\G$).


\def\conseq{\textcolor{gray}{\blacklozenge}}
\def\boe{\blacklozenge}
\definecolor{Gray}{gray}{0.85}
\newcommand{\CSMDelAgMA}{{%
{\Pol$^{\conseq}$~[Prop.~\ref{prop:CSM-DelAg-MA}]}}}
\newcommand{\CSMDelAccSM}{{%
{\Pol$^{\conseq}$~[Prop.~\ref{prop:CSM-DelAcc-SM}]}}}
\newcommand{\CSRAddAgExistsSM}{\NPc~[Thm.~\ref{thm:CSR-AddAg-ExistsSM}]}
\newcommand{\CSMAddAgMA}{\NPc~[Thm.~\ref{thm:CSM-AddAg-MA}]}
\newcommand{\CSMAddAgPSM}{\NPc~[Thm.~\ref{thm:CSM-AddAg-MA}]}
\newcommand{\CSRAddAgMA}{\NPc~[Cor.~\ref{cor:CSR-AddAg-MA}]}
\newcommand{\CSRDelAgMP}{\Pol~[Thm.~\ref{thm:CSR-DelAg-MP+MA}]}
\newcommand{\CSRDelAgMA}{\Pol~[Thm.~\ref{thm:CSR-DelAg-MP+MA}]}
\newcommand{\CSRAddAgSM}{\NPc~[Thm.~\ref{thm:CSR-AddAg-SM}]}

\newcommand{\SMDelAgMA}{%
\Pol & [P.~\ref{prop:CSM-DelAg-MA},${\conseq}$]}
\newcommand{\SMDelAccSM}{%
\Pol & [P.~\ref{prop:CSM-DelAcc-SM}, ${\conseq}$]}
\newcommand{\SRAddAgExistsSM}{\NPc & [T.~\ref{thm:CSR-AddAg-ExistsSM}]}
\newcommand{\SMAddAgMA}{\NPc  &[T.~\ref{thm:CSM-AddAg-MA}]}
\newcommand{\SMAddAgPSM}{\NPc & [T.~\ref{thm:CSM-AddAg-MA}]}
\newcommand{\SRAddAgMA}{\NPc & [C.~\ref{cor:CSR-AddAg-MA}]}
\newcommand{\SRDelAgMP}{\Pol  &[T.~\ref{thm:CSR-DelAg-MP+MA}]}
\newcommand{\SRDelAgMA}{\Pol & [T.~\ref{thm:CSR-DelAg-MP+MA}]}
\newcommand{\SRAddAgSM}{\NPc  &[T.~\ref{thm:CSR-AddAg-SM}]}
\newcommand{\boecite}{[$\blacklozenge$]}
\newcommand{\MScite}{[$\heartsuit$]}
%
\newcommand{\trivialwcite}{\multicolumn{2}{c}{\trivial}}
\newcommand{\tancite}{[$\dag$]}
\newcommand{\BMMcite}{[$\spadesuit$]}
\newcommand{\ABMcite}{[$\star$]}

%
\newcolumntype{a}{>{}c} 
\def\trivial{---}
\def\NPc{$\np$c}
\def\Pol{in $\p$}
\def\own{\mathbf{A}}
\begin{table*}[t!]
\caption{Complexity results for \textsc{Control in Stable Marriage (\textup{or} Roommates)}-$\A$-$\G$ problems. 
The problems \textsc{CSM}-$\A$-$\ESM$ are omitted, as a stable matching always exists in an instance of SM. 
Results by \protect\citet{boe-bre-hee-nie:j:control-bribery-in-SM} and easy consequences thereof 
are marked with~\boecite\ and with~$\conseq$, respectively.
Results by \protect\citet{tan:j:maximum-stable-matching,tan:j:stable-partitions} are marked with~\tancite, by \protect\citet{mni-sch:j:SM-covering-constraints} with~\MScite, by \protect\citet{abr-bir-man:c:almost-stable-roommates} with~\ABMcite, and by \protect\citet{bir-man-mit:j:size-vs-stability} with~\BMMcite.
``{\NPc}'' stands for $\np$-complete.}
\label{tab:summary-for-SM}
\centering
%
%
%
%
%
%
%
%
%
%
%
%
\begin{tabular}{@{}r r@{\hspace*{1.5mm}}l c r@{\hspace*{1.5mm}}l c r@{\hspace*{1.5mm}}l c r@{\hspace*{1.5mm}}l c r@{\hspace*{1.5mm}}l c r@{\hspace*{1.5mm}}l @{} }
%
  \toprule
 %
 $\G$ & \multicolumn{2}{@{}c@{}}{\centering CSM-$\AddAg$-$\G$} &  &   \multicolumn{2}{@{}c@{}}{\centering CSR-$\AddAg$-$\G$} &  & \multicolumn{2}{@{}c@{}}{\centering CSM-$\DelAg$-$\G$}  &  &\multicolumn{2}{@{}c@{}}{\centering CSR-$\DelAg$-$\G$}  & &\multicolumn{2}{@{}c@{}}{\centering CSM-$\DelAcc$-$\G$} &   &\multicolumn{2}{@{}c@{}}{\centering CSR-$\DelAcc$-$\G$}\\ \midrule
  $\MA$ & \SMAddAgMA &&  \SRAddAgMA &&  \SMDelAgMA &&  \SRDelAgMA && \quad \NPc & \MScite &&  ~~\NPc & \MScite \\
  $\MP$ & \NPc &  \boecite &&  \NPc & \boecite &&  \Pol & \boecite && \SRDelAgMP && \NPc & \boecite && \NPc & \boecite \\
  $\SM$ &  \Pol &  \boecite &&  \SRAddAgSM && \NPc & \boecite && \NPc & \boecite  &&  \Pol  & \boecite && \SMDelAccSM\\
  $\ESM$ & \trivialwcite &&  \SRAddAgExistsSM && \trivialwcite && \Pol &\tancite && \trivialwcite && \NPc & \ABMcite\\
  $\PSM$ & \SMAddAgPSM &&  \SRAddAgExistsSM &&  \Pol & \tancite && \Pol & \tancite && \NPc & \BMMcite && \NPc & \BMMcite\\
 \bottomrule
\end{tabular}
\end{table*}

%
%
%
%
\paragraph{Our contributions and related work.}
We answer all open questions for both the marriage and roommates settings that arise from the three control actions and five control goals described above.
Table~\ref{tab:summary-for-SM} gives an overview of the computational complexity of all 27 problems we consider. 

Although \citet{boe-bre-hee-nie:j:control-bribery-in-SM} appear to be the first to explicitly and systematically focus on manipulation, control, and bribery in the context of matching under preferences, the literature contains several earlier work where these problems were studied, usually from some other aspect or with different motivations.
\protect\citet{tan:j:maximum-stable-matching,tan:j:stable-partitions} investigated the problem of finding a \myemph{maximum} stable matching in an instance of \SR, which is a matching that is as large as possible while being stable within the set of matched agents. It is straightforward that this problem is equivalent to \CSR-$\DelAg$-$\SM$.


We remark that some of the problems \textsc{CSR}-$\DelAcc$-$\mathcal{G}$ have been studied in  the context of \myemph{almost stable} matchings
%
where we aim to obtain a matching that satisfies certain constraints and has as few blocking pairs as possible. 
Namely, an instance of \textsc{Stable Roommates} admits a matching with at most $k$ blocking pairs if and only if we can ensure the existence of a stable matching by $k$ deletions of acceptability; 
\citet{abr-bir-man:c:almost-stable-roommates}
proved that approximating this problem within~$n^{1/2-\varepsilon}$ for any~$\varepsilon>0$ is $\np$-hard, where $n$ is the number of agents.
\citet{bir-man-mit:j:size-vs-stability}
showed even stronger inapproximability for the problem of finding a maximum-cardinality matching that minimizes the number of blocking pairs; %
%
in fact, their proof holds for the marriage setting where the desired matching is a perfect matching, implying intractability for \textsc{CSM}-$\DelAcc$-$\PSM$.
%
%
\citet{mni-sch:j:SM-covering-constraints} examined the question of finding a matching that covers a given set of distinguished agents while respecting a threshold for the number of blocking pairs.
Among others, they showed that even for the marriage setting, it is $\np$-hard to find a matching
that has at most $k$ blocking pairs and  covers a single distinguished agent~\citep[Theorem~5]{mni-sch:j:SM-covering-constraints}, implying hardness for \CSM-$\DelAcc$-$\MA$.

 Further intractability and parameterized results were provided by \citet{che-herm-sor-yed:c:-par-stable} for \CSR-$\DelAcc$-$\MP$, by \citet{gup-jai:c:manipulation-matching} for weighted and destructive variants of many of the problems in Table~\ref{tab:summary-for-SM}, and by 
%
\citet{kam:j:super-stable-agent-deletion} 
for the problem of finding a super-stable matching by deleting as few agents as possible in a setting where preference can contain ties.



\section{Preliminaries}
\label{sec:prelim}

An instance~$I$ of \SR\ consists of a set~$U$ of agents and a collection $\{\succ_u\}_{u \in U}$ of preferences where each preference list~$\succ_u$ is a strict linear order over the set~$A(u) \subseteq U \setminus \{u\}$ of agents that $u$ finds \myemph{acceptable}.
For instance, $x\succ_u y$ means that $u$ prefers~$x$ to $y$.
We assume that the acceptability relation is symmetric, meaning that $x$ finds $y$ acceptable if and only if $y$ finds $x$ acceptable.
For the acceptability relation of each instance~$I$, there exists an \myemph{associated undirected graph~$G=(U,E)$} over the set of agents such that for each two agents there is 
%
an edge in~$E$ connecting them if and only if they find each other acceptable.
A \myemph{matching} in~$I$ is a subset~$M \subseteq E$ of pairwise disjoint edges.
For some pair~$\{u,v\} \in M$, we will use the notation $M(u)=v$ and $M(v)=u$ and say that $M$ \myemph{covers}~$u$ and $v$. 
We write $M(u)=\emptyset$ to indicate that $u$ is \myemph{unmatched} by~$M$, that is, $M$ does not cover agent~$u$.
A \myemph{blocking pair} for some matching~$M$ in~$I$ is a pair~$\{u,v\} \in E$ of agents such that 
(i) either $M(u)=\emptyset$ or $v \succ_u M(u)$, and similarly,
(ii) either $M(v)=\emptyset$ or $u \succ_v M(v)$.
A matching is \myemph{stable} if it admits no blocking pair.

Given an instance~$I=(U,(\succ_u)_{u\in U})$ of \SR\ and a subset~$W$ of agents, we use \myemph{$I-W$} to refer to the instance obtained by \myemph{deleting the agents} in~$W$ from~$I$, that is, 
$I-W \coloneqq (U \setminus W,(\succ'_u)_{u\in U \setminus W})$ where $\succ'_u$ is the restriction of~$\succ_u$ to~$U \setminus W$.
Similarly, given a subset~$F \subseteq E$ of agent pairs,
the instance obtained from~$I$ by \myemph{deleting the acceptability} for the pairs in~$F$ is
\myemph{$I-F$} $\coloneqq (U,(\succ'_u)_{u\in U})$, where for each agent~$u \in U$, the set of acceptable agents becomes $A'(u)\coloneqq A(u) \setminus \{v\colon \{u,v\} \in F\}$
and thus $\succ'_u$ is the restriction of~$\succ_u$ to $A'(u)$.

\subsection{Problem definitions.}
Let us now formally define the control problem \CSR-$\A$-$\G$ for
each control action~$\A \in \{\AddAg,\DelAg,\DelAcc\}$ and each goal~$\G \in \{\MA, \MP, \SM, \ESM,\PSM\}$.

\paragraph{Adding agents.} First, let us focus first on the control action of agent addition~$\AddAg$.
For this control action, the input not only contains the \myemph{original} set~$U$ of agents but also a set of \myemph{addable}~$U'$ agents, and the preferences of the agents are given over the whole market over $U \cup U'$.
For the control goals~$\ESM$ and~$\PSM$, the definitions are as follows:

\decprob{\CSR-$\AddAg$-$\ESM$ (\normalfont{resp.} \CSR-$\AddAg$-$\PSM$)}{
  An instance~$I=(U\cup U',(\succ_u)_{u\in U\cup U'})$ of \SR, where $U$
  denotes the set of \myemph{original} agents and $U'$ the set of \myemph{addable} agents, and an integer \myemph{budget}~$\ell$.}
{
  Is there a subset~$W\subseteq U'$ of at most~$\ell$ agents such that $I-(U'' \setminus W)$ 
  admits a stable (resp., \myemph{perfect} and stable) matching?
}


By contrast, the control goals $\MA$ and $\MP$ require an additional input:

\decprob{\CSR-$\AddAg$-$\MA$ (resp.\ \CSR-$\AddAg$-$\MP$)}{
  An instance~$I=(U\cup U',(\succ_u)_{u\in U\cup U'})$ of \textsc{Stable Roommates}, %
  %
  an integer \myemph{budget}~$\ell$, and 
  an agent~${u^* \in U}$ (resp., a pair $e^* \in E$ of agents).}
{
  Is there a subset~$W \subseteq U$ of at most~$\ell$ agents such that $I-(U'' \setminus W)$ admits a stable matching~$M'$
  that covers~$u^*$ (resp., contains~$e^*$)?
}

Finally, for the control goal~$\SM$, following \citet{boe-bre-hee-nie:j:control-bribery-in-SM}, the additional input is a \myemph{perfect} matching, which not only covers the original but also the addable agents.
The reason for this is that the control action~$\AddAg$ may alter the agent set of an instance.


\decprob{\CSR-$\AddAg$-$\SM$}{
  An instance~$I=(U\cup U',(\succ_u)_{u\in U\cup U'})$ of \textsc{Stable Roommates}, %
  %
  an integer \myemph{budget}~$\ell$, and  
  a perfect matching~$M$ of $I$.}
{
  Is there a subset~$W \subseteq U$ of at most~$\ell$ agents such that $I-(U'' \setminus W)$ admits a stable matching~$M'$
  with $M'\subseteq M$?
}

\paragraph{Deleting agents.}
The control problems regarding the control action of deleting agents are defined analogously, where for $\SM$, we again assume the input matching to be perfect: 

\decprob{\CSR-$\DelAg$-$\SM$}{
  An instance~$I=(U,(\succ_u)_{u\in U})$ of \SR,
  an integer \myemph{budget}~$\ell$,
  and a perfect matching~$M$ of~$I$.}
{
  Is there a subset~$W \subseteq U$ of at most~$\ell$ agents such that $I-(U'' \setminus W)$ admits a stable matching~$M'$
  with $M'\subseteq M$?
}
We omit the other problem definitions for the sake of brevity.

\paragraph{Deleting acceptability.} The control problems \CSR-$\DelAcc$-$\G$ are defined analogously: instead of a set of at most~$\ell$ agents $W \subseteq U$, we ask for a set of at most~$\ell$ edges~$F \subseteq E$ so that the instance~$I-F$ satisfies our goals.
We remark that, contrasting the previous two control actions, the input matching for \CSR-$\DelAcc$-$\SM$ is simply a matching:

\decprob{\CSR-$\DelAcc$-$\SM$}{
  An instance~$I=(U,(\succ_u)_{u\in U})$ of \SR,
  an integer \myemph{budget}~$\ell$,
  and a matching~$M$ in~$I$.}
{
  Is there a subset~$F$ of at most~$\ell$ acceptable pairs such that $M$ is a stable matching in $I-F$?
}
We omit the other problem definitions for the sake of brevity.

The marriage variants of the control problems are defined analogously, with the input containing a \textsc{Stable Marriage} instance instead of a \SR\ instance.

%
%
%


%
%
%
%
%
%
%
%
%
%
%

%
%
%
%
%
%

%
%
%
%
%
%
%



%
%
%
%
%


\section{Straightforward Consequences of Known Results}
In this section, we explain some results that 
%
can be derived as easy consequences of the work by \citet{boe-bre-hee-nie:j:control-bribery-in-SM}.
\label{sec:app-SM-conseq}

\begin{proposition}
\label{prop:CSM-DelAg-MA}
\textsc{CSM}-$\DelAg$-$\MA$ is in $\p$.
\end{proposition}

\begin{proof}
To solve the \textsc{CSM}-$\DelAg$-$\MA$ problem, we will use the polynomial-time algorithm given for \textsc{CSM}-$\DelAg$-$\MP$ by \citet{boe-bre-hee-nie:j:control-bribery-in-SM}. 
Let our input of \textsc{CSM}-$\DelAg$-$\MA$ be an instance $I$ of \textsc{Stable Marriage}, an agent~$a$ in~$I$, and an integer~$\ell$ determining our budget on the number of control actions that can be used, i.e., the number of agents we can delete.

Our approach is simply to check for each possible partner~$b$ of~$a$ whether we can delete at most~$\ell$ agents from~$I$ so that in the remainder the pair $\{a,b\}$ is contained in some stable matching.
Clearly, this holds for some agent~$b$ if and only if our original instance is a ``yes''-instance (that is, there is a set of at most~$\ell$ agents whose deletion from~$I$ yields an instance that admits a stable matching, where $a$ is matched.)
Thus, for each agent~$b$ in the preference list of~$a$, we solve the \textsc{CSM}-$\DelAg$-$\MP$ problem on input $(I,\{a,b\},\ell)$. If we obtain that $(I,\{a,b\},\ell)$ is a ``yes''-instance of \textsc{CSM}-$\DelAg$-$\MP$ for some agent~$b$, then we return ``yes'', otherwise we return ``no.'' It is clear that this algorithm is correct and runs in polynomial time.
\end{proof}

\begin{proposition}
  \label{prop:CSM-DelAcc-SM}
\textsc{CSR}-$\DelAcc$-$\SM$ can be solved in linear time.
\end{proposition}

\begin{proof}
Let our input for \CSR-$\DelAcc$-$\SM$ be an instance~$I$ of \SR, a matching~$M$ in~$I$, and our budget~$\ell$ on the number of acceptability deletions we can perform. Note that $(I,\ell, M)$ is a ``yes''-instance of \CSR-$\DelAcc$-$\SM$ if and only if the number of blocking pairs for~$M$  is at most~$\ell$: Clearly, we need to delete all blocking pairs to make~$M$ stable, and deleting all of them is sufficient to ensure that $M$ becomes stable. Since we can compute the number of blocking pairs for~$M$ in linear  time, the result follows.

We remark that this simple fact was also observed in~\cite[Observation~1]{boe-bre-hee-nie:j:control-bribery-in-SM}, though stated only  for the bipartite case.
\end{proof}


\section{New Results}
 In this section, we present the results contributed by this paper. 
\label{sec:app-SM-own}
We start with the control action of adding agents in Section~\ref{sec:add-ag}, 
and then proceed with the control action of deleting agents in Section~\ref{sec:del-ag}.

\subsection{Control by adding agents}
\label{sec:add-ag}

Let us turn our focus on control by adding agents. The first result shows that making an agent matched in a stable matching is no easier than making a pair matched.
The reduction is inspired by the one given by \citet{boe-bre-hee-nie:j:control-bribery-in-SM} for $\MP$ case.

\begin{theorem}
\label{thm:CSM-AddAg-MA}
\textsc{CSM}-$\AddAg$-$\MA$ 
and \textsc{CSM}-$\AddAg$-$\PSM$ are $\np$-complete.
\end{theorem}

\begin{proof}
  It is clear that both problems are in~$\np$:
  Given a set of agent additions resulting in a modified instance~$I'$, it suffices to compute a single stable matching in~$I'$ and check whether it is perfect or whether it matches our given agent,
  because by the well-known Rural Hospitals Theorem~\citep{irv-lea-gus:j:efficient-algorithm-for-optimal-stable-marriage},
  every stable matching in~$I'$ covers the same set of agents.

  We present a polynomial-time many-to-one reduction that can be obtained as a modification of the reduction in~\cite[Theorem~1]{boe-bre-hee-nie:j:control-bribery-in-SM}; the modified reduction shows $\np$-hardness for both \textsc{CSM}-$\AddAg$-$\MA$
  and \textsc{CSM}-$\AddAg$-$\PSM$. 
%
  For convenience, we present the modified reduction in full detail. 

  We reduce from the following $\np$-complete \textsc{Clique} problem~\citep{gar-joh:b:int}.
  \decprob{Clique}
  {An undirected graph~$G$ and a non-negative integer~$k$.}
  {Does~$G$ admit a \myemph{clique} of size at least~$k$, i.e., a complete subgraph~$K$ with at least~$k$ vertices?}

  Let $G=(V,E)$ be the input graph and~$k$ the size of the desired clique. For each vertex~$v \in V$, we introduce a woman~$w_v$ and a man~$m'_v$. For each edge~$e \in E$, we introduce a woman~$w_e$ and two men~$m_e$ and $m'_e$. We further create a set $S=\{s_1,\dots,s_{\binom{k}{2}}\}$ of   \myemph{selector women}, a set $D=\{d_1,\dots,d_{|V|-k}\}$ of  \myemph{dummy men}, a woman~$w^*$, and a man~$m^*$. 
The preferences of the agents are as follows (with each set in the preference lists interpreted as an arbitrary but fixed, strict linear order of its elements): 
\[
\begin{array}{lll}
m^*: & S\succ w^*; \\
w^*: & m^*; \\
s: & \{m_e: e \in E\}\succ m^* & \textrm{ for each $s \in S$}; \\
m_e: & w_e\succ \{w_v: v \in e\}\succ S & \textrm{ for each $e \in E$}; \\
w_e: & m'_e \succ m_e; & \textrm{ for each $e \in E$}; \\
m'_e: & w_e; & \textrm{ for each $e \in E$}; \\
w_v: & m'_v\succ \{m_e:e \ni v\}\succ D & \textrm{ for each $v \in V$}; \\
m'_v: & w_v & \textrm{ for each $v \in V$;} \\
d: & \{m_v: v \in V\} & \textrm{ for each $d \in D$.}
%
\end{array}
\]

We let~$M'_V\coloneqq \{m'_v:v \in V\}$ and $M'_E \coloneqq \{m'_e:e \in E\}$ and set $M'_V \cup M'_E$ as the set of addable agents, with the remaining agents forming the set of original agents. 
Note that the addable agents are all men.
Our goal in \textsc{CSM}-$\AddAg$-$\MA$ is to match agent~$w^*$, while our goal in \textsc{CSM}-$\AddAg$-$\PSM$ is to obtain a perfect and stable matching. 
Finally, we set our budget, that is, the number of agents from $M'_V \cup M'_E$ that we can add to~$I$, as $\ell \coloneqq k+\binom{k}{2}$. 

To see the correctness of the reduction, first assume that
$G$ admits a clique of order~$k$; let $V(K)$ and $E(K)$ denote its vertices and edges, respectively. 
We show that adding the $k+\binom{k}{2}$ men $\{m'_v:v  \in V(K)\} \cup \{m'_e:e \in E(K)\}$ yields an instance~$I_K$ that admits a perfect and stable matching~$M$ that, in particular, matches~$w^*$ as well. Clearly, every stable matching in the resulting instance~$I_K$ contains all pairs in $F'\coloneqq\{\{m'_v,w_v\}:v  \in V(K)\} \cup \{\{m'_e,w_e\}:e \in E(K)\}$, 
because $F'$  matches every involved agent to its best choice. We construct $M$ as follows.
We match $w^*$ to $m^*$, match all selector women to the men in $\{m_e:e  \in E(K)\}$ in a stable way, add the pairs in the matching~\myemph{$F$} $\coloneqq \{\{m_e,w_e\}:e \in E \setminus E(K)\}$ together with those in~$F'$ to~$M$, 
and finally, we match the women in~$\{w_v:v \in V \setminus V(K)\}$ to the $|V|-k$ dummy men in a stable way.\footnote{By matching a set $X$ of women with a set~$Y$ of men \myemph{in a stable way}, we mean matching them according to an arbitrary stable matching in the SM instance induced by $X \cup Y$. Note that we only do this when preferences in this sub-instance are complete, and hence, each of its stable matchings matches all agents in~$X \cup Y$.}

It is straightforward to verify that $M$ covers every agent and is stable: 
\begin{itemize}
\item No blocking pair contains the distinguished woman~$w^*$, because she is matched with her best choice. %
\item No blocking pair contains a selector woman~$s \in S$ and some man $m_e$: If $e \notin E(K)$, then this follows from $\{m_e,w_e\} \in F$, otherwise we have $M(m_e) \in S$, and thus $\{s,m_e\}$ does not block~$M$ due to the stability of the restriction of $M$ to~$S \cup \{m_e:e \in E(K)\}$.
\item No blocking pair contains woman~$w_e$ corresponding to an edge~$e$, because it is matched to her best choice in~$I_K$: If $e \in E(K)$, then she is matched to~$m'_e$;otherwise, she is matched to~$m_e$ and $m'_e$ is not present.
%
%

\item No blocking pair contains a woman~$w_v$: 
If $v \in V(K)$, then it is matched to~$m'_v$, her best choice.
If $v \notin V(K)$, then $M(w_v) \in D$.
In this case, she cannot block with any other dummy agent  due to the stability of the restriction of $M$ to~$D \cup \{w_v:v \in V \setminus V(K)\}$.
Moreover, each non-dummy man acceptable to~$w_v$ is a man~$m_e$ corresponding to an edge incident to~$v$ and hence not in $E(K)$, which means that $m_e$ is matched to his best choice by $\{m_e,w_e\} \in F$. 
\end{itemize}
Since no woman is involved in a blocking pair, $M$ is indeed a stable matching in~$I_K$.

For the other direction, assume that by adding a set of at most~$k+\binom{k}{2}$ men from~$M'_V \cup M'_E$ to~$I$, we obtain an instance~$I'$ that admits a matching~$M$ in which $w^*$ is matched (which always holds if $I'$ admits a perfect matching). 
We first show that in this case we may assume w.l.o.g.\ that $M$ contains no pair $\{m_e,w_v\}$ for some edge~$e$ incident to some vertex~$v$ in~$G$; we call such pairs \myemph{problematic}. Indeed, if ${\{m_e,w_v\} \in M}$, then since $\{m_e,w_e\}$ cannot block~$M$, we know that ${M(w_e)=m'_e}$. Then we modify~$I'$ and~$M$  by removing the addable agent~$m'_e$ from~$I'$ and adding~$m'_v$ instead, setting also
$M'={\big(M \setminus \{\{m_e,w_v\}, \{m_e', w_e\}\}\big) \cup \{\{m_e,w_e\},\{m'_v, w_v\}\}}$. It is clear that the resulting instance~$I''$ can still be obtained by adding at most $k+\binom{k}{2}$ agents to~$I$, and $M'$ is a stable matching in~$I''$ covering~$w^*$ and containing less problematic pairs than~$M$. Hence, repeating this operation as long as necessary, we end up with an instance~$\hat{I}$ and a stable matching~$\hat{M}$ in~$\hat{I}$ that covers~$w^*$ but contains no problematic pairs.
Let $X_V$ and~$X_E$ denote the set of vertices and edges in~$G$ that correspond to the addable men that are present in~$\hat{I}$.


Observe first that $w^*$ can only be matched to~$m^*$, that is, $\{m^*,w^*\} \in \hat{M}$.
This means that all selector women must be matched in~$\hat{M}$, as an unmatched selector woman would form a blocking pair with~$m^*$. Hence, there are exactly $\binom{k}{2}$ men in~$\{m_e:e \in E\}$ matched to selector women; let $P$ denote the set of these men.
Then, $|P|=\binom{k}{2}$.
Note that if $m_e \in P$, then $m'_e$ must be present in~$I'$ and matched to~$w_e$ by~$\hat{M}$ (that is, ${e \in X_E}$), as otherwise $\{m_e,w_e\}$ would block~$\hat{M}$.
%
%
Furthermore, for each $m_e\in P$, each woman in~$\{w_v:v  \in e\}$ must also be matched by~$\hat{M}$ to
someone that is more preferred to~$m_e$ as otherwise $\{m_e,w_v\}$ would block~$\hat{M}$.
Since~$\hat{M}$ contains no problematic pair, this means that for each $m_e\in P$ and each~$v\in E$,
the man~$m'_v$ must be present in~$I'$ and matched to~$w_v$ (that is, $v \in X_V$).
Hence, for each edge~$e$ in $\hat{E}\coloneqq \{e:m_e \in P\} \subseteq X_E$, both endpoints of~$e$ must be in~$X_V$. By $|X_E|+|X_V| \leq \binom{k}{2}+k$, we get that the $\binom{k}{2}$ edges in~$\hat{E}$ must have altogether at most $k$ endpoints, i.e., they must form a clique of size~$k$ in~$G$.
\end{proof}


Note that \textsc{CSR}-$\AddAg$-$\MA$ is in~$\np$, because even in an instance of \SR, all stable matchings cover the same set of agents~\citep[Theorem 4.5.2]{gus-irv:b:the-stable-marriage-problem}.
This leads to the following consequence of Theorem~\ref{thm:CSM-AddAg-MA}. 
\begin{corollary}\label{cor:CSR-AddAg-MA}
\textsc{CSR}-$\AddAg$-$\MA$ 
%
is $\np$-complete.
\end{corollary}

In contrast to the marriage case, making a matching stable by adding a fewest number of agents in the roommates setting is NP-hard.
\begin{theorem}\label{thm:CSR-AddAg-SM}
\textsc{CSR}-$\AddAg$-$\SM$ is $\np$-complete.
\end{theorem}

\begin{proof}
It is clear that the problem is in $\np$, so we give a polynomial-time many-to-one reduction from the following $\np$-complete problem~\textsc{Independent Set} to prove its $\np$-hardness. 
  \decprob{Independent Set}
  {An undirected graph~$G$ and a non-negative integer~$k$.}
  {Does~$G$ admit a subset of at least~$k$ vertices that form an \myemph{independent set}, that is,
    these vertices are pairwise non-adjacent?}

Let $G=(V,E)$ and $k$ be our input of \textsc{Independent Set}. 

For each vertex~$v \in V$, we introduce three agents~$a_v,b_v,c_v$ and their respective copies $a'_v,b'_v,c'_v$.
We let the input matching~$M$ match each agent in~$A=\{a_v,b_v,c_v:v \in V\}$ with its copy in $A'=\{a'_v,b'_v,c'_v:v \in V\}$. 
Let the preferences of the agents be defined by setting the following for each $v \in V$:
\[
\begin{array}{llll}
a_v: & a'_v\succ b_v \succ c_v; \qquad\qquad
  & a'_v: & \{a'_u:\{u,v\} \in E\}\succ a_v \\
b_v: & b'_v\succ a_v;
  & b'_v: & b_v; \\
c_v: & c'_v\succ a_v;
  & c'_v: & c_v. \\
\end{array}
\]
We set $A'$ as the set of addable agents, and we set $\ell=2|V|-k$ as our budget.



First assume that $G$ contains a set~$X$ of~$k$ vertices that form an independent set. Let $I_X$ denote the instance obtained by adding the agents in \[A'_X\coloneqq\{a'_v:v \in X\} \cup \{b'_v,c'_v:v \in V \setminus X\}.\]
Thus, $I_X$ is the instance induced by the agent set $A \cup A'_X$. 
Note that $|A'_X|=|X|+2|V \setminus X|=\ell$.

Define the matching \[M_X\coloneqq\{\{p,p'\}:p \in A,p' \in A'_X\}\] in~$I_X$.
Observe that $M_X \subseteq M$ is stable in~$I_X$:
\begin{itemize}
\item No agent~$a_v$ is contained in a blocking pair: If $v \in X$, then $a_v$ is matched to~$a'_v$, her best choice in~$I_X$.
If $v \in V \setminus X$, then $b_v$ and $c_v$ are the only agents acceptable for~$a_v$ in~$I_X$, and they are both matched to their respective copies, which are their respective best choice in~$I_X$.
\item Neither $b_v$ nor $c_v$ is contained in a blocking pair: If $v \in X$, then $a_v$ is matched to its best choice (its copy~$a'_v$), so does not form a blocking pair with~$b_v$ or~$c_v$; If $v \in V \setminus X$, then both $b_v$ and $c_v$ are matched to their copies which are their respective best choice.
\item No blocking pair contains two agents from $A'_X$, say $a'_v$ and $a'_u$: if $u,v \in X$, then $a'_v$ and $a'_u$ are not acceptable to each other, because $\{u,v\} \notin E$ since $X$ is an independent set.
\end{itemize}
This shows that our instance of \textsc{CSR}-$\AddAg$-$\SM$ is a ``yes''-instance.

For the other direction, assume that there is a set $A^\star \subseteq A'$ of at most $\ell$ agents such that the instance~$I'$ induced by the agents in~$A \cup A^\star$ contains a matching $M' \subseteq M$ that is stable in~$I'$.
We claim that $a'_v \in A^\star$ or $\{b'_v,c'_v\} \subseteq A^\star$ holds for each vertex $v \in V$. 
To see this, assume $a'_v \notin A^\star$. This means that $a_v$ is not matched by~$M'$. 
Hence, $M'$ must match both $b_v$ and~$c_v$, as otherwise $\{a_v,b_v\}$ or $\{a_v,c_v\}$ would block~$M'$. Therefore, we get $\{b'_v,c'_v\} \subseteq A^\star$, as required.

Let us define $X\coloneqq\{v:a'_v \in A^\star\}$. Note that \[2|V|-k=\ell \geq |A^\star| \geq |X|+2 |V \setminus X|=2|V| -|X|\]
by the previous paragraph, so we get $|X| \geq k$. It remains to see that $X$ is an independent set: indeed, if two vertices~$u$ and~$v$ of~$X$ are adjacent in~$G$, then $M'$ matches $a_u$ and~$a_v$ with their respective copies; however, then $\{a'_v,a'_u\}$ is a blocking pair for~$M'$.
Hence, $X$ is an independent set of size at least~$k$ in~$G$, proving the correctness of our reduction.
\end{proof}



\begin{theorem}
\label{thm:CSR-AddAg-ExistsSM}
\textsc{CSR}-$\AddAg$-$\ESM$ and \textsc{CSR}-$\AddAg$-$\PSM$ are $\np$-complete.
\end{theorem}

\begin{proof}
It is clear that both problems are in $\np$, because checking whether an  instance~$I$ of SR admits a stable matching can be done in polynomial time~\citep{irv:j:efficient-algorithm-for-stable-roommates-problem}, and since all stable matchings in~$I$ have the same size~\citep{gus-irv:b:the-stable-marriage-problem}, finding any stable matching by, e.g., Irving's algorithm~\citep{irv:j:efficient-algorithm-for-stable-roommates-problem} suffices to check whether a perfect and stable matching exists.

We present a reduction from the \textsc{Independent Set} problem which shows the $\np$-hardness of both problems. Let the graph $G=(V,E)$ and the integer $k$  be our input for \textsc{Independent Set}. For each vertex $v \in V$, let $N(v)$ denote the set of vertices adjacent to~$v$ in~$G$.

We construct an instance of \SR\ over agent set $A\coloneqq V \cup \{s_j,a_j,b_j:j=1,\dots,k\}$. The preferences of these agents are as follows (with each set in the preference lists interpreted as an arbitrary but fixed, strict linear order of its elements): 
\[
\begin{array}{lll}
v: & N(v)\succ  s_1\succ \dots \succ s_k
  & \textrm{ for each $v \in V$;} \\
s_i: & V \succ a_i \succ  b_i 
  & \textrm{ for each $i \in \{1,\dots,k\}$;} \\
a_i: & b_i\succ  s_i
  & \textrm{ for each $i \in \{1,\dots,k\}$;} \\  
b_i: & s_i\succ  a_i 
  & \textrm{ for each $i \in \{1,\dots,k\}$.}   
\end{array}
\]
The set of original agents is~$A \setminus V$, and $V$ is the set of addable agents.
We set $k$ as our budget, i.e., the number of agents that can be added to~$I$ from among~$V$.

We claim that $G$ contains an independent set of size~$k$ if and only if there are $k$ agents in~$V$ whose addition to~$I$ yields in instance that contains a stable matching; moreover, such a stable matching is necessarily perfect.

Assume first that there is an independent set~$X$ of size~$k$ in~$G$.
Let $I'$ be the SR instance obtained by adding the agents of~$X$ to~$I$. We show that $I'$ admits a stable and perfect matching. Let us match the agents of $S\coloneqq\{s_1,\dots,s_k\}$ to agents of~$X$ in a stable manner: Since the sub-instance of~$I$ induced by $S \cup X$ is an instance of \textsc{Stable Marriage} with complete preference lists, there exists a stable matching~$M_X$ that covers all  agents in~$S \cup X$ and admits no blocking pair formed by agents in~$S \cup X$.
Then the matching $M\coloneqq\{\{a_i,b_i\}:i =1,\dots,k\} \cup M_X$  covers all agents in~$I'$ and is also stable in~$I'$. 

Assume now that there is a set~$X$ of at most~$k$ agents in~$V$ whose addition to~$I$ yields an instance~$I'$ that admits a stable matching~$M$. Observe that for each $i \in \{1,\dots,k\}$, the matching $M$ must match $s_i$ to an agent in~$V$ due to the circular preferences of $a_i,b_i,$ and $s_i$. Thus, $X$ must contain exactly~$k$ agents from~$V$. It suffices now to note that $X$ forms an independent set in~$G$: assuming that two vertices of~$X$, say $x$ and $y$, are adjacent, the pair $\{x,y\}$ would block~$M$, a contradiction proving the correctness of our reduction.
\end{proof}

\subsection{Control by deleting agents}
\label{sec:del-ag}
Let us now present our last result for control by deleting agents.
\citet{boe-bre-hee-nie:j:control-bribery-in-SM} showed that making a pair included in a stable matching is polynomial-time solvable for the marriage setting. 
We extend their approach to the roommates setting.
The crucial idea is to use the concept of stable partitions by \citet{tan:j:stable-partitions} and a result by \citet{tan-hsu:j:-generalization-of-the-stable-matching-problem} regarding deleting an agent; see also \citep[Sections~4.3.2 and~4.3.4]{man:b:algorithmics-of-matching-under-preferences} for the necessary background on stable partitions.

\begin{theorem}\label{thm:CSR-DelAg-MP+MA}
\textsc{CSR}-$\DelAg$-$\MP$ and \textsc{CSR}-$\DelAg$-$\MA$ are in~$\p$.
\end{theorem}

\def\J{\mathcal{J}}
\begin{proof}
We present a polynomial-time algorithm for the \textsc{CSR}-$\DelAg$-$\MP$ problem. Using the arguments of Proposition~\ref{prop:CSM-DelAg-MA}, this implies the polynomial-time solubility of \textsc{CSR}-$\DelAg$-$\MA$ as well.
As already mentioned, we will use the structural properties of stable partitions  and ideas from the algorithm given by \citet{boe-bre-hee-nie:j:control-bribery-in-SM} for the bipartite version of our problem, \textsc{CSM}-$\DelAg$-$\MP$.

Let $\J=(I,\ell, \{a,b\})$ be our input instance where
$I$ is an instance of \SR, 
$\ell$ is our budget, i.e., the number of agents we are allowed to delete, and
$\{a,b\}$ is the (acceptable) pair in~$I$ that we aim to include in the desired matching. We say that a set~$S$ of agents is a \myemph{solution} for~$\J$ if removing $S$ from~$I$ yields an instance, denoted by $I-S$, where some stable matching contains~$\{a,b\}$.

Let $A^\star$ and $B^\star$  be the set of agents which are preferred by~$a$ to~$b$, or by~$b$ to~$a$, respectively (note that $A^\star$ and~$B^\star$ may not be disjoint). Furthermore, let $F$ contain those pairs $\{x,y\}$ where 
%
\begin{compactenum}[(i)]
  \item $x\in A^{\star}$, and $y=a$ or $x$ prefers $a$ to~$y$, or 
  \item $x\in B^{\star}$, and $y=b$ or $x$ prefers $b$ to~$y$.
\end{compactenum}
Formally, let 
\begin{align*}
F\coloneqq & 
\{\{x,y\} \mid x\in A^{\star}, (y=a \textup{ or } a \succ_x y)\} \\
& \cup 
\{\{x,y\}\mid x\in B^{\star}, (y=b \textup{ or } b \succ_x y)\}.
\end{align*}
Let $I^\star$ denote the instance obtained from~$I$ by deleting all pairs in~$F$ from the set of acceptable pairs,
i.e., $I^{\star}\coloneqq I-F$.
%


Our algorithm is the following: 
\begin{itemize}
\item[Step 1.] Compute the set~$F$ and the instance~$I^\star$.
\item[Step 2.] Compute a stable partition~$\Pi$ for~$I^\star$, and let $U$ denote the set of singleton parties in~$U$ consisting of an agent in~$A^\star \cup B^\star$. 
\item[Step 3.] Return ``yes'' if $r+|U| \leq \ell$ where $r$ is the number of odd parties of size at least~$3$ in~$\Pi$; otherwise, return ``no.''
\end{itemize}

Note that Step~2 can be performed in linear time using Tan's algorithm~\citep{tan:j:stable-roommates} for computing a stable partition. 
Hence, the algorithm runs in polynomial time. It remains to show its correctness. 
We start with the following claim.

\begin{claim}
\label{clm:characterize-sols}
 A set $S$ of agents is a solution for~$\J$ if and only if $I^\star-S$ admits a stable matching that matches every agent in~$(A^\star \cup B^\star) \setminus S$.
\end{claim}

\begin{claimproof}
Assume that $S$ is a solution for~$\J$. Then there exists a stable matching~$M$ in~$I \setminus S$ containing $\{a,b\}$. First, let us show that $M \cap F=\emptyset$. Suppose for the sake of contradiction that $\{x,y\} \in M \cap F$. Then $x \in A^\star \cup B^\star$, by symmetry we may assume $x \in A^\star$. By the definition of~$F$, we infer that either $y=a$ or $x$ prefers~$a$ to~$y$. The former is not possible, because $M(a)=b \notin A^\star$, and the latter is not possible either, because it would mean that $\{x,a\}$ blocks~$M$: $x$ prefers~$a$ to~$y=M(x)$ and $a$ prefers~$x$ to~$M(a)=b$ by the definition of~$A^\star$. This contradiction proves that $M$ is a matching disjoint from~$F$, i.e., it is a matching in~$I^\star-S$. 

It remains to see that $M$ matches every agent in the set~$(A^\star \cup B^\star) \setminus S$. Suppose for the sake of contradiction that $x \in (A^\star \cup B^\star) \setminus S$ is unmatched by~$M$. By symmetry, we may assume $x \in A^\star$. However, then the pair $\{x,a\}$ is a blocking pair, a contradiction. Thus, $M$ has the desired property.

For the other direction, assume that for some agent set~$S$, the instance $I^\star -S$ admits a stable matching~$M$ that matches every agent in~$(A^\star \cup B^\star) \setminus S$. In this case, $M$ matches $a$ to~$b$ because they are each other's best choice in~$I^\star$ due to the deletion of the acceptability of all pairs in~$F$.
\end{claimproof}
 
To show the correctness of our algorithm, assume first that it returns ``yes''. Let $S$ contain an arbitrarily chosen agent from each odd party of size at least~$3$ in~$\Pi$ and all singleton parties consisting of an agent in~$A^\star \cup B^\star$.  
Then deleting $S$ from~$I^\star$ yields an instance~$I'$ whose stable partition~$\Pi'$ has no odd parties and whose singleton parties are all disjoint from~$A^\star \cup B^\star$, due to well-known properties of stable partitions~\citep{tan:j:stable-partitions} (see also \citep[Proposition~4.11]{man:b:algorithmics-of-matching-under-preferences}). This means that in all stable matchings for~$I^\star-S$, the set of unmatched agents (i.e., those agents that form singletons in~$\Pi'$) is disjoint from~$A^\star \cup B^\star$. By Claim~\ref{clm:characterize-sols}, $S$ is therefore a solution, and since it has size at most~$\ell$ by Step~3, we obtain that $\J$ is a ``yes''-instance.
 
It remains to prove that whenever $S$ is a solution of size at most~$\ell$, then the algorithm returns ``yes''.
Since $S$ is a solution, by Claim~\ref{clm:characterize-sols} we know that $I^\star-S$ admits a stable matching that matches all agents in~$(A^\star \cup B^\star) \setminus S$. This means that the stable partition~$\Pi_S$ of~$I^\star-S$ contains no odd parties of size at least~$3$, and contains no singleton odd parties formed by some agent in~$(A^\star \cup B^\star) \setminus S$; let us call such odd parties \myemph{forbidden}. However, it is known that deleting an agent from an instance of \SR\ results in an instance whose stable partitions contain all odd parties of the original instance except, possibly, for one (see \cite[Corollary~3.9]{tan-hsu:j:-generalization-of-the-stable-matching-problem}). 
Since $\Pi$ has exactly $|U|+r$ forbidden odd parties, deleting less than~$r+|U|$ agents from~$I^\star$ will result in an instance that has at least one forbidden odd party. This implies $\ell \geq |S| \geq r+|U|$, implying that  the algorithm returns ``yes'' in Step~3.
\end{proof}


\section*{Acknowledgement}
We are grateful for J\"org Rothe for enjoyable and fruitful discussions, and for his invitation for preparing a survey on control problems in the area of matching under preferences;  the work contained in this paper originated in our efforts to write that survey. 
J.~Chen is supported by the Vienna Science and Technology Fund (WWTF)~[10.47379/VRG18012].
I.~Schlotter is supported by the Hungarian Academy of Sciences under its Momentum Programme
(LP2021-2) and its J\'anos Bolyai Research Scholarship. 


%
\bibliographystyle{ACM-Reference-Format}
\bibliography{control}

\end{document}


%
%
%
%
