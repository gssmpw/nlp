\section{Related Works}
\label{sec:related}
The vast majority of work on point cloud-based learning has focused on three dimensional problems, such as mapping and interpreting sensor readings in computer vision or localization tasks ~\cite{pointnet, pointnet++, pointtransformer, pointmlp, dgcnn}. Early methods such as PointNet~\cite{pointnet} and its successor PointNet++~\cite{pointnet++} introduced permutation-invariant models with the ability to extract local and global features from 3D point clouds. Newer methods such as Dynamic Graph Convolutional Neural Network (DGCNN)~\cite{dgcnn} leverage dynamic graph representations to better extract rich features from input point clouds, while PointMLP~\cite{pointmlp} and Point Transformer~\cite{pointtransformer} use standard Multilayer Perceptrons and local self-attention mechanism as the building blocks for better performance in classification and regression tasks. Previous methods provide motivation for our line of inquiry, but since the biological domains in which we are interested frequently have orders of magnitude more dimensions ranging from dozens in proteomics datasets to thousands in transcriptomic datasets, we require a new approach that is not limited by architectural decisions and is computationally efficient when working with high-dimensional data. Despite the advancements in these existing methods, they are fundamentally designed for 3D point clouds and struggle with performance and scalability in a high dimensional setting as they rely on spatial heuristics, because local neighborhoods become less meaningful in high-dimension. As opposed to this, {\modelname} is able to scale to high-dimensional data while also preserving the geometry of the dataset.

Single-cell analysis has greatly benefited from graph-based methods that learns global structure from high-dimensional data. These methods typically rely on a \emph{single} graph construct to infer relationships between cellular states. Methods such as UMAP~\cite{mcinnes2020umapuniformmanifoldapproximation} and t-SNE~\cite{JMLR:v9:vandermaaten08a} perform non-linear dimensionality reduction by constructing a neighborhood graph and embedding cells into a low-dimensional space. On the other hand, PHATE~\cite{moon2019visualizing} is a dimensionality reduction method that captures both global and local nonlinear structure but only constructs a \emph{single} graph from the data. While these methods have been useful in understanding various biological processes, they may fail to organize cells based on processes governed by subsets of dimensions with just a single connectivity structure. In contrast to this, {\modelname} has the ability to model distinct cellular processes by leveraging its multi-view framework, preserving important geometric properties.