\section{Background}\label{sec:background}
This section introduces the quantum circuit cutting theory and identifies its key challenges.

\subsection{Circuit Cutting Theory}
\begin{figure*}[t]
    \centering
    \includegraphics[width=\linewidth]{figures/cutting_example.pdf}
    \caption{Example of cutting a $5$-qubit quantum circuit with one cut to divide it into two smaller subcircuits.
    (Left) The red cross indicates the cutting point.
    Subcircuit $1$ is shaded dark and subcircuit $2$ is shaded light.
    (Right) The dashed arrow between the subcircuits shows the path undertaken by the qubit wire being cut.
    The one cut needs to permute through the $\{I,X,Y,Z\}$ bases to reconstruct the unknown cut state.
    The two subcircuits require no quantum communications can now be executed independently in any order on multiple $3$-qubit QPUs.}
    \label{fig:cutting_example}
\end{figure*}

While we direct readers to~\cite{peng2020simulating} for a detailed derivation and proof of the physics theory,
we provide an intuitive understanding of the cutting process to identify the key challenges.

Figure~\ref{fig:cutting_example} shows an example of cutting a simple quantum circuit.
The left panel shows a $5$-qubit quantum circuit.
Each horizontal line is a qubit wire.
Boxes incident on a single qubit wire are single qubit quantum gates.
Boxes incident on two qubit wires are $2$-qubit quantum gates.
Without cutting, this circuit requires a QPU with at least five good enough qubits to execute all the quantum gates before too many errors accumulate.

Circuit cutting divides a large quantum circuit into smaller subcircuits.
For instance, a cut marked by a red cross can split the circuit into two distinct subcircuits.
This allows multiple less powerful $3$-qubit QPUs to run these independent subcircuits in parallel and in any sequence,
as there are no quantum connections necessary between them.
The process involves making vertical cuts along the qubit wires.
Generally, multiple cuts are used to separate a large quantum circuit into several smaller subcircuits.

Circuit cutting decomposes any unknown quantum states at cut points into a linear combination of their Pauli bases.
Specifically, QPUs run four variations of subcircuit $1$, each measures the upstream cut qubit $q_2$ in one of the $\{I,X,Y,Z\}$ Pauli bases.
One detail to note is that measuring in $I$ and $Z$ bases uses the same single-qubit rotations,
hence only $3$ subcircuits per upstream cut qubit are needed.
Correspondingly, QPUs run four variations of subcircuit $2$,
each initializes the downstream cut qubit $q_2'$ in one of the $\{\ket{0},\ket{1},\ket{+},\ket{i}\}$ states.
The $\{I,X,Y,Z\}$ Pauli bases are further constructed as a linear combination from the initialization states.
Measuring (initializing) a qubit in different bases simply means appending (prepending) various single qubit rotations on the qubit,
which are standard operations with little overhead and does not complicate the subcircuits.
Overall, the procedure produces four subcircuit $1$ outputs of $p_1^e$, for cut edge $e\in\{I,X,Y,Z\}$,
and similarly for $p_2^e$.

Circuit cutting involves classically reconstructing the quantum state outputs by combining the outputs of the subcircuits.
Prior works~\cite{peng2020simulating,tang2021cutqc} demonstrate that the binary state probability distribution $P$ of the original uncut circuit is equivalent to $\sum_{e}p_1^e\otimes p_2^e$.
This process effectively replaces quantum interactions between subcircuits with classical co-processing.

We introduce the following notations to describe the general quantum circuit cutting scenario:
\begin{enumerate}
    \item An $n$-qubit quantum circuit undergoes multiple cuts $e\in E$,
    for a total of $|E|$ cuts.
    \item Cuts in $E$ divide the input circuit into $m$ completely separated subcircuits $C_i\in\left\{C_1,\ldots,C_{m}\right\}$.
    \item $E_i\subseteq E$ represents the subset of cut edges in $E$ on $C_i$.
    In general, a subcircuit $C_i$ does not touch all the cut edges.
    Note that $C_i$ is only initialized and measured differently if $e\in E_i$ changes bases.
    \item $p_i^{\{e\}}$ represents the binary state probability output of $C_i$ for a particular set of cut edge bases $\forall e\in E_i$.
\end{enumerate}

The physics theory~\cite{peng2020simulating} dictates that the binary state probability vector output $P$ of the original $n$-qubit circuit is given by:
\begin{equation}
    P=\textcolor{blue}{\sum_{\{e_0\ldots e_{|E|-1}|e_i\in\{I,X,Y,Z\}\}}\otimes_{j=1}^{m}}\textcolor{red}{p_j^{\{e|e\in E_j\}}}\in\mathbb{R}^{2^n}\label{eq:reconstruction}
\end{equation}
where $\otimes$ is the tensor product between a pair of subcircuit binary state output vectors.
The red part of equation~\ref{eq:reconstruction} represents the QPU computations,
and the blue part of the equation represents the classical co-processing.
Equation~\ref{eq:reconstruction} shows that a full reconstruction permutes each cut edge $e$ through the $\{I,X,Y,Z\}$ bases,
for a total of $4^{|E|}$ permutation combinations.

\subsection{Circuit Cutting Challenges}
Equation~\ref{eq:reconstruction} clearly demonstrates the key challenges of applying quantum circuit cutting at useful scales:

\begin{enumerate}
    \item The classical co-processing overhead scales exponentially with the number of cuts as $4^{|E|}$ and hence bottlenecks the runtime.
    \item Reconstructing the full binary state probability distribution $P$ computes an $\mathbb{R}^{2^n}$ vector,
    which scales exponentially.
    \item Identifying high-quality cuts that reduce the classical co-processing overhead is crucial;
    however, this task constitutes an NP-complete constrained graph partition problem when applied to arbitrary quantum circuits.
\end{enumerate}

TensorQC combines frontier classical techniques in tensor networks and quantum computing to address the co-processing scalability challenges.
In addition, TensorQC models the hybrid runtime of distributing quantum circuits and proposes graph partition heuristics to automatically find cuts.