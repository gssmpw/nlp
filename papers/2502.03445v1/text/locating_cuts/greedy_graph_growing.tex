\subsection{Greedy Graph Growing}

\begin{algorithm}[t]
    \DontPrintSemicolon
    \SetAlgoLined
    \caption{Greedy Graph Growing}\label{alg:graph_growing}
    \KwIn{Quantum circuit DAG $g$.
    Max number of qubits on QPUs $w_{\max}$.
    Max number of gates supported by QPUs $s_{\max}$.
    Threshold number of contraction edges $K_{t}$.
    Merging cost cutoff $Q_{\max}$.}
    Initialize a DAG $g'=[E,V]$ with only the $2$-qubit gates in $g$.\;
    Compute the cost of merging $\forall e\in E$ using Algorithm~\ref{alg:compute_merging_cost}.\;
    Select the edge $e_{\min}$ with the minimum merging cost for which $Q_e\leq Q_{\max}$.\;
    \While{$\exists e_{\min}$}{
        Merge the edge $e_{\min}$ to create a new partition.\;
        Update the merging costs of all neighboring edges on the new partition using Algorithm~\ref{alg:compute_merging_cost}.\;
        Select $e_{\min}$ if there are valid edges remaining.\;
    }
\end{algorithm}

\begin{algorithm}[t]
    \DontPrintSemicolon
    \SetAlgoLined
    \caption{Compute Edge Merging Cost}\label{alg:compute_merging_cost}
    \KwIn{DAG $g=[E,V]$.
    Subset $E'\subseteq E$.
    Various threshold values.}
    \For{$e\in E'$}{
        Merge the two vertices connected by $e$ in $g$.\;
        Count the number of qubits $w_{trial}$ and gates $s_{trial}$ in the newly merged trial partition.\;
        Count the max number of contraction edges $K_{trial}$ between the trial partition and its neighbors.\;
        Compute the sum of costs $Q_e$ from $w_{trial}$, $s_{trial}$ and $K_{trial}$ according to Table~\ref{table:merging_cost}.\;
        Un-merge $e$.\;
        }
\end{algorithm}

\begin{table}[t]
    \centering
    \begin{tabular}{ |c|c|c| } 
        \hline
         & Subcircuit Widths \& Sizes & \#Contraction Edges \\
        \hline
        $\leq T$ & $x/T$ & $x/T$ \\ 
        \hline
        $>T$ & Not allowed & $4^{x-T}+1$ \\ 
        \hline
       \end{tabular}
    \caption{The cost of merging an edge depends on the metric value $x$,
    its respective threshold $T$,
    and whether it is a quantum hard constraint or a classical penalty.
    Breaking quantum constraints thresholds is not allowed while breaking classical cost thresholds imposes heavy penalties.}
    \label{table:merging_cost}
\end{table}

Constrained graph partition problems,
which involve partitioning nodes in a way that satisfies certain conditions,
are believed to have no polynomial time solutions~\cite{andreev2006balanced, karypis1998fast}.
Prior works~\cite{tang2021cutqc} employed Mixed Integer Programming (MIP) to minimize the number of cuts in a quantum circuit.
However, this approach is notably slow and fails to efficiently scale beyond circuits with approximately $100$ qubits.
Furthermore, traditional graph partition heuristics like METIS~\cite{karypis1998multilevelk} only deal with static graphs.
These algorithms are not suitable for quantum circuits because the subcircuit qubit counts increase with each cut,
making them difficult to treat as static input variables.

Instead, TensorQC has developed heuristic methods to approximate the total runtime objective~\ref{eq:cut_objective}.
According to the tensor network contraction complexity formula~\ref{eq:tn_upper_bound},
the maximum number of contraction edges at any step in an optimal contraction sequence provides an upper bound on the cost.
Yet, identifying an optimal contraction sequence remains NP-hard,
making it challenging to determine the precise $K_{\max}$.

Exhaustive counting does not scale.
Instead, TensorQC simplifies the approach by counting the number of cut edges needed to contract any pair of neighboring subcircuits,
using this as a proxy for $T_{classical}$.
For example, Figure~\ref{fig:cut_dag} requires counting the number of cut edges if we were to contract
subcircuits $1,2$, subcircuits $1,3$, or subcircuits $2,3$.
This method scales effectively,
as it limits the counting to no more than $|E|$ pairs of neighboring subcircuits,
providing a reasonable estimate of contraction costs across various sequences.

In addition, the number of cuts on each subcircuit has a significant exponential impact on the quantum runtime $T_{QPU}$.
Hence, counting the cut edges also indirectly approximates the quantum runtime $T_{QPU}$.

To optimize the total runtime,
we introduce a threshold $K_{t}$ for the number of cut edges.
The heuristics heavily penalize any pair of neighboring subcircuits exceeding this threshold,
with $K_{i,j}>K_{t}, \forall i,j\in \{1,\ldots,m\}$,
where $K_{i,j}$ is the number of contraction edges for a pair of neighboring subcircuits.

Algorithm~\ref{alg:graph_growing} details the heuristic approach,
with Algorithm~\ref{alg:compute_merging_cost} serving as a subroutine.
At a high level, this algorithm evaluates the cost of merging potential edges and selects the most advantageous merger.
The merging process continues until further merging would exceed QPU capabilities or incur prohibitive classical costs.
The cost of a potential merge is calculated by aggregating both quantum and classical impacts,
as summarized in Table~\ref{table:merging_cost}.