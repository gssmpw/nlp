\subsection{Problem Formulation}
\begin{figure}[t]
    \centering
    \begin{subfigure}{0.5\textwidth}
        \centering
        \includegraphics[width=\textwidth]{figures/circuit.pdf}
        \caption{QAOA benchmark circuit solving the maximum independent set problem for a random Erdos-Renyi graph with $5$ qubits.}
        \label{fig:circuit}
    \end{subfigure}
    \begin{subfigure}{0.5\textwidth}
        \centering
        \includegraphics[width=\textwidth]{figures/cut_dag.pdf}
        \caption{DAG representation of~\ref{fig:circuit} without the single-qubit gates.
        The shaded vertices indicate the two-qubit quantum gates.
        The directed edges indicate the time flow of the qubit wires.
        Qubits in the initial states (left column) evolve through the quantum gates and are measured as output (right column).
        The colored shades delineate an example partition.}
        \label{fig:cut_dag}
    \end{subfigure}
       \caption{Finding cuts for a quantum circuit reduces to finding a partition of the quantum gates.
       The dashed qubit wires across different partitions are the cut edges.}
       \label{fig:tensor_graph_example}
\end{figure}

\subsubsection{Reducing to Graph Partition}
Quantum circuits can be depicted as directed acyclic graphs (DAGs).
Figure~\ref{fig:circuit} illustrates a $5$-qubit quantum circuit,
while Figure~\ref{fig:cut_dag} displays its corresponding DAG representation.
In the DAG, single-qubit gates are omitted to clearly highlight the circuit's topology.
The vertices in the DAG represent the input/output qubits and gates,
and the edges denote the connections between these qubits and gates through qubit wires.

Identifying cuts within this framework is akin to partitioning the two-qubit quantum gates.
The placement of cut edges precisely determines the boundaries of subcircuits.
In the context of DAG partitioning,
the focus is primarily on how vertices are interconnected;
therefore, single-qubit gates,
which do not influence the overall topology,
are excluded from the cut finding process.
These gates are instead assigned to the same subcircuit as their adjacent two-qubit gate.
Additionally, input and output qubit vertices should not be isolated in the partitioning process.

\subsubsection{Constraints}
QPU hardware is constrained by both the limited size and quality of qubits,
presenting two constraints.
The available sizes of QPUs set strict upper limits on the width of each subcircuit.
Furthermore, as Noisy Intermediate-Scale Quantum (NISQ) QPUs are error-prone,
they can support only a limited number of quantum gates before too many errors accumulate in each subcircuit.
Looking ahead, while fault-tolerant QPUs are expected to ease the limitations on subcircuit size significantly,
they will still be subject to circuit width constraints.

\subsubsection{Problem Statement}
TensorQC aims to identify cuts that optimize contraction costs while adhering to the stringent limitations imposed by QPU hardware.
To this end, we formalize the problem of finding cuts within a quantum circuit as follows:

Consider the DAG $g$ representing a quantum circuit,
the maximum number of qubits $w_{\max}$ that an available QPU can handle,
and the maximum number of gates $s_{\max}$ that the QPU can support.
Assuming that partitioning the quantum gates in $g$ results in $m$ distinct partitions,
the task is to find a partition that meets the following criteria:
\begin{enumerate}
    \item $0<w_i\leq w_{\max}, \forall i\in \{1,\ldots,n\}$, where $w_i$ is the number of qubits in each partition.
    \item $0<s_i\leq s_{\max}, \forall i\in \{1,\ldots,n\}$, where $s_i$ is the number of gates in each partition.
\end{enumerate}
The objective is to minimize the total runtime:
\begin{equation}
    T\equiv T_{QPU} + T_{classical}\label{eq:cut_objective}
\end{equation}