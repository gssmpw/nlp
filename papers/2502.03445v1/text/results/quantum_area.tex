\subsection{Quantum Area}
\begin{figure}[t]
    \centering
    \includegraphics[width=\linewidth]{figures/quantum_area_ratio.pdf}
    \caption{Quantum areas reductions of cutting versus not cutting for the same experiments in Figure~\ref{fig:total_runtime}.
    The plots show the quantum area of the largest subcircuit resulted from cutting as a percentage of the uncut benchmark.}
    \label{fig:quantum_area}
\end{figure}

Figure~\ref{fig:quantum_area} shows the quantum area reduction from cutting.
Quantum area for TensorQC is defined to be that of the largest subcircuit produced from cutting.
TensorQC only requires QPUs to support running the largest subcircuit at reasonable accuracies.
As a result, TensorQC reduces the QPU resource requirements by more than $10\times$ for various benchmarks.
QPU resource requirements reduction is at the expense of the extra cuts finding,
QPU executions of the subcircuits,
and tensor network contraction runtimes in Figure~\ref{fig:total_runtime}.

TensorQC opens up new potentials to run large scale quantum benchmarks using cutting and a set of less powerful QPUs.
Lower quantum area requirements translate to the ability to tolerate smaller and noisier QPUs.
Furthermore, in the fault-tolerant regime,
lower quantum areas translate to a looser requirement on the logical qubit error rate.
Depending on different quantum error correction solutions,
this implies much reduced physical qubit counts and error threshold requirements.