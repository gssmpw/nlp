\section{Related Work}\label{sec:related_work}
Many quantum compiler works exist to improve the performance of standalone QPUs to evaluate quantum circuits~\cite{nannicini2021optimal,tan2020optimality,murali2019noise,ding2020systematic,murali2020software,tang2024alpharouter}.
Quantum error correction is the key to build reliable QPUs~\cite{fowler2012surface,bravyi2012subsystem,javadi2017optimized,yoder2017surface,litinski2019game}.
QAOA uses classical computing to tune quantum circuit hyper-parameters to solve optimization problems~\cite{farhi2014quantum,tomesh2021coreset}.
However, they still rely entirely on QPUs to compute the quantum circuits.

Prior circuit cutting implementations~\cite{tang2021cutqc} rely on parallelization techniques for faster compute
while performing direct reconstruction of Equation~\ref{eq:reconstruction} with high overhead.
The various post-processing algorithms proposed in this paper go beyond what is possible from such techniques
and reduce the overhead itself.
Several small-scale demonstrations apply circuit cutting for chemical molecule simulations~\cite{eddins2022doubling},
variational quantum solvers~\cite{yuan2021quantum} and noise mitigation~\cite{basu2021qer}.

In addition, there are theoretical proposals to decompose a chemistry Hamiltonian on the algorithm level before translating it to a quantum circuit~\cite{eddins2022doubling}.
It is also theoretically possible to decompose $2$-qubit quantum gates~\cite{piveteau2023circuit},
instead of cutting quantum wires as in TensorQC.
There are proposals to manufacture distributed QPUs with quantum links~\cite{smith2022scaling,ang2024arquin}.
Quantum-linked QPUs essentially act as a large monolithic QPU and face even more hardware development challenges than building monolithic QPUs.
All these related works are completely orthogonal to TensorQC,
combining these techniques is a promising open question.