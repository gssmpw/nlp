%%
%% This is file `sample-sigconf-authordraft.tex',
%% generated with the docstrip utility.
%%
%% The original source files were:
%%
%% samples.dtx  (with options: `all,proceedings,bibtex,authordraft')
%% 
%% IMPORTANT NOTICE:
%% 
%% For the copyright see the source file.
%% 
%% Any modified versions of this file must be renamed
%% with new filenames distinct from sample-sigconf-authordraft.tex.
%% 
%% For distribution of the original source see the terms
%% for copying and modification in the file samples.dtx.
%% 
%% This generated file may be distributed as long as the
%% original source files, as listed above, are part of the
%% same distribution. (The sources need not necessarily be
%% in the same archive or directory.)
%%
%%
%% Commands for TeXCount
%TC:macro \cite [option:text,text]
%TC:macro \citep [option:text,text]
%TC:macro \citet [option:text,text]
%TC:envir table 0 1
%TC:envir table* 0 1
%TC:envir tabular [ignore] word
%TC:envir displaymath 0 word
%TC:envir math 0 word
%TC:envir comment 0 0
%%
%%
%% The first command in your LaTeX source must be the \documentclass
%% command.
%%
%% For submission and review of your manuscript please change the
%% command to \documentclass[manuscript, screen, review]{acmart}.
%%
%% When submitting camera ready or to TAPS, please change the command
%% to \documentclass[sigconf]{acmart} or whichever template is required
%% for your publication.
%%
%%
\documentclass[sigconf]{acmart}
% \documentclass[sigconf,authordraft]{acmart}
% \documentclass[manuscript,review,anonymous]{acmart}
% \usepackage{amsmath}
\usepackage{algorithm}
\usepackage{algorithmic}
% \usepackage{tabularx}
% \usepackage{array}
% \usepackage{placeins}
% \usepackage{booktabs}
\usepackage{enumitem}
\usepackage{graphicx}
\usepackage{subfig}



\usepackage{xcolor}
% \usepackage{hyperref}
\definecolor{REVISEcolor}{HTML}{0000FF}
% \newcommand{\REVISE}[1]{\textcolor{REVISEcolor}{#1}}
\newcommand{\REVISE}[0]{}

%% \BibTeX command to typeset BibTeX logo in the docs
\AtBeginDocument{%
  \providecommand\BibTeX{{%
    Bib\TeX}}}

%% Rights management information.  This information is sent to you
%% when you complete the rights form.  These commands have SAMPLE
%% values in them; it is your responsibility as an author to replace
%% the commands and values with those provided to you when you
%% complete the rights form.

\copyrightyear{2025}
\acmYear{2025}
\setcopyright{acmlicensed}\acmConference[CHI '25]{CHI Conference on Human Factors in Computing Systems}{April 26-May 1, 2025}{Yokohama, Japan}
\acmBooktitle{CHI Conference on Human Factors in Computing Systems (CHI '25), April 26-May 1, 2025, Yokohama, Japan}
\acmDOI{10.1145/3706598.3714176}
\acmISBN{979-8-4007-1394-1/25/04}


%%
%% Submission ID.
%% Use this when submitting an article to a sponsored event. You'll
%% receive a unique submission ID from the organizers
%% of the event, and this ID should be used as the parameter to this command.
%%\acmSubmissionID{123-A56-BU3}

%%
%% For managing citations, it is recommended to use bibliography
%% files in BibTeX format.
%%
%% You can then either use BibTeX with the ACM-Reference-Format style,
%% or BibLaTeX with the acmnumeric or acmauthoryear sytles, that include
%% support for advanced citation of software artefact from the
%% biblatex-software package, also separately available on CTAN.
%%
%% Look at the sample-*-biblatex.tex files for templates showcasing
%% the biblatex styles.
%%

%%
%% The majority of ACM publications use numbered citations and
%% references.  The command \citestyle{authoryear} switches to the
%% "author year" style.
%%
%% If you are preparing content for an event
%% sponsored by ACM SIGGRAPH, you must use the "author year" style of
%% citations and references.
%% Uncommenting
%% the next command will enable that style.
%%\citestyle{acmauthoryear}


%%
%% end of the preamble, start of the body of the document source.
\begin{document}

%%
%% The "title" command has an optional parameter,
%% allowing the author to define a "short title" to be used in page headers.
\title{CalliSense: An Interactive Educational Tool for Process-based Learning in Chinese Calligraphy}

%%
%% The "author" command and its associated commands are used to define
%% the authors and their affiliations.
%% Of note is the shared affiliation of the first two authors, and the
%% "authornote" and "authornotemark" commands
%% used to denote shared contribution to the research.

\author{Xinya Gong} 
\orcid{0009-0005-6414-9351} 
\email{gongxinya123@gmail.com} 
\affiliation{% 
  \institution{Department of Computer Science and Engineering, Southern University of Science and Technology} 
  \country{China} 
}
\authornote{Equal contribution.}

\author{Wenhui Tao} 
\orcid{0009-0003-2645-6444} 
\email{wenhui9703@gmail.com} 
\affiliation{% 
  \institution{Department of Computer Science and Engineering, Southern University of Science and Technology} 
  \country{China} 
}
\authornotemark[1] % share the authornote

\author{Yuxin Ma} 
\orcid{0000-0003-0484-668X} 
\email{mayx@sustech.edu.cn} 
\affiliation{% 
  \institution{Department of Computer Science and Engineering, Southern University of Science and Technology} 
  \country{China} 
}
\authornote{Corresponding author.}

%%
%% By default, the full list of authors will be used in the page
%% headers. Often, this list is too long, and will overlap
%% other information printed in the page headers. This command allows
%% the author to define a more concise list
%% of authors' names for this purpose.
\renewcommand{\shortauthors}{Trovato et al.}

%%
%% The abstract is a short summary of the work to be presented in the
%% article.
\begin{abstract}
  Process-based learning is crucial for the transmission of intangible cultural heritage, especially in complex arts like Chinese calligraphy, where mastering techniques cannot be achieved by merely observing the final work. To explore the challenges faced in calligraphy heritage transmission, we conducted semi-structured interviews (N=8) as a formative study. Our findings indicate that the lack of calligraphy instructors and tools makes it difficult for students to master brush techniques, and teachers struggle to convey the intricate details and rhythm of brushwork. To address this, we collaborated with calligraphy instructors to develop an educational tool that integrates writing process capture and visualization, showcasing the writing rhythm, hand force, and brush posture. Through empirical studies conducted in multiple teaching workshops, we evaluated the system's effectiveness with teachers (N=4) and students (N=12). The results show that the tool significantly enhances teaching efficiency and aids learners in better understanding brush techniques.
\end{abstract}

%%
%% The code below is generated by the tool at http://dl.acm.org/ccs.cfm.
%% Please copy and paste the code instead of the example below.
%%
\begin{CCSXML}
<ccs2012>
   <concept>
       <concept_id>10003120.10003121.10003129</concept_id>
       <concept_desc>Human-centered computing~Interactive systems and tools</concept_desc>
       <concept_significance>500</concept_significance>
       </concept>
 </ccs2012>
\end{CCSXML}

\ccsdesc[500]{Human-centered computing~Interactive systems and tools}

%%
%% Keywords. The author(s) should pick words that accurately describe
%% the work being presented. Separate the keywords with commas.
\keywords{Chinese Calligraphy, learning system, visualization}
%% A "teaser" image appears between the author and affiliation
%% information and the body of the document, and typically spans the
%% page.
\begin{teaserfigure}
  \includegraphics[width=\textwidth]{design_fig/teaser.jpg}
  \caption{An overview the visual teaching assistant system for the entire calligraphy process begins with the user writing in a real scenario, where sensors and cameras capture the entire writing process. The captured strokes will then undergo a series of data processing steps, including temporal alignment, stroke segmentation, skeleton extraction, etc., and the captured writing parameters will correspond one-to-one with their own dots. CalliSense will visualize these parameters and has comparison features for both students and teachers, ultimately helping students enhance their awareness of brushwork and understand the details of stroke execution.}
  % \Description{Enjoying the baseball game from the third-base seats. Ichiro Suzuki preparing to bat.}
  \label{fig:teaser}
\end{teaserfigure}

% \received{20 February 2007}
% \received[revised]{12 March 2009}
% \received[accepted]{5 June 2009}

%%
%% This command processes the author and affiliation and title
%% information and builds the first part of the formatted document.
\maketitle
\section{Introduction}

Video generation has garnered significant attention owing to its transformative potential across a wide range of applications, such media content creation~\citep{polyak2024movie}, advertising~\citep{zhang2024virbo,bacher2021advert}, video games~\citep{yang2024playable,valevski2024diffusion, oasis2024}, and world model simulators~\citep{ha2018world, videoworldsimulators2024, agarwal2025cosmos}. Benefiting from advanced generative algorithms~\citep{goodfellow2014generative, ho2020denoising, liu2023flow, lipman2023flow}, scalable model architectures~\citep{vaswani2017attention, peebles2023scalable}, vast amounts of internet-sourced data~\citep{chen2024panda, nan2024openvid, ju2024miradata}, and ongoing expansion of computing capabilities~\citep{nvidia2022h100, nvidia2023dgxgh200, nvidia2024h200nvl}, remarkable advancements have been achieved in the field of video generation~\citep{ho2022video, ho2022imagen, singer2023makeavideo, blattmann2023align, videoworldsimulators2024, kuaishou2024klingai, yang2024cogvideox, jin2024pyramidal, polyak2024movie, kong2024hunyuanvideo, ji2024prompt}.


In this work, we present \textbf{\ours}, a family of rectified flow~\citep{lipman2023flow, liu2023flow} transformer models designed for joint image and video generation, establishing a pathway toward industry-grade performance. This report centers on four key components: data curation, model architecture design, flow formulation, and training infrastructure optimization—each rigorously refined to meet the demands of high-quality, large-scale video generation.


\begin{figure}[ht]
    \centering
    \begin{subfigure}[b]{0.82\linewidth}
        \centering
        \includegraphics[width=\linewidth]{figures/t2i_1024.pdf}
        \caption{Text-to-Image Samples}\label{fig:main-demo-t2i}
    \end{subfigure}
    \vfill
    \begin{subfigure}[b]{0.82\linewidth}
        \centering
        \includegraphics[width=\linewidth]{figures/t2v_samples.pdf}
        \caption{Text-to-Video Samples}\label{fig:main-demo-t2v}
    \end{subfigure}
\caption{\textbf{Generated samples from \ours.} Key components are highlighted in \textcolor{red}{\textbf{RED}}.}\label{fig:main-demo}
\end{figure}


First, we present a comprehensive data processing pipeline designed to construct large-scale, high-quality image and video-text datasets. The pipeline integrates multiple advanced techniques, including video and image filtering based on aesthetic scores, OCR-driven content analysis, and subjective evaluations, to ensure exceptional visual and contextual quality. Furthermore, we employ multimodal large language models~(MLLMs)~\citep{yuan2025tarsier2} to generate dense and contextually aligned captions, which are subsequently refined using an additional large language model~(LLM)~\citep{yang2024qwen2} to enhance their accuracy, fluency, and descriptive richness. As a result, we have curated a robust training dataset comprising approximately 36M video-text pairs and 160M image-text pairs, which are proven sufficient for training industry-level generative models.

Secondly, we take a pioneering step by applying rectified flow formulation~\citep{lipman2023flow} for joint image and video generation, implemented through the \ours model family, which comprises Transformer architectures with 2B and 8B parameters. At its core, the \ours framework employs a 3D joint image-video variational autoencoder (VAE) to compress image and video inputs into a shared latent space, facilitating unified representation. This shared latent space is coupled with a full-attention~\citep{vaswani2017attention} mechanism, enabling seamless joint training of image and video. This architecture delivers high-quality, coherent outputs across both images and videos, establishing a unified framework for visual generation tasks.


Furthermore, to support the training of \ours at scale, we have developed a robust infrastructure tailored for large-scale model training. Our approach incorporates advanced parallelism strategies~\citep{jacobs2023deepspeed, pytorch_fsdp} to manage memory efficiently during long-context training. Additionally, we employ ByteCheckpoint~\citep{wan2024bytecheckpoint} for high-performance checkpointing and integrate fault-tolerant mechanisms from MegaScale~\citep{jiang2024megascale} to ensure stability and scalability across large GPU clusters. These optimizations enable \ours to handle the computational and data challenges of generative modeling with exceptional efficiency and reliability.


We evaluate \ours on both text-to-image and text-to-video benchmarks to highlight its competitive advantages. For text-to-image generation, \ours-T2I demonstrates strong performance across multiple benchmarks, including T2I-CompBench~\citep{huang2023t2i-compbench}, GenEval~\citep{ghosh2024geneval}, and DPG-Bench~\citep{hu2024ella_dbgbench}, excelling in both visual quality and text-image alignment. In text-to-video benchmarks, \ours-T2V achieves state-of-the-art performance on the UCF-101~\citep{ucf101} zero-shot generation task. Additionally, \ours-T2V attains an impressive score of \textbf{84.85} on VBench~\citep{huang2024vbench}, securing the top position on the leaderboard (as of 2025-01-25) and surpassing several leading commercial text-to-video models. Qualitative results, illustrated in \Cref{fig:main-demo}, further demonstrate the superior quality of the generated media samples. These findings underscore \ours's effectiveness in multi-modal generation and its potential as a high-performing solution for both research and commercial applications.
\section{Related Work}

\subsection{Large 3D Reconstruction Models}
Recently, generalized feed-forward models for 3D reconstruction from sparse input views have garnered considerable attention due to their applicability in heavily under-constrained scenarios. The Large Reconstruction Model (LRM)~\cite{hong2023lrm} uses a transformer-based encoder-decoder pipeline to infer a NeRF reconstruction from just a single image. Newer iterations have shifted the focus towards generating 3D Gaussian representations from four input images~\cite{tang2025lgm, xu2024grm, zhang2025gslrm, charatan2024pixelsplat, chen2025mvsplat, liu2025mvsgaussian}, showing remarkable novel view synthesis results. The paradigm of transformer-based sparse 3D reconstruction has also successfully been applied to lifting monocular videos to 4D~\cite{ren2024l4gm}. \\
Yet, none of the existing works in the domain have studied the use-case of inferring \textit{animatable} 3D representations from sparse input images, which is the focus of our work. To this end, we build on top of the Large Gaussian Reconstruction Model (GRM)~\cite{xu2024grm}.

\subsection{3D-aware Portrait Animation}
A different line of work focuses on animating portraits in a 3D-aware manner.
MegaPortraits~\cite{drobyshev2022megaportraits} builds a 3D Volume given a source and driving image, and renders the animated source actor via orthographic projection with subsequent 2D neural rendering.
3D morphable models (3DMMs)~\cite{blanz19993dmm} are extensively used to obtain more interpretable control over the portrait animation. For example, StyleRig~\cite{tewari2020stylerig} demonstrates how a 3DMM can be used to control the data generated from a pre-trained StyleGAN~\cite{karras2019stylegan} network. ROME~\cite{khakhulin2022rome} predicts vertex offsets and texture of a FLAME~\cite{li2017flame} mesh from the input image.
A TriPlane representation is inferred and animated via FLAME~\cite{li2017flame} in multiple methods like Portrait4D~\cite{deng2024portrait4d}, Portrait4D-v2~\cite{deng2024portrait4dv2}, and GPAvatar~\cite{chu2024gpavatar}.
Others, such as VOODOO 3D~\cite{tran2024voodoo3d} and VOODOO XP~\cite{tran2024voodooxp}, learn their own expression encoder to drive the source person in a more detailed manner. \\
All of the aforementioned methods require nothing more than a single image of a person to animate it. This allows them to train on large monocular video datasets to infer a very generic motion prior that even translates to paintings or cartoon characters. However, due to their task formulation, these methods mostly focus on image synthesis from a frontal camera, often trading 3D consistency for better image quality by using 2D screen-space neural renderers. In contrast, our work aims to produce a truthful and complete 3D avatar representation from the input images that can be viewed from any angle.  

\subsection{Photo-realistic 3D Face Models}
The increasing availability of large-scale multi-view face datasets~\cite{kirschstein2023nersemble, ava256, pan2024renderme360, yang2020facescape} has enabled building photo-realistic 3D face models that learn a detailed prior over both geometry and appearance of human faces. HeadNeRF~\cite{hong2022headnerf} conditions a Neural Radiance Field (NeRF)~\cite{mildenhall2021nerf} on identity, expression, albedo, and illumination codes. VRMM~\cite{yang2024vrmm} builds a high-quality and relightable 3D face model using volumetric primitives~\cite{lombardi2021mvp}. One2Avatar~\cite{yu2024one2avatar} extends a 3DMM by anchoring a radiance field to its surface. More recently, GPHM~\cite{xu2025gphm} and HeadGAP~\cite{zheng2024headgap} have adopted 3D Gaussians to build a photo-realistic 3D face model. \\
Photo-realistic 3D face models learn a powerful prior over human facial appearance and geometry, which can be fitted to a single or multiple images of a person, effectively inferring a 3D head avatar. However, the fitting procedure itself is non-trivial and often requires expensive test-time optimization, impeding casual use-cases on consumer-grade devices. While this limitation may be circumvented by learning a generalized encoder that maps images into the 3D face model's latent space, another fundamental limitation remains. Even with more multi-view face datasets being published, the number of available training subjects rarely exceeds the thousands, making it hard to truly learn the full distibution of human facial appearance. Instead, our approach avoids generalizing over the identity axis by conditioning on some images of a person, and only generalizes over the expression axis for which plenty of data is available. 

A similar motivation has inspired recent work on codec avatars where a generalized network infers an animatable 3D representation given a registered mesh of a person~\cite{cao2022authentic, li2024uravatar}.
The resulting avatars exhibit excellent quality at the cost of several minutes of video capture per subject and expensive test-time optimization.
For example, URAvatar~\cite{li2024uravatar} finetunes their network on the given video recording for 3 hours on 8 A100 GPUs, making inference on consumer-grade devices impossible. In contrast, our approach directly regresses the final 3D head avatar from just four input images without the need for expensive test-time fine-tuning.


\section{Problem Formulation}

We illustrate our system's motivation through a scenario that highlights how data workers monitor global data states during EDA using a basic Jupyter Notebook, along with the associated challenges they encounter. 
Following this, we review related studies to assess the progress made and the remaining gaps in current tools.
From these insights, we derive design requirements to guide the development of our system.

\subsection{Motivating Scenario}
\label{sec:motivate_scenario}

The scenario demonstrates the challenges encountered while writing codes from scratch for EDA and reviewing an existing script to identify data errors and bugs.

\begin{figure*}[!htb]
    \centering
    \includegraphics[width=\textwidth]{figures/motivating_scenario.png}
    \caption{Motivating scenario. The EDA process will encounter a series of challenges that require tedious efforts on programming when conducting data profiling (A), data cleaning (B), data analysis and error discovery (C), and locating error codes (D).}
    \label{fig:motivating_scenario}
\end{figure*}

Alice, a data analyst, is tasked with cleaning and analyzing a dataset on Google Play Store Apps. She begins by importing the libraries and loading the dataset. 

\textbf{Profiling and data cleaning. } 
After loading the dataset, Alice manually prints various profiling information from the data table, such as column data types and null value counts \textbf{(C1)}. 
Based on the information, she identifies several columns that require further processing. 
For example, string columns like ``Installs'' and ``Size'' should be converted to numerical formats. 
Alice writes data transformation scripts to adjust these columns.
To verify the changes, she visualizes the distribution of the transformed columns using visualization libraries, such as matploblib and seaborn, and prints the transformed tables. 
Since multiple columns require transformation, and each transformation may involve several steps, this process quickly becomes repetitive and labor-intensive, as she must repeatedly copy, edit, and run visualization scripts to monitor the results \textbf{(C1)}.
Additionally, inserting visualization scripts between transformation steps increases the overall length of the notebook, making it cumbersome to navigate through the notebook and track data changes \textbf{(C2)}. 
Alice has to scroll through the lengthy notebook to check and compare the results.


\textbf{Discover, locate and fix the error. } 
After cleaning the data, Alice begins to explore multiple data patterns by grouping data and creating visualizations. 
However, when she visualizes the mean ratings of free versus paid apps, she notices an unexpected zero value in the ``Type'' column. 
Alice wonders where this issue originated and which line of scripts may have caused it. 
She first searches and confirms there are no direct transformation scripts applied to the ``Type'' column. 
Next, she tries to scroll through the notebook to identify the intermediate data table that preceded the current one. 
However, it is difficult to find and recognize due to the lack of clear indications \textbf{(C3)}.
Moreover, as the intermediate data tables have been overwritten, she can not create visualizations on them to revisit their features on the ``Type'' column \textbf{(C4)}. 
The only visualizations available are those created earlier, which do not include the ``Type'' column. 
Therefore, Alice has to re-run each cell and manually add the visualization for the ``Type'' distribution until she identifies the code line responsible \textbf{(C5)}, which encounters multiple visualizations to insert and compare \textbf{(C2)}. 
Finally, after checking the distribution differences step by step, she discovers that the issue is caused by a transformation applied to the ``Size'' column, which mistakenly set many data rows to completely zero.
After locating the problematic code, Alice fixes it and reruns the subsequent cells to verify the correction. 
However, as the cells are re-executed, the previous visualizations are overwritten, forcing her to compare the results mentally, which makes it difficult to track the changes before and after the fix \textbf{(C4)}.

In summary, the traditional workflow to monitor the global data states during EDA encounters the following challenges:
\begin{enumerate}[label={\textbf{C\arabic*.}}]

\item \textbf{Tedious manual programming for data visualization. } Users have to manually create and update visualizations scripts, which is both time-consuming and prone to errors.

\item \textbf{Cognitive overload from comparison among cells.} To locate the special line of code that causes the important data changes, users must compare visualizations at each step of the analysis, leading to an overwhelming amount of information to remember and assess.

\item \textbf{Lack of clear indication in the relationships between data tables.} Users struggle to identify where a current data table originated or how it was transformed throughout the analysis, as these relationships are buried in the script and not visually represented.

\item \textbf{Loss of execution history and previous data states.} During EDA, previous states of data may be overwritten, and the history of re-executed cells may be obscured, making it difficult to revisit and compare data states over time.

\item \textbf{Repeatedly copying and editing of the same visualization codes.} To monitor the data states in specific columns, users repeatedly copy and paste the same visualization code, which is inefficient and redundant.

\end{enumerate}

\subsection{Progress and Gaps in Current Tools}
Several studies have attempted to enhance EDA by providing visualizations that help users better understand data states (\textbf{C1}). 
The most notable examples are LUX~\cite{lee2021lux}, Solas~\cite{epperson2022leveraging}, and AutoProfiler~\cite{autoprofiler}.
LUX and Solas recommend visualizations based on data insights and provenance, offering a quick overview of patterns and trends that align with the user's operations and specified intents. 
AutoProfiler focuses on displaying basic information about the data, such as the distribution and profiling of each column. 
We consider both data insights and profiling valuable and incorporate these factors into our chart recommendations.

On the other hand, both LUX and AutoProfiler primarily focus on local data states and lack support for understanding the global state of the data (\textbf{C2-C5}). 
LUX requires users to write specific visualization intent codes to display the current state of a data table. 
To monitor global state changes, users must repeatedly insert the codes, which only simplifies the process of writing visualization codes compared to traditional workflows. 
AutoProfiler eliminates the need for code insertion by automatically updating visualizations and profiling to reflect the most recent state. 
However, it only shows the latest state of the data and does not support revisiting or comparing previous states.
In this study, we enable users to use recommended charts as ``sight glasses'' to trace the global data flow, enabling users to track and understand how data evolves throughout the entire EDA process.

\subsection{Design Requirements}

Based on the challenges identified in the motivating scenario and the progress and gaps in current tools, we derive the following design requirements to guide our system's development:

\begin{enumerate}[label={\textbf{R\arabic*.}}]

    \item \textbf{Effective Chart Recommendation.}
    The system should recommend charts that effectively represent the key aspects of the data, considering both data insights and profiling information. These recommendations should help users quickly grasp the state of their data at any given point, reducing the need for manual chart creation.

    \item \textbf{On-Demand Tracing of Changed Related Data.} The system should allow users to trace the data flow on-demand, enabling them to focus specifically on the related intermediate data tables where changes have occurred. This targeted approach helps users concentrate on critical transformations without being overwhelmed by irrelevant information.
    
    \item \textbf{Representation of Data Relationships.}
    Given the complexity and number of intermediate data tables in a notebook, it is essential to provide visual cues that clearly indicate the flow between these data tables. 
    This flow should include information about the transformations applied, such as \code{df\_2} being generated from \code{df\_1} through filtering. These visual hints will help users quickly understand how data evolves throughout the notebook.
    
    \item \textbf{Support for Revisiting Past Intermediate Tables.} The system should provide mechanisms to revisit past intermediate tables, even if they have been overwritten or hidden due to cell re-execution. This feature is crucial for allowing users to compare historical data states with current ones, ensuring that they can accurately track changes and understand the impact of each transformation. 

    \item \textbf{Consistent Tracing of Charts.} 
    To facilitate a comprehensive understanding of how data changes over time, the system should support consistent tracing of data states. This involves tracking the changes in specific charts or visual representations as they reflect the underlying data across multiple transformations. 
\end{enumerate}

\section{Data Collection and Processing}
Considering the need for a complete record of both the strokes and the brushstoke process \textbf{(DC1, DC4)}, we design a data collection methodology that utilizes cameras to capture the brushstoke process and sensors to collect brush posture and finger pressure \textbf{(DC2)}. Algorithmic techniques are then applied to align the brush parameters with the stroke positions \textbf{(DC2)}. After discussions with experts, the specific collection parameters and their significance were finalized, as shown in the Table \ref{tab:calligraphy-measurements}. The overall data workflow of the system is illustrated in Figure \ref{fig:overview}, with the detailed design and corresponding descriptions discussed in the following sections.

















\subsection{Image Data Collection}
In this step, the goal is to capture raw video footage of the writing process, laying the groundwork for subsequent data processing. 
\subsubsection{Equipment Overview}
To minimize user reliance on specific equipment, we use widely available smartphones as the primary data collection tool. We have set up two recording devices, referred to as \textbf{Device A} and \textbf{Device B} (Figure \ref{fig:device A and B}). \textbf{Device A} is placed parallel to the table surface to monitor the contact between the brush tip and the paper, detecting writing motions. \textbf{Device B} is fixed above the table, covering the entire writing area to capture the full brushstroke process.

\begin{figure}[t!]
    \centering
    \includegraphics[width=0.48\textwidth]{design_fig/whole_system.png}
    \caption{The data collection setup includes pressure sensors, inertial sensors, and cameras. The sensors are connected via Arduino to capture data, which is then synchronized and processed on the computer.}
    \label{fig:whole system}
\end{figure}

\begin{figure}[t!]
    \centering
    \includegraphics[width=0.4\textwidth]{design_fig/Device_A_and_B.png}
    \caption{Capture the brushstoke process using two smartphones from different angles: Device A and Device B}
    \label{fig:device A and B}
\end{figure}

\subsubsection{Image Preprocessing}
After \textbf{Device B} recording the whole brushstoke process, the videos are sent to a data preprocessing stage. To correct the perspective distortion captured by the overhead camera, we use the perspective transformation algorithm from the OpenCV library~\cite{opencv_library}. This algorithm corrects image distortion by defining source and destination points, calculating a perspective transformation matrix, and remapping the image to a top-down view (Figure \ref{fig:view_correct}). This step ensures that the writing trajectory is presented at the correct scale and angle in the video, accurately restoring the original writing result during subsequent stroke visualization \textbf{(DC1)}. 

\begin{figure}[htbp]
    \centering
    \includegraphics[width=0.4\textwidth]{design_fig/image_correct.png}
    \caption{Stroke Images Before and After Correction}
    \label{fig:view_correct}
\end{figure}



\subsection{Sensor Data Collection}
In this step, the inertial sensor and pressure sensor are used to capture the brush's posture and the hand's applied force, respectively (Figure \ref{fig:whole system}). 
 
\subsubsection{Brush posture collection}
The MPU6050 inertial sensor was used to measure the brush's tilt and rotation, and it was attached to the brush head. The sensor's built-in Digital Motion Processor (DMP) module can directly process data from the accelerometer and gyroscope to calculate the object's orientation. This not only reduces the computational load on external processors but also provides real-time orientation data~\cite{widagdo2017limb}. 

\subsubsection{Finger force collection}
Previous research inferred hand force by detecting arm muscle activity~\cite{10269740, 10340786}, but we required more precise finger force data due to the critical role of the fingers in writing. Therefore, we selected a circular resistive pressure sensor (short tail RP-C7.6-ST-LF2), which offers high sensitivity, small size, quick response, and low cost. Since the brushstoke process requires rotating the brush, we opted to attach the sensors to the fingers rather than directly to the brush shaft. This allows users to practice with any standard brush without the need for modifications. We initially considered wrapping a strip-shaped sensor around the shaft, but due to the slender nature of the brush handle, the excessive curvature of the sensor could lead to measurement errors.
\begin{figure}[t!]
  \centering
  \includegraphics[width=0.4\textwidth]{design_fig/finger.png}
  \caption{Design sketch: the pressure each finger applied to the brush}
  \label{fig: finger}
\end{figure}
Based on the posture of holding the brush handle, we attached four sensors to the thumb, index finger, middle finger, and ring finger using finger sleeves to measure the pressure each finger applied to the brush handle (Figure~\ref{fig: finger}). However, during subsequent iterations, expert suggested that the force exerted by the hand is primarily applied to the brush shaft, indirectly altering the contact between the brush tip and the paper, thus affecting the quality of the strokes. Such details are usually not emphasized in typical teaching settings, where only the overall force application needs to be monitored. Additionally, monitoring the force of multiple fingers might confuse beginners, leading them to believe they need to intentionally replicate different force applications from all four fingers, thereby creating unnecessary learning challenges.

As a result, we simplified the original four sensors to a single one, placed on the thumb. The thumb, being positioned opposite the other three fingers, serves as the primary point of force during writing and can largely represent the overall force exerted by the hand on the brush.


\subsection{Time Alignment and Skeleton Extraction}
 \begin{figure*}[htbp]
    \centering
    \includegraphics[width=0.9\linewidth]{design_fig/iswriting.png}
    \caption{Determine if the brush is in writing mode based on the distance between the brush tip and the paper}
    \label{fig: Pen up and Down}
\end{figure*}
\subsubsection{Time Alignment}
In order to match the collected strokes with the sensor data, we need to assign a timestamp to each stroke point and align the stroke's time labels with the sensor data. The most straightforward approach is to record the time when the ink appears, thus corresponding the ink with the timeline, as shown in Figure~\ref{fig: TimeAlignment}.

\begin{figure}[H]
    \centering
    \includegraphics[width=0.6\linewidth]{design_fig/timeallign.png}
    \caption{In the figure, each point of the handwriting is annotated with the corresponding timestamp information. The entire time frame is assumed to be 500 units, providing context for the timing of each point. }
    \label{fig: TimeAlignment}
\end{figure}

Once the stroke's timestamps are assigned, we perform stroke segmentation to analyze the writing process of each stroke in detail \textbf{(DC2)}. Traditional stroke segmentation methods typically handle flat Chinese character images~\cite{10.1145/3548608.3559239, YAO2004631}. However, the rules for segmenting strokes vary between different calligraphy styles. For example, in the character ``Min'', the horizontal and vertical strokes in Clerical Script are made with two separate strokes, while in Regular Script, they are completed in one stroke (Figure \ref{fig: image_of_min_charactor}a).

To enhance the system's adaptability to different calligraphy styles, we adopt an image-based method, segmenting strokes based on the start and end of each brushstroke in the writing process. Specifically, a side-view camera (Device A) combined with OpenCV's CSRT-based tracker is used to detect the contact between the brush tip and the paper (Figure~\ref{fig: Pen up and Down}). The time points of each brush's lifting and landing, combined with the stroke's time data, allow for stroke-by-stroke classification while maintaining the stroke order (Figure~\ref{fig: strokepointsgroup}). Taking the character "Yong" as an example, it is divided into five strokes based on the state of contact between the brush and the paper (Figure \ref{fig: image_of_min_charactor}b).



% \begin{figure}[htbp]
%     \centering
%     \includegraphics[width=0.8\linewidth]{design_fig/Min.png}
%     \caption{A Chinese character can have different stroke segmentation methods.}
%     \label{fig: image_of_min_charactor}
% \end{figure}


% \begin{figure}[H]
%     \centering
%     \includegraphics[width=0.8\linewidth]{group.png}
%     \caption{Pixel points are assigned to brush lift/drop intervals, isolating a stroke in the character ``Yong''.}
%     \label{fig: strokepointsgroup}
% \end{figure}

% \begin{figure}[H]
%     \centering
%     \includegraphics[width=0.9\linewidth]{design_fig/Stroke_splitting.png}
%     \caption{The character ``Yong'' is segmented into different individual strokes.}
%     \label{fig: Stroke splitting}
% \end{figure}

\begin{figure}[!htbp]
\centering
    \subfloat[A Chinese character can have different stroke segmentation methods.]{ \includegraphics[width=0.8\linewidth]{design_fig/Min.png}}\\
    \subfloat[The character ``Yong'' is segmented into different individual strokes.]{\includegraphics[width=0.9\linewidth]{design_fig/Stroke_splitting.png}}
    \caption{Segmentation of Calligraphy Strokes.}
    \label{fig: image_of_min_charactor}
\end{figure}


\begin{figure}[!htbp]
    \centering
    \includegraphics[width=0.8\linewidth]{group.png}
    \caption{Pixel points are assigned to brush lift/drop intervals, isolating a stroke in the character ``Yong''.}
    \label{fig: strokepointsgroup}
\end{figure}





\subsubsection{Skeleton Extraction}
Chinese calligraphy is inherently an art of lines. To clearly highlight key positions in visualizations for students, extracting the skeleton of Chinese characters became a key focus of our work. 
We apply a centroid-based algorithm on the incremental ink deposition to extract the skeleton (Figure \ref{fig: Ink increment}). By connecting centroids over time, a smooth skeleton structure is obtained (Figure \ref{fig: stroke2skeleton}). Ultimately, the writing speed is represented by the offset of the ink's center point in each time frame.

\begin{figure}[H]
    \centering
    \includegraphics[width=0.4\textwidth]{design_fig/Ink_increment.png}
    \caption{By tracking the added ink traces, time frames are assigned to different positions of the character. The pixel positions from each frame are averaged to generate central axis points, which are then connected to form the character's central axis line.}
    \label{fig: Ink increment}
\end{figure}



\begin{figure}[H]
    \centering
    \includegraphics[width=0.9\linewidth]{design_fig/stroke2skeleton.png}
    \caption{Skeleton extraction of each stroke in the character ``Yong''}
    \label{fig: stroke2skeleton}
\end{figure}





















\section{Visualization Design}
Our visualization framework sequentially covers glyph observation, rhythm observation, and detailed stroke analysis. During the iterative design process, expert feedback was incorporated, emphasizing that calligraphy instruction typically begins with an overall view of the character's form.
 Therefore, the initial interface displays only the character's overall structure, concealing brushwork details (Figure \ref{fig:Structure Analysis}). This design encourages learners to first understand the overall composition and identify gaps between their work and the teacher's, allowing for targeted improvement in the next stage. This approach aligns with the cognitive process of traditional Chinese calligraphy training~\cite{dong2008creation}. Starting with glyph critiques also helps beginners unfamiliar with brushwork techniques to build confidence in a domain they may find challenging.

During the iterative process, an interesting phenomenon was discovered: rhythm changes in dance and music were frequently used by calligraphy experts to elucidate the importance of varying writing pressure and speed. Calligraphy and dance are often compared in studies due to their similar rhythmic qualities~\cite{szeto2010calligraphic}. Building on the conclusions from the previous formative study\textbf{ (DC3)}, an overview of writing rhythm is introduced after the character structure analysis. Users can then select specific strokes and proceed to the next interface to analyze individual lines (Figure \ref{fig:Rhythm and Stroke Analysis}). This design also follows the standard framework of information visualization: (a) provide an overview first, (b) allow zoom and filter, and (c) offer details on demand\cite{shneiderman2003eyes}. Overall, a comparison between each part for the teacher and the student was provided, enabling the student to identify where the issues with posture are present \textbf{(DC5)}.

\subsection{Glyph Comparison}

\begin{figure}[t]
    \centering
    \includegraphics[width=\linewidth]{design_fig/interface/Structure_Analysis.jpg}
    \caption{Step 1: Character Shape Comparison. The left side shows the teacher's writing, while the right side displays the student's work. Students can use the teacher's writing as a reference to identify structural issues in their own writing. Additionally, they can choose to use two types of guidelines—``Structural Boundary'' and ``Form Boundary''—to assist in their assessment.}
    \label{fig:Structure Analysis}
\end{figure}

To provide a more intuitive comparison between the teacher and student, we added the commonly used guiding grid – the ``Mi Zi Ge'' grid\cite{chinese_calligraphy_grids}. The grid not only serves as a reference but also helps observe the relative positions of the strokes.

Additionally, two optional boundary boxes were incorporated (Figure \ref{fig:bounding box}). The first boundary connects the four extremities of the character, showing the relative positions of the strokes. The second boundary uses these four points as edges to form a grid, allowing for a clearer view of the character's proportions and its central position. With these tools, the teacher and student can drag and compare characters to further analyze differences in writing techniques (Figure \ref{fig:drag}).

\begin{figure}[H]
    \centering
    \includegraphics[width=\linewidth]{design_fig/interface/drag.png}
    \caption{Compare the structure of students' and teachers' handwriting through dragging.}
    \label{fig:drag}
\end{figure}

% \vspace{-5mm}

\begin{figure}[H]
    \centering

    \includegraphics[width=\linewidth]{design_fig/interface/bounding_box.png}
    \caption{Two types of bounding boxes observe the structure of Chinese characters from different perspectives.}
    \label{fig:bounding box}
\end{figure}

% \vspace{-5mm}


\begin{figure}[htbp]
    \centering
    \includegraphics[width=\linewidth]{design_fig/interface/Rhythm_and_Stroke_Analysis.jpg}
    \caption{Step 2 and Step 3: Rhythm Analysis(Left) and Stroke Analysis (Right). These steps allow for the analysis of the speed and force variations throughout the writing of a complete character on the left, and the detailed brush technique used for a specific stroke on the right.}
    \label{fig:Rhythm and Stroke Analysis}
\end{figure}

\subsection{Rhythm Comparison}

To minimize confusion, finger pressure is represented using a heatmap, while speed is color-coded on the skeleton (Figure \ref{fig: Rhythm}) ~\cite{howe1983temporal}. In Figure \ref{fig: Rhythm}, the teacher's finger pressure varies throughout the stroke, typically applying force at the start and turning points. In contrast, the student's grip remained tense throughout the earlier strokes, only showing improvement in the final stroke. Similarly, speed can be analyzed in this way.
\begin{figure}[H]
    \centering
    \includegraphics[width=0.5\textwidth]{design_fig/interface/Rhythm.png}
    \caption{In the rhythm view, observe the changes in the teacher's and student's finger pressure (left) and speed (right) throughout the entire character.}
    \label{fig: Rhythm}
\end{figure}

\begin{figure}[H]
    \centering
    \includegraphics[width=0.3\textwidth]{design_fig/interface/time_bar.png}
    \caption{Drag the timeline to view the brush tip status at specific points of the character.}
    \label{fig:time bar}
\end{figure}

Center stroke (zhongfeng) is a fundamental principle in Chinese calligraphy ~\cite{yi2021beginner, yang2009animating, yang2013animating}, and therefore requires direct observation. To facilitate this, the writing video was retained, allowing teachers to integrate the brushstroke process with the form of the center stroke technique (Figure \ref{fig:time bar}). This helps students understand and connect the two aspects more effectively.

\subsection{Line Quality Analysis}

Placed on the same page as the previous rhythm analysis view, allowing users to observe brushstroke parameters while simultaneously comparing changes in the brush-tip video. This integrated approach enables a comprehensive analysis of stroke quality and the brushwork process.

\subsubsection{Brush Rotation}
In Chinese calligraphy, to maintain the ``center stroke'' (zhongfeng) and ensure smooth writing, the calligrapher must continuously adjust the position of the brush tip on the paper\cite{chiang1974chinese}. This adjustment is achieved by rotating the brush handle: on the one hand, the rotation ensures that the brush tip remains aligned with the center of the stroke, and on the other hand, when the brush tip begins to splay, the rotation consolidates the tip, ensuring that the force is concentrated at the tip. This focus on brush handle rotation reflects the calligrapher's exploration of brush-tip control. Therefore, the rotation of the brush handle in writing is worth demonstrating.

Initially, an attempt was made to decompose the posture of the brush handle into yaw, roll, and pitch\cite{fitzpatrick2010validation}, which is a common method for analyzing movement. The idea of displaying these values on a dashboard was considered, but isolated parameters at a single moment in time provided little explanatory value. We also explored displaying the brush handle posture curves (e.g., yaw, roll, and pitch) for both teachers and students side by side \cite{10.1145/3476124.3488645, 10.1145/1878083.1878098}. However, it was found that these curves overly abstracted the posture information, rendering them less accessible to users without a background in data visualization.

Ultimately, a decision was made to visualize the rotation directly on the written strokes. Initially, the plan was to capture and display the amount of rotation at corresponding locations (Figure \ref{fig:roll}). However, given the possibility of brush rotation during advancement, marking the absolute orientation of the brush handle at sampled points along the stroke's central axis using arrow symbols was chosen. This approach offers a more intuitive understanding of the rotation process.

\begin{figure}[htbp]
    \centering
    \includegraphics[width=0.8\linewidth]{design_fig/interface/roll.png}
    \caption{One of the sketches during the iterative process: A rotating arrow is used to indicate that a rotation has occurred at a certain position in the handwriting, and this is represented with a pulse diagram.}
    \label{fig:roll}
\end{figure}

Given the cylindrical nature of the brush shaft and the absence of a fixed front-facing direction, the reverse direction of the first stroke's extension is adopted as the initial orientation, aligning with customary writing practices. The calibration is done at the start of each stroke (Figure \ref{fig:Rotation}). Figure \ref{fig:Rotation} shows that the teacher rotates (left) the brush clockwise while advancing it, whereas the student (rignt) first rotates the brush counterclockwise, followed by a slight clockwise rotation, without achieving the same 'brush wrapping' technique as the teacher.

\begin{figure}[htbp]
    \centering
    \includegraphics[width=0.8\linewidth]{design_fig/interface/Rotation.png}
    \caption{In the stroke detail view, the brush rotation of both the student and the teacher while writing the same stroke is visualized through the rotation view.}
    \label{fig:Rotation}
\end{figure}

\subsubsection{Brush Tilt}
The tilt angle of the brush handle directly affects the way the brush tip interacts with the paper (Figure \ref{fig:brush tilt}), thereby altering the friction during writing, which significantly impacts the quality of the strokes. Typically, strokes with greater friction appear darker, thicker, and have sharper edges, while those with less friction are lighter, thinner, and have more blurry edges (Figure \ref{fig:stroke splitting}). Therefore, to explore the conditions that contribute to the texture of a particular stroke, it is essential to display the tilt of the brush handle in various directions. Building on the discussion from the previous section, the tilt direction of the brush handle must be directly visualized on the stroke.
\begin{figure}[ht]
        \centering
        \includegraphics[width=0.8\linewidth]{design_fig/Brush_Tilt.png}
        \caption{The texture of lines varies with two different pen angles.}
        \label{fig:brush tilt}
\end{figure}

\begin{figure}[th]
        \centering
        \includegraphics[width=0.98\linewidth]{design_fig/interface/Tilt.png}
        \caption{The short lines on the strokes represent the brush's projection on the paper. From the visualization, it's clear that the brush's tilt direction differs between the teacher and the student when writing the same stroke. The teacher tends to tilt the brush against the writing direction to add more strength to the lines, while the student completely overlooks this technique.}
        \label{fig:stroke splitting}
\end{figure}

To ensure ease of comprehension for users, a 3D model of the brush hovering above each sampled point on the stroke was initially planned to be rendered to represent the brush's position at that location. However, since the strokes lie on a horizontal plane, the 3D brush could become obscured or cause perspective distortion due to overlapping brushes in the foreground and background \cite{munzner2015visualization}. As a result, the projection of the brush handle onto the plane of the paper was ultimately chosen as the most suitable visualization method.


\subsubsection{Comparison Curves for Speed and Pressure}
The speed of the brush and the pressure applied by the fingers are critical variables that influence the quality of calligraphy strokes. These two factors respectively determine the force of the brush and the duration of contact with the paper surface. The visualization of pressure and speed parameters in this view continued with the encoding approach from the rhythm view. The difference is that the upper and lower limits of the color scale were adjusted for each stroke to enhance the visibility of pressure variations within individual strokes. Considering that both pressure and speed are unidimensional numerical variables, they are better suited for comparison through curves rather than posture data. To prevent confusion and decrease cognitive load~\cite{keller2006information}, scatter plots were selected to illustrate pressure changes, while curves were used to represent speed trends.

Calligraphy does not follow rigid brushwork rules; much like riding a bicycle, where direction is adjusted based on road conditions, the writing process requires continuous adjustments in brush posture based on the state of the brush. Thus, the primary requirement was to display the general range and trends. In the charts, specific numerical values for pressure and speed were omitted, and a tiered representation was chosen instead.

First, the pressure view is introduced, wherein the bottom scatter plot displays the range of pressure levels for both teachers and students. When hovering the mouse over a scatter point, a small red dot will appear on the stroke to indicate the corresponding position on the ink trace. The x-axis of the scatter plot represents the position within the stroke, and scatter points aligned vertically represent roughly the same position on the stroke, facilitating comparison. As the mouse moves across the chart, a vertical line and the corresponding red dot on the stroke will move in sync, indicating the matching position on the ink trace.

Through the visualization, we can observe that when writing a short downward stroke (pie), the teacher applies greater pressure with the fingers in the initial phase, gradually relaxing to maintain a moderate pressure level (Figure \ref{fig: Pressure}). In contrast, the student consistently applies high pressure throughout the stroke. The hover function was utilized to identify the point of maximum pressure difference between the teacher and the student, which was approximately located at the 2/5 mark of the stroke. This feature provides clear guidance on the area where the student requires adjustment.

\begin{figure}[t]
    \centering
    \includegraphics[width=\linewidth]{design_fig/interface/Pressure.png}
    \caption{electing a left-falling stroke for pressure analysis, the scatter plot clearly shows the pressure variation trends and the differences between the student and the teacher. The X-axis represents the relative position of the stroke, and users can hover over the vertical line of any sample point. Simultaneously, a small red dot in the character will be positioned at the corresponding location, making comparison and analysis easier.}
    \label{fig: Pressure}
\end{figure}

Next is the speed view, where the X-axis also represents the stroke position. The curve shows the speed variation throughout the stroke, allowing for a comparison with the teacher's speed (Figure \ref{ref: Speed}). The aim was also to observe the variation in writing speeds of the teacher and the student over time. To accomplish this, time was set as the x-axis, analogous to a stopwatch, with two rows of horizontal scatter points representing the writing progress of both the teacher and the student. Furthermore, color saturation was employed as an additional layer of encoding to align the sequence between the top and bottom of the ink trace and the graph.

In the speed comparison (left), the teacher's writing shows a rhythm of faster movement in the middle and slower at both ends, while the student writes quickly at the start, with little variation in speed afterward. This indicates that the student's initial and finishing strokes are not executed properly. However, the right chart reveals that the student's overall writing speed is three times slower than the teacher's, suggesting hesitation at the beginning, which leads to sluggish and lifeless strokes \cite{chiang1974chinese}.



\begin{figure}[H]
    \centering
    \includegraphics[width=\linewidth]{design_fig/interface/Speed.jpg}
    \caption{By comparing the writing speed of the teacher and the student in the same stroke segment using line charts (left) and their writing positions at the same time points (right). }
    \label{ref: Speed}
\end{figure}


\section{Evaluation}

% \saidur{Working on it}




\begin{table*}[!t]
% \small
\centering
\caption{Summary of Results for EMBER Domain-IL Experiments.}
\vspace{-0.2cm}
\label{tab:ember_DIL}

\begin{tabular}{p{1.1cm}|l|c|c|c|c|c|c|c} 

% \toprule 

\multirow{2}{*}{\textbf{Group}} & \multirow{2}{*}{\textbf{Method}} & \multicolumn{7}{c}{\textbf{Budget}} \\ \cline{3-9}

&  & 1K & 10K & 50K & 100K & 200K & 300K & 400K \\ \midrule

\multirow{3}{*}{Baselines} 
& Joint  & \multicolumn{7}{c}{96.4$\pm$0.3} \\ 
& None   & \multicolumn{7}{c}{93.1$\pm$0.1} \\ 
& GRS    & 93.6$\pm$0.3 & 94.1$\pm$1.3 & 95.3$\pm$0.2 & 95.3$\pm$0.7 & 95.9$\pm$0.1 & 95.8$\pm$0.6 & 96.0$\pm$0.3 \\ 
\midrule

\multirow{4}{*}{\parbox{0.7cm}{Prior \\ Work}} 
& ER~\cite{er}     & 80.6$\pm$0.1 & 73.5$\pm$0.5 & 70.5$\pm$0.3 & 69.9$\pm$0.1 & 70.0$\pm$0.1 & 70.0$\pm$0.1 & 70.0$\pm$0.1 \\ 
& AGEM~\cite{agem}   & 80.5$\pm$0.1 & 73.6$\pm$0.2 & 70.4$\pm$0.3 & 70.0$\pm$0.1 & 70.0$\pm$0.2 & 70.0$\pm$0.1 & 70.0$\pm$0.1 \\ 
& GR~\cite{gr}     & \multicolumn{7}{c}{93.1$\pm$0.2} \\ 
& RtF~\cite{rtf}    & \multicolumn{7}{c}{93.2$\pm$0.2} \\ 
& BI-R~\cite{BIR}   & \multicolumn{7}{c}{93.4$\pm$0.1} \\ 
\midrule

\multirow{4}{*}{\system}      
& \system-R         & \textbf{93.7$\pm$0.1} & \textbf{94.7$\pm$0.1} & \textbf{95.4$\pm$0.1} & \textbf{95.3$\pm$0.6} & \textbf{96.0$\pm$0.1} & \textbf{96.1$\pm$0.1} & \textbf{96.1$\pm$0.1} \\ 
& \system-U         & \textbf{93.6$\pm$0.2} & 94.0$\pm$0.2 & 95.1$\pm$0.1 & \textbf{95.3$\pm$0.1} & 95.5$\pm$0.1 & 95.7$\pm$0.1 & 95.8$\pm$0.1 \\  \cline{2-9}
& MADAR$^{\theta}$-R & \textbf{93.6$\pm$0.1} & \textbf{94.4$\pm$0.3} & \textbf{95.3$\pm$0.2} & \textbf{95.8$\pm$0.1} & \textbf{96.1$\pm$0.1} & \textbf{96.1$\pm$0.1} & \textbf{96.1$\pm$0.1} \\ 
& MADAR$^{\theta}$-U & 93.5$\pm$0.2 & 94.1$\pm$0.2 & 94.9$\pm$0.1 & 95.2$\pm$0.2 & 95.6$\pm$0.1 & 95.7$\pm$0.1 & 95.7$\pm$0.1 \\ 

\bottomrule

\end{tabular}
\vspace{-0.2cm}
\end{table*}









\begin{figure}[!t]
    \centering
    \begin{subfigure}{0.485\linewidth}
        \centering
        \includegraphics[width=1.0\linewidth]{figures_TIFS/EMBER_IFS_DIL_RATIO.pdf}
        \label{fig:EMBER_DIL_IFS_R}
        \vspace{-0.4cm}
        \caption{MADAR Ratio}
    \end{subfigure}
    \hfill
    \begin{subfigure}{0.485\linewidth}
        \centering
        \includegraphics[width=1.0\linewidth]{figures_TIFS/EMBER_IFS_DIL_UNIFORM.pdf}
        \label{fig:EMBER_DIL_IFS_U}
        \vspace{-0.4cm}
        \caption{MADAR Uniform}
    \end{subfigure}
    \vfill
    \begin{subfigure}{0.485\linewidth}
        \centering
        \includegraphics[width=1.0\linewidth]{figures_TIFS/EMBER_AWS_DIL_RATIO.pdf}
        \label{fig:EMBER_DIL_AWS_R}
        \vspace{-0.4cm}
        \caption{MADAR$^\theta$ Ratio}
    \end{subfigure}
    \hfill
    \begin{subfigure}{0.485\linewidth}
        \centering
        \includegraphics[width=1.0\linewidth]{figures_TIFS/EMBER_AWS_DIL_UNIFORM.pdf}
        \label{fig:EMBER_DIL_AWS_U}
        \vspace{-0.4cm}
        \caption{MADAR$^\theta$ Uniform}
    \end{subfigure}

    \caption{EMBER Domain-IL: Comparison of the MADAR-R, MADAR-U, MADAR$^\theta$-R, and MADAR$^\theta$-U with Joint baseline.}
    \label{fig:ember_DIL}
    \vspace{-0.3cm}
\end{figure}





\begin{table*}[!t]
\centering
\caption{Summary of Results for AZ Domain-IL Experiments.}
\vspace{-0.3cm}
\label{tab:az_DIL}
\begin{tabular}{p{1.1cm}|l|c|c|c|c|c|c|c} 

% \toprule 

\multirow{2}{*}{\textbf{Group}} & \multirow{2}{*}{\textbf{Method}} & \multicolumn{7}{c}{\textbf{Budget}} \\ \cline{3-9}

&  & 1K & 10K & 50K & 100K & 200K & 300K & 400K \\ \midrule

\multirow{3}{*}{Baselines} 
& Joint  & \multicolumn{7}{c}{97.3$\pm$0.1} \\ 
& None   & \multicolumn{7}{c}{94.4$\pm$0.1} \\ 
& GRS    & 95.3$\pm$0.1 & 96.4$\pm$0.1 & 96.9$\pm$0.1 & 97.1$\pm$0.1 & 97.1$\pm$0.1 & 97.2$\pm$0.1 & 97.2$\pm$0.1 \\ 
\midrule

\multirow{4}{*}{\parbox{0.7cm}{Prior \\ Work}} 
& ER~\cite{er}     & 40.4$\pm$0.1 & 40.1$\pm$0.1 & 41.1$\pm$0.2 & 42.6$\pm$0.1 & 44.0$\pm$0.1 & 45.9$\pm$0.1 & 48.6$\pm$1.1 \\ 
& AGEM~\cite{agem}   & 45.4$\pm$0.1 & 47.4$\pm$0.2 & 49.2$\pm$0.2 & 53.7$\pm$0.6 & 54.2$\pm$0.3 & 54.8$\pm$0.4 & 56.7$\pm$0.3 \\ 
& GR~\cite{gr}     & \multicolumn{7}{c}{93.3$\pm$0.4} \\ 
& RtF~\cite{rtf}     & \multicolumn{7}{c}{93.4$\pm$0.2} \\ 
& BI-R~\cite{BIR}     & \multicolumn{7}{c}{93.5$\pm$0.1} \\ 
\midrule

\multirow{4}{*}{\system}      
& \system-R         & \textbf{95.8$\pm$0.1} & \textbf{96.6$\pm$0.1} & \textbf{96.9$\pm$0.1} & \textbf{97.0$\pm$0.1} & \textbf{97.0$\pm$0.1} & \textbf{97.0$\pm$0.1} & \textbf{97.0$\pm$0.1} \\ 
& \system-U         & \textbf{95.7$\pm$0.1} & 95.5$\pm$0.1 & 95.2$\pm$0.2 & 95.2$\pm$0.1 & 95.4$\pm$0.1 & 95.8$\pm$0.2 & 96.3$\pm$0.2 \\ \cline{2-9}
& MADAR$^{\theta}$-R & \textbf{95.8$\pm$0.2} & \textbf{96.6$\pm$0.1} & \textbf{96.9$\pm$0.1} & \textbf{96.9$\pm$0.1} & \textbf{97.1$\pm$0.1} & \textbf{97.1$\pm$0.1} & \textbf{97.2$\pm$0.1} \\ 
& MADAR$^{\theta}$-U & 95.6$\pm$0.1 & 96.1$\pm$0.1 & 96.6$\pm$0.1 & 96.8$\pm$0.1 & \textbf{97.0$\pm$0.1} & \textbf{97.1$\pm$0.1} & \textbf{97.1$\pm$0.1} \\ 

\bottomrule

\end{tabular}
\vspace{-0.3cm}
\end{table*}



\begin{figure}[!t]
    \centering
    \begin{subfigure}{0.485\linewidth}
        \centering
        \includegraphics[width=1.0\linewidth]{figures_TIFS/AZ_IFS_DIL_RATIO.pdf}
        \label{fig:AZ_DIL_IFS_R}
        \vspace{-0.4cm}
        \caption{MADAR Ratio}
    \end{subfigure}
    \hfill
    \begin{subfigure}{0.485\linewidth}
        \centering
        \includegraphics[width=1.0\linewidth]{figures_TIFS/AZ_IFS_DIL_UNIFORM.pdf}
        \label{fig:AZ_DIL_IFS_U}
        \vspace{-0.4cm}
        \caption{MADAR Uniform}
    \end{subfigure}
    \hfill
    \begin{subfigure}{0.485\linewidth}
        \centering
        \includegraphics[width=1.0\linewidth]{figures_TIFS/AZ_AWS_DIL_RATIO.pdf}
        \label{fig:AZ_DIL_AWS_R}
        \vspace{-0.4cm}
        \caption{MADAR$^\theta$ Ratio}
    \end{subfigure}
    \hfill
    \begin{subfigure}{0.485\linewidth}
        \centering
        \includegraphics[width=1.0\linewidth]{figures_TIFS/AZ_AWS_DIL_UNIFORM.pdf}
        \label{fig:AZ_DIL_AWS_U}
        \vspace{-0.4cm}
        \caption{MADAR$^\theta$ Uniform}
    \end{subfigure}

    \caption{AZ Domain-IL: Comparison of the MADAR-R, MADAR-U, MADAR$^\theta$-R, and MADAR$^\theta$-U with Joint baseline.}
    \label{fig:az_DIL}
    \vspace{-0.3cm}
\end{figure}






% \subsection{Experimental Setup, Datasets, and Baselines}


We present the results of our \system\ framework and MADAR$^\theta$ in the Domain-IL, Class-IL, and Task-IL scenarios using the EMBER and AZ datasets discussed in Section~\ref{sec:dataset}. To denote our techniques, we use the following abbreviations: {\bf \system-R} for \system-Ratio, {\bf \system-U} for \system-Uniform, {\bf MADAR$^\theta$-R} for MADAR$^\theta$-Ratio, and {\bf MADAR$^\theta$-U} for MADAR$^\theta$-Uniform.

For all three scenarios, we compare \system\ against widely studied replay-based continual learning (CL) techniques, including experience replay (ER)\cite{er}, average gradient episodic memory (AGEM)\cite{agem}, deep generative replay (GR)\cite{gr}, Replay-through-Feedback (RtF)\cite{rtf}, and Brain-inspired Replay (BI-R)\cite{BIR}. Additionally, we evaluate \system\ against iCaRL\cite{icarl}, a replay-based method specifically designed for Class-IL. For the Class-IL and Task-IL scenarios, we additionally compare \system\ with Task-specific Attention Modules in Lifelong Learning (TAMiL)\cite{tamil}. Furthermore, we benchmark MADAR against MalCL\cite{malcl}, a method specifically designed for Class-IL. Notably, most recent work focuses primarily on Class-IL and Task-IL scenarios, limiting direct comparisons in the Domain-IL scenario. In our results tables, the best-performing methods and those within the error margin of the top results are highlighted. 

%Finally, we built upon the codebase provided by \cite{continual-learning-malware} for implementation and evaluation.


% In this study, we utilize large-scale malware datasets, including the EMBER dataset~\cite{ember}, a widely recognized benchmark for Windows malware classification, and two Android malware datasets derived from AndroZoo~\cite{AndroZoo}, which were specifically curated for this research. Our approach is evaluated against two primary baselines:

% \begin{smitemize}
%     \item \textbf{None}: A baseline where the model is trained sequentially on each new task without employing any continual learning (CL) techniques, serving as an informal lower bound.
%     \item \textbf{Joint}: A baseline where the model is trained on both new and previously seen data at each step, representing an informal upper bound. While resource-intensive, the \textbf{Joint} baseline consistently achieves robust performance.
% \end{smitemize}

% Additionally, we introduce a third baseline: \textbf{Global Reservoir Sampling (GRS)}. This method is based on reservoir sampling~\cite{vitter1985random} and builds upon prior work by \cite{continual-learning-malware}. GRS provides an unbiased representation of class distributions and serves as a strong benchmark for comparing our diversity-aware approach.




% We now present the results of our \system framework for both \system and MADAR$^\theta$ in the Domain-IL, Class-IL, and Task-IL scenarios for EMBER and AZ datasets. We use the following four abbreviations to denote our techniques---{\bf \system-R} for \system-Ratio, {\bf ~\system-U} for \system-Uniform, {\bf MADAR$^\theta$-R} for MADAR$^\theta$-Ratio, and {\bf ~MADAR$^\theta$-U} for MADAR$^\theta$-Uniform.  For all three scenarios, we compare \system\ with the most widely studied replay-based CL techniques: experience replay (ER)~\cite{er}, average gradient episodic memory (AGEM)~\cite{agem}, deep generative replay (GR)~\cite{gr}, Replay-through-Feedback (RtF)~\cite{rtf}, and Brain-inspired Replay (BI-R)~\cite{BIR}. In addition, we compare \system\ with iCaRL~\cite{icarl}, a replay-based technique specifically designed for Class-IL. Furthermore, we compare \system with Task-specific Attention Modules in Lifelong learning (TAMiL)~\cite{bhat2023task} which is designed for Class-IL and Task-IL scenarios. In addition, we also compare MADAR with MalCL~\cite{malcl} specifically designed for Class-IL. We observe that recent works mostly focus on Class-IL and Task-IL scenarios which limits what we can compare with in the Domain-IL scenario. The results of the best-performing method, as well as those within the error range of the best results, are highlighted in the results tables. We built upon the code of the prior work by \cite{continual-learning-malware}.

% In this study, we use large-scale Windows and Android malware datasets: EMBER~\cite{ember}, a Windows malware dataset from prior work, recognized as a standard benchmark for malware classification, and two new Android malware datasets derived from AndroZoo~\cite{AndroZoo}, specifically assembled for this research.

% We adopt two baselines for comparison: {\em None} and {\em Joint}.  {\em None} sequentially trains the model on each new task without any CL techniques, serving as an informal minimum baseline. By contrast, {\em Joint} uses all new and prior data for training at each step, acting as an informal maximum baseline. Despite its resource demands, {\em Joint} ensures strong performance throughout the dataset. We also introduce an additional baseline -- Global Reservoir Sampling (GRS) built upon {\em reservoir sampling}~\cite{vitter1985random} and \cite{continual-learning-malware}. GRS provides an unbiased sampling of the underlying class distributions and serves as a strong point of comparison for our diversity-aware approach.

% In this study, we utilize large-scale malware datasets, including the EMBER dataset~\cite{ember}, a widely used benchmark for Windows malware classification, and two Android malware datasets derived from AndroZoo~\cite{AndroZoo}, specifically assembled for this research. We compare our approach against two baselines: {\em None}, where the model is trained sequentially on each new task without any CL techniques, serving as an informal lower bound; and {\em Joint}, which trains on both new and previous data at each step, representing an informal upper bound. Although resource-intensive, {\em Joint} ensures consistently strong results. Additionally, we introduce another baseline -- Global Reservoir Sampling (GRS), an approach based on {\em reservoir sampling}~\cite{vitter1985random} and \cite{continual-learning-malware}, which provides an unbiased representation of class distributions and serves as a strong point of comparison for our diversity-aware approach.


\subsection{Domain-IL}
\label{domainilexps}

%% #of training samples --> 674994
%As shown in Table~\ref{tab:combined_DIL}, a



In EMBER, we have 12 tasks, each representing the monthly data distribution spanning January--December 2018. Our results, detailed in Table~\ref{tab:ember_DIL}, provide a comprehensive view of each method's performance, reported as the average accuracy over all tasks $\mathbf{\overline{AP}}$. Additionally, Figure~\ref{fig:ember_DIL} illustrates the progression of average accuracy over time compared to the \textit{Joint} baseline. 

The informal lower and upper performance bounds for this configuration are approximated by the \textit{None} and \textit{Joint} methods, achieving $\mathbf{\overline{AP}}$ scores of 93.1\% and 96.4\%, respectively. Meanwhile, \textit{GRS} serves as a strong baseline, providing unbiased sampling without incorporating sample diversity awareness.

% In EMBER, we have 12 tasks, each representing the monthly data distribution spanning January--December 2018. Our results, detailed in Table~\ref{tab:ember_DIL}, present a nuanced view of each method's performance, reported as the average accuracy over all tasks $\mathbf{\overline{AP}}$. In addition, Figure~\ref{fig:ember_DIL} represents the progression of average accuracy as the task progresses compared with {joint} baseline. The informal lower and upper performance bounds for this configuration can be approximated by the {\em None} and {\em Joint} methods, which get $\mathbf{\overline{AP}}$ of 93.1\% and 96.4\%, respectively. Meanwhile, {\em GRS} represents a strong baseline for unbiased sampling without awareness of sample diversity.

At a lower budget of 1K, \system-R, \system-U, and MADAR$^\theta$-R exhibit competitive performance, all achieving $\mathbf{\overline{AP}}$ of over $93.6$\%, significantly outperforming prior approaches. This highlights their ability to effectively utilize limited resources. In particular, \system-R achieves the highest accuracy at this budget, with $\mathbf{\overline{AP}}$ of $93.7\%$.

As the memory budget increases, the performance of all \system\ and MADAR$^\theta$ variants improves steadily. At a budget of 200K, \system-R and MADAR$^\theta$-R achieve an impressive $\mathbf{\overline{AP}}$ of $96.0\%$ and $96.1\%$, respectively, closely approaching the $96.4\%$ achieved by the \textit{Joint} baseline, which utilizes over 670K samples. Uniform strategies, including \system-U and MADAR$^\theta$-U, are only slightly behind, with $\mathbf{\overline{AP}}$ values of $95.5\%$ and $95.6\%$, respectively.

% At lower budget of 1K, GRS, \system-R, and \system-U exhibit competitive performance, all significantly better than prior work with $\mathbf{\overline{AP}}$ above $93.6$\%, indicating their effective utilization of limited resources. ER and AGEM performed far below even the \emph{None} baseline, while GR could only match it. For higher budgets, GRS and \system\ methods all show excellent performance. At a 200K budget, \system-R yields $\mathbf{\overline{AP}}$ of $96.0$\%, close to the $96.4$\% reached by the Joint baseline that used over 670K samples. GRS is competitive, while Uniform strategies are only slightly behind.




\begin{table*}[!t]
\centering
\caption{Summary of Results for EMBER Class-IL Experiments.}
\vspace{-0.3cm}
\label{tab:ember_CIL}
\begin{tabular}{p{1.1cm}|l|c|c|c|c|c|c|c} 

% \toprule 

\multirow{2}{*}{\textbf{Group}} & \multirow{2}{*}{\textbf{Method}} & \multicolumn{7}{c}{\textbf{Budget}} \\ \cline{3-9}

&  & 100 & 500 & 1K & 5K & 10K & 15K & 20K \\ \midrule

\multirow{3}{*}{Baselines} 
& Joint  & \multicolumn{7}{c}{86.5$\pm$0.4} \\ 
& None   & \multicolumn{7}{c}{26.5$\pm$0.2} \\ 
& GRS    & 51.9$\pm$0.4 & 70.3$\pm$0.5 & 75.4$\pm$0.7 & 82.0$\pm$0.2 & 83.5$\pm$0.1 & 84.3$\pm$0.3 & 84.6$\pm$0.2 \\ \midrule

\multirow{6}{*}{\parbox{0.7cm}{Prior \\ Work}} 
& TAMiL~\cite{tamil}  & 32.2$\pm$0.3 & 33.1$\pm$0.2 & 35.3$\pm$0.2 & 36.7$\pm$0.1 & 38.2$\pm$0.3 & 37.2$\pm$0.2 & 38.8$\pm$0.2 \\ 
& iCaRL~\cite{icarl}  & 53.9$\pm$0.7 & 58.7$\pm$0.7 & 60.0$\pm$1.0 & 63.9$\pm$1.2 & 64.6$\pm$0.8 & 65.5$\pm$1.0 & 66.8$\pm$1.1 \\ 
& ER~\cite{er}     & 27.5$\pm$0.1 & 27.8$\pm$0.1 & 28.0$\pm$0.1 & 27.9$\pm$0.1 & 28.0$\pm$0.1 & 28.0$\pm$0.1 & 28.2$\pm$0.1 \\ 
& AGEM~\cite{agem}   & 27.3$\pm$0.1 & 27.4$\pm$0.1 & 27.7$\pm$0.1 & 28.5$\pm$0.1 & 28.2$\pm$0.1 & 28.3$\pm$0.1 & 28.2$\pm$0.1 \\ 
& GR~\cite{gr}     & \multicolumn{7}{c}{26.8$\pm$0.2} \\ 
& RtF~\cite{rtf}   & \multicolumn{7}{c}{26.5$\pm$0.1} \\ 
& BI-R~\cite{BIR}   & \multicolumn{7}{c}{26.9$\pm$0.1} \\ 
& MalCL~\cite{malcl}   & \multicolumn{7}{c}{54.5$\pm$0.3} \\ 
\midrule

\multirow{4}{*}{\system} 
& \system-R & \textbf{68.0$\pm$0.4} & 73.6$\pm$0.2 & 76.0$\pm$0.3 & 81.5$\pm$0.2 & 83.2$\pm$0.2 & 83.8$\pm$0.2 & 84.0$\pm$0.2 \\ 
& \system-U & 66.4$\pm$0.4 & \textbf{76.5$\pm$0.2} & \textbf{79.4$\pm$0.4} & \textbf{83.8$\pm$0.2} & \textbf{84.8$\pm$0.1} & \textbf{85.5$\pm$0.1} & \textbf{85.8$\pm$0.3} \\ \cline{2-9}
& MADAR$^{\theta}$-R & {\bf 67.9$\pm$0.3} & 72.7$\pm$0.5 & 72.7$\pm$0.5 & 81.7$\pm$0.2 & 83.2$\pm$0.1 & 83.9$\pm$0.1 & 84.5$\pm$0.2 \\ 
& MADAR$^{\theta}$-U & 67.5$\pm$0.3 & {\bf 76.4$\pm$0.4} & {\bf 78.5$\pm$0.4} & {\bf 84.1$\pm$0.1} & {\bf 85.3$\pm$0.1} & {\bf 85.8$\pm$0.2} & {\bf 86.2$\pm$0.2} \\ 

\bottomrule

\end{tabular}
\vspace{-0.2cm}
\end{table*}



\begin{figure}[!t]
    \centering
    \begin{subfigure}{0.485\linewidth}
        \centering
        \includegraphics[width=1.0\linewidth]{figures_TIFS/EMBER_CIL_IFS_RATIO.pdf}
        \label{fig:EMBER_CIL_IFS_R}
        \vspace{-0.4cm}
        \caption{MADAR Ratio}
    \end{subfigure}
    \hfill
    \begin{subfigure}{0.485\linewidth}
        \centering
        \includegraphics[width=1.0\linewidth]{figures_TIFS/EMBER_CIL_IFS_UNIFORM.pdf}
        \label{fig:EMBER_CIL_IFS_U}
        \vspace{-0.4cm}
        \caption{MADAR Uniform}
    \end{subfigure}
    \vfill
    \begin{subfigure}{0.485\linewidth}
        \centering
        \includegraphics[width=1.0\linewidth]{figures_TIFS/EMBER_CIL_AWS_RATIO.pdf}
        \label{fig:EMBER_CIL_AWS_R}
        \vspace{-0.4cm}
        \caption{MADAR$^\theta$ Ratio}
    \end{subfigure}
    \hfill
    \begin{subfigure}{0.485\linewidth}
        \centering
        \includegraphics[width=1.0\linewidth]{figures_TIFS/EMBER_CIL_AWS_UNIFORM.pdf}
        \label{fig:EMBER_CIL_AWS_U}
        \vspace{-0.4cm}
        \caption{MADAR$^\theta$ Uniform}
    \end{subfigure}

    \caption{EMBER Class-IL: Comparison of the MADAR-R, MADAR-U, MADAR$^\theta$-R, and MADAR$^\theta$-U with Joint baseline.}
    \label{fig:ember_CIL}
    \vspace{-0.3cm}
\end{figure}


For the experiments with AZ-Domain, we consider 9 tasks, each representing a yearly data distribution from 2008 to 2016. The performance of each method is presented in Table~\ref{tab:az_DIL} as $\mathbf{\overline{AP}}$ and compared to two baselines: \textit{None}, which achieves $94.4\%$, and \textit{Joint}, which reaches $97.3\%$. Additionally, Figure~\ref{fig:az_DIL} illustrates the progression of average accuracy across tasks, highlighting the comparison with the \textit{Joint} baseline.

Similar to the results observed with EMBER, our MADAR techniques consistently outperform prior methods such as ER, AGEM, GR, RtF, and BI-R across all budget levels. For lower budgets, such as 1K, \system-R achieves $\mathbf{\overline{AP}}$ of $95.8\%$ and coming within 1.5\% of the \textit{Joint} baseline.

At higher budgets, ranging from 100K to 400K, \system-R continues to demonstrate high $\mathbf{\overline{AP}}$ scores of up to $97.0\%$, closely matching GRS and only marginally below the \textit{Joint} baseline, which requires significantly more training samples (680K). Notably, MADAR$^\theta$-R exhibits comparable performance, reaching a peak $\mathbf{\overline{AP}}$ of $97.2\%$ at the highest budget level, further underscoring the efficacy of our diversity-aware approach.



% For the experiments with AZ-Domain, we have 9 tasks, each representing a year from 2008 to 2016. The performance of each method is shown in Table~\ref{tab:az_DIL} as $\mathbf{\overline{AP}}$ and compared with two baselines: {\em None} at $94.4\pm0.1$ and {\em Joint} at $97.3\pm0.1$. Additionally, Figure~\ref{fig:az_DIL} illustrates the progression of average accuracy as tasks progress, compared to the \textit{Joint} baseline. 

% As with EMBER, we find that our MADAR techniques greatly surpass previous methods like ER, AGEM, GR, RtF, and BI-R for every budget level. For lower budgets like 1K, \system-R slightly outperforms GRS and is within 1.5\% of {\em Joint}. For higher budgets (100K-400K), \system-R perform well -- in line with GRS and just slightly below {\em Joint}, which requires 680K training samples. 


% In summary, our results empirically depict the effectiveness of MADAR's diversity-aware sample selection in maximizing the efficiency and effectiveness of a malware classifier in Domain-IL. \system-R is either better or on par with GRS and significantly better than prior work.

In summary, these results empirically demonstrate the effectiveness of MADAR's diversity-aware sample selection in enhancing the efficiency and accuracy of malware classification in Domain-IL scenarios. \system-R and MADAR$^\theta$-R, in particular, consistently either yield on-par or outperform GRS while delivering significant improvements over prior methods.












\begin{table*}[!t]
\centering
\caption{Summary of Results for AZ Class-IL Experiments.}
\vspace{-0.3cm}
\label{tab:az_CIL}
\begin{tabular}{p{1.1cm}|l|c|c|c|c|c|c|c} 

% \toprule 

\multirow{2}{*}{\textbf{Group}} & \multirow{2}{*}{\textbf{Method}} & \multicolumn{7}{c}{\textbf{Budget}} \\ \cline{3-9}

&  & 100 & 500 & 1K & 5K & 10K & 15K & 20K \\ \midrule

\multirow{3}{*}{Baselines} 
& Joint  & \multicolumn{7}{c}{94.2$\pm$0.1} \\ 
& None   & \multicolumn{7}{c}{26.4$\pm$0.2} \\ 
& GRS    & 43.8$\pm$0.7 & 62.9$\pm$0.8 & 70.2$\pm$0.4 & 83.0$\pm$0.3 & 86.4$\pm$0.2 & 88.2$\pm$0.2 & 89.1$\pm$0.2 \\ \midrule

\multirow{6}{*}{\parbox{0.7cm}{Prior \\ Work}} 
& TAMiL~\cite{tamil}  & 53.4$\pm$0.3 & 55.2$\pm$0.3 & 57.6$\pm$0.3 & 60.8$\pm$0.2 & 63.5$\pm$0.1 & 65.3$\pm$0.5 & 67.7$\pm$0.3 \\ 
& iCaRL~\cite{icarl}  & 43.6$\pm$1.2 & 54.9$\pm$1.0 & 61.7$\pm$0.7 & 77.2$\pm$0.4 & 81.5$\pm$0.6 & 83.4$\pm$0.5 & 84.6$\pm$0.5 \\ 
& ER~\cite{er}     & 50.8$\pm$0.7 & 58.3$\pm$0.6 & 58.9$\pm$0.2 & 59.2$\pm$0.8 & 62.9$\pm$0.7 & 63.1$\pm$0.5 & 64.2$\pm$0.4 \\ 
& AGEM~\cite{agem}   & 27.3$\pm$0.7 & 28.0$\pm$1.4 & 27.1$\pm$0.3 & 28.0$\pm$0.6 & 28.2$\pm$1.0 & 29.8$\pm$2.6 & 28.0$\pm$0.8 \\ 
& GR~\cite{gr}     & \multicolumn{7}{c}{22.7$\pm$0.3} \\ 
& RtF~\cite{rtf}    & \multicolumn{7}{c}{22.9$\pm$0.3} \\ 
& BI-R~\cite{BIR}   & \multicolumn{7}{c}{23.4$\pm$0.2} \\ 
& MalCL~\cite{malcl}   & \multicolumn{7}{c}{59.8$\pm$0.4} \\ 
\midrule

\multirow{4}{*}{\system} 
& \system-R & \textbf{59.4$\pm$0.6} & 67.8$\pm$0.9 & 71.9$\pm$0.5 & 82.9$\pm$0.2 & 86.3$\pm$0.1 & 88.2$\pm$0.2 & 89.1$\pm$0.1 \\ 
& \system-U & 57.3$\pm$0.5 & \textbf{70.4$\pm$0.4} & \textbf{76.2$\pm$0.2} & \textbf{86.8$\pm$0.1} & \textbf{89.8$\pm$0.1} & \textbf{91.0$\pm$0.1} & \textbf{91.5$\pm$0.1} \\ \cline{2-9}
& MADAR$^{\theta}$-R & {\bf 58.8$\pm$0.3} & 66.2$\pm$0.7 & 71.0$\pm$0.7 & 81.2$\pm$0.3 & 85.1$\pm$0.2 & 86.9$\pm$0.2 & 88.1$\pm$0.1 \\ 
& MADAR$^{\theta}$-U & 58.5$\pm$0.7 & {\bf 70.1$\pm$0.2} & {\bf 74.7$\pm$0.2} & {\bf 85.5$\pm$0.1} & {\bf 88.7$\pm$0.1} & {\bf 90.3$\pm$0.2} & {\bf 90.7$\pm$0.1} \\ 

\bottomrule

\end{tabular}
\vspace{-0.2cm}
\end{table*}








\begin{figure}[!t]
    \centering
    \begin{subfigure}{0.485\linewidth}
        \centering
        \includegraphics[width=1.0\linewidth]{figures_TIFS/AZ_CIL_IFS_RATIO.pdf}
        \label{fig:AZ_CIL_IFS_R}
        \vspace{-0.4cm}
        \caption{MADAR Ratio}
    \end{subfigure}
    \hfill
    \begin{subfigure}{0.485\linewidth}
        \centering
        \includegraphics[width=1.0\linewidth]{figures_TIFS/AZ_CIL_IFS_UNIFORM.pdf}
        \label{fig:AZ_CIL_IFS_U}
        \vspace{-0.4cm}
        \caption{MADAR Uniform}
    \end{subfigure}
    \vfill
    \begin{subfigure}{0.485\linewidth}
        \centering
        \includegraphics[width=1.0\linewidth]{figures_TIFS/AZ_CIL_AWS_RATIO.pdf}
        \label{fig:AZ_CIL_AWS_R}
        \vspace{-0.4cm}
        \caption{MADAR$^\theta$ Ratio}
    \end{subfigure}
    \hfill
    \begin{subfigure}{0.485\linewidth}
        \centering
        \includegraphics[width=1.0\linewidth]{figures_TIFS/AZ_CIL_AWS_UNIFORM.pdf}
        \label{fig:AZ_CIL_AWS_U}
        \vspace{-0.4cm}
        \caption{MADAR$^\theta$ Uniform}
    \end{subfigure}

    \caption{AZ Class-IL: Comparison of the MADAR-R, MADAR-U, MADAR$^\theta$-R, and MADAR$^\theta$-U with Joint baseline.}
    \label{fig:az_CIL}
    \vspace{-0.3cm}
\end{figure}





\subsection{Class-IL}
\label{classilexps}



In this set of experiments with EMBER, we consider 11 tasks, starting with 50 classes (representing distinct malware families) in the initial task, and incrementally adding five new classes in each subsequent task. Table~\ref{tab:ember_CIL} presents the performance of each method, measured by average accuracy $\mathbf{\overline{AP}}$. The \textit{None} and \textit{Joint} baselines achieve $\mathbf{\overline{AP}}$ values of $26.5\%$ and $86.5\%$, respectively, providing informal lower and upper bounds. Figure~\ref{fig:ember_CIL} illustrates the progression of average accuracy across tasks, showing how the \system\ and MADAR$^\theta$ methods compare to the \textit{Joint} baseline.

At a very low budget of just 100 samples, \system-R achieves a notable $\mathbf{\overline{AP}}$ of $68.0\%$, outperforming GRS and prior methods by a significant margin. As the budget increases, \system-U emerges as the top performer, achieving $\mathbf{\overline{AP}}$ values of $76.5\%$ and $79.4\%$ at 1K and 10K budgets, respectively, surpassing all other methods, including GRS. 

%For example, at a 10K budget, \system-U outperforms GRS, which achieves $83.5\%$, with an $\mathbf{\overline{AP}}$ of $84.8\%$.

At higher budgets, \system-U and MADAR$^\theta$-U continue to excel, with MADAR$^\theta$-U achieving the best results overall. At a 20K budget, MADAR$^\theta$-U reaches an $\mathbf{\overline{AP}}$ of $86.2\%$, nearly equaling the \textit{Joint} baseline, which uses over {\bf 150 times} more training samples. These results clearly demonstrate the effectiveness of MADAR's diversity-aware sample selection and the effectiveness of \system-U and MADAR$^\theta$-U in leveraging limited resources.

In contrast, prior methods such as ER, AGEM, GR, RtF, and BI-R fail to exceed 30\% $\mathbf{\overline{AP}}$, while more advanced techniques like TAMiL and MalCL achieve only $38.2\%$ and $54.8\%$, respectively. Even iCaRL, designed specifically for Class-IL, achieves only $64.6\%$, further highlighting the significant advantage of our approaches in the malware domain.


% In this set of experiments with EMBER, we have 11 tasks, where the initial task starts with 50 classes---one for each of 50 malware families---and five classes are added in each subsequent task. The performance of these methods, detailed in Table~\ref{tab:az_CIL}, is measured by average accuracy $\mathbf{\overline{AP}}$ with {\em None} and {\em Joint} training baselines at an $\mathbf{\overline{AP}}$ of $26.5\pm0.2$ and $86.5\pm0.4$, respectively. Additionally, Figure~\ref{fig:ember_CIL} illustrates the progression of average accuracy across tasks, highlighting the comparison with the \textit{Joint} baseline. 

% For a very low budget of 100 samples, \system methods greatly outperform GRS, with \system-R getting 16\% higher $\mathbf{\overline{AP}}$. For more reasonable budgets, however, the uniform variant \system-U offers the best performance. For example, with a 10K budget, \system-U yields at least 84.8\% $\mathbf{\overline{AP}}$, which is better than GRS at 83.5\% $\mathbf{\overline{AP}}$. They also fare far better than all prior works, with ER, AGEM, GR, RtF, and BI-R below 30\%, TAMiL at 38.2\%, MalCL at 54.8\% and iCaRL at only 64.6\%. These poor results for the prior methods are in line with other findings in the malware domain~\cite{continual-learning-malware}. For a budget of 20K, \system-U reaches $85.8\pm0.3$, nearly as good as the Joint baseline that uses a maximum budget over 150 times larger.



In the Class-IL setting of AZ-Class, we consider 11 tasks. The summary results of all experiments are provided in Table~\ref{tab:az_CIL}, with comparisons against the \textit{None} and \textit{Joint} baselines, which achieve $\mathbf{\overline{AP}}$ scores of $26.4\%$ and $94.2\%$, respectively. Figure~\ref{fig:az_CIL} illustrates the progression of average accuracy across tasks, showing how each method performs relative to the \textit{Joint} baseline.

As shown in Table~\ref{tab:az_CIL}, among the prior methods, iCaRL performs best across most budget configurations, outperforming techniques such as MalCL, TAMiL, ER, AGEM, GR, RtF, and BI-R. Therefore, we focus on comparing MADAR's performance with iCaRL. At a low budget of 100 samples, iCaRL and GRS achieve less than $44\%$ $\mathbf{\overline{AP}}$, while all MADAR methods surpass $57\%$. In particular, \system-R and MADAR$^\theta$-R achieve $\mathbf{\overline{AP}}$ scores of $59.4\%$ and $58.8\%$, respectively, highlighting their efficiency at low-resource levels.

As the budget increases, all methods improve, but \system-U consistently delivers the best results. At a budget of 1K, \system-U achieves the highest $\mathbf{\overline{AP}}$ at $70.4\%$, followed closely by MADAR$^\theta$-U at $70.1\%$. This trend continues as budgets increase, with \system-U outperforming all other methods, achieving $\mathbf{\overline{AP}}$ scores of $89.8\%$ at 10K and $91.5\%$ at 20K. Compared to GRS, which achieves $90.1\%$ at 20K, and iCaRL, which trails at $84.6\%$, \system-U demonstrates clear superiority. MADAR$^\theta$-U also performs GRS reaching $90.7\%$ at 20K.



% We have 11 tasks for the Class-IL setting of AZ-Class. The summary results of all the experiments are shown in Table~\ref{tab:az_CIL} and benchmarked against {\em None} and {\em Joint} with $\mathbf{\overline{AP}}$ of $26.4\pm0.2$ and $94.2\pm0.1$, respectively. Figure~\ref{fig:az_CIL} illustrates the progression of average accuracy across tasks, highlighting the comparison with the \textit{Joint} baseline. 


% As we can from Table~\ref{tab:az_CIL} that, among TAMiL, iCaRL, ER, AGEM, GR, RtF, and BI-R, iCaRL outperforms in most of the budget configurations. Therefore, we discuss the results of MADAR in comparison with iCaRL. For a low budget of 100, iCaRL and GRS get less than 44\%, while all MADAR methods achieve over 57\%. As budgets increase, all methods improve, with \system-U offering the best results at every budget from 1K to 20K. At 20K, it reaches $91.5\pm0.1\%$, which is 1.4\% higher than GRS and 6.9\% higher than iCaRL.



In summary, our experiments demonstrate the effectiveness of \system's diversity-aware replay techniques in the Class-IL setting for both EMBER and AZ datasets. While GRS shows significant improvement with larger budgets, \system's uniform variants consistently outperform it across all budget levels. These results underscore \system's ability to mitigate catastrophic forgetting and enhance malware classification performance, even in resource-constrained environments.

% In summary, our experiments clearly demonstrate the effectiveness of \system's diversity-aware replay techniques in Class-IL for both EMBER and AZ datasets. Additionally, while GRS shows significant improvement with an increased budget, the uniform variants of \system  are more effective at every budget level. \system  significantly improves performance in malware classification by mitigating catastrophic forgetting, and they do so using fewer resources.












\begin{table*}[!t]
\centering
\caption{Summary of Results for EMBER Task-IL Experiments.}
\vspace{-0.3cm}
\label{tab:ember_TIL}
\begin{tabular}{p{1.1cm}|l|c|c|c|c|c|c|c} 

% \toprule 

\multirow{2}{*}{\textbf{Group}} & \multirow{2}{*}{\textbf{Method}} & \multicolumn{7}{c}{\textbf{Budget}} \\ \cline{3-9}

&  & 100 & 500 & 1K & 5K & 10K & 15K & 20K \\ \midrule

\multirow{3}{*}{Baselines} 
& Joint  & \multicolumn{7}{c}{97.0$\pm$0.3} \\ 
& None   & \multicolumn{7}{c}{74.6$\pm$0.7} \\ 
& GRS    & 86.9$\pm$0.3 & 87.4$\pm$0.3 & 93.6$\pm$0.3 & 94.4$\pm$0.2 & 94.7$\pm$0.3 & 94.9$\pm$0.1 & 95.0$\pm$0.1 \\ \midrule

\multirow{6}{*}{\parbox{0.7cm}{Prior \\ Work}} 
& TAMiL~\cite{tamil}  & 72.8$\pm$0.1 & 81.5$\pm$0.3 & 86.9$\pm$0.2 & 88.1$\pm$0.3 & 90.3$\pm$0.1 & 93.2$\pm$0.3 & 94.2$\pm$0.7 \\ 
& ER~\cite{er}     & 67.4$\pm$0.3 & 84.9$\pm$0.2 & 89.5$\pm$0.5 & 93.9$\pm$0.2 & 94.8$\pm$0.2 & 95.2$\pm$0.1 & 95.4$\pm$0.1 \\ 
& AGEM~\cite{agem}   & 79.6$\pm$0.2 & 81.7$\pm$0.2 & 83.8$\pm$0.4 & 84.9$\pm$0.2 & 86.1$\pm$0.2 & 88.9$\pm$0.2 & 89.3$\pm$0.1 \\ 
& GR~\cite{gr}     & \multicolumn{7}{c}{79.8$\pm$0.3} \\ 
& RtF~\cite{rtf}    & \multicolumn{7}{c}{77.8$\pm$0.2} \\ 
& BI-R~\cite{BIR}   & \multicolumn{7}{c}{87.2$\pm$0.3} \\ \midrule

\multirow{4}{*}{\system} 
& \system-R & 92.1$\pm$0.2 & 92.3$\pm$0.9 & 93.8$\pm$0.2 & 94.2$\pm$0.1 & 94.8$\pm$0.2 & {\bf 95.7$\pm$0.2} & {\bf 95.6$\pm$0.1} \\ 
& \system-U & {\bf 93.4$\pm$0.2} & {\bf 93.7$\pm$0.3} & {\bf 93.9$\pm$0.3} & {\bf 94.8$\pm$0.2} & {\bf 95.6$\pm$0.1} & {\bf 95.7$\pm$0.1} & {\bf 95.8$\pm$0.2} \\ \cline{2-9}
& MADAR$^{\theta}$-R & {\bf 93.1$\pm$0.2} & {\bf 93.3$\pm$0.1} & {\bf 93.6$\pm$0.1} & 94.3$\pm$0.1 & 94.6$\pm$0.2 & 94.8$\pm$0.2 & 94.7$\pm$0.3 \\ 
& MADAR$^{\theta}$-U & {\bf 93.2$\pm$0.1} & 93.1$\pm$0.2 & {\bf 93.8$\pm$0.2} & {\bf 94.4$\pm$0.1} & {\bf 94.8$\pm$0.1} & {\bf 95.3$\pm$0.2} & {\bf 95.5$\pm$0.3} \\ 

\bottomrule

\end{tabular}
\vspace{-0.3cm}
\end{table*}



\begin{figure}[!t]
    \centering
    \begin{subfigure}{0.485\linewidth}
        \centering
        \includegraphics[width=1.0\linewidth]{figures_TIFS/EMBER_TIL_IFS_RATIO.pdf}
        \label{fig:EMBER_TIL_IFS_R}
        \vspace{-0.4cm}
        \caption{MADAR Ratio}
    \end{subfigure}
    \hfill
    \begin{subfigure}{0.485\linewidth}
        \centering
        \includegraphics[width=1.0\linewidth]{figures_TIFS/EMBER_TIL_IFS_UNIFORM.pdf}
        \label{fig:EMBER_TIL_IFS_U}
        \vspace{-0.4cm}
        \caption{MADAR Uniform}
    \end{subfigure}
    \vfill
    \begin{subfigure}{0.485\linewidth}
        \centering
        \includegraphics[width=1.0\linewidth]{figures_TIFS/EMBER_TIL_AWS_RATIO.pdf}
        \label{fig:EMBER_TIL_AWS_R}
        \vspace{-0.4cm}
        \caption{MADAR$^\theta$ Ratio}
    \end{subfigure}
    \hfill
    \begin{subfigure}{0.485\linewidth}
        \centering
        \includegraphics[width=1.0\linewidth]{figures_TIFS/EMBER_TIL_AWS_UNIFORM.pdf}
        \label{fig:EMBER_TIL_AWS_U}
        \vspace{-0.4cm}
        \caption{MADAR$^\theta$ Uniform}
    \end{subfigure}

    \caption{EMBER Task-IL: Comparison of the MADAR-R, MADAR-U, MADAR$^\theta$-R, and MADAR$^\theta$-U with Joint baseline.}
    \label{fig:ember_TIL}
    \vspace{-0.3cm}
\end{figure}

























\subsection{Task-IL}
\label{taskilexps-ember}


In this set of experiments with EMBER, we consider 20 tasks, with 5 new classes added in each task. The summarized results are shown in Table~\ref{tab:ember_TIL}, where performance is reported as the average accuracy over all tasks ($\mathbf{\overline{AP}}$). It is worth noting that Task-IL is considered the easiest scenario in continual learning~\cite{van2022three, BIR}. The \textit{None} and \textit{Joint} methods serve as informal lower and upper bounds, achieving $\mathbf{\overline{AP}}$ scores of $74.6\%$ and $97\%$, respectively. Figure~\ref{fig:ember_TIL} visualizes the progression of average accuracy across tasks, highlighting comparisons with the \textit{Joint} baseline.

As shown in Table~\ref{tab:ember_TIL}, ER consistently outperforms TAMiL, A-GEM, GR, RtF, and BI-R across all budget configurations and even surpasses GRS in some cases. However, \system\ variants significantly outperform all prior methods, particularly under lower budget constraints (100–1K). For example, \system-U achieves the highest $\mathbf{\overline{AP}}$ of $93.4\%$ and $93.7\%$ at budgets of 100 and 1K, respectively, outperforming GRS and all other approaches. Similarly, MADAR$^\theta$-U performs competitively, with $\mathbf{\overline{AP}}$ of $93.2\%$ at a 100 budget and $93.8\%$ at 1K.

As the budget increases, the performance gap among \system, ER, and GRS narrows; however, \system\ variants continue to either outperform or match other techniques. Notably, the \system-U variant of MADAR achieves the best overall performance at a budget of 20K, attaining a $\mathbf{\overline{AP}}$ of $95.8\%$, which closely approaches the \textit{Joint} baseline. Similarly, \system-R yields $\mathbf{\overline{AP}}$ of $95.6\%$ at 20K.



% In this set of experiments with EMBER, we have 20 tasks with 5 new classes in each task. Table~\ref{tab:ember_TIL} shows a summarized view of this set of experiments, where the performances are presented as the average accuracy over all tasks ($\mathbf{\overline{AP}}$). Note that Task-IL is considered the easiest scenario of continual learning~\cite{van2022three, BIR}. The {\em None} and {\em Joint} methods, which are the informal lower and upper bounds of this configuration, attain $\overline{AP}$ of $74.6\%$ and $\overline{AP}$ of $97.03\%$, respectively. Figure~\ref{fig:ember_TIL} illustrates the progression of average accuracy across tasks, showing how each method performs relative to the \textit{Joint} baseline.

% As we can see from Table~\ref{tab:combined_TIL}, ER outperforms TAMiL, A-GEM, GR, RtF, and BI-R in all budget configurations and outperforms GRS for few configurations. \system, on the other hand, outperforms all the prior methods significantly in lower budget constraints ($100$–$1K$). For instance, \system-U reaches $\mathbf{\overline{AP}}$ of 93.9\% with only 1K replay samples, compared with 93.6\% for GRS. The performance gap among MADAR, ER, and GRS gets closer as the budget increases; however, \system  variants continue to either outperform or perform on par with other techniques. In particular, the \system-U variant of MADAR outperforms all the other techniques and attains $\mathbf{\overline{AP}}$ of 95.8\% with a 20K replay budget, which is close to joint level performance.


Task-IL for AZ consists of 20 tasks, each with 5 non-overlapping classes. The results are summarized in Table~\ref{tab:az_TIL} and benchmarked against the \textit{None} and \textit{Joint} baselines, which achieve $\mathbf{\overline{AP}}$ values of $74.5\%$ and $98.8\%$, respectively. Figure~\ref{fig:az_TIL} illustrates the progression of average accuracy across tasks, showing how each method performs relative to the \textit{Joint} baseline.

As seen in Table~\ref{tab:az_TIL}, ER consistently outperforms TAMiL, AGEM, GR, RtF, BI-R, and GRS across most budget configurations, making it a strong baseline for comparison. At a low budget of 100 samples, \system-U achieves $\mathbf{\overline{AP}}$ of $88.1\%$, which is 4.5\% higher than ER's performance. Similarly, MADAR$^\theta$-U demonstrates competitive performance, achieving $87.9\%$ at the same budget.

As the budget increases, \system-U continues to deliver the best performance, reaching $\mathbf{\overline{AP}}$ scores of $94.5\%$ at a 1K budget and $98.1\%$ at a 10K budget, outperforming all other methods, including ER and GRS. At the highest budget of 20K, \system-U achieves an $\mathbf{\overline{AP}}$ of $98.7\%$, surpassing ER by 1.2\% and nearly matching the \textit{Joint} baseline. Notably, MADAR$^\theta$-U also performs well, achieving $98.1\%$. In contrast, \system-R and MADAR$^\theta$-R perform slightly lower but remain competitive, with $\mathbf{\overline{AP}}$ values of $97.9\%$ and $96.9\%$ at a 20K budget, respectively. These results indicate that ratio-based methods, while effective, are slightly less robust than uniform sampling in this scenario.

In summary, \system-U and MADAR$^\theta$-U consistently demonstrate better performance across most of the budget levels, particularly excelling at low-resource settings and achieving near-optimal results at higher budgets. These findings underscore the effectiveness of \system\ framework in Task-IL scenarios and their ability to approach joint-level performance with significantly fewer resources.


% Task-IL for AZ contains 20 tasks, each with 5 non-overlapping classes. Our results are shown in Table~\ref{tab:az_TIL}, compared against the {\em None} and {\em Joint} benchmarks, with $\mathbf{\overline{AP}}$ of 74.5\% and 98.8\%, respectively. Figure~\ref{fig:az_TIL} illustrates the progression of average accuracy across tasks, showing how each method performs relative to the \textit{Joint} baseline. As with EMBER, ER outperforms TAMiL, AGEM, GR, RtF, BI-R, and GRS for most budgets, so we use it for comparison. For a low budget of 100, \system-U achieves an $\overline{AP}$ of 88.1\%, 4.5\% higher than that of ER. For a higher budget of 20K, \system-U attains an $\overline{AP}$ of 98.7\%, which is 1.2\% higher than that of ER and very close to the joint level performance of 98.8\%.


% Overall, mirroring the success seen with the EMBER dataset, our proposed techniques also surpass previous work in Task-IL in the context of the AZ-Class dataset. Additionally, while ER and GRS shows significant improvement with an increased budget, the uniform variant of IFS of MADAR is more effective at every budget level.








\begin{table*}[!t]
\centering
\caption{Summary of Results for AZ Task-IL Experiments.}
\vspace{-0.3cm}
\label{tab:az_TIL}
\begin{tabular}{p{1.1cm}|l|c|c|c|c|c|c|c} 

% \toprule 

\multirow{2}{*}{\textbf{Group}} & \multirow{2}{*}{\textbf{Method}} & \multicolumn{7}{c}{\textbf{Budget}} \\ \cline{3-9}

&  & 100 & 500 & 1K & 5K & 10K & 15K & 20K \\ \midrule

\multirow{3}{*}{Baselines} 
& Joint  & \multicolumn{7}{c}{98.8$\pm$0.2} \\ 
& None   & \multicolumn{7}{c}{74.5$\pm$0.2} \\ 
& GRS    & 85.2$\pm$0.1 & 89.2$\pm$0.2 & 90.8$\pm$0.1 & 91.6$\pm$0.2 & 93.5$\pm$0.1 & 93.9$\pm$0.1 & 95.2$\pm$0.1 \\ \midrule

\multirow{6}{*}{\parbox{0.7cm}{Prior \\ Work}} 
& TAMiL  & 80.5$\pm$0.4 & 85.3$\pm$0.6 & 91.5$\pm$0.2 & 92.1$\pm$0.1 & 93.5$\pm$0.1 & 94.0$\pm$0.2 & 94.8$\pm$0.2 \\ 
& ER     & 83.6$\pm$0.2 & 90.2$\pm$0.1 & 92.3$\pm$0.3 & 95.6$\pm$0.1 & 96.2$\pm$0.1 & 96.8$\pm$0.2 & 97.5$\pm$0.2 \\ 
& AGEM   & 76.7$\pm$0.5 & 82.8$\pm$0.2 & 85.3$\pm$0.1 & 85.6$\pm$0.2 & 86.7$\pm$0.2 & 88.9$\pm$0.2 & 91.3$\pm$0.3 \\ 
& GR     & \multicolumn{7}{c}{75.6$\pm$0.2} \\ 
& RtF    & \multicolumn{7}{c}{74.2$\pm$0.3} \\ 
& BI-R   & \multicolumn{7}{c}{85.4$\pm$0.2} \\ \midrule

\multirow{4}{*}{\system} 
& \system-R & 86.0$\pm$0.3 & 90.3$\pm$0.2 & 92.4$\pm$0.1 & 95.8$\pm$0.2 & 96.7$\pm$0.1 & 97.1$\pm$0.1 & 97.9$\pm$0.2 \\ 
& \system-U & {\bf 88.1$\pm$0.3} & {\bf 92.9$\pm$0.2} & {\bf 94.5$\pm$0.3} & {\bf 97.2$\pm$0.2} & {\bf 98.1$\pm$0.1} & {\bf 98.5$\pm$0.1} & {\bf 98.7$\pm$0.1} \\ \cline{2-9}
& MADAR$^{\theta}$-R & 87.3$\pm$0.3 & {\bf 90.6$\pm$0.2} & 93.2$\pm$0.2 & 95.7$\pm$0.2 & 95.9$\pm$0.1 & 96.6$\pm$0.1 & 96.9$\pm$0.1 \\ 
& MADAR$^{\theta}$-U & {\bf 87.9$\pm$0.2} & {\bf 90.8$\pm$0.2} & {\bf 93.6$\pm$0.1} & {\bf 96.2$\pm$0.3} & {\bf 97.2$\pm$0.2} & {\bf 97.5$\pm$0.2} & {\bf 98.1$\pm$0.1} \\ 

\bottomrule

\end{tabular}
\vspace{-0.3cm}
\end{table*}



\begin{figure}[!t]
    \centering
    \begin{subfigure}{0.45\linewidth}
        \centering
        \includegraphics[width=1.0\linewidth]{figures_TIFS/AZ_TIL_IFS_RATIO.pdf}
        \label{fig:AZ_TIL_IFS_R}
        \vspace{-0.4cm}
        \caption{MADAR Ratio}
    \end{subfigure}
    \hfill
    \begin{subfigure}{0.45\linewidth}
        \centering
        \includegraphics[width=1.0\linewidth]{figures_TIFS/AZ_TIL_IFS_UNIFORM.pdf}
        \label{fig:AZ_TIL_IFS_U}
        \vspace{-0.4cm}
        \caption{MADAR Uniform}
    \end{subfigure}
    \vfill
    \begin{subfigure}{0.45\linewidth}
        \centering
        \includegraphics[width=1.0\linewidth]{figures_TIFS/AZ_TIL_AWS_RATIO.pdf}
        \label{fig:AZ_TIL_AWS_R}
        \vspace{-0.4cm}
        \caption{MADAR$^\theta$ Ratio}
    \end{subfigure}
    \hfill
    \begin{subfigure}{0.45\linewidth}
        \centering
        \includegraphics[width=1.0\linewidth]{figures_TIFS/AZ_TIL_AWS_UNIFORM.pdf}
        \label{fig:AZ_TIL_AWS_U}
        \vspace{-0.4cm}
        \caption{MADAR$^\theta$ Uniform}
    \end{subfigure}

    \caption{AZ Task-IL: Comparison of the MADAR-R, MADAR-U, MADAR$^\theta$-R, and MADAR$^\theta$-U with Joint baseline.}
    \label{fig:az_TIL}
    \vspace{-0.3cm}
\end{figure}


\subsection{Analysis and Discussion}\label{diss}


Our results demonstrate that MADAR yields markedly better performances compared to previous methods for both the EMBER and AZ datasets across all CL settings. This clearly indicates that diversity-aware replay is effective in preserving the stability of a CL-based system for malware classification, while prior CL techniques largely fail to achieve acceptable performance.


\paragraphX{\bf MADAR in low-budget settings.} In Domain-IL, MADAR achieves competitive performance even with a 1K budget, surpassing prior work by over 3 percentage points in EMBER and AZ. At higher budgets, ratio-based selection (\system-R and MADAR$^{\theta}$-R) achieves near Joint baseline performance (96.4\% in EMBER and 97.3\% in AZ) while using significantly fewer resources. This demonstrates MADAR’s efficiency in leveraging limited samples to achieve robust classification.


\paragraphX{\bf MADAR is both effective and scalable.} Traditional CL methods, including ER and AGEM, experience significant performance degradation as tasks increase. In contrast, MADAR maintains high accuracy across 20 Task-IL tasks, with \system-U achieving 95.8\% in EMBER and 98.7\% in AZ at a 20K budget, nearly matching the {\em Joint} baseline.




\paragraphX{\bf Ratio vs. Uniform Budgeting.} A consistent trend across our experiments is that ratio-based selection performs best in Domain-IL, whereas uniform-based selection is superior in Class-IL and Task-IL. MADAR$^{\theta}$-U reaches 91.5\% in AZ at 20K, significantly outperforming iCaRL and TAMiL. Furthermore, in EMBER, \system-U achieves near {\em Joint} baseline performance at just a 5K budget, underscoring the effectiveness of uniform selection in class-incremental settings. Intuitively, this makes sense because ratio budgeting for binary classification in the Domain-IL setting naturally captures the contributions of each family to the overall malware distribution. Additionally, since there are many small families in the Domain-IL datasets, uniformly sampling from them consumes budget while offering little improvement in malware coverage. In contrast, our Class-IL and Task-IL experiments perform classification across families, which is better supported by Uniform budgeting to maintain class balance and ensure coverage over all families. Moreover, in most settings we can leverage efficient representations using MADAR$^\theta$ to scale the approach regardless of feature dimension without significant loss of performance.



\paragraphX{\bf GRS remains a strong baseline at high budgets.} While MADAR consistently outperforms GRS in low-resource settings, GRS performs comparably at higher budgets, particularly in Domain-IL. This suggests that diversity-aware replay is most impactful when the number of available samples per class is limited, whereas uniform selection provides sufficient representation at larger budgets.















\if 0
Our results demonstrate that MADAR yields markedly better performances compared to previous methods for both the EMBER and AZ datasets across all CL settings. This clearly indicates that diversity-aware replay is effective in preserving the stability of a CL-based system for malware classification, while prior CL techniques largely fail to achieve acceptable performance.


In the Domain-IL scenario, MADAR consistently achieves better performance than all other methods, particularly at lower budgets. For example, MADAR's uniform and ratio variants surpass other methods with $\mathbf{\overline{AP}}$ values exceeding $93.6\%$ in EMBER and $95.7\%$ in AZ at a 1K budget. As the memory budget increases, the ratio-based variants (\system-R and MADAR$^\theta$-R) excel, approaching the \textit{Joint} baselines of $96.4\%$ for EMBER and $97.3\%$ for AZ. Notably, these results are achieved with significantly fewer replay samples compared to the \textit{Joint} baseline, highlighting MADAR's efficiency in leveraging limited resources.


In the Class-IL scenario, MADAR achieves remarkable improvements over prior methods, including iCaRL and TAMiL, on both EMBER and AZ datasets. For EMBER, \system-U achieves near \textit{Joint} baseline performance with a budget as low as 5K, outperforming iCaRL  method with fewer resources. Similarly, in AZ, MADAR$^\theta$-U reaches an impressive $\mathbf{\overline{AP}}$ of $91.5\%$ at a 20K budget, significantly surpassing prior techniques. Across both datasets, uniform variants (\system-U and MADAR$^\theta$-U) consistently outperform other methods, demonstrating their effectiveness in managing resources and adapting to evolving class distributions.


In the Task-IL scenario, MADAR outperforms prior methods by a significant margin for both the EMBER and AZ datasets, confirming that diversity-aware replay is effective for this scenario. For EMBER, \system-U achieves $\mathbf{\overline{AP}}$ values of $95.8\%$ at a 20K budget, effectively matching \textit{Joint} performance with a fraction of the resources. For AZ, MADAR$^\theta$-U attains $98.7\%$ at 20K, further underscoring the efficacy of diversity-aware techniques in resource-constrained settings.These findings highlight that the MADAR framework, particularly the uniform variant, not only matches but often exceeds the effectiveness of existing techniques, confirming its robustness across various budget levels in Task-IL.


The Ratio variants worked better for Domain-IL experiments, while Uniform variants worked well in Class-IL and Task-IL. Intuitively, this makes sense because ratio budgeting for binary classification in the Domain-IL setting naturally captures the contributions of each family to the overall malware distribution. Additionally, since there are many small families in the Domain-IL datasets, uniformly sampling from them consumes budget while offering little improvement in malware coverage. In contrast, our Class-IL and Task-IL experiments perform classification across families, which is better supported by Uniform budgeting to maintain class balance and ensure coverage over all families. Moreover, in most settings we can leverage efficient representations using MADAR$^\theta$ to scale the approach regardless of feature dimension without significant loss of performance.


Our results show that GRS performs very well, in some cases closer to the performances of MADAR. Indeed, uniform random sampling should be expected to be a strong baseline, since it provides an unbiased estimate of the true underlying distribution. MADAR is particularly effective in Class-IL and Task-IL, and for lower budgets in Domain-IL, while GRS generally performs as well as MADAR in higher-budget Domain-IL settings. We hypothesize that MADAR's diversity-aware approach is more important when the number of samples per class is limited. In our Domain-IL experiments, larger budgets enable a sufficient representation of the distributions of both classes with uniform selection, making MADAR useful only at smaller budget sizes. 
\fi 

















\section{Discussion}
%\subsection{Explore the Parameters of Chinese Calligraphy Brush}

\subsection{Capturing Brush Motion in Calligraphy}
When discussing techniques for capturing the process of brush calligraphy, we evaluated multiple possibilities and ultimately selected a solution that meets research needs while balancing flexibility and cost-effectiveness.

Firstly, we observed that hand movements exhibit minimal explicit variation (as the hand maintains a consistent grip on the brush), so capturing hand gestures was not prioritized. Instead, the movement characteristics of the brush itself stood out as the primary focus of our research. The brush shaft, with its relatively simple shape, is easier to abstract and model, prompting us to explore various techniques for capturing its motion.

One direct approach involves inferring the brush’s motion trajectory by analyzing the written output. For example, key positions such as stroke starting and ending points can be processed systematically to reconstruct writing movements~\cite{10.1145/3526114.3558657, 10.1145/3613904.3642792}. While this method is effective for creating stylistically consistent fonts, it falls short in capturing the personalized nuances in learners’ writing and the subtle motions of the brush. Another approach is leveraging optical motion capture devices, such as Leap Motion~\cite{weichert2013analysis} to track the brush. By marking key points on the brush, these devices can measure its position and tilt~\cite{Matsumaru_2017jaciii, 10.1145/3029798.3038422}. This method offers exceptional precision (up to 0.01mm) but cannot directly capture brush rotation, requiring additional sensors for complete data. To minimize the complexity and cost of the equipment while maintaining a natural writing experience, we opted to reduce auxiliary device usage~\cite{chang2007simplicity, mcinnerney2004online}. However, in non-teaching scenarios focused on detailed analysis, this high-precision technology could still be considered.

Contact-based motion sensing devices, such as the PHANToM Omni~\cite{silva2009phantom} and Force Dimension Omega~\cite{chen2018patient}, provide an alternative by recording and reproducing writing motions while offering haptic feedback to help learners perceive subtle actions~\cite{10.1145/1255047.1255063, nishino2011calligraphy}. However, these devices are not specifically designed for brush calligraphy, often requiring additional force to operate, which may compromise the natural writing experience. Moreover, their high cost limits accessibility and widespread adoption.

After weighing the options, we ultimately chose to attach small inertial sensors to the brush. This approach captures critical spatial characteristics such as tilt and rotation while maintaining low cost~\cite{10.1145/3559400.3565595}. This technique has been widely applied in motion analysis for activities such as golf~\cite{king2008wireless, nam2013golf, fitzpatrick2010validation} and archery~\cite{phang2024archery, zhao2016archery}, where the tools are structurally simple but require precise motion tracking. Although some errors exist~\cite{fedorov2015using}, they can be effectively mitigated through algorithmic compensation~\cite{albaghdadi2019optimized, 10724704, 9289769, 10.1007/978-981-13-2553-3_36} enhancing the adaptability of the approach. Our findings demonstrate that the collected data sufficiently reveal trends in brush tilt and rotation and can be used to compare students’ and instructors’ writing. Thus, considering the balance of research needs, device availability, flexibility, and cost-effectiveness, we adopted this method.


In addition to motion capture technologies, accurately capturing the written characters themselves is equally important in calligraphy studies. To address this, we adopted top-down cameras to record the writing process, which aligns with the intuitive way humans observe calligraphy results. Furthermore, for detecting the lifting motion of the brush, we chose not to rely on the previously adopted inertial sensors. This is because variations in writing pressure during strokes make it challenging to define a clear boundary between writing and non-writing heights. Instead, we used a side-view camera (a smartphone) to directly detect contact between the brush tip and the paper. This combination of top-down and side-view visual capture ensures comprehensive data collection while preserving the natural flow of the writing process.


\subsection{Multilevel Support for Calligraphy Learning}
% Our system provides tiered scaffolding to support learners at different stages. For beginners, it clearly displays subtle details that are hard to grasp, helping them master key elements of calligraphy from the outset. By breaking down techniques into steps, learners gradually comprehend complex strokes and build a solid foundation.

% For more experienced learners, the system offers advanced support. As they have a preliminary understanding of the rhythm, brushstroke transitions, and hand force, the detailed dynamic data provided by the system, such as brush rotation and stroke angles, helps them analyze these intricate movements in depth. This targeted feedback enables learners to refine their skills and understanding, unlocking the full potential of the system.

% Additionally, the system offers unmatched traceability compared to traditional teaching methods. Learners can pause, replay key movements, and save their practice records for future analysis and improvement. Through dynamic breakdowns of each stroke, learners can clearly see how multiple movements combine to form a complete stroke, greatly enhancing the effectiveness of kinesthetic learning.

% \REVISE{1. Added the characteristic that ``calligraphy learning is imitation-centric,'' outlining the progression from imitation to personalized creative expression.  
% 2. Highlighted the value of ``exploring the writing process of others’ works'' as a source of inspiration for advanced learners.  
% 3. Emphasized that ``writing involves multiple focal points,'' underscoring the system’s indispensable role in facilitating reflection on complex movements.  }



CalliSense offers a layered scaffolding approach tailored to the needs of learners at different stages, grounded in the imitation-centric nature of calligraphy learning. From beginner to advanced levels, learners progress by first deeply understanding the detailed brushwork of exemplary models and gradually integrating personal styles to achieve creative expression.

For beginners, the system clearly visualizes the intricate details that are often challenging to grasp, helping them identify key elements of calligraphy from the start. By breaking down techniques into manageable steps, learners can build a solid foundation as they gradually comprehend the complexity of brushstrokes. For advanced learners, CalliSense provides more sophisticated support. As they develop a foundational understanding of brushwork and rhythm, the system delivers detailed process data—such as brush rotation and stroke angles—that enables deeper analysis of these intricate movements, refining their technique and comprehension. Even for those who have mastered all brushwork techniques, exploring and appreciating the writing process behind others’ creations can provide valuable inspiration.

Moreover, self-reflection on brushwork details is an indispensable part of practice. Writing involves multiple focal points, making it difficult for learners to thoroughly review every action without assistance. CalliSense ensures traceability of the writing process, allowing learners to pause, replay key movements, and save practice records for future analysis and improvement. By dynamically deconstructing each stroke, the system helps learners clearly understand how multiple actions combine to form complete brushstrokes, significantly enhancing the effectiveness of kinesthetic learning.



\subsection{Integration of Independent Learning and Teacher-Guided Scenarios}
\textbf{Guided and self-directed learning.} In terms of application scenarios, the primary design goal of CalliSense is to support teacher-guided learning while holding potential for independent use as a self-learning tool. On one hand, the visualization of calligraphy processes aids teachers in explaining techniques. On the other hand, students can engage in autonomous practice by comparing their writing data with pre-saved examples from their instructors. However, despite efforts to ensure the system's hardware and software adaptability to both scenarios, CalliSense's current capabilities are not yet sufficient to fully support independent learning, particularly for beginners.
User interviews revealed challenges in standardizing calligraphy evaluation. Students often struggle to discern which differences in their writing require correction and which are acceptable when comparing their data with that of their teacher. Additionally, the complexity of calligraphy skills makes it difficult for learners to identify their current focus without explicit guidance. Supporting self-learning through CalliSense could involve designing features such as guided learning sequences~\cite{brydges2010comparing, duschl2011learning} or gamification to enhance engagement~\cite{suh2018enhancing, mohamad2018gamification}.

\textbf{Post-class learning tool.} Although CalliSense is currently centered on classroom support, its role in post-class review warrants further exploration. In this scenario, students—having identified key areas to improve—can independently experiment. Beyond mimicking the teacher’s techniques, they may explore variations in brush techniques and observe their impact on line quality. Repeated practice can lead to unintentional moments of success; for example, a student may accidentally replicate a line with texture strikingly similar to that of their teacher. These serendipitous moments, often fleeting, can be captured and analyzed using CalliSense’s ability to record detailed brushstroke parameters.

\textbf{ Peer comparison.} The system’s comparison functionality could also extend to peer interactions, where ``incorrect demonstrations'' play a vital role in learning~\cite{BOOTH201324}. By comparing brushstroke samples among classmates, students could deepen their understanding of calligraphy techniques.


\subsection{Diverse Applications Beyond the Classroom}
While initially designed for calligraphy teaching, the system's applications extend far beyond the classroom. By capturing fine dynamic details and visualizing data, it unveils hidden techniques in traditional calligraphy, surpassing the limitations of static demonstrations and addressing a long-standing challenge in calligraphy research: quantifying stroke effects and formation mechanisms~\cite{shi2023aesthetics}. We focused on teaching because of the high information transmission requirements, making it ideal for simplifying complex techniques. With parameters covering key variables like brush position, rotation, force, and speed, the system is well-suited for creation analysis, calligraphy exchanges, and academic discussions, showcasing its adaptability across a range of scenarios.

\subsection{Process Data Collection and Cross-Disciplinary Applications}
This study introduces a new method for collecting data on the calligraphy creation process, \REVISE{which is closely related to the concept of externalizing tacit knowledge\cite{ahmad2011influence, Virtanen_2011}}. While Chinese calligraphy emphasizes the final result, learners must understand how to achieve optimal stroke quality through brush posture, grip strength, and writing speed. \REVISE{These aspects of knowledge are often difficult to articulate or teach, as they are deeply embedded in the calligrapher’s muscle memory and sensory experience. Our system visualizes and analyzes these aspects of the creation process, helping learners connect hand control with stroke expression and providing precise feedback to facilitate skill mastery. By making these implicit elements accessible and comprehensible, our approach not only bridges the gap between expert practice and novice learning but also preserves and transmits the intricate art of calligraphy in a more structured and teachable form.}

This method also applies to other fields requiring fine motor control. A similar application is painting, which, like calligraphy, relies on brush control to affect stroke quality. Other areas include sports like golf, where posture recognition is crucial, and musical instrument performance, such as playing the violin. Additionally, by capturing brushstroke and hand force data, virtual pens in virtual reality (VR) and augmented reality (AR) can enhance learning experiences and support remote teaching.

% In healthcare, this system can help detect early signs of neurological disorders by analyzing hand movements, and infer emotional states based on force exertion. In safety monitoring, it can be used to assess drivers' habits and potential risks.

\subsection{Feasibility of Crowdsourced Data Collection}
The low cost of the system's equipment makes it accessible for widespread use, enabling large-scale data collection through crowdsourcing. Calligraphy enthusiasts, researchers, and students can contribute to data collection using simple devices, helping build a diverse calligraphy dataset. This approach can gather samples from various countries and regions, capturing writing habits across different skill levels and forming a representative dataset. By showcasing the dynamic parameters of brush movements throughout the writing process, this dataset addresses a gap in previous ICH research, particularly the often-overlooked factor of hand force. This process-based data collection not only offers new tools for calligraphy preservation but also provides significant technical methods for broader ICH research.

\subsection{Limitations and Future Work}
This study has limitations in capturing fine brush details in small characters (such as Xiaokai), as the strokes are too small to accurately record subtle variations. However, the brush techniques for small characters can be transferred from large character practice, making the system still valuable in large character training. Currently, the system relies on teacher explanations to interpret complex brush parameters. In the future, large language models (LLMs) could be introduced to automatically recognize and interpret these parameters, reducing dependence on teachers and enhancing system autonomy.

In terms of image recognition, overlapping strokes can sometimes cause errors in complex lighting or reflective conditions. Future research should focus on optimizing skeletonization and time alignment algorithms to improve system adaptability in these challenging environments. Future studies could also explore ink dynamics\cite{Matsumaru_2017jaciii}, investigating the relationship between ink density and brushstrokes, and generating dynamic textures based on parameter mapping. Additionally, developing real-time feedback that combines touch, sound, and visual cues~\cite{10.1145/3281505.3281604, 10.1145/3305367.3327993} could help learners instantly correct mistakes during practice. The system could further support long-term progress analysis, tracking and visualizing learners' brush movements over time to assist teachers and students in evaluating progress. In remote teaching, comparing student data with standard datasets would enable personalized feedback, improving the accuracy of guidance.


Mastering complex brush techniques often requires long-term practice. While this study focuses on the short-term effectiveness of CalliSense in enhancing brushwork learning, mastering short-term skills may provide a foundation for long-term skill consolidation and transfer~\cite{billing2007teaching}. Additionally, during the user research phase, we observed that clarifying key calligraphy brushwork concepts within the CalliSense curriculum significantly facilitates subsequent repetitive practice. This approach aligns closely with the principles of Scaffolding Theory~\cite{van2002scaffolding}. Future research could design longitudinal follow-ups to evaluate students' skill retention and application over time. For instance, tracking students' handwriting performance one month or longer after using the system, as well as assessing whether they can flexibly apply learned techniques across different calligraphy styles. Additionally, building on the positive findings from current student feedback, future studies could explore how the system might support self-directed long-term learning. By observing learning outcomes over an extended period, further evidence could be gathered on the role of clear visual feedback in fostering sustained progress in calligraphy learning.


\section{Conclusion}
This paper introduces CalliSense, an interactive educational tool designed to support Chinese calligraphy brushstroke techniques through process-based learning. Semi-structured studies revealed that brushstroke details are often difficult to detect, especially for beginners who tend to overlook these techniques. To address this, we developed a comprehensive solution that utilizes low-cost devices, such as smartphones, pressure sensors, and inertial sensors, to capture key parameters during the writing process. These parameters are aligned with the written characters and visualized through an intuitive interface. 

User studies demonstrate that CalliSense significantly enhances students' understanding of critical brushstroke techniques compared to traditional teaching methods, reducing their neglect of important details and strengthening their brushstroke awareness. Additionally, the system provides teachers with an effective tool for communicating complex writing actions, improving instructional efficiency. Overall, our work presents an innovative approach to calligraphy education, not only enhancing learners' awareness and execution of brushstroke techniques but also offering technological support for the preservation of intangible cultural heritage. Future work will focus on further optimizing the system and exploring its applications in broader cultural and educational contexts.




%%
%% The acknowledgments section is defined using the "acks" environment
%% (and NOT an unnumbered section). This ensures the proper
%% identification of the section in the article metadata, and the
%% consistent spelling of the heading.
\begin{acks}
This research was partially supported by the National Natural Science Foundation of China (No. 62202217), Guangdong Basic and Applied Basic Research Foundation (No. 2023A1515012889), and Guangdong Key Program (No. 2021QN02X794). We thank all of our study participants for their insightful discussions and feedback. We acknowledge the partial use of a large language model (LLM), specifically ChatGPT, to assist in the writing process. The LLM was employed as a tool for polishing the manuscript to enhance the clarity and quality of the text.
\end{acks}

%%
%% The next two lines define the bibliography style to be used, and
%% the bibliography file.
\bibliographystyle{ACM-Reference-Format}
\bibliography{sample}


%%
%% If your work has an appendix, this is the place to put it.
% \appendix


% \section{Online Resources}

% Nam id fermentum dui. Suspendisse sagittis tortor a nulla mollis, in
% pulvinar ex pretium. Sed interdum orci quis metus euismod, et sagittis
% enim maximus. Vestibulum gravida massa ut felis suscipit
% congue. Quisque mattis elit a risus ultrices commodo venenatis eget
% dui. Etiam sagittis eleifend elementum.

% Nam interdum magna at lectus dignissim, ac dignissim lorem
% rhoncus. Maecenas eu arcu ac neque placerat aliquam. Nunc pulvinar
% massa et mattis lacinia.

\end{document}
\endinput
%%
%% End of file `sample-sigconf-authordraft.tex'.
