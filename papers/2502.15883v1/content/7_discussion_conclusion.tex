\section{Discussion}
%\subsection{Explore the Parameters of Chinese Calligraphy Brush}

\subsection{Capturing Brush Motion in Calligraphy}
When discussing techniques for capturing the process of brush calligraphy, we evaluated multiple possibilities and ultimately selected a solution that meets research needs while balancing flexibility and cost-effectiveness.

Firstly, we observed that hand movements exhibit minimal explicit variation (as the hand maintains a consistent grip on the brush), so capturing hand gestures was not prioritized. Instead, the movement characteristics of the brush itself stood out as the primary focus of our research. The brush shaft, with its relatively simple shape, is easier to abstract and model, prompting us to explore various techniques for capturing its motion.

One direct approach involves inferring the brush’s motion trajectory by analyzing the written output. For example, key positions such as stroke starting and ending points can be processed systematically to reconstruct writing movements~\cite{10.1145/3526114.3558657, 10.1145/3613904.3642792}. While this method is effective for creating stylistically consistent fonts, it falls short in capturing the personalized nuances in learners’ writing and the subtle motions of the brush. Another approach is leveraging optical motion capture devices, such as Leap Motion~\cite{weichert2013analysis} to track the brush. By marking key points on the brush, these devices can measure its position and tilt~\cite{Matsumaru_2017jaciii, 10.1145/3029798.3038422}. This method offers exceptional precision (up to 0.01mm) but cannot directly capture brush rotation, requiring additional sensors for complete data. To minimize the complexity and cost of the equipment while maintaining a natural writing experience, we opted to reduce auxiliary device usage~\cite{chang2007simplicity, mcinnerney2004online}. However, in non-teaching scenarios focused on detailed analysis, this high-precision technology could still be considered.

Contact-based motion sensing devices, such as the PHANToM Omni~\cite{silva2009phantom} and Force Dimension Omega~\cite{chen2018patient}, provide an alternative by recording and reproducing writing motions while offering haptic feedback to help learners perceive subtle actions~\cite{10.1145/1255047.1255063, nishino2011calligraphy}. However, these devices are not specifically designed for brush calligraphy, often requiring additional force to operate, which may compromise the natural writing experience. Moreover, their high cost limits accessibility and widespread adoption.

After weighing the options, we ultimately chose to attach small inertial sensors to the brush. This approach captures critical spatial characteristics such as tilt and rotation while maintaining low cost~\cite{10.1145/3559400.3565595}. This technique has been widely applied in motion analysis for activities such as golf~\cite{king2008wireless, nam2013golf, fitzpatrick2010validation} and archery~\cite{phang2024archery, zhao2016archery}, where the tools are structurally simple but require precise motion tracking. Although some errors exist~\cite{fedorov2015using}, they can be effectively mitigated through algorithmic compensation~\cite{albaghdadi2019optimized, 10724704, 9289769, 10.1007/978-981-13-2553-3_36} enhancing the adaptability of the approach. Our findings demonstrate that the collected data sufficiently reveal trends in brush tilt and rotation and can be used to compare students’ and instructors’ writing. Thus, considering the balance of research needs, device availability, flexibility, and cost-effectiveness, we adopted this method.


In addition to motion capture technologies, accurately capturing the written characters themselves is equally important in calligraphy studies. To address this, we adopted top-down cameras to record the writing process, which aligns with the intuitive way humans observe calligraphy results. Furthermore, for detecting the lifting motion of the brush, we chose not to rely on the previously adopted inertial sensors. This is because variations in writing pressure during strokes make it challenging to define a clear boundary between writing and non-writing heights. Instead, we used a side-view camera (a smartphone) to directly detect contact between the brush tip and the paper. This combination of top-down and side-view visual capture ensures comprehensive data collection while preserving the natural flow of the writing process.


\subsection{Multilevel Support for Calligraphy Learning}
% Our system provides tiered scaffolding to support learners at different stages. For beginners, it clearly displays subtle details that are hard to grasp, helping them master key elements of calligraphy from the outset. By breaking down techniques into steps, learners gradually comprehend complex strokes and build a solid foundation.

% For more experienced learners, the system offers advanced support. As they have a preliminary understanding of the rhythm, brushstroke transitions, and hand force, the detailed dynamic data provided by the system, such as brush rotation and stroke angles, helps them analyze these intricate movements in depth. This targeted feedback enables learners to refine their skills and understanding, unlocking the full potential of the system.

% Additionally, the system offers unmatched traceability compared to traditional teaching methods. Learners can pause, replay key movements, and save their practice records for future analysis and improvement. Through dynamic breakdowns of each stroke, learners can clearly see how multiple movements combine to form a complete stroke, greatly enhancing the effectiveness of kinesthetic learning.

% \REVISE{1. Added the characteristic that ``calligraphy learning is imitation-centric,'' outlining the progression from imitation to personalized creative expression.  
% 2. Highlighted the value of ``exploring the writing process of others’ works'' as a source of inspiration for advanced learners.  
% 3. Emphasized that ``writing involves multiple focal points,'' underscoring the system’s indispensable role in facilitating reflection on complex movements.  }



CalliSense offers a layered scaffolding approach tailored to the needs of learners at different stages, grounded in the imitation-centric nature of calligraphy learning. From beginner to advanced levels, learners progress by first deeply understanding the detailed brushwork of exemplary models and gradually integrating personal styles to achieve creative expression.

For beginners, the system clearly visualizes the intricate details that are often challenging to grasp, helping them identify key elements of calligraphy from the start. By breaking down techniques into manageable steps, learners can build a solid foundation as they gradually comprehend the complexity of brushstrokes. For advanced learners, CalliSense provides more sophisticated support. As they develop a foundational understanding of brushwork and rhythm, the system delivers detailed process data—such as brush rotation and stroke angles—that enables deeper analysis of these intricate movements, refining their technique and comprehension. Even for those who have mastered all brushwork techniques, exploring and appreciating the writing process behind others’ creations can provide valuable inspiration.

Moreover, self-reflection on brushwork details is an indispensable part of practice. Writing involves multiple focal points, making it difficult for learners to thoroughly review every action without assistance. CalliSense ensures traceability of the writing process, allowing learners to pause, replay key movements, and save practice records for future analysis and improvement. By dynamically deconstructing each stroke, the system helps learners clearly understand how multiple actions combine to form complete brushstrokes, significantly enhancing the effectiveness of kinesthetic learning.



\subsection{Integration of Independent Learning and Teacher-Guided Scenarios}
\textbf{Guided and self-directed learning.} In terms of application scenarios, the primary design goal of CalliSense is to support teacher-guided learning while holding potential for independent use as a self-learning tool. On one hand, the visualization of calligraphy processes aids teachers in explaining techniques. On the other hand, students can engage in autonomous practice by comparing their writing data with pre-saved examples from their instructors. However, despite efforts to ensure the system's hardware and software adaptability to both scenarios, CalliSense's current capabilities are not yet sufficient to fully support independent learning, particularly for beginners.
User interviews revealed challenges in standardizing calligraphy evaluation. Students often struggle to discern which differences in their writing require correction and which are acceptable when comparing their data with that of their teacher. Additionally, the complexity of calligraphy skills makes it difficult for learners to identify their current focus without explicit guidance. Supporting self-learning through CalliSense could involve designing features such as guided learning sequences~\cite{brydges2010comparing, duschl2011learning} or gamification to enhance engagement~\cite{suh2018enhancing, mohamad2018gamification}.

\textbf{Post-class learning tool.} Although CalliSense is currently centered on classroom support, its role in post-class review warrants further exploration. In this scenario, students—having identified key areas to improve—can independently experiment. Beyond mimicking the teacher’s techniques, they may explore variations in brush techniques and observe their impact on line quality. Repeated practice can lead to unintentional moments of success; for example, a student may accidentally replicate a line with texture strikingly similar to that of their teacher. These serendipitous moments, often fleeting, can be captured and analyzed using CalliSense’s ability to record detailed brushstroke parameters.

\textbf{ Peer comparison.} The system’s comparison functionality could also extend to peer interactions, where ``incorrect demonstrations'' play a vital role in learning~\cite{BOOTH201324}. By comparing brushstroke samples among classmates, students could deepen their understanding of calligraphy techniques.


\subsection{Diverse Applications Beyond the Classroom}
While initially designed for calligraphy teaching, the system's applications extend far beyond the classroom. By capturing fine dynamic details and visualizing data, it unveils hidden techniques in traditional calligraphy, surpassing the limitations of static demonstrations and addressing a long-standing challenge in calligraphy research: quantifying stroke effects and formation mechanisms~\cite{shi2023aesthetics}. We focused on teaching because of the high information transmission requirements, making it ideal for simplifying complex techniques. With parameters covering key variables like brush position, rotation, force, and speed, the system is well-suited for creation analysis, calligraphy exchanges, and academic discussions, showcasing its adaptability across a range of scenarios.

\subsection{Process Data Collection and Cross-Disciplinary Applications}
This study introduces a new method for collecting data on the calligraphy creation process, \REVISE{which is closely related to the concept of externalizing tacit knowledge\cite{ahmad2011influence, Virtanen_2011}}. While Chinese calligraphy emphasizes the final result, learners must understand how to achieve optimal stroke quality through brush posture, grip strength, and writing speed. \REVISE{These aspects of knowledge are often difficult to articulate or teach, as they are deeply embedded in the calligrapher’s muscle memory and sensory experience. Our system visualizes and analyzes these aspects of the creation process, helping learners connect hand control with stroke expression and providing precise feedback to facilitate skill mastery. By making these implicit elements accessible and comprehensible, our approach not only bridges the gap between expert practice and novice learning but also preserves and transmits the intricate art of calligraphy in a more structured and teachable form.}

This method also applies to other fields requiring fine motor control. A similar application is painting, which, like calligraphy, relies on brush control to affect stroke quality. Other areas include sports like golf, where posture recognition is crucial, and musical instrument performance, such as playing the violin. Additionally, by capturing brushstroke and hand force data, virtual pens in virtual reality (VR) and augmented reality (AR) can enhance learning experiences and support remote teaching.

% In healthcare, this system can help detect early signs of neurological disorders by analyzing hand movements, and infer emotional states based on force exertion. In safety monitoring, it can be used to assess drivers' habits and potential risks.

\subsection{Feasibility of Crowdsourced Data Collection}
The low cost of the system's equipment makes it accessible for widespread use, enabling large-scale data collection through crowdsourcing. Calligraphy enthusiasts, researchers, and students can contribute to data collection using simple devices, helping build a diverse calligraphy dataset. This approach can gather samples from various countries and regions, capturing writing habits across different skill levels and forming a representative dataset. By showcasing the dynamic parameters of brush movements throughout the writing process, this dataset addresses a gap in previous ICH research, particularly the often-overlooked factor of hand force. This process-based data collection not only offers new tools for calligraphy preservation but also provides significant technical methods for broader ICH research.

\subsection{Limitations and Future Work}
This study has limitations in capturing fine brush details in small characters (such as Xiaokai), as the strokes are too small to accurately record subtle variations. However, the brush techniques for small characters can be transferred from large character practice, making the system still valuable in large character training. Currently, the system relies on teacher explanations to interpret complex brush parameters. In the future, large language models (LLMs) could be introduced to automatically recognize and interpret these parameters, reducing dependence on teachers and enhancing system autonomy.

In terms of image recognition, overlapping strokes can sometimes cause errors in complex lighting or reflective conditions. Future research should focus on optimizing skeletonization and time alignment algorithms to improve system adaptability in these challenging environments. Future studies could also explore ink dynamics\cite{Matsumaru_2017jaciii}, investigating the relationship between ink density and brushstrokes, and generating dynamic textures based on parameter mapping. Additionally, developing real-time feedback that combines touch, sound, and visual cues~\cite{10.1145/3281505.3281604, 10.1145/3305367.3327993} could help learners instantly correct mistakes during practice. The system could further support long-term progress analysis, tracking and visualizing learners' brush movements over time to assist teachers and students in evaluating progress. In remote teaching, comparing student data with standard datasets would enable personalized feedback, improving the accuracy of guidance.


Mastering complex brush techniques often requires long-term practice. While this study focuses on the short-term effectiveness of CalliSense in enhancing brushwork learning, mastering short-term skills may provide a foundation for long-term skill consolidation and transfer~\cite{billing2007teaching}. Additionally, during the user research phase, we observed that clarifying key calligraphy brushwork concepts within the CalliSense curriculum significantly facilitates subsequent repetitive practice. This approach aligns closely with the principles of Scaffolding Theory~\cite{van2002scaffolding}. Future research could design longitudinal follow-ups to evaluate students' skill retention and application over time. For instance, tracking students' handwriting performance one month or longer after using the system, as well as assessing whether they can flexibly apply learned techniques across different calligraphy styles. Additionally, building on the positive findings from current student feedback, future studies could explore how the system might support self-directed long-term learning. By observing learning outcomes over an extended period, further evidence could be gathered on the role of clear visual feedback in fostering sustained progress in calligraphy learning.


\section{Conclusion}
This paper introduces CalliSense, an interactive educational tool designed to support Chinese calligraphy brushstroke techniques through process-based learning. Semi-structured studies revealed that brushstroke details are often difficult to detect, especially for beginners who tend to overlook these techniques. To address this, we developed a comprehensive solution that utilizes low-cost devices, such as smartphones, pressure sensors, and inertial sensors, to capture key parameters during the writing process. These parameters are aligned with the written characters and visualized through an intuitive interface. 

User studies demonstrate that CalliSense significantly enhances students' understanding of critical brushstroke techniques compared to traditional teaching methods, reducing their neglect of important details and strengthening their brushstroke awareness. Additionally, the system provides teachers with an effective tool for communicating complex writing actions, improving instructional efficiency. Overall, our work presents an innovative approach to calligraphy education, not only enhancing learners' awareness and execution of brushstroke techniques but also offering technological support for the preservation of intangible cultural heritage. Future work will focus on further optimizing the system and exploring its applications in broader cultural and educational contexts.


