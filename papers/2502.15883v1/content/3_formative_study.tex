\section{Formative Study}
The key to learning Chinese calligraphy lies in mastering fundamental techniques and maintaining consistent practice. The learning process begins with adopting the correct brush-holding posture, understanding basic strokes, and grasping the structure of characters\cite{zhang2023bringing, shi2019chinese}. Through repeated practice, students can develop muscle memory, improve stroke accuracy, and master the rhythm and fluidity of writing, which are crucial for the expressiveness of calligraphy. Incorporating cultural elements can make the learning process more engaging, while timely feedback and encouragement help students progress and express themselves\cite{hue2010aestheticism}. As students delve deeper into their studies, they can explore their own creativity within the traditional framework and enhance their understanding of calligraphy. This practice blends brushwork and ink control, allowing the mastery of angles, pressure, and speed to produce lines rich in personality and emotion\cite{chiang1974chinese}. To explore learning challenges in brush techniques, we conducted a formative study for design insights.


\subsection{Semi-structured Interviews}
Semi-structured interviews were conducted with eight calligraphy practitioners (four males, four females) to gain insights into their needs concerning the visualization of brushwork in calligraphy. The exploration focused on the following aspects:
\begin{enumerate}
    \item The challenges calligraphy practitioners face in their practice, particularly in brush handling, and which of these challenges are the most difficult to overcome.
    \item Identifying design opportunities for CalliSense to address these challenges.
\end{enumerate}

Our tool is designed for calligraphy learners that include practitioners of varying skill levels, ranging from beginners to calligraphy experts. Through the multi-level interviews, we aimed to understand the obstacles learners face in their practice and to gather insights from teachers on the common difficulties students encounter and the challenges they face in instruction. Additionally, it is believed that individuals who have planned to learn calligraphy but ultimately quit due to various reasons also offer valuable perspectives for the interviews. This group can reveal key barriers to learning traditional culture and help us identify and address potential issues.

The interviewees included two calligraphy experts (A1 and A2), who have been deeply involved in the field for 15 and 35 years, respectively, and are well-regarded in the community. Both experts have extensive experience in practicing and teaching calligraphy and will continue to serve as consultants, providing ongoing insights for our design process. The remaining six interviewees were volunteers recruited online. Four of them are calligraphy practitioners (B1–B4), representing different age groups  (mean = 33, SD = 13.4) and having practiced for over a year, with a basic understanding of calligraphy techniques. The remaining two interviewees (C1 and C2) are individuals who previously practiced calligraphy but have not continued for nearly a year. Though their experience was brief, they offer a valuable beginner's perspective.

The interviews began by asking the three groups about the difficulties they encountered in calligraphy practice and the methods they used to overcome them. This was followed by a discussion on the key aspects they focused on during practice. For the two calligraphy experts, we additionally explored their teaching experiences to further investigate the most challenging issues students usually meet in the learning process. 
 
\subsection{Analysis and Results}
Our semi-structured interviews were conducted over the phone, with each interview lasting approximately 30 minutes and recorded for later transcription. To analyze the interview data, we employed thematic analysis. First, the co-authors read through the transcripts to gain an overall understanding. After familiarizing themselves with the data, they independently performed open coding. Upon completing the coding, the co-authors shared their results, discussed their interpretations of the data, and reached a consensus on the final coding outcomes. 

\subsubsection{The Lack of Awareness in Brushwork}

In this category, we have identified two issues:

\textbf{Beginners' Tendency to Overlook the Importance of Brushwork:}
For beginners, understanding correct brushwork means knowing what constitutes proper stroke techniques, brush pressure, and brush movements, as well as how to achieve these effects. Teachers believe that while mastering brush techniques requires long-term practice, having the right awareness beforehand is more crucial to avoid forming bad habits that are hard to correct (A1: ``If students realize the importance of brushwork and keep practicing, they will improve steadily. The real danger is when students mistakenly believe they are doing it correctly. That's when it's a problem.''). Using the correct methods makes learning more efficient, while the wrong approach could lead to frustration and even quitting (B3: ``Finding the right teacher is important. If the method is correct, it becomes easy to apply knowledge in new situations. Otherwise, learning becomes difficult, and it's easy to give up.''). However, there is currently a shortage of calligraphy instructors, and learners often lack a full understanding of the importance of brushwork in the early stages, making it difficult to judge what constitutes a good piece of work (C1: ``When I practiced, I only focused on whether the characters looked right. I didn't pay attention to brushwork unless the teacher pointed it out.''). Therefore, it is crucial for calligraphy learners to recognize the key role that brushwork plays from the very beginning of their studies.

\textbf{Reliance on External Feedback}: Both calligraphy practitioners and teachers generally believe that beginners rely heavily on a teacher's guidance to avoid going astray (N=6) (A1: ``There's a widely shared, albeit somewhat `biased' saying in the calligraphy community: `Self-learning is tantamount to self-destruction.' ''). This is because learners can grasp brush techniques and variations in rhythm through a teacher's demonstration, something that is difficult to experience solely by following copybooks (N=7) (B2: ``When it comes to brushwork, I find a rolling stroke technique in cursive script particularly challenging. It requires the teacher's demonstration to fully understand it. It's hard to grasp from a copybook alone.''). Additionally, practitioners often find it difficult to correct their mistakes on their own during writing (N=5) (A2: ``I knew there was a problem, but I didn't know how to fix it. Learning helped me figure out how to make those corrections.'' B1: ``My hand trembles when I write, probably because I'm holding the brush too tightly, but I don't know why this happens.''). Even experienced calligraphy practitioners who have been practicing for years may need to attend specialized institutions and seek guidance from more experienced teachers to correct ingrained habits that are difficult to detect (A2: ``When I was practicing calligraphy, I always struggled to match the copybook style. It wasn't until I attended an institution that I realized my habits were off, and the learning process was quite painful.''). Therefore, it is essential for practitioners to receive feedback on their brushwork techniques during the writing process. 

\subsubsection{Difficult-to-Detect Brushwork Techniques} In addition, three aspects regarding learning specific techniques are identified:

\textbf{Difficulty in Observing Brush Techniques Through Strokes:} Unlike regular writing tools, the brush is challenging for beginners to master, and they may struggle with how to control it (N=7). (A2: ``Beginners find it hard to grasp the characteristics of the brush and don't know how to apply force.'') The static appearance of the characters makes it difficult to observe the brushwork process, which leads to situations where learners cannot find the connection between brush techniques and the resulting strokes when copying calligraphy models. They also don't know how to control the brush through hand movements, making it hard to achieve their writing goals (N=6). (B1: ``Strokes are really the result of brush-tip movements, but when I look at ancient calligraphy models, I can't see this. I don't know how to practice.'') Therefore, students need to be shown the brushwork process corresponding to specific calligraphy strokes.

\textbf{Demonstration Limitations:} Calligraphy instruction often involves numerous abstract terms, such as center stroke (zhongfeng), wrapping stroke (guofeng), and reverse stroke (nifeng). Students find it challenging to fully understand these terms through verbal explanations alone. To address this issue, teacher demonstrations are the most common and effective method. However, students have limited observational abilities and often miss critical details in the teacher's demonstrations (N=6). (A1: ``When discussing the concept of `center stroke', students might spread the brush bristles wide while writing a horizontal stroke, thinking this will create a thicker line. However, this approach does not align with the true principle of the center stroke.'')

During demonstrations, it is difficult for students to simultaneously analyze the stroke, hand movements, and the complex changes in the brush bristles. As a result, they still struggle to understand what kind of writing process corresponds to a specific calligraphy term. Therefore, when teaching brush techniques, it is essential to break down the movements of the brush in sufficient detail.

\textbf{Hand Force Cannot Be Observed:} Calligraphy writing relies on precise control of the brush by the hand, and learners typically rely on teacher demonstrations or videos to learn. However, not only is it difficult to capture the subtle movements of the brush handle, but the force applied by the hand to the brush is inherently hard to detect with the naked eye (N=5). (B1: ``When writing small characters, my hand often trembles. I'm not sure if I'm gripping the brush too tightly. The teacher didn't specifically mention this, and I can't figure it out.'' A1: ``I also realized that students were gripping the brush too tightly after teaching for a while. Now, I can roughly judge if they are applying force incorrectly based on the strokes, but younger teachers may not be able to do this.'')

Although teachers explain the brushwork details during demonstrations, they often do not cover all the key points and may not be aware of the specific aspects students are focusing on, making it challenging to address their learning needs. For instance, when writing long strokes, in order to achieve dynamic movement in the middle of the line and clean, crisp ends, the hand's force typically follows a pattern of tension—relaxation—tension. However, since the explanation often focuses on the tip of the brush, this technique is frequently overlooked by students. Therefore, the pressure applied by the hand to the brush needs to be demonstrated explicitly.

\subsubsection{Forgotten Writing Process: }In this category, two issues have been identified: 

\textbf{Forgetting Brushwork Details:} In traditional calligraphy instruction, to avoid interrupting students' writing, teachers typically provide feedback after the student has finished. However, by this time, the details of the writing process are often forgotten (B4: ``Sometimes I don't understand what the teacher is referring to, and I need to write it again to make sense of it''). On occasion, students even bring completed works to the teacher for critique. While experienced teachers can quickly identify issues, students, having forgotten their own writing process, struggle to connect the teacher's feedback with their performance at the time. As a result, they can only mechanically record the feedback and reflect on it later, missing the opportunity for real-time interaction with the teacher. Therefore, it is essential that students' writing process be more fully captured and reconstructed.

\textbf{Forgetting Writing Rhythm:} Writing rhythm is a critical element in calligraphy practice, as the distribution of hand force and writing speed throughout the character directly impacts the overall appearance of the Chinese character\cite{wang2024standards, kraus1991brushes}. In well-resourced teaching environments, instructors typically ask students to rewrite problematic strokes and provide feedback. However, even with rewrites, it is challenging for students to fully recreate the original writing rhythm. On the one hand, students may consciously adjust their natural writing state when being observed by the teacher. On the other hand, the distribution of force and speed during writing is complex, making it difficult to recall the overall rhythm from a macro perspective through rewriting alone. Therefore, students need an ``overview'' perspective to review and comprehend the complete writing rhythm.

\subsection{Design Consideration}
Based on the findings in the formative study, we identified five design considerations to build a system that supports designers' reference recombination process during early-stage ideation: 

\textbf{DC1: Capture the Complete Writing Process} 
To accurately reconstruct the writing motion, the system should capture the entire writing process of a character, including detailed brush techniques, ensuring that the dynamic changes of each stroke and brush movement are recorded.

\textbf{DC2: Correlate Strokes with Brush Techniques}
Although the quality of the lines can be directly assessed through the writing results, the specific brush techniques used to create these lines are equally important. The system should be capable of precisely pinpointing specific segments of the strokes and reviewing corresponding brush details, such as brush posture and finger pressure. At the same time, it is essential to ensure that the writing process aligns with the final strokes for accurate analysis.

\textbf{DC3: Review Overall Writing Rhythm}
Writing rhythm influences the internal contrast and variation within a character. Therefore, the system should support reviewing the overall rhythm after writing is completed, allowing users to observe the variations in hand force and writing speed throughout the character.

\begin{table*}
\caption{Brush Measurement Parameters and Their Impact in Calligraphy}
\label{tab:calligraphy-measurements}
\setlength{\tabcolsep}{4pt}   % 调整列间距
\renewcommand{\arraystretch}{1.5}  % 调整行距
\begin{tabular}{p{3cm} p{4cm} p{9cm}} % 使用 m{} 使内容垂直居中
\hline
\textbf{Measurement Object}   & \textbf{Measurement Parameter}  & \textbf{Impact and Significance in Calligraphy} \\ \hline
\textbf{Brush Handle}  
    & Tilt  
    & Affects the friction between the brush and the paper, which in turn alters the strength and expressiveness of the strokes. \\ \cline{2-3} 
    & Rotation  
    & Influences the position and organization of the brush tip, thereby impacting the texture of the strokes. \\ \cline{2-3} 
    & Speed  
    & Affects the contact time between the brush and the paper, altering ink absorption and subsequently affecting stroke thickness and ink intensity. \\ \hline
\textbf{Fingers}  
    &  
    & Affects how the brush hairs interact with the paper, thereby influencing the texture of the strokes. \\ \hline
\textbf{Brush Tip}  
    &  
    & Affects the force applied through the brush to the paper, leading to variations in the texture of the strokes. \\ \hline
\end{tabular}
\vspace{10pt}
\end{table*}

\begin{figure*}[t!]
  \centering
  \includegraphics[width=\textwidth]{design_fig/dataprocessing.png}
  \caption{Overview of the work: The entire process from data collection and processing to visualization}
  \label{fig:overview}
\end{figure*}

\textbf{DC4: Examine Brush Tip Dynamics}
The brush tip is a critical factor that directly affects stroke quality. The system should provide a clear visualization of the brush tip's behavior during writing.

\textbf{DC5: Identify Brush Technique Errors Through Comparison}
The system should enable users to compare their brush techniques with those of the instructor, helping to quickly identify errors in brush posture, hand force, writing rhythm, and other aspects, while providing effective feedback for improvement.

Based on these five design principles, we propose the CalliSense system, an interactive tool for visualizing the calligraphy writing process. The system consists of two main components: (1) a camera and sensor suite for capturing the writing process, and (2) a web interface for visualizing the brushstroke process and allowing user control.

