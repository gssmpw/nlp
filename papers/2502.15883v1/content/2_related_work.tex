\section{Related Work}
\subsection{Motion Capture and Analysis in Intangible Cultural Heritage}
The motion processes in intangible cultural heritage, such as calligraphy, dance, and traditional crafts, are important for skill transmission~\cite{bortolotto2007objects, lenzerini2011intangible, vecco2010definition}. With digital technologies including motion capture, sensors, and visual recognition, such processes can now be recorded accurately, which provides valuable data for research, education, and preservation.

Traditionally, intangible cultural heritage was passed down orally or documented in books. Early digital recordings focused on audio and video~\cite{pietrobruno2009cultural}, such as the preservation of Tiwi songs and dances in public archives~\cite{inbook}. Additionally, cultural soundscapes~\cite{samuels2010soundscapes} have been used to document and recreate the sound environments of cultures~\cite{noviandri2023cultural, Bartalucci_2020, doi:10.1080/13527258.2016.1138237}. Techniques like photogrammetry and optical motion capture have been employed to document heritage, as seen in the Quanzhou Chest-Clapping Dance~\cite{kirchhofer2011cultural, chen2014}. Modern approaches, such as wearable devices~\cite{protopapadakis2020digitizing}, multi-camera LiDAR\cite{caterina_balletti__2023}, and multimodal platforms like i-Treasures\cite{8255779} and CHROMATA, enable detailed modeling, analysis, and immersive virtual experiences\cite{9480948, selmanovic2018vr}, offering innovative ways to preserve and display intangible cultural heritage.

In the field of Chinese calligraphy, various technologies have been introduced to aid learning and replicate traditional writing processes. For example, although they differ from the actual writing experience, plotters and robotic arms attempt to replicate the movements of calligraphers by simulating the reactive force of real handwriting~\cite{nishino2011calligraphy, 10.1145/3613904.3642792, 10.1145/3526114.3558657}. Additionally, remote haptic systems guide students in using calligraphy brushes via network-controlled devices to ensure high-quality tactile feedback~\cite{10.1145/1255047.1255063}. Specially designed brushes equipped with sensor technology are also used to record writing trajectories~\cite{10.1145/3029798.3038422, Matsumaru_2017jaciii} and convey some aspects of the motion~\cite{10.1145/3559400.3565595}. However, the authentic texture (Figure \ref{fig:lantingji_xu}) and feel of traditional brushes remain difficult to replicate with digital brushes. Furthermore, the digital exploration of calligraphy, including multi-target detection~\cite{10.1007/978-981-15-3867-4_27}, style classification~\cite{10.1007/978-3-031-41685-9_5}, and visual appreciation~\cite{zhang2023visual}, is enriching both the teaching and cultural preservation of calligraphy. 

\begin{figure}[t!]
  \centering
  \includegraphics[width=\columnwidth]{design_fig/lantingji_xu_image.png}
  \caption{Reproduction of the famous Chinese stele inscription \textit{Lantingji Xu} from {Wikipedia}\protect\footnotemark}
  \label{fig:lantingji_xu}
\end{figure}







Although existing technologies help preserve intangible cultural heritage such as dance and sculpture, the recording of the calligraphy writing process remains limited and often focuses on imitation rather than authentic documentation. Current robotic arms and plotters attempt to replicate the writing process, but they fall short of capturing the true creative act. Additionally, modern technology primarily focuses on recording the character form, with little research on how brush posture and hand pressure affect the outcome.

For motion-based intangible heritage, most studies focus on body posture, with less attention paid to the control and movement of tools. As the medium connecting the hand to the paper, the brush's role deserves further exploration. CalliSense aims to document the brush's movements during natural writing and explore the relationship between motion and line formation.



\subsection{Visual Feedback Through Motion Analysis in Kinesthetic Learning}
\footnotetext{\url{https://zh.wikipedia.org/wiki/File:LantingXu.jpg\#file}}
Proprioceptive demonstration is a practice that enhances an individual's perception of their own position, movement, and posture through bodily actions and sensory feedback~\cite{tuthill2018proprioception, 10.1080/00222895.1974.10734977, winter2022effectiveness}. Individuals can experience specific movement trajectories guided by external forces or actively execute actions relying on their own sensory feedback~\cite{wong2012can, FARRER2003609}. Richard A. Magill's concept of kinesthetic feedback emphasizes the natural perception of body position, speed, and force, helping to adjust and optimize movements for more precise control, supporting the effectiveness of learning calligraphy skills through bodily perception in a natural writing state~\cite{magill2010motor}. 

In the field of human-computer interaction, kinesthetic feedback relies on efficient motion capture. In skill motion capture involving tools, such as archery, sensor technologies are widely applied. For example, Zhao et al. used wearable devices and accelerometer-based methods to analyze archers' "release" actions~\cite{zhao2016archery}. Similarly, Phang et al. utilized inertial measurement units (IMU) and fast Fourier transform (FFT) methods to examine the micro-movements of the bow~\cite{phang2024archery}. In golf~\cite{king2008wireless, fitzpatrick2010validation}, Nam et al. applied inertial sensors and stereo camera techniques to track the motion trajectory of golf clubs, demonstrating the effectiveness of these technologies in capturing detailed tool dynamics for skill enhancement~\cite{nam2013golf}.

In terms of feedback, numerous studies have demonstrated that providing rich, visual feedback through technology can significantly improve skill training. For example, in areas such as dance, painting, and musical instrument performance, techniques like beat comparison~\cite{10.1145/3274247.3274514}, shadow generation~\cite{10.1145/2010324.1964922}, hand motion capture~\cite{10.1145/3544549.3585838}, and the use of virtual avatars~\cite{10.1007/978-3-030-36126-6_7} and projections~\cite{5557840} for learning and feedback have proven effective. These methods allow learners to more intuitively understand and adjust their movements. Additionally, AR real-time feedback, visual cues, and multimodal feedback are widely used in fitness and music instruction~\cite{10.1145/3411764.3445649, 10.1145/3491102.3517735, 10.1145/3411764.3445595}. In golf, similar to Chinese calligraphy, rigid body devices are used to study various feedback mechanisms, including body angle visualization~\cite{10.1145/1878083.1878098}, ground shadows~\cite{10.1145/3305367.3327993}, sound cues~\cite{10.1145/3305367.3327993, 10.1145/3281505.3281604}, posture comparison waveforms~\cite{10.1145/3476124.3488645}, and multimodal feedback comparisons~\cite{10.1145/3427332}.

In Chinese calligraphy research, visual enhancement techniques have proven effective in improving character training. For instance, augmented reality overlays digital information onto the real world, providing real-time guidance and feedback for calligraphy learning, thus enhancing efficiency and experience~\cite{10.1145/1935701.1935769}. Interactive projection systems also help learners practice character shapes effectively by projecting calligraphy fonts and stroke sequences in real-time~\cite{9122337}. Other studies use virtual paper and digital brushes to offer real-time feedback and animation demonstrations~\cite{sym11091071}.
Current research largely focuses on the correctness of the final result, with limited attention to the detailed process of brush handling. Even when some studies attempt to correct the writer's posture, they often emphasize overall body posture rather than the fine control of the hand over the brush~\cite{10.1145/1935701.1935769}, or simply stress following the correct stroke path~\cite{10.1145/3029798.3038422}. Other studies have explored brush pressure and movement speed~\cite{10.1145/3377325.3377534}, but have not fully analyzed how these parameters affect the quality of the strokes, nor do they offer specific guidance for students practicing key techniques. Additionally, no human-computer interaction technology currently records and replays the detailed brushwork of students' writing processes in their natural state. In the next section, we will delve into the challenges of process learning in calligraphy and demonstrate how the CalliSense system addresses these issues.

