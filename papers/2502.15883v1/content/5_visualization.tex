\section{Visualization Design}
Our visualization framework sequentially covers glyph observation, rhythm observation, and detailed stroke analysis. During the iterative design process, expert feedback was incorporated, emphasizing that calligraphy instruction typically begins with an overall view of the character's form.
 Therefore, the initial interface displays only the character's overall structure, concealing brushwork details (Figure \ref{fig:Structure Analysis}). This design encourages learners to first understand the overall composition and identify gaps between their work and the teacher's, allowing for targeted improvement in the next stage. This approach aligns with the cognitive process of traditional Chinese calligraphy training~\cite{dong2008creation}. Starting with glyph critiques also helps beginners unfamiliar with brushwork techniques to build confidence in a domain they may find challenging.

During the iterative process, an interesting phenomenon was discovered: rhythm changes in dance and music were frequently used by calligraphy experts to elucidate the importance of varying writing pressure and speed. Calligraphy and dance are often compared in studies due to their similar rhythmic qualities~\cite{szeto2010calligraphic}. Building on the conclusions from the previous formative study\textbf{ (DC3)}, an overview of writing rhythm is introduced after the character structure analysis. Users can then select specific strokes and proceed to the next interface to analyze individual lines (Figure \ref{fig:Rhythm and Stroke Analysis}). This design also follows the standard framework of information visualization: (a) provide an overview first, (b) allow zoom and filter, and (c) offer details on demand\cite{shneiderman2003eyes}. Overall, a comparison between each part for the teacher and the student was provided, enabling the student to identify where the issues with posture are present \textbf{(DC5)}.

\subsection{Glyph Comparison}

\begin{figure}[t]
    \centering
    \includegraphics[width=\linewidth]{design_fig/interface/Structure_Analysis.jpg}
    \caption{Step 1: Character Shape Comparison. The left side shows the teacher's writing, while the right side displays the student's work. Students can use the teacher's writing as a reference to identify structural issues in their own writing. Additionally, they can choose to use two types of guidelines—``Structural Boundary'' and ``Form Boundary''—to assist in their assessment.}
    \label{fig:Structure Analysis}
\end{figure}

To provide a more intuitive comparison between the teacher and student, we added the commonly used guiding grid – the ``Mi Zi Ge'' grid\cite{chinese_calligraphy_grids}. The grid not only serves as a reference but also helps observe the relative positions of the strokes.

Additionally, two optional boundary boxes were incorporated (Figure \ref{fig:bounding box}). The first boundary connects the four extremities of the character, showing the relative positions of the strokes. The second boundary uses these four points as edges to form a grid, allowing for a clearer view of the character's proportions and its central position. With these tools, the teacher and student can drag and compare characters to further analyze differences in writing techniques (Figure \ref{fig:drag}).

\begin{figure}[H]
    \centering
    \includegraphics[width=\linewidth]{design_fig/interface/drag.png}
    \caption{Compare the structure of students' and teachers' handwriting through dragging.}
    \label{fig:drag}
\end{figure}

% \vspace{-5mm}

\begin{figure}[H]
    \centering

    \includegraphics[width=\linewidth]{design_fig/interface/bounding_box.png}
    \caption{Two types of bounding boxes observe the structure of Chinese characters from different perspectives.}
    \label{fig:bounding box}
\end{figure}

% \vspace{-5mm}


\begin{figure}[htbp]
    \centering
    \includegraphics[width=\linewidth]{design_fig/interface/Rhythm_and_Stroke_Analysis.jpg}
    \caption{Step 2 and Step 3: Rhythm Analysis(Left) and Stroke Analysis (Right). These steps allow for the analysis of the speed and force variations throughout the writing of a complete character on the left, and the detailed brush technique used for a specific stroke on the right.}
    \label{fig:Rhythm and Stroke Analysis}
\end{figure}

\subsection{Rhythm Comparison}

To minimize confusion, finger pressure is represented using a heatmap, while speed is color-coded on the skeleton (Figure \ref{fig: Rhythm}) ~\cite{howe1983temporal}. In Figure \ref{fig: Rhythm}, the teacher's finger pressure varies throughout the stroke, typically applying force at the start and turning points. In contrast, the student's grip remained tense throughout the earlier strokes, only showing improvement in the final stroke. Similarly, speed can be analyzed in this way.
\begin{figure}[H]
    \centering
    \includegraphics[width=0.5\textwidth]{design_fig/interface/Rhythm.png}
    \caption{In the rhythm view, observe the changes in the teacher's and student's finger pressure (left) and speed (right) throughout the entire character.}
    \label{fig: Rhythm}
\end{figure}

\begin{figure}[H]
    \centering
    \includegraphics[width=0.3\textwidth]{design_fig/interface/time_bar.png}
    \caption{Drag the timeline to view the brush tip status at specific points of the character.}
    \label{fig:time bar}
\end{figure}

Center stroke (zhongfeng) is a fundamental principle in Chinese calligraphy ~\cite{yi2021beginner, yang2009animating, yang2013animating}, and therefore requires direct observation. To facilitate this, the writing video was retained, allowing teachers to integrate the brushstroke process with the form of the center stroke technique (Figure \ref{fig:time bar}). This helps students understand and connect the two aspects more effectively.

\subsection{Line Quality Analysis}

Placed on the same page as the previous rhythm analysis view, allowing users to observe brushstroke parameters while simultaneously comparing changes in the brush-tip video. This integrated approach enables a comprehensive analysis of stroke quality and the brushwork process.

\subsubsection{Brush Rotation}
In Chinese calligraphy, to maintain the ``center stroke'' (zhongfeng) and ensure smooth writing, the calligrapher must continuously adjust the position of the brush tip on the paper\cite{chiang1974chinese}. This adjustment is achieved by rotating the brush handle: on the one hand, the rotation ensures that the brush tip remains aligned with the center of the stroke, and on the other hand, when the brush tip begins to splay, the rotation consolidates the tip, ensuring that the force is concentrated at the tip. This focus on brush handle rotation reflects the calligrapher's exploration of brush-tip control. Therefore, the rotation of the brush handle in writing is worth demonstrating.

Initially, an attempt was made to decompose the posture of the brush handle into yaw, roll, and pitch\cite{fitzpatrick2010validation}, which is a common method for analyzing movement. The idea of displaying these values on a dashboard was considered, but isolated parameters at a single moment in time provided little explanatory value. We also explored displaying the brush handle posture curves (e.g., yaw, roll, and pitch) for both teachers and students side by side \cite{10.1145/3476124.3488645, 10.1145/1878083.1878098}. However, it was found that these curves overly abstracted the posture information, rendering them less accessible to users without a background in data visualization.

Ultimately, a decision was made to visualize the rotation directly on the written strokes. Initially, the plan was to capture and display the amount of rotation at corresponding locations (Figure \ref{fig:roll}). However, given the possibility of brush rotation during advancement, marking the absolute orientation of the brush handle at sampled points along the stroke's central axis using arrow symbols was chosen. This approach offers a more intuitive understanding of the rotation process.

\begin{figure}[htbp]
    \centering
    \includegraphics[width=0.8\linewidth]{design_fig/interface/roll.png}
    \caption{One of the sketches during the iterative process: A rotating arrow is used to indicate that a rotation has occurred at a certain position in the handwriting, and this is represented with a pulse diagram.}
    \label{fig:roll}
\end{figure}

Given the cylindrical nature of the brush shaft and the absence of a fixed front-facing direction, the reverse direction of the first stroke's extension is adopted as the initial orientation, aligning with customary writing practices. The calibration is done at the start of each stroke (Figure \ref{fig:Rotation}). Figure \ref{fig:Rotation} shows that the teacher rotates (left) the brush clockwise while advancing it, whereas the student (rignt) first rotates the brush counterclockwise, followed by a slight clockwise rotation, without achieving the same 'brush wrapping' technique as the teacher.

\begin{figure}[htbp]
    \centering
    \includegraphics[width=0.8\linewidth]{design_fig/interface/Rotation.png}
    \caption{In the stroke detail view, the brush rotation of both the student and the teacher while writing the same stroke is visualized through the rotation view.}
    \label{fig:Rotation}
\end{figure}

\subsubsection{Brush Tilt}
The tilt angle of the brush handle directly affects the way the brush tip interacts with the paper (Figure \ref{fig:brush tilt}), thereby altering the friction during writing, which significantly impacts the quality of the strokes. Typically, strokes with greater friction appear darker, thicker, and have sharper edges, while those with less friction are lighter, thinner, and have more blurry edges (Figure \ref{fig:stroke splitting}). Therefore, to explore the conditions that contribute to the texture of a particular stroke, it is essential to display the tilt of the brush handle in various directions. Building on the discussion from the previous section, the tilt direction of the brush handle must be directly visualized on the stroke.
\begin{figure}[ht]
        \centering
        \includegraphics[width=0.8\linewidth]{design_fig/Brush_Tilt.png}
        \caption{The texture of lines varies with two different pen angles.}
        \label{fig:brush tilt}
\end{figure}

\begin{figure}[th]
        \centering
        \includegraphics[width=0.98\linewidth]{design_fig/interface/Tilt.png}
        \caption{The short lines on the strokes represent the brush's projection on the paper. From the visualization, it's clear that the brush's tilt direction differs between the teacher and the student when writing the same stroke. The teacher tends to tilt the brush against the writing direction to add more strength to the lines, while the student completely overlooks this technique.}
        \label{fig:stroke splitting}
\end{figure}

To ensure ease of comprehension for users, a 3D model of the brush hovering above each sampled point on the stroke was initially planned to be rendered to represent the brush's position at that location. However, since the strokes lie on a horizontal plane, the 3D brush could become obscured or cause perspective distortion due to overlapping brushes in the foreground and background \cite{munzner2015visualization}. As a result, the projection of the brush handle onto the plane of the paper was ultimately chosen as the most suitable visualization method.


\subsubsection{Comparison Curves for Speed and Pressure}
The speed of the brush and the pressure applied by the fingers are critical variables that influence the quality of calligraphy strokes. These two factors respectively determine the force of the brush and the duration of contact with the paper surface. The visualization of pressure and speed parameters in this view continued with the encoding approach from the rhythm view. The difference is that the upper and lower limits of the color scale were adjusted for each stroke to enhance the visibility of pressure variations within individual strokes. Considering that both pressure and speed are unidimensional numerical variables, they are better suited for comparison through curves rather than posture data. To prevent confusion and decrease cognitive load~\cite{keller2006information}, scatter plots were selected to illustrate pressure changes, while curves were used to represent speed trends.

Calligraphy does not follow rigid brushwork rules; much like riding a bicycle, where direction is adjusted based on road conditions, the writing process requires continuous adjustments in brush posture based on the state of the brush. Thus, the primary requirement was to display the general range and trends. In the charts, specific numerical values for pressure and speed were omitted, and a tiered representation was chosen instead.

First, the pressure view is introduced, wherein the bottom scatter plot displays the range of pressure levels for both teachers and students. When hovering the mouse over a scatter point, a small red dot will appear on the stroke to indicate the corresponding position on the ink trace. The x-axis of the scatter plot represents the position within the stroke, and scatter points aligned vertically represent roughly the same position on the stroke, facilitating comparison. As the mouse moves across the chart, a vertical line and the corresponding red dot on the stroke will move in sync, indicating the matching position on the ink trace.

Through the visualization, we can observe that when writing a short downward stroke (pie), the teacher applies greater pressure with the fingers in the initial phase, gradually relaxing to maintain a moderate pressure level (Figure \ref{fig: Pressure}). In contrast, the student consistently applies high pressure throughout the stroke. The hover function was utilized to identify the point of maximum pressure difference between the teacher and the student, which was approximately located at the 2/5 mark of the stroke. This feature provides clear guidance on the area where the student requires adjustment.

\begin{figure}[t]
    \centering
    \includegraphics[width=\linewidth]{design_fig/interface/Pressure.png}
    \caption{electing a left-falling stroke for pressure analysis, the scatter plot clearly shows the pressure variation trends and the differences between the student and the teacher. The X-axis represents the relative position of the stroke, and users can hover over the vertical line of any sample point. Simultaneously, a small red dot in the character will be positioned at the corresponding location, making comparison and analysis easier.}
    \label{fig: Pressure}
\end{figure}

Next is the speed view, where the X-axis also represents the stroke position. The curve shows the speed variation throughout the stroke, allowing for a comparison with the teacher's speed (Figure \ref{ref: Speed}). The aim was also to observe the variation in writing speeds of the teacher and the student over time. To accomplish this, time was set as the x-axis, analogous to a stopwatch, with two rows of horizontal scatter points representing the writing progress of both the teacher and the student. Furthermore, color saturation was employed as an additional layer of encoding to align the sequence between the top and bottom of the ink trace and the graph.

In the speed comparison (left), the teacher's writing shows a rhythm of faster movement in the middle and slower at both ends, while the student writes quickly at the start, with little variation in speed afterward. This indicates that the student's initial and finishing strokes are not executed properly. However, the right chart reveals that the student's overall writing speed is three times slower than the teacher's, suggesting hesitation at the beginning, which leads to sluggish and lifeless strokes \cite{chiang1974chinese}.



\begin{figure}[H]
    \centering
    \includegraphics[width=\linewidth]{design_fig/interface/Speed.jpg}
    \caption{By comparing the writing speed of the teacher and the student in the same stroke segment using line charts (left) and their writing positions at the same time points (right). }
    \label{ref: Speed}
\end{figure}

