\section{Introduction}
The preservation and transmission of intangible cultural heritage (ICH) depend heavily on the effective teaching of skills and the documentation of the learning process. In this context, kinesthetic learning plays a critical role~\cite{TechnologyIntervention}, as mastering a skill involves not only understanding the final product but also perceiving and receiving feedback on the complex motor processes behind it~\cite{begel2004kinesthetic, magill2010motor}. In fields like artistic creation and traditional craftsmanship, while the final work is visible, the creation process often remains difficult to capture, making skill transmission particularly challenging~\cite{10.1145/3613904.3642205}. To address this, recent research has introduced a variety of technological solutions, such as gesture datasets are collected through motion segmentation in craft activities\cite{app10207325}, AR technology is used for interactive experiences\cite{app12199859}, and digital storytelling is conducted using virtual avatars\cite{https://doi.org/10.1111/j.1467-8535.2009.00991.x}. These technologies support multidimensional feedback that enhances learners' cognition and performance, which ultimately facilitates the effective transmission of ICH skills.

This paper takes the learning of brush techniques in Chinese calligraphy as a case study to investigate how capturing movement processes and providing visual feedback can support learning motor skills in the context of ICH. Our core research question is: What factors create barriers in the teaching and learning calligraphy brush techniques? Through a formative study, we found that beginner calligraphy students tend to focus too much on character shapes and often overlook the importance of brush control, which leads to a lack of understanding on proper brush techniques. This can result in the development of difficult-to-correct habits. Moreover, students often rely too heavily on external feedback and struggle to independently identify and correct mistakes. Since the dynamic process of brush control cannot be fully understood by observing static calligraphy, students also have difficulty capturing the details of the brush techniques during teacher demonstrations, especially the subtle timing of finger exertion, which is often neglected. Post-demonstration explanations are less effective due to students' fading memory of the brushstoke process, which further complicates the learning experience. Thus, capturing and clearly demonstrating the dynamic changes in brush control and finger force during writing is crucial for effective calligraphy learning.

To address these issues, a comprehensive solution, CalliSense, is proposed for capturing and visualizing the brushstroke process in Chinese calligraphy. Aiming for a widely applicable approach that allows experts and learners alike to contribute to the documentation of ICH motor skills, we opt for low-cost, accessible tools-specifically, smartphones, along with affordable pressure sensors and inertial sensors. These tools capture essential writing parameters such as brush posture, finger pressure, and writing speed, aligning them with the final written strokes. 
Ink deposition increments are used to mark time for tracking brush shape changes, while stroke points are filtered and interpolated to resolve brush occlusion and generate an accurate skeletal trajectory.
The captured process data is then visualized by incorporating traditional teaching routines, which also aligns with the typical practices of teachers and learners~\cite{ifenthaler2016learning}. The visualization design consists of three modules: character comparison, writing rhythm analysis, and stroke examination. Users are allowed to transit between modules seamlessly and receive a more holistic comprehension of the brushstroke process.
To validate the effectiveness of our system design, we organized four teaching workshops with 4 calligraphy instructors and 12 beginner students (i.e., students interested in calligraphy but with little experience). By comparing traditional teaching methods with those incorporating our CalliSense system, results showed that the system significantly aided students in understanding key aspects of brush techniques, especially those previously overlooked details.


In summary, our work makes the following contributions:

\begin{enumerate}
    \item A qualitative interviews with calligraphy teachers and learners of various skill levels to understand the challenges in the current transmission of calligraphy;
    \item CalliSense, a comprehensive system that includes the capturing of writing processes as well as a visualization interface to reveal brush control in Chinese calligraphy;
    \item A user study to assess the practicality and effectiveness of the system;
    \item Innovative design insights for using digital technology to facilitate the transmission of motor skills in intangible cultural heritage.
\end{enumerate}
