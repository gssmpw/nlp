\section{User Study}
The user study analyzes our system from two perspectives. First, the study will assess whether students can enhance their understanding of basic brush techniques through visualization and subsequently improve their calligraphy skills. Second, it will examine whether the tool effectively aids teachers in conveying complex calligraphy movements to students, thereby improving teaching efficiency. We compare the results of teaching the same skill both with and without the assistance of the CalliSense system, using the non-CalliSense method as the baseline.

\subsection{Participants}
We recruited participants through electronic posters, flyers, and the snowball sampling method~\cite{naderifar2017snowball}. A total of 6 calligraphy teachers and 28 students with varying levels of calligraphy experience and professional backgrounds signed up. After screening, we selected 4 teachers (T1-T4) with 1 to 2 years of teaching experience, whose teaching abilities were comparable and sufficient to cover the most basic calligraphy concepts. Additionally, we chose 12 students (S1-S12) with similar backgrounds in calligraphy, all of whom had less than one year of practice experience. Each participant received a reward of 100 CNY (approximately 14.05 USD).


\subsection{Study Design and Procedure}
The 12 students were randomly assigned to the four teachers, with each teacher guiding three students. Four teaching workshops were conducted, where each teacher tried both the CalliSense-assisted and traditional teaching methods to complete two different teaching tasks. To assess the students' learning outcomes, we used a within-subjects design and administered questionnaires after each teaching condition to evaluate their understanding of basic calligraphy concepts. Following the experiment, semi-structured interviews were conducted with both the teachers and students.

When designing the user study process, we consulted a calligraphy expert with 35 years of writing and teaching experience and established two guiding principles:
1) Ensure that the teaching content represents widely agreed-upon basic calligraphy knowledge. As artistic creation can involve personal expression, we needed to confirm that our system is applicable to most teaching scenarios;
2) Ensure that the content is suitable for beginners to understand.

The expert also noted that while it is difficult to fully master brush techniques in just 30 minutes, it is sufficient to demonstrate what correct brushwork looks like. Therefore, our evaluation focused more on participants' conceptual understanding of brush techniques rather than their actual writing performance.

We designed the study to have equal-length sessions for the same teaching tasks to assess the system's effectiveness and efficiency. The teaching approach followed a common method where students write a complete Chinese character, and the teacher provides feedback and corrections. In our study, we specifically chose the ``Yishan Stele'' (Stele of Mountain Yi)~\cite{metmuseum_yishan_stele} as the practice subject. The ``Yishan Stele'', one of China's earliest stone inscriptions, is renowned for its uniform strokes and consistent brush techniques. Since it consists of simple strokes and uses relatively basic brush techniques, it is often used as introductory material for beginner calligraphers. Notably, the stele includes both curved and long strokes, offering an excellent opportunity to practice central-axis brushwork (zhongfeng). Given that our participants had limited calligraphy experience, we carefully selected the character ``Zi'' with a moderate number of strokes that incorporates both curved and long lines, allowing them to better grasp fundamental calligraphy skills (Figure \ref{fig:Yishan Stele}).

\begin{figure}[h]
    \centering
    \includegraphics[width=0.4\textwidth]{design_fig/yishanbei_and_zi.png}
    \caption{The Yishan Stele (left) and the selected character ``Zi'' from the Yishan Stele (right).}
    \label{fig:Yishan Stele}
\end{figure}

By practicing these strokes, learners can master two key concepts: 
1. The concept of ``rolling the brush'' (guofeng), where rotating the brush handle helps achieve the central-axis brushwork~\cite{wang2024formation}. 
2. Controlling the texture of the strokes by adjusting the writing speed and the tilt of the brush handle~\cite{qian2007towards, long2001art}.

Based on this, we designed two teaching tasks aligned with common foundational calligraphy techniques. The first task focuses on teaching the ``central-axis'' (zhongfeng) brush technique, which involves the concept of ``rolling the brush'' (guofeng) by rotating the brush handle to concentrate the brush tip. The second task teaches the concept of ``controlling stroke texture'', where brush quality is managed through the tilt of the brush handle, writing speed, and hand pressure. While the brush technique parameters may not seem evenly distributed between the two tasks, the concept of ``rolling the brush'' is relatively difficult to grasp, as confirmed by pre-experiment test questionnaires. Therefore, the cognitive load required for both tasks is similar. To avoid learning effects, a Latin square design was used. The teaching tasks, combined with the presence or absence of the CalliSense system, created four distinct process flows (Figure \ref{fig:user study process}), which were randomly assigned to the four teachers.

\begin{figure}[t]
    \centering
    \includegraphics[width=0.44\textwidth]{design_fig/UserStudy.png}
    \caption{User Study Process. The four columns represent four different user research processes, which will be randomly assigned to four different teachers.}
    \label{fig:user study process}
\end{figure}

The final procedure is as follows: Prior to the experiment, participants completed an informed consent form and a demographic survey. Students also took a pre-test to assess their knowledge of the calligraphy concepts covered in the teaching tasks, while teachers used this time to learn how to use the system. Afterward, the teachers conducted the first teaching task. Upon completion, students filled out a mid-task questionnaire to assess their understanding of the concepts taught in the first task, and they rated the class for engagement, involvement, and clarity. The second teaching task followed, and after it was completed, students filled out a post-task questionnaire to evaluate their understanding of the second task, along with class ratings.

Finally, both teachers and students completed the NASA TLX~\cite{hart2006nasa} and SUS questionnaires~\cite{10.5555/2835587.2835589} and participated in semi-structured interviews to discuss the differences between traditional teaching methods and CalliSense, their experience with the system, and how it specifically aided calligraphy learning. We also encouraged participants to think aloud during the experiment, and all verbal feedback was recorded for analysis.

\subsection{Analysis and Results}

\subsubsection{System Usage Overview:}

Overall, both teachers and students agreed that CalliSense effectively aids students in mastering Chinese calligraphy techniques. Using the NASA TLX questionnaire (scale 1–10), we evaluated the system's impact on teaching effectiveness (Figure \ref{fig: NASA}). In terms of performance (brushwork learning), students showed significant improvement in their brush technique when using CalliSense (Md=8, IQR=3) compared to the baseline (without CalliSense) (Md=4, IQR=2). They were able to pinpoint specific areas for improvement instead of relying on vague impressions. One participant even brought their own brush and paper, attempting to recreate their usual practice setup with the system's sensors. They described the experience as viewing their practice from a fresh perspective (S5).

Frustration levels were generally low (Md=1, IQR=1), although one student remarked that, despite initially thinking their brushwork was quite good, the experience revealed overlooked issues. This student was surprised by the number of brushwork details that needed attention but saw this as a positive discovery in their calligraphy learning (S9). Both mental (Md=4, IQR=2.5) and physical demands (Md=2, IQR=1) were notably reduced. Another student mentioned that the ability to clearly identify areas for improvement reduced their previous anxiety (S10). Participants generally found CalliSense enjoyable to use, with improvements in time demands (Md=2.5, IQR=3) and effort (Md=5, IQR=4). However, effort varied significantly, mainly due to the complexity of the ``wrapped brush'' technique, which created individual differences based on experience and learning strategies (S10).

Additionally, the SUS questionnaire was used to evaluate system usability, with an average score of 78, significantly higher than the benchmark of 68, indicating good usability. Apart from the interface's ability to convey information clearly, the sensors were considered lightweight and barely noticeable, especially on the fingers (N=6). However, one teacher noted that the wire at the top of the inertial sensor could interfere with cursive script writing, which requires more fluid brush movements (T4).
\begin{figure*}[h]
    \centering
        \includegraphics[width=0.49\textwidth]{design_fig/before_1.png}
        \includegraphics[width=0.49\textwidth]{design_fig/after_1.png}
    \caption{Participants rated system usability using the NASA Task Load Index (1-10 scale), Top: Ratings for the traditional classroom, Bottom: Ratings for the classroom using CalliSense.}
    \label{fig: NASA}
\end{figure*}


\subsubsection{Knowledge Mastery Evaluation:}

We also examined how well students grasped the key concepts in the instructional tasks, asking them to self-assess their understanding of five fundamental brushwork techniques: Q1 - using the brush's center stroke (zhongfeng), Q2 - the ``encircling'' technique (guofeng), Q3 - how brush tilt affects line quality, Q4 - the impact of hand pressure on line formation, and Q5 - how writing speed influences the stroke. These aspects encompass all the parameters demonstrated by the CalliSense system. The results showed significant improvement in these five dimensions when CalliSense was used, compared to traditional classroom instruction (Figure \ref{fig:knowledge points test}). This improvement can be attributed to the system's ability to visually present data, allowing students to clearly identify areas for improvement and track their progress.


Some students even noted that with the recorded data from the system, they no longer needed constant guidance from the teacher and could independently focus on targeted practice, such as writing speed and brush angle (S7, S12). Additionally, several students mentioned that just one use of the system was enough to shift their understanding, helping them recognize which aspects were most critical and where they should focus more attention. For instance, some students had not previously understood the importance of writing speed, but the system-generated charts made this concept clear to them (S3). The structured visualization not only summarized the entire practice process from beginner to advanced levels but also highlighted that enhancing awareness of brush control details was more crucial than simply practicing individual strokes~\cite{zhang2023bringing}.

Furthermore, students appreciated the comparison feature in CalliSense, which allowed them to consciously replicate their teacher's movements (S6). Interestingly, one student independently discovered correlations between different visualized parameters, such as noticing that hand pressure decreases during brush rotation because the ``encircling'' technique requires a more relaxed grip (S11). This realization deepened their understanding of how grip tension affects the quality of the strokes. These findings suggest that CalliSense not only helps students master specific skills but also encourages them to think critically about brush control, thereby enhancing their overall understanding of calligraphy techniques.

\begin{figure}[t]
    \centering
    \includegraphics[width=0.4\textwidth]{design_fig/userstudy/learning_result.png}
    \caption{User research results show that students' mastery of 5 calligraphy brush knowledge points after learning in two different classroom settings. Compared to traditional classrooms, students' learning outcomes for the 5 knowledge points have significantly improved in classrooms equipped with CalliSense.}
    \label{fig:knowledge points test}
\end{figure}


\subsubsection{Classroom Experience:}

Students' subjective experience in class can influence how effectively they absorb knowledge. We asked students to rate two classroom experiences on scales including ``Q6: I found the class enjoyable'', ``Q7: I found the class relaxing'' and ``Q8: I felt engaged in the learning process'' with scores ranging from 1 (strongly disagree) to 10 (strongly agree). Results showed that classes with CalliSense scored higher in Q6-enjoyment (Md=8, IQR=2) and Q8-engagement (Md=7, IQR=2.5) compared to baseline classes. In class, students were eager to use the writing equipment, curious to see how their brushwork would be visualized. The immediate feedback from the system significantly boosted their engagement. After each writing session, they were able to see their process right away, and the concrete feedback from visualizations was more persuasive than vague praise or self-assessment (S1, S11). While Q7-relaxation (Md=7, IQR=3) was similar between CalliSense and baseline classes, both scored relatively high.

\subsubsection{Teacher Feedback:}

Teacher insights were mainly gathered through interviews. Teachers generally agreed that without tools like CalliSense, it was difficult to observe so many fine details (N=4). The system's visualizations helped them better understand students' writing processes. However, one teacher noted that using the system might require adjusting their teaching strategy (T2). They usually help beginners by repeatedly demonstrating ``incomplete strokes'' rather than analyzing each stroke individually. With the introduction of the system, they may need to rethink this approach. Still, they believed the adjustment was worthwhile, as the system allowed them to focus more on the core elements of calligraphy learning.

Some teachers also admitted that they had not previously paid enough attention to students' pressure control when holding the brush but now realized its critical importance (N=2). Another teacher highly appreciated CalliSense, seeing its potential for online teaching (T3). Moreover, the system's strengths in personalized teaching became evident. As each student faced unique challenges in writing, the system's detailed brushwork parameters allowed teachers to pinpoint specific weaknesses, rather than providing broad guidance. This targeted instruction not only improved teaching efficiency but also accelerated students' progress in calligraphy.

 
