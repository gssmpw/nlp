\begin{figure}[t]
\centering
\includegraphics[width=\linewidth, trim= 0 140 0 0, clip]{figs/teaser/Teaser.pdf}\\
\vspace{1em}
\resizebox{\linewidth}{!}{
\setlength{\tabcolsep}{4pt}

\begin{tabular}{@{}llccc@{}}
    \toprule
    & & CD ($10^{-2}$) $\downarrow$ & F-Score $\uparrow$ & Latency (s) $\downarrow$ \\
    \midrule
    K-Nearest Neighbors & oracle & 3.2 & 97.5 & 3.2 \\
    Ours~(Minkowski)~\cite{choy20194d} & grid-based & 3.8 & 96.2 & 1.5 \\
    \bf Ours & serialization & 3.3 &  97.4 & 1.7 \\
    \bottomrule
\end{tabular}

}
\caption{
To locally predict the SDF value that (implicitly) reconstructs the surface, the pivotal operation is to \textit{aggregate} the information (i.e.~features) of nearby points.
(left) Working on the point cloud directly is difficult, as there is no simple way to implement multi-scale architectures suitable for large scale point cloud processing.
(middle)~State-of-the-art methods therefore opt to quantize the input point cloud to a voxel grid, and employ established sparse CNN backbones, but quantization leads to information loss.
(right) By fetching approximate neighbors via serialization we can fetch the local context efficiently \textit{and} avoid information loss.
{We summarize the performance of representative works on a large scale outdoor dataset (\carla), and show that our method achieves the best performance in both time efficiency (latency) and accuracy (CD and F-score); for additional details see~\Cref{sec:results}. }
}
\label{fig:teaser}
\vspace{-1em}
\end{figure}


\endinput
