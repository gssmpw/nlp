\section{Introduction}
% level1: Molecule design is important; and 3D molecule design is important and hot
Discovering new molecules is crucial for designing new drugs and materials. However, brute-force searching in the astronomical chemical space poses significant challenges. Deep learning offers a promising approach to navigating this complexity efficiently. Particularly, 3D molecule generation, which involves generating a molecular graph and its coordinates within 3D space, is vital for structure-based drug design and inverse molecule design to achieve desired quantum chemical properties.



% level2: what is a molecule, and how past 3D molecule works approach this problem and their limitations. but these methods still face limitations when trying to generate chemically valid molecules
A 3D molecule is characterized by a set of atoms, their 3D spatial coordinates, and the chemical bonds connecting the atoms. Earlier efforts in 3D molecule generation focus on generating the atoms and their 3D coordinates~\citep{EDM, EEGSDE} without considering bond predictions. Thus, they easily violate valency rules dictated by bonds, hardly ensuring chemical validity.
% but these methods often fail to produce chemically valid molecules, typically violating the chemical rules of valency dictated by the bonds. 
To overcome this limitation, recent studies have been designed to simultaneously generate atoms, coordinates, and bonds \citep{JODO, MiDi}. Their superior performances demonstrate that incorporating predictions of bond connections can significantly enhance the chemical validity of the generated 3D structures. \yuan{Other works do xxx.} Despite these advancements, generating chemically valid structures for larger 3D molecules, as evidenced in the GEOM-DRUGs dataset \citep{GEOM}, remains a challenging task~\citep{JODO}.

% 动机 1是为了 validity,基于已知的 valid 的分子,我们的 3D 模型能够取得更好的表现
% 动机 2是为了 利用大量的预训练数据
% 动机 3是为了 rethink 这个 task


% level3: 1D molecule generator can easily generate 100% valid molecules. why? because it is hard to predict the conformer of a molecule. Therefore, we propose a approach that disentangles the tasks of molecule generation and 3D conformer prediction.
% why generating in 1D is easier
Conversely, generating a chemically valid 1D molecule, which is defined by a set of atoms and bonds without 3D coordinates, is significantly easier~\citep{moses}. These 1D molecules are often represented using sequential chemical language, such as SMILES~\citep{SMILES} and SELFIES~\citep{SELFIES}. Notably, the SELFIES guarantees 100\% chemical validity of its described molecules through its robust grammar. This higher rate of chemical validity stems from the absence of 3D structures, avoiding the valency rule violations that are common in \yuan{predicted 3D coordinates.}
% eliminating the common issue of violating chemical valency rules in 3D space. 
However, this absence of 3D structures limits the direct applicability of 1D molecular descriptors in 3D molecule generation.
% this absence also limits the usage of 1D molecular descriptors in 3D molecule generation. 
A natural solution to this challenge is to employ a 3D conformer predictor~\citep{torsion} to predict the 3D coordinates of molecules from a 1D molecule generator. Moreover, another advantage of utilizing 1D molecules is their relative abundance compared to 3D molecules. For example, as of 2024, \yuan{the xxx database contains xxx 1D molecules but only xxx 3D molecules}. This disparity highlights the potential to enhance 3D molecule generation by leveraging the extensive repository of 1D molecules through transfer learning or representation learning.

Nonetheless, to the best of our knowledge, few prior research has thoroughly explored this direction for 3D molecule generation, potentially due to the insufficient performance of prior 3D molecular conformer predictors. This research gap raises several critical questions: (1) \textit{Can we accurately predict the 3D structures of newly generated \yuan{1D molecules}?} (2) \textit{What challenges emerge when translating the outputs of a 1D molecule generator into 3D molecules using a 3D conformer predictor?} (3) \textit{Are 1D molecule generators capable of capturing the 3D distributional intricacies for molecule generation?}


% Put:
% However, these 1D descriptors cannot describe 3D coordinates, limiting their usage in 3D molecule generation. To address this limitation, it natually arises to employ a 3D conformer prediction module~\citep{torsion} to predict the 3D coordinates of atoms, which are  generated by a model trained on 1D molecular descriptors beforehand. However, to our knowledge, no prior research has thoroughly explored this approach, potentially due to the insufficient performance of prior 3D conformer predictors. This gap leaves several research questions open: ...
% Here

% level4: what we study
To mitigate the research gap and answer the questions above, in this study, we propose a \yuan{foundation model} for 3D molecules -- \textbf{NExT-Mol}: 3D Diffusion Meets 1D Language Modeling for 3D Molecule Generation. NExT-Mol includes three key components: (1) We pretrain a \underline{Mo}lecular \underline{Llama} language model (\textbf{MoLlama})~\citep{Llama-2,zhang2024tinyllama} on a large collection of \yuan{(xxx B)} 1D molecules. MoLlama excels in autoregressive 1D molecule generation, effectively capturing the distribution of 1D molecular characteristics, such as the fragment similarity and scaffold structures~\citep{moses}. MoLlama serves as a \yuan{robust foudation model} for generating 1D molecules for the subsequent 3D conformer prediction. (2) We introduce the \textbf{D}iffusion \textbf{M}olecular \textbf{Trans}former (\textbf{DMTrans}), a novel diffusion model for predicting 3D molecular conformers. Our results demonstrate that DMTrans sets new state-of-the-arts for 3D conformer prediction, accurately revealing 3D structures of 1D molecules generated by MoLlama with improved chemical validity. (3) NExT-Mol includes a cross-modal projector and the corresponding training strategy dedicated to integrate the pre-trained representations of MoLlama to improve DMT's performances in 3D conformer prediction. 

Collectively, the NExT-Mol foundation model demonstates state-of-the-art performances across diverse tasks, including \textit{de novo} 3D molecule generation, conditional 3D molecule generation targeting quantum chemical properties, and 3D conformer prediction, evidenced by its performance on the GEOM-QM9 and GEOM-DRUGs datasets~\citep{GEOM}.

Our contributions can be summarized as follows:
\begin{itemize}[leftmargin=*]
\item \textbf{The NExT-Mol Foundation Model.} Our proposed NExT-Mol foundation includes two novel components: an extensively pretrained and tested molecule language model MoLlama to generate 1D molecule sequences; a structural diffusion module DMTrans that can accurately predict molecular structures. Their architectures and training strategies are tailored for molecule generation.
\item \textbf{State-of-the-Art Performances Across Various Tasks.} NExT-Mol demonstrate state-of-the-art perforamnce across diverse tasks including \textit{de novo} 3D molecule generation, conditional 3D molecule generation, and 3D conformer prediction. The strong performance demonstrates the effectiveness of our proposed method. 
\item \textbf{Rethinking 3D Molecule Generation.} To answer the ealier proposed three research questions, we present \yuan{extensive analysis} to rethink the evaluation metrics (\eg molecule validity and RMSD of conformer prediction) and the model design choices for 3D molecule generation. We find that \yuan{xxx.}
\end{itemize}
