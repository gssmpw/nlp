\section{Related works}
Collaborative perception is a foundational technology in ICV, enabling vehicles to share and fuse sensory data to enhance situational awareness. Despite significant advancements, challenges such as data asynchrony, communication efficiency, and the impact of information freshness persist, necessitating further investigation. 


Substantial progress has been achieved in addressing data sharing and fusion challenges within collaborative perception. The study by ____ identifies critical issues, including data asynchrony, and proposes a dynamic communication graph framework to minimize latency. Similarly, the work in ____ emphasizes the role of wireless communication in enhancing perception accuracy through environmental information fusion. Furthermore, edge-assisted frameworks such as  ____ demonstrate how V2X communication can effectively improve environmental awareness in dynamic vehicular networks.

Optimizing communication efficiency remains a vital focus for collaborative perception, particularly in bandwidth-constrained environments. The Where2comm framework proposed in ____, reduces bandwidth consumption by prioritizing spatially significant data for transmission. Complementing this, the How2comm framework in ____ utilizes mutual information-based mechanisms and delay compensation strategies to optimize multi-agent collaboration. Additionally, frameworks like What2comm ____ and When2com ____ dynamically balance communication overhead and perception accuracy through innovative feature selection and communication graph grouping methods.

Publicly available datasets play an essential role in advancing collaborative perception research by providing standardized benchmarks to evaluate fusion strategies and real-world performance. For instance, the OPV2V dataset introduced by ____ , supports V2V communication research with over 11,000 frames and multiple fusion pipelines. Similarly, the DAIR-V2X dataset proposed by ____ enables studies on asynchronous fusion and communication delays using diverse and synchronized scenarios. These datasets are instrumental in addressing challenges such as communication variability and the alignment of multi-source data.

The concept of AoI has emerged as a critical metric for maintaining information freshness in collaborative perception systems. The work in  ____ explores the interplay between AoI, latency, and reliability, emphasizing the impact of dynamic vehicular conditions on AoI variations. An AoI-driven power allocation strategy is proposed in ____, balancing AoI reduction with system throughput to ensure accurate and timely information updates. In edge-enabled environments, the study ____ addresses the dual challenges of AoI and computational delays, proposing efficient resource allocation methods to maintain information freshness. Additionally, ____ employs deep reinforcement learning to optimize wireless resource allocation and minimize AoI in dynamic vehicular networks.  Moreover, the study by ____ focuses on minimizing the tail distribution of AoI, targeting ultra-reliable, low-latency vehicular communications. These studies collectively demonstrate the critical role of AoI in enhancing the timeliness and accuracy of collaborative perception.


To further enhance the performance of collaborative perception, several studies investigate the joint optimization of communication and computation processes. The active learning-based approach in ____ dynamically allocates resources to minimize AoI while ensuring communication reliability. Similarly, the work in ____ proposes joint strategies for optimizing AoI and completion time, establishing a theoretical foundation for balancing timeliness and efficiency. Finally, the study ____ explores the integration of trajectory planning and resource allocation in UAV-assisted vehicular networks, demonstrating the potential of multi-modal collaboration to enhance information freshness and perception performance.