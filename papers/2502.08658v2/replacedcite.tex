\section{Related Works\label{2}
}
\subsection{Vehicle-level dynamics modeling}
%physics
The physics-driven car following model is defined as a rigorous derivation process, with most parameters having clear physical meanings. Classic car following models such as full velocity difference model (FVD) ____ and intelligent driver model (IDM) ____, greatly promoted the fundamental research of car following behavior. Subsequently, there are various perspectives to model car following behavior, such as stochasticity ____ and heterogeneity ____. 
%Data
In terms of data-driven models, Ma et al. considered reaction delay and established a sequence to sequence learning based car-following model ____. Liao et al. develops a personalized car-following model using a memory-based deep reinforcement learning approach, specifically integrating LSTM with twin delayed deep deterministic policy gradients, to optimize driving behavior ____. Jiang et al. presented a stochastic learning approach using approximate Bayesian computation to integrate multiple car-following models, achieving more accurate vehicle trajectory reproduction for both human-driven and automated vehicles ____. 
%PIDL
Innovatively, Mo et al. introduced a physics-informed deep learning encoded with physics-based models, providing a new perspective and tool for car following research ____. Afterwards, Transformer neural network architecture with self-attention was applied to the PIDL-based car following architecture, achieving superior performance ____. However, within the PIDL-based car following models, physical prior knowledge only played a secondary role in training deep neural networks, where the neural network remains the main focus of inference. An additional drawback is that these models idealize assumptions for modeling actual traffic flow and lack of the analysis of traffic characteristics of real traffic flow.

\subsection{Platoon-level dynamics modeling}
While most research has focused on predicting short-term trajectories of individual vehicles, there is a growing need to model platoon-level car-following dynamics to better understand the interactions and collective behavior of multiple vehicles. For example, Lin et al. analyzed the spatiotemporal error propagation from the perspective of platoon generation, and proposed an long short-term memory network structure with improved sampling mechanism ____. Lin et al. worked on providing enhanced state representation for multi-agents reinforcement learning, incorporating intricate relative vehicle relationships to improve performance in platoon-following scenarios ____. Liu et al. combined physics knowledge with deep learning and stochasticity to design a quantile-regression physics-informed deep learning car-following model, which comprehensively characterizes the real laws of car following behavior in a platoon ____. Furthermore, Rowan et al. attempted to capture mesoscopic and macroscopic traffic phenomena as broader traffic features for model training ____, and constructed a multiscale car-following framework for vehicle platoons. Tian et al. proposed an AI-based Koopman framework to model the unknown nonlinear platoon dynamics. This framework linearizes the platoon system, aiming to analyze how disturbances propagates over vehicle platoon while achieving accurate prediction ____. However, due to the existence of operators, this framework lacks scalability. Overall, there is a lack of platoon-level dynamic modeling, particularly in capturing the behavioral interaction features across the physical space of vehicles at the platoon-level. More importantly, the existing platoon-level dynamics modeling lacks of addressing the scalability and analyzability, issues simultaneously, which motivates the focus of this paper.