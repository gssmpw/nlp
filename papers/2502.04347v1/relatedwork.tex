\section{Related Work}
LLMs have been widely applied and validated for their ability to identify and fix vulnerabilities \cite{Napoli_2023_Evaluating}. In the field of smart contract security, various methods based on LLMs have been proposed, and certain effects have been achieved.
Firstly, Boi et al. \cite{Boi_2024_VulnHunt-GPT} proposed VulnHunt-GPT, a method that uses the GPT-3 to identify common vulnerabilities in OWASP standards smart contracts.s. Similarly, Xia et al. \cite{Xia_2024_AuditGPT} introduced a tool called AuditGPT, which utilizes LLM to automatically and comprehensively verify the ERC rules of smart contracts. They break down large, complex audit processes into small, manageable tasks and design prompts for each ERC rule type to improve audit performance. 
In terms of fuzz testing, Shou et al. \cite{Shou_2024_LLM4Fuzz} presented LLM4FUZZ, which employs LLMs to effectively guide fuzz activities in smart contracts and determine their priorities. This approach can enhance test efficiency and coverage compared to traditional fuzz testing methods.

However, the direct use of pre-trained LLMs is no longer sufficient on some occasions, so many studies have chosen to fine-tune LLMs to meet specific needs. For example, Liu et al. \cite{Liu_2024_PropertyGPT} proposed PropertyGPT,  a system for formal verification of smart contracts by retrieving enhanced property generation. This system utilizes the capability of LLM to automatically generate comprehensive properties of smart contracts, including invariants, preconditions, and postconditions, thus improving the security of smart contracts. On the other hand, Storhaug et al. \cite{Storhaug_2023_Efficient} proposed a novel vulnerability-bound decoding method to reduce the amount of vulnerable code generated by the model. They fine-tuned the LLM to include vulnerability tags when generating code and then prohibited these tags during the decoding process to avoid producing vulnerable codes. Finally, Yang et al. \cite{Yang_2024_Automated} collected a large number of tagged smart contract vulnerabilities and fine-tuned Llama-2-13B for the automatic detection of vulnerabilities in the decentralized finance (DeFi) domain's smart contracts \cite{li2024defitail}.

In conclusion, while LLMs show great potential for improving smart contract security, the full realization of their potential requires continuous fine-tuning and adaptation of these models in specific contexts.