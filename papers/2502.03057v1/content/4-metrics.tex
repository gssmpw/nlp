\label{sec::eval}
Using the method proposed in this paper, we annotated the left eye of six different users from the dataset~\cite{angelopoulos2020event}.

We reviewed and annotated 114222 frames, each containing a varying number of events. The total number of frames reviewed was 315213; however, many did not contain the minimum number of events necessary to recognize and detect an eye. These empty frames likely resulted from periods when the eyes were still and did not generate events.

\begin{table}[t]
\centering
    
\resizebox{\textwidth}{!}{
\setlength{\tabcolsep}{10pt}
\begin{tabular}{c|ccccccc|c|}
 \hline
 \hline
 Statistic& User4 & User5 & User10 & User15 & User18 & User19 & User20 & Total\\
 \hline
 Frame analyzed   & 44020 & 41233 & 46818 & 46008 & 43470 & 43125 & 50539 & 315213\\
 Annotated Frame & 13643  & 16836 & 26440 & 12967 & 11109 & 9109 & 24118 & 114222\\
% Max Events   & 6290 &   5573 & 9022 & 12563 & 7247 & 7646 & 14368 &XXX\\
 Saccade Counts & 406  & 317 & 288 & 317 & 238 & 248 & 287 &2101\\
 Blink Counts & 18 & 3 & 32 & 29 & 14 & 10 & 14 &120\\
 Eye Center Position & 4447 & 2911 & 3400 & 2981 & 2510 & 2917 & 3139 &22305\\
 \hline
 \hline
\end{tabular}
}
\caption{Table reporting core statistics about each annotated user of the Angelopoulos dataset, computed with the proposed method.\label{annotations_statistics}}
\end{table}

In the 114222 annotated frames, it was possible to recognize either an open eye or a blink movement. From these frames, we annotated 22305 distinct eye center positions, detected 2101 saccade movements, and identified 120 blink movements.
All these elements are summarized in Tab~\ref{annotations_statistics}, where it is also possible to see this information data for each annotated user.

These annotations can be used during the training and testing of eye-tracking algorithms that utilize event-camera data to also detect various types of eye movements, such as blinks and saccades.

