\documentclass[runningheads]{llncs}

% ---------------------------------------------------------------
% Include basic ECCV package
 
% TODO REVIEW: Insert your submission number below by replacing '*****'
% TODO FINAL: Comment out the following line for the camera-ready version
%\usepackage[review,year=2024,ID=*****]{eccv}
% TODO FINAL: Un-comment the following line for the camera-ready version
\usepackage{eccv}

% OPTIONAL: Un-comment the following line for a version which is easier to read
% on small portrait-orientation screens (e.g., mobile phones, or beside other windows)
%\usepackage[mobile]{eccv}


% ---------------------------------------------------------------
% Other packages

% Commonly used abbreviations (\eg, \ie, \etc, \cf, \etal, etc.)
\usepackage{eccvabbrv}

% Include other packages here, before hyperref.
% \usepackage{graphicx}
% \usepackage{tabularx}
% \usepackage{booktabs}

% The "axessiblity" package can be found at: https://ctan.org/pkg/axessibility?lang=en
\usepackage[accsupp]{axessibility}  % Improves PDF readability for those with disabilities.


% ---------------------------------------------------------------
% Hyperref package

% It is strongly recommended to use hyperref, especially for the review version.
% Please disable hyperref *only* if you encounter grave issues.
% hyperref with option pagebackref eases the reviewers' job, but should be disabled for the final version.
%
% If you comment hyperref and then uncomment it, you should delete
% main.aux before re-running LaTeX.
% (Or just hit 'q' on the first LaTeX run, let it finish, and you
%  should be clear).

% TODO FINAL: Comment out the following line for the camera-ready version
%\usepackage[pagebackref,breaklinks,colorlinks,citecolor=eccvblue]{hyperref}
% TODO FINAL: Un-comment the following line for the camera-ready version
\usepackage{hyperref}

% Support for ORCID icon
\usepackage{orcidlink}


\usepackage{graphicx}
\usepackage{booktabs}
\usepackage{makecell}
\usepackage{multirow}
\usepackage{adjustbox}
\usepackage{threeparttable}



%personal addendum
\usepackage{tikz}
\usetikzlibrary{positioning, fit}

\pgfdeclarelayer{background}
\pgfsetlayers{background,main}

\usepackage{comment}
\usepackage{caption} 
\captionsetup[table]{skip=10pt}
\usepackage{subcaption}


\begin{document}

% ---------------------------------------------------------------
% TODO REVIEW: Replace with your title
%\title{Finding the eye: an improved ground truth for eye tracking dataset} 
\title{High-frequency near-eye ground truth for event-based eye tracking}

% TODO REVIEW: If the paper title is too long for the running head, you can set
% an abbreviated paper title here. If not, comment out.
\titlerunning{High-frequency near-eye ground truth for event-based eye tracking}

% TODO FINAL: Replace with your author list. 
% Include the authors' OCRID for the camera-ready version, if at all possible.
\author{Andrea Simpsi\inst{1}\thanks{Corresponding author.}\orcidlink{0000-0002-2700-4808} \and
Andrea Aspesi\inst{1,2}\orcidlink{0009-0000-4936-5973} \and
Simone Mentasti\inst{1}\orcidlink{0000-0001-7059-413X} \and
Luca Merigo\inst{2}\orcidlink{0000-0001-9103-1774} \and
Tommaso Ongarello\inst{2} \and
Matteo Matteucci\inst{1}\orcidlink{0000-0002-8306-6739}}

% TODO FINAL: Replace with an abbreviated list of authors.
\authorrunning{A.~Simpsi et al.}
% First names are abbreviated in the running head.
% If there are more than two authors, 'et al.' is used.

% TODO FINAL: Replace with your institution list.
\institute{
    Department of Electronics, Information and Bioengineering (DEIB)\\
    Politecnico di Milano\\
    Via Ponzio 34/5, 20133 Milan, Italy\\
    \email{\{name.surname\}@polimi.it}
    \and
    EssilorLuxottica Italia S.p.A.\\
    Piazzale Cadorna 3, 20123 Milan, Italy\\
    \email{\{name.surname\}@luxottica.com}
}

\maketitle


\begin{abstract}
 Event-based eye tracking is a promising solution for efficient and low-power eye tracking in smart eyewear technologies. However, the novelty of event-based sensors has resulted in a limited number of available datasets, particularly those with eye-level annotations, crucial for algorithm validation and deep-learning training. This paper addresses this gap by presenting an improved version of a popular event-based eye-tracking dataset. We introduce a semi-automatic annotation pipeline specifically designed for event-based data annotation. Additionally, we provide the scientific community with the computed annotations for pupil detection at $200Hz$.
  \keywords{Smart Eyewear \and Eye Tracking }
\end{abstract}


\section{Introduction}
\section{Introduction}

Despite the remarkable capabilities of large language models (LLMs)~\cite{DBLP:conf/emnlp/QinZ0CYY23,DBLP:journals/corr/abs-2307-09288}, they often inevitably exhibit hallucinations due to incorrect or outdated knowledge embedded in their parameters~\cite{DBLP:journals/corr/abs-2309-01219, DBLP:journals/corr/abs-2302-12813, DBLP:journals/csur/JiLFYSXIBMF23}.
Given the significant time and expense required to retrain LLMs, there has been growing interest in \emph{model editing} (a.k.a., \emph{knowledge editing})~\cite{DBLP:conf/iclr/SinitsinPPPB20, DBLP:journals/corr/abs-2012-00363, DBLP:conf/acl/DaiDHSCW22, DBLP:conf/icml/MitchellLBMF22, DBLP:conf/nips/MengBAB22, DBLP:conf/iclr/MengSABB23, DBLP:conf/emnlp/YaoWT0LDC023, DBLP:conf/emnlp/ZhongWMPC23, DBLP:conf/icml/MaL0G24, DBLP:journals/corr/abs-2401-04700}, 
which aims to update the knowledge of LLMs cost-effectively.
Some existing methods of model editing achieve this by modifying model parameters, which can be generally divided into two categories~\cite{DBLP:journals/corr/abs-2308-07269, DBLP:conf/emnlp/YaoWT0LDC023}.
Specifically, one type is based on \emph{Meta-Learning}~\cite{DBLP:conf/emnlp/CaoAT21, DBLP:conf/acl/DaiDHSCW22}, while the other is based on \emph{Locate-then-Edit}~\cite{DBLP:conf/acl/DaiDHSCW22, DBLP:conf/nips/MengBAB22, DBLP:conf/iclr/MengSABB23}. This paper primarily focuses on the latter.

\begin{figure}[t]
  \centering
  \includegraphics[width=0.48\textwidth]{figures/demonstration.pdf}
  \vspace{-4mm}
  \caption{(a) Comparison of regular model editing and EAC. EAC compresses the editing information into the dimensions where the editing anchors are located. Here, we utilize the gradients generated during training and the magnitude of the updated knowledge vector to identify anchors. (b) Comparison of general downstream task performance before editing, after regular editing, and after constrained editing by EAC.}
  \vspace{-3mm}
  \label{demo}
\end{figure}

\emph{Sequential} model editing~\cite{DBLP:conf/emnlp/YaoWT0LDC023} can expedite the continual learning of LLMs where a series of consecutive edits are conducted.
This is very important in real-world scenarios because new knowledge continually appears, requiring the model to retain previous knowledge while conducting new edits. 
Some studies have experimentally revealed that in sequential editing, existing methods lead to a decrease in the general abilities of the model across downstream tasks~\cite{DBLP:journals/corr/abs-2401-04700, DBLP:conf/acl/GuptaRA24, DBLP:conf/acl/Yang0MLYC24, DBLP:conf/acl/HuC00024}. 
Besides, \citet{ma2024perturbation} have performed a theoretical analysis to elucidate the bottleneck of the general abilities during sequential editing.
However, previous work has not introduced an effective method that maintains editing performance while preserving general abilities in sequential editing.
This impacts model scalability and presents major challenges for continuous learning in LLMs.

In this paper, a statistical analysis is first conducted to help understand how the model is affected during sequential editing using two popular editing methods, including ROME~\cite{DBLP:conf/nips/MengBAB22} and MEMIT~\cite{DBLP:conf/iclr/MengSABB23}.
Matrix norms, particularly the L1 norm, have been shown to be effective indicators of matrix properties such as sparsity, stability, and conditioning, as evidenced by several theoretical works~\cite{kahan2013tutorial}. In our analysis of matrix norms, we observe significant deviations in the parameter matrix after sequential editing.
Besides, the semantic differences between the facts before and after editing are also visualized, and we find that the differences become larger as the deviation of the parameter matrix after editing increases.
Therefore, we assume that each edit during sequential editing not only updates the editing fact as expected but also unintentionally introduces non-trivial noise that can cause the edited model to deviate from its original semantics space.
Furthermore, the accumulation of non-trivial noise can amplify the negative impact on the general abilities of LLMs.

Inspired by these findings, a framework termed \textbf{E}diting \textbf{A}nchor \textbf{C}ompression (EAC) is proposed to constrain the deviation of the parameter matrix during sequential editing by reducing the norm of the update matrix at each step. 
As shown in Figure~\ref{demo}, EAC first selects a subset of dimension with a high product of gradient and magnitude values, namely editing anchors, that are considered crucial for encoding the new relation through a weighted gradient saliency map.
Retraining is then performed on the dimensions where these important editing anchors are located, effectively compressing the editing information.
By compressing information only in certain dimensions and leaving other dimensions unmodified, the deviation of the parameter matrix after editing is constrained. 
To further regulate changes in the L1 norm of the edited matrix to constrain the deviation, we incorporate a scored elastic net ~\cite{zou2005regularization} into the retraining process, optimizing the previously selected editing anchors.

To validate the effectiveness of the proposed EAC, experiments of applying EAC to \textbf{two popular editing methods} including ROME and MEMIT are conducted.
In addition, \textbf{three LLMs of varying sizes} including GPT2-XL~\cite{radford2019language}, LLaMA-3 (8B)~\cite{llama3} and LLaMA-2 (13B)~\cite{DBLP:journals/corr/abs-2307-09288} and \textbf{four representative tasks} including 
natural language inference~\cite{DBLP:conf/mlcw/DaganGM05}, 
summarization~\cite{gliwa-etal-2019-samsum},
open-domain question-answering~\cite{DBLP:journals/tacl/KwiatkowskiPRCP19},  
and sentiment analysis~\cite{DBLP:conf/emnlp/SocherPWCMNP13} are selected to extensively demonstrate the impact of model editing on the general abilities of LLMs. 
Experimental results demonstrate that in sequential editing, EAC can effectively preserve over 70\% of the general abilities of the model across downstream tasks and better retain the edited knowledge.

In summary, our contributions to this paper are three-fold:
(1) This paper statistically elucidates how deviations in the parameter matrix after editing are responsible for the decreased general abilities of the model across downstream tasks after sequential editing.
(2) A framework termed EAC is proposed, which ultimately aims to constrain the deviation of the parameter matrix after editing by compressing the editing information into editing anchors. 
(3) It is discovered that on models like GPT2-XL and LLaMA-3 (8B), EAC significantly preserves over 70\% of the general abilities across downstream tasks and retains the edited knowledge better.

\section{Related Works}
\label{sec::soa}
In this section, we discuss the datasets that provide event-camera data specifically for eye-tracking applications. As previously anticipated, due to the novelty of the sensor and application, the number of available datasets is extremely low.

%\subsection{Angelopoulos} \label{angelopoulos_soa}
%todo~\cite{angelopoulos2020event}
One of the first datasets for the eye-tracking task is the one presented by Angelopoulos in~\cite{angelopoulos2020event}. This dataset has been created using data obtained by the DAVIS sensor, which provides both greyscale and event-based data. The dataset is composed of several eye movements obtained from 27 different users. 
The ground truth provided is unfortunately limited due to the difficulties in generating a pupil-level annotation from the sparse event data. Indeed, Angelopoulos's dataset provides as ground truth the looked point on a screen, not the pupil position in the image frame. This type of annotation has been widely used for many years in the eye-tracking field~\cite{holmqvist_eye_movements} but introduces additional calibration and nonlinearities in the system, which makes a proper evaluation of the pure pupil detection and tracking algorithm less accurate.

%\subsection{Retina} \label{retina_soa}
%todo~\cite{bonazzi2024retina}
A recent addition is the Ini-30 dataset~\cite{bonazzi2024retina}, obtained by recording 30 volunteers using a glass frame equipped with two DVXplorer event cameras. To the best of our knowledge, this is the first event-based dataset where labels are provided at high-frequency directly at frame level and where a screen was not employed to guide the users during the acquisition process and with labels dire. The glasses cameras captured natural eye movements; then, data were manually labeled with a variable sampling period ranging from 20.0ms to 235.77ms. The resulting 50Hz labeling frequency, however, is not high enough to capture all eye movements, as presented in the next sections. Additionally, the absence of a standardized protocol during acquisitions, although enhancing variability, might lead to imbalances in the data, thereby affecting its usability.


%\subsection{EV-Eye} \label{eveye_soa}
%todo~\cite{eveyepaper}
Another dataset was recently presented as part of the EV-Eye paper \cite{eveyepaper}. Similarly to \cite{angelopoulos2020event}, two DAVIS346 are used to capture events and also record near-eye grayscale image sequences at a frame rate of 25fps. In addition, it contains gaze references provided by Tobii Pro Glasses 3. This additional metadata comprises PoGs and pupil diameters of the users at $100Hz$. The dataset is composed by 48 participants (28 male and 20 female) aged between 21 and 35 years. Labels are also provided at sensor level as in~\cite{bonazzi2024retina}, but only at slow frequency as estimated from near-eye grayscale images.


%\subsection{AIS 2024 Challenge} \label{challenge_soa}
%todo \cite{wang2024eventbasedeyetrackingais}
Recently a new event-based dataset has also been presented, the 3ET+ dataset~\cite{wang2024eventbasedeyetrackingais}. In this scenario the data have been obtained using a DVXplorer Mini event camera. The recording consists of 13 distinct users, each exhibiting various eye movements. Unlike the previous datasets, 3ET+ provides the ground truth annotated at 100Hz. Moreover, they provide two different labels: one binary value to indicate the blink status, and the human-labeled pupil center coordinates. However, the data were collected without an IR-pass filter, so they contain events of object reflections mixed with eye movements. This could impact the generalization capability of deep learning algorithms trained on this dataset.

This work advances the current state of the art by providing additional annotations to the Angelopoulos dataset~\cite{angelopoulos2020event}. Specifically, we retrieve the center of the pupil on the image frame at a rate of 200 Hz, a frequency sufficiently high to model all documented eye movements~\cite{holmqvist_eye_movements}. Additionally, we include annotations describing the current state of the eye, labeling the presence of a blink and the status of the saccade.


\section{Ground truth semi-automatic generation}
\label{sec::ann}
In this section, we describe the methodology used to generate pupil position and blink annotations from Angelopoulos's dataset~\cite{angelopoulos2020event}. First, the annotations are computed using an automated process. These annotations are subsequently validated and, if necessary, corrected through manual verification.

\begin{figure}[t]
  \centering
  \includegraphics[width=0.9\textwidth]{image/Polarity_frame.png}
  \caption{Examples of frames generated accumulating events at 200Hz}
  \label{fig:Polarity_frame}
\end{figure}

\subsection{Automatic annotation } \label{automatic_annotation }
A notable drawback of event cameras is the difficulty of directly applying traditional frame-based algorithms to their output data. This issue arises because event cameras generate data as a continuous stream of asynchronous events. Unlike frame-based cameras, event cameras only capture changes in individual pixels, without producing complete images. This limitation impacts the data visualization and the annotation process. Therefore, the first step in the automatic annotation pipeline is to generate RGB frames from the event stream. These frames possess all the characteristics of traditional camera images, allowing them to be visualized and annotated by both the automatic pipeline and the users.

\begin{figure}
\centering
\includegraphics[width=0.80\textwidth, trim=80px 19px 170px 0px, clip]{image/Pipeline_auto.pdf}
\caption{Schema of the automatic annotation pipeline. The system takes the frame generated by accumulating events as input and first predicts if there has been eye movement. If a saccade is detected, the pupil center is determined using a template matching strategy followed by RANSAC estimation.}
\label{fig:auto_annotation_pipeline}
\end{figure}

To generate the event frames, we aggregated events at a sampling rate of $200 Hz$ (i.e., a frame every $5ms$). This means that all events, which represent changing pixels, in a $5ms$ window are summed to reconstruct an RGB image of the eye.  This sampling rate was selected to capture all relevant eye movements. Indeed, saccadic movements typically span from $30 ms$ to $80 ms$, while microsaccadic movements occur within a $10 ms$ to $30 ms$ range~\cite{holmqvist_eye_movements}. The need for such a high sampling frequency to accurately capture all types of eye movements introduces substantial challenges in the annotation process. As the sampling frequency increases, there is a proportional increase in the volume of frames that require annotation, complicating the data processing workflow.

The result of the accumulation process are RGB frames where the only colored points are the events accumulated during the sampling period. The color of the points indicates the polarity of each event: green for positives and red for the negative ones. The output of this first phase of the pipeline is shown in Fig.~\ref{fig:Polarity_frame}. 


The next phase involves automatically generating the raw annotations for each frame. Due to the high frequency at which the data are processed, a significant portion of the frames contain limited information, with most colored pixels attributable to noise. This phenomenon occurs when the eye is nearly stationary; if no movement is detected by the sensor, no events are generated. In these scenarios, where only noise data are available, automatic annotation is problematic due to the lack of a clear pupil. In this scenario, manual inspection would only increase the effort required by the human operator without adding additional benefit to the final result.


%Then, we start by generating a base annotation using an automatic process, in order to reduce manual rework to a minimum. Most of the frames, at this frequency, contain noise and not eye movements, so it would not be feasible to manually inspect these frames. 

%The automatic annotation pipeline is shown in Fig.~\ref{fig:auto_annotation_pipeline}, and it's divided into three major steps:
%\begin{enumerate}
%    \item We need to detect if either a saccade or blink is ongoing.
%    \item If a saccade is ongoing, we find a tentative eye center using a pattern-matching methodology.
%    \item We further improve the center using RANSAC fit on the events contained in a ROI centered on the tentative center.
%\end{enumerate}
%These three steps utilize the method described in~\cite{mentasti2024event}. The saccade classification is performed using a running mean with the same parameters outlined in the article. Similarly, the pattern-matching methodology employs a slightly improved kernel compared with the one presented in that study. Although the RANSAC fit remains the same, it uses a higher number of iterations in this scenario: 1000 iterations.


The automatic annotation pipeline is shown in Fig.~\ref{fig:auto_annotation_pipeline}, and it is divided into three major steps inspired by the method described in:~\cite{mentasti2024event}:
\begin{itemize}
\item \textbf{Eye Movement Detection}: In this step, we need to detect if either a saccade or a blink is ongoing. We can detect the presence of a saccade using a threshold technique. If the events in the frame are more than 150, there is a saccade; otherwise, it is not a saccade. Also, the first time that a frame is detected as a saccade, it is labeled as \textit{SACCADE\_START}. Similarly, the frame is labeled as \textit{SACCADE\_END} if it is the last frame before switching to the frames without detected saccades.
\item \textbf{Match Template:} If a saccade is ongoing, the match template step finds a tentative eye center using a pattern-matching methodology. To do this, we use a 2D convolution similar to~\cite{mentasti2024event}, but with a slightly improved kernel compared with the one presented in the original work. In particular, to improve system performance, we employ only 8 different templates that represent different movement directions. These kernels are applied to the polarity images (i.e., the frames reconstructed integrating events data). The output of this phase is shown in Fig.~\ref{fig:convolutions}. Then, each result is multiplied to create heatmaps, identifying movement direction. The highest-scoring template indicates the motion direction; a bounding box, representing the Region Of Interest (ROI) of the frame, is built centered at the maximum point of the heatmap. The result is visible in~\ref{fig:tentative_center}.
\item \textbf{RANSAC Estimation}: After the match template step, we execute the Random Sample Consensus (RANSAC) algorithm~\cite{fischler1981random}, considering only the events inside the ROI detected earlier. In particular, we employ RANSAC to fit an ellipse described using the equation of a generic conic with the f parameter fixed to -1. Although the RANSAC fit follows the method described in~\cite{mentasti2024event}, it uses 1000 iterations in this scenario to prioritize reliable results over fast execution. The final result is visible in~\ref{fig:ransac_fit}.
\end{itemize}

\begin{figure}[t]
    \centering
    
    \begin{subfigure}[t]{0.32\textwidth}
        \centering
        \includegraphics[width=\linewidth]{image/convolutions_r_cropped.png}
        \caption{Result of a convolution from match template phase, the light blue zone indicates where the pupil can be located.}
        \label{fig:convolutions}
    \end{subfigure}
    \hfill % optional, for better spacing
    \begin{subfigure}[t]{0.32\textwidth}
        \centering
        \includegraphics[width=\linewidth]{image/tentative_center_cropped.png}
        \caption{Expected pupil center after the match template step.}
        \label{fig:tentative_center}
    \end{subfigure}
    \hfill % optional, for better spacing
    \begin{subfigure}[t]{0.32\textwidth}
        \centering
        \includegraphics[width=\linewidth]{image/ransac_fit_cropped.png}
        \caption{Estimated pupil center after the RANSAC estimation step.}
        \label{fig:ransac_fit}
    \end{subfigure}
    
    \caption{Results from different operations performed during the pupil center localization. As we can see, the RANSAC estimation step increases the accuracy of the estimation made by the match template step.}
    \label{fig:predictor_3g}
\end{figure}


\subsection{Fine Tuning Annotation using Human Annotators} \label{fine_tune_annotation}

After generating and automatically annotating each frame, as described in the previous section, the labels are inspected by human annotators with expertise in eye-tracking and event cameras to validate and correct the annotations.

The annotation fine-tuning process is divided into two parts. The first part aims to assign each frame a specific label describing the state of the eye, such as a saccade or blink. In this stage, each annotator reviews the automatic annotations generated in the previous step. Annotators are presented with each frame that contains a number of events exceeding a pre-set threshold, which varies between users. This method excludes all frames with a low number of events, where the eye's state is not clearly defined, and the human annotator would be unable to provide accurate additional information.

Using a simple keyboard shortcut-based interface, each annotator is able to modify the eye center's position, to change the saccade status (\textit{SACCADE\_START}, \textit{SACCADE\_IN\_PROGRESS}, \textit{SACCADE\_END}) and select the current blink status (\textit{BLINK\_STATUS}, \textit{BLINK\_IN\_PROGRESS}, \textit{BLINK\_END}). Having a base annotation greatly speed-up the review process, as for many frames the base annotation itself is correct.

In the second stage, the quality of the labels is evaluated using another custom-made tool. Considering the high frequency of annotations, we expect extremely small eye movements to occur between frames. To leverage this, we introduced a metric based on the difference in eye position between consecutive frames. A threshold for this distance is set to identify potential label anomalies, under the assumption that pupil movement should not exceed this distance within a $5 ms$ interval. The output from this phase is displayed as an interactive plot (see Fig.~\ref{fig:anomaly_detection}). Reviewers can then zoom into specific plot regions where the threshold is breached and directly jump to the corresponding labels for correction.



\begin{figure}[t]
  \centering
  \includegraphics[width=0.9\textwidth]{image/anomaly_detection_detail.png}
  \caption{Interactive plot for annotation correction. Delta in x,y from an eye position to the next is shown in blue and red. Saccades are marked in violet (rising edge when the saccade starts, falling edge when the saccade ends). Blinks are indicated similarly in orange. At the end of the sequence, a possible anomaly is marked in green.}
  \label{fig:anomaly_detection}
\end{figure}


This second step is performed iteratively until the plot no longer exhibits any significant anomalies. Some peak distances may persist due to eye movements occurring during blinks. The presence of these peaks is largely attributed to different physiological mechanisms; for example, downward eye movements during blinks are well-documented in the literature~\cite{Khazali2017}. Moreover, saccades and blinks can be simultaneous.



\section{Dataset statistics}
\label{sec::eval}
Using the method proposed in this paper, we annotated the left eye of six different users from the dataset~\cite{angelopoulos2020event}.

We reviewed and annotated 114222 frames, each containing a varying number of events. The total number of frames reviewed was 315213; however, many did not contain the minimum number of events necessary to recognize and detect an eye. These empty frames likely resulted from periods when the eyes were still and did not generate events.

\begin{table}[t]
\centering
    
\resizebox{\textwidth}{!}{
\setlength{\tabcolsep}{10pt}
\begin{tabular}{c|ccccccc|c|}
 \hline
 \hline
 Statistic& User4 & User5 & User10 & User15 & User18 & User19 & User20 & Total\\
 \hline
 Frame analyzed   & 44020 & 41233 & 46818 & 46008 & 43470 & 43125 & 50539 & 315213\\
 Annotated Frame & 13643  & 16836 & 26440 & 12967 & 11109 & 9109 & 24118 & 114222\\
% Max Events   & 6290 &   5573 & 9022 & 12563 & 7247 & 7646 & 14368 &XXX\\
 Saccade Counts & 406  & 317 & 288 & 317 & 238 & 248 & 287 &2101\\
 Blink Counts & 18 & 3 & 32 & 29 & 14 & 10 & 14 &120\\
 Eye Center Position & 4447 & 2911 & 3400 & 2981 & 2510 & 2917 & 3139 &22305\\
 \hline
 \hline
\end{tabular}
}
\caption{Table reporting core statistics about each annotated user of the Angelopoulos dataset, computed with the proposed method.\label{annotations_statistics}}
\end{table}

In the 114222 annotated frames, it was possible to recognize either an open eye or a blink movement. From these frames, we annotated 22305 distinct eye center positions, detected 2101 saccade movements, and identified 120 blink movements.
All these elements are summarized in Tab~\ref{annotations_statistics}, where it is also possible to see this information data for each annotated user.

These annotations can be used during the training and testing of eye-tracking algorithms that utilize event-camera data to also detect various types of eye movements, such as blinks and saccades.



\section{Conclusions}
\label{sec::end}
With this work, we presented an improved version of the original dataset proposed by Angelopoulos in~\cite{angelopoulos2020event}, together with the pipeline used to generate improved annotations at 200Hz, providing what we believe could be a valuable resource for the scientific community.

The semi-automatic pipeline presented provides a method that reduces both the time required to annotate an event-camera dataset and the potential human error that might arise from a completely manual annotation process. This method can also be applied to future datasets, giving more precise annotations that can help develop and test new algorithms for event-based eye-tracking.

Simultaneously, with the increasing interest in event-based technologies, particularly in low-power applications, we aim to expedite research in real-time eye-tracking using purely event-driven approaches.


\begin{comment}
\[
\mathbf{X} =
\begin{bmatrix}
x \\
y \\
\dot{x} \\
\dot{y}
\end{bmatrix}
\]

\[
\mathbf{F} =
\begin{bmatrix}
1 &  0 & \Delta t & 0 \\
0 & 1 & 0 & \Delta_t\\
0 & 0 & 1 & 0\\
0 & 0 & 0 & 1
\end{bmatrix}
\]

\[
\mathbf{H} =
\begin{bmatrix}
1 & 0 & 0 & 0\\
0 & 1 & 0 & 0\\
\end{bmatrix}
\]

\[
\mathbf{Q} =
\begin{bmatrix}
1 & 0 & 0 & 0\\
0 & 1 & 0 & 0\\
0 & 0 & 1 & 0\\
0 & 0 & 0 & 1
\end{bmatrix}
\]

\[
\mathbf{R} =
\begin{bmatrix}
1 & 0\\
0 & 1\\
\end{bmatrix}
\]

\[
\mathbf{B} = 0
\]
\end{comment}

\section*{Acknowledgments}
This paper was carried out in the EssilorLuxottica Smart Eyewear Lab, a Joint Research Center between EssilorLuxottica and Politecnico di Milano.

%\clearpage\mbox{}Page \thepage\ of the manuscript.
%\clearpage\mbox{}Page \thepage\ of the manuscript.
%\clearpage\mbox{}Page \thepage\ of the manuscript.
%\clearpage\mbox{}Page \thepage\ of the manuscript.
%\clearpage\mbox{}Page \thepage\ of the manuscript. This is the last page.
%\par\vfill\par
%Now we have reached the maximum length of an ECCV \ECCVyear{} submission (excluding references).
%References should start immediately after the main text, but can continue past p.\ 14 if needed.
\clearpage  % TODO REVIEW/FINAL: This \clearpage needs to be removed from both review and camera-ready versions.


% ---- Bibliography ----
%
% BibTeX users should specify bibliography style 'splncs04'.
% References will then be sorted and formatted in the correct style.
%
\bibliographystyle{splncs04}
\bibliography{main}
\end{document}
