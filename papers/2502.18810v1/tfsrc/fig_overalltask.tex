\begin{figure*}[!t]
    \centering     
    \includegraphics[width=0.85\linewidth]{figure/taskoverall.pdf}
    \caption{Illustration of the basic pipeline for LLM knowledge unlearning and its audit. 
    % (1) The unlearning process demonstrates the knowledge removal workflow: An initial LLM is fine-tuned on the complete training dataset to produce a Finetuned-LLM. Upon receiving unlearning requests, the training data is partitioned into retain and forget datasets. The Target LLM—which would ideally be obtained by training directly on the retain dataset from scratch—represents the desired outcome. However, due to the prohibitive computational cost of complete retraining, unlearning algorithms are instead applied to the Finetuned-LLM to produce an Unlearned LLM that approximates the Target LLM's behavior.  (2) The auditing phase, where specially constructed test queries derived from the forget dataset evaluate the effectiveness of knowledge removal and detect potential knowledge residuals in the Unlearned LLM.
    }
    \label{fig:overalltask}
    % \vspace{-12pt}
\end{figure*}